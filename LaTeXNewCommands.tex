\newcommand{\lcaption}[1]{\caption{#1}\label{cap:#1}}
\newcommand{\lsec}[1]{\MakeUppercase{#1}\label{#1}}
\newcommand{\lssec}[1]{\MakeUppercase{#1}\label{#1}}
\newcommand{\lsssec}[1]{#1\label{#1}}

\newcommand{\ssecfake}[1]{\par \vspace{1em}{\Large\bfseries\raggedright #1} \vspace{1em} \par}
\newcommand{\sssecfake}[1]{\par \vspace{0.5em}{}{\large\bfseries\raggedright #1} \vspace{0.5em} \par}
\newcommand{\sssecfakehref}[2]{\par \vspace{0.5em}{}{\large\bfseries\raggedright \hyperlink{#1}{#2}} \vspace{0.5em} \par}
\newcommand{\pref}[1]{page \pageref{#1}}
\newcommand{\pdref}[1]{#1 (page \pageref{#1})}
\newcommand{\pcref}[1]{#1, page \pageref{#1}}
\newcommand{\tref}[1]{Table \ref{cap:#1}: #1 (\pref{cap:#1})}
\newcommand{\trefnp}[1]{Table \ref{cap:#1}: #1}
\newcommand{\trefcp}[1]{Table \ref{cap:#1}: #1, \pref{cap:#1}}
\newcommand{\trefnum}[1]{Table \ref{cap:#1}}

\newcommand{\minus}{\texttt{-{}}}
\newcommand{\plus}{\texttt{+}}    %Use a smaller positive marker for use directly adjacent to numbers
\newcommand{\add}{\texttt{+}{\footnotesize{ }}}    %Use a smaller plus sign
\newcommand{\sub}{\texttt{- }}
\newcommand{\mult}{x}
\newcommand{\mtimes}{x }
\newcommand{\ki}{\emph{ki}\xspace}
\newcommand{\Ki}{\emph{Ki}\xspace}
%NOTED BUG: Spell names with mixed capitalization (Dispel magic) will not work properly.
\newcommand{\spell}[1]{\emph{\mbox{\hyperlink{spell:#1}{#1}}}}    %Italicize spells
\newcommand{\spellindirect}[2]{\emph{\hyperlink{spell:#1}{#2}}}
\newcommand{\ritual}[1]{\emph{#1}}    %Italicize spells
\newcommand{\cd}[1]{\par \textbf{#1}:}    %Bold class fluff description headers
\newcommand{\cf}[2]{\par \textbf{#2:}\label{#1:#2}}    %Bold class feature headers
\newcommand{\cfl}[3]{\par \textbf{#2 -- #3:}\label{#1:#3}}    %class feature with level
\newcommand{\cft}[3]{\par \textbf{#2 (#3):}\label{#1:#2}}    %class feature with extraordinary/supernatural type
\newcommand{\cfnl}[1]{\par \textbf{#1:}}    %class feature with no label
\newcommand{\cflt}[4]{\par \textbf{#2 -- #3 (#4):}\label{#1:#3}}    %class feature with level and extraordinary/supernatural type
\newcommand{\subcf}[1]{\par \emph{#1}:}    %Italicize sub-class-feature headers
\newcommand{\subsk}[1]{\par \emph{#1}:}    %Italicize sub-skill headers

\DeclareDocumentCommand{\parhead}{s m o}
{\par \IfBooleanT{#1}{\noindent}%
    \IfNoValueTF{#3}
    {\textbf{#2}:}
    {\textbf{#2} (#3):}\xspace
}
\newcommand{\subparhead}[1]{\par \emph{#1}:}
\newcommand{\itemhead}[1]{\item \textbf{#1}}
\newcommand{\thead}[1]{\textbf{#1}}    %Bold the header of tables
\newcommand{\x}{---}    %A null entry in a table
\newcommand{\ccol}{\centering\arraybackslash}    %Center columns
\newcommand{\lcol}{\raggedright\arraybackslash}
\newcommand{\rcol}{\raggedleft\arraybackslash}
\newcommand{\booktitle}[1]{\emph{#1}}
%\newcommand{\footnotetemp}[1]{\tiny{[#1]}}    %I don't actually know how to make footnotes yet, so placeholder
\newcommand{\footnotetemp}[1]{\textsuperscript{#1}}
\newcommand{\fn}[1]{\textsuperscript{#1}}
\newcommand{\M}{\textsuperscript{M}}
\newcommand{\F}{\textsuperscript{F}}
\newcommand{\mfx}[1]{\emph{#1}}    % For material components, focuses, and XP costs for spells
\newcommand{\subspell}[1]{\emph{#1}}    %For parts of spells
\newcommand{\magicitem}[1]{\emph{#1}}    %For the names of items
\newcommand{\mitem}[1]{\emph{#1}}
\newcommand{\shead}[1]{\> \textbf{#1}}    %For the headers of spells in spell lists
%\newcommand{\spellhead}[2][]{\item[#1] \textbf{\hyperlink{spell:#2}{#2}:}}    %Spell in a spell list
%\newcommand{\spellheadc}[3][]{\item[#1] \textbf{\hyperlink{spell:#3 #2}{#2, #3}:}}
%\newcommand{\spellheadrestricted}[1]{\item \textit{\textbf{\hyperlink{spell:#1}{#1}:}}}  %Restricted spell in a spell list
%\newcommand{\spellheadrestrictedc}[2]{\item \textit{\textbf{\hyperlink{spell:#2 #1}{#1, #2}:}}}  %Restricted spell in a spell list
\newcommand{\spellsection}[2]{\vspace{1em}\setlength\multicolsep{0pt}\begin{multicols}{2}
    \lowercase{\hypertarget{spell:#1}{}}\hypertarget{spell:#1}{\subsubsection{#1}}\columnbreak
    \begin{flushright}\large\textbf{\nth{#2} Level}\end{flushright}\end{multicols}
}
\newcommand{\spellsectioncomma}[3]{\vspace{1em}\setlength\multicolsep{0pt}\begin{multicols}{2}
    \lowercase{\hypertarget{spell:#2 #1}{}}\hypertarget{spell:#2 #1}{\subsubsection{#1, #2}}\columnbreak
    \begin{flushright}\large\textbf{\nth{#3} Level}\end{flushright}\end{multicols}
}
\newcommand{\invocationsection}[1]{\lowercase{\hypertarget{spell:#1}{}}\hypertarget{spell:#1}{\subsubsection{#1}}}
\newcommand{\invocationsectioncomma}[2]{\lowercase{\hypertarget{spell:#2 #1}{}}\hypertarget{spell:#2 #1}{\subsubsection{#1, #2}}}

\newcommand{\degree}{\ensuremath{^\circ}\xspace}    %Degree symbol
\newcommand{\tind}{\hspace{1em}}

\newcommand{\appdescription}[7]{\subsubsection{#1} \parhead{Price (Level)} #2 gp (#3) \parhead{Body Location} #4 \parhead{Aura, Caster Level} #5, #6 \parhead{Activation} #7 \par}
\newcommand{\appdescriptiondc}[8]{\appdescription{#1}{#2}{#3}{#4}{#5}{#6}{#7} \parhead{Special Attack (Attack Bonus)} #8 \par}
\newcommand{\impdescription}[6]{\subsubsection{#1} \parhead{Price (Level)} #2 gp (#3) \parhead{Aura, Caster Level} #4, #5 \parhead{Activation} #6 \par}
\newcommand{\impdescriptionnoact}[5]{\subsubsection{#1} \parhead{Price (Level)} #2 gp (#3) \parhead{Aura, Caster Level} #4, #5 \par}
\newcommand{\impdescriptiondc}[7]{\impdescription{#1}{#2}{#3}{#4}{#5}{#6} \parhead{Special Attack (Attack Bonus)} #7 \par}
\newcommand{\impdescriptiondcnoact}[6]{\impdescription{#1}{#2}{#3}{#4}{#5} \parhead{Special Attack (Attack Bonus)} #6 \par}
%Spell creation requirements, without descriptors
%School, subschool, spell level, caster level, craft skill
\newcommand{\mitemreq}[5]{\par \textit{Creation Requirements:} #1 (#2) #3; caster level #4 or #5}
\newcommand{\mitemreqdesc}[6]{\par \textit{Creation Requirements:} #1 (#2) [#3] #4; caster level #5 or #6}
%Legacy
\newcommand{\tooldescription}[6]{\subsubsection{#1} \parhead{Price (Level)} #2 gp (#3) \parhead{Caster Level} #4 \parhead{Aura} #5 \parhead{Activation} #6 \par}
%one shot items should not have scaling DCs
\newcommand{\tooldescriptiondcos}[7]{\tooldescription{#1}{#2}{#3}{#4}{#5}{#6} \parhead{Special Attack (Attack Bonus)} #7\par}
\newcommand{\tooldescriptiondc}[7]{\tooldescription{#1}{#2}{#3}{#4}{#5}{#6} \parhead{Special Attack (Attack Bonus)} #7\par}

\newcommand{\prereq}[1]{\subparhead{Prerequisites:} #1}
\newcommand{\bpl}[3]{(\plus #1) #2 gp (#3)}
\newcommand{\bcl}[2]{(\plus #1) #2}

%Monster commands for format consistency
%Starting with \\ instead of \par yields an indent
\newcommand{\montypes}[5]{\begin{mstatblock}\par #1 #2 #3 \hfill \textbf{CR} #4 [#5]}
\newcommand{\montypessubtypes}[6]{\begin{mstatblock}\par #1 #2 #3 (#4) \hfill \textbf{CR} #5 [#6]}
\newcommand{\monsenses}[2]{\par \textbf{Init} #1; Perception #2}
\newcommand{\monsensesspellcraft}[3]{\par \textbf{Init} #1; Perception #2, Spellcraft #3}
\newcommand{\monsensesspecial}[3]{\par \textbf{Init} #1; Perception #2; \textbf{Senses} #3}
\newcommand{\monsensesfull}[4]{\par \textbf{Init} #1; Perception #2, Spellcraft #3; \textbf{Senses} #4}
\newcommand{\monaura}[1]{\par \textbf{Aura} #1}
\newcommand{\monlanguages}[1]{\par \textbf{Languages} #1}
\newcommand{\monac}[5]{\par \textbf{AC} #1, touch #2, flat-footed #3; \textbf{CMD} #4 \\ (#5)}
\newcommand{\monacspecial}[6]{\par \textbf{AC} #1, touch #2, flat-footed #3; \textbf{CMD} #4; #5 \\ (#6)}
\newcommand{\monhp}[2]{\par \textbf{HP} #1 (#2 HV)}
\newcommand{\mondr}[1]{\textbf{DR} #1}
\newcommand{\monhpdr}[3]{\mhp{#1}{#2}; \mdr{#3}}
\newcommand{\monimmune}[1]{\par \textbf{Immune} #1}
\newcommand{\monresist}[1]{\par \textbf{Resist} #1}
\newcommand{\monsr}[1]{\textbf{SR} #1}
\newcommand{\monresistsr}[2]{\mres{#1}; \msr{#2}}
\newcommand{\monsaves}[3]{\par \textbf{Fort} \plus #1, \textbf{Ref} \plus #2, \textbf{Will} \plus #3}
\newcommand{\monsavesnegative}[3]{\par \textbf{Fort} #1, \textbf{Ref} #2, \textbf{Will} #3}
\newcommand{\monspeed}[1]{\textbf{Speed} #1}
\newcommand{\monmelee}[1]{\par \textbf{Melee} #1}
\newcommand{\monrange}[1]{\par \textbf{Ranged} #1}
\newcommand{\monspace}[2]{\par \textbf{Space} #1 ft.; \textbf{Reach} #2 ft.}
\newcommand{\monspacenofeet}[2]{\par \textbf{Space} #1; \textbf{Reach} #2}
\newcommand{\moncmb}[2]{\par \textbf{BAB} \plus #1; \textbf{CMB} \plus #2}
\newcommand{\moncmbnegative}[2]{\par \textbf{BAB} \plus #1; \textbf{CMB} #2}
\newcommand{\monspecialact}[1]{\par \textbf{Special Actions} #1}
\newcommand{\monsla}[5]{\par \textbf{Spell-Like Abilities (Sp)} (CL #1, DC #2, #3/day, #4--based): \\ #5}
\newcommand{\monpsi}[5]{\par \textbf{Psionics (Sp)} (CL #1, DC #2, #3/day, #4--based): \\ #5}
\newcommand{\monslamore}[1]{\\ #1}
\newcommand{\monattributes}[6]{\par \textbf{Attributes} Str #1, Dex #2, Con #3, Int #4, Wis #5, Cha #6}
\newcommand{\monsa}[1]{\par \textbf{SA} #1}
\newcommand{\monsq}[1]{\par \textbf{SQ} #1}
\newcommand{\monfeats}[1]{\par \textbf{Feats} #1}
\newcommand{\monskills}[1]{\par \textbf{Skills} #1}
\newcommand{\monitems}[1]{\par \textbf{Items} #1}
\newcommand{\monability}[2]{\par \textbf{#1} #2}
%http://tex.stackexchange.com/questions/128636/center-hrule-in-the-middle-of-the-page
\newcommand{\monlinerule}{\par\noindent\hfil\rule[0.5em]{0.9\columnwidth}{0.5pt}\hfil\par}
\newcommand{\mondescription}[1]{\par \end{mstatblock}\sssecfake{Description}\label{#1:Description}}
\newcommand{\mondescriptionnoblock}[1]{\par \sssecfake{Description}\label{#1:Description}}
\newcommand{\monbehavior}[1]{\sssecfake{Combat}\label{#1:Behavior}}

\newcommand{\wilditem}{\addtocounter{enumi}{1} \item}
\newcommand{\wilditemplus}{\addtocounter{enumi}{3} \item}

\newcommand{\spelldesc}[1]{\par \textit{#1}}
\newcommand{\spelldmg}[1]{\parhead*{Damage} #1}
\newcommand{\spellheal}[1]{\parhead*{Healing} #1}
\newcommand{\spellattack}[1]{\parhead*{Attack} #1}
\newcommand{\spellattacktarget}[2]{\spelltabularcompressed \thead{Attack:} #1 & \thead{Target:} #2 \\\end{tabularx}}
\newcommand{\spellattacktargets}[2]{\spelltabularcompressed \thead{Attack:} #1 & \thead{Targets:} #2 \\\end{tabularx}}
\newcommand{\spellattackseparatetarget}[1]{\parhead{Target} #1}
\newcommand{\spellattackseparatetargets}[1]{\parhead{Targets} #1}

\newcommand{\spelltargetattack}[2]{\spelltabularcompressed \thead{Target:} #1 & \thead{Attack:} #2 \\\end{tabularx}}
\newcommand{\spelltargetsattack}[2]{\spelltabularcompressed \thead{Targets:} #1 & \thead{Attack:} #2 \\\end{tabularx}}

%\newcommand{\spellfocus}[1]{\par\noindent \textit{Focus:} #1}
\newcommand{\spellmat}[1]{\par\noindent \textit{Material Components:} #1}

\newcommand{\spelltwocol}[2]{\spelltabularcompressed #1 & #2\end{tabularx}}
\newcommand{\spellinfo}[2]{\spelltwocol{#1}{\thead{Lists:} #2}}

\newcommand{\spellschool}[1]{#1}
\newcommand{\spelllists}[1]{\parhead*{Lists} #1}
\newcommand{\spellcmp}[1]{\parhead*{Components} #1}
\newcommand{\spelltime}[1]{\parhead*{Casting Time} #1}
\newcommand{\spellrng}[1]{\parhead*{Range} #1}

\newcommand{\spellarea}[1]{\parhead*{Area} #1}
\newcommand{\spellburst}[1]{\parhead*{Burst} #1}
\newcommand{\spellemanation}[1]{\parhead*{Emanation} #1}
\newcommand{\spelllimit}[1]{\parhead*{Limit} #1}
\newcommand{\spellzone}[1]{\parhead*{Zone} #1}
\newcommand{\spellzoneandburst}[1]{\parhead*{Zone and Burst} #1}

%\newcommand{\spelltgt}[1]{\parhead*{Target} #1}
%\newcommand{\spelltgts}[1]{\parhead*{Targets} #1}
\newcommand{\spelltgtortgts}[1]{\parhead*{Target or Targets} #1}
\newcommand{\spelltgtorarea}[1]{\parhead*{Target or Area} #1}
\newcommand{\spelltgteffarea}[1]{\parhead*{Target or Area} #1}

\newcommand{\spelldur}[1]{\parhead*{Duration} #1}
\newcommand{\spellfocus}[1]{\parhead*{Focus} #1}
\newcommand{\spellsr}[1]{\parhead*{Spell Resistance} #1}

\newcommand{\rngpers}{Personal}
\newcommand{\rngtouch}{Touch}
\newcommand{\rngclose}{Close \reminder{30 ft.}\xspace}
\newcommand{\rngmed}{Medium \reminder{100 ft.}\xspace}
\newcommand{\rnglong}{Long \reminder{300 ft.}\xspace}
\newcommand{\rngfar}{Long \reminder{300 ft.}\xspace}
\newcommand{\rngext}{Extreme \reminder{1,000 ft.}\xspace}
\newcommand{\areasmall}{Small \reminder{10 ft.}\xspace}
\newcommand{\areamed}{Medium \reminder{20 ft.}\xspace}
\newcommand{\arealarge}{Large \reminder{50 ft.}\xspace}
\newcommand{\durshort}{Short \reminder{Concentration \add 5 rounds}\xspace}
\newcommand{\durmed}{Medium \reminder{5 minutes}\xspace}
\newcommand{\durlong}{Long \reminder{1 hour}\xspace}
\newcommand{\durext}{Extreme \reminder{12 hours}\xspace}

\newcommand{\confusionexplanation}{Each turn, a confused creature has a random chance to take one of four actions: babble incoherently, flee from the caster as if panicked, attack the nearest creature, or act normally. A confused character who can't carry out the indicated action does nothing but babble incoherently. Any confused character who is attacked automatically attacks its attackers on its next turn, as long as it is still confused when its turn comes.\xspace}

\newcommand{\bonusscalingdescription}{This bonus increases to \plus3 at 8th level, to \plus4 at 14th level, and finally to \plus5 at 20th level.\xspace}
\newcommand{\spellbonusscalingdescription}{This bonus increases to \plus3 at 8th caster level, to \plus4 at 14th caster level, and finally to \plus5 at 20th caster level.\xspace}
\newcommand{\babscalingdescription}{This bonus increases to \plus3 at base attack bonus \plus8, to \plus4 at base attack bonus \plus12, and finally to \plus5 at base attack bonus \plus16.\xspace}

\newcommand{\feat}[2]{\hypertarget{feat:#1}{\sssecfakehref{ft:#1}{#1 [#2]}}\label{feat:#1}}
\newcommand{\featpre}{\parhead{Prerequisite} }
\newcommand{\featpres}{\parhead{Prerequisites} }
\newcommand{\featben}{\parhead{Benefit} }
\newcommand{\featspecial}{\parhead{Special} }
\newcommand{\featref}[1]{\hyperlink{feat:#1}{\hypertarget{ft:#1}{#1}}}
%featref no target
\newcommand{\featrefnt}[1]{\hyperlink{feat:#1}{#1}}
\newcommand{\featpref}[1]{\pageref{feat:#1}}
%\newcommand{\featref}[1]{}
\newcommand{\stylereq}{\parhead{Style Requirement}}

\newcommand{\featbanenotes}{You can only apply the benefits of a single Bane feat against a particular foe.}

\newcommand{\skill}[2]{\subsection{#1 (#2)}\label{#1}}

\newcommand{\reminder}[1]{\textcolor{darkgray}{\textit{(#1)}}}

\newcommand{\vulnerableexplanation}{A vulnerable creature takes a \minus2 penalty to attacks, defenses, and checks.\xspace}
\newcommand{\vulneffect}{\minus2 to attacks, defenses, and checks\xspace}
\newcommand{\conditionlink}[1]{\hyperlink{condition:#1}{#1}\xspace}
\newcommand{\bewildered}{\conditionlink{bewildered} \reminder{\vulneffect}\xspace}
\newcommand{\blinded}{\conditionlink{blinded} \reminder{unable to see, moves at half speed, defenseless}\xspace}
\newcommand{\bloodied}{\conditionlink{bloodied} \reminder{half hit points}\xspace}
\newcommand{\confused}{\conditionlink{confused} \reminder{randomly babbles, flees, attacks nearest, or acts normally}\xspace}
\newcommand{\dazzled}{\conditionlink{dazzled} \reminder{20\% miss chance, \minus4 to visual Perception}\xspace}
\newcommand{\exhausted}{\conditionlink{exhausted} \reminder{move at half speed, unable to sprint or charge, \minus4 to physical attacks, defenses, and checks}}
\newcommand{\entangled}{\conditionlink{entangled} \reminder{move at half speed, unable to sprint or charge, \minus2 to physical attacks, defenses, and checks}\xspace}
\newcommand{\fascinated}{\conditionlink{fascinated} \reminder{Unable to act unless threatened}}
\newcommand{\fatigued}{\conditionlink{fatigued} \reminder{unable to sprint or charge, \vulneffect}\xspace}
\newcommand{\frightened}{\conditionlink{frightened} \reminder{flees, \vulneffect}\xspace}
\newcommand{\dazed}{\conditionlink{dazed} \reminder{unable to act}\xspace}
\newcommand{\deafened}{\conditionlink{deafened} \reminder{unable to hear}\xspace}
\newcommand{\helpless}{\conditionlink{helpless} \reminder{physical defenses are 10, vulnerable to coup de grace}\xspace}
\newcommand{\ignited}{\conditionlink{ignited} \reminder{d6 damage/round, Dex DC 10 to extinguish, \vulneffect}\xspace}
\newcommand{\immobilized}{\conditionlink{immobilized} \reminder{Unable to leave its location}\xspace}
\newcommand{\nauseated}{\conditionlink{nauseated} \reminder{unable to act in action phase, \vulneffect}\xspace}
\newcommand{\negativelevels}{\conditionlink{negative levels} \reminder{\minus1 to attacks, special defenses, and checks, \minus5 hit points, lose spell slot}}
\newcommand{\panicked}{\conditionlink{panicked} \reminder{flees, \vulneffect}\xspace}
\newcommand{\paralyzed}{\conditionlink{paralyzed} \reminder{helpless, unable to move}\xspace}
\newcommand{\prone}{\conditionlink{prone} \reminder{\minus4 to melee attack and defense, \plus4 vs ranged attacks, move action to stand}}
\newcommand{\shaken}{\conditionlink{shaken} \reminder{\vulneffect}\xspace}
\newcommand{\sickened}{\conditionlink{sickened} \reminder{\vulneffect}\xspace}
\newcommand{\slowed}{\conditionlink{slowed} \reminder{unable to act in movement phase, \minus2 to physical attacks, defenses, and checks}\xspace}
\newcommand{\staggered}{\conditionlink{staggered} \reminder{unable to act in movement phase, \vulneffect}\xspace}
\newcommand{\stunned}{unable to act, \vulneffect\xspace}
\newcommand{\unencumbered}{\conditionlink{unencumbered} \reminder{not carrying a heavy load or wearing medium or heavy armor}\xspace}
\newcommand{\monkunencumbered}{not wearing armor or encumbered by weight (see \pcref{Encumbrance})\xspace}
\newcommand{\vulnerable}{\conditionlink{vulnerable} \reminder{\vulneffect}\xspace}

\newcommand{\condition}[1]{\parhead{#1}\label{#1}\hypertarget{condition:#1}{}}

\newcommand{\concealment}{concealment \reminder{\plus4 to physical defenses}}

\newcommand{\destructivespellnotes}{If a destructive spell deals enough damage to an interposing barrier to shatter or breaks through it, its effects may continue beyond the barrier if the area permits; otherwise, it stops at the barrier just as any other spell effect does.\xspace}
\newcommand{\cursespellnotes}{Curses cannot be dispelled with \spell{dispel magic}, but can be removed with \spell{break enchantment} or \spell{remove curse}.\xspace}
\newcommand{\firespellnotes}{Fire spells shed light equivalent to a torch. They do not function underwater. They can ignite combustable materials, such as dry wood and cloth.\xspace}
\newcommand{\forcespellnotes}{Force spells also affect the Ethereal Plane.}
\newcommand{\fogspellnotes}{Fog spells do not function underwater and can be dispersed by wind. Localized wind, such as from a \spell{gust of wind} spell, only disperses the fog where it overlaps the fog. A fire spell burns away the fog in the area into which it deals damage.\xspace}
\newcommand{\fogwindspellnotes}{A moderate wind (11\add mph) disperses the fog in 5 rounds; a strong wind (21\add mph) disperses the fog in 1 round.\xspace}
\newcommand{\sensorspellnotes}{A magic sensor is a floating, invisible sphere approximately one inch in diameter. It has 1 hit point and an Armor defense of 8. Its special defenses are the same as yours.\xspace}
\newcommand{\mobilesensorspellnotes}[1]{\sensorspellnotes The sensor has a #1 foot fly speed with perfect maneuverability. It is unable to enter another plane of existence, even through a \spell{gate} or similar magical portal.}
\newcommand{\norepeatspellnotes}{You can only affect any individual creature with this spell once per 24 hours.\xspace}
\newcommand{\norepeatnotes}{You can only affect any individual creature with this ability once per 24 hours.\xspace}

\newcommand{\spelltabular}{\par\noindent\begin{tabularx}{\columnwidth}{>{\lcol}X l}}
\newcommand{\spelltabularcompressed}{\par\noindent\begin{tabularx}{\linewidth}{@{}>{\lcol}X l}}
\newcommand{\spelltablecolumns}{>{\lcol}X l}

\newcommand{\spelllevelschool}[2]{\spelllvl{#1} & #2 \\}
\newcommand{\spelllevelnew}[3]{\thead{\nth{#1} level} #2 & #3 \\}

\newcommand{\dismissable}{(Dismissable)}
\newcommand{\shapeable}{(S)}
\newcommand{\rngunrestricted}{(Unrestricted)}
\newcommand{\undispellable}{(Undispellable)}

% http://tex.stackexchange.com/questions/135617/how-to-i-reduce-space-before-and-after-hrule
\newcommand{\HRule}[2][\medskipamount]{\par
  \vspace*{\dimexpr-\parskip-\baselineskip+#1}
  \noindent\rule{\linewidth}{#2}\par
  \vspace*{\dimexpr-\parskip-.5\baselineskip+#1}}

\newcommand{\spellline}{\par\HRule[0.5\medskipamount]{0.2pt}}
% weird spacing before twocol
\newcommand{\spelllinespaced}{\spellline\vspace{0.5\medskipamount}}

%\newcommand{\spelleffect}{\parhead*{Effect}}
%\newcommand{\spellsuccess}{\parhead*{Success}}
%\newcommand{\spellfailure}{\parhead*{Failure}\xspace}
\newcommand{\spellnotes}{\spellline\par\noindent\textit{Notes}:\xspace}
\newcommand{\spellspecial}{\parhead*{Special}}
\newcommand{\spellcheck}[1]{\parhead*{Check} #1}

\DeclareDocumentCommand{\spellfailure}{o}
{
    \IfNoValueTF{#1}
    {\parhead*{Failure}}
    {\parhead*{Failure}[#1]}
}
\DeclareDocumentCommand{\spellsuccess}{o}
{
    \IfNoValueTF{#1}
    {\parhead*{Success}}
    {\parhead*{Success}[#1]}
}
\DeclareDocumentCommand{\spelleffect}{o}
{
    \IfNoValueTF{#1}
    {\parhead*{Effect}}
    {\parhead*{Effect}[#1]}
}
\DeclareDocumentCommand{\spelltgt}{m o}
{
    \IfNoValueTF{#2}
    {\parhead*{Target} #1}
    {\parhead*{#2 Target} #1}
}
\DeclareDocumentCommand{\spelltgts}{m o}
{
    \IfNoValueTF{#2}
    {\parhead*{Targets} #1}
    {\parhead*{#2 Targets} #1}
}

\ExplSyntaxOn
\DeclareDocumentCommand{\ritualcostparser}{m}
{
    \ifcase#1\relax
    \or 5~gp    %1
    \or 20~gp   %2
    \or 50~gp   %3
    \or 125~gp  %4
    \or 300~gp  %5
    \or 750~gp  %6
    \or 1500~gp %7
    \or 3000~gp %8
    \or 7500~gp %9
    \else ???
    \fi
}
\ExplSyntaxOff

\ExplSyntaxOn
\DeclareDocumentCommand{\ritualcost}{s m o}
{
    \IfBooleanF{#1}{\spellline}
    \spellmat{\ritualcostparser#2\xspace
        \IfValueTF{#3}
        {#3}
        {in~ritual~components.}
    }
}
\ExplSyntaxOff

\ExplSyntaxOn
\DeclareDocumentCommand{\ifstr}{m m m o}
{
    \ifnum \pdfstrcmp{#1}{#2}=0 #3
    \else \IfValueT{#4}{#4}
    \fi
}
\ExplSyntaxOff

\ExplSyntaxOn
\DeclareDocumentCommand{\spellhead}{o s m o}
{
    \IfValueTF{#1}
    {\item[#1]}
    {\item}
    \IfBooleanTF{#2}
    {
        \textit{
            \textbf{
                \IfValueTF{#4}
                {\hyperlink{spell:#4~#3}{#3,~#4}}
                {\hyperlink{spell:#3}{#3}}
            }
        }
    }
    {
        \textbf{
            \IfValueTF{#4}
            {\hyperlink{spell:#4~#3}{#3,~#4}}
            {\hyperlink{spell:#3}{#3}}
        }
    }:
}
\ExplSyntaxOff
