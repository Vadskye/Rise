\chapter{Classes}
Your character's class represents the things your character has chosen to train in. This choice determines a great deal about your character's abilities.

\section{Class Introductions}
There are eleven classes in Rise.
\begin{itemize*}
  \item Barbarians are mighty warriors who can enter a deadly battlerage.
  \item Clerics are divine spellcasters who draw power from their veneration of a deity or ideal.
  \item Druids are nature spellcasters who draw power from their veneration of the natural world.
  \item Fighters are highly disciplined warriors who excel in physical combat of any variety.
  \item Monks are agile masters of ``\ki'' who hone their personal abilities to strike down foes and perform supernatural feats.
  \item Paladins are holy warriors whose devotion to their deity grants them the ability to discern and smite evil.
  \item Rangers are skilled hunters who bridge the divide between nature and civilization.
  \item Rogues are exceptionally skillful characters known for their ability to strike at their foe's weak points in combat.
  \item Sorcerers are arcane spellcasters with an intuitive and flexible understanding of magic.
  \item Spellwarped wield a unique blend of martial skill and narrowly focused magical abilities.
  \item Wizards are arcane spellcasters with a highly studied and deep understanding of magic.
\end{itemize*}

\subsection{Class Description Format}

\parhead{Alignment} Some classes have alignment restrictions. See Chapter 6: Description for a description of what alignments are.

\parhead{Hit Value} At each level, you gain hit points equal to your class's Hit Value \add half your Constitution.

\parhead{Class Skills} These are skills that members of this class are typically good at.

\parhead{Skill Points} This is the number of skill points that members of this class get.

\parhead{Weapon and Armor Proficiencies} These are the types of equipment that members of this class are trained in using.

\parhead{Base Attack Progression} This measures how skilled a character is in combat. A character adds his base attack bonus to all attacks he makes, and half his base attack bonus to his physical defenses, which represents how hard he is to hit. There are three progressions: Good, Average, or Poor. The effects of each progression are described on \trefnp{Base Progressions}.

A high base attack bonus can grant additional attacks, as described in \pcref{Combat}.

\parhead{Base Defense Progressions}\label{Base Defense Progressions} This measures how resistant members of the class are to unusual kinds of attacks. There are three kinds of special defenses. Your Fortitude defense represents your ability to resist attacks to your body, like poisons and diseases. Your Reflex defense represents your ability to avoid attacks, such as pit traps or explosions. Your Will defense represents your ability to resist mental influence, like fearsome creatures and enchantment spells. There are three progressions: Good, Average, or Poor. The effects of each progression are described on \trefnp{Base Progressions}.

\begin{dtable}
    \lcaption{Base Progressions}
    \begin{tabularx}{\columnwidth}{l l X}
        \thead{Progression} & \thead{Attack Bonus} & \thead{Special Defense Bonus} \\
        Good & Class level & Class level \add 2 \\
        Average & Three-quarters class level & Three-quarters class level \add 1 \\
        Poor & One-half class level & One-half class level \\
    \end{tabularx}
\end{dtable}

\parhead{Class Features} The class features that a character gets for being a member of the class.

\section{Class Descriptions}

\subsection{Barbarian}
\begin{dtable*}
\lcaption{The Barbarian}
\begin{tabularx}{\textwidth}{>{\ccol}p{\levelcol} >{\ccol}p{\babcolgood} *{3}{>{\ccol}p{\babcolgood}} X}
\thead{Level} & \thead{Base Attack Bonus} & \thead{Fort} & \thead{Ref} & \thead{Will} & \thead{Special} \\
1st & \plus1         & \plus3 & \plus1 & \plus0 & Damage reduction, rage \plus2 \\
2nd & \plus2         & \plus4 & \plus2 & \plus1 & Endurance, fast movement \\
3rd & \plus3         & \plus5 & \plus3 & \plus1 & Channeled rage, uncanny dodge \\
4th & \plus4         & \plus6 & \plus4 & \plus2 & Grit \\
5th & \plus5         & \plus7 & \plus4 & \plus2 & Improved damage reduction \\
6th & \plus6/\plus1  & \plus8 & \plus5 & \plus3 & Improved uncanny dodge, channeled rage \\
7th & \plus7/\plus2  & \plus9 & \plus6 & \plus3 & Larger than life \\
8th & \plus8/\plus3  & \plus10& \plus7 & \plus4 & Rage \plus3 \\
9th & \plus9/\plus4  & \plus11& \plus8 & \plus4 & Channeled rage \\
10th& \plus10/\plus5 & \plus12& \plus8 & \plus5 & Greater uncanny dodge \\
11th& \plus11/\plus6/\plus1  & \plus13 & \plus9 & \plus5 & Tireless rage \\
12th& \plus12/\plus7/\plus2  & \plus14 & \plus10& \plus6 & Channeled rage, chaotic rage \\
13th& \plus13/\plus8/\plus3  & \plus15 & \plus10& \plus6 & Indomitable will \\
14th& \plus14/\plus9/\plus4  & \plus16 & \plus11& \plus7 & Rage \plus4 \\
15th& \plus15/\plus10/\plus5 & \plus17 & \plus12& \plus7 & Channeled rage, greater damage reduction \\
16th& \plus16/\plus11/\plus6/\plus1 & \plus18 & \plus13& \plus8 & Improved grit \\
17th& \plus17/\plus12/\plus7/\plus2 & \plus19 & \plus13& \plus8 & Larger than belief \\
18th& \plus18/\plus13/\plus8/\plus3 & \plus20 & \plus14& \plus9 & Channeled rage \\
19th& \plus19/\plus14/\plus9/\plus4 & \plus21 & \plus15& \plus9 & Deathless rage \\
20th& \plus20/\plus15/\plus10/\plus5& \plus22 & \plus16 & \plus10 & Limitless rage, rage \plus5
\end{tabularx}
\end{dtable*}

\cd{Alignment} Any nonlawful.

\cd{Hit Value} 7.

\sssecfake{Class Skills}
The barbarian's class skills (and the key attribute for each skill) are
Athletics (Str), Climb (Str), Swim (Str), Acrobatics (Dex), Ride (Dex), Perception (Wis), Survival (Wis), Creature Handling (Cha), and Intimidate (Cha).
\cf{Bbn}{Skill Points at 1st Level} 4

\sssecfake{Class Features}

All of the following are class features of the barbarian.
\cf{Bbn}{Weapon and Armor Proficiency}  A barbarian is proficient with simple weapons, any four other weapon groups, light armor, medium armor, and shields (except tower shields).

\cft{Bbn}{Damage Reduction}{Ex} A barbarian has the ability to shrug off some amount of injury from attacks. He has physical damage reduction equal to his barbarian level. This damage reduction allows the barbarian to ignore the first points of physical damage he takes each round.

\cft{Bbn}{Rage}{Ex} A barbarian can fly into a rage as a swift action. While in a rage, he gains a \plus2 bonus to Strength and Charisma, and gains 2 temporary hit points per barbarian level. However, he takes a \minus2 penalty to physical defenses. The extra hit points gained from raging are lost before any other hit points (see \pcref{Temporary Hit Points}). While raging, a barbarian cannot take any action that requires patience or concentration, such as casting spells.

A barbarian's ability to maintain his rage depends on his willpower. A fit of rage lasts for a number of rounds equal to 5 \add the barbarian's (newly increased) Charisma. He may prematurely end his rage, and it ends automatically if he becomes unconscious. At the end of the rage, the barbarian takes nonlethal damage equal to the number of temporary hit points he gained by raging. If the barbarian has any temporary hit points remaining at the end of his rage, the nonlethal damage is dealt to those hit points before they go away. In addition, he becomes fatigued until he rests for 5 minutes. A fatigued character can neither sprint nor charge and is \vulnerable. The barbarian cannot enter a rage while he is fatigued from his previous rage.

The amount by which the barbarian's attributes increase, and the number of temporary hit points gained per barbarian level, is called the barbarian's rage bonus. A barbarian can fly into a rage a number of times per day equal to his rage bonus. The barbarian's rage bonus improves to \plus3 at 8th barbarian level, to \plus4 at 14th barbarian level, and finally to \plus5 at 20th barbarian level. His penalty to physical defenses while raging remains the same.

\cfl{Bbn}{2nd}{Endurance} A barbarian gains Endurance (see \hyperlink{feat:Endurance}{Endurance}) as a bonus feat. If he already has Endurance, he may gain any other feat for which he qualifies as a bonus feat.

\cflt{Bbn}{2nd}{Fast Movement}{Ex} A barbarian increases his land speed by 10 feet while unencumbered.

\cflt{Bbn}{3rd}{Uncanny Dodge}{Ex} A barbarian can react to danger before his senses would normally allow him to do so. He is not helpless while unaware of an attack.

If a barbarian already has uncanny dodge from a different class, he stacks those class levels to determine whether he gains improved uncanny dodge (see below) instead.

\cfl{Bbn}{3rd}{Channeled Rage} The barbarian gains the ability to enter a channeled rage whenever he rages. Each channeled rage grants the barbarian additional abilities while in that rage or changes the nature of his rage. All bonuses granted by channeled rages apply only while the barbarian is in that channeled rage.

A barbarian can only be in one channeled rage at a time. By spending an additional use of his rage ability, he can change which channeled rage he is in without exiting the rage, but this does not extend the duration of the rage.

At his 6th barbarian level, and every three barbarian levels thereafter, the barbarian gains an additional channeled rage. Some channeled rages require a minimum barbarian level, as indicated before the name of the ability. All channeled rage abilities are extraordinary abilities unless otherwise noted. The barbarian's special attack bonus with rage powers is equal to his barbarian level \add his Charisma.

\subcf{Athletic Rage} The barbarian adds his rage bonus to his Athletics, Climb, and Swim checks. Additionally, he is always treated as having a running start when jumping.

\subcf{Agile Rage} The barbarian's rage increases his Dexterity in place of his Strength.

\subcf{Endless Rage} The barbarian's rage lasts for an additional 5 rounds.

%Needs to be stronger
\subcf{Fearless Rage} The barbarian becomes immune to fear and harmful morale effects. All allies who can see or hear him gain a \plus4 bonus to special defenses against fear and harmful morale effects.

\subcf{Savage Rage} The barbarian gains the unarmed warrior ability (see Unarmed Warrior, \pref{Mnk:Unarmed Warrior}), increasing his power with unarmed attacks (1d6 damage for a Medium barbarian).

\subcf{Wary Rage} The barbarian only suffers a \minus1 penalty to physical defenses for raging.

\subcf{Willful Rage} The barbarian adds his rage bonus to his Will defense.

\subcf{6th -- Terrifying Rage (Su)} Whenever the barbarian makes a physical attack, he may also make a special attack vs. Will against all enemies he threatens. A successful attack makes a creature \shaken for 5 rounds. This ability can only affect any individual creature once per 24 hours.

\subcf{6th -- Unstoppable Rage} Each round, the barbarian can shove an opponent as a swift action that does not provoke attacks of opportunity.

\subcf{9th -- Overwhelming Rage (Su)} The barbarian counts as two creatures for the purpose of determining overwhelm penalties. This does not allow him to overwhelm a creature by himself.

\subcf{12th -- Overpowering Rage} The barbarian adds his rage bonus to his maneuver attacks.

\subcf{12th -- Spellbreaker Rage (Su)} The barbarian gains spell resistance while raging. To affect a creature with spell resistance using a spell or spell-like ability, a special attack is always required. The defense used against the attack is indicated by the spell. If the attack fails, the spell has no effect on the barbarian.

\subcf{15th -- Mindless Rage} The barbarian becomes immune to mind-affecting spells and effects.

\subcf{18th -- Invulnerable Rage} The barbarian doubles his damage reduction. This replaces his rage bonus to Strength.

\cflt{Bbn}{4th}{Grit}{Ex} The barbarian's resilience allows him to shrug off magical effects. If he resists a attack against his Fortitude that deals half damage on a failed attack, he instead takes no damage.

\cflt{Bbn}{5th}{Improved Damage Reduction}{Ex} The barbarian can apply half his damage reduction against any attack which deal damage, including magical and supernatural attacks.

\cflt{Bbn}{6th}{Improved Uncanny Dodge}{Ex} The barbarian is always treated as being threatened by two fewer creatures than he actually is for the purpose of determining overwhelm penalties. This defense can deny a barbarian the ability to sneak attack the barbarian.
\par If a character already has improved uncanny dodge from a second class and gains improved uncanny dodge, the character stacks those class levels to determine if he should gain greater uncanny dodge.

\cflt{Bbn}{7th}{Larger than Life}{Ex} A barbarian holds the strength of a giant in the body of a man (or woman). The barbarian is treated as being one size category larger than he actually is for the purpose of maneuvers he performs or is the target of, checks that are affected by size (such as Strength checks to break down doors), and whether a creature's special attacks based on size can affect him if doing so is advantageous to him. In addition, though he uses weapons of the same size, his weapons deal damage as if they were one size category larger, including natural weapons and unarmed strikes. The barbarian's space and reach remain those of a creature of his actual size. The benefits of this class feature stack with the effects of spells and abilities that increase the barbarian's size category.

\cflt{Bbn}{10th}{Greater Uncanny Dodge}{Ex} The barbarian no longer suffers overwhelm penalties, regardless of the number of foes surrounding him.

\cflt{Bbn}{11th}{Tireless Rage}{Ex} The barbarian no longer becomes fatigued at the end of his rage.

\cflt{Bbn}{12th}{Chaotic Rage}{Ex} The barbarian gains the ability to change channeled rage abilities at will, without consuming an additional use of his rage ability. He may not change channeled rages in this way more than once per round.

\cflt{Bbn}{13th}{Indomitable Will}{Ex} The barbarian becomes immune to compulsion and domination spells and effects.

\cflt{Bbn}{15th}{Greater Damage Reduction}{Ex} The barbarian can apply his full damage reduction against any attack which deal damage, including magical and supernatural attacks.

\cflt{Bbn}{16th}{Improved Grit}{Ex} The barbarian's fortitude knows no bounds. If he fails to resist an attack against his Fortitude that deals half damage on a failed attack, he takes only half damage from the attack.

\cflt{Bbn}{17th}{Larger than Belief}{Ex} The barbarian's larger than life ability improves. He is treated as being two size categories larger than he actually is.

\cflt{Bbn}{19th}{Deathless Rage}{Ex} The raging barbarian can scorn death and unconsciousness. As long as his rage continues, he is not staggered at 0 hit points, and cannot take critical damage. However, every 50 points of damage he takes in excess of his hit points reduces the duration of his rage by one round, and the Endless Rage channeled rage ability does not extend the duration of his rage if he is at 0 hit points. Once his rage ends, the effects of the barbarian's wounds apply normally if they have not been healed. This ability does not prevent death from sources other than damage, such as from the \spell{finger of death} or \spell{disintegrate} spells.

\cflt{Bbn}{20th}{Limitless Rage}{Ex} The barbarian may rage at will. He no longer has any limitation on the number of times he can rage each day. He may still rage no more than once per encounter.

\subsection{Cleric}
\begin{dtable*}
\lcaption{The Cleric}
\begin{tabularx}{\textwidth}{>{\ccol}p{2em} >{\ccol}p{7em} *{3}{>{\ccol}p{\savecol}} >{\lcol}X}
\thead{Level} & \thead{Base Attack Bonus} & \thead{Fort} & \thead{Ref} & \thead{Will} & \thead{Special} \\
1st & \plus0 & \plus1 & \plus0 & \plus3 & Matters of faith, lesser domain aspect \\
2nd & \plus1 & \plus2 & \plus1 & \plus4         & Channel energy \\
3rd & \plus2 & \plus3 & \plus1 & \plus5         & Lesser domain aspect \\
4th & \plus3 & \plus4 & \plus2 & \plus6         & \x \\
5th & \plus3 & \plus4 & \plus2 & \plus7         & Channeled domain power \\
6th & \plus4 & \plus5 & \plus3 & \plus8         & \x  \\
7th & \plus5 & \plus6 & \plus3 & \plus9         & Channeled domain power  \\
8th & \plus6/\plus1 & \plus7 & \plus4 & \plus10    & \x  \\
9th & \plus6/\plus1 & \plus7 & \plus4 & \plus11    & Domain aspect  \\
10th & \plus7/\plus2 & \plus8 & \plus5 & \plus12    & \x  \\
11th & \plus8/\plus3 & \plus9 & \plus5 & \plus13   & Domain aspect  \\
12th & \plus9/\plus4 & \plus10& \plus6 & \plus14    & \x  \\
13th & \plus9/\plus4 & \plus10& \plus6 & \plus15    & Greater channeled domain power  \\
14th & \plus10/\plus5 & \plus11& \plus7 & \plus16    & \x  \\
15th & \plus11/\plus6/\plus1 & \plus12& \plus7 & \plus17 & Greater channeled domain power  \\
16th & \plus12/\plus7/\plus2 & \plus13& \plus8 & \plus18 & \x  \\
17th & \plus12/\plus7/\plus2 & \plus13& \plus8 & \plus19 & Domain mastery  \\
18th & \plus13/\plus8/\plus3 & \plus14& \plus9 & \plus20 & \x  \\
19th & \plus14/\plus9/\plus4 & \plus15& \plus9 & \plus21 & Domain mastery  \\
20th & \plus15/\plus10/\plus5 & \plus16 & \plus10 & \plus22 & \x  \\
\end{tabularx}
\end{dtable*}

\cd{Alignment} A cleric's alignment must be within one step of his deity's (that is, it may be one step away on either the lawful-chaotic axis or the good-evil axis, but not both). A cleric may not be neutral unless his deity's alignment is also neutral.

\cd{Hit Value} 5.

\sssecfake{Class Skills}
The cleric's class skills (and the key attribute for each skill) are Knowledge (arcana) (Int), Knowledge (local) (Int), Knowledge (religion) (Int), Knowledge (the planes) (Int), Heal (Wis), Sense Motive (Wis), Spellcraft (Wis), Persuasion (Cha), and Intimidate (Cha).

\cf{Clr}{Skill Points at 1st Level} 2.

\sssecfake{Class Features}
All of the following are class features of the cleric.

\cf{Clr}{Weapon and Armor Proficiency}  Clerics are proficient with simple weapons, any two other weapon groups, light and medium armor, and shields (except tower shields).

\cf{Clr}{Bonus Languages} A cleric's bonus language options include Celestial, Abyssal, and Infernal (the languages of good, chaotic evil, and lawful evil outsiders, respectively). These choices are in addition to the bonus languages available to the character because of his race.

\cf{Clr}{Domains} A cleric chooses two domains, which represent his personal spiritual inclinations. If he has a deity, he must choose his domains from among those his deity offers. A cleric's choice of domains has broad effects on the cleric's spellcasting and supernatural abilities. Each domain has an associated domain attribute which is used for the domain's abilities. The domains and their attributes are listed below. If a domain offers a choice of multiple attributes, the cleric must choose which attribute to use when he gains the domain, and that choice cannot normally be changed.

\subcf{Air} Dexterity or Wisdom
\subcf{Chaos} Charisma
\subcf{Death} Constitution
\subcf{Destruction} Strength or Charisma
\subcf{Earth} Constitution or Wisdom
\subcf{Evil} Charisma
\subcf{Fire} Dexterity or Wisdom
\subcf{Good} Wisdom or Charisma
\subcf{Knowledge} Intelligence
\subcf{Law} Wisdom
\subcf{Magic} Intelligence
\subcf{Protection} Constitution or Wisdom
\subcf{Strength} Strength
\subcf{Travel} Any physical attribute
\subcf{Trickery} Dexterity or Charisma
\subcf{Vitality} Constitution
\subcf{War} Any 
\subcf{Water} Dexterity or Wisdom

\cf{Clr}{Spells} A cleric casts divine spells using his Charisma. To learn or cast a spell, a cleric must have a Charisma at least equal to half the spell's level. A cleric's magic attack bonus equals half his caster level \add his Charisma.

Like other spellcasters, the number of spells a cleric knows and can cast each day is limited. These limitations are given below on \trefnp{Cleric Spells per Day} and \trefnp{Cleric Spells Known}. A cleric's spells are drawn from the divine spell list (\pcref{Divine Spells}), as well as from his domains (\pcref{Cleric Domains}). Two spells at every spell level must be drawn from the cleric's domains; sometimes these spells are normal spells on the cleric's spell list, but often they are only accessible by the domain. A cleric may also choose spells from his domain lists with his normal spells known.

Clerics meditate or pray for their spells. Each cleric must choose a time at which he must spend 1 hour each day performing a ritual, worshipping, or quietly contemplating to regain his daily allotment of spells. They do not need to rest to regain spells.

A cleric can't cast spells of an alignment opposed to his own or his deity's (if he has one). Spells associated with particular alignments are indicated by the chaos, evil, good, and law descriptors in their spell descriptions.

\begin{dtable}
    \lcaption{Cleric Spells per Day} 
    \centering
    \begin{tabularx}{\columnwidth}{>{\ccol}X *{9}{>{\ccol}p{\spellcol}}}
        & \multicolumn{9}{c}{\thead{---{}---{}---{}---{}---{}---{}---{}---Spell Level---{}---{}---{}---{}---{}---{}---{}---}} \\
        \thead{Level} & \thead{1st} & \thead{2nd} & \thead{3rd} & \thead{4th} & \thead{5th} & \thead{6th} & \thead{7th} & \thead{8th} & \thead{9th} \\
        1st & 3 & \x & \x & \x & \x & \x & \x & \x & \x \\
        2nd & 4 & \x & \x & \x & \x & \x & \x & \x & \x \\
        3rd & 5 & \x & \x & \x & \x & \x & \x & \x & \x \\
        4th & 6 & 3 & \x & \x & \x & \x & \x & \x & \x \\
        5th & 6 & 4 & \x & \x & \x & \x & \x & \x & \x \\
        6th & 6 & 5 & 3 & \x & \x & \x & \x & \x & \x \\
        7th & 6 & 6 & 4 & \x & \x & \x & \x & \x & \x \\
        8th & 6 & 6 & 5 & 3 & \x & \x & \x & \x & \x \\
        9th & 6 & 6 & 6 & 4 & \x & \x & \x & \x & \x \\
        10th & 6 & 6 & 6 & 5 & 3 & \x & \x & \x & \x \\
        11th & 6 & 6 & 6 & 6 & 4 & \x & \x & \x & \x \\
        12th & 6 & 6 & 6 & 6 & 5 & 3 & \x & \x & \x \\
        13th & 6 & 6 & 6 & 6 & 6 & 4 & \x & \x & \x \\
        14th & 6 & 6 & 6 & 6 & 6 & 5 & 3 & \x & \x \\
        15th & 6 & 6 & 6 & 6 & 6 & 6 & 4 & \x & \x \\
        16th & 6 & 6 & 6 & 6 & 6 & 6 & 5 & 3 & \x \\
        17th & 6 & 6 & 6 & 6 & 6 & 6 & 6 & 4 & \x \\
        18th & 6 & 6 & 6 & 6 & 6 & 6 & 6 & 5 & 3 \\
        19th & 6 & 6 & 6 & 6 & 6 & 6 & 6 & 6 & 4 \\
        20th & 6 & 6 & 6 & 6 & 6 & 6 & 6 & 6 & 6 \\
    \end{tabularx}
\end{dtable}

\begin{dtable}
\lcaption{Cleric Spells Known}
\centering
\begin{tabularx}{\columnwidth}{>{\ccol}X *{9}{>{\ccol}p{\spellcol}}}
& \multicolumn{9}{c}{\thead{---{}---{}---{}---{}---{}---{}---{}---Spell Level---{}---{}---{}---{}---{}---{}---{}---}} \\
\thead{Level} & \thead{1st} & \thead{2nd} & \thead{3rd} & \thead{4th} & \thead{5th} & \thead{6th} & \thead{7th} & \thead{8th} & \thead{9th} \\
1st  & 0\plus 2 & \x & \x & \x & \x & \x & \x & \x & \x \\
2nd  & 1\plus 2 & \x & \x & \x & \x & \x & \x & \x & \x \\
3rd  & 2\plus 2 & \x & \x & \x & \x & \x & \x & \x & \x \\
4th  & 2\plus 2 & 0\plus 2 & \x & \x & \x & \x & \x & \x & \x \\
5th  & 3\plus 2 & 1\plus 2 & \x & \x & \x & \x & \x & \x & \x \\
6th  & 3\plus 2 & 1\plus 2 & 0\plus 2 & \x & \x & \x & \x & \x & \x \\
7th  & 3\plus 2 & 2\plus 2 & 1\plus 2 & \x & \x & \x & \x & \x & \x \\
8th  & 3\plus 2 & 2\plus 2 & 1\plus 2 & 0\plus 2 & \x & \x & \x & \x & \x \\
9th  & 3\plus 2 & 2\plus 2 & 2\plus 2 & 1\plus 2 & \x & \x & \x & \x & \x \\
10th & 3\plus 2 & 2\plus 2 & 2\plus 2 & 1\plus 2 & 0\plus 2 & \x & \x & \x & \x \\
11th & 3\plus 2 & 2\plus 2 & 2\plus 2 & 2\plus 2 & 1\plus 2 & \x & \x & \x & \x \\
12th & 3\plus 2 & 2\plus 2 & 2\plus 2 & 2\plus 2 & 1\plus 2 & 0\plus 2 & \x & \x & \x \\
13th & 3\plus 2 & 2\plus 2 & 2\plus 2 & 2\plus 2 & 2\plus 2 & 1\plus 2 & \x & \x & \x \\
14th & 3\plus 2 & 2\plus 2 & 2\plus 2 & 2\plus 2 & 2\plus 2 & 1\plus 2 & 0\plus 2 & \x & \x \\
15th & 3\plus 2 & 2\plus 2 & 2\plus 2 & 2\plus 2 & 2\plus 2 & 2\plus 2 & 1\plus 2 & \x & \x \\
16th & 3\plus 2 & 2\plus 2 & 2\plus 2 & 2\plus 2 & 2\plus 2 & 2\plus 2 & 1\plus 2 & 0\plus 2 & \x \\
17th & 3\plus 2 & 2\plus 2 & 2\plus 2 & 2\plus 2 & 2\plus 2 & 2\plus 2 & 1\plus 2 & 1\plus 2 & \x \\
18th & 3\plus 2 & 2\plus 2 & 2\plus 2 & 2\plus 2 & 2\plus 2 & 2\plus 2 & 1\plus 2 & 1\plus 2 & 0\plus 2 \\
19th & 3\plus 2 & 2\plus 2 & 2\plus 2 & 2\plus 2 & 2\plus 2 & 2\plus 2 & 1\plus 2 & 1\plus 2 & 1\plus 2 \\
20th & 3\plus 2 & 2\plus 2 & 2\plus 2 & 2\plus 2 & 2\plus 2 & 2\plus 2 & 1\plus 2 & 1\plus 2 & 1\plus 2
\end{tabularx}
\end{dtable}

\begin{dtable!*}
\lcaption{Deities}
\begin{tabularx}{\textwidth}{X l X}
\thead{Deity} & \thead{Alignment} & \thead{Domains} \\
Guftas, horse god of justice & Lawful good & Good, Law, Strength, Travel \\
Lucied, paladin god of justice & Lawful good & Destruction, Good, Leadership, War \\
Simor, fighter god of protection & Lawful good & Good, Law, Life, Protection \\
Vanya, centaur god of nature & Neutral good & Good, Strength, War, Wild \\
Brushtwig, pixie god of creativity & Chaotic good & Chaos, Good, Trickery \\
Chavi, god of stories & Chaotic good & Chaos, Knowledge, Leadership, Trickery \\
Ivan Ivanovitch, bear god of strength & Chaotic good & Chaos, Strength, War, Wild \\
Raphael, monk god of retribution & Lawful neutral & Death, Law, Protection, Travel \\
Declan, god of fire & True neutral & Destruction, Fire, Knowledge, Magic \\
%Kalten, half-orc god of smiths
Kurai, shaman god of nature & True neutral & Air, Earth, Fire, Water \\
Murdoc, god of mercenaries & Chaotic neutral & Destruction, Knowledge, Leadership, War\\
Daeghul, god of slaughter & Chaotic evil & Destruction, Evil, Magic, War \\
%Ribo, halfling god of trickery & Chaotic neutral & Chaos, Trickery, Water
\end{tabularx}
\end{dtable!*}

\cf{Clr}{Rituals} Clerics, like other spellcasters, can perform rituals to create unique magical effects (see \pcref{Rituals}). A cleric begins play with a ritual book containing one divine ritual of his choice (see \pcref{Divine Rituals}).

\cft{Clr}{Lesser Domain Aspect}{Su} A cleric's abilities are shaped by his domains. Each domain grants a lesser domain aspect. Lesser domain aspects are not activated. Options for domain aspects are listed at \pcref{Lesser Domain Aspects}.

At his 2nd cleric level, the cleric gains an additional lesser domain aspect from one of his domains.

\cft{Clr}{Matters of Faith}{Ex} A cleric gains a \plus10 bonus to Knowledge (religion) checks made concerning his faith, such as questions about his deity or philosophy, religious rites, holy sites, and so on. Further, he is treated as being trained in Knowledge (religion) when making such checks, whether or not he actually is.

\cflt{Clr}{2nd}{Channel Energy}{Su} By channeling the power of his faith through his holy (or unholy) symbol, a cleric can act as a powerful conduit of divine energy. He must choose whether to channel positive or negative energy. Once this choice is made, it cannot be reversed.

When a cleric channels energy, he affects all creatures in a \areamed radius burst centered on him, including himself if he desires. The cleric may choose to exclude a number of other creatures from the effect equal to 1 \add half his Wisdom. The amount of damage dealt (if negative energy is channeled) or healed (if positive energy is channeled) is equal to 1d6 damage per two cleric levels. The cleric must make a special attack vs. Fortitude against unwilling targets. The energy burst deals or heals half damage if the attack fails.

Channeling energy is a standard action that does not provoke attacks of opportunity. A cleric can channel energy a number of times per day equal to 3 \add half his Charisma. A cleric must be able to present his holy symbol to use this ability. The abilities used 

\cflt{Clr}{5th}{Channeled Domain Power}{Su} The cleric gains a channeled domain power from one of his domains. Unless otherwise stated, using a channelled domain power is identical to using channel energy and consumes a use of the cleric's channel energy ability. Instead of channeling positive or negative energy, the cleric instead gains the effect of the channelled domain power. If a channeled domain power deals damage, it functions like channeling negative energy unless otherwise noted. If a channeled domain power heals damage, it functions like channeling postive energy unless otherwise noted. His attack bonus with a channeled domain power is equal to his cleric level \add his domain attribute. The channeled domain powers are described at \pcref{Channeled Domain Powers}.

At his 7th cleric level, the cleric gains an additional channeled domain power from one of his domains.

\cflt{Clr}{9th}{Domain Aspect}{Su} The cleric gains a domain aspect from one of his domains. Domain aspects do not require an action to activate. Options for domain aspects are listed at \pcref{Domain Aspects}.

At his 11th cleric level, the cleric gains an additional domain aspect from one of his domains.

\cflt{Clr}{13th}{Greater Channeled Domain Power}{Su} The cleric gains a greater channeled domain power from one of his domains. Using a greater channeled domain power consumes two uses of the cleric's channel energy ability. Instead of channeling positive or negative energy, the cleric instead gains the effect of the greater channeled domain power. Options for greater channeled domain powers are listed at \pcref{Greater Channeled Domain Powers}.

At his 15th cleric level, the cleric gains an additional greater channeled domain power from one of his domains.

\cflt{Clr}{17th}{Domain Mastery}{Su} The cleric gains a domain mastery from one of his domains. Options for domain masteries are listed at \pcref{Domain Masteries}.

At his 19th cleric level, the cleric gains an additional domain mastery from one of his domains.

\ssecfake{Cleric Domain Abilities}

\subsubsection{Lesser Domain Aspects}\label{Lesser Domain Aspects}

\subcf{Air} The cleric adds Athletics to his cleric class skill list. In addition, he gains a bonus equal to half his cleric level on Athletics checks made to jump.
\subcf{Chaos} Any lawful creature the cleric touches is \bewildered for 5 rounds.
\subcf{Death} The cleric adds half his cleric level to the maximum amount of critical damage he can take without dying.
\subcf{Destruction} The cleric can ignore half of the hardness of any object or creature he damages, whether with spells or weapons.
\subcf{Earth} The cleric gains Endurance as a bonus feat.
\subcf{Evil} Any good creature the cleric touches is \sickened for 5 rounds.
\subcf{Fire} The cleric gains fire and cold damage reduction equal to twice his cleric level. This damage reduction allows him to ignore the first points of fire or cold damage he would take each round.
\subcf{Good} Any evil creature the cleric touches is \shaken for 5 rounds.
\subcf{Knowledge} The cleric adds all Knowledge skills to his cleric class skill list.
\subcf{Law} Any chaotic creature the cleric touches is \vulnerable for 5 rounds.
\subcf{Magic} The cleric gains an additional spell slot at his highest level of spells.
\subcf{Protection} The cleric gains his choice of Covering Fire or Guardian as a bonus feat.
\subcf{Strength} The cleric adds Athletics, Climb, and Swim to his cleric class skill list.
\subcf{Travel} The cleric adds Knowledge (geography) and Survival to his cleric class skill list.
\subcf{Trickery} The cleric adds Bluff and Disguise to his cleric class skill list.
\subcf{Vitality} The cleric gains a bonus equal to half his cleric level on Heal checks.
\subcf{War} The cleric gains Weapon Focus with his deity's favored weapon group as a bonus feat. If he does not have a deity, he gains it in a weapon group related to his alignment or ideals.
\subcf{Water} The cleric adds Swim to his cleric class skill list and halves the penalties he takes for fighting underwater.
\subcf{Wild} The cleric adds Creature Handling, Knowledge (nature), and Survival to his cleric class skill list.

\subsubsection{Channeled Domain Powers}\label{Channeled Domain Powers}

\subcf{Air} The cleric channels electrical energy. This deals electricity damage and heals creatures with the air subtype. A failed attack deals half damage.
\subcf{Chaos} The cleric channels anarchic energy. Roll randomly each time this power is used to determine whether it functions as channeling negative energy or channeling positive energy, except that it always heals chaotic creatures and harms lawful creatures. The cleric may choose which creatures to exclude from the effect after determining which effect is generated.
\subcf{Death} The cleric channels negative energy, except that any creatures dealt critical damage by this power are instantly killed. This is a death effect.
\subcf{Destruction} The cleric channels destructive energy. This functions like channeling negative energy, except that it deals sonic damage.
\subcf{Earth} The cleric channels seismic energy, making a Reflex attack to deal physical bludgeoning damage to all creatures in the area on the ground. A failed attack deals half damage.
\subcf{Evil} The cleric channels negative energy, except that it has no effect on evil creatures.
\subcf{Fire} The cleric channels fiery energy, making a Reflex attack to deal fire damage to all creatures in the area. Creatures with the fire subtype are healed instead. A failed attack deals half damage.
\subcf{Good} The cleric channels positive energy, except that it has no effect on evil creatures.
\subcf{Knowledge} The cleric channels insight from knowledge. Each subject is not healed or damaged, but it gains an enhancement bonus equal to half the cleric's level on the next physical attack or check that it makes. If this bonus is not used within 5 rounds, it is wasted.
\subcf{Law} The cleric channels axiomatic energy, making an attack vs. Will to deal damage to all creatures within a 40 ft. cube centered on the cleric. The attack deals 4 damage per two cleric levels. A failed attack deals half damage. This has no effect on lawful creatures.
\subcf{Magic} The cleric channels magical energy. This heals creatures who can cast spells and deals damage to creatures who cannot.
\subcf{Protection} The cleric grants each subject temporary hit points equal to half the amount that channeling positive energy would have healed. Undead gain temporary hit points as well. The temporary hit points last for 5 rounds.
\subcf{Strength} The cleric channels energy as normal, except that he also gains a \plus2 enhancement bonus to Strength for 1 round. At 8th, 14th, and 20th cleric level, this bonus increases by \plus1. 
\subcf{Travel} The cleric channels positive energy. All subjects gain a \plus10 foot enhancement bonus to movement speed for 1 round.
\subcf{Trickery} The cleric channels trickery and confusion, making an attack vs. Will against each creature in the area. A successful attack deals and heals no damage, but makes the creature \bewildered for 5 rounds. This is a mind-affecting effect.
\subcf{Vitality} The cleric channels energy as normal, except that he deals or heals an extra 1d6 damage.
\subcf{War} The cleric channels energy as normal, except that he can exclude two additional creatures from the effect.
\subcf{Wild} The cleric channels positive energy, except that the cleric can decide whether it acts as positive or negative energy to animals and plants.

\subsubsection{Domain Aspects}\label{Domain Aspects}

\subcf{Air -- Glide} The cleric gains a glide speed equal to his land speed. See \pcref{Gliding}, for more details.
\subcf{Chaos -- Uncertain Fate} Whenever the cleric would take 10, he instead rolls 2d20 and uses whichever roll he prefers.
\subcf{Death -- Lifedrinker} Whenever the cleric kills a creature with a death effect other than \spell{death knell}, he automatically gains the benefits of a \spell{death knell} spell as if it was cast on the creature he killed.
\subcf{Destruction -- }
\subcf{Earth -- Anchored} The cleric gains a \plus4 bonus to maneuver defense against overrun, shove, and trip attempts while standing on solid ground.
\subcf{Evil -- Unholy Aura} Good creaures are \sickened as long as they are adjacent to the cleric.
\subcf{Fire -- Friendly Fire} All of the cleric's fire spells and abilities deal only half damage to his allies.
\subcf{Good -- Purifying Aura} Evil creatures are \shaken as long as they are adjacent to the cleric.
\subcf{Knowledge -- Knowledge Mastery} The cleric may choose a number of Knowledge skills equal to his Intelligence (minimum 1). He may take 10 with those skills if he is not in danger or rushed.
\subcf{Law -- Certain Triumph} Whenever the cleric would take 10, he may instead take 12, treating any roll lower than a 12 as if it had been a 12.
\subcf{Magic -- Magic Feat} The cleric gains a bonus magic feat or metamagic feat.
\subcf{Protection -- Faithful Shield} The cleric may maintain concentration on Abjuration (Shielding) effects as a swift action.
\subcf{Strength -- Legendary Strength} The cleric gains Legendary Strength as a bonus feat, even if he does not meet the prerequisites.
\subcf{Travel -- Rapid Traveller} The cleric gains a \plus10 foot bonus to his base land speed.
\subcf{Trickery -- Legendary Liar} The cleric gains Legendary Liar as a bonus feat, even if he does not meet the prerequisites.
\subcf{Vitality -- }
\subcf{War -- Weapon Specialization} The cleric gains Weapon Specialization in his deity's favored weapon group as a bonus feat. If he does not have a deity, he gains it in a weapon group related to his alignment or ideals.
\subcf{Water -- Water Breathing} The cleric may breathe and speak normally while underwater, as the \spell{water breathing} ritual. He may also pass through boggy or wet areas with no penalty to his movement speed.
\subcf{Wild -- Favored Terrain} The cleric gains a favored terrain, as the ranger class feature (see \pcref{Rgr:Favored Terrain}).

\subsubsection{Greater Channeled Domain Powers}\label{Greater Channeled Domain Powers}

\subcf{Air -- Mantle of Air} As a swift action, the cleric can surround himself in a mantle of air for 5 rounds. Thrown and projectile weapons have a 50\% chance to miss him while this effect is active. Unusually large weapons, such as a giant's boulders, may suffer a decreased miss chance as appropriate to their size.
\subcf{Chaos -- Invoke Chaos} This power functions like the Chaos channeled domain power, except that it randomly generates negative energy, positive energy, or both. If both effects are generated, the cleric may exclude creatures separately from each effect. 
\subcf{Death -- Invoke Death} This power functions like the Death channeled domain power, except that any creature brought to 0 hit points by this effect immediately dies. This is a death effect.
\subcf{Destruction -- Tide of Destruction} The cleric channels destructive energy as the Destruction channeled domain power, except that any creature damaged by the effect is also filled with a destructive resonance for 5 rounds. The first time each round that each subject takes damage, that damage is increased by half the cleric's level.
\subcf{Earth -- Mantle of Earth} As a swift action, the cleric can surround himself in a mantle of earth for 5 rounds. The mantle grants him physical damage reduction equal to his cleric level. This allows him to ignore the first points of damage he would take each round. If he is struck by an adamantine weapon, he cannot use his damage reduction for 1 round.
\subcf{Evil -- Invoke Evil} The cleric channels negative energy, except that it heals evil creatures. 
\subcf{Fire -- Mantle of Fire} As a swift action, the cleric can surround himself in a mantle of fire for 5 rounds. He gains the effect of a \spell{fire shield} spell, with a caster level equal to his cleric level \add his domain attribute.
\subcf{Good -- Invoke Good} The cleric channels positive energy, except that it deals divine damage to evil creatures. 
\subcf{Knowledge -- See the Truth} As a swift action, the cleric can gain the benefit of the \spell{true seeing} spell until the end of his turn.
\subcf{Law -- Invoke Law} This power functions like the Law channeled domain power, except that the attack automatically succeeds against chaotic creatures.
\subcf{Magic -- Invoke Magic} This power functions like the Magic channeled domain power, except that affected spellcasters receive a \plus5 enhancement bonus to the next spell they cast. If this bonus is not used within 5 rounds, it is wasted. 
\subcf{Protection -- Invoke Sanctuary} This power functions like the Protection channeled domain power, except that each subject also receives the benefit of a \spell{sanctuary} spell for 5 rounds. If a subject attacks, the \spell{sanctuary} is broken for that creature, but not for any other subject. 
\subcf{Strength -- Invoke Strength} As a swift action, the cleric can add his cleric level as an enhancement bonus to his Strength until the end of his turn.
\subcf{Travel -- Invoke Speed} As a swift action, the cleric can double his movement speed with all forms of movement until the end of his turn. This is considered an enhancement bonus. In addition, he does not provoke attacks of opportunity for any movement he makes during this time.
\subcf{Trickery -- Swift Invisibility} As a swift action, the cleric can gain the benefit of the \spell{invisibility} spell until the end of his turn.
\subcf{Vitality -- }
\subcf{War -- Warmaster's Boon} The cleric can use this power as part of casting a spell that affects a single creature other than himself. The spell also affects the cleric. This lasts for the normal duration of the spell or for a number of rounds equal to the cleric's domain attribute, whichever is shorter.
\subcf{Water -- Aquatic Globe} The cleric creates water out of thin air in an immobile \areamed radius spread centered on his original location for 5 rounds. Everything within the area is underwater. After 1 round, the sphere grows to fill an \arealarge radius spread. At the end of the duration, the water evaporates, leaving no trace that it was ever there.
\subcf{Wild -- Wild Aspect} When the cleric gains this ability, he chooses one wild aspect ability, as if he were a were a druid of a level equal to his cleric level (see Wild Aspect, \pref{Drd:Wild Aspect}). When he uses this ability, he may embody that wild aspect. This effect lasts as long as that wild aspect would normally last.

\subsubsection{Domain Masteries}\label{Domain Masteries}

\subcf{Air -- Flight} The cleric gains a fly speed (good maneuverability) equal to his land speed. He may remain flying for up to 5 rounds at a time. After that, he must land for 1 round before he can fly again. See \pcref{Flying}, for more details.
\subcf{Chaos -- Avatar of Luck} Once per round, the cleric can gain a \plus1d6 bonus to any physical attack or check. He may declare the use of the ability after failing the roll, but before any additional effects are resolved, potentially making it succeed where it would have failed.
\subcf{Death -- Deathfeeder} The cleric constantly radiates a \areamed radius emanation of death. Whenever a creature dies within the area, he gains the benefits of the \spell{death knell} spell as if it had been cast on the creature.
\subcf{Destruction -- Ruinbringer} The cleric's attacks and spells ignore all damage reduction and hardness (but not damage immunity).
\subcf{Earth -- Earth Glide} The cleric gains the earth glide ability, as an earth elemental.
\subcf{Evil -- Avatar of Evil} The cleric continuously gains the benefits of the \spell{protection from good} spell, with a caster level equal to his cleric level \add his domain attribute. If the effect is dispelled or suppressed, he can resume it as a swift action.
\subcf{Fire -- Flaming Soul} The cleric gains the fire subtype, making him immune to fire but giving him a 50\% vulnerability to cold damage. In addition, whenever he deals fire damage to a creature, the creature is \ignited for 5 rounds.
\subcf{Good -- Avatar of Good} The cleric continuously gains the benefits of the \spell{protection from evil} spell, with a caster level equal to his cleric level \add his domain attribute. If the effect is dispelled or suppressed, he can resume it as a swift action.
\subcf{Knowledge -- Combat Insight} The cleric gains a \plus2 bonus to physical attacks, checks, and special defenses against creatures he has identified with a successful Knowledge check.
\subcf{Law -- Avatar of Order} Once per round, if the cleric rolls less than a 10 on a d20, he may treat the result as if it were a 10, potentially causing him to succeed where he would have failed. You must declare the use of this ability before any additional effects from the roll are resolved.
\subcf{Magic -- Spellfeeder} The cleric gains spell resistance. To affect a creature with spell resistance using a spell or spell-like ability, a special attack is always required. The defense used against the attack is indicated by the spell. If the attack fails, the spell has no effect on the cleric. In addition, whenever the cleric resists a spell with his spell resistance, he regains a spell slot of a level up to one lower than the level of the resisted spell.
\subcf{Protection -- Martyr's Gift} The cleric constantly radiates a \areamed radius emanation of protective energy. Whenever a creature within the area takes damage, the cleric can choose to take half of that damage instead, as the \spell{share pain} spell.
\subcf{Strength -- Might of the Gods} The cleric gains the larger than life ability, as the barbarian class feature (see Larger than Life, \pref{Bbn:Larger than Life}).
\subcf{Travel -- Perfect Stride} The cleric gains perfect stride, as the ranger class feature. He constantly acts as if he were under the effect of a \spell{freedom} spell, except that it does not allow him to act normally underwater.
\subcf{Trickery -- Exemplar of Deceit} The cleric continuously gains the benefits of the \spell{nondetection} spell, with a caster level equal to his cleric level \add his domain attribute, except that it also protects him from any spells or effects which would prevent him from lying or reveal his lies. If the effect is dispelled or suppressed, he can resume it as a swift action.
\subcf{Vitality -- }
\subcf{War -- Warmaster's Power} The cleric continuously gains the benefits of the \spell{divine power} spell, with a caster level equal to his cleric level \add his domain attribute. If the effect is dispelled or suppressed, he can resume it as a swift action.
\subcf{Water -- Water's Flow} As a swift action, the cleric can transform himself into a rushing flow of water with a volume roughly equal to his normal volume until the end of his turn. In this form, he may move wherever water could go, but he cannot take other actions, such as jumping, attacking, or casting spells. His speed is halved when moving uphill and doubled when moving downhill. He does not provoke attacks of opportunity for moving, and has physical damage reduction equal to his cleric level. He may return to his normal form as a free action.
\par If the water is split, he may reform from anywhere the water has reached, to as little as a single ounce of water. If not even an ounce of water exists contiguously, his body reforms from the largest available parts of water, cut into pieces of appropriate size. This usually causes the cleric to die.
\subcf{Wild -- Natural Power} Whenever the cleric is in natural terrain, he gains a \plus2 bonus to caster level and the improved natural casting ability, as the druid class feature (see Improved Natural Casting, \pref{Drd:Improved Natural Casting}).

\subsubsection{Ex-Clerics}
A cleric who grossly violates the code of conduct required by his god loses all spells and supernatural cleric class features. He cannot thereafter gain levels as a cleric of that god until he atones (see the \spell{atonement} spell description).

\subsection{Druid}
\begin{dtable}
\lcaption{The Druid}
\begin{tabularx}{\columnwidth}{>{\ccol}p{\levelcol} >{\centering}p{\babcolavg} *{3}{>{\ccol}p{\savecol}} >{\ccol}X}
\thead{Level} & \thead{Base Attack Bonus} & \thead{Fort} & \thead{Ref} & \thead{Will} & \thead{Special} \\
1st & \plus0 & \plus3 & \plus0 & \plus1 & Nature sense, wild speech \\
2nd & \plus1 & \plus4 & \plus1 & \plus2 & Woodland stride \\
3rd & \plus2 & \plus5 & \plus1 & \plus3 & Wild aspect \\
4th & \plus3 & \plus6 & \plus2 & \plus4 & Venom immunity \\
5th & \plus3 & \plus7 & \plus2 & \plus4 & Natural casting, wild aspect \\
6th & \plus4 & \plus8 & \plus3 & \plus5 & Wild speech (plants) \\
7th & \plus5 & \plus9 & \plus3 & \plus6 & Wild aspect \\
8th & \plus6/\plus1 & \plus10& \plus4 & \plus7 & Improved wild speech, multiple wild aspect \\
9th & \plus6/\plus1 & \plus11& \plus4 & \plus7 & Wild aspect \\
10th & \plus7/\plus2 & \plus12& \plus5 & \plus8 & Improved natural casting \\
11th & \plus8/\plus3 & \plus13 & \plus5 & \plus9  & Natural aspect \\
12th & \plus9/\plus4 & \plus14 & \plus6 & \plus10 & A thousand faces \\
13th & \plus9/\plus4 & \plus15 & \plus6 & \plus10 & Natural aspect \\
14th & \plus10/\plus5 & \plus16 & \plus7 & \plus11 & Elemental speech, timeless body \\
15th & \plus11/\plus6/\plus1 & \plus17 & \plus7 & \plus12 & Natural aspect \\
16th & \plus12/\plus7/\plus2 & \plus18 & \plus8 & \plus13 & Commanding wild speech\\
17th & \plus12/\plus7/\plus2 & \plus19 & \plus8 & \plus13 & Natural aspect \\
18th & \plus13/\plus8/\plus3 & \plus20 & \plus9 & \plus14 & Totemic aspect \\
19th & \plus14/\plus9/\plus4 & \plus21 & \plus9 & \plus15 & Natural aspect \\
20th & \plus15/\plus10/\plus5 & \plus22 & \plus10 & \plus16 & Greater natural casting \\
\end{tabularx}
\end{dtable}

\cd{Alignment} Neutral good, lawful neutral, neutral, chaotic
neutral, or neutral evil.
\cd{Hit Value} 5.
 \sssecfake{Class Skills}
The druid's class skills (and the key attribute for each skill) are Athletics (Str), Climb (Str), Swim (Str), Ride (Dex), Stealth (Dex), Knowledge (geography), Knowledge (nature) (Int), Heal (Wis), Perception (Wis), Survival (Wis), and Creature Handling (Cha).
\cf{Drd}{Skill Points at 1st Level} 4

\sssecfake{Class Features}
All of the following are class features of the druid.

\cf{Drd}{Weapon and Armor Proficiency} Druids are proficient with simple weapons, any one other weapon group, scimitars, sickles, and slings.
\par Druids are proficient with light and medium armor, but are prohibited from wearing metal armor; thus, they may wear only padded, leather, or hide armor. (A druid may also wear wooden armor that has been altered by the \spell{ironwood} spell so that it functions as though it were steel. See the \spell{ironwood} spell description.) Druids are proficient with shields (except tower shields) but must use only wooden ones.
\par A druid who wears prohibited armor or carries a prohibited shield is unable to cast druid spells or use any of her supernatural class abilities while doing so and for 24 hours thereafter.

\cf{Drd}{Bonus Languages} A druid's bonus language options include Sylvan, the language of magical woodland creatures. This choice is in addition to the bonus languages available to the character because of her race.

A druid also knows Druidic, a secret language known only to druids, which she learns upon becoming a 1st-level druid. Druidic is a free language for a druid; that is, she knows it in addition to her regular allotment of languages and it doesn't take up a language slot. Druids are forbidden to teach this language to nondruids. Druidic has its own alphabet.

\cf{Drd}{Spells} A druid casts nature spells using her Wisdom. To learn or cast a spell, a druid must have a Wisdom at least equal to half the spell's level. A druid's magic attack bonus equals half her caster level \add her Wisdom.

Like other spellcasters, the number of spells a druid knows and can cast each day is limited. These limitations are given below on \trefnp{Druid Spells per Day} and \trefnp{Druid Spells Known}. A druid's spells are drawn from the nature spell list (see \pcref{Nature Spells}).

Druids meditate or pray for their spells. Each druid must choose a time at which she must spend 1 hour each day performing a ritual, worshipping, or quietly contemplating to regain her daily allotment of spells. They do not need to rest to regain spells.

\begin{dtable}
    \lcaption{Druid Spells per Day} 
    \centering
    \begin{tabularx}{\columnwidth}{>{\ccol}X *{9}{>{\ccol}p{\spellcol}}}
        & \multicolumn{9}{c}{\thead{---{}---{}---{}---{}---{}---{}---{}---Spell Level---{}---{}---{}---{}---{}---{}---{}---}} \\
        \thead{Level} & \thead{1st} & \thead{2nd} & \thead{3rd} & \thead{4th} & \thead{5th} & \thead{6th} & \thead{7th} & \thead{8th} & \thead{9th} \\
        1st & 3 & \x & \x & \x & \x & \x & \x & \x & \x \\
        2nd & 4 & \x & \x & \x & \x & \x & \x & \x & \x \\
        3rd & 5 & \x & \x & \x & \x & \x & \x & \x & \x \\
        4th & 6 & 3 & \x & \x & \x & \x & \x & \x & \x \\
        5th & 6 & 4 & \x & \x & \x & \x & \x & \x & \x \\
        6th & 6 & 5 & 3 & \x & \x & \x & \x & \x & \x \\
        7th & 6 & 6 & 4 & \x & \x & \x & \x & \x & \x \\
        8th & 6 & 6 & 5 & 3 & \x & \x & \x & \x & \x \\
        9th & 6 & 6 & 6 & 4 & \x & \x & \x & \x & \x \\
        10th & 6 & 6 & 6 & 5 & 3 & \x & \x & \x & \x \\
        11th & 6 & 6 & 6 & 6 & 4 & \x & \x & \x & \x \\
        12th & 6 & 6 & 6 & 6 & 5 & 3 & \x & \x & \x \\
        13th & 6 & 6 & 6 & 6 & 6 & 4 & \x & \x & \x \\
        14th & 6 & 6 & 6 & 6 & 6 & 5 & 3 & \x & \x \\
        15th & 6 & 6 & 6 & 6 & 6 & 6 & 4 & \x & \x \\
        16th & 6 & 6 & 6 & 6 & 6 & 6 & 5 & 3 & \x \\
        17th & 6 & 6 & 6 & 6 & 6 & 6 & 6 & 4 & \x \\
        18th & 6 & 6 & 6 & 6 & 6 & 6 & 6 & 5 & 3 \\
        19th & 6 & 6 & 6 & 6 & 6 & 6 & 6 & 6 & 4 \\
        20th & 6 & 6 & 6 & 6 & 6 & 6 & 6 & 6 & 6 \\
    \end{tabularx}
\end{dtable}

\begin{dtable}
    \lcaption{Druid Spells Known}
    \centering
    \begin{tabularx}{\columnwidth}{X *{9}{p{1.1em}}}
        & \multicolumn{9}{c}{\thead{---{}---{}---{}---{}---{}---{}---Spell Level---{}---{}---{}---{}---{}---{}---}} \\
        \thead{Level} & \thead{1st} & \thead{2nd} & \thead{3rd} & \thead{4th} & \thead{5th} & \thead{6th} & \thead{7th} & \thead{8th} & \thead{9th} \\
        1st  & 1 & \x & \x & \x & \x & \x & \x & \x & \x \\
        2nd  & 2 & \x & \x & \x & \x & \x & \x & \x & \x \\
        3rd  & 3 & \x & \x & \x & \x & \x & \x & \x & \x \\
        4th  & 3 & 1 & \x & \x & \x & \x & \x & \x & \x \\
        5th  & 4 & 2 & \x & \x & \x & \x & \x & \x & \x \\
        6th  & 4 & 2 & 1 & \x & \x & \x & \x & \x & \x \\
        7th  & 4 & 3 & 2 & \x & \x & \x & \x & \x & \x \\
        8th  & 4 & 3 & 2 & 1 & \x & \x & \x & \x & \x \\
        9th  & 4 & 3 & 3 & 2 & \x & \x & \x & \x & \x \\
        10th & 4 & 3 & 3 & 2 & 1 & \x & \x & \x & \x \\
        11th & 4 & 3 & 3 & 3 & 2 & \x & \x & \x & \x \\
        12th & 4 & 3 & 3 & 3 & 2 & 1 & \x & \x & \x \\
        13th & 4 & 3 & 3 & 3 & 3 & 2 & \x & \x & \x \\
        14th & 4 & 3 & 3 & 3 & 3 & 2 & 1 & \x & \x \\
        15th & 4 & 3 & 3 & 3 & 3 & 3 & 2 & \x & \x \\
        16th & 4 & 3 & 3 & 3 & 3 & 3 & 2 & 1 & \x \\
        17th & 4 & 3 & 3 & 3 & 3 & 3 & 2 & 2 & \x \\
        18th & 4 & 3 & 3 & 3 & 3 & 3 & 2 & 2 & 1 \\
        19th & 4 & 3 & 3 & 3 & 3 & 3 & 2 & 2 & 2 \\
        20th & 4 & 3 & 3 & 3 & 3 & 3 & 2 & 2 & 2
    \end{tabularx}
\end{dtable}

\cf{Drd}{Rituals} Druids, like other spellcasters, can perform rituals to create unique magical effects (see \pcref{Rituals}). A druid begins play with a ritual book containing one nature ritual of her choice (see \pcref{Nature Rituals}).

\cft{Drd}{Nature Sense}{Ex} A druid gains a \plus2 bonus on Knowledge (nature) and Survival checks. In addition, she can make those checks as if she were trained.

\cft{Drd}{Wild Speech}{Su} One of the first lessons a druid learns is how to commune with natural creatures. A druid can speak with animals a number of times per day equal to half her druid level \add her Charisma (minimum 1). Each time she uses this ability, she chooses a kind of animal, such as owl or wolf. She can then speak to and understand animals of that type for a number of minutes equal to her druid level.

This ability doesn't make the animals any more friendly or cooperative than normal. Furthermore, wary and cunning animals are likely to be terse and evasive, while the more stupid ones make inane comments. If an animal is friendly toward the druid, she may be able to convince it to do some favor or service.

\cflt{Drd}{2nd}{Woodland Stride}{Ex} The druid may move through any sort of undergrowth (such as natural thorns, briars, overgrown areas, and similar terrain) at her normal speed and without taking damage or suffering any other impairment. However, plants magically manipulated to impede motion still affect her.

\cflt{Drd}{3rd}{Wild Aspect}{Su} The druid gains the ability to embody an aspect of an animal. She can embody a wild aspect a number of times per day equal to half her druid level \add her Constitution (minimum 1), but she can only have one wild aspect active at once. If she attempts to embody a new wild aspect, her old wild aspect is dismissed.

Embodying a wild aspect is a standard action. Wild aspects last for 5 minutes, or until the druid dismisses them (a swift action).

At her 5th druid level, and every odd druid level thereafter, the druid gains an additional wild aspect. Some wild aspects have minimum druid levels, as indicated in the title of the aspect. The list of wild aspects is given below. All wild aspects are supernatural abilities unless otherwise noted.

The descriptions below describe the effects of the aspect. With many aspects, the druid's appearance also changes to match the aspect, but this is not described. Different druids change in different ways. For example, one druid might gain unusually large eyes when embodying the low-light vision aspect, while another might change her irises into slits, like a cat, when embodying the same aspect. The changes made are up to the druid, but cannot be used to gain an additional substantive benefit beyond the effects given in the description of the aspect.

\subcf{Animal Affinity} The druid gains a \plus2 enhancement bonus to Creature Handling and Ride checks. This bonus increases by \plus1 at 8th, 14th, and 20th druid level.
\subcf{Bear's Endurance} The druid gains a \plus2 enhancement bonus to Constitution.
\subcf{Bite} The druid's mouth transforms, allowing it to perform a bite attack.The attack deals 1d8 damage for a Medium druid. (See \pcref{Natural Weapons}, for details about natural weapons.)
\subcf{Bull's Strength} The druid gains a \plus2 enhancement bonus to Strength.
\subcf{Cat's Grace} The druid gains a \plus2 enhancement bonus to Dexterity.
\subcf{Claws} The druid's hands transform, allowing them to perform claw attacks. Each claw deals 1d6 damage for a Medium druid. (see \pcref{Natural Weapons}, for details about natural weapons).
\subcf{Constrict} The druid's body transforms, allowing her to perform a constrict attack. The attack deals 1d10 damage for a Medium druid, but it can only be used against a foe she is grappling with. (See \pcref{Natural Weapons}, for details about natural weapons). 
\subcf{Eagle's Splendor} The druid gains a \plus2 enhancement bonus to Charisma.
\subcf{Fox's Cunning} The druid gains a \plus2 enhancement bonus to Intelligence.
\subcf{Gore} The druid's head transforms, allowing it to perform a gore attack. The attack deals 1d8 damage for a Medium druid. (See \pcref{Natural Weapons}, for details about natural weapons.)
\subcf{Lope} The druid gains the ability to move on all four limbs. When doing so, she increases her speed by 20 feet, but she cannot use her hands for anything except moving. When not moving on four legs, her ability to use her hands is unchanged.
\subcf{Low-light Vision} The druid gains low-light vision. She can see twice as far as a human in starlight, moonlight, torchlight, and similar conditions of poor illumination. She retains the ability to distinguish color and detail under these conditions.
\subcf{Owl's Wisdom} The druid gains a \plus2 enhancement bonus to Wisdom.
\subcf{Shrink} The druid shrinks by a size category, halving her height, length, and width, and dividing her weight by 8. This gives her a \minus10 foot penalty to her speed, a \plus1 bonus to physical attacks and defenses, a \minus4 penalty to maneuver attack and defense, and a \plus4 bonus to Stealth. This is a size-affecting effect.
\subcf{Talons} The druid's feet transform, allowing them to perform a talon attack.
\subcf{5th -- Scent} The druid gains the scent ability, granting her a \plus10 bonus to scent-based Perception checks (see \pcref{Perception}).
\subcf{5th -- Glide} The druid grows wings, granting her a glide speed equal to her land speed. See \pcref{Gliding}, for more details.
\subcf{5th -- Grow} The druid grows by a size category, doubling her height, length, and width, and multiplying her weight by 8. This gives her a \plus10 foot bonus to her speed, a \minus1 penalty to physical attacks and defenses, a \plus4 bonus to maneuver attack and defense, and a \minus4 penalty to Stealth. This is a size-affecting effect.
\subcf{7th -- Natural Grab} If the druid hits with a natural attack, she may attempt to grapple her foe as an immediate action. She cannot provoke an attack of opportunity for the grapple attempt.
\subcf{7th -- Natural Knockback} If the druid hits with a natural attack, she may attempt to shove her foe as an immediate action. She cannot provoke an attack of opportunity for the shove attempt. She cannot move with the struck creature to push it back farther. 
\subcf{7th -- Natural Trip} If the druid hits with a natural attack, she may attempt to trip her foe as an immediate action. She cannot provoke an attack of opportunity for the trip attempt.
\subcf{7th -- Slither} The druid gains a climb speed equal to half her base land speed. She does not need to use her hands to climb in this way. See \pcref{Climbing}, for more details.
\subcf{9th -- Climb} The druid gains a climb speed equal to her base land speed. See \pcref{Climbing}, for more details.
\subcf{9th -- Limited Flight} The druid grows wings, granting her a fly speed equal to her land speed with average maneuverability. See \pcref{Flying}, for more details. She can only fly for a number of rounds equal to 3 \add half her Constitution. After that limit is reached, she must rest for 5 minutes.
\subcf{9th -- Venom} When the druid embodies this apsect, she transforms one of her natural weapons to become poisonous. If she hits with that natural attack, she may inject poison into the struck creature as an immediate action. At the end of each round, she makes a special attack vs. Fortitude. If the attack succeeds, the creature takes 2 points of Constitution damage. The poison lasts until the druid fails the attack twice.
\subcf{9th -- Wolfpack} Overwhelmed foes the druid threatens increase their overwhelm penalties by 1.

\cflt{Drd}{4th}{Venom Immunity}{Ex} The druid gains immunity to all poisons.

\cflt{Drd}{5th}{Natural Casting}{Ex} Whenever the druid casts a druidic area spell that would emanate from her, such as a cone or line spell, she may cause the spell to originate from any position within 10 feet of her. All other aspects of the spell are unchanged.

For example, a druid casting \spell{gust of wind} could create a line of wind originating from 10 feet to her right. The line would extend 50 feet out from that point, as normal. If the druid cause the line of wind to blow to the left, she could potentially be affected by the wind.

\cflt{Drd}{6th}{Wild Speech (Plants)}{Ex} The druid can also converse with plants and plant creatures using her wild speech ability. A regular plant's sense of its surroundings is limited, so it won't be able to give (or recognize) detailed descriptions of creatures or answer questions about events outside its immediate vicinity.

\cflt{Drd}{8th}{Commanding Wild Speech}{Su} As a standard action that consumes a use of her wild speech ability, the druid can attempt to charm a creature she is speaking with using her wild speech ability. If she succeeds at a special attack vs. Will, the target is charmed. The effect lasts for the duration of the conversation, and for 1 hour thereafter. This ability is not mind-affecting, and can affect creatures or even objects of any kind that the druid can use her wild speech to converse with. The attack automatically succeeds against non-intelligent objects. The paladin's special attack bonus with commanding wild speech is equal to her druid level \add her Charisma.

A charmed creature or object regards the druid as a trusted friend and ally. The druid cannot control the subject as if it were an automaton, but it perceives her words and actions in the most favorable way. She can try to give the subject orders, but she must succeed at a Persuasion check to convince it to do anything it wouldn't ordinarily do. (Retries are not allowed.) Treat the subject as a friend (a \plus10 relationship modifier) for the purpose of the Persuasion check. An affected creature never obeys suicidal or obviously harmful orders, but it might be convinced that something very dangerous is worth doing.

\cflt{Drd}{8th}{Multiple Wild Aspect}{Su} The druid can maintain two wild aspects at once. If she attempts to embody a new wild aspect while she already has her maximum number of wild aspects active, her oldest wild aspect is dismissed. At her 12th druid level, and every 4 druid levels thereafter, the druid may keep an additional wild aspect active simultaneously.

\cflt{Drd}{10th}{Improved Natural Casting}{Ex} The druid expands the range of her natural casting ability. She can cause area spells to originate from up to \rngclose range away from her.

\cflt{Drd}{11th}{Natural Aspect}{Su} The druid gains the ability to embody aspects of the natural world, including the elements, in addition to those of animals. She adds the options below to the list of abilities she can gain with her wild aspect ability.

%\subcf{Aerial Combat} The druid gains a \plus2 bonus to physical attacks while either she or her target is airborne.
%\subcf{Aquatic Combat} The druid gains a \plus2 bonus to physical attacks while both she and her target are touching naturally occuring water (including rain, but not fog or ice). 
%\subcf{Earthen Combat} The druid gains a \plus2 bonus to physical attacks with melee attacks while both she and her target are touching natural earth or stone.
%\subcf{Fiery Combat} The druid gains a \plus2 bonus to physical attacks while she or her target is ignited.
%\subcf{Earthen Power} The druid gains a \plus2 bonus to caster level while she is touching natural earth or stone.
\subcf{Heart of Air} The druid can breathe in any environment, and is immune to \spell{sickening cloud} and similar effects. In addition, she falls at half speed and takes no falling damage.
\subcf{Heart of Earth} The druid gains the effects of the \spell{stoneskin} spell, with a caster level equal to her druid level \add her Constitution.
\subcf{Heart of Fire} The druid gains the effects of the warm version of the \spell{fire shield} spell, with a caster level equal to her druid level \add her Constitution.
\subcf{Heart of the Sun} The druid constantly radiates bright light out to a 100 foot radius (and shadowy illumination for an additional 100 feet). The illumination is so bright that she becomes hard to look at. Any creature attacking her from within the radius of bright light becomes dazzled for 5 rounds after the attack.
\subcf{Heart of Oak} The druid gains the effects of the \spell{stoneskin} spell, with a caster level equal to her druid level \add her Constitution, except that the damage reduction is overcome by fire instead of adamantine weapons.
\subcf{Heart of Water} The druid gains the effects of the \spell{freedom} spell.
\subcf{15th -- Air Mantle}  The druid is surrounded by a mantle of air. Thrown and projectile weapons have a 50\% chance to miss her while this effect is active. Unusually large weapons, such as a giant's boulders, may suffer a decreased miss chance as appropriate to their size.
\subcf{15th -- Aqueous Step}  Wherever the druid moves, she leaves a path of animated water that can grab creatures. Whenever a creature crosses the path, the druid makes a Reflex attack to trip the creature, causing it to fall prone and waste the rest of its movement. Her attack bonus is equal to her druid level \add her Constitution.
\subcf{15th -- Flaming Step} Wherever the druid moves, she leaves a path of burning flame behind her that lasts for 1 round. Whenever a creature crosses the path, the druid makes a Reflex attack to deal damage to the creature. The attack deals 1d6 points of fire damage per two druid levels. Her attack bonus is equal to her druid level \add her Constitution. A failed attack deals half damage.
\subcf{15th -- Lifegiving Step} Wherever the druid moves, she leaves a path of small, living plants that entangle foes for 1 round.  Whenever a creature crosses the path, the druid makes a Reflex attack to entangle the creature, causing it to waste the rest of its movement. Her attack bonus is equal to her druid level \add her Constitution. The plants appear on any surface, and will continue to grow if they can survive, though they may die quickly if they appear on inhospitable terrain.
\subcf{17th -- Flight} The druid gains a fly speed equal to her land speed, with good maneuverability. She may remain flying for up to 5 rounds at a time. After that, she must land for 1 round before she can fly again. See \pcref{Flying}, for more details.
\subcf{17th -- Flaming Soul} The druid gains the fire subtype, making her immune to fire but giving her a 50\% vulnerability to cold damage. In addition, whenever she deals fire damage to a creature, the creature is \ignited for 5 rounds.
\subcf{17th -- Sunblessed Rejuvenation} The druid gains fast healing equal to her druid level as long as she remains in sunlight or touches a plant of her size or larger.
\subcf{17th -- Sunscour} This aspect functions like the heart of the sun natural aspect, except that it also suppresses shadow effects and the visual components of illusions within the area of bright light. 
\subcf{17th -- Water's Flow} As a swift action, the druid can transform herself into a rushing flow of water with a volume roughly equal to her normal volume until the end of her turn. In this form, she may move wherever water could go, but she cannot take other actions, such as jumping, attacking, or casting spells. Her speed is halved when moving uphill and doubled when moving downhill. She does not provoke attacks of opportunity for moving, and has physical damage reduction equal to her druid level. She may return to her normal form as a free action.
\par If the water is split, she may reform from anywhere the water has reached, to as little as a single ounce of water. If not even an ounce of water exists contiguously, her body reforms from the largest available parts of water, cut into pieces of appropriate size. This usually causes the druid to die.

\cflt{Drd}{12th}{A Thousand Faces}{Su} The druid gains the ability to change her appearance at will, as if using the \spell{disguise self} spell. This affects the druid's body but not her possessions. It is not an illusory effect, but a minor physical alteration of the druid's appearance, within the limits described for the spell.

\cflt{Drd}{14th}{Elemental Speech}{Su} The druid gains the ability to speak with one of the elements that make up the natural world with her wild speech ability. When she gains this ability, she chooses whether she can speak with natural air, earth, fire, or water. That choice cannot thereafter be changed.

\cflt{Drd}{14th}{Timeless Body}{Ex} The druid no longer takes attribute penalties for aging and cannot be magically aged. Any penalties she may have already incurred, however, remain in place. Bonuses still accrue, and the druid still dies of old age when her time is up.

\cflt{Drd}{16th}{Dominating Wild Speech}{Ex} When the druid uses her commanding wild speech ability, she can dominate the subject (as the \spell{dominate person} spell) instead of charming it. This ability is not mind-affecting, and can affect creatures of any kind that the druid can converse with. The attack automatically succeeds against non-intelligent objects.

\cflt{Drd}{18th}{Totemic Aspect}{Su} The druid can choose any one wild aspect (but not natural aspect). She permanently gains the abilities of that aspect, as if she was constantly manifesting it. She may suppress or resume this effect as a swift action.

\cflt{Drd}{20th}{Greater Natural Casting}{Ex} The druid may cause area spells to originate from any point within \rngmed range of her, as the natural casting ability.

\subsubsection{Ex-Druids}
A druid who ceases to revere nature, changes to a prohibited alignment, or teaches the Druidic language to a nondruid loses all spells and supernatural druid class features. She cannot thereafter gain levels as a druid until she atones (see the \spell{atonement} ritual).

\subsection{Fighter}
\begin{dtable}
\lcaption{The Fighter}
\begin{tabularx}{\columnwidth}{>{\ccol}p{\levelcol} >{\ccol}p{\babcolgood} *{3}{>{\ccol}p{\savecol}} >{\lcol}X}
\thead{Level} & \thead{Base Attack Bonus} & \thead{Fort} & \thead{Ref} & \thead{Will} & \thead{Special} \\
1st & \plus1                         & \plus3 & \plus0 & \plus1 & Armor discipline \\
2nd & \plus2                         & \plus4 & \plus1 & \plus2 & Bonus feat \\
3rd & \plus3                         & \plus5 & \plus1 & \plus3 & Weapon discipline \\
4th & \plus4                         & \plus6 & \plus2 & \plus4 & Adaptive style feat \\
5th & \plus5                         & \plus7 & \plus2 & \plus4 & Combat discipline \\
6th & \plus6/\plus1                  & \plus8 & \plus3 & \plus5 & Bonus feat \\
7th & \plus7/\plus2                  & \plus9 & \plus3 & \plus6 & Improved armor discipline \\
8th & \plus8/\plus3                  & \plus10& \plus4 & \plus7 & Adaptive style feat \\
9th & \plus9/\plus4                  & \plus11& \plus4 & \plus7 & Improved weapon discipline\\
10th & \plus10/\plus5                & \plus12& \plus5 & \plus8 & Battlemaster, bonus feat\\
11th & \plus11/\plus6/\plus1         & \plus13 & \plus5 & \plus9 & Improved combat discipline\\
12th & \plus12/\plus7/\plus2         & \plus14 & \plus6 & \plus10& Adaptive style feat \\
13th & \plus13/\plus8/\plus3         & \plus15 & \plus6 & \plus10& Greater armor discipline \\
14th & \plus14/\plus9/\plus4         & \plus16 & \plus7 & \plus11& Bonus feat, improved adaptive style \\
15th & \plus15/\plus10/\plus5        & \plus17 & \plus7 & \plus12& Greater weapon discipline\\
16th & \plus16/\plus11/\plus6/\plus1 & \plus18 & \plus8 & \plus13& Adaptive style feat \\
17th & \plus17/\plus12/\plus7/\plus2 & \plus19 & \plus8 & \plus13& Greater combat discipline\\
18th & \plus18/\plus13/\plus8/\plus3 & \plus20 & \plus9 & \plus14& Bonus feat, improved battlemaster\\
19th & \plus19/\plus14/\plus9/\plus4 & \plus21 & \plus9 & \plus15& True discipline \\
20th & \plus20/\plus15/\plus10/\plus5& \plus22 & \plus10 & \plus16 & Adaptive style feat, greater adaptive style
\end{tabularx}
\end{dtable}

\cd{Alignment} Any.

\cd{Hit Value} 6.

\sssecfake{Class Skills}
The fighter's class skills (and the key attribute for each skill) are Athletics (Str), Climb (Str), Swim (Str), Ride (Dex), and Intimidate (Cha).
\cf{Ftr}{Skill Points at 1st Level} 2.

\sssecfake{Class Features}
All of the following are class features of the fighter.
 \cf{Ftr}{Weapon and Armor Proficiency}  A fighter is proficient with simple weapons, any four other weapon groups,  all armor (heavy, medium, and light) and shields (including tower shields).

\cf{Ftr}{Armor Discipline} A fighter's training grants him additional capability in armor. He must choose to improve his agility or his resilience in armor. This applies to all armor discipline abilities the fighter has. If he improves his agility, he reduces his armor check penalty by 2 and reduces his arcane spell failure by 5\% while wearing body armor. If he improves his resilience, he gains nonlethal damage reduction equal to his fighter level while wearing body armor. This damage reduction allows him to ignore the first points of nonlethal damage he takes each round.

\cfl{Ftr}{2nd}{Bonus Feat} The fighter gets a bonus combat-oriented feat. This bonus feat must be drawn from the feats noted as combat feats on \tref{Combat Feats}. A fighter must still meet all prerequisites for a bonus feat, including attribute score and base attack bonus minimums. The fighter gains an additional bonus feat at his 6th fighter level and every four fighter levels thereafter (6th, 10th, 14th, and 18th).

\cfl{Ftr}{3rd}{Weapon Discipline} The fighter's training grants him additional capability when using his weapons. He may choose a weapon group, or he may choose to train equally with all weapons. If he chooses a weapon group, he gains a \plus1 bonus to physical attacks with weapons from that group.
\par If he chooses not to focus on a specific group of weapons, he gains the ability to become proficient with any weapon group if he spends 8 hours training with a weapon from that group. He may only keep this proficiency with one weapon group at a time; if he trains with a new weapon group, he loses his proficiency in the previous group.

\cfl{Ftr}{4th}{Adaptive Style Feats} The fighter gains a flexible bonus feat which he can change each day. The fighter chooses a number of combat feats equal to half his fighter level \add his Intelligence. These feats comprise his adaptive style feat pool. He gains any one feat from the pool as a bonus feat (for which he must still meet the normal prerequisites). At the start of each day, the fighter may train for an hour. If he does so, he may choose to change his current adaptive style feat to one of the other feats in his adaptive style feat pool, assuming he meets the prerequisites for the feat.  The fighter gains an additional adaptive style feat at his 8th fighter level and every four fighter levels thereafter (8th, 12th, 16th, and 20th).

\par An adaptive style feat may be used normally as prerequisites for other feats or abilities.  However, if an adaptive style feat is used as a prerequisite, it cannot be changed until the fighter no longer needs to use it as a prerequisite, such as might happen if the fighter takes the feat as a normal feat or bonus feat.

\par In order to gain an adaptive style feat, it must be reasonably possible to do training related to the new feat. For example, a fighter could not gain Weapon Focus in axes without at least one axe available to train with.

\cfl{Ftr}{5th}{Combat Discipline} The fighter can use his superior training and focus to keep fighting in the face of debilitating effects. When a fighter is initially affected by one of the conditions listed on \trefnp{Combat Discipline Conditions}, he may use his combat discipline ability to instead suffer the mitigated condition one column to the right. This lasts until the end of the original condition or for one round per two fighter levels, at which point he suffers the effects of the original condition unless he uses his combat discipline ability again.
\par Using combat discipline takes no action, and can be done at any time, even when it isn't the fighter's turn. A fighter may use this ability a number of times per day equal to 3 \add his Charisma. However, he cannot mitigate more than one condition at a time. If the fighter attempts to mitigate a new condition, the old condition takes its full effect.

\begin{dtable}
\lcaption{Combat Discipline Conditions}
\begin{tabularx}{\columnwidth}{*{4}{>{\lcol}X}}
\thead{Original Condition} & \thead{Mitigated Condition} & \thead{Mitigated Condition} & \thead{Mitigated Condition} \\
Panicked & Frightened & Shaken & None  \\
Petrified  & Paralyzed & Slowed & None \\
Stunned       & Dazed & Staggered & None \\
Blinded & Dazzled & None  & \x \\
Confused & Bewildered & None & \x \\
Exhausted & Fatigued & None & \x \\
Nauseated & Sickened & None & \x \\
Ability damage\fn{1} & None & \x & \x \\
Ability penalty\fn{1} & None & \x & \x \\
Entangled & None & \x & \x \\
Deafened & None & \x & \x \\
Fascinated & None & \x & \x \\
Ignited\fn{2} & None & \x & \x \\
Immobilized & None & \x & \x \\
Negative level\fn{3} & None & \x & \x \\
Vulnerable & None & \x & \x \\
\end{tabularx}
1. Allows the fighter to mitigate up to half his fighter level in ability damage or penalties per use of combat discipline. \\
2. Mitigates the penalties, but does not prevent the fighter from taking d6 fire damage per round until the fire is put out. \\
3. Allows the fighter to ignore a single negative level per use of combat discipline.
\end{dtable}

\par A fighter cannot use this ability more than once against a single source. For example, if a fighter is exhausted by a \spell{ray of exhaustion} spell, he can use this ability to downgrade the exhaustion to fatigue, but he can't then expend a second use to negate the fatigue. The lesser condition that this ability imposes may be cured or removed normally, but doing so does not affect the resurgence of the condition the fighter was originally afflicted with. If a fighter uses this ability to mitigate or negate a condition which he must suffer as a sacrifice or cost to gain some benefit, he automatically forfeits the benefit he would have gained.

\cfl{Ftr}{7th}{Improved Armor Discipline} The fighter's training with his armor improves. If he chose agility, he reduces his armor check penalty by 4 and decreases his arcane spell failure by 15\%. This does not stack with the effects of armor discipline. In addition, he treats all body armor as if it were one encumbrance category lighter than it is.
\par This ability means heavy armor is treated as medium armor, medium armor is treated as light armor, and light armor is treated as being unarmored. This can remove the halving of the fighter's Dexterity bonus, if appropriate for the new encumbrance of the fighter's armor. This allows the fighter to qualify for class features using the reduced armor encumbrance category. For example, a fighter 9 / wizard 2 who reduces his encumbrance in light armor could cast spells without any arcane spell failure in light armor.

If the fighter chose resilience, his damage reduction from armor discipline applies against all physical damage.

\cfl{Ftr}{9th}{Improved Weapon Discipline} The fighter's training in his chosen weapons improves. He gains a \plus4 bonus to resist disarm and sunder attempts when using his chosen weapons. If he chose a specific weapon group, he gains a \plus2 bonus to physical attacks with weapons from that group. If he did not, he can apply all weapon group-specific feats he has to any weapon group that he trains with for 8 hours. He retains this benefit for one week after the training.

\cfl{Ftr}{10th}{Battlemaster} The fighter can improve his allies' combat abilities. As a standard action, he may grant the use of one of his combat feats to allies within \rngclose range of him who can see and hear him. He can affect a number of allies equal to 1 \add his Intelligence (minimum 1). Affected allies must meet all prerequisites for the granted feat, except that they can ignore any feat prerequisites. The effect lasts for 5 rounds. The fighter can use this ability a number of times per day equal to 3 \add his Charisma.

\cfl{Ftr}{11th}{Improved Combat Discipline} The fighter's ability to keep fighting despite negative influence improves. He can reduce conditions by two steps instead of one when using combat discipline. For example, a stunned fighter who used combat discipline would instead be staggered.
\par In addition, a fighter may use combat discipline to reduce any penalties he suffers to his attacks, damage, checks, or defenses by 2, even if the source of the penalty is not listed on the combat discipline chart.
\par The fighter may also mitigate up to two conditions at once.

\cfl{Ftr}{13th}{Greater Armor Discipline} The fighter's training in his chosen armor becomes still greater. If he chose agility, he reduces his armor check penalty by 6 and decreases his arcane spell failure by 30\% while wearing armor of any kind. In addition, he treats all armor as if it were two encumbrance categories lighter than it actually is whenever doing so would be beneficial to him. This does not stack with the benefits of armor discipline or improved armor discipline.

If the fighter chose resilience, he adds half his Constutition to his Armor defense while wearing armor of any kind. He loses this bonus when he is helpless or unarmored.

\cfl{Ftr}{14th}{Improved Adaptive Style} The fighter's ability to adapt to situations improves. He need only spend 1 minute training to change a single adaptive style feat. He may continue training as he wishes, changing one adaptive style feat per minute.

\cfl{Ftr}{15th}{Greater Weapon Discipline} The fighter's training in his chosen weapons becomes still greater. He increases the critical threat range and critical multiplier of his chosen weapons by 1. This increase applies after and stacks with any other effects that affect critical threat range or critical multiplier. Thus, a fighter using the Heartseeker combat style and wielding a longsword would have a critical threat range of 16-20 (x3), while a similar fighter would have a critical threat range of 18-20 (x5) with a heavy pick.

\cfl{Ftr}{17th}{Greater Combat Discipline} The fighter's ability to keep fighting despite influence becomes still greater. When using combat discipline, he may ignore any condition listed on the combat discipline chart. In addition, he may use combat discipline to be \dazed rather than suffer any non-damaging condition not listed on the chart.

\par The fighter may also mitigate up to three conditions at once.

\cfl{Ftr}{18th}{Improved Battlemaster} The fighter can improve his allies' combat abilities more effectively. When using his battlemaster ability, he can grant two feats at once. In addition, he can use his battlemaster ability as a swift action.

\cfl{Ftr}{19th}{True Discipline} The fighter's discipline in his chosen area is beyond equal. He must choose either weapon discipline, armor discipline, or combat discipline. Depending on which discipline he chooses, he gains a different bonus.
\subcf{True Weapon Discipline} The fighter can take 10 on the first attack he makes each round and automatically confirms all critical threats while using his chosen weapons.
\subcf{True Armor Discipline} If the fighter chose agility, he no longer suffers armor check penalties or arcane spell failure with any armor. He ignores the encumbrance of all armor, causing him to be treated as unarmored whenever doing so is beneficial to him. In addition, he applies the defense bonus from any body armor he wears to his Reflex defense.

If the fighter chose resilience, he applies his damage reduction to all damage, including from magical attacks. In addition, he applies the defense bonus from any body armor he wears to his Fortitude defense.
\subcf{True Combat Discipline} The fighter can use combat discipline to be staggered instead of suffering any nondamaging negative effect with a duration. He may also mitigate up to four conditions at once.

\cfl{Ftr}{20th}{Greater Adaptive Style} The fighter's ability to react to situations is unparalleled. He need only spend a swift action to exchange an adaptive style feat.

\subsection{Monk}
\begin{dtable}
\lcaption{The Monk}
\begin{tabularx}{\columnwidth}{>{\ccol}p{\levelcol} >{\ccol}p{\babcolavg} *{3}{>{\ccol}p{\savecol}} >{\lcol}X}
\thead{Level} & \thead{Base Attack Bonus} & \thead{Fort} & \thead{Ref} & \thead{Will} & \thead{Special} \\
1st & \plus1                    & \plus1 & \plus3 & \plus3    & Unarmed warrior, unfettered defense \\
2nd & \plus2                    & \plus2 & \plus4 & \plus4    & Fast movement, \Ki power \\
3rd & \plus3                    & \plus3 & \plus5 & \plus5    & Evasion, uncanny dodge \\
4th & \plus4                    & \plus4 & \plus6 & \plus6    & \Ki strike, Perfect Health\\
5th & \plus5                    & \plus4 & \plus7 & \plus7    & \Ki power, still mind \\
6th & \plus6/\plus1                    & \plus5 & \plus8 & \plus8    & Improved uncanny dodge \\
7th & \plus7/\plus2                    & \plus6 & \plus9 & \plus9    & Bodily perfection \\
8th & \plus8/\plus3             & \plus7 & \plus10& \plus10   & \Ki power, perfect speech \\
9th & \plus9/\plus4             & \plus7 & \plus11& \plus11   & Improved evasion \\
10th & \plus10/\plus5            & \plus8 & \plus12& \plus12   & Greater uncanny dodge \\
11th & \plus11/\plus6/\plus1            & \plus9  & \plus13 & \plus13 & \Ki power \\
12th & \plus12/\plus7/\plus2            & \plus10 & \plus14& \plus14 & Perfect soul \\
13th & \plus13/\plus8/\plus3            & \plus10 & \plus15 & \plus15 & Improved bodily perfection \\
14th & \plus14/\plus9/\plus4            & \plus11 & \plus16& \plus16 & \Ki power \\
15th & \plus15/\plus10/\plus5    & \plus12 & \plus17 & \plus17 & Timeless \\
16th & \plus16/\plus11/\plus6/\plus1    & \plus13 & \plus18 & \plus18 & Perfect mind \\
17th & \plus17/\plus12/\plus7/\plus2    & \plus13 & \plus19 & \plus19 & \Ki power\\
18th & \plus18/\plus13/\plus8/\plus3    & \plus14 & \plus20 & \plus20 & Uncanny foresight \\
19th & \plus19/\plus14/\plus9/\plus4    & \plus15 & \plus21 & \plus21 & Greater bodily perfection \\
20th & \plus20/\plus15/\plus10/\plus5   & \plus16 & \plus22 & \plus22 & True perfection \\
\end{tabularx}
\end{dtable}

\cd{Alignment} Any nonchaotic.

\cd{Hit Value} 6

\sssecfake{Class Skills}
The monk's class skills (and the key attribute for each skill) are Athletics (Str), Climb (Str), Swim (Str), Acrobatics (Dex), Escape Artist (Dex), Stealth (Dex), Heal (Wis), Perception (Wis), Survival (Wis), and Perform (Cha).
\cf{Mnk}{Skill Points at 1st Level} 4

\sssecfake{Class Features}
All of the following are class features of the monk.
\cf{Mnk}{Weapon and Armor Proficiency}
Monks are proficient with simple weapons, monk weapons, and any one other weapon group. Monks are not proficient with any armor or shields. When wearing armor, using a shield, or carrying a medium or heavy load, a monk loses the benefit of her unfettered defense, fast movement, and \ki abilities.

\cft{Mnk}{Unfettered Defense}{Ex} A monk knows how to react intuitively to avoid blows. When unarmored and unencumbered, the monk may add her Wisdom to her physical defenses. She loses this bonus when she is helpless.

\begin{comment}  %Made AC too high
 \cfnl{\Ki Ward (Ex)}\label{Mnk:Ki Ward (Ex)} When unarmored and unencumbered, a monk gains a \plus1 armor bonus to AC at 2nd level. This bonus increases by 1 for every two monk levels thereafter (\plus2 at 4th, \plus3 at 6th, etc.).

\par The monk loses this bonus when she is
immobilized or helpless, when she wears any armor, when she carries a shield, or when she carries a medium or heavy load.
\end{comment}

\cf{Mnk}{Improved Unarmed Strike} A monk gains Improved Unarmed Strike as a bonus feat. She is treated as armed even while unarmed, and her unarmed strikes can deal lethal damage if she desires.

\cft{Mnk}{Unarmed Warrior}{Ex} A monk's unarmed attacks are exceptionally deadly. She deals damage with her unarmed strikes as if she were two size categories larger (1d6 for a Medium creature, or 1d4 for a small creature).

A monk's attacks may be with either fist interchangeably or even from elbows, knees, and feet. This means that a monk may even make unarmed strikes with her hands full. A monk's unarmed strike can be treated as a manufactured weapon, natural weapon, or both for the purpose of spells and effects, whichever is more beneficial for the monk. This allows monks to make unarmed strikes as if they were fighting with two weapons at once (see \pcref{Two-Weapon Fighting}). Monks can also use gauntlets, including enchanted gauntlets. The damage dealt by gauntlets is the same as the damage dealt by the monk's normal unarmed strike.

\cflt{Mnk}{2nd}{Fast Movement}{Ex} The monk increases her land speed by 10 feet while unencumbered.

\cflt{Mnk}{2nd}{Ki Power}{Su} The monk gains the ability to channel her \ki energy to temporarily enhance her abilities. She may use any combination of \ki powers she knows a number of times per day equal to half her monk level \add her Wisdom (minimum 1), but no more than once per round.

The monk chooses one \ki power from the list below. At her 5th monk level, and every three monk levels thereafter, the monk gains an additional \ki power. Some \ki powers have minimum monk levels, as indicated in the title of the power. All \ki powers are supernatural abilities unless otherwise noted. The monk's special attack bonus with \ki powers is equal to her monk level \add her Wisdom.

\subcf{Flurry of Blows} As part of a standard attack, the monk can make an additional attack at a \minus5 penalty.
\subcf{Stunning Fist} As part of an unarmed attack, the monk can strike a weak point to interfere with her foe's \ki. If the attack deals damage, the monk makes a special attack vs. Fortitude to make the struck creature \staggered for 1 round. If the creature is bloodied, it is \stunned for 1 round instead.
\subcf{Surpass Limits} As a swift action, the monk can surpass the physical limitations of her body. Until the start of her next turn, she adds her Wisdom to checks based on Strength and Dexterity. This does not affect checks based on multiple attributes, such as initiative checks.
\subcf{5th -- Speed Boost} As a swift action, the monk can force her muscles to strain beyond their normal limits. Until the end of her turn, she gains increases her land speed by 20 feet, and a gains \plus2 bonus to her dodge defense modifier.
\subcf{5th -- Wholeness of Body} As a standard action, the monk can correct the flow of energy within her body. She heals 1d6 points of damage per two monk levels.
\subcf{8th -- Rapid Step} As a move action, the monk can move up to her speed without provoking attacks of opportunity.
\subcf{8th -- Slow Fall} As a swift action while falling, the monk can slow the rate at which she falls to 60 feet per round, which is too slow to hurt her if she hits the ground. She must be touching a solid object to use this ability. The effect lasts until the end of her turn.
\subcf{11th -- Diamond Fists} As a swift action, the monk can empower her unarmed strike with incredible force. Until the start of her next turn, she adds half her Wisdom to damage and treats her unarmed strike as if it were an adamantine weapon for the purpose of overcoming damage reduction and hardness.
\subcf{11th -- Flash Step} As a move action, the monk can slip between spaces, allowing her to teleport to anywhere she can see within 30 feet. If her line of effect is blocked, even by an invisible barrier, or if this would somehow place her inside a solid object, the ability fails. 
\subcf{14th -- Empty Step} As a swift action, the monk can step into the Ethereal Plane until the end of her turn, as the \spell{ethereal jaunt} spell. She may return as a free action.
\subcf{14th -- Quivering Palm} As a standard action, the monk can make a single unarmed strike. If the attack deals damage, the struck creature is \sickened by the disruption of the \ki within in its body for 5 rounds. At any point during that time, the monk can will the struck target to die (a free action). If she does, and the creature is bloodied, she makes a special attack vs. Fortitude. If the attack succeeds, the creature loses all its hit points and takes 9 critical damage, rendering it unconscious and dying (see \pcref{Dying}).
\subcf{17th -- Moment of Perfection} The monk can align herself with the universe to achieve a single moment of perfection. As a swift action, she can add her monk level as an enhancement bonus to any single physical attack or opposed check. Alternately, when she is physically attacked, she can use an immediate action to add her monk level to her physical defenses against the attack. After using this ability, she must wait five minutes before she can use it again.
\subcf{17th -- Empty Body} As a move action, the monk can step into the Ethereal Plane for 5 rounds, as the \spell{ethereal jaunt} spell. For the duration of the effect, she may switch between the planes as a move action.

\cflt{Mnk}{3rd}{Evasion}{Ex} If the monk resists a Reflex attack that normally deals half damage when resisted, she instead takes no damage. Evasion can be used only if a monk is unencumbered. A helpless monk does not gain the benefit of evasion.

\cflt{Mnk}{3rd}{Uncanny Dodge}{Ex} The monk can react to danger before her senses would normally allow her to do so. She is not helpess when unaware of an attack.

If a monk already has uncanny dodge from a different class, she stacks those class levels to determine whether she gains improved uncanny dodge (see below) instead.

\cflt{Mnk}{4th}{Ki Strike}{Su} The monk's attacks are empowered with \ki. While unarmored and unencumbered, she treats all weapons she wields, including her unarmed strike, as if they were \plus1 weapons, gaining a \plus1 enhancement bonus to attack and damage. This also grants her a \plus1 enhancement bonus with maneuvers. At her 7th monk level, and every three monk levels thereafter, the bonus granted by the monk's \ki strike ability improves by 1. 

\cfl{Mnk}{4th}{Perfect Health} The monk gains the Perfect Health feat as a bonus feat, making her immune to diseases of all kinds. If she already has the feat, she gains another feat of her choice for which she qualifies.

\cflt{Mnk}{5th}{Still Mind}{Ex} The monk may add half her Wisdom to her Will defense in place of half her Intelligence.

\cflt{Mnk}{6th}{Improved Uncanny Dodge}{Ex} The monk can no longer be overwhelmed as easily; she can react to multiple opponents as easily as she can react to a single attacker. The monk is always treated as being threatened by two fewer creatures than she actually is for the purpose of determining overwhelm penalties.
\par If a character already has uncanny dodge (see above) from a second class and gains improved uncanny dodge, the character stacks those class levels to determine if she should gain greater uncanny dodge.

\cflt{Mnk}{7th}{Bodily Perfection}{Ex} The monk gains a \plus1 bonus to her Strength, Dexterity, and Constitution.

\cflt{Mnk}{8th}{Perfect Speech}{Ex} The monk gains the ability to speak with and understand the speech of any living creature. This grants her no special ability to speak to or understand creatures that do not speak, such as animals.

\cflt{Mnk}{9th}{Improved Evasion}{Ex} The monk's evasion ability improves. If she is affected by Reflex attack that would deal half damage when resisted, she takes only half damage from the attack, even if the attack succeeds. Like evasion, improved evasion can be used only if a monk is unencumbered. A helpless monk does not gain the benefit of improved evasion.

\cflt{Mnk}{10th}{Greater Uncanny Dodge}{Ex} The monk can no longer be overwhelmed, regardless of the number of foes surrounding her.

\cflt{Mnk}{12th}{Perfect Soul}{Ex} The monk gains spell resistance. To affect a creature with spell resistance using a spell or spell-like ability, a special attack is always required. The defense used against the attack is indicated by the spell. If the attack fails, the spell has no effect on the monk.

\cfl{Mnk}{13th}{Improved Bodily Perfection} The bonus granted by the monk's bodily perfection ability increases to \plus2.

\cflt{Mnk}{15th}{Timeless}{Ex} The monk no longer takes penalties to her attribute scores for aging, and cannot be magically aged. She also gains the benefits of being middle-aged if she did not already possess them, granting her a \plus1 bonus to her Intelligence, Wisdom, and Charisma. Any aging penalties she has are removed. The monk still dies of old age when her time is up.

\cflt{Mnk}{16th}{Perfect Mind}{Ex} The monk becomes immune to hostile mind-affecting effects.

%weak after changes?
\cflt{Mnk}{18th}{Uncanny Foresight}{Su} The monk gains the ability to react to situations without premeditation or thought. She is never surprised or unaware, and always acts in the first round of combat.

\cfl{Mnk}{19th}{Greater Bodily Perfection} The bonus granted by the monk's bodily perfection ability increases to \plus3.

\cfl{Mnk}{20th}{True Perfection} The monk becomes inhumanly perfect. If she rolls less than a 5 on any d20 roll, it is treated as a 5. In addition, she is treated as an outsider rather than as a humanoid for the purpose of spells and magical effects whenever doing so is advantageous to her.

\subsubsection{Ex-Monks}
A monk who becomes chaotic cannot gain new levels as a monk, but retains all monk abilities.

\subsection{Paladin}
\begin{dtable*}
\lcaption{The Paladin}
\begin{tabularx}{\textwidth}{>{\ccol}p{\levelcol} >{\ccol}p{\babcolgood} *{3}{>{\ccol}p{\savecolpoof}} X}
\thead{Level} & \thead{Base Attack Bonus} & \thead{Fort} & \thead{Ref} & \thead{Will} & \thead{Special} \\
1st  & \plus1                        & \plus3  & \plus0 & \plus3 & Discernment, smite \\
2nd  & \plus2                        & \plus4  & \plus1 & \plus4 & Smite power, lay on hands \\
3rd  & \plus3                        & \plus5  & \plus1 & \plus5 & Divine gift \\
4th  & \plus4                        & \plus6  & \plus2 & \plus6 & Discern foe \\
5th  & \plus5                        & \plus7  & \plus2 & \plus7 & Smite power \\
6th  & \plus6/\plus1                 & \plus8  & \plus3 & \plus8 & Divine gift \\
7th  & \plus7/\plus2                 & \plus9  & \plus3 & \plus9 & Pass judgment \\
8th  & \plus8/\plus3                 & \plus10 & \plus4 & \plus10& Smite power \\
9th  & \plus9/\plus4                 & \plus11 & \plus4 & \plus11& Divine gift \\
10th & \plus10/\plus5                & \plus12 & \plus5 & \plus12& Certain smite \\
11th & \plus11/\plus6/\plus1         & \plus13 & \plus5 & \plus13& Smite power \\
12th & \plus12/\plus7/\plus2         & \plus14 & \plus6 & \plus14& Divine gift \\
13th & \plus13/\plus8/\plus3         & \plus15 & \plus6 & \plus15& \\
14th & \plus14/\plus9/\plus4         & \plus16 & \plus7 & \plus16& Smite power \\
15th & \plus15/\plus10/\plus5        & \plus17 & \plus7 & \plus17& Divine gift \\
16th & \plus16/\plus11/\plus6/\plus1 & \plus18 & \plus8 & \plus18&  \\
17th & \plus17/\plus12/\plus7/\plus2 & \plus19 & \plus8 & \plus19& Smite power \\
18th & \plus18/\plus13/\plus8/\plus3 & \plus20 & \plus9 & \plus20& Divine gift \\
19th & \plus19/\plus14/\plus9/\plus4 & \plus21 & \plus9 & \plus21& Martyr's glorious retribution \\
20th & \plus20/\plus15/\plus10/\plus5& \plus22 &\plus10 &\plus22 & Greater smite, smite power \\
\end{tabularx}
\end{dtable*}

\cd{Alignment} Any other than true neutral.
\cd{Hit Value} 6

\sssecfake{Class Skills}
The paladin's class skills (and the key attribute for each skill) are Ride (Dex), Knowledge (local) (Int), Knowledge (religion) (Int), Heal (Wis), Sense Motive (Wis), Intimidate (Cha), and Persuasion (Cha).
\cf{Pal}{Skill Points at 1st Level} 2.

\sssecfake{Class Features}
All of the following are class features of the paladin.
 \cf{Pal}{Weapon and Armor Proficiency}  Paladins are proficient with simple weapons,  any two other weapon groups,  all types of armor (heavy, medium, and light), and with  shields (including tower shields). A paladin is also proficient with the favored weapon group of her deity. If she does not follow a deity, she is proficient with any other weapon group of her choice.

\cft{Pal}{Alignment Devotion}{Su} A paladin is devoted to a specific alignment. She must choose one of her alignment components: good, evil, lawful, or chaotic. The alignment she chooses is her devoted alignment. A paladin's class features are affected by this choice. She excels at slaying creatures with alignments opposed to her devoted alignment. A paladin's devoted alignment cannot be changed without extraordinary repurcussions to the paladin.

\cft{Pal}{Discernment}{Su} A paladin can discern truths about creatures she sees as a swift action. When she uses this ability, she learns which creatures within a \arealarge cone have her devoted alignment. The paladin does not learn anything else about their alignments.

The paladin may use her discernment ability a number of times per day equal to half her paladin level \add her Wisdom (minimum 1).

\cft{Pal}{Smite}{Su} As part of an attack action, a paladin may attempt to smite with one physical melee attack. If the creature does not share her devoted alignment, she adds her Charisma (if positive) to her attack roll, and deals bonus damage equal to her paladin level. If she smites a creature who shares her devoted alignment, she deals no damage at all (not even normal weapon damage), but the use of the ability is still spent.

A paladin can smite a number of times per day equal to half her paladin level \add her Charisma (minimum 1), but may only do so once per round.

\cfl{Pal}{2nd}{Smite Power} The paladin gains an smite power to augment her smite attack. Whenever she smites, she chooses one smite power she knows to augment her attack. If the struck creature's alignment is opposed to her devoted alignment, the creature also suffers the effect of the paladin's smite power.

The paladin chooses a single smite power from the list below. At 5th level, and every 3 levels thereafter, she gains an additional smite power. Some smite powers have minimum paladin levels, as indicated in the title of the ability. All smite powers are supernatural abilities unless otherwise noted. The paladin's special attack bonus with smite powers is equal to her paladin level \add her Charisma.

\parhead{Any Alignment}
\subcf{Bewildering} The paladin's smite challenges her foe's mind. If the paladin succeeds on a special attack vs. Will, the struck creature is \bewildered for 5 rounds.
\subcf{Resounding} The paladin's smite knocks her foe off its feet. The struck creature is knocked prone.
\subcf{Sickening} The paladin's smite strikes her foe where it is weak. If the paladin succeeds on a special attack vs. Will, the struck creature is \sickened for 5 rounds.
\subcf{5th -- Blinding} The paladin's smite manifests as a bright light. If the paladin succeeds on a special attack vs. Will, the struck creature is \blinded for 1 round. Creatures vulnerable to light (such as vampires) take extra damage equal to twice her paladin level.
\subcf{5th -- Impeding} The paladin's smite traps her foe in place, unable to escape her wrath. The struck creature's speed is reduced to 5 feet for 1 round.
\subcf{5th -- Penetrating} The paladin's smite punches through her enemies' defenses. The attack reduces the foe's damage reduction by an amount equal to the paladin's level for 1 round, regardless of the type of damage reduction.
\subcf{8th -- Seeking} The paladin's smite is uncannily guided to its target. The attack ignores any miss chance, though the weapon must still be physically able to strike the target.
\subcf{8th -- Staggering} The paladin's smite hits with incredible force. If the paladin succeeds on a Fortitude attack, the struck creature is \staggered for 5 rounds.
\subcf{11th -- Dazing} The paladin's smite shatters her foe's ability to think. If the paladin succeeds on a Fortitude attack, the struck creature is \dazed for 1 round.
\subcf{11th -- Coercing} The paladin's smite compels her foe to briefly join her cause. If she succeeds at an attack vs. Will, the struck creature must obey a \spell{suggestion}, as the spell, of the paladin's choice. The effect lasts for one round.
\subcf{14th -- Dispelling} The paladin's smite strips away her foe's magical protections. The struck creature is subject to a targeted \spell{dispel magic} with a dispel bonus equal to the paladin's level \add her Charisma.
\subcf{17th -- Brilliant} The paladin's smite cannot be turned aside by mortal defenses. The smite attack is made against the enemy's Reflex defense.
\subcf{20th -- Converting} The paladin's smite shows her foe the error of its ways. If the struck creature is bloodied, and the paladin succeeds at an attack vs. Will, the struck creature's alignment changes. It gains the paladin's devoted alignment for 1 week. After that time, it can choose to return to its original alignment, or keep its new alignment permanently.

\parhead{Good}
\subcf{5th -- Challenging} The paladin's smite compels her foe's attention. The struck creature takes a \minus4 penalty on attacks against all creatures other than the paladin.

\parhead{Evil}
\subcf{5th -- Executing} The paladin's smite takes her foe's life. If the paladin succeeds on a Fortitude attack, the struck creature is affected by the \spell{death knell} spell.
\subcf{8th -- Agonizing} The paladin's smite inflicts debilitating pain. The struck creature is affected by the \spell{agony} spell.

\cflt{Pal}{2nd}{Lay On Hands}{Su} The paladin can channel divine power to heal the wounds of her allies and inflict wounds on her foes. As a standard action, she can touch a creature. If the creature's alignment is not opposed to her devoted alignment, the creature is healed. If the creature's alignment is opposed to the paladin's devoted alignment, the creature takes divine damage. This ability heals or inflicts 1d8 hit points of damage per paladin level.

The paladin can lay on hands a number of times per day equal to 1 \add half her Charisma (minimum 1). When using lay on hands on an unwilling target, the paladin must make a touch attack. If she hits, she must make a special attack vs. Fortitude. A failed attack heals or inflicts half damage. The paladin's special attack bonus with lay on hands is equal to her paladin level \add her Charisma.

\cfl{Pal}{3rd}{Divine Gift} The paladin's devotion to her ideals is rewarded with a divine gift which improves her abilities. She chooses a single divine gift from the list below. At 6th level, and every 3 levels thereafter, she gains an additional divine gift. Some divine gifts have minimum paladin levels, as indicated in the title of the ability. All divine gifts are supernatural abilities unless otherwise noted. The paladin's special attack bonus with divine gifts is equal to her paladin level \add her Charisma.

\parhead{Any Alignment}
\subcf{Divine Health} The paladin is immune to poison and disease.
\subcf{Unbending Mind} The paladin is immune to charm and domination effects.
\subcf{Unshakeable Courage} The paladin is immune to fear and negative morale effects.
\subcf{6th -- Divine Grace} If the paladin resists an attack against her Fortitude, Reflex, or Will that normally deals half damage when resisted, she instead takes no damage.
\subcf{6th -- Shielded Senses} The paladin is immune to sight-dependent and sound-dependent effects, whenever that is beneficial to her.
\subcf{9th -- Implacable Resolve} The paladin is immune to compulsion and inhibition effects.
\subcf{12th -- Improved Divine Grace} If the paladin resists an attack against her Fortitude, Reflex, or Will that has any partial effect when resisted, she instead suffers no effect. The paladin must have the divine grace gift in order to select this gift.

\parhead{Chaotic Divine Gifts}
\subcf{Chaotic Mind} The paladin is unaffected by effects which detect truth, lies, or alignment. Such spells never detect the paladin, just as if she was not there at all.
\subcf{6th -- Confusion} As a standard action, the paladin can expend a use of her smite ability to confuse a creature, as the \spell{confusion} spell.
\subcf{6th -- Break the Chains} As a standard action, the paladin can expend a use of her smite ability to break all shackles, bindings, and locks within a \arealarge radius of her. Nonmagical objects are automatically broken. To break magical objects, the paladin must make a special attack against a DC of 10 \add the object's caster level.
\subcf{9th -- Free the Mind} When the paladin uses her lay on hands ability, she can also choose to remove all enchantment or illusion effects affecting the touched creature.
\subcf{9th -- Uncertain Fate} Whenever the paladin would take 10, she instead rolls 2d20 and uses whichever roll she prefers.
\subcf{12th -- Mass Confusion} As a standard action, the paladin can expend a use of her smite ability to confuse many creatures, as the \spell{mass confusion} spell.
\subcf{12th -- Freedom of Movement} The paladin continuously gains the benefit of the \spell{freedom} spell.

\parhead{Evil Divine Gifts}
\subcf{6th -- Executioner} The paladin gains a \plus2 bonus to physical damage against bloodied creatures.
\subcf{9th -- Debilitating Aura} All creatures within a \areamed radius of the paladin are \vulnerable.

\parhead{Good Divine Gifts}
\subcf{6th -- Shield Other} The paladin gains the ability to take damage for her allies. When she uses her lay on hands ability, she may also choose to take half of the damage that ally will take, as the \spell{share pain} spell.
\subcf{9th -- Martyr's Shield} Whenever an ally within 100 feet would take critical damage, the paladin can expend a use of her lay on hands ability as an immediate action. If she does, the ally takes no damage, and the paladin takes that damage as regular damage instead.
\subcf{12th -- Peaceful Guardian} The paladin may give up a use of her smite ability to gain a use of her lay on hands ability.
\subcf{15th -- Aura of Protection} The paladin continuously radiates a \spell{magic circle against evil}, as the spell.

\parhead{Lawful Divine Gifts}
\subcf{Command} As a standard action, the paladin can expend a use of her smite ability to issue a \spell{command}, as the spell.
\subcf{9th -- Enforcer} The paladin gains a \plus2 bonus to her attacks against creatures who are currently breaking the law. The paladin must be aware of the offense in order to gain this bonus.
\subcf{12th -- Edict} As a standard action, the paladin can expend a use of her smite ability to issue an \spell{edict}, as the spell.
\subcf{15th -- Truthbearer} The paladin is immune to unreal effects on her, such as phantasms. Additionally, she automatically sees through figments. Unlike normally seeing through figments, the paladin does not receive any indication that the figment would otherwise be there - the figment simply does not exist for the paladin.
\subcf{18th -- Prohibition} As a standard action, the paladin can expend a use of her smite ability to issue a \spell{prohibition}, as the spell.

\cflt{Pal}{4th}{Divine Command}{Su} The paladin can call upon her divine connections to control the actions of others. As a swift action, she may 

\cfl{Pal}{4th}{Discern Foe} When the paladin uses her discernment ability, she also learns which creatures within the cone have the alignment opposed to her devoted alignment.

\cflt{Pal}{7th}{Pass Judgment}{Su} The paladin gains the ability to pass judgment on those she deems unworthy. Once per day as a swift action, she may pass judgment on a creature within 100 feet of her. For the purpose of spells and effects, the creature is treated as if it had the alignment opposed to the paladin's devoted alignment. This effect lasts for one day per paladin level, or until the paladin changes her mind about the subject. This does not change the creature's actions or behavior, but the creature is subject to the paladin's smite attack, would detect as that alignment under the inspection of a \spell{detect alignment} spell, and so on.

No attack is required for this effect, and it cannot be dispelled, but a \spell{remove curse}, \spell{miracle}, or \spell{wish} spell can remove it. The paladin can use this ability an additional time per day at her 10th paladin level and every third level thereafter. A paladin should be careful when using this ability, as persecution of the innocent can lead overzealous paladins to fall.

\cflt{Pal}{10th}{Certain Smite}{Su} If the paladin smites a creature who is not opposed to her devoted alignment, the smite attempt is not used.

\cflt{Pal}{19th}{Martyr's Glorious Retribution}{Su} If the paladin dies in the devoted service of her devoted alignment, she may choose to have her fallen body erupt in an immense burst of divine power. If she does, her body is almost completely consumed, preventing her from being raised with \spell{raise dead} and similar effects that require an intact body. This has two effects. First, a \spell{sunburst} spell immediately takes effect over the area where the paladin fell. Second, a \spell{storm of vengeance} spell begins to take effect, centered on the same area. The spell lasts for 10 rounds, and the lightning strikes target the paladin's enemies. Both of these effects harm only the paladin's foes, and do not harm her allies. However, her allies' vision is still impeded by the \spell{storm of vengeance}.

\cflt{Pal}{20th}{Greater Smite}{Su} The paladin can use two smite powers on every smite attack she makes.

\subsubsection{Ex-Paladins}
A paladin who ceases to follow her devoted alignment loses all supernatural paladin class features. She may not gain any additional paladin levels. She regains her abilities and advancement potential if she atones for her violations (see the \spell{atonement} ritual), as appropriate.

\subsection{Ranger}

\begin{dtable}
\lcaption{The Ranger}
\begin{tabularx}{\columnwidth}{>{\ccol}p{\levelcol} >{\ccol}p{\babcolgood} *{3}{>{\ccol}p{\savecol}} >{\lcol}X}
\thead{Level} & \thead{Base Attack Bonus} & \thead{Fort} & \thead{Ref} & \thead{Will} & \thead{Special} \\
1st  & \plus1                        & \plus3  & \plus1  & \plus1 & Quarry \plus2, Track, wild speech \\
2nd  & \plus2                        & \plus4  & \plus2  & \plus2 & Danger sense, favored terrain \\
3rd  & \plus3                        & \plus5  & \plus3  & \plus3 & Ranger lore \\
4th  & \plus4                        & \plus6  & \plus4  & \plus4 & Low-light vision, tracking expert \\
5th  & \plus5                        & \plus7  & \plus4  & \plus4 & Free stride, tenacious hunter \\
6th  & \plus6/\plus1                 & \plus8  & \plus5  & \plus5 & Favored terrain, ranger lore \\
7th  & \plus7/\plus2                 & \plus9  & \plus6  & \plus6 & Guide \\
8th  & \plus8/\plus3                 & \plus10 & \plus7  & \plus7 & Darkvision, quarry \plus3  \\
9th  & \plus9/\plus4                 & \plus11 & \plus7  & \plus7 & Ranger lore \\
10th & \plus10/\plus5                & \plus12 & \plus8  & \plus8 & Favored terrain (planar) \\
11th & \plus11/\plus6/\plus1         & \plus13 & \plus9  & \plus9 & Hidden hunter\\
12th & \plus12/\plus7/\plus2         & \plus14 & \plus10 & \plus10& Blindsense, ranger lore  \\
13th & \plus13/\plus8/\plus3         & \plus15 & \plus10 & \plus10& Terrain mastery  \\
14th & \plus14/\plus9/\plus4         & \plus16 & \plus11 & \plus11& Favored terrain (planar), quarry \plus4 \\
15th & \plus15/\plus10/\plus5        & \plus17 & \plus12 & \plus12& Ranger lore \\
16th & \plus16/\plus11/\plus6/\plus1 & \plus18 & \plus13 & \plus13& Blindsight  \\
17th & \plus17/\plus12/\plus7/\plus2 & \plus19 & \plus13 & \plus14& Terrain mastery, unerring hunter \\
18th & \plus18/\plus13/\plus8/\plus3 & \plus20 & \plus14 & \plus14& Ranger lore, favored terrain (planar)  \\
19th & \plus19/\plus14/\plus9/\plus4 & \plus21 & \plus15 & \plus15& Perfect stride  \\
20th & \plus20/\plus15/\plus10/\plus5& \plus22 & \plus16 & \plus16& Quarry \plus5, truesight
\end{tabularx}
\end{dtable}

\cd{Alignment} Any.

\cd{Hit Value} 6.

\sssecfake{Class Skills}
The ranger's class skills (and the key attribute for each skill) are Athletics (Str), Climb (Str), Swim (Str), Acrobatics (Dex), Escape Artist (Dex), Ride (Dex), Stealth (Dex), Knowledge (dungeoneering) (Int), Knowledge (geography) (Int), Knowledge (nature) (Int), Heal (Wis), Perception (Wis), Survival (Wis), and Creature Handling (Cha).
\cf{Rgr}{Skill Points at 1st Level} 6.

\sssecfake{Class Features}

All of the following are class features of the ranger.

\cf{Rgr}{Weapon and Armor Proficiency}  A ranger is proficient with simple weapons, any two weapon groups, light and medium armor, and shields (except tower shields). He is also proficient with his choice of bows, crossbows, or thrown weapons.

\cft{Rgr}{Quarry}{Ex} A ranger is a deadly hunter. As a swift action, a ranger may designate any foe he sees as his quarry. A ranger gains a \plus2 bonus to physical attacks, Perception checks, and Survival checks against his quarry. However, while a ranger is pursuing a quarry, he takes a \minus2 penalty on the same rolls against any target other than his quarry. A ranger may give up pursuing a quarry as a free action. He may not have more than one quarry at once; if he designates a new quarry, the old target is no longer considered his quarry. If the ranger does not see his quarry for more than a week, it is no longer considered his quarry.

\par The amount by which a ranger's physical attacks and skill checks increase against his quarry is called his quarry bonus. A ranger can designate a quarry a number of times per day equal to his quarry bonus. The ranger's quarry bonus improves to \plus3 at his 8th ranger level, to \plus4 at 14th ranger level, and finally to \plus5 at 20th ranger level. His penalties against targets other than his quarry remains the same.

\cf{Rgr}{Track} A ranger gains Track as a bonus feat (see \featref{Track}).

\cft{Rgr}{Wild Speech}{Su} A ranger has the ability to communicate with animals. This ability functions like the druid ability of the same name (see Wild Speech, \pref{Drd:Wild Speech}). A ranger can use this ability a number of times per day equal to half his ranger level \add his Charisma.

\cflt{Rgr}{2nd}{Danger Sense}{Ex} The ranger has an intuitive sense that alerts him to danger, giving him a \plus2 bonus to initiative checks. This bonus increases by 1 at his 5th ranger level and every 3 ranger levels thereafter.
\par If a character has danger sense from a multiple classes, the character stacks those levels to determine his bonus from danger sense.

\cflt{Rgr}{2nd}{Favored Terrain}{Ex} The ranger becomes particularly attuned to certain kinds of terrain. He chooses one kind of terrain to select as a favored terrain from the list below. Usually, rangers favor their home terrain, but a ranger may choose any kind of terrain that he has personally experienced at least once. At his 6th ranger level, and every four ranger levels thereafter, the ranger gains an additional favored terrain.
\par While in a favored terrain, a ranger gains a \plus2 bonus to Perception, Stealth, and Survival checks. If he desires, he may leave no trace of his passage, causing attempts to track him to take a \minus20 penalty. In addition, his experience with his favored terrain grants the ranger a single ability, regardless of whether he is currently in that terrain or not. The options for favored terrains are listed below.
\subcf{Aquatic} The ranger gains Skill Focus (Swim) as a bonus feat, and halves the penalties he takes for fighting underwater.
\subcf{Cold} The ranger gains cold damage reduction equal to his ranger level. This allows him to ignore the first points of cold damage he would take each round.
\subcf{Desert} The ranger gains fire damage reduction equal to his ranger level. This allows him to ignore the first points of fire damage he would take each round.
\subcf{Forest} The ranger gains Skill Focus (Stealth) as a bonus feat.
\subcf{Mountains} The ranger gains Skill Focus (Climb) as a bonus feat, and takes half damage from falling damage.
\subcf{Plains} The ranger gains Skill Focus (Perception) as a bonus feat.
\subcf{Swamp} The ranger can move at full speed in water-related difficult terrain.
\subcf{Underground} The ranger gains Blind-Fight as a bonus feat.
\subcf{Urban} The ranger gains Skill Focus (Persuasion) as a bonus feat.

\cfl{Rgr}{3rd}{Ranger Lore} The ranger gains an ability drawn from ancient ranger lore. At his 6th ranger level, and every three ranger levels thereafter, the ranger gains an additional ranger lore ability. Some ranger lores have minimum ranger levels, as indicated in the title of the ability. The list of ranger lores is given below. All ranger lore abilities are extraordinary abilities unless otherwise noted.

\subcf{Combat Style} The ranger is skilled with the traditional ranger combat styles. He gains the Precise Shot and Two-Weapon Fighting feats if he meets the prerequisites. However, the benefits of this lore apply only when the ranger is unencumbered.

\subcf{Evasion} If the ranger resists a Reflex attack that normally deals half damage when resisted, he instead takes no damage. Evasion can be used only if a ranger is unencumbered. A helpless ranger does not gain the benefit of evasion.

\subcf{Fast Movement} The ranger increases his movement speed by 10 feet when unencumbered.

\subcf{Favored Enemy} The ranger increases his quarry bonus by \plus2 against creatures of a particular kind. The possible creature options are listed on \trefnp{Favored Enemy Options}. The ranger may select this lore multiple times, choosing a different favored enemy each time.

\begin{dtable}
    \lcaption{Favored Enemy Options}
\begin{tabularx}{\columnwidth}{X X}
Animals and vermin & Humanoids (uncivilized) \\
Dragons & Oozes and plants \\
Fey & Outsiders (inner planes) \\
Giants and monstrous humanoids & Outsiders (outer planes) \\
Humanoids (civilized)  & Undead and constructs \\
\end{tabularx}
\end{dtable}

\subcf{Master of the Hunt} The ranger may use a standard action to share the benefits of his quarry ability with all allies who can see and hear him for 5 rounds. The bonus his allies get is considered an enhancement bonus, and they do not suffer penalties against targets other than the quarry.

\subcf{Scent} The ranger gains scent. 

\subcf{6th -- Improved Combat Style} The ranger increases his skill in the traditional ranger combat styles. He adds half his Wisdom to damage when using ranged attacks or when attacking with two weapons at once. Natural weapons qualify for this purpose if the ranger attacks with two natural weapons at once.

The ranger must have the combat style lore to select this lore. The benefits of this lore apply only when the ranger is unencumbered.

\subcf{9th -- Hail of Arrows} A number of times per day equal to 1 \add half the ranger's Constitution, he may take a full-round action to fire a single projectile at every enemy within a \areamed radius. All enemies must be within one range increment of the ranger. This lore can be used with any projectile weapon that the ranger can reload as a free action.

\subcf{9th -- Storm of Blades} A number of times per day equal to 1 \add half the ranger's Constitution, he may take a standard action to make a single melee attack against every enemy he threatens.

\subcf{12th -- Camouflage}\label{Camouflage} The ranger can use the Stealth skill to hide in any of his favored terrains, even if the terrain does not grant cover or concealment.

\subcf{12th -- Greater Combat Style} The ranger's abilities with traditional ranger combat styles improves again. He gains the Two-Weapon Rend and Manyshot feats if he meets the prerequisites (see \featref{Two-Weapon Rend} and \featref{Manyshot}). He must have the improved combat style lore to choose this lore. The benefits of this lore only apply if the ranger is unencumbered.

\subcf{12th -- Improved Evasion} The ranger's ability to avoid damage improves. If he is affected by Reflex attack that would deal half damage when resisted, he takes only half damage from the attack, even if the attack succeeds. Like evasion, improved evasion can be used only if a ranger is unencumbered. A helpless ranger does not gain the benefit of improved evasion.

\subcf{15th -- Combat Style Mastery} The ranger's abilities with traditional ranger combat styles reach their peak. When using a ranged weapon, he can take a move action to study the weak points of a foe within one range increment. If he does, the next attack he makes against that foe is made against Reflex defense, if it is made within 1 round. When wielding two weapons at once, he gains the pounce ability, allowing him to take a full attack action at the end of a charge.

The ranger must have the greater combat style lore to choose this lore. The benefits apply only if the ranger is unencumbered.

\cflt{Rgr}{4th}{Low-light Vision}{Ex} The ranger's sight improves, allowing him to see in conditions of dim light more easily. He gains low-light vision, as the elf racial ability. If he already has low-light vision, he doubles its benefit, allowing him to see four times as far as a human in poor illumination.

\cflt{Rgr}{4th}{Tracking Expert}{Ex} The ranger's ability to track his foes improves. He may always take 10 on Survival checks made to track, even if conditions would otherwise prevent this. Additionally, he can move at his normal speed while following tracks without taking the normal \minus5 penalty. He takes only a \minus10 penalty (instead of the normal \minus20) when moving at up to twice normal speed while tracking.

\cflt{Rgr}{5th}{Free Stride}{Ex} The ranger can move through any sort of natural terrain that slows or impedes movement at his normal speed without suffering any sort of impairment. If a skill check, such as Climb or Swim, would normally be required to move through the terrain, this ability does not help.

\cflt{Rgr}{5th}{Tenacious Hunter}{Ex} The ranger's ability to pursue his quarry improves. He adds his quarry bonus to his dodge defense modifier and his special defenses against attacks that his quarry makes.

\cflt{Rgr}{7th}{Guide}{Ex} Whenever the ranger is in his favored terrain, all allies that can see and hear the ranger gain his favored terrain bonuses in that terrain as well.

\cflt{Rgr}{8th}{Darkvision}{Ex} The ranger's sight improves again, and he gains the ability to see even when there is no light at all. He gains darkvision out to 60 feet, as the dwarf ability. If he already has darkvision, he increases its range by 60 feet.

\cflt{Rgr}{10th}{Favored Terrain (Planar)}{Ex} The ranger may choose any plane as a favored terrain in addition to his normal options whenever he gains a new favored terrain. He is immune to any hostile planar effects from any plane he has chosen as favored terrain. In addition, he gains a \plus2 bonus to Knowledge checks relating to the plane and is always treated as trained in Knowledge (planar) for the purpose of such checks.

\cflt{Rgr}{11th}{Hidden Hunter}{Su} The ranger becomes even more difficult for his quarry to detect. He adds his quarry bonus to his Stealth checks against his quarry. In addition, he continuously benefits from the effect of the \spell{nondectection} spell against all attempts that his quarry makes to detect him magically. The effect uses a caster level equal to his ranger level \add his Wisdom.

\cflt{Rgr}{12th}{Blindsense}{Ex} The ranger's perceptions are so finely honed that he can sense his enemies without seeing them. He gains the blindsense ability out to 60 feet. This ability allows him to sense the presence and location of objects and foes within 60 feet without seeing them. If he already has the blindsense ability, he increases its range by 60 feet.

\cflt{Rgr}{13th}{Terrain Mastery}{Ex} The ranger gains a greater degree of mastery over some of his favored terrains. He chooses a single kind of terrain that he has already chosen as a favored terrain. At his 17th ranger level, he chooses an additional kind of terrain to master.
\par While in that terrain, his bonuses on Perception, Stealth, and Survival checks increase to \plus4. In addition, he gains another ability based on that terrain that is constantly active, whether or not he is currently in the terrain. The options for terrain masteries are given below.
\subcf{Aquatic} The ranger gains a swim speed equal to his base land speed. If he already has a swim speed, he increases his swim speed by 10 feet.
\subcf{Cold} The ranger becomes immune to fatigue.
\subcf{Desert} The ranger becomes immune to fatigue.
\subcf{Forest} The ranger may use his wild speech ability to communicate with plants, as the druid ability.
\subcf{Mountains} The ranger gains a climb speed equal to his land speed. If he already has a climb speed, he increases his climb speed by 10 feet.
\subcf{Plains} The ranger increases his land speed by 10 feet.
\subcf{Swamp} The ranger becomes immune to nausea.
\subcf{Underground} The ranger increases the range of his darkvision and blindsense by 60 feet.
\subcf{Urban} The ranger can use the Stealth skill to hide behind creatures granting him active cover, just like he can hide behind passive cover. 

\cflt{Rgr}{16th}{Blindsight}{Ex} The ranger gains the ability to ``see'' perfectly without his eyes in a 60 foot radius around him. With this ability, he can fight just as well with his eyes closed as with them open. If he already has the blindsight ability, he increases its range by 60 feet.

\cflt{Rgr}{17th}{Unerring Hunter}{Su} The ranger's ability to hunt down his quarry improves to supernatural levels. Once per day, the ranger may concentrate for a full round to duplicate the effects of the \spell{discern location} spell targeted at his quarry.

\cflt{Rgr}{17th}{Hide in Plain Sight}{Ex} While in any of his favored terrains, the ranger can use the Stealth skill to hide even while being observed, taking a \minus5 penalty to the Stealth check. He still needs cover or concealment to hide.

If the ranger has the Camouflage ranger lore (see \pcref{Camouflage}), this allows the ranger to attempt to hide in almost any situation, as long as he is in one of his favored terrains.

\cflt{Rgr}{19th}{Perfect Stride}{Su} The ranger's ability to surpass obstacles becomes unparalleled. He constantly acts as if he were under the effect of a \spell{freedom} spell, except that it does not allow him to act normally underwater.

\cflt{Rgr}{20th}{Truesight}{Su} The ranger's perceptions are accurate enough to defeat even powerful magic. He gains the ability to see all things as they actually are, as the \spell{true seeing} spell, out to a range of 60 feet.

\subsection{Rogue}
\begin{dtable*}
\lcaption{The Rogue}
\begin{tabularx}{\textwidth}{>{\ccol}p{\levelcol} >{\ccol}p{\babcolgood} *{3}{>{\ccol}p{\babcolgood}} X}
\thead{Level} & \thead{Base Attack Bonus} & \thead{Fort} & \thead{Ref} & \thead{Will} & \thead{Special} \\
1st  & \plus0                & \plus0 & \plus3  & \plus1 & Sneak attack \plus1d6 \\
2nd  & \plus1                & \plus1 & \plus4  & \plus2 & Danger sense, skill talent \\
3rd  & \plus2                & \plus1 & \plus5  & \plus3 & Sneak attack \plus2d6, uncanny dodge \\
4th  & \plus3                & \plus2 & \plus6  & \plus4 & Combat trick \\
5th  & \plus3                & \plus2 & \plus7  & \plus4 & Persistent sneak attack, sneak attack \plus3d6 \\
6th  & \plus4                & \plus3 & \plus8  & \plus5 & Improved uncanny dodge, skill talent \\
7th  & \plus5                & \plus3 & \plus9  & \plus6 & Sneak attack \plus4d6 \\
8th  & \plus6/\plus1         & \plus4 & \plus10 & \plus7 & Combat trick \\
9th  & \plus6/\plus1         & \plus4 & \plus11 & \plus7 & Sneak attack \plus5d6 \\
10th & \plus7/\plus2         & \plus5 & \plus12 & \plus8 & Greater uncanny dodge, skill talent \\
11th & \plus8/\plus3         & \plus5 & \plus13 & \plus9 & Sneak attack \plus6d6 \\
12th & \plus9/\plus4         & \plus6 & \plus14 & \plus10 & Combat trick \\
13th & \plus9/\plus4         & \plus6 & \plus15 & \plus10 & Skill exemplar, sneak attack \plus7d6 \\
14th & \plus10/\plus5        & \plus7 & \plus16 & \plus11 & Skill talent \\
15th & \plus11/\plus6/\plus1 & \plus7 & \plus17 & \plus12 & Sneak attack \plus8d6 \\
16th & \plus12/\plus7/\plus2 & \plus8 & \plus18 & \plus13 & Combat trick \\
17th & \plus12/\plus7/\plus2 & \plus8 & \plus19 & \plus13 & Skill exemplar, sneak attack \plus9d6 \\
18th & \plus13/\plus8/\plus3 & \plus9 & \plus20 & \plus14 & Skill talent \\
19th & \plus14/\plus9/\plus4 & \plus9 & \plus21 & \plus15 & Sneak attack \plus10d6 \\
20th & \plus15/\plus10/\plus5& \plus10& \plus22 & \plus16 & Ambush master, combat trick \\
\end{tabularx}
\end{dtable*}

\cd{Alignment} Any.

\cd{Hit Value} 5.

\sssecfake{Class Skills}
The rogue's class skills (and the key attribute for each skill) are
Athletics (Str), Climb (Str), Swim (Str), Acrobatics (Dex), Escape Artist (Dex),  Sleight of Hand (Dex), Stealth (Dex), Craft (Int), Devices (Int), Disguise (Int), Forgery (Int), Knowledge (dungeoneering), Knowledge (local) (Int), Linguistics (Int), Perception (Wis), Sense Motive (Wis), Bluff (Cha), Persuasion (Cha), Disguise (Cha), Intimidate (Cha), and Perform (Cha).
\cf{Rog}{Skill Points at 1st Level} 8.

\sssecfake{Class Features}
All of the following are class features of the rogue.

\cf{Rog}{Weapon and Armor Proficiency} Rogues are proficient with simple weapons, any two weapon groups, light armor, and bucklers. They are also proficient with saps.

\cf{Rog}{Sneak Attack} If a rogue can catch an opponent when he is unable to defend himself effectively from her attack, she can strike a vital spot for extra damage. She can choose to deal 1d6 points of extra damage if the target is unaware or is suffering overwhelm penalties from being surrounded by enemies (see \pcref{Overwhelm}).

This extra damage is only dealt the first time that the rogue makes a sneak attack against that particular creature in the encounter. Additional sneak attacks against the same creature deal no additional damage.

The extra damage increases by 1d6 at her 3rd rogue level and every two rogue levels thereafter. Unlike most damage bonuses, this extra damage is not multiplied if the rogue scores a critical hit.

\par Ranged attacks can count as sneak attacks only if the target is within 30 feet. A rogue can't strike with deadly accuracy from beyond that range.

With a sap (blackjack) or an unarmed strike, a rogue can make a sneak attack that deals nonlethal damage instead of lethal damage. She cannot use a weapon that deals lethal damage to deal nonlethal damage in a sneak attack, not even with the usual \minus4 penalty.

A rogue can only sneak attack creatures with a discernible body structure -- oozes, incorporeal creatures, and some plants lack vital areas to attack. Any creature that is immune to critical hits is not vulnerable to sneak attacks. The rogue must be able to see the target well enough to pick out a vital spot and must be able to reach such a spot. A rogue cannot sneak attack while striking the limbs of a creature whose vitals are beyond reach.

\cflt{Rog}{2nd}{Danger Sense}{Ex} The rogue has an intuitive sense that alerts her to danger, giving her a \plus2 bonus to initiative checks. This bonus increases by 1 at her 5th rogue level and every 3 rogue levels thereafter.
\par If a character has danger sense from a multiple classes, the character stacks those levels to determine her bonus from danger sense.

\cflt{Rog}{2nd}{Skill Talent}{Ex} The rogue's skills improve. She gains an additional skill point, which she can place in any skill, and a bonus skill feat for which she qualifies. At her 6th rogue level, and every four rogue levels thereafter, she gains an additional skill point and skill feat.

\cflt{Rog}{3rd}{Uncanny Dodge}{Ex} The rogue can react to danger before her senses would normally allow her to do so. She is not helpless when unaware of an attack.

If a rogue already has uncanny dodge from a different class, she stacks those class levels to determine whether she gains improved uncanny dodge (see below) instead.

\cfl{Rog}{4th}{Combat Tricks} The rogue gains a single combat trick to aid her and confound her foes. Tricks marked with an asterisk are called ambush tricks. Ambush tricks only function on the first sneak attack the rogue makes against a particular creature in an encounter.

The rogue chooses a single combat trick from the list below. At her 8th rogue level, and every four rogue levels thereafter, the rogue gains an additional combat trick. Some combat tricks have minimum rogue levels, as indicated in the title of the ability. All combat tricks are extraordinary abilities unless otherwise noted. The rogue's special attack bonus with combat tricks is equal to her rogue level \add her Intelligence.

\subcf{Brutal Ambush*} The rogue rolls d8s instead of d6s for her sneak attack dice. This is an ambush attack, and only functions once per creature.

\subcf{Combat Feat} The rogue gains a combat feat for which she qualifies (see Feats). This trick can be selected multiple times.

\subcf{Distracting Attack} A creature damaged by the rogue's sneak attack takes a penalty on its Concentration checks equal to number of sneak attack dice the rogue would roll. This penalty lasts for 5 rounds.

\subcf{Evasion} If the rogue resists a Reflex attack that normally deals half damage when resisted, she instead takes no damage. Evasion can be used only if a rogue is unencumbered. A helpless rogue does not gain the benefit of evasion.

\subcf{Distant Precision} The rogue can make sneak attacks from up to 100 feet away.

\subcf{Merciful Blows} The rogue suffers no penalty to physical attacks when attacking for nonlethal damage, and can deal her full sneak attack damage when attacking nonlethally.

\subcf{Swift Poisoner} The rogue can apply poison to a weapon she is holding as a swift action.

\subcf{Tricky Maneuver} When performing a maneuver against a creature she would be able to sneak attack, the rogue gains a bonus to attack equal to the number of sneak attack dice she would roll. The benefits of this trick apply even against creatures immune to critical hits.

\subcf{8th -- Bewildering Ambush*} A creature damaged by this sneak attack is \bewildered for 5 rounds. This is an ambush attack, and only functions once per creature.

\subcf{8th -- Defensive Roll} When the rogue would take physical damage that reduces her to 0 hit points, she can attempt to roll with the blow to reduce the damage as an immediate action. If the damage dealt is less than her Reflex defense, she takes half damage, and the damage is nonlethal. She must be aware of the attack to use this ability. This ability can be used number of times per day equal to 1 \add half the rogue's Constitution (minimum 1).

\subcf{8th -- Hamstring*} A creature damaged by this sneak attack has its land speed halved for 5 rounds. Despite the name, this can be used on creatures who do not have hamstrings. This is an ambush attack, and only works once per creature.

\subcf{8th -- Slippery Mind} Whenever an attack for a mind-affecting spell or effect succeeds against the rogue by less than 5, she is affected normally at first. One round later, the rogue is instead affected as if the attack had failed. This does not help against instantaneous effects.

\subcf{12th -- Ambush Strike*} If the rogue uses a Strike feat (see \pcref{Strike Feats}) on this sneak attack, she may add her Intelligence to the special attack for the strike. This is an ambush attack, and only works once per creature.

\subcf{12th -- Assassination} To use this ability, the rogue must spend a full round studying a creature within 30 feet of her that has not noticed her and who is not in combat. If she make a melee sneak attack against that target within 1 round, her attack deals maximum damage, including her sneak attack damage. If the target becomes aware of her presence before she attacks, this ability has no benefit.

\subcf{12th -- Confusing Ambush*} The rogue makes a special attack vs. Will to confuse or bewilder the struck creature for 1 round. A healthy creature is \bewildered, while a creature bloodied after the damage from this attack is \confused. A foe confused by this effect does not automatically attack the rogue on its next turn, even though she attacked it. This is an ambush attack, and only works once per creature, regardless of the result of the attack.

\subcf{12th -- Crippling Ambush*} A creature damaged by this sneak attack takes Strength damage equal to the rogue's Intelligence. This is an ambush attack, and only works once per creature.

\subcf{12th -- Dispelling Ambush (Su)*} A creature damaged by this sneak attack is affected by a targeted \spell{dispel magic}. The caster level for this ability is equal to her rogue level \add her Intelligence. This is an ambush attack, and only works once per creature.

\subcf{12th -- Improved Evasion} If the rogue is affected by Reflex attack that would deal half damage when resisted, she takes only half damage from the attack, even if the attack succeeds. Like evasion, improved evasion can be used only if a rogue is unencumbered. A helpless rogue does not gain the benefit of improved evasion.

\subcf{12th -- Perfect Precision} The rogue has no range limit on her sneak attacks. She must the distant precision combat trick to gain this trick.

\subcf{16th -- Deadly Ambush*} The rogue makes a special attack vs. Fortitude against the struck creature. If the attack succeeds, and the creature is bloodied after the damage dealt by this attack, it loses all its hit points and takes 9 critical damage, causing it to begin dying. If the creature is not bloodied, it suffers no ill effect. This is an ambush attack, and only works once per creature, regardless of the result of the attack.

\subcf{16th -- Hide in Plain Sight} The rogue can use the Stealth skill to hide even while being observed, taking a \minus5 penalty to the Stealth check. She still needs cover or concealment to hide.

\subcf{16th -- Paralyzing Ambush*} The rogue makes a special attack vs. Fortitude against the struck creature. If the attack succeeds, and the creature is bloodied after the damage dealt by this attack, it is paralyzed for 5 rounds. If the creature is not bloodied, it suffers no ill effect. This is an ambush attack, and only works once per creature, regardless of the result of the attack.

\subcf{16th -- Opportunist} Once per round, the rogue can make an attack of opportunity against a creature that has just taken physical damage from another creature's attack. This attack counts as one of the rogue's attacks of opportunity for that round.

\subcf{20th -- Dual Ambush} The rogue can apply the benefits of two ambush attacks to a single sneak attack.

\subcf{20th -- Lingering Ambush} The rogue increases the duration of the effects of her ambush attacks by a multiplier of 10. This has no effect on ambush attacks that have no duration.

\cflt{Rog}{5th}{Persistent Sneak Attack}{Ex} The rogue learns how to strike vital spots more consistently. She gains half her sneak attack dice (rounded down) when making sneak attacks against a creature she has already successfully dealt sneak attack damage to in the encounter. For example, a 5th level rogue would deal 3d6 points of extra damage on her first sneak attack against a creature, and 1d6 points of damage on every subsequent sneak attack against the same creature.

\cflt{Rog}{6th}{Improved Uncanny Dodge}{Ex} The rogue can no longer be overwhelmed as easily; she can react to multiple opponents as easily as she can react to a single attacker. The rogue is always treated as being threatened by two fewer creatures than she actually is for the purpose of determining overwhelm penalties. 

\par If a character already has improved uncanny dodge from a second class and gains improved uncanny dodge, the character stacks those class levels to determine if she should gain greater uncanny dodge.

\cflt{Rog}{10th}{Greater Uncanny Dodge}{Ex} The rogue can no longer be overwhelmed, regardless of the number of foes surrounding her.

\cflt{Rog}{13th}{Skill Exemplar}{Ex} The rogue gains a \plus5 bonus with a single skill of her choice. At her 17th rogue level, she may gain this bonus with an additional skill.

\cflt{Rog}{20th}{Endless Sneak Attack}{Ex} The rogue deals full sneak attack damage on her first successful sneak each round against the same creature, rather than only on her first sneak attack in the encounter against that creature.

\subsection{Sorcerer}
\begin{dtable*}
\lcaption{The Sorcerer}
\begin{tabularx}{\textwidth}{>{\ccol}p{\levelcol} >{\ccol}p{7em} *{3}{>{\ccol}p{\savecol}} >{\lcol}X}
\thead{Level} & \thead{Base Attack Bonus} & \thead{Fort} & \thead{Ref} & \thead{Will} & \thead{Special} \\
1st & \plus0 & \plus0 & \plus0 & \plus3 & Arcane invocation \\
2nd & \plus1 & \plus1 & \plus1 & \plus4     & Arcane invocation, spellsurge \plus2 \\
3rd & \plus1 & \plus1 & \plus1 & \plus5     & Expanded spell knowledge \\
4th & \plus2 & \plus2 & \plus2 & \plus6     & Spellblend invocation \\
5th & \plus2 & \plus2 & \plus2 & \plus7     & Expanded spell knowledge \\
6th & \plus3 & \plus3 & \plus3 & \plus8     & Spellsurge \\
7th & \plus3 & \plus3 & \plus3 & \plus9     & Expanded spell knowledge \\
8th & \plus4 & \plus4 & \plus4 & \plus10    & Defensive spellblend, spellsurge \plus3 \\
9th & \plus4 & \plus4 & \plus4 & \plus11    & Expanded spell knowledge \\
10th & \plus5 & \plus5 & \plus5 & \plus12    & Spellsurge, versatile spellcaster \\
11th & \plus5 & \plus5 & \plus5 & \plus13    & Expanded spell knowledge \\
12th & \plus6/\plus1 & \plus6 & \plus6 & \plus14& Offensive spellblend \\
13th & \plus6/\plus1 & \plus6 & \plus6 & \plus15& Expanded spell knowledge \\
14th & \plus7/\plus2 & \plus7 & \plus7 & \plus16& Spellsurge, spellsurge \plus4 \\
15th & \plus7/\plus2 & \plus7 & \plus7 & \plus17& Expanded spell knowledge \\
16th & \plus8/\plus3 & \plus8 & \plus8 & \plus18 & Rapid spellblend \\
17th & \plus8/\plus3 & \plus8 & \plus8 & \plus19 & Expanded spell knowledge \\
18th & \plus9/\plus4 & \plus9 & \plus9 & \plus20 & Spellsurge \\
19th & \plus9/\plus4 & \plus9 & \plus9 & \plus21 & Expanded spell knowledge \\
20th & \plus10/\plus5 & \plus10& \plus10& \plus22 & Spellsurge \plus5 \\
\end{tabularx}
\end{dtable*}

\cd{Alignment} Any.

\cd{Hit Value} 4

\sssecfake{Class Skills}
The sorcerer's class skills (and the key attribute for each skill) are Knowledge (arcana) (Int), Knowledge (the planes), Spellcraft (Wis), and Intimidate (Cha).
\cf{Sor}{Skill Points at 1st Level} 2.

\sssecfake{Class Features}
All of the following are class features of the sorcerer.

 \cf{Sor}{Weapon and Armor Proficiency}  Sorcerers are proficient with simple weapons  and one other weapon group.  They are not proficient with any type of armor or shield. Armor of any type interferes with a sorcerer's arcane gestures, which can cause his spells with somatic components to fail.

\cf{Sor}{Spells} A sorcerer casts arcane spells using his Charisma.  To learn or cast a spell, a sorcerer must have a Charisma at least equal to the spell's level. A sorcerer's magic attack bonus equals half his caster level \add his Charisma.

Like other spellcasters, the number of spells a sorcerer knows and can cast each day is limited. These limitations are given below on \trefnp{Sorcerer Spells per Day} and \trefnp{Sorcerer Spells Known}. A sorcerer's spells are drawn from the common spells on the arcane spell list (see \pcref{Arcane Spells}). Sorcerers need to rest for eight hours in order to regain spells.

\begin{dtable}
    \lcaption{Sorcerer Spells per Day}
    \centering
    \begin{tabularx}{\columnwidth}{>{\ccol}X *{9}{>{\ccol}p{\spellcol}}}
        & \multicolumn{9}{c}{\thead{---{}---{}---{}---{}---{}---{}---{}---Spell Level---{}---{}---{}---{}---{}---{}---{}---}} \\
        \thead{Level} & \thead{1st} & \thead{2nd} & \thead{3rd} & \thead{4th} & \thead{5th} & \thead{6th} & \thead{7th} & \thead{8th} & \thead{9th} \\
        1st & 3 & \x & \x & \x & \x & \x & \x & \x & \x \\
        2nd & 4 & \x & \x & \x & \x & \x & \x & \x & \x \\
        3rd & 5 & \x & \x & \x & \x & \x & \x & \x & \x \\
        4th & 6 & 3 & \x & \x & \x & \x & \x & \x & \x \\
        5th & 6 & 4 & \x & \x & \x & \x & \x & \x & \x \\
        6th & 6 & 5 & 3 & \x & \x & \x & \x & \x & \x \\
        7th & 6 & 6 & 4 & \x & \x & \x & \x & \x & \x \\
        8th & 6 & 6 & 5 & 3 & \x & \x & \x & \x & \x \\
        9th & 6 & 6 & 6 & 4 & \x & \x & \x & \x & \x \\
        10th & 6 & 6 & 6 & 5 & 3 & \x & \x & \x & \x \\
        11th & 6 & 6 & 6 & 6 & 4 & \x & \x & \x & \x \\
        12th & 6 & 6 & 6 & 6 & 5 & 3 & \x & \x & \x \\
        13th & 6 & 6 & 6 & 6 & 6 & 4 & \x & \x & \x \\
        14th & 6 & 6 & 6 & 6 & 6 & 5 & 3 & \x & \x \\
        15th & 6 & 6 & 6 & 6 & 6 & 6 & 4 & \x & \x \\
        16th & 6 & 6 & 6 & 6 & 6 & 6 & 5 & 3 & \x \\
        17th & 6 & 6 & 6 & 6 & 6 & 6 & 6 & 4 & \x \\
        18th & 6 & 6 & 6 & 6 & 6 & 6 & 6 & 5 & 3 \\
        19th & 6 & 6 & 6 & 6 & 6 & 6 & 6 & 6 & 4 \\
        20th & 6 & 6 & 6 & 6 & 6 & 6 & 6 & 6 & 6 \\
    \end{tabularx}
\end{dtable}

\begin{dtable}
    \lcaption{Sorcerer Spells Known}
    \begin{tabularx}{\columnwidth}{>{\ccol}X *{9}{>{\ccol}p{\spellcol}}}
        & \multicolumn{9}{c}{\thead{---{}---{}---{}---{}---{}---{}---{}---Spell Level---{}---{}---{}---{}---{}---{}---{}---}} \\
        \thead{Level} & \thead{1st} & \thead{2nd} & \thead{3rd} & \thead{4th} & \thead{5th} & \thead{6th} & \thead{7th} & \thead{8th} & \thead{9th} \\
        1st  & 1 & \x & \x & \x & \x & \x & \x & \x & \x \\
        2nd  & 2 & \x & \x & \x & \x & \x & \x & \x & \x \\
        3rd  & 3 & \x & \x & \x & \x & \x & \x & \x & \x \\
        4th  & 3 & 1 & \x & \x & \x & \x & \x & \x & \x \\
        5th  & 4 & 2 & \x & \x & \x & \x & \x & \x & \x \\
        6th  & 4 & 2 & 1 & \x & \x & \x & \x & \x & \x \\
        7th  & 4 & 3 & 2 & \x & \x & \x & \x & \x & \x \\
        8th  & 4 & 3 & 2 & 1 & \x & \x & \x & \x & \x \\
        9th  & 4 & 3 & 3 & 2 & \x & \x & \x & \x & \x \\
        10th & 4 & 3 & 3 & 2 & 1 & \x & \x & \x & \x \\
        11th & 4 & 3 & 3 & 3 & 2 & \x & \x & \x & \x \\
        12th & 4 & 3 & 3 & 3 & 2 & 1 & \x & \x & \x \\
        13th & 4 & 3 & 3 & 3 & 3 & 2 & \x & \x & \x \\
        14th & 4 & 3 & 3 & 3 & 3 & 2 & 1 & \x & \x \\
        15th & 4 & 3 & 3 & 3 & 3 & 3 & 2 & \x & \x \\
        16th & 4 & 3 & 3 & 3 & 3 & 3 & 2 & 1 & \x \\
        17th & 4 & 3 & 3 & 3 & 3 & 3 & 2 & 2 & \x \\
        18th & 4 & 3 & 3 & 3 & 3 & 3 & 2 & 2 & 1 \\
        19th & 4 & 3 & 3 & 3 & 3 & 3 & 2 & 2 & 2 \\
        20th & 4 & 3 & 3 & 3 & 3 & 3 & 2 & 2 & 2
    \end{tabularx}
\end{dtable}

\cf{Sor}{Rituals} Sorcerers, like other spellcasters, can perform rituals to create unique magical effects (see \pcref{Rituals}). A sorcerer begins play with a ritual book containing one arcane ritual of his choice (see \pcref{Arcane Rituals}).

\cf{Sor}{Arcane Invocation} All sorcerers master at least one arcane invocation. An arcane invocation allows a sorcerer to exert magical influence without expending the effort required to cast a spell. The sorcerer may choose to learn one invocation of her choice from the list of arcane invocations described in Chapter 11: Spells.

At his 2nd sorcerer level, the sorcerer learns a second arcane invocation of his choice.

\cfl{Sor}{2nd}{Spellsurge} The sorcerer learns how to alter how he casts spells by initiating a spellsurge. He can initiate a spellsurge as a swift action. While in a spellsurge, he gains bonuses with some types of magic or aspects of spellcasting, but takes penalties to others. A spellsurge lasts for 5 rounds. A sorcerer can attempt to end a spellsurge as a swift action by making a Concentration check against a DC equal to 10 \add his caster level (including any bonuses or penalties from the spellsurge that affect all spells). Success means he ends the spellsurge, and failure means he is unable to return to normal.

The value of the numerical bonuses and penalties granted by a spellsurge is called a sorcerer's spellsurge bonus. Initially, his spellsurge bonus is \plus2, granting him \plus2 bonuses and inflicting \minus2 penalties. A sorcerer can initiate a spellsurge a number of times per day equal to his spellsurge bonus. His spellsurge bonus improves to \plus3 at his 8th sorcerer level, to \plus4 at 14th sorcerer level, and finally to \plus5 at 20th sorcerer level.

At his 6th sorcerer level, and every four sorcerer levels thereafter, the sorcerer learns an additional spellsurge. Some spellsurges have minimum sorcerer levels, as indicated in the title of the ability. The list of spellsurges is given below. All spellsurges are extraordinary abilities unless otherwise specified.

\subcf{Brilliant Surge} The sorcerer gains a bonus to his caster level with light spells, but takes a penalty to his caster level with Illusion spells.
\subcf{Empathic Surge} The sorcerer gains a bonus to caster level with mind-affecting spells, but takes a penalty to Will defense and to caster level with spells that deal damage.
\subcf{Energetic Surge} The sorcerer gains a bonus to caster level with acid, cold, electricity, and fire spells, but takes a penalty to his caster level with other spells.
\subcf{6th -- Focused Surge} The sorcerer gains a bonus to his concentration checks, but takes a penalty to his caster level with all spells.
\subcf{6th -- Patient Surge} The sorcerer gains a bonus to his caster level with all spells, but all spells he casts require at least a full-round action to cast. This does not increase the casting time of spells with a casting time of a full-round action or longer.
\subcf{10th -- Extending Surge} The sorcerer doubles the range on all his spells, but takes a penalty to his caster level with all spells.
\subcf{10th -- Silent Surge} The sorcerer can cast spells as if they did not have verbal components, but takes a penalty to his caster level with all spells.
\subcf{10th -- Stilled Surge} The sorcerer can cast spells as if they did not have somatic components, but takes a penalty to his caster level with all spells. 
\subcf{14th -- Lifebound Surge} The sorcerer gains a bonus to his caster level with all spells, but takes 1 point of Constitution damage each time he casts a spell.
\subcf{14th -- Lingering Surge} The sorcerer gains a bonus to his caster level with spells that have a non-instantaneous duration, but takes a penalty to caster level with spells that are instantaneous.
\subcf{14th -- Resistant Surge} Whenever the sorcerer casts a spell, he gains spell resistance for 1 round. However, he takes a penalty to his caster level with all spells. To affect a creature with spell resistance using a spell or spell-like ability, a special attack is always required. The defense used against the attack is indicated by the spell. If the attack fails, the spell has no effect on the sorcerer.
\subcf{18th -- Widening Surge} The sorcerer doubles the area affected by his spells that affect an area, but takes a penalty to his caster level with all spells.

\cflt{Sor}{3rd}{Expanded Spell Knowledge}{Ex} The sorcerer learns how to cast a particularly esoteric spell. He may choose a restricted spell from the arcane spell list with a spell level of no more than half his sorcerer level (normally, the highest level spell he can cast) and add it to his spell list. He must still use a spell known to learn it, as normal. At his 5th sorcerer level, and every odd sorcerer level thereafter, he can gain access to an additional restricted spell.

\cflt{Sor}{4th}{Spellblend Invocation}{Ex} The sorcerer may combine his arcane invocations with his spells. As a full-round action, the sorcerer may cast a spell that affects only himself using a spell slot one level higher than what the spell would normally require. The spell must have a casting time of 1 standard action or less. If he does, he may also use an arcane invocation as part of the same action. The arcane invocation need not target the sorcerer.

\cflt{Sor}{8th}{Defensive Spellblend}{Ex} The sorcerer may combine two spells together. As a full-round action, the sorcerer may cast two spells at once, resolving each spell's effects separately. The spells cast in this way must have a casting time of 1 standard action or less, and must be at least three spell levels apart, such as a 1st-level spell and a 4th-level spell. In addition, one of the two spells must affect only the sorcerer. Using improved spellblend costs a spell slot of one level higher than the highest level spell being cast.

\cflt{Sorc}{10th}{Versatile Spellcaster}{Ex} The sorcerer's intuitive grasp of magic allows him to be flexible in his use of arcane energy. He can use two sorcerer spell slots of the same level to cast a spell or use an ability requiring a sorcerer spell slot of one level higher.

\cflt{Sor}{12th}{Offensive Spellblend}{Ex} The sorcerer may combine a damaging spell and a nondamaging spell together. As a full-round action, the sorcerer may cast two spells at once, resolving each spell's effects separately. The spells cast in this way must have a casting time of 1 standard action or less, and must at least three spell levels apart, such as a 1st-level spell and a 4th-level spell. One spell must be a damaging spell, and the other must not. Using offensive spellblend costs a spell slot of two levels higher than the highest level spell being cast.

\cflt{Sor}{16th}{Endless Surge}{Ex} There is no limit to the duration of a sorcerer's spellsurge ability. If a spellsurge lasts longer than 5 rounds, he automatically succeeds on any Concentration check he makes to end the surge.

\cflt{Sor}{20th}{Dual Surge}{Ex} The sorcerer may enter two spellsurges at once. He gains the benefits and suffers the consequences of both surges. However, while using two surges at once, he takes a \minus5 penalty to Concentration checks to end a spellsurge.

\subsection{Spellwarped}
\begin{dtable}
    \lcaption{The Spellwarped}
    \begin{tabularx}{\columnwidth}{>{\ccol}p{\levelcol} >{\ccol}p{\babcolavg} *{2}{>{\ccol}p{\savecolpoof}} >{\lcol}X}
        \thead{Level} & \thead{Base Attack Bonus} & \thead{Good Defense}\fn{1} & \thead{Normal Defenses}\fn{1} & \thead{Special} \\
        1st & \plus0                    & \plus2  & \plus1  & Innate magic, invoke power, spellwarp pool\\
        2nd & \plus1                    & \plus3  & \plus2  & Spellwarped body, surge of power \\
        3rd & \plus2                    & \plus4  & \plus3  & Attuned senses, spellwarped aspect \\
        4th & \plus3                    & \plus5  & \plus4  & Invoke power, resist magic \\
        5th & \plus3                    & \plus6  & \plus4  & Manipulate magic \\
        6th & \plus4                    & \plus7  & \plus5  & Invoke power \\
        7th & \plus5                    & \plus8  & \plus6  & Spellwarped aspect \\
        8th & \plus6/\plus1             & \plus9  & \plus7  & Invoke power \\
        9th & \plus6/\plus1             & \plus10 & \plus7  & Spell resistance\\
        10th & \plus7/\plus2            & \plus11 & \plus8  & Invoke power \\
        11th & \plus8/\plus3            & \plus12 & \plus9  & Spellwarped aspect \\
        12th & \plus9/\plus4            & \plus13 & \plus10 & Invoke power \\
        13th & \plus9/\plus4            & \plus14 & \plus10 & Improved manipulate magic \\
        14th & \plus10/\plus5           & \plus15 & \plus11 & Invoke power \\
        15th & \plus11/\plus6/\plus1    & \plus16 & \plus12 & Spellwarped aspect \\
        16th & \plus12/\plus7/\plus2    & \plus17 & \plus13 & Invoke power \\
        17th & \plus12/\plus7/\plus2    & \plus19 & \plus13 & Mass surge of power \\
        18th & \plus13/\plus8/\plus3    & \plus20 & \plus14 & Invoke power \\
        19th & \plus14/\plus9/\plus4    & \plus21 & \plus15 & Permanent surge of power, spellwarped aspect \\
        20th & \plus15/\plus10/\plus5   & \plus22 & \plus16 & Invoke power \\
    \end{tabularx}
    1 Each spellwarped has a good defense determined by his choice of innate magic.
\end{dtable}

\cd{Alignment} Any.

\cd{Hit Value} 5.

\ssecfake{Class Skills}
The spellwarped's class skills (and the key attribute for each skill) are Swim (Str), Ride (Dex), Knowledge (arcana) (Int), Spellcraft (Wis), and Intimidate (Cha). He gains additional class skills based on his choice of innate magic.
\cf{Spl}{Skill Points at 1st Level} 4.

\sssecfake{Class Features}
All of the following are class features of the spellwarped.

\cf{Spl}{Weapon and Armor Proficiency}
A spellwarped is proficient with simple weapons, any two weapon groups, light and medium armor, and shields (except tower shields).

\cft{Spl}{Innate Magic}{Ex} Each spellwarped draws his power from a particular kind of magic. This is a choice made when the first level of the class is taken, and it cannot thereafter be changed. The choices are listed below.
\subcf{Alteration} The spellwarped can manipulate the physical forms of creatures. His good defense is Fortitude, his key attribute is Intelligence, and he treats Athletics, Escape Artist, and Disguise as class skills. An alteration spellwarped may be called an alterer, bodywarper, or shifter.
\subcf{Pyromancy} The spellwarped can manipulate fire and heat. His good defense is Will, his key attribute is Charisma, and he treats Acrobatics, Athletics, and Perform as class skills. A pyromancy spellwarped may be called a pyromancer.
\subcf{Telekinesis} The spellwarped can manipulate objects and creatures with his mind. His good defense is Will, his key attribute is Intelligence, and he treats Craft, Devices, and Sleight of Hand as class skills. A telekinesis spellwarped may be called a telekine.
\subcf{Temporal} The spellwarped can manipulate time. His good defense is Reflex, his key attribute is Wisdom, and he treats Acrobatics, Perception, and Sleight of Hand as class skills. A temporal spellwarped may be called a temporalist or timewarper.

\cft{Spl}{Spellwarp Pool}{Su} A spellwarped has the ability to tap into the latent magic within his body to generate magical effects. He has a maximum number of spellwarp points equal to half his spellwarped level \add his Constitution (minimum 1 point). Each hour, he regains a number of spellwarp points equal to his key attribute. As long as he has at least one spellwarp point remaining, he gains a minor ability based on his choice of magic.
\subcf{Alteration -- Alter Appearance} The spellwarped can change minor aspects of his appearance at will -- removing a mole or lengthening his beard slightly. This can grant him a \plus2 bonus to Disguise checks. Major changes are not possible.
\subcf{Pyromancy -- Ember} The spellwarped can snap his fingers as a swift action to create a small ember of flame in his hand for 5 minutes. This ember casts light as a torch, and can deal 1 point of fire damage with a successful touch attack. The ember can be dismissed as a swift action or extinguished as a move action.
\subcf{Telekinesis -- Object Manipulation} The spellwarped can concentrate as a standard action to move objects within five feet of him telekinetically. He can slowly lift or manipulate one object by up to one foot per round. The object can weigh up to five pounds. This level of control is insufficient to make skill checks or wield a weapon or shield effectively.
\subcf{Temporal -- Time Awareness} The spellwarped always knows exactly what time it is, and can track the passage of time precisely without effort.

\cf{Spl}{Invoke Power} A spellwarped can invoke his innate magic to generate powerful effects by spending a spellwarp point. He chooses a single power at 1st level from those available based on his choice of innate magic.

At his 4th spellwarped level, and every two spellwarped levels thereafter, he gains an additional power. Some powers have minimum spellwarped levels, as indicated in the title of the ability. The list of powers is given at \pcref{Spellwarped Powers}. All spellwarped powers are supernatural abilities unless otherwise noted. The spellwarped's special attack bonus with spellwarped powers is equal to his spellwarped level \add his key attribute.

\cflt{Spl}{2nd}{Surge of Power}{Su} The spellwarped can invoke a surge of magical power that allows him to embody his innate magic more fully for 5 rounds. To invoke a surge of power, he must spend a spellwarp point as a a swift action. The effect of his surge depends on his choice of innate magic, as described below.
\subcf{Alteration -- Alter Body} The spellwarped enhances his physical ability. He gains a \plus2 enhancement bonus to a physical attribute of his choice. This bonus increases by 1 at 8th, 14th, and 20th spellwarped level.
\subcf{Pyromancy -- Flame Aura} The spellwarped emanates an aura of fire for 5 rounds. At the start of each of his turns, creatures adjacent to him take one point of fire damage per spellwarped level.
\subcf{Telekinesis -- Kinetic Deflection} The spellwarped reflexibly deflects attacks away with his mind. He gains a \plus2 bonus to his shield defense modifier. This bonus stacks with the bonus from using a shield. At 8th, 14th, and 20th spellwarped level, the shield bonus increases by 1.
\subcf{Temporal -- Accelerate Movement} The spellwarped accelerates his movement and reactions. He gains a \plus2 enhancement bonus to his dodge defense modifier and a \plus10 foot enhancement bonus to his movement speed. At 8th, 14th, and 20th spellwarped level, the dodge bonus increases by 1 and the speed bonus increases by 10 feet.

\cflt{Spl}{2nd}{Spellwarped Body}{Ex} The spellwarped's body is fundamentally altered by exposure to magic. He shows signs of the magic coursing through his body: strangely or inconsistently colored hair, natural skin markings which often resemble runes, and so on. Anyone observing the spellwarped can make a Perception or Spellcraft check with a DC equal to 20 \sub his spellwarped level to recognize that the character is a spellwarped. In addition, the spellwarped gains an ability based on his innate magic.
\subcf{Alteration -- Augment Skin} The spellwarped gains a \plus1 bonus to his armor defense modifier. This bonus increases by 1 at his 10th and 20th spellwarped levels.
\subcf{Pyromancy -- Energy Resistance} The spellwarped gains cold and fire damage reduction equal to twice his spellwarped level, allowing him to ignore the first points of cold or fire damage he takes each round.
\subcf{Telekinesis -- Tactile Telekinesis} The spellwarped gains a \plus1 bonus to Strength and Dexterity-based skill checks. This bonus increases by 1 at his 5th spellwarped level and every 5 spellwarped levels thereafter.
\subcf{Temporal -- Accelerate Reaction} The spellwarped gains a \plus2 bonus to initiative checks. This bonus increases by 1 at his 5th spellwarped level and every 3 spellwarped levels thereafter.

\cflt{Spl}{3rd}{Attuned Senses}{Su} The spellwarped learns to recognize the telltale signs of his chosen magic. He must concentrate as a standard action to use this ability, and he may do so any number of times per day.
\subcf{Alteration -- Perceive Alteration} The spellwarped can discern the true form of all creatures within 50 feet of him for 1 round, ignoring any effects which magically alter their shapes. This also grants him a \plus5 bonus to Perception checks to see through disguises.
\subcf{Pyromancy -- Flame of Life} The spellwarped can see the life-fire that lies within all living creatures, allowing him to clearly see all living creatures within 50 feet of him for 1 round. This ability can reveal creatures hiding in concealment and defeat figments and glamers such as \spell{invisibility}, but does not reveal creatures hiding behind cover. It also allows the spellwarped to see unusually warm objects, such as fires.
\subcf{Telekinesis -- Spatial Awareness} The spellwarped can feel the forms of all objects and creatures around him, granting blindsense out to a 50 foot range for 1 round.
\subcf{Temporal -- Accelerated Search} The spellwarped can accelerate his mind to immediately search everything within a 10 foot radius of him with the Perception skill as a standard action. Alternately, he may use this ability to read a book ten times as fast as normal.

\cflt{Spl}{3rd}{Spellwarped Aspect}{Su} The spellwarped gains a new ability based on his continued exposure to magical energy. Most aspects are specific to particular kinds of innate magic, but some aspects can be taken by any spellwarped. These aspects are listed under the General heading.

At his 7th spellwarped level, and every four spellwarped levels thereafter, the spellwarped gains an additional spellwarped aspect. Some aspects require a minimum spellwarped level, as indicated in the title of the ability. The full list of spellwarped aspects is given below.

\parhead{General}
\subcf{Spellwarped Soul} The spellwarped may use his character level in place of his spellwarped level to determine the effects of his spellwarped abilities, including the damage dealt and the special attack bonus. This does not affect the number of spellwarp points he has available.
\subcf{7th -- Expanded Senses} The range of the spellwarped's attuned senses ability doubles.
\subcf{11th -- Accelerated Recovery} The spellwarped regains spellwarp points once per 10 minutes, rather than once per hour.
\subcf{11th -- Rapid Senses} The spellwarped can constantly gain the benefit of his attuned senses ability. He can toggle his enhanced senses on or off as a swift action. If the ability does not have a duration, such as the temporal attuned senses ability, this aspect has no effect.

\parhead{Alteration}
\subcf{Damage Reduction} The spellwarped gains physical damage reduction against his choice of piercing, slashing, or bludgeoning damage. The amount of damage resisted is equal to half his spellwarped level, allowing him to ignore the first points of damage he takes each round. If he is hit by an adamantine weapon, he cannot use his damage reduction for 1 round.
\subcf{7th -- Improved Damage Reduction} The spellwarped's damage reduction applies against all forms of physical damage. The spellwarped must have the damage reduction aspect to gain this aspect.
\subcf{7th -- Alter Movement} The spellwarped gains his choice of the Legendary Balance, Legendary Climber, Legendary Leaper, or Legendary Swimmer feats, even if he does not meet the prerequisites. He may select this aspect multiple times, choosing a different bonus feat each time.
\subcf{11th -- Alter Size} When the spellwarped uses his surge of power, he can increase or decrease by a size category, as he chooses. The size alteration lasts as long as his surge of power does. This is a size-affecting effect, and does not stack with other size-affecting effects.
\subcf{15th -- Fast Healing} While his surge of power is active, the spellwarped gains fast healing equal to half his spellwarped level, allowing him to heal damage each round. This does not affect critical damage.

\parhead{Pyromancy}
\subcf{Improved Ember} When the spellwarped uses his ember ability, he can strengthen the fire so that it illuminates up to a 40 foot radius with bright illumination. He can also throw the ember up to 100 feet. It burns for up to 5 rounds on its own before becoming extinguished.
\subcf{Intense Flames} The spellwarped's attacks can ignore an amount of fire damage reduction equal to his spellwarped level \add his Charisma.
\subcf{7th -- Flame Eater} When the spellwarped resists fire damage with his spellwarped body ability, he gains temporary hit points equal to the damage resisted for 5 minutes.

\parhead{Telekinesis}
\subcf{Improved Object Manipulation} When the spellwarped uses his object manipulation ability, he can affect objects within 10 feet, with a weight limit of up to two pounds per spellwarped level. He has enough control to make checks with a DC of up to 10.
\subcf{7th -- Shieldbearer} The spellwarped may wield shields, except tower shields, telekinetically. The shield floats in his square, granting him its bonus to his shield defense modifier. He does not need a free hand to wield the shield and suffers no armor check penalty or arcane spell failure from it. The shield follows him as he moves. If it is forcibly removed from his square, he loses control over it and it falls to the ground.
\subcf{11th -- Mind Armory} The spellwarped may control a number of weapons equal to half his Intelligence with his mind blade ability. This does not allow him to make additional attacks per round, but he may attack interchangeably with any weapon he controls. Each weapon threatens an area and contributes to overwhelm penalties, just as with his normal mind blade ability.

\parhead{Temporal}
\subcf{Evasion} If the spellwarped resists a Reflex attack that normally deals half damage when resisted, he instead takes no damage. Evasion can be used only if a spellwarped is unencumbered. A helpless spellwarped does not gain the benefit of evasion.
\subcf{Fast Movement} The spellwarped gains a \plus10 foot bonus to movement speed.
\subcf{Uncanny Dodge} The spellwarped is not helpless when unaware of an attack.
\subcf{7th -- Accelerate Attack} While his surge of power is active, the spellwarped can make an additional attack at a \minus5 penalty when making a full attack.

\cflt{Spl}{4th}{Resist Magic}{Ex} The power of the magic with the spellwarped offers him some measure of protection against hostile magical effects. He gains a \plus1 bonus to special defenses against spells and spell-like abilities. This bonus increases by \plus1 at his 8th spellwarped level and every 4 spellwarped levels thereafter.

\cflt{Spl}{5th}{Manipulate Magic}{Su} The spellwarped can channel his innate magic to manipulate other forms of magic. Using this ability costs a spellwarp point. 
\subcf{Alteration -- Absorption} As an immediate action, when the spellwarped makes successfully resists an attack against his Fortitude from a spell or spell-like ability, he may absorb the magic harmlessly into his body. The spell has no effect on him, even if it would normally have an effect on a failed attack.
\subcf{Pyromancy -- Fuel the Flame} As an immediate action, when the spellwarped is affected by a spell or spell-like ability, he may channel its energy into a burst of flame around him. Creatures within a \areasmall radius of the spellwarped take fire damage equal to his spellwarped level. The spell still has its normal effect on the spellwarped.
\subcf{Telekinesis -- Mind over Matter} As an immediate action, when the spellwarped is subject to an attack against his Fortitude from a spell or spell-like ability, he may use his Will defense instead.
\subcf{Temporal -- Accelerate Magic} As a swift action, the spellwarped can increase or decrease the duration of any spell or spell-like ability affecting him by two rounds. This can end the effect immediately if it has no time remaining. The spellwarped can't increase the duration beyond twice the spell's original duration.

\cflt{Spl}{9th}{Spell Resistance}{Ex} The magic within the spellwarped allows him to completely ignore other magic, granting him spell resistance. To affect a creature with spell resistance using a spell or spell-like ability, a special attack is always required. The defense used against the attack is indicated by the spell. If the attack fails, the spell has no effect on the spellwarped.

\cflt{Spl}{13th}{Improved Manipulate Magic}{Su} The spellwarped can use his manipulate magic ability to affect any ally within \rngmed range of him.

\cflt{Spl}{17th}{Mass Surge of Power}{Su} The spellwarped can share the benefits of his surge of power with his allies. When he uses his surge of power, he can also affect up to five additional creatures within \rngmed range of him.

\cflt{Spl}{19th}{Permanent Surge of Power}{Su} The spellwarped can maintain the full power of his innate magic without limit. He can gain the effects of his surge of power indefinitely. He may toggle the ability on or off as a swift action at will, without expending spellwarp points. This does not allow him to activate his mass surge of power ability at will, and his allies only gain the benefits for 5 rounds.

\ssecfake{Spellwarped Powers}\label{Spellwarped Powers}

\subsubsection{Alteration Powers}
\parhead{1st -- Lesser Reduction} The spellwarped makes an special attack vs. Fortitude against a creature within \rngclose range. A successful attack causes the creature to become one size category smaller for 2 rounds. This has the following effects:
  \begin{itemize*} 
    \item \minus10 ft. penalty to movement speed.
    \item \minus4 penalty to maneuver attack and defense.
    \item \plus1 bonus to other physical attacks and defenses.
    \item \plus4 bonus to Stealth.
  \end{itemize*}
This is a size-affecting effect.
\parhead{4th -- Reduction} This power functions like the lesser reduction power, except that the foe is reduced for 5 rounds.
\parhead{6th -- Purge} The spellwarped can end any single effect or condition on him which requires a successful attack against his Fortitude. If he identify the effects on him, such as with a Spellcraft check, he may freely choose which effect to end. If he cannot differentiate the effects, choose the effect ended randomly.
\parhead{8th -- Body Bludgeon} The spellwarped elongates and distorts a part of his body and strikes a foe with it. The foe must be within his reach, as if he were wielding a reach weapon. He must make a physical attack with that part of his body. If he hits, he deals 1d6 bludgeoning damage per spellwarped level \add his Strength. In addition, whether he hits or misses, he may make a shove attack on the creature that does not provoke attacks of opportunity. He need not move with the creature to push it back.
\parhead{8th -- Enlargement} This power functions like the \spell{enlarge person} spell, except that it can affect creatures of any type.
\parhead{10th -- Amorphous Body} The spellwarped transforms his body into an amorphous form for 1 round. In this form, he gains several benefits. He gains a \plus20 bonus against grapple attacks, is immune to critical hits, takes no penalties for squeezing, and can move through spaces that are no more than two inches in width, though doing so forces him to move at half speed.
\parhead{10th -- Heal Wounds} As a standard action, the spellwarped can spend two spellwarp points to remove his own injuries by transforming himself into a healthier version of his body. He heals 1d6 points of damage per spellwarped level. This also removes any of the following conditions: blinded, diseased, exhausted, fatigued, nauseated, sickened, and poisoned.
\parhead{12th -- Baleful Polymorph} This attack functions like the \spell{baleful polymorph} spell.
\parhead{14th -- Flight} As a swift action, the spellwarped can spend two spellwarp points to grow wings to fly for 5 rounds. His fly speed is equal to his base land speed, and his maneuverability is good. See \pcref{Flying}, for more details. At the end of the duration, the wings are subsumed back into his body. This ability is draining to use, and the spellwarped must wait for five minutes after using it before he can use it again.
\parhead{14th -- Greater Amorphous Body} This power functions like the amorphous form power, except that it costs two spellwarp points and lasts for 5 rounds.
\parhead{16th -- Bludgeon the Horde} This attack functions like the body bludgeon attack, except that he may attack all foes within his reach, as if he were wielding a reach weapon. He deals 1d6 bludgeoning damage per two spellwarped levels \add his Strength to each foe.
\parhead{18th -- }
\parhead{20th -- }

\subsubsection{Pyromancy Powers}
\parhead{1st -- Lesser Ignite} As a standard action, the spellwarped makes a Reflex attack to deal damage to a foe within \rngclose range. This attack deals 1d6 points of fire damage \add 1 per spellwarped level. A failed attack deals half damage.
\parhead{1st -- Weapon of Flame} As a swift action, the spellwarped can create a weapon made of flame that lasts for 5 rounds. The weapon is sized appropriately for him, and may take the form of any weapon he is proficient with. He can attack with the weapon as if it were a normal weapon of its type, except that he adds half his Charisma to damage in place of half his Strength, and all damage dealt with the weapon is fire damage.
\par The flame weapon gains a \plus1 enhancement bonus to attack and damage at 4th spellwarped level. At his 7th level, and every 3 spellwarped levels thereafter, the bonus increases by 1. If it leaves his hand, it is extinguished 1 round later.
\parhead{4th -- Ignite} This attack functions like the lesser ignite attack, except that it deals 1d6 points of fire damage per spellwarped level, and a successful attack also makes the target \ignited.
\parhead{6th -- Ignite Weapon} As a swift action, the spellwarped can set one of his weapon on fire for 5 rounds. During this time, the spellwarped adds half his Charisma to damage with the weapon he wields in addition to half his Strength. This bonus damage is fire damage. If ignites a weapon created using his weapon of flame power, he adds his full Charisma to damage with the weapon instead of half his Charisma.
\parhead{6th -- Fiery Protection} As a standard action, the spellwarped can bestow fire and cold damage reduction equal to twice his spellwarped level on a creature within 30 feet of him. The protection lasts for 1 hour.
\parhead{8th -- Conflagration} As a standard action, the spellwarped can release a powerful explosion of flame. He makes a Reflex attack to deal damage to all enemies and objects within a \areamed radius spread of him. This attack deals 1d6 fire damage per two spellwarped levels. A failed attack deals half damage.
\parhead{10th -- Fire Shield} As a standard action, the spellwarped can wreath himself in flame for 5 rounds. Any creature that hits him with its body or a melee weapon takes 1d6 fire damage per two spellwarped levels. Each individual creature can take this damage only once per round.
\parhead{10th -- Flameheart} As a standard action, the spellwarped can become a being of pure fire for 1 round. In this form, he is immune to physical damage and can pass through openings as small as one inch at no movement penalty. However, he cannot attack normally or use any of his items, as they meld into his body. He may invoke any of his spellwarped powers normally. In this form, he can make a touch attack as a standard action to deal 1d6 points of fire damage per spellwarped level.
\parhead{12th -- Firestride} As a move action, the spellwarped can may teleport to any active flame of at least Tiny size within \rngmed range. When he does so, he immolates himself and disappears into a pile of ash before stepping out from the flame unharmed. An ordinary torch is sufficient flame to teleport to, but not a candle.
\parhead{14th -- Flight of the Phoenix} As a swift action, the spellwarped can spend two spellwarp points to fly on wings of flame for 5 rounds. His fly speed is equal to his base land speed, and his maneuverability is good. See \pcref{Flying}, for more details. At the end of the duration, the wings are extinguished. This ability is draining to use, and the spellwarped must wait for five minutes after using it before he can use it again.
\parhead{14th -- Greater Flameheart} This power functions like the flameheart power, except that it lasts for 5 rounds.
\parhead{16th -- } 
\parhead{18th -- Phoenix Revival} When the spellwarped takes critical damage, he may spend five spellwarp points as an immediate action, even if the critical damage would be sufficient to kill him. If he does, he ignores the critical damage he just took and dissolves into a pile of ash for 5 rounds. During this time, he can take no actions. If the pile of ash remains intact after 5 rounds, the spellwarped is restored to his normal body, with zero hit points but with all critical damage healed. However, if the pile of ash is dispersed, the spellwarped dies. The ash cannot be harmed by fire damage, and if the pile of ash would take at least 10 points of fire damage during a round, the spellwarped returns one round sooner. The spellwarped may take his normal actions immediately after being restored.
\parhead{20th -- Immolate} As a standard action, the spellwarped makes a special attack vs. Fortitude against a foe within \rngclose range to consume it in flames from the inside out. This attack deals d6 points of fire damage per spellwarped level, and if the creature is bloodied after it takes this damage, it immediately dies. A failed attack deals half damage and leaves a bloodied creature with 0 hit points.

\subsubsection{Telekinesis Powers}
\parhead{1st -- Lesser Crush} As a standard action, the spellwarped makes a special attack vs. Fortitude against a creature within \rngclose range. If his attack succeeds, he crushes it with telekinetic force for 1d6 points of physical damage \add 1 per spellwarped level. A failed attack deals half damage.
\parhead{1st -- Mind Blade} As a swift action, the spellwarped can telekinetically wield an unattended weapon within \rngclose range for 5 rounds. The weapon must be a light or medium weapon appropriate for his size. This allows him to attack with the weapon just as if he were holding it in one hand, except that he uses his Intelligence in place of his Strength. In all other respects, this functions as if he were wielding the weapon normally, including contributing to overwhelm penalties and taking attacks of opportunity. The weapon floats in midair and threatens all squares adjacent to it, and he may make attacks of opportunity with the weapon or with a weapon he wields in his hands, but not both.
As a move action, he may move the weapon up to 30 feet in any direction, even vertically. If the weapon goes outside of \rngclose range, he loses control of it and it falls to the ground.
\parhead{4th -- Crush} This power functions like the lesser crush attack, except that it deals 1d6 points of physical damage per spellwarped level. In addition, if his attack succeeds, the target is also \sickened for 5 rounds.
\parhead{4th -- Mighty Mind Blade} This power functions like the mind blade power, except that the spellwarped may also use a heavy weapon appropriate for his size, allowing him to attack with it just as if he were holding it in two hands. The spellwarped must have the mind blade power to select this power.
\parhead{6th -- Distant Manipulation} As a standard action, the spellwarped can mentally exert influence at up to \rngclose range. This allows him to take any standard action which he could normally take with his hands, using his Intelligence in place of his Strength or Dexterity, as appropriate. He may take actions that require more than a standard action to complete by spending the same amount of time concentrating, spending one spellwarp point per two rounds that he spends concentrating.
\subcf{6th -- Dual Mind Blade} This power functions like his mind blade power, except that the spellwarped may wield two weapons at once. They must stay in the same space, and he may make two-weapon fighting attacks with the weapons, just as if he was wielding them with two hands. The spellwarped must have the mind blade power to select this power.
\parhead{8th -- }
\parhead{10th -- Telekinetic Force} This power functions like the \spell{telekinetic force} spell, using his Intelligence as his casting attribute.
\parhead{12th -- Strangle} As a standard action, the spellwarped can make a special attack vs. Fortitude against a creature within \rngclose range to crush its windpipe. This attack deals 1d6 damage per spellwarped level. If the target is bloodied after the damage is dealt, it is nauseated for 1 round. A failed attack deals half damage, and prevents the foe from being nauseated. This power costs two spellwarp points.
\parhead{14th -- }
\parhead{16th -- } 
\parhead{18th -- }
\parhead{20th -- Mass Strangle} This power functions like the strangle power, except that it costs three spellwarp points and the spellwarped can affect any creatures within a \areasmall radius.

\subsubsection{Temporal Powers}
\parhead{1st -- Lesser Timetheft} As a standard action, the spellwarped can attempt to steal time. He makes a special attack vs. Will against an adjacent creature to force it to skip an action. If it is bloodied, it skips a standard action, while if it is healthy, it skips a move action.
\parhead{4th -- Slow} As a standard action, the spellwarped makes a special attack vs. Will against a creature within \rngclose range, making it \slowed for 5 rounds.
\parhead{4th -- Timetheft} This power functions like lesser timetheft, except that the spellwarped regains a spellwarp point if the attack is successful.
\parhead{6th -- Accelerate Movement} As a standard action, the spellwarped can accelerate a creature within \rngclose range so much that he can seem to pause time for everyone but the subject. This allows the subject to immediately take a single move action. During this move action, the subject does not provoke attacks of opportunity and can move through squares occupied by enemies without penalty. If the spellwarped uses this power on himself, it only requires a move action to activate.
\parhead{6th -- Disjointed Time} As a standard action, the spellwarped makes a special attack vs. Fortitude against a single creature within \rngmed range to chaotically disrupt its local flow of time. If the attack succeeds, the creature takes a \minus4 penalty to attacks, defenses, and checks for 5 rounds.
\parhead{8th -- Haste} As a standard action, the spellwarped dramatically accelerates a creature within \rngclose range for 5 rounds. It doubles all of its movement speeds (to a maximum of an additional 30 feet of movement), and can take an additional attack at a \minus5 penalty when it makes a standard attack. The increase to movement speed is considered an enhancement bonus.
\parhead{8th -- Pause Time} As a standard action, the spellwarped makes a special attack vs. Will against a single creature within \rngclose range to completely stop time for it for 5 rounds. The affected creature can take no actions and cannot be moved, damaged, or even affected in any way until the effect ends. The spellwarped may dismiss the effect as a swift action.
\parhead{10th -- Accelerate Movement, Greater} This power functions like the accelerate movement power, except that it costs two spellwarp points and can be used as a move action. If the spellwarped uses this power on himself, it only requires a swift action to activate.
\parhead{12th -- Timestream} The spellwarped manipulates time in a \arealarge line that extends out from him for 5 rounds. All creatures and objects that pass through the line are \slowed for 1 round. The spellwarped can exclude his allies from the effect. The timestream is virtually invisible, requiring a DC 30 Perception check to notice in a clear environment, though objects passing through the effect can make it obvious.
\parhead{14th -- Inhuman Speed} As a move action, the spellwarped can accelerate himself to immense speed, allowing him to move up to five times his speed. During this time, he does not provoke attacks of opportunity, can move through squares occupied by enemies without penalty, and can treat liquids as if they were solid ground.
\parhead{14th -- Mass Slow} This power functions like the slow power, except that it costs two spellwarp points and affects up to five creatures within \rngclose range of the spellwarped.
\parhead{16th -- Time Reversal} As a swift action, the spellwarped can spend a spellwarp point to create a ``time lock.'' The time lock persists for one round. As a standard action, he can make a special attack vs. Will against a creature within \rngmed range to bring it backwards through time to the point at which the time lock was created. An affected creature is perfectly restored to the point immediately after the time lock was created. The effects of any actions that the creature took in the intervening time are undone, any damage it dealt or took is removed, it is restored to its original location, and the creature is restored in all other ways, just as if the intervening time had never occured. The spellwarped cannot reverse time for himself in this way.
\parhead{16th -- Supreme Acceleration} As a standard action, the spellwarped can spend three spellwarp points to accelerate himself so much that he can take an additional round of actions immediately. During this round, all creatures he attacks are treated as helpless, but he cannot perform a coup de grace or similar ability; such an act requires more care and precision than is possible with such immense speed. After using this ability, he must wait 5 rounds before he can use it again.
\parhead{18th -- Mass Haste} This power functions like the haste power, except that it costs two spellwarp points and affects up to five creatures within \rngclose range of the spellwarped.
\parhead{18th -- Time Stop} As a standard action, the spellwarped can spend two spellwarp points to step into an alternate timestream, causing him to speed up so greatly that all other creatures seem frozen. He can act for 1d3\plus1 rounds of apparent time. During this time, all objects and creatures are frozen in place and are completely invulnerable to the spellwarped, though he may affect them with spellwarped powers normally. After using this ability, he must wait 5 rounds before he can use it again.
\parhead{20th -- Sever Time} As a standard action, the spellwarped can spend two spellwarp points to completely stop time for a single creature for 5 rounds. This functions like the pause time power, except that no attack is required.

\subsection{Wizard}
\begin{dtable*}
\lcaption{The Wizard}
\begin{tabularx}{\textwidth}{>{\ccol}p{\levelcol} >{\ccol}p{7em} *{3}{>{\ccol}p{\savecol}} >{\lcol}X}
\thead{Level} & \thead{Base Attack Bonus} & \thead{Fort} & \thead{Ref} & \thead{Will} & \thead{Special} \\
1st & \plus0 & \plus0 & \plus0 & \plus3 & Arcane invocation \\
2nd & \plus1 & \plus1 & \plus1 & \plus4     & Arcane invocation, magic feat \\
3rd & \plus1 & \plus1 & \plus1 & \plus5     & Arcane insight \\
4th & \plus2 & \plus2 & \plus2 & \plus6     & Invocation sequencer \\
5th & \plus2 & \plus2 & \plus2 & \plus7     & Arcane insight \\
6th & \plus3 & \plus3 & \plus3 & \plus8     & Magic feat \\
7th & \plus3 & \plus3 & \plus3 & \plus9     & Arcane insight \\
8th & \plus4 & \plus4 & \plus4 & \plus10    & Defensive sequencer \\
9th & \plus4 & \plus4 & \plus4 & \plus11    & Arcane insight \\
10th & \plus5 & \plus5 & \plus5 & \plus12    & Magic feat \\
11th & \plus5 & \plus5 & \plus5 & \plus13    & Arcane insight \\
12th & \plus6/\plus1 & \plus6 & \plus6 & \plus14& Contingency \\
13th & \plus6/\plus1 & \plus6 & \plus6 & \plus15& Arcane insight \\
14th & \plus7/\plus2 & \plus7 & \plus7 & \plus16& Magic feat \\
15th & \plus7/\plus2 & \plus7 & \plus7 & \plus17& Arcane insight \\
16th & \plus8/\plus3 & \plus8 & \plus8 & \plus18 & Offensive sequencer \\
17th & \plus8/\plus3 & \plus8 & \plus8 & \plus19 & Arcane insight \\
18th & \plus9/\plus4 & \plus9 & \plus9 & \plus20& Magic feat \\
19th & \plus9/\plus4 & \plus9 & \plus9 & \plus21 & Arcane insight \\
20th & \plus10/\plus5 & \plus10& \plus10& \plus22 & Chain contingency \\
\end{tabularx}
\end{dtable*}

\cd{Alignment} Any.
\cd{Hit Value} 4.

\sssecfake{Class Skills}
The wizard's class skills (and the key attribute for each skill) are
Knowledge (all skills, taken individually) (Int), Linguistics (Int), and Spellcraft (Wis).
\cf{Wiz}{Skill Points at 1st Level} 4

\sssecfake{Class Features}

All of the following are class features of the wizard.

\cf{Wiz}{Weapon and Armor Proficiency} Wizards are proficient with simple weapons, but not with any type of armor or shield. Armor of any type interferes with a wizard's movements, which can cause her spells with somatic components to fail.

\cf{Wiz}{Bonus Languages} A wizard may learn Draconic in addition to the bonus languages available to the character because of her race (see Chapter 2: Races). Many ancient tomes of magic are written in Draconic, and apprentice wizards often learn it as part of their studies.

\cf{Wiz}{Spells} A wizard casts arcane spells using her Intelligence. To learn or cast a spell, a wizard must have an Intelligence at least equal to the spell's level. A wizard's magic attack bonus equals half her caster level \add her Intelligence.

Like other spellcasters, the number of spells a wizard knows and can cast each day is limited. These limitations are given below on \trefnp{Wizard Spells per Day} and \trefnp{Wizard Spells Known}. A wizard's spells are drawn from the common spells on the arcane spell list (see \pcref{Arcane Spells}). Wizards need to rest for eight hours in order to regain spells. 

\begin{dtable}
    \lcaption{Wizard Spells per Day} 
    \centering
    \begin{tabularx}{\columnwidth}{>{\ccol}X *{9}{>{\ccol}p{\spellcol}}}
        & \multicolumn{9}{c}{\thead{---{}---{}---{}---{}---{}---{}---{}---Spell Level---{}---{}---{}---{}---{}---{}---{}---}} \\
        \thead{Level} & \thead{1st} & \thead{2nd} & \thead{3rd} & \thead{4th} & \thead{5th} & \thead{6th} & \thead{7th} & \thead{8th} & \thead{9th} \\
        1st & 3 & \x & \x & \x & \x & \x & \x & \x & \x \\
        2nd & 4 & \x & \x & \x & \x & \x & \x & \x & \x \\
        3rd & 5 & \x & \x & \x & \x & \x & \x & \x & \x \\
        4th & 6 & 3 & \x & \x & \x & \x & \x & \x & \x \\
        5th & 6 & 4 & \x & \x & \x & \x & \x & \x & \x \\
        6th & 6 & 5 & 3 & \x & \x & \x & \x & \x & \x \\
        7th & 6 & 6 & 4 & \x & \x & \x & \x & \x & \x \\
        8th & 6 & 6 & 5 & 3 & \x & \x & \x & \x & \x \\
        9th & 6 & 6 & 6 & 4 & \x & \x & \x & \x & \x \\
        10th & 6 & 6 & 6 & 5 & 3 & \x & \x & \x & \x \\
        11th & 6 & 6 & 6 & 6 & 4 & \x & \x & \x & \x \\
        12th & 6 & 6 & 6 & 6 & 5 & 3 & \x & \x & \x \\
        13th & 6 & 6 & 6 & 6 & 6 & 4 & \x & \x & \x \\
        14th & 6 & 6 & 6 & 6 & 6 & 5 & 3 & \x & \x \\
        15th & 6 & 6 & 6 & 6 & 6 & 6 & 4 & \x & \x \\
        16th & 6 & 6 & 6 & 6 & 6 & 6 & 5 & 3 & \x \\
        17th & 6 & 6 & 6 & 6 & 6 & 6 & 6 & 4 & \x \\
        18th & 6 & 6 & 6 & 6 & 6 & 6 & 6 & 5 & 3 \\
        19th & 6 & 6 & 6 & 6 & 6 & 6 & 6 & 6 & 4 \\
        20th & 6 & 6 & 6 & 6 & 6 & 6 & 6 & 6 & 6 \\
    \end{tabularx}
\end{dtable}

\begin{dtable}
\lcaption{Wizard Spells Known}
\begin{tabularx}{\columnwidth}{>{\ccol}X *{9}{>{\ccol}p{\spellcol}}}
& \multicolumn{9}{c}{\thead{---{}---{}---{}---{}---{}---{}---{}---Spell Level---{}---{}---{}---{}---{}---{}---{}---}} \\
\thead{Level} & \thead{1st} & \thead{2nd} & \thead{3rd} & \thead{4th} & \thead{5th} & \thead{6th} & \thead{7th} & \thead{8th} & \thead{9th} \\
1st  & 1 & \x & \x & \x & \x & \x & \x & \x & \x \\
2nd  & 2 & \x & \x & \x & \x & \x & \x & \x & \x \\
3rd  & 3 & \x & \x & \x & \x & \x & \x & \x & \x \\
4th  & 3 & 1 & \x & \x & \x & \x & \x & \x & \x \\
5th  & 4 & 2 & \x & \x & \x & \x & \x & \x & \x \\
6th  & 4 & 2 & 1 & \x & \x & \x & \x & \x & \x \\
7th  & 4 & 3 & 2 & \x & \x & \x & \x & \x & \x \\
8th  & 4 & 3 & 2 & 1 & \x & \x & \x & \x & \x \\
9th  & 4 & 3 & 3 & 2 & \x & \x & \x & \x & \x \\
10th & 4 & 3 & 3 & 2 & 1 & \x & \x & \x & \x \\
11th & 4 & 3 & 3 & 3 & 2 & \x & \x & \x & \x \\
12th & 4 & 3 & 3 & 3 & 2 & 1 & \x & \x & \x \\
13th & 4 & 3 & 3 & 3 & 3 & 2 & \x & \x & \x \\
14th & 4 & 3 & 3 & 3 & 3 & 2 & 1 & \x & \x \\
15th & 4 & 3 & 3 & 3 & 3 & 3 & 2 & \x & \x \\
16th & 4 & 3 & 3 & 3 & 3 & 3 & 2 & 1 & \x \\
17th & 4 & 3 & 3 & 3 & 3 & 3 & 2 & 2 & \x \\
18th & 4 & 3 & 3 & 3 & 3 & 3 & 2 & 2 & 1 \\
19th & 4 & 3 & 3 & 3 & 3 & 3 & 2 & 2 & 2 \\
20th & 4 & 3 & 3 & 3 & 3 & 3 & 2 & 2 & 2
\end{tabularx}
\end{dtable}

\cf{Wiz}{Rituals} Wizards, like other spellcasters, can perform rituals to create unique magical effects (see \pcref{Rituals}). A wizard begins play with a ritual book containing two arcane rituals of her choice (see \pcref{Arcane Rituals}).

\cf{Wiz}{Arcane Invocation} All wizards master at least one arcane invocation. An arcane invocation allows the wizard to exert magical influence without expending the effort required to cast a spell. The wizard may choose to learn one invocation of her choice from the list of arcane invocations described in Chapter 11: Spells. Specialist wizards must choose one of the invocations granted by their specialist school.

At her 2nd wizard level, the wizard gains a second arcane invocation, which can be chosen from any non-prohibited school.

\cfl{Wiz}{2nd}{Magic Feat} The wizard gains a bonus magic feat or metamagic feat of her choice. He must meet the prerequisites for the feat as normal. At her 6th wizard level, and every 4 wizard levels thereafter, he gains an additional magic feat or metamagic feat.

\cflt{Wiz}{3rd}{Arcane Insight}{Ex} The wizard gains a greater understanding of magic. Generalist wizards gain expanded spell knowledge, as the sorcerer class feature. Specialist wizards may choose a spell of their chosen school from the arcane spell list, including restricted spells, and add it to their spells known. The spell's level must not be higher than half her wizard level -- normally, the highest level of spells that the wizard can cast. At her 5th wizard level, and every odd wizard level thereafter, the wizard gains a new arcane insight.

\cflt{Wiz}{4th}{Invocation Sequencer}{Ex} The wizard gains the ability to create a sequence of a spell and invocation which she can cast together later. To create an invocation sequencer, the wizard must cast a spell which affects only herself and an arcane invocation, which may affect any target. The spell must have a casting time of 1 standard action or less. Neither has any effect immediately. The wizard may later use a full-round action to cast both the spell and the invocation at once, choosing the target of the invocation at that time.
\par The wizard may initially have only one sequencer active at any time. At her 8th wizard level, and every 4 wizard levels thereafter, she may keep an additional sequencer active at once. If she creates a new sequencer, it replaces one of her previous sequencers. She may choose which old sequencer is replaced.

\cflt{Wiz}{8th}{Defensive Sequencer}{Ex} The wizard gains the ability to create a sequence of two spells which she can cast rapidly later. To create a defensive sequencer, the wizard must cast two spells, one of which affects only herself. The spells must be at least three levels apart, and both must have a casting time of 1 standard action or less. Neither has any effect immediately. The wizard may later use a full-round action to cast both spells at once, resolving their effects separately. The number of sequencers a wizard may have active at once is limited, as described in the defensive sequencer ability.

\cflt{Wiz}{12th}{Contingency}{Ex} The wizard gains the ability to prepare a spell so it takes effect automatically if specific circumstances arise. To prepare a contingency, the wizard must spend 1 minute preparing the spell, which consumes the a spell slot two levels higher than the spell's level. During this casting time, the wizard specifies what circumstances cause the spell to take effect.

The contingency can be set to trigger in response to any circumstances that a typical human observing the wizard and her situation could detect. For example, a wizard could specify ``when I fall at least 50 feet'' or ``when I become bloodied,'' but not ``when there is an invisible creature within 50 feet of me'' or ``when I am at 17 hit points or fewer.'' The more specific the required circumstances, the better -- vague requirements, such as ``when I am in danger,'' may cause the contingency to trigger unexpectedly or fail to trigger at all. If a wizard attempts to specify multiple separate triggering conditions, such as ``when I take damage or when an enemy is adjacent to me,'' the contingency will randomly ignore all but one of the conditions.

The spell must have a casting time of 1 standard action or less, and it must target the wizard or have its area centered on the wizard. Any spells which require decisions, such as \spell{dimension door}, must have those decisions made at the time the contingency is created. The wizard cannot change those decisions when the contingency takes effect.

A wizard can have only one contingency active at a time. If he creates another contingency, it replaces his old contingency.

\cflt{Wiz}{16th}{Offensive Sequencer}{Ex} The wizard gains the ability to create a sequence of two spells which she can cast rapidly later. To create an offensive sequencer, the wizard must cast two spells. The spells must be at least three levels apart, and both must have a casting time of 1 standard action or less. One spell must be a damaging spell, and the other must not. Finally, both spells must be of the same school. Neither has any effect immediately. The wizard may later use a full-round action to cast both spells at once, resolving their effects separately. The number of sequencers a wizard may have active at once is limited, as described in the defensive sequencer ability.

\cflt{Wiz}{20th}{Chain Contingency}{Ex} The wizard may ready a sequencer in his contingency instead of a single spell. This sequencer counts against his limit of available sequencers.

\section{Character Advancement}

As your character accomplishes challenges and defeats foes, he gains experience. If your character has enough experience, he gains a level. When you gain a level, you can increase your character's level in your current class or in any other class, and gain the benefits described for each class. Rules for taking levels in multiple classes are described in \pref{Multiclass Characters}, below.

A character that increases in level gains additional benefits. Every odd level, including 1st level, he gains a feat (see \pcref{Feats}). Every even level, he increases one of his attributes by one. He cannot increase the same attribute twice in a row. If a character has multiple classes, these benefits are gained based on the total character level, not based on class level. The experience required to reach a level, and the benefits gained at each level, are shown on \trefnp{Character Advancement}.

\begin{dtable}
\lcaption{Character Advancement}
\begin{tabularx}{\columnwidth}{*{4}{>{\ccol}X}}
  \thead{Character level} & \thead{XP} & \thead{Feats} & \thead{Attribute Increases\fn{1}} \\
1st & 0 & 1st & \x \\
2nd & 2,000 & \x & 1st \\
3rd & 5,000 & 2nd & \x \\
4th & 9,000 & \x & 2nd \\
5th & 15,000 & 3rd & \x \\
6th & 23,000 & \x & 3rd \\
7th & 35,000 & 4th & \x \\
8th & 51,000 & \x & 4th \\
9th & 75,000 & 5th & \x \\
10th & 105,000 & \x & 5th \\
11th & 155,000 & 6th & \x \\
12th & 220,000 & \x & 6th \\
13th & 315,000 & 7th & \x \\
14th & 445,000 & \x & 7th \\
15th & 635,000 & 8th & \x \\
16th & 890,000 & \x & 8th \\
17th & 1,300,000 & 9th & \x \\
18th & 1,800,000 & \x & 9th \\
19th & 2,550,000 & 10th & \x \\
20th & 3,600,000 & \x & 10th
\end{tabularx}
1. The same attribute cannot be increased twice in a row.
\end{dtable}

\section{Multiclass Characters}\label{Multiclass Characters}
A character may add new classes as he or she progresses in level, thus becoming a multiclass character. The class abilities from a character's different classes combine to determine a multiclass character's overall abilities. Multiclassing improves a character's versatility at the expense of focus.

\subsection{Class And Level Features}
As a general rule, the abilities of a multiclass character are the sum
of the abilities of each of the character's classes.

\parhead{Level} ``Character level'' is a character's total number of levels. It is used to determine when feats and attribute score boosts are gained, as noted on \tref{Character Advancement}. Whenever a creature's ``level'' is specified, without reference to a particular class, the character level is used.

\par ``Class level'' is a character's level in a particular class. For a character whose levels are all in the same class, character level and class level are the same.

\parhead{Hit Points} A character gains hit points from each class as his or her class level increases, adding the new hit points to the previous total.

\parhead{Base Attack Bonus} Add your character's levels in classes that grant the same base attack bonus progressions together, then sum those base attack bonuses to find your total base attack bonus. If a character would have a higher base attack bonus by treating a level with an average base attack bonus progression as a level with a poor base attack bonus progression, he or she may do so. For example, a rogue 1 / wizard 1 would have a base attack bonus of 1.

\par For example, a 2nd-level rogue/2nd-level wizard would have a \plus2 base attack bonus. She would get a \plus1 base attack bonus from two levels in a class with average base attack bonus progression (rogue) and a \plus1 base attack bonus from two levels in a class with poor base attack bonus progression (wizard). That gives a total base attack bonus of 1 \add 1, or \plus2. In contrast, a 2nd-level rogue/2nd-level cleric would have a \plus3 base attack bonus, because she would have four levels in classes with average base attack bonus progression.

\parhead{Defenses} Add your character's levels in classes that grant the same base defense bonus progressions together, then sum those base defense bonuses to find your total base defense bonus.

\par For example, a 3rd-level rogue/2nd-level ranger has a base Fortitude bonus of \plus5 (\plus1 from 3 levels in a class with a poor Fortitude defense and \plus4 from two levels in a class with a good Fortitude defense), a base Reflex bonus of \plus7 (from five levels in classes with good Reflex defenses), and a base Will bonus of \plus2 (from five levels in classes with poor Will defenses).

\parhead{Skills} When taking the first level in a class, if that class gives more skill points than the most skill points the character already received from a class, the character immediately gets skill points equal to the difference. For example, if a fighter took a level in rogue, he would immediately get the difference between the rogue's 8 skill points at 1st level and the fighter's 2 skill points at first level, for a total of 6 skill points.

\parhead{Class Features} A multiclass character gets all the class features
of all his or her classes but must also suffer the consequences of the
special restrictions of all his or her classes.

\par In some cases, two classes can have virtually identical abilities.
Use the following guidelines to determine how abilities stack.
\begin{itemize*}
\item If two identical class features are not based on level and are not gained following a specific pattern, they do not stack.
\item If two identical class features are not explicitly based on level, but both classes gain them in a predictable pattern, the levels of the two classes stack for determining when the next improvement to the class feature will be gained.
\item If two identical class features are explicitly based on level, the levels of the two classes stack for determining the power of the ability.
\item If two identical class features say how they stack, those rules trump any other rules.
\end{itemize*}
These are some examples of how to use these guidelines.
\begin{itemize*}
\item Both a druid and a ranger gain woodland stride. A druid/ranger who has woodland stride from both classes has the same woodland stride ability as a druid or ranger would.
\item Both a barbarian and a rogue get danger sense. A barbarian/rogue adds his barbarian and rogue levels together to determine his bonus from danger sense.
\item Both a barbarian and a rogue get uncanny dodge. If a barbarian/rogue would gain uncanny dodge from both classes, she instead gains improved uncanny dodge, because uncanny dodge explicitly states how it stacks.
\end{itemize*}

\parhead{Weapon and Armor Proficiency} A character uses only the highest number of weapon proficiencies granted by her classes. If a class grants proficiency with specific weapon groups, that is counted as a chosen weapon group for the purpose of the number of weapon proficiencies the character may choose. For example, a fighter/paladin would have three weapon groups of her choice, plus the weapon group of her favored deity.

However, if a class grants proficiency with a specific weapon, it is not counted against the number of weapon groups the character gains from that class. For example, a rogue/fighter gains proficiency with four weapon groups of his choice, and is additionally proficient with saps.

\subsection{Spellcasters and Multiclassing}\label{Spellcasters and Multiclassing}
The character gains spells from all of his or her spellcasting classes separately and tracks his spells per day, spells known, and caster level separately with each class.

Characters with magical ability gain a special benefit when multiclassing. Such a character must choose a specific spellcasting class he has. For every two levels that a character has in nonmagical classes, up to the number of levels he has in his chosen spellcasting class, he increases his spellcasting ability with that class. This increases his spells per day, spells known, and caster level as if he had gained a level in his chosen spellcasting class. No class features or other abilities can be gained in this way. 

For example, Gish, a 2nd level fighter / 2th level wizard, would would have the spells per day, spells known, and caster level of a 3rd level wizard. If he gained two more fighter levels, his spellcasting ability would not increase.

