\chapter{Classes and Backgrounds}
Your character's class represents the things your character has chosen to train in. This choice determines a great deal about your character's abilities.

\section{Class Introductions}
There are eleven classes in Rise.
\begin{itemize*}
  \item Barbarians are mighty warriors who can enter a deadly battlerage.
  \item Bards are inspiring jacks-of-all-trades who can cast spells or fight while using their bardic music to help their allies.
  \item Clerics are divine spellcasters who draw power from their veneration of a deity or ideal.
  \item Druids are divine spellcasters who draw power from their veneration of nature.
  \item Fighters are highly disciplined warriors who excel in physical combat of any variety.
  \item Monks are agile masters of ``\ki'' who hone their personal abilities to strike down foes and perform supernatural feats.
  \item Paladins are holy warriors whose devotion to their deity grants them the ability to discern and smite evil.
  \item Rangers are skilled hunters who bridge the divide between nature and civilization.
  \item Rogues are exceptionally skillful characters known for their ability to strike at their foe's weak points in combat.
  \item Sorcerers are arcane spellcasters with an intuitive and flexible understanding of magic.
  \item Wizards are arcane spellcasters with a highly studied and deep understanding of magic.
\end{itemize*}
\begin{comment}
\subsection{Description Format}

\parhead{Alignment} Some classes have alignment restrictions. See Chapter 6: Description for a description of what alignments are.

\parhead{Hit Value} Your starting hit points are equal to your class's Hit Value \add half your Constitution.

\parhead{Class Skills} These are skills that members of this class are typically good at.

\parhead{Skill Points} This is the number of skill points that members of this class get.

\parhead{Weapon and Armor Proficiencies} These are the types of equipment that members of this class are trained in using.

\parhead{Base Attack Progression} Measures how good members of the class are in combat. There are three progressions: Good, Average, or Poor. A character in a class with a Good base attack bonus has a \plus1 base attack bonus. A character's base attack bonus is added to all attacks the character makes.

\parhead{Base Saving Throw Progressions} Measures how resistant members of the class are to unusual kinds of attacks. There are three progressions: Good, Average, or Poor. A character in a class with a Good saving throw progression has a \plus3 base saving throw bonus, while a character with an Average saving throw progression has a \plus1 base saving throw bonus. A character's base saving throw bonus is added to all saving throws the character makes of that kind.

There are three kinds of saving throws. Fortitude saves represent your ability to resist attacks on your body, like poisons and diseases. Reflex saves represent your ability to avoid attacks, such as pit traps or explosions. Will saves represent your ability to resist mental influence, like fearsome creatures and enchantment spells.

\parhead{Class Features} The class features that a character gets for being a member of the class.

\subsection{Barbarian}

\parhead{Hit Value} 7
\parhead{Class Skills} Climb (Str), Jump (Str), Swim (Str), Acrobatics (Dex), Ride (Dex), Perception (Wis), Survival (Wis), Creature Handling (Cha), and Intimidate (Cha)
\parhead{Skill Points} 4
\parhead{Weapon and Armor Proficiencies} A barbarian is proficient with simple weapons, any four other weapon groups, light armor, medium armor, and shields (except tower shields).
\parhead{Base Attack Progression} Good
\parhead{Base Saving Throw Progressions} Good Fortitude, average Reflex, poor Will
\sssecfake{Class Features} 

\cf{Bbn}{Damage Reduction (Ex)} A barbarian gains the ability to shrug off some amount of injury from attacks. He has physical damage reduction equal to his barbarian level, allowing him to the first points of damage that would be dealt to him each round.

\cf{Bbn}{Rage (Ex)} A barbarian can fly into a rage once per day as a swift action. While in a rage, he gains a \plus2 competence bonus to weapon damage rolls and Fortitude and Will saves. In addition, he gains 2 temporary hit points per barbarian level, but he takes a \minus2 penalty to Armor Class. These extra hit points are lost before any other hit points. (For more information, see \pref{Temporary Hit Points}.)

While raging, a barbarian cannot use any Intelligence or Charisma-based skills other than Intimidate. Additionally, he cannot use any abilities that require patience or concentration, such as spellcasting.

\par A fit of rage lasts for a number of rounds up to 5 \add half the barbarian's Constitution, though it can be ended earlier if the barbarian wishes to or goes unconscious. At the end of the rage, the barbarian takes nonlethal damage equal to the number of temporary hit points he gained by raging, loses his rage bonuses and restrictions, and becomes fatigued until he rests for a minute. A fatigued character can neither run nor charge and is vulnerable, giving it a \minus2 penalty to attack rolls, saving throws, checks, DCs, and AC. If the barbarian has any temporary hit points remaining at the end of his rage, the nonlethal damage is dealt to those hit points before they go away.

\subsection{Bard}
\parhead{Hit Value} 5
\parhead{Class Skills} Sleight of Hand (Dex), Forgery (Int), Knowledge (all skills, taken individually) (Int), Linguistics (Int), Perception (Wis), Sense Motive (Wis), Spellcraft (Wis), Bluff (Cha), Persuasion (Cha), Disguise (Cha), and Perform (Cha).
\parhead{Skill Points} 8 
\parhead{Weapon and Armor Proficiencies} Bards are proficient with simple weapons,   any two other weapon groups,  light armor, medium armor, and shields (except tower shields).  
\parhead{Base Attack Progression} Average
\parhead{Base Saving Throw Progressions} Average Fortitude, average Reflex, average Will
\sssecfake{Class Features}

\cf{Brd}{Bardic Knowledge} A bard may make a special bardic knowledge check with a bonus equal to his bard level \add his Intelligence to see whether he knows some relevant information about local notable people, legendary items, or noteworthy places.

A successful bardic knowledge check will not reveal the powers of a magic item but may give a hint as to its general function. A bard may not take 10 or take 20 on this check; this sort of knowledge is essentially random.

\begin{dtable}
\begin{tabularx}{\columnwidth}{l X}
\thead{DC} & \thead{Type of Knowledge} \\
10  & Common, known by at least a substantial minority drinking; common legends of the local population. \\
20  & Uncommon but available, known by only a few people legends. \\
25  & Obscure, known by few, hard to come by. \\
30  & Extremely obscure, known by very few, possibly forgotten by most who once knew it, possibly known only by those who don't understand the significance of the knowledge.
\end{tabularx}
\end{dtable}

 \cf{Brd}{Bardic Music} Once per day, a bard can use his song or poetics as a standard action to produce magical effects on those around him (usually including himself, if desired). While these abilities fall under the category of bardic music and the descriptions discuss singing or playing instruments, they can all be activated by any audible performance.
 
Each ability requires both a minimum bard level and a minimum number of ranks in the Perform skill to qualify; if a bard does not have the required number of ranks in at least one Perform skill, he does not gain the bardic music ability until he acquires the needed ranks.

A bard can typically maintain a song for up to 5 minutes before needing to rest. A deaf bard has a 20\% chance to fail when attempting to use bardic music. If he fails, the attempt still counts against his daily limit. 

\subcf{Countersong (Su)} A bard with 3 or more ranks in a Perform skill can use his music or poetics to counter magical effects that depend on sound (but not spells that simply have verbal components). Each round of the countersong, he makes a Perform check. Any creature within 30 feet of the bard (including the bard himself) that is affected by a language-dependent, sound-dependent, or sonic magical attack may use the bard's Perform check result in place of its saving throw if, after the saving throw is rolled, the Perform check result proves to be higher. If a creature within range of the countersong is already under the effect of a noninstantaneous magical attack of those kinds, it gains another saving throw against the effect each round it hears the countersong, but it must use the bard's Perform check result for the save. Countersong has no effect against effects that don't allow saves.

\subcf{Fascinate (Sp)} A bard with 3 or more ranks in a Perform skill can
use his music or poetics to cause   nearby creatures  to become
fascinated with him. Each creature to be fascinated must be within
\rngmed range, able to see and hear the bard, and able to pay attention to
him. The bard must also be able to see the creature. The distraction
of a nearby combat or other dangers prevents the ability from
working.

\par To use the ability, a bard makes a Perform check. His check result
is the DC for each affected creature's Will save against the effect. If a
creature's saving throw succeeds, the bard cannot attempt to
fascinate that creature again for 24 hours. If its saving throw fails,
the creature sits quietly and listens to the song, taking no other
actions, for as long as the bard continues to play and concentrate (up
to a maximum of 1 round per bard level). While fascinated, a target
takes a \minus4 penalty on skill checks made as reactions, such as Perception checks. Any potential threat requires the bard to make
another Perform check and allows the creature a new saving throw
against a DC equal to the new Perform check result. Any obvious
threat, such as someone drawing a weapon, casting a spell, or aiming
a ranged weapon at the target, automatically breaks the effect. 

Fascinate is an enchantment (compulsion), mind-affecting ability.

\cf{Brd}{Spells} A bard casts arcane spells, which are drawn from the bard spell list. 

He can cast any spell he knows by expending a spell slot of that spell's level or higher. 

Every bard spell has a verbal component (singing, reciting, or music).

To learn or cast a spell, a bard must have a Charisma score at least equal to the level of the spell. The Difficulty Class for a saving throw against a bard's spell is 10 \add 1/2 the bard's caster level \add the bard's Charisma.

Like other spellcasters, a bard can cast only a certain number of spells of each spell level per day. His base daily spell allotment is given on \trefnp{The Bard}.

The bard's selection of spells is extremely limited. A bard begins play knowing four 0-level spells of his choice. At most new bard levels, he gains one or more new spells, as indicated on \trefnp{Bard Spells Known}.

At every bard level, a bard can choose to learn a new spell in place of one he already knows. In effect, the bard ``loses'' the old spell in exchange for the new one. The new spell's level must be the same as that of the spell being exchanged, and it must be at least two levels lower than the highest-level bard spell the bard can cast. A bard may swap only a single spell at any given level, and must choose whether or not to swap the spell at the same time that he gains new spells known for the level.

A bard's magic level is equal to his bard level.

\subsection{Cleric}
\parhead{Hit Value} 5.
\parhead{Class Skills} Knowledge (arcana) (Int), Knowledge (local) (Int), Knowledge (religion) (Int), Knowledge (the planes) (Int), Heal (Wis), Sense Motive (Wis), Spellcraft (Wis), Persuasion (Cha), and Intimidate (Cha).
\parhead{Skill Points} 2.
\parhead{Weapon and Armor Proficiencies} Simple weapons, any two other weapon groups, light and medium armor, and shields (except tower shields).
\parhead{Base Attack Progression} Average.
\parhead{Base Saving Throw Progressions} Average Fortitude, poor Reflex, good Will.
\sssecfake{Class Features}

\cf{Clr}{Domains} A cleric chooses two domains that embody his particular form of devotion. If he worships a deity, he chooses from among his deity's domains. Domains provide the cleric with special abilities and affect the spells a cleric can cast. A cleric gains one granted power from those offered by his two domains. A list of domains and their associated special abilities is given below.
\subcf{Air} The cleric takes half damage from falling damage.
\subcf{Chaos} The cleric gains Spell Focus (Chaos) as a bonus feat.
\subcf{Death} The cleric does not suffer partial effects if he makes his saving throw against death effects. For example, he is not sickened by \spell{finger of death} if he makes his Fortitude save.
\subcf{Destruction} The cleric can ignore half of the hardness of any object he damages, whether with spells or weapons.
\subcf{Earth} The cleric gains Endurance as a bonus feat.
\subcf{Evil} The cleric gains Spell Focus (Evil) as a bonus feat.
\subcf{Fire} The cleric gains fire damage reduction 5.
\subcf{Good} The cleric gains Spell Focus (Good) as a bonus feat.
\subcf{Knowledge} The cleric adds all Knowledge skills to his cleric class skill list.
\subcf{Law} The cleric gains Spell Focus (Law) as a bonus feat.
\subcf{Magic} The cleric gains Spell Focus (any) as a bonus feat.
\subcf{Protection} When the cleric uses the aid another ability to improve an ally's armor class, he grants a \plus4 circumstance bonus instead of a \plus2 circumstance bonus.
\subcf{Strength} The cleric adds Climb, Jump, and Swim to his cleric class skill list.
\subcf{Travel} The cleric adds Survival to his cleric class skill list.
\subcf{Trickery} The cleric adds Bluff and Disguise to his cleric class skill list.
\subcf{Vitality} The cleric gains Spell Focus (Vitalism) as a bonus feat.
\subcf{War} The cleric gains Weapon Focus with his deity's favored weapon group as a bonus feat, even if he doesn't meet the prerequisites. If he does not have a deity, he gains it in a weapon group related to his alignment or ideals.
\subcf{Water} The cleric gains a swim speed equal to his base land speed.

\cf{Clr}{Spells}  A cleric casts divine spells.

However, his alignment may restrict him from casting certain spells opposed to his moral or ethical beliefs; see Chaotic, Evil, Good, and Lawful Spells, below. A cleric must choose and prepare his spells in advance (see below).

Most clerics use Charisma to cast spells. Clerics of specific deities may use a different casting ability, which affects all aspects of the cleric's spellcasting accordingly.

To learn or cast a spell, a cleric must have a Charisma score at least equal to the spell's level. The Difficulty Class for a saving throw against a cleric's spell is 10 \add the spell level \add the cleric's Charisma.

Like other spellcasters, a cleric can cast only a certain number of spells of each spell level per day.

\par A cleric's selection of spells is limited. A cleric begins play knowing one 1st-level spell of his choice, plus two spells chosen from the list of spells offered by his domains.

At each new cleric level, he gains one or more new spells, as indicated on \trefnp{Cleric Spells Known}. Two spells at every spell level must be drawn from the cleric's domains; sometimes these spells are normal spells on the cleric's spell list, but often they are additions to the spell list. A cleric may also choose spells from his domain lists with his normal spells known.

\par A cleric can cast any spell he knows at any time, assuming he
has not yet used up his spells per day for that spell level. For
example, at 1st level, the cleric Cadius can cast four 1st-level
spells per day. He knows four 1st-level spells: one of his choice, two from his domains, and either \spell{cure light wounds} or \spell{inflict light wounds} depending on his alignment (see \trefnp{Cleric Spells Known}). Thus, on any given day, he can cast some combination of the five spells a total of four times.

\par A cleric may use a higher-level slot to cast a lower-level spell if he so chooses. For example, if an 8th-level cleric has used up all his 3rd-level spell slots for the day but wants to cast another third level spell, he could use a 4th-level slot to do so. The spell is still treated as its actual level, not the level of the slot used to cast it.

\par At each cleric level, a cleric can choose to learn a
new spell in place of one he already knows. In effect, the cleric ``loses''
the old spell in exchange for the new one. The new spell's
level must be the same as that of the spell being exchanged, and the spells
must be of the same type; regular
spells cannot be exchanged with domain spells, and the spells a cleric knows
from his alignment cannot be exchanged in this way at all. A cleric may swap only a single
spell at any given level, and must choose whether or not to swap the
spell at the same time that he gains new spells known for the level.

Clerics meditate or pray for their spells. Each cleric must choose a time at which he must spend 1 hour each day in quiet contemplation or supplication to regain his daily allotment of spells. Time spent resting has no effect on whether a cleric can prepare spells. A cleric may prepare and cast any spell on the cleric spell list, provided that he can cast spells of that level, but he must choose which spells to prepare during his daily meditation.

A cleric's magic level is equal to his cleric level.


\end{comment}

\begin{comment}

\section{Class And Level Bonuses}

\parhead{Base Save Bonus} The two numbers given in this column on
\tref{Base Save and Base Attack Bonuses} apply to saving throws. Whether a character uses the first (good) bonus or the second (poor) bonus depends on his or her class and the type of saving throw being attempted.
\begin{cp}%
For example, fighters
get the lower bonus on Reflex and Will saves and the higher bonus
on Fortitude saves, while rogues get the lower bonus on Fortitude
and Will saves and the higher bonus on Reflex saves. Monks are
equally good at all three types of saving throws. See each class's
description to find out which bonus applies to which category of
saves. If a character has more than one class, the base save
bonuses for each class are mostly cumulative; for more information, see
\pdref{Multiclass Characters}.
\end{cp}

\parhead{Base Attack Bonus} On an attack roll, apply the bonus from the
appropriate column on \tref{Base Save and Base Attack Bonuses} according to the class to which the
character belongs. Whether a character uses the first (good) base
attack bonus, the second (average) base attack bonus, or the third
(poor) base attack bonus depends on his or her class.
\begin{cp}%
Barbarians, fighters, paladins, and rangers have a good base attack bonus, so they
use the first Base Attack Bonus column. Clerics, druids, monks, and
rogues have an average base attack bonus, so they use the second
column. Sorcerers and wizards have a poor base attack bonus, so
they use the third column.
\end{cp}
Numbers after a slash indicate additional
attacks at reduced bonuses.
\begin{cp}%
For example, ``\plus12/\plus7/\plus2'' means three attacks per
round, with an attack bonus of \plus12 for the first attack, \plus7 for the
second, and \plus2 for the third.
\end{cp}
Any modifiers on attack rolls apply to
all these attacks normally, but bonuses do not grant extra attacks.
\begin{cp}%
For example, when Lidda the halfling rogue is 2nd level, she has a
base attack bonus of \plus1. With a thrown weapon, she adds her
Dexterity (\plus3), her size bonus (\plus1), and a racial bonus (\plus1) for
a total of \plus6. Even though a \plus6 base attack bonus would grant an
additional attack at \plus1, raising that number to \plus6 via ability, racial,
size, weapon, or other bonuses doesn't grant Lidda an additional
attack.
\end{cp}
If a character has more than one class (see \pref{Multiclass Characters}), the base attack bonuses for each class are cumulative.

\end{comment}
\section{Character Advancement}

\begin{dtable}
\lcaption{Base Save and Base Attack Bonuses}
\begin{tabularx}{\columnwidth}{>{\ccol}p{2em} >{\ccol}X >{\ccol}X >{\ccol}p{6.25em} >{\ccol}p{5.1em} >{\ccol}p{3.5em}}
\thead{Class Level} & \thead{Base Save Bonus (Good)} & \thead{Base Save Bonus (Poor)} & \thead{Base Attack Bonus (Good)} & \thead{Base Attack Bonus (Average)} & \thead{Base Attack Bonus (Poor)} \\
1st & \plus2   & \plus0 & \plus1 & \plus0 & \plus0 \\
2nd & \plus3   & \plus1 & \plus2 & \plus1 & \plus1 \\
3rd & \plus4   & \plus1 & \plus3 & \plus2 & \plus1 \\
4th & \plus5   & \plus2 & \plus4 & \plus3 & \plus2 \\
5th & \plus5   & \plus2 & \plus5 & \plus3 & \plus2 \\
6th & \plus6   & \plus3 & \plus6/\plus1 & \plus4 & \plus3 \\
7th & \plus7   & \plus3 & \plus7/\plus2 & \plus5 & \plus3 \\
8th & \plus8   & \plus4 & \plus8/\plus3 & \plus6/\plus1 & \plus4 \\
9th & \plus8   & \plus4 & \plus9/\plus4 & \plus6/\plus1 & \plus4 \\
10th & \plus9  & \plus5 & \plus10/\plus5 & \plus7/\plus2 & \plus5 \\
11th & \plus10 & \plus5 & \plus11/\plus6/\plus1 & \plus8/\plus3 & \plus5 \\
12th & \plus11 & \plus6 & \plus12/\plus7/\plus2 & \plus9/\plus4 & \plus6/\plus1 \\
13th & \plus11 & \plus6 & \plus13/\plus8/\plus3 & \plus9/\plus4 & \plus6/\plus1 \\
14th & \plus12 & \plus7 & \plus14/\plus9/\plus4 & \plus10/\plus5 & \plus7/\plus2 \\
15th & \plus13 & \plus7 & \plus15/\plus10/\plus5 & \plus11/\plus6/\plus1 & \plus7/\plus2 \\
16th & \plus14 & \plus8 & \plus16/\plus11/\plus6/\plus1 & \plus12/\plus7/\plus2 & \plus8/\plus3 \\
17th & \plus14 & \plus8 & \plus17/\plus12/\plus7/\plus2 & \plus12/\plus7/\plus2 & \plus8/\plus3 \\
18th & \plus15 & \plus9 & \plus18/\plus13/\plus8/\plus3 & \plus13/\plus8/\plus3 & \plus9/\plus4 \\
19th & \plus16 & \plus10 & \plus19/\plus14/\plus9/\plus4 & \plus14/\plus9/\plus4 & \plus9/\plus4 \\
20th & \plus17 & \plus10 & \plus20/\plus15/\plus10/\plus5 & \plus15/\plus10/\plus5 & \plus10/\plus5 \\
\end{tabularx}
\end{dtable}

%XP listed here is placeholder; need an intelligently designed XP system.

\begin{dtable}
\lcaption{Character Advancement and Level-Dependent Bonuses}
\begin{tabularx}{\columnwidth}{*{5}{>{\ccol}X}}
  \thead{Character level} & \thead{XP} & \thead{Feats} & \thead{Attribute Increases\fn{1}} \\
1st & 0 & 1st & \x & \x \\
2nd & 2,000 & \x & 1st \\
3rd & 5,000 & 2nd & \x \\
4th & 9,000 & \x & 2nd \\
5th & 15,000 & 3rd & \x \\
6th & 23,000 & \x & 3rd \\
7th & 35,000 & 4th & \x \\
8th & 51,000 & \x & 4th \\
9th & 75,000 & 5th & \x \\
10th & 105,000 & \x & 5th \\
11th & 155,000 & 6th & \x \\
12th & 220,000 & \x & 6th \\
13th & 315,000 & 7th & \x \\
14th & 445,000 & \x & 7th \\
15th & 635,000 & 8th & \x \\
16th & 890,000 & \x & 8th \\
17th & 1,300,000 & 9th & \x \\
18th & 1,800,000 & \x & 9th \\
19th & 2,550,000 & 10th & \x \\
20th & 3,600,000 & \x & 10th
\end{tabularx}
1. You cannot increase the same attribute twice in a row.
\end{dtable}

\begin{comment}

\parhead{XP} This column on \trefnum{Base Save and Base Attack Bonuses} shows the experience point total
needed to attain a given character level -- that is, the total of all the
character's level in classes. (A character's level in a particular class is called his
or her class level.) For any character (including a multiclass one), XP
determines overall character level, not individual class levels.

\parhead{Feats} Every character gains one feat at 1st level and another at
every level divisible by three (3rd, 6th, 9th, 12th, 15th, and 18th
level). These feats are in addition to any bonus feats granted as class
features (see the class descriptions later in this chapter) and the
bonus feat granted to all humans. See Chapter 5: Feats for more
information about feats.

\parhead{Ability Increases} Upon attaining any level divisible by four
(4th, 8th, 12th, 16th, and 20th level), a character increases one of his
or her attribute scores by 1 point. The player chooses which ability
score to improve.
\begin{cp}%
For example, a sorcerer with a starting Charisma of
16 might increase this to 17 at 4th level. At 8th level, the same
character might increase his Charisma score again (from 17 to 18) or
could choose to improve some other ability instead.
\end{cp}
The ability improvement is permanent.

For multiclass characters, feats and attribute score increases are
gained according to character level, not class level.
\begin{cp}%
Thus, a 3rd-level wizard/1st-level fighter is a 4th-level character overall and eligible
for her first attribute score boost.
\end{cp}
\end{comment}
%
\section{Class Descriptions}

\begin{comment}

%
\ssecfake{Game Rule Information}

\begin{cp}%
Following the general class description comes game rule information.
Not all of the following categories apply to every class.
\parhead{Abilities} The Abilities entry tells you which abilities are most
important for a character of that class.
Players are welcome to ``play against type,'' but a typical character of that class will have his or her
highest attribute scores where they'll do the most good (or, in game
world terms, be attracted to the class that most suits his or her
talents or for which he or she is best qualified).
\end{cp}
\parhead{Alignment} A few classes restrict a character's possible alignments.
\begin{cp}%
For example, a bard must have a nonlawful alignment.
\end{cp}
An entry of ``Any'' means that characters of this class are not restricted in alignment.
 \parhead{Hit Value} The number of hit points the class gets at each level.

\begin{dtable}
\lcaption{Hit Values By Class}
    \begin{tabularx}{\columnwidth}{l >{\lcol}X}
        \thead{HV} & \thead{Class}                             \\
        4      & Sorcerer, wizard                  \\
        5      & Bard, cleric, druid, monk, rogue \\
        6     & Fighter, paladin, ranger          \\
        7     & Barbarian                        
    \end{tabularx}
\end{dtable}

A character gets a number of hit points based on his class at
each level, adds any Constitution to that number,
and adds the result to his or her hit point total. Thus, a
character has the same number of Hit Values as levels. For his or
her first level, a character gets double the listed value (although Constitutions,
positive or negative, still apply normally).

For example, Daria gets a Hit Value of 5 because she's a druid. At 1st
level, she gets 10 hit points. Since she has a Constitution score of 13, she
applies a \plus1 bonus, raising her hit points to 11. When she
reaches 2nd level (and every level thereafter), Daria's
player adds six hit points to her total hit points:
five from her class and one from her Constitution bonus.

If your character has a Constitution penalty and gets a result of 0
or lower after the penalty is applied to the Hit Value, ignore the
roll and add 1 to your character's hit point total anyway. It is not
possible to lose hit points (or not receive any) when gaining a level,
even for a character with a rotten Constitution score.

\parhead{Class Table} This table details how a character improves as he or
she gains levels in the class.
\begin{cp}%
Some of this material is repeated from
\tref{Base Save and Base Attack Bonuses}, but with more detail
on how the numbers apply to that class.
Class tables typically include the following:
\subparhead{Level} The character's level in that class.

\subparhead{Base Attack Bonus} The character's base attack bonus and number
of attacks.

\subparhead{Fort Save} The base save bonus on Fortitude saving throws. The
character's Constitution also applies.

\subparhead{Ref Save} The base save bonus on Reflex saving throws. The character's Dexterity also applies.

\subparhead{Will Save} The base save bonus on Will saving throws. The character's Wisdom also applies.

\subparhead{Special} Level-dependent class abilities, each explained in the Class Features section that follows.

\subparhead{Spells per Day} How many spells of each spell level the character
can cast each day. If the entry is ``\x'' for a given level of spells, the
character may not cast any spells of that level.
(Bonus spells for wizards are based on Intelligence, bonus spells for
clerics, druids, and rangers are based on Wisdom, and bonus spells for sorcerers, bards, and paladins are based off Charisma. See
\trefcp{Attributes and Bonus Spells}.) The character may cast that many spells plus
any bonus spells each day.
\end{cp}

\parhead{Class Skills} This section of a class description gives the class's
list of class skills and the number of skill points the character gets at
at 1st level.
  \parhead{Class Features}  Special characteristics of the class. When applicable,
this section also mentions restrictions and disadvantages of
the class. Class features include some or all of the following.

\subparhead{Weapon and Armor Proficiency} This section details which   weapon groups  and armor types the character is proficient with.
\begin{cp}%
Regardless of
training, cumbersome armor interferes with certain skills (such as
Climb) and with the casting of most arcane spells. Characters can
become proficient with other weapon or armor types by acquiring
the appropriate Armor Proficiency (light, medium, heavy), Shield
Proficiency,   Tower Shield Proficiency, Weapon Proficiency, or Exotic Weapon Proficiency   feats. (See Chapter 5: Feats.)
\end{cp}

\subparhead{Magic Level} Each level that a character gains in a class that provides spells increases that character's magic level. A character's magic level in a particular class is what determines how many spells per day and spells known the character has. Multiclass characters gain a special benefit to their magic level (see \pref{Multiclass Characters}).

\subparhead{Spells} Wizards, sorcerers, clerics, druids, and bards use spells.
Fighters, barbarians, rogues, and monks do not. Paladins and rangers
gain the ability to use spells at 4th level; at that point, their magic level becomes equal to their class level.

\subparhead{Other Features} Each class has certain unique capabilities.
\begin{cp}%
Some abilities are supernatural or spell-like. Using a spell-like
ability is essentially like casting a spell (but without components; see
\pref{Components}), and it provokes attacks of opportunity.
Using a supernatural ability is not like casting a spell. (See Chapter
8: Combat, especially \pdref{Attacks of Opportunity}, and \pdref{Use Special Ability}.)
\end{cp}

\subparhead{Ex-Members} If, for some reason, a character is forced to give up
this class, these are the rules for what happens. Unless otherwise
noted in the class description, an ex-member of a class retains any
weapon and armor proficiencies he or she has gained.

\end{comment}

\subsection{Barbarian}
\begin{dtable*}
\lcaption{The Barbarian}
\begin{tabularx}{\textwidth}{>{\ccol}p{\levelcol} >{\ccol}p{\babcolgood} *{3}{>{\ccol}p{\babcolgood}} X}
\thead{Level} & \thead{Base Attack Bonus} & \thead{Fort Save} & \thead{Ref Save} & \thead{Will Save} & \thead{Special} \\
1st & \plus1         & \plus3 & \plus1 & \plus0 & Damage reduction, rage \plus2 \\
2nd & \plus2         & \plus4 & \plus2 & \plus1 & Fast movement, uncanny dodge \\
3rd & \plus3         & \plus5 & \plus3 & \plus1 & Endurance, channeled rage \\
4th & \plus4         & \plus6 & \plus4 & \plus2 & Grit \\
5th & \plus5         & \plus7 & \plus4 & \plus2 & Resiliency \\
6th & \plus6/\plus1  & \plus8 & \plus5 & \plus3 & Improved uncanny dodge, channeled rage \\
7th & \plus7/\plus2  & \plus9 & \plus6 & \plus3 & Larger than life \\
8th & \plus8/\plus3  & \plus10& \plus7 & \plus4 & Rage \plus3 \\
9th & \plus9/\plus4  & \plus11& \plus8 & \plus4 & Channeled rage \\
10th& \plus10/\plus5 & \plus12& \plus8 & \plus5 & Greater uncanny dodge \\
11th& \plus11/\plus6/\plus1  & \plus13 & \plus9 & \plus5 & Tireless rage \\
12th& \plus12/\plus7/\plus2  & \plus14 & \plus10& \plus6 & Channeled rage \\
13th& \plus13/\plus8/\plus3  & \plus15 & \plus10& \plus6 & Indomitable will \\
14th& \plus14/\plus9/\plus4  & \plus16 & \plus11& \plus7 & Rage \plus4 \\
15th& \plus15/\plus10/\plus5 & \plus17 & \plus12& \plus7 & Channeled rage \\
16th& \plus16/\plus11/\plus6/\plus1 & \plus18 & \plus13& \plus8 & Improved grit \\
17th& \plus17/\plus12/\plus7/\plus2 & \plus19 & \plus13& \plus8 & Larger than belief \\
18th& \plus18/\plus13/\plus8/\plus3 & \plus20 & \plus14& \plus9 & Channeled rage \\
19th& \plus19/\plus14/\plus9/\plus4 & \plus21 & \plus15& \plus9 & Deathless rage \\
20th& \plus20/\plus15/\plus10/\plus5& \plus22 & \plus16 & \plus10 & Rage \plus5
\end{tabularx}
\end{dtable*}

\cd{Alignment} Any nonlawful.

\cd{Hit Value} 7.

\sssecfake{Class Skills}
The barbarian's class skills (and the key attribute for each skill) are
Climb (Str), Jump (Str), Swim (Str), Acrobatics (Dex), Ride (Dex), Perception (Wis), Survival (Wis), Creature Handling (Cha), and Intimidate (Cha).
\cf{Bbn}{Skill Points at 1st Level} 4

\sssecfake{Class Features}

All of the following are class features of the barbarian.
\cf{Bbn}{Weapon and Armor Proficiency}  A barbarian is proficient with simple weapons, any four other weapon groups, light armor, medium armor, and shields (except tower shields).

\cf{Bbn}{Damage Reduction (Ex)} A barbarian gains the ability to shrug off some amount of injury from attacks. He ignores an amount of damage each round equal to his barbarian level. Damage in excess of this value is dealt normally. Damage reduction can reduce damage to 0 but not below 0.

\begin{comment}
\cf{Bbn}{Illiteracy} Barbarians are the only characters who do not
automatically know how to read and write. A barbarian may spend a skill point to gain the ability to read and write all languages he is able to speak.
A barbarian who gains a level in any other class automatically gains literacy. Any other
character who gains a barbarian level does not lose the literacy he or she already had.
\end{comment}

\cf{Bbn}{Rage (Ex)} A barbarian can fly into a rage a certain number of times per day. While in a rage, he temporarily gains a \plus2 competence bonus to weapon damage rolls and Fortitude and Will saves. In addition, he gains 2 temporary hit points per barbarian level, but he takes a \minus2 penalty to Armor Class.
\par The barbarian's rage bonus increases to \plus3 (and 3 temporary hit points per barbarian level) at 8th level, to \plus4 at 14th level, and to \plus5 at 20th level. The penalty to Armor Class remains the same.
\par The extra hit points gained from raging are lost before any other hit points. (For more information, see \pref{Temporary Hit Points}.)

While raging, a barbarian
cannot use any Charisma-, Dexterity-, or Intelligence-based
skills other than Acrobatics, Escape Artist, Intimidate, and
Ride, or any abilities that require
patience or concentration, nor can he cast spells or activate
magic items that require a command word, a spell trigger (such as a wand), or spell
completion (such as a scroll) to function. He can use any feat he has except Combat Expertise, item creation feats, and metamagic feats.

\par A fit of rage lasts for a number of rounds equal to 5 \add half the character's Constitution.   A barbarian may prematurely end his
rage.  A barbarian's rage ends if he becomes unconscious. At the end of the rage, the barbarian takes nonlethal damage equal to the number of temporary hit points he gained by raging, loses his rage bonuses and restrictions, and becomes fatigued (\minus2 penalty to Strength, \minus2 penalty to Dexterity, can't charge or run) for the duration of the current encounter.  If the barbarian has any temporary hit points remaining at the end of his rage, the nonlethal damage is dealt to those hit points before they go away.

\par A barbarian can fly into a rage once per day, plus an additional number of times per day equal to half his Constitution (minimum 0), to a maximum number of rages per day equal to his barbarian class level. He may not rage more than once per encounter. Entering a rage takes no time itself, but a barbarian can do it only during his action, not in response to someone else's action.

\cf{Bbn}{Fast Movement (Ex)} At 2nd level, a barbarian's land speed becomes faster than the norm for his race by \plus10 feet. This benefit applies only when he is wearing no armor, light armor, or medium  armor and not carrying a heavy load. Apply this bonus before modifying the barbarian's speed because of any load carried or armor worn. This bonus is a competence bonus.

\cf{Bbn}{Uncanny Dodge (Ex)} Starting at 2nd level, a barbarian can react to danger before his senses would normally allow him to do so. He may apply his Dexterity and dodge modifier to his armor class while flat-footed.

If a barbarian already has uncanny dodge from a different class, he stacks those levels to determine whether he gains improved uncanny dodge (see below) instead.

\cf{Bbn}{Endurance} A barbarian gains Endurance (see \pref{Endurance [General]}) as a bonus feat at 3rd level. If he already has Endurance, he may gain any other feat for which he qualifies as a bonus feat.

\cf{Bbn}{Channeled Rage} At 3rd level, a barbarian gains the ability to enter a channeled rage whenever he rages. Each channeled rage grants the barbarian additional abilities while in that rage or changes the nature of his rage.
\par A barbarian can only be in one channeled rage at a time. By spending an additional use of his rage ability, he can change which channeled rage he is in without exiting the rage, but this does not reset the duration of the rage. A barbarian chooses one channeled rage that he may enter at 3rd level, plus one every three levels thereafter.
\par Channeled rage abilities are (Ex) abilities unless otherwise noted. All bonuses granted by channeled rages apply only while the barbarian is in that channeled rage.

\subcf{Acrobatic Rage} The barbarian adds his rage bonus as a competence bonus to his Climb, Jump, and Swim checks. Additionally, he is always treated as having a running start when jumping.

\subcf{Agile Rage} The barbarian adds his rage bonus as a competence bonus to his Reflex saves.

\subcf{Endless Rage} The barbarian's rage lasts for an additional 5 rounds.

%Needs to be stronger
\subcf{Fearless Rage} The barbarian becomes immune to fear effects.

\subcf{Intimidating Rage} The barbarian adds his rage bonus as a competence bonus to his Intimidate checks. Any foe he intimidates remains shaken until the barbarian ends his rage.

\subcf{Mighty Rage} The barbarian adds his rage bonus as a competence bonus to his Strength. This replaces his competence bonus to attack and weapon damage rolls.

%\subcf{Overpowering Rage} The barbarian adds his rage bonus as a competence bonus to attack when performing all combat maneuvers.

\subcf{Spellbreaker Rage (Su)} The barbarian gains spell resistance equal to twice his rage bonus.

\subcf{Wary Rage} The barbarian does not suffer a \minus2 penalty to AC for raging.

\cf{Bbn}{Grit (Ex)} At 4th level, a barbarian's resilience allows him to shrug off magical effects. If he makes a successful Fortitude save against an attack that normally deals half damage on a successful save, he instead takes no damage.

\cf{Bbn}{Resiliency (Ex)} At 5th level, a barbarian adds half his Constitution to his damage reduction.

\cf{Bbn}{Improved Uncanny Dodge (Ex)} At 6th level and higher, a barbarian is always treated as being threatened by two fewer creatures than he actually is for the purpose of determining overwhelm penalties. This defense can deny a rogue the ability to sneak attack the barbarian.
\par If a character already has uncanny dodge (see above) from a second class and gains improved uncanny dodge, the character stacks those levels to determine if he should gain greater uncanny dodge, and to determine the minimum level a rogue must be to flank the character.

\cf{Bbn}{Larger than Life (Ex)} A barbarian of 7th level or higher holds the strength of a giant in the body of a man (or woman). The barbarian is treated as being one size category larger than he actually is for the purpose of combat maneuvers he performs or is the target of, checks that are affected by size (such as Strength checks to break down doors), and whether a creature's special attacks based on size can affect him if doing so is advantageous to him. In addition, though he uses weapons of the same size, his weapons deal damage as if they were one size category larger. The barbarian's space and reach remain those of a creature of his actual size. The benefits of this class feature stack with the effects of spells and abilities that increase the barbarian's size category.

\cf{Bbn}{Greater Uncanny Dodge (Ex)} At 10th level and higher, a barbarian no longer suffers overwhelm penalties, regardless of the number of foes surrounding him.

\cf{Bbn}{Tireless Rage (Ex)} At 11th level and higher, a barbarian no longer becomes fatigued at the end of his rage.

\cf{Bbn}{Indomitable Will (Ex)} At 13th level, a barbarian becomes immune to compulsion and domination spells and effects.

\cf{Bbn}{Improved Grit (Ex)} At 16th level, a barbarian's fortitude knows no bounds. If he fails a Fortitude save against an effect that deals half damage on a successful save, he takes only half damage.

\cf{Bbn}{Larger than Belief (Ex)} At 17th level, the barbarian's larger than life ability improves. He is treated as being two size categories larger than he actually is.

\cf{Bbn}{Deathless Rage (Ex)} At 19th level and higher, a raging barbarian can scorn death and unconsciousness. As long as his rage continues, he is not disabled at 0 hit points, and does not go unconscious at negative hit points. Even if his negative hit points below his Constitution score, the barbarian continues to fight normally until his rage ends. At that point, the effects of his wounds apply normally if they have not been healed. This ability does not prevent death from spell effects such as \spell{finger of death} or \spell{disintegrate}.

\subsubsection{Ex-Barbarians}
A barbarian who becomes lawful loses the ability to rage and cannot
gain more levels as a barbarian. He retains all the other benefits of
the class.

 \subsection{Bard}
\begin{dtable*}
\lcaption{The Bard}
\begin{tabularx}{\textwidth}{>{\ccol}p{\levelcol} >{\ccol}p{\babcolgood} *{3}{>{\ccol}p{\savecolpoof}} >{\lcol}X *{6}{>{\ccol}p{\spellcol}}}
& & & & & & \multicolumn{6}{c}{\thead{---{}---{}---Spells per Day---{}---{}---}} \\
\thead{Level} & \thead{Base Attack Bonus} & \thead{Fort Save} & \thead{Ref Save} & \thead{Will Save} & \thead{Special} & \thead{1st} & \thead{2nd} & \thead{3rd} & \thead{4th} & \thead{5th} & \thead{6th} \\
1st & \plus0 & \plus1 & \plus1 & \plus1 & Bardic music (standard), bardic knowledge, countersong, fascinate & \x & \x & \x & \x & \x & \x \\
2nd & \plus1 & \plus2 & \plus2 & \plus2 & Inspire courage 1/round \plus1d6        & 2 & \x & \x & \x & \x & \x \\
3rd & \plus2 & \plus3 & \plus3 & \plus3 & Inspire competence            & 3 & \x & \x & \x & \x & \x \\
4th & \plus3 & \plus4 & \plus4 & \plus4 & \x                    & 3 & \x & \x & \x & \x & \x \\
5th & \plus3 & \plus4 & \plus4 & \plus4 & Inspire spellpower & 4 & 2 & \x & \x & \x & \x \\
6th & \plus4 & \plus5 & \plus5 & \plus5 & Bardic music (move), inspire courage 2/round         & 4 & 3 & \x & \x & \x & \x \\
7th & \plus5 & \plus6 & \plus6 & \plus6 & Suggestion      & 4 & 3 & 1 & \x & \x & \x \\
8th & \plus6/\plus1 & \plus7 & \plus7 & \plus7 & Inspire courage \plus1d8, lyrical freedom   & 4 & 4 & 2 & \x & \x & \x \\
9th & \plus6/\plus1 & \plus7 & \plus7 & \plus7 & Inspire greatness        & 4 & 4 & 3 & \x & \x & \x \\
10th & \plus7/\plus2 & \plus8 & \plus8 & \plus8 & Combine songs & 4 & 4 & 3 & 1 & \x & \x \\
11th & \plus8/\plus3 & \plus9 & \plus9 & \plus9 & Inspire courage 3/round, song of freedom  & 4 & 4 & 4 & 2 & \x & \x \\
12th & \plus9/\plus4 & \plus10& \plus10 & \plus10& Bardic music (swift)  & 4 & 4 & 4 & 3 & \x & \x \\
13th & \plus9/\plus4 & \plus10& \plus10 & \plus10& Inspire heroics       & 4 & 4 & 4 & 3 & 1 & \x \\
14th & \plus10/\plus5 & \plus11& \plus11 & \plus11& Inspire courage \plus1d10 & 4 & 4 & 4 & 4 & 2 & \x \\
15th & \plus11/\plus6/\plus1 & \plus12& \plus12 & \plus12& Improved countersong    & 5 & 4 & 4 & 4 & 3 & \x \\
16th & \plus12/\plus7/\plus2 & \plus13& \plus13 & \plus13& Inspire courage 4/round            & 5 & 5 & 4 & 4 & 3 & 1 \\
17th & \plus12/\plus7/\plus2 & \plus13& \plus13 & \plus13& Mass suggestion & 5 & 5 & 5 & 4 & 4 & 2 \\
18th & \plus13/\plus8/\plus3 & \plus14& \plus14 & \plus14& Song of life & 5 & 5 & 5 & 5 & 4 & 3 \\
19th & \plus14/\plus9/\plus4 & \plus15& \plus15 & \plus15& Song of death  & 5 & 5 & 5 & 5 & 5 & 4 \\
20th & \plus15/\plus10/\plus5 & \plus16 & \plus16 & \plus16 & Inspire courage \plus2d6 & 5 & 5 & 5 & 5 & 5 & 5
\end{tabularx}
\end{dtable*}
\cd{Alignment} Any nonlawful.

\cd{Hit Value} 5.

\sssecfake{Class Skills}
The bard's class skills (and the key attribute for each skill) are Sleight of Hand (Dex), Forgery (Int), Knowledge (all skills, taken individually) (Int), Linguistics (Int), Perception (Wis), Sense Motive (Wis), Spellcraft (Wis), Bluff (Cha), Persuasion (Cha), Disguise (Cha), and Perform (Cha).
\cf{Brd}{Skill Points at 1st Level} 8.

\sssecfake{Class Features}

All of the following are class features of the bard.

 \cf{Brd}{Weapon and Armor Proficiency}  Bards are proficient with simple weapons,   any two other weapon groups,  light armor, medium armor, and shields (except tower shields).  They are also proficient with rapiers.

\par A bard can cast bard spells while wearing light or medium armor without incurring the normal arcane spell failure chance. However, like any other arcane spellcaster, a bard wearing heavy armor or using a shield incurs a chance of arcane spell failure if the spell in question has a somatic component (most do). A multiclass bard still incurs the normal arcane spell failure chance for arcane spells received from other classes.

\cf{Brd}{Spells} A bard casts arcane spells, which are drawn from the bard spell list. He can cast any spell he knows without preparing it ahead of time. Every bard spell has a verbal component (singing, reciting, or music). To learn or cast a spell, a bard must have a Charisma score at least equal to the level of the spell. The Difficulty Class for a saving throw against a bard's spell is 10 \add 1/2 the bard's caster level \add the bard's Charisma.

Like other spellcasters, a bard can cast only a certain number of spells of each spell level per day. His base daily spell allotment is given on \trefnp{The Bard}.

The bard's selection of spells is extremely limited. A bard begins play knowing four 0-level spells of his choice. At most new bard levels, he gains one or more new spells, as indicated on \trefnp{Bard Spells Known}.

At every bard level, a bard can choose to learn a new spell in place of one he already knows. In effect, the bard ``loses'' the old spell in exchange for the new one. The new spell's level must be the same as that of the spell being exchanged, and it must be at least two levels lower than the highest-level bard spell the bard can cast. A bard may swap only a single spell at any given level, and must choose whether or not to swap the spell at the same time that he gains new spells known for the level.

As noted above, a bard need not prepare his spells in advance. He can cast any spell he knows at any time, assuming he has not yet used up his allotment of spells per day for the spell's level.

 \par A bard may use a higher-level slot to cast a lower-level spell if he so chooses. For example, if an 8th-level bard has used up all his 2nd-level spell slots for the day but wanted to cast another one, he could use a 3rd-level slot to do so. The spell is still treated as its actual level, not the level of the slot used to cast it.

A bard's magic level is equal to his bard level.
\begin{dtable}
\lcaption{Bard Spells Known}
\begin{tabularx}{\columnwidth}{>{\centering}X *{6}{>{\ccol}p{\spellcolpoof}}}
& \multicolumn{6}{c}{\thead{---{}---{}---{}---{}---Spells Known---{}---{}---{}---{}---}} \\
\thead{Level} & \thead{1st} & \thead{2nd} & \thead{3rd} & \thead{4th} & \thead{5th} & \thead{6th} \\
1st  & \x & \x & \x & \x & \x & \x \\
2nd  & 1 & \x & \x & \x & \x & \x \\
3rd  & 2 & \x & \x & \x & \x & \x \\
4th  & 3 & \x & \x & \x & \x & \x \\
5th  & 3 & 1 & \x & \x & \x & \x \\
6th  & 3 & 2 & \x & \x & \x & \x \\
7th  & 4 & 3 & \x & \x & \x & \x \\
8th  & 4 & 3 & 1 & \x & \x & \x \\
9th  & 4 & 3 & 2 & \x & \x & \x \\
10th & 4 & 3 & 3 & \x & \x & \x \\
11th & 4 & 3 & 3 & 1 & \x & \x \\
12th & 4 & 3 & 3 & 2 & \x & \x \\
13th & 4 & 3 & 3 & 3 & \x & \x \\
14th & 4 & 3 & 3 & 3 & 1 & \x \\
15th & 4 & 3 & 3 & 3 & 2 & \x \\
16th & 4 & 3 & 3 & 3 & 3 & \x \\
17th & 4 & 3 & 3 & 3 & 3 & 1 \\
18th & 4 & 3 & 3 & 3 & 3 & 2 \\
19th & 4 & 3 & 3 & 3 & 3 & 3 \\
20th & 4 & 3 & 3 & 3 & 3 & 3 \\
\end{tabularx}
\end{dtable}

\cf{Brd}{Bardic Knowledge} A bard may make a special bardic knowledge check with a bonus equal to his bard level \add his Intelligence to see whether he knows some relevant information about local notable people, legendary items, or noteworthy places.

A successful bardic knowledge check will not reveal the powers of a magic item but may give a hint as to its general function. A bard may not take 10 or take 20 on this check; this sort of knowledge is essentially random.

\begin{dtable}
\begin{tabularx}{\columnwidth}{l X}
\thead{DC} & \thead{Type of Knowledge} \\
10  & Common, known by at least a substantial minority drinking; common legends of the local population. \\
20  & Uncommon but available, known by only a few people legends. \\
25  & Obscure, known by few, hard to come by. \\
30  & Extremely obscure, known by very few, possibly forgotten by most who once knew it, possibly known only by those who don't understand the significance of the knowledge.
\end{tabularx}
\end{dtable}

 \cf{Brd}{Bardic Music}
%Don't like ``half+half'' formulas
A number of times per day equal to half the bard's level \add half the bard's Charisma,  a bard can use his song or poetics to produce magical effects on those around him (usually including himself, if desired). While these abilities fall under the category of bardic music and the descriptions discuss singing or playing instruments, they can all be activated by reciting poetry, chanting, singing lyrical songs, singing melodies, whistling, playing an instrument, or playing an instrument in combination with some spoken performance. Each ability requires both a minimum bard level and a minimum number of ranks in the Perform skill to qualify; if a bard does not have the required number of ranks in at least one Perform skill, he does not gain the bardic music ability until he acquires the needed ranks.

\par Starting a bardic music effect is a standard action at 1st level. At 6th level, a bard can start his music as a move action, and at 12th level, a bard can start his music as a swift action. Regardless of the action used, the bard cannot initiate more than one bardic music per round (but see the combine songs ability below).  Some bardic music abilities require concentration, which means the bard must
take a standard action each round to maintain the ability. Even while using bardic music that doesn't require concentration, a bard cannot cast spells, activate magic items by spell completion (such as scrolls), or activate magic items by magic word (such as wands). Just as for casting a spell with a verbal component, a deaf bard has a 20\% chance to fail when attempting to use bardic music. If he fails, the attempt still counts against his daily limit.

\subcf{Countersong (Su)} A bard with 3 or more ranks in a Perform skill can use his music or poetics to counter magical effects that depend on sound (but not spells that simply have verbal components). Each round of the countersong, he makes a Perform check. Any creature within 30 feet of the bard (including the bard himself) that is affected by a sonic or language-dependent magical attack may use the bard's Perform check result in place of its saving throw if, after the saving throw is rolled, the Perform check result proves to be higher. If a creature within range of the countersong is already under the effect of a noninstantaneous sonic or language-dependent magical attack, it gains another saving throw against the effect each round it hears the countersong, but it must use the bard's Perform check result for the save. Countersong has no effect against effects that don't allow saves. The bard may keep up the countersong for 5 rounds.

\subcf{Fascinate (Sp)} A bard with 3 or more ranks in a Perform skill can
use his music or poetics to cause   nearby creatures  to become
fascinated with him. Each creature to be fascinated must be within
90 feet, able to see and hear the bard, and able to pay attention to
him. The bard must also be able to see the creature. The distraction
of a nearby combat or other dangers prevents the ability from
working.

\par To use the ability, a bard makes a Perform check. His check result
is the DC for each affected creature's Will save against the effect. If a
creature's saving throw succeeds, the bard cannot attempt to
fascinate that creature again for 24 hours. If its saving throw fails,
the creature sits quietly and listens to the song, taking no other
actions, for as long as the bard continues to play and concentrate (up
to a maximum of 1 round per bard level). While fascinated, a target
takes a \minus4 penalty on skill checks made as reactions, such as Perception checks. Any potential threat requires the bard to make
another Perform check and allows the creature a new saving throw
against a DC equal to the new Perform check result. Any obvious
threat, such as someone drawing a weapon, casting a spell, or aiming
a ranged weapon at the target, automatically breaks the effect. \spell{Fascinate} is an enchantment (compulsion), mind-affecting ability.

\subcf{Inspire Courage (Su)}  A bard of 2nd level or higher with 4 or more ranks in a Perform skill can use song or poetics to inspire courage in his allies (including himself ), bolstering them against fear and improving their combat abilities. To be affected, an ally must be able to hear the bard sing. The effect lasts for as long as the ally hears the bard sing and for 5 rounds thereafter. An affected ally receives a \plus1d6 enhancement bonus on a single attack roll per round. The bard grants the bonus again each round at the beginning of his turn. Each ally may choose which of their attacks receives the bonus, but the choice must be made before the attack is rolled. At 6th, 11th, and 16th level the ally may add gain the bonus on an additional attack in that round. The bonus granted increases to 1d8 at 8th level, 1d10 at 14th level, and 2d6 at 20th level. In addition, this song counters the influence of fear effects just as \emph{countersong} counters the influence of sonic effects. Inspire courage is a mind-affecting ability.

 \subcf{Inspire Competence (Su)}  A bard of 3rd level or higher with 6 or more ranks in a Perform skill can use his music or poetics to help an ally succeed at a task. The ally must be within 30 feet and able to see and hear the bard. The bard must also be able to see the ally.

The ally gets a \plus1d6 enhancement bonus on skill checks with a particular skill as long as he or she continues to hear the bard's music. The bonus granted by this ability increases at the same rate as the bonus granted by the inspire courage ability. Certain uses of this ability are infeasible. The effect lasts as long as the bard concentrates, up to a maximum of 2 minutes. A bard can't inspire competence in himself. Inspire competence is a mind-affecting ability.

 \subcf{Inspire Spellpower (Su)} A bard of 5th level or higher with 8 or more ranks in
a Perform skill can use his music or poetics to help his allies (including himself) channel their magical
energy. All allies who can hear the bard play gain a \plus1 enhancement bonus to their caster level. This bonus increases by \plus1 at 10th, 15th, and 20th level. This bonus lasts as long as the ally hears the bard sing and for 5 rounds afterwards. Inspire spellpower is a mind-affecting ability.

 \subcf{Suggestion (Sp)} A bard of 7th level or higher with 10 or more ranks in a Perform skill can make a \spell{suggestion} (as the spell) to a creature
that he has already fascinated (see above). Using this ability does not
break the bard's concentration on the \spell{fascinate} effect, nor does it
allow a second saving throw against the \spell{fascinate} effect. A Will saving throw (DC 10 \add 1/2 bard level \add bard's Cha) negates the effect. This ability affects only a single creature (but see \spell{mass suggestion}, below). \spell{Suggestion} is an enchantment (compulsion), mind-affecting, language dependent ability.

 \subcf{Inspire Greatness (Su)} A bard of 9th level or higher with 12 or more ranks
in a Perform skill can use music or poetics to inspire greatness
in his allies (including himself), granting them extra fighting capability. To be affected, an ally must be able to hear the bard sing. The effect lasts for as
long as the ally hears the bard sing and for 5 rounds thereafter. A
creature inspired with greatness gains twice the bard's level in temporary hit points and can take an extra attack when making a full attack. The attack is made using the creature's full base attack bonus, plus any modifiers appropriate to the situation. This effect is not cumulative with similar effects. Inspire greatness is a mind-affecting ability.

\subcf{Song of Freedom (Sp)} A bard of 11th level or higher with 14 or more ranks in a Perform skill can use music or poetics to create an
effect equivalent to the \spell{break enchantment} spell (caster level equals
the character's bard level). Using this ability requires 1 minute of
uninterrupted concentration and music, and it functions on a single
target within 30 feet. A bard can't use \spell{song of freedom} on himself.

 \subcf{Inspire Heroics (Su)} A bard of 13th level or higher with 16 or more ranks in a Perform skill can use music or poetics to inspire tremendous heroism in himself and all allies who can hear him, allowing them to fight bravely even against overwhelming odds. To inspire heroics, a bard must sing and the allies must hear the bard sing for a full round. A creature so inspired gains a \plus4 enhancement bonus on saving throws and dodge modifier. This bonus increases to \plus5 at 17th level. The effect lasts for as long as the ally hears the bard sing and for up to 5 rounds thereafter. Inspire heroics is a mind-affecting ability.

 \subcf{Improved Countersong (Su)} A bard of 15th level or higher with 18 or more ranks in a Perform skill can use music or poetics to free allies from all forms of mental influence.This ability functions as \emph{countersong}, except it protects against all mind-affecting effects.

\subcf{Mass Suggestion (Sp)} This ability functions like \spell{suggestion}, above,
except that a bard of   17th level or higher with 20 or more ranks in a Perform skill  can make the suggestion simultaneously to any number
of creatures that he has already fascinated (see above). Mass suggestion
is an enchantment (compulsion), mind-affecting, language-dependent ability.

 \subcf{Song of Life (Su)} A bard of 18th level or higher with 21 or more ranks in a Perform skill can use music or poetics to restore life to a recently fallen ally. A creature to be restored must be within 30 feet of the bard, and sound must be able to travel from the bard to the creature. To use this ability, the bard makes a Perform check as a full-round action against a base DC of 30. The DC increases by 2 for every round that the creature has been dead. If the bard succeeds, the creature is returned to life as if a \spell{raise dead} spell had been cast on them, except that the creature suffers no penalty for having died.

\subcf{Song of Death (Su)} A bard of 19th level with rank 22 or higher in a Perform skill can use music or poetics to cause one enemy to suffer greatly from an overwhelming outburst of joy or sorrow. To be affected, the target must be able to see and hear the bard perform for 1 full round and be within 30 feet. The bard must spend a full-round action performing. The target receives a Will save (DC 10 \add 1/2 the bard's level \add the bard's Cha) to negate the effect. If the creature's saving throw succeeds, it is bewildered for 5 rounds, and the bard cannot use song of death on that creature again for 24 hours. If a creature's saving throw fails, it dies if it is bloodied; otherwise, it is staggered for 5 rounds. Song of death is a mind-affecting death effect.

\cf{Brd}{Lyrical Freedom (Ex)} A bard of 8th level or higher can cast bard spells with verbal components and activate magic items by magic word while maintaining his bardic music ability. He still cannot activate magic items by spell completion.

\cf{Brd}{Combine Songs (Ex)} A bard of 10th level or higher can combine two types of bardic
music to provide the benefits of both. The bard chooses two music abilities and activates both
using the same action. If either or both require concentration, the bard can maintain
concentration on both by using one standard action each round to concentrate. The normal stacking
rules apply to music abilities combined with this ability.
\subsubsection{Ex-Bards}
A bard who becomes lawful in alignment cannot progress in levels as a bard, though he retains all his bard abilities.

\subsection{Cleric}
\begin{dtable*}
\lcaption{The Cleric}
\begin{tabularx}{\textwidth}{>{\ccol}p{2em} >{\ccol}p{7em} *{3}{>{\ccol}p{\savecol}} >{\lcol}X *{9}{>{\ccol}p{1.2em}}}
& & & & & & \multicolumn{9}{c}{\thead{---{}---{}---{}---{}---{}---{}---Spells per Day---{}---{}---{}---{}---{}---}} \\
\thead{Level} & \thead{Base Attack Bonus} & \thead{Fort Save} & \thead{Ref Save} & \thead{Will Save} & \thead{Special} & \thead{1st} & \thead{2nd} & \thead{3rd} & \thead{4th} & \thead{5th} & \thead{6th} & \thead{7th} & \thead{8th} & \thead{9th} \\
1st & \plus0 & \plus1 & \plus0 & \plus3 & Matters of faith, lesser domain aspect, spontaneous casting
& 3 & \x & \x & \x & \x & \x & \x & \x & \x \\
2nd & \plus1 & \plus2 & \plus1 & \plus4         & Channel energy, lesser domain aspect
& 4 & \x & \x & \x & \x & \x & \x & \x & \x \\
3rd & \plus2 & \plus3 & \plus1 & \plus5         & Domain power
& 5 & \x & \x & \x & \x & \x & \x & \x & \x \\
4th & \plus3 & \plus4 & \plus2 & \plus6         & Domain power
& 6 & 3 & \x & \x & \x & \x & \x & \x & \x \\
5th & \plus3 & \plus4 & \plus2 & \plus7         & Channelled domain power
& 6 & 4 & \x & \x & \x & \x & \x & \x & \x \\
6th & \plus4 & \plus5 & \plus3 & \plus8         & \x 
& 6 & 5 & 3 & \x & \x & \x & \x & \x & \x \\
7th & \plus5 & \plus6 & \plus3 & \plus9         & Domain aspect 
& 6 & 6 & 4 & \x & \x & \x & \x & \x & \x \\
8th & \plus6/\plus1 & \plus7 & \plus4 & \plus10    & \x 
& 6 & 6 & 5 & 3 & \x & \x & \x & \x & \x \\
9th & \plus6/\plus1 & \plus7 & \plus4 & \plus11    & Channelled domain power 
& 6 & 6 & 6 & 4 & \x & \x & \x & \x & \x \\
10th & \plus7/\plus2 & \plus8 & \plus5 & \plus12    & \x 
& 6 & 6 & 6 & 5 & 3 & \x & \x & \x & \x \\
11th & \plus8/\plus3 & \plus9 & \plus5 & \plus13   & Domain aspect 
& 6 & 6 & 6 & 6 & 4 & \x & \x & \x & \x \\
12th & \plus9/\plus4 & \plus10& \plus6 & \plus14    & \x 
& 6 & 6 & 6 & 6 & 5 & 3 & \x & \x & \x \\
13th & \plus9/\plus4 & \plus10& \plus6 & \plus15    & Greater channeled domain power 
& 6 & 6 & 6 & 6 & 6 & 4 & \x & \x & \x \\
14th & \plus10/\plus5 & \plus11& \plus7 & \plus16    & \x 
& 6 & 6 & 6 & 6 & 6 & 5 & 3 & \x & \x \\
15th & \plus11/\plus6/\plus1 & \plus12& \plus7 & \plus17 & Greater channeled domain power 
& 6 & 6 & 6 & 6 & 6 & 6 & 4 & \x & \x \\
16th & \plus12/\plus7/\plus2 & \plus13& \plus8 & \plus18 & \x 
& 6 & 6 & 6 & 6 & 6 & 6 & 5 & 3 & \x \\
17th & \plus12/\plus7/\plus2 & \plus13& \plus8 & \plus19 & Domain mastery 
& 6 & 6 & 6 & 6 & 6 & 6 & 6 & 4 & \x \\
18th & \plus13/\plus8/\plus3 & \plus14& \plus9 & \plus20 & \x 
& 6 & 6 & 6 & 6 & 6 & 6 & 6 & 5 & 3 \\
19th & \plus14/\plus9/\plus4 & \plus15& \plus9 & \plus21 & Domain mastery 
& 6 & 6 & 6 & 6 & 6 & 6 & 6 & 6 & 4 \\
20th & \plus15/\plus10/\plus5 & \plus16 & \plus10 & \plus22 & \x 
& 6 & 6 & 6 & 6 & 6 & 6 & 6 & 6 & 6 \\
\end{tabularx}
\end{dtable*}

\cd{Alignment} A cleric's alignment must be within one step of his deity's (that is, it
may be one step away on either the lawful-chaotic axis or the good-evil
axis, but not both). A cleric may not be neutral unless his deity's alignment is also neutral.

\cd{Hit Value} 5.

\sssecfake{Class Skills}
The cleric's class skills (and the key attribute for each skill) are Knowledge (arcana) (Int), Knowledge (local) (Int), Knowledge (religion) (Int), Knowledge (the planes) (Int), Heal (Wis), Sense Motive (Wis), Spellcraft (Wis), Persuasion (Cha), and Intimidate (Cha).

\cf{Clr}{Domains and Class Skills} A cleric who chooses the Plant domain adds Knowledge (nature) (Int) to the cleric class skills listed above. A cleric who chooses the Travel domain adds Survival (Wis) to the list. A cleric who chooses the Trickery domain adds Bluff (Cha) and Disguise (Cha) to the list. See Deity, Domains, and Domain Spells, below, for more information.
\cf{Clr}{Skill Points at 1st Level} 2.

 \sssecfake{Class Features}
All of the following are class features of the cleric.

 \cf{Clr}{Weapon and Armor Proficiency}  Clerics are proficient with simple weapons, any two other weapon groups, light and medium armor, and shields (except tower shields).

\cf{Clr}{Spells}  A cleric casts divine spells, which are drawn from the cleric spell list. However, his alignment may restrict him from casting certain spells opposed to his moral or ethical beliefs; see Chaotic, Evil, Good, and Lawful Spells, below. A cleric must choose and prepare his spells in advance (see below).

Most clerics use Charisma to cast spells. Clerics of specific deities may use a different casting ability, which affects all aspects of the cleric's spellcasting accordingly.

To learn or cast a spell, a cleric must have a Charisma score at least equal to the spell's level. The Difficulty Class for a saving throw against a cleric's spell is 10 \add the spell level \add the cleric's Charisma.

Like other spellcasters, a cleric can cast only a certain number of spells of each spell level per day.

\par A cleric's selection of spells is limited. A cleric begins play knowing one 1st-level spell of his choice, plus two spells chosen from the list of spells offered by his domains.

At each new cleric level, he gains one or more new spells, as indicated on \trefnp{Cleric Spells Known}. Two spells at every spell level must be drawn from the cleric's domains; sometimes these spells are normal spells on the cleric's spell list, but often they are additions to the spell list. A cleric may also choose spells from his domain lists with his normal spells known.

\par A cleric can cast any spell he knows at any time, assuming he
has not yet used up his spells per day for that spell level. For
example, at 1st level, the cleric Cadius can cast four 1st-level
spells per day. He knows four 1st-level spells: one of his choice, two from his domains, and either \spell{cure light wounds} or \spell{inflict light wounds} depending on his alignment (see \trefnp{Cleric Spells Known}). Thus, on any given day, he can cast some combination of the five spells a total of four times.

\par A cleric may use a higher-level slot to cast a lower-level spell if he so chooses. For example, if an 8th-level cleric has used up all his 3rd-level spell slots for the day but wants to cast another third level spell, he could use a 4th-level slot to do so. The spell is still treated as its actual level, not the level of the slot used to cast it.

\par At each cleric level, a cleric can choose to learn a
new spell in place of one he already knows. In effect, the cleric ``loses''
the old spell in exchange for the new one. The new spell's
level must be the same as that of the spell being exchanged, and the spells
must be of the same type; regular
spells cannot be exchanged with domain spells, and the spells a cleric knows
from his alignment cannot be exchanged in this way at all. A cleric may swap only a single
spell at any given level, and must choose whether or not to swap the
spell at the same time that he gains new spells known for the level.

Clerics meditate or pray for their spells. Each cleric must choose a time at which he must spend 1 hour each day in quiet contemplation or supplication to regain his daily allotment of spells. Time spent resting has no effect on whether a cleric can prepare spells. A cleric may prepare and cast any spell on the cleric spell list, provided that he can cast spells of that level, but he must choose which spells to prepare during his daily meditation.

A cleric's magic level is equal to his cleric level.

\begin{dtable}
\lcaption{Cleric Spells Known}
\centering
\begin{tabularx}{\columnwidth}{>{\ccol}X *{10}{>{\ccol}p{\spellcol}}}
& \multicolumn{10}{c}{\thead{---{}---{}---{}---{}---{}---{}---{}---Spells Known---{}---{}---{}---{}---{}---{}---{}---}} \\
\thead{Level} & \thead{1st} & \thead{2nd} & \thead{3rd} & \thead{4th} & \thead{5th} & \thead{6th} & \thead{7th} & \thead{8th} & \thead{9th} \\
1st  & 0\plus 2 & \x & \x & \x & \x & \x & \x & \x & \x \\
2nd  & 1\plus 2 & \x & \x & \x & \x & \x & \x & \x & \x \\
3rd  & 2\plus 2 & \x & \x & \x & \x & \x & \x & \x & \x \\
4th  & 2\plus 2 & 0\plus 2 & \x & \x & \x & \x & \x & \x & \x \\
5th  & 3\plus 2 & 1\plus 2 & \x & \x & \x & \x & \x & \x & \x \\
6th  & 3\plus 2 & 1\plus 2 & 0\plus 2 & \x & \x & \x & \x & \x & \x \\
7th  & 3\plus 2 & 2\plus 2 & 1\plus 2 & \x & \x & \x & \x & \x & \x \\
8th  & 3\plus 2 & 2\plus 2 & 1\plus 2 & 0\plus 2 & \x & \x & \x & \x & \x \\
9th  & 3\plus 2 & 2\plus 2 & 2\plus 2 & 1\plus 2 & \x & \x & \x & \x & \x \\
10th & 3\plus 2 & 2\plus 2 & 2\plus 2 & 1\plus 2 & 0\plus 2 & \x & \x & \x & \x \\
11th & 3\plus 2 & 2\plus 2 & 2\plus 2 & 2\plus 2 & 1\plus 2 & \x & \x & \x & \x \\
12th & 3\plus 2 & 2\plus 2 & 2\plus 2 & 2\plus 2 & 1\plus 2 & 0\plus 2 & \x & \x & \x \\
13th & 3\plus 2 & 2\plus 2 & 2\plus 2 & 2\plus 2 & 2\plus 2 & 1\plus 2 & \x & \x & \x \\
14th & 3\plus 2 & 2\plus 2 & 2\plus 2 & 2\plus 2 & 2\plus 2 & 1\plus 2 & 0\plus 2 & \x & \x \\
15th & 3\plus 2 & 2\plus 2 & 2\plus 2 & 2\plus 2 & 2\plus 2 & 2\plus 2 & 1\plus 2 & \x & \x \\
16th & 3\plus 2 & 2\plus 2 & 2\plus 2 & 2\plus 2 & 2\plus 2 & 2\plus 2 & 1\plus 2 & 0\plus 2 & \x \\
17th & 3\plus 2 & 2\plus 2 & 2\plus 2 & 2\plus 2 & 2\plus 2 & 2\plus 2 & 1\plus 2 & 1\plus 2 & \x \\
18th & 3\plus 2 & 2\plus 2 & 2\plus 2 & 2\plus 2 & 2\plus 2 & 2\plus 2 & 1\plus 2 & 1\plus 2 & 0\plus 2 \\
19th & 3\plus 2 & 2\plus 2 & 2\plus 2 & 2\plus 2 & 2\plus 2 & 2\plus 2 & 1\plus 2 & 1\plus 2 & 1\plus 2 \\
20th & 3\plus 2 & 2\plus 2 & 2\plus 2 & 2\plus 2 & 2\plus 2 & 2\plus 2 & 1\plus 2 & 1\plus 2 & 1\plus 2
\end{tabularx}
\end{dtable}

\begin{dtable!*}
\lcaption{Deities}
\begin{tabularx}{\textwidth}{X l X}
\thead{Deity} & \thead{Alignment} & \thead{Domains} \\
Guftas, horse god of justice & Lawful good & Animal, Law, Strength, Travel \\
Lucied, paladin god of justice & Lawful good & Destruction, Good, Leadership, War \\
Simor, fighter god of protection & Lawful good & Good, Law, Life, Protection \\
Vanya, centaur god of nature & Neutral good & Good, Plant, Strength, War \\
Brushtwig, pixie god of creativity & Chaotic good & Chaos, Dragon, Good, Trickery \\
Chavi, bard god of stories & Chaotic good & Chaos, Knowledge, Leadership, Trickery \\
Ivan Ivanovitch, bear god of strength & Chaotic good & Animal, Chaos, Strength, War \\
Raphael, monk god of retribution & Lawful neutral & Death, Law, Protection, Travel \\
%Fizzlewhim Alamuare Liet'us-Alaras Stark Eivonor
Declan, god of fire & True neutral & Destruction, Fire, Knowledge, Magic \\
%Kalten, half-orc god of smiths
Kurai, shaman god of nature & True neutral & Air, Earth, Fire, Water \\
Murdoc, bard god of mercenaries & Chaotic neutral & Destruction, Knowledge, Leadership, War\\
Daeghul, god of slaughter & Chaotic evil & Destruction, Evil, Magic, War \\
%Ribo, halfling god of trickery & Chaotic neutral & Chaos, Trickery, Water
\end{tabularx}
\end{dtable!*}
\cf{Clr}{Deity, Domains, and Domain Spells} A cleric's deity influences his alignment, what magic he can perform, his values, and how others see him. A cleric chooses two domains from among those belonging to his deity. A cleric can select an alignment domain (Chaos, Evil, Good, or Law) only if his alignment matches that domain.

If a cleric is not devoted to a particular deity, he still selects two domains to represent his spiritual inclinations and abilities. The restriction on alignment domains still applies.

\par Each domain grants a cleric access to two spells at each spell level, from 1st on up. A cleric adds all four domain spells to his spell list (if they were not already on his spell list). Each domain also has several associated powers. At 1st level, a cleric can choose one domain power from one of his two domains. Clerics gain access to more domain granted powers as they increase in level.

\par For example, Cadius is a 1st-level cleric. He chooses Good
and Law as his two domains. He gets the granted powers of one of his selected domains.
The Good domain gives him access to \spell{bless} and \spell{protection from evil} as domain spells which he can learn, and could allow him to cast all spells with the good descriptor at \plus1 caster level (as if he were one level higher as a cleric) as a granted power. The Law domain gives him access to \spell{command} and \spell{protection from chaos} as domain spells, and could allow him to cast all law spells at \plus1 caster level as a granted power. Cadius can choose either the granted power from the Good domain or the granted power from the Law domain. He can learn any two of those four spells for his domain spells known, plus any one additional spell, which he can choose from either the cleric spell list or from the remaining two spells offered by his domains. When Cadius readies his spells for the day, he gets two 1st-level spell slots for being a 1st-level cleric and one bonus 1st-level spell slot for having a high Charisma score (15).

\cf{Clr}{Spontaneous Casting} Clerics are particularly adept at channeling positive and negative energy. A cleric may choose to spontaneously cast \spellindirect{cure light wounds}{cure} spells (any spell with ``cure'' in its name), which channel positive energy to heal wounds, or \spellindirect{inflict light wounds}{inflict} spells (any spell with ``inflict'' in its name), which channel negative energy to cause wounds. This choice adds each \spellindirect{cure light wounds}{cure} or \spellindirect{inflict light wounds}{inflict} spell to the cleric's list of spells known once the cleric gains the ability to cast spells of the appropriate level.

\cf{Clr}{Chaotic, Evil, Good, and Lawful Spells} A cleric can't cast spells of an alignment opposed to his own or his deity's (if he has one). Spells associated with particular alignments are indicated by the chaos, evil, good, and law descriptors in their spell descriptions.

\cf{Clr}{Lesser Domain Aspect (Su)} A cleric's abilities are shaped by his domains. Each domain is associated with a lesser domain aspect. Lesser domain aspects are not activated. Options for domain aspects are listed below.
\subcf{Air} The cleric adds Jump to his cleric class skill list and gains electricity damage reduction equal to his cleric level.
\subcf{Chaos} Whenever the cleric rolls a 2 on a d20 roll, he immediately rerolls.
\subcf{Death} The cleric does not suffer partial effects if he makes his saving throw against death effects. For example, he is not staggered by \spell{finger of death} if he makes his Fortitude save.
\subcf{Destruction} The cleric can ignore half of the hardness of any object he damages, whether with spells or weapons.
\subcf{Earth} The cleric gains Endurance as a bonus feat.
\subcf{Evil} The cleric gains Skill Focus (Intimidate) as a bonus feat.
\subcf{Fire} The cleric gains fire damage reduction equal to twice his cleric level.
\subcf{Good} The cleric gains Skill Focus (Persuasion) as a bonus feat.
\subcf{Knowledge} The cleric adds all Knowledge skills to his cleric class skill list.
\subcf{Law} Whenever the cleric rolls a 2 on a d20 roll, he treats it as if he had rolled a 10.
\subcf{Magic} The cleric gains an additional spell slot at his highest level of spells.
\subcf{Protection} You gain either Covering Fire or Guardian as a bonus feat, as you choose.
\subcf{Strength} The cleric adds Climb, Jump, and Swim to his cleric class skill list.
\subcf{Travel} The cleric adds Knowledge (geography) and Survival to his cleric class skill list.
\subcf{Trickery} The cleric adds Bluff and Disguise to his cleric class skill list.
\subcf{Vitality} The cleric gains a \plus2 competence bonus to attack rolls with Vitalism spells.
\subcf{War} The cleric gains Weapon Focus with his deity's favored weapon group as a bonus feat, even if he doesn't meet the prerequisites. If he does not have a deity, he gains it in a weapon group related to his alignment or ideals.
\subcf{Water} The cleric adds Swim to his cleric class skill list and gains a swim speed equal to his base land speed.

A cleric gains an additional lesser domain aspect at 2nd level.

\begin{comment}
\cf{Clr}{Turn or Rebuke Undead (Su)} Any cleric, regardless of alignment, has the power to affect undead creatures by channeling the power of his faith through his holy (or unholy) symbol.

Turning or rebuking undead requires a standard action and affects all undead within a large (50 foot) radius of the cleric to flee, as if panicked. Undead receive a Will save to negate the effect. The DC for this Will save is equal to 10 \add 1/2 the cleric's turning level \add the cleric's Cha. A cleric's turning level is normally equal to his class level. Intelligent undead receive a new saving throw each round to end the effect.

A cleric that turns undead causes the affected creatures to flee for 1 minute as if panicked. If the cleric's turning level is at least twice the HD of an affected undead, it is instead destroyed utterly.

A cleric that rebukes undead causes the affected creatures to cower for 1 minute. If the cleric's turning level is at least twice the HD of an affected undead, it instead falls under the control of the cleric for one day per turning level. (Since intelligent undead receive a new saving throw each round, only unintelligent undead can be consistently controlled in this fashion.)

A good cleric (or a neutral cleric who worships a good deity) can turn or destroy undead creatures. An evil cleric (or a neutral cleric who worships an evil deity) instead rebukes or commands such creatures. A neutral cleric of a neutral deity must choose whether his turning ability functions as that of a good cleric or an evil cleric. Once this choice is made, it cannot be reversed.

A cleric may attempt to turn or rebuke undead a number of times per day
equal to 3 \add his Charisma value.
\end{comment}

 \cf{Clr}{Matters of Faith (Ex)} A cleric gains a \plus10 competence bonus to Knowledge (religion) checks made concerning his faith, such as questions about his deity or philosophy, religious rites, holy sites, and so on. Further, he is treated as being trained in Knowledge (religion) when making such checks, whether or not he actually is.

\cf{Clr}{Channel Energy (Su)} At 2nd level, by channeling the power of his faith through his holy (or unholy) symbol, a cleric can act as a powerful conduit of divine energy. He cleric must choose whether to channel positive or negative energy. Once this choice is made, it cannot be reversed. This decision also determines whether the cleric casts spontaneous \spellindirect{cure light wounds}{cure} or \spellindirect{inflict light wounds}{inflict} spells (see below).

When a cleric channels energy, it affects all creatures in a \areamed radius burst centered on him, including himself. The cleric may choose to exclude a number of creatures from the effect equal to 1 \add half his Wisdom. The amount of damage dealt (if negative energy is channeled) or healed (if positive energy is channeled) is equal to 1d6 damage per two cleric levels. Each affected creature can make a Will save to halve the damage. The DC of this save is equal to 10 \add half the cleric's level \add the cleric's Charisma.

Channeling energy is a standard action that does not provoke attacks of opportunity. A cleric can channel energy a number of times per day equal to 3 \add half his Charisma. A cleric must be able to present his holy symbol to use this ability. The abilities used 

\cf{Clr}{Domain Power (Su)} At 3rd level, a cleric gains a domain power from one of his domains. Using a domain power requires a standard action that does not provoke attacks of opportunity unless otherwise noted. All domain powers can be used at will unless otherwise noted. If a domain power allows a saving throw, the DC is equal to 10 \add half the cleric's level \add the cleric's Charisma.
\par The domain powers for each domain are described below.

\subcf{Air -- Lightning Arc} The cleric fires an arc of lightning at a creature within \rngmed range. If he hits on a ranged touch attack, the creature takes d8 electricity damage \add d8 per four cleric levels after 1st.
\subcf{Chaos -- Touch of Chaos} A creature within \rngclose range is imbued with the power of chaos for 5 rounds. Any attack roll, saving throw, or check it makes where it rolls an odd number gets a \minus2 penalty. Creatures affected by this power are treated as lawful for the purpose of the cleric's chaotic spells and effects.
\subcf{Death -- Death's Door} If the cleric succeeds on a melee touch attack, he rolls d6 per cleric level. If the total at least equals the touched creature's current hit points, it loses all of its hit points and is disabled. If the total at least equals the creature's critical damage, it is instantly slain. This is a death effect.
\subcf{Destruction -- Destructive Resonance} A touched creature or object takes d10 sonic damage \add d10 per four cleric levels after 1st.
\subcf{Earth -- Raise Earth} As a swift action, the cleric can command the earth to raise up any creature within \rngclose range, allowing it to stand from prone without taking an action. This ability only works on natural earth or stone.
\subcf{Evil -- Touch of Evil} A touched creature is sickened and is treated as good for the purpose of the cleric's evil spells and effects. This effect lasts for 5 rounds.
\subcf{Fire -- Firebolt} The cleric fires a bolt of fire at a creature within \rngclose range. If he hits on a ranged touch attack, the creature takes d8 fire damage \add d8 per four cleric levels after 1st.
\subcf{Good -- Holy Touch} A touched creature is dazzled and treated as evil for the purpose of good spells and effects that the cleric creates. The effect lasts for 5 rounds.
\subcf{Knowledge -- Minor Vision} A touched creature is granted a brief vision of the future, giving it a \plus2 enhancement bonus to attack rolls, weapon damage rolls, and checks for 1 round.
\subcf{Law -- Touch of Order} A creature within \rngclose range is imbued with the leveling power of order for 5 rounds. Any attack roll, saving throw, or check it makes with a roll of 11 or higher gets a \minus2 penalty. In addition, it is treated as being chaotic for the purpose of the cleric's lawful spells and effects.
\subcf{Magic -- Breach Defenses} The cleric fires a magical ray at a creature within \rngclose range. If he hits, the target takes d6 damage \add 1 per cleric level. In addition, it takes a \minus2 penalty to saving throws against  the cleric's spells for 1 round.
\subcf{Nature -- Wild Speech} The cleric gains wild speech, as the druid ability, with a druid level equal to half his cleric level (minimum 1). This power can be used a number of times per day equal to half the cleric's level.
\subcf{Protection -- Martyr's Touch} For the next 5 rounds, the cleric takes half of the damage the touched creature would take, as the \spell{shield other} spell. If the cleric get farther than \rngmed range from the touched creature during this time, the effect is broken.
\subcf{Strength -- Surge of Strength} The cleric gains a \plus2 enchancement bonus to Strength for 1 round. \bonusscalingdescription This power can be used a number of times per day equal to 1 \add half the cleric's Strength.
\subcf{Travel -- Free Stride} As a swift action, the cleric can gain the ability to move through difficult terrain at full speed for 1 round. This power can be used a number of times per day equal to 1 \add half the cleric's Dexterity.
\subcf{Trickery -- Liar's Boon} As a swift action, the cleric can gain a \plus3 enhancement bonus to Bluff and Disguise checks for 5 minutes. This power can be used a number of times per day equal to 1 \add half the cleric's Charisma.
\subcf{Vitality -- Vital Reach} As a swift action, the cleric can cause a dying creature within \rngmed range to stabilize or take 1 critical damage, as he desires.
\subcf{War -- Warrior's Boon} A creature touched creature gains a \plus2 enhancement bonus to attack rolls for 1 round. \bonusscalingdescription This power can be used a number of times per day equal to 1 \add half the cleric's Charisma.
\subcf{Water -- Drowning Orb} The cleric fires an orb of water at a creature within \rngclose range. If he hits on a ranged touch attack, the orb attempts to force itself into the creature's mouth and nose to drown it, dealing d8 nonlethal damage \add d8 per four cleric levels after 1st. In addition, the creature must make a Fortitude save or be forced to hold its breath for 5 rounds.

The cleric gains an additional domain power from one of his domains at 4th level.

\cf{Clr}{Channelled Domain Power (Su)} At 5th level, a cleric gains a channeled domain power from one of his domains. Unless otherwise stated, using a channelled domain power is identical to using channel energy and consumes a use of the cleric's channel energy ability. Instead of channeling positive or negative energy, the cleric instead gains the effect of the channelled domain power. If a channeled domain power deals damage, it functions like channeling negative energy unless otherwise noted. If a channeled domain power heals damage, it functions like channeling postive energy unless otherwise noted. The channeled domain powers are described below.
\subcf{Air} The cleric channels electrical energy. This deals electricity damage and heals creatures with the air subtype. A Reflex save halves the damage.
\subcf{Chaos} The cleric channels anarchic energy. Roll randomly each time this power is used to determine whether it functions as channeling negative energy or channeling positive energy, except that it always heals chaotic creatures and harms lawful creatures.
\subcf{Death} The cleric channels negative energy, except that any creatures dealt critical damage by this power are instantly killed, with no saving throw allowed. This is a death effect.
\subcf{Destruction} The cleric channels destructive energy. This deals untyped damage and allows a Fortitude save for half damage.
\subcf{Earth} The cleric channels seismic energy. This deals physical bludgeoning damage to all creatures on the ground. A Reflex save halves the damage.
\subcf{Evil} The cleric channels negative energy, except that it has no effect on evil creatures.
\subcf{Fire} The cleric channels fiery energy. This deals fire damage and heals creatures with the fire subtype. A Reflex save halves the damage.
\subcf{Good} The cleric channels positive energy, except that it has no effect on evil creatures.
\subcf{Knowledge} The cleric channels knowledge itself. This functions like channeling positive energy, except that each affected creature gains a bonus equal to half the cleric's level on the next attack roll, damage roll, saving throw, or check that it makes instead of being healed. If this bonus is not used within 5 rounds, it is wasted.
\subcf{Law} The cleric channels axiomatic energy. This deals 4 damage per two cleric levels to all creatures within a 40 ft. cube centered on the cleric. A Will save halves the damage, and it has no effect on lawful creatures.
\subcf{Magic} The cleric channels magical energy. This heals creatures who can cast spells and deals damage to creatures who cannot.
\subcf{Nature} The cleric channels positive energy, except that the cleric can decide whether it acts as positive or negative energy to animals and plants.
\subcf{Protection} The cleric grants each affected creature temporary hit points equal to half the amount that channelling positive energy would have healed. Undead gain temporary hit points as well. The temporary hit points last for 5 rounds.
\subcf{Strength} The cleric channels positive energy. Creatures at their maximum hit points after being healed gain a \plus1 enhancement bonus to Strength. This bonus increases to \plus2 at 11th level.
\subcf{Travel} The cleric channels positive energy. All healed creatures can ignore difficult terrain for 1 round.
\subcf{Trickery} The cleric channels negative energy. It deals half damage, but all damaged creatures who fail their Will saves are bewildered for 5 rounds. This is a mind-affecting effect.
\subcf{Vitality} The cleric channels energy as normal, except that he gains a \plus2 circumstance bonus to his cleric level.
\subcf{War} The cleric channels energy as normal, except that he can exclude two additional creatures from the effect.

\par The cleric gains an additional channelled domain power at 9th level.

\cf{Clr}{Domain Aspect (Su)} At 7th level, a cleric gains a domain aspect from one of his domains. Domain aspects do not require an action to activate. Options for domain aspects are listed below.

\subcf{Air -- Stormwalker} The cleric suffers no penalties for inclement weather or severe winds and takes half damage from falling damage.
\subcf{Chaos -- Fortune's Friend} Whenever the cleric rolls randomly for an effect, such as when he is affected by the \spell{confusion} spell, he may roll twice and take whichever result he prefers.
\subcf{Death -- Lifedrinker} Whenever the cleric kills a creature with a death effect other than \spell{death knell}, he automatically gains the benefits of a \spell{death knell} spell as if it was cast on the creature he killed.
\subcf{Destruction -- Swordcleaver} Whenever the cleric breaks or destroys an object with a melee attack, he may take a free melee attack on a creature adjacent to him at the same attack bonus.
\subcf{Earth -- Anchored} The cleric gains a \plus4 enhancement bonus to CMD against bull rush, overrun, and trip attempts while standing on solid ground.
\subcf{Evil -- Malevolent Magic} Good creatures take a \minus2 penalty to saving throws against the cleric's spells.
\subcf{Fire -- Flamebearer} The cleric gains Spell Focus (Fire) as a bonus feat.
\subcf{Good -- Purifying Magic} Evil creatures take a \minus2 penalty to saving throws against the cleric's spells.
\subcf{Knowledge -- Knowledge Mastery} The cleric may choose a number of Knowledge skills equal half to his Intelligence (minimum 1). He may take 10 with those skills if he is not in danger or rushed.
\subcf{Law -- Certain Triumph} Whenever the cleric would take 10, he may instead take 12, treating any roll lower than a 12 as if it had been a 12.
\subcf{Magic -- Metamagic Feat} The cleric gains a bonus metamagic feat.
\subcf{Nature -- Favored Terrain} The cleric gains a favored terrain, as the ranger class feature.
\subcf{Protection -- Faithful Shield} The cleric may maintain concentration on Abjuration (Shielding) effects as a swift action.
\subcf{Strength -- Mighty Magic} The cleric can add half his Strength to his casting attribute to meet the minimum attribute requirements to cast spells.
\subcf{Travel -- Rapid Traveller} The cleric gains a \plus10 foot competence bonus to his base land speed.
\subcf{Trickery -- }
\subcf{Vitality -- }
\subcf{War -- Weapon Specialization} The cleric gains Weapon Specialization in his deity's favored weapon group as a bonus feat. If he does not have a deity, he gains it in a weapon group related to his alignment or ideals.
\subcf{Water -- Water Breathing} The cleric may breathe and speak normally while underwater, as the \spell{water breathing} ritual. He may also pass through boggy or wet areas with no penalty to his movement speed.

\par The cleric gains an additional domain aspect from one of his domains at 11th level.

\cf{Clr}{Greater Channelled Domain Power (Su)} At 13th level, a cleric gains a greater channelled domain power from one of his domains. Using a greater channelled domain power consumes two uses of the cleric's channel energy ability. Instead of channeling positive or negative energy, the cleric instead gains the effect of the greater channelled domain power. Options for greater channeled domain powers are listed below.
\subcf{Air -- Mantle of Air} As a swift action, the cleric can surround himself in a mantle of air for 5 rounds. Thrown and projectile weapons have a 50\% chance to miss him while this effect is active. Unusually large weapons, such as a giant's boulders, may suffer a decreased miss chance as appropriate to their size.
\subcf{Chaos -- Anarchic Weapon} As a swift action, the cleric can imbue a touched weapon with the anarchic weapon property for 5 rounds.
\subcf{Death -- Channel Death} The cleric channels negative energy as the Death channelled domain power, except that any creature brought to 0 hit points by this effect immediately dies. This is a death effect.
\subcf{Destruction -- Tide of Destruction} The cleric channels destructive energy as the Destruction channelled domain power, except that any creature damaged by the effect is also filled with a destructive resonance for 5 rounds. the first time each round that each affected creature takes damage, that damage is increased by half the cleric's level. This is considered a circumstance bonus to damage.
\subcf{Earth -- Mantle of Earth} As a swift action, the cleric can surround himself in a mantle of earth for 5 rounds. He gains physical damage reduction equal to half his cleric level that is only overcome by adamantine weapons.
\subcf{Evil -- Unholy Weapon} As a swift action, the cleric can imbue a touched weapon with the unholy weapon property for 5 rounds.
\subcf{Fire -- Mantle of Fire} As a swift action, the cleric can surround himself in a mantle of fire for 5 rounds. He gains the effect of a \spell{fire shield} spell, with a caster level equal to his cleric level.
\subcf{Good -- Holy Weapon} As a swift action, the cleric can imbue a touched weapon with the holy weapon property for 5 rounds.
\subcf{Knowledge -- See the Truth} As a swift action, the cleric can gain the benefit of the \spell{true seeing} spell for 1 round.
\subcf{Law -- Axiomatic Weapon} As a swift action, the cleric can imbue a touched weapon with the axiomatic weapon property for 5 rounds.
\subcf{Life -- Persistent Life} The cleric can restore life, as the \spell{raise dead} spell, to a touched corpse that died no more than 5 rounds previously.
\subcf{Magic -- Enhance Metamagic} The cleric can use this power as part of casting a spell affected by a metamagic feat. If he does, the spell costs a spell slot of one level lower than normal, and applying the metamagic does not increase the casting time of the spell.
\subcf{Nature -- Wild Aspect} When the cleric gains this ability, he chooses one wild aspect ability, as if he were a were a druid of a level equal to his cleric level. When he uses this ability, he may assume that wild aspect. This effect lasts as long as that wild aspect would normally last.
\subcf{Protection -- Mass Sanctuary} The cleric channels protective energy as the Protection channelled domain power, except that instead of being healed, each affected creature separately gains the benefit of a \spell{sanctuary} spell for 5 rounds. The save DC of the \spell{sanctuary} is equal to 10 \add half the cleric's level \add the cleric's Charisma. If a member of the group attacks, the effect is broken for that creature, but not for the whole group.
\subcf{Strength -- Surge of Strength} As a swift action, the cleric can add his cleric level as an enhancement bonus to his Strength for a single round.
\subcf{Travel -- Uninhibited Movement} As a swift action, the cleric can gain the ability to move without provoking any attacks of opportunity for a single round.
\subcf{Trickery -- Swift Invisibility} As a swift action, the cleric can gain the benefit of the \spell{invisibility} spell for a single round.
\subcf{War -- Warmaster's Boon} The cleric can use this power as part of casting a spell that targets himself with a duration of \durshort or longer and a range greater than personal. If he uses the power, the spell affects all of his allies within an \areamed radius. However, the spell lasts for no longer than half his Charisma score in rounds.
\subcf{Water -- Aquatic Globe} The cleric creates water out of thin air in an immobile \areamed radius emanation from his location for 5 rounds. Everything within the area is treated as if it were underwater. At the end of the duration, the water evaporates, leaving no trace that it was ever there.

\par The cleric gains an additional greater channelled domain power at 15th level.

\cf{Clr}{Domain Mastery (Su)} At 17th level, a cleric gains a domain mastery from one of his domains. Options for domain masteries are listed below.

\subcf{Air -- Flight} The cleric gains a fly speed (good maneuverability) equal to his land speed. He may remain flying for up to 5 rounds at a time. After that, he must land for 1 round before he can fly again.
\subcf{Chaos -- Avatar of Luck} Once per round, the cleric can add d6 as a circumstance bonus to any attack roll or check. He may declare the use of the ability after failing the roll, but before any additional effects are resolved, potentially making it succeed where it would have failed.
\subcf{Death -- Deathfeeder} The cleric constantly radiates a \areamed radius emanation of death. Whenever a creature dies within the area, he gains the benefits of the \spell{death knell} spell as if it had been cast on the creature.
\subcf{Destruction -- Ruinbringer} The cleric's attacks and spells ignore all damage reduction and hardness (but not damage immunity).
\subcf{Earth -- Earth Glide} The cleric gains the earth glide ability, as an earth elemental.
\subcf{Evil -- Avatar of Evil} The cleric continuously gains the benefits of the \spell{protection from good} spell, with a caster level equal to his cleric level. If the effect is dispelled or suppressed, he can resume it as a swift action.
\subcf{Fire -- Flame Incarnate} The cleric gains the fire subtype, making him immune to fire but giving him a 50\% vulnerability to cold damage. When ever he uses fire spells or effects, he may freely exclude areas or creatures within the area of effect.
\subcf{Good -- Avatar of Good} The cleric continuously gains the benefits of the \spell{protection from evil} spell, with a caster level equal to his cleric level. If the effect is dispelled or suppressed, he can resume it as a swift action.
\subcf{Knowledge -- Combat Insight} The cleric gains a \plus2 circumstance bonus to attack rolls, checks, and saving throws against creatures he has identified with a successful Knowledge check.
\subcf{Law -- Avatar of Order} Once per round, the cleric can take 5 on an attack roll or check, even while stressed or distracted. He may declare the use of this ability after rolling below a 5, but before any additional effects are resolved, potentially causing it to succeed where it would have failed.
\subcf{Life -- Fountain of Life} The cleric gains fast healing 1. All of the cleric's healing spells and abilities restore critical damage as easily as if it were hit point damage.
\subcf{Magic -- Spellbreaker} The cleric gains spell resistance 10.
\subcf{Nature -- Natural Power} Whenever the cleric is in natural terrain, he gains a \plus2 enhancement bonus to caster level and the improved natural casting ability, as the druid class feature.
\subcf{Protection -- Martyr's Gift} The cleric constantly radiates a \areamed radius emanation of protective energy. Whenever a creature within the area takes damage, the cleric can choose to take half of that damage instead, as the \spell{shield other} spell.
\subcf{Strength -- Might of the Gods} The cleric gains the larger than life ability, as the barbarian class feature. In addition, he may use his Strength as his casting attribute, and to determine the saving throw DC of his channel energy abilities.
\subcf{Travel -- Perfect Stride} The cleric gains perfect stride, as the ranger class feature. He constantly acts as if he were under the effect of a \spell{freedom} spell, except that it does not allow him to act normally underwater.
\subcf{Trickery -- Exemplar of Deceit} The cleric continuously gains the benefits of the \spell{nondetection} spell, with a caster level equal to his cleric level, except that it also protects him from any spells or effects which would prevent him from lying or reveal his lies. If the effect is dispelled or suppressed, he can resume it as a swift action.
\subcf{War -- Warmaster's Favor} The cleric continuously gains the benefits of the \spell{divine favor} spell, with a caster level equal to his cleric level. If the effect is dispelled or suppressed, he can resume it as a swift action.
\subcf{Water -- Water's Flow} At any time, the cleric can transform into a rushing flow of water as a move action that does not provoke attacks of opportunity. As part of the action, he may move up to his movement speed in any direction that water could go. His speed is halved when moving uphill and doubled when moving downhill. He does not provoke attacks of opportunity during this movement, and has physical damage reduction 10 while in this form. At the end of his movement, he regains his normal form.

\par The cleric gains an additional domain mastery at 19th level.

\cf{Clr}{Bonus Languages} A cleric's bonus language options include
Celestial, Abyssal, and Infernal (the languages of good, chaotic evil,
and lawful evil outsiders, respectively). These choices are in addition to the bonus languages available to the character because of his race.

\subsubsection{Ex-Clerics}
A cleric who grossly violates the code of conduct required by his god loses all spells and class features, except for armor and shield proficiencies and proficiency with simple weapons. He cannot thereafter gain levels as a cleric of that god until he atones (see the \spell{atonement} spell description).

 \subsection{Druid}
\begin{dtable*}
\lcaption{The Druid}
\begin{tabularx}{\textwidth}{>{\ccol}p{\levelcol} >{\centering}p{\babcolavg} *{3}{>{\ccol}p{\savecol}} >{\ccol}X *{9}{>{\ccol}p{\spellcol}}}
& & & & & & \multicolumn{9}{c}{\thead{---{}---{}---{}---{}---{}---{}---Spells per Day---{}---{}---{}---{}---{}---}} \\
\thead{Level} & \thead{Base Attack Bonus} & \thead{Fort Save} & \thead{Ref Save} & \thead{Will Save} & \thead{Special} & \thead{1st} & \thead{2nd} & \thead{3rd} & \thead{4th} & \thead{5th} & \thead{6th} & \thead{7th} & \thead{8th} & \thead{9th} \\
1st & \plus0 & \plus3 & \plus0 & \plus1 & Nature sense, natural casting, wild speech
& 3 & \x & \x & \x & \x & \x & \x & \x & \x \\
2nd & \plus1 & \plus4 & \plus1 & \plus2 & Woodland stride
& 4 & \x & \x & \x & \x & \x & \x & \x & \x \\
3rd & \plus2 & \plus5 & \plus1 & \plus3 & Wild aspect
& 5 & \x & \x & \x & \x & \x & \x & \x & \x \\
4th & \plus3 & \plus6 & \plus2 & \plus4 & Wild speech (plants)
& 6 & 3 & \x & \x & \x & \x & \x & \x & \x \\
5th & \plus3 & \plus7 & \plus2 & \plus4 & Wild aspect
& 6 & 4 & \x & \x & \x & \x & \x & \x & \x \\
6th & \plus4 & \plus8 & \plus3 & \plus5 & Venom immunity
& 6 & 5 & 3 & \x & \x & \x & \x & \x & \x \\
7th & \plus5 & \plus9 & \plus3 & \plus6 & Wild aspect
& 6 & 6 & 4 & \x & \x & \x & \x & \x & \x \\
8th & \plus6/\plus1 & \plus10& \plus4 & \plus7 & Improved wild speech
& 6 & 6 & 5 & 3 & \x & \x & \x & \x & \x \\
9th & \plus6/\plus1 & \plus11& \plus4 & \plus7 & Wild aspect
& 6 & 6 & 6 & 4 & \x & \x & \x & \x & \x \\
10th & \plus7/\plus2 & \plus12& \plus5 & \plus8 & Improved natural casting
& 6 & 6 & 6 & 5 & 3 & \x & \x & \x & \x \\
11th & \plus8/\plus3 & \plus13 & \plus5 & \plus9  & Greater wild aspect
& 6 & 6 & 6 & 6 & 4 & \x & \x & \x & \x \\
12th & \plus9/\plus4 & \plus14 & \plus6 & \plus10 & A thousand faces
& 6 & 6 & 6 & 6 & 5 & 3 & \x & \x & \x \\
13th & \plus9/\plus4 & \plus15 & \plus6 & \plus10 & Greater wild aspect
& 6 & 6 & 6 & 6 & 6 & 4 & \x & \x & \x \\
14th & \plus10/\plus5 & \plus16 & \plus7 & \plus11 & Timeless body
& 6 & 6 & 6 & 6 & 6 & 5 & 3 & \x & \x \\
15th & \plus11/\plus6/\plus1 & \plus17 & \plus7 & \plus12 & Greater wild aspect
& 6 & 6 & 6 & 6 & 6 & 6 & 4 & \x & \x \\
16th & \plus12/\plus7/\plus2 & \plus18 & \plus8 & \plus13 & Greater wild speech
& 6 & 6 & 6 & 6 & 6 & 6 & 5 & 3 & \x \\
17th & \plus12/\plus7/\plus2 & \plus19 & \plus8 & \plus13 & Greater wild aspect
& 6 & 6 & 6 & 6 & 6 & 6 & 6 & 4 & \x \\
18th & \plus13/\plus8/\plus3 & \plus20 & \plus9 & \plus14 & Totem aspect
& 6 & 6 & 6 & 6 & 6 & 6 & 6 & 5 & 3 \\
19th & \plus14/\plus9/\plus4 & \plus21 & \plus9 & \plus15 & Greater wild aspect
& 6 & 6 & 6 & 6 & 6 & 6 & 6 & 6 & 4 \\
20th & \plus15/\plus10/\plus5 & \plus22 & \plus10 & \plus16 & Greater natural casting
& 6 & 6 & 6 & 6 & 6 & 6 & 6 & 6 & 6 \\
\end{tabularx}
\end{dtable*}
\cd{Alignment} Neutral good, lawful neutral, neutral, chaotic
neutral, or neutral evil.
\cd{Hit Value} 5.
 \sssecfake{Class Skills}
The druid's class skills (and the key attribute for each skill) are Climb (Str), Jump (Str), Swim (Str), Ride (Dex), Stealth (Dex), Knowledge (geography), Knowledge (nature) (Int), Heal (Wis), Perception (Wis), Survival (Wis), and Creature Handling (Cha).
\cf{Drd}{Skill Points at 1st Level} 4

\sssecfake{Class Features}
All of the following are class features of the druid.

\cf{Drd}{Weapon and Armor Proficiency} Druids are proficient with simple weapons, any one other weapon group, scimitars, sickles, and slings.
\par Druids are proficient with light and medium armor, but are prohibited from wearing
metal armor; thus, they may wear only padded, leather, or hide armor. (A druid may also
wear wooden armor that has been altered by the \spell{ironwood} spell so that it
functions as though it were steel. See the \spell{ironwood} spell description, \pref{Ironwood}) Druids are proficient with shields (except tower shields) but must use only wooden ones.
\par A druid who wears prohibited armor or carries a prohibited shield is unable to cast druid spells or use any of her supernatural class abilities while doing so and for 24 hours thereafter.

 \cf{Drd}{Spells}  A druid casts divine spells, which are drawn from the druid spell list. Her alignment may restrict her from casting certain spells opposed to her moral or ethical beliefs; see Chaotic, Evil, Good, and Lawful Spells, below.

\par To learn or cast a spell, a druid must have a Wisdom score at least equal to the spell's level. The Difficulty Class for a saving throw against a druid's spell is 10 \add the spell level \add the druid's Wisdom.

\par Like other spellcasters, a druid can cast only a certain number of spells of each spell level per day. Her base daily spell allotment is
given on \trefnp{The Druid}.

 \par A druid's selection of spells is limited. A druid
begins play knowing three 0-level spells (also called orisons) and two
1st-level spells of the druid's choice. At each new druid level, she gains
one or more new spells, as indicated on \trefnp{Druid Spells Known}.

\par A druid learns and casts spells the way a cleric does, though she
does not have any domains (but see Spontaneous Casting, below). A druid may cast any spell she knows
on the druid spell list (\pref{Druid Spells}).

A druid's magic level is equal to her druid level.
\begin{dtable}
\lcaption{Druid Spells Known}
\centering
\begin{tabularx}{\columnwidth}{X *{10}{p{1.1em}}}
& \multicolumn{10}{c}{\thead{---{}---{}---{}---{}---{}---{}---Spells Known---{}---{}---{}---{}---{}---{}---}} \\
\thead{Level} & \thead{1st} & \thead{2nd} & \thead{3rd} & \thead{4th} & \thead{5th} & \thead{6th} & \thead{7th} & \thead{8th} & \thead{9th} \\
1st  & 1 & \x & \x & \x & \x & \x & \x & \x & \x \\
2nd  & 2 & \x & \x & \x & \x & \x & \x & \x & \x \\
3rd  & 3 & \x & \x & \x & \x & \x & \x & \x & \x \\
4th  & 3 & 1 & \x & \x & \x & \x & \x & \x & \x \\
5th  & 4 & 2 & \x & \x & \x & \x & \x & \x & \x \\
6th  & 4 & 2 & 1 & \x & \x & \x & \x & \x & \x \\
7th  & 4 & 3 & 2 & \x & \x & \x & \x & \x & \x \\
8th  & 4 & 3 & 2 & 1 & \x & \x & \x & \x & \x \\
9th  & 4 & 3 & 3 & 2 & \x & \x & \x & \x & \x \\
10th & 4 & 3 & 3 & 2 & 1 & \x & \x & \x & \x \\
11th & 4 & 3 & 3 & 3 & 2 & \x & \x & \x & \x \\
12th & 4 & 3 & 3 & 3 & 2 & 1 & \x & \x & \x \\
13th & 4 & 3 & 3 & 3 & 3 & 2 & \x & \x & \x \\
14th & 4 & 3 & 3 & 3 & 3 & 2 & 1 & \x & \x \\
15th & 4 & 3 & 3 & 3 & 3 & 3 & 2 & \x & \x \\
16th & 4 & 3 & 3 & 3 & 3 & 3 & 2 & 1 & \x \\
17th & 4 & 3 & 3 & 3 & 3 & 3 & 2 & 2 & \x \\
18th & 4 & 3 & 3 & 3 & 3 & 3 & 2 & 2 & 1 \\
19th & 4 & 3 & 3 & 3 & 3 & 3 & 2 & 2 & 2 \\
20th & 4 & 3 & 3 & 3 & 3 & 3 & 2 & 2 & 2
\end{tabularx}
\end{dtable}

%\cf{Drd}{Spontaneous Casting}  A druid can always channel stored spell energy into summoning spells. She can spend any spell slot in order to cast any \spell{summon nature's ally} spell of the same level or lower.

\cf{Drd}{Natural Casting (Ex)} A druid's spells channel of nature itself. Though her energy is a necessary component to bring natural power to bear, she need not be the focus of its might. Whenever she casts a druid area spell that would emanate from her, such as a cone or line spell, she may cause the spell to originate from any position within 10 feet of her. All other aspects of the spell are unchanged.

For example, a druid casting \spell{gust of wind} could create a line of wind originating from 10 feet to her right. The line would extend 50 feet out from that point, as normal. If the druid cause the line of wind to blow to the left, she could potentially be affected by the wind.

\cf{Drd}{Bonus Languages} A druid's bonus language options include Sylvan, the language of magical woodland creatures. This choice is in addition to the bonus languages available to the character because of her race.

A druid also knows Druidic, a secret language known only to druids, which she learns upon becoming a 1st-level druid. Druidic is a free language for a druid; that is, she knows it in addition to her regular allotment of languages and it doesn't take up a language slot. Druids are forbidden to teach this language to nondruids.

Druidic has its own alphabet.

\cf{Drd}{Nature Sense (Ex)} A druid gains a \plus2 competence bonus on Knowledge (nature) and Survival checks. In addition, she can make those checks as if she were trained.

\cf{Drd}{Wild Speech (Su)} One of the first lessons a druid learns is how to commune with natural creatures. A druid can speak with animals a number of times per day equal to half her druid level \add her Charisma (minimum 1). Each time she uses this ability, she chooses a kind of animal, such as owl or wolf. She can then speak to and understand animals of that type for a number of minutes equal to her druid level.

This ability doesn't make the animals any more friendly or cooperative than normal. Furthermore, wary and cunning animals are likely to be terse and evasive, while the more stupid ones make inane comments. If an animal is friendly toward the druid, she may be able to convince it to do some favor or service.

\cf{Drd}{Woodland Stride (Ex)} Starting at 2nd level, a druid may move through any sort of undergrowth (such as natural thorns, briars, overgrown areas, and similar terrain) at her normal speed and without taking damage or suffering any other impairment. However, thorns, briars, and overgrown areas that have been magically manipulated to impede motion still affect her.

 \cf{Drd}{Wild Aspect (Su)} At 3rd level, a druid gains the ability to embody an aspect of an animal. Embodying a wild aspect is a standard action and doesn't provoke an attack of opportunity. Wild aspects last for 10 minutes per druid level, or until the druid dismisses the effect, unless otherwise stated. The druid can use this ability a number of times per day equal to half her druid class level \add her Constitution.
\par When a druid embodies a wild aspect, she gains all abilities from an animal that she knows. For example, a druid devoted to bulls, who chose the Lope and Gore aspects, would gain both abilities with a single use of wild aspect. A druid devoted do both bulls and eagles, who chose the Lesser Flight aspect from the eagle and the Gore aspect from the bull, would have to use two uses of wild aspect to use both abilities.
\par The descriptions below describe the effects of the aspect. With many aspects, the druid's appearance also changes to match the aspect, but this is not described. Different druids change in different ways. For example, one druid might gain unusually large eyes when embodying the low-light vision aspect, while another might change her irises into slits, like a cat, when embodying the same aspect. The changes made are up to the druid, but cannot be used to gain an additional substantive benefit beyond the effects given in the description of the aspect.
\par When she initially chooses a wild aspect, a druid can only choose from the first aspects listed for each animal. At 5th, 7th, and 9th level, a druid learns how to take on a new wild aspect. At 5th level, she can choose from the first or seconds aspects listed. At 7th level, she can also choose from the third aspect listed, and at 9th level, she may choose any aspect listed for any animal. At higher levels, a druid may choose a greater aspect instead; see below.
\par Each wild aspect also grants an ability based on the number of abilities the druid has from that aspect. These bonuses are only granted when the druid has taken on that wild aspect.

\subcf{Animal}
While embodying the animal aspect, the druid gains an enhancement bonus to any attribute score of her choice equal to the number of animal abilities she possesses.
\begin{wildaspect}
\wilditem Low-Light Vision: The druid gains low-light vision.
\wilditem Scent: The druid gains the scent ability.
\wilditem Natural Attunement: The druid gains a \plus4 enhancement bonus to Creature Handling, Ride, and Survival checks.
\wilditem Animal Affinity: When the druid first acquires this ability, she may choose any wild aspect for which she qualifies. She may learn it and treat it as if it were an animal aspect.
\end{wildaspect}

\subcf{Ape}
While embodying the ape aspect, the druid gains an enhancement bonus to Strength equal to the number of ape abilities she possesses.
\begin{wildaspect}
\wilditem Climb: The druid gains a climb speed equal to her base land speed.
\wilditem Claws: The druid gains two claw attacks which can be used as light natural weapons. A Medium druid deals d6 damage with each claw.
\wilditem Rend: If the druid hits with both claw attacks, she latches on to her opponent's body and tears the flesh. This attack deals damage appropriate to a scimitar appropriate for the druid's size.
\wilditem Improved Grab: When the druid hits a foe with an unarmed strike or natural attack, she may attempt to grapple her foe as a swift action without provoking an attack of opportunity.
\end{wildaspect}

\subcf{Bear}
While embodying the bear aspect, the druid gains an enhancement bonus to Constitution equal to the number of bear abilities she possesses.
\begin{wildaspect}
\wilditem Bite: The druid gains a bite attack which can be used as a heavy natural weapon. A Medium druid deals d8 damage with the bite.
\wilditem Claws: The druid gains two claw attacks which can be used as light natural weapons. A Medium druid deals d6 damage with each claw.
\wilditem
\wilditem Improved Grab: When the druid hits a foe with an unarmed strike or natural attack, she may attempt to grapple her foe as a swift action without provoking an attack of opportunity.
\end{wildaspect}

\subcf{Bull}
While embodying the bull aspect, the druid gains an enhancement bonus to Strength equal to the number of bull abilities she possesses.
\begin{wildaspect}
\wilditem Lope: The druid gains the ability to move on all four legs. When doing so, she gains a \plus20 foot competence bonus to her speed, but she cannot use her hands for anything except moving. When not moving on four legs, her ability to use her hands is unchanged.
\wilditem Gore: The druid gains a gore attack which can be used as a heavy natural weapon. A Medium druid deals d8 damage with a gore.
\wilditem Rush: When the druid hits a foe with an unarmed strike or natural attack on a charge, she may attempt to bull rush her foe as a swift action without provoking an attack of opportunity.
\wilditem
\end{wildaspect}

\subcf{Cat}
While embodying the cat aspect, the druid gains an enhancement bonus to Dexterity equal to the number of cat abilities she possesses.
\begin{wildaspect}
\wilditem Lope: The druid gains the ability to move on all four legs. When doing so, she gains a \plus20 foot competence bonus to her speed, but she cannot use her hands for anything except moving. When not moving on four legs, her ability to use her hands is unchanged.
\wilditem Bite: The druid gains a bite attack which can be used as a heavy natural weapon. A Medium druid deals d8 damage with a bite.
\wilditem Stealth: The druid gains a \plus4 enhancement bonus to Stealth checks.
\wilditem Pounce: During the first round of combat, the druid can make a full attack after charging. She gains no bonus to attack rolls from charging, but still takes the normal penalty to AC if applicable.
\end{wildaspect}

\subcf{Eagle}
While embodying the eagle aspect, the druid gains an enhancement bonus to Charisma equal to the number of eagle abilities she possesses.
\begin{wildaspect}
\wilditem Wings, partial: The druid gains a glide speed equal to her base land speed. While gliding, she cannot use her hands for anything except moving.
\wilditem Talons: The druid gains talons which can be used as a heavy natural weapon. A Medium druid deals d8 damage with her talons.
\wilditem Dive: When the druid hits a foe with an unarmed strike or natural attack on a charge while gliding or flying down, she deals double damage. If she can make multiple attacks on the charge, this effect only applies to the first attack.
\wilditem Wings, full: The druid gains a fly speed equal to her base land speed with average maneuverability. While flying she cannot use her hands for anything except moving. She can only fly for a number of rounds equal 3 \add half her Constitution. After that limit is reached, she must rest for 5 minutes to recuperate.
\end{wildaspect}

\subcf{Fox}
While embodying the fox aspect, the druid gains an enhancement bonus to Intelligence equal to the number of fox abilities she possesses.
\begin{wildaspect}
\wilditem Lope: The druid gains the ability to move on all four legs. When doing so, she gains a \plus20 foot competence bonus to her speed, but she cannot use her hands for anything except moving. When not moving on four legs, her ability to use her hands is unchanged.
\wilditem Bite: The druid gains a bite attack which can be used as a primary natural weapon. A Medium druid deals d8 damage with a bite.
\wilditem Stealth: The druid gains a \plus4 enhancement bonus on Stealth checks.
\wilditem
\end{wildaspect}

\subcf{Owl}
For each owl ability she possesses, the druid gains a \plus1 enhancement bonus to Wisdom  while embodying the owl aspect.
\begin{wildaspect}
\wilditem Wings, partial: The druid gains a glide speed equal to her base land speed. While gliding, she cannot use her hands for anything except moving.
\wilditem Talons: The druid gains talons which can be used as a heavy natural weapon. A Medium druid deals d8 damage with her talons.
\wilditem Senses: The druid gains a \plus4 enhancement bonus on Perception checks.
\wilditem Wings, full: The druid gains a fly speed equal to her base land speed with average maneuverability. While flying, she cannot use her hands for anything except moving. She can fly for a number of rounds equal to 3 \add half her Constitution. After that, she must rest for 5 minutes to recuperate.
\end{wildaspect}

\subcf{Serpent}
While embodying the serpent aspect, the druid gains an enhancement bonus to grapple attacks and saving throw DCs with any poison-based ability she uses equal to the number of serpent abilities she possesses.
\begin{wildaspect}
\wilditem Slither: The druid gains a climb speed equal to half her base land speed. She does not need to use her hands to climb in this way.
\wilditem Bite: The druid gains a bite attack which can be used as a heavy natural weapon. A Medium druid deals d8 damage with a bite.
\wilditem Constrict: After making a successful grapple attack to grapple or damage her foe, the druid may constrict her foe as a swift action. A Medium druid deals d8 \add 1-1/2 her Strength when constricting.
\wilditem Venom: The druid's natural attacks, and any weapons she wields, become coated in poison. The poison deals initial and secondary damage of 1d4 Constitution damage. A Fortitude save (DC 10 \add 1/2 the druid's level \add the druid's Con) negates the damage as normal for poison. This ability lasts for one round per druid level.
\end{wildaspect}

\subcf{Wolf}
While embodying the wolf aspect, the druid gains an circumstance bonus to weapon damage against overwhelmed foes and an enhancement bonus to trip attacks equal to the number of wolf abilities she possesses.
\begin{wildaspect}
\wilditem Lope: The druid gains the ability to move on all four legs. When doing so, she gains a \plus20 foot competence bonus to her speed, but she cannot use her hands for anything except moving. When not moving on four legs, her ability to use her hands is unchanged.
\wilditem Bite: The druid gains a bite attack which can be used as a heavy natural weapon. A Medium druid deals d8 damage with a bite.
\wilditem Trip: When the druid hits a foe with an unarmed strike or natural attack, she may attempt to trip her foe as a swift action without provoking an attack of opportunity.
\wilditem Wolfpack: Foes overwhelmed by the druid increase their overwhelm penalties by 1.
\end{wildaspect}

%\cf{Drd}{Animal Affinity (Ex)} At 4th level, a druid may choose to be treated as an animal for the purpose of spells or abilities which only affect animals, such as \spell{reduce animal}. This does not prevent her from being affected by spells or abilities which would normally affect her.

\cf{Drd}{Venom Immunity (Ex)} At 4th level, a druid gains immunity to all poisons.

\cd{Drd}{Wild Speech (Plants) (Ex)} At 6th level, a druid can also converse with plants and plant creatures using her wild speech ability. A regular plant's sense of its surroundings is limited, so it won't be able to give (or recognize) detailed descriptions of creatures or answer questions about events outside its immediate vicinity.

\cf{Drd}{Improved Wild Speech (Su)} At 8th level, anything the druid speaks with with using her wild speech ability must make a Will save to avoid being charmed, as the \spell{charm person} spell, by the druid. The effect lasts for the duration of the conversation, and for 1 hour thereafter. This ability is not mind-affecting, and can affect creatures or even objects of any kind that the druid can converse with. Objects are always considered to fail their Will save.  The druid can choose not to exert this influence.

\cf{Drd}{Improved Natural Casting (Ex)} At 10th level, a druid can cause area spells to originate from up to \rngclose range away from her, as the natural casting ability.

\cf{Drd}{Greater Wild Aspect (Su)} At 11th level, a druid gains the ability to assume aspects of the natural world, including the elements, in addition to those of animals. This ability functions as wild aspect, and assuming a greater wild aspect consumes a use of wild aspect, but the druid may choose from a different list of abilities. Unlike with wild aspects, greater wild aspects must be learned in order within an aspect; a druid cannot gain the air mantle aspect unless she has the endless air aspect. A druid can suppress or resume any greater wild aspect ability as a swift action.

Whenever the druid learns a greater wild aspect, she may choose to learn a wild aspect instead.
\subcf{Air}
\begin{greaterwildaspect}
\wilditem Profusion of Air: The druid constantly exudes good, clean air. She can breathe in any environment, and is immune to \spell{sickening cloud} and similar effects. In addition, she may use her wild speech ability to speak with any natural air.
\wilditem Air Mantle: The druid is surrounded by a mantle of air. Thrown and projectile weapons have a 50\% chance to miss her while this effect is active. Unusually large weapons, such as a giant's boulders, may suffer a decreased miss chance as appropriate to their size.
\wilditemplus Flight: The druid gains a fly speed equal to her land speed, with good maneuverability. She may remain flying for up to 5 rounds at a time. After that, she must land for 1 round before she can fly again.
\end{greaterwildaspect}

\subcf{Earth}
\begin{greaterwildaspect}
\wilditem Earthen Profusion: The druid constantly exudes fresh, solid earth wherever she steps. She gains a \plus4 enhancement bonus to CMD and Fortitude saves against effects that would move her. In addition, she may use her wild speech ability to speak with any natural earth or stone.
\wilditem Earth Mantle: The druid is surrounded by a mantle of earth. She gains a \plus1 enhancement bonus to natural armor class per four druid levels and damage reduction 5/adamantine.
\wilditemplus Earth Glide: The druid gains the earth glide ability, as an earth elemental. She may remain partly within the earth while fighting, granting her cover at no penalty to her own actions.
\end{greaterwildaspect}

\subcf{Fire}
\begin{greaterwildaspect}
\wilditem Flaming Profusion: Wherever the druid moves, she leaves a path of burning flame behind her that lasts until the end of her next turn. A creature who crosses the path takes 1d6 points of fire damage per two druid levels. It can make a Reflex save to halve the damage, with a DC equal to 10 \add half the druid's level \add the druid's Charisma. In addition, the druid can her her wild speech ability to speak with any natural fire.
\wilditem Fire Mantle: The druid is surrounded by a mantle of fire. This functions as a warm \spell{fire shield} spell, with a caster level equal to the druid's level.
\wilditemplus Immolation: The druid gains the fire subtype, making her immune to fire but giving her a 50\% vulnerability to cold damage. Whenever she deals fire damage to a creature, she ignites the creature for 5 rounds. An ignited creature takes a \minus2 penalty to attack rolls, saving throws, checks, and AC, and takes d6 damage per round from the fire. If the creature takes a full-round action, it can attempt a Reflex save to put out the fire with a DC of 10 \add half the druid's level \add the druid's Charisma. A creature hit by the druid's fire multiple times is not ignited multiple times; only the most recent effect is used.
\end{greaterwildaspect}

\subcf{Plant}
\begin{greaterwildaspect}
\wilditem Profusion of Life: Wherever the druid moves, she leaves a path of small, living plants that entangle foes until the end of her next turn. A creature crossing the path must make a Reflex save with a DC equal to 10 \add half the druid's level \add the druid's Charisma or be entangled by the plants and unable to complete the movement. The plants appear on any surface, and will continue to grow if they can survive, though they may die quickly if they appear on inhospitable terrain.
\wilditem Plant Body: The druid's body takes on plantlike characteristics. She gains a \plus1 enhancement bonus to natural armor class per four druid levels and has a 50\% chance to ignore critical hits and sneak attacks.
\wilditemplus Rejuvenation: The druid gains fast healing 5 as long as she remains in sunlight or touches a plant of her size or larger.
\end{greaterwildaspect}

\subcf{Sun}
\begin{greaterwildaspect}
\wilditem Profusion of Light: The druid constantly radiates light in a \arealarge radius. This is treated as true sunlight, not ordinary magical light. Creatures and objects vulnerable to sunlight must make a Fortitude save every round to resist the effect on themselves, with a DC equal to 10 \add half the druid's level \add the druid's Charisma.
\wilditem Mantle of Light: The druid glows so brightly that she becomes hard to look at. She gains concealment against all attacks, and any creature attacking her from within the radius of her profusion of light is dazzled for 5 rounds after the attack (no save). She cannot use this ability while suppressing her profusion of light ability, and she cannot use this concealment to hide.
\wilditemplus Piercing Radiance: The druid's illumination radius with her profusion of light ability increases to 100 feet. All visual illusions and shadow effects within the radius are suppressed except those that the druid chooses to allow. Any creature within the radius who attacks the druid is blinded for 1 round after the attack. A successful Fortitude save with a DC of 10 \add half the druid's level \add the druid's Charisma prevents the creature from being blinded for the next round.
\end{greaterwildaspect}

\subcf{Water}
\begin{greaterwildaspect}
\wilditem Aqueous Profusion: Wherever the druid moves, she leaves a path of animated water that can grab creatures and cause them to trip. A creature crossing the path must make a Reflex save with a DC equal to 10 \add half the druid's level \add the druid's Charisma or fall prone and be unable to complete the movement. In addition, the druid can speak with natural water using her wild speech ability.
\wilditem Watery Mantle: The druid becomes surrounded by an animate mantle of water that reaches out to deflect incoming blows. She gains a \plus10 enhancement bonus to resist grapple attacks and a \plus5 shield bonus.
\wilditemplus Water's Flow: At any time during the duration of this aspect, the druid can transform into a rushing flow of water as a move action that does not provoke attacks of opportunity. As part of the action, she may move up to her movement speed in any direction that water could go. Her speed is halved when moving uphill and doubled when moving downhill. She does not provoke attacks of opportunity during this movement, and has physical damage reduction 10 while in this form. At the end of her movement, she regains her normal form.
\end{greaterwildaspect}

\cf{Drd}{A Thousand Faces (Su)} At 12th level, a druid gains the ability to change
her appearance at will, as if using the \spell{disguise self} spell. This affects the druid's body but not her possessions. It is not an illusory effect, but a minor physical alteration of the druid's appearance, within the limits described for the spell.

\cf{Drd}{Timeless Body (Ex)} After attaining 15th level, a druid no longer takes attribute score penalties for aging and cannot be magically aged. Any penalties she may have already incurred, however, remain in place. Bonuses still accrue, and the druid still dies of old age when her time is up.

\cf{Drd}{Greater Wild Speech (Ex)} At 16th level, the druid can use her wild speech ability to control creatures' actions. If the druid converses with a creature using her wild speech ability, she may spend a standard action and an additional wild speech use to dominate one creature she is speaking with, as the \spell{dominate person} spell. A successful Will save negates this effect. This ability is not mind-affecting, and can affect creatures of any kind that the druid can converse with.

\cf{Drd}{Totem Aspect (Su)} At 18th level, the druid can choose any one wild aspect (but not greater aspect). She permanently gains the abilities of that aspect, as if she was constantly manifesting it. She may suppress or resume this effect as a swift action. If the druid has multiple abilites from that aspect, she may suppress or resume them each individually.

\cf{Drd}{Greater Natural Casting (Ex)} At 20th level, a druid may cause area spells to originate from any point within \rngmed range of her, as the natural casting ability.

\begin{comment}
%Animal companions are complicated and unnecessary. They are going away. They can be regained with an alternate class feature.

 \cf{Drd}{Animal Companion (Ex)}  A druid may begin play with an animal companion selected from the following list: badger, camel, dire rat, dog, riding dog, eagle, hawk, horse (light or heavy), owl, pony, snake (Small or Medium viper), or wolf. If the campaign takes place wholly or partly in an aquatic environment, the following creatures are also available: crocodile, porpoise, Medium shark, and squid. This animal is a loyal companion that accompanies the druid on her adventures as appropriate for its kind.

\par A 1st-level druid's companion is completely typical for its kind except as noted below. As a druid advances in level, the animal's power increases as shown on the table. If a druid releases her companion from service, she may gain a new one by performing a ceremony requiring 24 uninterrupted hours of prayer. This ceremony can also replace an animal companion that has perished.

\begin{dtable!*}
\lcaption{Animal Companion Base Statistics}
\begin{tabularx}{\textwidth}{p{\levelcol} p{2.5em} p{3em} *{3}{p{\savecol}} p{\savecol} p{4em} p{3em} p{4.5em} p{3em} p{3em} X}
Class Level & Hit Value & Base Attack Bonus & Fort Save & Ref Save & Will Save & Feats & Natural Armor Bonus & Ability (Good) & Ability (Average) & Ability (Poor) & Bonus Tricks & Special \\
1st & 1 & \plus0 & \plus2 & \plus2 & \plus0 & 1 & \plus0 & 14 & 12 & 8 & 1 & Link, share spells \\
2nd & 2 & \plus1 & \plus3 & \plus3 & \plus0 & 1 & \plus0 & 15 & 12 & 8 & 1 & \x \\
3rd & 3 & \plus2 & \plus3 & \plus3 & \plus1 & 2 & \plus2 & 15 & 13 & 8 & 2 & Evasion \\
4th & 4 & \plus3 & \plus4 & \plus4 & \plus1 & 2 & \plus2 & 16 & 13 & 9 & 2 & Attribute score increase \\
5th & 4 & \plus3 & \plus4 & \plus4 & \plus1 & 2 & \plus2 & 16 & 13 & 9 & 2 & \x \\
6th & 5 & \plus3 & \plus4 & \plus4 & \plus1 & 2 & \plus4 & 17 & 14 & 9 & 3 & Devotion \\
7th & 6 & \plus4 & \plus5 & \plus5 & \plus2 & 3 & \plus4 & 17 & 14 & 9 & 3 & \x \\
8th & 7 & \plus5 & \plus5 & \plus5 & \plus2 & 3 & \plus4 & 18 & 14 & 10 & 3 & \x \\
9th & 8 & \plus6 & \plus6 & \plus6 & \plus2 & 3 & \plus6 & 18 & 15 & 10 & 4 & Attribute score increase, Multiattack\\
10th & 8 & \plus6 & \plus6 & \plus6 & \plus2 & 3 & \plus6 & 19 & 15 & 10 & 4 & \x \\
11th & 9 & \plus6 & \plus6 & \plus6 & \plus3 & 4 & \plus6 & 19 & 15 & 10 & 4 & \x \\
12th & 10 & \plus7 & \plus7 & \plus7 & \plus3 & 4 & \plus8 & 20 & 16 & 11 & 5 & \x \\
13th & 11 & \plus8 & \plus7 & \plus7 & \plus3 & 4 & \plus8 & 20 & 16 & 11 & 5 & Improved evasion \\
14th & 12 & \plus9 & \plus8 & \plus8 & \plus4 & 5 & \plus8 & 21 & 16 & 11 & 5 & Attribute score increase \\
15th & 12 & \plus9 & \plus8 & \plus8 & \plus4 & 5 & \plus10 & 21 & 17 & 11 & 6 & \x \\
16th & 13 & \plus9 & \plus8 & \plus8 & \plus4 & 5 & \plus10 & 22 & 17 & 12 & 6 & \x \\
17th & 14 & \plus10 & \plus9 & \plus9 & \plus4 & 5 & \plus10 & 22 & 17 & 12 & 6 & \x \\
18th & 15 & \plus11 & \plus9 & \plus9 & \plus5 & 6 & \plus12 & 23 & 18 & 12 & 7 & \x \\
19th & 16 & \plus12 & \plus10 & \plus10 & \plus5 & 6 & \plus12 & 23 & 18 & 12 & 7 & Attribute score increase \\
20th & 16 & \plus12 & \plus10 & \plus10 & \plus5 & 6 & \plus12 & 24 & 18 & 13 & 7 & \x
\end{tabularx}
\end{dtable!*}
  \subsubsection{Animal Companions}
\begin{dtable!*}
\lcaption{Animal Companion Choices}
\begin{tabularx}{\textwidth}{l >{\lcol}p{4em} >{\lcol}p{6.5em} *{3}{l} *{2}{>{\lcol}p{6em}} >{\lcol}X}
Animal & Size & Speed & Strength & Dexterity & Constitution & Primary Attacks & Secondary Attacks & Starting Feat \\
Badger & Small & 30', burrow 10' & Poor & Good & Good & 2 claws & Bite & Weapon Finesse \\
Bird & Tiny or Small & 10', fly 60' (average) & Poor & Good & Average & 2 talons & Bite & Alert \\
Camel & Large & 50' & Good & Average & Good & \x & Bite & Endurance \\
Dire rat & Small & 40', swim 20' & Average & Good & Average & Bite & \x & Weapon Finesse \\
Dog & Medium & 40' & Good & Good & Average & Bite & \x & Track \\
Light horse & Large & 60' & Average & Average & Good & \x & 2 hooves & Run \\
Heavy horse & Large & 50' & Good & Average & Good & \x & 2 hooves & Run \\
Pony & Medium & 40' & Average & Average & Average & \x & 2 hooves & Endurance \\
Viper & Small or Medium & 20', climb 20', swim 20' & Poor & Good & Average & Bite (poison) & \x & Weapon Finesse \\
Wolf & Medium & 50' & Average & Good & Good & Bite (trip) & \x & Track
\end{tabularx}
\end{dtable!*}

\par An animal companion's abilities are determined by the druid's level and its animal racial traits. Table: Animal Companion Base Statistics determines many of the base statistics of the animal companion. Animal companions remain creatures of the animal type for purposes of determining which spells can affect them.

\parhead{Class Level} This is the character's druid level. The druid's class levels stack with levels of any other classes that are entitled to an animal companion for the purpose of determining the companion's statistics.

\parhead{HD} This is the total number of value 5 Hit Values the animal companion possesses, each of which gains a Constitution, as normal.

\parhead{Base Attack Bonus} This is the animal companion's base attack bonus. An animal companion's base attack bonus is the same as that of a druid of a level equal to the animal's HV. Animal companions do not gain additional attacks using their natural weapons for a high base attack bonus.

\parhead{Fort/Ref/Will Save} These are the animal companion's base saving throw bonuses. An animal companion has good Fortitude and Reflex saves.

\parhead{Skills} Animal companions have two skill points when they are chosen. See the Monster Manual for more information on what skills animals treat as class skills.

\parhead{Feats} This is the total number of feats possessed by an animal companion. The first feat an animal takes is determined by its type; after that, the druid may choose the animal's feats. Animal companions are incapable of selecting many feats due to their natural limitations, and they may only select feats from the Player's Handbook and Monster Manual. The DM should approve any feat selected by an animal.

\parhead{Natural Armor Bonus} The number noted here is an improvement to the animal companion's existing natural armor bonus.

\parhead{Attribute Scores (Good/Average/Poor)} These are the physical attribute scores an animal companion might have in its physical attributes. Different animal companions have different abilities that they excel at, so they use different progressions. For example, a snake has good Dexterity, average Constitution, and poor Strength, so the snake of a level 6 druid has a 9 Strength, a 17 Dexterity, and a 14 Constitution.

\par While an animal companion's physical abilities depend on the kind of animal it is, its mental abilities do not. An animal companion has the intelligence of an animal and is not very persuasive, but can be wise.
Every animal companion has an intelligence of 2, a Wisdom of 12, and a Charisma of 6.

\parhead{Bonus Tricks} The value given in this column is the total number of ``bonus'' tricks that the animal knows in addition to any that the druid might choose to teach it (see the Creature Handling skill for more details on how to teach an animal tricks). These bonus tricks don't require any training time or Creature Handling checks, and they don't count against the normal limit of tricks known by the animal. The druid selects these bonus tricks, and once selected, they can't be changed.
\parhead{Special} This includes a number of abilities gained by animal companions as they increase in power. Each of these bonuses is described below.

\subcf{Link (Ex)} A druid can handle her animal companion as a free action, or push it as a move action, even if she doesn't have any ranks in the Creature Handling skill. The druid gains a \plus4  circumstance bonus  on all wild empathy checks and Creature Handling checks made regarding her animal companion.

\subcf{Share Spells (Ex)} The druid may cast a spell with a target of ``You'' on her animal companion (as a spell with a range of touch) instead of on himself. A druid may cast spells on her animal companion even if the spells normally do not affect creatures of the companion's type (animal). Spells cast in this way must come from a class that grants an animal companion. This ability does not allow the animal to share abilities that are not spells, even if they function like spells.

\subcf{Evasion (Ex)} If an animal companion is subjected to an attack that normally allows a Reflex save for half damage, it takes no damage if it makes a successful saving throw.
  \subcf{Attribute Score Increase (Ex)} The animal companion adds +1 to one of its attribute scores.

\subcf{Devotion (Ex)} An animal companion gains a +4 competence bonus on Will saves.
\subcf{Multiattack} An animal companion gains Multiattack as a bonus feat if it has three or more natural attacks and does not already have that feat. If it does not have the requisite three or more natural attacks, the animal companion instead gains a second attack with one of its natural weapons, albeit at a \minus5 penalty.

\subcf{Improved Evasion (Ex)} When subjected to an attack that allows a Reflex saving throw for half damage, an animal companion takes no damage if it makes a successful saving throw and only half damage if the saving throw fails. While helpless, the animal companion does not gain the benefit of this ability.
  \subsubsection{Animal Companion Choices}
Each animal companion has different starting sizes, speed, attacks, and attribute scores. All animal attacks are made using the creature's full base attack bonus unless otherwise noted. When an animal companion attacks, it adds its full Strength with its primary attacks and half its Strength with its secondary attacks. Secondary attacks take a \minus5 penalty. If the animal companion only has one primary attack and no secondary attacks, it adds 1-1/2 \mtimes its Strength instead.

\par Medium sized animals deal d4 damage with claw, hoof, and talon attacks, and d6 damage with bite attacks. Small sized animals deal d3 damage with claw, hoof, talon attacks, and d4 damage with bite attacks. Large sized animals deal d6 damage with claw, hoof, and talon attacks, and d8 damage with bite attacks.

\par Some animal companions have special abilities. Every animal companion has low-light vision, and every animal companion that is not a bird has scent. Vipers have a poisonous bite that deals initial and secondary damage of 1d6 Con. The save DC is equal to 10 \add 1/2 the viper's HV \add the viper's Con. A wolf that hits with a bite attack can attempt to trip the opponent as a free action without provoking attacks of opportunity.
\end{comment}
\subsubsection{Ex-Druids}
A druid who ceases to revere nature, changes to a prohibited alignment, or teaches the Druidic language to a nondruid loses all spells and druid abilities (not including weapon, armor, and shield proficiencies). She cannot thereafter gain levels as a druid until she atones (see the \spell{atonement} spell description).

 \subsection{Fighter}
\begin{dtable}
\lcaption{The Fighter}
\begin{tabularx}{\columnwidth}{>{\ccol}p{\levelcol} >{\ccol}p{\babcolgood} *{3}{>{\ccol}p{\savecol}} >{\lcol}X}
\thead{Level} & \thead{Base Attack Bonus} & \thead{Fort Save} & \thead{Ref Save} & \thead{Will Save} & \thead{Special} \\
1st & \plus1                         & \plus3 & \plus0 & \plus1 & Armor discipline \\
2nd & \plus2                         & \plus4 & \plus1 & \plus2 & Bonus feat \\
3rd & \plus3                         & \plus5 & \plus1 & \plus3 & Weapon discipline \\
4th & \plus4                         & \plus6 & \plus2 & \plus4 & Adaptive style feat \\
5th & \plus5                         & \plus7 & \plus2 & \plus4 & Combat discipline \\
6th & \plus6/\plus1                  & \plus8 & \plus3 & \plus5 & Bonus feat \\
7th & \plus7/\plus2                  & \plus9 & \plus3 & \plus6 & Improved armor discipline \\
8th & \plus8/\plus3                  & \plus10& \plus4 & \plus7 & Adaptive style feat \\
9th & \plus9/\plus4                  & \plus11& \plus4 & \plus7 & Improved weapon discipline\\
10th & \plus10/\plus5                & \plus12& \plus5 & \plus8 & Bonus feat, improved adaptive style\\
11th & \plus11/\plus6/\plus1         & \plus13 & \plus5 & \plus9 & Improved combat discipline\\
12th & \plus12/\plus7/\plus2         & \plus14 & \plus6 & \plus10& Adaptive style feat \\
13th & \plus13/\plus8/\plus3         & \plus15 & \plus6 & \plus10& Greater armor discipline \\
14th & \plus14/\plus9/\plus4         & \plus16 & \plus7 & \plus11& Bonus feat \\
15th & \plus15/\plus10/\plus5        & \plus17 & \plus7 & \plus12& Greater weapon discipline\\
16th & \plus16/\plus11/\plus6/\plus1 & \plus18 & \plus8 & \plus13& Adaptive style feat \\
17th & \plus17/\plus12/\plus7/\plus2 & \plus19 & \plus8 & \plus13& Greater combat discipline\\
18th & \plus18/\plus13/\plus8/\plus3 & \plus20 & \plus9 & \plus14& Bonus feat \\
19th & \plus19/\plus14/\plus9/\plus4 & \plus21 & \plus9 & \plus15& True discipline \\
20th & \plus20/\plus15/\plus10/\plus5& \plus22 & \plus10 & \plus16 & Adaptive style feat, greater adaptive style
\end{tabularx}
\end{dtable}
\cd{Alignment} Any.
 \cd{Hit Value} 6.

\sssecfake{Class Skills}
The fighter's class skills (and the key attribute for each skill) are Climb (Str), Jump (Str), Swim (Str), Ride (Dex), and Intimidate (Cha).
\cf{Ftr}{Skill Points at 1st Level} 2.

\sssecfake{Class Features}
All of the following are class features of the fighter.
 \cf{Ftr}{Weapon and Armor Proficiency}  A fighter is proficient with simple weapons,  any four other weapon groups,  all armor (heavy, medium, and light) and shields (including tower shields).

\cf{Ftr}{Armor Discipline} At 1st level, a fighter's training grants him additional capability when using his armor. He may choose an armor category (light, medium, or heavy), a shield category (buckler, light, medium, or tower), or he may choose to train equally with all kinds of armor and shields. Whether or not he chooses a specific armor category, he reduces his armor check penalty by 1 and reduces his arcane spell failure by 5\% when using his chosen armor. These benefits apply separately to armor and shields, if the fighter uses both and chose not to focus in a particular armor category.
\par If the fighter chose a particular armor category, he gains a \plus1 competence bonus to his dodge modifier while using armor of that category.

\cf{Ftr}{Bonus Feats} At 2nd level, a fighter gets a bonus combat-oriented feat. The fighter gains an additional bonus feat at 6th level and every four fighter levels thereafter (6th, 10th, 14th, and 18th). These bonus feats must be drawn from the feats noted as combat feats on \tref{Feats}. A fighter must still meet all prerequisites for a bonus feat, including attribute score and base attack bonus minimums.

These bonus feats are in addition to the feat that a character of any class gets from advancing levels. A fighter is not limited to the list of combat feats when choosing these feats.

\cf{Ftr}{Weapon Discipline} At 3rd level, a fighter's training grants him additional capability when using his weapons. He may choose a weapon group, or he may choose to train equally with all weapons. If he chooses a weapon group, he increases the attack bonus granted by the Weapon Focus feat (if he has that feat) by 1 when using weapons from that group.
\par If he chooses not to focus on a specific group of weapons, he gains the ability to become proficient with any weapon group if he spends 8 hours training with a weapon from that group. In addition, he gains a \plus1 competence bonus to attack rolls with all weapons he is proficient with. He may only gain these benefits with one weapon group at a time; if he trains with a new weapon group, he loses his proficiency and Weapon Focus in the previous group.

\cf{Ftr}{Adaptive Style Feats} At 4th level, a fighter gets a flexible bonus feat
which must be drawn from the list of combat feats. A fighter must still
meet all prerequisites for any bonus feat chosen. At the start of each day, the fighter may train for an hour. If he does so, he may choose to change his adaptive style feats to any other feats for which he meets the prerequisites.
The fighter gains an additional adaptive style feat at 8th level and every four fighter levels thereafter (8th, 12th, 16th, and 20th).
\par Adaptive style feats may be used normally as prerequisites for other feats or abilities.
However, an adaptive style feat is used as a prerequisite, it cannot be changed until the
fighter no longer needs to use it as a prerequisite, such as might happen if the fighter takes
the feat as a normal feat or bonus feat.
\par In order to gain a new adaptive style feat, it must be reasonably possible to do training related to the new feat. For example, a fighter could not gain Weapon Focus in axes without at least one axe available.

\cf{Ftr}{Combat Discipline} At 5th level, a fighter can his superior training and focus to keep fighting in the face of debilitating effects. When a fighter is initially affected by one of the conditions listed below, he may mitigate or negate the condition for up to one round per two fighter levels. At the end of that time, if the original condition would still affect the fighter, it does so normally, though he may use combat discipline again (if he has uses remaining) to continue to fight off the effect.
\par Using combat discipline takes no action, and can be done at any time, even when it isn't the fighter's turn. A fighter may use this ability a number of times per day equal to half his fighter class level \add his Constitution.

\begin{dtable}
\lcaption{Combat Discipline Conditions}
\begin{tabularx}{\columnwidth}{*{4}{>{\lcol}X}}
\thead{Condition} & \thead{Condition} & \thead{Condition} & \thead{Condition} \\
   Panicked & Frightened & Shaken & None  \\
   Petrified  & Paralyzed & Slowed & None \\
   Stunned       & Dazed & Staggered & None \\
&  Blinded & Dazzled & None \\
&  Confused & Bewildered & None \\
&  Exhausted & Fatigued & None\\
&  Nauseated & Sickened & None \\
&& Ability damage\fn{1} & None \\
&& Ability penalty\fn{1} & None \\
&& Entangled & None \\
&& Deafened & None \\
&& Fascinated & None \\
&& Ignited\fn{2} & None \\
&& Immobilized & None \\
&& Negative level\fn{3} & None \\
\end{tabularx}
1. Allows the fighter to mitigate up to half his fighter level in ability damage or penalties per use of combat discipline. \\
2. Mitigates the penalties, but does not prevent the fighter from taking d6 fire damage per round until the fire is put out. \\
3. Allows the fighter to ignore a single negative level per use of combat discipline.
\end{dtable}

\par A fighter cannot use this ability more than once against a single source. For example, if a fighter is exhausted by a \spell{ray of exhaustion} spell, he can use this ability to downgrade the exhaustion to fatigue, but he can't then expend a second use to negate the fatigue. The lesser condition that this ability imposes may be cured or removed normally, but doing so does not affect the resurgence of the condition the fighter was originally afflicted with. If a fighter uses this ability to mitigate or negate a condition which he must suffer as a sacrifice or cost to gain some benefit, he automatically forfeits the benefit he would have gained.

\cf{Ftr}{Improved Armor Discipline} At 7th level, a fighter's training in his chosen armor category (or with all armor categories) improves. He reduces the armor check penalty by 2 and decreases the arcane spell failure by 10\%. In addition, he treats his chosen armor (or armors) as if it were one encumbrance category lighter than it is.
\par This ability means heavy armor is treated as medium, medium armor is treated as light armor, and light armor is treated as being unarmored. Likewise, tower shields are treated as heavy shields (and no longer impose a \minus2 penalty to attack rolls), heavy shields are treated as light shields, and both light shields and bucklers are treated as being unarmored. This can remove the halving of the fighter's Dexterity bonus, if appropriate for the new encumbrance of the fighter's armor.
\par This allows the fighter to qualify for class features using the reduced armor encumbrance category. For example, a fighter 9 / bard 2 who reduces his encumbrance in medium armor could cast without any arcane spell failure in medium armor.
\par A fighter who chose a specific armor category gains a \plus2 competence bonus to his dodge modifier while using armor of that category.

\cf{Ftr}{Improved Weapon Discipline} At 9th level, a fighter's training in his chosen weapons improves. He gains a \plus4 competence bonus to resist disarm and sunder attempts when using his chosen weapons. If he chose a specific weapon group, he now increases the attack bonus gained from the Weapon Specialization feat by 2. If he did not, he can apply all weapon group-specific feats he has to any weapon group that he trains with for 8 hours, and he gains a \plus2 competence bonus to attack rolls with any weapon he is proficient with.

\cf{Ftr}{Improved Adaptive Style} At 10th level, a fighter's ability to adapt to situations improves. He need only spend 1 minute training to change a single adaptive style feat. He may continue training as he wishes, changing one adaptive style feat per minute.

\cf{Ftr}{Improved Combat Discipline} At 11th level, a fighter's ability to keep fighting despite negative influence improves. He can reduce conditions by two steps instead of one when using combat discipline. For example, a stunned fighter who used combat discipline would instead be staggered.
\par In addition, a fighter may use combat discipline to reduce any penalties he suffers to attribute scores, attack rolls, weapon damage rolls, skill checks, or ability checks that come from negative effects not listed on the combat discipline chart by 2. This cannot be used to reduce the effects of ability damage or drain.

\cf{Ftr}{Greater Armor Discipline} At 13th level, a fighter's training in his chosen armor becomes still greater. He reduces his armor check penalty by 3 and decreases his arcane spell failure by 15\% when using his chosen armor. In addition, he treats his chosen armor as if it were two encumbrance categories lighter than it actually is whenever doing so would be beneficial to him.
\par A fighter who chose a specific armor category gains a \plus3 competence bonus to his dodge modifier while using armor of that category.

\cf{Ftr}{Greater Weapon Discipline} At 15th level, a fighter's training in his chosen weapons becomes still greater. He increases the critical threat range and critical multiplier of his chosen weapons by 1. This increase applies after and stacks with any other effects that affect critical threat range or critical multiplier. Thus, a fighter with Improved Critical using a kukri would have a critical threat range of 14-20 (x3), while a similar fighter would have a critical threat range of 18-20 (x5) with a heavy pick.

\cf{Ftr}{Greater Combat Discipline} At 17th level, a fighter's ability to keep fighting despite influence becomes still greater. When using combat discipline, he may ignore any condition listed on the combat discipline chart. In addition, he may use combat discipline to be dazed rather than suffer any non-damaging condition not listed on the chart.

\cf{Ftr}{True Discipline} At 19th level, a fighter's discipline in his chosen area is beyond equal. He must choose either weapon discipline, armor discipline, or combat discipline. Depending on which discipline he chooses, he gains a different bonus.
\subcf{True Weapon Discipline} The fighter can take 10 on any attacks he makes at his full base attack bonus and automatically confirms all critical threats while using his chosen weapons.
\subcf{True Armor Discipline} The fighter no longer suffers armor check penalty or suffers arcane spell failure with his chosen armor. In addition, he treats his chosen armor as if it were three encumbrance categories lighter than it actually is.
\par A fighter who chose a specific armor category gains a \plus4 competence bonus to his dodge modifier while using armor of that category.
\subcf{True Combat Discipline} The fighter can use combat discipline to be staggered instead of suffering any nondamaging negative effect with a duration.

\cf{Ftr}{Greater Adaptive Style} At 20th level, a fighter's ability to react to situations is unparalleled. He need only spend a full-round action training to exchange an adaptive style feat. He may continue training as he wishes, changing one adaptive style feat per round.

\subsection{Monk}
\begin{dtable*}
\lcaption{The Monk}
\begin{tabularx}{\textwidth}{>{\ccol}p{\levelcol} >{\ccol}p{\babcolavg} *{3}{>{\ccol}p{\savecol}} >{\lcol}X >{\ccol}p{4.5em} >{\ccol}p{6em}}
\thead{Level} & \thead{Base Attack Bonus} & \thead{Fort Save} & \thead{Ref Save} & \thead{Will Save} & \thead{Special} & \thead{Unarmed Damage} & \thead{Unarmored Speed Bonus} \\
1st & \plus0                    & \plus1 & \plus3 & \plus3    & Enlightened defense, flurry of blows, unarmed strike & 1d6  & \plus0 ft. \\
2nd & \plus1                    & \plus2 & \plus4 & \plus4    & Bonus feat, \ki strike, uncanny dodge                                  & 1d6  & \plus0 ft. \\
3rd & \plus2                    & \plus3 & \plus5 & \plus5    & Bonus feat, still mind, wholeness of body                               & 1d6  & \plus10 ft. \\
4th & \plus3                    & \plus4 & \plus6 & \plus6    & Evasion, slow fall                         & 1d8  & \plus10 ft. \\
5th & \plus3                    & \plus4 & \plus7 & \plus7    & \Ki strike (magic)                                   & 1d8  & \plus10 ft. \\
6th & \plus4                    & \plus5 & \plus8 & \plus8    & Bonus feat, improved uncanny dodge                                          & 1d8  & \plus20 ft. \\
7th & \plus5                    & \plus6 & \plus9 & \plus9    & Diamond body                                         & 1d8  & \plus20 ft. \\
8th & \plus6/\plus1             & \plus7 & \plus10& \plus10   & Tongue of the sun and moon                           & 1d10 & \plus20 ft. \\
9th & \plus6/\plus1             & \plus7 & \plus11& \plus11   & Improved evasion                                     & 1d10 & \plus30 ft. \\
10th & \plus7/\plus2            & \plus8 & \plus12& \plus12   & Greater uncanny dodge, \ki strike (lawful)                   & 1d10 & \plus30 ft. \\
11th & \plus8/\plus3            & \plus9  & \plus13 & \plus13 & Abundant step                                      & 1d10 & \plus30 ft. \\
12th & \plus9/\plus4            & \plus10 & \plus14& \plus14 & Diamond soul                                         & 2d6  & \plus40 ft. \\
13th & \plus9/\plus4            & \plus10 & \plus15 & \plus15 & Quivering palm                                       & 2d6  & \plus40 ft. \\
14th & \plus10/\plus5           & \plus11 & \plus16& \plus16 & Empty step                                           & 2d6  & \plus40 ft. \\
15th & \plus11/\plus6/\plus1    & \plus12 & \plus17 & \plus17 & Timeless body                                        & 2d6  & \plus50 ft. \\
16th & \plus12/\plus7/\plus2    & \plus13 & \plus18 & \plus18 & \Ki strike (adamantine)                              & 2d8  & \plus50 ft. \\
17th & \plus12/\plus7/\plus2    & \plus13 & \plus19 & \plus19 & Moment of perfection                                 & 2d8  & \plus50 ft. \\
18th & \plus13/\plus8/\plus3    & \plus14 & \plus20 & \plus20 & Empty body                                           & 2d8  & \plus60 ft. \\
19th & \plus14/\plus9/\plus4    & \plus15 & \plus21 & \plus21 & Empty soul                                           & 2d8  & \plus60 ft. \\
20th & \plus15/\plus10/\plus5   & \plus16 & \plus22 & \plus22 & \Ki strike (epic), perfect self                      & 2d10 & \plus60 ft. \\
\end{tabularx}
\end{dtable*}

\cd{Alignment} Any lawful.

\cd{Hit Value} 5

\sssecfake{Class Skills}
The monk's class skills (and the key attribute for each skill) are Climb (Str), Jump (Str), Swim (Str), Acrobatics (Dex), Escape Artist (Dex), Stealth (Dex), Knowledge (religion) (Int), Perception (Wis), Sense Motive (Wis), and Persuasion (Cha).
\cf{Mnk}{Skill Points at 1st Level} 4

\sssecfake{Class Features}
All of the following are class features of the monk.
 \cf{Mnk}{Weapon and Armor Proficiency}
Monks are proficient with simple weapons, monk weapons,  and any one other weapon group.  Monks are not proficient with any armor or shields. When wearing armor, using a shield, or carrying a medium or heavy load, a monk loses the benefit of her  enlightened defense,  fast movement, and flurry of blows abilities.
 \cf{Mnk}{Enlightened Defense (Ex)}  When unarmored and unencumbered, the monk adds her Wisdom (if positive) to her AC.

\par This bonus to AC applies even against touch attacks or when the monk is flat-footed. She loses this bonus when she is immobilized or helpless, when she wears any armor, when she carries a shield, or when she carries a medium or heavy load.

\begin{comment}    %Made AC too high
 \cfnl{\Ki Ward (Ex)}\label{Mnk:Ki Ward (Ex)} When unarmored and unencumbered, a monk gains a \plus1 armor bonus to AC at 2nd level. This bonus increases by 1 for every two monk levels thereafter (\plus2 at 4th, \plus3 at 6th, etc.).

\par The monk loses this bonus when she is
immobilized or helpless, when she wears any armor, when she carries a shield, or when she carries a medium or heavy load.
\end{comment}

 \cf{Mnk}{Flurry of Blows (Ex)}  When unarmored, a monk may strike with a flurry of blows. When making flurry of blows attack, if the monk hits a foe by 5 or more, she may deal extra damage to the same foe as if she had hit with two attacks. She may instead attempt to direct the extra damage to a different foe. If the attack also exceeds that foe's armor class by 5 or more, both foes are damaged by the monk's attack. A monk must use a full attack action to strike with a flurry of blows, and she may only use a flurry of blows to make melee attacks.

\par When using flurry of blows, a monk may attack only with unarmed strikes or with special monk weapons (kama, nunchaku, quarterstaff, sai, and siangham, but not shuriken). She may attack with unarmed strikes and special monk weapons interchangeably as desired.

In the case of the quarterstaff, each end counts as a separate weapon for the purpose of using the flurry of blows ability. Even though the quarterstaff requires two hands to use, a monk may still intersperse unarmed strikes with quarterstaff strikes, assuming that she has enough attacks in her flurry of blows routine to do so.

\par While using a flurry of blows, a monk may use weapons in either hand interchangeably, and she applies her full Strength to her damage rolls for all successful attacks, whether she wields a weapon in one or both hands. However, a monk may not use a two-weapon fighting style (see \pref{Two-Weapon Fighting [Combat]}) and flurry of blows at the same time.

\cf{Mnk}{Unarmed Strike} At 1st level, a monk gains Improved Unarmed Strike as a bonus feat. A monk's attacks may be with either fist interchangeably or even from elbows, knees, and feet. This means that a monk may even make unarmed strikes with her hands full. There is no such thing as an off-hand attack for a monk striking unarmed. A monk may thus apply her full Strength bonus on damage rolls for all her unarmed strikes.

Usually a monk's unarmed strikes deal lethal damage, but she can choose to deal nonlethal damage instead with no penalty on her attack roll. She has the same choice to deal lethal or nonlethal damage while grappling.

A monk's unarmed strike is treated both as a manufactured weapon and a natural weapon for the purpose of spells and effects that enhance or improve either manufactured weapons or natural weapons.

A monk also deals more damage with her unarmed strikes than a normal person would, as shown on \trefnp{The Monk}. The unarmed damage given is for Medium monks. A Small monk deals less damage than the amount given there with her unarmed attacks, while a Large monk deals more damage; see \trefnp{Small or Large Monk Unarmed Damage}.

Monks can use gauntlets, including enchanted gauntlets, but they must use the damage of the gauntlets in place of their normal unarmed strike damage if they do so.

\begin{dtable}
\lcaption{Small or Large Monk Unarmed Damage}
\begin{tabularx}{\columnwidth}{>{\lcol}p{5em} >{\lcol}X >{\lcol}X}
\thead{Level} & \thead{Damage (Small Monk)} & \thead{Damage (Large Monk)} \\
1st-3rd  & 1d4 & 1d8 \\
4th-7th  & 1d6 & 1d10 \\
8th-11th & 1d8 & 2d6 \\
12th-15th & 1d10 & 2d8 \\
16th-19th & 2d6 & 2d10 \\
20th   & 2d8 & 4d6 \\
\end{tabularx}
\end{dtable}

\cf{Mnk}{Bonus Feat} At 2nd level, a monk may select either Improved Grapple or Stunning Fist as a bonus feat. At 3rd level, she may select either Combat Reflexes or Deflect Arrows as a bonus feat. At 6th level, she may select any combat maneuver feat. A monk need not have any of the prerequisites normally required for these feats to select them.

\cf{Mnk}{Ki Strike (Su)} At 2nd level, a monk can guide her strikes with \ki. She may add half her Wisdom to her attack rolls with unarmed strikes, special monk weapons, and combat maneuvers. This is in addition to the normal Strength or Dexterity, as appropriate.

\cf{Mnk}{Uncanny Dodge (Ex)} Starting at 2nd level, a monk can react to danger before her senses would normally allow her to do so. She may apply her Dexterity and dodge modifier to her armor class while flat-footed.

If a monk already has uncanny dodge from a different class, she stacks those levels to determine whether she gains improved uncanny dodge (see below) instead.

\cf{Mnk}{Evasion (Ex)} At 2nd level or higher, if a monk makes a successful Reflex saving throw against an attack that normally deals half damage on a successful save, she instead takes no damage. Evasion can be used only if a monk is wearing light armor or no armor. A helpless monk does not gain the benefit of evasion.

\cf{Mnk}{Fast Movement (Ex)} At 3rd level, a monk gains a competence bonus to her speed, as shown on \tref{The Monk}. A monk in armor or carrying a medium or heavy load loses this extra speed.

\cf{Mnk}{Still Mind (Ex)} A monk of 3rd level or higher may add half her Wisdom to Will saves in place of half her Intelligence.

\cf{Mnk}{Ki Strike (Magic) (Su)} At 5th level, a monk's attacks are empowered with \ki. Her unarmed strikes and any monk weapons she uses are treated as magic weapons for the purpose of dealing damage to creatures with damage reduction.

\cf{Mnk}{Slow Fall (Ex)}  At 4th level or higher, a monk within arm's reach of any solid object can use it to slow her descent. When first using this ability, she takes damage as if the fall were 20 feet shorter than it actually is. The monk's ability to slow her fall (that is, to reduce the effective distance of the fall when next to a solid object )increases by 5 feet per monk level thereafter until 20th level, when she can use a nearby object to slow her descent and fall any distance without harm.

\cf{Mnk}{Wholeness of Body (Su)}  At 4th level or higher, a monk can heal her own wounds. She can heal a number of hit points of damage equal to her monk level \mtimes her Wisdom bonus each day, and she can spread this healing out among several uses.  Using this ability is a swift action that does not provoke attacks of opportunity.

\cf{Mnk}{Improved Uncanny Dodge (Ex)} At 6th level and higher, a monk can no longer be overwhelmed as easily; she can react to multiple opponents as easily as she can react to a single attacker. The monk is always treated as being threatened by two fewer creatures than she actually is for the purpose of determining overwhelm penalties.
\par If a character already has uncanny dodge (see above) from a
second class and gains improved uncanny dodge, the character stacks those levels to determine if she should gain greater uncanny dodge, and to determine the minimum level a rogue must be to flank the character.

\cf{Mnk}{Diamond Body (Su)} At 7th level, a monk gains immunity to poisons and diseases of all kinds, including supernatural and magical diseases (such as mummy rot and lycanthropy).

\cf{Mnk}{Tongue of the Sun and Moon (Ex)} A monk of 8th level or higher can speak to and understand the speech of any living creature. This grants her no special ability to speak to or understand creatures that do not speak, such as animals.

\cf{Mnk}{Improved Evasion (Ex)} At 9th level, a monk's evasion ability improves. She still takes no damage on a successful Reflex saving throw against attacks, but henceforth she takes only half damage on a failed save. A helpless monk does not gain the benefit of improved evasion.

\cf{Mnk}{Greater Uncanny Dodge (Ex)} At 10th level and higher, a monk can no longer be overwhelmed, regardless of the number of foes surrounding her.

\cf{Mnk}{Improved Ki Strike (Su)} At 10th level, a monk may add half her Wisdom to damage with unarmed attacks and monk weapons. In addition, those weapons are treated as being lawful for the purpose of dealing damage to creatures with damage reduction.

\cf{Mnk}{Abundant Step (Su)} At 11th level or higher, a monk can slip magically between spaces, as if using the spell \spell{dimension slide}, a number of times per day equal to half her Wisdom. Using this ability is a move action that does not provoke attacks of opportunity. Her caster level for this effect is equal to her monk level.

\cf{Mnk}{Diamond Soul (Ex)} At 12th level, a monk gains spell resistance. In order to affect the monk with a spell, a spellcaster must make a spell penetration check (d20 \add half caster level \add casting attribute) that equals or exceeds the monk's spell resistance \add relevant saving throw modifier.

\cf{Mnk}{Quivering Palm (Su)} Starting at 13th level, a monk can set up vibrations within the body of another creature that can thereafter be fatal if the monk so desires. She can use this quivering palm attack a number of times per day equal to her Wisdom, but no more than once per round, and she must announce her intent before making her attack roll. Constructs, oozes, plants, undead, incorporeal creatures, and creatures immune to critical hits cannot be affected. Otherwise, if the monk strikes successfully and the target takes damage from the blow, the quivering palm attack succeeds. Thereafter the monk can try to slay the victim at any later time, as long as the attempt is made within a number of days equal to her monk level. To make such an attempt, the monk merely wills the target to die (a free action), and unless the target makes a Fortitude saving throw (DC 10 \add 1/2 the monk's level \add the monk's Wis), it dies. If the saving throw is successful, the target is no longer in danger from that particular quivering palm attack, but it may still be affected by another one at a later time.

\cf{Mnk}{Empty Step (Su)} At 14th level or higher, a monk can assume an ethereal state for brief periods, as if using the \spell{ethereal jaunt} spell, except that the effect only lasts for a single round, a number of times per day equal to her Wisdom. Using this ability is a swift action.

\cf{Mnk}{Ki Strike (Adamantine) (Su)} At 15th level, a monk's unarmed attacks and monk weapons are treated as adamantine weapons for the purpose of dealing damage to creatures with damage reduction and bypassing hardness.

\cf{Mnk}{Timeless Body (Ex)} Upon attaining 15th level, a monk no longer takes penalties to her attribute scores for aging and cannot be magically aged. Any such penalties that she has already taken, however, remain in place. Bonuses still accrue, and the monk still dies of old age when her time is up.

\cf{Mnk}{Moment of Perfection (Su)} At 17th level, a monk can align herself perfectly with the universe to achieve a single moment of perfection. She can add her monk level as an enhancement bonus to any one attack roll, opposed skill or ability check, or saving throw, or to her AC against any one attack, as if calling upon the effect of a \spell{moment of prescience} spell with a caster level equal to her monk level. She can use this ability a number of times per day equal to her Wisdom, but she must take a short rest between each use of this ability.

\cf{Mnk}{Empty Body (Su)} At 18th level, a monk using her empty step ability can stay ethereal for a number of rounds equal to her monk level. She may dismiss the ability and become material as a swift action.

\cf{Mnk}{Empty Soul (Su)} At 19th level, a monk achieves a state of complete emptiness, reacting to all situations without premeditation or thought. She continuously gains the benefits of the \spell{foresight} spell upon her person at all times.

\cf{Mnk}{Ki Strike (Epic) (Su)} At 20th level, a monk's unarmed attacks and monk weapons are treated as epic weapons for the purpose of dealing damage to creatures with damage reduction.

\cf{Mnk}{Perfect Self} At 20th level, a monk becomes a magical creature. She is forevermore treated as an outsider rather than as a humanoid for the purpose of spells and magical effects   whenever doing so is advantageous to her. For instance, \spell{charm person} does not affect her, but she can still be affected by \spell{enlarge person}, and she can still be brought back from the dead.

Additionally, the monk gains damage reduction 10/chaotic, which allows her to ignore the first 10 points of damage from any attack made by a non-chaotic weapon or by any natural attack made by a creature that doesn't have similar damage reduction.

\subsubsection{Ex-Monks}
A monk who becomes nonlawful cannot gain new levels as a monk, but retains all monk abilities.

 \subsection{Paladin}
\begin{dtable*}
\lcaption{The Paladin}
\begin{tabularx}{\textwidth}{>{\ccol}p{\levelcol} >{\ccol}p{\babcolgood} *{3}{>{\ccol}p{\savecolpoof}} X *{4}{>{\ccol}p{\spellcolpoof}}}
& & & & & & \multicolumn{4}{c}{\thead{---{}---Spells per Day---{}---}} \\
\thead{Level} & \thead{Base Attack Bonus} & \thead{Fort Save} & \thead{Ref Save} & \thead{Will Save} & \thead{Special} & \thead{1st} & \thead{2nd} & \thead{3rd} & \thead{4th} \\
1st  & \plus1                        & \plus3  & \plus0 & \plus3 & Aura of good, detect evil, smite evil & \x & \x & \x & \x \\
2nd  & \plus2                        & \plus4  & \plus1 & \plus4 & Improved smite, lay on hands & \x & \x & \x & \x \\
3rd  & \plus3                        & \plus5  & \plus1 & \plus5 & Aura of courage, bulwark of defense, divine health & \x & \x & \x & \x \\ 
4th  & \plus4                        & \plus6  & \plus2 & \plus6 & Divine grace & 1 & \x & \x & \x \\
5th  & \plus5                        & \plus7  & \plus2 & \plus7 & Detect chaos, improved smite & 2 & \x & \x & \x \\
6th  & \plus6/\plus1                 & \plus8  & \plus3 & \plus8 & Aura of resolve, holy ward & 3 & \x & \x & \x \\
7th  & \plus7/\plus2                 & \plus9  & \plus3 & \plus9 & Smite chaos & 3 & \x & \x & \x \\
8th  & \plus8/\plus3                 & \plus10 & \plus4 & \plus10& Improved smite & 3 & 1 & \x & \x \\
9th  & \plus9/\plus4                 & \plus11 & \plus4 & \plus11& Aura of determination, improved bulwark of defense & 3 & 2 & \x & \x \\
10th & \plus10/\plus5                & \plus12 & \plus5 & \plus12& Discern lies & 3 & 3 & \x & \x \\
11th & \plus11/\plus6/\plus1         & \plus13 & \plus5 & \plus13& Improved smite & 4 & 3 & \x & \x \\
12th & \plus12/\plus7/\plus2         & \plus14 & \plus6 & \plus14& Aura of protection & 4 & 3 & 1 & \x \\
13th & \plus13/\plus8/\plus3         & \plus15 & \plus6 & \plus15& Forgiving smite & 4 & 3 & 2 & \x \\
14th & \plus14/\plus9/\plus4         & \plus16 & \plus7 & \plus16& Improved smite & 4 & 3 & 3 & \x \\
15th & \plus15/\plus10/\plus5        & \plus17 & \plus7 & \plus17& Aura of warding & 4 & 4 & 3 & \x \\
16th & \plus16/\plus11/\plus6/\plus1 & \plus18 & \plus8 & \plus18& Glory of the martyr & 4 & 4 & 3 & 1 \\
17th & \plus17/\plus12/\plus7/\plus2 & \plus19 & \plus8 & \plus19& Improved smite & 4 & 4 & 3 & 2 \\
18th & \plus18/\plus13/\plus8/\plus3 & \plus20 & \plus9 & \plus20& Greater aura of warding & 4 & 4 & 3 & 3 \\
19th & \plus19/\plus14/\plus9/\plus4 & \plus21 & \plus9 & \plus21& Martyr's retribution & 4 & 4 & 4 & 3 \\
20th & \plus20/\plus15/\plus10/\plus5& \plus22 &\plus10 &\plus22 & Greater smite, improved smite & 4 & 4 & 4 & 4 \\
\end{tabularx}
\end{dtable*}

\cd{Alignment} Lawful good.
\cd{Hit Value} 6

\sssecfake{Class Skills}
The paladin's class skills (and the key attribute for each skill) are Ride (Dex), Knowledge (local) (Int), Knowledge (religion) (Int), Heal (Wis), Sense Motive (Wis), Intimidate (Cha), and Persuasion (Cha).
\cf{Pal}{Skill Points at 1st Level} 2.

\sssecfake{Class Features}
All of the following are class features of the paladin.
 \cf{Pal}{Weapon and Armor Proficiency}  Paladins are proficient with
simple weapons,  any two other weapon groups,  all types of armor (heavy, medium, and light), and with  shields (including tower shields). A paladin is also proficient with the favored weapon group of her deity. If she does not follow a deity, she is proficient with any other weapon group of her choice.
\cf{Pal}{Aura of Good (Ex)} The power of a paladin's aura of good (see the \spell{detect good} spell) is equal to her paladin level.

\cfnl{\spell{Detect Evil} (Sp)}\label{Pal:Detect Evil (Sp)} A paladin can use \spell{detect evil}, as the spell, a number of times per day equal to half the paladin's level \add the paladin's Wisdom (minimum 1).

At 5th level, the paladin can also use \spell{detect chaos}. At 10th level, the paladin can also use \spell{discern lies}.

\cf{Pal}{Smite (Su)} As part of an attack action, a paladin may attempt to smite evil with one normal melee attack. She adds her Charisma bonus (if any) as a circumstance bonus to attack. If she hits, the paladin deals gains a circumstance bonus to damage equal to her paladin level and gains a special effect.  A paladin can smite evil a number of times per day equal to half the paladin's class level \add the paladin's Charisma, but may only do so once per round.

If the paladin smites a creature that is not evil, the smite attack deals no damage at all (not even normal weapon damage), but the use of the ability is still spent.

\cf{Pal}{Improved Smite (Su)} At 2nd level, and every three levels thereafter, the paladin can select one improved smiting ability. Each improved smiting ability adds an effect to the paladin's smite ability. Whenever the paladin smites, she chooses one improved smiting ability she has and adds its effect to her smite. The save DC for a paladin's improved smiting abilities is 10 \add 1/2 the paladin's level \add the paladin's Charisma.
\par At 2nd level, the paladin can select the following improved smiting effects.
\subcf{Blinding} The paladin's smite manifests as a bright light. A creature struck by the smite must make a Will save to avoid being blinded for one round per four paladin levels (minimum 1). Creatures vulnerable to light (such as vampires) take extra damage equal to twice the paladin's level.
\subcf{Resounding} The paladin's smite knocks his foes off their feet. A creature struck by the smite must make a Reflex save to avoid being pushed back five feet per four paladin levels and being knocked prone.
\subcf{Staggering} The paladin's smite hits with incredible force. A creature struck by the smite must make a Fortitude save to avoid being staggered for one round per four paladin levels (minimum 1).

At 5th level, the paladin adds the following improved smiting ability to the list of those that can be selected.
\subcf{Holy} The paladin's smite is filled with exceptional divine energy. The attack ignores all damage reduction of evil creatures. Undead and evil outsiders take extra damage equal to twice the paladin's level.
\subcf{Penetrating} The paladin's smite punches through her enemies' defenses. The attack ignores a number of points of damage reduction equal to the paladin's level, regardless of the type of damage reduction.
\subcf{Seeking} The paladin's smite is uncannily guided to its target. The attack ignores any miss chance, though the weapon must still be able to strike the target.

At 8th level, the paladin adds the following improved smiting ability to the list of those that can be selected.
\subcf{Dazing} The paladin's smite shatters her foe's ability to concentrate. A creature struck by the smite must make a Fortitude save to avoid being dazed for one round.
\subcf{Impeding} The paladin's smite traps her foe in place, unable to escape her wrath. A creature struck by the smite must make a Reflex save to avoid having all its movement speeds reduced to 5 feet for one round.
\subcf{Coercing} The paladin's smite forces her foe to join the cause of righteousness, if only for a moment. A creature struck by the smite must make a Will save or else be affected by a \spell{suggestion}, as the spell, of the paladin's choice. The effect lasts for one round.

At 11th level, the paladin adds the following improved smiting ability to the list of those that can be selected.
\subcf{Axiomatic} The paladin's smite is filled with exceptionally lawful divine energy. The attack ignores all damage reduction of chaotic creatures. Aberrations and chaotic outsiders take extra damage equal to twice the paladin's level.

At 14th level, the paladin adds the following improved smiting ability to the list of those that can be selected.
\subcf{Dispelling} The paladin's smite strips away her foe's magical protections. A creature struck by the smite is subject to a targeted \spell{dispel magic} with a bonus equal to the paladin's level.

At 17th level, the paladin adds the following improved smiting ability to the list of those that can be selected.
\subcf{Brilliant} The paladin's smite cannot be turned aside by mortal defenses. The smite is made against the enemy's touch armor class.

\cf{Pal}{Lay On Hands (Su)} Beginning at 2nd level, a paladin can heal wounds (her own or those of others) with a touch. Each use heals 1d8 hit points per paladin level as a standard action. The paladin can lay on hands a number of times per day equal to 1 \add half her Charisma (minimum 1). Against undead creatures, this ability instead deals positive energy damage. A touch attack is required to hit unwilling targets, and a successful Will save halves the damage or healing received.

 \cf{Pal}{Aura of Courage (Su)}   Beginning at 3rd level, a paladin is immune to fear (magical or otherwise). Each ally within 10 feet of her gains a \plus4 enhancement bonus on saving throws against fear effects. This ability functions while the paladin is conscious, but not if she is unconscious or dead.

\cf{Pal}{Divine Health (Ex)} At 3rd level, a paladin gains immunity to all diseases, including supernatural and magical diseases.

\cf{Pal}{Divine Grace (Su)} At 4th level, whenever a paladin makes a successful save against an attack that would normally have a partial effect or deal half damage on a successful save, she instead ignores that aspect of the attack.

\cf{Pal}{Spells} Beginning at 4th level, a paladin gains the ability to cast a small number of divine spells, which are drawn from the paladin spell list.

Paladins do not require somatic components to cast spells, even if the spell would normally require a somatic component. A paladin need only request the favor of her deity to invoke divine magic.

\par To learn or cast a spell, a paladin must have a Charisma score at least equal to the spell's level. The Difficulty Class for a saving throw against a paladin's spell is 10 \add the spell level \add the paladin's Charisma.

\par Like other spellcasters, a paladin can cast only a certain number of spells of each spell level per day. Her base daily spell allotment is given on \trefnp{The Paladin}.

A paladin learns and casts spells the way a cleric does, though she does not have access to any domain spells or granted powers, as a cleric does, and cannot sacrifice a readied spell slot to cast a \spell{cure} spell in its place. A paladin may learn and cast any spell on the paladin spell list, provided that she can cast spells of that level.

A paladin's selection of spells is limited. A paladin begins play knowing no spells, but gains one or more new spells at certain levels, as indicated on \trefnum{Paladin Spells Known}.

\par At 5th level, and each level after that, a paladin can choose to learn one new spell in place of one she already knows. In effect, the paladin ``loses'' the old spell in exchange for the new one. The new spell's level must be the same as that of the spell being exchanged. A paladin may swap only a single spell at any given level, and must choose whether or not to swap the spell at the same time that she gains new spells known for the level.

\par Through 3rd level, a paladin has no caster level or magic level. At 4th level and higher, her caster level is equal to her paladin level / 2, and her magic level is equal to her paladin level.
\begin{dtable}
\lcaption{Paladin Spells Known}
\begin{tabularx}{\columnwidth}{X *{4}{>{\ccol}X}}
& \multicolumn{4}{c}{\thead{---{}---{}---{}---{}---Spells Known---{}---{}---{}---{}---}} \\
\thead{Level} & \thead{1st} & \thead{2nd} & \thead{3rd} & \thead{4th} \\
1st  & \x & \x & \x & \x \\
2nd  & \x & \x & \x & \x \\
3rd  & \x & \x & \x & \x \\
4th  & 1 & \x & \x & \x \\
5th  & 2 & \x & \x & \x \\
6th  & 2 & \x & \x & \x \\
7th  & 3 & \x & \x & \x \\
8th  & 3 & 1 & \x & \x \\
9th  & 3 & 2 & \x & \x \\
10th & 4 & 2 & \x & \x \\
11th & 4 & 2 & \x & \x \\
12th & 4 & 3 & 1 & \x \\
13th & 4 & 3 & 2 & \x \\
14th & 4 & 3 & 2 & \x \\
15th & 4 & 3 & 2 & \x \\
16th & 4 & 3 & 3 & 1 \\
17th & 4 & 3 & 3 & 2 \\
18th & 4 & 3 & 3 & 2 \\
19th & 4 & 3 & 3 & 2 \\
20th & 4 & 3 & 3 & 3 \\
\end{tabularx}
\end{dtable}

\begin{comment}
 \cf{Pal}{Special Mount (Sp)}  Upon reaching 5th level, a paladin gains the service of an unusually intelligent, strong, and loyal steed to serve her in her crusade against evil (see below). This mount is usually a heavy warhorse (for a Medium paladin) or a warpony (for a Small paladin).

Once per day, as a full-round action, a paladin may magically call her mount from the celestial realms in which it resides. This ability is the equivalent of a spell of a level equal to one-third the paladin's level. The mount immediately appears adjacent to the paladin and remains for 2 hours per paladin level; it may be dismissed at any time as a swift action. The mount is the same creature each time it is summoned, though the paladin may release a particular mount from service.

Each time the mount is called, it appears in full health, regardless of any damage it may have taken previously. The mount also appears wearing or carrying any gear it had.

\par Should the paladin's mount die, it immediately disappears, leaving behind any equipment it was carrying. The paladin may not summon another mount for thirty days or until she gains a paladin level, whichever comes first, even if the mount is somehow returned from the dead. During this thirty-day period, the paladin  gains a negative level that cannot be removed in any way.
\end{comment}

\cf{Pal}{Aura of Resolve (Su)} At 6th level, the paladin becomes immune to charm effects. Each ally within 10 feet of her gains a \plus4 enhancement bonus on saving throws against charm effects.

\cf{Pal}{Holy Ward (Sp)} A paladin is first and foremost a defender of her allies, and relies upon \spell{shield other} to defend her allies against harm from threats she cannot block with her skill at arms. A paladin of 6th level or higher who spends at least 10 points of healing from her lay on hands ability in a single round can bestow a \spell{shield other} effect on the healed ally, using her paladin level as her caster level.

\cf{Pal}{Smite Chaos (Su)} At 7th level, the paladin gains the ability to smite chaotic creatures as well as evil creatures with her smite ability, but she must choose which to smite before making the attack. If the paladin attempts to smite a chaotic creature, and that creature is not chaotic, the smite attack deals no damage at all (not even normal weapon damage), but the use of the ability is still spent.

\cf{Pal}{Aura of Determination (Su)} At 9th level, the paladin becomes immune to compulsion effects. Each ally within 10 feet of her gains a \plus4 enhancement bonus on saving throws against compulsion effects.

\cf{Pal}{Aura of Protection (Su)} At 12th level, the paladin continuously radiates a \spell{magic circle against evil}, as the spell.

\cf{Pal}{Forgiving Smite (Su)} At 13th level, if a paladin smites a creature who is not evil or chaotic, the smite attempt is not wasted.

\cf{Pal}{Glory of the Martyr (Su)} At 14th level, if the paladin dies while fighting evil or protecting her allies, her fallen body erupts in a burst of positive energy, granting all her allies within 100 feet the benefit of a \spell{heal} spell.

\cf{Pal}{Aura of Warding (Su)} At 15th level, the paladin continuously radiates a \spell{lesser globe of invulnerability}, as the spell. The effect travels with the paladin.

\cf{Pal}{Mighty Aura (Su)} At 16th level, the radius of a paladin's auras expands to 20 feet.

\cf{Pal}{Greater Aura of Warding (Su)} At 18th level, the paladin radiates a \spell{globe of invulnerability}, as the spell, instead of a \spell{lesser globe of invulnerability}. The effect continues to travel with the paladin.

\cf{Pal}{Martyr's Retribution (Su)} At 19th level, if the paladin dies while fighting evil or protecting her allies, she can choose to make the explosion of positive energy from her glory of the martyr ability painful to her foes. If she does, her body is almost completely consumed by holy power, preventing her from being raised with \spell{raise dead} and similar effects that require a body. This has two effects. First, a \spell{sunburst} spell immediately takes effect over the area where the paladin fell. Second, a \spell{storm of vengeance} spell begins to take effect, centered on the same area. The spell lasts for 10 rounds, and the lightning strikes focus on the paladin's enemies. Both of these effects harm only the paladin's foes, and do no damage to her allies. However, her allies' vision is still impeded by the \spell{storm of vengeance}.

\cf{Pal}{Greater Smite (Su)} At 20th level, the paladin can apply two improved smiting abilities to every smite attack she makes.

\cf{Pal}{Code of Conduct} A paladin must be of lawful good alignment
and loses all class abilities if she ever willingly commits an evil act.
Additionally, a paladin's code requires that she respect legitimate
authority, act with honor (not lying, not cheating, not using poison,
and so forth), help those in need (provided they do not use the help
for evil or chaotic ends), and punish those who harm or threaten
innocents.

\cf{Pal}{Associates} While she may adventure with characters of any good or neutral alignment, a paladin will never knowingly associate with evil characters, nor will she continue an association with someone who consistently offends her moral code. A paladin may accept only henchmen, followers, or cohorts who are lawful good.

\subsubsection{Ex-Paladins}
A paladin who ceases to be lawful good, who willfully commits an evil act, or who grossly violates the code of conduct loses all paladin spells and abilities (including the service of the paladin's mount, but not weapon, armor, and shield proficiencies). She may not progress any farther in levels as a paladin. She regains her abilities and advancement potential if she atones for her violations (see the \spell{atonement} spell description), as appropriate.

Like a member of any other class, a paladin may be a multiclass character, but multiclass paladins face a special restriction. A paladin who gains a level in any class other than paladin may never again raise her paladin level, though she retains all her paladin abilities.

\begin{comment}
\subsubsection{The Paladin's Mount}  The paladin's mount is superior to a normal mount of its kind and has
special powers.  This mount functions as a druid's animal companion, using the paladin's class level as her druid level, with some exceptions (see below).  The standard mount for a Medium paladin is a heavy warhorse, and the standard mount for a Small paladin is a warpony. Another kind of mount, such as a riding dog (for a halfling paladin) or a Large shark (for a paladin in an aquatic campaign) may be allowed as well.

A paladin's mount is treated as a magical beast, not an animal, for the purpose of all effects that depend on its type (though it retains an animal's HD, base attack bonus, saves, skill points, and feats).

\cf{Pal}{Paladin's Mount Basics} Use the base statistics for a horse or pony, as given in the druid's animal companion class ability, but make changes to take into account the abilities given below.

\subcf{War-trained} The paladin's mount has been trained for war, making it proficient in all types of armor. In addition, the warhorse and warpony mounts have two primary hoof attacks instead of the attacks a horse and pony would have. More exotic mounts may see similar improvements, depending on the nature of the mount.

\subcf{Intelligence} The mount's Intelligence score.

\subcf{Empathic Link (Su)} The paladin has an empathic link with her mount out to a distance of up to 1 mile. The paladin cannot see through the mount's eyes, but they can communicate empathically.

Note that even intelligent mounts see the world differently from humans, so misunderstandings are always possible.

Because of this empathic link, the paladin has the same connection to an item or place that her mount does, just as with a master and his familiar (see Familiars).

\subcf{Improved Evasion (Ex)} When subjected to an attack that normally allows a Reflex saving throw for half damage, a mount takes no damage if it makes a successful saving throw and half damage if the saving throw fails.

\subcf{Share Spells} At the paladin's option, she may have any spell (but not any spell-like ability) she casts on herself also affect her mount.

The mount must be within 5 feet at the time of casting to receive the benefit. If the spell or effect has a duration other than instantaneous, it stops affecting the mount if it moves farther than 5 feet away and will not affect the mount again even if it returns to the paladin before the duration expires. Additionally, the paladin may cast a spell with a target of ``You'' on her mount (as a touch range spell) instead of on herself. A paladin and her mount can share spells even if the spells normally do not affect creatures of the mount's type (magical beast).

\subcf{Share Saving Throws} For each of its saving throws, the mount uses its own base save bonus or the paladin's, whichever is higher. The mount applies its own attributes to saves, and it doesn't share any other bonuses on saves that the master might have.

\subcf{Improved Speed (Ex)} The mount's speed increases by 10 feet.

\subcf{Command (Sp)} Once per day per two paladin levels of its master, a mount can use this ability to command other any normal animal of approximately the same kind as itself (for warhorses and warponies, this category includes donkeys, mules, and ponies), as long as the target creature has fewer Hit Values than the mount. This ability functions like the \spell{command} spell, but the mount must make a DC 17 Concentration check to succeed if it's being ridden at the time. If the check fails, the ability does not work that time, but it still counts against the mount's daily uses. Each target may attempt a Will save (DC 10 \add 1/2 paladin's level \add paladin's Cha) to negate the effect.

\subcf{Spell Resistance (Ex)} A mount's spell resistance equals its master's paladin level \add 5. To affect the mount with a spell, a spellcaster must get a result on a caster level check (1d20 \add caster level) that equals or exceeds the mount's spell resistance.

\subsubsection{Sample Paladin's Mounts}
The statistics below are for the warhorse and warpony when they are first acquired, at 5th level.

\textbf{Heavy Warhorse} Large animal; HV 4\mult5\plus12; hp 32; Init \plus1; Spd
50 ft.; AC 12, touch 10, flat-footed 11; CMD 21; Base Atk \plus3; CMB \plus10; Atk \plus5 melee
(1d6\plus3, hoof); Full Atk \plus5/\plus5 melee (1d6\plus4, 2 hooves); Space/Reach 10 ft./5 ft.; SQ empathic link, improved evasion, low-light vision, scent, share spells, share saving throws; SV Fort \plus7, Ref \plus5, Will \plus2; Str 16, Dex 13, Con 16, Int 6, Wis 12, Cha 6.
\par A special mount's skills and feats depend on the choices the paladin makes.

\textbf{Warpony} Medium animal; HV 4\mult5\plus4; hp 24; Init \plus1; Spd 40
ft.; AC 13, touch 11, flat-footed 12; CMD 15; Base Atk \plus3; CMB \plus4; Atk \plus4 melee
(1d3\plus1, hoof); Full Atk \plus3/\plus3 melee (1d3\plus2, 2 hooves); Space/Reach 5
ft./5 ft.; SQ empathic link, improved evasion, low-light vision, scent, share spells, share saving throws; SV Fort \plus5, Ref \plus5, Will \plus2; Str 13, Dex 13, Con 13, Int 6, Wis 12, Cha 6.
\par A special mount's skills and feats depend on the choices the paladin makes.
\end{comment}

 \subsection{Ranger}

\begin{dtable}
\lcaption{The Ranger}
\begin{tabularx}{\columnwidth}{>{\ccol}p{\levelcol} >{\ccol}p{\babcolgood} *{3}{>{\ccol}p{\savecol}} >{\lcol}X}
\thead{Level} & \thead{Base Attack Bonus} & \thead{Fort Save} & \thead{Ref Save} & \thead{Will Save} & \thead{Special} \\
1st  & \plus1                        & \plus3  & \plus1  & \plus1 & Quarry \plus2, Track, wild speech \\
2nd  & \plus2                        & \plus4  & \plus2  & \plus2 & Danger sense, favored terrain \\
3rd  & \plus3                        & \plus5  & \plus3  & \plus3 & Ranger lore \\
4th  & \plus4                        & \plus6  & \plus4  & \plus4 & Low-light vision, tracking expert \\
5th  & \plus5                        & \plus7  & \plus4  & \plus4 & Free stride, tenacious hunter \\
6th  & \plus6/\plus1                 & \plus8  & \plus5  & \plus5 & Favored terrain, ranger lore \\
7th  & \plus7/\plus2                 & \plus9  & \plus6  & \plus6 & Guide \\
8th  & \plus8/\plus3                 & \plus10 & \plus7  & \plus7 & Darkvision, quarry \plus3  \\
9th  & \plus9/\plus4                 & \plus11 & \plus7  & \plus7 & Ranger lore \\
10th & \plus10/\plus5                & \plus12 & \plus8  & \plus8 & Favored terrain (planar) \\
11th & \plus11/\plus6/\plus1         & \plus13 & \plus9  & \plus9 & Hidden hunter\\
12th & \plus12/\plus7/\plus2         & \plus14 & \plus10 & \plus10& Blindsense, advanced lore  \\
13th & \plus13/\plus8/\plus3         & \plus15 & \plus10 & \plus10& Terrain mastery  \\
14th & \plus14/\plus9/\plus4         & \plus16 & \plus11 & \plus11& Favored terrain (planar), quarry \plus4 \\
15th & \plus15/\plus10/\plus5        & \plus17 & \plus12 & \plus12& Advanced ranger lore \\
16th & \plus16/\plus11/\plus6/\plus1 & \plus18 & \plus13 & \plus13& Blindsight  \\
17th & \plus17/\plus12/\plus7/\plus2 & \plus19 & \plus13 & \plus14& Terrain mastery, unerring hunter \\
18th & \plus18/\plus13/\plus8/\plus3 & \plus20 & \plus14 & \plus14& Advanced ranger lore, favored terrain (planar)  \\
19th & \plus19/\plus14/\plus9/\plus4 & \plus21 & \plus15 & \plus15& Perfect stride  \\
20th & \plus20/\plus15/\plus10/\plus5& \plus22 & \plus16 & \plus16& Quarry \plus5, truesight
\end{tabularx}
\end{dtable}

\cd{Alignment} Any.

\cd{Hit Value} 6.

\sssecfake{Class Skills}
The ranger's class skills (and the key attribute for each skill) are Climb (Str), Jump (Str), Swim (Str), Acrobatics (Dex), Escape Artist (Dex), Ride (Dex), Stealth (Dex), Knowledge (dungeoneering) (Int), Knowledge (geography) (Int), Knowledge (nature) (Int), Heal (Wis), Perception (Wis), Survival (Wis), and Creature Handling (Cha).
\cf{Rgr}{Skill Points at 1st Level} 8.

\sssecfake{Class Features}

All of the following are class features of the ranger.

\cf{Rgr}{Weapon and Armor Proficiency}  A ranger is proficient with simple weapons, any two weapon groups, light and medium armor, and shields (except tower shields).   He is also proficient with his choice of bows, crossbows, or thrown weapons.

\cf{Rgr}{Quarry (Ex)} A ranger is a deadly hunter. As a swift action, a ranger may designate any foe he sees as his quarry. A ranger gains a \plus2 competence bonus to attack rolls, Perception checks, and Survival checks against his quarry. However, while a ranger is pursuing a quarry, he takes a \minus2 penalty on the same rolls against any target other than his quarry. A ranger may give up pursuing a quarry at any time. He may not have more than one quarry at once; if he designates a new quarry, the old target is no longer considered his quarry. If the ranger does not see his quarry for more than a week, it is no longer considered his quarry.

\par A ranger can designate a quarry a number of times per day equal to 1 \add half her Wisdom, to a maximum number of uses per day equal to his ranger class level. The ranger's quarry bonus improves to \plus3 at 8th level, to \plus4 at 14th level, and finally to \plus5 at 20th level.

\cf{Rgr}{Track} A ranger gains Track as a bonus feat.

\cf{Rgr}{Wild Speech (Su)} A ranger has the ability to communicate with animals. This ability functions like the druid ability of the same name. A ranger can use this ability a number of times per day equal to half his ranger level \add his Charisma.

\cf{Rgr}{Danger Sense (Ex)} Starting at 2nd level, a ranger has an intuitive sense that alerts him to danger, giving him a competence bonus on initiative checks equal to half his ranger level.
\par If a character has danger sense from a multiple classes, the character stacks those levels to determine his bonus from danger sense.

\cf{Rgr}{Favored Terrain (Ex)} At 2nd level, a ranger becomes particularly attuned to certain kinds of terrain. He chooses one kind of terrain to select as a favored terrain from the list below. Usually, rangers favor their home terrain, but a ranger may choose any kind of terrain that he has personally experienced at least once. At 6th, 10th, 14th, and 18th level, the ranger may choose an additional favored terrain.
\par While in a favored terrain, a ranger gains a \plus2 competence bonus to Perception, Stealth, and Survival checks. If he desires, he may leave no trace of his passage, causing attempts to track him to take a \minus20 penalty. In addition, his experience with his favored terrain grants the ranger a single ability, regardless of whether he is currently in that terrain or not. The options for favored terrains are listed below.
\subcf{Aquatic} The ranger gains a swim speed equal to his base land speed. If he already has a swim speed, he gains a \plus10 competence bonus to his swim speed.
\subcf{Cold} The ranger gains cold damage reduction 5.
\subcf{Desert} The ranger becomes immune to heat effects and exhaustion. Anything that would make him exhausted makes him fatigued instead.
\subcf{Forest} The ranger gains Skill Focus (Stealth) as a bonus feat.
\subcf{Mountains} The ranger gains a climb speed equal to his base land speed. If he already has a climb speed, he gains a \plus10 competence bonus to his climb speed.
\subcf{Plains} The ranger gains Skill Focus (Perception) as a bonus feat.
\subcf{Swamp} 
\subcf{Underground} The ranger gains Blind-Fight as a bonus feat.
\subcf{Urban} The ranger gains Skill Focus (Persuasion) as a bonus feat.

\cf{Rgr}{Ranger Lore} At 3rd level, a ranger can choose an additional ability drawn from ancient ranger lore. All ranger lore abilities are extraordinary abilities unless specified otherwise. At 6th level and every 3 levels thereafter, the ranger gains an additional ranger lore ability. He may choose from any of the following options.

\subcf{Combat Style} The ranger is skilled with the traditional ranger combat styles. He gains the Precise Shot and Two-Weapon Fighting feats if he meets the prerequisites. However, the benefits of this lore apply only when the ranger uses light or no armor.

\subcf{Evasion} If the ranger makes a successful Reflex saving throw against an attack that normally deals half damage on a successful save, he instead takes no damage. A helpless ranger does not gain the benefit of evasion.

\subcf{Fast Movement} The ranger gains a \plus10 foot competence bonus to movement speed.

\subcf{Favored Enemy} The ranger increases his quarry bonus by \plus2 against creatures of a particular kind. The possible creature options are listed below.

\begin{dtable}
\begin{tabularx}{\columnwidth}{X X}
Animals and vermin & Humanoids (uncivilized) \\
Dragons & Oozes and plants \\
Fey & Outsiders (inner planes) \\
Giants and monstrous humanoids & Outsiders (outer planes) \\
Humanoids (civilized)  & Undead and constructs \\
\end{tabularx}
\end{dtable}

\subcf{Master of the Hunt} The ranger may use a standard action to share the benefits of his quarry ability with all allies who can see and hear him. The bonus his allies get is considered an enhancement bonus.

\subcf{Improved Combat Style} The ranger increases his skill in the traditional ranger combat styles. He adds half his Wisdom to damage when using ranged attacks or when attacking with two weapons at once. Natural weapons qualify for this purpose if the ranger attacks with two natural weapons at once.

The ranger must have the combat style lore to select this lore. The benefits of this lore apply only when the ranger uses light or no armor.

\subcf{Scent} The ranger gains scent, as the monster ability.

\subcf{Trapfinding} The ranger gains trapfinding, as the rogue skill trick.

\begin{comment}
\cf{Rgr}{Animal Companion (Ex)}:  At 4th level, a ranger gains an animal
companion selected from the following list: badger, camel, dire rat,
dog, riding dog, eagle, hawk, horse (light or heavy), owl, pony, snake
(Small or Medium viper), or wolf. If the campaign takes place
wholly or partly in an aquatic environment, the following creatures may be added
to the ranger's list of options: crocodile,
porpoise, Medium shark, and squid. This animal is a loyal
companion that accompanies the ranger on his adventures as
appropriate for its kind.

\par This ability functions like the druid ability of the same name (see \pref{Drd:Animal Companion (Ex)}), except that the ranger's effective druid level is equal to his ranger level \sub 3.
\par For example, the animal companion of an 8th-level
ranger would be the equivalent of a 5th-level druid's animal
companion.
\end{comment}

\cf{Rgr}{Low-light Vision (Ex)} At 4th level, a ranger's sight improves, allowing him to see in conditions of dim light more easily. He gains low-light vision, as the elf racial ability. If he already has low-light vision, he doubles its benefit, allowing him to see four times as far as a human in poor illumination.

\cf{Rgr}{Tracking Expert (Ex)} At 4th level, a ranger's ability to track his foes improves. He may always take 10 on Survival checks made to track, even if conditions would otherwise prevent this. Additionally, he can move at his normal speed while following tracks without taking the normal \minus5 penalty. He takes only a \minus10 penalty (instead of the normal \minus20) when moving at up to twice normal speed while tracking.

\begin{comment}
\cf{Rgr}{Spells} Beginning at 4th level, a ranger gains the ability to cast a small number of divine spells, which are drawn from the ranger spell list (\pref{Ranger Spells}).

Rangers do not require verbal components to cast spells, even if the spell would normally require a verbal component. Rangers need only pay respect to nature to invoke their divine magic.

To learn or cast a spell, a ranger must have a Wisdom score at least equal to 10 \add the spell level (Wis 11 for 1st-level spells, Wis 12 for 2nd-level spells, and so forth). The Difficulty Class for a saving throw against a ranger's spell is 10 \add the spell level \add the ranger's Wisdom.

Like other spellcasters, a ranger can cast only a certain number of spells of each spell level per day. His base daily spell allotment is given on \trefnp{The Ranger}. In addition, he receives bonus spells per day if he has a high Wisdom score (see \trefcp{Attributes and Bonus Spells}).

 A ranger learns and casts spells the way a cleric does, though he does not have access to any domain spells or granted powers, as a cleric does, and cannot sacrifice a readied spell slot to cast a \spell{cure} spell in its place. A ranger may learn and cast any spell on the ranger spell list, provided that he can cast spells of that level.

A ranger's selection of spells is limited. He knows as many spells at each level as a paladin does, and can swap spells in the same way as the paladin.

\par Through 3rd level, a ranger has no caster level or magic level. At 4th level and higher, his caster level is equal to his ranger level / 2, and his magic level is equal to his ranger level.

\begin{dtable}
\lcaption{Ranger Spells Known}
\begin{tabularx}{\columnwidth}{X *{4}{>{\ccol}X}}
& \multicolumn{4}{c}{\thead{---{}---{}---{}---{}---Spells Known---{}---{}---{}---{}---}} \\
\thead{Level} & \thead{1st} & \thead{2nd} & \thead{3rd} & \thead{4th} \\
1st  & \x & \x & \x & \x \\
2nd  & \x & \x & \x & \x \\
3rd  & \x & \x & \x & \x \\
4th  & 1 & \x & \x & \x \\
5th  & 2 & \x & \x & \x \\
6th  & 2 & \x & \x & \x \\
7th  & 3 & \x & \x & \x \\
8th  & 3 & 1 & \x & \x \\
9th  & 3 & 2 & \x & \x \\
10th & 4 & 2 & \x & \x \\
11th & 4 & 2 & 1 & \x \\
12th & 4 & 3 & 2 & \x \\
13th & 4 & 3 & 2 & \x \\
14th & 4 & 3 & 2 & 1 \\
15th & 4 & 3 & 3 & 2 \\
16th & 4 & 3 & 3 & 2 \\
17th & 4 & 3 & 3 & 2 \\
18th & 4 & 3 & 3 & 3 \\
19th & 4 & 3 & 3 & 3 \\
20th & 4 & 3 & 3 & 3 \\
\end{tabularx}
\end{dtable}
\end{comment}

\cf{Rgr}{Free Stride (Ex)} At 5th level, a ranger can move through any sort of natural terrain that slows or impedes movement at his normal speed without suffering any sort of impairment. If a skill check, such as Climb or Swim, would normally be required to move through the terrain, this ability does not help.

\cf{Rgr}{Tenacious Hunter (Ex)} At 5th level, a ranger's ability to pursue his quarry improves. He adds his quarry bonus as a competence bonus to his dodge modifier and saving throws against attacks that his quarry makes.

\cf{Rgr}{Guide (Ex)} At 7th level, whenever the ranger is are in his favored terrain, all allies that can see and hear the ranger gain his favored terrain bonuses in that terrain as well.

\cf{Rgr}{Darkvision (Ex)} At 8th level, a ranger's sight improves again, and he gains the ability to see even when there is no light at all. He gains darkvision out to 60 feet, as the dwarf ability. If he already has darkvision, he increases its range by 60 feet.

\cf{Rgr}{Greater Combat Style (Ex)} At 10th level, a ranger's aptitude in combat improves again. He is treated as having the Ambidexterity feat (\pref{Ambidexterity [Combat]}) and the Manyshot feat (\pref{Manyshot [Combat]}), even if he does not have the normal prerequisites for those feats.
\par As before, the benefits of the ranger's chosen style apply only when he wears light or no armor. He loses all benefits of his combat style when wearing medium or heavy armor.

\cf{Rgr}{Favored Terrain (Planar) (Ex)} After 10th level, a ranger may choose any plane as a favored terrain in addition to his normal options whenever he gains a new favored terrain. He is immune to any hostile planar effects from any plane he has chosen as favored terrain. In addition, he gains a \plus2 competence bonus to Knowledge checks relating to the plane and is always treated as trained in Knowledge (planar) for the purpose of such checks.

\cf{Rgr}{Hidden Hunter (Su)} At 11th level, the ranger becomes even more difficult for his quarry to detect. He adds his quarry bonus to his Stealth checks against his quarry. In addition, he continuously benefits from the effect of the \spell{nondectection} spell against all attempts that his quarry makes to detect him magically. The effect uses a caster level equal to his ranger level.

\cf{Rgr}{Advanced Ranger Lore} After 12th level, the ranger can choose an advanced ranger lore in place of a regular ranger lore. All advanced lore abilities are extraordinary abilities unless otherwise indicated. His options for advanced ranger lores are listed below.

\subcf{Camouflage} The ranger can use the Hide skill in any of his favored terrains, even if the terrain does not grant cover or concealment.

\subcf{Combat Style Mastery} The ranger's abilities with traditional ranger combat styles reach their peak. When using a ranged weapon, he can take a move action to study the weak points of a foe within one range increment. If he does, the next attack he makes against that foe, if it is made in the same turn, is made as a ranged touch attack. When wielding two weapons at once, he gains the pounce ability, allowing him to take a full attack action at the end of a charge.

The ranger must have the greater combat style lore to choose this lore. The benefits apply only if the ranger is wearing light or no armor.

\subcf{Greater Combat Style} The ranger's abilities with traditional ranger combat styles improves again. He gains the Two-Weapon Rend and Manyshot feats if he meets the prerequisites. He must have the improved combat style lore to choose this lore. The benefits of this lore only apply if the ranger is wearing light or no armor.

\subcf{Hail of Arrows} A number of times per day equal to 1 \add half the ranger's Constitution, he may take a full-round action to fire a single arrow at every enemy within a \areamed radius. All enemies must be within one range increment of the ranger. This lore can be used with any ranged weapon that the ranger is capable of making a full attack with.

\subcf{Improved Evasion} The ranger's ability to avoid damage improves. He still takes no damage on a successful Reflex saving throw against attacks, but henceforth he takes only half damage on a failed save. A helpless ranger does not gain the benefit of improved evasion.

\subcf{Storm of Blades} A number of times per day equal to 1 \add half the ranger's Constitution, he may take a standard action to make a single melee attack against every enemy he threatens.

\cf{Rgr}{Blindsense (Ex)} At 12th level, a ranger's perceptions are so finely honed that he can sense his enemies without seeing them. He gains the blindsense ability out to 60 feet. This ability allows him to sense the presence and location of objects and foes within 60 feet without seeing them. If he already has the blindsense ability, he increases its range by 60 feet.

\cf{Rgr}{Terrain Mastery (Ex)} At 13th level, a ranger gains a greater degree of mastery over some of his favored terrains. He chooses a single kind of terrain that he has already chosen as a favored terrain. At 17th level, he chooses an additional kind of terrain to master.
\par While in that terrain, his competence bonuses on Perception, Stealth, and Survival checks increase to \plus4. In addition, he gains another ability based on that terrain that is constantly active, whether or not he is currently in the terrain. The options for terrain masteries are given below.
\subcf{Aquatic} The ranger does not suffer penalties for fighting and moving underwater.
\subcf{Cold} The ranger gains cold damage reduction 30.
\subcf{Desert} The ranger becomes immune to fatigue.
\subcf{Forest} The ranger may use his wild speech ability to communicate with plants, as the druid ability.
\subcf{Mountains} The ranger is always treated as if he had a running start when making Jump checks. In addition, he takes only half damage from falling damage.
\subcf{Plains} The ranger gains a \plus10 competence bonus to his land speed.
\subcf{Swamp} The ranger becomes immune to nausea.
\subcf{Underground} The ranger increases the range of his darkvision and blindsense by 60 feet.
\subcf{Urban} The ranger can treat cities as being natural terrain for the purpose of his camouflage and hide in plain sight abilities.

\cf{Rgr}{Blindsight (Ex)} At 16th level, a ranger gains the ability to ``see'' perfectly without his eyes in a 60 foot radius around him. With this ability, he can fight just as well with his eyes closed as with them open. If he already has the blindsight ability, he increases its range by 60 feet.

\cf{Rgr}{Unerring Hunter (Su)} At 17th level, a ranger's ability to hunt down his quarry improves to supernatural levels. Once per day, the ranger may concentrate for a full round to duplicate the effects of the \spell{discern location} spell targeted at his quarry.

\cf{Rgr}{Hide in Plain Sight (Ex)} While in any sort of natural terrain, a ranger of 17th level or higher can use the Hide skill even while being observed. He still needs cover or concealment to hide.

\cf{Rgr}{Perfect Stride (Su)} At 19th level, a ranger's ability to surpass obstacles becomes unparalleled. He constantly acts as if he were under the effect of a \spell{freedom} spell, except that it does not allow him to act normally underwater.

\cf{Rgr}{Truesight (Su)} At 20th level, a ranger's perceptions are accurate enough to defeat even powerful magic. He gains the ability to see all things as they actually are, as the \spell{true seeing} spell, out to a range of 60 feet.

\subsection{Rogue}
\begin{dtable*}
\lcaption{The Rogue}
\begin{tabularx}{\textwidth}{>{\ccol}p{\levelcol} >{\ccol}p{\babcolgood} *{3}{>{\ccol}p{\babcolgood}} X}
\thead{Level} & \thead{Base Attack Bonus} & \thead{Fort Save} & \thead{Ref Save} & \thead{Will Save} & \thead{Special} \\
1st  & \plus0                & \plus0 & \plus3  & \plus0 & Sneak attack \plus1d6 \\
2nd  & \plus1                & \plus1 & \plus4  & \plus1 & Uncanny dodge, skill trick, danger sense \\
3rd  & \plus2                & \plus1 & \plus5  & \plus1 & Ambush attack \plus1d6 \\
4th  & \plus3                & \plus2 & \plus6  & \plus2 & Evasion, combat trick \\
5th  & \plus3                & \plus2 & \plus7  & \plus2 & Skill trick, sneak attack \plus2d6 \\
6th  & \plus4                & \plus3 & \plus8  & \plus3 & Improved uncanny dodge \\
7th  & \plus5                & \plus3 & \plus9  & \plus3 & Ambush attack \plus2d6, combat trick\\
8th  & \plus6/\plus1         & \plus4 & \plus10 & \plus4 & Skill trick \\
9th  & \plus6/\plus1         & \plus4 & \plus11 & \plus4 & Sneak attack \plus3d6 \\
10th & \plus7/\plus2         & \plus5 & \plus12 & \plus5 & Greater uncanny dodge, combat trick \\
11th & \plus8/\plus3         & \plus5 & \plus13 & \plus5 & Advanced skill trick, ambush attack \plus3d6 \\
12th & \plus9/\plus4         & \plus6 & \plus14 & \plus6 & Jack of all trades \\
13th & \plus9/\plus4         & \plus6 & \plus15 & \plus6 & Sneak attack \plus4d6, advanced combat trick \\
14th & \plus10/\plus5        & \plus7 & \plus16 & \plus7 & Advanced skill trick \\
15th & \plus11/\plus6/\plus1 & \plus7 & \plus17 & \plus7 & Ambush attack \plus4d6 \\
16th & \plus12/\plus7/\plus2 & \plus8 & \plus18 & \plus8 & Advanced combat trick \\
17th & \plus12/\plus7/\plus2 & \plus8 & \plus19 & \plus8 & Advanced skill trick, sneak attack \plus5d6 \\
18th & \plus13/\plus8/\plus3 & \plus9 & \plus20 & \plus9 & Master of all trades \\
19th & \plus14/\plus9/\plus4 & \plus9 & \plus21 & \plus9 & Advanced combat trick, ambush attack \plus5d6 \\
20th & \plus15/\plus10/\plus5& \plus10& \plus22 & \plus10& Ambush master, advanced skill trick
\end{tabularx}
\end{dtable*}

\cd{Alignment} Any.

\cd{Hit Value} 5.

\sssecfake{Class Skills}
The rogue's class skills (and the key attribute for each skill) are
Climb (Str), Jump (Str), Swim (Str), Acrobatics (Dex), Escape Artist (Dex),  Sleight of Hand (Dex), Stealth (Dex), Craft (Int), Devices (Int), Forgery (Int), Knowledge (dungeoneering), Knowledge (local) (Int), Linguistics (Int), Perception (Wis), Sense Motive (Wis), Bluff (Cha), Persuasion (Cha), Disguise (Cha), Intimidate (Cha), and Perform (Cha).
\cf{Rog}{Skill Points at 1st Level} 12.

\sssecfake{Class Features}
All of the following are class features of the rogue.

\cf{Rog}{Weapon and Armor Proficiency} Rogues are proficient with simple weapons,  any two weapon groups,  light armor,  and bucklers. A rogue is also proficient with saps.

\cf{Rog}{Sneak Attack}  If a rogue can catch an opponent when he is unable to defend himself effectively from her attack, she can strike a vital spot for extra damage. The rogue's attack deals extra damage if the target is flat-footed or is suffering overwhelm penalties.  This extra damage is 1d6 at 1st level, and it increases by 1d6 every four rogue levels thereafter. Should the rogue score a critical hit with a sneak attack, this extra damage is not multiplied. This damage bonus is treated as a circumstance bonus.

\par Ranged attacks can count as sneak attacks only if the target is within 30 feet. A rogue can't strike with deadly accuracy from beyond that range.

With a sap (blackjack) or an unarmed strike, a rogue can make a sneak attack that deals nonlethal damage instead of lethal damage. She cannot use a weapon that deals lethal damage to deal nonlethal damage in a sneak attack, not even with the usual \minus4 penalty.

A rogue can sneak attack only living creatures with discernible anatomies -- oozes, plants, and incorporeal creatures lack vital areas to attack. Any creature that is immune to critical hits is not vulnerable to sneak attacks. The rogue must be able to see the target well enough to pick out a vital spot and must be able to reach such a spot. A rogue cannot sneak attack while striking a creature with concealment or striking the limbs of a creature whose vitals are beyond reach.

\cf{Rog}{Danger Sense (Ex)} Starting at 2nd level, a rogue has an intuitive sense that alerts her to danger, giving her a competence bonus on initiative checks equal to half her rogue levels.
\par If a character has danger sense from a multiple classes, the character stacks those levels to determine her bonus from danger sense.

\cf{Rog}{Skill Tricks} As a rogue gains experience, she gains additional insight into how to perfect her skills. Starting at 2nd level, a rogue gains one skill trick. She gains an additional skill trick at 5th level and every three levels thereafter. A rogue cannot select an individual skill trick more than once unless otherwise stated. All skill tricks are (Ex) abilities unless otherwise noted.

\subcf{Fast Acrobatics} The rogue can move at full speed using the Acrobatics skill without penalty, and can move at half speed using the Climb skill without penalty.

\subcf{Fast Stealth} The rogue can move at up to half speed using the Stealth skill without penalty, or up to full speed at a \minus5 penalty.

\subcf{Kip Up} The rogue can stand up from a prone position as a swift action. The rogue must be wearing light armor to perform this trick.

\subcf{Knowledgeable Strike} The rogue may sneak attack almost any foe she identifies with a successful Knowledge check (see the Knowledge skill, \pref{Knowledge (Int; Trained Only)}). She still may not sneak attack incorporeal foes, even if she successfully identifies them.

\subcf{Ledge Walker} The rogue may move along narrow surfaces at full speed using the Acrobatics skill without penalty. In addition, she is not flat-footed when using Acrobatics to move along narrow surfaces.

\subcf{Lingering Poison} When the rogue applies poison to a weapon, it lasts for twice as many doses as it normally would.

\subcf{Quick Disable} It takes the rogue half the normal amount of time to disable a trap or open a lock using the Devices skill (minimum 1 round).

\subcf{Quick Search} The rogue can search an area (with the Perception skill) as a move action rather than as a full-round action.

\subcf{Rogue Crawl} While prone, the rogue can move at half speed. This movement provokes attacks of opportunity as normal.

\subcf{Skill Feat} The rogue gains a bonus skill feat (see Feats). A rogue can select this trick multiple times.

\subcf{Standing Leap} The rogue is always treated as if she had a running start when making Jump checks.

\subcf{Swift Poisoner} The rogue can apply poison to a weapon as a move action instead of a standard action.

\subcf{Trapfinding} As a full-round action, a rogue may move up 10 feet while searching every square within 10 feet of her for traps. If a rogue detects a trap partway through her movement, she may immediately stop moving.

\subcf{Trap Sense} Whenever the rogue comes within 10 feet of a trap, she receives an immediate Perception check to notice the trap. This check should be made secretly, so the rogue does not know whether she failed to notice a trap.

\cf{Rog}{Uncanny Dodge (Ex)} Starting at 2nd level, a rogue can react to danger before her senses would normally allow her to do so. She may apply her Dexterity and dodge modifier to her armor class while flat-footed.

If a rogue already has uncanny dodge from a different class, she stacks those levels to determine whether she gains improved uncanny dodge (see below) instead.

\cf{Rog}{Ambush Attack (Ex)} At 3rd level, a rogue learns how to deal extra damage when she ambushes her foe. The first time that a rogue successfully sneak attacks a particular foe in an encounter, the attack is considered an ambush attack. After she has delivered an ambush attack against that foe, she cannot make any more ambush attacks against that same foe for the rest of the encounter. A rogue can deliver no more than one ambush attack per round. This extra damage is 1d6 at 3rd level, and it increases by 1d6 every four rogue levels thereafter. This damage is treated as being sneak attack damage.

\cf{Rog}{Combat Tricks} As a rogue gains experience, she learns a small number of talents that aid her and confound her foes. Starting at 3rd level, a rogue gains one combat trick. She gains an additional combat trick at 6th level and every three levels thereafter. A rogue cannot select an individual combat trick more than once unless otherwise stated. All combat tricks are (Ex) abilities unless otherwise noted.

\par Tricks marked with an asterisk are called ambush tricks. Ambush tricks add additional effects to the rogue's ambush attacks. If an ambush trick allows a saving throw, the DC is 10 \add half the rogue's level \add the rogue's Intelligence.

\subcf{Brutal Strike*} The rogue rolls d8s instead of d6s for her sneak attack dice on this attack.

\subcf{Combat Feat} The rogue gains a combat feat (see Feats).

\subcf{Dazzling Strike*} A foe damaged by this attack must make a Fortitude save or be dazzled for 5 rounds.

\subcf{Dispelling Strike (Su)*} A foe damaged by this attack is affected by a targeted \spell{dispel magic} which affects only the highest-level spell effect active on the target. The caster level for this ability is equal to the rogue's level.

\subcf{Distracting Strike*} A foe damaged by this attack must make a Will save or be bewildered for 5 rounds.

\subcf{Extended Precision} The rogue can make sneak attacks from up to 60 feet away.

\subcf{Gut Punch*} A foe damaged by this attack must make a Fortitude save or be sickened for 5 rounds.

\subcf{Hamstring*} A foe damaged by this attack has its speed with a single movement mode halved for 5 rounds. Despite the name, this can be used on foes who do not have hamstrings, though it only affects ground movement. Some forms of movement, such as magical flight, cannot be impeded by this attack.

\subcf{Merciful Blows} The rogue suffers no penalty to attack rolls when making attacking for nonlethal damage, and can deal her full sneak attack damage.

\subcf{Tricky Maneuver} When performing combat maneuvers against foes she would be able to sneak attack, the rogue gains a circumstance bonus to attack equal to the number of sneak attack dice she would normally roll. The benefits of this trick apply even against foes immune to critical hits.

\cf{Rog}{Improved Uncanny Dodge (Ex)} At 6th level and higher, a rogue can no longer be overwhelmed as easily; she can react to multiple opponents as easily as she can react to a single attacker. The rogue is always treated as being threatened by two fewer creatures than she actually is for the purpose of determining overwhelm penalties. 

\par If a character already has uncanny dodge (see above) from a second class and gains improved uncanny dodge, the character stacks those levels to determine if she should gain greater uncanny dodge, and to determine the minimum level a rogue must be to sneak attack the character.

\cf{Rog}{Greater Uncanny Dodge (Ex)} At 10th level and higher, a rogue can no longer be overwhelmed, regardless of the number of foes surrounding her.

\cf{Rog}{Advanced Skill Trick} At 11th level, and every three levels thereafter, the rogue may choose any of the following advanced skill tricks in addition to her other options for skill tricks. All advanced skill tricks are (Ex) abilities unless otherwise noted.

\subcf{Disable Spell (Su)} The rogue can use Devices to dispel any currently active spell as if it were a magical trap. Doing so requires a full-round action that provokes attacks of opportunity. The DC to disable the spell is equal to 10 \add the caster level of the spell \add the level of the spell. If the spell is not subject to \spell{dispel magic}, it cannot be dispelled using this ability.

\subcf{Exemplar} The rogue must choose one skill that she already has Skill Focus in. She gains a \plus6 competence bonus with the skill, and may reroll it three times per day. This replaces the benefit of the Skill Focus feat for that skill. A single roll can never be rerolled more than once.

\subcf{Hide in Plain Sight} The rogue can use the Hide even while being observed. She still needs cover or concealment to hide.

\subcf{Rogue's Luck} Three times per day, the rogue can reroll any skill check. A single roll can never be rerolled more than once.

\subcf{Skill Mastery} The rogue becomes so certain in the use of certain skills that she can use them reliably even under adverse conditions. Upon gaining this trick, she selects a number of skills equal to 3 \add half her Intelligence. When making a skill check with one of these skills, she may take 10 even if stress and distractions would normally prevent her from doing so. A rogue may gain this trick multiple times, selecting additional skills for it to apply to each time.

\cf{Rog}{Advanced Combat Trick} At 12th level, and every three levels thereafter, the rogue may choose any of the following advanced combat tricks in addition to her other options for combat tricks. All advanced combat tricks are (Ex) abilities unless otherwise noted.

\subcf{Crippling Strike} A rogue with this trick can ambush attack opponents with such precision that her blows weaken and hamper them. An opponent damaged by one of her ambush attacks takes 2 points of Strength damage. Ability points lost to damage return on their own at the rate of 1 point per day for each damaged ability. If the rogue has no ambush attacks, she may use crippling strike as an ambush attack.

\subcf{Defensive Roll}  The rogue can roll with a potentially lethal blow to take less damage from it than she otherwise would.  A number of times per day equal to half the rogue's Wisdom, when she would be reduced to 0 or fewer hit points by damage in combat (from a weapon or other blow, not a spell or special ability), the rogue can attempt to roll with the damage. To use this ability, the rogue must attempt a Reflex saving throw (DC = damage dealt). If the save succeeds, she takes only half damage from the blow, and the damage is nonlethal; if it fails, she takes full damage. She must be aware of the attack and able to react to it in order to execute her defensive roll -- if she is flat-footed, she can't use this ability. Since this effect would not normally allow a character to make a Reflex save for half damage, the rogue's evasion ability does not apply to the defensive roll.

\subcf{Distant Precision} The rogue has no range limit on her sneak attacks. A rogue must have selected the extended precision combat trick before choosing this trick.

\subcf{Improved Evasion} This talent works like evasion, except that while the rogue still takes no damage on a successful Reflex saving throw against attacks, henceforth she takes only half damage on a failed save. A helpless rogue does not gain the benefit of improved evasion.

\subcf{Opportunist}  Once per round, the rogue can make an attack of opportunity against an opponent who has just been struck for damage in melee by another character. This attack counts as one of the rogue's attacks of opportunity for that round.

\subcf{Slippery Mind}  This ability represents the rogue's ability to wriggle free from magical effects that would otherwise control or compel her. If a rogue with slippery mind is affected by a mind-affecting spell or effect and fails her saving throw, she can attempt it again 1 round later at the same DC. She gets only this one extra chance to succeed on her saving throw.

\cf{Rog}{Jack of All Trades (Ex)} At 16th level, a rogue treats all skills as class skills.

\cf{Rog}{Master of All Trades (Ex)} At 18th level, a rogue is treated as having as having at least one skill point in all skills, except for trained-only skills. These ``phantom'' skill points give her ranks in the skills normally, but do not otherwise count as skill points.

\cf{Rog}{Ambush Master (Ex)} A 20th level rogue has achieved such a mastery of tricky combat that she can combine the effects of two different ambush tricks into a single ambush attack.

\subsection{Sorcerer}
\begin{dtable*}
\lcaption{The Sorcerer}
\begin{tabularx}{\textwidth}{>{\ccol}p{\levelcol} >{\ccol}p{7em} *{3}{>{\ccol}p{\savecol}} >{\lcol}X *{9}{>{\ccol}p{\spellcol}}}
& & & & & & \multicolumn{9}{c}{\thead{---{}---{}---{}---{}---{}---{}---Spells per Day---{}---{}---{}---{}---{}---}} \\
\thead{Level} & \thead{Base Attack Bonus} & \thead{Fort Save} & \thead{Ref Save} & \thead{Will Save} & \thead{Special} & \thead{1st} & \thead{2nd} & \thead{3rd} & \thead{4th} & \thead{5th} & \thead{6th} & \thead{7th} & \thead{8th} & \thead{9th} \\
1st & \plus0 & \plus0 & \plus0 & \plus3 & Arcane invocation, Rapid Metamagic, versatile spellcaster
& 3 & \x & \x & \x & \x & \x & \x & \x & \x \\
2nd & \plus1 & \plus1 & \plus1 & \plus4     & Arcane invocation
& 4 & \x & \x & \x & \x & \x & \x & \x & \x \\
3rd & \plus1 & \plus1 & \plus1 & \plus5     & Expanded spell knowledge
& 5 & \x & \x & \x & \x & \x & \x & \x & \x \\
4th & \plus2 & \plus2 & \plus2 & \plus6     & Spellblend
& 6 & 3 & \x & \x & \x & \x & \x & \x & \x \\
5th & \plus2 & \plus2 & \plus2 & \plus7     & Expanded spell knowledge
& 6 & 4 & \x & \x & \x & \x & \x & \x & \x \\
6th & \plus3 & \plus3 & \plus3 & \plus8     & \x
& 6 & 5 & 3 & \x & \x & \x & \x & \x & \x \\
7th & \plus3 & \plus3 & \plus3 & \plus9     & Expanded spell knowledge
& 6 & 6 & 4 & \x & \x & \x & \x & \x & \x \\
8th & \plus4 & \plus4 & \plus4 & \plus10    & Improved spellblend
& 6 & 6 & 5 & 3 & \x & \x & \x & \x & \x \\
9th & \plus4 & \plus4 & \plus4 & \plus11    & Expanded spell knowledge
& 6 & 6 & 6 & 4 & \x & \x & \x & \x & \x \\
10th & \plus5 & \plus5 & \plus5 & \plus12    & \x 
& 6 & 6 & 6 & 5 & 3 & \x & \x & \x & \x \\
11th & \plus5 & \plus5 & \plus5 & \plus13    & Expanded spell knowledge
& 6 & 6 & 6 & 6 & 4 & \x & \x & \x & \x \\
12th & \plus6/\plus1 & \plus6 & \plus6 & \plus14& Versatile spellblend
& 6 & 6 & 6 & 6 & 5 & 3 & \x & \x & \x \\
13th & \plus6/\plus1 & \plus6 & \plus6 & \plus15& Expanded spell knowledge
& 6 & 6 & 6 & 6 & 6 & 4 & \x & \x & \x \\
14th & \plus7/\plus2 & \plus7 & \plus7 & \plus16& \x
& 6 & 6 & 6 & 6 & 6 & 5 & 3 & \x & \x \\
15th & \plus7/\plus2 & \plus7 & \plus7 & \plus17& Expanded spell knowledge
& 6 & 6 & 6 & 6 & 6 & 6 & 4 & \x & \x \\
16th & \plus8/\plus3 & \plus8 & \plus8 & \plus18 & Spellsurge
& 6 & 6 & 6 & 6 & 6 & 6 & 5 & 3 & \x \\
17th & \plus8/\plus3 & \plus8 & \plus8 & \plus19 & Expanded spell knowledge
& 6 & 6 & 6 & 6 & 6 & 6 & 6 & 4 & \x \\
18th & \plus9/\plus4 & \plus9 & \plus9 & \plus20 & \x
& 6 & 6 & 6 & 6 & 6 & 6 & 6 & 5 & 3 \\
19th & \plus9/\plus4 & \plus9 & \plus9 & \plus21 & Expanded spell knowledge
& 6 & 6 & 6 & 6 & 6 & 6 & 6 & 6 & 4 \\
20th & \plus10/\plus5 & \plus10& \plus10& \plus22 & Improved spellsurge
& 6 & 6 & 6 & 6 & 6 & 6 & 6 & 6 & 6 \\
\end{tabularx}
\end{dtable*}
\cd{Alignment} Any.

\cd{Hit Value} 4

\sssecfake{Class Skills}
The sorcerer's class skills (and the key attribute for each skill) are Knowledge (arcana) (Int), Knowledge (the planes), Spellcraft (Wis), and Intimidate (Cha).
\cf{Sor}{Skill Points at 1st Level} 2.

\sssecfake{Class Features}
All of the following are class features of the sorcerer.

 \cf{Sor}{Weapon and Armor Proficiency}  Sorcerers are proficient with simple weapons  and one other weapon group.  They are not proficient with any type of armor or shield. Armor of any type interferes with a sorcerer's arcane gestures, which can cause his spells with somatic components to fail.

\begin{dtable}
\lcaption{Sorcerer Spells Known}
\begin{tabularx}{\columnwidth}{>{\ccol}X *{10}{>{\ccol}p{\spellcol}}}
& \multicolumn{10}{c}{\thead{---{}---{}---{}---{}---{}---{}---{}---Spells Known---{}---{}---{}---{}---{}---{}---{}---}} \\
\thead{Level} & \thead{1st} & \thead{2nd} & \thead{3rd} & \thead{4th} & \thead{5th} & \thead{6th} & \thead{7th} & \thead{8th} & \thead{9th} \\
1st  & 1 & \x & \x & \x & \x & \x & \x & \x & \x \\
2nd  & 2 & \x & \x & \x & \x & \x & \x & \x & \x \\
3rd  & 3 & \x & \x & \x & \x & \x & \x & \x & \x \\
4th  & 3 & 1 & \x & \x & \x & \x & \x & \x & \x \\
5th  & 4 & 2 & \x & \x & \x & \x & \x & \x & \x \\
6th  & 4 & 2 & 1 & \x & \x & \x & \x & \x & \x \\
7th  & 4 & 3 & 2 & \x & \x & \x & \x & \x & \x \\
8th  & 4 & 3 & 2 & 1 & \x & \x & \x & \x & \x \\
9th  & 4 & 3 & 3 & 2 & \x & \x & \x & \x & \x \\
10th & 4 & 3 & 3 & 2 & 1 & \x & \x & \x & \x \\
11th & 4 & 3 & 3 & 3 & 2 & \x & \x & \x & \x \\
12th & 4 & 3 & 3 & 3 & 2 & 1 & \x & \x & \x \\
13th & 4 & 3 & 3 & 3 & 3 & 2 & \x & \x & \x \\
14th & 4 & 3 & 3 & 3 & 3 & 2 & 1 & \x & \x \\
15th & 4 & 3 & 3 & 3 & 3 & 3 & 2 & \x & \x \\
16th & 4 & 3 & 3 & 3 & 3 & 3 & 2 & 1 & \x \\
17th & 4 & 3 & 3 & 3 & 3 & 3 & 2 & 2 & \x \\
18th & 4 & 3 & 3 & 3 & 3 & 3 & 2 & 2 & 1 \\
19th & 4 & 3 & 3 & 3 & 3 & 3 & 2 & 2 & 2 \\
20th & 4 & 3 & 3 & 3 & 3 & 3 & 2 & 2 & 2
\end{tabularx}
\end{dtable}
\cf{Sor}{Spells}  A sorcerer casts arcane spells, which are drawn primarily from the sorcerer/wizard spell list \pref{Sorcerer/Wizard Spells}. To learn or cast a spell, a sorcerer must have a Charisma score at least equal to the spell's level. The Difficulty Class for a saving throw against a sorcerer's spell is 10 \add the spell level \add the sorcerer's Charisma.

\par Like other spellcasters, a sorcerer can cast only a certain number
of spells of each spell level per day. His base daily spell allotment is
given on \trefnp{The Sorcerer}.

\par A sorcerer's selection of spells is limited. A sorcerer
begins play knowing four 0-level spells and two
1st-level spells of his choice. At each new sorcerer level, he gains
one or more new spells, as indicated on \trefnp{Sorcerer Spells Known}. These new spells can be common spells chosen from
the sorcerer/wizard spell list \pref{Sorcerer/Wizard Spells}, or they can be unusual
spells that the sorcerer has gained some understanding of by study.

\par At each level, a sorcerer can choose to learn a
new spell in place of one he already knows. In effect, the sorcerer
``loses'' the old spell in exchange for the new one. The new spell's
level must be the same as that of the spell being exchanged. A sorcerer may swap only a single
spell at any given level, and must choose whether or not to swap the
spell at the same time that he gains new spells known for the level.

\par A sorcerer can cast any spell he knows at any time, assuming he has the spell slot to cast it.

\par A sorcerer may use a higher-level slot to cast a lower-level spell if he so chooses. For example, if an 8th-level sorcerer has used up all his 3rd-level spell slots for the day but wants to cast another 3rd-level spell, he could use a 4th-level slot to do so. The spell is still treated as its actual level, not the level of the slot used to cast it.

A sorcerer's magic level is equal to his sorcerer level.
\cf{Sor}{Arcane Invocation} All sorcerers master at least one arcane invocation. An arcane invocation allows a sorcerer to exert magical influence without expending the effort required to cast a spell. The sorcerer may choose to learn one invocation of her choice from the list of arcane invocations described in Chapter 11: Spells. At 2nd level, the sorcerer learns a second arcane invocation of his choice.

In order to cast an arcane invocation, a sorcerer must wield an arcane implement, a special item with a small amount of magical energy that is attuned to the wizard. Almost any item of Tiny size or larger can function as an arcane implement. A sorcerer begins play with an arcane implement attuned to him.

\cf{Sor}{Rapid Metamagic} A sorcerer gains Rapid Metamagic as a bonus feat at 1st level, even if he does not meet the prerequisites.

\cf{Sorc}{Versatile Spellcaster (Ex)} A sorcerer's intuitive grasp of magic allows him to be flexible in his use of arcane energy. A sorcerer can use two sorcerer spell slots of the same level to use an ability requiring a sorcerer spell slot of one level higher. For example, Serric the sorcerer can use two 0th-level spell slots to cast a single 1st-level spell he knows.

If Serric were 6th level, he could use two 2nd level spell slots to use the spellblend ability, casting a 2nd level spell and an arcane invocation in the same action.

\cf{Sor}{Expanded Spell Knowledge (Ex)} At 3rd level, and every odd level afterwards, a sorcerer can learn how to cast a particularly esoteric spell. He may choose a restricted spell from the sorcerer/wizard spell list with a level of no more than half his sorcerer level (normally, the highest level spell he can cast) and add it to his spell list. He must still use a spell known to learn it, as normal.

\cf{Sor}{Spellblend (Ex)} At 4th level, a sorcerer may combine his arcane invocations with his spells. As a full-round action, the sorcerer may cast a spell that affects only himself using a spell slot one level higher than what the spell would normally require. If he does, he may also use an arcane invocation as part of the same action. The arcane invocation need not target the sorcerer.

\cf{Sor}{Improved Spellblend (Ex)} At 8th level, a sorcerer may combine two spells together. As a full-round action, the sorcerer may cast two spells at once, resolving each spell's effects separately. The spells cast in this way must be at least three spell levels apart, such as a 1st-level spell and a 4th-level spell. In addition, one of the two spells must affect only the sorcerer. Using improved spellblend costs a spell slot of one level higher than the highest level spell being cast.

\cf{Sor}{Versatile Spellblend (Ex)} At 12th level, a sorcerer may combine any two spells together. When using spellblend or improved spellblend, the sorcerer may cast any spells, regardless of whether they affect only the sorcerer. However, using versatile spellblend costs a spell slot of two levels higher than the highest level spell being cast.

\cf{Sor}{Spellsurge (Ex)} At 16th level, a sorcerer may enter a trance-like state once per day in which he can surpass his normal limits. A spellsurge trance lasts for one minute. During the trance, the sorcerer may use his versatile spellblend ability by expending a spell slot of one level lower than the highest level spell being cast. However, the sorcerer is forced to expel the arcane energy welling up inside him, and is forced to use his versatile spellblend ability with all of his actions. The sorcerer can suppress this effect for a round with a DC 25 Will save, but he takes nonlethal damage equal to his caster level if he does so.

\par At 19th level, a sorcerer can enter a spellsurge trance an additional time per day.

\cf{Sor}{Improved Spellsurge (Ex)} At 20th level, a sorcerer in a spellsurge trance can use his spellblend ability as a standard action instead of as a full-round action.

\subsection{Wizards}
\begin{dtable*}
\lcaption{The Wizard}
\begin{tabularx}{\textwidth}{>{\ccol}p{\levelcol} >{\ccol}p{7em} *{3}{>{\ccol}p{\savecol}} >{\lcol}X *{9}{>{\ccol}p{\spellcol}}}
& & & & & & \multicolumn{9}{c}{\thead{---{}---{}---{}---{}---{}---{}---Spells per Day---{}---{}---{}---{}---{}---}} \\
\thead{Level} & \thead{Base Attack Bonus} & \thead{Fort Save} & \thead{Ref Save} & \thead{Will Save} & \thead{Special} & \thead{1st} & \thead{2nd} & \thead{3rd} & \thead{4th} & \thead{5th} & \thead{6th} & \thead{7th} & \thead{8th} & \thead{9th} \\
1st & \plus0 & \plus0 & \plus0 & \plus3 & Arcane invocation, ritual master
& 3 & \x & \x & \x & \x & \x & \x & \x & \x \\
2nd & \plus1 & \plus1 & \plus1 & \plus4     & Arcane invocation, Scribe Scroll
& 4 & \x & \x & \x & \x & \x & \x & \x & \x \\
3rd & \plus1 & \plus1 & \plus1 & \plus5     & Arcane insight
& 5 & \x & \x & \x & \x & \x & \x & \x & \x \\
4th & \plus2 & \plus2 & \plus2 & \plus6     & Spell sequencer
& 6 & 3 & \x & \x & \x & \x & \x & \x & \x \\
5th & \plus2 & \plus2 & \plus2 & \plus7     & Arcane attunement (1 item), arcane insight
& 6 & 4 & \x & \x & \x & \x & \x & \x & \x \\
6th & \plus3 & \plus3 & \plus3 & \plus8     & \x
& 6 & 5 & 3 & \x & \x & \x & \x & \x & \x \\
7th & \plus3 & \plus3 & \plus3 & \plus9     & Arcane insight
& 6 & 6 & 4 & \x & \x & \x & \x & \x & \x \\
8th & \plus4 & \plus4 & \plus4 & \plus10    & Improved spell sequencer
& 6 & 6 & 5 & 3 & \x & \x & \x & \x & \x \\
9th & \plus4 & \plus4 & \plus4 & \plus11    & Arcane insight
& 6 & 6 & 6 & 4 & \x & \x & \x & \x & \x \\
10th & \plus5 & \plus5 & \plus5 & \plus12    & Arcane attunement (2 items)
& 6 & 6 & 6 & 5 & 3 & \x & \x & \x & \x \\
11th & \plus5 & \plus5 & \plus5 & \plus13    & Arcane insight
& 6 & 6 & 6 & 6 & 4 & \x & \x & \x & \x \\
12th & \plus6/\plus1 & \plus6 & \plus6 & \plus14& Contingency
& 6 & 6 & 6 & 6 & 5 & 3 & \x & \x & \x \\
13th & \plus6/\plus1 & \plus6 & \plus6 & \plus15& Arcane insight
& 6 & 6 & 6 & 6 & 6 & 4 & \x & \x & \x \\
14th & \plus7/\plus2 & \plus7 & \plus7 & \plus16& \x
& 6 & 6 & 6 & 6 & 6 & 5 & 3 & \x & \x \\
15th & \plus7/\plus2 & \plus7 & \plus7 & \plus17& Arcane attunement (3 items), arcane insight
& 6 & 6 & 6 & 6 & 6 & 6 & 4 & \x & \x \\
16th & \plus8/\plus3 & \plus8 & \plus8 & \plus18 & Versatile spell sequencer
& 6 & 6 & 6 & 6 & 6 & 6 & 5 & 3 & \x \\
17th & \plus8/\plus3 & \plus8 & \plus8 & \plus19 & Arcane insight
& 6 & 6 & 6 & 6 & 6 & 6 & 6 & 4 & \x \\
18th & \plus9/\plus4 & \plus9 & \plus9 & \plus20& \x
& 6 & 6 & 6 & 6 & 6 & 6 & 6 & 5 & 3 \\
19th & \plus9/\plus4 & \plus9 & \plus9 & \plus21 & Arcane insight
& 6 & 6 & 6 & 6 & 6 & 6 & 6 & 6 & 4 \\
20th & \plus10/\plus5 & \plus10& \plus10& \plus22 & Arcane attunement (4 items), chain contingency
& 6 & 6 & 6 & 6 & 6 & 6 & 6 & 6 & 6 \\
\end{tabularx}
\end{dtable*}
\cd{Alignment} Any.

\cd{Hit Value} 4.

\sssecfake{Class Skills}
The wizard's class skills (and the key attribute for each skill) are
Knowledge (all skills, taken individually) (Int), Linguistics (Int), and Spellcraft (Wis).
\cf{Wiz}{Skill Points at 1st Level} 4

\sssecfake{Class Features}

All of the following are class features of the wizard.

\cf{Wiz}{Weapon and Armor Proficiency} Wizards are proficient with simple weapons, but not with any type of
armor or shield. Armor of any type interferes with a wizard's movements, which can cause her spells with somatic components to fail.

\cf{Wiz}{Ritual Master} Wizards are thoroughly trained in study and memorization, and can perform difficult rituals with more ease than others. A wizard gains the Ritual Master feat as a bonus feat at 1st level, even if she does not have the prerequisites.

 \cf{Wiz}{Spells}  A wizard casts arcane spells, which are drawn from the sorcerer/wizard spell list. To learn or cast a spell, the wizard must have an Intelligence score
at least equal to the spell's level. The Difficulty Class for a saving throw against a wizard's
spell is 10 \add the spell level \add the wizard's Intelligence.

Like other spellcasters, a wizard can cast only a certain number of
spells of each spell level per day. Her base daily spell allotment
is given on \trefnp{The Wizard}.

 \par A wizard's selection of spells is limited. A wizard
begins play knowing four 0-level spells and two
1st-level spells of her choice. At each new wizard level, she gains
one or more new spells, as indicated on \trefnp{Wizard Spells Known}.

\par A wizard learns and casts spells the way a sorcerer does. A wizard may cast any spell she knows.

A wizard's magic level is equal to his wizard level.
\begin{dtable}
\lcaption{Wizard Spells Known}
\begin{tabularx}{\columnwidth}{>{\ccol}X *{10}{>{\ccol}p{\spellcol}}}
& \multicolumn{10}{c}{\thead{---{}---{}---{}---{}---{}---{}---{}---Spells Known---{}---{}---{}---{}---{}---{}---{}---}} \\
\thead{Level} & \thead{1st} & \thead{2nd} & \thead{3rd} & \thead{4th} & \thead{5th} & \thead{6th} & \thead{7th} & \thead{8th} & \thead{9th} \\
1st  & 1 & \x & \x & \x & \x & \x & \x & \x & \x \\
2nd  & 2 & \x & \x & \x & \x & \x & \x & \x & \x \\
3rd  & 3 & \x & \x & \x & \x & \x & \x & \x & \x \\
4th  & 3 & 1 & \x & \x & \x & \x & \x & \x & \x \\
5th  & 4 & 2 & \x & \x & \x & \x & \x & \x & \x \\
6th  & 4 & 2 & 1 & \x & \x & \x & \x & \x & \x \\
7th  & 4 & 3 & 2 & \x & \x & \x & \x & \x & \x \\
8th  & 4 & 3 & 2 & 1 & \x & \x & \x & \x & \x \\
9th  & 4 & 3 & 3 & 2 & \x & \x & \x & \x & \x \\
10th & 4 & 3 & 3 & 2 & 1 & \x & \x & \x & \x \\
11th & 4 & 3 & 3 & 3 & 2 & \x & \x & \x & \x \\
12th & 4 & 3 & 3 & 3 & 2 & 1 & \x & \x & \x \\
13th & 4 & 3 & 3 & 3 & 3 & 2 & \x & \x & \x \\
14th & 4 & 3 & 3 & 3 & 3 & 2 & 1 & \x & \x \\
15th & 4 & 3 & 3 & 3 & 3 & 3 & 2 & \x & \x \\
16th & 4 & 3 & 3 & 3 & 3 & 3 & 2 & 1 & \x \\
17th & 4 & 3 & 3 & 3 & 3 & 3 & 2 & 2 & \x \\
18th & 4 & 3 & 3 & 3 & 3 & 3 & 2 & 2 & 1 \\
19th & 4 & 3 & 3 & 3 & 3 & 3 & 2 & 2 & 2 \\
20th & 4 & 3 & 3 & 3 & 3 & 3 & 2 & 2 & 2
\end{tabularx}
\end{dtable}

\cf{Wiz}{Bonus Languages} A wizard may learn Draconic in addition to the bonus languages available to the character because of her race (see Chapter 2: Races). Many ancient tomes of magic are written in Draconic, and apprentice wizards often learn it as part of their studies.

\cf{Wiz}{Arcane Invocation} All wizards master at least one arcane invocation. An arcane invocation allows the wizard to exert magical influence without expending the effort required to cast a spell. The wizard may choose to learn one invocation of her choice from the list of arcane invocations described in Chapter 11: Spells. Specialist wizards must choose one of the invocations granted by their specialist school. At 2nd level, the wizard gains a second arcane invocation, which can be chosen from any non-prohibited school.

In order to cast an arcane invocation, a wizard must wield an arcane implement, a special item with a small amount of magical energy that is attuned to the wizard. Almost any item of Tiny size or larger can function as an arcane implement. A wizard begins play with an arcane implement attuned to him.

\cf{Wiz}{Scribe Scroll} At 2nd level, a wizard gains Scribe Scroll as a bonus feat. This feat enables her to create magic scrolls (see \pref{Scribe Scroll [Item Creation]}).

\cf{Wiz}{Arcane Attunement (Su)} At 5th level, a wizard gains the ability to use arcane items like scrolls and wands with particular skill. At the beginning of each day, the wizard may attune himself to a single item with a spell trigger or spell completion activation method. If he does, he gains one of several benefits, depending on the type of magic item he attunes himself to.
\subparhead{Scroll} The wizard casts the spell from the scroll at his full caster level if it is higher than the scroll's caster level, and using his full attribute.
\subparhead{Wand} The wizard casts the spell from the wand at his full caster level if it is higher than the wand's caster level. In addition, once per day, the wizard may cast a spell from the wand using a spell slot of the appropriate level instead of a charge.
\subparhead{Staff} The wizard may cast spells from the staff using spell slots of the appropriate level instead charges. If the spell would require multiple charges when cast from the staff, the spell slot consumed must be one level higher than the spell's normal level for each charge beyond the first.
\par If other items exist that can produce similar effects, the wizard may gain appropriate bonuses for attuning himself to them, using the above examples as a guide.

\par At 10th, 15th, and 20th level, the wizard gains the ability to attune herself to an additional item.

\cf{Wiz}{Arcane Insight (Ex)} At 3rd level, and every odd level afterwards, a wizard gains a greater understanding of magic. Generalist wizards gain expanded spell knowledge, as the sorcerer class feature. Specialist wizards may choose a spell of their chosen school from the sorcerer/wizard spell list, including restricted spells, and add it to their spells known. The spell's level must not be higher than half the wizard's class level -- normally, the highest level of spells that the wizard can cast.

\cf{Wiz}{Spell Sequencer (Ex)} At 4th level, a wizard gains the ability to create a sequence of a spell and invocation which she can cast rapidly later. To create a spell sequencer, the wizard must cast a spell which affects only herself and an arcane invocation, which may affect any target. Neither has any effect immediately. The wizard may later use a full-round action to cast both the spell and the invocation at once, choosing the target of the invocation at that time.
\par The wizard may have only one spell sequencer active at any time. If she creates a new spell sequencer, it replaces her existing spell sequencer.

\cf{Wiz}{Improved Spell Sequencer (Ex)} At 8th level, a wizard gains the ability to create a sequence of two spells which she can cast rapidly later. To create an improved spell sequencer, the wizard must cast two spells, one of which affects only herself. The spells must be at least three levels apart. Neither has any effect immediately. the wizard may later use a full-round action to cast both spells at once.
\par The wizard may have only one spell sequencer or improved spell sequencer active at any time.

 \subsubsection{School Specialization}
A school is one of eight groupings of spells, each defined by a common theme. If desired, a wizard may specialize in one school of magic (see below). Specialization allows a wizard to gain access to extra spells from her chosen school, but she then never learns to cast spells from some other schools.

\par A specialist wizard adds all restricted sorcerer/wizard spells of her school to her personal spell list.  The wizard must choose whether to specialize and, if she does so, choose her specialty at 1st level. At this time, she must also give up two other schools of magic (unless she chooses to specialize in divination; see below), which become her prohibited schools. A wizard can never give up divination to fulfill this requirement.

Spells of the prohibited school or schools are not available to the wizard, and she
can't even cast such spells from scrolls or fire them from wands. She
may not change either her specialization or her prohibited schools later.
\par The eight schools of arcane magic are abjuration, conjuration,
divination, enchantment, evocation, illusion, necromancy, and transmutation. Spells that do not fall into any of these schools are called universal spells.
\subcf{Abjuration} Spells that protect, block, banish, or deal with magic itself. An abjuration specialist is called an abjurer.
\subcf{Conjuration} Spells that bring creatures or materials to the caster. A
conjuration specialist is called a conjurer.
\subcf{Divination} Spells that reveal information. A divination specialist is
called a diviner. Unlike the other specialists, a diviner must give up only
one other school.
\subcf{Enchantment} Spells that affect the mind or grant the caster power over another being. An enchantment specialist is called an enchanter.
\subcf{Evocation} Spells that manipulate energy or other powers to produce effects. An evocation specialist is called an evoker.
\subcf{Illusion} Spells that alter perception or create false images. An illusion
specialist is called an illusionist.
\subcf{Necromancy} Spells that manipulate, create, or destroy life or life force.
A necromancy specialist is called a necromancer.
\subcf{Transmutation} Spells that transform the recipient physically or change
its properties in a more subtle way. A transmutation specialist is called a
transmuter.
\subcf{Universal} Not a school, but a category for spells that all wizards can
learn. A wizard cannot select universal as a specialty school or as a
prohibited school. Only a limited number of spells fall into this category.

\subsubsection{Arcane Spells And Armor}
Wizards and sorcerers do not know how to wear armor effectively.

If desired, they can wear armor anyway (though they'll be clumsy in it), or they can gain training in the proper use of armor (with the various Armor Proficiency feats -- light, medium, and heavy -- and the Shield Proficiency feat), or they can multiclass to add a class that grants them armor proficiency. Even if a wizard or sorcerer is wearing armor with which he or she is proficient, however, it might still interfere with spellcasting.

Armor restricts the complicated gestures that a wizards or sorcerer must make while casting any spell that has a somatic component (most do). The armor and shield descriptions list the arcane spell failure chance for different armors and shields.

By contrast, bards not only know how to wear light and medium armor effectively, but they can also ignore the arcane spell failure chance for such armor. A bard wearing armor heavier than light or using any type of shield incurs the normal arcane spell failure chance, even if he becomes proficient with that armor.

If a spell doesn't have a somatic component, an arcane spellcaster can cast it with no problem while wearing armor. Such spells can also be cast even if the caster's hands are bound or if he or she is grappling (although Concentration checks still apply normally). Also, the metamagic feat Still Spell allows a spellcaster to cast a spell at one spell level higher than normal without the somatic component. This also provides a way to cast a spell while wearing armor without risking arcane spell failure.

\begin{comment}

\section{Experience And Levels}
Experience points (XP) measure how much your character has
learned and how much he or she has grown in personal power.
\begin{cp}%
Your character earns XP by defeating monsters and other opponents. The
DM assigns XP to the characters at the end of each adventure based
on what they have accomplished. Characters accumulate XP from
one adventure to another. When a character earns enough XP, he or
she attains a new character level (see \trefcp{Experience and Level-Dependent Benefits}).
\end{cp}

\parhead{Advancing a Level} When your character's XP total reaches at
least the minimum XP needed for a new character level (see Table
3-2), he or she ``goes up a level.''
\begin{cp}%
For example, when Tordek obtains
1,000 or more XP, he becomes a 2nd-level character. As soon as he
accumulates a total of 3,000 XP or higher (2,000 more than he had
when he gained 2nd level), he reaches 3rd level. Going up a level
provides the character with several immediate benefits (see below).
\end{cp}

\begin{cp}%
\section{Level Advancement}
Each character class description includes a table that shows how the
class features and statistics increase as a member of that class
advances in level. When your character attains a new level, make
these changes.
\begin{enumerate*}
\itemhead{Choose Class:} A typical character has only one class, and when
he or she attains a new level, it is a new level in that class. If your
character has more than one class or wants to acquire a new class,
you choose which class goes up one level. The other class or classes
stay at the previous level. (See \pcref{Multiclass Characters}.)

\itemhead{Base Attack Bonus:} The base attack bonus for fighters,
barbarians, rangers, and paladins increase by 1 every level. The base
attack bonus for other characters increases at a slower rate. If your
character's base attack bonus changes, record it on your character
sheet.

\itemhead{Base Save Bonuses:} Like base attack bonuses, base save
bonuses improve at varying rates as characters increase in level.
Check your character's base save bonuses for the class that has
advanced in level to see if any of them have increased by 1. Some
base save bonuses increase at every even-numbered level; others
increase at every level divisible by three.

\itemhead{Attribute Score:} If your character has just attained 4th, 8th, 12th,
16th, or 20th character level, choose one of his or her attribute scores
and raise it by 1 point. It's the overall character level, not the class level, that counts for this
adjustment.

\par If your character's Constitution increases by 1 (see Table
1-1: Attributes and Bonus Spells, page 8), add \plus1 to his or her
hit point total for every character level below the one just attained.
For example, if you raise your character's Constitution from 11 to 12
at 4th level, he or she gets \plus3 hit points (one each for 1st, 2nd, and
3rd levels).

\itemhead{Hit Points:} Add your character's Hit Value and Constitution
modifier to his or her hit points. Even if the
character has a Constitution penalty, always add at least 1 hit point
upon gaining a new level.

\itemhead{Skill Ranks:} Each character gains skill ranks as they level up.
How many skill ranks your character has in each skill depends on how many skill points
you put into each skill when you created your character. See Chapter 4: Skills.

\itemhead{Feats:} Upon attaining 3rd level and at every third level
thereafter (6th, 9th, 12th, 15th, and 18th level), the character gains
one feat of your choice (see \trefcp{Feats}). The character
must meet any prerequisites for that feat in order to select it. As with
attribute score increases, it is the overall character level, not the class
level, that determines when a character gets a new feat.

\itemhead{Spells:} Spellcasting characters gain the ability to cast more
spells as they advance in levels. Each class description for a spell-
casting class includes a Spells per Day section (on the class table)
that shows the base number of spells (without bonus spells for high
attribute scores) of a given spell level that a character can cast at each
class level. See your character's class description in this chapter for
details.

\itemhead{Class Features:} Check your character's class description in
this chapter for any new capabilities your character may receive.
Many characters gain special attacks or new special powers as they
advance in levels.
\end{enumerate*}
\end{cp}
\end{comment}
 \section{Multiclass Characters}
A character may add new classes as he or she progresses in level, thus becoming a multiclass character. The class abilities from a character's different classes combine to determine a multiclass character's overall abilities. Multiclassing improves a character's versatility at the expense of focus.

 \subsection{Class And Level Features}
As a general rule, the abilities of a multiclass character are the sum
of the abilities of each of the character's classes.

\parhead{Level} ``Character level'' is a character's total number of levels. It is
used to determine when feats and attribute score boosts are gained, as
noted on \tref{Experience and Level-Dependent Benefits}

\par ``Class level'' is a character's level in a particular class. For a
character whose levels are all in the same class, character level and
class level are the same.

\parhead{Hit Points} A character gains hit points from each class as his or
her class level increases, adding the new hit points to the previous
total.

 \parhead{Base Attack Bonus} Add your character's levels in classes that grant
the same base attack bonus progressions together, then sum those
base attack bonuses to find your total base attack bonus. If a character would have a higher base attack bonus by treating a level with an average base attack bonus progression as a level with a poor base attack bonus progression, he or she may do so. For example, a rogue 1 / wizard 1 would have a base attack bonus of 1. A resulting value
of \plus6 or higher provides the character with multiple attacks.

\par For example, a 3rd-level rogue/5th-level wizard would have a \plus4 base attack bonus. He would get a \plus2 base attack bonus from 3 levels in a class with average base attack bonus progression (rogue) and a \plus2 base attack bonus from 5 levels in a class with poor base attack bonus progression (wizard). That gives a total base attack bonus of 2 \add 2, or \plus4. In contrast, a 3rd-level rogue/5th-level cleric would have a \plus6 base attack bonus, because she would have 8 levels in classes with average base attack bonus progression. A base attack bonus of \plus6 allows a second attack with a bonus of \plus1 (given as \plus6/\plus1 on \trefnum{Base Save and Base Attack Bonuses}), even though neither the \plus2 from the rogue levels alone nor the \plus3 from the cleric levels alone would normally allow an extra attack.

\parhead{Saving Throws} Add your character's levels in classes that grant
the same base save bonus progressions together, then sum those
base save bonuses to find your total base save bonus.

\par For example, a 3rd-level rogue/5th-level ranger has a \plus5 base save bonus on Fortitude saving throws (\plus1 from 3 levels in a class with a poor Fortitude save and \plus4 from 5 levels in a class with a good Fortitude save), a \plus6 base save bonus on Reflex saving throws (from 8 levels in classes with good Reflex saves), and a \plus2 base save bonus on Will saving throws (from 8 levels in classes with poor Will saves).

\parhead{Skills} When taking the first level in a class, if that class gives more skill points than the most skill points the character already received from a class, the character immediately gets skill points equal to the difference. For example, if a fighter took a level in rogue, he would immediately get the difference between the rogue's 12 skill points at 1st level and the fighter's 2 skill points at first level, for a total of 10 skill points.

\par In addition, if the new class grants the character class skills associated with an ability that the character previously had no class skills from, the character gains skill points for the ability as normal.

\parhead{Class Features} A multiclass character gets all the class features
of all his or her classes but must also suffer the consequences of the
special restrictions of all his or her classes. (\emph{Exception:} A character
who acquires the barbarian class does not become illiterate.)

\par In some cases, two classes can have virtually identical abilities.
Use the following guidelines to determine how abilities stack.
\begin{itemize*}
\item If two identical class features are not based on level and are not gained following a specific pattern, they do not stack.
\item If two identical class features are not explicitly based on level, but both classes gain them in a predictable pattern, the levels of the two classes stack for determining when the next improvement to the class feature will be gained.
\item If two identical class features are explicitly based on level, the levels of the two classes stack for determining the power of the ability.
\item If two identical class features say how they stack, those rules trump any other rules.
\end{itemize*}
These are some examples of how to use these guidelines.
\begin{itemize*}
\item Both a druid and a ranger gain woodland stride. A druid/ranger who has woodland stride from both classes has the same woodland stride ability as a druid or ranger would.
\item Both a barbarian and a rogue get danger sense. A barbarian/rogue adds his barbarian and rogue levels together to determine his bonus from danger sense.
\item Both a barbarian and a rogue get uncanny dodge. If a barbarian/rogue would gain uncanny dodge from both classes, she instead gains improved uncanny dodge, because uncanny dodge explicitly states how it stacks.
\end{itemize*}

\parhead{Weapon and Armor Proficiency} A character uses only the highest number of weapon proficiencies granted by her classes. If a class grants proficiency with specific weapon groups, that is counted as a chosen weapon group for the purpose of the number of weapon proficiencies the character may choose. For example, a fighter/paladin would have three weapon groups of her choice, plus the weapon group of her favored deity.

However, if a class grants proficiency with a specific weapon, it is not counted against the number of weapon groups the character gains from that class. For example, a rogue/fighter gains proficiency with four weapon groups of his choice, and is additionally proficient with saps.

\parhead{Feats} A multiclass character gains a feat every three character levels, regardless of individual class level.

\parhead{Ability Increases} A multiclass character increases one attribute score by 1 point every four character levels, regardless of individual class level.

\subsection{Magic Level and Multiclassing} The character gains spells from all of his or her spellcasting classes separately and tracks his magic level and caster level separately with each class.

Characters with magical ability gain a special benefit when multiclassing. For every two levels that a character has in nonmagical classes, the character increases his magic level as if he had gained a level in one of his magical classes. The same applies to caster level. If the character has multiple magical classes, he chooses a single class to receive this benefit each time it is gained. However, the character can only count a number of levels in nonmagical classes equal to the number of levels he has in his chosen magical class for the purpose of this benefit.

For example, Gish, a 2nd level fighter / 4th level wizard, would have a wizard magic level of 5. If he gained two more fighter levels, his wizard magic level would increase to 6. Gaining two additional fighter levels, making him a 6th level fighter / 4th level wizard, would not increase his magic level.

\section{Backgrounds}
Your class describes who you are and what you have trained in. But your character's background also forms an important part of who he or she is. The backgrounds listed here provide possible ideas for who your character was before becoming an adventurer. A character may choose up to two backgrounds.

Each background is associated with one or more skills. Characters gain the skills associated with their backgrounds as class skills, no matter what classes they take. A character with only a single background gains a free skill point which can only be spent on a skill from that background.

The backgrounds listed here are merely suggestions. You may choose to create a new background. A new background created in this way may be associated with any two skills of your choice, provided that they make sense together, as determined by the DM.

%The backgrounds do not all need to have an equal number of associated skills. Backgrounds which are unlikely to become adventurers can have fewer skills. Max of 3 skills from a single background. NPCs have one background.

\subsection{Society Backgrounds}

\subsubsection{City Watch}
\parhead{Skills} Knowledge (local), Perception

\subsubsection{Merchant}
\parhead{Skills} Persuasion, Knowledge (local)

\subsubsection{Priest}
\parhead{Skill} Heal, Knowledge (religion), Perform (oratory).

\subsubsection{Scholar}
\parhead{Skill} Knowledge (any two), Linguistics.

\subsubsection{Scribe}
\parhead{Skill} Forgery, Knowledge (any one), Linguistics.

\subsubsection{Spy}
\parhead{Skills} Bluff, Disguise, Forgery.

\subsection{Military Backgrounds}

\subsubsection{Border Guard}
\parhead{Skill} Knowledge (geography, nature), Survival.

\subsubsection{Cavalry}
\parhead{Skill} Creature Handling, Ride.

\subsubsection{Diplomat}
\parhead{Skills} Bluff, Persuasion, Sense Motive.

\subsubsection{Engineer}
\parhead{Skill} Craft (any one), Knowledge (engineering).

\subsubsection{Infiltrator}
\parhead{Skills} Disguise, Stealth

\subsubsection{Saboteur}
\parhead{Skills} Devices, Stealth

\subsubsection{Scout}
\parhead{Skills} Acrobatics, Perception

\subsection{Other Backgrounds}

\subsubsection{Bandit}
\parhead{Skills} Intimidate, Stealth

\subsubsection{Commoner}
\parhead{Skill} Profession (any one).

\subsubsection{Explorer}
\parhead{Skills} Knowledge (geography), Survival.

\subsubsection{Hermit}
\parhead{Skill} Knowledge (nature), Survival.

\subsubsection{Minstrel}
\parhead{Skill} Perform (any one).

\subsubsection{Primitive}
\parhead{Skill} Survival.

\subsubsection{Smith}
\parhead{Skill} Craft (all)

\subsubsection{Thief}
\parhead{Skills} Devices, Stealth

\subsubsection{Virtuoso}
\parhead{Skills} Perform (all)
