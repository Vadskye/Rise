\section{Rituals}

\spellsection{Alarm}
\spellschool{Abjuration (Warding) [Trap]}
\spellskill{Arcana}
\spelllvl{Arcane 1, Divine 1}
\spelltime{1 minute}
\spellrng{\rngclose}
\spellarea{\arealarge radius emanation centered on a point in space}
\spelldur{\durext (D)}
\spellsave{None}
\spellsr{No}
\begin{spelleffect}
\ritual{Alarm} sounds a mental or audible alarm each time a creature of Tiny or larger size enters the warded area. If you set a password as you cast the spell, a creature that speaks the password before entering the area does not trigger the alarm. You decide at the time of casting whether the alarm will be mental or audible.
\par \subspell{Mental Alarm} A mental alarm alerts you (and only you) so long as you remain within 1 mile of the warded area. You note a single mental ``ping'' that awakens you from normal sleep but does not otherwise disturb concentration. A \ritual{silence} spell has no effect on a mental alarm.
\par \subspell{Audible Alarm} An audible alarm produces the sound of a hand bell. It is typically clearly audible up to 100 feet away. The sound lasts for 1 round. 
\end{spelleffect}
\begin{spellnotes}
A \spell{silence} spell or similar effect can prevent the alarm from being heard. Ethereal or astral creatures do not trigger the alarm. \ritual{Alarm} can be made permanent with a permanency spell.
\end{spellnotes}

\spellsection{Animal Messenger}
\spelldesc{You compel a Tiny animal to go to a spot you designate.}
\spellschool{Enchantment (Compulsion) [Mind-Affecting]}
\spellskill{Nature}
\spelllvl{Arcane 2, Divine 2}
\spelltime{1 minute; see text}
\spellrng{\rngclose}
\spelltgt{One Tiny animal}
\spelldur{One week}
\spellsave{None; see text}
\spellsr{Yes (Will)}
\begin{spelleffect}
As soon as you begin casting the ritual, the animal approaches you and awaits your bidding. You can mentally impress on the animal a certain place well known to you or an obvious landmark. The directions must be simple, because the animal depends on your knowledge and can't find a destination on its own. During the casting of the ritual, you can attach some small item or note to the messenger. The animal then goes to the designated location and waits there, straying only to gather food and water as necessary, until the duration of the ritual expires. During this period of waiting, the messenger allows others to approach it and remove any scroll or token it carries.
\end{spelleffect}
\begin{spellnotes}
The most common use for this ritual is to get an animal to carry a message to your allies. The animal cannot be one tamed or trained by someone else, including such creatures as familiars and animal companions. The intended recipient gains no special ability to communicate with the animal or read any attached message (if it's written in a language he or she doesn't know, for example).
\end{spellnotes}
\spellfocus{Food desirable to the animal}

\spellsection{Arcane Lock}
\spellschool{Abjuration/Transmutation (Alteration, Warding)}
\spellskill{Arcana}
\spelllvl{Arcane 2}
\spelltime{1 minute}
\spellrng{Touch}
\spelltgt{The door, chest, or portal touched, up to 30 sq. ft./level in size}
\spelldur{Permanent}
\spellsave{None}
\spellsr{No}
\begin{spelleffect}
The target object is magically locked. You can freely pass your own \ritual{arcane lock} without affecting it; otherwise, a door or object secured with this ritual can be opened only by breaking in or with a successful \ritual{dispel magic} or \ritual{knock} spell. Add 10 to the normal DC to break open a door or portal affected by this ritual.
\end{spelleffect}
\begin{spellnotes}
A \ritual{knock} spell does not remove an arcane lock; it only suppresses the effect for 10 minutes.
\end{spellnotes}

\spellsection{Arcane Mark}
\spelldesc{You inscribe your personal rune or mark on a creature or object.}
\spellschool{Universal}
\spellskill{None}
\spelllvl{Arcane 1}
\spelltime{1 minute}
\spellrng{\rngtouch}
\spelleff{One personal rune or mark, all of which must fit within 1 sq. ft.}
\spelldur{Permanent}
\spellsave{None}
\spellsr{No}
\begin{spelleffect}
You etch your personal rune or mark upon any substance without harm to the material upon which it is placed. Your personal rune or mark can consist of no more than six characters. The writing can be visible or invisible.
\end{spelleffect}
\begin{spellnotes}
If an \spell{arcane mark} is placed on a living being, normal wear gradually causes the effect to fade in about a month. Arcane mark must be cast on an object prior to casting \spell{instant summons} on the same object.

This ritual does not require training any specific skill to learn, and no check is required to perform the ritual.
\end{spellnotes}

\spellsection{Augury}
\spellschool{Divination (Knowledge)}
\spellskill{Religion}
\spelllvl{Divine 2}
\spelltime{1 minute}
\spellrng{Personal}
\spelltgt{You}
\spelldur{Instantaneous}
\begin{spelleffect}
An \ritual{augury} can tell you whether a particular action will bring good or bad results for you in the immediate future.
\par The base chance for receiving a meaningful reply is 70\% \add 1\% per caster level; this roll is made secretly. A question may be so straightforward that a successful result is automatic, or so vague as to have no chance of success. If the augury succeeds, you get one of four results:
\begin{itemize*}
\item Weal (if the action will probably bring good results).
\item Woe (for bad results).
\item Weal and woe (for both).
\item Nothing (for actions that don't have especially good or bad results).
\end{itemize*}
\par If the ritual fails, you get the ``nothing'' result. A cleric who gets the ``nothing'' result has no way to tell whether it was the consequence of a failed or successful \ritual{augury}.
\par The \ritual{augury} can see into the future only about half an hour, so anything that might happen after that does not affect the result. Thus, the result might not take into account the long-term consequences of a contemplated action. All \ritual{auguries} cast by the same person about the same topic use the same dice result as the first casting.
\end{spelleffect}

\spellsection{Bless Water}
\spellschool{Evocation (Channeling) [Good]}
\spellskill{Religion}
\spelllvl{Divine 1}
\spelltime{1 minute}
\spellrng{Touch}
\spelltgt{Flask of water touched}
\spelldur{Instantaneous}
\spellsave{Will negates (object)}
\spellsr{Yes (object)}
\begin{spelleffect}
This ritual imbues a flask (1 pint) of water with holy power, turning it into holy water.
\end{spelleffect}
\begin{spellnotes}
Holy water can be thrown as a splash weapon, dealing 2d4 points of damage to a struck undead creature or an evil outsider.
\end{spellnotes}
\spellmat{5 pounds of powdered silver (worth 25 gp).}

\spellsection{Comprehend Languages}
\spelldesc{You can understand any language.}
\spellschool{Divination (Communication)}
\spellskill{Linguistics}
\spelllvl{Arcane 2, Divine 2}
\spelltime{1 minute}
\spellrng{\rngpers}
\spelltgt{You}
\spelldur{\durlong}
\begin{spelleffect}
  You can understand the spoken words of creatures or read otherwise incomprehensible written messages. The ability to read does not necessarily impart insight into the material, merely its literal meaning. The spell enables you to understand or read an unknown language, not speak or write it.
  \par Written material can be read at the rate of one page (250 words) per minute. Magical writing cannot be read, though the spell reveals that it is magical. 
\end{spelleffect}
\begin{spellnotes}
  This ritual can be foiled by certain obscuring magic (such as the \spell{secret page} and \spell{illusory script} spells). It does not decipher codes or reveal messages concealed in otherwise normal text. You may be unable to understand dead or extremely obscure languages.

  \spell[comprehend languages]{Comprehend languages} can be made permanent with a \spell{permanency} ritual.
\end{spellnotes}

\spellsection{Consecrate}
\spellschool{Evocation (Channeling) [Good]}
\spellskill{Religion}
\spelllvl{Divine 2}
\spelltime{1 minute }
\spellrng{\rngclose}
\spellarea{\areamed radius emanation}
\spelldur{24 hours}
\spellsave{None}
\spellsr{No}
\begin{spelleffect}
This ritual blesses an area with holy power. Every undead creature in a \ritual{consecrated} area suffers minor disruption, giving it a \minus2 penalty on attack rolls, saves, checks, DCs, and armor class. Undead cannot be created within or summoned into a \ritual{consecrated} area.
\end{spelleffect}
\begin{spellnotes}
\par \ritual{Consecrate} counters and dispels \ritual{desecrate}.
\end{spellnotes}

\spellsection{Continual Flame}
\spellschool{Illusion (Figment) [Light]}
\spellskill{Arcana}
\spelllvl{Arcane 2, Divine 2}
\spelltime{1 minute}
\spellrng{Touch}
\spelltgt{Object touched}
\spelleff{Magical, heatless flame}
\spelldur{Permanent}
\spellsave{None}
\spellsr{No}
\begin{spelleffect}
A flame, equivalent in brightness to a torch, springs forth from an object that you touch. The effect looks like a regular flame, but it creates no heat and doesn't use oxygen. A \spell{continual flame} can be covered and hidden, but not smothered or quenched.
\end{spelleffect}

\spellsection{Create Food and Water}
\spellschool{Conjuration (Creation)}
\spellskill{Survival}
\spelllvl{Arcane 2, Divine 2}
\spelltime{10 minutes}
\spellrng{\rngclose}
\spelleff{Food and water to sustain three humans or one horse/level for 24 hours}
\spelldur{Instantaneous; see text}
\spellsave{None}
\spellsr{No}
\begin{spelleffect}
The food that this ritual creates is simple fare of your choice -- highly nourishing, if rather bland. Food so created decays and becomes inedible within 24 hours, although it can be kept fresh for another 24 hours by casting a \ritual{purify food and water} spell on it. The water created by this ritual is just like clean rain water, and it doesn't go bad as the food does.
\end{spelleffect}

\spellsection{Create Water}
\spelldesc{You create water to ease the thirst of you and your companions.}
\spellschool{Conjuration (Creation) [Water]}
\spellskill{Survival}
\spelllvl{Divine 1}
\spellrng{\rngclose}
\spelleff{Up to 2 gallons of water/level}
\spelldur{Instantaneous}
\spellsave{None}
\spellsr{No}
\begin{spelleffect}
This ritual generates wholesome, drinkable water, just like clean rain water. Water can be created in an area as small as will actually contain the liquid, or in an area three times as large -- possibly creating a downpour or filling many small receptacles.
\end{spelleffect}
\begin{spellnotes}
Conjuration spells can't create substances or objects within a creature. Water weighs about 8 pounds per gallon. One cubic foot of water contains roughly 8 gallons and weighs about 60 pounds.
\end{spellnotes}

\spellsection{Curse Water}
\spellschool{Evocation (Channeling) [Evil]}
\spellskill{Religion}
\spelllvl{Divine 1}
\spelltime{1 minute}
\spellrng{Touch}
\spelltgt{Flask of water touched}
\spelldur{Instantaneous}
\spellsave{Will negates (object)}
\spellsr{Yes (Will)}
\begin{spelleffect}
This ritual imbues a flask (1 pint) of water with unholy power, turning it into unholy water.
\end{spelleffect}
\begin{spellnotes}
Unholy water can be thrown as a splash weapon, dealing 2d4 points of damage to a struck good outsider. It functions like holy water in all other respects.
\end{spellnotes}
\spellmat{5 pounds of powdered silver (worth 25 gp).}

\spellsection{Desecrate}
\spellschool{Evocation (Channeling) [Evil]}
\spellskill{Religion}
\spelllvl{Divine 2}
\spelltime{1 minute}
\spellrng{\rngclose}
\spellarea{\areamed radius emanation}
\spelldur{24 hours}
\spellsave{None}
\spellsr{No}
\begin{spelleffect}
This ritual imbues an area with unholy power. Every undead creature in a \ritual{desecrated} area gains a \plus2 bonus to attack rolls, checks, and saving throws. An undead creature created within or summoned into such an area gains \plus1 hit points per HV.
\end{spelleffect}
\begin{spellnotes}
\ritual{Desecrate} counters and dispels \ritual{consecrate}.
\end{spellnotes}

\spellsection{Endure Elements}
\spellschool{Abjuration (Shielding)}
\spellskill{Survival}
\spelllvl{Arcane 1, Divine 1}
\spelltime{1 minute}
\spellrng{Touch}
\spelltgt{Creature touched}
\spelldur{\durext}
\spellsave{Will negates (harmless)}
\spellsr{Yes (harmless)}
\begin{spelleffect}
A creature protected by endure elements suffers no harm from being in a hot or cold environment. It can exist comfortably in conditions between \minus50 and 140 degrees Fahrenheit without having to make Fortitude saves. The creature's equipment is likewise protected.
\end{spelleffect}
\begin{spellnotes}
\par \spell{Endure elements} doesn't provide any protection from fire or cold damage, nor does it protect against other environmental hazards such as smoke, lack of air, and so forth.
\end{spellnotes}

\spellsection{Erase}
\spellschool{Transmutation (Alteration)}
\spellskill{Arcana, Forgery}
\spelllvl{Arcane 1}
\spelltime{1 minute}
\spellrng{Touch}
\spelltgt{One scroll or two pages}
\spelldur{Instantaneous}
\spellsave{See text}
\spellsr{No}
\begin{spelleffect}
This ritual removes writings of either magical or mundane nature from a scroll or from one or two pages of paper, parchment, or similar surfaces. Nonmagical writing is automatically erased. To erase magical writing, you must succeed on a caster level check against a DC of 11 \add the caster level of the magic. 
\end{spelleffect}
\begin{spellnotes}
With this spell, you can remove \spell{explosive runes}, a \spell{glyph of warding}, a \spell{sepia snake sigil}, or an \spell{arcane mark}, but not \spell{illusory script} or a \spellindirect{symbol of death}{symbol} spell. This ritual grants no special protection against activating such traps. If you fail to erase \spell{explosive runes}, a \spell{glyph of warding}, or a \spell{sepia snake sigil}, you accidentally activate that writing instead.
\end{spellnotes}

\spellsection{False Trap}
\spellschool{Illusion (Figment) [Unreal]}
\spellskill{Arcana}
\spelllvl{Arcane 2}
\spelltime{1 standard action}
\spellrng{Touch}
\spelltgt{Object touched}
\spelldur{Permanent (D)}
\spellsave{None}
\spellsr{No}
\begin{spelleffect}
This spell makes a lock or other small mechanism seem to be trapped to anyone who can detect traps. You place the spell upon any small mechanism or device, such as a lock, hinge, hasp, cork, cap, or ratchet. Any character who searches for traps can find the trap with a DC 10 Perception check. Of course, the effect is illusory and nothing happens if the trap is ``sprung"; its primary purpose is to frighten away thieves or make them waste precious time.
\par If another phantom trap is active within 50 feet when the ritual is performed, the ritual fails.
\end{spelleffect}

\spellsection{Find Traps}
\spelldesc{You grant your ally an intuitive insight into the workings of traps, allowing her to easily spot danger ahead.}
\spellschool{Divination (Knowledge)}
\spellskill{Craft, Engineering, Perception}
\spelllvl{Arcana 2, Divine 2}
\spellrng{Touch}
\spelltgt{One touched creature}
\spelldur{\durmed}
\begin{spelleffect}
The subject gains a bonus equal to one-half your caster level on Perception checks made to find traps. In addition, she gains the trapfinding ability as a rogue (if she did not already have it). As a full-round action, she may move up 10 feet while searching every square within 10 feet of her for traps. If she detects a trap partway through her movement, she may immediately stop moving.
\end{spelleffect}
\begin{spellnotes}
\spell{Find traps} grants no ability to disable any traps found.
\end{spellnotes}

\spellsection{Floating Disk}
\spellschool{Evocation [Force]}
\spellskill{Arcana}
\spelllvl{Arcane 1}
\spelltime{1 minute}
\spellrng{\rngmed}
\spelleff{2 ft. radius disk of force}
\spelldur{\durext (D)}
\spellsave{None}
\spellsr{No}
\begin{spelleffect}
You create a slightly concave, circular plane of force that follows you about and carries loads for you. The disk is 4 feet in diameter and 1 inch deep at its center. It can hold 100 pounds of weight per caster level. (If used to transport a liquid, its capacity is 2 gallons.) The disk floats approximately 3 feet above the ground at all times and remains level. It floats along horizontally within spell range and will accompany you at a rate of no more than your normal speed each round. If not otherwise directed, it maintains a constant interval of 5 feet between itself and you. The disk winks out of existence when the spell duration expires. The disk also winks out if you move beyond range or try to take the disk more than 3 feet away from the surface beneath it. When the disk winks out, whatever it was supporting falls to the surface beneath it.
\end{spelleffect}

\spellsection{Gentle Repose}
\spellschool{Transmutation (Temporal)}
\spellskill{Heal}
\spelllvl{Arcane 2, Divine 2}
\spelltime{10 minutes}
\spellrng{Touch}
\spelltgt{Corpse touched}
\spelldur{Thirty days (D)}
\spellsave{Will negates (object)}
\spellsr{Yes (object)}
\begin{spelleffect}
You preserve the remains of a dead creature so that they do not decay. Doing so effectively extends the time limit on raising that creature from the dead (see \spell{raise dead}). Days spent under the influence of this spell don't count against the time limit. Additionally, this spell makes transporting a fallen comrade more pleasant.
\par The spell also works on severed body parts, poisons drawn from living creatures, and the like.
\end{spelleffect}

\spellsection{Identify}
\spellschool{Divination (Knowledge)}
\spellskill{Arcana, Spellcraft}
\spelllvl{Arcane 1}
\spelltime{1 hour}
\spellrng{Touch}
\spelltgts{One touched object}
\spelldur{Instantaneous}
\spellsave{None}
\spellsr{No}
\begin{spelleffect}
The spell determines all magic properties of a single magic item, including how to activate those functions (if appropriate), and how many charges are left (if any).
\end{spelleffect}
\begin{spellnotes}
\par If used on a cursed item, \ritual{Identify} only reveals the properties the item appears to have, not the properties of the curse. \ritual{Identify} does not function when used on an artifact.
\end{spellnotes}

\spellsection{Magic Aura}
\spellschool{Illusion (Glamer) [Magic]}
\spellskill{Spellcraft}
\spelllvl{Arcane 1}
\spelltime{1 minute}
\spellrng{Touch}
\spelltgt{One touched object weighing up to 5 lb./level}
\spelldur{Thirty days (D)}
\spellsave{None; see text}
\spellsr{No}
\begin{spelleffect}
You alter an item's aura so that it registers  to  detect spells (and spells with similar  capabilities) as though it were less magical, or a magic item of a kind you specify, or the subject of a spell you specify. You can increase or decrease the strength of the aura of an item by an amount up to your caster level. For example, if you have a caster level of 8, you could alter a magic item with a caster level of 6 to make it seem as if it were nonmagical, or you could make it seem as if it had a caster level of 14.
\par If the object bearing \spell{magic aura}  has \spell{identify} cast on it or is similarly examined, the examiner recognizes that the aura is false and detects the object's actual qualities if he succeeds on a Will save.
\par Otherwise, he believes the aura and no  amount of testing reveals what the true  magic is.
\par If the targeted item's own aura is exceptionally powerful (if it is an artifact, for  instance), \spell{magic aura} doesn't work.
\subparhead{Note} A magic weapon, shield, or suit of  armor must be a masterwork item, so a  sword of average make, for example, looks  suspicious if it has a magical aura.
\end{spelleffect}

\spellsection{Magic Mouth}
\spellschool{Illusion (Glamer)}
\spellskill{Arcana}
\spelllvl{Arcane 1}
\spelltime{1 standard action}
\spellrng{\rngclose}
\spelltgt{One creature or object}
\spelldur{One month or until discharged (D)}
\spellsave{Will negates (object)}
\spellsr{Yes (object)}
\begin{spelleffect}
This spell imbues the chosen object or creature with an enchanted mouth that suddenly appears and speaks its message the next time a specified event occurs. The message, which must be twenty-five or fewer words long, can be in any language known by you and can be delivered over a period of 10 minutes. The mouth cannot utter verbal components, use command words, or activate magical effects. It does, however, move according to the words articulated; if it were placed upon a statue, the mouth of the statue would move and appear to speak. Of course, magic mouth can be placed upon a tree, rock, or any other object or creature.
\par The spell functions when specific conditions are fulfilled according to your command as set in the spell. Commands can be as general or as detailed as desired, although only visual and audible triggers can be used. Triggers react to what appears to be the case. Disguises and illusions can fool them. Normal darkness does not defeat a visual trigger, but magical darkness or invisibility does. Silent movement or magical silence defeats audible triggers. Audible triggers can be keyed to general types of noises or to a specific noise or spoken word. Actions can serve as triggers if they are visible or audible. A magic mouth cannot distinguish alignment, level, Hit Values, or class except by external garb.
\par The range limit of a trigger is 100 feet. Regardless of range, the mouth can respond only to visible or audible triggers and actions in line of sight or within hearing distance.
\end{spelleffect}
\begin{spellnotes}
\par \spell{Magic mouth} can be made permanent with a permanency ritual.
\end{spellnotes}

\spellsection{Mending}
\spellschool{Transmutation (Alteration)}
\spellskill{Craft, Engineering}
\spelllvl{Arcane 1, Divine 1}
\spelltime{1 minute}
\spellrng{Touch}
\spelltgt{One object of up to 1 lb.}
\spelldur{Instantaneous}
\spellsave{Will negates (harmless, object)}
\spellsr{Yes (Will)}
\begin{spelleffect}
Mending repairs small breaks or tears in objects (but not warps, such as might be caused by a \spell{warp wood} spell). It will weld broken metallic objects such as a ring, a chain link, a medallion, or a slender dagger, providing but one break exists.
\par Ceramic or wooden objects with multiple breaks can be invisibly rejoined to be as strong as new. A hole in a leather sack or a wineskin is completely healed over by \spell{mending}. The spell can repair items which are magical, but the item's magical abilities are not restored. 
\end{spelleffect}
\begin{spellnotes}
The spell cannot affect creatures (including constructs).
\end{spellnotes}

\spellsection{Misdirection}
\spellschool{Illusion (Glamer)}
\spellskill{Spellcraft}
\spelllvl{Arcane 2}
\spelltime{1 minute}
\spellrng{\rngclose}
\spelltgt{One creature or object, up to a 10 ft. cube in size}
\spelldur{\durext (D)}
\spellsave{None or Will negates; see text}
\spellsr{No}
\begin{spelleffect}
While performing the ritual, you choose another object within range. For the duration of the ritual, the subject of \spell{misdirection} is detected as if it were the other object. No saving throw is allowed against this effect. Detection spells provide information based on the second object rather than on the actual target of the detection unless the caster of the detection succeeds on a Will save. For instance, you could make yourself detect as a tree if one were within range at casting: not evil, not lying, not magical, neutral in alignment, and so forth.
\end{spelleffect}
\begin{spellnotes}
This spell does not affect other types of divination magic (\spell{augury}, \spell{clairaudience/clairvoyance}, and the like).
\end{spellnotes}

\spellsection{Mount}
\spellschool{Conjuration (Summoning)}
\spellskill{Nature}
\spelllvl{Arcane 1}
\spelltime{1 minute}
\spellrng{\rngclose}
\spelleff{One mount}
\spelldur{\durext (D)}
\spellsave{None}
\spellsr{No}
\begin{spelleffect}
You summon a light horse or a pony (your choice) to serve you as a mount. The steed serves willingly and well. The mount comes with a bit and bridle and a riding saddle.
\end{spelleffect}

\spellsection{Pass Without Trace}
\spellschool{Transmutation (Imbuement)}
\spellskill{Survival}
\spelllvl{Divine 1}
\spelltime{1 standard action}
\spellrng{Touch}
\spelltgts{One creature/level touched}
\spelldur{\durext (D)}
\spellsave{Will negates (harmless)}
\spellsr{Yes (harmless)}
\begin{spelleffect}
The subject or subjects can move through any type of terrain and leave neither footprints nor scent. Tracking the subjects is virtually impossible by nonmagical means; the DC is increased by 20.
\end{spelleffect}

\spellsection{Prestidigitation}
\spellschool{Universal}
\spellskill{None}
\spelllvl{Arcane 1}
\spellrng{Personal/\rngclose}
\spelltgt{See text}
\spelldur{1 hour}
\spellsave{See text}
\spellsr{No}
\begin{spelleffect}
Prestidigitations are minor tricks that novice spellcasters use for practice. Once cast, a \spell{prestidigitation} spell enables you to perform simple magical effects for 1 hour on objects or creatures within \rngclose range of you. The effects are minor and have severe limitations. A prestidigitation can slowly lift 1 pound of material. It can color, clean, or soil items in a 1-foot cube each round. It can chill, warm, or flavor 1 pound of nonliving material. It cannot deal damage or affect the concentration of spellcasters. \spell{Prestidigitation} can create small objects, but they look crude and artificial. The materials created by a prestidigitation spell are extremely fragile, and they cannot be used as tools, weapons, or spell components.
\end{spelleffect}
\begin{spellnotes}
A \spell{prestidigitation} lacks the power to duplicate any other spell effects. Any actual change to an object (beyond just moving, cleaning, or soiling it) persists only 1 hour. Attended objects, such as the clothes a creature is wearing, cannot be affected.

This ritual does not require any specific skill to learn or perform. You are always considered to have succeeded at the check to perform the ritual.
\end{spellnotes}

\spellsection{Purify Food and Drink}
\spellschool{Transmutation (Alteration)}
\spellskill{Survival}
\spelllvl{Divine 1}
\spellrng{Touch}
\spelltgt{5 cu. ft. of contaminated food and water}
\spelldur{Instantaneous}
\spellsave{Fortitude negates (object)}
\spellsr{Yes (Fortitude)}
\begin{spelleffect}
This spell makes spoiled, rotten, poisonous, or otherwise contaminated food and water pure and suitable for eating and drinking. This spell does not prevent subsequent natural decay or spoilage. Unholy water and similar food and drink of significance is spoiled by \spell{purify food and drink}, but the spell has no effect on creatures of any type or magical liquids, such as potions.
\end{spelleffect}
\begin{spellnotes}
Water weighs about 8 pounds per gallon. One cubic foot of water contains roughly 8 gallons and weighs about 60 pounds.
\end{spellnotes}

\spellsection{Read Magic}
\spellschool{Divination (Knowledge)}
\spellskill{Spellcraft}
\spelllvl{Arcane 1, Divine 1}
\spellrng{Personal}
\spelltgt{You}
\spelldur{\durlong}
\begin{spelleffect}
You gain the ability to decipher magical inscriptions on objects -- books, scrolls, weapons, and the like -- that would otherwise be unintelligible. This deciphering does not normally invoke the magic contained in the writing, although it may do so in the case of a cursed scroll. Furthermore, once the spell is cast and you have read the magical inscription, you are thereafter able to read that particular writing without recourse to the use of \spell{read magic}. The spell allows you to identify a \spell{glyph of warding} with a DC 13 Spellcraft check, a \spell{greater glyph of warding} with a DC 16 Spellcraft check, or any \spellindirect{symbol of death}{symbol} spell with a Spellcraft check (DC 10 \add spell level).
\end{spelleffect}
\begin{spellnotes}
\spell{Read magic} can be made permanent with a permanency spell.
\end{spellnotes}

\spellsection{Restoration, Lesser}
\spellschool{Necromancy (Life) [Healing, Positive]}
\spellskill{Heal}
\spelllvl{Divine 2}
\spelltime{1 minute}
\spellrng{Touch}
\spelltgt{Creature touched}
\spelldur{Instantaneous}
\spellsave{Will negates (harmless)}
\spellsr{Yes (harmless)}
\begin{spelleffect}
This ritual dispels any magical effects reducing one of the subject's attribute scores or cures up to 5 points of temporary ability damage to one of the subject's attribute scores. It also eliminates any fatigue or exhaustion suffered by the character.
\end{spelleffect}
\begin{spellnotes}
This ritual does not restore permanent ability drain.
\end{spellnotes}

\spellsection{Snare}
\spellschool{Transmutation (Alteration) [Trap]}
\spellskill{Survival}
\spelllvl{Divine 2}
\spelltime{1 minute}
\spellrng{Touch}
\spelltgt{Touched nonmagical circle of vine, rope, or thong with a 2 ft. diameter \add 2 ft./level}
\spelldur{Permanent until triggered or broken}
\spellsave{None}
\spellsr{No}
\begin{spelleffect}
This ritual enables you to make a snare that functions as a magic trap. The snare can be made from any supple vine, a thong, or a rope. When you cast \spell{snare} upon it, the cordlike object blends with its surroundings (Perception DC 23 to locate). One end of the snare is tied in a loop that contracts around one or more of the limbs of any creature stepping inside the circle.
\par If a strong and supple tree is nearby, the snare can be fastened to it. The spell causes the tree to bend and then straighten when the loop is triggered, dealing 1d6 points of damage to the creature trapped and lifting it off the ground by the trapped limb or limbs. If no such tree is available, the cordlike object tightens around the creature, dealing no damage but causing it to be entangled.
\par The snare is magical. To escape, a trapped creature must make a DC 23 Escape Artist check or a DC 23 Strength check that is a full-round action. The snare has AC 7 and 5 hit points. A successful escape from the snare breaks the loop and ends the spell.
\end{spelleffect}

\spellsection{Undetectable Alignment}
\spellschool{Abjuration (Shielding)}
\spellskill{Spellcraft}
\spelllvl{Arcane 2, Divine 2}
\spelltime{1 minute}
\spellrng{\rngclose}
\spelltgt{One creature or object}
\spelldur{\durext (D)}
\spellsave{Will negates (object)}
\spellsr{Yes (Will)}
\begin{spelleffect}
This ritual conceals the alignment of an object or a creature from all forms of divination.
\end{spelleffect}

\spellsection{Unseen Servant}
\spellschool{Conjuration/Evocation (Creation, Control)}
\spellskill{Arcana}
\spelllvl{Arcane 1}
\spelltime{1 minute}
\spellrng{\rngmed}
\spelleff{One invisible, mindless, shapeless servant}
\spelldur{\durlong (D)}
\spellsave{None}
\spellsr{No}
\begin{spelleffect}
An unseen servant is an invisible, mindless, shapeless force that performs simple tasks at your command. It can run and fetch things, open unstuck doors, and hold chairs, as well as clean and mend. The servant can perform only one activity at a time, but it repeats the same activity over and over again if told to do so as long as you remain within range. It can open only normal doors, drawers, lids, and the like. It has an effective Strength score of 2 (so it can lift 20 pounds or drag 100 pounds). It can trigger traps and such, but it can exert only 20 pounds of force, which is not enough to activate certain pressure plates and other devices. It can't perform any task that requires a skill check with a DC higher than 10 or that requires a check using a skill that can't be used untrained. Its speed is 15 feet.
\par The servant cannot attack in any way; it is never allowed an attack roll. It cannot be killed, but it dissipates if it takes 6 points of damage from area attacks. (It gets no saves against attacks.) If you attempt to send it beyond the spell's range (measured from your current position), the servant ceases to exist.
\end{spelleffect}

\spellsection{Whispering Wind}
\spellschool{Divination (Communication) [Air]}
\spellskill{Arcana, Nature}
\spelllvl{Arcane 2, Divine 2}
\spelltime{1 minute}
\spellrng{10 miles}
\spelldur{\durext or until discharged}
\spellsave{None}
\spellsr{No}
\begin{spelleffect}
You send a message or sound on the wind to a designated spot. The whispering wind travels to a specific location within range that is familiar to you, provided that it can find a way to the location. A whispering wind is as gentle and unnoticed as a zephyr until it reaches the location. It then delivers its whisper-quiet message or other sound. Note that the message is delivered regardless of whether anyone is present to hear it. The wind then dissipates.
\par You can prepare the spell to bear a message of no more than twenty-five words, cause the spell to deliver other sounds for 1 round, or merely have the whispering wind seem to be a faint stirring of the air. You can likewise cause the whispering wind to move as slowly as 1 mile per hour or as quickly as 1 mile per 10 minutes.
\end{spelleffect}
\begin{spellnotes}
This spell cannot speak verbal components, use command words, or activate magical effects.
\end{spellnotes}
