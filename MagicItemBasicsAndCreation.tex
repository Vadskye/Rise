\chapter{Magic Items}

The basic rules pertaining to magic items were given in the core book. For the sake of clarity, full details about the function of magic items are given here.

\section{Magic Item Rules}

\subsection{Magic Items and Detect Magic}

When \spell{detect magic} identifies a magic item's school of magic, this information refers to the school of the spell placed within the potion, scroll, or wand, or the prerequisite given for the item. The description of each item provides its aura strength and the school it belongs to.

If more than one spell is given as a prerequisite, use the highest-level spell. If no spells are included in the prerequisites, use the following default guidelines.

\begin{dtable}
\begin{dtabularx}{\columnwidth}{>{\ccol}X >{\ccol}X}
\thead{Item Nature} & \thead{School} \\
\hline
Armor and protection items & Abjuration \\
Weapons or offensive items & Transmutation \\
Bonus to attribute score, on skill check, etc. & Transmutation
\end{dtabularx}
\end{dtable}

\subsection{Magic Item Rarity}
Some magic items are more common than others. Naturally, cheaper items tend to be more common than more expensive ones, but certain items are harder to find than their cost alone would imply. Generally, the more complex an item, the more rare it is. For example, a \magicitem{ring of protection \plus3} is more common than a \magicitem{ring of spell storing, minor}, even though they both cost about the same amount.

Sometimes, the rarity of an item directly affects its cost. Potions of uncommon spells are 50\% more expensive than potions of common spells. For this purpose, an common spell is defined as being a spell which is on the cleric spell list, druid spell list, or sorcerer/wizard general spell list as defined in the core rules. Any other spells are considered to be uncommon.

\subsection{Size and Magic Items}

When an article of magic clothing or jewelry is discovered, most of the time size shouldn't be an issue. Many magic garments are made to be easily adjustable, or they adjust themselves magically to the wearer. Size should not keep characters of various kinds from using magic items.

There may be rare exceptions, especially with racial specific items.

\parhead{Armor and Weapon Sizes} Most armor and weapons found are Medium. Armor and weapons can be reforged by a skilled smith to change their size. Doing so costs 10\% of the cost of the item.

\subsection{Damaging Magic Items}

A magic item doesn't need to make a saving throw unless it is unattended, it is specifically targeted by the effect, or its wielder rolls a natural 1 on his save. Magic items should always get a saving throw against spells that might deal damage to them - even against attacks from which a nonmagical item would normally get no chance to save. Magic items use the same saving throw bonus for all saves, no matter what the type (Fortitude, Reflex, or Will). A magic item's saving throw bonus equals 2 \add one-half its caster level (round down). The only exceptions to this are intelligent magic items, which make Will saves based on their own Wisdom scores.

Magic items, unless otherwise noted, take damage as nonmagical items of the same sort. A damaged magic item continues to function, but if it is broken, its magic ceases to function until it is repaired. If it is destroyed, all its magical power is lost.

\subsection{Repairing Magic Items}

Some magic items take damage over the course of an adventure. Repairing a broken magic item costs 10\% of the value of the item. The \spell{make whole} ritual also repairs a damaged - but not completely broken - magic item.

\subsection{Intelligent Items}

Some magic items, particularly weapons, have an intelligence all their own. Only permanent magic items (as opposed to those with a single use or those with charges) can be intelligent. (This means that potions, scrolls, and wands, among other items, are never intelligent.)

In general, fewer than 1\% of magic items have intelligence.

\subsection{Cursed Items}

Some items are cursed - incorrectly made, or corrupted by outside forces. Cursed items might be particularly dangerous to the user, or they might be normal items with a minor flaw, an inconvenient requirement, or an unpredictable nature.

\section{Creating Magic Items}

To create magic items, spellcasters use special feats. They invest time, money, and their own personal energy (in the form of \my{negative levels}) in an item's creation.

Note that all items have prerequisites in their descriptions. These prerequisites must be met for the item to be created. Most of the time, they take the form of spells that must be known by the item's creator (although access through another magic item or spellcaster is allowed). If an item lists a spell, greater and mass versions of the same spell can also be used to meet that prerequisite. For example, a spellcaster with \spell{greater totemic power} can make \magicitem{gauntlets of ogre power}, even if the spellcaster does not have \spell{totemic power}.

While item creation costs are handled in detail below, note that normally the two primary factors are the caster level of the creator and the level of the spell or spells put into the item. A creator can create an item at a lower caster level than her own, but never lower than the minimum level needed to cast the needed spell. Using metamagic feats, a caster can place spells in items at a higher level than normal.

Magic supplies for items are always half of the base price in gp. For many items, the market price equals the base price.

Armor, shields, weapons, and items with a value independent of their magically enhanced properties add their item cost to the market price. The item cost does not influence the base price (which determines the cost of magic supplies and the experience point cost), but it does increase the final market price.

In addition, some items cast or replicate spells with costly material components. For these items, the market price equals the base price plus an extra price for the spell component costs. The cost to create these items is the magic supplies cost plus the costs for the components. Descriptions of these items include an entry that gives the total cost of creating the item.

The creator also needs a fairly quiet, comfortable, and well-lit place in which to work. Any place suitable for preparing spells is suitable for making items. Creating an item requires one day per 1,000 gp in the item's base price, with a minimum of at least one day. Potions are an exception to this rule; they always take just one day to brew. The character must spend the gold at the beginning of the construction process.

For the duration of the crafting process, the character suffers a negative level, as he infuses his own life energy and magical power into the item he creates. This negative level persists after the creation of the item for a length of time equal to the time required to craft the item.

The caster works for 8 hours each day. He cannot rush the process by working longer each day. However, the days need not be consecutive, and the caster can use the rest of his time as he sees fit.

A character can work on only one item at a time. If a character starts work on a new item, all materials used on the under-construction item are wasted.

The secrets of creating artifacts are long lost.

\begin{dtable!*}
\lcaption{Summary of Magic Item Creation Costs}
\begin{dtabularx}{\textwidth}{*{6}{>{\lcol}X}}
\thead{Magic Item}  & \thead{Feat}  & \thead{Item Cost}  & \thead{Material Components\fn{2}} & \thead{Magic Supplies Cost}  & \thead{Base Price} \\
\hline
Armor  & Craft Magic Arms and Armor & Armor  & Cost \mtimes 50 (usually none) & 1/2 the value on Table: Armor and Shields & Value on Table: Armor and Shields \\
Shield   & Craft Magic Arms and Armor & Shield  & \mtimes 50 (usually none) & 1/2 the value on Table: Armor and Shields & Value on Table: Armor and Shields \\
Weapon  & Craft Magic Arms and Armor & Weapon  & \mtimes 50 (usually none)  & 1/2 the value on Table: Weapons  & Value on Table: Weapons \\
Potion  & Brew Potion &   -   & Cost (usually none) & 1/2 \mtimes 25 \mtimes level of spell \mtimes level of caster & 25 \mtimes level of spell \mtimes level of caster \\
Ring  & Forge Ring  &  -  &  \mtimes 50  & Special, see Table: Estimating Magic Item Gold Price Values, below & Special, see Table: Estimating Magic Item Gold Price Values, below \\
Rod  & Craft Rod  & \fn{1}  & \mtimes 50 (often none) & Special, see Table: Estimating Magic Item Gold Price Values, below & Special, see Table: Estimating Magic Item Gold Price Values, below \\
Scroll  & Scribe Scroll  &  -   & Cost (usually none) & 1/2 \mtimes 12.5 \mtimes level of spell \mtimes level of caster & 12.5 \mtimes level of spell \mtimes level of caster \\
Staff  & Craft Staff  & Quarterstaff (0 gp) & \mtimes 50 / (\# of charges used to activate spell) & See Creating Staffs, below  & See Creating Staffs, below \\
Wand  & Craft Wand  &  -   & \mtimes 50  &  1/2 \mtimes 375 \mtimes level of spell \mtimes level of caster   & 375 \mtimes level of spell \mtimes level of caster \\
Wondrous Item  & Craft Wondrous Item  & -  & \mtimes 50 (usually none) & Special, see Table: Estimating Magic Item Gold Price Values, below & Special, see Table: Estimating Magic Item Gold Price Values, below \\
\end{dtabularx}
1 Rods usable as weapons must include the weapon cost. \\
2 This cost is only for spells activated by the item that have costly material components. Having a spell with a costly component as a prerequisite does not automatically incur this cost if the item doesn't actually cast the spell.  \\
An item's market price is the sum of the item cost, spell component costs, and the base price.
\end{dtable!*}

\begin{dtable!*}
\lcaption{Estimating Magic Item Gold Piece Values}
\begin{dtabularx}{\textwidth}{*{3}{>{\lcol}X}}
\thead{Effect}  & \thead{Base Price}  & \thead{Example} \\
\hline
Ability bonus  & Bonus squared  \mtimes 2,000 gp  & Gloves of Dexterity \plus2 \\
Armor bonus  & Bonus squared  \mtimes 1,000 gp  & \plus1 chainmail \\
Bonus spell  & Spell level  \mtimes 2,000 gp  & Pearl of power \\
AC bonus (deflection)  & Bonus squared  \mtimes 1,000 gp  & Ring of protection \plus3 \\
AC bonus (dodge)  & Bonus squared  \mtimes 1,500 gp & Ioun stone, dusty rose prism \\
AC Bonus (natural armor)  & Bonus squared  \mtimes 1,000 gp  & Amulet of natural armor \plus1 \\
Save bonus  & Bonus squared  \mtimes 1,000 gp  & Cloak of resistance \plus5 \\
Skill bonus  & Bonus squared  \mtimes 100 gp  & Cloak of elvenkind \\
%SR 10 should be level 5 item.
%SR 15 should be level 10 item.
%SR 20 should be level 15 item.
%SR 25 should be level 20 item.
%Formula: (SR-12)*SR*200
Spell resistance  & SR \mtimes 5000  & Mantle of spell resistance \\
Weapon bonus  & Bonus squared  \mtimes 1,000 gp  & \plus1 longsword \\
Attack bonus & Bonus squared \mtimes 1,000 gp & Bracers of archery \\
Damage bonus & Bonus squared \mtimes 1,000 gp &  \\
\thead{Spell Effect}  & \thead{Base Price}  & \thead{Example} \\
%I would prefer the numbers at 20 and 40, but the difference is not worth the inconvenient prices.
Single use, spell completion  & Spell level \mtimes caster level  \mtimes 25 gp  & Scroll of \spell{haste} \\
Single use, use-activated  & Spell level \mtimes caster level  \mtimes 50 gp  & Potion of \spell{cure light wounds} \\
25 charges, spell trigger  & Spell level \mtimes caster level  \mtimes 250 gp  & Wand of \spell{fireball} \\
Command word or other activation & Spell level \mtimes caster level  \mtimes 1,000 gp\fn{1} & Cape of the mountebank \\
Use-activated or continuous  & Spell level  \mtimes caster level  \mtimes 1,000 gp\fn{2} & Lantern of revealing \\
\thead{Special}  & \thead{Base Price Adjustment}  & \thead{Example} \\
Emulates rare spell\fn{3} & Increase effective spell level by 1\fn{4} & Ring of blinking \\
Emulates personal or emanation spell & Increase effective spell level by 1\fn{4} & Lantern of revealing \\
Charges per day  & See Daily Charge Price Modifiers below  & Boots of translocation \\
Unusual space limitation\fn{5} & Multiply entire cost by 1.5  & Helm of translocation \\
No space limitation\fn{6}  & Multiply entire cost by 2  & Ioun stone \\
Multiple different abilities  & Multiply lower item costs by 2  & Helm of brilliance \\
Multiple related abilities & Multiply first lower cost by 0.75, then divide other lower costs by 2 & Staff of abjuration \\
Charged (25 charges)  & 1/2 unlimited use base price  & Ring of the ram \\
\thead{Component}  & \thead{Extra Cost}  & \thead{Example} \\
Armor, shield, or weapon  & Add cost of item  & \plus1 composite longbow \\
Spell has material component cost  & Add directly into price of item per charge\fn{7}  & Wand of stoneskin \\
\end{dtabularx}
\parhead{Spell Level} A 0-level spell is half the value of a 1st-level spell for determining price.
\parhead{Caster Level} A 1st-level spell is considered to have a caster level of 2 for determining price. \\
1 If the activation time is a swift action, multiply the cost by 2. If it is an immediate action, multiply the cost by 3. Ignore these modifiers if the spell being emulated originally had a swift or immediate casting time. If the item bestows an effect on its user, the cost is calculated differently. If the effect has a duration measured in rounds, multiply the cost by 1.5. If the effect has a duration measured in minutes or longer, price it as a continuous item instead of as an activated item. If the effect has a duration of 10 minutes/level or more, the cost cannot be reduced by giving the item limited charges per day. \\
2 If a continuous item has an effect based on a spell with a duration measured in rounds, multiply the cost by 2. If the duration of the spell is 1 minute/level, multiply the cost by 1.5. If the spell has a duration of 1 hour/level or greater, multiply the cost by 0.75. If a use-activated item does not take any action to activate, multiply the cost by 2. \\
3 Includes any spell not on the generally available spell list for clerics, druids, sorcerers, and wizards. \\
4 Increase caster level to match the spell's new level.
5 See Body Slot Affinities, below. \\
6 An item that does not take up one of the spaces on a body costs double. \\
7 If item is continuous or unlimited, not charged, determine cost as if it had 50 charges. If it has some daily limit, determine as if it had 25 charges.
\end{dtable!*}

\begin{dtable}
\lcaption{Limited Use Price Modifiers}
\begin{dtabularx}{\columnwidth}{>{\ccol}X >{\ccol}X}
\thead{Charges} & \thead{Cost Modifier} \\
\hline
1/week & Divide by 5 \\
1/day & Divide by 4 \\
2/day & Divide by 3 \\
3/day & Divide by 2 \\
5/day & Divide by 1.5 \\
\end{dtabularx}
\end{dtable}

\subsection{Magic Item Gold Piece Values}

Many factors must be considered when determining the price of new magic items. The easiest way to come up with a price is to match the new item to an item that is already priced that price as a guide. Otherwise, use the guidelines summarized on Table: Estimating Magic Item Gold Piece Values.

\parhead{Multiple Similar Abilities} For items with multiple similar abilities that don't take up space on a character's body use the following formula: Calculate the price of the single most costly ability, then add 75\% of the value of the next most costly ability, plus one-half the value of any other abilities.

\parhead{Multiple Different Abilities} Abilities such as an attack roll bonus or saving throw bonus and a spell-like function are not similar, and their values are simply added together to determine the cost. For items that do take up a space on a character's body each additional power not only has no discount but instead has a 50\% increase in price.

\parhead{0-Level Spells} When multiplying spell levels to determine value, 0- level spells should be treated as 1/2 level.

\parhead{Other Considerations} Once you have a final cost figure, reduce that number if the following condition applies:
\begin{itemize}
\itemhead{Item Requires Skill to Use:} Some items require a specific skill to get them to function. This factor should reduce the cost about 10\%.
\end{itemize}

Prices presented in the magic item descriptions (the gold piece value following the item's caster level) are the market value, which is generally twice what it costs the creator to make the item.

Since different classes get access to certain spells at different levels, the prices for two characters to make the same item might actually be different. An item is only worth two times what the caster of lowest possible level can make it for. Calculate the market price based on the lowest possible level caster, no matter who makes the item.

Not all items adhere to these formulas directly. The reasons for this are several. First and foremost, these few formulas aren't enough to truly gauge the exact differences between items. The price of a magic item may be modified based on its actual worth. The formulas only provide a starting point. The pricing of scrolls assumes that, whenever possible, a wizard or cleric created it. Potions and wands follow the formulas exactly. Staffs follow the formulas closely, and other items require at least some judgment calls.

\subsection{Caster Level}
The caster level of a magic item is both a measure of how powerful the item is and a requirement for how powerful a character must be to create it. The caster level of an item is generally based on the minimum caster level of the spell (or spells) required to forge the item, but several modifiers apply to the caster level. These are listed below. As always, discretion is necessary when creating items, and these guidelines are not always followed exactly.
\begin{dtable}
\lcaption{Item Caster Level Modifiers}
\begin{dtabularx}{\columnwidth}{>{\lcol}X l}
\thead{Condition} & \thead{CL Modifier} \\
\hline
Item effect is significantly weaker than prerequisite spell & \minus2 CL \\
Item effect is significantly stronger than prerequisite spell & \plus2 CL \\
Item has multiple effects (per significant effect beyond the first) & \plus2 CL \\
Item effect is charged (daily or otherwise) & \plus2 CL \\
Item effect has \mult1.5 cost multipler & \plus2 CL \\
Item effect is unlimited use or continuous & \plus4 CL \\
Item effect is a more powerful part of an increasingly powerful sequence of effects (such as light, moderate, and heavy fortification) & \plus4 CL \\
Item effect has \mult2 cost multipler & \plus4 CL
\end{dtabularx}
\end{dtable}

\subsection{Masterwork Items}

Masterwork items are extraordinarily well-made items. They are more expensive, but they benefit the user with improved quality. They are not magical in any way. However, only masterwork items may be enhanced to become magic armor and weapons. (Items that are not weapons or armor may or may not be masterwork items.)

\subsection{Creating Magic Armor}

To create magic armor, a character needs a heat source and some iron, wood, or leatherworking tools. He also needs a supply of materials, the most obvious being the armor or the pieces of the armor to be assembled. Armor to be made into magic armor must be masterwork armor, and the masterwork cost is added to the base price to determine final market value. Additional magic supplies costs for the materials are subsumed in the cost for creating the magic armor - half the base price of the item.

Creating magic armor has a special prerequisite: The creator's caster level must be at least three times the enhancement bonus of the armor. If an item has both an enhancement bonus and a special ability, the higher of the two caster level requirements must be met.

Magic armor or a magic shield must have at least a \plus1 enhancement bonus to have any of the abilities listed on Table: Armor Special Abilities and Table: Shield Special Abilities.

If spells are involved in the prerequisites for making the armor, the creator must have prepared the spells to be cast (or must know the spells, in the case of a sorcerer), must provide any material components or focuses the spells require, and must pay any XP costs required for the spells. The act of working on the armor triggers the prepared spells, making them unavailable for casting during each day of the armor's creation. (That is, those spell slots are expended from his currently prepared spells, just as if they had been cast.)

Creating some armor may entail other prerequisites beyond or other than spellcasting. See the individual descriptions for details.

Crafting magic armor requires one day for each 1,000 gp value of the base price.

\parhead{Item Creation Feat Required} Craft Magic Arms and Armor.

\subsection{Creating Magic Weapons}

To create a magic weapon, a character needs a heat source and some iron, wood, or leatherworking tools. She also needs a supply of materials, the most obvious being the weapon or the pieces of the weapon to be assembled. Only a masterwork weapon can become a magic weapon, and the masterwork cost is added to the total cost to determine final market value. Additional magic supplies costs for the materials are subsumed in the cost for creating the magic weapon - half the base price given on Table: Weapons, according to the weapon's total effective bonus.

Creating a magic weapon has a special prerequisite: The creator's caster level must be at least three times the enhancement bonus of the weapon. If an item has both an enhancement bonus and a special ability the higher of the two caster level requirements must be met.

A magic weapon must have at least a \plus1 enhancement bonus to have any of the abilities listed on Table: Melee Weapon Special Abilities or Table: Ranged Weapon Special Abilities.

If spells are involved in the prerequisites for making the weapon, the creator must have prepared the spells to be cast (or must know the spells, in the case of a sorcerer) but need not provide any material components or focuses the spells require. The act of working on the weapon triggers the prepared spells, making them unavailable for casting during each day of the weapon's creation. (That is, those spell slots are expended from his currently prepared spells, just as if they had been cast.)

At the time of creation, the creator must decide if the weapon glows or not as a side-effect of the magic imbued within it. This decision does not affect the price or the creation time, but once the item is finished, the decision is binding.

Creating magic double-headed weapons is treated as creating two weapons when determining cost, time, XP, and special abilities.

Creating some weapons may entail other prerequisites beyond or other than spellcasting. See the individual descriptions for details.

Crafting a magic weapon requires one day for each 1,000 gp value of the base price.

\parhead{Item Creation Feat Required} Craft Magic Arms and Armor.

\subsection{Creating Potions}

The creator of a potion needs a level working surface and at least a few containers in which to mix liquids, as well as a source of heat to boil the brew. In addition, he needs ingredients. The costs for materials and ingredients are subsumed in the cost for brewing the potion - 25 gp \mtimes  the level of the spell \mtimes  the level of the caster.

All ingredients and materials used to brew a potion must be fresh and unused. The character must pay the full cost for brewing each potion. (Economies of scale do not apply.)

The imbiber of the potion is both the caster and the target. Spells with a range of personal cannot be made into potions.

The creator must have prepared the spell to be placed in the potion (or must know the spell, in the case of a sorcerer) and must provide any material component or focus the spell requires.

Material components are consumed when he begins working, but a focus is not. (A focus used in brewing a potion can be reused.) The act of brewing triggers the prepared spell, making it unavailable for casting until the character has rested and regained spells. (That is, that spell slot is expended from his currently prepared spells, just as if it had been cast.) Brewing a potion requires one day.

\parhead{Item Creation Feat Required} Brew Potion.

\begin{twait}
Potion Base Prices (By Brewer's Class) \\
Spell Level & Clr, Drd, Wiz & Sor & Brd & Pal, Rgr* \\
0 & 25 gp & 25 gp & 25 gp &  -  \\
1st & 50 gp & 50 gp & 100 gp & 100 gp \\
2nd & 300 gp & 400 gp & 400 gp & 400 gp \\
3rd & 750 gp & 900 gp & 1,050 gp & 750 gp \\
* Caster level is half class level. \\
Prices assume that the potion was made at the minimum caster level. &

Base Cost to Brew a Potion (By Brewer's Class) \\
Spell Level & Clr, Drd, Wiz & Sor & Brd & Pal, Rgr* \\
0 & 12 gp 5 sp

\plus1 XP & 12 gp 5 sp

\plus1 XP & 12 gp 5 sp

\plus1 XP &  -  \\
1st & 25 gp

\plus2 XP & 25 gp

\plus2 XP & 50 gp

\plus4 XP & 50 gp

\plus4 XP \\
2nd & 150 gp

\plus12 XP & 200 gp

\plus16 XP & 200 gp

\plus16 XP & 200 gp

\plus16 XP \\
3rd & 375 gp

\plus30 XP & 450 gp

\plus36 XP & 525 gp

\plus42 XP & 375 gp

\plus30 XP \\
* Caster level is half class level. \\
Costs assume that the creator makes the potion at the minimum caster level. &

\end{twait}

\subsection{Creating Rings}

To create a magic ring, a character needs a heat source. He also needs a supply of materials, the most obvious being a ring or the pieces of the ring to be assembled. The cost for the materials is subsumed in the cost for creating the ring. Ring costs are difficult to formularize. Refer to Table: Estimating Magic Item Gold Piece Values and use the ring prices in the ring descriptions as a guideline. Creating a ring generally costs half the ring's market price.

Rings that duplicate spells with costly material components add in the value of 50 \mtimes the spell's component cost. Having a spell with a costly component as a prerequisite does not automatically incur this cost. The act of working on the ring triggers the prepared spells, making them unavailable for casting during each day of the ring's creation. (That is, those spell slots are expended from his currently prepared spells, just as if they had been cast.)

Creating some rings may entail other prerequisites beyond or other than spellcasting. See the individual descriptions for details.

Forging a ring requires one day for each 1,000 gp of the base price.

\parhead{Item Creation Feat Required} Forge Ring.

\subsection{Creating Rods}

To create a magic rod, a character needs a supply of materials, the most obvious being a rod or the pieces of the rod to be assembled. The cost for the materials is subsumed in the cost for creating the rod. Rod costs are difficult to formularize. Refer to Table: Estimating Magic Item Gold Piece Values and use the rod prices in the rod descriptions as a guideline. Creating a rod costs half the market value listed.

If spells are involved in the prerequisites for making the rod, the creator must have prepared the spells to be cast (or must know the spells, in the case of a sorcerer) but need not provide any material components or focuses the spells require. The act of working on the rod triggers the prepared spells, making them unavailable for casting during each day of the rod's creation. (That is, those spell slots are expended from his currently prepared spells, just as if they had been cast.)

Creating some rods may entail other prerequisites beyond or other than spellcasting. See the individual descriptions for details.

Crafting a rod requires one day for each 1,000 gp of the base price.

\parhead{Item Creation Feat Required} Craft Rod.

\subsection{Creating Scrolls}

To create a scroll, a character needs a supply of choice writing materials, the cost of which is subsumed in the cost for scribing the scroll - 12.5 gp \mtimes the level of the spell \mtimes the level of the caster.

All writing implements and materials used to scribe a scroll must be fresh and unused. A character must pay the full cost for scribing each spell scroll no matter how many times she previously has scribed the same spell.

The creator must have prepared the spell to be scribed (or must know the spell, in the case of a sorcerer) and must provide any material component or focus the spell requires. A material component is consumed when she begins writing, but a focus is not. (A focus used in scribing a scroll can be reused.) The act of writing triggers the prepared spell, making it unavailable for casting until the character has rested and regained spells. (That is, that spell slot is expended from her currently prepared spells, just as if it had been cast.)

Scribing a scroll requires one day per each 1,000 gp of the base price.

\parhead{Item Creation Feat Required} Scribe Scroll.

\begin{twait}

Scroll Base Prices (By Scriber's Class) \\
Spell Level & Clr, Drd, Wiz & Sor & Brd & Pal, Rgr* \\
0 & 12 gp 5 sp & 12 gp 5 sp & 12 gp 5 sp &  -  \\
1st & 25 gp & 25 gp & 50 gp & 50 gp \\
2nd & 150 gp & 200 gp & 200 gp & 200 gp \\
3rd & 375 gp & 450 gp & 525 gp & 375 gp \\
4th & 700 gp & 800 gp & 1,000 gp & 700 gp \\
5th & 1,125 gp & 1,250 gp & 1,625 gp &  -  \\
6th & 1,650 gp & 1,800 gp & 2,400 gp &  -  \\
7th & 2,275 gp & 2,450 gp &  -  &  -  \\
8th & 3,000 gp & 3,200 gp &  -  &  -  \\
9th & 3,825 gp & 4,050 gp &  -  &  -  \\
* Caster level is half class level. \\
Prices assume that the scroll was made at the minimum caster level. &

Base Magic Supplies and XP Cost to Scribe a Scroll (By Scriber's Class) \\
Spell Level & Clr, Drd, Wiz & Sor & Brd & Pal, Rgr* \\
0 & 6 gp 2 sp 5 cp

\plus1 XP & 6 gp 2 sp 5 cp

\plus1 XP & 6 gp 2 sp 5 cp

\plus1 XP &  -  \\
1st & 12 gp 5 sp

\plus1 XP & 12 gp 5 sp

\plus1 XP & 25 gp

\plus1 XP & 25 gp

\plus2 XP \\
2nd & 75 gp

\plus6 XP & 100 gp

\plus8 XP & 100 gp

\plus8 XP & 100 gp

\plus8 XP \\
3rd & 187 gp 5 sp

\plus15 XP & 225 gp

\plus18 XP & 262 gp 5 sp

\plus21 XP & 187 gp 5 sp

\plus15 XP \\
4th & 350 gp

\plus28 XP & 400 gp

\plus32 XP & 500 gp

\plus40 XP & 350 gp

\plus28 XP \\
5th & 562 gp 5 sp

\plus45 XP & 625 gp

\plus50 XP & 812 gp 5 sp

\plus65 XP &  -  \\
6th & 826 gp

\plus66 XP & 900 gp

\plus72 XP & 1,200 gp

\plus96 XP &  -  \\
7th & 1,135 gp 5 sp

\plus91 XP & 1,225 gp

\plus98 XP &  -  &  -  \\
8th & 1,500 gp

\plus120 XP & 1,600 gp

\plus128 XP &  -  &  -  \\
9th & 1,912 gp 5 sp

\plus153 XP & 2, 025 gp

\plus162 XP &  -  &  -  \\
* Caster level is half class level. \\
Costs assume that the creator makes the scroll at the minimum caster level. &

\end{twait}

\subsection{Creating Staffs}

To create a magic staff, a character needs a supply of materials, the most obvious being a staff or the pieces of the staff to be assembled.

The cost for the materials is subsumed in the cost for creating the staff - 250 gp \mtimes the level of the highest-level spell \mtimes the level of the caster, plus 75\% of the value of the next most costly ability (187.5 gp \mtimes the level of the spell \mtimes the level of the caster), plus one-half of the value of any other abilities (125 gp \mtimes the level of the spell \mtimes the level of the caster). Staffs are always fully charged (50 charges) when created.

If desired, a spell can be placed into the staff at only half the normal cost, but then activating that particular spell costs 2 charges from the staff. The caster level of all spells in a staff must be the same, and no staff can have a caster level of less than 8th, even if all the spells in the staff are low-level spells.

The creator must have prepared the spells to be stored (or must know the spell, in the case of a sorcerer) and must provide any focus the spells require as well as material component costs sufficient to activate the spell a maximum number of times (50 divided by the number of charges one use of the spell expends). This is in addition to the XP cost for making the staff itself. Material components are consumed when he begins working, but focuses are not. (A focus used in creating a staff can be reused.) The act of working on the staff triggers the prepared spells, making them unavailable for casting during each day of the staff's creation. (That is, those spell slots are expended from his currently prepared spells, just as if they had been cast.)

Creating a few staffs may entail other prerequisites beyond spellcasting. See the individual descriptions for details.

Crafting a staff requires one day for each 1,000 gp of the base price.

\parhead{Item Creation Feat Required} Craft Staff.

\subsection{Creating Wands}

To create a magic wand, a character needs a small supply of materials, the most obvious being a baton or the pieces of the wand to be assembled. The cost for the materials is subsumed in the cost for creating the wand - 375 gp \mtimes the level of the spell \mtimes the level of the caster. Wands are always fully charged (50 charges) when created.

The creator must have prepared the spell to be stored (or must know the spell, in the case of a sorcerer) and must provide any focuses the spell requires. Fifty of each needed material component are required, one for each charge. Material components are consumed when she begins working, but focuses are not. (A focus used in creating a wand can be reused.) The act of working on the wand triggers the prepared spell, making it unavailable for casting during each day devoted to the wand's creation. (That is, that spell slot is expended from her currently prepared spells, just as if it had been cast.)

Crafting a wand requires one day per each 1,000 gp of the base price.

\parhead{Item Creation Feat Required} Craft Wand.

\begin{twait}

Wand Base Prices (By Crafter's Class) \\
Spell Level & Clr, Drd, Wiz & Sor & Brd & Pal, Rgr* \\
0 & 375 gp & 375 gp & 375 gp &  -  \\
1st & 750 gp & 750 gp & 1,500 gp & 1,500 gp \\
2nd & 4,500 gp & 6,000 gp & 6,000 gp & 6,000 gp \\
3rd & 11,250 gp & 13,500 gp & 15,750 gp & 11,250 gp \\
4th & 21,000 gp & 24,000 gp & 30,000 gp & 21,000 gp \\
* Caster level is half class level. \\
Prices assume that the wand was made at the minimum caster level. &

Base Magic Supplies and XP Cost to Craft a Wand (By Crafter's Class) \\
Spell Level & Clr, Drd, Wiz & Sor & Brd & Pal, Rgr* \\
0 & 187 gp 5 sp

\plus15 XP & 187 gp 5 sp

\plus15 XP & 187 gp 5 sp

\plus15 XP &  -  \\
1st & 375 gp

\plus30 XP & 375 gp

\plus30 XP & 750 gp

\plus60 XP & 750 gp

\plus60 XP \\
2nd & 2,250 gp

\plus180 XP & 3,000 gp

\plus240 XP & 3,000 gp

\plus240 XP & 3,000 gp

\plus240 XP \\
3rd & 5,625 gp

\plus450 XP & 6,750 gp

\plus540 XP & 7,875 gp

\plus630 XP & 5,625 gp

\plus450 XP \\
4th & 10,500 gp

\plus840 XP & 12,000 gp

\plus960 XP & 15,000 gp

\plus1200 XP & 10,500 gp

\plus840 XP \\
* Caster level is half class level. \\
Costs assume that the creator makes the wand at the minimum caster level. &

\end{twait}

\subsection{Creating Wondrous Items}

To create a wondrous item, a character usually needs some sort of equipment or tools to work on the item. She also needs a supply of materials, the most obvious being the item itself or the pieces of the item to be assembled. The cost for the materials is subsumed in the cost for creating the item. Wondrous item costs are difficult to formularize. Refer to Table: Estimating Magic Item Gold Piece Values and use the item prices in the item descriptions as a guideline. Creating an item costs half the market value listed.

If spells are involved in the prerequisites for making the item, the creator must have prepared the spells to be cast (or must know the spells, in the case of a sorcerer) but need not provide any material components or focuses the spells require, nor are any XP costs inherent in a prerequisite spell incurred in the creation of the item. The act of working on the item triggers the prepared spells, making them unavailable for casting during each day of the item's creation. (That is, those spell slots are expended from his currently prepared spells, just as if they had been cast.)

Creating some items may entail other prerequisites beyond or other than spellcasting. See the individual descriptions for details.

Crafting a wondrous item requires one day for each 1,000 gp of the base price.

\parhead{Item Creation Feat Required} Craft Wondrous Item.

\subsection{Intelligent Item Creation}

To create an intelligent item, a character must have a caster level of 15th or higher. Time and creation cost are based on the normal item creation rules, with the market price values on Table: Item Intelligence, Wisdom, Charisma, and Capabilities treated as additions to time and gp cost. The item's alignment is the same as its creator's. Determine other features randomly, following the guidelines in the relevant section.

\subsection{Adding New Abilities}

A creator can add new magical abilities to a magic item with no restrictions. The cost to do this is the same as if the item was not magical. Thus, a \plus1 longsword can be made into a \plus2 vorpal longsword, with the cost to create it being equal to that of a \plus2 vorpal sword minus the cost of a \plus1 sword.

If the item is one that occupies a specific place on a character's body the cost of adding any additional ability to that item increases by 50\%. For example, if a character adds the power to confer invisibility to her ring of protection \plus2, the cost of adding this ability is the same as for creating a ring of invisibility multiplied by 1.5.

\subsection{Body Slot Affinities}

Each location on the body, or body slot, has one or more affinities: a word or phrase that describes the general function or nature of magic items designed for that body slot. Body slot affinities are deliberately broad, abstract categorizations, because a hard-and-fast rule can't cover the great variety among wondrous items.

You can use the affinities in the list below to guide your decisions on which magic items should \my{be housed in what forms}. And when you design your own magic items, the affinities give you some guidance for what form a particular item should take.

Some body slots have different affinities for different specific items.

\begin{dtable}
\begin{dtabularx}{\columnwidth}{>{\lcol}X >{\lcol}X}
\thead{Body Slot} & \thead{Affinity} \\
\hline
Amulet & Physical protection \\
Belt & Physical improvement \\
Bracelets  & Allies \\
Bracers  & Combat \\
Boots  & Movement, speed \\
Cloak, cape, mantle & Transformation, protection \\
Gauntlets  & Destructive power \\
Gloves  & Finesse, speed \\
Eye lenses, goggles & Vision \\
Hat & Interaction \\
Headband, helmet & Mental improvement \\
Medallion, necklace & Discernment \\
Periapt, scarab & Magical protection \\
Phylactery & Morale, alignment \\
Robe  & Multiple effects \\
Shirt  & Physical improvement \\
Vest, vestment  & Class ability improvement \\
\end{dtabularx}
\end{dtable}

Wondrous items that don't match the affinity for a particular body slot should cost 50\% more than wondrous items that match the affinity.
