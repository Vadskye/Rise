\chapter{Skills}
\section{Skills Summary}
A character's skills describe the myriad of talents that people have.

\parhead{Skill Points} At 1st level, your character gains a certain number of skill points. Skill points represent your training in a particular area. You get a base allotment of 2, 4, 8, or 12 skill points, depending on your character's class. These skill points can be spent on any skills. In addition, you gain additional skill points based on your attributes that can only be spent on skills associated with the relevant attribute. For each attribute, you gain a number of skill points equal to half your attribute score.

\par If your character's attributes increase, she may immediately gain a skill point which can be spent on skills with that key attribute. However, enhancement bonuses, circumstance bonuses, and penalties of any kind do not affect a character's skill points. If she gains a level in a new class that has more skill points than any class she previously had, the character immediately gains skill points equal to the difference between the skill points provided by the two classes. These skill points can be spent on any skill.

\parhead{Spending Skill Points} If you place one skill point in a skill, you become trained in that skill. If you place two points in a skill, you become an expert in that skill. Your level of training in a skill determines how many skill ranks you have in that skill. Some skills can only be used if you are trained in them.

\parhead{Skill Ranks}
Skill ranks represent how capable your character is with a particular skill. Your character automatically gains skill ranks as she increases in level, as shown by \trefnp{Skill Ranks and Skill Training}.

\begin{dtable}
\lcaption{Skill Ranks and Skill Training}
\begin{tabularx}{\columnwidth}{>{\lcol}p{4.5em} >{\lcol}p{2.5em} >{\lcol}p{6.5em} >{\lcol}X}
\thead{Skill Training Level} & \thead{Skill Points Spent} & \thead{Cross-Class Skill Ranks} & \thead{Class Skill Ranks} \\
Untrained & 0 & \x & \x \\
Trained & 1 & 1/2 character level & 1/2 character level \add 2 \\
Expert & 2 & 1/2 character level \add 2 & Associated class levels \add 1/2 unassociated class levels \add 3 \\
\end{tabularx}
\end{dtable}

\parhead{Using Skills} To make a skill check, roll 1d20 \add skill rank \add key attribute \add bonuses and penalties.

\subparhead{Key Attribute} The attribute used in a skill check is noted in its description.

\subparhead{Bonuses and Penalties} Miscellaneous modifiers include racial bonuses, armor check penalties, bonuses provided by feats, and more.

\section{Using Skills}
When your character uses a skill, you make a skill check to see how well he or she does. The higher the result of the skill check, the better. Based on the circumstances, your result must match or beat a particular number (a DC or the result of an opposed skill check) for the check to be successful. The harder the task, the higher the number you need to roll.

Circumstances can affect your check. A character who is free to work without distractions can make a careful attempt and avoid simple mistakes. A character who has lots of time can try over and over again, thereby assuring the best outcome. If others help, the character may succeed where otherwise he or she would fail.

\begin{dtable!*}
\lcaption{Skills}
\begin{tabularx}{\textwidth}{>{\lcol}p{12em} *{11}{>{\ccol}X} >{\ccol}p{4em} >{\ccol}p{4em}}
    \thead{Skill}   & \thead{Bbn} & \thead{Clr} & \thead{Drd} & \thead{Ftr} & \thead{Mnk} & \thead{Pal} & \thead{Rgr} & \thead{Rog} & \thead{Sor} & \thead{Spl} & \thead{Wiz} & \thead{Untrained} & \thead{Key Ability} \\
Climb           & C  & cc & cc & C  & C  & cc & C  & C  & cc & cc & cc & Yes & Str\footnotetemp{1} \\
Jump            & C  & cc & cc & C  & C  & cc & C  & C  & cc & cc & cc & Yes & Str\footnotetemp{1} \\
Swim            & C  & cc & C  & C  & C  & cc & C  & C  & cc & C  & cc & Yes & Str\footnotetemp{2} \\
Acrobatics      & cc & cc & C  & cc & C  & cc & cc & C  & cc & cc & cc & Yes & Dex\footnotetemp{1} \\
Escape Artist   & cc & cc & cc & cc & C  & cc & cc & C  & cc & cc & cc & Yes & Dex\footnotetemp{1} \\
Ride            & cc & cc & cc & C  & cc & C  & cc & cc & cc & C  & cc & Yes & Dex \\
Sleight of Hand & cc & cc & cc & cc & cc & cc & cc & C  & cc & cc & cc & No & Dex\footnotetemp{1} \\
Stealth         & cc & cc & cc & cc & C  & cc & C  & C  & cc & cc & cc & Yes & Dex\footnotetemp{1} \\
Craft           & cc & cc & cc & cc & cc & cc & cc & cc & cc & cc & cc & Yes & Int \\
Devices         & cc & cc & cc & cc & cc & cc & cc & C  & cc & cc & cc & No & Int \\
Disguise        & cc & cc & cc & cc & cc & cc & cc & C  & cc & cc & cc & Yes & Int \\
Knowledge (arcana)          & cc & C  & cc & cc & C  & cc & cc & cc & C  & C  & C  & No & Int \\
Knowledge (dungeoneering)   & cc & cc & cc & cc & cc & cc & C  & C  & cc & cc & C  & No & Int \\
Knowledge (engineering)     & cc & cc & cc & cc & cc & cc & cc & cc & cc & cc & C  & No & Int \\
Knowledge (geography)       & cc & cc & cc & cc & cc & cc & C  & cc & cc & cc & C  & No & Int \\
Knowledge (local)           & cc & cc & cc & cc & cc & cc & cc & C  & cc & cc & C  & No & Int \\
Knowledge (nature)          & cc & cc & C  & cc & cc & cc & C  & cc & cc & cc & C  & No & Int \\
Knowledge (planes)          & cc & C  & cc & cc & cc & cc & cc & cc & C  & cc & C  & No & Int \\
Knowledge (religion)        & cc & C  & cc & cc & C  & C  & cc & cc & cc & cc & C  & No & Int \\
Linguistics     & cc & C  & cc & cc & cc & cc & cc & C  & cc & cc & C  & No & Int \\
\thead{Skill}   & \thead{Bbn} & \thead{Clr} & \thead{Drd} & \thead{Ftr} & \thead{Mnk} & \thead{Pal} & \thead{Rgr} & \thead{Rog} & \thead{Sor} & \thead{Spl} & \thead{Wiz} & \thead{Untrained} & \thead{Key Ability} \\
Heal            & cc & C  & C  & cc & cc & C  & C  & cc & cc & cc & cc & Yes & Wis \\
Perception      & C  & cc & C  & cc & C  & cc & C  & C  & cc & cc & cc & Yes & Wis \\
Profession      & cc & cc & cc & cc & cc & cc & cc & cc & cc & cc & cc & No & Wis\fn{3} \\
Sense Motive    & cc & C  & cc & cc & C  & C  & cc & C  & cc & cc & cc & Yes & Wis \\
Spellcraft      & cc & C  & C  & cc & cc & cc & cc & cc & C  &  C & C  & No & Wis \\
Survival        & C  & cc & C  & cc & cc & cc & C  & cc & cc & cc & cc & Yes & Wis \\
Bluff           & cc & cc & cc & cc & cc & cc & cc & C  & cc & cc & cc & Yes & Cha \\
Creature Handling   & C  & cc & C  & cc & cc & C  & C  & cc & cc & cc & cc & No & Cha \\
Intimidate      & C  & cc & cc & C  & cc & cc & cc & C  & C  &  C & cc & Yes & Cha \\
Perform         & cc & cc & cc & cc & cc & cc & cc & C  & cc & cc & cc & Yes & Cha \\
Persuasion      & cc & C  & C  & cc & C  & C  & cc & C  & cc & cc & cc & Yes & Cha \\
\end{tabularx}
1. Armor check penalty applies \\
2. Double armor check penalty applies \\
3. Varies depending on profession
\end{dtable!*}

\subsection{Skill Checks}
A skill check takes into account a character's training (skill rank), natural talent (attribute), and luck (the die roll). It may also take into account his or her race's knack for doing certain things (racial bonus) or what armor he or she is wearing (armor check penalty), or a certain feat the character possesses, among other things.

To make a skill check, roll 1d20 and add your character's skill modifier for that skill. The skill modifier incorporates the character's ranks in that skill and the attribute for that skill's key attribute, plus any other miscellaneous modifiers that may apply, including racial bonuses and armor check penalties. The higher the result, the better. Unlike with attack rolls and saving throws, a natural roll of 20 on the d20 is not an overwhelming success, and a natural roll of 1 is not an overwhelming failure.

\subsubsection{Difficulty Class}
Some checks are made against a Difficulty Class (DC). The DC is a number (set using the skill rules as a guideline) that you must score as a result on your skill check in order to succeed.

\begin{dtable}
\lcaption{Difficulty Class Examples}
\begin{tabularx}{\columnwidth}{p{8em} X}
Difficulty (DC) & Example (Skill Used) \\
Very easy (0) & Notice something large in plain sight (Perception) \\
Easy (5) & Hear a conversation from 50 feet away (Perception) \\
Average (10) & Palm a coin-sided object (Sleight of Hand) \\
Tough (15) & Rig a wagon wheel to fall off (Disable Device) \\
Challenging (20) & Swim in stormy water (Swim) \\
Formidable (25) & Climb a natural rock wall with no equipment (Climb) \\
Heroic (30) & Leap across a 30-foot chasm with a running start (Jump) \\
Nearly impossible (40) & Track a squad of orcs across hard ground after 24 hours of rainfall (Survival) \\
\end{tabularx}
\end{dtable}

\subsubsection{Opposed Checks}
An opposed check is a check whose success or failure is determined by comparing the check result to another character's check result. In an opposed check, the higher result succeeds, while the lower result fails. In case of a tie, the higher skill modifier wins. If these scores are the same, roll again to break the tie.

\begin{dtable}
\lcaption{Example Opposed Checks}
\begin{tabularx}{\columnwidth}{*{3}{>{\lcol}X}}
\thead{Task} & \thead{Skill (Key Ability)} & \thead{Opposing Skill (Key Ability)} \\
Lie & Bluff (Cha) & Sense Motive (Wis) \\
Create a false map & Craft (Int) & Craft (Int) or Perception (Wis) \\
Make a bully back down & Intimidate (Cha) & Special\footnotetemp{1} \\
Make someone look like someone look like someone else & Disguise (Int) & Perception (Wis) \\
Sneak up on someone & Stealth (Dex) & Perception (Wis) \\
Steal a coin pouch & Sleight of Hand (Dex) & Perception (Wis) \\
Tie a prisoner securely & Devices (Int)\fn{2} & Escape Artist (Dex) \\
\end{tabularx}
1 An Intimidate check is opposed by the target's Will save, not a skill check. See the Intimidate skill description for more information. \\
2 You can also tie a prisoner with a grapple attack. See \pcref{Grapple}. \\
\end{dtable}

\subsubsection{Trying Again}
In general, you can try a skill check again if you fail, and you can keep trying indefinitely. Some skills, however, have consequences of failure that must be taken into account. A few skills are virtually useless once a check has failed on an attempt to accomplish a particular task. For most skills, when a character has succeeded once at a given task, additional successes are meaningless.

\subsubsection{Untrained Skill Checks}
Generally, if your character attempts to use a skill he or she does not possess, you make a skill check as normal. The skill modifier doesn't have a skill rank added in because the character has no ranks in the skill. Any other applicable modifiers, such as the modifier for the skill's key attribute, are applied to the check.

Many skills can be used only by someone who is trained in them.

\subsubsection{Favorable and Unfavorable Conditions}
Some situations may make a skill easier or harder to use, resulting in a bonus or penalty to the skill modifier for a skill check or a change to the DC of the skill check.

The chance of success can be altered in four ways to take into account exceptional circumstances.
\begin{enumerate*}
\item Give the skill user a \plus2 circumstance bonus to represent conditions that improve performance, such as having the perfect tool for the job, getting help from another character (see Combining Skill Attempts), or possessing unusually accurate information.
\item Give the skill user a \minus2 circumstance penalty to represent conditions that hamper performance, such as being forced to use improvised tools or having misleading information.
\item Reduce the DC by 2 to represent circumstances that make the task easier, such as having a friendly audience or doing work that can be subpar.
\item  Increase the DC by 2 to represent circumstances that make the task harder, such as having an uncooperative audience or doing work that must be flawless.
\end{enumerate*}

Conditions that affect your character's ability to perform the skill change the skill modifier. Conditions that modify how well the character has to perform the skill to succeed change the DC. A bonus to the skill modifier and a reduction in the check's DC have the same result: They create a better chance of success. But they represent different circumstances, and sometimes that difference is important.

\subsubsection{Time and Skill Checks}
Using a skill might take a round, take no time, or take several rounds or even longer. Most skill uses are standard actions, move actions, or full-round actions. Types of actions define how long activities take to perform within the framework of a combat round (6 seconds) and how movement is treated with respect to the activity. Some skill checks are instant and represent reactions to an event, or are included as part of an action. These skill checks are not actions. Other skill checks represent part of movement.

 \subsubsection{Checks without Rolls}
A skill check represents an attempt to accomplish some goal, usually while under some sort of time pressure or distraction. Sometimes, though, a character can use a skill under more favorable conditions and eliminate the luck factor.
 \parhead{Taking 10} When your character is not being threatened or distracted,
and the skill check has no penalties for failure, you can't roll less than a 10. If you roll a d20 for your skill check, and you roll less than a 10, calculate your result as if you had rolled a 10. For many routine tasks, taking 10 makes them automatically successful, and you may not even roll at all.

Distractions or threats (such as combat) make it impossible for a character to take 10. Taking 10 is especially useful in situations where a particularly high roll wouldn't help.
\parhead{Taking 20} When you have plenty of time (generally 2 minutes for a skill that can normally be checked in 1 round, one full-round action, or one standard action), you are faced with no threats or distractions, and the skill being attempted carries no penalties for failure, you can take 20. In other words, eventually you will get a 20 on 1d20 if you roll enough times. Instead of rolling 1d20 for the skill check, just calculate your result as if you had rolled a 20.

Taking 20 means you are trying until you get it right, and it assumes that you fail many times before succeeding. Taking 20 takes twenty times as long as making a single check would take. If the check takes a variable amount of time, assume it took the average amount of time required to make a check.

Since taking 20 assumes that the character will fail many times before succeeding, if you did attempt to take 20 on a skill that carries penalties for failure, your character would automatically incur those penalties before he or she could complete the task. Common ``take 20'' skills include Escape Artist, Open Lock, and Search.

\parhead{Ability Checks and Caster Level Checks} The normal take 10 and take 20 rules apply for ability checks. Neither rule applies to caster level checks.
\parhead{Special Abilities} Some special abilities grant the ability to take 5, take 10, or some other number on specific checks or even attacks. This follows the same rules as taking 10, except that the character can typically use such abilities even while threatened or distracted.

\subsection{Group Skill Checks}
When multiple characters are trying to use the same skill simultaneously, they may be able to work together. Most skills can be done as a group skill check, where the more skilled characters assist the less skilled characters and work together to find the best result. There are two kinds of group skill checks.

\parhead{Collaborative Checks} When making a collaborative skill check, each member of the group gets their own result. Collaborative checks might include a group of scouts sneaking up on an enemy camp, a group of adventurers climbing a cliff, or a group of spies lying about their true allegiances.

When making a collaborative check, the group must choose a leader before making the check. Each member of the group can make the check using the higher of their skill ranks and half the leader's skill ranks. Other modifiers apply normally.

The group leader must remain in a position to help the other members of the group for the group members to gain this benefit. For example, if a group is making a collaborative check to leap across a chasm, the leader must jump last. If the group is making a collaborative Stealth check, the leader must remain adjacent to the other members of the group to correct their mistakes quietly. The exact circumstances depend on the situation. 

\parhead{Collective Checks} The group works together to get a single result. Collective checks might include a group of diplomats persuading a noble to go to war, a group of medics tending to a dying warrior, or a group of scouts keeping watch for enemy attacks.

When making a collective check, the group simply uses the highest result from any character making the check.

\parhead{Making Group Skill Checks} A group skill check is not always possible. If the whole group is leaping over a chasm at once to escape a dragon, there is not enough time to designate a leader and have them aid the others to make a collaborative check. In general, a group skill check is only possible if the group has at least five rounds to work together, though this time may vary widely depending on the situation and the skill being used. 

\subsection{Ability Checks}
Sometimes a character tries to do something to which no specific skill really applies. In these cases, you make an ability check. An ability check is a roll of 1d20 plus the appropriate attribute. Essentially, you're making an untrained skill check.

In some cases, an action is a straight test of one's ability with no luck involved. Just as you wouldn't make a height check to see who is taller, you don't make a Strength check to see who is stronger.

\section{Skill Descriptions}
This section describes each skill, including common uses and typical modifiers. Characters can sometimes use skills for purposes other than those noted here.

Here is the format for skill descriptions.

\subsection*{Skill Name}
The skill name line includes (in addition to the name of the skill) the following information.
\parhead{Key Ability} The abbreviation of the ability whose modifier applies to the skill check.
\parhead{Trained Only} If this notation is included in the skill name line, you must have at least 1 rank in the skill to use it. If it is omitted, the skill can be used untrained (with a rank of 0). If any special notes apply to trained or untrained use, they are covered in the Untrained section (see below).
\parhead{Armor Check Penalty} If this notation is included in the skill name line, an armor check penalty applies (when appropriate) to checks using this skill. If this entry is absent, an armor check penalty does not apply.

\par The skill name line is followed by a general description of what using the skill represents. After the description are a few other types of information:

\parhead{Check} What a character (``you'' in the skill description) can do with a successful skill check, and the check's DC.

\parhead{Action} The type of action using the skill requires, or the amount of time required for a check.

\parhead{Try Again} Any conditions that apply to successive attempts to use the skill successfully. If the skill doesn't allow you to attempt the same task more than once, or if failure carries an inherent penalty (such as with the Climb skill), you can't take 20. If this paragraph is omitted, the skill can be retried without any inherent penalty, other than the additional time required.

\parhead{Special} Any extra facts that apply to the skill, such as special effects deriving from its use or bonuses that certain characters receive because of class, feat choices, or race.

\parhead{Restriction} The full utility of certain skills is restricted to characters of certain classes or characters who possess certain feats. This entry indicates whether any such restrictions exist for the skill.

\parhead{Untrained} This entry indicates what a character without at least 1 skill point in the skill can do with it. If this entry doesn't appear, it means that the skill functions normally for untrained characters (if it can be used untrained) or that an untrained character can't attempt checks with this skill (for skills that are designated as ``Trained Only'').

\subsection{Acrobatics (Dex; Armor Check Penalty)}
Acrobatics represents your agility and coordination. All Acrobatics checks are made as part of movement, so they require no special action to perform.

\subsubsection{Agile Movement}
You can make a DC 20 Acrobatics check while running or charging to make a single turn of up to 90 degrees in the middle of the moveent. Failure indicates that you can't change direction, though you can continue your movement or stop. Failure by 6 or more indicates that you stop where you tried to change direction and fall prone.

\subsubsection{Balance}
You can make a Acrobatics check to move safely on a precarious surface. Success means you move along the surface at half speed. Failure by 5 or less means your action is wasted, and you do not move. Failure by 6 or more means you fall.

You are flat-footed while balancing, since you can't move to avoid a blow. If you take damage while balancing, you must make another Balance check against the same DC to avoid falling. If you take a \minus5 penalty, you can defend yourself normally while balancing. Accepting a \minus5 penalty can also allow you to move at full speed while balancing.

The DC varies with the surface, as follows.

\begin{itemize*}
  \item Uneven floor (flagstones, sloped floor): DC 10. This only applise when running or charging. Failure means you lose a move action (and possibly fall), but you can still take a standard action.
  \item One foot wide (or wider): DC 5. 
  \item Six inches wide: DC 10.
  \item Two inches wide: DC 15.
  \item One inch wide: DC 20. Halving the width of the surface further increases the DC by 5 each time.
\end{itemize*}

\subsubsection{Tumble}
If you are trained in Acrobatics, you can tumble past opponents in combat to reduce your odds of being hit. You can tumble as part of normal movement. If you do, you move at half speed and make an Acrobatics check. You may treat your check result as your Armor Class against attacks of opportunity provoked by the movement.

If you accept a \minus10 penalty, you can move at full speed while tumbling. If you accept a \minus20 penalty, you can tumble while running or charging.

\subsubsection{Mitigate Fall}
As you hit the ground after a fall, you can make an Acrobatics check to reduce falling damage. A DC 15 check allows you to treat a fall as if it were 10 feet shorter. For every 10 by which you beat that DC, you can reduce the falling damage by 10 additional feet.

\subsubsection{Acrobatics Modifiers}
Obstructed or otherwise treacherous surfaces, such as natural cavern floors or undergrowth, are tough to balance or tumble through. The DC for any Acrobatics check in such a square (except checks to mitigate falling damage) is modified as indicated below.
\begin{dtable}
\lcaption{Acrobatics Modifiers}
\begin{tabularx}{\columnwidth}{>{\lcol}X c}
\thead{Surface Is} & \thead{DC Modifier} \\
Lightly obstructed (scree, light rubble, shallow bog, light undergrowth)  & \plus2 \\
Lightly slippery (wet floor)  & \plus2 \\
Sloped or angled  & \plus2 \\
Slightly mobile (rope bridge) & \plus2 \\
Severely obstructed (dense rubble, dense undergrowth)  & \plus5 \\
Severely slippery (ice sheet, oiled floor)  & \plus5 \\
Very mobile (slack rope) & \plus5 \\
\end{tabularx}
\end{dtable}

\subsection{Athletics (Str; Armor Check Penalty)}
Athletics includes running, jumping, and general athleticism. All Athletics checks are made as part of movement, so they take no special action to perform.

\subsubsection{Long Jump}
When you make a long jump, choose a DC. You jump forward by a number of feet equal to your check result, to a maximum of the DC you chose. At the midpoint of the jump, you achieve a height equal to a quarter of that distance. If you fail by 6 or more, you fall prone after making the jump. If a failed jump would cause you to fall into a gap, you can make a DC 20 Climb check to catch the edge of the gap, provided you can reach it. 

A long jump assumes you have a running start, which requires that you move at least 20 feet in a straight line before attempting the jump. If you do not get a running start, your check result is halved.

A long jump is modified by your speed. You gain a \plus2 bonus per 5 feet faster than 30 feet, or a \minus3 penalty per 5 feet slower than 30 feet.

Distance moved by jumping is counted against your normal maximum movement in a round.

\subsubsection{High Jump}
When you make a high jump, choose a DC. You move forward by an amount to a quarter of to your check result, to a maximum of a quarter of the DC. At the midpoint of the jump, you gain a height equal to that distance. If you fail by 6 or more, you land prone after making the jump.

A high jump assumes you have a running start, which requires that you move at least 20 feet in a straight line before attempting the jump. If you do not get a running start, your check result is halved, and you do not move forward.

Distance moved by jumping is counted against your normal maximum movement in a round.

\subsk{High Jump} A high jump is a vertical leap made to reach a ledge high above or to grasp something overhead. The DC is equal to 4 times the distance to be cleared.

If you jumped up to grab something, success means you reached the desired height. If you wish to pull yourself up, you can do so with a move action and a DC 15 Climb check. If you fail the Jump check, you do not reach the height, and you land on your feet in the same spot from which you jumped. As with a long jump, the DC is doubled if you do not get a running start of at least 20 feet.

Obviously, the difficulty of reaching a given height varies according to the size of the character or creature. The maximum vertical reach (height the creature can reach without jumping) for an average creature of a given size is shown on the table below. (As a Medium creature, a typical human can reach 8 feet without jumping.) The maximum vertical reach for an individual creature is usually given by the 1 and 1/2 times the creature's height.

Quadrupedal creatures don't have the same vertical reach as a bipedal creature; treat them as being one size category smaller.

\begin{dtable}
\begin{tabularx}{\columnwidth}{>{\lcol}X >{\lcol}X}
    \thead{Creature Size}  & \thead{Vertical Reach} \\
Colossal  & 128 ft. \\
Gargantuan  & 64 ft. \\
Huge  & 32 ft. \\
Large  & 16 ft. \\
Medium  & 8 ft. \\
Small  & 4 ft. \\
Tiny  & 2 ft. \\
Diminutive  & 1 ft. \\
Fine  & 1/2 ft.
\end{tabularx}
\end{dtable}

\subsubsection{Hop Up}
You can jump up onto an object as tall as your waist, such as a table or small boulder, with a DC 10 Athletics check. Doing so counts as 5 feet of movement, so if your speed is 30 feet, you could move 25 feet, then hop up onto a counter. You do not need to get a running start to hop up.

\subsubsection{Jump Down}
If you make a DC 15 Athletics check to intentionally jump down from a height, you take falling damage as if you had dropped 10 fewer feet than you actually did.

\subsubsection{Sprint}

%If your Sprint check exceeds your Fortitude defense, you are fatigued?
A Sprint check allows you to move faster temporarily. A DC 15 Sprint check allows you to treat your speed as if it were 5 feet faster for 1 round. For every 5 points by which you beat that DC, you can increase your speed by an additional 5 feet. After sprinting, you are fatigued for 1 round (until the end of your next turn), making you vulnerable. While vulnerable, you take a \minus2 penalty to attack rolls, saving throws, checks, DCs, and AC.

If you take a \minus10 penalty to your Sprint check, you are not fatigued after sprinting. You can sprint in any movement mode that you can use.

\subsection{Bluff (Cha)}
You can use Bluff to convince people that you are telling the truth. It is usually used when you are lying. Using a Bluff check is part of conversation, so it requires no special action to perform. You cannot normally take 10 on Bluff checks.

\subsubsection{False Impression}
You can make a DC 15 Bluff check give others an incorrect impression of your attitude and thoughts. If you succeed, anyone who makes a DC 10 Sense Motive check receives whatever impression you wish to portray. If the creature's Sense Motive check exceeds your Bluff check, they recognize both the impression you intended to portray and your true attitude, and they can tell the difference.

\subsubsection{Lie}
When you say something which you know is untrue, you can make a Bluff check to avoid revealing your deception. Anyone witnessing you lie can make a Sense Motive check. If a creature's Sense Motive check exceeds your Bluff check, they realize that you are lying.

A creature that fails its Sense Motive check may choose not to believe you for other reasons. This check only prevents a creature from recognizing the lie based on your body language and behavior.

\subsubsection{Misdirection}
You can make a Bluff check as a move action to distract attention from you so you can hide. Anyone observing you must beat your Bluff check with a Perception check to keep an eye on you.

\subsubsection{Secret Message}
You can make a Bluff check to attempt to convey a hidden message to another character without others understanding it. The DC is 15 for simple messages and 20 for complex messages. If the message contains completely new information, the DC increases by 5. You can freely increase the DC to make the message more difficult to intercept, but doing so also makes it more difficult for the intended recipient to interpret.

Anyone witnessing the exchange must make a Sense Motive check against the same DC to identify the hidden message. Creatures who know how the message will be conveyed -- normally, the intended recipient -- receive a \plus10 bonus on this check. Exceptionally conplex hidden message systems may grant a bonus greater than \plus10.

\subsection{Climb (Str; Armor Check Penalty)}

\subsubsection{Climb}
You can make a Climb check as a move action to move up, down, or across a slope, a wall, or some other steep incline (or even a ceiling with handholds). Success means you move a number of feet equal to the size of your space, as described on \trefnp{Climb Speeds}. Failure by 5 or less means your action is wasted and you do not move. Failure by 6 or more means you fall.

You need both hands free to climb, but you may cling to a wall with one hand while you cast a spell or take some other action that requires only one hand. You are flat-footed while climbing, since you can't move to avoid a blow. If you take damage while climbing, you must make another Climb check against the same DC to avoid falling. If you take a \minus5 penalty, you can defend yourself normally while climbing. Accepting a \minus5 penalty can also allow you to move at double speed while climbing. 

\begin{dtable}
\lcaption{Climb Speeds}
\begin{tabularx}{\columnwidth}{l X l X}
  \thead{Size} & \thead{Speed} & \thead{Size} & \thead{Speed} \\
  Medium & 5 ft. && \\
  Small & 5 ft. & Large & 10 ft. \\
  Tiny & 2-1/2 ft. & Huge & 15 ft. \\
  Diminuitive & 1 ft. & Gargantuan & 20 ft. \\
  Fine & 1/2 ft. & Colossal & 40 ft. \\
\end{tabularx}
\end{dtable}

The DC of the check depends on the difficulty of the task and the conditions of the climbing surface, as shown on \trefnp{Climb DCs} and \trefnp{Climb Modifiers}.

\begin{dtable*}
\lcaption{Climb DCs}
\begin{tabularx}{\textwidth}{l X l}
    \thead{Climb DC} & \thead{Surface or Activity} & \thead{Example} \\
    10 & Surface with large hand and holds to stand on & Very rough rocks, ship's rigging \\
    15 & Surface with some hand and foot holds & Knotted rope, surface with pitons or carved holes, rough wall \\
    15 & Surface with only large hand holds & Pulling yourself up by your hands while dangling \\
    20 & Uneven surface with narrow hand and foot holds & Unknotted rope, unweathered natural rock, typical ruin wall, brick wall \\
    20 & Overhang or ceiling with only handholds & Tree limbs, butcher's ceiling with meat hooks \\
    25 & Rough surface with no holds & Weathered natural rock, well-made stone wall \\
    \x\fn{1} & Smooth surface & Glass window, \spell{wall of force} \\
    30 & Smooth surface with chimney setup & \x \\
    35 & Smooth surface corner & \x \\
\end{tabularx}
    1 A perfectly smooth, flat, surface cannot be climbed on its own.\\
\end{dtable*}

\begin{dtable}
\lcaption{Climb Modifiers}
\begin{tabularx}{\columnwidth}{l X}
\thead{Climb DC Modifier\footnotetemp{1}} & \thead{Example Surface or Activity} \\
\minus10 & Climbing a chimney (artificial or natural) or other location where you can brace against two opposite walls \\
\minus5 & Inclined surface (between 45 and 60 degrees) \\
\minus5 & Climbing a corner where you can brace against perpendicular walls \\
\plus2 & Surface is slightly slippery, such as damp stone \\
\plus5 & Surface is very slippery, such as ice or oil-covered stone 
\end{tabularx}
1 These modifiers are cumulative; use any that apply.
\end{dtable}

\subsubsection{Catch Falling Character}
While climbing, you can attempt to catch another character who is falling near you. To do so, you must make a successful grapple attack against the falling character. Most falling characters will choose to be flat-footed against this attack. If you succeed, you must make a Climb check against a DC equal to the wall's DC \add 10. Success means you catch the falling character, but his or her total weight, including equipment, cannot exceed your heavy load limit or you automatically fall. If you fail your Climb check by 4 or less, you fail to stop the character's fall but don't lose your grip on the wall. If you fail by 6 or more, you fail to stop the character's fall and begin falling as well.

\subsubsection{Stop Fall}
It is possible, but very difficult, to make a Climb check to stop yourself from falling when near a wall. To catch yourself while falling, make a Climb check against a DC equal to the wall's DC \add 20.

\subsubsection{Climb Speeds}
A creature with a climb speed moves by a distance equal to its climb speed when climbing. It has a \plus5 inherent bonus on all Climb checks. It can always choose to take 10 on Climb checks, even if rushed or threatened, and is not flat-footed while climbing. It cannot make an accelerated climb or use the run action while climbing.

\subsection{Craft (Int)}
Like Knowledge, Perform, and Profession, Craft is actually a number of separate skills. You could have several Craft skills, each with its own ranks, each purchased as a separate skill. Common Craft skills are listed below, with additional description for some skills.

\begin{itemize*}
  \item Alchemy (Alchemist's fire, tanglefoot bags, potions)
  \item Ceramics (Glass, pottery)
  \item Jewelry (gemcutting, amulets, rings)
  \item Leather
  \item Manuscripts (Books, official documents, scrolls)
  \item Metal
  \item Stone
  \item Textiles (Cloth, fabric)
  \item Traps
  \item Wood
\end{itemize*}

A Craft skill is specifically focused on physical objects. If nothing can be created by an endeavor, it probably falls under the heading of a Profession skill. Complex structures, such as buildings or siege engines, may require Knowledge (engineering) in addition to an appropriate Craft skill.

\subsubsection{Create Item}
You can use the Craft skill to create an item by expending time and material components. Creating an item often requires multiple consecutive Craft checks. Success on a check means you make progress on completing the item. If you make enough progress, you complete the item. Failure by 5 or less means you failed to make progress, but can try again without penalty. Failure by 6 or more means you botched the item, and negating all progress and ruining half of the material components. You can start again from scratch with your remaining components.

Each item takes a certain amount of working time to craft, as shown on \tref{Crafting Time}, and the expenditure of one quarter of the item's price in raw materials. In order to craft an item, you must make a Craft check against the item's Craft DC, as shown on \trefnp{Craft DCs}. If you succeed, you make progress on the item based on how long you spend crafting. For every 5 points by which you beat the Craft check, you accomplish twice as much work in the same amount of time. Once your total effective working time exceeds the time required to craft the item, you have finished the item.

All crafts require artisan's tools to give the best chance of success. If improvised tools are used, the check may be made with a penalty, or may be impossible, depending on the tools available and the item to be crafted. For example, crafting a bow with improvised woodworking tools would impose a \minus2 penalty, but cutting a diamond without specialized tools is impossible. Note that raw materials for some items, particularly alchemical items, may be hard to come by in some areas. A typical day's work consists of 8 hours of work.

\begin{comment}
To determine the time required to craft an item, consult the table below.
\begin{dtable*}
  \lcaption{Crafting Time}
  \begin{tabularx}{\columnwidth}{l X}
    \thead{Item Price} & \thead{Crafting Time} \\
    1gp or less & One hour \\
    10gp or less & Eight hours \\
    100gp or less & One week\fn{1} \\
    500gp or less & One month\fn{1} \\
    1000gp or less & Two months\fn{1} \\
  \end{tabularx}
  1 Assuming 8 hours of work each day.
\end{dtable*}

\begin{dtable}
    \lcaption{Craft DCs}
\begin{tabularx}{\columnwidth}{>{\lcol}X l >{\lcol}p{4em}}
\thead{Item} & \thead{Craft Skill} & \thead{Craft DC} \\
Acid & Alchemy\footnotetemp{1} & 15 \\
Alchemist's fire, smokestick, or tindertwig & Alchemy & 20 \\
Antitoxin, sunrod, tanglefoot bag, or thunderstone & Alchemy & 25 \\
Armor or shield & Metalsmithing or woodworking & 10 \add AC bonus \\
Longbow or shortbow & Woodworking & 12 \\
Composite longbow or composite shortbow & Woodworking & 15 \\
Composite longbow or composite shortbow with high strength rating & Woodworking & 15 \add  (2 \mtimes rating) \\
Crossbow & Weaponsmithing & 15 \\
Simple melee or thrown weapon & Metalsmithing or woodworking & 12 \\
Martial melee or thrown weapon & Metalsmithing or woodworking & 15 \\
Exotic melee or thrown weapon & Metalsmithing or woodworking & 18 \\
Mechanical trap & Trapmaking & Varies\footnotetemp{1} \\
Very simple item (wooden spoon) & Varies & 5 \\
Typical item (iron pot) & Varies & 10 \\
High-quality item (bell, average lock) & Varies & 15 \\
Complex or superior item (fine china, document with official seal)  & Varies & 20\plus \\
\end{tabularx}
1 Traps have their own rules for construction.
\end{dtable}
\end{comment}

\subsubsection{Create Forgery}
You can use the Craft skill to create false or defective versions of objects. For example, you may wish to forge official-looking documents with Craft (manuscripts), or you may need to make items in a rush to satisfy an employer. This functions like crafting the item from scratch, except that the Craft DC is 5 lower than normal, and you use one-half the item's price to determine the price of raw materials and the crafting time.

Forgeries which have a function, such as a weapon, are always defective in some way which makes them unsuitable for use. A forgery can be detected with the Identify Forgery uses of the Craft and Perception skills.

\subsubsection{Identify Forgery}
You can make a Craft check as a full-round action to evaluate whether an item is a forgery. The DC to identify a forgery is equal to the Craft check used to make the item. Success indicates that you correctly identify whether the item is a forgery or not. Failure indicates that you are unsure. Failure by more than 10 indicates that you incorrectly identify the item. The check is made secretly, so you can't be sure how good the result is.

\subsubsection{Repair Item}
You can make a Craft check to repair a broken item. This functions like crafting the item from scratch, except that you use one-tenth of the item's price to determine the price of raw materials and the crafting time.

\subsubsection{Other Tasks}
You can use the Craft skill for various tasks related to the object of your craft. For example, you could use Craft (woodworking) to sabotage a wagon so it will break at a later time, or Craft (alchemy) to purify spoiled food or ingredients. The DCs of such tasks, and what can be accomplished, can vary widely. In general, the more difficult the task, and the more loosely it is related to the skill, the higher the DC.

\subsection{Creature Handling (Cha; Trained Only)}
You can handle creatures without being able to speak with them, convincing them to do what you want or training them to follow commands. This skill can only be used with creatures with an Intelligence of \minus5 or lower.

Animals are easier to handle than other kinds of creatures. The DCs listed are for animals; the DC to handle other kinds of creatures are 5 higher.

\subsubsection{Handling Creatures}
You can use Creature Handling to control a creature's actions. Success indicates it does what you want on its next action. Failure indicates that your action is wasted, and the creature does not listen to you. Exceptional failure may make the creature hostile, depending on the circumstances.
\parhead{Pacify} As a standard action, you can make a check to pacify a creature. If the creature fails a Will save with a DC equal to your Creature Handling check, it will do nothing for 5 rounds. If it is threatened or damaged, this effect is automatically broken. Actively hostile creatures gain a \plus10 circumstance bonus on this saving throw. If you interfere with an action the creature is trained to perform while it is pacified, such as entering a room it is trained to guard, it receives a new saving throw each round with a \plus10 circumstance bonus. You an attempt to pacify a creature as a swift action by taking a \minus10 penalty on the check.
\parhead{Perform Trained Action} As a swift action, you can make a DC 10 check to convince a creature to perform an action it is trained to perform. Generally, wild animals are not trained in any actions, so this is not effective on them.
\parhead{Push} As a standard action, you can make a DC 25 check to convince a willing creature to perform an action it is not trained to perform, but which it is physically capable of performing. This also covers making a creature perform a forced march and similar activities. You can attempt to push a creature as a swift action by taking a \minus10 penalty on the check.

\subsubsection{Training Creatures}
If you are trained in Creature Handling, you can use it to train a creature. Success indicates that the creature learns a trick or becomes domesticated. Failure means your time is wasted, and you must try again. Exceptional failure may indicate that the animal becomes unable to learn the trick, becomes hostile, or other consequences depending on the situation. Training a creature takes a week or more; this requires spending at least four hours each day with the creature. It is not generally possible to accelerate the process by spending more time each day; the creature must take time to learn the new behavior. If the training is interrupted for a week or more, the attempt automatically fails.

\parhead{Teach a Trick} You can teach a creature a specific trick with one week of work and a successful Creature Handling check against the indicated DC. A creature can learn a number of tricks equal to its Intelligence \add 10. Thus, a creature with an Intelligence of \minus9 can learn a single trick, while a creature with an Intelligence of \minus5 can learn five tricks. Possible tricks (and their associated DCs) include, but are not necessarily limited to, the following.

\subsk{Attack (DC 20)} The creature attacks apparent enemies. You may point to a particular creature that you wish the creature to attack, and it will comply if able. Normally, a creature will attack only humanoids, monstrous humanoids, giants, or other creatures. Teaching a creature to attack all creatures (including such unnatural creatures as undead and aberrations) counts as two tricks.
\subsk{Come (DC 15)} The creature comes to you, even if it normally would not do so.
\subsk{Defend (DC 20)} The creature defends you (or is ready to defend you if no threat is present), even without any command being given. Alternatively, you can command the creature to defend a specific other character.
\subsk{Down (DC 15)} The creature breaks off from combat or otherwise backs down. A creature that doesn't know this trick continues to fight until it must flee (due to injury, a fear effect, or the like) or its opponent is defeated.
\subsk{Fetch (DC 15)} The creature goes and gets something. If you do not point out a specific item, the creature fetches some random object.
\subsk{Guard (DC 20)} The creature stays in place and prevents others from approaching.
\subsk{Heel (DC 15)} The creature follows you closely, even to places where it normally wouldn't go.
\subsk{Perform (DC 15)} The creature performs a variety of simple tricks, such as sitting up, rolling over, roaring or barking, and so on.
\subsk{Seek (DC 15)} The creature moves into an area and looks around for anything that is obviously alive or animate.
\subsk{Stay (DC 15)} The creature stays in place, waiting for you to return. It does not challenge other creatures that come by, though it still defends itself if it needs to.
\subsk{Track (DC 20)} The creature tracks the scent presented to it. (This requires the creature to have the scent ability)
\subsk{Work (DC 15)} The creature pulls or pushes a medium or heavy load.

\parhead{Rear a Wild Creature} To rear a creature means to raise a wild creature from infancy so that it becomes domesticated. The time required depends on the creature in question. The DC for this check is equal to 15 \add the Hit Values of the creature. You can rear as many as three creatures of the same kind at once without penalty. You can rear additional creatures, but you take a cumulative \minus2 penalty to checks you make to rear all of the animals. A successfully domesticated creature can be taught tricks at the same time it's being raised, or it can be taught as a domesticated creature later.
\subsection{Devices (Int; Trained Only)}
You can use this skill to manipulate mechanical devices such as locks, traps, and other contraptions. With enough skill, you can even manipulate magical devices.

The DC of a Devices check depends on the device being manipulated. In addition, some actions are easier than others, and modify the DC accordingly. DCs are listed on \trefnp{Devices DCs}.

\begin{dtable}
\lcaption{Device DCs}
\begin{tabularx}{\columnwidth}{>{\lcol}X c}
\thead{Device Type} & \thead{Base DC} \\
Simple device (wagon wheel) & 10 \\
Average device (door hinge) & 15 \\
Challenging device (typical lock or trap) & 20 \\
Difficult device (good lock) & 25 \\
Magic trap & 25 \add double spell level \\
Extraordinary device (masterwork lock, complex trap) & 30 \\
\end{tabularx}
\end{dtable}

\subsubsection{Activate Device}
As a standard action, you can make a Devices check to make a device perform a function it was designed to do, even if you lack the normal requirements. For example, you could lock or unlock a lock without its key, or activate a trap directly without using its triggering mechanism (hopefully without being in its line of fire).

\subsubsection{Create Bindings}
As a standard action, you can make a Devices check to tie a knot or create a binding. You can also take a full-round action to bind a helpless foe in rope or similar material. Your check result is equal to the DC to escape the binding.

\subsubsection{Break Device}
As a standard action, you can make a Devices check to break a device. The DC is 5 lower than normal. This is generally not subtle, such as jamming a lock or breaking a trap while triggering it in the process. It cannot change the state of a device, but you can prevent it from being used normally; a locked door will remain locked, but you can prevent it from being unlocked with its key. Failure indicates that the device continues to function. Failure by 6 or more may cause you to think that you successfully broke the device, while in fact it functions normally.

\subsubsection{Subvert Device}
As a standard action, you can make a Devices check to subvert the normal functionality of a device. The DC is 5 higher than normal. This could allow you to sabotage a hinge or lock so it breaks once it is used, disable a trap without triggering it, or disable a lock so it is perpetually closed or open. Failure by 5 or less means you were unsuccessful, and your action was wasted. Failure by 6 or more means you break the device immediately, think you sabotaged the device when you did not, or activate the trap, depending on the circumstances.

Once you have disabled a device, you can attempt to recover it for later use if you can carry it. This requires a separate Devices check to subvert the device. The DC is 5 higher than normal, as usual for a check to subvert a device.

\subsubsection{Special Circumstances}

You can attempt to leave no trace of your tampering while making a Devices check to activate or subvert a device. This increases the Devices DC by 5, but increases the Perception DC to notice the tampering by 10.

When dealing with traps, you are always considered to be ``threatened'' by the trap, preventing you from taking 10.


\subsection{Disguise (Int)}
Disguise represents your ability to create disguises to conceal the appearance of creatures or objects. This skill does not help you act appropriately while disguised; see Perform (acting) and Bluff.

\subsubsection{Conceal Object}
You can make a Disguise check as a full-round action to conceal an item on your person. The object must be at least two size categories smaller than you are. Your Disguise check is opposed by the Perception check of anyone observing you. If your Disguise check exceeds their Perception check, they fail to notice the item. A creature directly interacting you to search for the object, such as by frisking you, gains a \plus5 circumstance bonus to its Perception check.

\subsubsection{Disguise Creature}
You can make a Disguise check to change the appearance of a creature using makeup, costumes, and so forth. An observer can perceive the presence of a disguise with a Disguise or Perception check to identify a disguise that is higher than your Disguise check. The effectiveness of a disguise depends in part on how much you're attempting to change the creature's appearance, as shown on the table below. The modifiers are cumulative; use any that apply.

Creating a disguise takes 1d4 \mtimes 10 minutes. You can take a \minus10 penalty to reduce the time to 1d4 minutes, or a \minus20 penalty to reduce the time to 1d4 rounds.

The Disguise check is made secretly, so that you can't be sure how good the result is. However, you can attempt to judge its effectiveness with an identify disguise check using either Disguise or Perception. 

\begin{dtable}
\begin{tabularx}{\columnwidth}{l >{\ccol}X}
\thead{Characteristic} & \thead{Disguise Check Modifier} \\
Different gender & \minus2 \\
Different race or subtype & \minus2 \\
Different age category & \minus2\footnotetemp{1} \\
Different creature type & \minus5 \\
Additional limb & \minus5\fn{2} \\
Larger size category & \plus20\footnotetemp{2}
\end{tabularx}
1 Per step of difference between your actual age category and your
disguised age category. The steps are: young (younger than
adulthood), adulthood, middle age, old, and venerable. \\
2 Per limb. \\
3 Per step of difference between the original size category and the new size category.
\end{dtable}

\subsubsection{Emulate Creature}
You can make a Disguise check to make a creature look like another creature you have previously observed. This functions like a disguise creature check, but the result of your Disguise check can't exceed the result of a Perception check you have made to observe the creature. People viewing the disguise who know what that person looks like get a \plus5 circumstance bonus on their Spot checks to identify the disguise.

\subsubsection{Identify Disguise}
You can make a Disguise check to identify a disguise on another creature. The DC is equal to the Disguise check used to create the disguise. Success means you know that the creature is disguised. If you succeed by 10 or more, Success means you know that the creature is disguised. If you succeed by 10 or more, you can also discern the creature's true appearance beneath the disguise. You can make an identify disguise check against any individual creature once per hour.

\subsection{Escape Artist (Dex; Armor Check Penalty)}
Escape Artist represents your ability to escape bindings and move through small areas by contorting your body.

\subsubsection{Escape Bindings}
You can make an Escape Artist check as a full-round action to escape bindings and restraints. The DCs of various restraints are given on the table below.

\begin{dtable}
\begin{tabularx}{\columnwidth}{>{\lcol}X l}
\thead{Restraint}  & \thead{Escape Artist DC} \\
Ropes & Binder's grapple or Devices check \\
Net & 20 \\
Manacles  & 30 \\
Masterwork manacles  & 35 \\
Grappler & Grappler's grapple attack result	 \\
\spell{Entangle}, \spell{web}, or similar spells & Spell's save DC \\
\end{tabularx}
\end{dtable}

\subsubsection{Squeeze}
You can make an Escape Artist check as a full-round action to move one foot forward in a space too small to normally fit you. A DC 15 check allows you to fit into a space that can fit your head and shoulders, but which is too tight to allow crawling. A DC 30 check allows you fit into a space that can fit your head, but not your shoulders. Success indicates that you make progress through the space, while failure indicates that your action is wasted.

If you take a \minus10 penalty to your Escape Artist check, you can squeeze as a move action.

\subsection{Heal (Wis)}
Heal allows you to tend the wounds of others. In order to heal a creature, you must be able to see and touch it.

\subsubsection{Accelerate Recovery}
You can make a DC 15 Heal check to treat wounded people, allowing them to recover more quickly. Success means the patient recovers hit points or attribute damage at twice the normal rate: half the patient's hit points and one point of ability damage for 4 hours of rest, or all of the patient's hit points and two points of ability damage with 8 hours of rest. For every 5 points by which you beat the DC, you half the patient's recovery time again.

You can tend as many as six patients at a time. You need a few items and supplies (bandages, salves, and so on) that are easy to come by in settled lands. Accelerating a creature's recovery counts as light activity. %What does light activity mean?

\subsubsection{First Aid}
You can make a DC 15 Heal check as a standard action to stabilize a dying character. Success indicates that the patient becomes stable, and no longer needs to make Fortitude saves each round to stave off death. 

\subsubsection{Treat Poison or Disease}
You can make a Heal check to treat poison or disease in a character. The next time the character would make a saving throw against the poison or disease, it can use its saving throw result or your Heal check result, whichever is higher. A creature can only benefit from one such attempt to treat a poison on each saving throw.

Treating a poison takes a standard action. Treating a disease takes ten minutes of work.

\subsubsection{Treat Wound}
You can make a DC 15 Heal check as a standard action to treat some specific wounds, such as from a caltrop or \spell{spike growth} spell. Success usually indicates that the wound is gone, as indicated by the effect's description.

\subsection{Intimidate (Cha)}
You can use Intimidate to intimidate people.

You gain a \plus4 circumstance bonus on your Intimidate check for every size category that you are larger than your target. Conversely, you take a \minus4 penalty on your Intimidate check for every size category that you are smaller than your target. If your target doesn't know how large you are, this modifier does not apply. A character immune to fear (such as a paladin of 3rd level or higher) can't be intimidated, nor can nonintelligent creatures.

\subsubsection{Coercion}
You can make an Intimidate check to convince a creature to do what you want. This functions like a Persuasion check to form an agreement with a group, except that you are always considered an enemy of the group you are intimidating (\plus5 DC modifier). In addition, the DC is 5 lower if the group thinks your group is significantly stronger than them, or 5 higher if the group thinks your group is significantly weaker.

\subsubsection{Demoralize Foe}
As a standard action, you can make an Intimidate check to demoralize a foe within \rngmed range of you. The target must make a Will save against your check result. If it fails, it becomes shaken for 5 rounds.

\subsection{Knowledge (Int; Trained Only)}
Like the Craft, Profession, and Perform skills, Knowledge actually encompasses a number of unrelated skills. Knowledge represents a study of some body of lore, possibly an academic or even scientific discipline. Below are listed typical fields of study.
\begin{itemize*}
\item Arcana (ancient mysteries, magic traditions, arcane symbols,
cryptic phrases, constructs, dragons, magical beasts)
\item Engineering (architecture, buildings, bridges, fortifications, siege weapons)
\item Dungeoneering (aberrations, caverns, oozes, spelunking, subterranean monsters)
\item Geography (lands, terrain, climate, people, outdoor monsters)
\item Local (humanoids, legends, inhabitants, laws, customs, history, nobility, royalty)
\item Nature (animals, fey, giants, monstrous humanoids, plants, seasons and cycles, weather, vermin)
\item Planes (the Inner Planes, the Outer Planes, the Astral Plane,
the Ethereal Plane, outsiders, elementals, magic related to the planes, extraplanar monsters)
\item Religion (gods and goddesses, mythic history, ecclesiastic tradition, holy symbols, undead)
\end{itemize*}

You cannot take 10 or take 20 on Knowledge checks. You cannot retry Knowledge checks unless you are presented with significant new information about the subject that could jog your memory.

You can attempt Knowledge checks untrained, but your result cannot exceed 10, limiting you to only the most well-known facts. Particularly common or famous monsters, such as goblins or dragons, can be recognized with an untrained knowledge check of this sort. 

\subsubsection{Identify Monster}
You can make a Knowledge check reactively to identify a monster and recall its special powers or vulnerabilities. In general, the DC is equal to 10 \add the monster's CR. Success allows you to remember the monster's name and its most well-known features. For every 5 points by which you beat the DC, you remember an additional piece of useful information. Failure indicates you don't remember anything important about the monster. Failure by more than 10 indicates that you remember incorrect information.

\subsubsection{Answer Question/Recall Information}
You can make a Knowledge check reactively to remember information related to your field of study. The DC varies depending on the difficulty of the question. Answering a simple or trivial question that most students would be familiar with is DC 10. Answering a challenging question which would be beyond the reach of most initiates is DC 20. Answering a deeply complex or obscure question which might require information known only by the most learned masters in the field would be DC 30 or higher.

\subsection{Linguistics (Int; Trained Only)}
Linguistics represents your mastery of other languages.

\subsubsection{Decipher Script}
You can make a Linguistics check to decipher writing in an unfamiliar language or a message written in an incomplete or archaic form. The base DC is 20 for the simplest messages, 25 for standard texts, and 30 or higher for intricate, exotic, or very old writing. In addition, the DC increases by 5 if you do not know any languages that use the same alphabet as the writing being deciphered. Deciphering the equivalent of a single page of script takes 1 minute (ten consecutive full-round actions).

Success indicates that you understand the general content of a piece of writing about one page long (or the equivalent). Failure indicates that you fail to understand the writing. Failure by 6 or more forces you to make a DC 5 Wisdom check. If you fail the Wisdom check, you draw a false conclusion about the text. The DC of the Wisdom check may sometimes be higher or lower, depending on the particular writing in question.

Both the check to decipher and (if necessary) the Wisdom check are made secretly, so that you can't tell whether the conclusion you draw is true or false.

\subsubsection{Learn Language}
For every two ranks in Linguistics that you have, you may learn a new language, in addition to your starting languages race (or class). You don't make Linguistics checks to speak or understand languages. You either know a language or you don't. All characters with an Intelligence of \minus2 or higher are presumed to be literate, allowing them to read and write any language they speak. Each language has an alphabet, though sometimes several spoken languages share a single alphabet. Languages are summarized on the table below.

\begin{dtable}
    \lcaption{Languages}
\begin{tabularx}{\columnwidth}{l >{\lcol}X l}
\thead{Language}  & \thead{Typical Speakers}  & \thead{Alphabet} \\
Abyssal  & Demons, chaotic evil outsiders  & Infernal \\
Aquan  & Water-based creatures  & Elven \\
Auran  & Air-based creatures  & Draconic \\
Celestial  & Good outsiders  & Celestial \\
Common  & Humans, halflings, half-elves, half-orcs  & Common \\
Draconic  & Kobolds, troglodytes, lizardfolk, dragons & Draconic \\
Druidic  & Druids (only)  & Druidic \\
Dwarven  & Dwarves  & Dwarven \\
Elven  & Elves  & Elven \\
Giant  & Ogres, giants  & Dwarven \\
Gnome  & Gnomes  & Dwarven \\
Goblin  & Goblins, hobgoblins, bugbears  & Dwarven \\
Gnoll  & Gnolls  & Common \\
Halfling  & Halflings  & Common \\
Ignan  & Fire-based creatures  & Draconic \\
Infernal  & Devils, lawful evil outsiders  & Infernal \\
Orc  & Orcs  & Dwarven \\
Sylvan  & Dryads, brownies, leprechauns  & Elven \\
Terran  & Xorns and other earth-based creatures & Dwarven \\
Undercommon  & Drow & Elven
\end{tabularx}
\end{dtable}

\subsection{Perception (Wis)}
Perception represents your ability to observe things which you might otherwise fail to notice. It can be used to spot concealed things, to correctly identify sounds, or to smell the distinctive rotting flesh smell that accompanies undead.

Each creature has a variety of senses. The Perception skill governs the use of every sense. It is most commonly used with sight, hearing, and smell, though other uses are possible. Often, bonuses or penalites will apply only to certain senses. For example, it is extremely difficult to see when there is no light, but that has no effect on how difficult it is for you to hear sounds. In such cases, you only roll one Perception check, and you apply the modifiers separately for each sense. 

While sleeping, you take a \minus10 penalty to Perception.

\subsubsection{Active Attention}
As a move action, you can make a conscious effort to pay attention to events around you. You may make a Perception check to notice events, rather than simply using your modifier. Most creatures can't maintain this level of attention for longer than 5 rounds \add their Constitution before they have to relax for 5 rounds. %Poorly worded limitation

\subsubsection{Discern Illusion}
You automatically notice inconsistencies in illusion spells. The DC is equal to the save DC of the spell. Success means that you get a Will save to disbelieve the illusion as if you interacted with the illusion.

If the illusion is completely missing a sense that should logically be present, such as a \spell{silent image} of people in armor, the DC to identify the illusion with that sense is lowered by 10. 

\subsubsection{Notice Event}
You automatically notice events around you. The DC depends on the sense used and the obviousness of the event, as described on tables below. If your modifier exceeds the event's DC, you notice something, but you don't know any details -- only its general direction. If your modifier exceeds the DC by 5 or more, you can clearly identify what the event was. If your modifier is less than the DC, you don't notice anything.

\subsubsection{Identify Disguise}
You can make a Perception check as a move action to identify a disguise on another creature. The DC is equal to the Disguise check used to create the disguise. Success means you know that the creature is disguised. If you succeed by 10 or more, you can also discern the creature's true appearance beneath the disguise. You can make an identify disguise check against any individual creature once per hour.

\subsubsection{Identify Forgery}
You can make a Perception check as a full-round action to evaluate whether an item is a forgery. The DC to identify a forgery is equal to the Craft check used to make the item. Success indicates that you correctly identify whether the item is a forgery or not. Failure indicates that you are unsure. Failure by more than 10 indicates that you incorrectly identify the item. The check is made secretly, so you can't be sure how good the result is.

\subsubsection{Read Lips}
You can make a DC 15 sight-based Perception check to read a creature's lips. You must be able to understand the language spoken. Success means you can understand the general content of the message, but not the exact words. Success by 5 or more means you understand the exact words. spoken if you understand the language. Failure means you don't understand the message. Failure by 6 or more means you draw an incorrect conclusion about the message.

\subsubsection{Search}
You can spend a full-round action to make a Perception check to notice things in a single 5-ft. square within 10 feet of you. While doing so, you ignore size penalties that would affect the DC to notice anything within the square.

\subsubsection{Senses}

\parhead{Sight} The DC to see something depends on the obviousness of the sight, as shown on \trefnp{Sight-based DCs}, and other modifiers given at \trefnp{Perception DC Modifiers}.

The DC to notice an invisible creature with sight is 20 higher than normal. Noticing an invisible creature makes you aware of its presence, but doesn't let you see it perfectly.

\begin{dtable}
    \lcaption{Sight-based DCs}
    \begin{tabularx}{\columnwidth}{X l}
        \thead{Situation} & \thead{DC} \\
        Creature or object moving & \minus5\fn{1} \\
        Creature or object standing still & 0\fn{1} \\
        Creature trying to hide & Stealth check result \\ 
        Hidden trap or secret door & Craft or Devices check result \add 10 \\
        Magic trap & 25 \add double level of spell used to create trap \\
    \end{tabularx}
    1 Add the creature or object's special size modifier \\
\end{dtable}

\parhead{Sound} The DC to hear a sound depends on the intensity of the sound, as shown on \trefnp{Sound-based DCs}, and other modifiers given at \trefnp{Perception DC Modifiers}.

Background noise can make it more difficult to notice sounds. If there is significant background noise of a similar intensity to the sound to be detected, the DC increases by 5. If there is significant background noise of a much greater intensity than the sound to be detected, the DC increases by 10.

\begin{dtable}
    \lcaption{Sound-based DCs}
    \begin{tabularx}{\columnwidth}{X l}
        \thead{Situation} & \thead{DC} \\
        Creature shouting & \minus5\fn{1} \\
        Creature talking normally or fighting & 0\fn{1} \\ 
        Creature whispering & 10\fn{1} \\
        Creature standing still & 15\fn{1} \\
        Creature trying to be quiet & Stealth check result \\ 
    \end{tabularx}
    1 Add the creature or object's special size modifier \\
\end{dtable}

\parhead{Scent} The DC to smell something depends on the intensity of the scent, as shown on \trefnp{Scent-based DCs}, and other modifiers given at \trefnp{Perception DC Modifiers}.

The DCs given are for a creature with an ordinary sense of smell, like a human. A creature with an unusually keen sense of smell, like most animals, gains a \plus5 competence bonus to scent-based Perception checks. A creature with the scent ability, such as a dog, gains a \plus10 competence bonus to scent-based Perception checks.

Smell intensity can vary widely depending on the circumstances. In general, an unusually strong smell, such as a creature wearing perfume, has a DC which is 5 lower. An unusually weak smell, such as a creature who has just taken an unscented bath, has a DC which is 5 higher.

\begin{dtable}
    \lcaption{Scent-based DCs}
    \begin{tabularx}{\columnwidth}{X l}
        \thead{Situation} & \thead{DC} \\
        Strong scent (rotting flesh, pungent spices) & 0 \\
        Moderate scent (fresh food) & 5 \\
        Living creature\fn{1} & 10 \\
    \end{tabularx}
    1 Add the creature's special size modifier \\
\end{dtable}

\parhead{Other Senses} Other senses can exist, and creatures can make Perception checks to use those other senses appropriately.

\subsubsection{Modifiers}
All Perception checks share the same set of modifiers, in addition to certain modifiers which depend on the sense. These are noted on \trefnp{Perception DC modifiers}.

\begin{dtable}
    \lcaption{Perception DC Modifiers}
    \begin{tabularx}{\columnwidth}{X l}
        \thead{Number} & \thead{DC Modifier} \\
        One creature or object & \plus0 \\
        Two creatures or objects & \plus2 \\
        Five creatures or objects & \plus5 \\
        Twenty creatures or objects & \plus10 \\
        A hundred creatures or objects & \plus15 \\
        Five hundred creatures or objects & \plus20 \\
        \thead{Distance} & \thead{DC Modifier} \\
        Less than five feet away & \plus0 \\
        Five feet away & \plus2 \\
        Twenty feet away & \plus5 \\
        A hundred feet away & \plus10 \\
        Five hundred feet away & \plus15 \\
        Half a mile away & \plus20 \\
    \end{tabularx}
\end{dtable}

\subsection{Perform (Cha)}
\par Like Craft, Knowledge, and Profession, Perform is actually a number of separate skills. You could have several Perform skills, each with its own ranks, each purchased as a separate skill.

Each of the nine categories of the Perform skill includes a variety of methods, instruments, or techniques, a small list of which is provided for each category below.
\begin{itemize*}
\item Act (comedy, drama, mime)
\item Comedy (buffoonery, limericks, joke-telling)
\item Dance (ballet, waltz, jig)
\item Keyboard instruments (harpsichord, piano, pipe organ)
\item Oratory (epic, ode, storytelling)
\item Percussion instruments (bells, chimes, drums, gong)
\item String instruments (fiddle, harp, lute, mandolin)
\item Wind instruments (flute, pan pipes, recorder, shawm, trumpet)
\item Sing (ballad, chant, melody)
\end{itemize*}
\parhead{Check} You can impress audiences with your talent and skill.
\par A masterwork musical instrument gives you a \plus2 circumstance bonus on Perform checks that involve its use.
\parhead{Action} Varies. Trying to earn money by playing in public requires anywhere from an evening's work to a full day's performance.
\parhead{Try Again} Yes. Retries are allowed, but they don't negate previous failures, and an audience that has been unimpressed in the past is likely to be prejudiced against future performances. (Increase the DC by 2 for each previous failure.)

\subsection{Persuasion (Cha)}
You can use Persuasion to convince people to do or think what you want. Depending on how it is used, it represents a combination of verbal acuity, tact, argumentative ability, grace, etiquette, and personal magnetism. Using a Persuasion check usually takes at least a minute of sustained conversation. You cannot normally take 10 on Persuasion checks.

Persuasion checks are usually made against a group. For the purposes of a Persuasion check, a ``group'' consists of creatures who consider themselves to be allies. For example, in a king's court, you cannot simply influence the king alone; his trusted advisors are also part of the same group. It is possible to influence one group without influencing another. For example, the king and his advisors may be unpersuaded, but the prince may find your arguments compelling. Groups can also consist of a single creature. The DM decides what the groups are.

The base DC for a Persuasion check against a group is equal to 10 \add the highest level of any character in the group \add the highest Wisdom of any character in the group.

\subsubsection{Compel Belief}
You can make a Persuasion check to cause creatures to believe something you say. If you are lying, you must also make a Bluff check to lie. Success means the group believes what you are saying is true. What they choose to do with that information is up to them, however. Failure by less than 10 means they do not believe you, but they do not react poorly; perhaps they simply want more verification. You may be able to try again, depending on their patientce. Failure by 10 or more means the group reacts poorly, and you may have permanently damaged your credibility with them.

Your check is modified by how believable your argument is, as well as whether the group has has strong feelings about the truth of your story.

\begin{dtable}
  \lcaption{Believability Modifiers}
  \begin{tabularx}{\columnwidth}{X l}
    \thead{Description} & \thead{DC Modifier}  \\
    Expected to be true (``Nothing interesting happened while I was on patrol'') & \minus5 \\
    Plausible (``The mayor is too busy to see you now.'') & \plus0 \\
    Unlikely (``The north gate is under attack!'') & \plus5 \\
    Extremely unlikely (``The mayor is secretly a vampire.'') & \plus10 \\
    Virtually impossible (``You are secretly a vampire.'') & \plus20 \\
    Demonstratably untrue (``You are a frog.'') & \x\fn{1} \\
  \end{tabularx}
  1 You cannot convince someone of something that is proven to be false.
\end{dtable}

\begin{dtable}
  \lcaption{Motivation Modifiers}
  \begin{tabularx}{\columnwidth}{X l}
    \thead{Description} & \thead{DC Modifier} \\
    Target wants to believe (``That dress looks lovely on you.'') & \minus5 \\
    Target does not have strong feelings (``I'm busy.'') & \plus0 \\
    Target doesn't want the story to be true (``Your brother is a murderer.'') & \plus5 \\
  \end{tabularx}
\end{dtable}

\subsubsection{Form Agreement}
You can make a Persuasion check to convince a group to accept a deal you offer. This can be used to persuade the chamberlain to let you see the king, to negotiate peace between feuding barbarian tribes, or to convince the ogre mages that have captured you that they should ransom you back to your friends instead of twisting your limbs off one by one.

Success means the group will fulfill their end of the deal. Of course, if you don't fulfill your part, they are likely to react poorly. Failure by less than 10 means they did not accept the deal, but they may propose an alternate arrangement, or you can propose another deal without penalty. Failure by 10 or more means that there is virtually no chance to reach an agreement, and the group may become hostile or take other steps to end the conversation. They may have been insulted by how unfair the deal was, or you may have made a critical verbal misstep.

Your check is modified by your relationship with the group and how favorable the deal is.

\begin{dtable}
\begin{tabularx}{\columnwidth}{>{\lcol}X r}
\thead{Relationship} & \thead{Modifier} \\
Intimate: Someone who with whom you have an implicit trust.
Example: A lover or spouse. & \minus15 \\
Friend: Someone with whom you have a regularly positive personal relationship.
Example: A long-time buddy or a sibling. & \minus10 \\
Ally: Someone on the same team, but with whom you have no personal relationship.
Example: A cleric of the same religion or a knight serving the same king. & \minus5 \\
Acquaintance (Positive): Someone you have met several times with no particularly negative experiences. Example: The blacksmith that buys your looted equipment regularly. & \minus2 \\
Just Met: No relationship whatsoever.
Example: A guard at a castle or a traveler on a road. & \plus0 \\
Acquaintance (Negative): Someone you have met several times with no particularly positive experiences. Example: A town guard that has arrested you for drunkenness once or twice. & \plus2 \\
Enemy: Someone on an opposed team, with whom you have no personal relationship.
Example: A cleric of a philosophically-opposed religion or an orc bandit who is robbing you. & \plus5 \\
Personal Foe: Someone with whom you have a regularly antagonistic personal relationship.
Example: An evil warlord whom you are attempting to thwart, or a bounty hunter who is tracking you down for your crimes. & \plus10 \\
Nemesis: Someone who has sworn to do you, personally, harm. Example: The brother of a man you murdered in cold blood. & \plus15 \\
\end{tabularx}
\end{dtable}
\begin{dtable*}
\begin{tabularx}{\textwidth}{>{\lcol}X r}
\thead{Risk vs. Reward Judgement (Persuasion)} & \thead{Modifier} \\
Fantastic: The reward for accepting the deal is very worthwhile; the risk is either acceptable or extremely unlikely. The best-case scenario is a virtual guarantee. Example: An offer to pay a lot of gold for information that isn't important to the character. & \minus15 \\
Good: The reward is good and the risk is minimal. The subject is very likely to profit from the deal. Example: An offer to pay someone twice their normal daily wage to spend their evening in a seedy tavern with a reputation for vicious brawls and later report on everyone they saw there. & \minus10\\
Favorable: The reward is appealing, but there's risk involved. If all goes according to plan, though, the deal will end up benefiting the subject. Example: A request for a mercenary to aid the party in battle against a weak goblin tribe in return for a cut of the money and first pick of the magic items. & \minus5\\
Even: The reward and risk more of less even out; or the deal involves neither reward nor risk. Example: A request for directions to a place that isn't a secret. & \plus0 \\
Unfavorable: The reward is not enough compared to the risk involved. Even if all goes according to plan, chances are it will end badly for the subject. Example: A request to free a prisoner the target is guarding for a small amount of money. & \plus5\\
Bad: The reward is poor and the risk is high. The subject is very likely to get the raw end of the deal. Example: A request for a mercenary to aid the party in battle against an ancient red dragon for a small cut of any non-magical treasure. & \plus10 \\
Horrible: There is no conceivable way that the proposed plan could end up with the subject ahead or the worst-case scenario is guaranteed to occur. Example: An offer to trade a rusty kitchen knife for a shiny new longsword. & \plus15 \\
\end{tabularx}
\end{dtable*}

\subsubsection{Gather Information}
An evening's time, a few gold pieces for buying drinks and making friends, and a DC 10 Persuasion check get you a general idea of a city's major news items, assuming there are no obvious reasons why the information would be withheld. The higher your check result, the better the information.

If you want to find out about a specific rumor, or a specific item, or obtain a map, or do something else along those lines, the DC for the check is 15 to 25, or even higher.

\subsection{Profession (Wis; Trained Only)}
Like Craft, Knowledge, and Perform, Profession is actually a number of separate skills. You could have several Profession skills, each with its own ranks, each purchased as a separate skill. While a Craft skill represents ability in creating or making an item, a Profession skill represents an aptitude in a vocation requiring a broader range of less specific knowledge.

\parhead{Check} You can practice your trade and make a decent living, earning about half your Profession check result in gold pieces per week of dedicated work. You know how to use the tools of your trade, how to perform the profession's daily tasks, how to supervise helpers, and how to handle common problems.

\parhead{Action} Not applicable. A single check generally represents a week of work.
\parhead{Try Again} Varies. An attempt to use a Profession skill to earn an income cannot be retried. You are stuck with whatever weekly wage your check result brought you. Another check may be made after a week to determine a new income for the next period of time. An attempt to accomplish some specific task can usually be retried.
\parhead{Untrained} Untrained laborers and assistants (that is, characters without any ranks in Profession) earn an average of 1 silver piece per day. Most commoners have ranks in a Profession skill.

\subsection{Ride (Dex)}
If you attempt to ride a creature that is ill suited as a mount, you take a ?5 penalty on your Ride checks.
\parhead{Check} Typical riding actions don't require checks. You can saddle, mount, ride, and dismount from a mount without a problem. The following tasks do require checks.

\begin{dtable}
\begin{tabularx}{\columnwidth}{>{\lcol}X c >{\lcol}X c}
\thead{Task}  & \thead{Ride DC}  & \thead{Task}  & \thead{Ride DC} \\
Guide with knees  & 5 & Leap  & 15 \\
Stay in saddle  & 5 & Spur mount  & 15 \\
Fight with warhorse  & 10 &  Control mount in battle & 20 \\
Cover  & 15 & Fast mount or dismount & 20\fn{1} \\
Soft fall & 15 &  &
\end{tabularx}
1 Armor check penalty applies.
\end{dtable}
\subsk{Guide with Knees} You can react instantly to guide your mount with your knees so that you can use both hands in combat. Make your Ride check at the start of your turn. If you fail, you can use only one hand this round because you need to use the other to control your mount.

\subsk{Stay in Saddle} You can react instantly to try to avoid falling when your mount rears or bolts unexpectedly or when you take damage. This usage does not take an action

\subsk{Fight with Warhorse} If you direct your war-trained mount to attack in battle, you can still make your own attack or attacks normally. This usage is a free action.

\subsk{Cover} You can react instantly to drop down and hang alongside your mount, using it as cover. You can't attack or cast spells while using your mount as cover. If you fail your Ride check, you don't get the cover benefit. This usage does not take an action.

\subsk{Soft Fall} You can react instantly to try to take no damage when you fall off a mount -- when it is killed or when it falls, for example. If you fail your Ride check, you take 1d6 points of falling damage. This usage does not take an action.

\subsk{Leap} You can get your mount to leap obstacles as part of its movement. Use your Ride modifier or the mount's Jump modifier, whichever is lower, to see how far the creature can jump. If you fail your Ride check, you fall off the mount when it leaps and take the appropriate falling damage (at least 1d6 points). This usage does not take an action, but is part of the mount's movement.

\subsk{Spur Mount} You can spur your mount to greater speed with a move action. A successful Ride check increases the mount's speed by 10 feet for 1 round but deals 1 point of damage to the creature. You can use this ability every round, but each consecutive round of additional speed deals twice as much damage to the mount as the previous round (2 points, 4 points, 8 points, and so on).

\subsk{Control Mount in Battle} As a move action, you can attempt to control a light horse, pony, heavy horse, or other mount not trained for combat riding while in battle. If you fail the Ride check, you can do nothing else in that round. You do not need to roll for warhorses or warponies.

\subsk{Fast Mount or Dismount} You can attempt to mount or dismount from a mount of up to one size category larger than yourself as a free action, provided that you still have a move action available that round. If you fail the Ride check, mounting or dismounting is a move action. You can't use fast mount or dismount on a mount more than one size category larger than yourself.

\parhead{Action} Varies. Mounting or dismounting normally is a move action. Other checks are a move action, a free action, or no action at all, as noted above.
\parhead{Special} If you are riding bareback, you take a \minus5 penalty on Ride checks.
\par If your mount has a military saddle (\pref{Mounts and Related Gear}), you get a \plus2 circumstance bonus on Ride checks related to staying in the saddle.

\subsection{Sense Motive (Wis)}
\parhead{Check} A successful check lets you avoid being bluffed (see the Bluff skill). You can also use this skill to determine when ``something is up'' (that is, something odd is going on) or to assess someone's trustworthiness.

\begin{dtable}
\begin{tabularx}{\columnwidth}{>{\lcol}X >{\lcol}X}
Task Sense Motive & DC \\
Hunch & 20 \\
Sense enchantment & 25 or 15 \\
Discern secret message & Varies \\
\end{tabularx}
\end{dtable}

\subsk{Hunch} This use of the skill involves making a gut assessment of the social situation. You can get the feeling from another's behavior that something is wrong, such as when you're talking to an impostor. Alternatively, you can get the feeling that someone is trustworthy.

\subsk{Discern Mental Influence} If you or a creature you are interacting with is affected by beguilement, compulsion, or domination effect that alters their behavior, you can attempt a Sense Motive check against the save DC of the spell (including other modifiers as normal). If you succeed, you recognize the existence of unnatural mental influence. If you recognize such an effect on yourself, you can make a Will save to fight off the effect once each round.

This can only be used if the effect in question is actually affecting the creature's behavior at the time. For example, a person who has been given an unnatural aversion to cheese, as the \spell{aversion} spell, would generally not show signs of the enchantment unless presented with cheese. Therefore, you could not generally make a Sense Motive check to detect the effect if the creature was simply talking with you about the weather.

\subsk{Discern Secret Message} You may use Sense Motive to detect that a hidden message is being transmitted via the Bluff skill. In this case, your Sense Motive check is opposed by the Bluff check of the character transmitting the message. For each piece of information relating to the message that you are missing, you take a \minus2 penalty on your Sense Motive check. If you succeed by 4 or less, you know that something hidden is being communicated, but you can't learn anything specific about its content. If you beat the DC by 6 or more, you intercept and understand the message. If you fail by 4 or less, you don't detect any hidden communication. If you fail by 6 or more, you infer some false information.

\parhead{Action} Trying to gain information with Sense Motive generally takes at least 1 minute, and you could spend a whole evening trying to get a sense of the people around you.
\parhead{Try Again} No, though you may make a Sense Motive check for each Bluff check made against you.
\par If you have the Negotiator feat, you get a \plus2 competence bonus on Sense
Motive checks.

\subsection{Sleight Of Hand (Dex; Trained Only; Armor Check Penalty)}
\parhead{Check} A DC 10 Sleight of Hand check lets you palm a coin-sized, unattended object. Performing a minor feat of legerdemain, such as making a coin disappear, also has a DC of 10 unless an observer is determined to note where the item went.

When you use this skill under close observation, your skill check is opposed by the observer's Spot check. The observer's success doesn't prevent you from performing the action, just from doing it unnoticed. If the observer is actively defending himself from you, you can't make a Sleight of Hand check to steal from him. See the Disarm combat maneuver instead.

You can hide a tiny object (including a light weapon or an easily concealed ranged weapon, such as a dart, sling, or hand crossbow) on your body. The object must be at least two size categories smaller than you are. Your Sleight of Hand check is opposed by the Spot check of anyone observing you. The observer gains a \plus4 bonus for directly interacting with you to search for the object, such as by frisking you. You gain a bonus on your Sleight of Hand check when hiding extraordinarily small objects: Diminuitive objects grant a \plus2 circumstance bonus, and Fine objects grant a \plus4 circumstance bonus.

Drawing a hidden weapon is a standard action and doesn't provoke an attack of opportunity.

If you try to take something from another creature, you must make a DC 20 Sleight of Hand check to obtain it. The opponent makes a Spot check to detect the attempt, opposed by the same Sleight of Hand check result you achieved when you tried to grab the item. An opponent who succeeds on this check notices the attempt, regardless of whether you got the item.

You can also use Sleight of Hand to entertain an audience as though you were using the Perform skill. In such a case, your ``act'' encompasses elements of legerdemain, juggling, and the like.

\begin{dtable}
\begin{tabularx}{\columnwidth}{>{\lcol}X >{\lcol}X}
\thead{DC} & \thead{Task} \\
10 & Palm a coin-sized object, make a coin disappear \\
20 & Lift a small object from a person
\end{tabularx}
\end{dtable}

\parhead{Action} Any Sleight of Hand check is normally a standard action. However, you may perform a Sleight of Hand check as a swift action by taking a \minus20 penalty on the check.
\parhead{Try Again} Yes, but observers who have seen you fail before gain a \plus10 circumstance bonus to catch you again.
\parhead{Untrained} An untrained Sleight of Hand check is simply a Dexterity check. Without actual training, you can't succeed on any Sleight of Hand check with a DC higher than 10, except for hiding an object on your body.

\subsection{Spellcraft (Wis; Trained Only)}
Use this skill to identify spells as they are cast or spells already in place.

\begin{dtable}
\lcaption{Spellcraft DCs}
\begin{tabularx}{\columnwidth}{l >{\lcol}X l l}
\thead{Spellcraft DC}  & \thead{Task} & \thead{Action} & \thead{Retry} \\
10 & Notice the presence of any magical auras within 100 feet\fn{1} & Reflexive\fn{2} & Yes \\
10 \add spell level & When using \spell{read magic}, identify a \spell{glyph} or \spell{symbol} spell. & Reflexive & Yes \\
15 & Identify the number of magical auras within 100 feet & Move & Yes \\
15 & Identify the strength and location of an aura you have identified\fn{3} & Move & Yes \\
15 \add spell level\fn{4} & Identify the school of an aura you have identified & Move\fn{5} & Yes \\
15 \add spell level  & Identify a spell being cast. (You must see or hear the spell's verbal or somatic components.) & Reflexive & No  \\
15 \add spell level  & Determine the school of magic involved in the aura of a single item or creature you can see. (If the aura is not a spell effect, the DC is 15 \add one-half caster level.) & Move & Yes \\
20 \add spell level  & Identify a spell that's already in place and in effect. You must be able to see or detect the effects of the spell. & Reflexive & Yes \\
20 \add spell level  & Identify materials created or shaped by magic, such as noting that an iron wall is the result of a \spell{wall of iron} spell. & Reflexive & Yes \\
20 \add spell level  & Decipher a written spell (such as a scroll) without using \spell{read magic}. & Full-round & 1/day \\
25 \add spell level  & After being affected by against a spell targeted on you, determine what that spell was. & Reflexive & No \\
25  & Identify a potion. & 1 minute & No \\
30 or higher  & Understand a strange or unique magical effect, such as the effects of a magic stream. & Varies & No \\
\end{tabularx}
1 You cannot differentiate individual auras. However, you can ignore auras you are familiar with (such as magic items you or your companions wear). \\
2 You can make a check reflexively to perceive new information without taking an action. Actively paying attention to notice something requires a move action. \\
3 See \trefnp{Aura Strengths} for information on how strong auras are.
4 Or half caster level for effects which are not spells.
5 Can be done as part of the same action as identifying the strength and location of the aura.
\end{dtable}

\begin{dtable*}
\lcaption{Aura Strengths}
\begin{tabularx}{\textwidth}{>{\lcol}X *{4}{>{\lcol}p{9em}}}
& \multicolumn{4}{c}{\thead{---{}---{}---Aura Power---{}---{}---}} \\
\thead{Spell or Object} & \thead{Faint} & \thead{Moderate} & \thead{Strong} & \thead{Overwhelming} \\
Functioning spell (spell level) & 3rd or lower & 4th--6th & 7th--9th & 10th\add (deity-level) \\
Magic item (caster level) & 6th or lower & 7th--12th & 13th--20th & 21st\add (artifact)
\end{tabularx}
\end{dtable*}

\parhead{Lingering Auras}A magical aura can linger after its original source dissipates (in the case of a spell) or is destroyed (in the case of a magic item). If you use Spellcraft in such a location, you can an aura strength of ``dim'' (even weaker than a faint aura). How long the aura lingers at this dim level depends on its original power. Most auras only linger for a few rounds, but strong or overwhelming auras can linger for days.
\par Outsiders and elementals are not magical in themselves, but if they are summoned, the conjuration spell registers.

\parhead{Check} You can identify spells and magic effects. The DCs for Spellcraft checks relating to various tasks are summarized on the table above.
\parhead{Action} Varies, as noted above.
\parhead{Try Again} See above.
\parhead{Special} If you are a specialist wizard, you get a \plus2 competence bonus on Spellcraft checks when dealing with a spell or effect from your specialty school. You take a \minus5 penalty when dealing with a spell or effect from a prohibited school (and some tasks, such as learning a prohibited spell, are just impossible).

\subsection{Stealth (Dex; Armor Check Penalty)}
The Stealth skill is used to escape detection while moving or hiding. There are two components required to remain undetected: you must hide from sight, and you must be able to move silently so you cannot be overheard. These are not different skills, but merely different applications of the same skill.

It is possible to have bonuses or penalties to only the Hide part of a Stealth check, or to only the Move Silently part. For example, it is extremely easy to hide when there is no light, but that has no effect on how difficult it is for you to move silently. In such a case, roll the Stealth check once and add the Hide modifier separately from the Move Silently modifier. These are referred to as separate checks -- a Hide check and a Move Silently check -- though you only roll one die. If a creature beats your Move Silently check without beating your Hide check, they have heard you but not seen you, and vice versa.

\parhead{Check} Your Stealth check is opposed by the Perception check of anyone who might see you. If you move during your turn, you take a \minus5 penalty. When moving at a speed greater than one-half but less than your normal speed, you take a \minus10 penalty. It's practically impossible (\minus20 penalty) to remain unobserved while attacking, running or charging.

A creature larger or smaller than Medium takes a size bonus or penalty on Stealth checks depending on its size category: Fine \plus16, Diminutive \plus12, Tiny \plus8, Small \plus4, Large \minus4, Huge \minus8, Gargantuan \minus12, Colossal \minus16.

\subsubsection{Hide}
You need cover or concealment in order to attempt a Hide check. Total cover or total concealment usually (but not always; see Special, below) obviates the need for a Hide check, since nothing can see you anyway. When determining whether or not a character may attempt a hide check against a particular observer, do not consider any cover that does not impede the observer's vision or that would be hidden as a result of a successful check.

If people are observing you, even casually, you can't hide. You can run around a corner or behind cover so that you're out of sight and then hide, but the others then know at least where you went.

If your observers are momentarily distracted (such as by a Bluff check; see below), though, you can attempt to hide. While the others turn their attention from you, you can attempt a Hide check if you can get to a hiding place of some kind. (As a general guideline, the hiding place has to be within a quarter of your movement speed.) This check, however, is made at a \minus10 penalty because you have to move fast.

\subsk{Sniping} If you've already successfully hidden at least 10 feet from your target, you can make take a standard action to attack, then immediately hide again. You take a \minus20 penalty on your Hide check to conceal yourself after the attack.

\subsk{Creating a Diversion to Hide} You can use Bluff to help you hide. A successful Bluff check can give you the momentary diversion you need to attempt a Hide check while people are aware of you.

\parhead{Action} Usually none. Normally, you make a Hide check as part of movement, so it doesn't take a separate action. However, hiding immediately after an attack (see Sniping, above) is a move action.

\parhead{Special} If you are invisible, you gain a \plus40 bonus on Hide checks if you are immobile, or a \plus20 bonus on Hide checks if you're moving.

\subsubsection{Move Silently}
Noisy surfaces, such as bogs or undergrowth, are tough to move silently across. When you try to sneak across such a surface, you take a penalty on your Stealth check as indicated below.
\begin{dtable}
\begin{tabularx}{\columnwidth}{>{\lcol}X c}
\thead{Surface} & \thead{Check Modifier} \\
Noisy (scree, shallow or deep bog, undergrowth, dense rubble) & \minus2 \\
Very noisy (dense undergrowth, deep snow) & \minus5 \\
\end{tabularx}
\end{dtable}
\parhead{Action} None. A Move Silently check is included in your movement or other activity, so it is part of another action.

\subsection{Survival (Wis)}
\parhead{Check} You can keep yourself and others safe and fed in the wild. The table below gives the DCs for various tasks that require Survival checks.

Survival does not allow you to follow difficult tracks unless you are a ranger or have the Track feat (see the Restriction section below).

\begin{dtable}
\begin{tabularx}{\columnwidth}{c >{\lcol}X}
\thead{Survival DC}  & \thead{Task} \\
10  & Get along in the wild. Move up to one-half your overland speed while hunting and foraging (no food or water supplies needed). You can provide food and water for one other person for every 2 points by which your check result exceeds 10. \\
15  & Keep from getting lost or avoid natural hazards, such as quicksand. \\
15  & Predict the weather up to 24 hours in advance. For every 5 points by which your Survival check result exceeds 15, you can predict the weather for one additional day in advance. \\
Varies  & Follow tracks (see the Track feat).
\end{tabularx}
\end{dtable}

\parhead{Action} Varies. A single Survival check may represent activity over the course of hours or a full day. A Survival check made to find tracks is at least a full-round action, and it may take even longer.

\parhead{Try Again} Varies. For getting along in the wild or for gaining the Fortitude save bonus noted in the table above, you make a Survival check once every 24 hours. The result of that check applies until the next check is made. To avoid getting lost or avoid natural hazards, you make a Survival check whenever the situation calls for one. Retries to avoid getting lost in a specific situation or to avoid a specific natural hazard are not allowed. For finding tracks, you can retry a failed check after 10 minutes of searching, plus 10 more minutes for every 5 that you failed the check by.

\parhead{Restriction} While anyone can use Survival to find tracks (regardless of the DC), or to follow tracks when the DC for the task is 10 or lower, only a character with the Track feat can use Survival to follow tracks when the task has a higher DC.

\parhead{Special} If you have 5 or more ranks in Survival, you can automatically determine where true north lies in relation to yourself.

\subsection{Swim (Str; Armor Check Penalty)}
\parhead{Check} Make a Swim check once per round while you are in the water. Success means you may swim at up to one-half your speed (as a full-round action) or at one-quarter your speed (as a move action). If you fail by 4 or less, you make no progress through the water. If you fail by 6 or more, you go underwater.

If you are underwater, either because you failed a Swim check or because you are swimming underwater intentionally, you must hold your breath. You can hold your breath for a number of rounds equal to your Constitution score, but only if you do nothing other than take move actions or free actions. If you take a standard action or a full-round action (such as making an attack), the remainder of the duration for which you can hold your breath is reduced by 1 round. (Effectively, a character in combat can hold his or her breath only half as long as normal.) After that period of time, you must make a DC 10 Constitution check every round to continue holding your breath. Each round, the DC for that check increases by 1. If you fail the Constitution check, you begin to drown.

The DC for the Swim check depends on the water, as given on the table below.

\begin{dtable}
\begin{tabularx}{\columnwidth}{>{\lcol}X >{\lcol}X}
\thead{Water} & \thead{Swim DC} \\
Calm water & 10 \\
Rough water & 15 \\
Stormy water & 20\footnotetemp{1}
\end{tabularx}
1 You can't take 10 on a Swim check in stormy water, even if you aren't
otherwise being threatened or distracted.
\end{dtable}

Each hour that you swim, you must make a DC 20 Swim check or take 1d6 points of nonlethal damage from fatigue.

\subsk{Accelerated Swimming} If you beat the Swim DC by 10 or more, you can move half your speed (instead of one-quarter your speed).

\parhead{Action} Swimming is a kind of movement, so it's generally a move action. Each move action that includes any swimming requires a separate Swim check.
\parhead{Special} Swim checks are subject to double the normal armor check penalty and encumbrance penalty.

A creature with a swim speed can move through water at its indicated speed without making Swim checks. It gains a \plus5 inherent bonus on any Swim check to perform a special action or avoid a hazard. The creature always can choose to take 10 on a Swim check, even if distracted or endangered when swimming. Such a creature can use the run action while swimming, provided that it swims in a straight line.

\begin{comment}
\subsection{Use Rope (Dex)}
With this skill, you can make firm knots, undo tricky knots, and
bind prisoners with ropes.
\parhead{Check} Most tasks with a rope are relatively simple. The DCs for
various tasks utilizing this skill are summarized on the table below.

\begin{dtable}
\begin{tabularx}{\columnwidth}{l >{\lcol}X}
\thead{Use Rope DC} & \thead{Task} \\
10 & Tie a firm knot \\
10\footnotetemp{1} & Secure a grappling hook \\
15 & Tie a special knot, such as one that slips, slides slowly, or loosens with a tug \\
15 & Tie a rope around yourself one-handed \\
15 & Splice two ropes together \\
Varies & Bind a character
\end{tabularx}
1 Add 2 to the DC for every 10 feet the hook is thrown; see below.
\end{dtable}

\subsk{Secure a Grappling Hook} Securing a grappling hook requires a Use
Rope check (DC 10, \plus2 for every 10 feet of distance the grappling
hook is thrown, to a maximum DC of 20 at 50 feet). Failure by 4 or less means the grappling hook
initially holds, but comes loose after 1d4 rounds of supporting
weight. Failure by 6 or more means the hook fails to catch and falls, allowing you to
try again.  Your DM should make this check secretly, so that you don't
know whether the rope will hold your weight.
\subsk{Bind a Character} When you bind another character with a rope,
any Escape Artist check that the bound
character makes is opposed by your Use
Rope check. You get a \plus10 bonus on this
check because it is easier to bind
someone than to escape from bonds.
You don't even make your Use Rope
check until someone tries to escape.
\parhead{Action} Varies. Throwing a grappling
hook is a standard action that provokes an
attack of opportunity. Tying a knot, tying a
special knot, or tying a rope around
yourself one-handed is a full-round
action that provokes an attack of
opportunity. Splicing two ropes together
takes 5 minutes. Binding a character takes 1
minute.
\parhead{Special} A silk rope (\pref{Tools and Skill Kits}) gives you a \plus2 circumstance bonus on Use Rope checks. If you cast an  \spell{animate rope} spell on a rope, you get a \plus5
circumstance bonus on any Use Rope checks
you make when using that rope. These
bonuses stack.

If you have the Deft Hands feat, you get a \plus2 competence bonus
on Use Rope checks.
\end{comment}
