\chapter{Skills}
\section{Skills Summary}
A character's skills describe the myriad of talents that people have.

\parhead{Skill Points} At 1st level, your character gains a certain number of skill points. Skill points represent your training in a particular area. You get a base allotment of 2, 4, 8, or 12 skill points, depending on your character's class. These skill points can be spent on any skills. In addition, you gain additional skill points based on your attributes that can only be spent on skills associated with the relevant attribute. For each attribute, you gain a number of skill points equal to half your attribute score.

\par If your character's attributes increase, she may immediately gain a skill point which can be spent on skills with that key attribute. However, enhancement bonuses, circumstance bonuses, and penalties of any kind do not affect a character's skill points. If she gains a level in a new class that has more skill points than any class she previously had, the character immediately gains skill points equal to the difference between the skill points provided by the two classes. These skill points can be spent on any skill.

\parhead{Spending Skill Points} If you place one skill point in a skill, you become trained in that skill. If you place two points in a skill, you become an expert in that skill. Your level of training in a skill determines how many skill ranks you have in that skill. Some skills can only be used if you are trained in them.

\parhead{Skill Ranks}
Skill ranks represent how capable your character is with a particular skill. Your character automatically gains skill ranks as she increases in level, as shown by \trefnp{Skill Ranks and Skill Training}.

\begin{dtable}
\lcaption{Skill Ranks and Skill Training}
\begin{tabularx}{\columnwidth}{>{\lcol}p{4.5em} >{\lcol}p{2.5em} >{\lcol}p{6.5em} >{\lcol}X}
\thead{Skill Training Level} & \thead{Skill Points Spent} & \thead{Cross-Class Skill Ranks} & \thead{Class Skill Ranks} \\
Untrained & 0 & \x & \x \\
Trained & 1 & 1/2 character level & 1/2 character level \add 2 \\
Expert & 2 & 1/2 character level \add 2 & Associated class levels \add 1/2 unassociated class levels \add 3 \\
\end{tabularx}
\end{dtable}

\parhead{Using Skills} To make a skill check, roll 1d20 \add skill rank \add key attribute \add bonuses and penalties.

\subparhead{Key Attribute} The attribute used in a skill check is noted in its description.

\subparhead{Bonuses and Penalties} Miscellaneous modifiers include racial bonuses, armor check penalties, bonuses provided by feats, and more.

\section{Using Skills}
When your character uses a skill, you make a skill check to see how well he or she does. The higher the result of the skill check, the better. Based on the circumstances, your result must match or beat a particular number (a DC or the result of an opposed skill check) for the check to be successful. The harder the task, the higher the number you need to roll.

Circumstances can affect your check. A character who is free to work without distractions can make a careful attempt and avoid simple mistakes. A character who has lots of time can try over and over again, thereby assuring the best outcome. If others help, the character may succeed where otherwise he or she would fail.

\begin{dtable!*}
\lcaption{Skills}
\begin{tabularx}{\textwidth}{>{\lcol}p{12em} *{11}{>{\ccol}X} >{\ccol}p{4em} >{\ccol}p{4em}}
\thead{Skill}   & \thead{Bbn} & \thead{Brd} & \thead{Clr} & \thead{Drd} & \thead{Ftr} & \thead{Mnk} & \thead{Pal} & \thead{Rgr} & \thead{Rog} & \thead{Sor} & \thead{Wiz} & \thead{Untrained} & \thead{Key Ability} \\
Climb           & C  & cc & cc & cc & C  & C  & cc & C  & C  & cc & cc & Yes & Str\footnotetemp{1} \\
Jump            & C  & cc & cc & cc & C  & C  & cc & C  & C  & cc & cc & Yes & Str\footnotetemp{1} \\
Swim            & C  & cc & cc & C  & C  & C  & cc & C  & C  & cc & cc & Yes & Str\footnotetemp{2} \\
Acrobatics      & cc & cc & cc & C  & cc & C  & cc & cc & C  & cc & cc & Yes & Dex\footnotetemp{1} \\
Escape Artist   & cc & cc & cc & cc & cc & C  & cc & cc & C  & cc & cc & Yes & Dex\footnotetemp{1} \\
Ride            & cc & cc & cc & cc & C  & cc & C  & cc & cc & cc & cc & Yes & Dex \\
Sleight of Hand & cc & C  & cc & cc & cc & cc & cc & cc & C  & cc & cc & No & Dex\footnotetemp{1} \\
Stealth         & cc & C  & cc & cc & cc & C  & cc & C  & C  & cc & cc & Yes & Dex\footnotetemp{1} \\
Craft           & cc & cc & cc & cc & cc & cc & cc & cc & cc & cc & cc & Yes & Int \\
Devices         & cc & cc & cc & cc & cc & cc & cc & cc & C  & cc & cc & No & Int \\
Disguise        & cc & C  & cc & cc & cc & cc & cc & cc & C  & cc & cc & Yes & Int \\
Knowledge (arcana) & cc & C  & C  & cc & cc & C  & cc & cc & cc & C  & C  & No & Int \\
Knowledge (dungeoneering) & cc & C  & cc & cc & cc & cc & cc & C  & C  & cc & C  & No & Int \\
Knowledge (engineering) & cc & C  & cc & cc & cc & cc & cc & cc & cc & cc & C  & No & Int \\
Knowledge (geography) & cc & C  & cc & cc & cc & cc & cc & C  & cc & cc & C  & No & Int \\
Knowledge (local) & cc & C  & cc & cc & cc & cc & cc & cc & C  & cc & C  & No & Int \\
Knowledge (nature) & cc & C  & cc & C  & cc & cc & cc & C  & cc & cc & C  & No & Int \\
Knowledge (planes) & cc & C  & C  & cc & cc & cc & cc & cc & cc & C  & C  & No & Int \\
Knowledge (religion) & cc & C  & C  & cc & cc & C  & C  & cc & cc & cc & C  & No & Int \\
Linguistics     & cc & C  & cc & cc & cc & cc & cc & cc & C  & cc & C  & No & Int \\
\thead{Skill}   & \thead{Bbn} & \thead{Brd} & \thead{Clr} & \thead{Drd} & \thead{Ftr} & \thead{Mnk} & \thead{Pal} & \thead{Rgr} & \thead{Rog} & \thead{Sor} & \thead{Wiz} & \thead{Untrained} & \thead{Key Ability} \\
Heal            & cc & cc & C  & C  & cc & cc & C  & C  & cc & cc & cc & Yes & Wis \\
Perception      & C  & C  & cc & C  & cc & C  & cc & C  & C  & cc & cc & Yes & Wis \\
Profession      & cc & cc & cc & cc & cc & cc & cc & cc & cc & cc & cc & No & Wis\fn{3} \\
Sense Motive    & cc & C  & C  & cc & cc & C  & C  & cc & C  & cc & cc & Yes & Wis \\
Spellcraft      & cc & C  & C  & C  & cc & cc & cc & cc & cc & C  & C  & No & Wis \\
Survival        & C  & cc & cc & C  & cc & cc & cc & C  & cc & cc & cc & Yes & Wis \\
Bluff           & cc & C  & cc & cc & cc & cc & cc & cc & C  & cc & cc & Yes & Cha \\
Creature Handling   & C  & cc & cc & C  & cc & cc & C  & C  & cc & cc & cc & No & Cha \\
Intimidate      & C  & cc & cc & cc & C  & cc & cc & cc & C  & C  & cc & Yes & Cha \\
Perform         & cc & C  & cc & cc & cc & C  & cc & cc & C  & cc & cc & Yes & Cha \\
Persuasion      & cc & C  & C  & C  & cc & C  & C  & cc & C  & cc & cc & Yes & Cha \\
\end{tabularx}
1. Armor check penalty applies \\
2. Double armor check penalty applies \\
3. Varies depending on profession
\end{dtable!*}

\subsection{Skill Checks}
A skill check takes into account a character's training (skill rank), natural talent (attribute), and luck (the die roll). It may also take into account his or her race's knack for doing certain things (racial bonus) or what armor he or she is wearing (armor check penalty), or a certain feat the character possesses, among other things.

To make a skill check, roll 1d20 and add your character's skill modifier for that skill. The skill modifier incorporates the character's ranks in that skill and the attribute for that skill's key attribute, plus any other miscellaneous modifiers that may apply, including racial bonuses and armor check penalties. The higher the result, the better. Unlike with attack rolls and saving throws, a natural roll of 20 on the d20 is not an overwhelming success, and a natural roll of 1 is not an overwhelming failure.

\subsubsection{Difficulty Class}
Some checks are made against a Difficulty Class (DC). The DC is a number (set using the skill rules as a guideline) that you must score as a result on your skill check in order to succeed.

\begin{dtable}
\lcaption{Difficulty Class Examples}
\begin{tabularx}{\columnwidth}{p{8em} X}
Difficulty (DC) & Example (Skill Used) \\
Very easy (0) & Notice something large in plain sight (Perception) \\
Easy (5) & Hear a conversation from 50 feet away (Perception) \\
Average (10) & Palm a coin-sided object (Sleight of Hand) \\
Tough (15) & Rig a wagon wheel to fall off (Disable Device) \\
Challenging (20) & Swim in stormy water (Swim) \\
Formidable (25) & Climb a natural rock wall with no equipment (Climb) \\
Heroic (30) & Leap across a 30-foot chasm with a running start (Jump) \\
Nearly impossible (40) & Track a squad of orcs across hard ground after 24 hours of rainfall (Survival) \\
\end{tabularx}
\end{dtable}

\subsubsection{Opposed Checks}
An opposed check is a check whose success or failure is determined by comparing the check result to another character's check result. In an opposed check, the higher result succeeds, while the lower result fails. In case of a tie, the higher skill modifier wins. If these scores are the same, roll again to break the tie.

\begin{dtable}
\lcaption{Example Opposed Checks}
\begin{tabularx}{\columnwidth}{*{3}{>{\lcol}X}}
\thead{Task} & \thead{Skill (Key Ability)} & \thead{Opposing Skill (Key Ability)} \\
Lie & Bluff (Cha) & Sense Motive (Wis) \\
Create a false map & Craft (Int) & Craft (Int) or Perception (Wis) \\
Make a bully back down & Intimidate (Cha) & Special\footnotetemp{1} \\
Make someone look like someone look like someone else & Disguise (Int) & Perception (Wis) \\
Sneak up on someone & Stealth (Dex) & Perception (Wis) \\
Steal a coin pouch & Sleight of Hand (Dex) & Perception (Wis) \\
Tie a prisoner securely & Devices (Int)\fn{2} & Escape Artist (Dex) \\
\end{tabularx}
1 An Intimidate check is opposed by the target's Will save, not a skill check. See the Intimidate skill description for more information. \\
2 You can also tie a prisoner with a grapple attack. See \pcref{Grapple}. \\
\end{dtable}

\subsubsection{Trying Again}
In general, you can try a skill check again if you fail, and you can keep trying indefinitely. Some skills, however, have consequences of failure that must be taken into account. A few skills are virtually useless once a check has failed on an attempt to accomplish a particular task. For most skills, when a character has succeeded once at a given task, additional successes are meaningless.

\subsubsection{Untrained Skill Checks}
Generally, if your character attempts to use a skill he or she does not possess, you make a skill check as normal. The skill modifier doesn't have a skill rank added in because the character has no ranks in the skill. Any other applicable modifiers, such as the modifier for the skill's key attribute, are applied to the check.

Many skills can be used only by someone who is trained in them.

\subsubsection{Favorable and Unfavorable Conditions}
Some situations may make a skill easier or harder to use, resulting in a bonus or penalty to the skill modifier for a skill check or a change to the DC of the skill check.

The chance of success can be altered in four ways to take into account exceptional circumstances.
\begin{enumerate*}
\item Give the skill user a \plus2 circumstance bonus to represent conditions that improve performance, such as having the perfect tool for the job, getting help from another character (see Combining Skill Attempts), or possessing unusually accurate information.
\item Give the skill user a \minus2 circumstance penalty to represent conditions that hamper performance, such as being forced to use improvised tools or having misleading information.
\item Reduce the DC by 2 to represent circumstances that make the task easier, such as having a friendly audience or doing work that can be subpar.
\item  Increase the DC by 2 to represent circumstances that make the task harder, such as having an uncooperative audience or doing work that must be flawless.
\end{enumerate*}

Conditions that affect your character's ability to perform the skill change the skill modifier. Conditions that modify how well the character has to perform the skill to succeed change the DC. A bonus to the skill modifier and a reduction in the check's DC have the same result: They create a better chance of success. But they represent different circumstances, and sometimes that difference is important.

\subsubsection{Time and Skill Checks}
Using a skill might take a round, take no time, or take several rounds or even longer. Most skill uses are standard actions, move actions, or full-round actions. Types of actions define how long activities take to perform within the framework of a combat round (6 seconds) and how movement is treated with respect to the activity. Some skill checks are instant and represent reactions to an event, or are included as part of an action. These skill checks are not actions. Other skill checks represent part of movement.

 \subsubsection{Checks without Rolls}
A skill check represents an attempt to accomplish some goal, usually while under some sort of time pressure or distraction. Sometimes, though, a character can use a skill under more favorable conditions and eliminate the luck factor.
 \parhead{Taking 10} When your character is not being threatened or distracted,
and the skill check has no penalties for failure, you can't roll less than a 10. If you roll a d20 for your skill check, and you roll less than a 10, calculate your result as if you had rolled a 10. For many routine tasks, taking 10 makes them automatically successful, and you may not even roll at all.

Distractions or threats (such as combat) make it impossible for a character to take 10. Taking 10 is especially useful in situations where a particularly high roll wouldn't help.
\parhead{Taking 20} When you have plenty of time (generally 2 minutes for a skill that can normally be checked in 1 round, one full-round action, or one standard action), you are faced with no threats or distractions, and the skill being attempted carries no penalties for failure, you can take 20. In other words, eventually you will get a 20 on 1d20 if you roll enough times. Instead of rolling 1d20 for the skill check, just calculate your result as if you had rolled a 20.

Taking 20 means you are trying until you get it right, and it assumes that you fail many times before succeeding. Taking 20 takes twenty times as long as making a single check would take. If the check takes a variable amount of time, assume it took the average amount of time required to make a check.

Since taking 20 assumes that the character will fail many times before succeeding, if you did attempt to take 20 on a skill that carries penalties for failure, your character would automatically incur those penalties before he or she could complete the task. Common ``take 20'' skills include Escape Artist, Open Lock, and Search.

\parhead{Ability Checks and Caster Level Checks} The normal take 10 and take 20 rules apply for ability checks. Neither rule applies to caster level checks.
\parhead{Special Abilities} Some special abilities grant the ability to take 5, take 10, or some other number on specific checks or even attacks. This follows the same rules as taking 10, except that the character can typically use such abilities even while threatened or distracted.

\subsection{Group Skill Checks}
When multiple characters are trying to use the same skill simultaneously, they may be able to work together. Most skills can be done as a group skill check, where the more skilled characters assist the less skilled characters and work together to find the best result. There are two kinds of group skill checks.

\parhead{Collaborative Checks} When making a collaborative skill check, each member of the group gets their own result. Collaborative checks might include a group of scouts sneaking up on an enemy camp, a group of adventurers climbing a cliff, or a group of spies lying about their true allegiances.

When making a collaborative check, the group must choose a leader before making the check. Each member of the group can make the check using the higher of their skill ranks and half the leader's skill ranks. Other modifiers apply normally.

The group leader must remain in a position to help the other members of the group for the group members to gain this benefit. For example, if a group is making a collaborative check to leap across a chasm, the leader must jump last. If the group is making a collaborative Stealth check, the leader must remain adjacent to the other members of the group to correct their mistakes quietly. The exact circumstances depend on the situation. 

\parhead{Collective Checks} The group works together to get a single result. Collective checks might include a group of diplomats persuading a noble to go to war, a group of medics tending to a dying warrior, or a group of scouts keeping watch for enemy attacks.

When making a collective check, the group simply uses the highest result from any character making the check.

\parhead{Making Group Skill Checks} A group skill check is not always possible. If the whole group is leaping over a chasm at once to escape a dragon, there is not enough time to designate a leader and have them aid the others to make a collaborative check. In general, a group skill check is only possible if the group has at least five rounds to work together, though this time may vary widely depending on the situation and the skill being used. 

\subsection{Ability Checks}
Sometimes a character tries to do something to which no specific skill really applies. In these cases, you make an ability check. An ability check is a roll of 1d20 plus the appropriate attribute. Essentially, you're making an untrained skill check.

In some cases, an action is a straight test of one's ability with no luck involved. Just as you wouldn't make a height check to see who is taller, you don't make a Strength check to see who is stronger.

\section{Skill Descriptions}
This section describes each skill, including common uses and typical modifiers. Characters can sometimes use skills for purposes other than those noted here.

Here is the format for skill descriptions.

\subsection*{Skill Name}
The skill name line includes (in addition to the name of the skill) the following information.
\parhead{Key Ability} The abbreviation of the ability whose modifier applies to the skill check.
\parhead{Trained Only} If this notation is included in the skill name line, you must have at least 1 rank in the skill to use it. If it is omitted, the skill can be used untrained (with a rank of 0). If any special notes apply to trained or untrained use, they are covered in the Untrained section (see below).
\parhead{Armor Check Penalty} If this notation is included in the skill name line, an armor check penalty applies (when appropriate) to checks using this skill. If this entry is absent, an armor check penalty does not apply.

\par The skill name line is followed by a general description of what using the skill represents. After the description are a few other types of information:

\parhead{Check} What a character (``you'' in the skill description) can do with a successful skill check, and the check's DC.

\parhead{Action} The type of action using the skill requires, or the amount of time required for a check.

\parhead{Try Again} Any conditions that apply to successive attempts to use the skill successfully. If the skill doesn't allow you to attempt the same task more than once, or if failure carries an inherent penalty (such as with the Climb skill), you can't take 20. If this paragraph is omitted, the skill can be retried without any inherent penalty, other than the additional time required.

\parhead{Special} Any extra facts that apply to the skill, such as special effects deriving from its use or bonuses that certain characters receive because of class, feat choices, or race.

\parhead{Restriction} The full utility of certain skills is restricted to characters of certain classes or characters who possess certain feats. This entry indicates whether any such restrictions exist for the skill.

\parhead{Untrained} This entry indicates what a character without at least 1 skill point in the skill can do with it. If this entry doesn't appear, it means that the skill functions normally for untrained characters (if it can be used untrained) or that an untrained character can't attempt checks with this skill (for skills that are designated as ``Trained Only'').

\subsection{Acrobatics (Dex; Armor Check Penalty)}
The Acrobatics skill can be used for two purposes: balancing and tumbling. These are not different skills, but merely different applications of the same skill.
\subsubsection{Balance}
\parhead{Check} You can walk on a precarious surface. A successful check lets you move at half your speed along the surface for 1 round. A failure by 4 or less means your action is wasted, and you don't move. A failure by 5 or more means you fall. The difficulty varies with the surface, as follows.

\begin{itemize*}
  \item Uneven floor (flagstones, sloped floor): DC 10. This only applise when running or charging. Failure means you lose a move action (and possibly fall), but you can still take a standard action.
  \item One foot wide (or wider): DC 5. 
  \item Six inches wide: DC 10.
  \item Two inches wide: DC 15.
  \item One inch wide: DC 20. Halving the width of the surface further increases the DC by 5 each time.
\end{itemize*}

\subsk{Being Attacked while Balancing} You are considered flat-footed while balancing, since you can't move to avoid a blow. If you take damage or are forced to move while balancing, you must make another Acrobatics check against the same DC to remain standing.

\subsubsection{Tumble}
You can only attempt to tumble if you are trained in Acrobatics.
\parhead{Check} You can tumble past opponents in combat to reduce your odds of getting hit. You can tumble at one-half your normal speed as part of normal movement. If you do, make an Acrobatics check. You may treat your check result as your Armor Class against attacks of opportunity provoked by the movement.

You can also try to tumble to reduce falling damage as an immediate action. A DC 10 check allows you to treat a fall as if it were 10 feet shorter. For every 5 by which you beat that DC, you double the length of the fall that you can mitigate.

\subsubsection{Acrobatics Modifiers}
Obstructed or otherwise treacherous surfaces, such as natural cavern floors or undergrowth, are tough to balance or tumble through. The DC for any Acrobatics check in such a square (except checks to mitigate falling damage) is modified as indicated below.
\begin{dtable}
\lcaption{Acrobatics Modifiers}
\begin{tabularx}{\columnwidth}{>{\lcol}X c}
\thead{Surface Is} & \thead{DC Modifier} \\
Lightly obstructed (scree, light rubble, shallow bog, light undergrowth)  & \plus2 \\
Lightly slippery (wet floor)  & \plus2 \\
Sloped or angled  & \plus2 \\
Slightly mobile (rope bridge) & \plus2 \\
Severely obstructed (dense rubble, dense undergrowth)  & \plus5 \\
Severely slippery (ice sheet, oiled floor)  & \plus5 \\
Very mobile (slack rope) & \plus5 \\
\end{tabularx}
\end{dtable}

\subsk{Accelerated Movement} You can try to move more quickly than normal. You take a \minus5 penalty on the Acrobatics check, and failure by any amount on a balance check means you fall. If you succeed, you move at full speed instead of at half speed.

\parhead{Action} Balancing and tumbling are a part of movement, so an Acrobatics check is made as part of a move action. Reducing damage from a fall is an immediate action.

\parhead{Try Again} Varies. You can continue trying to balance on a precarious surface as long as you remain on the surface. You can try to reduce damage from a fall only once per fall.


\subsection{Athletics (Str; Armor Check Penalty)}
Athletics includes running, jumping, and general athleticism.

\subsubsection{Jump}

A Jump check allows you to jump. The DC depends on the type of jump you are attempting (see below).

Your Jump check is modified by your speed. If your speed is 30 feet then no modifier based on speed applies to the check. If your speed is less than 30 feet, you take a \minus6 penalty for every 10 feet of speed less than 30 feet. If your speed is greater than 30 feet, you gain a \plus4 bonus for every 10 feet beyond 30 feet.

All Jump DCs given here assume that you get a running start, which requires that you move at least 20 feet in a straight line before attempting the jump. If you do not get a running start, the DC for the jump is doubled.

Distance moved by jumping is counted against your normal maximum movement in a round.

If you have ranks in Jump and you succeed on a Jump check, you land on your feet (when appropriate). If you attempt a Jump check untrained, you land prone unless you beat the DC by 5 or more.

\subsk{Long Jump} A long jump is a horizontal jump, made across a gap like a chasm or stream. At the midpoint of the jump, you attain a vertical height equal to one-quarter of the horizontal distance. The DC for the jump is equal to the distance jumped (in feet).

If your check succeeds, you land on your feet at the far end. If you fail the check by less than 5, you don't clear the distance, but you can make a DC 15 Reflex save to grab the far edge of the gap. You end your movement grasping the far edge. If that leaves you dangling over a chasm or gap, getting up requires a move action and a DC 15 Climb check.

\begin{dtable}
\begin{tabularx}{\columnwidth}{>{\lcol}X >{\lcol}X}
\thead{Long Jump Distance} & \thead{Jump DC\footnotetemp{1}} \\
5 feet & 5 \\
10 feet & 10 \\
15 feet & 15 \\
20 feet & 20 \\
25 feet & 25 \\
30 feet & 30 \\
\end{tabularx}
1 Requires a 20-foot running start. Without a running start, double the DC.
\end{dtable}

\subsk{High Jump} A high jump is a vertical leap made to reach a ledge high above or to grasp something overhead. The DC is equal to 4 times the distance to be cleared.

If you jumped up to grab something, a successful check indicates that you reached the desired height. If you wish to pull yourself up, you can do so with a move action and a DC 15 Climb check. If you fail the Jump check, you do not reach the height, and you land on your feet in the same spot from which you jumped. As with a long jump, the DC is doubled if you do not get a running start of at least 20 feet.

\begin{dtable}
\begin{tabularx}{\columnwidth}{>{\lcol}X >{\lcol}X}
\thead{High Jump Distance\footnotetemp{1}}  & \thead{Jump DC\footnotetemp{2}} \\
1 foot  & 4 \\
2 feet  & 8 \\
3 feet  & 12 \\
4 feet  & 16 \\
5 feet  & 20 \\
6 feet  & 24 \\
7 feet  & 28 \\
8 feet  & 32 \\
\end{tabularx}
1 Not including vertical reach; see below.
2 Requires a 20-foot running start. Without a running start, double the DC.
\end{dtable}

Obviously, the difficulty of reaching a given height varies according to the size of the character or creature. The maximum vertical reach (height the creature can reach without jumping) for an average creature of a given size is shown on the table below. (As a Medium creature, a typical human can reach 8 feet without jumping.) The maximum vertical reach for an individual creature is usually given by the 1 and 1/2 times the creature's height.

Quadrupedal creatures don't have the same vertical reach as a bipedal creature; treat them as being one size category smaller.

\begin{dtable}
\begin{tabularx}{\columnwidth}{>{\lcol}X >{\lcol}X}
    \thead{Creature Size}  & \thead{Vertical Reach} \\
Colossal  & 128 ft. \\
Gargantuan  & 64 ft. \\
Huge  & 32 ft. \\
Large  & 16 ft. \\
Medium  & 8 ft. \\
Small  & 4 ft. \\
Tiny  & 2 ft. \\
Diminutive  & 1 ft. \\
Fine  & 1/2 ft.
\end{tabularx}
\end{dtable}

\subsk{Hop Up} You can jump up onto an object as tall as your waist, such as a table or small boulder, with a DC 10 Jump check. Doing so counts as 10 feet of movement, so if your speed is 30 feet, you could move 20 feet, then hop up onto a counter. You do not need to get a running start to hop up, so the DC is not doubled if you do not get a running start.

\subsk{Jumping Down} If you intentionally jump from a height, you take less damage than you would if you just fell. The DC to jump down from a height is 15. You do not have to get a running start to jump down, so the DC is not doubled if you do not get a running start.

If you succeed on the check, you take falling damage as if you had dropped 10 fewer feet than you actually did.

\parhead{Action} None. A Jump check is included in your movement, so it is part of a move action. If you run out of movement mid-jump, your next action (either on this turn or, if necessary, on your next turn) must be a move action to complete the jump.

\parhead{Special} Effects that increase your movement also increase your jumping distance, since your check is modified by your speed.

\subsubsection{Sprint}

%If your Sprint check exceeds your Fortitude defense, you are fatigued?
A Sprint check allows you to move faster temporarily. A DC 15 Sprint check allows you to treat your speed as if it were 5 feet faster for 1 round. For every 5 points by which you beat that DC, you increase your speed by an additional 5 feet. After sprinting, you are fatigued for 1 round, making you vulnerable. You take a \minus2 penalty to attack rolls, saving throws, checks, DCs, and AC.

You can sprint in any movement mode that you can use.

\parhead{Action} None. A Sprint check is included in your movement, so it is part of a move action.



\subsection{Bluff (Cha)}
You can use Bluff to convince people that you are telling the truth. It is usually used when you are lying.

\parhead{Check} A Bluff check is opposed by the target's Sense Motive check. See the accompanying tables for examples of different kinds of bluffs and the modifier to the target's Sense Motive check for each one. A successful Bluff check indicates that the target believes that you are telling the truth. A failed Bluff check usually indicates that the target is unconvinced. Failure by 5 or more means that they realize that you were intentionally lying.

A bluff requires interaction between you and the target. Creatures unaware of you cannot be bluffed.

\begin{dtable}
  \lcaption{Believability Modifiers}
  \begin{tabularx}{\columnwidth}{X l}
    \thead{Description} & \thead{Sense Motive Modifier}  \\
    Lie is  (``Nothing interesting happened while I was on patrol'') & \minus5 \\
    Plausible lie (``The mayor is too busy to see you now.'') & \plus0 \\
    Unlikely lie (``The north gate is under attack!'') & \plus5 \\
    Extremely unlikely lie (``The mayor is secretly a vampire.'') & \plus10 \\
    Virtually impossible lie (``You are secretly a vampire.'') & \plus20 \\
    Demonstratably untrue lie (``You are a frog.'') & \x\fn{1} \\
  \end{tabularx}
  1 A lie that is proven false can't work.
\end{dtable}

\begin{dtable}
  \lcaption{Motivation Modifiers}
  \begin{tabularx}{\columnwidth}{X l}
    \thead{Description} & \thead{Sense Motive Modifier} \\
    Target wants to believe the lie (``That dress looks lovely on you.'') & \minus5 \\
    Target does not have strong feelings about the lie (``I'm busy.'') & \plus0 \\
    Target doesn't want the lie to be true (``Your son is dead.'') & \plus5 \\
  \end{tabularx}
\end{dtable}
  
\subsk{Creating a Diversion to Hide} You can use the Bluff skill to help you hide. A successful Bluff check gives you the momentary diversion you need to attempt a Hide check while people are aware of you. This usage does not provoke an attack of opportunity.

\subsk{Delivering a Secret Message} You can use Bluff to get a message across to another character without others understanding it. The DC is 15 for simple messages, or 20 for complex messages, especially those that rely on getting across new information. Failure by 4 or less means you can't get the message across. Failure by 5 or more means that some false information has been implied or inferred. Anyone listening to the exchange can make a Sense Motive check opposed by the Bluff check you made to transmit in order to intercept your message (see \pcref{Sense Motive [Wis]}).

\parhead{Action} Varies. A Bluff check made as part of general interaction always takes at least 1 round (and is at least a full-round action), but it can take much longer if you try something elaborate. A Bluff check made to create a diversion to hide is a standard action. A Bluff check made to deliver a secret message doesn't take an action; it is part of normal communication.

\parhead{Try Again} Varies. Generally, a failed Bluff check in social interaction makes the target too suspicious for you to try again in the same circumstances. Retries are also allowed when you are trying to send a message, but you may attempt such a retry only once per round.

\subsection{Climb (Str; Armor Check Penalty)}

\parhead{Check} With a successful Climb check, you can advance up, down, or across a slope, a wall, or some other steep incline (or even a ceiling with handholds). The distance that you move is equal to the size of your space, as described on the table below.

\begin{dtable}
\lcaption{Climb Speeds}
\begin{tabularx}{\columnwidth}{l X l X}
  \thead{Size} & \thead{Speed} & \thead{Size} & \thead{Speed} \\
  Medium & 5 ft. && \\
  Small & 5 ft. & Large & 10 ft. \\
  Tiny & 2-1/2 ft. & Huge & 15 ft. \\
  Diminuitive & 1 ft. & Gargantuan & 20 ft. \\
  Fine & 1/2 ft. & Colossal & 40 ft. \\
\end{tabularx}
\end{dtable}

The DC of the check depends on the difficulty of the task and the conditions of the climbing surface. Consult the following tables to determine an appropriate DC.

\begin{dtable}
\lcaption{Climb DC Examples}
\begin{tabularx}{\columnwidth}{l X}
\thead{Climb DC} & \thead{Example Surface or Activity} \\
0 & A slope too steep to walk up, or a knotted rope with a wall
to brace against. \\
5 & A rope with a wall to brace against, or a knotted rope, or a
rope affected by the \spell{rope trick} spell.  \\
10 & A surface with ledges to hold on to and stand on, such as a
very rough wall or a ship's rigging. \\
15 & Any surface with adequate handholds and footholds
(natural or artificial), such as a very rough natural rock
surface or a tree, or a knotted rope, or pulling yourself
up when dangling by your hands. \\
20 & An uneven surface with limited handholds and
footholds, such an unknotted rope or a typical wall in a ruins. \\
25 & A rough surface, such as a natural rock wall or a brick wall. \\
30 & An uneven overhang or ceiling with limited handholds and footholds, such as a typical ceiling in a ruins. \\
35 & A rough overhand or ceiling, such as a natural rock ceiling \\
\x & A perfectly smooth, flat, surface cannot be climbed. \\
\end{tabularx}
\end{dtable}

\begin{dtable}
\lcaption{Climb Modifiers}
\begin{tabularx}{\columnwidth}{l X}
\thead{Climb DC Modifier\footnotetemp{1}} & \thead{Example Surface or Activity} \\
\minus5 & Inclined surface (between 45 and 60 degrees) \\
\minus5 & Climbing a chimney (artificial or natural) or other location
where you can brace against two opposite walls \\
\minus2 & Climbing a corner where you can brace against
perpendicular walls \\
\plus2 & Surface is slightly slippery, such as damp stone \\
\plus5 & Surface is very slippery, such as ice or oil-covered stone 
\end{tabularx}
1 These modifiers are cumulative; use any that apply.
\end{dtable}

A Climb check that fails by 4 or less means that you make no progress, and one that fails by 5 or more means that you fall from whatever height you have already attained. You need both hands free to climb, but you may cling to a wall with one hand while you cast a spell or take some other action that requires only one hand. While climbing, you can't move to avoid a blow, so you are flat-footed.

Any time you take damage while climbing, make a Climb check against the DC of the slope or wall. Failure means you fall from your current height and sustain the appropriate falling damage.

\subsk{Accelerated Climbing} You can try to climb more quickly than normal. You take a \minus5 penalty on the Climb check, and failure by any amount means you fall. If you succeed, you move twice as far as normal.

\subsk{Catching Yourself When Falling} It's practically impossible to catch yourself on a wall while falling. Make a Climb check (DC = wall's DC \add 20) to do so. It's much easier to catch yourself on a slope (DC = slope's DC \add 10).

\subsk{Catching a Falling Character While Climbing} If someone climbing above you or adjacent to you falls, you can attempt to catch the falling character if he or she is within your reach. Doing so requires a successful melee touch attack against the falling character (though he or she can voluntarily act flat-footed if desired). If you hit, you must immediately attempt a Climb check (DC = wall's DC \add 10). Success indicates that you catch the falling character, but his or her total weight, including equipment, cannot exceed your heavy load limit or you automatically fall. If you fail your Climb check by 4 or less, you fail to stop the character's fall but don't lose your grip on the wall. If you fail by 5 or more, you fail to stop the character's fall and begin falling as well.

\parhead{Action} Climbing is a kind of movement, so it's generally a move action. Each move action that includes any climbing requires a separate Climb check. Catching yourself or another falling character doesn't take an action.

\parhead{Special} You can use a rope to haul a character upward (or lower a character) through sheer strength. You can lift double your maximum load in this manner.

\par A creature with a climb speed has a \plus5 inherent bonus on all Climb checks. It can always choose to take 10 on Climb checks, even if rushed or threatened, and is not flat-footed while climbing. It cannot make an accelerated climb or use the run action while climbing.

\subsection{Craft (Int)}
Like Knowledge, Perform, and Profession, Craft is actually a number of separate skills. You could have several Craft skills, each with its own ranks, each purchased as a separate skill. Common Craft skills are listed below, with additional description for some skills.

\begin{itemize*}
  \item Alchemy (Alchemist's fire, tanglefoot bags, potions)
  \item Ceramics (Glass, pottery)
  \item Jewelry (gemcutting, amulets, rings)
  \item Leather
  \item Manuscripts (Books, official documents, scrolls)
  \item Metal
  \item Stone
  \item Textiles (Cloth, fabric)
  \item Traps
  \item Wood
\end{itemize*}

A Craft skill is specifically focused on physical objects. If nothing can be created by an endeavor, it probably falls under the heading of a Profession skill. Complex structures, such as buildings or siege engines, may require Knowledge (engineering) in addition to an appropriate Craft skill.

\parhead{Check} Craft can be used for several purposes. The most important check is to make an item of the appropriate type. The DC depends on the complexity of the item to be created. Your check result and the price of the item determine how long it takes to make a particular item. The item's finished price also determines the cost of raw materials.

All crafts require artisan's tools to give the best chance of success. If improvised tools are used, the check may be made with a penalty, or may be impossible, depending on the tools available and the item to be crafted. For example, crafting a bow with improvised woodworking tools would impose a \minus2 penalty, but cutting a diamond without specialized tools is impossible.

\begin{comment}
To determine the time required to craft an item, consult the table below.
\begin{dtable*}
  \lcaption{Crafting Time}
  \begin{tabularx}{\columnwidth}{l X}
    \thead{Item Price} & \thead{Crafting Time} \\
    1gp or less & 1 hour per sp (minimum 1 hour) \\
    10gp or less & 1 day per 2 gp (minimum 1 day) \\
    100gp or less & 5 days \add 1 day per 4 gp \\
    1000gp or less & 30 days \add 1 day per 100 gp \\
  \end{tabularx}
\end{dtable*}
\end{comment}

Each item takes a certain amount of working time to craft, as shown on \tref{Crafting Time}, and the expenditure of one quarter of the item's price in raw materials. In order to craft an item, you must make a Craft check against the item's Craft DC. If you succeed, you make progress on the item based on how long you spend crafting. For every 5 points by which you beat the Craft check, you accomplish twice as much work in the same amount of time. Once your total effective working time exceeds the time required to craft the item, you have finished the item.

If you fail a check by 4 or less, you make no progress that day. If you fail by 5 or more, you ruin the raw materials and have to pay the cost again.

Note that raw materials for some items, particularly alchemical items, may be hard to come by in some areas. A typical day's work consists of 8 hours of work.

\subsk{Repairing Items} Generally, you can repair an item by making checks against the same DC that it took to make the item in the first place. The cost of repairing an item is one-tenth of the item's price.

\subsk{Other Tasks} You can also use the Craft skill for other tasks related to the object of your craft. For example, you could use Craft (woodworking) to sabotage a wagon so it will break at a later time, or Craft (alchemy) to purify spoiled food or ingredients. The DCs of such tasks, and what can be accomplished, can vary widely. In general, the more difficult the task, and the more loosely it is related to the skill, the higher the DC. 

\subsk{Craft DCs} When you use the Craft skill to make a particular sort of item, the DC for checks involving the creation of that item are typically as given on the following table.

\begin{dtable}
\begin{tabularx}{\columnwidth}{>{\lcol}X l >{\lcol}p{4em}}
\thead{Item} & \thead{Craft Skill} & \thead{Craft DC} \\
Acid & Alchemy\footnotetemp{1} & 15 \\
Alchemist's fire, smokestick, or tindertwig & Alchemy & 20 \\
Antitoxin, sunrod, tanglefoot bag, or thunderstone & Alchemy & 25 \\
Armor or shield & Metalsmithing or woodworking & 10 \add AC bonus \\
Longbow or shortbow & Woodworking & 12 \\
Composite longbow or composite shortbow & Woodworking & 15 \\
Composite longbow or composite shortbow with high strength rating & Woodworking & 15 \add  (2 \mtimes rating) \\
Crossbow & Weaponsmithing & 15 \\
Simple melee or thrown weapon & Metalsmithing or woodworking & 12 \\
Martial melee or thrown weapon & Metalsmithing or woodworking & 15 \\
Exotic melee or thrown weapon & Metalsmithing or woodworking & 18 \\
Mechanical trap & Trapmaking & Varies\footnotetemp{1} \\
Very simple item (wooden spoon) & Varies & 5 \\
Typical item (iron pot) & Varies & 10 \\
High-quality item (bell, average lock) & Varies & 15 \\
Complex or superior item (fine china, document with official seal)  & Varies & 20\plus \\
\end{tabularx}
1 Traps have their own rules for construction.
\end{dtable}

\subsk{Forgeries} You can use the Craft skill to make false or defective versions of objects. For example, you may wish to forge official-looking documents with Craft (manuscripts), or you may need to make items in a rush to satisfy an employer. 

Forged items have a Craft DC 5 lower than the real thing, and require only half as many costly material components (allowing you to craft them twice as quickly). However, functional items are always defective in some way. A forgery can be detected by examining the item and making a Craft check, or by making a Perception check with a \minus10 penalty. The DC to identify a forgery is equal to the Craft check used to make the item.

\subsk{Magic Items} For every 5 ranks that you have in each Craft skill, choose a subschool of magic, as described in Chapter 10: Magic. If you have the Imbue Magic feat, you can craft items from that subschool of magic using that Craft skill. The Versatile Crafter feat increases the number of subschools you learn in each Craft skill to one per two ranks.

\parhead{Action} Does not apply. Craft checks are made by the day (see above).

\parhead{Try Again} Yes, but each time you miss by 5 or more, you ruin half the raw materials and have to pay half the original raw material cost again.

\subsection{Creature Handling (Cha; Trained Only)}
You can handle creatures, teaching them tricks and training them to follow instructions. This skill can only be used with creatures with an Intelligence of \minus5 or lower.
\parhead{Check} The DC depends on what you are trying to do.
\begin{dtable}
\begin{tabularx}{\columnwidth}{>{\lcol}X r}
\thead{Task}  & \thead{Creature Handling DC} \\
Handle a creature  & 10 \\
``Push'' a creature  & 25 \\
Teach a creature a trick  & 15 or 20\footnotetemp{1} \\
Train a creature for a general purpose  & 15 or 20\footnotetemp{1} \\
Rear a wild creature & 15 \add HD of creature \\
\end{tabularx}
1 See the specific trick or purpose below.
\end{dtable}
\begin{dtable}
\begin{tabularx}{\columnwidth}{>{\lcol}X c >{\lcol}X c}
\thead{General Purpose}  & \thead{DC}  & \thead{General Purpose}  & \thead{DC} \\
Combat riding  & 20  & Hunting  & 20 \\
Fighting  & 20  & Performance  & 15 \\
Guarding  & 20  & Riding  & 15 \\
Heavy labor & 15 & &
\end{tabularx}
\end{dtable}

\subsk{Handle a Creature} This task involves commanding a creature to perform a task or trick that it knows. If your check succeeds, the creature performs the task or trick on its next action.

\subsk{''Push'' a Creature} To push a creature means to get it to perform a task or trick that it doesn't know but is physically capable of performing. This category also covers making a creature perform a forced march or forcing it to hustle for more than 1 hour between sleep cycles. If your check succeeds, the creature performs the task or trick on its next action.

\subsk{Teach a Creature a Trick} You can teach a creature a specific trick with one week of work and a successful Creature Handling check against the indicated DC. A creature with an Intelligence score of 1 can learn a maximum of three tricks, while a creature with an Intelligence score of 2 can learn a maximum of six tricks. Possible tricks (and their associated DCs) include, but are not necessarily limited to, the following.

\par Attack (DC 20): The creature attacks apparent enemies. You may point to a particular creature that you wish the creature to attack, and it will comply if able. Normally, a creature will attack only humanoids, monstrous humanoids, giants, or other creatures. Teaching a creature to attack all creatures (including such unnatural creatures as undead and aberrations) counts as two tricks.
\par Come (DC 15): The creature comes to you, even if it normally would not do so.
\par Defend (DC 20): The creature defends you (or is ready to defend you if no threat is present), even without any command being given. Alternatively, you can command the creature to defend a specific other character.
\par Down (DC 15): The creature breaks off from combat or otherwise backs down. A creature that doesn't know this trick continues to fight until it must flee (due to injury, a fear effect, or the like) or its opponent is defeated.
\par Fetch (DC 15): The creature goes and gets something. If you do not point out a specific item, the creature fetches some random object.
\par Guard (DC 20): The creature stays in place and prevents others from approaching.
\par Heel (DC 15): The creature follows you closely, even to places where it normally wouldn't go.
\par Perform (DC 15): The creature performs a variety of simple tricks, such as sitting up, rolling over, roaring or barking, and so on.
\par Seek (DC 15): The creature moves into an area and looks around for anything that is obviously alive or animate.
\par Stay (DC 15): The creature stays in place, waiting for you to return. It does not challenge other creatures that come by,
though it still defends itself if it needs to.
\par Track (DC 20): The creature tracks the scent presented to it. (This requires the creature to have the scent ability)
\par Work (DC 15): The creature pulls or pushes a medium or heavy load.

\subsk{Train a Creature for a Purpose} Rather than teaching a creature individual tricks, you can simply train it for a general purpose. Essentially, a creature's purpose represents a preselected set of known tricks that fit into a common scheme, such as guarding or heavy labor. The creature must meet all the normal prerequisites for all tricks included in the training package. If the package includes more than three tricks, the creature must have an Intelligence score of \minus5.

A creature can be trained for only one general purpose, though if the creature is capable of learning additional tricks (above and beyond those included in its general purpose), it may do so. Training a creature for a purpose requires fewer checks than teaching individual tricks does, but no less time.

\par Combat Riding (DC 20): A creature trained to bear a rider into combat knows the tricks attack, come, defend, down, guard, and heel. Training a creature for combat riding takes six weeks. You may also ``upgrade'' a creature trained for riding to one trained for combat riding by spending three weeks and making a successful DC 20 Creature Handling check. The new general purpose and tricks completely replace the creature's previous purpose and any tricks it once knew. Warhorses and riding dogs are already trained to bear riders into combat, and they don't require any additional training for this purpose.
\par Fighting (DC 20): A creature trained to engage in combat knows the tricks attack, down, and stay. Training a creature for fighting takes three weeks.
\par Guarding (DC 20): A creature trained to guard knows the tricks attack, defend, down, and guard. Training a creature for guarding takes four weeks.
\par Heavy Labor (DC 15): A creature trained for heavy labor knows the tricks come and work. Training a creature for heavy labor takes two weeks.
\par Hunting (DC 20): A creature trained for hunting knows the tricks attack, down, fetch, heel, seek, and track. Training a creature for hunting takes six weeks.
\par Performance (DC 15): A creature trained for performance knows the tricks come, fetch, heel, perform, and stay. Training a creature for performance takes five weeks.
\par Riding (DC 15): A creature trained to bear a rider knows the tricks come, heel, and stay. Training a creature for riding takes three weeks.

\subsk{Rear a Wild creature} To rear a creature means to raise a wild creature from infancy so that it becomes domesticated. A handler can rear as many as three creatures of the same kind at once. A successfully domesticated creature can be taught tricks at the same time it's being raised, or it can be taught as a domesticated creature later.

\parhead{Action} Varies. Handling a creature is a move action, while pushing a creature is a full-round action. For tasks with specific time frames noted above, you must spend half this time (at the rate of 3 hours per day per creature being handled) working toward completion of the task before you attempt the Creature Handling check. If the check fails, your attempt to teach, rear, or train the creature fails and you need not complete the teaching, rearing, or training time. If the check succeeds, you must invest the remainder of the time to complete the teaching, rearing, or training. If the time is interrupted or the task is not followed through to completion, the attempt to teach, rear, or train the creature automatically fails.

\parhead{Try Again} Yes, except for rearing a creature.
\parhead{Special} Animals are easier to handle than other kinds of creatures. The DCs listed are for animals; the DC to handle other kinds of creatures are 5 higher.

\parhead{Untrained} If you have no ranks in Creature Handling, you can use a Charisma check to handle and push domestic creatures, but you can't teach, rear, or train creatures.

\subsection{Devices (Int; Trained Only)}
\parhead{Check} You can attempt to sabotage or manipulate mechanical devices. The DC depends on how complex the device is, and how difficult your intended action is.

\begin{dtable}
\lcaption{Device DCs}
\begin{tabularx}{\columnwidth}{>{\lcol}X c}
\thead{Device Type} & \thead{Base DC} \\
Simple device (wagon wheel) & 10 \\
Average device (typical lock, door hinge) & 15 \\
Challenging device (typical trap, good lock) & 20 \\
Extraordinary device (magical trap, masterwork lock) & 30\plus \\
\end{tabularx}
\end{dtable}

\begin{dtable}
\lcaption{Action DC Modifiers}
\begin{tabularx}{\columnwidth}{>{\lcol}X c}
\thead{Action Type} & \thead{DC Modifier} \\
Identify device (find traps or hidden mechanisms) & \minus5 \\
Break device (jam a lock or hinge, disable trap while triggering it) & \minus5 \\
Normal usage (lock a lock, reset a trap) & \plus0 \\
Manipulate device (open lock, disable trap without triggering it, set device to break when used) & \plus5 \\
Extraordinary manipulation (recover trap) & \plus10 \\
Leave no trace of tampering & \plus5 \\
\end{tabularx}
\end{dtable}

If the check succeeds, your action succeeds. If it fails by 4 or less, you think you have succeeded, but your attempt actually had no effect. If you fail by 5 or more, something goes wrong. If the device is a trap, you spring it. If you were attempting to sabotage a device, it triggers, breaks, or otherwise does something contrary to your intentions.

You can take 10 on most Devices checks, but when dealing with traps, you are always considered to be ``threatened'' by the trap. You may not take 20 when disabling a trap.

\par You can attempt to disable magic traps in addition to mundane traps. A magic trap generally has a DC of 25 \add double the spell level of the magic used to create it.

\subsk{Bindings and Knots} You can use the Devices skill to bind an opponent in rope if the opponent is not struggling. You can also use it to create bindings and tie knots in general.

\parhead{Action} The amount of time needed to make a Disable Device check depends on the task. A task of DC 10 or lower takes a full-round action. A task of DC 20 or lower takes 1d4 rounds, a task of DC 30 or lower takes 2d4 rounds, and so on. For every 5 by which you beat the DC, you reduce the time required by 1 round, to a minimum of a standard action.

\parhead{Try Again} Varies. You can retry if you have missed the check by 4 or less, though you must be aware that you have failed in order to try again. Retrying after failing by 5 or more is not always possible. if it is possible, it is harder, and the DC increases by at least 5.

\begin{figure*}[hbt!]
\subsection{Other Ways to Beat a Trap}
It's possible to ruin many traps without making a Disable Device check.
\parhead{Ranged Attack Traps} Once a trap's location is known, the obvious way to ruin it is to smash the mechanism -- assuming the mechanism can be accessed. Failing that, it's possible to plug up the holes from which the projectiles emerge. Doing this prevents the trap from firing unless its ammunition does enough damage to break through the plugs.

\parhead{Melee Attack Traps} These devices can be thwarted by smashing the mechanism or blocking the weapons, as noted above. Alternatively, if a character studies the trap as it triggers, he might be able to time his dodges just right to avoid damage.

\parhead{Pits} Disabling a pit trap generally ruins only the trapdoor, making it an uncovered pit. Filling in the pit or building a makeshift bridge across it is an application of manual labor, not the Disable Device skill. Characters could neutralize any spikes at the bottom of a pit by attacking them -- they break just as daggers do.

\parhead{Magic Traps} \spell{Dispel magic} and similar spells help here. Someone who succeeds on a caster level check against the level of the trap's creator suppresses the trap for 1d4 rounds. This works only with a targeted \spell{dispel magic}, not the area version (see the spell description).
\end{figure*}

\subsection{Disguise (Int)}
\parhead{Check} A Disguise check allows you to change the appearance of a creature using makeup, costumes, and so forth. It is opposed by the Perception skill; a Spot check higher than your Disguise check will reveal the disguise. The effectiveness of a disguise depends in part on how much you're attempting to change the creature's appearance, as shown on the table below. The modifiers are cumulative; use any that apply.

\begin{dtable}
\begin{tabularx}{\columnwidth}{l >{\ccol}X}
\thead{Characteristic} & \thead{Disguise Check Modifier} \\
Different gender & \minus2 \\
Different race or subtype & \minus2 \\
Different age category & \minus2\footnotetemp{1} \\
Different creature type & \minus5 \\
Larger size category & \plus20\footnotetemp{2}
\end{tabularx}
1 Per step of difference between your actual age category and your
disguised age category. The steps are: young (younger than
adulthood), adulthood, middle age, old, and venerable.
2 Per step of difference between the original size category and the new size category.
\end{dtable}

You make the Disguise check when you apply the disguise. The Disguise check is made secretly, so that you can't be sure how good the result is. However, you can attempt to judge its effectiveness with Perception.

If make a disguise to emulate a particular individual, you must know what that individual looks like. The result of your Disguise check to emulate that person cannot exceed the result of a Spot check to see that person, which must be made by you or someone helping you. People viewing the disguise who know what that person looks like get a \plus5 circumstance bonus on their Spot checks to identify the disguise.

\par Anyone seeing the disguise may recognize it with their normal ``take 0'' check. Usually, an individual makes an active Spot check to see through your disguise only if they are suspicious of you. A creature that remains suspicious of you may make a new check once per hour.

\parhead{Action} Creating a disguise requires 1d4\mtimes10 minutes of work. You can take a \minus10 penalty to reduce the time to 1d4 minutes, or a \minus20 penalty to reduce the time to 1d4 rounds.
\parhead{Try Again} Yes. You may try to redo a failed disguise, but if others know that a disguise was attempted, they'll be more suspicious.
\parhead{Special} Magic that alters a creature's form, such as \spell{disguise self}, grants a \plus10 enhancement bonus to Disguise checks made to disguise the creature (see the individual spell descriptions). Divination magic that allows people to see through illusions (such as true seeing) does not penetrate a mundane disguise, but it can negate the magical component of a magically enhanced one.

This skill does not help you act appropriately while disguised. See Perform (acting) and Bluff.

\subsection{Escape Artist (Dex; Armor Check Penalty)}

\parhead{Check} You can escape from bindings and restrictions, or move through very small areas. The table below gives the DCs to escape various forms of restraints.

\begin{dtable}
\begin{tabularx}{\columnwidth}{>{\lcol}X l}
\thead{Restraint}  & \thead{Escape Artist DC} \\
Ropes & Binder's grapple attack \plus10 \\
Net & 20 \\
\spell{Snare} spell  & 23 \\
Manacles  & 30 \\
Tight space  & 30 \\
Masterwork manacles  & 35 \\
Grappler & Grappler's grapple attack result	 \\
\spell{Animate rope} spell, \spell{entangle} spell, or \spell{web} spell & Spell's save DC \\
\end{tabularx}
\end{dtable}

\subsk{Ropes} Your Escape Artist check is opposed by the binder's grapple attack to bind you. Since it's easier to tie someone up than to escape from being tied up, the binder gets a \plus10 circumstance bonus on his or her check.
\subsk{Manacles and Masterwork Manacles} The DC for manacles is set by their construction.
\subsk{Tight Space} The DC noted on the table is for getting through a space where your head fits but your shoulders don't. If the space is long you may need to make multiple checks. You can't get through a space that your head does not fit through.

\subsk{Grappler} You can make an Escape Artist check opposed by your enemy's grapple check to get out of a grapple or out of a pinned condition (so that you're only grappling).

\parhead{Action} Making an Escape Artist check to escape restraints of any kind is a full-round action. Squeezing through a tight space takes at least 1 minute, and possibly longer, depending on how long the space is.

\parhead{Try Again} Varies. You can make another check after a failed check if you're squeezing your way through a tight space, making multiple checks. If the situation permits, you can make additional checks, or even take 20, as long as you're not being actively opposed.

\subsection{Heal (Wis)}
\parhead{Check} The DC and effect depend on the task you attempt.
\begin{dtable}
\begin{tabularx}{\columnwidth}{>{\lcol}X r}
\thead{Task} & \thead{Heal DC} \\
First aid & 15 \\
Long-term care  & 15 \\
Accelerate healing & 25 \\
Treat wound from caltrop, \spell{spike growth}, or \spell{spike stones} & 15 \\
Treat poison  & Poison's save DC \\
Treat disease  & Disease's save DC
\end{tabularx}
\end{dtable}
\subsk{First Aid} You usually use first aid to save a dying character. If a character has negative hit points and is losing hit points (at the rate of 1 per round, 1 per hour, or 1 per day), you can make him or her stable. A stable character regains no hit points but stops losing them.
\subsk{Long-Term Care} Providing long-term care means treating a wounded person for a day or more. If your Heal check is successful, the patient recovers hit points or attribute score points (lost to ability damage) at twice the normal rate: half the patient's hit points and one point of ability damage for 4 hours of rest, or all of the patient's hit points and two points of ability damage with 8 hours of rest.

You can tend as many as six patients at a time. You need a few items and supplies (bandages, salves, and so on) that are easy to come by in settled lands. Giving long-term care counts as light activity for the healer. You cannot give long-term care to yourself.
\par For every 5 by which you beat the DC, time your patients need to recover is halved. For example, a Heal check result of 25 would mean that the patients need only 2 hours of rest to fully heal.

\subsk{Treat Wound from Caltrop, Spike Growth, or Spike Stones}A creature wounded by stepping on a caltrop moves at one-half normal speed. A successful Heal check removes this movement penalty.

A creature wounded by a spike growth or spike stones spell must succeed on a Reflex save or take injuries that reduce his speed by one-third. Another character can remove this penalty by taking 10 minutes to dress the victim's injuries and succeeding on a Heal check against the spell's save DC. For every 5 by which you beat the DC, the time required to dress the injuries is halved.

\subsk{Treat Poison} To treat poison means to tend a single character who has been poisoned and who is going to take more damage from the poison (or suffer some other effect). Every time the poisoned character makes a saving throw against the poison, you make a Heal check. The poisoned character uses your check result or his or her saving throw, whichever is higher.

\subsk{Treat Disease} To treat a disease means to tend a single diseased character. Every time he or she makes a saving throw against disease effects, you make a Heal check. The diseased character uses your check result or his or her saving throw, whichever is higher.

\parhead{Action} Providing first aid, treating a wound, or treating poison is a standard action. Treating a disease or tending a creature wounded by a spike growth or spike stones spell takes 10 minutes of work. Providing long-term care requires 8 hours of light activity.

\parhead{Try Again} Varies. Generally speaking, you can't try a Heal check again without proof of the original check's failure. You can always retry a check to provide first aid, assuming the target of the previous attempt is still alive.

\subsection{Intimidate (Cha)}
\parhead{Check} You can convince a creature to do what you want with a successful check. This check is made in the same way as a Persuasion check, except that you are always considered an enemy of the creature you are intimidating (\plus5 DC modifier). If you fail the check by 5 or more, the target provides you with incorrect or useless information, or otherwise frustrates your efforts.

\subsk{Demoralize Opponent} You can also use Intimidate to weaken an opponent's resolve in combat. To do so, make an Intimidate check opposed by the target's Will save or Intimidate check. If the target fails, he becomes shaken for 1 minute or until he successfully deals damage to you. You can intimidate any opponent that can see you.

\parhead{Action} Varies. Changing another's behavior requires 1 minute of interaction. Intimidating an opponent in combat is a standard action.

\parhead{Try Again} Optional, but not recommended because retries usually do not work. Even if the initial check succeeds, the other character can be intimidated only so far, and a retry doesn't help. If the initial check fails, the other character has probably become more firmly resolved to resist the intimidator, and a retry is futile.

\parhead{Special} You gain a \plus4 bonus on your Intimidate check for every size category that you are larger than your target. Conversely, you take a \minus4 penalty on your Intimidate check for every size category that you are smaller than your target.

\par A character immune to fear (such as a paladin of 3rd level or higher) can't be intimidated, nor can nonintelligent creatures.

\subsection{Knowledge (Int; Trained Only)}
Like the Craft, Profession, and Perform skills, Knowledge actually encompasses a number of unrelated skills. Knowledge represents a study of some body of lore, possibly an academic or even scientific discipline. Below are listed typical fields of study.
\begin{itemize*}
\item Arcana (ancient mysteries, magic traditions, arcane symbols,
cryptic phrases, constructs, dragons, magical beasts)
\item Engineering (architecture, buildings, bridges, fortifications, siege weapons)
\item Dungeoneering (aberrations, caverns, oozes, spelunking, subterranean monsters)
\item Geography (lands, terrain, climate, people, outdoor monsters)
\item Local (humanoids, legends, inhabitants, laws, customs, history, nobility, royalty)
\item Nature (animals, fey, giants, monstrous humanoids, plants, seasons and cycles, weather, vermin)
\item Planes (the Inner Planes, the Outer Planes, the Astral Plane,
the Ethereal Plane, outsiders, elementals, magic related to the planes, extraplanar monsters)
\item Religion (gods and goddesses, mythic history, ecclesiastic tradition, holy symbols, undead)
\end{itemize*}
\parhead{Check} Answering a question within your field of study has a DC of 10 (for really easy questions), 15 (for basic questions), or 20 to 30 (for really tough questions).

In many cases, you can use this skill to identify monsters and their special powers or vulnerabilities. In general, the DC of such a check equals 10 \add the monster's CR. A successful check allows you to remember the monster's name and its most well-known features.

For every 5 points by which your check result exceeds the DC, you recall another piece of useful information.

\parhead{Action} Usually none. In most cases, making a Knowledge check doesn't take an action -- you simply know the answer or you don't.
\parhead{Try Again} No. The check represents what you know, and thinking about a topic a second time doesn't let you know something that you never learned in the first place.
\parhead{Untrained} An untrained Knowledge check is simply an Intelligence check. Without actual training, you know only common knowledge (DC 10 or lower). Particularly common or famous monsters, such as goblins or dragons, can be recognized with an untrained knowledge check of this sort.

\subsection{Linguistics (Int; Trained Only)}
\parhead{Check} You can decipher writing in an unfamiliar language or a message written in an incomplete or archaic form. The base DC is 20 for the simplest messages, 25 for standard texts, and 30 or higher for intricate, exotic, or very old writing. In addition, the DC increases by 5 if you do not know any languages that use the same alphabet as the writing being deciphered.

If the check succeeds, you understand the general content of a piece of writing about one page long (or the equivalent). If the check fails by 5 or more, make a DC 5 Wisdom check to see if you avoid drawing a false conclusion about the text. (Success means that you do not draw a false conclusion; failure means that you do.) The DC of the Wisdom check may sometimes be higher or lower, depending on the particular writing in question.

Both the check to decipher and (if necessary) the Wisdom check are made secretly, so that you can't tell whether the conclusion you draw is true or false.

In addition, for every two ranks in Linguistics that your character has, you may learn a new language. Languages work as follows.
\begin{itemize*}
\item You start at 1st level knowing one or two languages (based on your race), plus an additional number of languages equal to your starting Intelligence bonus.
\item You don't make Linguistics checks to speak or understand languages. You either know a language or you don't.
\item A literate character (anyone but a barbarian who has not spent skill points to become literate) can read and write any language she speaks. Each language has an alphabet, though sometimes several spoken languages share a single alphabet.
\end{itemize*}

Languages are summarized on the table below.

\begin{dtable}
\begin{tabularx}{\columnwidth}{l >{\lcol}X l}
\thead{Language}  & \thead{Typical Speakers}  & \thead{Alphabet} \\
Abyssal  & Demons, chaotic evil outsiders  & Infernal \\
Aquan  & Water-based creatures  & Elven \\
Auran  & Air-based creatures  & Draconic \\
Celestial  & Good outsiders  & Celestial \\
Common  & Humans, halflings, half-elves, half-orcs  & Common \\
Draconic  & Kobolds, troglodytes, lizardfolk, dragons & Draconic \\
Druidic  & Druids (only)  & Druidic \\
Dwarven  & Dwarves  & Dwarven \\
Elven  & Elves  & Elven \\
Giant  & Ogres, giants  & Dwarven \\
Gnome  & Gnomes  & Dwarven \\
Goblin  & Goblins, hobgoblins, bugbears  & Dwarven \\
Gnoll  & Gnolls  & Common \\
Halfling  & Halflings  & Common \\
Ignan  & Fire-based creatures  & Draconic \\
Infernal  & Devils, lawful evil outsiders  & Infernal \\
Orc  & Orcs  & Dwarven \\
Sylvan  & Dryads, brownies, leprechauns  & Elven \\
Terran  & Xorns and other earth-based creatures & Dwarven \\
Undercommon  & Drow & Elven
\end{tabularx}
\end{dtable}

\parhead{Action} Deciphering the equivalent of a single page of script takes 1 minute (ten consecutive full-round actions).

\parhead{Try Again} No.

\subsection{Perception (Wis)}
The Perception skill is used to observe things which you might otherwise fail to notice. It can be used to spot concealed things, or to listen to sounds accurately. These are not different skills, but merely different applications of the same skill.

It is possible to have bonuses or penalties to only the Spot part of a Perception check, or to only the Listen part. For example, it is extremely difficult to see when there is no light, but that has no effect on how difficult it is for you to hear sounds. In such a case, roll the Perception check once and add the Spot modifier separately from the Listen modifier. These are referred to as separate checks -- a Spot check and a Listen check -- though you only roll one die. If you succeed on a Listen check without succeeding on a Spot check, you have heard a sound but can't see what it is coming from, or vice versa.

Some creatures have additional senses, such as scent. These senses are also used with the Perception skill.

\subsubsection{Listen}
\parhead{Check} Your Listen check is either made against a DC that reflects how quiet the noise is that you might hear, or it is opposed by your target's Move Silently check (a form of Stealth check).
\begin{dtable}
\begin{tabularx}{\columnwidth}{l >{\lcol}X}
\thead{Listen DC}   &  \thead{Sound}  \\
\minus10   &  A battle  \\
0   &  People talking\footnotetemp{1}  \\
5   &  A person in medium armor walking at a slow pace (10 ft./round) trying not to make any noise.  \\
10   &  An unarmored person walking at a slow pace (15 ft./round) trying not to make any noise  \\
15   &  A 1st-level rogue using Move Silently to sneak past the listener  \\
15   &  People whispering\footnotetemp{1}  \\
19   &  A cat stalking  \\
30   &  An owl gliding in for a kill  \\
\end{tabularx}
1 If you beat the DC by 10 or more, you can make out what's being said, assuming that you understand the language.
\end{dtable}
\begin{dtable}
\begin{tabularx}{\columnwidth}{l >{\lcol}X}
\thead{Listen DC Modifier}  &  \thead{Condition} \\
\plus5  &  Through a door  \\
\plus15  &  Through a stone wall  \\
\plus1  &  Per 10 feet of distance  \\
\plus5  &  Listener distracted
\end{tabularx}
\end{dtable}
In the case of people trying to be quiet, the DCs given on the table could be replaced by Move Silently checks, in which case the indicated DC would be their average check result.

\subsk{Discern Illusion} If you are near an illusion spell which allows a Will save to disbelieve, you can attempt a Listen check against the save DC of the spell (including other modifiers as normal). If you succeed, you are considered to have interacted with it, and can make a Will save to disbelieve it accordingly. This can only be used if the illusion in question is or should be producing sound. For example, a silent illusion of a ghost should not produce sound, and therefore cannot be interacted with by a Listen check.

\parhead{Action} Varies. Every time you have a chance to hear something in a reactive manner (such as when someone makes a noise or you move into a new area), you can make a Listen check without using an action. Trying to hear something you failed to hear previously is a move action.
\parhead{Try Again} Yes. You can try to hear something that you failed to hear previously with no penalty.
\parhead{Special} When several characters are listening to the same thing, a single 1d20 roll can be used for all the individuals' Listen checks.
\par A fascinated creature takes a  \minus4 penalty on Listen checks made as reactions.
\par A sleeping character may make Listen checks at a \minus10 penalty. A successful check awakens the sleeper.

\subsubsection{Spot (Wis)}
\parhead{Check} The Spot skill is used to notice hidden creature or objects which are not readily apparent. Typically, your Spot check is opposed by the Hide check (a form of Stealth check) of the creature trying not to be seen or the DC to find a concealed object or trap. Sometimes a creature or object isn't intentionally hidden but is still difficult to see, so a successful Spot check is necessary to notice it.

A Spot check result higher than 20 generally lets you become aware of an invisible creature near you, though you can't actually see it.

Spot is also used to detect someone in disguise (see the Disguise skill), and to read lips when you can't hear or understand what someone is saying.

Spot checks may be called for to determine the distance at which an encounter begins. A penalty applies on such checks, depending on the distance between the two individuals or groups, and an additional penalty may apply if the character making the Spot check is distracted (not concentrating on being observant).

\begin{dtable}
\begin{tabularx}{\columnwidth}{>{\lcol}X >{\lcol}X}
\thead{Condition} & \thead{Penalty} \\
Per 10 feet of distance & \minus1 \\
Spotter distracted & \minus5 \\
\end{tabularx}
\end{dtable}

\subsk{Read Lips} To understand what someone is saying by reading lips, you must be within 30 feet of the speaker, be able to see him or her speak, and understand the speaker's language. (This use of the skill is language-dependent.) The base DC is 15, but it increases for complex speech or an inarticulate speaker. You must maintain a line of sight to the lips being read.

If your Spot check succeeds, you can understand the general content of a minute's worth of speaking, but you usually still miss certain details. If the check fails by 4 or less, you can't read the speaker's lips. If the check fails by 5 or more, you draw some incorrect conclusion about the speech. The check is rolled secretly in this case, so that you don't know whether you succeeded or missed by 5.

\subsk{Discern Illusion} If you are near an illusion spell which allows a Will save to disbelieve, you can reflexively make a Spot check against the save DC of the spell (including other modifiers as normal). If you succeed, you are considered to have interacted with it, and can make a Will save to disbelieve it accordingly. This can only be used if the illusion in question has a visual component. For example, an illusion of a wave of heat has no visual component, and therefore cannot be interacted with by a Spot check.

\subsk{Search} Noticing some things, such as hidden traps and secret doors, requires close examination and attention to detail. You generally must be within 10 feet of the object or surface to be searched. The table below gives DCs for typical searching tasks involving the Spot skill.

\begin{dtable}
\lcaption{Active Searching DCs}
\begin{tabularx}{\columnwidth}{>{\lcol}p{15em} >{\rcol}X}
\thead{Task} & \thead{Spot DC} \\
Ransack a chest full of junk to find a certain item & 10 \\
Notice a typical secret door or a simple trap  & 20 \\
Find a difficult nonmagical trap & 21 or higher \\
Find a magic trap & 25 \add double level of spell used to create trap \\
Notice a well-hidden secret door  & 30 \\
Find a footprint  & Varies\footnotetemp{1} \\
\end{tabularx}
1 A successful Spot check can find a footprint or similar sign of a creature's passage, but it won't let you find or follow a trail. See the Track feat for the appropriate DC.
\end{dtable}

\parhead{Action} Varies. Every time you have a chance to spot something in a reactive manner you can make a Spot check without using an action. Trying to spot something you failed to see previously is a move action. To read lips, you must concentrate for a full minute before making a Spot check, and you can't perform any other actions (other than moving at up to half speed) during this minute. It takes a full-round action to actively search a 5-foot-by-5-foot area or a volume of goods 5 feet on a side.
\parhead{Try Again} Yes. You can try to spot something that you failed to see previously at no penalty. You can attempt to read lips once per minute.
\parhead{Special} A fascinated creature takes a \minus4 penalty on Spot checks
made as reactions.

\par The rituals \spell{explosive runes},  \spell{fire trap},  \spell{glyph of warding},  \spell{symbol of X}, and \spell{teleportation circle} create magic traps  that can be found by making a successful Search check. \spell{Spike growth} and \spell{spike stones} create magic traps that can be found using Spot, but against which Disable Device checks do not succeed. See the individual spell descriptions for details.

\par Active abjuration spells within 10 feet of each other for 24 hours or more create barely visible energy fluctuations. These fluctuations reduce the DC to locate such abjuration spells by 4.

\subsection{Perform (Cha)}
\par Like Craft, Knowledge, and Profession, Perform is actually a number of separate skills. You could have several Perform skills, each with its own ranks, each purchased as a separate skill.

Each of the nine categories of the Perform skill includes a variety of methods, instruments, or techniques, a small list of which is provided for each category below.
\begin{itemize*}
\item Act (comedy, drama, mime)
\item Comedy (buffoonery, limericks, joke-telling)
\item Dance (ballet, waltz, jig)
\item Keyboard instruments (harpsichord, piano, pipe organ)
\item Oratory (epic, ode, storytelling)
\item Percussion instruments (bells, chimes, drums, gong)
\item String instruments (fiddle, harp, lute, mandolin)
\item Wind instruments (flute, pan pipes, recorder, shawm, trumpet)
\item Sing (ballad, chant, melody)
\end{itemize*}
\parhead{Check} You can impress audiences with your talent and skill.
\par A masterwork musical instrument gives you a \plus2 circumstance bonus on Perform checks that involve its use.
\parhead{Action} Varies. Trying to earn money by playing in public requires anywhere from an evening's work to a full day's performance. The bard's special Perform-based abilities are described in that class's description.
\parhead{Try Again} Yes. Retries are allowed, but they don't negate previous failures, and an audience that has been unimpressed in the past is likely to be prejudiced against future performances. (Increase the DC by 2 for each previous failure.)

\subsection{Persuasion (Cha)}
Use this skill to persuade the chamberlain to let you see the king, to negotiate peace between feuding barbarian tribes, or to convince the ogre mages that have captured you that they should ransom you back to your friends instead of twisting your limbs off one by one.

Persuasion includes negotiation, etiquette, social grace, tact, subtlety, and a way with words. A skilled character knows the formal and informal rules of conduct, social expectations, proper forms of address, and so on. This skill represents the ability to give others the right impression of oneself, to negotiate effectively, and to influence others.

\parhead{Check} You can propose a trade or agreement to another creature with your words; a Persuasion check can then persuade them that accepting it is a good idea. Either side of the deal may involve physical goods, money, services, promises, or abstract concepts like ``satisfaction.'' The DC for the Persuasion check is based on three factors: who the target is, the relationship between the target and the character making the check, and the risk vs. reward factor of the deal proposed.

\par The base DC for any Persuasion check is equal to 10 \add the level of the highest-level character in the group that you are trying to influence \add the Wisdom of the character in the group with the highest Wisdom. If your offer has no risk and no possible downside to the target (such when as asking for directions, in most situations), the base DC is simply 10 instead.

\par If you are going to lie about the offer, make a Bluff check in addition to the Persuasion check. If the Bluff check fails, the Persuasion check automatically fails, as the target recognizes your offer is not genuine.
\begin{dtable}
\begin{tabularx}{\columnwidth}{>{\lcol}X r}
\thead{Relationship} & \thead{Modifier} \\
Intimate: Someone who with whom you have an implicit trust.
Example: A lover or spouse. & \minus15 \\
Friend: Someone with whom you have a regularly positive personal relationship.
Example: A long-time buddy or a sibling. & \minus10 \\
Ally: Someone on the same team, but with whom you have no personal relationship.
Example: A cleric of the same religion or a knight serving the same king. & \minus5 \\
Acquaintance (Positive): Someone you have met several times with no particularly negative experiences. Example: The blacksmith that buys your looted equipment regularly. & \minus2 \\
Just Met: No relationship whatsoever.
Example: A guard at a castle or a traveler on a road. & \plus0 \\
Acquaintance (Negative): Someone you have met several times with no particularly positive experiences. Example: A town guard that has arrested you for drunkenness once or twice. & \plus2 \\
Enemy: Someone on an opposed team, with whom you have no personal relationship.
Example: A cleric of a philosophically-opposed religion or an orc bandit who is robbing you. & \plus5 \\
Personal Foe: Someone with whom you have a regularly antagonistic personal relationship.
Example: An evil warlord whom you are attempting to thwart, or a bounty hunter who is tracking you down for your crimes. & \plus10 \\
Nemesis: Someone who has sworn to do you, personally, harm. Example: The brother of a man you murdered in cold blood. & \plus15 \\
\end{tabularx}
\end{dtable}
\begin{dtable*}
\begin{tabularx}{\textwidth}{>{\lcol}X r}
\thead{Risk vs. Reward Judgement (Persuasion)} & \thead{Modifier} \\
Fantastic: The reward for accepting the deal is very worthwhile; the risk is either acceptable or extremely unlikely. The best-case scenario is a virtual guarantee. Example: An offer to pay a lot of gold for information that isn't important to the character. & \minus15 \\
Good: The reward is good and the risk is minimal. The subject is very likely to profit from the deal. Example: An offer to pay someone twice their normal daily wage to spend their evening in a seedy tavern with a reputation for vicious brawls and later report on everyone they saw there. & \minus10\\
Favorable: The reward is appealing, but there's risk involved. If all goes according to plan, though, the deal will end up benefiting the subject. Example: A request for a mercenary to aid the party in battle against a weak goblin tribe in return for a cut of the money and first pick of the magic items. & \minus5\\
Even: The reward and risk more of less even out; or the deal involves neither reward nor risk. Example: A request for directions to a place that isn't a secret. & \plus0 \\
Unfavorable: The reward is not enough compared to the risk involved. Even if all goes according to plan, chances are it will end badly for the subject. Example: A request to free a prisoner the target is guarding for a small amount of money. & \plus5\\
Bad: The reward is poor and the risk is high. The subject is very likely to get the raw end of the deal. Example: A request for a mercenary to aid the party in battle against an ancient red dragon for a small cut of any non-magical treasure. & \plus10 \\
Horrible: There is no conceivable way that the proposed plan could end up with the subject ahead or the worst-case scenario is guaranteed to occur. Example: An offer to trade a rusty kitchen knife for a shiny new longsword. & \plus15 \\
\end{tabularx}
\end{dtable*}
\parhead{Success or Failure} If the Persuasion check beats the DC, the subject accepts the proposal, with no changes or with minor (mostly idiosyncratic) changes. If the check fails by 5 or less, the subject does not accept the deal but may, at the DM's option, present a counter-offer that would push the deal up one place on the risk-vs.-reward list. For example, a counter-offer might make an Even deal Favorable for the subject. The character who made the Persuasion check can simply accept the counter-offer, if they choose; no further check will be required. If the check fails by 10 or more, the Persuasion is over; the subject will entertain no further deals, and may become hostile or take other steps to end the conversation.

\subsk{Gather Information} An evening's time, a few gold pieces for buying drinks and making friends, and a DC 10 Persuasion check get you a general idea of a city's major news items, assuming there are no obvious reasons why the information would be withheld. The higher your check result, the better the information.

If you want to find out about a specific rumor, or a specific item, or obtain a map, or do something else along those lines, the DC for the check is 15 to 25, or even higher.

\parhead{Action} Influencing others with Persuasion generally takes at least 1 full minute. In many situations, this time requirement may greatly increase. A rushed Persuasion check (such as an attempt to head off a fight between two angry warriors) can be made as a full-round action, but you take a \minus10 penalty on the check. Gathering information generally takes d4 hours.

\parhead{Try Again} If you alter the parameters of the deal you are proposing, you may try to convince the subject that this new deal is even better than the last one. This is essentially how people haggle. As long as you never fail the Persuasion check by 10 or more, you can continue to offer deals.

\subsection{Profession (Wis; Trained Only)}
Like Craft, Knowledge, and Perform, Profession is actually a number of separate skills. You could have several Profession skills, each with its own ranks, each purchased as a separate skill. While a Craft skill represents ability in creating or making an item, a Profession skill represents an aptitude in a vocation requiring a broader range of less specific knowledge.

\parhead{Check} You can practice your trade and make a decent living, earning about half your Profession check result in gold pieces per week of dedicated work. You know how to use the tools of your trade, how to perform the profession's daily tasks, how to supervise helpers, and how to handle common problems.

\parhead{Action} Not applicable. A single check generally represents a week of work.
\parhead{Try Again} Varies. An attempt to use a Profession skill to earn an income cannot be retried. You are stuck with whatever weekly wage your check result brought you. Another check may be made after a week to determine a new income for the next period of time. An attempt to accomplish some specific task can usually be retried.
\parhead{Untrained} Untrained laborers and assistants (that is, characters without any ranks in Profession) earn an average of 1 silver piece per day. Most commoners have ranks in a Profession skill.

\subsection{Ride (Dex)}
If you attempt to ride a creature that is ill suited as a mount, you take a ?5 penalty on your Ride checks.
\parhead{Check} Typical riding actions don't require checks. You can saddle, mount, ride, and dismount from a mount without a problem. The following tasks do require checks.

\begin{dtable}
\begin{tabularx}{\columnwidth}{>{\lcol}X c >{\lcol}X c}
\thead{Task}  & \thead{Ride DC}  & \thead{Task}  & \thead{Ride DC} \\
Guide with knees  & 5 & Leap  & 15 \\
Stay in saddle  & 5 & Spur mount  & 15 \\
Fight with warhorse  & 10 &  Control mount in battle & 20 \\
Cover  & 15 & Fast mount or dismount & 20\fn{1} \\
Soft fall & 15 &  &
\end{tabularx}
1 Armor check penalty applies.
\end{dtable}
\subsk{Guide with Knees} You can react instantly to guide your mount with your knees so that you can use both hands in combat. Make your Ride check at the start of your turn. If you fail, you can use only one hand this round because you need to use the other to control your mount.

\subsk{Stay in Saddle} You can react instantly to try to avoid falling when your mount rears or bolts unexpectedly or when you take damage. This usage does not take an action

\subsk{Fight with Warhorse} If you direct your war-trained mount to attack in battle, you can still make your own attack or attacks normally. This usage is a free action.

\subsk{Cover} You can react instantly to drop down and hang alongside your mount, using it as cover. You can't attack or cast spells while using your mount as cover. If you fail your Ride check, you don't get the cover benefit. This usage does not take an action.

\subsk{Soft Fall} You can react instantly to try to take no damage when you fall off a mount -- when it is killed or when it falls, for example. If you fail your Ride check, you take 1d6 points of falling damage. This usage does not take an action.

\subsk{Leap} You can get your mount to leap obstacles as part of its movement. Use your Ride modifier or the mount's Jump modifier, whichever is lower, to see how far the creature can jump. If you fail your Ride check, you fall off the mount when it leaps and take the appropriate falling damage (at least 1d6 points). This usage does not take an action, but is part of the mount's movement.

\subsk{Spur Mount} You can spur your mount to greater speed with a move action. A successful Ride check increases the mount's speed by 10 feet for 1 round but deals 1 point of damage to the creature. You can use this ability every round, but each consecutive round of additional speed deals twice as much damage to the mount as the previous round (2 points, 4 points, 8 points, and so on).

\subsk{Control Mount in Battle} As a move action, you can attempt to control a light horse, pony, heavy horse, or other mount not trained for combat riding while in battle. If you fail the Ride check, you can do nothing else in that round. You do not need to roll for warhorses or warponies.

\subsk{Fast Mount or Dismount} You can attempt to mount or dismount from a mount of up to one size category larger than yourself as a free action, provided that you still have a move action available that round. If you fail the Ride check, mounting or dismounting is a move action. You can't use fast mount or dismount on a mount more than one size category larger than yourself.

\parhead{Action} Varies. Mounting or dismounting normally is a move action. Other checks are a move action, a free action, or no action at all, as noted above.
\parhead{Special} If you are riding bareback, you take a \minus5 penalty on Ride checks.
\par If your mount has a military saddle (\pref{Mounts and Related Gear}), you get a \plus2 circumstance bonus on Ride checks related to staying in the saddle.

\subsection{Sense Motive (Wis)}
\parhead{Check} A successful check lets you avoid being bluffed (see the Bluff skill). You can also use this skill to determine when ``something is up'' (that is, something odd is going on) or to assess someone's trustworthiness.

\begin{dtable}
\begin{tabularx}{\columnwidth}{>{\lcol}X >{\lcol}X}
Task Sense Motive & DC \\
Hunch & 20 \\
Sense enchantment & 25 or 15 \\
Discern secret message & Varies \\
\end{tabularx}
\end{dtable}

\subsk{Hunch} This use of the skill involves making a gut assessment of the social situation. You can get the feeling from another's behavior that something is wrong, such as when you're talking to an impostor. Alternatively, you can get the feeling that someone is trustworthy.

\subsk{Discern Mental Influence} If you or a creature you are interacting with is affected by beguilement, compulsion, or domination effect that alters their behavior, you can attempt a Sense Motive check against the save DC of the spell (including other modifiers as normal). If you succeed, you recognize the existence of unnatural mental influence. If you recognize such an effect on yourself, you can make a Will save to fight off the effect once each round.

This can only be used if the effect in question is actually affecting the creature's behavior at the time. For example, a person who has been given an unnatural aversion to cheese, as the \spell{aversion} spell, would generally not show signs of the enchantment unless presented with cheese. Therefore, you could not generally make a Sense Motive check to detect the effect if the creature was simply talking with you about the weather.

\subsk{Discern Secret Message} You may use Sense Motive to detect that a hidden message is being transmitted via the Bluff skill. In this case, your Sense Motive check is opposed by the Bluff check of the character transmitting the message. For each piece of information relating to the message that you are missing, you take a \minus2 penalty on your Sense Motive check. If you succeed by 4 or less, you know that something hidden is being communicated, but you can't learn anything specific about its content. If you beat the DC by 5 or more, you intercept and understand the message. If you fail by 4 or less, you don't detect any hidden communication. If you fail by 5 or more, you infer some false information.

\parhead{Action} Trying to gain information with Sense Motive generally takes at least 1 minute, and you could spend a whole evening trying to get a sense of the people around you.
\parhead{Try Again} No, though you may make a Sense Motive check for each Bluff check made against you.
\par If you have the Negotiator feat, you get a \plus2 competence bonus on Sense
Motive checks.

\subsection{Sleight Of Hand (Dex; Trained Only; Armor Check Penalty)}
\parhead{Check} A DC 10 Sleight of Hand check lets you palm a coin-sized, unattended object. Performing a minor feat of legerdemain, such as making a coin disappear, also has a DC of 10 unless an observer is determined to note where the item went.

When you use this skill under close observation, your skill check is opposed by the observer's Spot check. The observer's success doesn't prevent you from performing the action, just from doing it unnoticed. If the observer is actively defending himself from you, you can't make a Sleight of Hand check to steal from him. See the Disarm combat maneuver instead.

You can hide a tiny object (including a light weapon or an easily concealed ranged weapon, such as a dart, sling, or hand crossbow) on your body. The object must be at least two size categories smaller than you are. Your Sleight of Hand check is opposed by the Spot check of anyone observing you. The observer gains a \plus4 bonus for directly interacting with you to search for the object, such as by frisking you. You gain a bonus on your Sleight of Hand check when hiding extraordinarily small objects: Diminuitive objects grant a \plus2 circumstance bonus, and Fine objects grant a \plus4 circumstance bonus.

Drawing a hidden weapon is a standard action and doesn't provoke an attack of opportunity.

If you try to take something from another creature, you must make a DC 20 Sleight of Hand check to obtain it. The opponent makes a Spot check to detect the attempt, opposed by the same Sleight of Hand check result you achieved when you tried to grab the item. An opponent who succeeds on this check notices the attempt, regardless of whether you got the item.

You can also use Sleight of Hand to entertain an audience as though you were using the Perform skill. In such a case, your ``act'' encompasses elements of legerdemain, juggling, and the like.

\begin{dtable}
\begin{tabularx}{\columnwidth}{>{\lcol}X >{\lcol}X}
\thead{DC} & \thead{Task} \\
10 & Palm a coin-sized object, make a coin disappear \\
20 & Lift a small object from a person
\end{tabularx}
\end{dtable}

\parhead{Action} Any Sleight of Hand check is normally a standard action. However, you may perform a Sleight of Hand check as a swift action by taking a \minus20 penalty on the check.
\parhead{Try Again} Yes, but observers who have seen you fail before gain a \plus10 circumstance bonus to catch you again.
\parhead{Untrained} An untrained Sleight of Hand check is simply a Dexterity check. Without actual training, you can't succeed on any Sleight of Hand check with a DC higher than 10, except for hiding an object on your body.

\subsection{Spellcraft (Wis; Trained Only)}
Use this skill to identify spells as they are cast or spells already in place.

\begin{dtable}
\lcaption{Spellcraft DCs}
\begin{tabularx}{\columnwidth}{l >{\lcol}X l l}
\thead{Spellcraft DC}  & \thead{Task} & \thead{Action} & \thead{Retry} \\
10 & Notice the presence of any magical auras within 100 feet\fn{1} & Reflexive\fn{2} & Yes \\
10 \add spell level & When using \spell{read magic}, identify a \spell{glyph} or \spell{symbol} spell. & Reflexive & Yes \\
15 & Identify the number of magical auras within 100 feet & Move & Yes \\
15 & Identify the strength and location of an aura you have identified\fn{3} & Move & Yes \\
15 \add spell level\fn{4} & Identify the school of an aura you have identified & Move\fn{5} & Yes \\
15 \add spell level  & Identify a spell being cast. (You must see or hear the spell's verbal or somatic components.) & Reflexive & No  \\
15 \add spell level  & Determine the school of magic involved in the aura of a single item or creature you can see. (If the aura is not a spell effect, the DC is 15 \add one-half caster level.) & Move & Yes \\
20 \add spell level  & Identify a spell that's already in place and in effect. You must be able to see or detect the effects of the spell. & Reflexive & Yes \\
20 \add spell level  & Identify materials created or shaped by magic, such as noting that an iron wall is the result of a \spell{wall of iron} spell. & Reflexive & Yes \\
20 \add spell level  & Decipher a written spell (such as a scroll) without using \spell{read magic}. & Full-round & 1/day \\
25 \add spell level  & After being affected by against a spell targeted on you, determine what that spell was. & Reflexive & No \\
25  & Identify a potion. & 1 minute & No \\
30 or higher  & Understand a strange or unique magical effect, such as the effects of a magic stream. & Varies & No \\
\end{tabularx}
1 You cannot differentiate individual auras. However, you can ignore auras you are familiar with (such as magic items you or your companions wear). \\
2 You can make a check reflexively to perceive new information without taking an action. Actively paying attention to notice something requires a move action. \\
3 See \trefnp{Aura Strengths} for information on how strong auras are.
4 Or half caster level for effects which are not spells.
5 Can be done as part of the same action as identifying the strength and location of the aura.
\end{dtable}

\begin{dtable*}
\lcaption{Aura Strengths}
\begin{tabularx}{\textwidth}{>{\lcol}X *{4}{>{\lcol}p{9em}}}
& \multicolumn{4}{c}{\thead{---{}---{}---Aura Power---{}---{}---}} \\
\thead{Spell or Object} & \thead{Faint} & \thead{Moderate} & \thead{Strong} & \thead{Overwhelming} \\
Functioning spell (spell level) & 3rd or lower & 4th--6th & 7th--9th & 10th\add (deity-level) \\
Magic item (caster level) & 6th or lower & 7th--12th & 13th--20th & 21st\add (artifact)
\end{tabularx}
\end{dtable*}

\parhead{Lingering Auras}A magical aura can linger after its original source dissipates (in the case of a spell) or is destroyed (in the case of a magic item). If you use Spellcraft in such a location, you can an aura strength of ``dim'' (even weaker than a faint aura). How long the aura lingers at this dim level depends on its original power. Most auras only linger for a few rounds, but strong or overwhelming auras can linger for days.
\par Outsiders and elementals are not magical in themselves, but if they are summoned, the conjuration spell registers.

\parhead{Check} You can identify spells and magic effects. The DCs for Spellcraft checks relating to various tasks are summarized on the table above.
\parhead{Action} Varies, as noted above.
\parhead{Try Again} See above.
\parhead{Special} If you are a specialist wizard, you get a \plus2 competence bonus on Spellcraft checks when dealing with a spell or effect from your specialty school. You take a \minus5 penalty when dealing with a spell or effect from a prohibited school (and some tasks, such as learning a prohibited spell, are just impossible).

\subsection{Stealth (Dex; Armor Check Penalty)}
The Stealth skill is used to escape detection while moving or hiding. There are two components required to remain undetected: you must hide from sight, and you must be able to move silently so you cannot be overheard. These are not different skills, but merely different applications of the same skill.

It is possible to have bonuses or penalties to only the Hide part of a Stealth check, or to only the Move Silently part. For example, it is extremely easy to hide when there is no light, but that has no effect on how difficult it is for you to move silently. In such a case, roll the Stealth check once and add the Hide modifier separately from the Move Silently modifier. These are referred to as separate checks -- a Hide check and a Move Silently check -- though you only roll one die. If a creature beats your Move Silently check without beating your Hide check, they have heard you but not seen you, and vice versa.

\parhead{Check} Your Stealth check is opposed by the Perception check of anyone who might see you. If you move during your turn, you take a \minus5 penalty. When moving at a speed greater than one-half but less than your normal speed, you take a \minus10 penalty. It's practically impossible (\minus20 penalty) to remain unobserved while attacking, running or charging.

A creature larger or smaller than Medium takes a size bonus or penalty on Stealth checks depending on its size category: Fine \plus16, Diminutive \plus12, Tiny \plus8, Small \plus4, Large \minus4, Huge \minus8, Gargantuan \minus12, Colossal \minus16.

\subsubsection{Hide}
You need cover or concealment in order to attempt a Hide check. Total cover or total concealment usually (but not always; see Special, below) obviates the need for a Hide check, since nothing can see you anyway. When determining whether or not a character may attempt a hide check against a particular observer, do not consider any cover that does not impede the observer's vision or that would be hidden as a result of a successful check.

If people are observing you, even casually, you can't hide. You can run around a corner or behind cover so that you're out of sight and then hide, but the others then know at least where you went.

If your observers are momentarily distracted (such as by a Bluff check; see below), though, you can attempt to hide. While the others turn their attention from you, you can attempt a Hide check if you can get to a hiding place of some kind. (As a general guideline, the hiding place has to be within a quarter of your movement speed.) This check, however, is made at a \minus10 penalty because you have to move fast.

\subsk{Sniping} If you've already successfully hidden at least 10 feet from your target, you can make take a standard action to attack, then immediately hide again. You take a \minus20 penalty on your Hide check to conceal yourself after the attack.

\subsk{Creating a Diversion to Hide} You can use Bluff to help you hide. A successful Bluff check can give you the momentary diversion you need to attempt a Hide check while people are aware of you.

\parhead{Action} Usually none. Normally, you make a Hide check as part of movement, so it doesn't take a separate action. However, hiding immediately after an attack (see Sniping, above) is a move action.

\parhead{Special} If you are invisible, you gain a \plus40 bonus on Hide checks if you are immobile, or a \plus20 bonus on Hide checks if you're moving.

\subsubsection{Move Silently}
Noisy surfaces, such as bogs or undergrowth, are tough to move silently across. When you try to sneak across such a surface, you take a penalty on your Stealth check as indicated below.
\begin{dtable}
\begin{tabularx}{\columnwidth}{>{\lcol}X c}
\thead{Surface} & \thead{Check Modifier} \\
Noisy (scree, shallow or deep bog, undergrowth, dense rubble) & \minus2 \\
Very noisy (dense undergrowth, deep snow) & \minus5 \\
\end{tabularx}
\end{dtable}
\parhead{Action} None. A Move Silently check is included in your movement or other activity, so it is part of another action.

\subsection{Survival (Wis)}
\parhead{Check} You can keep yourself and others safe and fed in the wild. The table below gives the DCs for various tasks that require Survival checks.

Survival does not allow you to follow difficult tracks unless you are a ranger or have the Track feat (see the Restriction section below).

\begin{dtable}
\begin{tabularx}{\columnwidth}{c >{\lcol}X}
\thead{Survival DC}  & \thead{Task} \\
10  & Get along in the wild. Move up to one-half your overland speed while hunting and foraging (no food or water supplies needed). You can provide food and water for one other person for every 2 points by which your check result exceeds 10. \\
15  & Keep from getting lost or avoid natural hazards, such as quicksand. \\
15  & Predict the weather up to 24 hours in advance. For every 5 points by which your Survival check result exceeds 15, you can predict the weather for one additional day in advance. \\
Varies  & Follow tracks (see the Track feat).
\end{tabularx}
\end{dtable}

\parhead{Action} Varies. A single Survival check may represent activity over the course of hours or a full day. A Survival check made to find tracks is at least a full-round action, and it may take even longer.

\parhead{Try Again} Varies. For getting along in the wild or for gaining the Fortitude save bonus noted in the table above, you make a Survival check once every 24 hours. The result of that check applies until the next check is made. To avoid getting lost or avoid natural hazards, you make a Survival check whenever the situation calls for one. Retries to avoid getting lost in a specific situation or to avoid a specific natural hazard are not allowed. For finding tracks, you can retry a failed check after 10 minutes of searching, plus 10 more minutes for every 5 that you failed the check by.

\parhead{Restriction} While anyone can use Survival to find tracks (regardless of the DC), or to follow tracks when the DC for the task is 10 or lower, only a character with the Track feat can use Survival to follow tracks when the task has a higher DC.

\parhead{Special} If you have 5 or more ranks in Survival, you can automatically determine where true north lies in relation to yourself.

\subsection{Swim (Str; Armor Check Penalty)}
\parhead{Check} Make a Swim check once per round while you are in the water. Success means you may swim at up to one-half your speed (as a full-round action) or at one-quarter your speed (as a move action). If you fail by 4 or less, you make no progress through the water. If you fail by 5 or more, you go underwater.

If you are underwater, either because you failed a Swim check or because you are swimming underwater intentionally, you must hold your breath. You can hold your breath for a number of rounds equal to your Constitution score, but only if you do nothing other than take move actions or free actions. If you take a standard action or a full-round action (such as making an attack), the remainder of the duration for which you can hold your breath is reduced by 1 round. (Effectively, a character in combat can hold his or her breath only half as long as normal.) After that period of time, you must make a DC 10 Constitution check every round to continue holding your breath. Each round, the DC for that check increases by 1. If you fail the Constitution check, you begin to drown.

The DC for the Swim check depends on the water, as given on the table below.

\begin{dtable}
\begin{tabularx}{\columnwidth}{>{\lcol}X >{\lcol}X}
\thead{Water} & \thead{Swim DC} \\
Calm water & 10 \\
Rough water & 15 \\
Stormy water & 20\footnotetemp{1}
\end{tabularx}
1 You can't take 10 on a Swim check in stormy water, even if you aren't
otherwise being threatened or distracted.
\end{dtable}

Each hour that you swim, you must make a DC 20 Swim check or take 1d6 points of nonlethal damage from fatigue.

\subsk{Accelerated Swimming} If you beat the Swim DC by 10 or more, you can move half your speed (instead of one-quarter your speed).

\parhead{Action} Swimming is a kind of movement, so it's generally a move action. Each move action that includes any swimming requires a separate Swim check.
\parhead{Special} Swim checks are subject to double the normal armor check penalty and encumbrance penalty.

A creature with a swim speed can move through water at its indicated speed without making Swim checks. It gains a \plus5 inherent bonus on any Swim check to perform a special action or avoid a hazard. The creature always can choose to take 10 on a Swim check, even if distracted or endangered when swimming. Such a creature can use the run action while swimming, provided that it swims in a straight line.

\begin{comment}
\subsection{Use Rope (Dex)}
With this skill, you can make firm knots, undo tricky knots, and
bind prisoners with ropes.
\parhead{Check} Most tasks with a rope are relatively simple. The DCs for
various tasks utilizing this skill are summarized on the table below.

\begin{dtable}
\begin{tabularx}{\columnwidth}{l >{\lcol}X}
\thead{Use Rope DC} & \thead{Task} \\
10 & Tie a firm knot \\
10\footnotetemp{1} & Secure a grappling hook \\
15 & Tie a special knot, such as one that slips, slides slowly, or loosens with a tug \\
15 & Tie a rope around yourself one-handed \\
15 & Splice two ropes together \\
Varies & Bind a character
\end{tabularx}
1 Add 2 to the DC for every 10 feet the hook is thrown; see below.
\end{dtable}

\subsk{Secure a Grappling Hook} Securing a grappling hook requires a Use
Rope check (DC 10, \plus2 for every 10 feet of distance the grappling
hook is thrown, to a maximum DC of 20 at 50 feet). Failure by 4 or less indicates that the grappling hook
initially holds, but comes loose after 1d4 rounds of supporting
weight. Failure by 5 or more indicates that the hook fails to catch and falls, allowing you to
try again.  Your DM should make this check secretly, so that you don't
know whether the rope will hold your weight.
\subsk{Bind a Character} When you bind another character with a rope,
any Escape Artist check that the bound
character makes is opposed by your Use
Rope check. You get a \plus10 bonus on this
check because it is easier to bind
someone than to escape from bonds.
You don't even make your Use Rope
check until someone tries to escape.
\parhead{Action} Varies. Throwing a grappling
hook is a standard action that provokes an
attack of opportunity. Tying a knot, tying a
special knot, or tying a rope around
yourself one-handed is a full-round
action that provokes an attack of
opportunity. Splicing two ropes together
takes 5 minutes. Binding a character takes 1
minute.
\parhead{Special} A silk rope (\pref{Tools and Skill Kits}) gives you a \plus2 circumstance bonus on Use Rope checks. If you cast an  \spell{animate rope} spell on a rope, you get a \plus5
circumstance bonus on any Use Rope checks
you make when using that rope. These
bonuses stack.

If you have the Deft Hands feat, you get a \plus2 competence bonus
on Use Rope checks.
\end{comment}
