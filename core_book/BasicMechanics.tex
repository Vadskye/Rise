\chapter{Basic Mechanics}

\section{Attributes}\label{Attributes}
    Each character has six \glossterm[attribute]{attributes}: Strength (Str), Dexterity (Dex), Constitution (Con), Intelligence (Int), Perception (Per), and Willpower (Wil).
    Each attribute represents a character's raw talent in that area.
    A 0 in an attribute represents average human capacity.
    That doesn't mean that every commoner has a 0 in every attribute; not everyone is average, after all.

    \subsection{Attribute Descriptions}

        \subsubsection{Strength (Str)}\label{Strength}
            Strength measures muscle and physical power.
            \begin{itemize}
                \item Strength determines how much a character can carry (see \tref{Weight Limits}).
                \item Strength can be used to attack with melee and thrown weapons (see \pcref{Physical Accuracy}).
                \item Strength can be used to deal damage with all physical attacks (see \pcref{Physical Damage}).
                \item Strength can be used for Fortitude defense (see \pcref{Defenses}).
                \item Strength can be used for Climb, Jump, Sprint, and Swim skill checks (see \pcref{Skills}). If your Strength is negative, you take a penalty to all Strength-based skill checks equal to your Strength.
                \item For every 5 Strength you have, you gain a \plus1 bonus to damage with physical attacks. If your Strength is negative, you take a penalty to damage with physical attacks equal to half your Strength.
            \end{itemize}

        \subsubsection{Dexterity (Dex)}\label{Dexterity}
            Dexterity measures hand-eye coordination, agility, and reflexes.
            \begin{itemize}
                \item Dexterity can be used to attack with light melee and thrown weapons (see \pcref{Physical Accuracy}).
                \item Dexterity can be used for all \glossterm{physical defenses} (see \pcref{Defenses}).
                \item Dexterity can be used for Acrobatics, Escape Artist, Ride, Sleight of Hand, and Stealth skill checks (see \pcref{Skills}). If your Dexterity is negative, you take a penalty to all Dexterity-based skill checks equal to your Dexterity.
                \item For every 5 Dexterity you have, you gain a \plus1 bonus to \glossterm{physical defenses}. If your Dexterity is negative, you take a penalty to your physical defenses equal to half your Dexterity.
            \end{itemize}

        \subsubsection{Constitution (Con)}\label{Constitution}
            Constitution represents your character's health and stamina.
            \begin{itemize}
                \item Constitution can be used for Armor and Fortitude defenses (see \pcref{Defenses}).
                \item For every 2 Constitution you have, you gain a \plus1 bonus to Fortitude defense. If your Constitution is negative, you take a penalty to Fortitude defense equal to your Constitution.
            \end{itemize}

        \subsubsection{Intelligence (Int)}\label{Intelligence}
            Intelligence determines how well your character learns and reasons.
            It affects your character's knowledge in many areas.

            \begin{itemize}
                \item Intelligence is added to Craft, Disguise, Heal, Knowledge, and Linguistics skill checks (see \pcref{Skills}). If your Intelligence is negative, you take a penalty to all Intelligence-based skill checks equal to your Intelligence.
                \item Intelligence can be used for Mental defense (see \pcref{Defenses}).
                \item You gain bonus languages at 1st level equal to your starting Intelligence (see \pcref{Languages}).
                \item For every 2 Intelligence you have, you gain an extra skill point. If your Intelligence is negative, you take a penalty to your skill points equal to your Intelligence.
            \end{itemize}

            \par An animal has an Intelligence score of \minus6 or lower.
            A creature of humanlike intelligence has a score of at least a \minus5 Intelligence.

        \subsubsection{Perception (Per)}\label{Perception}
            Perception describes a character's ability to observe and be aware of one's surroundings.
            \begin{itemize}
                \item Perception can be used to attack with projectile weapons (see \pcref{Physical Accuracy}).
                \item Perception can be used for Awareness, Creature Handling, Sense Motive, Spellcraft, and Survival skill checks (see \pcref{Skills}). If your Perception is negative, you take a penalty to all Perception-based skill checks equal to your Perception.
                \item Perception can be used for Reflex defense (see \pcref{Defenses}).
                \item For every 5 Perception you have, you gain a \plus1 bonus to accuracy with \glossterm{physical attacks}.
                    If your Perception is negative, you take a penalty to accuracy with physical attacks equal to half your Perception.
            \end{itemize}

        \subsubsection{Willpower (Wil)}\label{Willpower}
            Willpower measures a character's ability to endure mental hardships.
            \begin{itemize}
                \item Willpower can be used for Mental defense (see \pcref{Defenses}).
                \item Many special abilities are based on Willpower.
                \item For every 2 Willpower you have, you gain a \plus1 bonus to Mental defense. If your Willpower is negative, you take a penalty to Mental defense equal to your Willpower.
            \end{itemize}

    \subsection{Using Attributes}

        \subsubsection{Choosing Attributes to Use}
            In many cases, multiple attributes can be used for the same thing.
            For example, both Strength and Dexterity can be used to attack with light weapons such as daggers.
            Whenever more than one attribute could be used, you must choose which one to use (usually, the higher attribute).

    \subsection{Determining Attributes}
        There are several options for how to determine attribute scores.

        \subsubsection{Predefined Attribute Scores}
            This is the simplest method.
            Simply take the following set of attribute scores and distribute them as you choose among your character's abilities:

            4, 4, 1, 1, 0, 0

            This set of attribute scores is called the ``elite array''.
            For more extreme characters, you may use the ``savant array'':

            5, 2, 1, 1, 0, 0.

            Finally, for more well-balanced characters, you may use the ``balanced array'':

            4, 3, 2, 2, 0, 0

            Any of these distributions can be altered by taking penalties to any attributes given as 0.
            For each penalty you take, you gain an additional \glossterm{skill point} (see \pcref{Skills}).

        \subsubsection{Point Buy}
            With this method, you can fully control your character's attribute scores to match what you want your character to be.
            All your character's attribute scores start at 0.
            You get 12 points to distribute among your character's attributes.
            Attributes can be bought according to the costs on \trefnp{Attribute Score Point Costs}.

            \parhead{Impaired Attributes}\label{Impaired Attributes}
            You can start with up to two attributes below 0.
            If you do, you compensate for your impairment in that area with additional talents in other areas.
            For each point below 0, you gain an additional \glossterm{skill point}.

            \begin{dtable}
                \lcaption{Attribute Score Point Costs}
                \begin{dtabularx}{\columnwidth}{X X}
                    \tb{Attribute Score}        & \tb{Point Cost} \\
                    \hline
                    \minus2\fn{1}               & 0\fn{2}         \\
                    \minus1\fn{1}               & 0\fn{3}         \\
                    0                           & 0               \\
                    1 \add one-quarter level    & 1               \\
                    2 \add half level           & 2               \\
                    3 \add three-quarters level & 3               \\
                    3 \add level                & 5               \\
                    4 \add level                & 8               \\
                \end{dtabularx}
                1 You cannot reduce more than two attributes below 0 in this way. \\
                2 You gain two \glossterm{skill points}. \\
                3 You gain one skill point. \\
            \end{dtable}

\section{Combat Overview}\label{Combat Overview}

    Combat takes place in a series of ``rounds'', which represent about six seconds of action.
    In combat, creatures attack each other (see \pcref{Attacks}) and defend themselves (see \pcref{Defenses}), while moving around the battlefield (see \pcref{Movement and Positioning}).
    When your defenses fail, you can get hurt (see \pcref{Injury, Death, and Healing}).
    In unusual situations, you might become more or less likely to succeed at your actions (see \pcref{Circumstances, Bonuses, and Penalties}).

    \subsection{Combat Prowess}\label{Combat Prowess}
        Every character has a \glossterm{combat prowess}, which represents how skilled they are in physical combat.
        You can add your combat prowess to your accuracy and damage with physical attacks (see \pcref{Physical Accuracy}, and \pcref{Physical Damage}).
        You can also add your combat prowess to your physical defenses (see \pcref{Defense Values}).
        In addition, your combat prowess may grant you additional \glossterm{strikes} during a round (see \pcref{Multiple Attacks}).

        As your character gains levels, her combat prowess will increase.
        Gaining levels in physical classes, like fighter, will cause it to increase faster, while gaining levels in non-physical classes, like wizard, will cause it to increase slower.

    \subsection{Combat Phases}

        Each round of a combat is divided into two phases: a movement phase and an action phase.
        During each phase, all characters declare their actions simultaneously, and then those actions are resolved simultaneously.
        After both phases are complete, the round ends.

        \subsubsection{The Movement Phase}\label{The Movement Phase}

            The movement phase takes place first in the round.
            During the movement phase, all creatures can move a distance equal to their \glossterm{speed} (see \pcref{Movement and Positioning} for details).
            In addition to moving, creatures can take minor actions that require motion, such as drawing a weapon.
            These actions are called \glossterm{move actions}.

            You can take any number of move actions during the movement phase, as long as all of those actions can be performed simultaneously.
            For example, you can walk your speed and draw your sword in a single movement phase.
            However, you cannot draw a sword and equip a shield in the same phase.
            Equipping a shield takes two hands, leaving you with no free hand to draw your sword.

            Once all creatures are done moving, the action phase begins.

        \subsubsection{The Action Phase}\label{The Action Phase}

            During the action phase, each creature can take a single \glossterm{standard action}.

            \parhead{Standard Action} You can use a standard action to attack with a weapon, cast a spell, drink a potion, and do most other things that take concentration and effort.

    \subsection{Resolving Actions}\label{Resolving Actions}

        Within each phase, actions of all creatures are simultaneously resolved in the following order.
        Italicized steps are less common, and can usually be skipped.

        \begin{enumerate*}
            \item Choose actions.
            \item Determine targets affected by actions.
            \item \textit{Resolve swift actions}.
            \item Check action success.
                Example: Making attack rolls.
            \item Determine action results.
                Example: Making damage rolls.
            \item Apply action results.
                Examples: Reducing hit points, moving character locations, and applying penalties.
                Effects that trigger when damage is dealt, such as Concentration checks (see \pcref{Concentration}), are resolved now.
            \item \textit{Choose delayed actions.}
            \item \textit{Determine targets affected by delayed actions.}
            \item \textit{Check delayed action success.}
            \item \textit{Determine delayed action results.}
            \item \textit{Apply delayed action results.}
        \end{enumerate*}

        In the vast majority of cases, there is no need to go through this order explicitly.
        Combats will run much faster if attack and damage rolls are generally made and announced at the same time as the actions are chosen, even before all characters have explicitly stated their actions.
        The order of resolution matters when creatures take actions that directly conflict with each other.

        \subsubsection{Conflicting Actions}\label{Conflicting Actions}

            Sometimes, actions that occur within the same resolution step can conflict with each other.
            There are two main methods for resolving these conflicts.

            \parhead{Mutually Exclusive Actions} Sometimes, actions that should take place at the same time directly conflict with each other.
            This most commonly happens when two creatures move to the same place.
            In this case, each involved character rolls initiative.
            The creature with the highest initiative result succeeds.
            All other creatures come as close as possible to completing their intended action.

            Your initiative check is calculated as follows:

            \begin{figure}[h]
                \centering Dexterity or Perception \add other bonuses and penalties
            \end{figure}

            For example, if two creatures were racing to reach a door, they would both roll initiative.
            The winner would reach the door and stop in their intended square, and the loser would stop adjacent to their intended square.

            \parhead{Conditionally Impossible Actions} In rare cases, one action may make another action impossible if the first action succeeds.
            However, unlike with mutually exclusive actions, the second action would not make the first action impossible.
            This usually happens if a creature moves during the action phase while being attacked.
            If the attack trips or deals enough damage to kill the moving creature, its movement becomes impossible.
            In this case, the second action is negated, and the creature takes no action during that action phase.

    \subsection{Special Actions}

        \parhead{Swift and Immediate Actions}\label{Swift and Immediate Actions} Each round, you can take a single swift or immediate action.
        Swift and immediate actions can be taken in either the movement or action phase.
        Swift actions must be declared along with any other actions you intend to take during that phase.
        They are resolved early in the phase, before other actions resolve.

        Immediate actions do not need to be declared ahead of time.
        Instead, abilities that can be used as immediate actions specify triggering conditions that allow the action to be taken.
        Immediate actions are resolved immediately, before the triggering action resolves.
        If multiple swift or immediate actions are taken simultaneously, they are resolved using the normal rules for resolving simultaneous actions.

        \parhead{Free Actions}\label{Free Actions} Each round, you can take any number of free actions.
        Free actions can be taken in either the movement or action phase.
        Like swift actions, free actions must be declared along with any other actions you intend to take during that phase.

        \parhead{Full-Round Actions} A full-round action requires your character's full attention.
        Most full-round actions involve a combination of movement and concentrated effort, such as \glossterm{charging} to strike a distant foe.
        Unless otherwise specified, you perform any movement required for the action during the movement phase, and the rest of the action during the action phase.

        \parhead{Delaying}\label{Delaying}
        During the action phase, you can delay your action instead of acting immediately.
        If you delay, you do nothing until after the actions of all other creatures have been resolved.
        At that point, you can declare and resolve your actions for the action phase.
        If multiple creatures delay, all their actions are declared and resolved in the normal action resolution order, as if they were part of a shared ``delay phase''.
        You cannot delay during the movement phase.

        Some abilities cause actions to be delayed, such as charging (see \pcref{Charge}).
        If you use an ability that causes actions to be delayed after you have already delayed, any actions which would be delayed are ignored.
        For example, if you charge after delaying, you would not be able to attack after the charge, making it generally pointless.

    \subsection{Attacks}\label{Attacks}
        An \glossterm{attack} is anything that affects another creatures in a potentially harmful way.
        There are two kinds of attacks: \glossterm{physical} attacks, such as striking with a weapon, and \glossterm{magical} attacks, such as spells.
        Most attacks require making an \glossterm{attack roll} against a \glossterm{defense}.
        To make an attack roll, you roll 1d20, adding your \glossterm{accuracy} with the attack to the roll.
        If the result of the attack roll equals or exceeds the defense, the attack succeeds.

        \subsubsection{Standard Attack}\label{Standard Attack}
            As a standard action, you can make a single \glossterm{strike} with a weapon you are wielding against an enemy.
            If you're using a melee weapon, you must \glossterm{threaten} your target.
            If you're using a ranged weapon, the target must be within the weapon's maximum \glossterm{range}.

            To make a strike, make an attack roll, as with most other atacks.
            If your attack roll beats the target's Armor defense, your foe takes damage.

        \subsubsection{Physical Accuracy}\label{Physical Accuracy}
            Your accuracy with physical attacks is equal to the following:

            \begin{figure}[h]
                \centering Combat prowess or attack attribute \add proficiency bonus \add size modifier \add other bonuses and penalties
            \end{figure}

            \parhead{Attack Attribute} You can use Strength to attack with melee and thrown weapons, Dexterity to attack with melee and thrown weapons that are light, and Perception to attack with projectile weapons.
            \parhead{Proficiency Bonus} You gain a \plus4 bonus to accuracy with a weapon you are proficient with.
            \parhead{Size Modifier} Your size modifier is described in \tref{Size in Combat}.

        \subsubsection{Physical Damage}\label{Physical Damage}
            If your strike hits, you deal damage equal to the following:
            \begin{figure}[h]
                \centering Weapon damage die \add half combat prowess or half Strength \add other bonuses and penalties
            \end{figure}

            \parhead{Dealing Nonlethal Damage} You can attempt to strike nonlethally with any weapon.
            If you hit, you deal half damage as \glossterm{nonlethal damage} (see \pcref{Nonlethal Damage}).

        \subsubsection{Reach}\label{Reach}
            Normally, you can make melee attacks against anyone within five feet of you.
            The range at which you can make melee attacks is called your \glossterm{reach}, and the area that you can attack into is called your \glossterm{threatened area}.
            Reach for larger and smaller creatures is determined by size, as shown on \trefnp{Size in Combat}.

    \subsection{Defenses}\label{Defenses}
        Usually, when you are attacked, the attacker has to make an attack roll against a specific defense.
        If the attack roll is at least as high as that defense, the attack succeeds.
        There are three physical defenses and two non-physical defenses.
        \begin{itemize}
            \item Armor defense (AD): Your Armor defense protects you from normal physical attacks, such as attempts to hit you with a sword.
                It is the most commonly used defense.
                Armor defense is a physical defense.
            \item Reflex defense: Your Reflex protects you from attacks you have to avoid, such as explosions or falling rocks.
                Reflex defense is a physical defense.
            \item Fortitude defense: Your Fortitude defense protects you from attacks you have to physically endure or resist, such as poisons and deadly spells.
                Fortitude defense is not a physical defense.
            \item Mental defense: Your Mental defense protects you from attacks you have to mentally endure or resist, such as terrifying creatures and magical manipulation.
                Mental defense is not a physical defense.
        \end{itemize}

        \subsubsection{Defense Values}\label{Defense Values}

            Each of your defenses is calculated in the following way:

            \begin{figure}[h]
                \centering 10 \add Base defense bonus or defense attribute \add size modifier \add other bonuses and penalties
            \end{figure}

            The attributes and relevant bonuses which apply to each defense are described in \trefnp{Defense Calculations}.
            Your base defense bonus for Armor defense is equal to your combat prowess.
            Your base defense bonus for Fortitude, Reflex, and Mental defense is equal to your character level.
            It does not include the defense bonuses granted by your \glossterm{base class}.

            \begin{dtable!*}
                \lcaption{Defense Calculations}
                \begin{dtabularx}{\textwidth}{l l l l l >{\lcol}X}
                    \tb{Defense Name} & \tb{Defense Bonus} & \tb{Attributes} & \tb{Body Armor Modifier} & \tb{Shield Modifier} & \tb{Size Modifier} \\
                    \hline
                    Armor defense     & Combat prowess       & Dex or Con & Yes & Yes & Yes     \\
                    Fortitude defense & Level & Con or Str & No  & No  & No      \\
                    Reflex defense    & Level    & Dex or Per & No  & Yes & Yes     \\
                    Mental defense    & Level    & Wil or Int & No  & No  & No      \\
                \end{dtabularx}
            \end{dtable!*}

            \parhead{Natural Armor} Creatures with unusually tough skin or thick hide, including most monsters, gain bonuses to their Armor defense.
            These bonuses stack with any armor such creatures might wear.
            \parhead{Size Modifiers} Your size modifier and special size modifier are described on \tref{Size in Combat}.

    \subsection{Movement and Positioning}\label{Movement and Positioning}

        \subsubsection{Taking up Space}
            A typical human takes up a 5-ft.\ by 5-ft.\ space in combat.
            For convenience, this is often called a \glossterm{square}.
            Differently sized creatures can take up more or less space, as indicated on \tref{Size in Combat}.
            Normally, other creatures can't be in any squares you occupy.

            Sometimes, movement and distance are represented in squares.
            A 30-ft.\ movement is the same thing as moving six squares.

        \subsubsection{Moving}

            When you move, you can travel a number of feet up to your speed in any direction.
            For simplicity, all movement is measured in five-foot increments.
            While it is possible to be more precise than that, it's generally not worth the complexity.

        \subsubsection{Measuring Movement}

            \parhead{Diagonals} When measuring distance, the first diagonal counts as five feet of movement, and the second counds as ten feet of movement.
            The third costs five feet, the fourth costs ten feet, and so on.
            You can move diagonally past corners and enemies.

        \subsubsection{Moving Near Foes}\label{Moving Near Foes}
            All squares threatened by any foes cost double the normal movement cost to move out of.

    \subsection{Injury, Death, and Healing}\label{Injury, Death, and Healing}

        \subsubsection{Hit Points}\label{Hit Points}
            Your hit points measure how hard you are to kill.
            No matter how many hit points you lose, your character isn't significantly hindered until your hit points drop to 0.
            When you run out of hit points, your actions are limited and you might die.

            Your hit points are equal to your \glossterm{hit value} \x your level.
            Your hit value is calculated as follows:

            \begin{figure}[h]
                \centering Half Fortitude defense or half Mental defense \add other bonuses or penalties
            \end{figure}

            \parhead{Temporary Modifiers} Temporary effects which alter your Fortitude or Mental defenses, including changes to your Constitution or Willpower, do not alter your maximum or current hit points.
            Your maximum number of hit points is determined when you gain a level, and generally does not change between levels.
            Some effects specifically modify your maximum hit points, such as the \spell{curse of blood and bone} spell.

        \subsubsection{Losing Hit Points}
            When you take lethal damage, you subtract that damage from your hit points.
            \parhead{What Hit Points Represent} Hit points represent a combination of durability, luck, divine providence, and sheer determination, depending on the nature of your character.
            When you take 10 damage from an orc with a greataxe, the axe did not literally carve into your skin without affecting your ability to fight.
            Instead, you avoided the worst of the blow, but it bruised you through your armor, the effort to dodge the blow fatigued your character, or it barely nicked you through sheer luck -- and everyone's luck runs out eventually.

        \subsubsection{Critical Damage}\label{Critical Damage}
            When you take damage while you are disabled (see \pcref{Disabled}), that damage represents serious physical injury to your body.
            This is called \glossterm{critical damage}.
            You suffer a penalty to accuracy, checks, and defenses equal to the amount of critical damage you have.

            While you have critical damage, magical healing which would normally restore hit points cannot restore your hit points, though it can stabilize you, preventing you from dying.
            In addition, if you take damage that would reduce your hit points to 0 while you have any critical damage, any excess damage from the attack is dealt directly as critical damage.

        \subsubsection{Overkill Damage}\label{Overkill Damage}
            Normally, when you take damage that would reduce your hit points to 0, any excess damage from the attack is wasted.
            However, some attacks deal such massive damage that you begin dying immediately, rather than just becoming staggered.
            If you take damage in excess of your \glossterm{bloodied} hit point total in a single round, any damage past what would reduce your hit points to 0 is dealt as \glossterm{critical damage}.

        \subsubsection{Stages of Injury}

            \parhead{Healthy} When you are above half hit points, you suffer no significant effects from losing hit points.
            If you take damage, you can become bloodied (see Bloodied, below).

            \parhead{Bloodied} When you drop to half your hit points or below, you are \bloodied.
            If you take additional damage, you can become disabled (see Disabled, below).

            \parhead{Disabled}\label{Disabled} At the end of each round, if you have no hit points remaining after resolving all other effects in the round, you become \disabled.
            While disabled, you are \staggered, and you are vulnerable to taking critical damage.

            At the end of each round you are disabled, if you have received more damage than healing, that damage becomes critical damage.
            This causes you to begin dying (see Dying, below).
            If you have received more healing than damage, you stop being disabled.

            \parhead{Dying}\label{Dying} While you are dying, you must make an attack against your own Fortitude defense at the end of every round.
            This is called a \glossterm{stabilization roll}.
            No bonuses or penalties apply to the roll, but \glossterm{critical damage} can penalize your Fortitude defense.
            If you fail to resist the attack once, you fall unconscious.
            If you fail to resist the attack three times, you die.
            If you resist the attack three times, you stabilize.

            If you receive magical healing of any kind while dying, you become partially stabilized.
            While partially stabilized, you must make an attack against your Fortitude once per minute, instead of once per round.

            An ally can make a Heal check to tend to you while you are dying.
            The Heal check result can be used in place of your Fortitude defense, although the critical damage you have taken applies as a penalty to the Heal check result as well.

            \parhead{Stable}\label{Stable}
            If you have taken critical damage but managed to stave off death, you become stable.
            As long as you have critical damage, magical healing has no effect on your hit points, though some magical effects can heal critical damage.
            If you became unconscious while dying, you regain consciousness as soon as you have hit points.

        \subsubsection{Healing}
            After taking damage, you can recover hit points through natural healing or through magical healing.
            In any case, you can't regain hit points past your full normal hit point total.

            \parhead{Natural Healing} With half an hour of rest, you recover one quarter of your hit points.
            Any significant interruption (such as combat or the like) during your rest prevents you from healing.
            If you rest for two hours, you recover all your hit points.

            \parhead{Magical Healing} Various abilities and spells can restore hit points.
            However, only certain spells can heal critical damage, as specified in the spell description.
            Unless a spell says it can cure critical damage, it cannot -- though it can still stabilize dying characters.
            Magical healing has no effect on the hit points of creatures with critical damage.

            \parhead{Healing While Disabled} While you are disabled, any healing you receive cancels out damage you receive in the same phase on a one-for-one basis.
            This can prevent you from taking critical damage if you are damaged while disabled.

            \parhead{Healing Critical Damage} Critical damage takes much longer to heal than hit point damage.
            Resting for 8 hours restores an amount of critical damage equal to 1 \add half the character's Constitution (minimum 1).
            A character can have both hit points and critical damage.
            As long as a character has critical damage, he is staggered, even if he is at full hit points.

        \subsubsection{Nonlethal Damage}\label{Nonlethal Damage}
            Some attacks and environmental effects deal nonlethal damage.
            Nonlethal damage is not subtracted from your hit points.
            Instead, it is tracked separately.
            If your nonlethal damage exceeds your hit points, you become staggered, just as if you were at 0 hit points.
            If you take additional damage while staggered, you fall unconscious.
            However, you do not begin dying unless your hit points are actually below 0.

            \parhead{Healing Nonlethal Damage}
            You heal half your hit points in nonlethal damage with 1 hour of rest.
            When a spell or a magical ability cures hit point damage, it also removes an equal amount of nonlethal damage.

    \subsection{Temporary Hit Points}\label{Temporary Hit Points}
        Certain effects give a character temporary hit points which act as a protection against damage.
        Whenever a character takes damage, if he has temporary hit points, the damage is applied to his temporary hit points first.
        Any excess damage is then applied to his hit points as normal.
        Temporary hit points are not ``real'' hit points, and cannot be healed.
        If a character has temporary hit points from multiple effects, only the highest value is used.

    \subsection{Circumstances, Bonuses, and Penalties}

        \subsubsection{Overwhelm}\label{Overwhelm}
            When a creature is being attacked by multiple foes at once, it is less able to defend itself.
            A creature is considered overwhelmed if it is being threatened by more than one creature.
            Multiple creatures occupying the same square count as a single creature when determining overwhelm penalties.
            If a creature is overwhelmed, it takes a penalty to physical defenses equal to the number of creatures threatening it.

            \parhead{Ignoring Attackers} You can freely ignore a creature attacking you.
            If you do, you are treated as being \unaware against that creature.
            In exchange, it does not contribute to overwhelm penalties against you.

        \subsubsection{Range Increments}\label{Range Increments}
            Physical ranged attacks often have a specific \glossterm{range increment}.
            A range increment is always measured in feet.
            You take a \minus2 penalty to accuracy with the ranged attack for each full range increment between you and your target.
            For example, when using a longbow with a range increment of 100 feet against a target 170 feet away, you take a \minus2 penalty to accuracy.

        \subsubsection{Size in Combat}\label{Size in Combat}
            Size affects your space and reach in combat.
            In addition, your physical attacks and defenses are affected by your size modifier.
            These effects are shown on \trefnp{Size in Combat}.

            \begin{dtable*}
                \lcaption{Size in Combat}
                \begin{dtabularx}{\textwidth}{l l l l l X}
                    \tb{Size} & \tb{Space\fn{1}} & \tb{Reach\fn{1}} & \tb{Size Modifier\fn{2}} & \tb{Special Size Modifier\fn{3}} & \tb{Example Creature} \\
                    \hline
                    Fine              & 1/2 ft.    & 0          & \plus8  & \minus16 & Fly                      \\
                    Diminutive        & 1 ft.      & 0          & \plus4  & \minus12 & Toad                     \\
                    Tiny              & 2-1/2 ft.  & 0          & \plus2  & \minus8  & Cat                      \\
                    Small             & 5 ft.      & 5 ft.      & \plus1  & \minus4  & Halfling                 \\
                    Medium            & 5 ft.      & 5 ft.      & \plus0  & \plus0   & Human                    \\
                    Large (tall)      & 10 ft.     & 10 ft.     & \minus1 & \plus4   & Ogre                     \\
                    Large (long)      & 10 ft.     & 5 ft.      & \minus1 & \plus4   & Horse                    \\
                    Huge (tall)       & 15 ft.     & 15 ft.     & \minus2 & \plus8   & Cloud giant              \\
                    Huge (long)       & 15 ft.     & 10 ft.     & \minus2 & \plus8   & Bulette                  \\
                    Gargantuan (tall) & 20 ft.     & 20 ft.     & \minus4 & \plus12  & 50-ft.\ animated statue  \\
                    Gargantuan (long) & 20 ft.     & 15 ft.     & \minus4 & \plus12  & Kraken                   \\
                    Colossal (tall)   & 30\add ft. & 30\add ft. & \minus8 & \plus16  & Colossal animated object \\
                    Colossal (long)   & 30\add ft. & 20\add ft. & \minus8 & \plus16  & Great wyrm red dragon    \\
                \end{dtabularx}
                1 Creatures can vary in space and reach.  These are simply typical values.  \\
                2 Modifies physical accuracy and defenses, except for maneuvers. \\
                3 Modifies Fortitude defense. The opposite modifier applies to Stealth.  \\
            \end{dtable*}

            Unusually large or small creatures also have other special rules apply to them, as described in \pcref{Special Size Rules}.
            In addition, larger creatures deal more damage with weapons, and smaller creatures deal less damage with weapons, as described in \tref{Weapon Damage and Size}.

        \subsubsection{Total Defense}\label{Total Defense}
            As a standard action, you can focus entirely on defense, granting you a \plus4 bonus to your defenses for 1 round.

    \subsection{Special Rules}\label{Special Rules}

        \subsubsection{Critical Success and Failure}\label{Critical Success and Failure}
            A natural 1 (the d20 comes up 1) on an attack roll is treated as rolling a \minus10.
            A natural 20 (the d20 comes up 20) is treated as rolling a 30.
            Under normal circumstances, a natural 1 automatically misses, and a natural 20 automatically hits.

        \subsubsection{Critical Hits}\label{Critical Hits}
            When you roll a natural 20 on an attack roll and hit, you have scored a critical hit.
            Roll the damage for the attack twice, including all modifiers.
            The result is the total damage for the attack.

            \parhead{Critical Multiplier} Your critical multiplier is the multiplier you add to your damage when you score a critical hit.
            Your critical multiplier is normally 2, which means your critical hits deal double damage.
            If your critical multiplier increases by 1, you deal triple damage on a critical hit (rolling the dice three times).

            \parhead{Critical Range} Your critical range is the number of die rolls that you can score a critical hit on.
            Your critical range is normally 1, which means you score a critical hit by rolling a 20.
            If your critical range increases by 1, you score a critical hit on a 19 or 20.
            Your attack must still hit to be considered a critical hit.

        \subsubsection{Unarmed Combat}\label{Unarmed Combat}
            Every creature can attack with its body using an unarmed attack.
            You are not proficient with your unarmed attack, so you are usually \defenseless while unarmed.
            In addition, an unarmed attack always deals nonlethal damage.
            You may use any appropriate part of your body to make an unarmed attack -- fists, feet, elbows, and so on.
            However, you only have one unarmed attack.
            You cannot dual-wield unarmed attacks as if you were fighting with two weapons at once (see \pcref{feat:Two-Weapon Fighting}).

            An unarmed attack is a type of natural weapon.
            Spells and abilities that affect natural weapons can affect your unarmed attack.
            Gauntlets can also be worn to increase the power of an unarmed attack (see \pcref{Unarmed Weapons}).

            If you have the Unarmed Fighting feat, you become proficient with your unarmed attack, and can deal lethal damage with it (see \featpcref{Unarmed Fighting}).

\section{Legend Points}\label{Legend Points}

    As your character gains levels, she may gain legend points.
    Legend points allow you to change fate to ensure your character succeeds.
    Certain abilities can also grant offensive or defensive legend points.

    \subsection{Using Legend Points}

        Offensive legend points can be used to reroll any attack or check your character makes.
        You use the higher of the two results.
        You may choose to reroll after knowing whether the roll succeeded or failed.

        Defensive legend points can be used to reroll any attack or check made against your character.
        You use the higher of the two results.
        You may choose to reroll after knowing whether the roll succeeded or failed.

        Legend points which are not specifically offensive or defensive are called general legend points, and can be used for either purpose.
        Using a legend point is not an action, and can be done at any time.
        You cannot use more than one legend point for any single roll.

    \subsection{Gaining Legend Points}

        You gain legend points as you gain levels, as described in \pcref{Character Advancement}.
        Magic weapons and armor can grant additional legend points, as well as certain spells.

    \subsection{Restoring Legend Points}

        At dawn each day, you regain all missing legend points, up to your maximum number of legend points of each type.
        This does not require rest or any specific action.

        It is possible to regain legend points during the day by performing extraordinary actions worthy of legends.

    \subsection{Legendary Foes}
        Some monsters and humanoid enemies you fight may have their own legend points.
        In addition, some monsters have such legendary might that they can prevent characters from using legend points near them.
        Legend points from spells and magical abilities may still be used near such creatures, but not legend points from magic items or physical abilities, including the legend points you gain by increasing your level.
        % this makes no sense but has to work that way?

\section{Character Advancement}\label{Character Advancement}

    As your character accomplishes challenges and defeats foes, he gains experience.
    If your character has enough experience, he gains a level.
    When you gain a level, you can increase your character's level in your current class or in any other class, and gain the benefits described for each class.
    Rules for taking levels in multiple classes are described in \pref{Multiclass Characters}, below.

    A character that increases in level gains additional benefits.
    \begin{itemize}
        \item At 1st level, and every 4 levels thereafter, you gain a feat (see \pcref{Feats}).
        \item At 3rd level, and every 4 levels thereafter, you gain a \glossterm{legend point}.
    \end{itemize}

    If a character has multiple classes, these benefits are gained based on the total character level, not based on class level.
    The experience required to reach a level, and the benefits gained at each level, are shown on \trefnp{Character Advancement}.

    \begin{dtable}
        \lcaption{Character Advancement}
        \begin{dtabularx}{\columnwidth}{*{3}{>{\lcol}l} >{\lcol}X}
            \tb{Level} & \tb{XP}   & \tb{Feats} & \tb{Legend Points} \\
            \hline
            1st        & 0         & 1st\fn{1}        & \tdash             \\
            2nd        & 2,000     & \tdash     & \tdash             \\
            3rd        & 5,000     & \tdash        & 1st             \\
            4th        & 9,000     & \tdash     & \tdash                \\
            5th        & 15,000    & 2nd        & \tdash            \\
            6th        & 23,000    & \tdash     & \tdash             \\
            7th        & 35,000    & \tdash        & 2nd             \\
            8th        & 51,000    & \tdash     & \tdash                \\
            9th        & 75,000    & 3rd        & \tdash             \\
            10th       & 105,000   & \tdash     & \tdash            \\
            11th       & 155,000   & \tdash        & 3rd             \\
            12th       & 220,000   & \tdash     & \tdash                \\
            13th       & 315,000   & 4th        & \tdash             \\
            14th       & 445,000   & \tdash     & \tdash             \\
            15th       & 635,000   & \tdash        & 4th            \\
            16th       & 890,000   & \tdash     & \tdash                \\
            17th       & 1,300,000 & 5th        & \tdash             \\
            18th       & 1,800,000 & \tdash     & \tdash             \\
            19th       & 2,550,000 & \tdash       & 5th             \\
            20th       & 3,600,000 & \tdash     & \tdash               \\
        \end{dtabularx}
        1. All races also grant a bonus feat at 1st level. The feat must be chosen from a specific list of racial bonus feats. \\
    \end{dtable}

        \cf{Clr}[4th]{Devotion Feat}[Su]
        The cleric chooses a feat that he meets the prerequisites for.
        As long as his devotion pool is at least half full, he gains that feat as a bonus feat.
        At each level, the cleric may change this feat to a different feat he qualifies for at his newly increased level.

        At 14th level, the cleric gains a second devotion feat.
        The cleric may not normally use these devotion feats as prerequisites for other feats or abilities.
        However, he he may use one devotion feat to meet a prerequiste for his second devotion feat.
