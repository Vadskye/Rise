\chapter{What Makes Rise Different?}
  If you have played other tabletop roleplaying games, you may wonder what makes Rise unique.
  Rise has five fundamental principles that differentiate it from other TTRPGs: minimal resource management, simultaneous combat, optional complexity, unbounded scaling, and a bounded action economy.

  \subsection{Minimal Resource Management}
    Many games make use of resources like mana, spell slots, or timed cooldowns to limit how often characters can use their abilities.
    These systems have fundamental problems that undercut the fun and flow of a TTRPG, and Rise essentially does not use resources to limit character ability usage.
    In Rise, characters can cast spells or use special attacks any number of times in a row without consuming resources.

    Some systems have resources that are designed to ebb and flow in the course of a typical combat.
    You might expend mana to use a powerful spell, and then regain mana over time by using weaker spells or fulfilling certain conditions.
    Alternately, you might use a spell and then wait some number of in-game turns before you can use that same spell again.
    This can be fiddly to track and hard to recover from if you forget what happened to your resource pool, which is why this approach is more common in video games than in TTRPGs.
    More importantly, this system has no clear way to handle ability usage outside of combat.
    It effectively gives unlimited ability usage when time is no obstacle, but only in an awkward and convoluted way.
    This category of system is unsuitable for Rise because it is too fiddly in combat and doesn't make sense out of combat.

    Some systems have finite-use resources that are tied to the expenditure of in-game time, such as taking long rests, or session breaks.
    You might spend a spell slot to use a powerful spell, and then be unable to cast that spell again until your character rests for some period of time.
    This can be manageable from a complexity perspective if the number of unique resources is small.
    However, it can get dangerously convoluted if characters have a large number of separate or partially interchangeable resource pools, such as using separate pools for individual spell levels.

    The real problem is that this limitation requires you to make your decisions based on not just the current situation, but also on your prediction of all future situations you will encounter before you have the opportunity to rest.
    This contributes significantly to the tactical complexity of deciding each individual action in combat, which slows down the pace of the game.
    It is also punishing to newer players who have less experience with the metagaming required to deduce how many resources an individual fight is worth.
    This strategic complexity is compounded if hit points are treated as an additional resource, since you now have to trade off the potential impact of one limited resource against another limited resource.

    Optimization of resource usage can be unintuitive and out of character, but failure to correctly manage your resources can leave you with no useful abilities remaining.
    This concern can be exacerbated if some characters are extremely resource-intensive while others have no meaningful resources to track.
    No one likes being forced to hide from a difficult fight or take only insignificant actions while your more resource-savvy or resource-independent allies continue using dramatic and powerful abilities.
    It can also add stress to the party dynamics when one character frequently asks for long rests after fights because they expended resources and no one else needs to rest.
    This category of system is unsuitable for Rise because it creates complexity in ways that detract from the fun and narrative of a game instead of adding to it.

    Rise does not use resources to limit normal actions in combat.
    The vast majority of spells, special martial attacks, and other abilities that affect enemies or your environment can be used any number of times.
    There are a small number of abilities with one-round cooldowns, and a universal ability that can only be used once per short rest.
    However, there is no time tracking in the system longer than ``next round''.
    Small cooldowns are a fine-grained balancing tool that allow characters to have powerful abilities which would have detrimental effects for the game if they could be used every turn.

    Rise does use a single universal resource, called ``fatigue'', that recovers based on long rests.
    This allows some opportunity for characters to invest extra effort into specific difficult fights, and to become tired after a long day.
    Normal damage taken during a fight is easily recovered after a ten minute rest.
    This means that you typically don't have to track state between fights.
    However, a GM can prevent that rest time with multiple sequential fights to increase difficulty and drama.

    Overall, Rise uses resource limitations very sparingly.
    This allows it to gain some of their benefits while avoiding the detrimental effects that come from making resource limitations a fundamental part of the system.

  \subsectiongraphic{Simultaneous Combat}{width=\columnwidth}{introduction/simultaneous combat}
    In most TTRPGs, combat takes place in a series of turns.
    When your turn comes up, you take all of your actions, and then you wait through everyone else's turn until your turn comes again.
    This system has one foundational disadvantage: it is very, very slow.
    Rise uses a simultaneous combat system that dramatically increases the pace of combat.

    Imagine a typical 4-5 player game with 1-2 enemy groups using a traditional turn-based initiative system.
    In this scenario, you have to wait through about 5 turns before it comes back to your turn.
    This number can increase significantly in large-scale fights.
    Each of those 5 or so turns can meaningfully change the battlefield situation on its own by moving, weakening, or defeating various enemies and allies.
    The state of the battlefield at the end of last turn is often drastically different than the state of that battlefield at the start of your new turn.
    Player coordination can be challenging, since they must coordinate in the specific order assigned by the initiative system, and enemy turns can intervene to ruin coordinated plans.

    In theory, every player should accurately track the unfolding battlefield state through each of the intervening turns.
    That would mean everyone would know what to do when their turn comes up.
    In practice, many players find that difficult or impossible.
    Instead, at the start of each of their turns, they ask or try to figure out how the situation has changed.
    Not everyone asks this explicitly, but it must always be analyzed anew.

    Once a player understands the current battlefield state, they can finally decide their actions.
    This typically involves both movement and any number of sequential attacks, so there are many factors to consider.
    Everyone else must wait and do nothing while this happens.
    Once the active player has decided their actions, those actions must be fully rolled and resolved before combat can proceed.
    Even the next player in the initiative order may not be able to make accurate plans during this time, since the die rolls can change those plans.
    All of this combines to make even short combats take an hour or more, and six-person adventuring groups can feel dangerously bloated.

    Rise works differently.
    Combat in Rise is broken up into two phases: the movement phase and the action phase.
    During the movement phase, all creatures move simultaneously, and no attacks are possible.
    Characters can declare certain simple reactive movements like ``stay adjacent to this enemy'' to ensure that they end up in a reasonable position regardless of enemy actions.
    If the movements of characters conflict in impossible ways, initiative checks can temporarily force a linear order of resolution.
    Each player declares their own actions in an arbitrary order as soon as they decide them, so people are not forced to wait and do nothing while slower players contemplate their choices.
    Player coordination is easy, since all actions are happening together.

    During the action phase, players resolve their actions sequentially, but in an arbitrary order of the players' choice.
    This allows slower players to make their decisions when they are ready, while allowing faster players to resolve their actions first.
    Since movement during the action phase is rare, and enemies cannot unexpectedly move, players are typically able to decide their actions much more quickly and easily even when they have a large number of unique abilities to choose from.
    Once all players have resolved their actions, they learn what their enemies did.
    Those actions all resolve simultaneously, so enemy actions cannot interrupt player actions and vice versa.
    Attackers are always responsible for rolling instead of using ``saving throws'' or similar mechanics that force defenders to roll dice.
    All of this means that players can choose and resolve their actions simultaneously and efficiently, minimizing total time spent in combat while still allowing significant tactical complexity.

    The start of each phase still requires a general assessment from all acting players about the current state of the battlefield, which takes just as much time as the assessment in a classic initiative system.
    However, the time required for this tactical analysis only increases marginally as the number of players and enemies in the game increases.
    This allows Rise to handle large player counts or large enemy hordes without becoming glacially slow.
    Combat in Rise flows by quickly, making it much easier to balance time between combat and non-combat encounters within the same game session - or to run through multiple separate, individually challenging combats without sacrificing the pace and energy of the game.

  \subsection{Optional Complexity}
    Many games operate at a consistent level of complexity.
    Many rules-light games are always simple, and many rules-dense games are always complex.
    This is a perfectly reasonable design philosophy.
    Among other benefits, it makes it easy to know what to expect from the game, which helps give the game a well-defined niche.

    Rise is designed to allow players to choose their own level of complexity.
    This broadens its potential audience by allowing people with very different play styles or tolerances for complexity to enjoy the same game together.
    This goal is manifested in several key ways in Rise's design:
    \begin{raggeditemize}
      \item Core gameplay is designed to be simple.
      \item Character creation is deeply interconnected.
      \item Complexity is not tied to narrative roles.
      \item Character power does not require complexity.
    \end{raggeditemize}

    \subsubsection{Simple Core Gameplay}
      The core gameplay loop must be simple.
      You can contribute in combat by relying on one or two standard attacks that you use in all circumstances.
      In narrative situations, you can just roll the skills you have trained, and ignore other options.
      Engaging with the system more deeply than that is a choice, not a requirement.

    \subsubsection{Interconnected Character Creation}
      Character creation and build optimization is a better place to store complexity.
      Creating a Rise character involves a number of decisions, each of which can have nuanced ramifications on other aspects of the system.
      If you are just trying to build a character that matches a desired narrative, you can generally approach each decision in isolation.

      For example, you can decide that your character is intelligent and agile but not very strong or durable, because that is the concept you want.
      That decision has consequences, such as changing how many trained skills you have and what your defenses are.
      If you approach each decision sequentially, each one is relatively easy to make, and doesn't require deep system knowledge.
      On the other hand, trying to mathematically optimize a character requires thinking about many aspects of the system at once.
      This results in a system that is easy to learn but hard to master.

      Even for simple characters, the process of character creation is still one of the most complicated aspects of Rise.
      That is why Rise provides (or will provide, once that section is done) an extensive selection of premade characters for a wide variety of narrative archetypes.
      Each premade character includes advice for how to play that character and level them up.
      The premade characters make the system more accessible to people who don't want to to deal with the complexity of creating a character from scratch.

    \subsubsection{Complexity and Narrative}
      Complexity and simplicity should not be directly connected to a character's concept or narrative.`
      For example, it would be a bad idea to define a system where martial characters are simple and spellcasters are complicated.
      Both of those are rich and evocative narrative constructs.
      Many people who don't enjoy complexity will want to play spellcasters, and many people who enjoy complexity will want to play martial characters.
      Gameplay complexity must be more finely tuned and localized than those sweeping strokes.

      In Rise, gameplay complexity is generally generated by acquiring a large number of increasingly situational abilities.
      Every class has some archetypes that grant additional abilities known and some archetypes that grant additional passive abilities.
      If you like having a lot of unique abilities, you can have a high Intelligence to maximize your insight points, and focus on learning spells and maneuvers that attack your enemies or have situational effects.
      If you like minimizing complexity, you can instead choose archetypes or learn spells that simply grant you passive benefits, and focus on one or two standard attacks that you specialize in.
      Some feats give you new abilities and new circumstances to pay attention to that make you more effective, while others simply increase your passive statistics and defenses.

      Rise specifically handles complexity for martial characters and spellcasters slightly differently.
      Martial characters in Rise typically have fairly simple individual abilities.
      However, they can use those abilities with a variety of meaningfully different weapons.
      A martial character with four unique attacks and three different weapons has twelve different options in combat.
      In addition, martial characters can typically make better use of universal abilities, such as shoving and grappling.

      Spellcasters have more complex and varied individual abilities.
      They also tend to have more abilities that have significant narrative effects.
      However, their abilities are more isolated.
      There is no spellcaster equivalent of martial weapons that would multiply their number of distinct abilities in combat.
      The result of this design is that both martial characters and spellcasters can be very simple or very complicated.
      However, they approach complexity in different ways, ensuring that they feel narratively distinct.

    \subsubsection{Complexity and Power}
      All of this customization of complexity would be mostly pointless if complexity was strongly correlated with character power.
      If exceptionally complicated or hyper-specialized characters were obviously and consistently more effective than other characters, it would push everyone to use those characters.
      Rise structures the tradeoffs between gaining raw power and gaining additional options balanced enough that neither is always superior.

      There will always be some benefit from build optimization and system mastery.
      Players who are deeply familiar with Rise will be able to build characters with more relevant strengths and fewer relevant weaknesses.
      However, the gap between optimized characters and ``normal'' characters is limited.
      There will always be specific contexts where one character's mechanics are superior to another's.
      For example, a specialized defensive melee character may excel in a duel in a confined space.
      However, it may be irrelevant against cavalry archers on an open field.
      Characters in Rise cannot drastically change their capabilities each day, so they will always have moments to shine and moments of weakness.

  \subsectiongraphic{Unbounded Scaling}{width=\columnwidth}{introduction/unbounded scaling}
    Some systems uses bounded bonuses for accuracy or other game statistics.
    Bounded scaling means that every character of the same power level - or in some systems, of any power level - has a similar chance of success with any given skill check or attack roll.
    This can frequently cause narratively inappropriate and even comical events, and Rise explicitly rejects this philosophy.

    Imagine a typical party of four players, with one character being exceptionally skilled at a particular task.
    Perhaps the rogue is exceptionally skilled at lying, or a barbarian is exceptionally skilled at climbing.
    If ``exceptionally skilled'' only means that they have a \plus5 bonus on a d20 compared to \plus0 from the rest of the party, the exceptionally skilled character will only get the best result in the party half the time.
    The other half of the time, some other character with no relevant skills will meet or exceed the skilled character's result - sometimes by a dramatic margin.
    When failure compared to rank amateurs happens this often, it becomes hard to take seriously the idea that any character can be exceptionally skilled at anything.

    Rise characters can have dramatic statistical differences between each other, even at low levels.
    It uses a d10 as the fundamental die, which makes every bonus more significant.
    In addition, a 1st-level character can easily reach a \plus6 bonus with a skill check that is particularly relevant to their character.
    This means that a skilled character can beat a party of rank amateurs 80\% of the time, and at higher levels their success becomes completely guaranteed.
    Likewise, the difference in Mental defense between a powerful sorcerer and a cowardly rogue can allow mind-affecting attacks to almost always hit a rogue while almost never hitting the sorcerer.
    These statistical differences do not always grow with level, but they remain significant at every level.

    One advantage of systems with bounded scaling is that it is easier to guarantee that every character is relevant in any situation.
    Even if your character has no useful abilities of any kind, you might sometimes succeed on important actions through sheer luck.
    However, this design philosophy often breaks the symmetry between magical and non-magical characters.
    Magical characters can often use extremely specific and powerful abilities that are impossible for nonmagical characters to duplicate.
    If magical characters also have similar odds of success with all generic mechanics of the game, they will almost certainly have far more influence over the narrative of the game than any nonmagical character can hope to match.

    The philosophy of Rise is that it's okay for some characters to be irrelevant in specific contexts.
    It's good to give people time in the spotlight where their character's abilities help solve the specific problem that the group is facing when no other character could.
    Rise encourages that, and makes it impossible for one character to be relevant in \textit{all} contexts.
    Each character has their own strengths and weaknesses, and if you try to be good at everything, you'll fall behind people who specialize in a particular area.
    This will naturally rotate the spotlight between different characters, allowing each player to feel relevant and important in turn.

    This dramatic scaling is also used to govern the power of characters over time, in addition to the power of characters relative to each other.
    Rise attempts to model a massive power range for player characters.
    They are expected to start their journeys at level 1 as little more than commoners, and by level 21 they are effectively demigods who can alter the fate of entire worlds.
    This is a critical part of the narrative fabric of Rise, and it is reflected in the statistics and abilities of characters.
    If a level 1 kobold posed even a tiny threat to a level 21 character, the mechanics of the game would sabotage the purported narrative of power and growth.
    In Rise, overall character power doubles approximately every two to three levels.
    The system takes some care to avoid bloating numbers to unwieldy levels on this journey, and the use of the d10 as the standard die helps immensely.

  \subsection{Bounded Action Economy}
    It is dangerous to to give characters too many actions each turn.
    Each additional action a character can take increases how difficult it is for a player to decide what to do on their turn.
    In addition, each additional action increases the complexity of the change between the start of the turn and the end of the turn.
    This is especially risky with Rise's simultaneous initiative system, which combines the actions taken by all characters into a single resolution process.

    Rise places significant limitations on how many relevant actions each character can take on their turn.
    Generally, characters can only move during the movement phase and then take one significant action each turn.
    Some characters can use a minor action to accomplish something useful.
    However, that essentially marks the end of action economy scaling, even up to the maximum level.

    Detrimental effects that could deny actions are also heavily limited.
    Total action denial effects are only usable by high level characters, and even then they only work against weak enemies or enemies that have already been significantly damaged.
    Taking actions is fun, and sitting quietly while everyone else does things can be very frustrating.
    Similarly, completely removing an enemy's ability to act can easily remove the tension from a fight before it's actually over.
