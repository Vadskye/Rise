\chapter{Magic Items}

Magic items are objects that have been imbued with magical energy. They can take almost any form, and their potential uses are only as limited as the magic that created them.

\section{Magic Item Types}
    Magic items are divided into four broad categories:
    \begin{itemize}
        \item Weapons are used to make physical attacks. They provide access to their abilities when wielded.
            A \mitem{flaming longsword} and a \mitem{vampiric scythe} are weapons.
        \item Implements are used to cast spells. They provide access to their abilities when wielded.
            A \mitem{staff of fire} and a \mitem{staff of time} are implements.
        \item Apparel items are usually not used individually. They provide access to their abilities when worn.
            A \mitem{flaming burst full plate} and a \mitem{ring of protection} are apparel items.
        \item Tools provide access to their abilities when used in some way.
            A \mitem{bag of carrying} is a tool.
    \end{itemize}

\section{Using Magic Items}

    \subsection{Item Activation}

        Some magic items have to be explicitly activated to have unusual effects.
        For example, a \mitem{healing belt} can be activated to heal you.
        Other magic items constantly have magical effects.
        For example, a \mitem{flaming longsword} is on fire.

        The description of a magic item effect will specify what mechanical actions must be taken, if any, to activate the effects of the item.
        For example, a healing belt requires spending an \glossterm{action point} as a \glossterm{standard action}.
        However, the item description will not specify the exact nature of the action.
        Different items, even if they have the same effect, can have different physical actions that are required to activate the item.
        These minor actions can come in one of the following forms:
        \begin{itemize}
           \item Command word: You must speak a specific word that the item will hear and react to.
                For example, you may need to say the word ``healing'' in Elven to activate a \mitem{belt of healing}.
            \item Mental command: You must mentally direct the item to activate, such as by visualizing the item or thinking a particular word.
                % TODO: does this item exist
                For example, you may need to imagine a warm blanket around you to activate a \mitem{ring of elemental endurance}.
            \item Physical motion: You must perform a specific physical motion, usually involving the item in some way.
                For example, you may need to rapidly stomp one foot on the ground to activate \mitem{boots of speed}.
        \end{itemize}

        % TODO: table of random item activations?

    \subsection{Item Limitations}

        There are two restrictions on your ability to use magic items.
        First, you cannot equip two apparel items that take up the same physical location on your body.
        For example, you cannot equip two different gauntlet sets and gain the effects of both, but you could equip several amulets or up to ten rings.
        Second, most magic items other than tools require you to attune to them to gain their effect.
        You can attune to a magic item with the \textit{item attunement} ability, below.

        \subsubsection{Item Slots}
            Normally, attuning to an ability or magic item reduces your maximum action points by one.
            However, you can attune to a certain number of items without penalizing your action points.
            You start with two \glossterm{item slots}, and acquire more as you gain levels (see \tref{Character Advancement}).

            When you attune to a magic item, you can spend an \glossterm{item slot} instead of an \glossterm{action point}.
            If you do, your maximum action points are not affected by your attunement.
            This is summarized in the description of the \textit{item attunement} ability.

            Dismissing your attunement to an item does not restore the spent item slot.
            You regain all spent \glossterm{item slots} from dismissed attunements after a \glossterm{long rest}.

        \subsubsection{Item Attunement}
            As a standard action, you can spend an \glossterm{action point} or an \glossterm{item slot} to use this ability.
            \begin{ability}
                \begin{spelltargetinginfo}
                    \spellquicktargeting{One magic item}{\rngtouch}
                \end{spelltargetinginfo}
                \begin{spelleffects}
                    \spelleffect Any abilities the target has that require attunement become active, allowing you to use its full potential.
                    \spelldur Attunement. If you spent an \glossterm{item slot} to use this ability, attuning to this ability does not reduce your maximum \glossterm{action points}.
                \end{spelleffects}
            \end{ability}

    \subsubsection{Item Power}
        The \glossterm{power} of an item depends on its level.
        If the item is not being attuned to by a creature, its power is equal to its level.
        If it is being attuned to, its power is equal to its level or the level of the attuning creature, whichever is higher.

        An item's \glossterm{power} also affects its defenses.
        Its Fortitude and Mental defenses are equal to 10 \add its \glossterm{power}.
        Its Armor defense and Reflex defense are not affected by its \glossterm{power}, and are solely determined by its size and shape.

    \subsection{Removing Magic Items}
        Unless otherwise noted, magic items that have effects on the creature using the item must continue to be worn or held as long as the effect lasts.
        If a magic item is removed before its duration ends the effect ends.
        Items which are consumed when used or which do not affect their user are unaffected by this rule.

\section{Item Description Format}
    TODO

\section{Apparel}

    \lowercase{\hypertarget{item:Amulet of Mighty Fists}{}}\label{item:Amulet of Mighty Fists}
\hypertarget{item:Amulet of Mighty Fists}{\subsubsection{Amulet of Mighty Fists\hfill\nth{6}}}
You gain a \plus1d bonus to \glossterm{strike damage} with \glossterm{unarmed attacks} and natural weapons.
\parhead*{Tags} \glossterm{Enhancement}
\parhead*{Materials} Jewelry
\lowercase{\hypertarget{item:Amulet of Mighty Fists, Greater}{}}\label{item:Amulet of Mighty Fists, Greater}
\hypertarget{item:Amulet of Mighty Fists, Greater}{\subsubsection{Amulet of Mighty Fists, Greater\hfill\nth{14}}}
You gain a \plus2d bonus to \glossterm{strike damage} with \glossterm{unarmed attacks} and natural weapons.
\parhead*{Tags} \glossterm{Enhancement}
\parhead*{Materials} Jewelry
\lowercase{\hypertarget{item:Armor of Energy Resistance}{}}\label{item:Armor of Energy Resistance}
\hypertarget{item:Armor of Energy Resistance}{\subsubsection{Armor of Energy Resistance\hfill\nth{4}}}
You have \glossterm{damage reduction} equal to the item's \glossterm{power} against \glossterm{energy damage}.
Whenever you resist energy with this item, it sheds light as a torch until the end of the next round.
The color of the light depends on the energy damage resisted: blue for cold, yellow for electricity, red for fire, and brown for sonic.
\parhead*{Tags} \glossterm{Shielding}
\parhead*{Materials} Bone, metal
\lowercase{\hypertarget{item:Armor of Energy Resistance, Greater}{}}\label{item:Armor of Energy Resistance, Greater}
\hypertarget{item:Armor of Energy Resistance, Greater}{\subsubsection{Armor of Energy Resistance, Greater\hfill\nth{12}}}
This item functions like the \mitem{armor of energy resistance} item, except that the damage reduction is equal to twice the item's \glossterm{power}.
\parhead*{Tags} \glossterm{Shielding}
\parhead*{Materials} Bone, metal
\lowercase{\hypertarget{item:Armor of Fortification}{}}\label{item:Armor of Fortification}
\hypertarget{item:Armor of Fortification}{\subsubsection{Armor of Fortification\hfill\nth{7}}}
You gain a \plus5 bonus to defenses when determining whether a \glossterm{strike} gets a \glossterm{critical hit} against you instead of a normal hit.
\parhead*{Tags} \glossterm{Imbuement}
\parhead*{Materials} Bone, metal
\lowercase{\hypertarget{item:Armor of Fortification, Greater}{}}\label{item:Armor of Fortification, Greater}
\hypertarget{item:Armor of Fortification, Greater}{\subsubsection{Armor of Fortification, Greater\hfill\nth{15}}}
This item functions like the \mitem{armor of fortification} item, except that the bonus increases to \plus10.
\parhead*{Tags} \glossterm{Imbuement}
\parhead*{Materials} Bone, metal
\lowercase{\hypertarget{item:Armor of Fortification, Mystic}{}}\label{item:Armor of Fortification, Mystic}
\hypertarget{item:Armor of Fortification, Mystic}{\subsubsection{Armor of Fortification, Mystic\hfill\nth{12}}}
This item functions like the \mitem{armor of fortification} item, except that it applies against all attacks instead of only against; \glossterm{strikes}.
\parhead*{Tags} \glossterm{Imbuement}
\parhead*{Materials} Bone, metal
\lowercase{\hypertarget{item:Armor of Invulnerability}{}}\label{item:Armor of Invulnerability}
\hypertarget{item:Armor of Invulnerability}{\subsubsection{Armor of Invulnerability\hfill\nth{8}}}
You have \glossterm{damage reduction} equal to this item's \glossterm{power} against damage from \glossterm{physical attacks}.
\parhead*{Tags} \glossterm{Shielding}
\parhead*{Materials} Bone, metal
\lowercase{\hypertarget{item:Armor of Invulnerability, Greater}{}}\label{item:Armor of Invulnerability, Greater}
\hypertarget{item:Armor of Invulnerability, Greater}{\subsubsection{Armor of Invulnerability, Greater\hfill\nth{16}}}
This item functions like the \mitem{armor of invulnerability} item, except that the damage reduction is equal to twice the item's \glossterm{power}.
You have \glossterm{damage reduction} equal to the item's \glossterm{power} against damage from \glossterm{physical attacks}.
\parhead*{Tags} \glossterm{Shielding}
\parhead*{Materials} Bone, metal
\lowercase{\hypertarget{item:Armor of Magic Resistance}{}}\label{item:Armor of Magic Resistance}
\hypertarget{item:Armor of Magic Resistance}{\subsubsection{Armor of Magic Resistance\hfill\nth{14}}}
You have \glossterm{magic resistance} equal to 5 + the item's \glossterm{power}.
\parhead*{Tags} \glossterm{Shielding}
\parhead*{Materials} Bone, metal
\lowercase{\hypertarget{item:Assassin's Cloak}{}}\label{item:Assassin's Cloak}
\hypertarget{item:Assassin's Cloak}{\subsubsection{Assassin's Cloak\hfill\nth{7}}}
At the end of each round, if you took no actions that round, you become \glossterm{invisible} until the end of the next round.
\parhead*{Tags} \glossterm{Glamer}
\parhead*{Materials} Textiles
\lowercase{\hypertarget{item:Assassin's Cloak, Greater}{}}\label{item:Assassin's Cloak, Greater}
\hypertarget{item:Assassin's Cloak, Greater}{\subsubsection{Assassin's Cloak, Greater\hfill\nth{17}}}
At the end of each round, if you did not attack a creature that round, you become \glossterm{invisible} until the end of the next round.
\parhead*{Tags} \glossterm{Glamer}
\parhead*{Materials} Textiles
\lowercase{\hypertarget{item:Astral Boots}{}}\label{item:Astral Boots}
\hypertarget{item:Astral Boots}{\subsubsection{Astral Boots\hfill\nth{16}}}
Whenever you move, you can teleport the same distance instead.
This does not change the total distance you can move, but you can teleport in any direction, even vertically.
You cannot teleport to locations you do not have \glossterm{line of sight} and \glossterm{line of effect} to.
\parhead*{Tags} \glossterm{Teleportation}
\parhead*{Materials} Bone, leather, metal
\lowercase{\hypertarget{item:Belt of Healing}{}}\label{item:Belt of Healing}
\hypertarget{item:Belt of Healing}{\subsubsection{Belt of Healing\hfill\nth{1}}}
When you use the \textit{recover} action, you heal \plus1d hit points.
\parhead*{Tags} \glossterm{Life}
\parhead*{Materials} Leather, textiles
\lowercase{\hypertarget{item:Belt of Healing, Greater}{}}\label{item:Belt of Healing, Greater}
\hypertarget{item:Belt of Healing, Greater}{\subsubsection{Belt of Healing, Greater\hfill\nth{8}}}
When you use the \textit{recover} action, you heal \plus2d hit points.
\parhead*{Tags} \glossterm{Life}
\parhead*{Materials} Leather, textiles
\lowercase{\hypertarget{item:Belt of Heroic Recovery}{}}\label{item:Belt of Heroic Recovery}
\hypertarget{item:Belt of Heroic Recovery}{\subsubsection{Belt of Heroic Recovery\hfill\nth{6}}}
% TODO: timing?
As an \glossterm{immediate action} when you get a \glossterm{critical hit}, you can take the \textit{recover} action.
\parhead*{Tags} \glossterm{Life}
\parhead*{Materials} Leather, textiles
\lowercase{\hypertarget{item:Boots of Earth's Embrace}{}}\label{item:Boots of Earth's Embrace}
\hypertarget{item:Boots of Earth's Embrace}{\subsubsection{Boots of Earth's Embrace\hfill\nth{4}}}
While you are standing on solid ground, you are immune to effects that would force you to move.
This does not protect you from other effects of those attacks, such as damage.
\parhead*{Tags} \glossterm{Earth}, \glossterm{Enhancement}
\parhead*{Materials} Bone, leather, metal
\lowercase{\hypertarget{item:Boots of Freedom}{}}\label{item:Boots of Freedom}
\hypertarget{item:Boots of Freedom}{\subsubsection{Boots of Freedom\hfill\nth{6}}}
You are immune to effects that restrict your mobility.
This removes all penalties you would suffer for acting underwater, except for those relating to using ranged weapons.
This does not prevent you from being \grappled, but you gain a \plus10 bonus to your defense against \glossterm{grapple} attacks.
\parhead*{Tags} \glossterm{Imbuement}
\parhead*{Materials} Bone, leather, metal
\lowercase{\hypertarget{item:Boots of Freedom, Greater}{}}\label{item:Boots of Freedom, Greater}
\hypertarget{item:Boots of Freedom, Greater}{\subsubsection{Boots of Freedom, Greater\hfill\nth{12}}}
These boots function like \mitem{boots of freedom}, except that you are also immune to being \grappled.
\parhead*{Tags} \glossterm{Imbuement}
\parhead*{Materials} Bone, leather, metal
\lowercase{\hypertarget{item:Boots of Gravitation}{}}\label{item:Boots of Gravitation}
\hypertarget{item:Boots of Gravitation}{\subsubsection{Boots of Gravitation\hfill\nth{8}}}
While these boots are within 5 feet of a solid surface, gravity pulls you towards the solid surface closest to your boots rather than in the normal direction.
This can allow you to walk easily on walls or even ceilings.
\parhead*{Tags} \glossterm{Imbuement}
\parhead*{Materials} Bone, leather, metal
\lowercase{\hypertarget{item:Boots of Speed}{}}\label{item:Boots of Speed}
\hypertarget{item:Boots of Speed}{\subsubsection{Boots of Speed\hfill\nth{5}}}
You gain a \plus10 foot bonus to your speed in all your movement modes, up to a maximum of double your normal speed.
\parhead*{Tags} \glossterm{Temporal}
\parhead*{Materials} Bone, leather, metal
\lowercase{\hypertarget{item:Boots of Speed, Greater}{}}\label{item:Boots of Speed, Greater}
\hypertarget{item:Boots of Speed, Greater}{\subsubsection{Boots of Speed, Greater\hfill\nth{13}}}
You gain a \plus30 foot bonus to your speed in all your movement modes, up to a maximum of double your normal speed.
\parhead*{Tags} \glossterm{Temporal}
\parhead*{Materials} Bone, leather, metal
\lowercase{\hypertarget{item:Boots of Water Walking}{}}\label{item:Boots of Water Walking}
\hypertarget{item:Boots of Water Walking}{\subsubsection{Boots of Water Walking\hfill\nth{7}}}
You treat the surface of all liquids as if they were firm ground.
Your feet hover about an inch above the liquid's surface, allowing you to traverse dangerous liquids without harm as long as the surface is calm.
If you are below the surface of the liquid, you rise towards the surface at a rate of 60 feet per round.
Thick liquids, such as mud and lava, may cause you to rise more slowly.
\parhead*{Tags} \glossterm{Imbuement}
\parhead*{Materials} Bone, leather, metal
\lowercase{\hypertarget{item:Boots of the Winterlands}{}}\label{item:Boots of the Winterlands}
\hypertarget{item:Boots of the Winterlands}{\subsubsection{Boots of the Winterlands\hfill\nth{2}}}
You can travel across snow and ice without slipping or suffering movement penalties for the terrain.
% TODO: degree symbol?
In addition, the boots keep you warn, protecting you in environments as cold as \minus50 Fahrenheit.
\parhead*{Tags} \glossterm{Enhancement}
\parhead*{Materials} Bone, leather, metal
\lowercase{\hypertarget{item:Bracers of Archery}{}}\label{item:Bracers of Archery}
\hypertarget{item:Bracers of Archery}{\subsubsection{Bracers of Archery\hfill\nth{1}}}
You are proficient with bows.
\parhead*{Tags} \glossterm{Enhancement}
\parhead*{Materials} Bone, leather, metal, wood
\lowercase{\hypertarget{item:Bracers of Armor}{}}\label{item:Bracers of Armor}
\hypertarget{item:Bracers of Armor}{\subsubsection{Bracers of Armor\hfill\nth{2}}}
You gain a \plus2 bonus to Armor defense.
The protection from these bracers is treated as body armor, and it does not stack with any other body armor you wear.
\parhead*{Tags} \glossterm{Shielding}
\parhead*{Materials} Bone, leather, metal, wood
\lowercase{\hypertarget{item:Bracers of Repulsion}{}}\label{item:Bracers of Repulsion}
\hypertarget{item:Bracers of Repulsion}{\subsubsection{Bracers of Repulsion\hfill\nth{4}}}
Whenever a creature hits you with a melee \glossterm{strike} during the \glossterm{action phase},
you can spend an \glossterm{action point} to use this item as an \glossterm{immediate action}.
If you do, you make a \glossterm{shove} attack against that creature during the \glossterm{delayed action phase}, using this item's power in place of your Strength.
\parhead*{Tags} \glossterm{Telekinesis}
\parhead*{Materials} Bone, leather, metal, wood
\lowercase{\hypertarget{item:Bracers of Repulsion, Greater}{}}\label{item:Bracers of Repulsion, Greater}
\hypertarget{item:Bracers of Repulsion, Greater}{\subsubsection{Bracers of Repulsion, Greater\hfill\nth{11}}}
This item functions like the \mitem{bracers of repulsion} item, except that it does not cost an action point to use.
\parhead*{Tags} \glossterm{Telekinesis}
\parhead*{Materials} Bone, leather, metal, wood
\lowercase{\hypertarget{item:Cloak of Mist}{}}\label{item:Cloak of Mist}
\hypertarget{item:Cloak of Mist}{\subsubsection{Cloak of Mist\hfill\nth{8}}}
Fog constantly fills an \areamed radius emanation from you.
This fog does not fully block sight, but it provides \concealment.
If a 5-foot square of fog takes fire damage equal to half this item's \glossterm{power}, the fog disappears from that area until the end of the next round.
\parhead*{Tags} \glossterm{Fog}, \glossterm{Manifestation}
\parhead*{Materials} Textiles
\lowercase{\hypertarget{item:Cloak of Mist, Greater}{}}\label{item:Cloak of Mist, Greater}
\hypertarget{item:Cloak of Mist, Greater}{\subsubsection{Cloak of Mist, Greater\hfill\nth{16}}}
A thick fog constantly fills an \areamed radius emanation from you.
This fog completely blocks sight beyond 10 feet.
Within that range, it still provides \concealment.
If a 5-foot square of fog takes fire damage equal to this item's \glossterm{power}, the fog disappears from that area until the end of the next round.
\parhead*{Tags} \glossterm{Fog}, \glossterm{Manifestation}
\parhead*{Materials} Textiles
\lowercase{\hypertarget{item:Crown of Flame}{}}\label{item:Crown of Flame}
\hypertarget{item:Crown of Flame}{\subsubsection{Crown of Flame\hfill\nth{5}}}
This crown is continuously on fire.
The flame sheds light as a torch.
You and all allies within an \arealarge radius emanation from you are immune to fire damage.
\parhead*{Tags} \glossterm{Fire}
\parhead*{Materials} Bone, metal
\lowercase{\hypertarget{item:Crown of Frost}{}}\label{item:Crown of Frost}
\hypertarget{item:Crown of Frost}{\subsubsection{Crown of Frost\hfill\nth{11}}}
At the end of each \glossterm{action phase}, you make a Power vs. Fortitude attack against all enemies within an \areamed radius emanation from you.
A hit deals cold \glossterm{standard damage} \minus3d.
Each creature that takes damage in this way is \fatigued until the end of the next round.
\parhead*{Tags} \glossterm{Cold}
\parhead*{Materials} Bone, metal
\lowercase{\hypertarget{item:Crown of Lightning}{}}\label{item:Crown of Lightning}
\hypertarget{item:Crown of Lightning}{\subsubsection{Crown of Lightning\hfill\nth{7}}}
This crown continuously crackles with electricity.
The constant sparks shed light as a torch.
At the end of each \glossterm{action phase}, you make a Power vs. Reflex attack against all enemies within an \areamed radius emanation from you.
A hit deals electricity \glossterm{standard damage} \minus3d.
\parhead*{Tags} \glossterm{Electricity}
\parhead*{Materials} Bone, metal
\lowercase{\hypertarget{item:Crown of Thunder}{}}\label{item:Crown of Thunder}
\hypertarget{item:Crown of Thunder}{\subsubsection{Crown of Thunder\hfill\nth{9}}}
The crown constantly emits a low-pitched rumbling.
To you and your allies, the sound is barely perceptible.
However, all enemies within an \arealarge radius emanation from you hear the sound as a deafening, continuous roll of thunder.
The noise blocks out all other sounds quieter than thunder, causing them to be \deafened while they remain in the area and until the end of the next round after they leave.
\parhead*{Tags} \glossterm{Sonic}
\parhead*{Materials} Bone, metal
\lowercase{\hypertarget{item:Featherlight Armor}{}}\label{item:Featherlight Armor}
\hypertarget{item:Featherlight Armor}{\subsubsection{Featherlight Armor\hfill\nth{4}}}
This armor's \glossterm{encumbrance penalty} is reduced by 2.
\parhead*{Tags} \glossterm{Enhancement}
\parhead*{Materials} Bone, metal
\lowercase{\hypertarget{item:Featherlight Armor, Greater}{}}\label{item:Featherlight Armor, Greater}
\hypertarget{item:Featherlight Armor, Greater}{\subsubsection{Featherlight Armor, Greater\hfill\nth{10}}}
This armor's \glossterm{encumbrance penalty} is reduced by 4.
\parhead*{Tags} \glossterm{Enhancement}
\parhead*{Materials} Bone, metal
\lowercase{\hypertarget{item:Gauntlet of the Ram}{}}\label{item:Gauntlet of the Ram}
\hypertarget{item:Gauntlet of the Ram}{\subsubsection{Gauntlet of the Ram\hfill\nth{2}}}
If you hit on a \glossterm{strike} with this gauntlet during the \glossterm{action phse}, you can attempt to \glossterm{shove} your foe during the \glossterm{delayed action phase}.
Making a strike with this gauntlet is equivalent to an \glossterm{unarmed attack}.
You do not need to move with your foe to push it back the full distance.
\parhead*{Tags} \glossterm{Telekinesis}
\parhead*{Materials} Bone, metal, wood
\lowercase{\hypertarget{item:Gauntlet of the Ram, Greater}{}}\label{item:Gauntlet of the Ram, Greater}
\hypertarget{item:Gauntlet of the Ram, Greater}{\subsubsection{Gauntlet of the Ram, Greater\hfill\nth{7}}}
This item functions like the \mitem{gauntlet of the ram}, except that you gain a bonus to the \glossterm{shove} attack equal to the damage you dealt with the \glossterm{strike}.
\parhead*{Tags} \glossterm{Telekinesis}
\parhead*{Materials} Bone, metal, wood
\lowercase{\hypertarget{item:Gauntlets of Improvisation}{}}\label{item:Gauntlets of Improvisation}
\hypertarget{item:Gauntlets of Improvisation}{\subsubsection{Gauntlets of Improvisation\hfill\nth{2}}}
You gain a \plus1d bonus to damage with \glossterm{improvised weapons}.
\parhead*{Tags} \glossterm{Enhancement}
\parhead*{Materials} Bone, metal, wood
\lowercase{\hypertarget{item:Gauntlets of Improvisation, Greater}{}}\label{item:Gauntlets of Improvisation, Greater}
\hypertarget{item:Gauntlets of Improvisation, Greater}{\subsubsection{Gauntlets of Improvisation, Greater\hfill\nth{7}}}
This item functions like the \mitem{gauntlets of improvisation}, except that the damage bonus is increased to \plus2d.
\parhead*{Tags} \glossterm{Enhancement}
\parhead*{Materials} Bone, metal, wood
\lowercase{\hypertarget{item:Greatreach Bracers}{}}\label{item:Greatreach Bracers}
\hypertarget{item:Greatreach Bracers}{\subsubsection{Greatreach Bracers\hfill\nth{9}}}
Your \glossterm{reach} is increased by 5 feet.
\parhead*{Tags} \glossterm{Imbuement}
\parhead*{Materials} Bone, leather, metal, wood
\lowercase{\hypertarget{item:Greatreach Bracers, Greater}{}}\label{item:Greatreach Bracers, Greater}
\hypertarget{item:Greatreach Bracers, Greater}{\subsubsection{Greatreach Bracers, Greater\hfill\nth{17}}}
Your \glossterm{reach} is increased by 10 feet.
\parhead*{Tags} \glossterm{Imbuement}
\parhead*{Materials} Bone, leather, metal, wood
\lowercase{\hypertarget{item:Hexproof Cloak}{}}\label{item:Hexproof Cloak}
\hypertarget{item:Hexproof Cloak}{\subsubsection{Hexproof Cloak\hfill\nth{18}}}
All \glossterm{magical} abilities that target you directly fail to affect you.
This does not protect you from abilities that affect an area.
\parhead*{Tags} \glossterm{Thaumaturgy}
\parhead*{Materials} Textiles
\lowercase{\hypertarget{item:Hexward Cloak}{}}\label{item:Hexward Cloak}
\hypertarget{item:Hexward Cloak}{\subsubsection{Hexward Cloak\hfill\nth{10}}}
You gain a \plus5 bonus to defenses against \glossterm{magical} abilities that target you directly.
This does not protect you from abilities that affect an area.
\parhead*{Tags} \glossterm{Thaumaturgy}
\parhead*{Materials} Textiles
\lowercase{\hypertarget{item:Hidden Armor}{}}\label{item:Hidden Armor}
\hypertarget{item:Hidden Armor}{\subsubsection{Hidden Armor\hfill\nth{4}}}
As a standard action, you can use this item.
If you do, it appears to change shape and form to assume the shape of a normal set of clothing.
You may choose the design of the clothing.
The item retains all of its properties, including weight and sound, while disguised in this way.
Only its visual appearance is altered.
Alternately, you may return the armor to its original appearance.
\parhead*{Tags} \glossterm{Glamer}
\parhead*{Materials} Bone, metal
\lowercase{\hypertarget{item:Hidden Armor, Greater}{}}\label{item:Hidden Armor, Greater}
\hypertarget{item:Hidden Armor, Greater}{\subsubsection{Hidden Armor, Greater\hfill\nth{9}}}
This item functions like the \mitem{hidden armor} item, except that the item also makes sound appropriate to its disguised form while disguised.
\parhead*{Tags} \glossterm{Alteration}
\parhead*{Materials} Bone, metal
\lowercase{\hypertarget{item:Mask of Air}{}}\label{item:Mask of Air}
\hypertarget{item:Mask of Air}{\subsubsection{Mask of Air\hfill\nth{9}}}
If you breathe through this mask, you breathe in clean, fresh air, regardless of your environment.
This can protect you from inhaled poisons and similar effects.
\parhead*{Tags} \glossterm{Imbuement}
\parhead*{Materials} Textiles
\lowercase{\hypertarget{item:Mask of Water Breathing}{}}\label{item:Mask of Water Breathing}
\hypertarget{item:Mask of Water Breathing}{\subsubsection{Mask of Water Breathing\hfill\nth{4}}}
You can breathe water through this mask as easily as a human breaths air.
This does not grant you the ability to breathe other liquids.
\parhead*{Tags} \glossterm{Imbuement}
\parhead*{Materials} Textiles
\lowercase{\hypertarget{item:Ring of Elemental Endurance}{}}\label{item:Ring of Elemental Endurance}
\hypertarget{item:Ring of Elemental Endurance}{\subsubsection{Ring of Elemental Endurance\hfill\nth{2}}}
You can exist comfortably in conditions between \minus50 and 140 degrees Fahrenheit without any ill effects.
You suffer the normal penalties in temperatures outside of that range.
\parhead*{Tags} \glossterm{Shielding}
\parhead*{Materials} Bone, jewelry, metal, wood
\lowercase{\hypertarget{item:Ring of Energy Resistance}{}}\label{item:Ring of Energy Resistance}
\hypertarget{item:Ring of Energy Resistance}{\subsubsection{Ring of Energy Resistance\hfill\nth{6}}}
You have \glossterm{damage reduction} equal to the ring's \glossterm{power} against \glossterm{energy damage}.
Whenever you resist energy with this ability, the ring sheds light as a torch until the end of the next round.
The color of the light depends on the energy damage resisted: blue for cold, yellow for electricity, red for fire, and brown for sonic.
\parhead*{Tags} \glossterm{Shielding}
\parhead*{Materials} Bone, jewelry, metal, wood
\lowercase{\hypertarget{item:Ring of Energy Resistance, Greater}{}}\label{item:Ring of Energy Resistance, Greater}
\hypertarget{item:Ring of Energy Resistance, Greater}{\subsubsection{Ring of Energy Resistance, Greater\hfill\nth{14}}}
This item functions like the \mitem{ring of energy resistance}, except that the damage reduction is equal to twice the item's \glossterm{power}.
\parhead*{Tags} \glossterm{Shielding}
\parhead*{Materials} Bone, jewelry, metal, wood
\lowercase{\hypertarget{item:Ring of Nourishment}{}}\label{item:Ring of Nourishment}
\hypertarget{item:Ring of Nourishment}{\subsubsection{Ring of Nourishment\hfill\nth{3}}}
You continuously gain nourishment, and no longer need to eat or drink.
This ring must be worn for 24 hours before it begins to work.
\parhead*{Tags} \glossterm{Creation}
\parhead*{Materials} Bone, jewelry, metal, wood
\lowercase{\hypertarget{item:Ring of Protection}{}}\label{item:Ring of Protection}
\hypertarget{item:Ring of Protection}{\subsubsection{Ring of Protection\hfill\nth{8}}}
You gain a \plus1 bonus to Armor defense.
\parhead*{Tags} \glossterm{Shielding}
\parhead*{Materials} Bone, jewelry, metal, wood
\lowercase{\hypertarget{item:Ring of Regeneration}{}}\label{item:Ring of Regeneration}
\hypertarget{item:Ring of Regeneration}{\subsubsection{Ring of Regeneration\hfill\nth{11}}}
At the end of each \glossterm{action phase}, you heal hit points equal to this item's \glossterm{power}.
Only damage taken while wearing the ring can be healed in this way.
\parhead*{Tags} \glossterm{Life}
\parhead*{Materials} Bone, jewelry, metal, wood
\lowercase{\hypertarget{item:Ring of Sustenance}{}}\label{item:Ring of Sustenance}
\hypertarget{item:Ring of Sustenance}{\subsubsection{Ring of Sustenance\hfill\nth{7}}}
You continuously gain nourishment, and no longer need to eat or drink.
In addition, you need only one-quarter your normal amount of sleep (or similar activity, such as elven trance) each day.
The ring must be worn for 24 hours before it begins to work.
\parhead*{Tags} \glossterm{Creation}, \glossterm{Temporal}
\parhead*{Materials} Bone, jewelry, metal, wood
\lowercase{\hypertarget{item:Seven League Boots}{}}\label{item:Seven League Boots}
\hypertarget{item:Seven League Boots}{\subsubsection{Seven League Boots\hfill\nth{12}}}
As a standard action, you can spend an \glossterm{action point} to use this item.
If you do, you teleport exactly 25 miles in a direction you specify.
If this would place you within a solid object or otherwise impossible space, the boots will shunt you up to 1,000 feet in any direction to the closest available space.
If there is no available space within 1,000 feet of your intended destination, the effect fails and you take \glossterm{standard damage} \minus1d.
\parhead*{Tags} \glossterm{Teleportation}
\parhead*{Materials} Bone, leather, metal
\lowercase{\hypertarget{item:Shield of Arrow Catching}{}}\label{item:Shield of Arrow Catching}
\hypertarget{item:Shield of Arrow Catching}{\subsubsection{Shield of Arrow Catching\hfill\nth{5}}}
Whenever a creature within a \areamed radius emanation from you would be attacked by a ranged weapon, the attack is redirected to target you instead.
Resolve the attack as if it had initially targeted you, except that the attack is not affected by cover or concealment.
This item can only affect projectiles and thrown objects that are Small or smaller.
\parhead*{Tags} \glossterm{Telekinesis}
\parhead*{Materials} Bone, metal, wood
\lowercase{\hypertarget{item:Shield of Arrow Catching, Greater}{}}\label{item:Shield of Arrow Catching, Greater}
\hypertarget{item:Shield of Arrow Catching, Greater}{\subsubsection{Shield of Arrow Catching, Greater\hfill\nth{10}}}
This item functions like the \mitem{shield of arrow catching} item, except that it affects a \arealarge radius from you.
In addition, you may choose to exclude creature from this item's effect, allowing projectiles to target nearby foes normally.
\parhead*{Tags} \glossterm{Telekinesis}
\parhead*{Materials} Bone, metal, wood
\lowercase{\hypertarget{item:Shield of Arrow Deflection}{}}\label{item:Shield of Arrow Deflection}
\hypertarget{item:Shield of Arrow Deflection}{\subsubsection{Shield of Arrow Deflection\hfill\nth{2}}}
As an \glossterm{immediate action} when you are attacked by a ranged \glossterm{strike}, you can use this item.
If you do, you gain a \plus5 bonus to Armor defense against the attack.
You must be aware of the attack to deflect it in this way.
This item can only affect projectiles and thrown objects that are Small or smaller.
\parhead*{Tags} \glossterm{Telekinesis}
\parhead*{Materials} Bone, metal, wood
\lowercase{\hypertarget{item:Shield of Arrow Deflection, Greater}{}}\label{item:Shield of Arrow Deflection, Greater}
\hypertarget{item:Shield of Arrow Deflection, Greater}{\subsubsection{Shield of Arrow Deflection, Greater\hfill\nth{12}}}
This item functions like the \mitem{shield of arrow deflection} item, except that the defense bonus increases to \plus10.
\parhead*{Tags} \glossterm{Telekinesis}
\parhead*{Materials} Bone, metal, wood
\lowercase{\hypertarget{item:Shield of Bashing}{}}\label{item:Shield of Bashing}
\hypertarget{item:Shield of Bashing}{\subsubsection{Shield of Bashing\hfill\nth{2}}}
% Should this be strike damage?
You gain a \plus1d bonus to damage with \glossterm{physical attacks} using this shield.
\parhead*{Tags} \glossterm{Enhancement}
\parhead*{Materials} Bone, metal, wood
\lowercase{\hypertarget{item:Shield of Bashing, Greater}{}}\label{item:Shield of Bashing, Greater}
\hypertarget{item:Shield of Bashing, Greater}{\subsubsection{Shield of Bashing, Greater\hfill\nth{11}}}
% Should this be strike damage?
You gain a \plus2d bonus to damage with \glossterm{physical attacks} using this shield.
\parhead*{Tags} \glossterm{Enhancement}
\parhead*{Materials} Bone, metal, wood
\lowercase{\hypertarget{item:Shield of Boulder Catching}{}}\label{item:Shield of Boulder Catching}
\hypertarget{item:Shield of Boulder Catching}{\subsubsection{Shield of Boulder Catching\hfill\nth{8}}}
This item functions like the \mitem{shield of arrow catching} item, except that it can affect projectile and thrown objects of up to Large size.
\parhead*{Tags} \glossterm{Telekinesis}
\parhead*{Materials} Bone, metal, wood
\lowercase{\hypertarget{item:Shield of Boulder Deflection}{}}\label{item:Shield of Boulder Deflection}
\hypertarget{item:Shield of Boulder Deflection}{\subsubsection{Shield of Boulder Deflection\hfill\nth{6}}}
This item functions like the \mitem{shield of arrow deflection} item, except that it can affect projectiles and thrown objects of up to Large size.
\parhead*{Tags} \glossterm{Telekinesis}
\parhead*{Materials} Bone, metal, wood
\lowercase{\hypertarget{item:Shield of Mystic Reflection}{}}\label{item:Shield of Mystic Reflection}
\hypertarget{item:Shield of Mystic Reflection}{\subsubsection{Shield of Mystic Reflection\hfill\nth{12}}}
As an \glossterm{immediate action} when you are targeted by a targeted \glossterm{magical} ability, you can spend an \glossterm{action point} to use this ability.
If you do, the ability targets the creature using the ability instead of you.
Any other targets of the ability are affected normally.
\parhead*{Tags} \glossterm{Thaumaturgy}
\parhead*{Materials} Bone, metal, wood
\lowercase{\hypertarget{item:Throwing Gloves}{}}\label{item:Throwing Gloves}
\hypertarget{item:Throwing Gloves}{\subsubsection{Throwing Gloves\hfill\nth{4}}}
% TODO: reference basic "not designed to be thrown" mechanics?
You can throw any item as if it was designed to be thrown.
This does not improve your ability to throw items designed to be thrown, such as darts.
\parhead*{Tags} \glossterm{Enhancement}
\parhead*{Materials} Leather
\lowercase{\hypertarget{item:Torchlight Gloves}{}}\label{item:Torchlight Gloves}
\hypertarget{item:Torchlight Gloves}{\subsubsection{Torchlight Gloves\hfill\nth{2}}}
These gloves shed light as a torch.
As a \glossterm{standard action}, you may choose to suppress or resume the light from either or both gloves.
\parhead*{Tags} \glossterm{Figment}, \glossterm{Light}
\parhead*{Materials} Leather
\lowercase{\hypertarget{item:Vanishing Cloak}{}}\label{item:Vanishing Cloak}
\hypertarget{item:Vanishing Cloak}{\subsubsection{Vanishing Cloak\hfill\nth{8}}}
As a standard action, you can spend an \glossterm{action point} to use this item.
If you do, you teleport to an unoccupied location within \rngmed range of your original location.
In addition, you become \glossterm{invisible} unitl the end of the next round.
If your intended destination is invalid, or if your teleportation otherwise fails, you still become invisible.
\parhead*{Tags} \glossterm{Glamer}, \glossterm{Teleportation}
\parhead*{Materials} Textiles
\lowercase{\hypertarget{item:Winged Boots}{}}\label{item:Winged Boots}
\hypertarget{item:Winged Boots}{\subsubsection{Winged Boots\hfill\nth{10}}}
You gain a \glossterm{fly speed} equal to your land speed.
However, the boots are not strong enough to keep you aloft indefinitely.
At the end of each round, if you are not standing on solid ground, the magic of the boots fails and you fall normally.
The boots begin working again at the end of the next round, even if you have not yet hit the ground.
\parhead*{Tags} \glossterm{Imbuement}
\parhead*{Materials} Bone, leather, metal
