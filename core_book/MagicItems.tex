\chapter{Magic Items}

Magic items are objects that have been imbued with magical energy. They can take almost any form, and their potential uses are only as limited as the magic that created them.

\section{Magic Item Types}
    Magic items are divided into four broad categories:
    \begin{itemize}
        \item Weapons are used to make physical attacks. They provide access to their abilities when wielded.
            A \mitem{flaming longsword} and a \mitem{vampiric scythe} are weapons.
        \item Implements are used to cast spells. They provide access to their abilities when wielded.
            A \mitem{staff of fire} and a \mitem{staff of time} are implements.
        \item Apparel items are usually not used individually. They provide access to their abilities when worn.
            A \mitem{flaming burst full plate} and a \mitem{ring of protection} are apparel items.
        \item Tools provide access to their abilities when used in some way.
            A \mitem{bag of carrying} is a tool.
    \end{itemize}

\section{Using Magic Items}

    \subsection{Item Activation}

        Some magic items have to be explicitly activated to have unusual effects.
        For example, a \mitem{healing belt} can be activated to heal you.
        Other magic items constantly have magical effects.
        For example, a \mitem{flaming longsword} is on fire.

        The description of a magic item effect will specify what mechanical actions must be taken, if any, to activate the effects of the item.
        For example, a healing belt requires spending an \glossterm{action point} as a \glossterm{standard action}.
        However, the item description will not specify the exact nature of the action.
        Different items, even if they have the same effect, can have different physical actions that are required to activate the item.
        These minor actions can come in one of the following forms:
        \begin{itemize}
           \item Command word: You must speak a specific word that the item will hear and react to.
                For example, you may need to say the word ``healing'' in Elven to activate a \mitem{belt of healing}.
            \item Mental command: You must mentally direct the item to activate, such as by visualizing the item or thinking a particular word.
                % TODO: does this item exist
                For example, you may need to imagine a warm blanket around you to activate a \mitem{ring of elemental endurance}.
            \item Physical motion: You must perform a specific physical motion, usually involving the item in some way.
                For example, you may need to rapidly stomp one foot on the ground to activate \mitem{boots of speed}.
        \end{itemize}

        % TODO: table of random item activations?

    \subsection{Item Limitations}

        There are three restrictions on your ability to use magic items.
        First, you cannot equip two apparel items that take up the same physical location on your body.
        For example, you cannot equip two different gauntlet sets and gain the effects of both, but you could equip several amulets or up to ten rings.
        Second, most magic items other than tools require you to attune to them to gain their effect.
        You can attune to a magic item with the \textit{item attunement} ability, below.
        Third, you cannot attune to two items with the same name, or if one is simply a Greater or Lesser version of the other.

        \subsubsection{Item Slots}
            Normally, attuning to an ability or magic item reduces your maximum action points by one.
            However, you can attune to a certain number of items without penalizing your action points.
            You start with two \glossterm{item slots}, and acquire more as you gain levels (see \tref{Character Advancement}).

            When you attune to a magic item, you can spend an \glossterm{item slot} instead of an \glossterm{action point}.
            If you do, your maximum action points are not affected by your attunement.
            This is summarized in the description of the \textit{item attunement} ability.

            Dismissing your attunement to an item does not restore the spent item slot.
            You regain all spent \glossterm{item slots} from dismissed attunements after a \glossterm{long rest}.

        \subsubsection{Item Attunement}
            As a standard action, you can spend an \glossterm{action point} or an \glossterm{item slot} to use this ability.
            \begin{ability}
                \begin{spelltargetinginfo}
                    \spellquicktargeting{One magic item}{\rngtouch}
                \end{spelltargetinginfo}
                \begin{spelleffects}
                    \spelleffect Any abilities the target has that require attunement become active, allowing you to use its full potential.
                    \spelldur Attunement. If you spent an \glossterm{item slot} to use this ability, attuning to this ability does not reduce your maximum \glossterm{action points}.
                \end{spelleffects}
            \end{ability}

    \subsubsection{Item Power}
        The \glossterm{power} of an item depends on its level.
        If the item is not being attuned to by a creature, its power is equal to its level.
        If it is being attuned to, its power is equal to its level or the level of the attuning creature, whichever is higher.

        An item's \glossterm{power} also affects its defenses.
        Its Fortitude and Mental defenses are equal to 10 \add its \glossterm{power}.
        Its Armor defense and Reflex defense are not affected by its \glossterm{power}, and are solely determined by its size and shape.

    \subsection{Removing Magic Items}
        Unless otherwise noted, magic items that have effects on the creature using the item must continue to be worn or held as long as the effect lasts.
        If a magic item is removed before its duration ends the effect ends.
        Items which are consumed when used or which do not affect their user are unaffected by this rule.

\section{Item Description Format}
    TODO

\section{Apparel}

    
\lowercase{\hypertarget{item:Amulet of Health}{}}\label{item:Amulet of Health}
\hypertarget{item:Amulet of Health}{\subsubsection{Amulet of Health\hfill\nth{2} (125 gp)}}

You gain a \glossterm{magic bonus} equal to this item's \glossterm{power} to your \glossterm{wound threshold}.



\vspace{0.25em}
\spelltwocol{\textbf{Type}: Amulet}{}
\textbf{Materials}: Jewelry


\lowercase{\hypertarget{item:Amulet of Health, Greater}{}}\label{item:Amulet of Health, Greater}
\hypertarget{item:Amulet of Health, Greater}{\subsubsection{Amulet of Health, Greater\hfill\nth{8} (2,750 gp)}}

You gain a \glossterm{magic bonus} equal to twice this item's \glossterm{power} to your \glossterm{wound threshold}.



\vspace{0.25em}
\spelltwocol{\textbf{Type}: Amulet}{}
\textbf{Materials}: Jewelry


\lowercase{\hypertarget{item:Amulet of Mighty Fists}{}}\label{item:Amulet of Mighty Fists}
\hypertarget{item:Amulet of Mighty Fists}{\subsubsection{Amulet of Mighty Fists\hfill\nth{8} (2,750 gp)}}

You gain a \plus2 \glossterm{magic bonus} to \glossterm{power} with \glossterm{unarmed attacks} and natural weapons.



\vspace{0.25em}
\spelltwocol{\textbf{Type}: Amulet}{}
\textbf{Materials}: Jewelry


\lowercase{\hypertarget{item:Amulet of Mighty Fists, Greater}{}}\label{item:Amulet of Mighty Fists, Greater}
\hypertarget{item:Amulet of Mighty Fists, Greater}{\subsubsection{Amulet of Mighty Fists, Greater\hfill\nth{16} (85,000 gp)}}

You gain a \plus4 \glossterm{magic bonus} to \glossterm{power} with \glossterm{unarmed attacks} and natural weapons.



\vspace{0.25em}
\spelltwocol{\textbf{Type}: Amulet}{}
\textbf{Materials}: Jewelry


\lowercase{\hypertarget{item:Amulet of Nondetection}{}}\label{item:Amulet of Nondetection}
\hypertarget{item:Amulet of Nondetection}{\subsubsection{Amulet of Nondetection\hfill\nth{6} (1,200 gp)}}

You gain a \plus4 bonus to defenses against abilities with the \glossterm{Detection} or \glossterm{Scrying} tags.



\vspace{0.25em}
\spelltwocol{\textbf{Type}: Amulet}{}
\textbf{Materials}: Jewelry


\lowercase{\hypertarget{item:Amulet of Nondetection, Greater}{}}\label{item:Amulet of Nondetection, Greater}
\hypertarget{item:Amulet of Nondetection, Greater}{\subsubsection{Amulet of Nondetection, Greater\hfill\nth{14} (37,000 gp)}}

You gain a \plus8 bonus to defenses against abilities with the \glossterm{Detection} or \glossterm{Scrying} tags.



\vspace{0.25em}
\spelltwocol{\textbf{Type}: Amulet}{}
\textbf{Materials}: Jewelry


\lowercase{\hypertarget{item:Amulet of the Planes}{}}\label{item:Amulet of the Planes}
\hypertarget{item:Amulet of the Planes}{\subsubsection{Amulet of the Planes\hfill\nth{12} (16,000 gp)}}

When you perform the \ritual{plane shift} ritual, this amulet provides all action points required.
This does not grant you the ability to perform the \ritual{plane shift} ritual if you could not already.



\vspace{0.25em}
\spelltwocol{\textbf{Type}: Amulet}{}
\textbf{Materials}: Jewelry


\lowercase{\hypertarget{item:Armor of Energy Resistance}{}}\label{item:Armor of Energy Resistance}
\hypertarget{item:Armor of Energy Resistance}{\subsubsection{Armor of Energy Resistance\hfill\nth{5} (800 gp)}}

You gain a \glossterm{magic bonus} equal to half the item's \glossterm{power} to \glossterm{resistances} against \glossterm{energy damage}.
When you resist energy damage, it sheds light as a torch until the end of the next round.
The color of the light depends on the energy damage resisted: green for acid, blue for cold, yellow for electricity, and red for fire.



\vspace{0.25em}
\spelltwocol{\textbf{Type}: Body armor}{}
\textbf{Materials}: Bone, metal


\lowercase{\hypertarget{item:Armor of Energy Resistance, Greater}{}}\label{item:Armor of Energy Resistance, Greater}
\hypertarget{item:Armor of Energy Resistance, Greater}{\subsubsection{Armor of Energy Resistance, Greater\hfill\nth{14} (37,000 gp)}}

This item functions like the \mitem{armor of energy resistance} item, except that the bonus is equal to the item's \glossterm{power}.



\vspace{0.25em}
\spelltwocol{\textbf{Type}: Body armor}{}
\textbf{Materials}: Bone, metal


\lowercase{\hypertarget{item:Armor of Fortification}{}}\label{item:Armor of Fortification}
\hypertarget{item:Armor of Fortification}{\subsubsection{Armor of Fortification\hfill\nth{7} (1,800 gp)}}

You gain a \plus5 bonus to defenses when determining whether a \glossterm{strike} gets a \glossterm{critical hit} against you instead of a normal hit.



\vspace{0.25em}
\spelltwocol{\textbf{Type}: Body armor}{}
\textbf{Materials}: Bone, metal


\lowercase{\hypertarget{item:Armor of Fortification, Greater}{}}\label{item:Armor of Fortification, Greater}
\hypertarget{item:Armor of Fortification, Greater}{\subsubsection{Armor of Fortification, Greater\hfill\nth{15} (55,000 gp)}}

This item functions like the \mitem{armor of fortification} item, except that the bonus increases to \plus10.



\vspace{0.25em}
\spelltwocol{\textbf{Type}: Body armor}{}
\textbf{Materials}: Bone, metal


\lowercase{\hypertarget{item:Armor of Fortification, Mystic}{}}\label{item:Armor of Fortification, Mystic}
\hypertarget{item:Armor of Fortification, Mystic}{\subsubsection{Armor of Fortification, Mystic\hfill\nth{12} (16,000 gp)}}

This item functions like the \mitem{armor of fortification} item, except that it applies against all attacks instead of only against; \glossterm{strikes}.



\vspace{0.25em}
\spelltwocol{\textbf{Type}: Body armor}{}
\textbf{Materials}: Bone, metal


\lowercase{\hypertarget{item:Armor of Invulnerability}{}}\label{item:Armor of Invulnerability}
\hypertarget{item:Armor of Invulnerability}{\subsubsection{Armor of Invulnerability\hfill\nth{8} (2,750 gp)}}

The armor's bonus to \glossterm{resistances} based on its armor type is doubled.



\vspace{0.25em}
\spelltwocol{\textbf{Type}: Body armor}{}
\textbf{Materials}: Bone, metal


\lowercase{\hypertarget{item:Armor of Invulnerability, Greater}{}}\label{item:Armor of Invulnerability, Greater}
\hypertarget{item:Armor of Invulnerability, Greater}{\subsubsection{Armor of Invulnerability, Greater\hfill\nth{14} (37,000 gp)}}

The armor's bonus to \glossterm{resistances} based on its armor type is tripled.



\vspace{0.25em}
\spelltwocol{\textbf{Type}: Body armor}{}
\textbf{Materials}: Bone, metal


\lowercase{\hypertarget{item:Armor of Invulnerability, Supreme}{}}\label{item:Armor of Invulnerability, Supreme}
\hypertarget{item:Armor of Invulnerability, Supreme}{\subsubsection{Armor of Invulnerability, Supreme\hfill\nth{20} (400,000 gp)}}

The armor's bonus to \glossterm{resistances} based on its armor type is quadrupled.



\vspace{0.25em}
\spelltwocol{\textbf{Type}: Body armor}{}
\textbf{Materials}: Bone, metal


\lowercase{\hypertarget{item:Assassin's Cloak}{}}\label{item:Assassin's Cloak}
\hypertarget{item:Assassin's Cloak}{\subsubsection{Assassin's Cloak\hfill\nth{7} (1,800 gp)}}

At the end of each round, if you took no actions that round, you become \glossterm{invisible} until the end of the next round.



\vspace{0.25em}
\spelltwocol{\textbf{Type}: Cloak}{\parhead*{Tags} \glossterm{Sensation}}
\textbf{Materials}: Textiles


\lowercase{\hypertarget{item:Assassin's Cloak, Greater}{}}\label{item:Assassin's Cloak, Greater}
\hypertarget{item:Assassin's Cloak, Greater}{\subsubsection{Assassin's Cloak, Greater\hfill\nth{17} (125,000 gp)}}

At the end of each round, if you did not attack a creature that round, you become \glossterm{invisible} until the end of the next round.



\vspace{0.25em}
\spelltwocol{\textbf{Type}: Cloak}{\parhead*{Tags} \glossterm{Sensation}}
\textbf{Materials}: Textiles


\lowercase{\hypertarget{item:Astral Boots}{}}\label{item:Astral Boots}
\hypertarget{item:Astral Boots}{\subsubsection{Astral Boots\hfill\nth{16} (85,000 gp)}}

When you move, you can teleport the same distance instead.
This does not change the total distance you can move, but you can teleport in any direction, even vertically.
You cannot teleport to locations you do not have \glossterm{line of sight} and \glossterm{line of effect} to.



\vspace{0.25em}
\spelltwocol{\textbf{Type}: Boots}{}
\textbf{Materials}: Bone, leather, metal


\lowercase{\hypertarget{item:Belt of Healing}{}}\label{item:Belt of Healing}
\hypertarget{item:Belt of Healing}{\subsubsection{Belt of Healing\hfill\nth{6} (1,200 gp)}}

As a standard action, you can use this belt to regain a \glossterm{hit point}.
You can only use this item once between \glossterm{short rests}.



\vspace{0.25em}
\spelltwocol{\textbf{Type}: Belt}{}
\textbf{Materials}: Leather, textiles


\lowercase{\hypertarget{item:Belt of Healing, Greater}{}}\label{item:Belt of Healing, Greater}
\hypertarget{item:Belt of Healing, Greater}{\subsubsection{Belt of Healing, Greater\hfill\nth{14} (37,000 gp)}}

This item functions like the \textit{belt of healing}, except that you can regain two \glossterm{hit points} instead of one.



\vspace{0.25em}
\spelltwocol{\textbf{Type}: Belt}{}
\textbf{Materials}: Leather, textiles


\lowercase{\hypertarget{item:Boots of Earth's Embrace}{}}\label{item:Boots of Earth's Embrace}
\hypertarget{item:Boots of Earth's Embrace}{\subsubsection{Boots of Earth's Embrace\hfill\nth{4} (500 gp)}}

While you are standing on solid ground, you are immune to effects that would force you to move.
This does not protect you from other effects of those attacks, such as damage.



\vspace{0.25em}
\spelltwocol{\textbf{Type}: Boots}{}
\textbf{Materials}: Bone, leather, metal


\lowercase{\hypertarget{item:Boots of Elvenkind}{}}\label{item:Boots of Elvenkind}
\hypertarget{item:Boots of Elvenkind}{\subsubsection{Boots of Elvenkind\hfill\nth{4} (500 gp)}}

You gain a \plus2 \glossterm{magic bonus} to the Stealth skill (see \pcref{Stealth}).



\vspace{0.25em}
\spelltwocol{\textbf{Type}: Boots}{}
\textbf{Materials}: Bone, leather, metal


\lowercase{\hypertarget{item:Boots of Freedom}{}}\label{item:Boots of Freedom}
\hypertarget{item:Boots of Freedom}{\subsubsection{Boots of Freedom\hfill\nth{3} (250 gp)}}

You are immune to magical effects that restrict your mobility.
This does not prevent physical obstacles from affecting you, such as \glossterm{difficult terrain}.



\vspace{0.25em}
\spelltwocol{\textbf{Type}: Boots}{}
\textbf{Materials}: Bone, leather, metal


\lowercase{\hypertarget{item:Boots of Freedom, Greater}{}}\label{item:Boots of Freedom, Greater}
\hypertarget{item:Boots of Freedom, Greater}{\subsubsection{Boots of Freedom, Greater\hfill\nth{9} (4,000 gp)}}

You are immune to all effects that restrict your mobility, including nonmagical effects such as \glossterm{difficult terrain}.
This removes all penalties you would suffer for acting underwater, except for those relating to using ranged weapons.
This does not prevent you from being \grappled, but you gain a \plus10 bonus to defenses against the \textit{grapple} ability (see \pcref{Grapple}).



\vspace{0.25em}
\spelltwocol{\textbf{Type}: Boots}{}
\textbf{Materials}: Bone, leather, metal


\lowercase{\hypertarget{item:Boots of Gravitation}{}}\label{item:Boots of Gravitation}
\hypertarget{item:Boots of Gravitation}{\subsubsection{Boots of Gravitation\hfill\nth{8} (2,750 gp)}}

While these boots are within 5 feet of a solid surface, gravity pulls you towards the solid surface closest to your boots rather than in the normal direction.
This can allow you to walk easily on walls or even ceilings.



\vspace{0.25em}
\spelltwocol{\textbf{Type}: Boots}{}
\textbf{Materials}: Bone, leather, metal


\lowercase{\hypertarget{item:Boots of Speed}{}}\label{item:Boots of Speed}
\hypertarget{item:Boots of Speed}{\subsubsection{Boots of Speed\hfill\nth{6} (1,200 gp)}}

You gain a \plus10 foot \glossterm{magic bonus} to your land speed, up to a maximum of double your normal speed.



\vspace{0.25em}
\spelltwocol{\textbf{Type}: Boots}{}
\textbf{Materials}: Bone, leather, metal


\lowercase{\hypertarget{item:Boots of Speed, Greater}{}}\label{item:Boots of Speed, Greater}
\hypertarget{item:Boots of Speed, Greater}{\subsubsection{Boots of Speed, Greater\hfill\nth{10} (6,500 gp)}}

You gain a \plus20 foot \glossterm{magic bonus} to your land speed, up to a maximum of double your normal speed.



\vspace{0.25em}
\spelltwocol{\textbf{Type}: Boots}{}
\textbf{Materials}: Bone, leather, metal


\lowercase{\hypertarget{item:Boots of Speed, Supreme}{}}\label{item:Boots of Speed, Supreme}
\hypertarget{item:Boots of Speed, Supreme}{\subsubsection{Boots of Speed, Supreme\hfill\nth{14} (37,000 gp)}}

You gain a \plus30 foot \glossterm{magic bonus} to your land speed, up to a maximum of double your normal speed.



\vspace{0.25em}
\spelltwocol{\textbf{Type}: Boots}{}
\textbf{Materials}: Bone, leather, metal


\lowercase{\hypertarget{item:Boots of Water Walking}{}}\label{item:Boots of Water Walking}
\hypertarget{item:Boots of Water Walking}{\subsubsection{Boots of Water Walking\hfill\nth{7} (1,800 gp)}}

You treat the surface of all liquids as if they were firm ground.
Your feet hover about an inch above the liquid's surface, allowing you to traverse dangerous liquids without harm as long as the surface is calm.

If you are below the surface of the liquid, you rise towards the surface at a rate of 60 feet per round.
Thick liquids, such as mud and lava, may cause you to rise more slowly.



\vspace{0.25em}
\spelltwocol{\textbf{Type}: Boots}{}
\textbf{Materials}: Bone, leather, metal


\lowercase{\hypertarget{item:Boots of the Skydancer}{}}\label{item:Boots of the Skydancer}
\hypertarget{item:Boots of the Skydancer}{\subsubsection{Boots of the Skydancer\hfill\nth{7} (1,800 gp)}}

As a \glossterm{free action}, you can activate these boots.
When you do, you may treat air as if it were solid ground to your feet for the rest of the current phase.
You may selectively choose when to treat the air as solid ground, allowing you to walk or jump on air freely.
After using this ability, you cannot use it again until these boots touch the ground.



\vspace{0.25em}
\spelltwocol{\textbf{Type}: Boots}{\parhead*{Tags} \glossterm{Swift}}
\textbf{Materials}: Bone, leather, metal


\lowercase{\hypertarget{item:Boots of the Skydancer, Greater}{}}\label{item:Boots of the Skydancer, Greater}
\hypertarget{item:Boots of the Skydancer, Greater}{\subsubsection{Boots of the Skydancer, Greater\hfill\nth{13} (25,000 gp)}}

This item functions like the \magicitem{boots of the skydancer}, except that the ability lasts until the end of the round.
In addition, you can use this item twice before the boots touch the ground.



\vspace{0.25em}
\spelltwocol{\textbf{Type}: Boots}{\parhead*{Tags} \glossterm{Swift}}
\textbf{Materials}: Bone, leather, metal


\lowercase{\hypertarget{item:Boots of the Winterlands}{}}\label{item:Boots of the Winterlands}
\hypertarget{item:Boots of the Winterlands}{\subsubsection{Boots of the Winterlands\hfill\nth{2} (125 gp)}}

You can travel across snow and ice without slipping or suffering movement penalties for the terrain.
% TODO: degree symbol?
In addition, the boots keep you warn, protecting you in environments as cold as \minus50 Fahrenheit.



\vspace{0.25em}
\spelltwocol{\textbf{Type}: Boots}{}
\textbf{Materials}: Bone, leather, metal


\lowercase{\hypertarget{item:Bracers of Archery}{}}\label{item:Bracers of Archery}
\hypertarget{item:Bracers of Archery}{\subsubsection{Bracers of Archery\hfill\nth{1} (50 gp)}}

You are proficient with bows.



\vspace{0.25em}
\spelltwocol{\textbf{Type}: Bracers}{}
\textbf{Materials}: Bone, leather, metal, wood


\lowercase{\hypertarget{item:Bracers of Archery, Greater}{}}\label{item:Bracers of Archery, Greater}
\hypertarget{item:Bracers of Archery, Greater}{\subsubsection{Bracers of Archery, Greater\hfill\nth{7} (1,800 gp)}}

You are proficient with bows.
In addition, you gain a \plus1 \glossterm{magic bonus} to \glossterm{accuracy} with ranged \glossterm{strikes}.



\vspace{0.25em}
\spelltwocol{\textbf{Type}: Bracers}{}
\textbf{Materials}: Bone, leather, metal, wood


\lowercase{\hypertarget{item:Bracers of Armor}{}}\label{item:Bracers of Armor}
\hypertarget{item:Bracers of Armor}{\subsubsection{Bracers of Armor\hfill\nth{2} (125 gp)}}

You gain a \plus2 bonus to Armor defense.
The protection from these bracers is treated as body armor, and it does not stack with any other body armor you wear.



\vspace{0.25em}
\spelltwocol{\textbf{Type}: Bracers}{}
\textbf{Materials}: Bone, leather, metal, wood


\lowercase{\hypertarget{item:Bracers of Repulsion}{}}\label{item:Bracers of Repulsion}
\hypertarget{item:Bracers of Repulsion}{\subsubsection{Bracers of Repulsion\hfill\nth{7} (1,800 gp)}}

As a standard action, you can activate these bracers.
When you do, they emit a telekinetic burst of force.
Make an attack vs. Fortitude against everything within a \areasmall radius burst from you.
If you use this item during the \glossterm{delayed action phase},
you gain a \plus4 bonus to \glossterm{accuracy} with this attack against any creature that attacked you during the \glossterm{action phase}.
On a hit, you \glossterm{knockback} each target up to 20 feet.



\vspace{0.25em}
\spelltwocol{\textbf{Type}: Bracers}{}
\textbf{Materials}: Bone, leather, metal, wood


\lowercase{\hypertarget{item:Bracers of Repulsion, Greater}{}}\label{item:Bracers of Repulsion, Greater}
\hypertarget{item:Bracers of Repulsion, Greater}{\subsubsection{Bracers of Repulsion, Greater\hfill\nth{15} (55,000 gp)}}

This item functions like the \mitem{bracers of repulsion} item, except that it targets everything within a \arealarge radius burst.



\vspace{0.25em}
\spelltwocol{\textbf{Type}: Bracers}{}
\textbf{Materials}: Bone, leather, metal, wood


\lowercase{\hypertarget{item:Circlet of Blasting}{}}\label{item:Circlet of Blasting}
\hypertarget{item:Circlet of Blasting}{\subsubsection{Circlet of Blasting\hfill\nth{5} (800 gp)}}

As a standard action, you can activate this circlet.
If you do, make an attack vs. Armor against a creature or object within \rngmed range.
\hit The target takes fire \glossterm{standard damage}.



\vspace{0.25em}
\spelltwocol{\textbf{Type}: Circlet}{}
\textbf{Materials}: Bone, metal


\lowercase{\hypertarget{item:Circlet of Blasting, Greater}{}}\label{item:Circlet of Blasting, Greater}
\hypertarget{item:Circlet of Blasting, Greater}{\subsubsection{Circlet of Blasting, Greater\hfill\nth{10} (6,500 gp)}}

This item functions like the \textit{circlet of blasting}, except that it gains a \plus1d bonus to damage.



\vspace{0.25em}
\spelltwocol{\textbf{Type}: Circlet}{}
\textbf{Materials}: Bone, metal


\lowercase{\hypertarget{item:Circlet of Blasting, Supreme}{}}\label{item:Circlet of Blasting, Supreme}
\hypertarget{item:Circlet of Blasting, Supreme}{\subsubsection{Circlet of Blasting, Supreme\hfill\nth{16} (85,000 gp)}}

This item functions like the \textit{circlet of blasting}, except that it gains a \plus2d bonus to damage.



\vspace{0.25em}
\spelltwocol{\textbf{Type}: Circlet}{}
\textbf{Materials}: Bone, metal


\lowercase{\hypertarget{item:Circlet of Persuasion}{}}\label{item:Circlet of Persuasion}
\hypertarget{item:Circlet of Persuasion}{\subsubsection{Circlet of Persuasion\hfill\nth{4} (500 gp)}}

You gain a \plus2 \glossterm{magic bonus} to the Persuasion skill (see \pcref{Persuasion}).



\vspace{0.25em}
\spelltwocol{\textbf{Type}: Circlet}{}
\textbf{Materials}: Bone, metal


\lowercase{\hypertarget{item:Cloak of Mist}{}}\label{item:Cloak of Mist}
\hypertarget{item:Cloak of Mist}{\subsubsection{Cloak of Mist\hfill\nth{8} (2,750 gp)}}

Fog constantly fills a \areamed radius emanation from you.
This fog does not fully block sight, but it provides \concealment.

If a 5-foot square of fog takes fire damage equal to half this item's \glossterm{power}, the fog disappears from that area until the end of the next round.



\vspace{0.25em}
\spelltwocol{\textbf{Type}: Cloak}{\parhead*{Tags} \glossterm{Manifestation}}
\textbf{Materials}: Textiles


\lowercase{\hypertarget{item:Cloak of Mist, Greater}{}}\label{item:Cloak of Mist, Greater}
\hypertarget{item:Cloak of Mist, Greater}{\subsubsection{Cloak of Mist, Greater\hfill\nth{16} (85,000 gp)}}

A thick fog constantly fills a \areamed radius emanation from you.
This fog completely blocks sight beyond 10 feet.
Within that range, it still provides \concealment.

If a 5-foot square of fog takes fire damage equal to this item's \glossterm{power}, the fog disappears from that area until the end of the next round.



\vspace{0.25em}
\spelltwocol{\textbf{Type}: Cloak}{\parhead*{Tags} \glossterm{Manifestation}}
\textbf{Materials}: Textiles


\lowercase{\hypertarget{item:Crater Boots}{}}\label{item:Crater Boots}
\hypertarget{item:Crater Boots}{\subsubsection{Crater Boots\hfill\nth{10} (6,500 gp)}}

% This only works if you only take falling damage during the movement phase, which seems possible?
When you take \glossterm{falling damage}, make an attack vs Reflex against everything within a \areasmall radius from you.
\hit Each target takes damage as if they had fallen the same distance that you fell.
This roll is made separately from the damage roll to determine your falling damage.
\crit As above, and each target is knocked \glossterm{prone}.
This does not deal double damage on a critical hit.



\vspace{0.25em}
\spelltwocol{\textbf{Type}: Boots}{}
\textbf{Materials}: Bone, leather, metal


\lowercase{\hypertarget{item:Crown of Flame}{}}\label{item:Crown of Flame}
\hypertarget{item:Crown of Flame}{\subsubsection{Crown of Flame\hfill\nth{9} (4,000 gp)}}

This crown is continuously on fire.
The flame sheds light as a torch.

You and your \glossterm{allies} within a \arealarge radius emanation from you
gain a \glossterm{magic bonus} equal to this item's \glossterm{power} to \glossterm{resistances} against fire damage.



\vspace{0.25em}
\spelltwocol{\textbf{Type}: Crown}{}
\textbf{Materials}: Bone, metal


\lowercase{\hypertarget{item:Crown of Frost}{}}\label{item:Crown of Frost}
\hypertarget{item:Crown of Frost}{\subsubsection{Crown of Frost\hfill\nth{13} (25,000 gp)}}

At the end of each \glossterm{action phase}, you make an attack vs. Fortitude against all enemies within a \areamed radius emanation from you.
At hit deals cold \glossterm{standard damage} \minus2d.



\vspace{0.25em}
\spelltwocol{\textbf{Type}: Crown}{}
\textbf{Materials}: Bone, metal


\lowercase{\hypertarget{item:Crown of Lightning}{}}\label{item:Crown of Lightning}
\hypertarget{item:Crown of Lightning}{\subsubsection{Crown of Lightning\hfill\nth{7} (1,800 gp)}}

This crown continuously crackles with electricity.
The constant sparks shed light as a torch.

At the end of each \glossterm{action phase}, you make an attack vs. Fortitude against all enemies within a \areamed radius emanation from you.
A hit deals electricity \glossterm{standard damage} \minus3d.



\vspace{0.25em}
\spelltwocol{\textbf{Type}: Crown}{}
\textbf{Materials}: Bone, metal


\lowercase{\hypertarget{item:Crown of Thunder}{}}\label{item:Crown of Thunder}
\hypertarget{item:Crown of Thunder}{\subsubsection{Crown of Thunder\hfill\nth{11} (10,000 gp)}}

The crown constantly emits a low-pitched rumbling.
To you and your \glossterm{allies}, the sound is barely perceptible.
However, all other creatures within a \arealarge radius emanation from you hear the sound as a deafening, continuous roll of thunder.
The noise blocks out all other sounds quieter than thunder, causing them to be \deafened while they remain in the area.



\vspace{0.25em}
\spelltwocol{\textbf{Type}: Crown}{}
\textbf{Materials}: Bone, metal


\lowercase{\hypertarget{item:Featherlight Armor}{}}\label{item:Featherlight Armor}
\hypertarget{item:Featherlight Armor}{\subsubsection{Featherlight Armor\hfill\nth{4} (500 gp)}}

This armor's \glossterm{encumbrance} is reduced by 1.



\vspace{0.25em}
\spelltwocol{\textbf{Type}: Body armor}{}
\textbf{Materials}: Bone, metal


\lowercase{\hypertarget{item:Featherlight Armor, Greater}{}}\label{item:Featherlight Armor, Greater}
\hypertarget{item:Featherlight Armor, Greater}{\subsubsection{Featherlight Armor, Greater\hfill\nth{10} (6,500 gp)}}

This armor's \glossterm{encumbrance} is reduced by 2.



\vspace{0.25em}
\spelltwocol{\textbf{Type}: Body armor}{}
\textbf{Materials}: Bone, metal


\lowercase{\hypertarget{item:Gauntlet of the Ram}{}}\label{item:Gauntlet of the Ram}
\hypertarget{item:Gauntlet of the Ram}{\subsubsection{Gauntlet of the Ram\hfill\nth{2} (125 gp)}}

When you make a \glossterm{strike} with this gauntlet, you also compare the attack result to the target's Fortitude defense.
On a hit, you \glossterm{knockback} the target up to 10 feet.
Making a strike with this gauntlet is equivalent to an \glossterm{unarmed attack}.



\vspace{0.25em}
\spelltwocol{\textbf{Type}: Gauntlet}{}
\textbf{Materials}: Bone, metal, wood


\lowercase{\hypertarget{item:Gauntlet of the Ram, Greater}{}}\label{item:Gauntlet of the Ram, Greater}
\hypertarget{item:Gauntlet of the Ram, Greater}{\subsubsection{Gauntlet of the Ram, Greater\hfill\nth{7} (1,800 gp)}}

This item functions like the \mitem{gauntlet of the ram}, except that you \glossterm{knockback} the target up to 30 feet.



\vspace{0.25em}
\spelltwocol{\textbf{Type}: Gauntlet}{}
\textbf{Materials}: Bone, metal, wood


\lowercase{\hypertarget{item:Gauntlets of Improvisation}{}}\label{item:Gauntlets of Improvisation}
\hypertarget{item:Gauntlets of Improvisation}{\subsubsection{Gauntlets of Improvisation\hfill\nth{2} (125 gp)}}

You gain a \plus1d \glossterm{magic bonus} to damage with \glossterm{improvised weapons}.



\vspace{0.25em}
\spelltwocol{\textbf{Type}: Gauntlet}{}
\textbf{Materials}: Bone, metal, wood


\lowercase{\hypertarget{item:Gauntlets of Improvisation, Greater}{}}\label{item:Gauntlets of Improvisation, Greater}
\hypertarget{item:Gauntlets of Improvisation, Greater}{\subsubsection{Gauntlets of Improvisation, Greater\hfill\nth{7} (1,800 gp)}}

This item functions like the \mitem{gauntlets of improvisation}, except that the damage bonus is increased to \plus2d.



\vspace{0.25em}
\spelltwocol{\textbf{Type}: Gauntlet}{}
\textbf{Materials}: Bone, metal, wood


\lowercase{\hypertarget{item:Gloves of Spell Investment}{}}\label{item:Gloves of Spell Investment}
\hypertarget{item:Gloves of Spell Investment}{\subsubsection{Gloves of Spell Investment\hfill\nth{7} (1,800 gp)}}

When you cast a spell that does not have the \glossterm{AP}, \glossterm{Attune}, \glossterm{Sustain} tags,
you can invest the magic of the spell in these gloves.
If you do, the spell does not have its normal effect.

As a standard action, you can activate these gloves.
When you do, you cause the effect of the last spell invested in the gloves.
This does not require \glossterm{concentration} or \glossterm{somatic components}.
After you use a spell in this way, the energy in the gloves is spent, and you must invest a new spell to activate the gloves again.

If you remove either glove from your hand, the magic of the spell invested in the gloves is lost.



\vspace{0.25em}
\spelltwocol{\textbf{Type}: Gloves}{}
\textbf{Materials}: Leather


\lowercase{\hypertarget{item:Gloves of Spell Investment, Greater}{}}\label{item:Gloves of Spell Investment, Greater}
\hypertarget{item:Gloves of Spell Investment, Greater}{\subsubsection{Gloves of Spell Investment, Greater\hfill\nth{13} (25,000 gp)}}

This item functions like the \mitem{gloves of spell investment}, except that you can store up to two spells in the gloves.
When you activate the gauntlets, you choose which spell to use.



\vspace{0.25em}
\spelltwocol{\textbf{Type}: Gloves}{}
\textbf{Materials}: Leather


\lowercase{\hypertarget{item:Greatreach Bracers}{}}\label{item:Greatreach Bracers}
\hypertarget{item:Greatreach Bracers}{\subsubsection{Greatreach Bracers\hfill\nth{9} (4,000 gp)}}

Your \glossterm{reach} is increased by 5 feet.



\vspace{0.25em}
\spelltwocol{\textbf{Type}: Bracers}{}
\textbf{Materials}: Bone, leather, metal, wood


\lowercase{\hypertarget{item:Greatreach Bracers, Greater}{}}\label{item:Greatreach Bracers, Greater}
\hypertarget{item:Greatreach Bracers, Greater}{\subsubsection{Greatreach Bracers, Greater\hfill\nth{17} (125,000 gp)}}

Your \glossterm{reach} is increased by 10 feet.



\vspace{0.25em}
\spelltwocol{\textbf{Type}: Bracers}{}
\textbf{Materials}: Bone, leather, metal, wood


\lowercase{\hypertarget{item:Hexproof Amulet, Greater}{}}\label{item:Hexproof Amulet, Greater}
\hypertarget{item:Hexproof Amulet, Greater}{\subsubsection{Hexproof Amulet, Greater\hfill\nth{15} (55,000 gp)}}

You gain a \plus4 bonus to defenses against \glossterm{magical} abilities that target you directly.
This does not protect you from abilities that affect an area.



\vspace{0.25em}
\spelltwocol{\textbf{Type}: Amulet}{}
\textbf{Materials}: Jewelry


\lowercase{\hypertarget{item:Hexward Amulet}{}}\label{item:Hexward Amulet}
\hypertarget{item:Hexward Amulet}{\subsubsection{Hexward Amulet\hfill\nth{9} (4,000 gp)}}

You gain a \plus2 bonus to defenses against \glossterm{magical} abilities that target you directly.
This does not protect you from abilities that affect an area.



\vspace{0.25em}
\spelltwocol{\textbf{Type}: Amulet}{}
\textbf{Materials}: Jewelry


\lowercase{\hypertarget{item:Hidden Armor}{}}\label{item:Hidden Armor}
\hypertarget{item:Hidden Armor}{\subsubsection{Hidden Armor\hfill\nth{4} (500 gp)}}

As a standard action, you can use this item.
If you do, it appears to change shape and form to assume the shape of a normal set of clothing.
You may choose the design of the clothing.
The item retains all of its properties, including weight and sound, while disguised in this way.
Only its visual appearance is altered.

Alternately, you may return the armor to its original appearance.



\vspace{0.25em}
\spelltwocol{\textbf{Type}: Body armor}{\parhead*{Tags} \glossterm{Sensation}}
\textbf{Materials}: Bone, metal


\lowercase{\hypertarget{item:Hidden Armor, Greater}{}}\label{item:Hidden Armor, Greater}
\hypertarget{item:Hidden Armor, Greater}{\subsubsection{Hidden Armor, Greater\hfill\nth{9} (4,000 gp)}}

This item functions like the \mitem{hidden armor} item, except that the item also makes sound appropriate to its disguised form while disguised.



\vspace{0.25em}
\spelltwocol{\textbf{Type}: Body armor}{\parhead*{Tags} \glossterm{Sensation}}
\textbf{Materials}: Bone, metal


\lowercase{\hypertarget{item:Lifekeeping Belt}{}}\label{item:Lifekeeping Belt}
\hypertarget{item:Lifekeeping Belt}{\subsubsection{Lifekeeping Belt\hfill\nth{7} (1,800 gp)}}

You gain a \plus1 \glossterm{magic bonus} to \glossterm{wound rolls}.



\vspace{0.25em}
\spelltwocol{\textbf{Type}: Belt}{}
\textbf{Materials}: Leather, textiles


\lowercase{\hypertarget{item:Lifekeeping Belt, Greater}{}}\label{item:Lifekeeping Belt, Greater}
\hypertarget{item:Lifekeeping Belt, Greater}{\subsubsection{Lifekeeping Belt, Greater\hfill\nth{13} (25,000 gp)}}

You gain a \plus2 \glossterm{magic bonus} to \glossterm{wound rolls}.



\vspace{0.25em}
\spelltwocol{\textbf{Type}: Belt}{}
\textbf{Materials}: Leather, textiles


\lowercase{\hypertarget{item:Lifekeeping Belt, Supreme}{}}\label{item:Lifekeeping Belt, Supreme}
\hypertarget{item:Lifekeeping Belt, Supreme}{\subsubsection{Lifekeeping Belt, Supreme\hfill\nth{19} (280,000 gp)}}

You gain a \plus3 \glossterm{magic bonus} to \glossterm{wound rolls}.



\vspace{0.25em}
\spelltwocol{\textbf{Type}: Belt}{}
\textbf{Materials}: Leather, textiles


\lowercase{\hypertarget{item:Mask of Air}{}}\label{item:Mask of Air}
\hypertarget{item:Mask of Air}{\subsubsection{Mask of Air\hfill\nth{9} (4,000 gp)}}

If you breathe through this mask, you breathe in clean, fresh air, regardless of your environment.
This can protect you from inhaled poisons and similar effects.



\vspace{0.25em}
\spelltwocol{\textbf{Type}: Mask}{}
\textbf{Materials}: Textiles


\lowercase{\hypertarget{item:Mask of Water Breathing}{}}\label{item:Mask of Water Breathing}
\hypertarget{item:Mask of Water Breathing}{\subsubsection{Mask of Water Breathing\hfill\nth{4} (500 gp)}}

You can breathe water through this mask as easily as a human breaths air.
This does not grant you the ability to breathe other liquids.



\vspace{0.25em}
\spelltwocol{\textbf{Type}: Mask}{}
\textbf{Materials}: Textiles


\lowercase{\hypertarget{item:Ocular Circlet}{}}\label{item:Ocular Circlet}
\hypertarget{item:Ocular Circlet}{\subsubsection{Ocular Circlet\hfill\nth{3} (250 gp)}}

As a \glossterm{standard action}, you can concentrate to use this item.
If you do, a \glossterm{scrying sensor} appears floating in the air in an unoccupied square within \rngclose range.
As long as you \glossterm{sustain} the effect as a standard action, you see through the sensor instead of from your body.

While viewing through the sensor, your visual acuity is the same as your normal body,
except that it does not share the benefits of any \glossterm{magical} effects that improve your vision.
You otherwise act normally, though you may have difficulty moving or taking actions if the sensor cannot see your body or your intended targets, effectively making you \blinded.



\vspace{0.25em}
\spelltwocol{\textbf{Type}: Circlet}{\parhead*{Tags} \glossterm{Scrying}}
\textbf{Materials}: Bone, metal


\lowercase{\hypertarget{item:Ocular Circlet, Greater}{}}\label{item:Ocular Circlet, Greater}
\hypertarget{item:Ocular Circlet, Greater}{\subsubsection{Ocular Circlet, Greater\hfill\nth{9} (4,000 gp)}}

This item functions like the \mitem{ocular circlet}, except that it only takes a \glossterm{minor action} to activate and sustain the item's effect.
In addition, the sensor appears anywhere within \rngmed range.



\vspace{0.25em}
\spelltwocol{\textbf{Type}: Circlet}{\parhead*{Tags} \glossterm{Scrying}}
\textbf{Materials}: Bone, metal


\lowercase{\hypertarget{item:Protective Armor}{}}\label{item:Protective Armor}
\hypertarget{item:Protective Armor}{\subsubsection{Protective Armor\hfill\nth{7} (1,800 gp)}}

You gain a \plus1 \glossterm{magic bonus} to Armor defense.



\vspace{0.25em}
\spelltwocol{\textbf{Type}: Body armor}{}
\textbf{Materials}: Bone, metal


\lowercase{\hypertarget{item:Protective Shield}{}}\label{item:Protective Shield}
\hypertarget{item:Protective Shield}{\subsubsection{Protective Shield\hfill\nth{7} (1,800 gp)}}

You gain a \plus1 \glossterm{magic bonus} to Armor defense.



\vspace{0.25em}
\spelltwocol{\textbf{Type}: Shield}{}
\textbf{Materials}: Bone, metal, wood


\lowercase{\hypertarget{item:Ring of Angel's Grace}{}}\label{item:Ring of Angel's Grace}
\hypertarget{item:Ring of Angel's Grace}{\subsubsection{Ring of Angel's Grace\hfill\nth{9} (4,000 gp)}}

You gain \plus2 \glossterm{magic bonus} to Mental defense.
In addition, if you fall at least 20 feet, ephemeral angel wings spring from your back.
The wings slow your fall to a rate of 60 feet per round, preventing you from taking \glossterm{falling damage}.



\vspace{0.25em}
\spelltwocol{\textbf{Type}: Ring}{}
\textbf{Materials}: Bone, jewelry, metal, wood


\lowercase{\hypertarget{item:Ring of Elemental Endurance}{}}\label{item:Ring of Elemental Endurance}
\hypertarget{item:Ring of Elemental Endurance}{\subsubsection{Ring of Elemental Endurance\hfill\nth{2} (125 gp)}}

You can exist comfortably in conditions between \minus50 and 140 degrees Fahrenheit without any ill effects.
You suffer the normal penalties in temperatures outside of that range.



\vspace{0.25em}
\spelltwocol{\textbf{Type}: Ring}{}
\textbf{Materials}: Bone, jewelry, metal, wood


\lowercase{\hypertarget{item:Ring of Energy Resistance}{}}\label{item:Ring of Energy Resistance}
\hypertarget{item:Ring of Energy Resistance}{\subsubsection{Ring of Energy Resistance\hfill\nth{6} (1,200 gp)}}

You gain a \glossterm{magic bonus} equal to half this item's \glossterm{power} to \glossterm{resistances} against \glossterm{energy damage}.
When you resist energy with this ability, the ring sheds light as a torch until the end of the next round.
The color of the light depends on the energy damage resisted: green for acid, blue for cold, yellow for electricity, and red for fire.



\vspace{0.25em}
\spelltwocol{\textbf{Type}: Ring}{}
\textbf{Materials}: Bone, jewelry, metal, wood


\lowercase{\hypertarget{item:Ring of Energy Resistance, Greater}{}}\label{item:Ring of Energy Resistance, Greater}
\hypertarget{item:Ring of Energy Resistance, Greater}{\subsubsection{Ring of Energy Resistance, Greater\hfill\nth{15} (55,000 gp)}}

This item functions like the \mitem{ring of energy resistance}, except that the bonus is equal to the item's \glossterm{power}.



\vspace{0.25em}
\spelltwocol{\textbf{Type}: Ring}{}
\textbf{Materials}: Bone, jewelry, metal, wood


\lowercase{\hypertarget{item:Ring of Nourishment}{}}\label{item:Ring of Nourishment}
\hypertarget{item:Ring of Nourishment}{\subsubsection{Ring of Nourishment\hfill\nth{3} (250 gp)}}

You continuously gain nourishment, and no longer need to eat or drink.
This ring must be worn for 24 hours before it begins to work.



\vspace{0.25em}
\spelltwocol{\textbf{Type}: Ring}{\parhead*{Tags} \glossterm{Creation}}
\textbf{Materials}: Bone, jewelry, metal, wood


\lowercase{\hypertarget{item:Ring of Protection}{}}\label{item:Ring of Protection}
\hypertarget{item:Ring of Protection}{\subsubsection{Ring of Protection\hfill\nth{8} (2,750 gp)}}

This ring creates a transluscent shield-like barrier that floats in front of you, deflecting enemy attacks.
You gain a \plus1 \glossterm{magic bonus} to Armor and Reflex defenses.
This does not stack with the defense bonus from any shields you use.



\vspace{0.25em}
\spelltwocol{\textbf{Type}: Ring}{}
\textbf{Materials}: Bone, jewelry, metal, wood


\lowercase{\hypertarget{item:Ring of Protection, Greater}{}}\label{item:Ring of Protection, Greater}
\hypertarget{item:Ring of Protection, Greater}{\subsubsection{Ring of Protection, Greater\hfill\nth{16} (85,000 gp)}}

This item functions like the \magicitem{ring of protection}, except that the bonus increases to \plus2.



\vspace{0.25em}
\spelltwocol{\textbf{Type}: Ring}{}
\textbf{Materials}: Bone, jewelry, metal, wood


\lowercase{\hypertarget{item:Ring of Regeneration}{}}\label{item:Ring of Regeneration}
\hypertarget{item:Ring of Regeneration}{\subsubsection{Ring of Regeneration\hfill\nth{8} (2,750 gp)}}

As a standard action, you can use this ring to enhance your healing.
When you do, you gain a \plus1 bonus to the \glossterm{wound roll} of your most recent \glossterm{vital wound}.



\vspace{0.25em}
\spelltwocol{\textbf{Type}: Ring}{}
\textbf{Materials}: Bone, jewelry, metal, wood


\lowercase{\hypertarget{item:Ring of Regeneration, Greater}{}}\label{item:Ring of Regeneration, Greater}
\hypertarget{item:Ring of Regeneration, Greater}{\subsubsection{Ring of Regeneration, Greater\hfill\nth{12} (16,000 gp)}}

At the end of each round, you gain a \plus1 bonus to the \glossterm{wound roll} of your most recent \glossterm{vital wound}.



\vspace{0.25em}
\spelltwocol{\textbf{Type}: Ring}{}
\textbf{Materials}: Bone, jewelry, metal, wood


\lowercase{\hypertarget{item:Ring of Sustenance}{}}\label{item:Ring of Sustenance}
\hypertarget{item:Ring of Sustenance}{\subsubsection{Ring of Sustenance\hfill\nth{7} (1,800 gp)}}

You continuously gain nourishment, and no longer need to eat or drink.
In addition, you need only one-quarter your normal amount of sleep (or similar activity, such as elven trance) each day.

The ring must be worn for 24 hours before it begins to work.



\vspace{0.25em}
\spelltwocol{\textbf{Type}: Ring}{\parhead*{Tags} \glossterm{Creation}}
\textbf{Materials}: Bone, jewelry, metal, wood


\lowercase{\hypertarget{item:Seven League Boots}{}}\label{item:Seven League Boots}
\hypertarget{item:Seven League Boots}{\subsubsection{Seven League Boots\hfill\nth{12} (16,000 gp)}}

As a standard action, you can spend an \glossterm{action point} to activate these boots.
If you do, you teleport exactly 25 miles in a direction you specify.
If this would place you within a solid object or otherwise impossible space, the boots will shunt you up to 1,000 feet in any direction to the closest available space.
If there is no available space within 1,000 feet of your intended destination, the effect fails and you take \glossterm{standard damage} \minus1d.



\vspace{0.25em}
\spelltwocol{\textbf{Type}: Boots}{}
\textbf{Materials}: Bone, leather, metal


\lowercase{\hypertarget{item:Shield of Arrow Catching}{}}\label{item:Shield of Arrow Catching}
\hypertarget{item:Shield of Arrow Catching}{\subsubsection{Shield of Arrow Catching\hfill\nth{5} (800 gp)}}

When a creature within a \areamed radius emanation from you would be attacked by a ranged weapon, the attack is redirected to target you instead.
Resolve the attack as if it had initially targeted you, except that the attack is not affected by cover or concealment.
This item can only affect projectiles and thrown objects that are Small or smaller.



\vspace{0.25em}
\spelltwocol{\textbf{Type}: Shield}{}
\textbf{Materials}: Bone, metal, wood


\lowercase{\hypertarget{item:Shield of Arrow Catching, Greater}{}}\label{item:Shield of Arrow Catching, Greater}
\hypertarget{item:Shield of Arrow Catching, Greater}{\subsubsection{Shield of Arrow Catching, Greater\hfill\nth{10} (6,500 gp)}}

This item functions like the \mitem{shield of arrow catching} item, except that it affects a \arealarge radius from you.
In addition, you may choose to exclude creature from this item's effect, allowing projectiles to target nearby foes normally.



\vspace{0.25em}
\spelltwocol{\textbf{Type}: Shield}{}
\textbf{Materials}: Bone, metal, wood


\lowercase{\hypertarget{item:Shield of Arrow Deflection}{}}\label{item:Shield of Arrow Deflection}
\hypertarget{item:Shield of Arrow Deflection}{\subsubsection{Shield of Arrow Deflection\hfill\nth{2} (125 gp)}}

As a \glossterm{minor action}, you can activate this shield.
If you do, you gain a \plus5 \glossterm{magic bonus} to Armor defense against ranged \glossterm{physical attacks} from weapons or projectiles that are Small or smaller.
This is a \glossterm{Swift} ability, and it lasts until the end of the round.



\vspace{0.25em}
\spelltwocol{\textbf{Type}: Shield}{}
\textbf{Materials}: Bone, metal, wood


\lowercase{\hypertarget{item:Shield of Arrow Deflection, Greater}{}}\label{item:Shield of Arrow Deflection, Greater}
\hypertarget{item:Shield of Arrow Deflection, Greater}{\subsubsection{Shield of Arrow Deflection, Greater\hfill\nth{8} (2,750 gp)}}

You gain a \plus5 \glossterm{magic bonus} to Armor defense against ranged \glossterm{physical attacks} from weapons or projectiles that are Small or smaller.



\vspace{0.25em}
\spelltwocol{\textbf{Type}: Shield}{}
\textbf{Materials}: Bone, metal, wood


\lowercase{\hypertarget{item:Shield of Bashing}{}}\label{item:Shield of Bashing}
\hypertarget{item:Shield of Bashing}{\subsubsection{Shield of Bashing\hfill\nth{2} (125 gp)}}

You gain a \plus1d \glossterm{magic bonus} to damage with \glossterm{strikes} using this shield.



\vspace{0.25em}
\spelltwocol{\textbf{Type}: Shield}{}
\textbf{Materials}: Bone, metal, wood


\lowercase{\hypertarget{item:Shield of Bashing, Greater}{}}\label{item:Shield of Bashing, Greater}
\hypertarget{item:Shield of Bashing, Greater}{\subsubsection{Shield of Bashing, Greater\hfill\nth{12} (16,000 gp)}}

You gain a \plus2d \glossterm{magic bonus} to damage with \glossterm{strikes} using this shield.



\vspace{0.25em}
\spelltwocol{\textbf{Type}: Shield}{}
\textbf{Materials}: Bone, metal, wood


\lowercase{\hypertarget{item:Shield of Boulder Catching}{}}\label{item:Shield of Boulder Catching}
\hypertarget{item:Shield of Boulder Catching}{\subsubsection{Shield of Boulder Catching\hfill\nth{8} (2,750 gp)}}

This item functions like the \mitem{shield of arrow catching} item, except that it can affect projectile and thrown objects of up to Large size.



\vspace{0.25em}
\spelltwocol{\textbf{Type}: Shield}{}
\textbf{Materials}: Bone, metal, wood


\lowercase{\hypertarget{item:Shield of Boulder Deflection}{}}\label{item:Shield of Boulder Deflection}
\hypertarget{item:Shield of Boulder Deflection}{\subsubsection{Shield of Boulder Deflection\hfill\nth{6} (1,200 gp)}}

This item functions like the \mitem{shield of arrow deflection} item, except that it can affect weapons and projectiles of up to Large size.



\vspace{0.25em}
\spelltwocol{\textbf{Type}: Shield}{}
\textbf{Materials}: Bone, metal, wood


\lowercase{\hypertarget{item:Shield of Boulder Deflection, Greater}{}}\label{item:Shield of Boulder Deflection, Greater}
\hypertarget{item:Shield of Boulder Deflection, Greater}{\subsubsection{Shield of Boulder Deflection, Greater\hfill\nth{12} (16,000 gp)}}

This item functions like the \mitem{greater shield of arrow deflection} item, except that it can affect weapons and projectiles of up to Large size.



\vspace{0.25em}
\spelltwocol{\textbf{Type}: Shield}{}
\textbf{Materials}: Bone, metal, wood


\lowercase{\hypertarget{item:Shield of Mystic Reflection}{}}\label{item:Shield of Mystic Reflection}
\hypertarget{item:Shield of Mystic Reflection}{\subsubsection{Shield of Mystic Reflection\hfill\nth{12} (16,000 gp)}}

As a standard action, you can activate this shield.
When you do, any \glossterm{targeted} \glossterm{magical} abilities that would target you this round are redirected to target the creature using that ability instead of you.
Any other targets of the ability are affected normally.
This is a \glossterm{Swift} ability, so it affects any abilities targeting you in the phase you activate the item.



\vspace{0.25em}
\spelltwocol{\textbf{Type}: Shield}{}
\textbf{Materials}: Bone, metal, wood


\lowercase{\hypertarget{item:Throwing Gloves}{}}\label{item:Throwing Gloves}
\hypertarget{item:Throwing Gloves}{\subsubsection{Throwing Gloves\hfill\nth{4} (500 gp)}}

% TODO: reference basic "not designed to be thrown" mechanics?
You can throw any item as if it was designed to be thrown.
This does not improve your ability to throw items designed to be thrown, such as darts.



\vspace{0.25em}
\spelltwocol{\textbf{Type}: Gloves}{}
\textbf{Materials}: Leather


\lowercase{\hypertarget{item:Titan Gauntlets}{}}\label{item:Titan Gauntlets}
\hypertarget{item:Titan Gauntlets}{\subsubsection{Titan Gauntlets\hfill\nth{13} (25,000 gp)}}

You gain a \plus1d \glossterm{magic bonus} to damage with \glossterm{strikes}.



\vspace{0.25em}
\spelltwocol{\textbf{Type}: Gauntlet}{}
\textbf{Materials}: Bone, metal, wood


\lowercase{\hypertarget{item:Torchlight Gloves}{}}\label{item:Torchlight Gloves}
\hypertarget{item:Torchlight Gloves}{\subsubsection{Torchlight Gloves\hfill\nth{2} (125 gp)}}

These gloves shed light as a torch.
As a \glossterm{standard action}, you may snap your fingers to suppress or resume the light from either or both gloves.



\vspace{0.25em}
\spelltwocol{\textbf{Type}: Gloves}{}
\textbf{Materials}: Leather


\lowercase{\hypertarget{item:Vanishing Cloak}{}}\label{item:Vanishing Cloak}
\hypertarget{item:Vanishing Cloak}{\subsubsection{Vanishing Cloak\hfill\nth{13} (25,000 gp)}}

As a standard action, you can activate this cloak.
When you do, you teleport to an unoccupied location within \rngmed range of your original location.
In addition, you become \glossterm{invisible} until the end of the next round.

If your intended destination is invalid, or if your teleportation otherwise fails, you still become invisible.



\vspace{0.25em}
\spelltwocol{\textbf{Type}: Cloak}{\parhead*{Tags} \glossterm{Sensation}}
\textbf{Materials}: Textiles


\lowercase{\hypertarget{item:Winged Boots}{}}\label{item:Winged Boots}
\hypertarget{item:Winged Boots}{\subsubsection{Winged Boots\hfill\nth{10} (6,500 gp)}}

You gain a \glossterm{fly speed} equal to your \glossterm{base speed}.
However, the boots are not strong enough to keep you aloft indefinitely.
At the end of each round, if you are not standing on solid ground, the magic of the boots fails and you fall normally.
The boots begin working again at the end of the next round, even if you have not yet hit the ground.



\vspace{0.25em}
\spelltwocol{\textbf{Type}: Boots}{}
\textbf{Materials}: Bone, leather, metal

