\chapter{Magic Items}

Magic items are objects that have been imbued with magical energy. They can take almost any form, and their potential uses are only as limited as the magic that created them.

\section{Magic Item Types}
    Magic items are divided into four broad categories:
    \begin{itemize}
        \item Weapons are used to make physical attacks. They provide access to their abilities when wielded.
            A \mitem{flaming longsword} and a \mitem{vampiric scythe} are weapons.
        \item Implements are used to cast spells. They provide access to their abilities when wielded.
            A \mitem{staff of fire} and a \mitem{staff of time} are implements.
        \item Apparel items are usually not used individually. They provide access to their abilities when worn.
            A \mitem{flaming burst full plate} and a \mitem{ring of protection} are apparel items.
        \item Tools provide access to their abilities when used in some way.
            A \mitem{bag of carrying} is a tool.
    \end{itemize}

\section{Using Magic Items}

    \subsection{Item Activation}

        Some magic items have to be explicitly activated to have unusual effects.
        For example, the \mitem{seven league boots} can be activated to teleport you across great distances.
        Other magic items constantly have magical effects.
        For example, a \mitem{flaming} sword is on fire.

        The description of a magic item effect will specify what mechanical actions must be taken, if any, to activate the effects of the item.
        For example, a healing belt requires spending an \glossterm{action point} as a \glossterm{standard action}.
        However, the item description will not specify the exact nature of the action.
        Different items, even if they have the same effect, can have different physical actions that are required to activate the item.
        These minor actions can come in one of the following forms:
        \begin{itemize}
           \item Command word: You must speak a specific word that the item will hear and react to.
                For example, you may need to say the word ``healing'' in Elven to activate an item that heals you.
            \item Mental command: You must mentally direct the item to activate, such as by visualizing the item or thinking a particular word.
                % TODO: does this item exist
                For example, you may need to imagine a warm blanket around you to activate an item that protects you from cold damage or environmental effects.
            \item Physical motion: You must perform a specific physical motion, usually involving the item in some way.
                For example, you may need to rapidly stomp one foot on the ground to activate an item that allows you to move faster.
        \end{itemize}

        % TODO: table of random item activations?

    \subsection{Item Limitations}

        There are three restrictions on your ability to use magic items.
        First, you cannot equip two apparel items that take up the same physical location on your body.
        For example, you cannot equip two different gauntlet sets and gain the effects of both, but you could equip several amulets or up to ten rings.
        Second, most magic items other than tools require you to attune to them to gain their effect.
        You can attune to a magic item with the \textit{item attunement} ability, below.
        Third, you cannot attune to two items with the same name, or if one is simply a Greater or Lesser version of the other.

        \subsubsection{Item Slots}\label{Item Slots}
            Normally, attuning to an ability or magic item reduces your maximum action points by one.
            However, you can attune to a certain number of items without penalizing your action points.
            You start with one \glossterm{item slot}, and acquire more as you gain levels (see \tref{Character Advancement}).

            When you attune to a magic item, you can spend an \glossterm{item slot} instead of an \glossterm{action point}.
            This is summarized in the description of the \textit{item attunement} ability.

            Dismissing your attunement to an item does not restore the spent item slot.
            You regain all spent \glossterm{item slots} from dismissed attunements after a \glossterm{long rest}.

        \subsubsection{Item Attunement}

            As a standard action, you can spend an \glossterm{action point} or an \glossterm{item slot} to use the \textit{item attunement} ability to attune to items.

            \begin{ability}{Item Attunement}[\glossterm{Attune} (self)]
                Choose a magic item you are touching.
                Any abilities the target has that require attunement become active, allowing you to use its full potential.

                If you spent an \glossterm{item slot} to use this ability, attuning to this ability does not affect the number of action points you recover when you rest.
            \end{ability}

            \parhead{Shared Item Attunement} Multiple creatures can attune to the same item simultaneously.
            Since most items only function while worn or wielded, this does not usually allow multiple creatures to gain the benefits of the item.
            However, the creatures can swap the item between them without having to reattune to it each time.

    \subsubsection{Item Power}\label{Item Power}
        The \glossterm{power} of an item depends on its level.
        If the item is not being attuned to by a creature, its power is equal to its level.
        If a creature is attuning to the item, its power is equal to its level or the level of the attuning creature, whichever is higher.

        An item's \glossterm{power} also affects its defenses.
        Its Fortitude and Mental defenses are equal to 10 \add its \glossterm{power}.
        Its Armor defense and Reflex defense are not affected by its \glossterm{power}, and are solely determined by its size and shape.

    \subsection{Removing Magic Items}
        Unless otherwise noted, magic items that have effects on the creature using the item must continue to be worn or held as long as the effect lasts.
        If a magic item has an ability with a duration, removing the item also ends the ability.
        Items which are consumed when used or which do not affect their user are unaffected by this rule.

\section{Item Description Format}
    TODO

\section{Apparel}

    
\begin{longtablewrapper}
\begin{longtable}{p{15em} p{3em} p{6em} p{25em} p{3em}}

\lcaption{Apparel Items} \\
\tb{Name} & \tb{Level} & \tb{Typical Price} & \tb{Description} & \tb{Page} \tableheaderrule
Bracers of Archery & \nth{1} & 50 gp & Grants bow proficiency & \pageref{item:Bracers of Archery} \\
Belt of Healing & \nth{2} & 125 gp & Grants healing & \pageref{item:Belt of Healing} \\
Boots of the Winterlands & \nth{2} & 125 gp & Eases travel in cold areas & \pageref{item:Boots of the Winterlands} \\
Bracers of Armor & \nth{2} & 125 gp & Grants invisible armor & \pageref{item:Bracers of Armor} \\
Gauntlets of Improvisation & \nth{2} & 125 gp & Grants \plus1d damage with improvised weapons & \pageref{item:Gauntlets of Improvisation} \\
Ring of Elemental Endurance & \nth{2} & 125 gp & Grants tolerance of temperature extremes & \pageref{item:Ring of Elemental Endurance} \\
Shield of Bashing & \nth{2} & 125 gp & Grants \plus2 power & \pageref{item:Shield of Bashing} \\
Torchlight Gloves & \nth{2} & 125 gp & Sheds light as a torch & \pageref{item:Torchlight Gloves} \\
Ocular Circlet & \nth{3} & 250 gp & Can allow you to see at a distance & \pageref{item:Ocular Circlet} \\
Ring of Nourishment & \nth{3} & 250 gp & Provides food and water & \pageref{item:Ring of Nourishment} \\
Boots of Earth's Embrace & \nth{4} & 500 gp & Grants immunity to forced movement & \pageref{item:Boots of Earth's Embrace} \\
Boots of Elvenkind & \nth{4} & 500 gp & Grants \plus2 Stealth & \pageref{item:Boots of Elvenkind} \\
Circlet of Persuasion & \nth{4} & 500 gp & Grants \plus2 Persuasion & \pageref{item:Circlet of Persuasion} \\
Gauntlet of the Ram & \nth{4} & 500 gp & Knocks back foe when used to strike & \pageref{item:Gauntlet of the Ram} \\
Mask of Water Breathing & \nth{4} & 500 gp & Allows breathing water like air & \pageref{item:Mask of Water Breathing} \\
Throwing Gloves & \nth{4} & 500 gp & Allows throwing any item accurately & \pageref{item:Throwing Gloves} \\
Acid Coated & \nth{5} & 800 gp & Deals acid damage to anything it touches & \pageref{item:Acid Coated} \\
Amulet of Translocation & \nth{5} & 800 gp & Grants ability to teleport up to 30 feet & \pageref{item:Amulet of Translocation} \\
Circlet of Blasting & \nth{5} & 800 gp & Can blast foe with fire & \pageref{item:Circlet of Blasting} \\
Hidden Armor & \nth{5} & 800 gp & Can look like normal clothing & \pageref{item:Hidden Armor} \\
Protective Armor & \nth{5} & 800 gp & Grants \plus1 Armor defense & \pageref{item:Protective Armor} \\
Protective Shield & \nth{5} & 800 gp & Grants \plus1 Armor defense & \pageref{item:Protective Shield} \\
Shield of Arrow Catching & \nth{5} & 800 gp & Redirects small nearby projectiles to hit you & \pageref{item:Shield of Arrow Catching} \\
Shield of Arrow Deflection & \nth{5} & 800 gp & Blocks small projectiles & \pageref{item:Shield of Arrow Deflection} \\
Translocation & \nth{5} & 800 gp & Grants ability to teleport up to 30 feet & \pageref{item:Translocation} \\
Agile & \nth{6} & 1,200 gp & Grants \plus2 Reflex defense & \pageref{item:Agile} \\
Amulet of Health & \nth{6} & 1,200 gp & Grants 2 additional hit points & \pageref{item:Amulet of Health} \\
Amulet of Nondetection & \nth{6} & 1,200 gp & Grants \plus4 to defenses against detection & \pageref{item:Amulet of Nondetection} \\
Boots of Speed & \nth{6} & 1,200 gp & Increases speed by ten feet & \pageref{item:Boots of Speed} \\
Featherlight Armor & \nth{6} & 1,200 gp & Reduces encumbrance by 1 & \pageref{item:Featherlight Armor} \\
Fortified & \nth{6} & 1,200 gp & Grants \plus2 Fortitude defense & \pageref{item:Fortified} \\
Quilled Cloak & \nth{6} & 1,200 gp & Deals damage to creatures that grapple you & \pageref{item:Quilled Cloak} \\
Willguard & \nth{6} & 1,200 gp & Grants \plus2 Mental defense & \pageref{item:Willguard} \\
Anchoring & \nth{7} & 1,800 gp & Protects you from most forced movement attacks & \pageref{item:Anchoring} \\
Armor of Fortification & \nth{7} & 1,800 gp & Reduces critical hits from strikes & \pageref{item:Armor of Fortification} \\
Boots of Water Walking & \nth{7} & 1,800 gp & Allows walking on liquids & \pageref{item:Boots of Water Walking} \\
Boots of the Skydancer & \nth{7} & 1,800 gp & Can walk on air & \pageref{item:Boots of the Skydancer} \\
Bracers of Archery, Greater & \nth{7} & 1,800 gp & Grants bow proficiency, \plus1 ranged accuracy & \pageref{item:Bracers of Archery, Greater} \\
Bracers of Repulsion & \nth{7} & 1,800 gp & Can knock nearby creatures back & \pageref{item:Bracers of Repulsion} \\
Crown of Lightning & \nth{7} & 1,800 gp & Continuously damages nearby enemies & \pageref{item:Crown of Lightning} \\
Gauntlet of the Ram, Greater & \nth{7} & 1,800 gp & Knocks back foe farther when use to strike & \pageref{item:Gauntlet of the Ram, Greater} \\
Gauntlets of Improvisation, Greater & \nth{7} & 1,800 gp & Grants \plus2d damage with improvised weapons & \pageref{item:Gauntlets of Improvisation, Greater} \\
Gloves of Spell Investment & \nth{7} & 1,800 gp & Can invest a spell to cast later & \pageref{item:Gloves of Spell Investment} \\
Hexward Amulet & \nth{7} & 1,800 gp & Grants \plus1 defenses against targeted magical attacks & \pageref{item:Hexward Amulet} \\
Lifekeeping Belt & \nth{7} & 1,800 gp & Grants \plus1 bonus to \glossterm{vital rolls} & \pageref{item:Lifekeeping Belt} \\
Ring of Sustenance & \nth{7} & 1,800 gp & Provides food, water, and rest & \pageref{item:Ring of Sustenance} \\
Amulet of Mighty Fists & \nth{8} & 2,750 gp & Grants \plus2 power with natural and unarmed attacks & \pageref{item:Amulet of Mighty Fists} \\
Armor of Energy Resistance & \nth{8} & 2,750 gp & Reduces energy damage & \pageref{item:Armor of Energy Resistance} \\
Armor of Invulnerability & \nth{8} & 2,750 gp & Reduces physical damage & \pageref{item:Armor of Invulnerability} \\
Assassin's Cloak & \nth{8} & 2,750 gp & Grants invisibility while inactive & \pageref{item:Assassin's Cloak} \\
Avian Cloak & \nth{8} & 2,750 gp & Grants a glide speed & \pageref{item:Avian Cloak} \\
Belt of Healing, Greater & \nth{8} & 2,750 gp & Grants more healing & \pageref{item:Belt of Healing, Greater} \\
Boots of Gravitation & \nth{8} & 2,750 gp & Redirects personal gravity & \pageref{item:Boots of Gravitation} \\
Cloak of Mist & \nth{8} & 2,750 gp & Fills nearby area with fog & \pageref{item:Cloak of Mist} \\
Ring of Protection & \nth{8} & 2,750 gp & Grants \plus1 to Armor and Reflex defenses & \pageref{item:Ring of Protection} \\
Shield of Boulder Catching & \nth{8} & 2,750 gp & Redirects large nearby projectiles to hit you & \pageref{item:Shield of Boulder Catching} \\
Shield of Boulder Deflection & \nth{8} & 2,750 gp & Can block large projectiles & \pageref{item:Shield of Boulder Deflection} \\
Crown of Flame & \nth{9} & 4,000 gp & Grants nearby allies immunity to fire damage & \pageref{item:Crown of Flame} \\
Greatreach Bracers & \nth{9} & 4,000 gp & Increases reach by five feet & \pageref{item:Greatreach Bracers} \\
Hidden Armor, Greater & \nth{9} & 4,000 gp & Can look and sound like normal clothing & \pageref{item:Hidden Armor, Greater} \\
Mask of Air & \nth{9} & 4,000 gp & Allows breathing in any environment & \pageref{item:Mask of Air} \\
Ocular Circlet, Greater & \nth{9} & 4,000 gp & Can allow you to see at a greater distance & \pageref{item:Ocular Circlet, Greater} \\
Ring of Angel's Grace & \nth{9} & 4,000 gp & Grants \plus2 Mental and slows falls & \pageref{item:Ring of Angel's Grace} \\
Boots of Speed, Greater & \nth{10} & 6,500 gp & Increases speed by twenty feet & \pageref{item:Boots of Speed, Greater} \\
Circlet of Blasting, Greater & \nth{10} & 6,500 gp & Can blast foe with intense fire & \pageref{item:Circlet of Blasting, Greater} \\
Crater Boots & \nth{10} & 6,500 gp & Deals your falling damage to enemies & \pageref{item:Crater Boots} \\
Titan Gauntlets & \nth{10} & 6,500 gp & Grants \plus2 \glossterm{mundane} power & \pageref{item:Titan Gauntlets} \\
Winged Boots & \nth{10} & 6,500 gp & Grants limited flight & \pageref{item:Winged Boots} \\
Amulet of Translocation, Greater & \nth{11} & 10,000 gp & Grants ability to teleport up to 100 feet & \pageref{item:Amulet of Translocation, Greater} \\
Crown of Thunder & \nth{11} & 10,000 gp & Continously deafens nearby enemies & \pageref{item:Crown of Thunder} \\
Shield of Arrow Catching, Greater & \nth{11} & 10,000 gp & Selectively redirects small nearby projectiles to hit you & \pageref{item:Shield of Arrow Catching, Greater} \\
Shield of Bashing, Greater & \nth{11} & 10,000 gp & Grants \plus4 power & \pageref{item:Shield of Bashing, Greater} \\
Translocation, Greater & \nth{11} & 10,000 gp & Grants ability to teleport up to 100 feet & \pageref{item:Translocation, Greater} \\
Amulet of the Planes & \nth{12} & 16,000 gp & Aids travel with \ritual{plane shift} & \pageref{item:Amulet of the Planes} \\
Armor of Fortification, Mystic & \nth{12} & 16,000 gp & Reduces critical hits from all attacks & \pageref{item:Armor of Fortification, Mystic} \\
Boots of Freedom & \nth{12} & 16,000 gp & Grants immunity to almost all mobility restrictions & \pageref{item:Boots of Freedom} \\
Featherlight Armor, Greater & \nth{12} & 16,000 gp & Reduces encumbrance by 2 & \pageref{item:Featherlight Armor, Greater} \\
Greater Quilled Cloak & \nth{12} & 16,000 gp & Deals more damage to creatures that grapple you & \pageref{item:Greater Quilled Cloak} \\
Ring of Energy Resistance & \nth{12} & 16,000 gp & Reduces energy damage & \pageref{item:Ring of Energy Resistance} \\
Seven League Boots & \nth{12} & 16,000 gp & Teleport seven leages with a step & \pageref{item:Seven League Boots} \\
Shield of Mystic Reflection & \nth{12} & 16,000 gp & React to reflect magical attacks & \pageref{item:Shield of Mystic Reflection} \\
Anchoring, Greater & \nth{13} & 25,000 gp & Protects you from all forced movement and teleportation attacks & \pageref{item:Anchoring, Greater} \\
Assassin's Cloak, Greater & \nth{13} & 25,000 gp & Grants longer invisibility while inactive & \pageref{item:Assassin's Cloak, Greater} \\
Boots of the Skydancer, Greater & \nth{13} & 25,000 gp & description & \pageref{item:Boots of the Skydancer, Greater} \\
Crown of Frost & \nth{13} & 25,000 gp & Continuously damages nearby enemies & \pageref{item:Crown of Frost} \\
Gloves of Spell Investment, Greater & \nth{13} & 25,000 gp & Can invest two spells to cast later & \pageref{item:Gloves of Spell Investment, Greater} \\
Hexproof Amulet, Greater & \nth{13} & 25,000 gp & Grants \plus2 defenses against targeted magical attacks & \pageref{item:Hexproof Amulet, Greater} \\
Lifekeeping Belt, Greater & \nth{13} & 25,000 gp & Grants \plus2 bonus to \glossterm{vital rolls} & \pageref{item:Lifekeeping Belt, Greater} \\
Vanishing Cloak & \nth{13} & 25,000 gp & Can teleport a short distance and grant invisibility & \pageref{item:Vanishing Cloak} \\
Amulet of Nondetection, Greater & \nth{14} & 37,000 gp & Grants \plus8 to defenses against detection & \pageref{item:Amulet of Nondetection, Greater} \\
Armor of Energy Resistance, Greater & \nth{14} & 37,000 gp & Significantly reduces energy damage & \pageref{item:Armor of Energy Resistance, Greater} \\
Armor of Invulnerability & \nth{14} & 37,000 gp & Significantly reduces physical damage & \pageref{item:Armor of Invulnerability} \\
Belt of Healing, Supreme & \nth{14} & 37,000 gp & Grants more healing & \pageref{item:Belt of Healing, Supreme} \\
Boots of Speed, Supreme & \nth{14} & 37,000 gp & Increases speed by thirty feet & \pageref{item:Boots of Speed, Supreme} \\
Protective Armor, Greater & \nth{14} & 37,000 gp & Grants \plus2 Armor defense & \pageref{item:Protective Armor, Greater} \\
Protective Shield, Greater & \nth{14} & 37,000 gp & Grants \plus2 Armor defense & \pageref{item:Protective Shield, Greater} \\
Shield of Arrow Deflection, Greater & \nth{14} & 37,000 gp & Blocks small projectiles & \pageref{item:Shield of Arrow Deflection, Greater} \\
Agile, Greater & \nth{15} & 55,000 gp & Grants \plus4 Reflex defense & \pageref{item:Agile, Greater} \\
Amulet of Health, Greater & \nth{15} & 55,000 gp & Grants 4 additional hit points & \pageref{item:Amulet of Health, Greater} \\
Armor of Fortification, Greater & \nth{15} & 55,000 gp & Drastically reduces critical hits from strikes & \pageref{item:Armor of Fortification, Greater} \\
Bracers of Repulsion, Greater & \nth{15} & 55,000 gp & Can knock many nearby creatures back & \pageref{item:Bracers of Repulsion, Greater} \\
Fortified, Greater & \nth{15} & 55,000 gp & Grants \plus4 Fortitude defense & \pageref{item:Fortified, Greater} \\
Ring of Regeneration & \nth{15} & 55,000 gp & Automatically removes vital wounds & \pageref{item:Ring of Regeneration} \\
Willguard, Greater & \nth{15} & 55,000 gp & Grants \plus4 Mental defense & \pageref{item:Willguard, Greater} \\
Amulet of Mighty Fists, Greater & \nth{16} & 85,000 gp & Grants \plus4 power with natural and unarmed attacks & \pageref{item:Amulet of Mighty Fists, Greater} \\
Astral Boots & \nth{16} & 85,000 gp & Allows teleporting instead of moving & \pageref{item:Astral Boots} \\
Circlet of Blasting, Supreme & \nth{16} & 85,000 gp & Can blast foe with supremely intense fire & \pageref{item:Circlet of Blasting, Supreme} \\
Cloak of Mist, Greater & \nth{16} & 85,000 gp & Fills nearby area with thick fog & \pageref{item:Cloak of Mist, Greater} \\
Ring of Protection, Greater & \nth{16} & 85,000 gp & Grants \plus2 to Armor and Reflex defenses & \pageref{item:Ring of Protection, Greater} \\
Amulet of Translocation, Supreme & \nth{17} & 125,000 gp & Grants ability to teleport up to 300 feet & \pageref{item:Amulet of Translocation, Supreme} \\
Greatreach Bracers, Greater & \nth{17} & 125,000 gp & Increases reach by ten feet & \pageref{item:Greatreach Bracers, Greater} \\
Shield of Boulder Deflection, Greater & \nth{17} & 125,000 gp & Blocks large projectiles & \pageref{item:Shield of Boulder Deflection, Greater} \\
Translocation, Supreme & \nth{17} & 125,000 gp & Grants ability to teleport up to 300 feet & \pageref{item:Translocation, Supreme} \\
Featherlight Armor, Supreme & \nth{18} & 190,000 gp & Reduces encumbrance by 2 & \pageref{item:Featherlight Armor, Supreme} \\
Supreme Quilled Cloak & \nth{18} & 190,000 gp & Deals even more damage to creatures that grapple you & \pageref{item:Supreme Quilled Cloak} \\
Hexproof Amulet, Supreme & \nth{19} & 280,000 gp & Grants \plus3 defenses against targeted magical attacks & \pageref{item:Hexproof Amulet, Supreme} \\
Lifekeeping Belt, Supreme & \nth{19} & 280,000 gp & Grants \plus3 bonus to \glossterm{vital rolls} & \pageref{item:Lifekeeping Belt, Supreme} \\
Titan Gauntlets, Greater & \nth{19} & 280,000 gp & Grants \plus4 \glossterm{mundane} power & \pageref{item:Titan Gauntlets, Greater} \\
Armor of Energy Resistance, Supreme & \nth{20} & 400,000 gp & Drastically reduces energy damage & \pageref{item:Armor of Energy Resistance, Supreme} \\
Armor of Invulnerability, Greater & \nth{20} & 400,000 gp & Drastically reduces physical damage & \pageref{item:Armor of Invulnerability, Greater} \\
Shield of Bashing, Supreme & \nth{20} & 400,000 gp & Grants \plus6 power & \pageref{item:Shield of Bashing, Supreme} \\

\end{longtable}
\end{longtablewrapper}


    \lowercase{\hypertarget{item:Amulet of Mighty Fists}{}}\label{item:Amulet of Mighty Fists}
\hypertarget{item:Amulet of Mighty Fists}{\subsubsection{Amulet of Mighty Fists\hfill\nth{6}}}
You gain a \plus1d bonus to \glossterm{strike damage} with \glossterm{unarmed attacks} and natural weapons.
\parhead*{Tags} \glossterm{Enhancement}
\parhead*{Materials} Jewelry
\lowercase{\hypertarget{item:Amulet of Mighty Fists, Greater}{}}\label{item:Amulet of Mighty Fists, Greater}
\hypertarget{item:Amulet of Mighty Fists, Greater}{\subsubsection{Amulet of Mighty Fists, Greater\hfill\nth{14}}}
You gain a \plus2d bonus to \glossterm{strike damage} with \glossterm{unarmed attacks} and natural weapons.
\parhead*{Tags} \glossterm{Enhancement}
\parhead*{Materials} Jewelry
\lowercase{\hypertarget{item:Armor of Energy Resistance}{}}\label{item:Armor of Energy Resistance}
\hypertarget{item:Armor of Energy Resistance}{\subsubsection{Armor of Energy Resistance\hfill\nth{4}}}
You have \glossterm{damage reduction} equal to the item's \glossterm{power} against \glossterm{energy damage}.
Whenever you resist energy with this item, it sheds light as a torch until the end of the next round.
The color of the light depends on the energy damage resisted: blue for cold, yellow for electricity, red for fire, and brown for sonic.
\parhead*{Tags} \glossterm{Shielding}
\parhead*{Materials} Bone, metal
\lowercase{\hypertarget{item:Armor of Energy Resistance, Greater}{}}\label{item:Armor of Energy Resistance, Greater}
\hypertarget{item:Armor of Energy Resistance, Greater}{\subsubsection{Armor of Energy Resistance, Greater\hfill\nth{12}}}
This item functions like the \mitem{armor of energy resistance} item, except that the damage reduction is equal to twice the item's \glossterm{power}.
\parhead*{Tags} \glossterm{Shielding}
\parhead*{Materials} Bone, metal
\lowercase{\hypertarget{item:Armor of Fortification}{}}\label{item:Armor of Fortification}
\hypertarget{item:Armor of Fortification}{\subsubsection{Armor of Fortification\hfill\nth{7}}}
You gain a \plus5 bonus to defenses when determining whether a \glossterm{strike} gets a \glossterm{critical hit} against you instead of a normal hit.
\parhead*{Tags} \glossterm{Imbuement}
\parhead*{Materials} Bone, metal
\lowercase{\hypertarget{item:Armor of Fortification, Greater}{}}\label{item:Armor of Fortification, Greater}
\hypertarget{item:Armor of Fortification, Greater}{\subsubsection{Armor of Fortification, Greater\hfill\nth{15}}}
This item functions like the \mitem{armor of fortification} item, except that the bonus increases to \plus10.
\parhead*{Tags} \glossterm{Imbuement}
\parhead*{Materials} Bone, metal
\lowercase{\hypertarget{item:Armor of Fortification, Mystic}{}}\label{item:Armor of Fortification, Mystic}
\hypertarget{item:Armor of Fortification, Mystic}{\subsubsection{Armor of Fortification, Mystic\hfill\nth{12}}}
This item functions like the \mitem{armor of fortification} item, except that it applies against all attacks instead of only against; \glossterm{strikes}.
\parhead*{Tags} \glossterm{Imbuement}
\parhead*{Materials} Bone, metal
\lowercase{\hypertarget{item:Armor of Invulnerability}{}}\label{item:Armor of Invulnerability}
\hypertarget{item:Armor of Invulnerability}{\subsubsection{Armor of Invulnerability\hfill\nth{8}}}
You have \glossterm{damage reduction} equal to this item's \glossterm{power} against damage from \glossterm{physical attacks}.
\parhead*{Tags} \glossterm{Shielding}
\parhead*{Materials} Bone, metal
\lowercase{\hypertarget{item:Armor of Invulnerability, Greater}{}}\label{item:Armor of Invulnerability, Greater}
\hypertarget{item:Armor of Invulnerability, Greater}{\subsubsection{Armor of Invulnerability, Greater\hfill\nth{16}}}
This item functions like the \mitem{armor of invulnerability} item, except that the damage reduction is equal to twice the item's \glossterm{power}.
You have \glossterm{damage reduction} equal to the item's \glossterm{power} against damage from \glossterm{physical attacks}.
\parhead*{Tags} \glossterm{Shielding}
\parhead*{Materials} Bone, metal
\lowercase{\hypertarget{item:Armor of Magic Resistance}{}}\label{item:Armor of Magic Resistance}
\hypertarget{item:Armor of Magic Resistance}{\subsubsection{Armor of Magic Resistance\hfill\nth{14}}}
You have \glossterm{magic resistance} equal to 5 + the item's \glossterm{power}.
\parhead*{Tags} \glossterm{Shielding}
\parhead*{Materials} Bone, metal
\lowercase{\hypertarget{item:Assassin's Cloak}{}}\label{item:Assassin's Cloak}
\hypertarget{item:Assassin's Cloak}{\subsubsection{Assassin's Cloak\hfill\nth{7}}}
At the end of each round, if you took no actions that round, you become \glossterm{invisible} until the end of the next round.
\parhead*{Tags} \glossterm{Glamer}
\parhead*{Materials} Textiles
\lowercase{\hypertarget{item:Assassin's Cloak, Greater}{}}\label{item:Assassin's Cloak, Greater}
\hypertarget{item:Assassin's Cloak, Greater}{\subsubsection{Assassin's Cloak, Greater\hfill\nth{17}}}
At the end of each round, if you did not attack a creature that round, you become \glossterm{invisible} until the end of the next round.
\parhead*{Tags} \glossterm{Glamer}
\parhead*{Materials} Textiles
\lowercase{\hypertarget{item:Astral Boots}{}}\label{item:Astral Boots}
\hypertarget{item:Astral Boots}{\subsubsection{Astral Boots\hfill\nth{16}}}
Whenever you move, you can teleport the same distance instead.
This does not change the total distance you can move, but you can teleport in any direction, even vertically.
You cannot teleport to locations you do not have \glossterm{line of sight} and \glossterm{line of effect} to.
\parhead*{Tags} \glossterm{Teleportation}
\parhead*{Materials} Bone, leather, metal
\lowercase{\hypertarget{item:Belt of Healing}{}}\label{item:Belt of Healing}
\hypertarget{item:Belt of Healing}{\subsubsection{Belt of Healing\hfill\nth{1}}}
When you use the \textit{recover} action, you heal \plus1d hit points.
\parhead*{Tags} \glossterm{Life}
\parhead*{Materials} Leather, textiles
\lowercase{\hypertarget{item:Belt of Healing, Greater}{}}\label{item:Belt of Healing, Greater}
\hypertarget{item:Belt of Healing, Greater}{\subsubsection{Belt of Healing, Greater\hfill\nth{8}}}
When you use the \textit{recover} action, you heal \plus2d hit points.
\parhead*{Tags} \glossterm{Life}
\parhead*{Materials} Leather, textiles
\lowercase{\hypertarget{item:Belt of Heroic Recovery}{}}\label{item:Belt of Heroic Recovery}
\hypertarget{item:Belt of Heroic Recovery}{\subsubsection{Belt of Heroic Recovery\hfill\nth{6}}}
% TODO: timing?
As an \glossterm{immediate action} when you get a \glossterm{critical hit}, you can take the \textit{recover} action.
\parhead*{Tags} \glossterm{Life}
\parhead*{Materials} Leather, textiles
\lowercase{\hypertarget{item:Boots of Earth's Embrace}{}}\label{item:Boots of Earth's Embrace}
\hypertarget{item:Boots of Earth's Embrace}{\subsubsection{Boots of Earth's Embrace\hfill\nth{4}}}
While you are standing on solid ground, you are immune to effects that would force you to move.
This does not protect you from other effects of those attacks, such as damage.
\parhead*{Tags} \glossterm{Earth}, \glossterm{Enhancement}
\parhead*{Materials} Bone, leather, metal
\lowercase{\hypertarget{item:Boots of Freedom}{}}\label{item:Boots of Freedom}
\hypertarget{item:Boots of Freedom}{\subsubsection{Boots of Freedom\hfill\nth{6}}}
You are immune to effects that restrict your mobility.
This removes all penalties you would suffer for acting underwater, except for those relating to using ranged weapons.
This does not prevent you from being \grappled, but you gain a \plus10 bonus to your defense against \glossterm{grapple} attacks.
\parhead*{Tags} \glossterm{Imbuement}
\parhead*{Materials} Bone, leather, metal
\lowercase{\hypertarget{item:Boots of Freedom, Greater}{}}\label{item:Boots of Freedom, Greater}
\hypertarget{item:Boots of Freedom, Greater}{\subsubsection{Boots of Freedom, Greater\hfill\nth{12}}}
These boots function like \mitem{boots of freedom}, except that you are also immune to being \grappled.
\parhead*{Tags} \glossterm{Imbuement}
\parhead*{Materials} Bone, leather, metal
\lowercase{\hypertarget{item:Boots of Gravitation}{}}\label{item:Boots of Gravitation}
\hypertarget{item:Boots of Gravitation}{\subsubsection{Boots of Gravitation\hfill\nth{8}}}
While these boots are within 5 feet of a solid surface, gravity pulls you towards the solid surface closest to your boots rather than in the normal direction.
This can allow you to walk easily on walls or even ceilings.
\parhead*{Tags} \glossterm{Imbuement}
\parhead*{Materials} Bone, leather, metal
\lowercase{\hypertarget{item:Boots of Speed}{}}\label{item:Boots of Speed}
\hypertarget{item:Boots of Speed}{\subsubsection{Boots of Speed\hfill\nth{5}}}
You gain a \plus10 foot bonus to your speed in all your movement modes, up to a maximum of double your normal speed.
\parhead*{Tags} \glossterm{Temporal}
\parhead*{Materials} Bone, leather, metal
\lowercase{\hypertarget{item:Boots of Speed, Greater}{}}\label{item:Boots of Speed, Greater}
\hypertarget{item:Boots of Speed, Greater}{\subsubsection{Boots of Speed, Greater\hfill\nth{13}}}
You gain a \plus30 foot bonus to your speed in all your movement modes, up to a maximum of double your normal speed.
\parhead*{Tags} \glossterm{Temporal}
\parhead*{Materials} Bone, leather, metal
\lowercase{\hypertarget{item:Boots of Water Walking}{}}\label{item:Boots of Water Walking}
\hypertarget{item:Boots of Water Walking}{\subsubsection{Boots of Water Walking\hfill\nth{7}}}
You treat the surface of all liquids as if they were firm ground.
Your feet hover about an inch above the liquid's surface, allowing you to traverse dangerous liquids without harm as long as the surface is calm.
If you are below the surface of the liquid, you rise towards the surface at a rate of 60 feet per round.
Thick liquids, such as mud and lava, may cause you to rise more slowly.
\parhead*{Tags} \glossterm{Imbuement}
\parhead*{Materials} Bone, leather, metal
\lowercase{\hypertarget{item:Boots of the Winterlands}{}}\label{item:Boots of the Winterlands}
\hypertarget{item:Boots of the Winterlands}{\subsubsection{Boots of the Winterlands\hfill\nth{2}}}
You can travel across snow and ice without slipping or suffering movement penalties for the terrain.
% TODO: degree symbol?
In addition, the boots keep you warn, protecting you in environments as cold as \minus50 Fahrenheit.
\parhead*{Tags} \glossterm{Enhancement}
\parhead*{Materials} Bone, leather, metal
\lowercase{\hypertarget{item:Bracers of Archery}{}}\label{item:Bracers of Archery}
\hypertarget{item:Bracers of Archery}{\subsubsection{Bracers of Archery\hfill\nth{1}}}
You are proficient with bows.
\parhead*{Tags} \glossterm{Enhancement}
\parhead*{Materials} Bone, leather, metal, wood
\lowercase{\hypertarget{item:Bracers of Armor}{}}\label{item:Bracers of Armor}
\hypertarget{item:Bracers of Armor}{\subsubsection{Bracers of Armor\hfill\nth{2}}}
You gain a \plus2 bonus to Armor defense.
The protection from these bracers is treated as body armor, and it does not stack with any other body armor you wear.
\parhead*{Tags} \glossterm{Shielding}
\parhead*{Materials} Bone, leather, metal, wood
\lowercase{\hypertarget{item:Bracers of Repulsion}{}}\label{item:Bracers of Repulsion}
\hypertarget{item:Bracers of Repulsion}{\subsubsection{Bracers of Repulsion\hfill\nth{4}}}
Whenever a creature hits you with a melee \glossterm{strike} during the \glossterm{action phase},
you can spend an \glossterm{action point} to use this item as an \glossterm{immediate action}.
If you do, you make a \glossterm{shove} attack against that creature during the \glossterm{delayed action phase}, using this item's power in place of your Strength.
\parhead*{Tags} \glossterm{Telekinesis}
\parhead*{Materials} Bone, leather, metal, wood
\lowercase{\hypertarget{item:Bracers of Repulsion, Greater}{}}\label{item:Bracers of Repulsion, Greater}
\hypertarget{item:Bracers of Repulsion, Greater}{\subsubsection{Bracers of Repulsion, Greater\hfill\nth{11}}}
This item functions like the \mitem{bracers of repulsion} item, except that it does not cost an action point to use.
\parhead*{Tags} \glossterm{Telekinesis}
\parhead*{Materials} Bone, leather, metal, wood
\lowercase{\hypertarget{item:Cloak of Mist}{}}\label{item:Cloak of Mist}
\hypertarget{item:Cloak of Mist}{\subsubsection{Cloak of Mist\hfill\nth{8}}}
Fog constantly fills an \areamed radius emanation from you.
This fog does not fully block sight, but it provides \concealment.
If a 5-foot square of fog takes fire damage equal to half this item's \glossterm{power}, the fog disappears from that area until the end of the next round.
\parhead*{Tags} \glossterm{Fog}, \glossterm{Manifestation}
\parhead*{Materials} Textiles
\lowercase{\hypertarget{item:Cloak of Mist, Greater}{}}\label{item:Cloak of Mist, Greater}
\hypertarget{item:Cloak of Mist, Greater}{\subsubsection{Cloak of Mist, Greater\hfill\nth{16}}}
A thick fog constantly fills an \areamed radius emanation from you.
This fog completely blocks sight beyond 10 feet.
Within that range, it still provides \concealment.
If a 5-foot square of fog takes fire damage equal to this item's \glossterm{power}, the fog disappears from that area until the end of the next round.
\parhead*{Tags} \glossterm{Fog}, \glossterm{Manifestation}
\parhead*{Materials} Textiles
\lowercase{\hypertarget{item:Crown of Flame}{}}\label{item:Crown of Flame}
\hypertarget{item:Crown of Flame}{\subsubsection{Crown of Flame\hfill\nth{5}}}
This crown is continuously on fire.
The flame sheds light as a torch.
You and all allies within an \arealarge radius emanation from you are immune to fire damage.
\parhead*{Tags} \glossterm{Fire}
\parhead*{Materials} Bone, metal
\lowercase{\hypertarget{item:Crown of Frost}{}}\label{item:Crown of Frost}
\hypertarget{item:Crown of Frost}{\subsubsection{Crown of Frost\hfill\nth{11}}}
At the end of each \glossterm{action phase}, you make a Power vs. Fortitude attack against all enemies within an \areamed radius emanation from you.
A hit deals cold \glossterm{standard damage} \minus3d.
Each creature that takes damage in this way is \fatigued until the end of the next round.
\parhead*{Tags} \glossterm{Cold}
\parhead*{Materials} Bone, metal
\lowercase{\hypertarget{item:Crown of Lightning}{}}\label{item:Crown of Lightning}
\hypertarget{item:Crown of Lightning}{\subsubsection{Crown of Lightning\hfill\nth{7}}}
This crown continuously crackles with electricity.
The constant sparks shed light as a torch.
At the end of each \glossterm{action phase}, you make a Power vs. Reflex attack against all enemies within an \areamed radius emanation from you.
A hit deals electricity \glossterm{standard damage} \minus3d.
\parhead*{Tags} \glossterm{Electricity}
\parhead*{Materials} Bone, metal
\lowercase{\hypertarget{item:Crown of Thunder}{}}\label{item:Crown of Thunder}
\hypertarget{item:Crown of Thunder}{\subsubsection{Crown of Thunder\hfill\nth{9}}}
The crown constantly emits a low-pitched rumbling.
To you and your allies, the sound is barely perceptible.
However, all enemies within an \arealarge radius emanation from you hear the sound as a deafening, continuous roll of thunder.
The noise blocks out all other sounds quieter than thunder, causing them to be \deafened while they remain in the area and until the end of the next round after they leave.
\parhead*{Tags} \glossterm{Sonic}
\parhead*{Materials} Bone, metal
\lowercase{\hypertarget{item:Featherlight Armor}{}}\label{item:Featherlight Armor}
\hypertarget{item:Featherlight Armor}{\subsubsection{Featherlight Armor\hfill\nth{4}}}
This armor's \glossterm{encumbrance penalty} is reduced by 2.
\parhead*{Tags} \glossterm{Enhancement}
\parhead*{Materials} Bone, metal
\lowercase{\hypertarget{item:Featherlight Armor, Greater}{}}\label{item:Featherlight Armor, Greater}
\hypertarget{item:Featherlight Armor, Greater}{\subsubsection{Featherlight Armor, Greater\hfill\nth{10}}}
This armor's \glossterm{encumbrance penalty} is reduced by 4.
\parhead*{Tags} \glossterm{Enhancement}
\parhead*{Materials} Bone, metal
\lowercase{\hypertarget{item:Gauntlet of the Ram}{}}\label{item:Gauntlet of the Ram}
\hypertarget{item:Gauntlet of the Ram}{\subsubsection{Gauntlet of the Ram\hfill\nth{2}}}
If you hit on a \glossterm{strike} with this gauntlet during the \glossterm{action phse}, you can attempt to \glossterm{shove} your foe during the \glossterm{delayed action phase}.
Making a strike with this gauntlet is equivalent to an \glossterm{unarmed attack}.
You do not need to move with your foe to push it back the full distance.
\parhead*{Tags} \glossterm{Telekinesis}
\parhead*{Materials} Bone, metal, wood
\lowercase{\hypertarget{item:Gauntlet of the Ram, Greater}{}}\label{item:Gauntlet of the Ram, Greater}
\hypertarget{item:Gauntlet of the Ram, Greater}{\subsubsection{Gauntlet of the Ram, Greater\hfill\nth{7}}}
This item functions like the \mitem{gauntlet of the ram}, except that you gain a bonus to the \glossterm{shove} attack equal to the damage you dealt with the \glossterm{strike}.
\parhead*{Tags} \glossterm{Telekinesis}
\parhead*{Materials} Bone, metal, wood
\lowercase{\hypertarget{item:Gauntlets of Improvisation}{}}\label{item:Gauntlets of Improvisation}
\hypertarget{item:Gauntlets of Improvisation}{\subsubsection{Gauntlets of Improvisation\hfill\nth{2}}}
You gain a \plus1d bonus to damage with \glossterm{improvised weapons}.
\parhead*{Tags} \glossterm{Enhancement}
\parhead*{Materials} Bone, metal, wood
\lowercase{\hypertarget{item:Gauntlets of Improvisation, Greater}{}}\label{item:Gauntlets of Improvisation, Greater}
\hypertarget{item:Gauntlets of Improvisation, Greater}{\subsubsection{Gauntlets of Improvisation, Greater\hfill\nth{7}}}
This item functions like the \mitem{gauntlets of improvisation}, except that the damage bonus is increased to \plus2d.
\parhead*{Tags} \glossterm{Enhancement}
\parhead*{Materials} Bone, metal, wood
\lowercase{\hypertarget{item:Greatreach Bracers}{}}\label{item:Greatreach Bracers}
\hypertarget{item:Greatreach Bracers}{\subsubsection{Greatreach Bracers\hfill\nth{9}}}
Your \glossterm{reach} is increased by 5 feet.
\parhead*{Tags} \glossterm{Imbuement}
\parhead*{Materials} Bone, leather, metal, wood
\lowercase{\hypertarget{item:Greatreach Bracers, Greater}{}}\label{item:Greatreach Bracers, Greater}
\hypertarget{item:Greatreach Bracers, Greater}{\subsubsection{Greatreach Bracers, Greater\hfill\nth{17}}}
Your \glossterm{reach} is increased by 10 feet.
\parhead*{Tags} \glossterm{Imbuement}
\parhead*{Materials} Bone, leather, metal, wood
\lowercase{\hypertarget{item:Hexproof Cloak}{}}\label{item:Hexproof Cloak}
\hypertarget{item:Hexproof Cloak}{\subsubsection{Hexproof Cloak\hfill\nth{18}}}
All \glossterm{magical} abilities that target you directly fail to affect you.
This does not protect you from abilities that affect an area.
\parhead*{Tags} \glossterm{Thaumaturgy}
\parhead*{Materials} Textiles
\lowercase{\hypertarget{item:Hexward Cloak}{}}\label{item:Hexward Cloak}
\hypertarget{item:Hexward Cloak}{\subsubsection{Hexward Cloak\hfill\nth{10}}}
You gain a \plus5 bonus to defenses against \glossterm{magical} abilities that target you directly.
This does not protect you from abilities that affect an area.
\parhead*{Tags} \glossterm{Thaumaturgy}
\parhead*{Materials} Textiles
\lowercase{\hypertarget{item:Hidden Armor}{}}\label{item:Hidden Armor}
\hypertarget{item:Hidden Armor}{\subsubsection{Hidden Armor\hfill\nth{4}}}
As a standard action, you can use this item.
If you do, it appears to change shape and form to assume the shape of a normal set of clothing.
You may choose the design of the clothing.
The item retains all of its properties, including weight and sound, while disguised in this way.
Only its visual appearance is altered.
Alternately, you may return the armor to its original appearance.
\parhead*{Tags} \glossterm{Glamer}
\parhead*{Materials} Bone, metal
\lowercase{\hypertarget{item:Hidden Armor, Greater}{}}\label{item:Hidden Armor, Greater}
\hypertarget{item:Hidden Armor, Greater}{\subsubsection{Hidden Armor, Greater\hfill\nth{9}}}
This item functions like the \mitem{hidden armor} item, except that the item also makes sound appropriate to its disguised form while disguised.
\parhead*{Tags} \glossterm{Alteration}
\parhead*{Materials} Bone, metal
\lowercase{\hypertarget{item:Mask of Air}{}}\label{item:Mask of Air}
\hypertarget{item:Mask of Air}{\subsubsection{Mask of Air\hfill\nth{9}}}
If you breathe through this mask, you breathe in clean, fresh air, regardless of your environment.
This can protect you from inhaled poisons and similar effects.
\parhead*{Tags} \glossterm{Imbuement}
\parhead*{Materials} Textiles
\lowercase{\hypertarget{item:Mask of Water Breathing}{}}\label{item:Mask of Water Breathing}
\hypertarget{item:Mask of Water Breathing}{\subsubsection{Mask of Water Breathing\hfill\nth{4}}}
You can breathe water through this mask as easily as a human breaths air.
This does not grant you the ability to breathe other liquids.
\parhead*{Tags} \glossterm{Imbuement}
\parhead*{Materials} Textiles
\lowercase{\hypertarget{item:Ring of Elemental Endurance}{}}\label{item:Ring of Elemental Endurance}
\hypertarget{item:Ring of Elemental Endurance}{\subsubsection{Ring of Elemental Endurance\hfill\nth{2}}}
You can exist comfortably in conditions between \minus50 and 140 degrees Fahrenheit without any ill effects.
You suffer the normal penalties in temperatures outside of that range.
\parhead*{Tags} \glossterm{Shielding}
\parhead*{Materials} Bone, jewelry, metal, wood
\lowercase{\hypertarget{item:Ring of Energy Resistance}{}}\label{item:Ring of Energy Resistance}
\hypertarget{item:Ring of Energy Resistance}{\subsubsection{Ring of Energy Resistance\hfill\nth{6}}}
You have \glossterm{damage reduction} equal to the ring's \glossterm{power} against \glossterm{energy damage}.
Whenever you resist energy with this ability, the ring sheds light as a torch until the end of the next round.
The color of the light depends on the energy damage resisted: blue for cold, yellow for electricity, red for fire, and brown for sonic.
\parhead*{Tags} \glossterm{Shielding}
\parhead*{Materials} Bone, jewelry, metal, wood
\lowercase{\hypertarget{item:Ring of Energy Resistance, Greater}{}}\label{item:Ring of Energy Resistance, Greater}
\hypertarget{item:Ring of Energy Resistance, Greater}{\subsubsection{Ring of Energy Resistance, Greater\hfill\nth{14}}}
This item functions like the \mitem{ring of energy resistance}, except that the damage reduction is equal to twice the item's \glossterm{power}.
\parhead*{Tags} \glossterm{Shielding}
\parhead*{Materials} Bone, jewelry, metal, wood
\lowercase{\hypertarget{item:Ring of Nourishment}{}}\label{item:Ring of Nourishment}
\hypertarget{item:Ring of Nourishment}{\subsubsection{Ring of Nourishment\hfill\nth{3}}}
You continuously gain nourishment, and no longer need to eat or drink.
This ring must be worn for 24 hours before it begins to work.
\parhead*{Tags} \glossterm{Creation}
\parhead*{Materials} Bone, jewelry, metal, wood
\lowercase{\hypertarget{item:Ring of Protection}{}}\label{item:Ring of Protection}
\hypertarget{item:Ring of Protection}{\subsubsection{Ring of Protection\hfill\nth{8}}}
You gain a \plus1 bonus to Armor defense.
\parhead*{Tags} \glossterm{Shielding}
\parhead*{Materials} Bone, jewelry, metal, wood
\lowercase{\hypertarget{item:Ring of Regeneration}{}}\label{item:Ring of Regeneration}
\hypertarget{item:Ring of Regeneration}{\subsubsection{Ring of Regeneration\hfill\nth{11}}}
At the end of each \glossterm{action phase}, you heal hit points equal to this item's \glossterm{power}.
Only damage taken while wearing the ring can be healed in this way.
\parhead*{Tags} \glossterm{Life}
\parhead*{Materials} Bone, jewelry, metal, wood
\lowercase{\hypertarget{item:Ring of Sustenance}{}}\label{item:Ring of Sustenance}
\hypertarget{item:Ring of Sustenance}{\subsubsection{Ring of Sustenance\hfill\nth{7}}}
You continuously gain nourishment, and no longer need to eat or drink.
In addition, you need only one-quarter your normal amount of sleep (or similar activity, such as elven trance) each day.
The ring must be worn for 24 hours before it begins to work.
\parhead*{Tags} \glossterm{Creation}, \glossterm{Temporal}
\parhead*{Materials} Bone, jewelry, metal, wood
\lowercase{\hypertarget{item:Seven League Boots}{}}\label{item:Seven League Boots}
\hypertarget{item:Seven League Boots}{\subsubsection{Seven League Boots\hfill\nth{12}}}
As a standard action, you can spend an \glossterm{action point} to use this item.
If you do, you teleport exactly 25 miles in a direction you specify.
If this would place you within a solid object or otherwise impossible space, the boots will shunt you up to 1,000 feet in any direction to the closest available space.
If there is no available space within 1,000 feet of your intended destination, the effect fails and you take \glossterm{standard damage} \minus1d.
\parhead*{Tags} \glossterm{Teleportation}
\parhead*{Materials} Bone, leather, metal
\lowercase{\hypertarget{item:Shield of Arrow Catching}{}}\label{item:Shield of Arrow Catching}
\hypertarget{item:Shield of Arrow Catching}{\subsubsection{Shield of Arrow Catching\hfill\nth{5}}}
Whenever a creature within a \areamed radius emanation from you would be attacked by a ranged weapon, the attack is redirected to target you instead.
Resolve the attack as if it had initially targeted you, except that the attack is not affected by cover or concealment.
This item can only affect projectiles and thrown objects that are Small or smaller.
\parhead*{Tags} \glossterm{Telekinesis}
\parhead*{Materials} Bone, metal, wood
\lowercase{\hypertarget{item:Shield of Arrow Catching, Greater}{}}\label{item:Shield of Arrow Catching, Greater}
\hypertarget{item:Shield of Arrow Catching, Greater}{\subsubsection{Shield of Arrow Catching, Greater\hfill\nth{10}}}
This item functions like the \mitem{shield of arrow catching} item, except that it affects a \arealarge radius from you.
In addition, you may choose to exclude creature from this item's effect, allowing projectiles to target nearby foes normally.
\parhead*{Tags} \glossterm{Telekinesis}
\parhead*{Materials} Bone, metal, wood
\lowercase{\hypertarget{item:Shield of Arrow Deflection}{}}\label{item:Shield of Arrow Deflection}
\hypertarget{item:Shield of Arrow Deflection}{\subsubsection{Shield of Arrow Deflection\hfill\nth{2}}}
As an \glossterm{immediate action} when you are attacked by a ranged \glossterm{strike}, you can use this item.
If you do, you gain a \plus5 bonus to Armor defense against the attack.
You must be aware of the attack to deflect it in this way.
This item can only affect projectiles and thrown objects that are Small or smaller.
\parhead*{Tags} \glossterm{Telekinesis}
\parhead*{Materials} Bone, metal, wood
\lowercase{\hypertarget{item:Shield of Arrow Deflection, Greater}{}}\label{item:Shield of Arrow Deflection, Greater}
\hypertarget{item:Shield of Arrow Deflection, Greater}{\subsubsection{Shield of Arrow Deflection, Greater\hfill\nth{12}}}
This item functions like the \mitem{shield of arrow deflection} item, except that the defense bonus increases to \plus10.
\parhead*{Tags} \glossterm{Telekinesis}
\parhead*{Materials} Bone, metal, wood
\lowercase{\hypertarget{item:Shield of Bashing}{}}\label{item:Shield of Bashing}
\hypertarget{item:Shield of Bashing}{\subsubsection{Shield of Bashing\hfill\nth{2}}}
% Should this be strike damage?
You gain a \plus1d bonus to damage with \glossterm{physical attacks} using this shield.
\parhead*{Tags} \glossterm{Enhancement}
\parhead*{Materials} Bone, metal, wood
\lowercase{\hypertarget{item:Shield of Bashing, Greater}{}}\label{item:Shield of Bashing, Greater}
\hypertarget{item:Shield of Bashing, Greater}{\subsubsection{Shield of Bashing, Greater\hfill\nth{11}}}
% Should this be strike damage?
You gain a \plus2d bonus to damage with \glossterm{physical attacks} using this shield.
\parhead*{Tags} \glossterm{Enhancement}
\parhead*{Materials} Bone, metal, wood
\lowercase{\hypertarget{item:Shield of Boulder Catching}{}}\label{item:Shield of Boulder Catching}
\hypertarget{item:Shield of Boulder Catching}{\subsubsection{Shield of Boulder Catching\hfill\nth{8}}}
This item functions like the \mitem{shield of arrow catching} item, except that it can affect projectile and thrown objects of up to Large size.
\parhead*{Tags} \glossterm{Telekinesis}
\parhead*{Materials} Bone, metal, wood
\lowercase{\hypertarget{item:Shield of Boulder Deflection}{}}\label{item:Shield of Boulder Deflection}
\hypertarget{item:Shield of Boulder Deflection}{\subsubsection{Shield of Boulder Deflection\hfill\nth{6}}}
This item functions like the \mitem{shield of arrow deflection} item, except that it can affect projectiles and thrown objects of up to Large size.
\parhead*{Tags} \glossterm{Telekinesis}
\parhead*{Materials} Bone, metal, wood
\lowercase{\hypertarget{item:Shield of Mystic Reflection}{}}\label{item:Shield of Mystic Reflection}
\hypertarget{item:Shield of Mystic Reflection}{\subsubsection{Shield of Mystic Reflection\hfill\nth{12}}}
As an \glossterm{immediate action} when you are targeted by a targeted \glossterm{magical} ability, you can spend an \glossterm{action point} to use this ability.
If you do, the ability targets the creature using the ability instead of you.
Any other targets of the ability are affected normally.
\parhead*{Tags} \glossterm{Thaumaturgy}
\parhead*{Materials} Bone, metal, wood
\lowercase{\hypertarget{item:Throwing Gloves}{}}\label{item:Throwing Gloves}
\hypertarget{item:Throwing Gloves}{\subsubsection{Throwing Gloves\hfill\nth{4}}}
% TODO: reference basic "not designed to be thrown" mechanics?
You can throw any item as if it was designed to be thrown.
This does not improve your ability to throw items designed to be thrown, such as darts.
\parhead*{Tags} \glossterm{Enhancement}
\parhead*{Materials} Leather
\lowercase{\hypertarget{item:Torchlight Gloves}{}}\label{item:Torchlight Gloves}
\hypertarget{item:Torchlight Gloves}{\subsubsection{Torchlight Gloves\hfill\nth{2}}}
These gloves shed light as a torch.
As a \glossterm{standard action}, you may choose to suppress or resume the light from either or both gloves.
\parhead*{Tags} \glossterm{Figment}, \glossterm{Light}
\parhead*{Materials} Leather
\lowercase{\hypertarget{item:Vanishing Cloak}{}}\label{item:Vanishing Cloak}
\hypertarget{item:Vanishing Cloak}{\subsubsection{Vanishing Cloak\hfill\nth{8}}}
As a standard action, you can spend an \glossterm{action point} to use this item.
If you do, you teleport to an unoccupied location within \rngmed range of your original location.
In addition, you become \glossterm{invisible} unitl the end of the next round.
If your intended destination is invalid, or if your teleportation otherwise fails, you still become invisible.
\parhead*{Tags} \glossterm{Glamer}, \glossterm{Teleportation}
\parhead*{Materials} Textiles
\lowercase{\hypertarget{item:Winged Boots}{}}\label{item:Winged Boots}
\hypertarget{item:Winged Boots}{\subsubsection{Winged Boots\hfill\nth{10}}}
You gain a \glossterm{fly speed} equal to your land speed.
However, the boots are not strong enough to keep you aloft indefinitely.
At the end of each round, if you are not standing on solid ground, the magic of the boots fails and you fall normally.
The boots begin working again at the end of the next round, even if you have not yet hit the ground.
\parhead*{Tags} \glossterm{Imbuement}
\parhead*{Materials} Bone, leather, metal

\section{Weapons} % or implements?
    Magic weapons improve a character's combat abilities.
    They must be wielded to gain their effects.

    \parhead{Ranged Weapons and Ammunition} Any magical properties of a projectile weapon also apply to all ammunition fired from that weapon.

    \subsection{Weapon Description}

        \begin{longtabuwrapper}
\begin{longtabu}{l l X l}
\lcaption{Weapon Items} \\
\tb{Name} & \tb{Level} & \tb{Description} & \tb{Page} \\
\bottomrule
Morphing & \nth{2} & Can change into similar weapon & \pageref{item:Morphing} \\
Merciful & \nth{3} & Deals subdual damage & \pageref{item:Merciful} \\
Returning & \nth{3} & Teleports back to you after being thrown & \pageref{item:Returning} \\
Freezing & \nth{4} & Deals cold damage, can fatigue & \pageref{item:Freezing} \\
Longshot & \nth{4} & Has twice the normal range increment & \pageref{item:Longshot} \\
Flaming & \nth{5} & Can deal \plus1d fire damage & \pageref{item:Flaming} \\
Thundering & \nth{5} & Deals sonic damage, can deafen & \pageref{item:Thundering} \\
Forceful & \nth{6} & Can shove struck foes & \pageref{item:Forceful} \\
Morphing, Greater & \nth{6} & Can change into any weapon & \pageref{item:Morphing, Greater} \\
Vampiric & \nth{6} & Heals you when dealing damage & \pageref{item:Vampiric} \\
Seeking & \nth{7} & Reduces miss chances & \pageref{item:Seeking} \\
Shocking & \nth{7} & Deals electicity damage, can daze & \pageref{item:Shocking} \\
Thieving & \nth{7} & Can absorb small items & \pageref{item:Thieving} \\
Defending & \nth{9} & Grants \plus1 Armor defense & \pageref{item:Defending} \\
Disorienting & \nth{9} & Can disorient struck foes & \pageref{item:Disorienting} \\
Phasing & \nth{9} & Can ignore obstacles when attacking & \pageref{item:Phasing} \\
Freezing, Greater & \nth{10} & Deals fatiguing cold damage & \pageref{item:Freezing, Greater} \\
Longshot, Greater & \nth{10} & Has three times the normal range increment & \pageref{item:Longshot, Greater} \\
Surestrike & \nth{10} & React to reroll missed attacks & \pageref{item:Surestrike} \\
Flaming, Greater & \nth{11} & Deals \plus1d fire damage & \pageref{item:Flaming, Greater} \\
Thundering, Greater & \nth{11} & Deals deafening sonic damage & \pageref{item:Thundering, Greater} \\
Forceful, Greater & \nth{12} & Shoves struck foes & \pageref{item:Forceful, Greater} \\
Vorpal & \nth{12} & Inflicts lethal critical hits & \pageref{item:Vorpal} \\
Fixating & \nth{13} & Grants accuracy bonus against struck foe & \pageref{item:Fixating} \\
Shocking, Greater & \nth{13} & Deals dazing electicity damage & \pageref{item:Shocking, Greater} \\
Soulreaving & \nth{13} & Deals delayed damage & \pageref{item:Soulreaving} \\
Thieving, Greater & \nth{13} & Can absorb large items & \pageref{item:Thieving, Greater} \\
Vampiric, Greater & \nth{14} & Drastically heals you when dealing damage & \pageref{item:Vampiric, Greater} \\
Disorienting, Greater & \nth{15} & Disorients struck foes & \pageref{item:Disorienting, Greater} \\
Heartseeker & \nth{17} & Rolls attacks twice & \pageref{item:Heartseeker} \\
\end{longtabu}
\end{longtabuwrapper}

        
\lowercase{\hypertarget{item:Concussive}{}}\label{item:Concussive}
\hypertarget{item:Concussive}{\subsubsection{Concussive\hfill\nth{4}}}

As a standard action, you can infuse this weapon with concussive force.
The next time you make a \glossterm{strike} with this weapon, if your attack result beats the target's Fortitude defense, it is \glossterm{dazed} as a \glossterm{condition}.



\parhead*{Materials} As weapon


\lowercase{\hypertarget{item:Cutthroat}{}}\label{item:Cutthroat}
\hypertarget{item:Cutthroat}{\subsubsection{Cutthroat\hfill\nth{4}}}

As a standard action, you can make a \glossterm{strike} with this weapon.
In addition to the normal effects of the strike, if your attack result beats the target's Fortitude defense, it is \glossterm{muted} as a \glossterm{condition}.



\parhead*{Materials} As weapon


\lowercase{\hypertarget{item:Defending}{}}\label{item:Defending}
\hypertarget{item:Defending}{\subsubsection{Defending\hfill\nth{9}}}

You gain a \plus1 \glossterm{magic bonus} to Armor defense.



\parhead*{Tags} \glossterm{Shielding}


\parhead*{Materials} As weapon


\lowercase{\hypertarget{item:Disorienting}{}}\label{item:Disorienting}
\hypertarget{item:Disorienting}{\subsubsection{Disorienting\hfill\nth{9}}}

This weapon shimmers with a chaotic pattern of colors.
As a \glossterm{minor action}, you can intensify the shimmering.
If you do, when you make a \glossterm{strike}  with this weapon and your attack result beats the target's Mental defense, it is \disoriented as a \glossterm{condition}.
This is a \glossterm{Swift} ability, and it lasts until the end of the round.



\parhead*{Tags} \glossterm{Compulsion}, \glossterm{Mind}


\parhead*{Materials} As weapon


\lowercase{\hypertarget{item:Disorienting, Greater}{}}\label{item:Disorienting, Greater}
\hypertarget{item:Disorienting, Greater}{\subsubsection{Disorienting, Greater\hfill\nth{15}}}

This weapon shimmers with a chaotic pattern of colors.
When you make a \glossterm{strike} with this weapon and your attack result beats the target's Mental defense, it is \disoriented as a \glossterm{condition}.



\parhead*{Tags} \glossterm{Compulsion}, \glossterm{Mind}


\parhead*{Materials} As weapon


\lowercase{\hypertarget{item:Fixating}{}}\label{item:Fixating}
\hypertarget{item:Fixating}{\subsubsection{Fixating\hfill\nth{13}}}

When you make a \glossterm{strike} with this weapon, you gain a \plus1 bonus to accuracy against the target.
This bonus lasts until you make a strike with this weapon against a different target.
This bonus can stack with itself, up to a maximum of \plus5.



\parhead*{Materials} As weapon


\lowercase{\hypertarget{item:Flaming}{}}\label{item:Flaming}
\hypertarget{item:Flaming}{\subsubsection{Flaming\hfill\nth{5}}}

This weapon is on fire.
It sheds light as a torch, and all damage dealt with it is fire damage in addition to its other types.
As a \glossterm{minor action}, you can kindle the flames.
If you do, you gain a \plus1d \glossterm{magic bonus} to \glossterm{strike damage} with this weapon.
This is a \glossterm{Swift} ability, and it lasts until the end of the round.



\parhead*{Tags} \glossterm{Fire}


\parhead*{Materials} As weapon


\lowercase{\hypertarget{item:Flaming, Greater}{}}\label{item:Flaming, Greater}
\hypertarget{item:Flaming, Greater}{\subsubsection{Flaming, Greater\hfill\nth{11}}}

This weapon is on fire.
It sheds light as a torch, and all damage dealt with it is fire damage in addition to its other types.
You gain a \plus1d \glossterm{magic bonus} to \glossterm{strike damage} with this weapon.



\parhead*{Tags} \glossterm{Fire}


\parhead*{Materials} As weapon


\lowercase{\hypertarget{item:Forceful}{}}\label{item:Forceful}
\hypertarget{item:Forceful}{\subsubsection{Forceful\hfill\nth{6}}}

This weapon feels heavy in the hand.
As a \glossterm{minor action}, you can intensify the weapon's heft.
If you do, when you make a \glossterm{strike} with this weapon, you can also use your attack result as a \glossterm{shove} attack agsint the target.
You do not need to move with your foe to move it the full distance of the shove.
This is a \glossterm{Swift} ability, and it lasts until the end of the round.



\parhead*{Materials} As weapon


\lowercase{\hypertarget{item:Forceful, Greater}{}}\label{item:Forceful, Greater}
\hypertarget{item:Forceful, Greater}{\subsubsection{Forceful, Greater\hfill\nth{12}}}

This weapon feels heavy in the hand.
When you make a \glossterm{strike} with this weapon, you can also use your attack result as a \glossterm{shove} attack agsint the target.
You do not need to move with your foe to move it the full distance of the shove.



\parhead*{Materials} As weapon


\lowercase{\hypertarget{item:Freezing}{}}\label{item:Freezing}
\hypertarget{item:Freezing}{\subsubsection{Freezing\hfill\nth{4}}}

This weapon is bitterly cold, and all damage dealt with it is cold damage in addition to its other types.
As a \glossterm{minor action}, you can intensify the cold.
If you do, when you make a \glossterm{strike} with this weapon and your attack result beats the target's Fortitude defense, the target is \fatigued as a \glossterm{condition}.
This is a \glossterm{Swift} ability, and it lasts until the end of the round.



\parhead*{Tags} \glossterm{Cold}


\parhead*{Materials} As weapon


\lowercase{\hypertarget{item:Freezing, Greater}{}}\label{item:Freezing, Greater}
\hypertarget{item:Freezing, Greater}{\subsubsection{Freezing, Greater\hfill\nth{10}}}

This weapon is bitterly cold, and all damage dealt with it is cold damage in addition to its other types.
When you make a \glossterm{strike} with this weapon, if your attack result beats the target's Fortitude defense, the target is \fatigued as a \glossterm{condition}.



\parhead*{Tags} \glossterm{Cold}


\parhead*{Materials} As weapon


\lowercase{\hypertarget{item:Heartseeker}{}}\label{item:Heartseeker}
\hypertarget{item:Heartseeker}{\subsubsection{Heartseeker\hfill\nth{17}}}

Whenever you make a \glossterm{strike} with this weapon, you can roll twice and take the higher result.



\parhead*{Tags} \glossterm{Knowledge}


\parhead*{Materials} As weapon


\lowercase{\hypertarget{item:Longshot}{}}\label{item:Longshot}
\hypertarget{item:Longshot}{\subsubsection{Longshot\hfill\nth{4}}}

Ranged attacks with this weapon have twice the normal \glossterm{range increment}.



\parhead*{Materials} As weapon


\lowercase{\hypertarget{item:Longshot, Greater}{}}\label{item:Longshot, Greater}
\hypertarget{item:Longshot, Greater}{\subsubsection{Longshot, Greater\hfill\nth{10}}}

Ranged attacks with this weapon have three times the normal \glossterm{range increment}.



\parhead*{Materials} As weapon


\lowercase{\hypertarget{item:Merciful}{}}\label{item:Merciful}
\hypertarget{item:Merciful}{\subsubsection{Merciful\hfill\nth{3}}}

This weapon deals \glossterm{subdual damage} instead of lethal damage.



\parhead*{Materials} As weapon


\lowercase{\hypertarget{item:Morphing}{}}\label{item:Morphing}
\hypertarget{item:Morphing}{\subsubsection{Morphing\hfill\nth{2}}}

As a standard action, you can spend an \glossterm{action point} to activate this item.
If you do, it changes shape into a new weapon of your choice from the same weapon group.



\parhead*{Tags} \glossterm{Shaping}


\parhead*{Materials} As weapon


\lowercase{\hypertarget{item:Morphing, Greater}{}}\label{item:Morphing, Greater}
\hypertarget{item:Morphing, Greater}{\subsubsection{Morphing, Greater\hfill\nth{6}}}

As a standard action, you can spend an \glossterm{action point} to activate this item.
If you do, it changes shape into a new weapon of your choice that you are proficient with.
This can only change into existing manufactured weapons, not improvised weapons (see \pcref{Weapons}).



\parhead*{Tags} \glossterm{Shaping}


\parhead*{Materials} As weapon


\lowercase{\hypertarget{item:Phasing}{}}\label{item:Phasing}
\hypertarget{item:Phasing}{\subsubsection{Phasing\hfill\nth{9}}}

\glossterm{Strikes} with this weapon can pass through a single solid obstacle of up to five feet thick on the way to their target.
This can allow you to ignore \glossterm{cover}, or even attack through solid walls.
It does not allow you to ignore armor, shields, or or similar items used by the target of your attacks.



\parhead*{Tags} \glossterm{Planar}


\parhead*{Materials} As weapon


\lowercase{\hypertarget{item:Returning}{}}\label{item:Returning}
\hypertarget{item:Returning}{\subsubsection{Returning\hfill\nth{3}}}

After being thrown, this weapon teleports back into your hand at the end of the current phase.
Catching a rebounding weapon when it comes back is a free action.
If you can't catch it, the weapon drops to the ground in the square from which it was thrown.



\parhead*{Tags} \glossterm{Teleportation}


\parhead*{Materials} As weapon


\lowercase{\hypertarget{item:Seeking}{}}\label{item:Seeking}
\hypertarget{item:Seeking}{\subsubsection{Seeking\hfill\nth{7}}}

This weapon automatically veers towards its intended target.
\glossterm{Strikes} with this weapon that would suffer a 50\% miss chance instead suffer a 20\% miss chance.
In addition, attacks that would otherwise suffer a 20\% miss chance instead suffer no miss chance.



\parhead*{Tags} \glossterm{Knowledge}


\parhead*{Materials} As weapon


\lowercase{\hypertarget{item:Shocking}{}}\label{item:Shocking}
\hypertarget{item:Shocking}{\subsubsection{Shocking\hfill\nth{7}}}

This weapon continuously crackles with electricity.
The constant sparks shed light as a torch, and all damage dealt with it is electricity damage in addition to its other types.
As a \glossterm{minor action}, you can intensify the electricity.
If you do, when you make a \glossterm{strike} with this weapon and your attack result beats the target's Fortitude defense, the target is \dazed as a \glossterm{condition}.
This is a \glossterm{Swift} ability, and it lasts until the end of the round.



\parhead*{Tags} \glossterm{Electricity}


\parhead*{Materials} As weapon


\lowercase{\hypertarget{item:Shocking, Greater}{}}\label{item:Shocking, Greater}
\hypertarget{item:Shocking, Greater}{\subsubsection{Shocking, Greater\hfill\nth{13}}}

This weapon continuously crackles with electricity.
The constant sparks shed light as a torch, and all damage dealt with it is electricity damage in addition to its other types.
When you make a \glossterm{strike} with this weapon, if your attack result beats the target's Fortitude defense, it is \dazed as a \glossterm{condition}.



\parhead*{Tags} \glossterm{Electricity}


\parhead*{Materials} As weapon


\lowercase{\hypertarget{item:Soulreaving}{}}\label{item:Soulreaving}
\hypertarget{item:Soulreaving}{\subsubsection{Soulreaving\hfill\nth{13}}}

This weapon is transluscent and has no physical presence for anyone except you.
It has no effect on objects or constructs, and creatures do not feel any pain or even notice attacks from it.
Attacks with this weapon ignore all damage reduction and hardness, but the damage is delayed instead of being dealt immediately.
Damage that would be dealt by the weapon can be delayed indefinitely.
While the damage is delayed, it cannot be removed by any means short of the destruction of this weapon or the creature's death.

As a \glossterm{minor action}, you can cut yourself with this weapon to activate it.
This deals no damage to you.
If you do, all delayed damage dealt by this weapon is converted into real damage.
Any such damage dealt in excess of a creature's hit points is dealt immediately as \glossterm{vital damage}.



\parhead*{Materials} As weapon


\lowercase{\hypertarget{item:Surestrike}{}}\label{item:Surestrike}
\hypertarget{item:Surestrike}{\subsubsection{Surestrike\hfill\nth{9}}}

You gain a \plus1 \glossterm{magic bonus} to accuracy with \glossterm{strikes} with this weapon.



\parhead*{Tags} \glossterm{Knowledge}


\parhead*{Materials} As weapon


\lowercase{\hypertarget{item:Thieving}{}}\label{item:Thieving}
\hypertarget{item:Thieving}{\subsubsection{Thieving\hfill\nth{7}}}

As a \glossterm{standard action}, you can spend an \glossterm{action point} to activate this weapon.
If you do, make a \glossterm{strike} or a \glossterm{disarm} attack.
If your disarm succeeds, or if your strike hit an unattended object, this weapon can absorb the struck object.
The object must be at least one size category smaller than the weapon.
An absorbed object leaves no trace that it ever existed.

This weapon can hold no more than three objects at once.
If you attempt to absorb an object while the weapon is full, the attempt fails.

As a standard action, you can retrieve the last item absorbed by the weapon.
The item appears in your hand, or falls to the ground if your hand is occupied.



\parhead*{Tags} \glossterm{Shaping}


\parhead*{Materials} As weapon


\lowercase{\hypertarget{item:Thieving, Greater}{}}\label{item:Thieving, Greater}
\hypertarget{item:Thieving, Greater}{\subsubsection{Thieving, Greater\hfill\nth{13}}}

This item functions like the \mitem{thieving} item, except that the maximum size category of object it can absorb is one size category larger than the weapon.



\parhead*{Tags} \glossterm{Shaping}


\parhead*{Materials} As weapon


\lowercase{\hypertarget{item:Thundering}{}}\label{item:Thundering}
\hypertarget{item:Thundering}{\subsubsection{Thundering\hfill\nth{5}}}

This weapon constantly emits a low-pitched rumbling noise and vibrates slightly in your hand.
All damage dealt with it is sonic damage in addition to its other types.
As a \glossterm{minor action}, you can intensify the vibration.
If you do, when you make a \glossterm{strike} with this weapon and your attack result beats the target's Fortitude defense, the target is \deafened as a \glossterm{condition}.
This is a \glossterm{Swift} ability, and it lasts until the end of the round.



\parhead*{Tags} \glossterm{Sonic}


\parhead*{Materials} As weapon


\lowercase{\hypertarget{item:Thundering, Greater}{}}\label{item:Thundering, Greater}
\hypertarget{item:Thundering, Greater}{\subsubsection{Thundering, Greater\hfill\nth{11}}}

This weapon constantly emits a low-pitched rumbling noise and vibrates slightly in your hand.
All damage dealt with it is sonic damage in addition to its other types.
When you make a \glossterm{strike} with this weapon and your attack result beats the target's Fortitude defense, the target is \deafened as a \glossterm{condition}.



\parhead*{Tags} \glossterm{Sonic}


\parhead*{Materials} As weapon


\lowercase{\hypertarget{item:Vampiric}{}}\label{item:Vampiric}
\hypertarget{item:Vampiric}{\subsubsection{Vampiric\hfill\nth{6}}}

When you deal damage to a living creature with a \glossterm{strike} with this weapon, you heal hit points equal to your level.



\parhead*{Tags} \glossterm{Life}


\parhead*{Materials} As weapon


\lowercase{\hypertarget{item:Vampiric, Greater}{}}\label{item:Vampiric, Greater}
\hypertarget{item:Vampiric, Greater}{\subsubsection{Vampiric, Greater\hfill\nth{14}}}

When you deal damage to a living creature with a \glossterm{strike} with this weapon, you heal hit points equal to twice your level.



\parhead*{Tags} \glossterm{Life}


\parhead*{Materials} As weapon


\lowercase{\hypertarget{item:Vorpal}{}}\label{item:Vorpal}
\hypertarget{item:Vorpal}{\subsubsection{Vorpal\hfill\nth{12}}}

Critical hits on \glossterm{strikes} with this weapon deal maximum damage.



\parhead*{Materials} As weapon


\section{Implements}

    Implements can take many forms: staffs, wands, holy symbols, and more.
    Like magic weapons, magic implements must be wielded to gain their effects.
    However, while weapons are used to deal damage to enemies, implements are used to cast spells.

    \parhead{Somatic Components} While wielding an implement, you may gesture with it and channel magic through it.
    These qualify as somatic components for the purpose of casting spells.

    \subsection{Implement Types}

        \subsubsection{Holy Symbols}

            \parhead{Physical Description} A typical holy symbol is a no larger than 4 inches in each dimension and can be easily held in the palm of a hand.
            Most holy symbols are metal, but they can be made from wood, bone, or even more exotic materials, depending on the deity they symbolize.

            \parhead{Special Rules} All holy symbols are implements for divine spells.
            Most holy symbols are designed to be worn as an amulet in addition to being held in the hand.
            A holy symbol worn on the body cannot be used to perform somatic components for spellcasting.
            However, it still grants its magical abilities as if it was being actively wielded.

        \subsubsection{Staffs}

            \parhead{Physical Description} A typical staff is 4 feet to 7 feet long and 2 inches to 3 inches thick, weighing about 5 pounds.
            Most staffs are wood, but a rare few are bone, metal, or even glass.
            (These are extremely exotic.)
            Staffs often have a gem or some device at their tip or are shod in metal at one or both ends.

            Staffs are often decorated with carvings or runes.
            A typical staff is like a walking stick, quarterstaff, or cudgel.
            It has AD 7, 10 hit points, hardness 5, and a break DR of 24.

        \subsubsection{Wands}

            \parhead{Physical Description} A typical wand is 6 inches to 12 inches long and about 1/4 inch thick, and usually weighs no more than 1 ounce.
            Most wands are wood, but some are bone.
            A rare few are metal, glass, or even ceramic, but these are quite exotic.
            Occasionally, a wand has a gem or some device at its tip, and most are decorated with carvings or runes.
            A typical wand has AD 7, 5 hit points, hardness 5, and a break DR of 16.

    \subsection{Implement Descriptions}

        
\begin{longtabuwrapper}
\begin{longtabu}{l l X l}
\lcaption{Implement Items} \\
\tb{Name} & \tb{Level} & \tb{Description} & \tb{Page} \\
\bottomrule
Wand of Spellpower & \nth{4} & Grants \plus1 power with a single spell & \pageref{item:Wand of Spellpower} \\
Staff of Transit & \nth{5} & Doubles your teleportation distance & \pageref{item:Staff of Transit} \\
Spellfeeding Staff & \nth{6} & Heals you when casting spells & \pageref{item:Spellfeeding Staff} \\
Staff of Spellpower & \nth{8} & Grants \plus1 power with spells & \pageref{item:Staff of Spellpower} \\
Staff of Sympathetic Shielding & \nth{8} & Shields you when shielding others & \pageref{item:Staff of Sympathetic Shielding} \\
Wand of Precision & \nth{8} & Grants \plus1 accuracy with a single spell & \pageref{item:Wand of Precision} \\
Staff of Precision & \nth{10} & Grants \plus1 accuracy with spells & \pageref{item:Staff of Precision} \\
Wand of Spellpower, Greater & \nth{10} & Grants \plus2 power with a single spell & \pageref{item:Wand of Spellpower, Greater} \\
Spellfeeding Staff, Greater & \nth{14} & Greatly heals you when casting spells & \pageref{item:Spellfeeding Staff, Greater} \\
Staff of Spellpower, Greater & \nth{14} & Grants \plus2 power with spells & \pageref{item:Staff of Spellpower, Greater} \\
Wand of Precision, Greater & \nth{14} & Grants \plus2 accuracy with a single spell & \pageref{item:Wand of Precision, Greater} \\
Greater Staff of Precision & \nth{16} & Grants \plus2 accuracy with spells & \pageref{item:Greater Staff of Precision} \\
Wand of Spellpower, Supreme & \nth{16} & Grants \plus3 power with a single spell & \pageref{item:Wand of Spellpower, Supreme} \\
Staff of Spellpower, Supreme & \nth{20} & Grants \plus3 power with spells & \pageref{item:Staff of Spellpower, Supreme} \\
\end{longtabu}
\end{longtabuwrapper}


        
\lowercase{\hypertarget{item:Extending Staff}{}}\label{item:Extending Staff}
\hypertarget{item:Extending Staff}{\subsubsection{Extending Staff\hfill\nth{10} (6,500 gp)}}

You double the range of your \glossterm{magical} abilities.



\vspace{0.25em}
\spelltwocol{\textbf{Type}: Staff}{}
\textbf{Materials}: Bone, wood


\lowercase{\hypertarget{item:Extending Staff, Greater}{}}\label{item:Extending Staff, Greater}
\hypertarget{item:Extending Staff, Greater}{\subsubsection{Extending Staff, Greater\hfill\nth{19} (280,000 gp)}}

You triple the range of your \glossterm{magical} abilities.



\vspace{0.25em}
\spelltwocol{\textbf{Type}: Staff}{}
\textbf{Materials}: Bone, wood


\lowercase{\hypertarget{item:Protective Staff}{}}\label{item:Protective Staff}
\hypertarget{item:Protective Staff}{\subsubsection{Protective Staff\hfill\nth{5} (800 gp)}}

You gain a \plus1 \glossterm{magic bonus} to Armor defense.



\vspace{0.25em}
\spelltwocol{\textbf{Type}: Staff}{}
\textbf{Materials}: Bone, wood


\lowercase{\hypertarget{item:Protective Staff, Greater}{}}\label{item:Protective Staff, Greater}
\hypertarget{item:Protective Staff, Greater}{\subsubsection{Protective Staff, Greater\hfill\nth{14} (37,000 gp)}}

You gain a \plus2 \glossterm{magic bonus} to Armor defense.



\vspace{0.25em}
\spelltwocol{\textbf{Type}: Staff}{}
\textbf{Materials}: Bone, wood


\lowercase{\hypertarget{item:Reaching Staff}{}}\label{item:Reaching Staff}
\hypertarget{item:Reaching Staff}{\subsubsection{Reaching Staff\hfill\nth{12} (16,000 gp)}}

Spells you cast with this staff automatically have the benefits of the Reach augment, if applicable (see \pcref{Augment Descriptions}).



\vspace{0.25em}
\spelltwocol{\textbf{Type}: Staff}{}
\textbf{Materials}: Bone, wood


\lowercase{\hypertarget{item:Spell Wand, 1st}{}}\label{item:Spell Wand, 1st}
\hypertarget{item:Spell Wand, 1st}{\subsubsection{Spell Wand, 1st\hfill\nth{5} (800 gp)}}

This wand grants you knowledge of a single 1st level spell.
You must have access to the \glossterm{mystic sphere} that spell belongs to.



\vspace{0.25em}
\spelltwocol{\textbf{Type}: Wand}{}
\textbf{Materials}: Bone, wood


\lowercase{\hypertarget{item:Spell Wand, 2nd}{}}\label{item:Spell Wand, 2nd}
\hypertarget{item:Spell Wand, 2nd}{\subsubsection{Spell Wand, 2nd\hfill\nth{9} (4,000 gp)}}

This item functions like a \mitem{spell wand}, except that it grants knowledge of a single 2nd level spell.



\vspace{0.25em}
\spelltwocol{\textbf{Type}: Wand}{}
\textbf{Materials}: Bone, wood


\lowercase{\hypertarget{item:Spell Wand, 3rd}{}}\label{item:Spell Wand, 3rd}
\hypertarget{item:Spell Wand, 3rd}{\subsubsection{Spell Wand, 3rd\hfill\nth{13} (25,000 gp)}}

This item functions like a \mitem{spell wand}, except that it grants knowledge of a single 3rd level spell.



\vspace{0.25em}
\spelltwocol{\textbf{Type}: Wand}{}
\textbf{Materials}: Bone, wood


\lowercase{\hypertarget{item:Spell Wand, 4th}{}}\label{item:Spell Wand, 4th}
\hypertarget{item:Spell Wand, 4th}{\subsubsection{Spell Wand, 4th\hfill\nth{17} (125,000 gp)}}

This item functions like a \mitem{spell wand}, except that it grants knowledge of a single 4th level spell.



\vspace{0.25em}
\spelltwocol{\textbf{Type}: Wand}{}
\textbf{Materials}: Bone, wood


\lowercase{\hypertarget{item:Staff of Expansion}{}}\label{item:Staff of Expansion}
\hypertarget{item:Staff of Expansion}{\subsubsection{Staff of Expansion\hfill\nth{7} (1,800 gp)}}

When you use a \glossterm{magical} ability that creates a \glossterm{zone} or \glossterm{emanation}, you can increase the size of the area by one size category, up to a maximum of \areahuge.
You can only increase the area of one ability at a time in this way.
If you increase the area of another ability or lose this staff, the area of the original ability returns to its normal size.



\vspace{0.25em}
\spelltwocol{\textbf{Type}: Staff}{}
\textbf{Materials}: Bone, wood


\lowercase{\hypertarget{item:Staff of Expansion, Greater}{}}\label{item:Staff of Expansion, Greater}
\hypertarget{item:Staff of Expansion, Greater}{\subsubsection{Staff of Expansion, Greater\hfill\nth{16} (85,000 gp)}}

This item functions like a \textit{staff of expansion}, except that it increases the area by two size categories.
In addition, the maximum area is a 200 foot radius, which is one size category larger than \areahuge.



\vspace{0.25em}
\spelltwocol{\textbf{Type}: Staff}{}
\textbf{Materials}: Bone, wood


\lowercase{\hypertarget{item:Staff of Focus}{}}\label{item:Staff of Focus}
\hypertarget{item:Staff of Focus}{\subsubsection{Staff of Focus\hfill\nth{6} (1,200 gp)}}

You reduce your \glossterm{focus penalty} by 1.



\vspace{0.25em}
\spelltwocol{\textbf{Type}: Staff}{}
\textbf{Materials}: Bone, wood


\lowercase{\hypertarget{item:Staff of Power}{}}\label{item:Staff of Power}
\hypertarget{item:Staff of Power}{\subsubsection{Staff of Power\hfill\nth{8} (2,750 gp)}}

You gain a \plus2 \glossterm{magic bonus} to \glossterm{power} with \glossterm{magical} abilities.



\vspace{0.25em}
\spelltwocol{\textbf{Type}: Staff}{}
\textbf{Materials}: Bone, wood


\lowercase{\hypertarget{item:Staff of Power, Greater}{}}\label{item:Staff of Power, Greater}
\hypertarget{item:Staff of Power, Greater}{\subsubsection{Staff of Power, Greater\hfill\nth{17} (125,000 gp)}}

You gain a \plus4 \glossterm{magic bonus} to \glossterm{power} with \glossterm{magical} abilities.



\vspace{0.25em}
\spelltwocol{\textbf{Type}: Staff}{}
\textbf{Materials}: Bone, wood


\lowercase{\hypertarget{item:Staff of Precision}{}}\label{item:Staff of Precision}
\hypertarget{item:Staff of Precision}{\subsubsection{Staff of Precision\hfill\nth{8} (2,750 gp)}}

You gain a \plus1 \glossterm{magic bonus} to \glossterm{accuracy}.



\vspace{0.25em}
\spelltwocol{\textbf{Type}: Staff}{}
\textbf{Materials}: Bone, wood


\lowercase{\hypertarget{item:Staff of Precision, Greater}{}}\label{item:Staff of Precision, Greater}
\hypertarget{item:Staff of Precision, Greater}{\subsubsection{Staff of Precision, Greater\hfill\nth{17} (125,000 gp)}}

You gain a \plus2 \glossterm{magic bonus} to \glossterm{accuracy}.



\vspace{0.25em}
\spelltwocol{\textbf{Type}: Staff}{}
\textbf{Materials}: Bone, wood


\lowercase{\hypertarget{item:Staff of Transit}{}}\label{item:Staff of Transit}
\hypertarget{item:Staff of Transit}{\subsubsection{Staff of Transit\hfill\nth{6} (1,200 gp)}}

Your \glossterm{magical} abilities have the maximum distance they can \glossterm{teleport} targets doubled.



\vspace{0.25em}
\spelltwocol{\textbf{Type}: Staff}{}
\textbf{Materials}: Bone, wood


\section{Tools}

    TODO

\section{Legacy Items}\label{Legacy Items}

    Over time, items associated with places and people of great power gain magical properties.
    This process takes place for you as you gain levels in addition to in the world as a whole.

    At 4th level, you choose a nonmagical item you own.
    That item becomes a \glossterm{legacy item}, and gains a magic item ability you choose.
    The ability's level must be no greater than 4th level.
    You do not have to \glossterm{attune} to your legacy item, and it does not consume an \glossterm{item slot}.
    If you choose a weapon or implement, you may choose any weapon or implement ability.
    If you choose any other item, you may choose any other ability.

    At 8th level, and every 4 levels thereafter, your legacy item increases in power again.
    You choose an ability of the appropriate type with a level no greater than your level when you choose the ability.
    You can choose a modified version of an existing ability on the item, such as the \textit{greater armor of invulnerability} ability if your legacy item already has the \textit{armor of invulnerability} ability.
    However, if you do so, you must change the lower level ability to be a different magic item ability.
    The new ability must meet the same maximum level requirement that it had when you first chose it.

    If you lose your legacy item, you must retrieve it to regain its power.
    There are rituals to facilitate this retrieval such as \ritual{seek legacy} and \ritual{retrieve legacy}.
    If your legacy item is \glossterm{destroyed}, you can designate a new item of the same type to be your legacy item, causing it to gain all of your legacy item abilities.
    Designating a new item in this way requires spending an \glossterm{action point} as a standard action while holding the replacement item.

    \parhead{Unique Legacy Items}
        Legacy items are fundamentally a reflection of the character who wields them.
        Their effects can be more unusual and complex than abilities on normal magic items, and they can have a larger effect on the way that character interacts with the world.
        As a player, you can work with your GM to create custom magical effects of an appropriate power that are a better reflection of your character's personality and powers than the magic item abilities that exist.

\section{Magic Item Creation}\label{Magic Item Creation}

    TODO
