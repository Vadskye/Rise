\section{Magic Item Creation}\label{Magic Item Creation}\label{Creating Magic Items}

    By investing time, money, and energy, spellcasters and crafters of great skill can imbue items with magical power. Learning how to perform this process requires either the Craft Magic Item (see page \featpref{Craft Magic Item}) or Enchant Item (see page \featpref{Enchant Item}) feats. In addition, each magical item has certain requirements that must be met before it can be crafted. These requirements can be met in one of two ways: by casting spells into the item, or by using an appropriate Craft skill.

    In almost all cases, ``creating'' a magic item actually refers to the act of enhancing an existing object with magical power.
    Creating entirely new items from raw materials is only possible with the Craft skill or specific rituals, such as the \spell{fabricate} ritual.

\subsection{Requirements}
Consider the requirements of a \mitem{flaming} weapon.

\textit{Creation Requirements:} Evocation [Fire]; 2nd level spells or Craft (as weapon) 9 ranks

This is composed of three parts: the school, the ability tags, and the creation requirements.

\subsubsection{Using Spells}
    To create an item with a spell, you must have the Enchant Item feat, and you must know a single spell that has the school and tags listed in the magic item's requirements. In addition, you must be able to learn and cast spells of the indicated level. For example, a wizard who knows the Fireball spell would be able to craft a \mitem{flaming} weapon, because \spell{fireball} is a 3rd level spell from the Evocation school with the \glossterm{Fire} tag. If an item has multiple tags, you must know a spell with the same combination of tags.
    
    The spell used can have additional tags or schools. For example, \spell{fire shield} also has the \glossterm{Shielding} tag, but it can still be used to craft a \mitem{flaming} weapon.

    Some magic items are more complex, requiring multiple schools or tags. You must meet all requirements for the item to craft it.

\subsubsection{Crafting}
    To craft an item, you must have the Craft Magic Item feat, and you must have at least as many ranks in the relevant Craft skill as the magic item requires.
    In addition, you must know how to create items with all tags present on the item (see page \featpref{Craft Magic Item}).

    Many magic items use the same Craft skill as the base item they are applied to. For example, almost all special properties of weapons require the craft skill of the weapon they are on, which may change depending on the weapon being enhanced. For example, adding the \mitem{flaming} ability to a longsword would require Craft (metal), but adding it to a quarterstaff would require Craft (wood).

    Some magic items are complex, requiring multiple tags or even multiple Craft skills. You must meet all requirements for the item to craft it.

\subsection{Creation Process}
Regardless of the type of item being created, item creation always has certain features in common.

\parhead{Raw Materials Cost} The cost of creating a magic item equals one-half the sale cost of the item.

Creating a magic item with a Craft skill also requires access to the normal tools required to craft items with that skill, such as a forge and anvil or an alchemist's laboratory.

\parhead{Negative Levels} Power and energy that a spellcaster would normally have is expended when making a magic item. While crafting a magic item, a spellcaster gains a negative level that cannot be removed. After the magic item is complete, the spellcaster still suffers the negative level for two days per day required to make the item. Scrolls and potions are less draining, and only bestow negative levels for one day per day required to make the item. These negative levels cannot be removed by any means.

\parhead{Time} Creating an item can require a significant time investment, based on the cost in raw materials required to create the item. The time required to craft an item is specified in the description of the ability that allows you to create items.

\parhead{Item Cost} Potions and scrolls directly reproduce spell effects, and the power and price of these items depends on the level of the spell they replicate.

\parhead{Extra Costs} Any potion or scroll that stores a spell with a costly material component also carries a commensurate cost. The creator must expend the material component when creating the item.

\par Some magic items similarly incur extra costs in material components, as noted in their descriptions.

\section{Determining Item Prices}

\subsection{Scaling Bonuses}
Items which give simple scaling bonuses are easy to price. Each bonus has a fixed price that depends on the statistic being enhanced, as shown on \trefnp{Scaling Item Costs}.

\begin{dtable*}
    \lcaption{Scaling Item Costs}
    \begin{dtabularx}{\textwidth}{X l l l l l}
        \tb{Item Effect} & \tb{\plus1 Bonus} & \tb{\plus2 Bonus} & \tb{\plus3 Bonus} & \tb{\plus4 Bonus} & \tb{\plus5 Bonus} \tableheaderrule
        Offensive legend point and weapon damage & 200 gp & 1,000 gp & 5,000 gp & 25,000 gp & 125,000 gp \\
        Armor defense\fn{1} & 100 gp & 500 gp & 2,500 gp & 12,500 gp & 62,500 gp \\
        offensive legend point and spell damage (single school) & 50 gp & 250 gp & 1,250 gp & 6,250 gp & 31,250 gp \\
        Offensive legend point and spell damage (all schools) & 150 gp & 750 gp & 3,750 gp & 18,750 gp & 93,750 gp \\
        Defensive legend point and temporary hit points & 100 gp & 500 gp & 2,500 gp & 12,500 gp & 62,500 gp \\
        \tb{Item Effect} & \tb{\plus2 Bonus} & \tb{\plus4 Bonus} & \tb{\plus6 Bonus} & \tb{\plus8 Bonus} & \tb{\plus10 Bonus} \\
        Skill (single) & 100 gp & 500 gp & 2,500 gp & 12,500 gp & 62,500 gp \\
    \end{dtabularx}
    1. Does not stack with Armor defense bonuses from physical armor. \\
    This table is approximate, and not intended for players to use to create new items.
\end{dtable*}

\subsection{Special Abilities}

Abilities more complicated than a simple bonus are more difficult to price. However, there are still consistent principles which can be followed. To assign a price to a special ability, follow the steps below.
\begin{enumerate*}
    \item Assign an effective spell level to the ability based on its power.
        \begin{itemize}
            \item Apparel items with abilities that affect the wearer are treated as being touch range when determining the level of the ability.
        \end{itemize}
    \item Decide how the ability will be activated.
    \item Determine the price, using \tref{Item Prices by Activation Method}.
\end{enumerate*}

%Patterns: Easy trigger is almost an item of 4 levels higher than when the
%spell could first be acquired (5th level spell is 14th level item, 4 levels
%higher than 10th level, when 5th level spells are acquired)
%Difficult trigger is gained around when the spell could normally first be
%cast.
\begin{dtable*}
    \lcaption{Item Prices by Activation Method}
    \begin{dtabularx}{\textwidth}{l X X X X}
        \tb{Spell Level} & \tb{Specific Action\fn{1} (Item Level)} & \tb{Triggered\fn{2} (Item Level)} & \tb{Continuous\fn{3} (Item Level)} \tableheaderrule
        Cantrip\fn{4} & 100 gp (2nd)      & 100 gp (3rd)      & 200 gp (3rd)      \\
        1st           & 200 gp (3rd)      & 200 gp (3rd)      & 800 gp (5th)      \\
        2nd           & 800 gp (5th)      & 800 gp (5th)      & 2,000 gp (8th)    \\
        3rd           & 2,000 gp (8th)    & 2,000 gp (8th)    & 5,000 gp (10th)   \\
        4th           & 5,000 gp (10th)   & 5,000 gp (10th)   & 12,000 gp (12th)  \\
        5th           & 12,000 gp (12th)  & 12,000 gp (12th)  & 30,000 gp (14th)  \\
        6th           & 30,000 gp (14th)  & 30,000 gp (14th)  & 60,000 gp (16th)  \\
        7th           & 60,000 gp (16th)  & 60,000 gp (16th)  & 140,000 gp (18th) \\
        8th           & 140,000 gp (18th) & 140,000 gp (18th) & 300,000 gp (20th) \\
        9th           & 300,000 gp (20th) & 300,000 gp (20th) & 700,000 gp (\tdash)   \\
    \end{dtabularx}
    1 Actiated with a time-consuming action, such as making a gesture or drinking a potion. \\
    2 Triggered effects should only modify existing actions, rather than constituting new actions, and should not have beneficial effects that last longer than the current round. \\
    3 Only effects that only target you can be made continuous. The spell level should be calculated as if it was touch range and \durext duration. \\
    4 Or other effects weaker than a 1st level spell. \\
    This table is approximate, and not intended for players to use to create new items.
\end{dtable*}
