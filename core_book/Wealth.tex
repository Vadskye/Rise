\chapter{Wealth}

\section{Wealth By Level}
    Characters can generally expect to have a certain amount of total wealth, gained through the course of their adventures. The below chart summarizes the amount of wealth a character can expect to have. This may take the form of currency, precious gems, magic items, land, or anything else of significant value. For the purpose of character wealth, magic items are considered to be worth their market price, regardless of how they were acquired.

    \begin{dtable}
        \lcaption{Character Wealth}
        \begin{dtabularx}{\columnwidth}{c >{\ccol}X >{\ccol}X}
            \tb{Level} & \tb{Total wealth} & \tb{Wealth gained at level} \tableheaderrule
            1  & 75 gp        & 75 gp      \\
            2  & 200 gp       & 125 gp     \\
            3  & 400 gp       & 200 gp     \\
            4  & 800 gp       & 400 gp     \\
            5  & 1,500 gp     & 700 gp     \\
            6  & 2,500 gp     & 1,000 gp   \\
            7  & 4,000 gp     & 1,500 gp   \\
            8  & 6,500 gp     & 2,500 gp   \\
            9  & 10,000 gp    & 3,500 gp   \\
            10 & 16,000 gp    & 6,000 gp   \\
            11 & 25,000 gp    & 9,000 gp   \\
            12 & 40,000 gp    & 15,000 gp  \\
            13 & 60,000 gp    & 20,000 gp  \\
            14 & 90,000 gp    & 30,000 gp  \\
            15 & 135,000 gp   & 45,000 gp  \\
            16 & 205,000 gp   & 70,000 gp  \\
            17 & 308,000 gp   & 103,000 gp \\
            18 & 460,000 gp   & 152,000 gp \\
            19 & 680,000 gp   & 220,000 gp \\
            20 & 1,000,000 gp & 320,000 gp \\
        \end{dtabularx}
    \end{dtable}

\section{Item Levels}

    Each item has a level associated with it.
    An item's level is generally correlated with the item's effectiveness and rarity.
    It determines the normal \glossterm{power} used for the item (see \pcref{Item Power}).
    The cost to buy an item in areas where buying items is possible is also generally determined by its level, as defined in \trefnp{Item Levels}.

    \subsection{Gear and Consumables}
        Long-term items that are expected to be worn or otherwise used repeatedly are more expensive than items that are destroyed immediately after being used.
        Although consumable items are cheaper, they still use their full item level for the purpose of determining their power, difficulty to craft or buy, and any other purposes.

    \subsection{Gearing with Item Levels}

        You can equip a character using by using a number of items of appropriate levels instead of by individually spending all of the wealth allotted to the character. To do so, give the character one item of each level, starting with the character's level and ending three levels lower, for a total of four items. If the character is lower than 4th level, add 1/2-level items as necessary to total 4 items.

        If you want more items, you can trade an item of one level for two items of a lower level.
        You can also trade two items of a lower level for an item of a higher level, but this should not be used to gain an item of a level higher than the character's level.
        Items can be traded according to the table below.

        \begin{dtable!*}
            \lcaption{Item Levels}
            \begin{dtabularx}{\textwidth}{c c c >{\ccol}X}
                \tb{Item Level} & \tb{Typical Gear Price} & \tb{Typical Consumable Price} & \tb{Worth two items of this level}\tableheaderrule
                1/2 & 10 gp      & 2 gp     & \tdash \\
                1   & 50 gp      & 10 gp    & 1/2    \\
                2   & 125 gp     & 25 gp    & 1      \\
                3   & 250 gp     & 50 gp    & 2      \\
                4   & 500 gp     & 100 gp   & 3      \\
                5   & 800 gp     & 160 gp   & 4      \\
                6   & 1,200 gp   & 240 gp   & 4      \\
                7   & 1,800 gp   & 360 gp   & 5      \\
                8   & 2,750 gp   & 550 gp   & 6      \\
                9   & 4,000 gp   & 800 gp   & 7      \\
                10  & 6,500 gp   & 1,300 gp  & 8      \\
                11  & 10,000 gp  & 2,000 gp  & 9      \\
                12  & 16,000 gp  & 3,200 gp  & 10     \\
                13  & 25,000 gp  & 5,000 gp  & 11     \\
                14  & 37,000 gp  & 7,400 gp  & 12     \\
                15  & 55,000 gp  & 11,000 gp & 13     \\
                16  & 85,000 gp  & 17,000 gp & 14     \\
                17  & 125,000 gp & 25,000 gp & 15     \\
                18  & 190,000 gp & 38,000 gp & 16     \\
                19  & 280,000 gp & 56,000 gp & 17     \\
                20  & 400,000 gp & 80,000 gp & 18     \\
            \end{dtabularx}
        \end{dtable!*}
