\chapter{Wealth}

\section{Wealth By Level}
Characters can generally expect to have a certain amount of total wealth, gained through the course of their adventures. The below chart summarizes the amount of wealth a character can expect to have. This may take the form of currency, precious gems, magic items, land, or anything else of significant value. For the purpose of character wealth, magic items are considered to be worth their market price, regardless of how they were acquired.

\begin{dtable}
    \lcaption{Character Wealth}
    \begin{dtabularx}{\columnwidth}{c >{\ccol}X >{\ccol}X}
        \tb{Level} & \tb{Total wealth} & \tb{Wealth gained at level} \\
        \bottomrule
        1  & 75 gp        & 75 gp      \\
        2  & 200 gp       & 125 gp     \\
        3  & 400 gp       & 200 gp     \\
        4  & 800 gp       & 400 gp     \\
        5  & 1,500 gp     & 700 gp     \\
        6  & 2,500 gp     & 1,000 gp   \\
        7  & 4,000 gp     & 1,500 gp   \\
        8  & 6,500 gp     & 2,500 gp   \\
        9  & 10,000 gp    & 3,500 gp   \\
        10 & 16,000 gp    & 6,000 gp   \\
        11 & 25,000 gp    & 9,000 gp   \\
        12 & 40,000 gp    & 15,000 gp  \\
        13 & 60,000 gp    & 20,000 gp  \\
        14 & 90,000 gp    & 30,000 gp  \\
        15 & 135,000 gp   & 45,000 gp  \\
        16 & 205,000 gp   & 70,000 gp  \\
        17 & 308,000 gp   & 103,000 gp \\
        18 & 460,000 gp   & 152,000 gp \\
        19 & 680,000 gp   & 220,000 gp \\
        20 & 1,000,000 gp & 320,000 gp \\
    \end{dtabularx}
\end{dtable}

\section{Item Levels}

Each item has a level associated with it. This level is different from its \glossterm{power}, and has no in-game significance; instead, it represents the level of character for which the item is appropriate. Item levels are based on the price of the item, using the table below.

\subsection{Using Item Levels}

You can equip a character using by using a number of items of appropriate levels instead of by individually spending all of the wealth allotted to the character. To do so, give the character one item of each level, starting with the character's level and ending five levels lower, for a total of six items. If the character is lower than 6th level, add 1/2-level items as necessary to total 6 items.

If you want more items, you can trade an item of one level for two items of a lower level. You can also trade two items of a lower level for an item of a higher level, but this should not be used to gain an item of a level higher than the character's level. Items can be traded according to the table below.

\begin{dtable}
    \lcaption{Item Levels}
    \begin{dtabularx}{\columnwidth}{c c >{\ccol}X}
        \tb{Item Level} & \tb{Market Price Range} & \tb{Worth two items of this level}\\
        \bottomrule
        1/2 & 0 gp -- 10 gp            & \tdash  \\
        1   & 11 gp -- 50 gp           & 1/2 \\
        2   & 51 gp -- 100 gp          & 1   \\
        3   & 101 gp -- 250 gp         & 2   \\
        4   & 251 gp -- 500 gp         & 3   \\
        5   & 501 gp -- 800 gp         & 4   \\
        6   & 801 gp -- 1,200 gp       & 4   \\
        7   & 1,201 gp -- 1,800 gp     & 5   \\
        8   & 1,801 gp -- 2,750 gp     & 6   \\
        9   & 2,751 gp -- 4,000 gp     & 7   \\
        10  & 4,001 gp -- 6,500 gp     & 8   \\
        11  & 6,501 gp -- 10,000 gp    & 9   \\
        12  & 10,001 gp -- 16,000 gp   & 10  \\
        13  & 16,001 gp -- 25,000 gp   & 11  \\
        14  & 25,001 gp -- 37,000 gp   & 12  \\
        15  & 37,001 gp -- 55,000 gp   & 13  \\
        16  & 55,001 gp -- 85,000 gp   & 14  \\
        17  & 85,001 gp -- 125,000 gp  & 15  \\
        18  & 125,000 gp -- 190,000 gp & 16  \\
        19  & 190,001 gp -- 280,000 gp & 17  \\
        20  & 280,001 gp -- 400,000 gp & 18  \\
    \end{dtabularx}
\end{dtable}
