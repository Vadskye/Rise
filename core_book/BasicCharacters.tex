\chapter{Characters}\label{Characters}

In Rise, you play as a character with a variety of unique abilities.
This section provides an overview of how characters are defined.
It assumes that the GM will provide a character for you or help you make your character.
For full details about how to create a character yourself, see the separate Comprehensive Codex.

\section{Species}
  There are eight common species.
  \begin{raggeditemize}
    \item Human: Common and capable of anything
    \item Dwarf: Slow and steady, with famous beards
    \item Elf: Graceful, frail, and extraordinarily long-lived
    \item Gnomes: Magically inclined tinkerers
    \item Half-elves: Diplomatic and versatile
    \item Half-orc: Strong and physically versatile
    \item Orc: Strong and intimidating
  \end{raggeditemize}

\section{Attributes}
  There are six fundamental attributes.
  A character's attributes have broad effects on many aspects of their statistics.
  \begin{raggeditemize}
    \item Strength: Muscle and physical power
    \item Dexterity: Hand-eye coordination, agility, and reflexes
    \item Constitution: Health and stamina
    \item Intelligence: Learning, reasoning, and versatility
    \item Perception: Observation, awareness, and precision
    \item Willpower: Mental resilience and magical potential
  \end{raggeditemize}

\section{Combat Statistics}\label{Combat Statistics}
  There are five main combat statistics.
  \begin{raggeditemize}
    \item Accuracy: This affects how likely you are to hit with attacks.
    \item Damage resistance: This affects how much damage you can shrug off easily.
    \item Defenses: These affects how likely you are to avoid being hit by attacks.
    \item Hit points: This affects how much damage you can take without vital injury.
      Some special abilities have detrimental effects on you if they make you lose hit points, but not if they only make you lose damage resistance.
    \item Power: This affects how much damage you deal with attacks.
      You have two types of power: \magical power and \glossterm{mundane} power.
  \end{raggeditemize}

\section{Resources}
  There are three main resources.
  \begin{raggeditemize}
    \item Attunement points: This affects how many spells and items you can benefit from simultaneously.
      It typically comes from wearing armor.
    \item Fatigue: This affects your accuracy and checks.
      It typically comes from using special abilities.
    \item Insight points: This affects how many special abilities you know, such as spells and maneuvers.
  \end{raggeditemize}

\section{Narrative Customization}
  A character also has an alignment, age, appearance, and similar traits that anchor them in the world.
  Generally, these do not have any mechanical effects, but they are an important part of designing a complete character.
  For details, see \pcref{Alignment}, \pcref{Personal Appearance}, and \pcref{Backgrounds}.

\section{Classes}
  Each character has at least one class.
  A character's class has a significant influence on their special abilities and narrative style.
  The eleven classes are briefly summarized below.

  \begin{raggeditemize}
    \item Barbarians are primal warriors who draw power from their physical prowess and unfettered emotions.
    \item Clerics are divine spellcasters who draw power from their veneration of a single deity.
    \item Druids are nature spellcasters who draw power from their veneration of the natural world.
    \item Fighters are highly disciplined warriors who excel in physical combat of any kind.
    \item Monks are agile masters of ``ki'' who hone their personal abilities to strike down foes and perform supernatural feats.
    \item Paladins are divinely empowered warriors who exemplify a particular alignment.
    \item Rangers are skilled hunters who bridge the divide between nature and civilization.
    \item Rogues are skillful and versatile characters known for their ability to strike at their foe's weak points in combat.
    \item Sorcerers are arcane spellcasters who draw power from their inherently magical nature.
    \item Votives are pact spellcasters who draw their power from a dangerous deal made with extraplanar creatures.
    \item Wizards are arcane spellcasters who study magic to unlock its powerful secrets.
  \end{raggeditemize}

\section{Skills}
  Skills represent the myriad of talents that people can have, such as cooking or swimming.
  Each character is trained in a certain number of skills from their class and Intelligence.
  The twenty-six skills are summarized below.

  \begin{raggeditemize}
    \item Awareness: This represents your ability to observe things which you might otherwise fail to notice.
    \item Balance: This represents your ability to maintain your balance and poise in difficult circumstances.
    \item Climb: This represents your ability to climb obstacles.
    \item Craft: This represent your ability to construct objects from raw materials.
    \item Creature Handling: This represents your ability to influence non-sapient creatures.
    \item Deception: This represents your ability to lie or otherwise mislead people without being caught.
    \item Deduction: This represents your ability to make logical deductions based on evidence.
    \item Devices: This represents your ability to to manipulate mechanical devices such as locks, traps, and other contraptions.
    \item Disguise: This represents your ability to create disguises to conceal the appearance of creatures or objects.
    \item Endurance: This represents your ability to persevere through physical trials.
    \item Flexibility: This represents your ability to escape bindings and move through small areas by contorting your body.
    \item Intimidate: This represents your ability to intimidate and coerce people into doing what you want.
    \item Jump: This represents your ability to jump.
    \item Knowledge: This represent your understanding of particular aspects of the world.
    \item Medicine: This represents your practical understanding of how to tend to the wounds of living creatures.
    \item Perform: This represent your ability to create particular forms of entertainment.
    \item Persuasion: This represents your ability to convince people to think what you want them to.
    \item Profession: This represent your practical understanding of a particular profession.
    \item Ride: This represents your ability to ride and control horses and other mounts.
    \item Sleight of Hand: This represents your ability to pick pockets, palm objects, and perform other feats of legerdemain.
    \item Social Insight: This represents your ability to read body language and emotion.
    \item Stealth: This represents your ability to escape detection while moving or taking large-scale actions.
    \item Survival: This represents your ability to take care of yourself and others in the wilderness, including the ability to follow tracks.
    \item Swim: This represents your ability to swim.
  \end{raggeditemize}
