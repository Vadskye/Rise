\chapter{Species}\label{Species}

Each character has a species.

\section{Species Traits}

\subsection{Species Bonus Feats}
Each species grants a bonus feat at 1st level. Most species can only choose from a small group of feats, listed in the description of the species. A character must meet any prerequisites for these bonus feats, as normal.

\subsection{Species and Languages}
All characters know how to speak Common. A dwarf, elf, gnome, half-elf, half-orc, or halfling also speaks a language unique to its species, as appropriate. For each point of Intelligence a character has at 1st level, they also know one additional language of their choice. See \pcref{Linguistics}, for details about languages.

\subsection{Small Characters}\label{Small Characters}
A Small character has the following effects based on their size.
\begin{itemize}
    \item \minus2 penalty to total Strength.
    \item \minus2 penalty to \glossterm{threat}.
    \item \plus2 bonus to Reflex defense.
    \item \plus4 bonus to Stealth.
\end{itemize}

In addition, a Small character generally has a move speed five feet slower than a Medium character. A Small character must also use smaller weapons than a Medium character.

\section{Species Descriptions}

\subsection{Humans}
\parhead{Size} Medium.
\parhead{Attributes} No change.
\parhead{Speed} 30 feet.
\parhead{Special Abilities}
\begin{itemize}
    \itemhead*{Flexible}: Humans gain one bonus \glossterm{insight point}. They can spend insight points to learn new special abilities (see \pcref{Insight Points}).
    \itemhead*{Skilled}: Humans gain 2 bonus \glossterm{skill points}. They can spend those skill points on any skills (see \pcref{Skills}).
\end{itemize}
\parhead{Species Bonus Feat} A human may choose any feat as a bonus feat.
\parhead{Automatic Language} Common.

\subsection{Dwarves}
\parhead{Size} Medium.
\parhead{Attributes} \plus1 starting Constitution, \minus1 starting Dexterity.
\parhead{Speed} 25 feet.
\parhead{Special Abilities}
\begin{itemize}
    \itemhead*{Darkvision}: Dwarves can see in the dark clearly up to 50 feet.   Beyond that, they can see dimly, treating areas of darkness as shadowy illumination. Darkvision does not function if a dwarf is in a brightly lit area, and does not resume functioning until the end of the next round after the dwarf leaves the brightly lit area.
    \itemhead*{Dwarven Endurance}: Wearing medium or heavy \glossterm{body armor} does not reduce a dwarf's movement speed (see \pcref{Moving in Armor}).
\end{itemize}
\parhead{Species Bonus Feat} Any from the following list: \featref*{Blindfighter}, \featref*{Class Versatility}, \featref*{Craft Specialization}, \featref*{Guardian}, \featref*{Iron Will}, \featref*{Martial Training}, \featref*{Regenerator}, \featref*{Toughness}.
\parhead{Automatic Languages} Common, Dwarven.

\subsection{Elves}
\parhead{Size} Medium.
\parhead{Attributes} \plus1 starting Dexterity, \minus1 starting Constitution.
\parhead{Speed} 30 feet.
\parhead{Special Abilities}
\begin{itemize}
    \itemhead*{Keen Senses}: \plus2 bonus to Awareness (see \pcref{Awareness}).
    \itemhead*{Low-light Vision}: Elves treat sources of light as if they had double their normal illumination range.
    \itemhead*{Trance}: Elves do not sleep, and are immune to sleep effects. Instead of sleeping, elves can trance for 4 hours. An elf in trance may make Perception-based checks at a \minus5 penalty. Elves must still avoid strenuous activity for 8 hours to heal, avoid fatigue, and gain other benefits of resting.
\end{itemize}
\parhead{Species Bonus Feat} Any from the following list: Any Casting feat (see \pcref{Casting Feats}), \featref*{Agility}, \featref*{Awareness Specialization}, \featref*{Class Versatility}, \featref*{Sniper}.
\parhead{Automatic Languages} Common, Elven.

\subsection{Gnomes}
\parhead{Size} Small. This gives several benefits and penalties, as described at \pcref{Small Characters}.
\parhead{Attributes} \plus1 starting Constitution. In addition, being Small gives gnomes a \minus2 penalty to total Strength.
\parhead{Speed} 25 feet.
\parhead{Special Abilities}
\begin{itemize}
    \itemhead*{Fae Light}: A gnome can use the \textit{fae light} ability as a \glossterm{standard action}.
        \begin{attuneability}{Fae Light}[\glossterm{Attune} (self)]
            A Tiny glowing orb appears at a location within \rngmed range.
            It sheds pale, bright light in a \areamed radius, and dim light for an additional 20 feet.
            The orb is intangible, and cannot be moved once placed.
        \end{attuneability}
    \itemhead*{Low-light Vision}: Gnomes treat sources of light as if they had double their normal illumination range.
    \itemhead*{Tinker}: Gnomes gain a \plus2 bonus to a Craft skill of their choice (see \pcref{Craft}).
\end{itemize}
\parhead{Species Bonus Feat} Any Casting feat (see \pcref{Casting Feats}), or any from the following list: \featref*{Blindfighter}, \featref*{Class Versatility}, \featref*{Craft Specialization}, \featref*{Stealth Specialization}, \featref*{Toughness}.
\parhead{Automatic Languages} Common, Gnome.

\subsection{Half-Elves}
\parhead{Size} Medium.
\parhead{Attributes} No change.
\parhead{Speed} 30 feet.
\parhead{Special Abilities}
\begin{itemize}
    \itemhead*{Dual Heritage}: For all effects related to species, a half-elf is considered both a human and an elf.
    \itemhead*{Hybrid Training}: Choose a class.
        You gain the \glossterm{class skills} of that class in addition to your existing class skills.
        In addition, you can exchange one class archetype from your class with one class archetype from that class.
        If that class has any basic class abilities which are not part of an archetype and do not have abilities of the same on other classes, such as a cleric's \textit{divine power}, you gain those abilities.
    \itemhead*{Low-light Vision}: Half-elves treat sources of light as if they had double their normal illumination range.
\end{itemize}
\parhead{Species Bonus Feat} Any Skill feat (see \pcref{Skill Feats}), or \featref*{Class Versatility}.
\parhead{Automatic Languages} Common, Elven.

\subsection{Half-Orcs}
\parhead{Size} Medium.
\parhead{Attributes} \plus1 starting Strength, \minus1 starting Intelligence.
\parhead{Speed} 30 feet.
\parhead{Special Abilities}
\begin{itemize}
    \itemhead*{Darkvision}: Half-orcs can see in the dark clearly up to 50 feet.   Beyond that, they can see dimly, treating areas of darkness as shadowy illumination. Darkvision does not function if a half-orc is in a brightly lit area, and does not resume functioning until the end of the next round after the half-orc leaves the brightly lit area.
    \itemhead*{Dual Heritage}: For all effects related to species, a half-orc is considered both a human and an orc.
    \itemhead*{Intimidating}: Half-orcs gain a \plus2 bonus to Intimidate (see \pcref{Intimidate}).
\end{itemize}
\parhead{Species Bonus Feat} Any Combat feat (see \pcref{Combat Feats}), \featref*{Class Versatility}, or \featref*{Toughness}.
\parhead{Automatic Languages} Common, Orc.

\subsection{Halflings}
\parhead{Size} Small. This gives several benefits and penalties, as described at \pcref{Small Characters}.
\parhead{Attributes} \plus1 starting Dexterity. In addition, being Small gives halflings a \minus2 penalty to total Strength.
\parhead{Speed} 25 feet.
\parhead{Special Abilities}
\begin{itemize}
    \itemhead*{Nimble Combatant}: Halflings gain a \plus1 bonus to Armor defense.
\end{itemize}
\parhead{Species Bonus Feat} Any from the following list: \featref*{Agility}, \featref*{Class Versatility}, \featref*{Climb Specialization}, \featref*{Iron Will}, \featref*{Jump Specialization}, \featref*{Stealth Specialization}.
\parhead{Automatic Languages} Common, Halfling.
