\chapter{Ship Combat}
    Normally, combat that happens on ships is resolved through local-scale combat rules.
    These rules work best if boarding actions are common, and if the main threat comes from other characters.
    If you aren't running a full naval campaign, these rules are generally fine, and most GMs will just handwave the initial exchange of fire between ships before they get in range for boarding actions.
    However, it can be useful to have more comprehensive rules for ship combat, where ships themselves frequently deal and suffer damage.

    This chapter presents optional rules to govern ship-based combat.
    They are designed to still emphasize the importance of individual player actions.
    It is primarily intended for naval campaigns where the players are important crew members and ship combat is expected to be common.
    These rules could also be used for other types of vehicles, such as flying ships in the Astral Plane or zeppelins.

    \section{Ship Statistics}
    Ships use the same basic framework for calculating their statistics as characters and monsters.
    A ship has a level that indicates its general power.
    This represents the sophistication of its construction, how advanced its weapons are, the general competence of its crew, and so on.
    A ship's level determines its hit points, damage resistance, accuracy, defenses, and power, as indicated in the table below.
    These values change based on the ship's attributes, as indicated in \pcref{Ship Attributes}.

    % NOTE: Keep in sync with Monster Advancement in Monsters
    \begin{dtable}
        \lcaption{Ship Statistics}
        \begin{dtabularx}{\columnwidth}{l l l l >{\lcol}X}
            \tb{Level} & \tb{HP} & \tb{DR} & \tb{Defenses} & \tb{Accuracy, Power} \tableheaderrule
            1st        & 8       & 4       & 3             & \tdash  \\
            2nd        & 10      & 5       & 4             & \plus1  \\
            3rd        & 12      & 6       & 4             & \plus1  \\
            4th        & 14      & 7       & 5             & \plus2  \\
            5th        & 16      & 8       & 5             & \plus2  \\
            6th        & 18      & 9       & 6             & \plus3  \\
            7th        & 20      & 10      & 6             & \plus3  \\
            8th        & 23      & 11      & 7             & \plus4  \\
            9th        & 26      & 13      & 7             & \plus4  \\
            10th       & 29      & 14      & 8             & \plus5  \\
            11th       & 32      & 16      & 8             & \plus5  \\
            12th       & 35      & 17      & 9             & \plus6  \\
            13th       & 40      & 20      & 9             & \plus6  \\
            14th       & 46      & 23      & 10            & \plus7  \\
            15th       & 52      & 26      & 10            & \plus7  \\
            16th       & 58      & 29      & 11            & \plus8  \\
            17th       & 64      & 32      & 11            & \plus8  \\
            18th       & 70      & 35      & 12            & \plus9  \\
            19th       & 80      & 40      & 12            & \plus9  \\
            20th       & 92      & 46      & 13            & \plus10 \\
            21st       & 104     & 52      & 13            & \plus10 \\
        \end{dtabularx} 
    \end{dtable}

    \begin{dtable}
        \lcaption{Elite Ship Statistics}
        \begin{dtabularx}{\columnwidth}{l l l l >{\lcol}X >{\lcol}X}
            \tb{Level} & \tb{HP} & \tb{DR} & \tb{Defenses} & \tb{Accuracy} & \tb{Power} \tableheaderrule
            1st        & 24      & 12      & 5             & \tdash        & \plus2  \\
            2nd        & 30      & 15      & 6             & \plus1        & \plus3  \\
            3rd        & 36      & 18      & 6             & \plus1        & \plus3  \\
            4th        & 42      & 21      & 7             & \plus2        & \plus4  \\
            5th        & 48      & 24      & 7             & \plus2        & \plus4  \\
            6th        & 54      & 27      & 8             & \plus3        & \plus5  \\
            7th        & 60      & 30      & 8             & \plus3        & \plus5  \\
            8th        & 69      & 34      & 9             & \plus4        & \plus6  \\
            9th        & 78      & 39      & 9             & \plus4        & \plus6  \\
            10th       & 87      & 43      & 10            & \plus5        & \plus7  \\
            11th       & 96      & 48      & 10            & \plus5        & \plus7  \\
            12th       & 105     & 52      & 11            & \plus6        & \plus8  \\
            13th       & 120     & 60      & 11            & \plus6        & \plus8  \\
            14th       & 138     & 69      & 12            & \plus7        & \plus9  \\
            15th       & 156     & 78      & 12            & \plus7        & \plus9  \\
            16th       & 174     & 87      & 13            & \plus8        & \plus10 \\
            17th       & 192     & 96      & 13            & \plus8        & \plus10 \\
            18th       & 210     & 105     & 14            & \plus9        & \plus11 \\
            19th       & 240     & 120     & 14            & \plus9        & \plus11 \\
            20th       & 276     & 138     & 15            & \plus10       & \plus12 \\
            21st       & 312     & 156     & 15            & \plus10       & \plus12 \\
        \end{dtabularx} 
    \end{dtable}

    \subsection{Ship HP and DR}
        Ships have hit points and damage resistance, just like characters.
        A ship's base hit points and damage resistance are based on its level, as listed in \tref{Ship Statistics}.
        In addition, a ship gains a bonus based on its Constitution, as listed in \pcref{Ship Attributes}.
        Armor can also give ships a bonus to their damage resistance, as listed in \pcref{Ship Armor}.

    \subsection{Elite Ships}\label{Elite Ships}
        Some ships are designated ``elite'' ships.
        Elite ships are much larger than ordinary ships.
        Even the smallest elite ship requires no less than 10 crew members to operate correctly.

        Elite ships are approximately four times as strong as an ordinary ships.
        They have a number of benefits and modifiers which make them superior to ordinary ships:
        % NOTE: Keep in sync with elite monsters
        \begin{itemize}
            \item Elite ships gain a \plus2 bonus to their \glossterm{power} and all \glossterm{defenses}.
            \item Elite ships have three times the \glossterm{hit points} and \glossterm{damage resistance} of standard ships.
            % \item Elite ships have 50\% more \glossterm{damage resistance} than standard ships. This is applied after the bonus they get from having higher hit points, since ship damage resistance is based on their hit points.
            % \item Elite ships can have a maximum starting attribute of 6 (see \pcref{Ship Attributes}).
            % elite actions should be about as strong as regular actions, so 2x more damage
            \item Elite ships can take an additional \glossterm{elite action} each round (see \pcref{Ship Actions}).
        \end{itemize}

    \subsection{Ship Attributes}
        Unlike characters, ships have no Intelligence attribute.
        However, they have Strength, Dexterity, Constitution, Perception, and Willpower attributes.
        These represent slightly different narrative concepts than they do for characters.
        In general, a ship's attributes represent a combination of its physical properties and the effectiveness of its crew.
        A highly advanced ship may still have low attributes when its crew is inexperienced or incompetent.

        \subsubsection{Strength}
            Strength measures the power of a ship's physical weapons.
            Ships with a high Strength have more damaging weapons, and a crew capable of keeping those weapons working effectively.
            Ships with a low Strength have ineffective weaponry, or a crew that services those weapons poorly.
            Strength has the following effects on ships:
            \begin{raggeditemize}
                \item Ships add their Strength to their \glossterm{mundane power}.
                \item Some weapons require a minimum ship Strength (see \pcref{Ship Weapons}).
            \end{raggeditemize}

            Unlike characters, a ship's Strength does not affect its carrying capacity.
            That is calculated entirely from its physical size and shape.

        \subsubsection{Dexterity}
            Dexterity measures a ship's agility.
            Ships with a high Dexterity can turn more sharply to avoid incoming fire and may be faster in short bursts.
            Ships with a low Dexterity are lumbering and slow to change direction, making them easy targets.
            Dexterity has the following effects on ships:
            \begin{raggeditemize}
                \item Ships add their Dexterity to their Armor defense.
                    This bonus can be reduced if the ship has medium or heavy armor (see \pcref{Ship Armor}).
                \item Ships add their Dexterity to their Reflex defense.
            \end{raggeditemize}

            As with characters, a ship's Dexterity does not affect its overall speed, simply its combat maneuverability.

        \subsubsection{Constitution}
            Constitution measures a ship's durability.
            Ships with a high Constitution are heavily reinforced and well crafted from sturdy materials.
            Ships with a low Constitution fall apart more easily, either because their construction is poor or because they were made from weak materials.
            Constitution has the following effects on ships:
            \begin{raggeditemize}
                \item Ships add twice their Constitution to their hit points.
                    At level 7, this bonus increases to three times the ship's Constitution.
                    At level 13, this bonus increases to six times the ship's Constitution.
                    At level 19, this bonus increases to twelve times the ship's Constitution.
                \item Ships add their Constitution to their Fortitude defense.
            \end{raggeditemize}

            Constitution is the attribute which is least affected by a ship's current crew, and most affected by its physical properties.

        \subsubsection{Perception}
            Perception measures a ship's awareness and precision.
            Ships with a high Perception have effective lookouts, excellent gunners, and weapons which are capable of swiftly repositioning for precise attacks.
            Ships with a low Perception are either unable to effectively observe their surroundings or unable to react effectively to those observations.
            Perception has the following effects on ships:
            \begin{raggeditemize}
                \item Ships add their Perception to their level to determine their \glossterm{accuracy} with almost all attacks (see \pcref{Accuracy}).
            \end{raggeditemize}

        \subsubsection{Willpower}
            Willpower measures the morale and emotional steadiness of a ship's crew.
            Ships with a high Willpower are better able to resist setbacks and frightening encounters.
            Ships with a low Willpower may panic and be driven off easily.
            Willpower has the following effects on ships:
            \begin{raggeditemize}
                \item Ships add their Willpower to their Mental defense.
            \end{raggeditemize}

            Willpower is the attribute which is most affected by a ship's current crew.

\section{Ship Size}\label{Ship Size}

        \begin{dtable*}
            \lcaption{Ship Size}
            \begin{dtabularx}{\textwidth}{l l l l l >{\lcol}X >{\lcol}X >{\lcol}X l}
                \tb{Size}  & \tb{Min Level} & \tb{Elite?} & \tb{Crew}\fn{1} & \tb{Armor}      & \tb{Space} & \tb{Speed} & \tb{Cargo}    & \tb{Item Rank} \tableheaderrule
                Medium     & 1              & No          & 1               & Light           & 5 ft.      & 30 ft.     & Small x2      & \tdash \\
                Large      & 1              & No          & 1---2           & Light           & 10 ft.     & 30 ft.     & Medium x2     & \tdash \\
                Huge       & 4              & Either      & 1---5           & Light or medium & 20 ft.     & 40 ft.     & Large x2      & \plus1\fn{2} \\
                Gargantuan & 7              & Either      & 2---20          & Any             & 40 ft.     & 50 ft.     & Huge x2       & \plus1\fn{2} \\
                Colossal   & 10             & Yes         & 10---100        & Any             & 80 ft.     & 60 ft.     & Gargantuan x2 & \plus3 \\
                Galleon    & 13             & Yes         & 50---500        & Any             & 160 ft.    & 80 ft.     & Colossal x2   & \plus3 \\
                Titan      & 16             & Yes         & 100---1000      & Any             & 320 ft.    & 100 ft.    & Galleon x2    & \plus4 \\
            \end{dtabularx}
            1. This range indicates the number of crew members that meaningfully contribute to the ship's functions, not the ship's maximum carrying capacity including passengers and cargo.
            It is either difficult or impossible to adequately control a large ship with less than the minimum crew listed here.
            Individual ships may have higher minimum crew requirements or lower maximum allowable crew based on their structure, at the GM's discretion. \\
            2. If the ship is Elite, increase its item rank by an additional \plus1.
        \end{dtable*}

        A ship's size does not directly affect its statistics.
        However, it has many effects on the ship's functionality.
        Larger ships are much more capable than smaller ships.
        Some of these effects are listed below in \trefnp{Ship Size}.
        In addition, advanced ship weapons often require a minimum ship size (see \pcref{Ship Weapons}).

        Some ships can be larger than most creatures and objects are usually defined.
        To track ship size beyond the limits of Colossal, additional Galleon and Titan categories are listed below.
        As usual, each size category represents a doubling of each dimension, and an eightfold increase in weight.
        Titan ships are unlikely to be present at all in many universes, and they require extensive magical reinforcement to function.
        The GM can decide whether their world is advanced enough to construct such monstrosities.

        Some examples of ships of a given size are given below.
        Since ships are typically named for their function and structure, not their size, this is only a rough guide.
        \begin{raggeditemize}
            \item Medium: Single-person kayak
            \item Large: Canoe, lifeboat
            \item Huge: Dinghy, outrigger canoe, punt, skiff
            \item Gargantuan: Felucca, small longship
            \item Colossal: Keelboat, large longship
        \end{raggeditemize}

\section{Ship Armor}
    Like characters, ships can have varying degrees of armor.
    Typically, even a heavily armored ship will not be literally covered in metal sheets.
    Instead, ship armor represents a heavily reinforced hull and extra layers of bracing and redundant infrastructure.

    There are three types of ship armor.
    \begin{raggeditemize}
        \item Light armor: Lightly armored ships are the default. They gain no special benefits or penalties.
        \item Medium armor: Ships with medium armor gain a \plus2 bonus to Armor defense.
            In addition, they have 50\% more damage resistance than a normal ship.
            However, they add only half their Dexterity bonus to their Armor defense.
            In addition, their movement speed is calculated as if they were one size category smaller (see \pcref{Ship Size}).
            Only Huge and larger ships can have medium armor.
        \item Heavy armor: Ships with heavy armor gain a \plus3 bonus to Armor defense.
            In addition, they have twice the damage resistance of a normal ship.
            However, they do not add their Dexterity bonus to their armor defense.
            In addition, their movement speed is calculated as if they were two size categories smaller (see \pcref{Ship Size}).
            Only Gargantuan and larger ships can have heavy armor.
    \end{raggeditemize}

%         \subsubsection{Customized Ship Armor}
%             Individual ships can have different configurations that change their defenses.
%             With customization, a ship can gain a \plus1 bonus to Armor defense, a \plus2 bonus to any other defense, or become \impervious to one of the following damage types: acid, cold, electricity, or fire.
%             In exchange, it must take a \minus1 penalty to Armor defense or a \minus2 penalty to any other defense.
%             Some rare and valuable ships may have unusual properties that grant additional benefits, at the GM's discretion.

\section{Ship Weapons}\label{Ship Weapons}
    Ships depend on having powerful weapons even more than martial characters do.
    They use weapon upgrades as their primary method of scaling damage rather than maneuvers or other special attacks.

    There are two ways that a GM can choose to use ship weapons.
    Real siege weapons used on ships had high crew requirements, slow firing rates, and extreme range.
    In practice, this can reduce ship combat to a slog of tracking reload times across multiple weapons and carefully maneuvering ship range to make the best use of varying weapon types.
    For GMs who want more realistic and unique ship combat, use the weapons listed in \tref{Realistic Ship Weapons}.
    For GMs who want ship combat to feel simpler and more similar to regular combat, use the weapons listed in \tref{Simplified Ship Weapons}.

    \begin{dtable!*}
        \lcaption{Simplified Ship Weapons}
        \begin{dtabularx}{\textwidth}{l X l l l l}
            \tb{Name} & \tb{Damage}                            & \tb{Targeting} & \tb{Tags}            & \tb{Ship Size} & \tb{Item Rank (Cost)} \tableheaderrule
            Longbow   & 1d6 piercing \add 1 per 2 power        & One target     & Projectile (90/270)  & Medium         & 1 (40 gp)      \\
            Arbalest  & 1d8 piercing \add 1 per 2 power        & One target     & Projectile (120/360) & Large          & 2 (200 gp)     \\
            Scorpion  & 1d8 piercing \add 1 per power          & One target     & Projectile (200/600) & Huge           & 3 (1,000 gp)   \\
            Ballista  & 1d8 piercing per 3 power               & One target     & Projectile (200/600) & Huge           & 4 (5,000 gp)   \\
            Catapult  & 1d6 bludgeoning plus 1d6 per 2 power   & One target     & Projectile (200/600) & Gargantuan     & 5 (25,000 gp)  \\
            Mangonel  & 1d8 bludgeoning per 2 power            & One target     & Projectile (200/600) & Gargantuan     & 6 (125,000 gp) \\
            Trebuchet & 1d10 bludgeoning plus 1d10 per 2 power & One target     & Projectile (200/600) & Colossal       & 7 (625,000 gp) \\
        \end{dtabularx}
    \end{dtable!*}

    % Baseline weapon is dr = rank with Projectile (90/270).
    % Dropping to Projectile (60/180) gives +1dr.
    % Increasing to Projectile (120/360) gives -1dr. Progression is 90 -> 120 -> 200 -> 300.
    \begin{dtable!*}
        \lcaption{Realistic Ship Weapons}
        \begin{dtabularx}{\textwidth}{l l l >{\lcol}X l l l}
            \tb{Name}        & \tb{Damage}                      & \tb{Targeting} & \tb{Tags}                             & \tb{Crew} & \tb{Ship Size} & \tb{Item Rank (Cost)} \tableheaderrule
            Longbow          & 1d6 piercing \add 1 per 2 power  & One target     & Projectile (90/270)                   & 1         & Medium         & 1 (40 gp)     \\
            Rock sling       & 1d8 bludgeoning \add 1 per power & One target     & Projectile (60/180)                   & 1         & Medium         & 2 (200 gp)    \\
            % +2dr for slow reload, -1dr for longer range
            Arbalest         & 1d8 piercing \add 1 per power    & One target     & Projectile (120/360), Slow Reload (2) & 1         & Medium         & 2 (200 gp)    \\
            % +2dr for slow reload, -2dr for range
            Scorpion         & 1d8 piercing \add 1 per power    & One target     & Projectile (200/600), Slow Reload (2) & 1         & Huge           & 3 (1,000 gp)  \\
            % +3dr for slow reload, -3dr for range
            Onager           & 1d8 bludgeoning \add 1 per power & One target     & Projectile (300/900), Slow Reload (3) & 8         & Gargantuan     & 3 (1,000 gp)  \\
            % +3dr for slow reload, -3dr for range
            Torsion ballista & 1d8 piercing per 3 power         & One target     & Projectile (300/900), Slow Reload (3) & 3         & Huge           & 4 (5,000 gp)  \\
            % -1dr for range
            Polybolos        & 1d8 piercing per 3 power         & One target     & Projectile (120/360)                  & 3         & Huge           & 5 (25,000 gp) \\
        \end{dtabularx}
    \end{dtable!*}

    \subsection{Ship Weapon Tags}
        Some weapon tags only apply to ship weapons.

        % +1dr per round
        \weapontagdef{Slow Reload} This weapon requires multiple rounds to after being fired.
        The number of rounds required to finish reloading is indicated in its description.
        For example, a weapon with Slow Reload (1) could be fired every other round.
        Only rounds where the weapon is being fully crewed count towards this reload time.
        The weapon can sustain its load for an arbitrary amount of time in combat before being fired.

        % +2dr for first round, +1dr for each subsequent round
        \weapontagdef{Slow Load} This weapon requires multiple rounds to load before it can fire for the first time.
        The number of rounds required to finish loading before it can be fired is indicated in its description.
        For example, a weapon with Slow Load (1) could be fired every other round.
        Only rounds where the weapon is being fully crewed count towards this load time.
        If the weapon is not fired during the round after its loading time is completed, the loading is wasted, and the weapon must be loaded again before it can fire.

\section{Ships Fighting Non-Ships}
    In some cases, ships might enter combat against non-ship foes.
    Ships take 10x less damage from non-ship attacks, rounded down as usual, so minor attacks will not even scratch them.
    Since ships are not creatures and are not alive, many special attacks have no effect on them.
    They also cannot gain \glossterm{conditions} from non-ship attacks by any means.
    When used against non-ship targets, weapons on Huge or larger ships have the \weapontag{Massive} tag, just like attacks from Huge or larger creatures (see \pcref{Very Large Creatures}).

\section{Ships and Gunpowder}
    In general, Rise avoids the use of guns and gunpowder.
    Those inventions do not fit into the traditional fantasy setting that Rise is built on, which emphasizes swords and bows and magic.
    For similar reasons, the ranged weapons of ships can be defined entirely with non-gunpowder weaponry common in the ancient world.
    These would include ballistas, catapults, scorpions, and similar siege weaponry.

    However, narrative tropes for ships and naval campaigns often have a more technology-heavy basis, with cannons and full broadsides.
    These stories tend to draw inspiration from the Golden Age of Piracy rather than medieval folklore.
    It can feel intuitively plausible to have cannons used as ship weaponry even when guns are never used by individuals.
    The default names for ship weapons assume that gunpowder is not being used.
    If technology has advanced to the point that gunpowder weapons are possible, you can assume that hull reinforcement has also increased at the same rate.
    All the GM has to do is change the name of the weapons to match their preferred technology level.
