\chapter{Ship Combat}
    Normally, combat that happens on ships is resolved through local-scale combat rules.
    These rules work best if boarding actions are common, and if the main threat comes from other characters.
    If you aren't running a full naval campaign, these rules are generally fine, and most GMs will just handwave the initial exchange of fire between ships before they get in range for boarding actions.
    However, it can be useful to have more comprehensive rules for ship combat, where ships themselves frequently deal and suffer damage.

    This chapter presents optional rules to govern ship-based combat.
    They are designed to still emphasize the importance of individual player actions.
    It is primarily intended for naval campaigns where the players are important crew members and ship combat is expected to be common.
    These rules could also be used for other types of vehicles, such as flying ships in the Astral Plane or zeppelins.

\section{Ship Statistics}
    Ships use the same basic framework for calculating their statistics as characters and monsters.
    A ship has a level that indicates its general power.
    This represents the sophistication of its construction, how advanced its weapons are, the general competence of its crew, and so on.
    A ship's level determines its hit points, damage resistance, accuracy, defenses, and power, as indicated in the table below.
    These values change based on the ship's attributes, as indicated in \pcref{Ship Attributes}.

    % NOTE: Keep in sync with Monster Advancement in Monsters
    \begin{dtable}
        \lcaption{Ship Statistics}
        \begin{compresseddtabularx}{\columnwidth}{l l l l l >{\lcol}X}
            \tb{Level} & \tb{HP} & \tb{DR} & \tb{Defenses} & \tb{Accuracy, Power} & \tb{Item Rank (Cost)} \tableheaderrule
            1st        & 8       & 4       & 3             & \tdash               & 1 (40 gp)      \\
            2nd        & 10      & 5       & 4             & \plus1               & 1 (40 gp)      \\
            3rd        & 12      & 6       & 4             & \plus1               & 1 (40 gp)      \\
            4th        & 14      & 7       & 5             & \plus2               & 2 (200 gp)     \\
            5th        & 16      & 8       & 5             & \plus2               & 2 (200 gp)     \\
            6th        & 18      & 9       & 6             & \plus3               & 2 (200 gp)     \\
            7th        & 20      & 10      & 6             & \plus3               & 3 (1,000 gp)   \\
            8th        & 23      & 11      & 7             & \plus4               & 3 (1,000 gp)   \\
            9th        & 26      & 13      & 7             & \plus4               & 3 (1,000 gp)   \\
            10th       & 29      & 14      & 8             & \plus5               & 4 (5,000 gp)   \\
            11th       & 32      & 16      & 8             & \plus5               & 4 (5,000 gp)   \\
            12th       & 35      & 17      & 9             & \plus6               & 4 (5,000 gp)   \\
            13th       & 40      & 20      & 9             & \plus6               & 5 (25,000 gp)  \\
            14th       & 46      & 23      & 10            & \plus7               & 5 (25,000 gp)  \\
            15th       & 52      & 26      & 10            & \plus7               & 5 (25,000 gp)  \\
            16th       & 58      & 29      & 11            & \plus8               & 6 (125,000 gp) \\
            17th       & 64      & 32      & 11            & \plus8               & 6 (125,000 gp) \\
            18th       & 70      & 35      & 12            & \plus9               & 6 (125,000 gp) \\
            19th       & 80      & 40      & 12            & \plus9               & 7 (625,000 gp) \\
            20th       & 92      & 46      & 13            & \plus10              & 7 (625,000 gp) \\
            21st       & 104     & 52      & 13            & \plus10              & 7 (625,000 gp) \\
        \end{compresseddtabularx} 
    \end{dtable}

    \begin{dtable}
        \lcaption{Elite Ship Statistics}
        \begin{compresseddtabularx}{\columnwidth}{l l l l l l >{\lcol}X}
            \tb{Level} & \tb{HP} & \tb{DR} & \tb{Defenses} & \tb{Accuracy} & \tb{Power} & \tb{Item Rank (Cost)} \tableheaderrule
            1st        & 24      & 12      & 5             & \tdash        & \plus2     & 1 (40 gp)      \\
            2nd        & 30      & 15      & 6             & \plus1        & \plus3     & 1 (40 gp)      \\
            3rd        & 36      & 18      & 6             & \plus1        & \plus3     & 1 (40 gp)      \\
            4th        & 42      & 21      & 7             & \plus2        & \plus4     & 2 (200 gp)     \\
            5th        & 48      & 24      & 7             & \plus2        & \plus4     & 2 (200 gp)     \\
            6th        & 54      & 27      & 8             & \plus3        & \plus5     & 2 (200 gp)     \\
            7th        & 60      & 30      & 8             & \plus3        & \plus5     & 3 (1,000 gp)   \\
            8th        & 69      & 34      & 9             & \plus4        & \plus6     & 3 (1,000 gp)   \\
            9th        & 78      & 39      & 9             & \plus4        & \plus6     & 3 (1,000 gp)   \\
            10th       & 87      & 43      & 10            & \plus5        & \plus7     & 4 (5,000 gp)   \\
            11th       & 96      & 48      & 10            & \plus5        & \plus7     & 4 (5,000 gp)   \\
            12th       & 105     & 52      & 11            & \plus6        & \plus8     & 4 (5,000 gp)   \\
            13th       & 120     & 60      & 11            & \plus6        & \plus8     & 5 (25,000 gp)  \\
            14th       & 138     & 69      & 12            & \plus7        & \plus9     & 5 (25,000 gp)  \\
            15th       & 156     & 78      & 12            & \plus7        & \plus9     & 5 (25,000 gp)  \\
            16th       & 174     & 87      & 13            & \plus8        & \plus10    & 6 (125,000 gp) \\
            17th       & 192     & 96      & 13            & \plus8        & \plus10    & 6 (125,000 gp) \\
            18th       & 210     & 105     & 14            & \plus9        & \plus11    & 6 (125,000 gp) \\
            19th       & 240     & 120     & 14            & \plus9        & \plus11    & 7 (625,000 gp) \\
            20th       & 276     & 138     & 15            & \plus10       & \plus12    & 7 (625,000 gp) \\
            21st       & 312     & 156     & 15            & \plus10       & \plus12    & 7 (625,000 gp) \\
        \end{compresseddtabularx} 
    \end{dtable}

    \subsection{Ship HP and DR}
        Ships have hit points and damage resistance, just like characters.
        A ship's base hit points and damage resistance are based on its level, as listed in \tref{Ship Statistics}.
        In addition, a ship gains a bonus based on its Constitution, as listed in \pcref{Ship Attributes}.
        Armor can also give ships a bonus to their damage resistance, as listed in \pcref{Ship Armor}.

    \subsection{Elite Ships}\label{Elite Ships}
        Some ships are designated ``elite'' ships.
        Elite ships are approximately four times as strong as an ordinary ships.
        They have a number of benefits and modifiers which make them superior to ordinary ships:
        % NOTE: Keep in sync with elite monsters
        \begin{itemize}
            \item Elite ships gain a \plus2 bonus to their \glossterm{power} and all \glossterm{defenses}.
            \item Elite ships have three times the \glossterm{hit points} and \glossterm{damage resistance} of standard ships.
            % \item Elite ships have 50\% more \glossterm{damage resistance} than standard ships. This is applied after the bonus they get from having higher hit points, since ship damage resistance is based on their hit points.
            % \item Elite ships can have a maximum starting attribute of 6 (see \pcref{Ship Attributes}).
            % elite actions should be about as strong as regular actions, so 2x more damage
            \item Elite ships can take an additional \glossterm{elite action} each round (see \pcref{Ship Actions}).
        \end{itemize}

\section{Ship Resources and Strain}
    Ships do not have normal resources like characters do, such as fatigue and insight points.
    However, ships do have a strain level.
    This functions similarly to a character's fatigue level.
    Some crew roles have special abilities that can increase a ship's strain level in exchange for beneficial effects (see \pcref{Crew Roles}).

    A ship's strain tolerance is equal to its Strength \add its Willpower.
    Ships take a penalty to their accuracy and defenses equal to their strain level \sub their strain tolerance.

\section{Taking Ship Damage}
    Ships suffer damage and vital wounds much like characters do.

    \subsection{Ship Vital Wounds}
        Ships gain vital wounds just like characters (see \pcref{Vital Wounds}).
        However, ships have different vital wound effects.

        \begin{dtable}
            \lcaption{Ship Vital Wound Effects}
            \begin{dtabularx}{\textwidth}{l X}
                \tb{Vital Roll} & \tb{Effect} \tableheaderrule
                0 or less  & The ship gains a leak (see \pcref{Taking on Water})             \\
                1          & The ship takes a \minus2 penalty to future vital rolls                    \\
                2          & The ship's weapons take a \minus1 penalty to \glossterm{accuracy}         \\
                3          & The ship has a \minus10 foot penalty to its speed with all movement modes \\
                4          & The ship's turning cost increases by 10 feet                              \\
                5          & The ship's maximum \glossterm{damage resistance} is halved                \\
                6          & The ship takes a \minus1 penalty to all \glossterm{defenses}              \\
                7          & The ship takes a \minus2 penalty to its Fortitude defense                \\
                8          & The ship takes a \minus2 penalty to its Reflex defense                   \\
                9          & The ship takes a \minus2 penalty to its Mental defense                   \\
                10 or more & No extra vital wound effect                                               \\
            \end{dtabularx}
        \end{dtable}

    \subsection{Taking On Water}
        Damaged ships can begin taking on water through leaks.
        This will eventually sink the ship without intervention by its crew.
        However, even a hole in the hull is not necessarily fatal to a ship.
        With constant effort to remove excess water, a crew can often keep a ship afloat long enough to repair it or reach dry land.

        \subsubsection{Time to Sink}
            A Medium ship with a leak becomes unusable after one minute.
            After that point, the crew cannot perform any ship tasks, and the ship is immobile in the water except for natural drifting.
            It generally takes another minute for the ship to fully sink.
            For each size category larger than Medium, the time required for the ship to become unusable and sink increases, as described below:
            \begin{itemize}
                \item Medium: Ten minutes
                \item Large: Thirty minutes
                \item Huge: One hour
                \item Gargantuan: Two hours
                \item Colossal: Four hours
                \item Galleon: Eight hours
                \item Titan: One day
            \end{itemize}

            Each additional leak multiplies the rate that the ship sinks.
            For example, a ship with three leaks would sink three times faster.

        \subsubsection{Bailing the Ship}
            A ship's crew can remove water from the ship to keep it from sinking.
            In general, it takes one quarter of the ship's minimum crew, working constantly, to counteract incoming water from one leak.
            This simply maintains the amount of water currently in the ship.
            With twice that many crew dedicated to the task of bailing, existing water in the ship can be removed at the same rate that a leak would add water in, allowing the crew to catch up on existing leaks.

\section{Repairing Ships}
    Unlike characters, ships do not automatically heal over time.
    Significant ship damage can be both time-consuming and expensive to repair.

    Ships are assumed to have one repair crew that can only perform one repair task a time.
    For large ships, that repair crew may have many members, but they can still only perform one repair task at a time.
    A skilled fixer can reduce repair times (see \pcref{Crew Roles}).
    At the GM's discretion, a relevant Craft skill check by a crew member can also reduce repair times.

    \subsection{Short Repair}
        The repair crew of a ship can execute a short repair with one hour of work and no significant material cost.
        This fully restores the ship's damage resistance and sets its strain level to zero.

    \subsection{Long Repair}
        The repair crew of a ship can execute a long repair with eight hours of work.
        This requires a relevant Craft check with a difficulty value equal to 5 \add the ship's item rank.
        If the ship is docked for repair, the repair crew automatically rolls a 10 on this check.
        A successful long repair fully restores the ship's hit points and damage resistance, and sets its strain level to zero.

        A long repair costs materials worth one consumable item with a rank that is two ranks lower than the ship's item rank.
        If you are using gold pieces, this roughly translates to a repair cost equal to one hundredth of the ship's total value.
        These repair materials can be prepurchased so they are available on the ship when it needs to be repaired.
        Ships that do not have these repair materials on hand must find a dock.

    \subsection{Vital Repair}
        The repair crew of a ship can execute a vital repair with 24 hours of work.
        This requires a relevant Craft check with a difficulty value equal to 10 \add the ship's item rank.
        If the ship is docked for repair, the repair crew automatically rolls a 10 on this check.
        A successful vital repair removes one vital wound.

        A vital repair costs materials worth one consumable item with a rank that is one rank lower than the ship's item rank.
        If you are using gold pieces, this roughly translates to a repair cost equal to one twentieth of the ship's total value.
        These repair materials can be prepurchased so they are available on the ship when it needs to be repaired.
        Ships that do not have these repair materials on hand must find a dock.

\section{Ship Attributes}
    Unlike characters, ships normally have no Intelligence attribute.
    However, they have Strength, Dexterity, Constitution, Perception, and Willpower attributes.
    These represent slightly different narrative concepts than they do for characters.

    In general, a ship's attributes represent a combination of its physical properties and the effectiveness of its crew.
    A highly advanced ship may still have low attributes when its crew is inexperienced or incompetent.

    Extremely rare magical ships may have an animating mind that can control the ship.
    For details, see \pcref{Intelligent Ships}.

    \subsection{Attribute Descriptions}

    \subsubsection{Strength}
        Strength measures the power of a ship's physical weapons.
        Ships with a high Strength have more damaging weapons, and a crew capable of keeping those weapons working effectively.
        Ships with a low Strength have ineffective weaponry, or a crew that services those weapons poorly.
        Strength has the following effects on ships:
        \begin{raggeditemize}
            \item Ships add their Strength to their \glossterm{mundane power}.
            \item Ships add their Strength to their strain tolerance.
        \end{raggeditemize}

        Unlike characters, a ship's Strength does not affect its carrying capacity.
        That is calculated entirely from its physical size and shape.

    \subsubsection{Dexterity}
        Dexterity measures a ship's agility.
        Ships with a high Dexterity can turn more sharply to avoid incoming fire and may be faster in short bursts.
        Ships with a low Dexterity are lumbering and slow to change direction, making them easy targets.
        Dexterity has the following effects on ships:
        \begin{raggeditemize}
            \item Ships add their Dexterity to their Armor defense.
                This bonus can be reduced if the ship has medium or heavy armor (see \pcref{Ship Armor}).
            \item Ships add their Dexterity to their Reflex defense.
        \end{raggeditemize}

        As with characters, a ship's Dexterity does not affect its overall speed, simply its combat maneuverability.

    \subsubsection{Constitution}
        Constitution measures a ship's durability.
        Ships with a high Constitution are heavily reinforced and well crafted from sturdy materials.
        Ships with a low Constitution fall apart more easily, either because their construction is poor or because they were made from weak materials.
        Constitution has the following effects on ships:
        \begin{raggeditemize}
            \item Ships add twice their Constitution to their hit points.
                At level 7, this bonus increases to three times the ship's Constitution.
                At level 13, this bonus increases to six times the ship's Constitution.
                At level 19, this bonus increases to twelve times the ship's Constitution.
            \item Ships add their Constitution to their Fortitude defense.
        \end{raggeditemize}

    \subsubsection{Perception}
        Perception measures a ship's awareness and precision.
        Ships with a high Perception have effective lookouts, excellent gunners, and weapons which are capable of swiftly repositioning for precise attacks.
        Ships with a low Perception are either unable to effectively observe their surroundings or unable to react effectively to those observations.
        Perception has the following effects on ships:
        \begin{raggeditemize}
            \item Ships add their Perception to their level to determine their \glossterm{accuracy} with almost all attacks (see \pcref{Accuracy}).
        \end{raggeditemize}

    \subsubsection{Willpower}
        Willpower measures the morale and emotional steadiness of a ship's crew.
        Ships with a high Willpower are better able to resist setbacks and frightening encounters.
        Ships with a low Willpower may panic and be driven off easily.
        Willpower has the following effects on ships:
        \begin{raggeditemize}
            \item Ships add their Willpower to their Mental defense.
            \item Ships add their Willpower to their strain tolerance.
        \end{raggeditemize}

        Willpower is the attribute which is most affected by a ship's current crew.

    \subsection{Determining Ship Attributes}
        As with characters, ships can use a predefined attribute array or use point buy to calculate attributes.
        Ships start with 10 attribute points instead of 14 (see \pcref{Attribute Point Buy}).
        However, a ship's attributes can be increased by crew roles (see \pcref{Crew Roles}).

        \subsubsection{Predefined Attribute Scores}
            If you don't want to individually allocate a ship's attribute scores, simply one of the following sets of attribute scores and distribute them as you choose among the ship's attributes, ignoring Intelligence:
            \begin{raggeditemize}
                % 5 + 3 + 1 + 1
                \item Standard: 3, 2, 1, 1, 0
                % 8 + 1 + 1
                \item Specialized: 4, 1, 1, 0, 0
                % 3 + 3 + 3 + 1 + 0
                \item Balanced: 2, 2, 2, 1, 0
            \end{raggeditemize}

    \subsection{Intelligent Ships}
        Intelligence measures a ship's capability for thought and internal control.
        Normal ships have no Intelligence, and a ship's crew does not affect its Intelligence.

        Intelligent ships have a number of skill points equal to 3 \add their Intelligence.
        These trained skills apply to any ship-related checks that the ship might make.
        Ships cannot make skill checks for skills that they are not trained in.
        For example, a ship that was trained in Awareness could be its own lookout, but a ship without Awareness trained would not be able to independenlty perceive its external surroundings.

        In addition, each point of intelligence above \minus5 contributes a number of effective crew members to the ship's operation equal to a quarter of the ship's minimum crew requirements.
        For example, a ship with an Intelligence of 0 would require no crew members to meet its minimum for navigation.
        A ship with an Intelligence of 4 would have a total automatic crew count equal to twice its minimum crew.

\section{Ship Size}\label{Ship Size}

        \begin{dtable*}
            \lcaption{Ship Size}
            \begin{compresseddtabularx}{\textwidth}{l l l l l l >{\lcol}X >{\lcol}X l l l}
                \tb{Size}  & \tb{Min Level} & \tb{Elite?} & \tb{Crew}\fn{1} & \tb{Armor}      & \tb{Weapons} & \tb{Space} & \tb{Speed} & \tb{Turning Cost} & \tb{Cargo}    & \tb{Item Rank} \tableheaderrule
                Medium     & 1              & No          & 1               & Light           & \tdash       & 5 ft.      & 30 ft.     & 10 ft.            & Small x2      & \tdash       \\
                Large      & 1              & No          & 1---2           & Light           & 1            & 10 ft.     & 30 ft.     & 15 ft.            & Medium x2     & \tdash       \\
                Huge       & 4              & Either      & 1---5           & Light or medium & 2            & 20 ft.     & 40 ft.     & 20 ft.            & Large x2      & \plus1\fn{2} \\
                Gargantuan & 7              & Either      & 2---20          & Any             & 3            & 40 ft.     & 50 ft.     & 30 ft.            & Huge x2       & \plus1\fn{2} \\
                Colossal   & 10             & Yes         & 10---100        & Any             & 4            & 80 ft.     & 60 ft.     & 40 ft.            & Gargantuan x2 & \plus3       \\
                Galleon    & 13             & Yes         & 50---500        & Any             & 6            & 160 ft.    & 80 ft.     & 60 ft.            & Colossal x2   & \plus3       \\
                Titan      & 16             & Yes         & 100---1000      & Any             & 8            & 320 ft.    & 100 ft.    & 80 ft.            & Galleon x2    & \plus4       \\
            \end{compresseddtabularx}
            1. This range indicates the number of crew members that meaningfully contribute to the ship's functions, not the ship's maximum carrying capacity including passengers and cargo.
            It is either difficult or impossible to adequately control a large ship with less than the minimum crew listed here.
            Individual ships may have higher minimum crew requirements or lower maximum allowable crew based on their structure, at the GM's discretion. \\
            2. If the ship is Elite, increase its item rank by an additional \plus1.
        \end{dtable*}

        A ship's size does not directly affect its statistics.
        However, it has many effects on the ship's functionality.
        Larger ships are much more capable than smaller ships.
        Some of these effects are listed below in \trefnp{Ship Size}.
        In addition, advanced ship weapons often require a minimum ship size (see \pcref{Ship Weapons}).

        Some ships can be larger than most creatures and objects are usually defined.
        To track ship size beyond the limits of Colossal, additional Galleon and Titan categories are listed below.
        As usual, each size category represents a doubling of each dimension, and an eightfold increase in weight.
        Titan ships are unlikely to be present at all in many universes, and they require extensive magical reinforcement to function.
        The GM can decide whether their world is advanced enough to construct such monstrosities.

        Some examples of ships of a given size are given below.
        Since ships are typically named for their function and structure, not their size, this is only a rough guide.
        \begin{raggeditemize}
            \item Medium: Single-person kayak
            \item Large: Canoe, lifeboat
            \item Huge: Dinghy, outrigger canoe, punt, skiff
            \item Gargantuan: Felucca, small longship
            \item Colossal: Keelboat, large longship
        \end{raggeditemize}

\section{Ship Movement}
    This section defines the rules that ships use to move, which are not identical to character movement.
    Real ships have a great deal of momentum, and their movement speed and direction cannot be quickly adjusted.
    This is too much of a hassle to represent fully, so Rise uses significantly simplified ship movement mechanics.
    However, ships still have more movement constraints than characters, including a concept of ship heading.

    \subsection{Ship Heading}
        A ship's heading always points in one of the eight standard cardinal directions: north, northeast, east, and so on.
        Forward-moving ships can only move within a 90 degree cone centered on their heading.
        For example, a ship with a heading of north could travel northwest or northeast, but not west or east.
        
        \subsubsection{Turning}
            A ship can change its heading by turning.
            Each ship has a turning cost based on its size.
            That cost is the number of feet that a ship must spend out of its movement to turn by 45 degrees.
            A ship can pay its turning cost twice, allowing it to rotate more quickly while typically making little or no forward progress.

        \subsubsection{Reversing}
            A ship can travel in reverse, allowing it to move within a 90 degree cone centered around the opposite direction of its heading.
            This has two restrictions.
            First, the ship must have not used more than half its movement during the previous round to travel forward.
            Second, the ship's speed is halved while travelling in reverse.

    \subsection{Movement Timing}
        Ships automatically move up to their movement speed during the \glossterm{movement phase}.
        They cannot move during the \glossterm{action phase}.

\section{Ship Armor}
    Like characters, ships can have varying degrees of armor.
    Typically, even a heavily armored ship will not be literally covered in metal sheets.
    Instead, ship armor represents a heavily reinforced hull and extra layers of bracing and redundant infrastructure.

    There are three types of ship armor.
    \begin{raggeditemize}
        \item Light armor: Lightly armored ships are the default. They gain no special benefits or penalties.
        \item Medium armor: Ships with medium armor gain a \plus2 bonus to Armor defense.
            In addition, they have 50\% more damage resistance than a normal ship.
            However, they add only half their Dexterity bonus to their Armor defense.
            In addition, their movement speed is calculated as if they were one size category smaller (see \pcref{Ship Size}).
            Only Huge and larger ships can have medium armor.
        \item Heavy armor: Ships with heavy armor gain a \plus3 bonus to Armor defense.
            In addition, they have twice the damage resistance of a normal ship.
            However, they do not add their Dexterity bonus to their armor defense.
            In addition, their movement speed is calculated as if they were two size categories smaller (see \pcref{Ship Size}).
            Only Gargantuan and larger ships can have heavy armor.
    \end{raggeditemize}

%         \subsubsection{Customized Ship Armor}
%             Individual ships can have different configurations that change their defenses.
%             With customization, a ship can gain a \plus1 bonus to Armor defense, a \plus2 bonus to any other defense, or become \impervious to one of the following damage types: acid, cold, electricity, or fire.
%             In exchange, it must take a \minus1 penalty to Armor defense or a \minus2 penalty to any other defense.
%             Some rare and valuable ships may have unusual properties that grant additional benefits, at the GM's discretion.

\section{Ship Weapons}\label{Ship Weapons}
    Ships depend on having powerful weapons even more than martial characters do.
    They use weapon upgrades as their primary method of scaling damage rather than maneuvers or other special attacks.
    The number of weapons a ship can have is limited based on its size, as seen in \tref{Ship Size}.

    There are two ways that a GM can choose to use ship weapons.
    Real siege weapons used on ships had high crew requirements, slow firing rates, and extreme range.
    In practice, this can reduce ship combat to a slog of tracking reload times across multiple weapons and carefully maneuvering ship range to make the best use of varying weapon types.
    For GMs who want more realistic and unique ship combat, use the weapons listed in \tref{Realistic Ship Weapons}.
    For GMs who want ship combat to feel simpler and more similar to regular combat, use the weapons listed in \tref{Simplified Ship Weapons}.

    The simplified ship weapons listed in the table normally attack Armor defense.
    Magical simplified ship weapons use the same statistics as regular ship weapons, except that that they have the \weapontag{Fixed} (200) and \weapontag{Mystic} tags instead of \weapontag{Projectile} (200/600).
    In addition, they can attack Fortitude, Reflex, or Mental defense instead of Armor defense.

    % Baseline weapon is dr = rank - 1 with Projectile (200/600).
    \begin{dtable!*}
        \lcaption{Simplified Ship Weapons}
        \begin{dtabularx}{\textwidth}{l X l l l l}
            \tb{Name}  & \tb{Damage}                           & \tb{Targeting} & \tb{Tags}            & \tb{Ship Size} & \tb{Item Rank (Cost)} \tableheaderrule
            Rock sling & 1d6 bludgeoning \add 1 per 2 power    & One target     & Projectile (200/600)  & Medium         & 2 (200 gp)     \\
            Scorpion   & 1d8 piercing \add 1 per 2 power       & One target     & Projectile (200/600) & Large           & 3 (1,000 gp)   \\
            Ballista   & 1d8 piercing \add 1 per power         & One target     & Projectile (200/600) & Huge           & 4 (5,000 gp)   \\
            Catapult   & 1d10 bludgeoning plus 1d6 per 3 power & One target     & Projectile (200/600) & Gargantuan     & 5 (25,000 gp)  \\
            Mangonel   & 1d8 bludgeoning plus 1d8 per 3 power  & One target     & Projectile (200/600) & Gargantuan     & 6 (125,000 gp) \\
            Trebuchet  & 2d8 bludgeoning plus 1d8 per 3 power  & One target     & Projectile (200/600) & Colossal       & 7 (625,000 gp) \\
        \end{dtabularx}
    \end{dtable!*}

    % Baseline weapon is dr = rank with Projectile (90/270) and Slow Reload (1).
    % Removing Slow Reload (1) is -2dr.
    % Dropping to Projectile (60/180) gives +1dr.
    % Increasing to Projectile (120/360) gives -1dr. Progression is 90 -> 120 -> 200 -> 300 -> 400.
    \begin{dtable!*}
        \lcaption{Realistic Ship Weapons}
        \begin{compresseddtabularx}{\textwidth}{l l l >{\lcol}X l l l}
            \tb{Name}                & \tb{Damage}                          & \tb{Defense} & \tb{Tags}                                   & \tb{Crew} & \tb{Ship Size} & \tb{Item Rank (Cost)} \tableheaderrule
            % +1dr for -1 range
            Rock sling               & 1d8 bludgeoning \add 1 per power     & Armor        & Projectile (60/180), Slow Reload (1)        & 1         & Medium         & 2 (200 gp)     \\
            % -1dr for +1 range
            Springald                & 1d6 piercing \add 1 per 2 power      & Armor        & Projectile (120/360), Slow Reload (1)       & 1         & Large          & 2 (200 gp)     \\
            % +1dr for reload -> load, -1dr for range
            Lightning caller\sparkle & 3d6 electricity                      & Reflex       & Fixed (200), Mystic, Slow Load (1) & 3         & Large          & 3 (1,000 gp)   \\
            % +2dr for slow reload, -2dr for range
            Scorpion                 & 1d8 piercing \add 1 per power        & Armor        & Projectile (200/600), Slow Reload (2)       & 1         & Huge           & 3 (1,000 gp)   \\
            % +3dr for slow reload, -3dr for range
            Onager                   & 1d8 bludgeoning \add 1 per power     & Armor        & Projectile (300/900), Slow Reload (3)       & 8         & Gargantuan     & 3 (1,000 gp)   \\
            % +1dr for reload -> load, +2dr for (1) to (2), -3dr for range
            Torsion ballista         & 1d8 piercing per 3 power             & Armor        & Projectile (300/900), Slow Load (2)         & 3         & Huge           & 4 (5,000 gp)   \\
            % +1dr for reload -> load, -1dr for range
            Flame caller\sparkle     & 4d6 fire                             & Reflex       & Fixed (200), Mystic, Slow Load (1) & 3         & Large          & 4 (5,000 gp)   \\
            % +1dr for reload -> load, -1dr for range
            Acid caller\sparkle      & 5d10 acid                            & Fortitude    & Fixed (200), Mystic, Slow Load (1) & 3         & Large          & 5 (25,000 gp)  \\
            % -2dr for removing reload
            Polybolos                & 1d8 piercing \add 1 per power        & Armor        & Projectile (90/270)                         & 3         & Huge           & 5 (25,000 gp)  \\
            % +2dr for slow reload, -2dr for range
            Mangonel                 & 1d8 bludgeoning plus 1d8 per 3 power & Armor        & Projectile (300/900), Slow Reload (2)       & 20        & Gargantuan     & 5 (25,000 gp)  \\
            % +1dr for reload -> load, -1dr for range
            Storm caller\sparkle     & 5d10 electricity                     & Reflex       & Fixed (200), Mystic, Slow Load (1) & 3         & Large          & 6 (125,000 gp) \\
            % +3dr for slow reload, -4dr for range
            Trebuchet                & 1d6 bludgeoning plus 1d6 per 2 power & Armor        & Projectile (400/1200), Slow Reload (3)      & 10        & Gargantuan     & 6 (125,000 gp) \\
            % +1dr for reload -> load, -1dr for range
            Meteor caller\sparkle    & 7d10 bludgeoning and fire            & Reflex       & Fixed (200), Mystic, Slow Load (1) & 3         & Large          & 7 (625,000 gp) \\
        \end{compresseddtabularx}
    \end{dtable!*}

    \subsection{Ship Weapon Tags}\label{Ship Weapon Tags}
        Some weapon tags only apply to ship weapons.

        \weapontagdef{Fixed} This weapon has a single fixed range limit.
        That number is given in parentheses, such as Fixed (200), and represents a number of feet of range.
        The weapon never suffers a \glossterm{longshot penalty}, but it cannot be used at all outside of its listed range.

        \weapontagdef{Mystic} This weapon can only be crewed by creatures capable of casting spells.
        The minimum spell rank of each creature must be no more than two ranks lower than this weapon's item rank.

        % +1dr for Slow Reload to Slow Load, +1dr for (1) to (2), then +1dr for each subsequent round
        \weapontagdef{Slow Load} This weapon requires multiple rounds to load before it can fire for the first time.
        The number of rounds required to finish loading before it can be fired is indicated in its description.
        For example, a weapon with Slow Load (1) could be fired every other round.
        Only rounds where the weapon is being fully crewed count towards this load time.
        If the weapon is not fired during the round after its loading time is completed, the loading is wasted, and the weapon must be loaded again before it can fire.

        % +2dr for (1) to (2), then +1dr for each subsequent round
        \weapontagdef{Slow Reload} This weapon requires multiple rounds to after being fired.
        The number of rounds required to finish reloading is indicated in its description.
        For example, a weapon with Slow Reload (1) could be fired every other round.
        Only rounds where the weapon is being fully crewed count towards this reload time.
        The weapon can sustain its load for an arbitrary amount of time in combat before being fired.

\section{Ships Fighting Non-Ships}
    In some cases, ships might enter combat against non-ship foes.
    Ships take one tenth of the normal damage from non-ship attacks, rounded down as usual, so minor attacks will not even scratch them.
    Since ships are not creatures and are not alive, many special attacks have no effect on them.
    They also cannot gain \glossterm{conditions} from non-ship attacks by any means.
    When used against non-ship targets, weapons on Huge or larger ships have the \weapontag{Massive} tag, just like attacks from Huge or larger creatures (see \pcref{Very Large Creatures}).

\section{Ships and Gunpowder}
    In general, Rise avoids the use of guns and gunpowder.
    Those inventions do not fit into the traditional fantasy setting that Rise is built on, which emphasizes swords and bows and magic.
    For similar reasons, the ranged weapons of ships can be defined entirely with non-gunpowder weaponry common in the ancient world.
    These would include ballistas, catapults, scorpions, and similar siege weaponry.

    However, narrative tropes for ships and naval campaigns often have a more technology-heavy basis, with cannons and full broadsides.
    These stories tend to draw inspiration from the Age of Sail and Golden Age of Piracy rather than medieval folklore.
    It can feel intuitively plausible to have cannons used as ship weaponry even when guns are never used by individuals.

    The default names for ship weapons assume that gunpowder is not being used.
    If technology has advanced to the point that gunpowder weapons are possible, you can assume that hull reinforcement has also increased at the same rate.
    All the GM has to do for ship vs ship combat is change the name of the weapons to match their preferred technology level.
    More advanced weaponry is more generally effective against non-ship targets, however.
    If you use gunpowder-based ship weapons, they deal double damage against non-ship targets.

\section{Crew Roles}
    Ships depend on their crew to function.
    There are many jobs that are necessary to make large ships function, including sailors, rowers, cooks, pages, and more.
    Fully defining life on a ship is outside the scope of this brief introduction to ship combat.
    However, some roles have outsized influence on the ship's effectiveness, such as the ship's captain and pilot.
    These roles provide a way for player characters to meaningfully influence the outcome of ship battles, even if their personal combat talents are irrelevant at those scales.

    Each crew role defined here functions like a class archetype, with seven progression ranks.
    A character can have any number of crew roles, but they can only fulfill one role at a time on a ship.
    In general, your highest crew role rank should not be significantly higher or lower than your highest rank in a class archetype, but they do not have to be kept exactly in sync.
    The GM can decide whether your crew role rank increases in lockstep with your class archetype rank, or whether it increases based on other factors.
    This could include practice and time spent on a ship, or money spent to buy ship improvements necessary to improve your crew role rank.

    Crew role ranks are limited by ship quality.
    Even the best pilot cannot dodge incoming fire with ease on a lifeboat.
    The maximum crew role rank for all crew on a ship is equal to the ship's item rank (see \pcref{Wealth and Item Ranks}).

    A sufficiently large ship can have more than one member of each crew role.
    This does not mean that the ship literally has multiple captains or pilots.
    Instead, the ship would have a hierarchy.
    It might have a boatswain and boatswain's mate, or a pilot and pilot's mate, with both characters able to fully use their crew role abilities.
    Colossal ships can have two people for each crew role, Galleon ships can have three, and Titan ships can have four.

    % TODO: do "ship-related" effects need to be defined in more detail?

    \subsection{Boatswain}
        This role is responsible for coordinating the ship's crew.
        It involves tracking everything that is happening on the ship and knowing where and how to intervene.

        \cf{Bsn}[1]{Ensure Competence} Whenever a crew member on the ship makes a ship-related check that you are aware of, you can increase your fatigue level by two.
        If you do, that creature gains the benefit of the \ability{desperate exertion} ability on that check (see \pcref{Desperate Exertion}).
        It still cannot apply that ability twice to the same check.

        \cf{Bsn}[2]{Specialized Encouragement} While you are in an appropriate location such that relevant crew can see or hear you, you can use a standard action to grant one of the following benefits:
        \begin{raggeditemize}
            \item Evasion: The ship \glossterm{briefly} gains a \plus1 bonus to all defenses.
            \item Travel: The ship \glossterm{briefly} gains a \plus10 foot bonus to its movement speed.
            \item Weapons: Attacks with the ship's weapons \glossterm{briefly} gain a \plus1 accuracy bonus.
            \item Desperate Rally: The ship \glossterm{briefly} gains the Evasion, Travel, and Weapons benefits from this ability.
                However, it increases its strain level by one.
        \end{raggeditemize}

        Changing which your location to be able to provide a different benefit with this ability typically takes one full round of movement for each size category by which the ship is larger than Huge.
        Specific ships may be easier or more difficult to navigate, at the GM's discretion.
        You cannot provide more than one benefit with this ability at once, even if you are perfectly located on the ship.

        \cf{Bsn}[3]{Collective Effort} Each other crew member who can see or hear you gains a \plus1 bonus to ship-related checks.
        If you are trained in a skill that a crew member is using to make a check, they gain a \plus2 bonus from this ability instead of \plus1.

        \cf{Bsn}[4]{Encouraging Presence} The ship gains a \plus1 bonus to its Willpower.

        \cf{Bsn}[5]{Ensure Competence+} When you use this ability to affect a creature other than yourself, you only increase your fatigue level by one.

        \cf{Bsn}[6]{Specialized Encouragement+} The bonuses you provide from this ability increase:
        \begin{raggeditemize}
            \item Evasion: The defense bonus increases to \plus2.
            \item Travel: The ship's turning cost is reduced by 10 feet, to a minimum of 10 feet.
            \item Weapons: The accuracy bonus increases to \plus2.
        \end{raggeditemize}

        \cf{Bsn}[7]{Collective Effort+} The bonus increases to \plus2, or to \plus4 for skills that you are trained in.

    \subsection{Fixer}
        This role is responsible for maintaining the ship's physical infrastructure.
        It represents a combination of carpentry, caulking, and similar repair jobs.
        In order to take this role on a ship, you must have at least one Craft skill relevant to the ship's construction.

        \cf{Fix}[1]{Quick Patch} As a standard action, you can attempt to patch the ship's defenses.
        This requires a Craft check relevant to the ship.
        The difficulty value is equal to 5 \add 5 for each time that you have used that same Craft skill with this ability since the ship's last short repair.
        Success means that the ship regains damage resistance equal to three times your rank in this crew role.

        \cf{Fix}[2]{Temporary Seal} You can make a Craft check relevant to the ship to temporarily stop a leak in a ship without expensive materials.
        The difficulty value is equal to 10 \add 10 for each time that you have used that same Craft skill with this ability since the ship's last vital repair.
        Success means that the leak is stopped for one hour.
        This does not remove the vital wound, and you can only use this ability to affect one leak on the ship.
        Normally, this check requires one minute of work.
        For every 5 points by which you succeed, the time required is halved.

        \cf{Fix}[3]{Rapid Repair} When you participate in repairing a ship, the repairs take half the normal time.
        This affects short repairs, long repairs, and vital repairs.

        \cf{Fix}[4]{Sturdy Reinforcement} The ship gains a \plus1 bonus to its Constitution.

        \cf{Fix}[5]{Quick Patch+} The damage resistance regained increases to four times your rank in this crew role.

        \cf{Fix}[6]{Temporary Seal+} You can use this ability to affect two leaks on the ship, rather than only one.

        \cf{Fix}[7]{Efficient Repair} When you participate in repairing a ship, the cost of that repair is reduced by one item rank.
        If you are using gold pieces, this roughly translates to a repair cost that is one fifth of the normal price.

    \subsection{Gunner}
        This role is responsible for aiming and firing ship weaponry.
        For complex weapons such as ballistas, they also direct the work of other crew members who are physically loading and aiming the weapons.

        \cf{Gun}[1]{Skilled Shot} Ship weapons you fire gain a \plus1 accuracy bonus.

        \cf{Gun}[2]{Customized Shot} Whenever you fire a ship weapon, you can choose one of the following effects.
        \begin{itemize}
            \item Arcing Shot: The \glossterm{longshot penalty} for the attack is reduced by 2.
            \item Desperate Shot: The attack rolls accuracy twice, and it gains a \plus2 accuracy bonus.
                However, the ship's \glossterm{strain level} increases by one.
            \item Direct Shot: The attack deals double damage.
                However, the attack suffers a \minus1 accuracy penalty, and its range limits are halved.
            \item Overloaded: The attack deals double damage.
                This can only be used with ship weapons that do not have the \weapontag{Slow Load} weapon tag (see \pcref{Ship Weapon Tags}).
                If the weapon has the \weapontag{Slow Reload} weapon tag, it takes twice as long to reload after this attack.
                Otherwise, it gains \weapontag{Slow Reload} (1) until it has been reloaded.
        \end{itemize}

        \cf{Gun}[3]{Weapons Coordination} As a \glossterm{minor action}, you can choose one ship weapon whose crew can see or hear you.
        That weapon gains your benefits from this crew role this round as if you were the one firing it.
        You cannot apply this ability to a weapon that is already benefiting from another gunner's effects.
        This ability has the \abilitytag{Swift} tag.

        \cf{Gun}[4]{Mighty Weaponry} The ship gains a \plus1 bonus to its Strength.

        \cf{Gun}[5]{Skilled Shot+} The accuracy bonus increases to \plus2.

        \cf{Gun}[6]{Customized Shot+} The benefits from this ability improve.
        \begin{itemize}
            \item Arcing Shot: The longshot penalty is fully removed.
            \item Desperate Shot: You roll accuracy three times instead of twice.
            \item Direct Shot: The accuracy penalty is removed.
            \item Overloaded: The attack also rolls damage twice, keeping the higher result.
        \end{itemize}

        \cf{Gun}[7]{Weapons Coordination+} When you use this ability, you can choose any number of ship weapons whose crew can see or hear you.

    \subsection{Lookout}
        This role is responsible for observing and reporting the ship's surroundings.
        It includes watching out for hazardous terrain, enemy ships, and similar dangers or opportunities.

        \cf{Lok}[1]{Vantage Point} You gain a \plus3 bonus to Awareness and Survival checks while in a crow's nest, or other equivalent location on the ship designed for a lookout.

        \cf{Lok}[2]{Detailed Scouting} If you look at a ship for one minute, you can make an Awareness check.
        The difficulty value is normally equal to 5, modified as normal for distance and vision conditions.
        If the ship is specifically designed to conceal its true nature, the difficulty value can increase to 10 or higher, at the GM's discretion.
        Success means you learn whether one of the following things is true.
        For every 5 points by which you succeed, you learn an additional piece of information.
        \begin{itemize}
            \item If the ship's item rank is greater than your ship's item rank.
            \item If the highest item rank among the ship's weapons is greater than the highest item rank among your ship's weapons.
            \item If the ship is elite.
            \item If the ship is mostly full of cargo.
            \item If the ship has more than half its maximum crew.
            \item If the ship has any vital wounds.
        \end{itemize}

        \cf{Lok}[3]{Far-Sighted Weapons} All ship weapons that can see or hear you reduce their \glossterm{longshot penalty} by 1.

        \cf{Lok}[4]{Clear Sighted} The ship gains a \plus1 bonus to its Perception.

        \cf{Lok}[5]{Vantage Point+} The skill bonuses increase to \plus6.

        \cf{Lok}[6]{Far-Sighted Weapons+} The longshot penalty reduction increases to 2.

        \cf{Lok}[7]{Detailed Scouting+} Using this ability does not take any time.
        It happens automatically whenever you see a ship.
        In addition, you automatically learn all pieces of information if the check succeeds, regardless of how much you succeed by.

    \subsection{Pilot}
        This role is responsible for steering the ship.
        They generally stay at the ship's wheel, but they may also direct the cut of sails or the direction of rowing.

        \cf{Pil}[1]{Desperate Sprint} As a standard action, you can increase the ship's strain level by one.
        If you do, the ship moves up to its speed during this action phase.
        Any ship can only be affected by this ability once per round.

        \cf{Pil}[2]{Evasive Maneuvers} As a standard action, you can choose one enemy ship you are aware of.
        Your ship gains a \plus1 bonus to its Armor and Reflex defenses against that enemy ship this round.
        This ability has the \ability{Swift} tag.

        \cf{Pil}[3]{Hard Turn} The ship's turning speed is reduced by 10 feet, to a minimum of 10 feet.
        In addition, whenever the ship moves, you can increase the ship's strain level by one.
        If you do, the ship turns up to 90 degrees without spending any movement.

        \cf{Pil}[4]{Agile Pilot} The ship gains a \plus1 bonus to its Dexterity.

        \cf{Pil}[5]{Evasive Maneuvers+} You can choose a second enemy ship at the same time.

        \cf{Pil}[6]{Hard Turn+} The turning speed reduction increases to 20 feet.

        \cf{Pil}[7]{Desperate Sprint+} When you use this ability, the ship also doubles its movement speed during the next movement phase.

    \subsection{Warder}
        This role is responsible for directly protecting the ship from danger with magic.

        In order to take this role on a ship, you must be able to cast spells.
        At the GM's discretion, you may also need to have access to mystic spheres that can plausibly be used to protect the ship.
        Most mystic spheres which have spells that affect objects can be used to justify this role.
        However, highly specialized mystic spheres like \sphere{enchantment} or \sphere{vivimancy} may not be sufficient.

        \cf{Wrd}[1]{Active Defense} As a standard action, you can give the ship a \plus2 bonus to all defenses during the current round.
        This ability has the \pcref{Swift} tag.
        It has no effect if the ship is already protected by another warder using this ability.

        \cf{Wrd}[2]{Specialized Ward} As a standard action, you can activate one of the following effects.
        This ability has the \abilitytag{Sustain} (minor) tag.
        \begin{raggeditemize}
            \item Desperate Fortification: Whenever the ship would take damage, that damage is halved.
                At the end of each round, if the ship took damage in excess of your rank in this crew role that round, it increases its strain level by two.
            \item Enhance Armor: The ship gains a \plus1 \glossterm{enhancement bonus} to its Armor defense.
            \item Enhance Resilience: The ship gains a \plus2 \glossterm{enhancement bonus} to its Fortitude defense and \glossterm{vital rolls}.
            \item Mystic Ward: The ship gains a \plus2 \glossterm{enhancement bonus} to all defenses against \magical attacks.
        \end{raggeditemize}

        \cf{Wrd}[3]{Ship of Magic} You gain a \plus1 accuracy with \magical abilities, including magical ship weapons, while on the ship.
        In addition, whenever you perform a ritual on the ship, you can increase the ship's strain level by one.
        If you do, the fatigue cost required to perform the ritual is reduced by an amount equal to twice your rank in this crew role.

        \cf{Wrd}[4]{Permanent Ward} The ship gains a bonus to its damage resistance equal to four times your rank in this crew role.

        \cf{Wrd}[5]{Active Defense+} The defense bonus increases to \plus3.

        \cf{Wrd}[6]{Ship of Magic+} The accuracy bonus increases to \plus2.
        In addition, the fatigue cost reduction increases to three times your rank in this crew role.

        \cf{Wrd}[7]{Specialized Ward} The benefits from this ability improve.
        \begin{raggeditemize}
            \item Desperate Fortification: The threshold for gaining fatigue increases to three times your rank in this crew role.
            \item Enhance Armor: The bonus increases to \plus2.
            \item Enhance Resilience: The bonuses increase to \plus4.
            \item Mystic Ward: The bonus increases to \plus4.
        \end{raggeditemize}
