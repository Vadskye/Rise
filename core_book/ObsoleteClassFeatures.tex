\subsubsection{Domain Influences}\label{Domain Influences}
\subcf{Air -- Voice of the Wind}
As a standard action, the cleric can spend a devotion point to command the air around him.
He can command it to perform a task that lasts up to 5 minutes, and which does not take place farther than 300 feet from him.
He cannot compel the air to go faster than 10 mph, which is not fast enough to significantly affect creatures in combat.
The air can only understand simple instructions, such as ``Blow southeast for one minute'' or ``Blow out the torch at the end of the hall''.
\subcf{Chaos -- Sow Chaos}
As a standard action, the cleric can spend a devotion point to cause an improbable event to occur.
He can visualize in general terms what he wants to happen, such as ``Make the bartender leave the bar''.
However, he has no control over the exact nature of the event, or even whether the event is beneficial for him.
Instead, roll a d20 when this ability is used.
A roll of 11 or higher means the event is generally beneficial for the cleric.
A roll of 5 or lower means the event is general detrimental to the cleric.
A roll from 6 to 10 means the event is neither good nor bad for the cleric.
\subcf{Death -- Reaper's Boon}
As a swift action, the cleric can spend a devotion point to summon the influence of Death or drive it away.
In either case, he targets a living creature within \rngmed range of him.
If he summons Death, the target dies if it fails a single \glossterm{stabilization roll}, but its death is painless.
If he drives away Death, the target stabilizes if it succeeds at a single stabilization roll.


        \subcf{6th -- Hidden Hunter}
        The ranger becomes even more difficult for his quarry to detect.
        He adds his quarry bonus to his Stealth checks against his quarry.
        In addition, he continuously benefits from the effect of the \spell{nondetection} ritual against all attempts that his quarry makes to detect him magically.
        The effect uses a spellpower equal to his hunting power.

        \subcf{Ooze Hunter}
        The ranger becomes immune to slime and engulf attacks.


        \subcf{Aberrant Hunter}
        The ranger gains a \plus2 bonus to Mental defense.
        In addition, he increases his quarry bonus by \plus2 against aberrations.

        \subcf{Aberrant Hunter}
        The ranger gains a \plus2 bonus to Mental defense.
        In addition, he increases his quarry bonus by \plus2 against aberrations.

        \subcf{Undead Destroyer}
        The ranger becomes immune to hostile \glossterm{Life} effects.
        In addition, he increases his quarry bonus by \plus2 against undead.

        \subcf{8th -- Dragonslayer}
        The ranger becomes immune to breath weapons.
        In addition, he increases his quarry bonus by \plus2 against dragons.

        \subcf{8th -- Golem Breaker}
        The ranger's attacks ignore an amount of hardness and damage reduction equal to half his hunting power.
        In addition, he increases his quarry bonus by \plus2 against constructs.

        \subcf{8th -- Giantslayer}
        The ranger gains a \plus10 bonus to Fortitude defense against attacks that would move him.  % weird bonus
        In addition, he increases his quarry bonus by \plus2 against creatures two or more size categories larger than him.

        \subcf{12th -- Divine Avenger}
        The ranger gains \glossterm{spell resistance} against divine spells equal to 10 \add his hunting power.
        In addition, he increases his quarry bonus by \plus2 against divine spellcasters.

        \subcf{12th -- Fey Stalker}
        The ranger becomes immune to hostile \glossterm{Mind} effects.
        In addition, he increases his quarry bonus by \plus2 against fey.

        \subcf{12th -- Mageslayer}
        The ranger gains \glossterm{spell resistance} against arcane spells equal to 10 \add his hunting power.
        In addition, he increases his quarry bonus by \plus2 against arcane spellcasters.

        \cf{Wiz}[5th]{Ritual Master}[Ex]
        When performing an arcane \glossterm{ritual}, the wizard may spend an arcane spell slot of the ritual's level or higher.
        If she does, she need only pay half the normal material component costs to perform the ritual.

        \cf{Ftr}[4th]{Opportunistic Strike} The fighter gains the ability to take advantage of opportunities his foes present him with.
        He must choose to focus on melee reactions or ranged reactions.
        If he improves his melee reactions, whenever a creature moves out of his \glossterm{threatened area}, he can make a \glossterm{strike} against that creature as an \glossterm{immediate action.}
        If the results of that strike would prevent the creature from moving, its movement is cancelled and the action is wasted.

        If he improves his ranged reactions, he can fire a quick shot at any target within one \glossterm{range increment} of him as a \glossterm{swift action}.
        However, this attack automatically misses unless the target is \defenseless, \helpless, or \unaware.

        \cf{Ftr}[8th]{Master of Opportunity}
        Once per round, the fighter can make a strike with his opportunistic strike ability without spending an action.
        He can combine this with an additional strike from spending an action, allowing him to quickly make multiple strikes against the same creature.

        \cf{Ftr}[12th]{Force Engagement}
        The fighter improves his ability to take advantage of opportunities with his opportunistic strike ability.
        If he chose to improve his melee reactions, he can also use the ability against creatures he threatens who do not attack him 

        \cf{Ftr}[16th]{Brutal Opportunist}
        The strikes a fighter makes with his opportunistic strike ability deal double damage.
