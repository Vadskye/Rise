
\section{Narrative Pacing}
    A typical game of Rise is broken up into two distinct concepts.
    Sometimes, you will interact with the world in broad, sweeping descriptions, as you describe in general terms what your character does.
    For example, you and your group may decide to travel overland for seven days to reach a distant city.
    Just like you don't need to mention every time your character breathes or takes a step, you don't need to track every second of time spent on that journey.
    Until something of dramatic significance happens, the GM should narrate the events of the journey in general terms to avoid spending time on irrelevant details.

    If something important happens, the pace of the game should change to focus on the current events.
    For example, if your group is ambushed by roaming bandits, the exact actions you take become important.
    This is called an encounter.
    During an encounter, the GM should ask each player to describe their character's actions more precisely.
    If the encounter is time-sensitive, such as combat, the GM should track the flow of time in \glossterm{rounds}.
    A round represents six seconds of time, so there are ten rounds per minute.

    Not all encounters require tracking time precisely.
    For example, if you and your party encounter a broken-down caravan carrying an angry prince, your exact actions can be important.
    Whether you show appropriate deference and fix his caravan, ignore him, or kill him to steal his jewelry can have important repurcussions in the game world.
    However, the exact time it takes to make that decision and execute on it is not usually important.

    Not every challenge needs to be handled with the narrative pacing of an encounter, especially if the outcome of the challenge is uninteresting or predetermined.
    Even combat can be described in broad terms, such as if you are significantly more powerful than your foes and your victory is assured.
    In all cases, the GM decides how to handle the flow of time and the narrative pacing.
    They may have information you don't about what is important and what is irrelevant.

\section{Principles of Rise}

    \begin{itemize}
        \item \textbf{The Game Master controls the world.} Everything about the game world is up to the GM\@: the people your character meets, the challenges they will overcome, and the very ground under their feet.
            Even the ``rules'' of the game are completely subject to the GM's whim.
        \item \textbf{You control your character.} A GM should never tell you how your character feels or what they try to do --- unless, of course, your character is being controlled by hostile magic or some other power.
        \item \textbf{Respect and trust are critical.} The GM has a great deal of power in Rise, and the players have to trust that the GM knows what they are doing.
            Likewise, the GM needs to let the players do what they want --- even if it doesn't suit their idea of what ``should'' happen.
            Some of the most memorable events happen when players do things that are totally unexpected.
        \item \textbf{Everything is flexible.} Rise contains hundreds of pages of rules.
            We think they're pretty good rules.
            But sometimes, you don't need them all --- or you think you've come up with something better.
            Do whatever works for you and your group.
        \item \textbf{Do what makes sense.} This book doesn't contain rules for how to drink water, sleep, or blink.
            If something isn't described explicitly here, assume that it works the same way it does in reality.
        \item \textbf{It's just a game, so have fun.}
    \end{itemize}

    \subsection{Rule of Drama}

        The Rule of Drama is simple: \textbf{Only dramatic actions matter}.
        In general, an action has dramatic significance if there is a consequence for failure, the exact time required to perform the task is important, or the game master says so.
        This has several effects.

        \subsubsection{Checks without Rolls}
            A check represents an attempt to accomplish a dramatically significant goal, usually while under some sort of time pressure or distraction.
            When there is no dramatic significance to the task, a check should not be rolled.

            \parhead{Taking 5}\label{Taking 5}
            Normally, if a check would be required but there is no dramatic significance to the check, assume the character rolled a 5 when determining whether the check is successful.
            This is called ``taking 5''.

            \parhead{Taking 10}\label{Taking 10}
            If a character would not succeed when taking 5, the character can try to ``take 10'' instead.
            Taking 10 requires spending ten times the amount of time normally required to accomplish the task.
            If a task takes a variable amount of time, assume it took the average amount of time required to make a check.
            In exchange, the character calculates their check result if they had rolled a 10.

            Essentially, taking 10 means the character repeatedly attempts the task until they succeed.
            It is possible to take 10 on a task that has consequences for failure, but taking 10 guarantees that those consequences occur.

        \subsubsection{Narrative Time}
            In most cases, the exact time of day and exactly how long an action takes is not very important.
            No one really cares whether the orc horde broke through the town gates at the crack of dawn, or at 6:55 AM\@.
            Likewise, travelling from a tavern in a city to the blacksmith probably takes a few minutes, depending on the size of the city, but it's not usually important.

            Rise contains rules which you can use to decide exactly how much time things take, but when it's not important, it's generally better to only worry about time in broad strokes.
            It makes everyone's life a bit easier --- especially for the GM\@.
