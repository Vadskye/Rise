\documentclass[twocolumn,oneside,letterpaper]{book} % consider draft option
% More standard font encoding, may help with some punctuation
\usepackage[T1]{fontenc}
% Generally recommended font
\usepackage{lmodern}
\usepackage{scrextend}
% Disable hyphenation to improve copy/paste
% \usepackage[none]{hyphenat}
\usepackage{xparse}
\usepackage{nth}
\usepackage{mathptmx}
\usepackage{enumitem}
\usepackage{comment}    %Provides comments
\usepackage{array}    %Provides \arraybackslash
\usepackage{tabularx}    %Provides almost all of the tables
\usepackage[dvipsnames,svgnames,usenames,table]{xcolor}    %Colors the tables
%Create a color for use with tables
\definecolor{tbrown}{RGB}{255,240,200}

%Set table captions to appear on the left side of the table
\usepackage[font=bf,justification=raggedright,singlelinecheck=off]{caption}
%Use more compact section headers
\usepackage[compact]{titlesec}
\usepackage{fancyhdr}
\usepackage{longtable}
\usepackage{tabu}
\usepackage{booktabs}
\usepackage{xspace}
\usepackage{changepage}
\usepackage{makecell}

% page config
\usepackage[letterpaper,margin=0.9in,left=0.5in,right=0.5in]{geometry}

%\usepackage{placeins} This made layout screwy
\usepackage{multicol}

% function math stuff
\usepackage{tikz}
% borders
\usepackage[framemethod=tikz,nobreak=true]{mdframed}

% variable width columns
%\usepackage{vwcol}

%Makes text look better
\usepackage{microtype}

\usepackage{siunitx}

\usepackage{xstring}
\usepackage{gensymb}
\usepackage{ragged2e}

\usepackage{graphicx}

%\newcommand{\Variable}{0}
%\newcommand{\storevalue}[1]{\renewcommand{\Variable}{#1}}

\newcommand{\lcaption}[1]{\caption{#1}\label{cap:#1}}
\newcommand{\lsec}[1]{\MakeUppercase{#1}\label{#1}}
\newcommand{\lssec}[1]{\MakeUppercase{#1}\label{#1}}
\newcommand{\lsssec}[1]{#1\label{#1}}

\newcommand{\ssecfake}[1]{\par \vspace{1em}{\Large\bfseries\raggedright #1} \vspace{1em} \par}
\newcommand{\sssecfake}[1]{\par \vspace{0.5em}{}{\large\raggedright #1} \vspace{0.5em} \par}
\newcommand{\sssecfakehref}[2]{\par \vspace{0.5em}{}{\large\raggedright \hyperlink{#1}{#2}} \vspace{0.5em} \par}
\newcommand{\pref}[1]{page \pageref{#1}}
\newcommand{\pdref}[1]{#1 (page \pageref{#1})}
\newcommand{\tref}[1]{Table \ref{cap:#1}: #1 (\pref{cap:#1})}
\newcommand{\trefnp}[1]{Table \ref{cap:#1}: #1}
\newcommand{\trefcp}[1]{Table \ref{cap:#1}: #1, \pref{cap:#1}}
\newcommand{\trefnum}[1]{Table \ref{cap:#1}}

\newcommand{\minus}{\texttt{-{}}}
\newcommand{\plus}{\texttt{+}}    %Use a smaller positive marker for use directly adjacent to numbers
\newcommand{\add}{\texttt{+}\xspace}    %Use a smaller plus sign
\newcommand{\sub}{\texttt{-}\xspace}
\newcommand{\mult}{x}
\newcommand{\mtimes}{x }
\newcommand{\x}{x\xspace}
\newcommand{\ki}{\emph{ki}\xspace}
\newcommand{\Ki}{\emph{Ki}\xspace}
%NOTED BUG: Spell names with mixed capitalization (Dispel magic) will not work properly.
\newcommand{\spell}[1]{\emph{\mbox{\hyperlink{spell:#1}{#1}}}}    %Italicize spells
\newcommand{\spellindirect}[2]{\emph{\hyperlink{spell:#1}{#2}}}
\newcommand{\ritual}[1]{\emph{#1}}    %Italicize spells
\newcommand{\cd}[1]{\par \textbf{#1}:}    %Bold class fluff description headers
\newcommand{\subcf}[1]{\parhead{#1}}    %Bold sub-class-feature headers
\newcommand{\subsk}[1]{\par \emph{#1}:}    %Italicize sub-skill headers

\newcommand{\subparhead}[1]{\par\emph{#1}:}
%\newcommand{\thead}[1]{\textbf{#1}}    %Bold the header of tables
\newcommand{\tb}[1]{\textbf{#1}}    %Bold the header of tables
\newcommand{\tdash}{\hskip 0.1em---\xspace}    %A null entry in a table
\newcommand{\ccol}{\centering\arraybackslash}    %Center columns
\newcommand{\lcol}{\raggedright\arraybackslash}
\newcommand{\rcol}{\raggedleft\arraybackslash}
\newcommand{\booktitle}[1]{\emph{#1}}
%\newcommand{\footnotetemp}[1]{\tiny{[#1]}}    %I don't actually know how to make footnotes yet, so placeholder
%\newcommand{\footnotetemp}[1]{\textsuperscript{#1}}
\newcommand{\fn}[1]{\textsuperscript{#1}}
\newcommand{\M}{\textsuperscript{M}}
\newcommand{\F}{\textsuperscript{F}}
\newcommand{\mfx}[1]{\emph{#1}}    % For material components, focuses, and XP costs for spells
\newcommand{\subspell}[1]{\emph{#1}}    %For parts of spells
\newcommand{\magicitem}[1]{\emph{#1}}    %For the names of items
\newcommand{\mitem}[1]{\emph{#1}}
\newcommand{\shead}[1]{\> \textbf{#1}}    %For the headers of spells in spell lists
%\newcommand{\spellhead}[2][]{\item[#1] \textbf{\hyperlink{spell:#2}{#2}:}}    %Spell in a spell list
%\newcommand{\spellheadc}[3][]{\item[#1] \textbf{\hyperlink{spell:#3 #2}{#2, #3}:}}
%\newcommand{\spellheadrestricted}[1]{\item \textit{\textbf{\hyperlink{spell:#1}{#1}:}}}  %Restricted spell in a spell list
%\newcommand{\spellheadrestrictedc}[2]{\item \textit{\textbf{\hyperlink{spell:#2 #1}{#1, #2}:}}}  %Restricted spell in a spell list
\newcommand{\spellsectionderp}[2]{\vspace{1em}\setlength\multicolsep{0pt}\begin{multicols}{2}
    \lowercase{\hypertarget{spell:#1}{}}\hypertarget{spell:#1}{\subsubsection{#1}}\columnbreak
    \begin{flushright}\large\textbf{\nth{#2} Level}\end{flushright}\end{multicols}
}
\newcommand{\spellsectioncomma}[3]{\vspace{1em}\setlength\multicolsep{0pt}\begin{multicols}{2}
    \lowercase{\hypertarget{spell:#2 #1}{}}\hypertarget{spell:#2 #1}{\subsubsection{#1, #2}}\columnbreak
    \begin{flushright}\large\textbf{\nth{#3} Level}\end{flushright}\end{multicols}
}
%\newcommand{\invocationsection}[1]{\lowercase{\hypertarget{spell:#1}{}}\hypertarget{spell:#1}{\subsubsection{#1}}}
%\newcommand{\invocationsectioncomma}[2]{\lowercase{\hypertarget{spell:#2 #1}{}}\hypertarget{spell:#2 #1}{\subsubsection{#1, #2}}}

\newcommand{\tind}{\hspace{1em}}

\newcommand{\appdescription}[7]{\subsubsection{#1} \parhead{Price (Level)} #2 gp (#3) \parhead{Body Location} #4 \parhead{Aura, Spellpower} #5, #6 \parhead{Activation} #7 \par}
\newcommand{\appdescriptiondc}[8]{\appdescription{#1}{#2}{#3}{#4}{#5}{#6}{#7} \parhead{Special Attack (Attack Bonus)} #8 \par}
\newcommand{\impdescription}[6]{\subsubsection{#1} \parhead{Price (Level)} #2 gp (#3) \parhead{Aura, Spellpower} #4, #5 \parhead{Activation} #6 \par}
\newcommand{\impdescriptionnoact}[5]{\subsubsection{#1} \parhead{Price (Level)} #2 gp (#3) \parhead{Aura, Spellpower} #4, #5 \par}
\newcommand{\impdescriptiondc}[7]{\impdescription{#1}{#2}{#3}{#4}{#5}{#6} \parhead{Special Attack (Attack Bonus)} #7 \par}
\newcommand{\impdescriptiondcnoact}[6]{\impdescription{#1}{#2}{#3}{#4}{#5} \parhead{Special Attack (Attack Bonus)} #6 \par}
%Spell creation requirements, without descriptors
%School, subschool, spell level, Spellpower, craft skill
\newcommand{\mitemreq}[5]{\par \textit{Creation Requirements:} #1 (#2) #3; spellpower #4 or #5}
\newcommand{\mitemreqdesc}[6]{\par \textit{Creation Requirements:} #1 (#2) [#3] #4; spellpower #5 or #6}
%Legacy
\newcommand{\tooldescription}[6]{\subsubsection{#1} \parhead{Price (Level)} #2 gp (#3) \parhead{Spellpower} #4 \parhead{Aura} #5 \parhead{Activation} #6 \par}
%one shot items should not have scaling DRs
\newcommand{\tooldescriptiondcos}[7]{\tooldescription{#1}{#2}{#3}{#4}{#5}{#6} \parhead{Special Attack (Attack Bonus)} #7\par}
\newcommand{\tooldescriptiondc}[7]{\tooldescription{#1}{#2}{#3}{#4}{#5}{#6} \parhead{Special Attack (Attack Bonus)} #7\par}

\newcommand{\prereq}[1]{\subparhead{Prerequisites:} #1}
\newcommand{\bpl}[3]{(\plus #1) #2 gp (#3)}
\newcommand{\bcl}[2]{(\plus #1) #2}

%Monster commands for format consistency
%Starting with \\ instead of \par yields an indent
\newcommand{\montypes}[5]{\begin{mstatblock}\par #1 #2 #3 \hfill \textbf{CR} #4 [#5]}
\newcommand{\montypessubtypes}[6]{\begin{mstatblock}\par #1 #2 #3 (#4) \hfill \textbf{CR} #5 [#6]}
\newcommand{\monsenses}[2]{\par \textbf{Init} #1; Perception #2}
\newcommand{\monsensesspellcraft}[3]{\par \textbf{Init} #1; Perception #2, Spellcraft #3}
\newcommand{\monsensesspecial}[3]{\par \textbf{Init} #1; Perception #2; \textbf{Senses} #3}
\newcommand{\monsensesfull}[4]{\par \textbf{Init} #1; Perception #2, Spellcraft #3; \textbf{Senses} #4}
\newcommand{\monaura}[1]{\par \textbf{Aura} #1}
\newcommand{\monlanguages}[1]{\par \textbf{Languages} #1}
%\newcommand{\monac}[5]{\par \textbf{AC} #1, touch #2, flat-footed #3; \textbf{CMD} #4 \\ (#5)}
%\newcommand{\monacspecial}[6]{\par \textbf{AC} #1, touch #2, flat-footed #3; \textbf{CMD} #4; #5 \\ (#6)}
\newcommand{\monhp}[2]{\par \textbf{HP} #1 (#2 HV)}
\newcommand{\mondr}[1]{\textbf{DR} #1}
\newcommand{\monhpdr}[3]{\mhp{#1}{#2}; \mdr{#3}}
\newcommand{\monimmune}[1]{\par \textbf{Immune} #1}
\newcommand{\monresist}[1]{\par \textbf{Resist} #1}
\newcommand{\monsr}[1]{\textbf{SR} #1}
\newcommand{\monresistsr}[2]{\mres{#1}; \msr{#2}}
\newcommand{\monsaves}[3]{\par \textbf{Fort} \plus #1, \textbf{Ref} \plus #2, \textbf{Will} \plus #3}
\newcommand{\monsavesnegative}[3]{\par \textbf{Fort} #1, \textbf{Ref} #2, \textbf{Will} #3}
\newcommand{\monspeed}[1]{\textbf{Speed} #1}
\newcommand{\monmelee}[1]{\par \textbf{Melee} #1}
\newcommand{\monrange}[1]{\par \textbf{Ranged} #1}
\newcommand{\monspace}[2]{\par \textbf{Space} #1 ft.; \textbf{Reach} #2 ft.}
\newcommand{\monspacenofeet}[2]{\par \textbf{Space} #1; \textbf{Reach} #2}
\newcommand{\moncmb}[2]{\par \textbf{BAB} \plus #1; \textbf{CMB} \plus #2}
\newcommand{\moncmbnegative}[2]{\par \textbf{BAB} \plus #1; \textbf{CMB} #2}
\newcommand{\monspecialact}[1]{\par \textbf{Special Actions} #1}
\newcommand{\monsla}[5]{\par \textbf{Spell-Like Abilities (Sp)} (CL #1, DR #2, #3/day, #4--based): \\ #5}
\newcommand{\monpsi}[5]{\par \textbf{Psionics (Sp)} (CL #1, DR #2, #3/day, #4--based): \\ #5}
\newcommand{\monslamore}[1]{\\ #1}
\newcommand{\monattributes}[6]{\par \textbf{Attributes} Str #1, Dex #2, Con #3, Int #4, Per #5, Wil #6}
\newcommand{\monsa}[1]{\par \textbf{SA} #1}
\newcommand{\monsq}[1]{\par \textbf{SQ} #1}
\newcommand{\monfeats}[1]{\par \textbf{Feats} #1}
\newcommand{\monskills}[1]{\par \textbf{Skills} #1}
\newcommand{\monitems}[1]{\par \textbf{Items} #1}
\newcommand{\monability}[2]{\par \textbf{#1} #2}
%http://tex.stackexchange.com/questions/128636/center-hrule-in-the-middle-of-the-page
\newcommand{\monlinerule}{\par\noindent\hfil\rule[0.5em]{0.9\columnwidth}{0.5pt}\hfil\par}
\newcommand{\mondescription}[1]{\par \end{mstatblock}\sssecfake{Description}\label{#1:Description}}
\newcommand{\mondescriptionnoblock}[1]{\par \sssecfake{Description}\label{#1:Description}}
\newcommand{\monbehavior}[1]{\sssecfake{Combat}\label{#1:Behavior}}

\newcommand{\wilditem}{\addtocounter{enumi}{1} \item}
\newcommand{\wilditemplus}{\addtocounter{enumi}{3} \item}

\newcommand{\spelldesc}[1]{\par\noindent \textit{#1}}
\newcommand{\spelldmg}[1]{\fieldhead{Damage} #1}
\newcommand{\spellheal}[1]{\fieldhead{Healing} #1}
\newcommand{\spellatk}[1]{\fieldhead{Attack} #1}
\newcommand{\spelltgr}[1]{\fieldhead{Trigger} #1}

%\newcommand{\spellfocus}[1]{\par\noindent \textit{Focus:} #1}
\newcommand{\spellmat}[1]{\par\noindent \textit{Material Components:} #1}

\newcommand{\spelltwocol}[2]{\spelltabularcompressed #1 & #2\end{tabularx}}

\newcommand{\spellschool}[1]{\noindent#1}
\newcommand{\spelllists}[1]{\fieldhead{Lists} #1}
\newcommand{\spellcmp}[1]{\fieldhead{Components} #1}
\newcommand{\spelltime}[1]{\fieldhead{Casting Time} #1}
\newcommand{\spellrng}[1]{\fieldhead{Range} #1}

\newcommand{\spellarea}[1]{\fieldhead{Area} #1}
\newcommand{\spellburst}[1]{\fieldhead{Burst} #1}
\newcommand{\spellemanation}[1]{\fieldhead{Emanation} #1}
\newcommand{\spelllimit}[1]{\fieldhead{Limit} #1}
\newcommand{\spellzone}[1]{\fieldhead{Zone} #1}
\newcommand{\spellzoneandburst}[1]{\fieldhead{Zone and Burst} #1}

%\newcommand{\spelltgt}[1]{\fieldhead{Target} #1}
%\newcommand{\spelltgts}[1]{\fieldhead{Targets} #1}
\newcommand{\spelltgtortgts}[1]{\fieldhead{Target or Targets} #1}
\newcommand{\spelltgtorarea}[1]{\fieldhead{Target or Area} #1}
\newcommand{\spelltgteffarea}[1]{\fieldhead{Target or Area} #1}

\newcommand{\spelldur}{\fieldhead{Duration}}
\newcommand{\spelltgrdur}[1]{\fieldhead{Trigger Duration} #1}
\newcommand{\spellfocus}[1]{\fieldhead{Focus} #1}
\newcommand{\spellsr}[1]{\fieldhead{Spell Resistance} #1}

\newcommand{\rngpers}{Personal}
\newcommand{\rngtouch}{Touch}
\newcommand{\rngclose}{Close \reminder{30 ft.}\xspace}
\newcommand{\rngmed}{Medium \reminder{100 ft.}\xspace}
\newcommand{\rnglong}{Long \reminder{300 ft.}\xspace}
\newcommand{\rngfar}{Long \reminder{300 ft.}\xspace}
\newcommand{\rngext}{Extreme \reminder{1,000 ft.}\xspace}
\newcommand{\areasmall}{Small \reminder{10 ft.}\xspace}
\newcommand{\areamed}{Medium \reminder{20 ft.}\xspace}
\newcommand{\arealarge}{Large \reminder{50 ft.}\xspace}
\newcommand{\areahuge}{Huge \reminder{100 ft.}\xspace}
\newcommand{\durbrief}{Brief \reminder{2 rounds}\xspace}
\newcommand{\durshort}{Short \reminder{Focus \add 5 rounds}\xspace}
\newcommand{\durmed}{Medium \reminder{5 minutes}\xspace}
\newcommand{\durlong}{Long \reminder{1 hour}\xspace}
\newcommand{\durext}{Extreme \reminder{12 hours}\xspace}
\newcommand{\durperm}{Permanent\xspace}
\newcommand{\durpersonallong}{\durshort}

\newcommand{\confusionexplanation}{A confused creature cannot take actions normally. If it is attacked, it automatically attacks a random attacker. Otherwise, at the beginning of each round, it randomly decides to take one of four actions that round: babble incoherently, flee from the caster as if panicked, attack the nearest creature, or act normally. A confused character who can't carry out the indicated action does nothing but babble incoherently.\xspace}

\newcommand{\bonusscalingdescription}{This bonus increases to \plus3 at 8th level, to \plus4 at 14th level, and finally to \plus5 at 20th level.\xspace}
\newcommand{\spellbonusscalingdescription}{This bonus increases to \plus3 at spellpower 8, to \plus4 at spellpower 14, and finally to \plus5 at spellpower 20.\xspace}
\newcommand{\babscalingdescription}{This bonus increases to \plus3 at combat prowess 8, to \plus4 at combat prowess 12, and finally to \plus5 at combat prowess 16.\xspace}

\newcommand{\feat}[2]{\hypertarget{feat:#1}{\sssecfakehref{ft:#1}{#1 [#2]}}\label{feat:#1}}
\newcommand{\featpre}{\parhead{Prerequisite} }
\newcommand{\featpres}{\parhead{Prerequisites} }
\newcommand{\featben}{\parhead{Benefit} }
\newcommand{\featspecial}{\parhead{Special} }
%featref no target
\newcommand{\featrefnt}[1]{\hyperlink{feat:#1}{#1}}
\newcommand{\featpref}[1]{\pageref{feat:#1}}
\newcommand{\mitempref}[1]{\pageref{item:#1}}
\newcommand{\featpcref}[1]{#1, page \pageref{feat:#1}}
\newcommand{\featpdref}[1]{#1 (page \pageref{feat:#1})}
%\newcommand{\featref}[1]{}
\newcommand{\stylereq}{\parhead{Style Requirement}}

\newcommand{\featbanenotes}{You can only apply the benefits of a single Bane feat against a particular foe.}

\newcommand{\skill}[2]{\section{#1 (#2)}\label{#1}}

\newcommand{\reminder}[1]{\textcolor{darkgray}{\textit{(#1)}}}

%\newcommand{\vulnerableexplanation}{A vulnerable creature takes a \minus2 penalty to attacks, defenses, and checks.\xspace}
%\newcommand{\vulneffect}{\minus2 to attacks, defenses, and checks\xspace}

\newcommand{\blinded}{\glossterm{blinded}\xspace}
\newcommand{\bloodied}{\glossterm{bloodied}\xspace}
\newcommand{\charmed}{\glossterm{charmed}\xspace}
\newcommand{\confused}{\glossterm{confused}\xspace}
\newcommand{\defenseless}{\glossterm{defenseless}\xspace}
\newcommand{\disoriented}{\glossterm{disoriented}\xspace}
\newcommand{\dominated}{\glossterm{dominated}\xspace}
\newcommand{\exhausted}{\glossterm{exhausted}\xspace}
\newcommand{\fascinated}{\glossterm{fascinated}\xspace}
\newcommand{\fatigued}{\glossterm{fatigued}\xspace}
\newcommand{\frightened}{\glossterm{frightened}\xspace}
\newcommand{\dazed}{\glossterm{dazed}\xspace}
\newcommand{\dazzled}{\glossterm{dazzled}\xspace}
\newcommand{\deafened}{\glossterm{deafened}\xspace}
\newcommand{\grappled}{\glossterm{grappled}\xspace}
\newcommand{\goaded}{\glossterm{goaded}\xspace}
\newcommand{\helpless}{\glossterm{helpless}\xspace}
\newcommand{\ignited}{\glossterm{ignited}\xspace}
\newcommand{\immobilized}{\glossterm{immobilized}\xspace}
\newcommand{\nauseated}{\glossterm{nauseated}\xspace}
\newcommand{\panicked}{\glossterm{panicked}\xspace}
\newcommand{\paralyzed}{\glossterm{paralyzed}\xspace}
\newcommand{\petrified}{\glossterm{petrified}\xspace}
\newcommand{\prone}{\glossterm{prone}\xspace}
\newcommand{\shaken}{\glossterm{shaken}\xspace}
\newcommand{\sickened}{\glossterm{sickened}\xspace}
\newcommand{\slowed}{\glossterm{slowed}\xspace}
\newcommand{\staggered}{\glossterm{staggered}\xspace}
\newcommand{\stunned}{\glossterm{stunned}\xspace}
\newcommand{\taunted}{\glossterm{taunted}\xspace}
\newcommand{\wounded}{\glossterm{wounded}\xspace}
\newcommand{\unaware}{\glossterm{unaware}\xspace}

\newcommand{\difficultterrain}{\glossterm{difficult terrain}\xspace}

\newcommand{\concealment}{concealment\xspace}

\newcommand{\destructivespellnotes}{If a destructive spell deals enough damage to an interposing barrier to shatter or breaks through it, its effects may continue beyond the barrier if the area permits; otherwise, it stops at the barrier just as any other spell effect does.\xspace}
\newcommand{\cursespellnotes}{Curses cannot be dispelled with \spell{dispel magic}, but can be removed with \spell{break enchantment} or \spell{remove curse}.\xspace}
\newcommand{\firespellnotes}{Fire spells shed light equivalent to a torch. They do not function underwater. They can ignite combustable materials, such as dry wood and cloth.\xspace}
\newcommand{\forcespellnotes}{Force spells also affect the Ethereal Plane.}
\newcommand{\fogspellnotes}{Fog spells do not function underwater and can be dispersed by wind. Localized wind, such as from a \spell{gust of wind} spell, only disperses the fog where it overlaps the fog. A fire spell burns away the fog in the area into which it deals damage.\xspace}
\newcommand{\fogwindspellnotes}{A moderate wind (11\add mph) disperses the fog in 5 rounds; a strong wind (21\add mph) disperses the fog in 1 round.\xspace}
\newcommand{\sensorspellnotes}{A magic sensor is a floating, invisible sphere approximately one inch in diameter. It has 1 hit point and an Armor defense of 8. Its special defenses are the same as yours.\xspace}
\newcommand{\subtlespellnotes}{Subtle spells have no obvious effects. A creature affected by a subtle spell is usually unaware that it is under magical influence. The DR to identify a subtle spell with the Spellcraft skill is 10 higher than normal (see \pcref{Spellcraft}).}
\newcommand{\mobilesensorspellnotes}[1]{\sensorspellnotes The sensor has a #1 foot fly speed with perfect maneuverability. It is unable to enter another plane of existence, even through a \spell{gate} or similar magical portal.}
\newcommand{\norepeatspellnotes}{You can only affect any individual creature with this spell once per 24 hours.\xspace}
\newcommand{\norepeatnotes}{You can only affect any individual creature with this ability once per 24 hours.\xspace}
\newcommand{\physicalspellnotes}{Physical spells cannot be dismissed or dispelled, and do not allow spell resistance.}
\newcommand{\sizingspellnotes}{Multiple magical effects which change size do not stack. Size increases offset size decreases on a one-for-one basis.}

\newcommand{\spelltablecolumns}{>{\lcol}X l}
\newcommand{\spelllevelschool}[2]{\spelllvl{#1} & #2 \\}
\newcommand{\spelllevelnew}[3]{\tb{\nth{#1} level} #2 & #3 \\}

\newcommand{\dismissable}{}
\newcommand{\shapeable}{(S)}
\newcommand{\rngunrestricted}{(Unrestricted)}
\newcommand{\undispellable}{(Undispellable)}

% http://tex.stackexchange.com/questions/135617/how-to-i-reduce-space-before-and-after-hrule
\newcommand{\HRule}[2][\medskipamount]{\par
  \vspace*{\dimexpr-\parskip-\baselineskip+#1}
  \noindent\rule{\linewidth}{#2}\par
  \vspace*{\dimexpr-\parskip-.5\baselineskip+#1}}

\newcommand{\spellline}{\par\HRule[0.5\medskipamount]{0.2pt}}
% weird spacing before twocol
\newcommand{\spelllinespaced}{\spellline\vspace{0.5\medskipamount}}

%\newcommand{\spelleffect}{\fieldhead{Effect}}
%\newcommand{\spellsuccess}{\fieldhead{Success}}
%\newcommand{\spellfailure}{\fieldhead{Failure}\xspace}
\newcommand{\spellnotes}{\par\noindent\textit{Notes}:\xspace}
\newcommand{\spellspecial}{\fieldhead{Special}}
\newcommand{\spellcheck}[1]{\fieldhead{Check} #1}

\ExplSyntaxOn

\DeclareDocumentCommand{\Create}{m}
{
    \newcounter{#1}
    \setcounter{#1}{0}
}

\DeclareDocumentCommand{\Set}{m m}
{
    \setcounter{#1}{#2}
}

\DeclareDocumentCommand{\Get}{m}
{
    \arabic{#1}
}

\DeclareDocumentCommand{\GetValue}{m}
{
    \value{#1}
}

\newcommand{\TempVar}{0}
\newcommand{\TempVarTwo}{0}

\DeclareDocumentCommand{\parhead}{s m o}
{
    \vspace{0.25em}
    \par \noindent
    \IfNoValueTF{#3}
    {\textbf{#2}:}
    {\textbf{#2}~[#3]:}\xspace
}

\DeclareDocumentCommand{\fieldhead}{m o}
{
    \par\noindent
    \IfNoValueTF{#2}
    {\textbf{#1}:}
    {\textbf{#1}~[#2]:}\xspace
}

\DeclareDocumentCommand{\spellfailure}{o}
{
    \IfNoValueTF{#1}
    {\fieldhead{Failure}}
    {\fieldhead{Failure}[#1]}
}
\DeclareDocumentCommand{\spellsuccess}{o}
{
    \IfNoValueTF{#1}
    {\fieldhead{Hit}}
    {\fieldhead{Hit}[#1]}
}
\DeclareDocumentCommand{\spellcritical}{o}
{
    \IfNoValueTF{#1}
    {\fieldhead{Critical~Hit}}
    {\fieldhead{Critical~Hit}[#1]}
}
\DeclareDocumentCommand{\spelleffect}{o}
{
    \IfNoValueTF{#1}
    {\fieldhead{Effect}}
    {\fieldhead{Effect}[#1]}
}
\DeclareDocumentCommand{\spelltgt}{m o}
{
    \IfNoValueTF{#2}
    {\fieldhead{Target} #1}
    {\fieldhead{#2~Target} #1}
}
\DeclareDocumentCommand{\spelltgts}{m o}
{
    \IfNoValueTF{#2}
    {\fieldhead{Targets} #1}
    {\fieldhead{#2~Targets} #1}
}

\DeclareDocumentCommand{\ritualcostparser}{m}
{
    \ifcase#1\relax
    \or 5~gp    %1
    \or 20~gp   %2
    \or 50~gp   %3
    \or 125~gp  %4
    \or 300~gp  %5
    \or 750~gp  %6
    \or 1,500~gp %7
    \or 3,000~gp %8
    \or 7,500~gp %9
    \else???
    \fi
}

\DeclareDocumentCommand{\ritualcost}{s m o}
{
    \IfBooleanF{#1}{\spellline}
    \spellmat{\ritualcostparser#2\xspace
        \IfValueTF{#3}
        {#3}
        {in~ritual~components.}
    }
}

% args:
%   magic item price or type of price
%   effective spell level
\DeclareDocumentCommand{\mitempricewithlevel}{m m}
{
    \IfSubStr{#1}{Passive}
    {
        %\renewcommand{\TempVarTwo}{\magicitempricepassive{#2}}
        \magicitempricepassive{#2}
    }
    {
        \IfSubStr{#1}{Active}
        {
            \magicitempriceactive{#2}
        }
        {
            \renewcommand{\TempVarTwo}{#1}
        }
    }

    \StrLen{\TempVarTwo}[\TempVar]
    \pgfmathparse{\TempVar > 4}
    \ifcase\pgfmathresult
        % handle normal magic items
        \fieldhead{Price~(Level)} \num{\TempVarTwo}~gp~(\cheapitemlevel{\TempVarTwo})
    \or
        % handle very expensive magic items
        % by passing the number with the right three digits removed
        \StrGobbleRight{\TempVarTwo}{3}[\TempVar]
        \fieldhead{Price~(Level)} \num{\TempVarTwo}~gp~(
            \magicitemexpensiveitemlevel{\TempVar}
        )
    \fi
}

\DeclareDocumentCommand{\mitempricetable}{m m}
{
    \IfSubStr{#1}{Passive}
    {
        %\renewcommand{\TempVarTwo}{\magicitempricepassive{#2}}
        \magicitempricepassive{#2}
    }
    {
        \IfSubStr{#1}{Active}
        {
            \magicitempriceactive{#2}
        }
        {
            \renewcommand{\TempVarTwo}{#1}
        }
    }
    \xdef\ItemCost{\num{\TempVarTwo}~gp}

    \StrLen{\TempVarTwo}[\TempVar]
    \pgfmathparse{\TempVar > 4}
    \ifcase\pgfmathresult
        % handle normal magic items
        \xdef\ItemLevel{\cheapitemlevel{\TempVarTwo}}
    \or
        % handle very expensive magic items
        % by passing the number with the right three digits removed
        \StrGobbleRight{\TempVarTwo}{3}[\TempVar]
        \xdef\ItemLevel{\magicitemexpensiveitemlevel{\TempVar}}
    \fi
    \ItemCost & \ItemLevel
}

% args:
%    spell level
\DeclareDocumentCommand{\magicitempriceactive}{m}
{
    \renewcommand{\TempVarTwo}{
        \ifcase#1~
        100    %0
        \or 200   %1
        \or 800   %2
        \or 2000  %3
        \or 5000  %4
        \or 12000  %5
        \or 30000 %6
        \or 60000 %7
        \or 140000 %8
        \or 300000 %9
        \else 0
        \fi
    }
}

% args:
%    spell level
\DeclareDocumentCommand{\magicitempricepassive}{m}
{
    \renewcommand{\TempVarTwo}{
        \ifcase#1~
        200   %0
        \or 800   %1
        \or 2000  %2
        \or 5000  %3
        \or 12000  %4
        \or 30000 %5
        \or 60000 %6
        \or 140000 %7
        \or 300000 %8
        \or 700000 %9
        \else 0
        \fi
    }
}

% If #1 = #2
%     print #3.
% Elseif #4 is provided
%     print #4
\DeclareDocumentCommand{\ifstr}{m m m o}
{
    \ifnum 0=\pdfstrcmp{#1}{#2}
        #3
    \else
        \IfValueT{#4}{#4}
    \fi
}

% args:
%   pre-spell header spell name, spell name modifier (Greater, etc.)
\DeclareDocumentCommand{\spellhead}{o m o}
{
    \IfValueTF{#1}
    {\item[#1]}
    {\item}
    \textbf{
        \IfValueTF{#3}
        {
            \hyperlink{spell:#3~#2}{#2,~#3}
            %\pageref{spell:#3~#2}
        }
        {
            \hyperlink{spell:#2}{#2}
            %\pageref{spell:#2}
        }
    }:
}

\DeclareDocumentCommand{\spellinfo}{m s m}
{
    \IfBooleanTF{#2}
    {
        \noindent#1
        \fieldhead{Lists} #3
    }
    {
        \spelltwocol{\fieldhead{Schools}~#1}{\fieldhead{Lists}~#3}
    }
}

\DeclareDocumentCommand{\spelltags}{m}
{
    \fieldhead{Tags} #1
}

\DeclareDocumentCommand{\imath}{m}
{
    %\pgfmathparse{int(floor(#1))}\pgfmathresult
    \pgfmathsetmacro{\TempVar}{int(floor(#1))}
    \TempVar
}

\DeclareDocumentCommand{\imathparse}{m}
{
    \pgfmathsetmacro{\TempVar}{int(floor(#1))}
}

\DeclareDocumentCommand{\imathresult}{}
{
    \TempVar
}

\DeclareDocumentCommand{\mathnth}{m}
{
    \pgfmathparse{int(floor(#1))}\nth{\pgfmathresult}
}

\DeclareDocumentCommand{\gooddefense}{m}
{
    \plus \imath{(#1*5/4)}
}

\DeclareDocumentCommand{\avgdefense}{m}
{
    \plus \imath{(#1*1)}
}

\DeclareDocumentCommand{\poordefense}{m}
{
    \plus \imath{(#1*3/4)}
}

\DeclareDocumentCommand{\babold}{m}
{
    \imathparse{int(floor((#1-1)/5+1))}

    \ifcase\imathresult \relax \or
        \plus #1 % case 0
    \or
        \plus #1 / \plus \imath{#1-5} % case 1
    \or
        \plus #1 / \plus \imath{#1-5} / \plus \imath{#1-10} % case 2
    \or
        \plus #1 / \plus \imath{#1-5} / \plus \imath{#1-10} / \plus \imath{#1-15} % case 3
    \or
        \plus #1 / \plus \imath{#1-5} / \plus \imath{#1-10} / \plus \imath{#1-15} / \plus \imath{#1-20} % case 3
    \fi
}

\DeclareDocumentCommand{\bab}{m o}
{
    \IfValueTF{#2}
    {
        \imathparse{int(floor((#1-1)/#2+1))}
    }
    {
        \imathparse{int(floor((#1-1)/5+1))}
    }

    \ifcase\imathresult
        {}#1 % case -1
    \or
        {}#1 % case 0
    \or
        {}#1~{}~(\mult2) % case 1
    \or
        {}#1~{}~(\mult3) % case 2
    \or
        {}#1~{}~(\mult4) % case 3
    \or
        {}#1~{}~(\mult5) % case 3
    \fi
}

\DeclareDocumentCommand{\goodbab}{m}
{
    \pgfmathparse{int(floor(#1+2))}
    \bab{\pgfmathresult}
}

\DeclareDocumentCommand{\avgbab}{m}
{
    \pgfmathparse{int(floor((#1*4/5+2)))}
    \bab{\pgfmathresult}
}

\DeclareDocumentCommand{\poorbab}{m}
{
    \pgfmathparse{int(floor((#1*2/3+1)))}
    \bab{\pgfmathresult}
}

\DeclareDocumentCommand{\alldefenseprogressionrow}{m}
{
    \nth{#1} & \gooddefense{#1} & \avgdefense{#1} & \poordefense{#1}
}

\DeclareDocumentCommand{\allbabprogressionrow}{m}
{
    \nth{#1} & \goodbab{#1} & \avgbab{#1} & \poorbab{#1}
}

\DeclareDocumentCommand{\barbarianprogressionrow}{m}
{
    \nth{#1} & \goodbab{#1}
}

\DeclareDocumentCommand{\clericprogressionrow}{m}
{
    \nth{#1} & \avgbab{#1}
}

\DeclareDocumentCommand{\druidprogressionrow}{m}
{
    \nth{#1} & \avgbab{#1}
}

\DeclareDocumentCommand{\halfdragonprogressionrow}{m}
{
    \nth{#1} & \goodbab{#1}
}

\DeclareDocumentCommand{\monkprogressionrow}{m}
{
    \nth{#1} & \goodbab{#1}
}

\DeclareDocumentCommand{\paladinprogressionrow}{m}
{
    \nth{#1} & \goodbab{#1}
}

\DeclareDocumentCommand{\rangerprogressionrow}{m}
{
    \nth{#1} & \goodbab{#1}
}

\DeclareDocumentCommand{\rogueprogressionrow}{m}
{
    \nth{#1} & \avgbab{#1}
}

\DeclareDocumentCommand{\spellwarpedprogressionrow}{m}
{
    \nth{#1} & \goodbab{#1}
}

\DeclareDocumentCommand{\mageprogressionrow}{m}
{
    \nth{#1} & \poorbab{#1}
}

\DeclareDocumentCommand{\fighterprogressionrow}{m}
{
    \nth{#1} & \goodbab{#1}
}

\DeclareDocumentCommand{\spelldamage}{m o}
{
    \IfValueTF{#2}{#2}{1d8}~
    \ifstr{#1}{#1~}{}
    damage~\plus1d~per~two~spellpower.
    % or alternately: standard damage based on your spellpower \add 2, etc.
}

\DeclareDocumentCommand{\spelldamageemp}{m o}
{
    \ifstr{#1}{}
    {
        \IfValueTF{#2}
        {1#2~damage~per~spellpower}
        {1d8~damage~per~spellpower}
    }
    {
        \IfValueTF{#2}
        {1#2~#1~damage~per~spellpower}
        {1d8~#1~damage~per~spellpower}
    }
}

\DeclareDocumentCommand{\spellquicktargeting}{s m m}
{
    \IfBooleanTF{#1}
    {
        \spelltwocol{\spelltgts{#2}}{\spellrng{#3}}
    }
    {
        \spelltwocol{\spelltgt{#2}}{\spellrng{#3}}
    }
}

\DeclareDocumentCommand{\spelltabular}{}
{
    \par\noindent
    \begin{tabularx}{\columnwidth}{>{\lcol}X l}
}

\DeclareDocumentCommand{\spelltabularcompressed}{}
{
    \renewcommand{\arraystretch}{0.5}
    \par\noindent
    \begin{tabularx}{\linewidth}{@{} >{\lcol}X l @{}}
}

\DeclareDocumentCommand{\spellmiscast}{m}
{
    \par\noindent\textit{#1~Miscast}:\xspace
}

% args:
%   class name
%   level, if any
%   ability name
%   ability type (Ex, Su, etc.), if any
\DeclareDocumentCommand{\cf}{s m o m o}
{
    \IfValueTF{#3}
    {
        \IfValueTF{#5}
        { \subsubsection{\nth{#3}~--~#4~(#5)} }
        { \subsubsection{\nth{#3}~--~#4} }
    }
    {
        \IfValueTF{#5}
        { \subsubsection{#4~(#5)} }
        { \subsubsection{#4} }
    }
    \IfBooleanF{#1}
    {
        \label{#2:#4}
    }
}

% feat level
% args:
%   level
%   name
%   ability type, if any
\DeclareDocumentCommand{\ff}{o m o}
{
    \IfValueTF{#1}
    {
        \IfValueTF{#3}
        { \nth{#1}~--~\textbf{#2}~(#3): }
        { \nth{#1}~--~\textbf{#2}: }
    }
    {
        \IfValueTF{#3}
        { \parhead{#2~(#3)} }
        { \parhead{#2} }
    }
}

\DeclareDocumentCommand{\itemhead}{s m}
{
    \item
    \IfValueTF{#1}
    {
        \textit{#2}
    }
    {
        \textbf{#2}
    }
}

\DeclareDocumentCommand{\altcf}{o m o}
{
    \cf*{}[#1]{#2}[#3]
}

\DeclareDocumentCommand{\featref}{s m}
{
    \hyperlink{feat:#2}{#2}
    \IfBooleanF{#1}
    {
        \hypertarget{ft:#2}{}
    }
}

\DeclareDocumentCommand{\mitemref}{s m}
{
    \IfBooleanF{#1}
    {
        \hypertarget{it:#2}{}
    }
    \hyperlink{item:#2}{#2}
}

\DeclareDocumentCommand{\term}{m}
{
    \textbf{#1}
}

% args:
%   #1: reference name of term, if different from display name
%   #2: display name of term
\DeclareDocumentCommand{\glossterm}{o m}
{
    \IfValueTF{#1}
    {
        % Enable this to test glossary links
        % \pref{#1}
        % TODO: remove \mbox before making major releases;
        % it fixes bizarre page break errors but makes text uglier
        \hyperlink{gloss:#1}{\textbf{\mbox{#2}}}
    }
    {
        % Enable this to test glossary links
        % \pref{#2}
        \hyperlink{gloss:#2}{\textbf{\mbox{#2}}}
    }
}

% args:
%   #1: name of term
%   #2: alternate name of term, only used for labels/targets
\DeclareDocumentCommand{\glossdef}{m o}
{
    \IfValueT{#2}
    {
        % Enable this to test glossary links
        % \label{#2}
        \hypertarget{gloss:#2}{}
    }
    % Enable this to test glossary links
    % \label{#1}
    \hypertarget{gloss:#1}{}
    \parhead{#1}
}

\DeclareDocumentCommand{\pcref}{o m}
{
    \IfValueTF{#1}
    {
        #2,~page~\pageref{#1}
    }
    {
        #2,~page~\pageref{#2}
    }
}

% args:
%   item price
\DeclareDocumentCommand{\cheapitemlevel}{m}
{
    % construct a statement that returns 0-20, incrementing by 1 for each
    % category of item level the item falls into
    \pgfmathparse{
        #1 <= 10 ? 0 : (
            #1 <= 50 ? 1 : (
                #1 <= 100 ? 2 : (
                    #1 <= 250 ? 3 : (
                        #1 <= 500 ? 4 : (
                            #1 <= 800 ? 5 : (
                                #1 <= 1200 ? 6 : (
                                    #1 <= 1800 ? 7 : (
                                        #1 <= 2750 ? 8 : (
                                            #1 <= 4000 ? 9 : (
                                                #1 <= 6500 ? 10 : (
                                                    #1 <= 10000 ? 11 : 12
                                                )
                                            )
                                        )
                                    )
                                )
                            )
                        )
                    )
                )
            )
        )
    }
    \ifstr{\pgfmathresult}{0}{1/2}[\nth{\pgfmathresult}]
}

\DeclareDocumentCommand{\magicitemexpensiveitemlevel}{m}
{
    \pgfmathparse{
        #1 <= 10 ? 11 : (
            #1 <= 16 ? 12 : (
                #1 <= 25 ? 13 : (
                    #1 <= 37 ? 14 : (
                        #1 <= 55 ? 15 : (
                            #1 <= 85 ? 16 : (
                                #1 <= 125 ? 17 : (
                                    #1 <= 190 ? 18 : (
                                        #1 <= 280 ? 19 : (
                                            #1 <= 400 ? 20 : 21
                                        )
                                    )
                                )
                            )
                        )
                    )
                )
            )
        )
    }
    \nth{\pgfmathresult}
}

% args:
%   effective spell level
\DeclareDocumentCommand{\magicitemaurastrength}{m}
{
    \pgfmathparse{
        #1 > 9 ? "Overwhelming" : (
            #1 > 6 ? "Strong" : (
                #1 > 3 ? "Moderate" : (
                    #1 >= 0 ? "Faint" : 0
                )
            )
        )
    }
    \pgfmathresult
}

% args:
%   star: if true, return spellpower with ``nth''
%   effective spell level
\DeclareDocumentCommand{\magicitemspellpower}{s m}
{
    \imathparse{max(#2*2,1)}
    \IfBooleanTF{#1}
    {
        \nth{\imathresult}
    }
    {
        \imathresult
    }
}

\DeclareDocumentCommand{\magicitemeffect}{o}
{
    \spelleffect[#1]
}

\DeclareDocumentCommand{\magicitemability}{m o}
{
    \fieldhead{#1~Ability}[#2]
}

\DeclareDocumentCommand{\spellaugment}{m m m o}
{
    \par\textbf{#1}~--~\textit{#2}:~#3
    \IfValueT{#4}{~This~is~a~#4~effect.}
}

\DeclareDocumentCommand{\augment}{m m}
{
    \vspace{0.5em}
    \par\noindent (\plus #1)~\textbf{#2}:\xspace
}

\DeclareDocumentCommand{\labeltext}{m}
{
    #1\label{#1}
}

\DeclareDocumentCommand{\hit}{}
{
    \subparhead{Hit}\xspace
}

\DeclareDocumentCommand{\crit}{}
{
    \subparhead{Critical~Hit}\xspace
}

\ExplSyntaxOff

\newcommand{\miscastexplode}{\spellmiscast{Explosive} The spell does not have its normal effect. Instead, the magical energy explodes, dealing 1d6 damage per two spellpower to you and all creatures adjacent to you.}
\newcommand{\miscastrandom}{\spellmiscast{Retargeting} The spell targets a random valid target within range (including yourself, if applicable).}
\newcommand{\miscastyou}{\spellmiscast{Localized} The spell targets you, or originates from your location, instead of its intended location. In addition, it only extends out 5 feet from your location, rather than its normal area.}

\newcommand{\classbasics}[1]{\par\noindent\textbf{#1}:}

\newcommand{\meh}[1]{\hypertarget{suspicious}{}#1}

\newcommand{\pari}{\par\noindent}

% for feat paths
\newcommand{\itempath}{\item\addtocounter{enumi}{1}\xspace}

\newcommand{\magical}{[\glossterm{Magical}]\xspace}

% create bordered frames
\newmdenv{fullborder}
\newmdenv[leftline=false,rightline=false]{topandbottomborders}

%Set the spacing of lists
\newenvironment{enumerate*}
{\begin{enumerate}
  \setlength{\leftmargin}{0em}
  \setlength{\topsep}{1pt}
  \setlength{\itemsep}{1pt}
  \setlength{\parskip}{0pt}
  \setlength{\parsep}{0pt}}
{\end{enumerate}}
\newenvironment{itemblank}
{\begin{itemize}[label={}]
  \setlength{\leftmargin}{0em}
  \setlength{\topsep}{1pt}
  \setlength{\itemsep}{1pt}
  \setlength{\parskip}{0pt}
  \setlength{\parsep}{0pt}}
{\end{itemize}}

%Define a custom table that uses that coloring
\newenvironment{dtable}
{\begin{table}[htb!]
  \small
  \rowcolors{1}{white}{tbrown}}
{\end{table}}
%Plus a custom table that takes up two columns
\newenvironment{dtable*}
{\begin{table*}[htb!]
  \small
  \rowcolors{1}{white}{tbrown}}
{\end{table*}}
%And one more for two-column tables that need fewer restrictions
\newenvironment{dtable!*}
{\begin{table*}[htb!]
    \small
    \rowcolors{1}{white}{tbrown}
}{
    \end{table*}
}

% small text for the feats table
\newenvironment{longtabuwrapper}
{
    \onecolumn
    \small
    \taburowcolors [2] 2{tbrown .. white}
}
{\twocolumn}

\newenvironment{spelltable}
{\begin{table}[htb!]}
{\end{table}}

%A list for sorcerer/wizard spells
\newenvironment{swspelllist}
{\begin{description}[nosep,font=\normalfont,leftmargin=4.25em,style=nextline,itemindent=-1em]}
{\end{description}}
%A list for normal spells
\newenvironment{spelllist}
{\begin{description}[nosep,font=\normalfont,leftmargin=2.25em,style=nextline,itemindent=-1em]}
{\end{description}}
%A list for rituals
\newenvironment{rituallist}
{\begin{description}[nosep,font=\normalfont,leftmargin=4.5em,style=nextline,itemindent=-1em]}
{\end{description}}


\newenvironment{mstatblock}{\leftskip 0.1in \parindent-0.1in}{}

\newenvironment{wildaspect}
{\begin{enumerate*}\setcounter{enumi}{1}}
{\end{enumerate*}}

\newenvironment{greaterwildaspect}
{\begin{enumerate*}\setcounter{enumi}{9}}
{\end{enumerate*}}

\newenvironment{spellhealthy}
{\parhead*{Healthy Effect}}
{}
\newenvironment{spellblood}
{\parhead*{Bloodied Effect}}
{}

\DeclareDocumentEnvironment{spellmargin}{}
{\par\setlength{\leftskip}{1em}}
{\par\setlength{\leftskip}{0pt}}

\DeclareDocumentEnvironment{spellmarginext}{}
{\par\setlength{\leftskip}{2em}}
{\par\setlength{\leftskip}{0pt}}

\ExplSyntaxOn

\DeclareDocumentEnvironment{fakehang}{}
{
    \setlength{\parindent}{2em}
    \everypar{\hangindent=1em}
}
{
    \par
}


% with a star, don't include \spellline
\DeclareDocumentEnvironment{spelltarget}{s o m t{l} o}
{\IfBooleanF{#1}{\spellline}
    \IfNoValueTF{#5}
    {\spelltgt{#3}[#2]}
    {
        \IfBooleanTF{#4}
        {\spelltgt{#3}[#2]\parhead{Attack} #5}
        {\spelltwocol{\spelltgt{#3}[#2]}{\spellatk{#5}}}
    }
    \begin{spellmargin}
}
{
    \end{spellmargin}
}

% same thing, but plural
\DeclareDocumentEnvironment{spelltargets}{s o m t{l} o}
{\IfBooleanF{#1}{\spellline}
    \IfNoValueTF{#5}
    {\spelltgts{#3}[#2]}
    {
        \IfBooleanTF{#4}
        {\spelltgts{#3}[#2]\parhead{Attack} #5}
        {\spelltwocol{\spelltgts{#3}[#2]}{\spellatk{#5}}}
    }
    \begin{spellmargin}
}
{
    \end{spellmargin}
}

\DeclareDocumentEnvironment{spelltrigger}{m o}
{
    \IfValueTF{#2}
    {
        \spelltwocol{\parhead{Trigger} #1}{#2}
    }
    {
        \parhead*{Trigger} #1
    }
    %\begin{spellmargin}
}
{
    %\end{spellmargin}
}

% with a star, include \spellline
\DeclareDocumentEnvironment{spellattack}{s m}
{
    \IfBooleanT{#1}{\spellline}
    \spellatk{#2}
    \begin{spellmargin}
}
{
    \end{spellmargin}
}

% with a star, include \spellline
\DeclareDocumentEnvironment{spellattacktriggered}{s m}
{
    \IfBooleanT{#1}{\spellline}
    \parhead*{Triggered~Attack} #2
    \begin{spellmargin}
}
{
    \end{spellmargin}
}

% with a star, include \spellline
\DeclareDocumentEnvironment{spelltriggeredeffect}{s m}
{
    \IfBooleanT{#1}{\spellline}
    \parhead*{Triggered~Effect} #2
    \begin{spellmargin}
}
{
    \end{spellmargin}
}

\DeclareDocumentEnvironment{spellheader}{s}
{
    %\IfBooleanF{#1}{\spellline}
}
{
    %\IfBooleanF{#1}{\spellline}
}

\DeclareDocumentEnvironment{spellcontent}{}
{
    \begin{thesamepage}
}
{
    \end{thesamepage}
}

\surroundwithmdframed[
    style=spellcontent,
    leftline=true,
    topline=true,
    rightline=true,
    bottomline=true,
]{spellcontent}

\DeclareDocumentEnvironment{ability}{m o}
{
    \begin{thesamepage}
    \spelltwocol{\textbf{#1}}{
        \IfValueTF{#2}{#2}{}
    }
}
{
    \end{thesamepage}
}

\surroundwithmdframed[
    leftline=true,
    topline=true,
    rightline=true,
    bottomline=true,
    skipabove=0.25em,
    skipbelow=0.5em,
    innerleftmargin=0.25em,
    innerrightmargin=0.5em,
    innertopmargin=-0.1em,  % using thesamepage gives it a top margin
    innerbottommargin=0.25em,
]{ability}

\newmdenv[
    style=spellcontent,
    leftline=true,
    topline=false,
    rightline=true,
    bottomline=true,
    skipabove=0,
]{spellsubcontent}

\newmdenv[
    style=colorenv,
    backgroundcolor=LightCyan,
]{spelltargetinginfo}

\newmdenv[
    style=colorenv,
    backgroundcolor=LightCyan,
]{augmenttargetinginfo}

\DeclareDocumentEnvironment{spelleffects}{}
{
    \begin{fakehang}
}
{
    \end{fakehang}
}
\surroundwithmdframed[
    style=colorenv,
    backgroundcolor=Lavender,
]{spelleffects}

\DeclareDocumentEnvironment{augmenteffects}{}
{
    \begin{fakehang}
}
{
    \end{fakehang}
}
\surroundwithmdframed[
    style=colorenv,
    backgroundcolor=Lavender,
]{augmenteffects}

\DeclareDocumentEnvironment{spellfooter}{}
{
    \begin{fakehang}
}
{
    \end{fakehang}
}

\surroundwithmdframed[
    style=colorenv,
    backgroundcolor=Gainsboro,
    leftline=true,
    rightline=true,
]{spellfooter}

\DeclareDocumentEnvironment{spelltriggercolor}{}
{
    \begin{fakehang}
}
{
    \end{fakehang}
}

\surroundwithmdframed[
    style=colorenv,
    backgroundcolor=Linen
]{spelltriggercolor}

\DeclareDocumentEnvironment{dtabularx}{m m}
{
    \tabularx{#1}{#2}%
}
{
    \endtabularx
}

\surroundwithmdframed{dtabularx}

\DeclareDocumentEnvironment{spellsection}{o m o}
{
    \vspace{1em}
    \begin{thesamepage}
    \setlength{\multicolsep}{0pt}
    \begin{multicols}{2}
        \IfValueTF{#1}
        {
            \lowercase{\hypertarget{spell:#1~#2}{}}\label{spell:#1~#2}
            \hypertarget{spell:#1~#2}
                {\subsubsection{#2,~#1}}
        }
        {
            \lowercase{\hypertarget{spell:#2}{}}\label{spell:#2}
            \hypertarget{spell:#2}
                {\subsection{#2}}
        }
        \IfValueTF{#3}
        {
            \columnbreak
            \begin{flushright}
                \large\textbf{\nth{#3}~Level}
            \end{flushright}
        }
        {
            \columnbreak
            \begin{flushright}
            \end{flushright}
        }
    \end{multicols}
}
{
    \end{thesamepage}
}

% args:
%   item name
%   price
%   effective spell level
%   [body location]
%   school and descriptors
%   [activation]
%   crafting skill
%   fluff text
\DeclareDocumentEnvironment{magicitemdef}{o m o m o m o m o}
{
    \begin{thesamepage}
    \IfValueTF{#1}
    {
        \hypertarget{item:#2}{\sssecfakehref{it:#2}{#2~[#1]}}
    }
    {
        \hypertarget{item:#2}{\sssecfakehref{it:#2}{#2}}
    }
    \label{item:#2}

    \begin{spellcontent}
    \begin{spelltargetinginfo}
        \IfValueT{#3}
        {
            \mitempricewithlevel{#3}{#4}
        }

        \parhead*{Base~Item~Power} \magicitemspellpower{#4}

        \IfValueT{#5}
        {
            \parhead*{Location} #5
        }

        \parhead*{Aura} \magicitemaurastrength{#4}~#6

        \IfValueT{#7}
        {
            \parhead*{Activation} #7
        }
        \par
    \end{spelltargetinginfo}
    \begin{spelleffects}
}
{
    \end{spelleffects}
    \end{spellcontent}
    \begin{spellfooter}
        \IfValueT{#9}{\par\noindent #9}
        \par\noindent \textit{Creation~Requirements:}~#6;~
        \nth{#4}~level~spells~or~Craft~(#8)~\imath{#4*2+5}~ranks
    \end{spellfooter}
    \end{thesamepage}
}

\DeclareDocumentEnvironment{thesamepage}{}
{
    \par\nobreak\vfil\penalty0\vfilneg
    \vtop{}
}
{
    \par\xdef\tpd{\the\prevdepth}
    \prevdepth=\tpd
}

\DeclareDocumentEnvironment{spellaugments}{}
{
    \parhead{Augments}
    \begin{fakehang}
}
{
    \end{fakehang}
}
\surroundwithmdframed[
    style=colorenv,
    backgroundcolor=MistyRose,
]{spellaugments}

\DeclareDocumentEnvironment{spellcantrip}{}
{
    \fieldhead{Cantrip}
    \begin{fakehang}
}
{
    \par~If~you~cast~this~spell~as~a~cantrip,~you~do~not~need~to~spend~an~\glossterm{action~point}~to~cast~it,~but~you~cannot~apply~any~augments~to~it.
    \end{fakehang}
}
\surroundwithmdframed[
    style=colorenv,
    backgroundcolor=MistyRose,
]{spellcantrip}

\DeclareDocumentEnvironment{feat}{m m}
{
    \hypertarget{feat:#1}{\sssecfakehref{ft:#1}{#1~[#2]}}
    \label{feat:#1}
}
{}

\DeclareDocumentEnvironment{monsection}{o m m o}
{
    \vspace{1em}
    \begin{thesamepage}
    \setlength\multicolsep{0pt}
    \begin{multicols}{2}
        \IfValueTF{#1}
        {
            \lowercase{\hypertarget{mon:#1~#2}{}}
            \label{mon:#1~#2}
            \hypertarget{mon:#1~#2}
                {\subsection{#2,~#1}}
        }
        {
            \lowercase{\hypertarget{mon:#2}{}}
            \label{mon:#2}
            \hypertarget{mon:#2}
                {\subsection{#2}}
        }

        \columnbreak
        \begin{flushright}
            \large\textbf{Level~#3}
            \IfValueTF{#4}
            {
                \textbf{~[CR~#4]}
            }
            {
                \textbf{~[CR~1]}
            }
        \end{flushright}
    \end{multicols}
}
{
    \end{thesamepage}
}

\ExplSyntaxOff


% allow variable vertical page length
\raggedbottom

%\DisableLigatures[f,i,t]{encoding = *, family = *}

% Standard spacing definitions
% \titlespacing*{\section} {0pt}{3.5ex plus 1ex minus .2ex}{2.3ex plus .2ex}
% \titlespacing*{\subsection} {0pt}{3.25ex plus 1ex minus .2ex}{1.5ex plus .2ex}
% \titlespacing*{\subsubsection}{0pt}{3.25ex plus 1ex minus .2ex}{1.5ex plus .2ex}
\titlespacing*{\section} {0pt}{1.5ex plus 1ex minus .2ex}{0.5ex plus .2ex}
\titlespacing*{\subsection} {0pt}{1.25ex plus 1ex minus .2ex}{0.5ex plus .2ex}
\titlespacing*{\subsubsection}{0pt}{1ex plus 1ex minus .2ex}{0.5ex plus .2ex}

%Format the page margins
\addtolength{\voffset}{-0.8in}
\addtolength{\textheight}{1.5in}
\addtolength{\hoffset}{-0.6in}
\addtolength{\textwidth}{0.8in}
\setlength{\oddsidemargin}{15.5pt}
\setlength{\evensidemargin}{15.5pt}
% widen middle gap width
\setlength{\columnsep}{2em}

%Leave little extra space between tables and the surrounding text
\setlength{\intextsep}{1em}
%And between paragraphs
\setlength{\parskip}{0.1em}
%Make sure the table of contents doesn't include every little thing
\setcounter{tocdepth}{2}

\newlength\levelcol
\newlength\spellcol
\newlength\spellcolpoof
\newlength\savecol
\newlength\savecolpoof
\newlength\babcolgood
\newlength\babcolavg
\newlength\babcolpoor
\setlength\levelcol{1.75em}
\setlength\spellcol{1.1em}
\setlength\spellcolpoof{2em}
\setlength\savecol{1.5em}
\setlength\savecolpoof{2em}
\setlength\babcolgood{4em}
\setlength\babcolavg{4em}
\setlength\babcolpoor{4em}

%Create a color for use with tables
\definecolor{tbrown}{RGB}{255,240,200}
% \tabulinesep=1mm

% make lists less poofy
\setlist{noitemsep,topsep=0pt,parsep=0pt,partopsep=0pt}

% setup for border frames
\ExplSyntaxOn
\mdfsetup{
    innerleftmargin=0,
    leftmargin=0,
    innertopmargin=0,
    skipabove=0,
    innerrightmargin=0,
    rightmargin=0,
    innerbottommargin=0,
    skipbelow=0,
}

% for spell content
\mdfdefinestyle{spellcontent}{
    innertopmargin=0,
    innerleftmargin=0,
    innerrightmargin=0,
    skipabove=2pt,
    innerbottommargin=0,
    skipbelow=0,
}

% for color-only environments
\mdfdefinestyle{colorenv}{
    leftline=false,
    topline=false,
    rightline=false,
    bottomline=false,
    innerleftmargin=2pt,
    leftmargin=0,
    innertopmargin=2pt,
    skipabove=0,
    innerrightmargin=2pt,
    rightmargin=0,
    innerbottommargin=2pt,
    skipbelow=0,
}

\makeatletter
 \newcommand{\hypertargetraised}[1]{\Hy@raisedlink{\hypertarget{#1}{}}}
\makeatother

\ExplSyntaxOff

\sisetup{group-minimum-digits = 4,group-separator = {,}}

% fix headheight warning
\setlength{\headheight}{14pt}

\setlength{\RaggedRightParindent}{\parindent}


\pagestyle{fancy}
\lhead{\nouppercase\leftmark}
\chead{\bfseries \large Rise}
\rhead{\nouppercase\rightmark}
\lfoot{Kevin Johnson}
\cfoot{\thepage}
\rfoot{v. 3.5.0}

% \includeonly{monsters}

\setcounter{tocdepth}{1}

\begin{document}
\setcounter{chapter}{-1}
\tableofcontents
\chapter*{Acknowledgements}
A number of marvelous people have helped me make this revision possible, and many clever minds produced ideas that I have incorporated into the new system. I cannot hope to properly credit everyone who contributed, knowingly or otherwise, but I can at least make an attempt here. The following individuals have my gratitude:

Connor Haines, for being a great brainstorming partner and excellent roommate.

Zontan Ferrah, for frequently telling me when my ideas were stupid. Sometimes he was even right.

Desmond Henderson, for his remarkably thorough and insightful feedback, and for being the first person brave enough to run a game with Rise.

Kyle McCauley, Linchaun Zhang, and Scott Kottkamp for helping to correct foolish errors that once littered this document and for providing ideas about how to improve the system.

Tarkisflux, for his excellent \href{http://dnd-wiki.org/wiki/Tome_of_Prowess_(3.5e_Sourcebook)}{Tome of Prowess}, feedback on spell design, and more.

Kholai, for keeping me on my toes in our monstrously long discussions of rules minutiae.

Douglas Milewski, for his insightful essay on the nature of fighters. It helped inspire me to think that I could make D\&D right.

Rich Burlew, for providing a forum for house rule enthusiasts like myself to gather and for his insightful articles that guided my thought.

Shannon Carty, for the donation of her vocabulary.

Dave Rosenberg, for his insight into the nature of druids.

Ryan Beck, for his helpful commentary and criticism.

Jordan MacAvoy, for his endless patience and creativity that repeatedly revealed flaws in the wording and design of this book.

Wizards of the Coast, for making a great game and releasing it under the OGL license, which makes all of this tinkering possible.

In addition, the art for this book came from \href{https://huggingface.co/spaces/stabilityai/stable-diffusion}{Stable Diffusion} and \href{https://openai.com/dall-e-2/}{Dall-E 2}, which is very exciting! We live in the future!

The emoji images used in this book come from Twemoji. They are used and redistributed under the CC-BY-4.0 license terms offered by the \href{https://twemoji.twitter.com/}{Twemoji} project.


\chapter{Introduction}

Rise is a tabletop role-playing game.
This chapter explains what that means, and how Rise is different from other existing games.

\section{What Is A Tabletop Role-Playing Game?}
    \includegraphics[width=\columnwidth]{introduction/what is a tabletop rpg}
    In tabletop role-playing games like Rise, you play a specific character of your own design.
    Your character can try to do anything you can imagine in a world that the game master, or GM, creates.
    Of course, you won't always succeed.
    The details of your character's capabilities are defined in the pages ahead; when you're done creating a character, it will have a personality of its own, along with strengths, weaknesses, and special abilities.
    Usually, your character will go on adventures with other characters, each of which is played by other players.
    Together, you will create and experience a story with the Game Master, or GM, who defines the universe that the player characters inhabit.

    \subsection{Describing Actions}
        Most of the time, when you're playing a game of Rise, you simply describe what you want your character to do.
        For example, you can say that your character steps out of their room in the inn and walks over to knock on a friend's door.
        Although Rise has rules that could govern some aspects of that scenario, such as an Awareness check to see if your friend notices you knocking, you wouldn't usually reference those rules explicitly.
        Even in the unlikely scenario that your friend doesn't notice you knock the first time, you can just knock again, so there's no point in worrying about the details.
        If something seems reasonable, it probably is, and you don't need to worry about the fiddly bits.

        Sometimes, when you describe what your character tries to do, the action has a narratively relevant chance of failure.
        Instead of knocking on the door to say hi, you might only have time to bang on it once to warn your sleeping friend about an attack from assassins.
        In that case, there's some chance that your friend is sleeping too deeply to notice the noise the first time you knock.
        You could try knocking again, just like in the first scenario, but in this scenario that failure would cost you valuable time to survive the attack.
        In that scenario, you would roll a die to determine whether you succeed in your action - or in this case, whether your friend would succeed in their attempt to notice you.

    \subsection{Using Specific Abilities}
        Instead of describing broadly what you want to have happen, you might choose one of a list of clearly defined abilities that your character can use.
        Every character has specific abilities unique to them, such as a wizard's spells known.
        There are also a number of simple abilities that anyone can use, such as the \ability{dirty trick} or \ability{trip} abilities.
        These universal abilities attempt to adequately describe a wide variety of reasonable improvised actions that you might try to use in combat.

        Explicitly defined abilities have rules for determining what happens when you use them.
        Some abilities, such as attacks in combat, require rolling dice to determine how effective they are.
        Of course, you can use your character's abilities at any time, not just in combat.
        Abilities such as the \spell{create water} or \spell{distant hand} spells can be used to solve other kinds of problems entirely.

    \subsection{Rolling Dice}
        Eventually, you'll have to determine whether something succeeds or fails.
        This can happen as part of using a specific ability that tells you exactly what to roll, or because you tried to narrate your character taking an action that has a dramatically relevant chance of failure.
        In either case, you'll roll a single ten-sided die, also known as 1d10.
        You'll add some modifier that represents how skilled your character is at the particular thing that they are trying to do.
        At the GM's discretion, they may also give the roll an extra bonus or penalty based on the circumstances that your character is in.
        If your die roll is high enough, your character succeeds at whatever they were trying to do.
        Otherwise, your character fails, which may sometimes have additional consequences.

        In Rise, it's entirely possible for characters to be so skilled that they succeed at what they are trying to do even if you roll a 1.
        Likewise, there are tasks that are so obviously impossible for your character that they cannot possibly succeed.
        In those cases, there's no reason to roll!
        Of course, the GM is the final arbiter of whether rolling is necessary.
        They may have information that the players do not.

        In some cases, you roll multiple dice at once.
        A collection of dice is called a \glossterm{dice pool}.
        Dice pools are written with the number of dice, followed by ``d'', followed by the size of dice to roll.
        For example, 2d6 means you roll two six-sided dice.

    \subsection{Why Use So Many Rules?}
        \includegraphics[width=\columnwidth]{introduction/so many rules}

        Tabletop role-playing games attempt to create rules to define how their universe works.
        Some games are intentionally vague or minimalist about their rules, which can be fun!
        Simple games are easy to start playing, and they try to avoid getting in the way of good role-playing.
        However, Rise takes a different approach.
        It spends a lot of effort - and words - attempting to define an internally consistent universe, and creating a large number of specific abilities that can be used in that universe.
        There are a few important advantages to taking this approach: establishing expectations, supporting multiple play styles, and assisting the GM.

        \subsubsection{Establishing Expectations}
            Different people can have very different ideas about what is realistic - or narratively appropriate - in a made-up fantasy universe.
            To some people, kicking in the tavern door and starting a brawl is just some good clean fun, and you'll take a few good punches and then laugh about it later that evening over drinks.
            But to other people, that might sound like a good way to find yourself imprisoned for the foreseeable future with all of your possessions confiscated by the town guard.
            Another interpretation of that scenario might see the brawler seriously injured with a broken bottle in the eye, leaving them partially blinded for weeks - or indefinitely.

            All of those ideas are valid, and they each match the narrative of a particular type of story.
            However, it's important that everyone sitting at a table and playing a game agrees about what to expect.
            Players can get confused or frustrated when their actions have consequences that feel arbitrary or unfair.
            Generally, games are more fun if everyone in the game shares a common set of expectations and conventions.
            Otherwise, games can devolve into disagreements about what is or isn't reasonable.

            One way to establish these expectations is to use a rules system like Rise that defines some expectations explicitly.
            If the scenario above happened in Rise, the last outcome of an incapacitating bottle to the eye shouldn't normally be possible, since the rules explicitly define how injury works.
            Knowing what is and isn't possible can help give players and GMs a useful set of guardrails for what they try to do in the universe.
            It's relatively easy to get everyone to agree about simple things that regular human people have experience with, like how difficult it is to climb a tree.
            However, Rise is full of superhuman people and monsters, and eventually you'll need to figure out how far a barbarian as strong as Hercules can throw a bear.
            Having a single authoritative resource to consult can cut off long disagreements about details that are difficult or impossible to determine objectively.

            Of course, different games played with a flexible rules system like Rise can have very different tones and themes.
            Either of the first two scenarios in the tavern are still plausible in different games, and a GM can use house rules to make vital wounds have more long-term consequences if they want.
            Using a rules system like Rise can help, but it is not the full answer by itself.
            The GM and players always share responsibility for establishing expectations about what genre a game will be, and conforming to those expectations to the extent that it makes the game more fun.

        \subsubsection{Supporting Multiple Play Styles}
            \includegraphics[width=\columnwidth]{introduction/multiple play styles}
            Some people deeply enjoy the process of role-playing itself.
            They enjoy the process of getting into a character and speaking in their voice, exploring their needs and desires, and building a narrative for them over time.
            These people often do not need the confines of a robust rules system, and can play equally well in games with minimal rules or none at all.

            Other people do not enjoy role-playing as an end in itself, or even at all.
            However, they may still enjoy the \textit{game} aspect of a role-playing game.
            Instead of playing a character for their personality and backstory, they may play a character for their unique mechanics and tactical advantages.

            Still other people may be interested in role-playing as a concept, but find it daunting.
            The blank page in front of you when you start painting a picture or writing an essay can be daunting, and that first step is often the hardest to take.
            Giving people a clearly defined set of abilities and specific tools for interacting with the world can enhance creativity by providing a safe space for interaction and experimentation.
            Even if you don't enjoy or feel confident in speaking in your character's voice, you can still engage with the narrative aspects of the adventure by casting a relevant spell or making a relevant skill check.
            People in this middle ground can sometimes enjoy deeper role-playing games while being feeling lost in role-playing games with minimal or nonexistent rules.

            One of the joys - and challenges - of Rise is drawing together people with very different desires and play styles to share a single experience.
            Rules-free role-playing games and tactical wargames can both have a narrower appeal than rules-heavy role-playing games like Rise, which try to provide something for everyone.
            You can run games with deep role-players alongside tactical gamers, and it can be a lot of fun.
            It does place a greater burden on the GM to provide the right ratio of content to keep everyone happy, and it does require the players to be patient when their preferred playstyle is put in the background to support the needs of other players.
            A well-blended game can also draw people out of their comfort zones slowly and safely over time as they observe and start to enjoy the playstyles of the other players in the game.

        \subsubsection{Assisting the GM}
            The Game Master carries an extra weight of responsibility to shape the flow of the game.
            Creating narratively consistent universes, appropriate challenges, and engaging storylines out of thin air is deeply challenging.
            If this job is too difficult, no one will want to do it, and then no one will play the game!
            Making the GM's job easier is a critical component of any role-playing game.

            There are several ways that Rise can make the GM's job easier.
            It provides information about the mechanics and tropes of the universe that the game takes place in, which helps establish expectations and resolve disputes that might come up during the game.
            It will provide a clear narrative foundation for the world and the characters in that world, which minimizes the up-front work required to run a game, once that section of the book is more complete.
            It will provide a wealth of pre-packaged challenges appropriate for players of any power level or play style, and advice for how to use those challenges appropriately, once that section of the book is more complete.
            The GM-focused sections are currently the most unfinished part of Rise, and this will be a more useful guide before Rise is done.

\section{What Makes Rise Different?}
    If you haven't played other tabletop role-playing games, feel free to skip this section.
    If you have, you may wonder what makes Rise unique in a crowded sea of games.
    Rise has five fundamental principles that differentiate it from other TTRPGs: minimal resource management, simultaneous combat, optional complexity, unbounded scaling, and a bounded action economy.

    \subsection{Minimal Resource Management}
        Many games make use of resources like mana, spell slots, or timed cooldowns to limit how often characters can use their abilities.
        These systems have fundamental problems that undercut the fun and flow of a TTRPG, and Rise essentially does not use resources to limit character ability usage.
        In Rise, characters can cast spells or use special attacks any number of times in a row without consuming resources.

        Some systems have resources that are designed to ebb and flow in the course of a typical combat.
        You might expend mana to use a powerful spell, and then regain mana over time by using weaker spells or fulfilling certain conditions.
        Alternately, you might use a spell and then wait some number of in-game turns before you can use that same spell again.
        This can be fiddly to track and hard to recover from if you forget what happened to your resource pool, which is why this approach is more common in video games than in TTRPGs.
        More importantly, this system has no clear way to handle ability usage outside of combat.
        It effectively gives unlimited ability usage when time is no obstacle, but only in an awkward and convoluted way.
        This category of system is unsuitable for Rise because it is too fiddly in combat and doesn't make sense out of combat.

        Some systems have finite-use resources that are tied to the expenditure of in-game time, such as taking long rests, or session breaks.
        You might spend a spell slot to use a powerful spell, and then be unable to cast that spell again until your character rests for some period of time.
        This can be manageable from a complexity perspective if the number of unique resources is small.
        However, it can get dangerously convoluted if characters have a large number of separate or partially interchangeable resource pools, such as using separate pools for individual spell levels.

        The real problem is that this limitation requires you to make your decisions based on not just the current situation, but also on your prediction of all future situations you will encounter before you have the opportunity to rest.
        This contributes significantly to the tactical complexity of deciding each individual action in combat, which slows down the pace of the game.
        It is also punishing to newer players who have less experience with the metagaming required to deduce how many resources an individual fight is worth.
        This strategic complexity is compounded if hit points are treated as an additional resource, since you now have to trade off the potential impact of one limited resource against another limited resource.

        Optimization of resource usage can be unintuitive and out of character, but failure to correctly manage your resources can leave you with no useful abilities remaining.
        This concern can be exacerbated if some characters are extremely resource-intensive while others have no meaningful resources to track.
        No one likes being forced to hide from a difficult fight or take only insignificant actions while your more resource-savvy or resource-independent allies continue using dramatic and powerful abilities.
        It can also add stress to the party dynamics when one character frequently asks for long rests after fights because they expended resources and no one else needs to rest.
        This category of system is unsuitable for Rise because it creates complexity in ways that detract from the fun and narrative of a game instead of adding to it.

        Rise does not use resources to limit normal actions in combat.
        The vast majority of spells, special martial attacks, and other abilities that affect enemies or your environment can be used any number of times.
        There are a small number of abilities with one-round cooldowns, and a universal ability that can only be used once per short rest.
        However, there is no time tracking in the system longer than ``next round''.
        Small cooldowns are a fine-grained balancing tool that allow characters to have powerful abilities which would have detrimental effects for the game if they could be used every turn.

        Rise does use a single universal resource, called ``fatigue'', that recovers based on long rests.
        This allows some opportunity for characters to invest extra effort into specific difficult fights, and to become tired after a long day.
        Normal damage taken during a fight is easily recovered after a ten minute rest.
        This means that you typically don't have to track state between fights.
        However, a GM can prevent that rest time with multiple sequential fights to increase difficulty and drama.

        Overall, Rise uses resource limitations very sparingly.
        This allows it to gain some of their benefits while avoiding the detrimental effects that come from making resource limitations a fundamental part of the system.

    \subsection{Simultaneous Combat}
        \includegraphics[width=\columnwidth]{introduction/simultaneous combat}
        In most TTRPGs, combat takes place in a series of turns.
        When your turn comes up, you take all of your actions, and then you wait through everyone else's turn until your turn comes again.
        This system has one foundational disadvantage: it is very, very slow.
        Rise uses a simultaneous combat system that dramatically increases the pace of combat.

        Imagine a typical 4-5 player game with 1-2 enemy groups using a traditional turn-based initiative system.
        In this scenario, you have to wait through about 5 turns before it comes back to your turn.
        This number can increase significantly in large-scale fights.
        Each of those 5 or so turns can meaningfully change the battlefield situation on its own by moving, weakening, or defeating various enemies and allies.
        The state of the battlefield at the end of last turn is often drastically different than the state of that battlefield at the start of your new turn.
        Player coordination can be challenging, since they must coordinate in the specific order assigned by the initiative system, and enemy turns can intervene to ruin coordinated plans.

        In theory, every player should accurately track the unfolding battlefield state through each of the intervening turns.
        That would mean everyone would know what to do when their turn comes up.
        In practice, many players find that difficult or impossible.
        Instead, at the start of each of their turns, they ask or try to figure out how the situation has changed.
        Not everyone asks this explicitly, but it must always be analyzed anew.

        Once a player understands the current battlefield state, they can finally decide their actions.
        This typically involves both movement and any number of sequential attacks, so there are many factors to consider.
        Everyone else must wait and do nothing while this happens.
        Once the active player has decided their actions, those actions must be fully rolled and resolved before combat can proceed.
        Even the next player in the initiative order may not be able to make accurate plans during this time, since the die rolls can change those plans.
        All of this combines to make even short combats take an hour or more, and six-person adventuring groups can feel dangerously bloated.

        Rise works differently.
        Combat in Rise is broken up into two phases: the movement phase and the action phase.
        During the movement phase, all creatures move simultaneously, and no attacks are possible.
        Characters can declare certain simple reactive movements like ``stay adjacent to this enemy'' to ensure that they end up in a reasonable position regardless of enemy actions.
        If the movements of characters conflict in impossible ways, initiative checks can temporarily force a linear order of resolution.
        Each player declares their own actions in an arbitrary order as soon as they decide them, so people are not forced to wait and do nothing while slower players contemplate their choices.
        Player coordination is easy, since all actions are happening together.

        During the action phase, players resolve their actions sequentially, but in an arbitrary order of the players' choice.
        This allows slower players to make their decisions when they are ready, while allowing faster players to resolve their actions first.
        Since movement during the action phase is rare, and enemies cannot unexpectedly move, players are typically able to decide their actions much more quickly and easily even when they have a large number of unique abilities to choose from.
        Once all players have resolved their actions, they learn what their enemies did.
        Those actions all resolve simultaneously, so enemy actions cannot interrupt player actions and vice versa.
        Attackers are always responsible for rolling instead of using ``saving throws'' or similar mechanics that force defenders to roll dice.
        All of this means that players can choose and resolve their actions simultaneously and efficiently, minimizing total time spent in combat while still allowing significant tactical complexity.

        The start of each phase still requires a general assessment from all acting players about the current state of the battlefield, which takes just as much time as the assessment in a classic initiative system.
        However, the time required for this tactical analysis only increases marginally as the number of players and enemies in the game increases.
        This allows Rise to handle large player counts or large enemy hordes without becoming glacially slow.
        Combat in Rise flows by quickly, making it much easier to balance time between combat and non-combat encounters within the same game session - or to run through multiple separate, individually challenging combats without sacrificing the pace and energy of the game.

    \subsection{Optional Complexity}
        Many games operate at a consistent level of complexity.
        Many rules-light games are always simple, and many rules-dense games are always complex.
        This is a perfectly reasonable design philosophy.
        Among other benefits, it makes it easy to know what to expect from the game, which helps give the game a well-defined niche.

        Rise is designed to allow players to choose their own level of complexity.
        This broadens its potential audience by allowing people with very different play styles or tolerances for complexity to enjoy the same game together.
        This goal is manifested in several key ways in Rise's design:
        \begin{itemize}
            \item Core gameplay is designed to be simple.
            \item Character creation is deeply interconnected.
            \item Complexity is not tied to narrative roles.
            \item Character power does not require complexity.
        \end{itemize}

        \subsubsection{Simple Core Gameplay}
            The core gameplay loop must be simple.
            You can contribute in combat by relying on one or two standard attacks that you use in all circumstances.
            In narrative situations, you can just roll the skills you have trained, and ignore other options.
            Engaging with the system more deeply than that is a choice, not a requirement.

        \subsubsection{Interconnected Character Creation}
            Character creation and build optimization is a better place to store complexity.
            Creating a Rise character involves a number of decisions, each of which can have nuanced ramifications on other aspects of the system.
            If you are just trying to build a character that matches a desired narrative, you can generally approach each decision in isolation.

            For example, you can decide that your character is intelligent and agile but not very strong or durable, because that is the concept you want.
            That decision has consequences, such as changing how many trained skills you have and what your defenses are.
            If you approach each decision sequentially, each one is relatively easy to make, and doesn't require deep system knowledge.
            On the other hand, trying to mathematically optimize a character requires thinking about many aspects of the system at once.
            This results in a system that is easy to learn but hard to master.

            Even for simple characters, the process of character creation is still one of the most complicated aspects of Rise.
            That is why Rise provides (or will provide, once that section is done) an extensive selection of premade characters for a wide variety of narrative archetypes.
            Each premade character includes advice for how to play that character and level them up.
            The premade characters make the system more accessible to people who don't want to to deal with the complexity of creating a character from scratch.

        \subsubsection{Complexity and Narrative}
            Complexity and simplicity should not be directly connected to a character's concept or narrative.`
            For example, it would be a bad idea to define a system where martial characters are simple and spellcasters are complicated.
            Both of those are rich and evocative narrative constructs.
            Many people who don't enjoy complexity will want to play spellcasters, and many people who enjoy complexity will want to play martial characters.
            Gameplay complexity must be more finely tuned and localized than those sweeping strokes.

            In Rise, gameplay complexity is generally generated by acquiring a large number of increasingly situational abilities.
            Every class has some archetypes that grant additional abilities known and some archetypes that grant additional passive abilities.
            If you like having a lot of unique abilities, you can have a high Intelligence to maximize your insight points, and focus on learning spells and maneuvers that attack your enemies or have situational effects.
            If you like minimizing complexity, you can instead choose archetypes or learn spells that simply grant you passive benefits, and focus on one or two standard attacks that you specialize in.
            Some feats give you new abilities and new circumstances to pay attention to that make you more effective, while others simply increase your passive statistics and defenses.

            Rise specifically handles complexity for martial characters and spellcasters slightly differently.
            Martial characters in Rise typically have fairly simple individual abilities.
            However, they can use those abilities with a variety of meaningfully different weapons.
            A martial character with four unique attacks and three different weapons has twelve different options in combat.
            In addition, martial characters can typically make better use of universal abilities, such as shoving and grappling.

            Spellcasters have more complex and varied individual abilities.
            They also tend to have more abilities that have significant narrative effects.
            However, their abilities are more isolated.
            There is no spellcaster equivalent of martial weapons that would multiply their number of distinct abilities in combat.
            The result of this design is that both martial characters and spellcasters can be very simple or very complicated.
            However, they approach complexity in different ways, ensuring that they feel narratively distinct.

        \subsubsection{Complexity and Power}
            All of this customization of complexity would be mostly pointless if complexity was strongly correlated with character power.
            If exceptionally complicated or hyper-specialized characters were obviously and consistently more effective than other characters, it would push everyone to use those characters.
            Rise structures the tradeoffs between gaining raw power and gaining additional options balanced enough that neither is always superior.

            There will always be some benefit from build optimization and system mastery.
            Players who are deeply familiar with Rise will be able to build characters with more relevant strengths and fewer relevant weaknesses.
            However, the gap between optimized characters and ``normal'' characters is limited.
            There will always be specific contexts where one character's mechanics are superior to another's.
            For example, a specialized defensive melee character may excel in a duel in a confined space.
            However, it may be irrelevant against cavalry archers on an open field.
            Characters in Rise cannot drastically change their capabilities each day, so they will always have moments to shine and moments of weakness.

    \subsection{Unbounded Scaling}
        \includegraphics[width=\columnwidth]{introduction/unbounded scaling}
        Some systems uses bounded bonuses for accuracy or other game statistics.
        Bounded scaling means that every character of the same power level - or in some systems, of any power level - has a similar chance of success with any given skill check or attack roll.
        This can frequently cause narratively inappropriate and even comical events, and Rise explicitly rejects this philosophy.

        Imagine a typical party of four players, with one character being exceptionally skilled at a particular task.
        Perhaps the rogue is exceptionally skilled at lying, or a barbarian is exceptionally skilled at climbing.
        If ``exceptionally skilled'' only means that they have a \plus5 bonus on a d20 compared to \plus0 from the rest of the party, the exceptionally skilled character will only get the best result in the party half the time.
        The other half of the time, some other character with no relevant skills will meet or exceed the skilled character's result - sometimes by a dramatic margin.
        When failure compared to rank amateurs happens this often, it becomes hard to take seriously the idea that any character can be exceptionally skilled at anything.

        Rise characters can have dramatic statistical differences between each other, even at low levels.
        It uses a d10 as the fundamental die, which makes every bonus more significant.
        In addition, a 1st-level character can easily reach a \plus6 bonus with a skill check that is particularly relevant to their character.
        This means that a skilled character can beat a party of rank amateurs 80\% of the time, and at higher levels their success becomes completely guaranteed.
        Likewise, the difference in Mental defense between a powerful sorcerer and a cowardly rogue can allow mind-affecting attacks to almost always hit a rogue while almost never hitting the sorcerer.
        These statistical differences do not always grow with level, but they remain significant at every level.

        One advantage of systems with bounded scaling is that it is easier to guarantee that every character is relevant in any situation.
        Even if your character has no useful abilities of any kind, you might sometimes succeed on important actions through sheer luck.
        However, this design philosophy often breaks the symmetry between magical and non-magical characters.
        Magical characters can often use extremely specific and powerful abilities that are impossible for nonmagical characters to duplicate.
        If magical characters also have similar odds of success with all generic mechanics of the game, they will almost certainly have far more influence over the narrative of the game than any nonmagical character can hope to match.

        The philosophy of Rise is that it's okay for some characters to be irrelevant in specific contexts.
        It's good to give people time in the spotlight where their character's abilities help solve the specific problem that the group is facing when no other character could.
        Rise encourages that, and makes it impossible for one character to be relevant in \textit{all} contexts.
        Each character has their own strengths and weaknesses, and if you try to be good at everything, you'll fall behind people who specialize in a particular area.
        This will naturally rotate the spotlight between different characters, allowing each player to feel relevant and important in turn.

        This dramatic scaling is also used to govern the power of characters over time, in addition to the power of characters relative to each other.
        Rise attempts to model a massive power range for player characters.
        They are expected to start their journeys at level 1 as little more than commoners, and by level 21 they are effectively demigods who can alter the fate of entire worlds.
        This is a critical part of the narrative fabric of Rise, and it is reflected in the statistics and abilities of characters.
        If a level 1 kobold posed even a tiny threat to a level 21 character, the mechanics of the game would sabotage the purported narrative of power and growth.
        In Rise, overall character power doubles approximately every two to three levels.
        The system takes some care to avoid bloating numbers to unwieldy levels on this journey, and the use of the d10 as the standard die helps immensely.

    \subsection{Bounded Action Economy}
        It is dangerous to to give characters too many actions each turn.
        Each additional action a character can take increases how difficult it is for a player to decide what to do on their turn.
        In addition, each additional action increases the complexity of the change between the start of the turn and the end of the turn.
        This is especially risky with Rise's simultaneous initiative system, which combines the actions taken by all characters into a single resolution process.

        Rise places significant limitations on how many relevant actions each character can take on their turn.
        Generally, characters can only move during the movement phase and then take one significant action each turn.
        Some characters can use a minor action to accomplish something useful.
        However, that essentially marks the end of action economy scaling, even up to the maximum level.

        Detrimental effects that could deny actions are also heavily limited.
        Total action denial effects are only usable by high level characters, and even then they only work against weak enemies or enemies that have already been significantly damaged.
        Taking actions is fun, and sitting quietly while everyone else does things can be very frustrating.
        Similarly, completely removing an enemy's ability to act can easily remove the tension from a fight before it's actually over.


\chapter{Adventuring}\label{Adventuring}

This chapter describes how to play Rise outside of combat situations.
It also provides general advice about how Rise games are typically run.
For information about how characters are defined, including their statistics, see \pcref{Characters}.
To run combat scenarios, see \pcref{Combat}.

\section{Defining the Undefined}
    This book does not attempt to include specific rules for every aspect of a realistic world.
    Unless defined otherwise - or if it's not worth the effort to look up Rise's exact rules in the flow of a game - you should assume that the universe works more or less like the real world does, and as long as everyone agrees that something is reasonable, it's not worth worrying about in more detail.

    For example, Rise does not have specific rules for how long it takes to eat a meal, the arc that a thrown ball takes through the air, or how much extra weight a well-made chandelier can hold without breaking.
    It's possible to imagine situations where each of those might be important to a game, however, so you'll have to guess what would be reasonable as obscure situations arise.
    The Game Master has the final word when defining ambiguities like this.

    \subsection{Resolving Ambiguity}\label{Resolving Ambiguity}
        When the rules are ambiguous about how they apply to you and no other creature, you decide how to resolve that ambiguity.
        For example, if an ability causes you to remove one of your \glossterm{vital wounds}, and you have more than one vital wound, you choose which vital wound is removed.
        When the rules are ambiguous in any other situation, the GM decides how to resolve that ambiguity.
        This includes situations where multiple creatures are relevant and situations where no particular creature is relevant.

\section{Making Checks}\label{Checks}\label{Making Checks}
    Checks are required to perform actions that have a chance of failure where the difficulty is not measured by the defense of another creature or object.
    For example, climbing a wall or remembering an obscure piece of trivia may require a check.

    To make a check, roll 1d10 and add your modifier with the check.
    You compare that result to a \glossterm{difficulty value} that represents the difficulty of the task.
    The more difficult the task, the higher the \glossterm{difficulty value} will be.
    If your result is equal to or higher than the \glossterm{difficulty value}, the check succeeds.
    This usually means you accomplish a task successfully.
    Otherwise, the check fails.
    This usually means that nothing happens, though sometimes there are specific consequences for failure.

    \subsection{Critical Success}
        If your check result is at least 10 higher than the \glossterm{difficulty value}, your check is a \glossterm{critical success}.
        Some checks have a special effect on a critical success.
        For example, a critical success while climbing means you move twice as quickly (see \pcref{Climb}).

    \subsection{Standard Difficulty Values}\label{Standard Difficulty Values}
        Most checks are made against a fixed \glossterm{difficulty value} that represents how hard the task is.
        Detailed rules for determining difficulty values in specific circumstances can be found in the Expanded Skills chapter from the Tome of Guidance.
        However, most of the time, it's not worth the effort to consult charts and tables to figure out how hard a task is.
        Instead, you can estimate it based on the guidelines below.

        \begin{itemize}
            \item DV 0 - Easy: Only an exceptionally incompetent or impaired person could possibly fail a DV 0 check. For example, this includes walking on rough ground without tripping (Balance) or noticing that a yelling, red-faced person is angry (Social Insight).
            \item DV 5 - Average: A typical human with no relevant skills should still succeed at a DV 5 check without much issue. However, it would be possible to fail in a stressful situation where time is limited if the person had no relevant training. For example, this includes climbing a ladder (Climb) or hearing the topic of a nearby conversation in a crowded bar (Awareness).
            \item DV 10 - Hard: A typical human with no relevant skills might succeed at a DV 10 check, but only if they were very lucky or had a lot of time on their hands. An experienced practioner might fail infrequently in stressful circumstances, but a world-class expert would never fail. For example, this includes swimming in fast-moving water (Swim) or providing first aid to mitigate a barely lethal wound (Medicine).
            \item DV 15 - Very Hard: Only an experienced practioner could succeed at a DV 15 check, and they would still need to get lucky if they were in a rush. Even a world-class expert at the peak of real-world human potential could fail, but only rarely. For example, this includes picking a well-made lock (Devices) or holding your breath for eight minutes while staying still (Endurance).
            \item DV 20 - Almost Impossible: A world-class expert like an Olympic medalist could succeed at a DV 20 check if they were lucky or patient. Succeeding consistently at tasks of this difficulty requires superhuman capabilities. For example, this includes climbing a weathered natural rock wall without equipment (Climb) or squeezing through a space with a diameter of only half a foot (Flexibility).
            \item DV 25\add - Impossible: No real-world human can succeed at a DV 25 check. This sort of feat is only possible for high-level Rise characters who have explicitly surpassed ordinary limitations. For example, this includes running at full speed along a slack rope (Balance) or climbing a sheer glass pane (Climb).
        \end{itemize}

    \subsection{Trying Again}
        You can think of checks as being broadly divided into two categories: checks that give you information, and checks that cause a change in the world around you.
        In general, you can retry checks that change your environment indefinitely until you succeed.
        The only major limiting factor to those checks is that failure sometimes also changes your environment in ways that may punish your failure or make it impossible to retry the check.
        For example, if you are trying to climb a cliff, you can keep trying until you succeed, but you may take \glossterm{falling damage} from falling off while halfway up the cliff.

        You generally cannot retry checks that give you information unless the situation changes in a way that is relevant to your check.
        This often takes the form of giving you new information.
        For example, if you've already examined a creature to determine whether they are disguised, you can't keep just keep staring that creature to make sure.
        However, if you splash the creature with water which washes away some makeup, you can try again now that you have more information.

        % TODO: it would be nice if this wasn't necessary
        In addition, checks that require a free action to make can never be made more than once for the same purpose within a round.

    \subsection{Opposed Checks}
        An opposed check involves multiple creatures competing to get the highest result.
        In case of a tie, all tied creatures roll again to break the tie.
        Usually, the creature with the highest result succeeds, while all other creatures either fail completely or simply succeed less effectively depending on the situation.

        Some opposed checks involve multiple creatures using the same skill to see who does the best job.
        For example, a climbing race up a wall might involve each participant rolling a Climb check, or you might make a Strength check to hold a door closed while another creature tries to shove it open.
        Alternately, it can involve creatures rolling opposite skills.
        For example, if you are trying to hide, you roll a Stealth check opposed by the Awareness check of any creatures who could notice you.

        Not all opposed checks require all participants to roll at the same time.
        For example, a creature who creates a disguise rolls the Disguise check at the time that the disguise is created.
        A creature who tries to notice the disguise would roll their Awareness check at the time they see the disguised creature.

    \subsection{Hidden Checks}
        The GM can always make checks on your character's behalf without telling you.
        Generally, this is used for observation-based skills.
        For example, it's very suspicious if the GM tells you to make an Awareness check and then tells you that you don't see anything interesting.
        One of the ways a GM can avoid that is by simply rolling a check on behalf of your character and only telling you the result if you succeed.

    \subsection{Helping On Checks}
        You can help an \glossterm{ally} make a check.
        To help an ally, you make a check of the same type against a \glossterm{difficulty value} that is 5 lower than the regular difficulty value.
        This has the same requirements, including time and physical contact, as the check would have if you made it yourself.
        For example, to help an ally climb a cliff, you must be able to touch your ally to guide them up.
        Success means that the ally gains a \plus2 bonus to the check.

        Multiple creatures can try to help the same person.
        At the GM's discretion, there may be a practical limit to how many people can assist with the same task.
        The bonus from multiple creatures helping does not stack.
        It just makes it more likely that the helping attempt will succeed.

    \subsection{Checks for Timed Tasks}
        For every 5 points by which you beat the \glossterm{difficulty value} to accomplish a timed task, the time required is usually halved.
        This only applies for tasks that have a base time requirement of at least one minute, if the GM agrees that it is relevant, and if there are no other specific ways in which your result is improved with higher check results.

\section{Resting}\label{Resting}
    \includegraphics[width=\columnwidth]{core mechanics/resting}

    When you have a moment to relax, you can rest to regain some of your expended resources.
    There are two main types of rests: a \glossterm{short rest} and a \glossterm{long rest}.
    Resting is not actually an ability in the same sense as most other abilities.
    You do not declare that you are using the ``short rest'' ability, and you do not have to differentiate between whether you intend to take a short rest or a long rest.
    The benefits of taking a short rest or long rest happen automatically after you spend enough time avoiding strenuous activity.
    Resting at night is often combined with sleeping, but you can rest at any time without sleeping.

    % TODO: clarify interrupted rests

    \subsection{Short Rest}\label{Short Rest}
        Resting for ten minutes is considered a \glossterm{short rest}.
        When you finish a short rest, you gain the following benefits.
        \begin{itemize}
            \item Your \glossterm{hit points} become equal to your maximum hit points.
            \item Your current \glossterm{damage resistance} becomes equal to your maximum damage resistance.
            \item You regain any \glossterm{attunement points} you released from \glossterm{deep attunement} effects (see \pcref{Deep Attunement}).
            \item You remove all \glossterm{conditions} affecting you (unless they cannot be removed normally).
            \item Some other abilities have specific effects that last until you finish a short rest.
        \end{itemize}

    \subsection{Long Rest}\label{Long Rest}
        Resting for eight hours is considered a \glossterm{long rest}.
        When you finish a long rest, you gain the following benefits.
        \begin{itemize}
            \item You remove one of your vital wounds (see \pcref{Removing Vital Wounds}).
                The Medicine skill can increase this healing (see \pcref{Accelerate Recovery}).
            \item Your \glossterm{fatigue level} becomes 0.
            \item Some other abilities have specific effects that last until you finish a long rest.
        \end{itemize}

        You can take multiple long rests consecutively to recover from extensive vital wounds.

    \subsection{Sleep and Fatigue}\label{Sleep and Fatigue}
        A typical creature needs a minimum of 6 hours of sleep for every 18 hours spent awake, and a minimum of 50 hours of sleep every week.
        You can stay awake beyond those limits with the Endurance skill (see \pcref{Stay Awake}).

\section{Communication and Languages}\label{Languages}\label{Communication and Languages}

    \parhead{Literacy}
    All characters with an Intelligence of \minus2 or higher are presumed to be literate, allowing them to read and write any language they speak. Each language has an alphabet, though sometimes several spoken languages share a single alphabet.

    \parhead{Language Rarity}\label{Language Rarity}
    Some languages are widely spoken in the world, while others are only encountered in unusual circumstances.
    Common languages are summarized on \trefnp{Common Languages}, below.
    Rare languages are summarized on \trefnp{Rare Languages}, below.
    Rare languages are more difficult to learn, and are usually only spoken by unusual creatures.

    \parhead{Learning Languages}\label{Learning Languages}
    Learning a language is a time-consuming process, and most characters only know a few languages based on their species.
    You can learn two common languages or one rare language in place of training a skill (see \pcref{Skills}).
    In addition, you can talk to your GM about knowing an additional language based on your character's personal background.

    \begin{dtable}
        \lcaption{Common Languages}
        \begin{dtabularx}{\columnwidth}{l >{\lcol}X l}
            \tb{Language} & \tb{Typical Speakers} & \tb{Alphabet} \tableheaderrule
            Common        & Civilized creatures   & Common   \\
            Draconic      & Dragons, kobolds      & Draconic \\
            Dwarven       & Dwarves               & Dwarven  \\
            Elven         & Elves                 & Elven    \\
            Giantish      & Ogres, giants         & Dwarven  \\
            Gnoll         & Gnolls                & Common   \\
            Gnome         & Gnomes                & Dwarven  \\
            Goblin        & Goblins, hobgoblins   & Dwarven  \\
            Halfling      & Halflings             & Common   \\
            Orcish        & Orcs                  & Dwarven  \\
        \end{dtabularx}
    \end{dtable}

    \begin{dtable}
        \lcaption{Rare Languages}
        \begin{dtabularx}{\columnwidth}{l >{\lcol}X l}
            \tb{Language}  & \tb{Typical Speakers}  & \tb{Alphabet} \tableheaderrule
            Abyssal     & Evil planeforged      & Abyssal  \\
            Aquan       & Water-based creatures & Elemental \\
            Auran       & Air-based creatures   & Elemental \\
            Celestial   & Good planeforged      & Celestial \\
            Ignan       & Fire-based creatures  & Elemental \\
            Sylvan      & Dryads, faeries       & Elven     \\
            Terran      & Earth-based creatures & Elemental \\
            Undercommon & Drow                  & Elven
        \end{dtabularx}
    \end{dtable}

\section{Ability Mechanics}\label{Ability Mechanics}

    \subsection{Magical and Mundane Abilities}\label{Magical and Mundane Abilities}

        There are two types of abilities: magical abilities and mundane abilities.

        \parhead{Magical Abilities}\label{Magical Abilities} A \magical ability is an ability fundamentally composed of or fuelled by magic.
        Magical abilities often have effects that would be impossible without magical intervention.
        Examples include \glossterm{spells}, a dragon's ability to fly, and a paladin's ability to smite foes.
        Abilities that are magical in nature are indicated with a \sparkle in their name.
        Abilities that are not magical are \glossterm{mundane}.

        \parhead{Mundane Abilities}\label{Mundane Abilities} A \glossterm{mundane} ability has some form of natural explanation and does not fundamentally originate from a magical source.
        Examples include weapon attacks, a dragon's frightful presence, and a barbarian's rage.
        Unless otherwise indicated, all abilities are mundane in nature.
        Abilities that are not mundane are \magical.

    \subsection{Targets}\label{Targets}
        Almost all abilities affect targets.
        A target of an ability is a creature directly affected by the ability in some way.
        Many abilities affect targets within a specific \glossterm{range}.

        \subsubsection{Targeted Abilities}\label{Targeted Abilities}
            Some abilities allow you to choose specific targets.
            There can be restrictions on the targets of the ability, such as ``a creature or object'' or ``an \glossterm{ally}''.
            These abilities are called \glossterm{targeted} abilities.

        \subsubsection{Area Abilities}
            Some abilities affect all valid targets within a given area.
            There can be restrictions on the targets of the ability, such as ``all creatures'' or ``all \glossterm{enemies}''.
            However, you cannot individually choose to include or exclude specific targets.
            These abilities are not \glossterm{targeted} abilities.

        \subsubsection{Invalid Targets}
            % clarify timing
            You can always attempt to use an ability on an invalid target.
            If the target is still invalid when the ability resolves, the ability automatically fails and has no effect on the target.

        \subsubsection{Primary and Secondary Targets}\label{Primary and Secondary Targets}
            Some abilities that affect multiple targets distinguish between their primary and secondary targets.
            For example, the \spell{chain lightning} spell affects secondary targets within a small radius around a primary target.
            If an ability does not mention secondary targets, all of its targets are primary targets.
            Unless otherwise specified, abilities have the same effect on their primary and secondary targets.
            However, some targeting rules are different between the two.

            First, \glossterm{line of effect} for secondary targets is always measured from the primary target, rather than from the ability's source.
            However, \glossterm{line of sight} is still measured from the ability's source.
            This can allow you to hit secondary targets behind walls if you can still see them or otherwise target them, and if there is no wall separating from the primary target.

            Second, secondary targets use the same \glossterm{longshot penalty} as the primary target, even if they are farther away.
            
    \subsection{Range}\label{Range}
        Many abilities can only affect targets or areas within a given \glossterm{range} of you.
        For abilities that affect specific targets, all targets must be within the range.
        For abilities that affect an area within a range, the area's \glossterm{point of origin} must be within the range (see \pcref{Point of Origin}).
        There are five common ranges: \shortrange, \medrange, \longrange, \distrange, and \extrange.
        Unless otherwise noted, all abilities with a range require both \glossterm{line of sight} and \glossterm{line of effect} to the point of origin or to all targets.

    \subsection{Touch}\label{Touch}
        Some abilities specify that you must touch a target.
        You can only touch creatures that are adjacent to you.
        Touching a creature that is not an \glossterm{ally} requires an attack against the target's Reflex defense, which is usually mentioned as part of the ability's description.
        Some creatures cannot be touched, such as \glossterm{incorporeal} creatures.

    \subsection{Area}\label{Area}

        Some abilities affect targets within an area.
        All areas have a \glossterm{point of origin}, an area shape, a measurement of their size in feet, and an area type (see \pcref{Point of Origin}).

        \subsubsection{Area Shapes}\label{Area Shapes}

            \parhead{Cone} A cone extends from the point of origin in a quarter-hemisphere, up to the given length.
            A square is affected by a cone if it is within the cone's 90 degree arc and all of the square's points of intersection are no more than the cone's length away from the cone's point of origin.

            \parhead{Cylinder} A cylinder extends out from the point of origin in a circle, up to the given radius.
            Cylinders also have a specific height.
            Unless otherwise specified, a cylinder's height is the same as its radius.
            Cylinders ignore obstacles that partially block line of effect, as long as there is a path around the obstacle that lies entirely within the ability's area.

            \parhead{Line} A line extends from the point of origin in a straight line, up to the given length.
            Lines also have a specific width and height.
            Unless otherwise specified, a line-shaped ability affects an area 5 feet wide and 5 feet high.
            The affected squares are chosen such that they stay close to the chosen line as possible.
            All squares affected by a line must be contiguous, so every square is adjacent to another affected square, disregarding diagonals.

            If a line-shaped effect has its area increased, only the length of the line increases unless otherwise noted.

            \parhead{Sphere} A sphere extends from the point of origin in all directions.
            Any ability which only specifies a radius for its area is sphere-shaped.

            \parhead{Wall} A wall is like a line, except that its width is not defined in squares.
            Narratively, all walls have a nonzero width.
            Mechanically, walls are considered to have no width and simply occupy the boundary between squares.
            Like lines, some walls are shapeable.

            All walls share the following common properties unless their description says otherwise.
            A wall's height is equal to half its length for straight walls, or half its radius for circular walls, to a minimum of 10 feet high.
            The entire wall is considered to be a single object, and is attacked and destroyed as a single unit.
            All of a wall's defenses are 0, but like other objects, they are immune to \glossterm{critical hits}.
            Most abilities that create walls indicate how many hit points the wall has.
            If an ability does not specify a wall's hit points, it does not have hit points and cannot be destroyed with damage.

            If you create a wall within a space too small to hold it, it fills as much of the space as possible, starting from the middle of the chosen space.
            This can allow you to completely block off small tunnels.

            Walls can normally be created within or adjacent to occupied squares, but not within solid objects.
            If a wall has hit points, it cannot be created inside the space of a single creature, but it can be created between two adjacent creatures.

            % \parhead{Specific Shapes} Some abilities specify a series of volumes that make up the area of the ability.
            % Most commonly, the volumes are cubes.
            % You may arrange the volumes as you want, with the restriction that each volume in the ability's area must be adjacent to one other volume in the ability's area.

        \subsubsection{Area Size}

            The area affected by many abilities falls into one of six sizes.
            Each size defines the extent to which the ability extends out from its origin, whether as a radius or as a length.
            Many abilities have specific sizes, as given in the ability description.

            \parhead{Tiny} Tiny areas extend 5 feet from their point of origin.
            \parhead{Small} Small areas extend 15 feet from their point of origin.
            \parhead{Medium} Medium areas extend 30 feet from their point of origin.
            \parhead{Large} Large areas extend 60 feet from their point of origin.
            \parhead{Huge} Huge areas extend 120 feet from their point of origin.
            \parhead{Gargantuan} Gargantuan areas extend 240 feet from their point of origin.

        \subsubsection{Area Types}\label{Area Types}

            \parhead{Burst} A burst ability has an immediate effect on all valid targets within an area.
            If an ability does not explicitly specify its area type, it is normally a burst effect.
            However, abilities that create wall-shaped areas are always zones.

            \parhead{Emanation} An emanation ability has effects within an area for the duration of the ability.
            It emanates from a specific creature or object, rather than a location.
            If that creature or object moves, the emanation moves with it.

            \parhead{Zone} A zone ability has effects within an area for the duration of the ability.
            Unless otherwise noted, it does not move after being created.

            When casting an area ability, you select the point where the ability originates.
            The point of origin of a ability is always a grid intersection.
            When determining whether a given creature is within the area of a ability, count out the distance from the point of origin in squares just as you do when moving a character or when determining the range for a ranged attack.
            The only difference is that instead of counting from the center of one square to the center of the next, you count from intersection to intersection.

            % This seems like an unnecessary complication, and is easily forgotten.
            % But it does solve some potentially unintuitive scenarios, like upgraded versions of spells not always being strictly better.
            % You can freely decrease a ability's area, provided that you decrease it uniformly across all of the ability's dimensions.
            % For example, you can cast a \spell{fireball} spell that affects a 5 foot radius if you choose to do so, but you can't cast a \spell{fireball} with any shape other than a sphere.

            You can count diagonally across a square, but remember that every second diagonal counts as 2 squares of distance.
            If the far edge of a square is within the ability's area, anything within that square is within the ability's area.
            If the ability's area only touches the near edge of a square, however, anything within that square is unaffected by the ability.

    \subsection{Ability Durations}\label{Ability Durations}

        An ability's duration determines how long its effect lasts.
        Abilities can have one of several different kinds of durations.

        % TODO: Is this necessary?
        % Should this clarify interactions with bursts/zones/emanations?
        If an ability targets creatures or objects directly, the effects travel with the subjects for the ability's duration, even if the subjects go outside the ability's initial range.
        If an ability creates or summons objects or creatures, they last for the duration of the ability, and are capable of moving outside the ability's initial range.
        Such effects can sometimes be destroyed prior to when their duration ends.

        \subsubsection{Attuned Abilities}
            Many abilities last as long as a creature \glossterm{attunes} to them.
            For details, see \pcref{Attunement}.

        \subsubsection{Conditions}\label{Conditions}
            Many abilities impose \glossterm{conditions} on their targets.
            A condition lasts until it is removed.
            You can remove conditions by taking a \glossterm{short rest} or using the \textit{recover} ability (see \pcref{Recover}).
            There are several other abilities that can also remove conditions.

        \subsubsection{Sustained Abilities}\label{Sustained Abilities}
            Sustained abilities have the \abilitytag{Sustain} tag.
            They last as long as you take an action to sustain them each round.
            The type of action required is always specified in the ability's tag, such as ``Sustain (standard)'' for a standard action, or in the ability's description.

            At the start of each \glossterm{action phase}, the ability is dismissed unless you take the appropriate action to sustain the ability.
            This happens before your normal turn, so you and your allies can't gain the benefits of a sustained ability without you sustaining it.

            Some sustained abilities include ``attuneable'' in the tag before the action type.
            When you use or sustain that ability, you can choose to \glossterm{attune} to it.
            If you do, it gains the \abilitytag{Attune} tag and loses the \abilitytag{Sustain} tag, so it stays active as long as you stay attuned to it.

            Taking an action to sustain an ability only allows you to sustain a single use of that ability.
            However, you can sustain multiple separate abilities at once if you have available actions.

            You can normally only sustain an ability for up to 5 minutes.
            After that time, the ability's effect is \glossterm{dismissed}.

        \subsubsection{Permanent Abilities}
            Some abilities last permanently.
            Such abilities never expire on their own, but can be \glossterm{dismissed} or removed by other abilities appropriately.

    \subsection{Combining Effects}
        Abilities do not generally affect the way another abilities function.
        However, sometimes multiple effects can be in conflict on a creature.
        If one effect makes another effect irrelevant or impossible, the latter effect is ignored.
        If two effects both conflict with each other, the most recent effect takes precedence, and the other is ignored.
        Unless otherwise noted, two different uses of the same ability are always considered to be conflicting with each other.

        All abilities will still have as much of their effect as possible.
        It is possible for an ability to be partially effective in this way.

    \subsection{Suppressing Abilities}\label{Suppressing Abilities}
        Abilities can be \glossterm{suppressed} by effects such as the \spell{suppress magic} spell.
        While an ability is suppressed, it has no effect.
        However, if it stops being suppressed, its effects continue as if they had not been interrupted.

    \subsection{Ability Tags}
        Many abilities have tags that describe the nature of the ability.
        Many of these tags have no game effect by themselves, but they govern how the ability interacts with spells, other abilities, unusual creatures, and so on.
        For a list of ability tags, see \pcref{Ability Tags}.

\section{Spell and Ritual Mechanics}\label{Spell and Ritual Mechanics}
    \includegraphics[width=\columnwidth]{core mechanics/spell and ritual mechanics}

    Spells and rituals share many common properties, defined here.

    \subsection{Spells}\label{Spells}
        % TODO: better description
        A \glossterm{spell} is a discrete magical effect with a name, a \glossterm{rank}, and an effect.
        Each \glossterm{mystic sphere} has a number of spells associated with it.
        An ability that gives you access to \glossterm{mystic spheres} will define how many spells you know.
        A spell's \glossterm{rank} is the minimum \glossterm{archetype rank} you must have in the relevant spellcasting archetype to be able to learn and cast the spell.

        \subsubsection{Cantrips}\label{Cantrips}
            Some \glossterm{mystic spheres} have minor spells called cantrips.
            Anyone who has access to a mystic sphere knows all cantrips from that sphere.

    \subsection{Rituals}\label{Rituals}
        \includegraphics[width=\columnwidth]{core mechanics/rituals}

        Each \glossterm{mystic sphere} has a number of \glossterm{rituals}.
        Some spellcasting characters can learn and perform rituals.
        Rituals are ceremonies that create magical effects.
        Like spells, each ritual has a name, a \glossterm{rank}, and an effect.
        Although rituals are similar to spells, abilities that affect spells do not affect rituals unless they say they do in their descriptions.
        % Oddly located
        A ritual's \glossterm{rank} is the minimum \glossterm{archetype rank} you must have in the relevant spellcasting archetype to be able to learn and perform the ritual.

        You don't memorize a ritual as you would a normal spell.
        Rituals are too complex for all but the most knowledgeable sages to commit to memory.
        To perform a ritual, you need to read from a book or a scroll containing it.

        \subsubsection{Rituals and Mystic Spheres}
            You must have access to the \glossterm{mystic sphere} a ritual is from in order to learn and perform the ritual efficiently.
            It is possible to learn and perform rituals from mystic spheres you do not have access to, but this comes with several requirements and limitations:
            \begin{itemize}
                \item Your \glossterm{magic source} must include the mystic sphere that the ritual belongs to.
                \item You treat the ritual as if it was one rank higher than its actual rank.
                    This means you must have access to higher rank spells to perform it, and the scribing cost is five times more expensive.
                \item The ritual requires double the normal \glossterm{fatigue level} to perform.
            \end{itemize}

        \subsubsection{Ritual Descriptions}
            % TODO: proper chapter references
            Rituals are described in the body of the \glossterm{mystic sphere} they are associated with, following the description of spells from that mystic sphere.

        \subsubsection{Scribing Rituals}
            A ritual book contains one or more rituals that you can use as frequently as you want, as long as you can spend the time and \glossterm{fatigue level} to perform the ritual.
            Scribing a ritual costs precious magical ink with a value equal to an item of the ritual's rank (see \tref{Item Ranks}).

        \subsubsection{Performing Rituals}
            To perform a ritual, you must have a ritual book containing the ritual and the material components required for the ritual.
            Some rituals cause the creatures performing them to increase their \glossterm{fatigue level}, as indicated in their descriptions.
            Other creatures can suffer this fatigue to help you perform rituals; see Ritual Participants, below.

            % The fatigue level cost for 24 hour rituals is equal to (ritual rank ^ 2) * 2.
            % Should this be specified explicitly?

        \subsubsection{Ritual Participants}
            Creatures can assist in the performance of rituals even if they are unable to perform rituals themselves.
            A creature that helps perform a ritual is called a ritual participant, and the creature performing the ritual is called the ritual leader.
            A ritual participant may increase their \glossterm{fatigue level} in place of or in addition to the fatigue level gained by the creature performing the ritual.
            If multiple creatures are willing to increase their fatigue level or attune to effects, the ritual leader decides which creatures increase their fatigue level or attune to the ritual's effects.

            The steps required to participate in rituals can be complex.
            Ritual participants must be given specific instructions for the actions they must perform during a ritual by a creature who knows how to perform the ritual.
            This instruction generally takes one tenth of the time required to perform the ritual.
            A creature cannot participate in rituals unless it has an Intelligence of at least 0, can speak at least one language, and has the fine motor control required to perform the \glossterm{somatic components} of rituals.

            Normally, a ritual participant can only contribute \glossterm{fatigue levels} up to a maximum of their \glossterm{fatigue tolerance}.
            If the participant has access to the same \glossterm{magic source} as the ritual, they can contribute any number of \glossterm{fatigue levels} (until they drop unconscious).
            Creatures willing to fatigue themselves generally tire at a rate no faster than one fatigue level per ten minutes spent performing the ritual.

            \parhead{Changing Ritual Participation}
            Rituals are deeply complex magic, and they cannot be abandoned or paused partway through.
            If the number of ritual participants in a ritual decreases below its initial value, the ritual fails at the end of the next round if the number of participants is not restored.
            However, ritual participants can transfer their participation to other creatures without disrupting the ritual.

            In order to transfer ritual participation, the new creature must be able to participate in the ritual.
            Similarly, the ritual leader can transfer their leadership to another creature.
            In addition to the requirements for transferring ritual participation, the new leader must know the ritual and be able to perform it themselves.

            Changing ritual participation and leadership is usually done when performing extraordinarily long or demanding rituals.

            \parhead{Attunement Rituals}
                Rituals with the \abilitytag{Attune} tag require a single ritual participant to \glossterm{attune} to the ritual's effect.
                Any ritual participant can attune to the effect, but only one ritual participant can attune to the effect unless otherwise noted in the ritual's description.
                For details, see \pcref{Attuned Abilities}.

        \subsubsection{Magical Writings}
            To record a spell in written form, a character uses complex notation that describes the magical forces involved in the spell.
            The notation constitutes a universal language that spellcasters have discovered, not invented.
            Each writer uses this universal system regardless of their native language or culture.
            However, each character uses the system in their own way.
            Another person's magical writing remains incomprehensible to even the most powerful spellcaster until they take the time to study and decipher it.

    % TODO: description
    \subsection{Categories of Magic}

        \subsubsection{Magic Sources}
            There are four \glossterm{magic sources} that characters can use to cast spells and perform rituals: arcane (cast by sorcerers and wizards), divine (cast by clerics and paladins), nature (cast by druids), and pact (cast by warlocks).
            Each magic source has a set of associated \glossterm{mystic spheres} (see Mystic Spheres, below).
            % TODO: more description

            \parhead{Characters with Multiple Magic Sources}
                A character can have access to multiple sources of magic through the use of abilities like the Hybrid Training ability (see \pcref{Half-Elves}).
                The \glossterm{mystic spheres}, spells, and rituals that character knows are tracked separately for each source of magic that character has access to.
                If you have access to the same spell or ritual from multiple sources, the two versions of the ability are generally considered to be the same ability.
                When you cast the spell or perform the ritual, you choose which source you are using for the ability.

        \subsubsection{Mystic Spheres}
            A \glossterm{mystic sphere} is a collection of thematically related magical effects that includes both \glossterm{spells} and \glossterm{rituals}.
            Each \glossterm{mystic sphere} can be associated with any number of \glossterm{magic sources}.
            The mystic spheres are listed at \pcref{Mystic Spheres}.

    \subsection{Ability Tags}
        All spells and rituals are \magical.
        In addition, all spells have the \abilitytag{Spell} \glossterm{ability tag}, and all rituals have the \abilitytag{Ritual} ability tag.
        Since spells and rituals are already clearly indicated in the Mystic Spheres chapter, the tags are omitted here to avoid unnecessary repetition.
        In some other places, such as monster descriptions, those tags may be used to indicate that specific abilities are considered spells or rituals despite not appearing in this chapter.

    \subsection{Casting Components}\label{Casting Components}
        Unless otherwise noted, all spells and rituals require \glossterm{verbal components} to cast or perform.
        In addition, spells and rituals from arcane and pact mystic sources require \glossterm{somatic components}.
        You cannot start casting a spell or performing a ritual without all required components.
        If you lose those components before the ability resolves, the spell fails with no effect.

        To provide the verbal component for a spell or ritual, you must speak in a strong voice with a volume at least as loud as ordinary conversation.
        To provide the somatic component for a spell or ritual, you must make a precise series of movements with at least one free hand.
        These movements involve your whole arm in addition to gestures with your fingers.

        \subsubsection{Somatic Component Failure}\label{Somatic Component Failure}
            Encumbrance from armor interferes with the \glossterm{somatic components} required to perform arcane spells, pact spells, and all rituals.
            When you cast a spell or perform a ritual that requires \glossterm{somatic components} while you have an \glossterm{encumbrance}, you must roll 1d10.
            If your result is less than or equal to your \glossterm{encumbrance}, the spell fails with no effect.
            When you perform a ritual, this roll must be repeated at the end of each round during the ritual.

    \subsection{Dismissal}\label{Dismissal}
        Many abilities can intentionally be ended early if you \glossterm{dismiss} it.
        When an ability is dismissed, all of its lingering effects immediately end.
        Unless otherwise noted, all \magical abilities with a duration can be dismissed, but \glossterm{mundane} abilities cannot be dismissed.
        This includes \glossterm{conditions}, \glossterm{brief} effects, and other abilities with more specific durations.
        You can dismiss abilities as a \glossterm{free action} that requires only mental effort.

    \subsection{Resurrecting the Dead}\label{Resurrecting the Dead}
        Several rituals have the power to restore dead characters to life.

        When a living creature dies, its soul departs its body, travels through the Astral Plane, and goes to abide on the appropriate plane or Divine Realm.
        This process is enforced by the Reapers, who escort souls to ensure they reach their intended destinations.
        Bringing a creature back from the dead means retrieving their soul and returning it to their body.

        \parhead{Deterioration of the Soul} While dead, souls gradually lose their cohesion and independent sense of self.
        A typical creature can maintain its existence in the afterlife for a number of years equal to 5 times the sum of its level and Willpower.
        This can vary significantly for individual creatures, and being tormented in the afterlife can significantly reduce this time.

        \parhead{Preventing Resurrection} Enemies can take steps to make it more difficult for a character to be returned from the dead.
        Except for \spell{true resurrection}, every ritual to raise the dead requires a body, so keeping or destroying the body is an effective deterrent.
        The \spell{soul bind} ritual prevents any sort of revivification unless the soul is first released.

        \parhead{Involuntary Resurrection} A soul cannot be returned to life if it does not wish to be.
        A soul infallibly knows the name, alignment, and patron deity (if any) of the character attempting to revive it and may refuse to return on that basis.

    \subsection{Functioning Like Other Spells}\label{Functioning Like Other Spells}
        Many spells and rituals say they ``function like'' some other spell or ritual, often with some noted changes.
        Except as otherwise noted, they retain all of the original effects and targets of the spell.
        However, they do not have the same rank upgrades as the original spell or ritual.

    \subsection{Impossible Spells and Rituals}
        When you try to use a spell or ritual in an impossible way, the ability fails with no effect.
        This most commonly happens if you attempt to declare an invalid target for a spell.

\section{Common Magical Effects}
    \subsection{Delayed and Repeated Effects}\label{Delayed and Repeated Effects}
        Some abilities cause an effect to happen in a future round, or in all subsequent rounds.
        These abilities often say that they happen ``during your action''.
        When you act during each action phase, you can decide when that effect happens, including before and after all of your other actions.
        For example, you could resolve the trigger first to see the results before deciding your actions, or you could act before the trigger to push another creature into a relevant area.

        You must still complete all of your actions, including these triggers, as part of your single turn.
        You can't resolve a trigger, then let your allies take actions, then take your own action afterwards.

    \subsection{Resurrection}\label{Resurrection}
        Some abilities can return dead creatures to life.
        This is called resurrection.

        A creature has no hit points or damage resistance when it returns to life.
        It is cured of all \glossterm{vital wounds}, \glossterm{conditions}, and other negative effects, but the body's shape is unchanged.
        Any missing or irreparably damaged limbs or organs remain missing or damaged.
        The creature may therefore die shortly after being resurrected if its body is excessively damaged.
        Some resurrection abilities can restore more damaged corpses to life, as indicated in their descriptions.

        Coming back from the dead is an ordeal.
        The creature's maximum \glossterm{fatigue tolerance} is reduced by 1.
        This penalty lasts for thirty days, or until the creature gains a level.
        If this would reduce a creature's maximum fatigue tolerance below 0, the creature cannot be resurrected.

        Resurrection is always voluntary.
        If a dead creature's soul refuses to return to life, no effect can compel it to be resurrected.
        Similarly, if a dead creature's soul has been subsumed into the planar essence of its afterlife plane, it has already been resurrected, or the soul is otherwise inaccessible, resurrection is impossible.

        Although you can resurrect creatures who have died of old age, it is usually pointless.
        They will die again before long from some malady resulting from their advanced age.

    \subsection{Shapeshifting}\label{Shapeshifting}
        When a creature shapeshifts, its physical body completely transforms into a different shape.
        It generally retains all of its original statistics and abilities, with the following exceptions.
        Some specific abilities that cause a creature to shapeshift have additional effects.
        % TODO: are more exceptions necessary?
        \begin{itemize}
            \item The creature's size changes to match the new form.
                This can change the creature's \glossterm{base speed}, Reflex defense, and other statistics as normal (see \pcref{Size Categories}).
            \item The creature's \glossterm{mundane} movement modes and natural weapons are replaced with the movement modes and natural weapons of its new shape.
            \item If the new shape is not normally capable of speech, the creature cannot speak.
                This may prevent it from casting spells with \glossterm{verbal components} and using similar abilities.
            \item The creature is limited by the number of \glossterm{free hands} present in the new form.
                In addition, it cannot gain more free hands by shapeshifting than it originally had in its base form.
                Even if you shapeshift to a form with many hands, you do not have the mental coordination necessary to use them all effectively.
            \item Any special properties that a creature had that were originally a result of its pure physical composition may be lost.
                For example, a ghost would stop being \trait{incorporeal} if it shapeshifted.
        \end{itemize}

        All of a shapeshifted creature's equipment that is physically incompatible with the creature's new shape meld into its body.
        This does not break \glossterm{attunement}, and the creature still gains the benefit of any magical properties of melded items.
        However, it does not gain the benefit of nonmagical properties from melded items.
        For example, a creature that shapeshifts into an amorphous gas would still benefit from all attuned effects from its equipped items, such as \mitem{boots of speed}.
        However, it would gain no benefit to its Armor defense or damage resistance from any melded body armor, and it would not be able to attack with any of its melded weapons.
        Items exceeding a creature's \glossterm{carrying capacity} are not melded, and simply fall to the ground in place.

        When a shapeshifted creature dies, it returns to its original form.

    \subsection{Teleportation}\label{Teleportation}
        Some abilities can \glossterm{teleport} creatures or objects.
        When you are teleported, you move through the Astral Plane and arrive at a new location.
        You can be teleported between two different locations on the same \glossterm{plane}, or between two different locations on different planes.
        If for some reason you cannot access the Astral Plane, you cannot be teleported.

        Anything being teleported must have both \glossterm{line of sight} and \glossterm{line of effect} to its destination.
        In addition, the destination of the teleportation must be an unoccupied location on a stable surface that can support the weight of the teleporting creature or object.
        If either of these conditions is not met, the teleportation fails without effect.
        Some teleportation abilities are less restricted, as indicated in their description.

        In general, you can teleport up slopes that are no more than 45 degrees.
        Steeper slopes prevent you from seeing stable ground to teleport to it.
        The GM can provide guidance for individual slopes, which may be easier or harder to navigate with teleportation.

        \subsubsection{Teleportation Noise}\label{Teleportation Noise}
            Creatures and objects that are teleported make a sound when they depart and arrive.
            This noise is caused by the displacement of air (or other substances) created by the teleportation.
            The base \glossterm{difficulty value} of an Awareness check to hear this sound for a Medium creature or object is 10.
            This difficulty value changes based on the size of the teleported creature or object:

            \begin{itemize}
                \item Fine: 30
                \item Diminutive: 25
                \item Tiny: 20
                \item Small: 15
                \item Medium: 10
                \item Large: 5
                \item Huge: 0
                \item Gargantuan: \minus5
                \item Colossal: \minus10
            \end{itemize}

        \subsubsection{Carrying Objects}
            When a creature is teleported, it can bring along equipment and held objects as long as two conditions are met.
            First, the combined weight of the objects cannot exceed the creature's maximum \glossterm{carrying capacity} (see \pcref{Weight Limits}).
            If a creature is teleported while carrying more than its maximum carrying capacity, all excess objects are left behind, starting with the heaviest object and proceeding in order of weight.

            Second, no object can extend more than two feet away from the creature's body.
            Any objects that extend beyond that distance are left behind.
            For example, a creature wearing handcuffs will arrive at its teleportation destination still wearing the handcuffs.
            However, a creature that is tied to a post by a long rope will arrive at its teleportation destination without the rope.

        \subsubsection{Astral Beacon}\label{Astral Beacon}
            Some abilities allow long-distance teleportation, such as the \ritual{overland teleportation} ritual.
            This sort of teleportation is much easier if you are travelling to an \glossterm{astral beacon}.
            The specific effects of an astral beacon are defined in the teleportation ability being used.
            An astral beacon covers an area, rather than a single point in space.

            Each astral beacon has a unique name.
            The name represents the beacon's precise location in the Astral Plane, so no two beacons can have identical names.
            For example, astral beacons created by rituals have their name defined by the precise color of ritual inks, details of drawn patterns, timing and inflection of ritual incantations, and similar subtleties.

            It is possible, though unlikely, to find astral beacons simply by wandering in the Astral Plane.
            They are similar in size and shape to \glossterm{scrying sensors}, but their appearance is visually distinct (for creatures who can see \trait{invisible} objects).
            Inspecting a beacon can reveal the location it points to, and destroying the beacon in the Astral Plane removes the astral beacon entirely.
            This is generally considered a hostile act, and may have consequences.

        \subsubsection{Horizontal Teleportation}
            Some planes have a curved primary surface.
            On those planes, ``horizontal'' teleportation isn't objectively horizontal.
            Instead, it is horizontal relative to the surface of the plane.

\section{Breaking Objects}
    There are two main ways of breaking objects.
    You can deal damage to objects with attacks, similarly to how you can deal damage to creatures.
    Alternately, you can attempt to sunder the object with sheer strength.

    \subsection{Damaging Objects}
        Objects have \glossterm{hit points} and \glossterm{damage resistance} like creatures.
        However, non-creature objects treat all damage they take as \glossterm{environmental damage} (see \pcref{Environmental Damage}).
        That means that all damage they take is reduced by their \glossterm{damage resistance} without subtracting from the remaining value of their damage resistance.

        An object becomes \glossterm{broken} if its \glossterm{hit points} are reduced to 0 (see \pcref{Broken and Destroyed Objects}).
        Objects cannot gain \glossterm{vital wounds}.
        Objects are also not normally subject to \glossterm{critical hits}.

    \subsection{Sundering Objects}
        As a standard action, you can attempt to break an object with raw strength instead of damage.
        This requires two hands.
        When you sunder an object, make a Strength check.
        The \glossterm{difficulty value} of the check is equal to the object's \glossterm{damage resistance}, \plus5 for each \glossterm{weight category} above Diminuitive.

    \subsection{Broken and Destroyed Objects}\label{Broken and Destroyed Objects}
        An object that is reduced to 0 \glossterm{hit points} becomes \glossterm{broken}.
        You can destroy an object by causing it to lose additional hit points equal to ten times its maximum hit points, or by succeeding at a check to sunder the object by 10.

        \parhead{Broken Objects}\label{Broken Objects}
        Broken objects cannot be used for their intended purpose, but still retain enough of their original form to be repaired without too much work.
        For example, a broken wall lies in pieces on the ground and no longer blocks passage, but can be repaired with far less effort than would be required to create a wall from scratch.
        Magic items that are broken retain their magical properties once fixed.
        Broken (but not destroyed) objects can be repaired with the Craft skill (see \pcref{Craft}).

        \parhead{Destroyed Objects}\label{Destroyed Objects}
        Destroyed object have been damaged beyond hope of any sort of repair short of crafting the object again from raw materials.
        For example, a destroyed wall is reduced to dust or small, useless chunks of rubble.
        Magic items that are destroyed irrevocably lose their magical properties.
        The remains of a destroyed object generally occupy a space one size category smaller than the original object.
        Destroyed objects can be rebuilt with the Craft skill, but it requires significant time and investment.

    \subsection{Relative Damage Resistance}\label{Relative Damage Resistance}
        When an object would take damage from a \glossterm{strike}, if the \glossterm{damage resistance} of the attacking object or creature is lower than the damage resistance of the defender, the attacking object or creature takes the damage instead.
        For example, if you try to break a stone wall with a wooden club, the club will break instead of the wall.
        % TODO: define hardness for creatures and their natural weapons; natural weapons should generally have higher hardness than creatures to avoid hardness reflection being common

    \subsection{Breaking Equipment}\label{Breaking Equipment}
        Normally, a character's equipment cannot be damaged or otherwise affected by attacks.
        This includes worn items, anything held in your hands, and anything in a secure storage like a small backpack.
        Such items are considered \glossterm{attended}.
        They are unaffected by damage caused by area effects, and cannot be targeted individually.
        Some abilities can specifically target \glossterm{attended} objects, as indicated in their descriptions.

        \subsubsection{Loose Equipment}
            Some items are explicitly \glossterm{loose equipment}.
            Loose equipment does not gain the protections listed above while worn as equipment.
            It can be individually targeted by attackers, and is affected by area effects just like any other object in the area.

\section{Poison}\label{Poison}
    Poisons are organic substances that are dangerous to living creatures.
    They can deal damage or inflict debilitating effects.
    Some effects which are not literally poisonous, such as animal venom or fungal spores, are considered poisons.
    Poisons are not \glossterm{conditions}, and cannot be removed by abilities that remove conditions (see \pcref{Conditions}).
    Common poisons are listed in \tref{Consumable Tools}.

    \subsection{Poison Effects}\label{Poison Effects}
        When you come in contact with a poison, you become \glossterm{poisoned}.
        As soon as you become poisoned, and at the end of each subsequent round, the poison makes an attack against your Fortitude defense.
        On a hit, the poison progresses to its next stage.
        On a critical hit, the poison progresses by two stages at once, to a maximum of the third stage.
        On a miss, you make progress towards removing the poison entirely.
        Once the poison misses you three times, you stop being poisoned.

        If you become poisoned again by the same poison, it does not intensify the effects of the poison.
        However, it cancels any progress you had made towards removing the poison.
        A poison is considered the same if it has the same name.
        If you are affected by multiple different poisons with the same name, but different accuracy bonuses, use the highest accuracy bonus.

        Poisons have no effect on non-living creatures.

        \subsubsection{Poison Stages}
            Poison effects are divided into stages.
            Becoming poisoned does not have any ill effects until the poison progresses to its first stage.

            Many poisons have a Stage 1 effect.
            This effect happens as soon as the poison's attack first succeeds against you.
            Some poisons also have a Stage 3 effect.

        \subsubsection{Poison Accuracy}
            A poison's accuracy depends on the way it was applied.
            Item-based poisons have a specific accuracy listed in their description.
            This accuracy does not depend on the skill of the creature inflicting the poison.

            Poisons inflicted by creature abilities use the creature's \glossterm{accuracy}.
            They may also have additional modifiers listed in the ability's description.
            For monsters, the poison's accuracy will be listed in the monster description.

    \subsection{Poison Transmission}\label{Poison Transmission}\label{Transmission}

        There are three ways that poisons can be contracted.

        \parhead{Contact} A contact poison affects any creature that touches it with bare skin.
        \parhead{Ingestion} An ingestion poison affects any creature that eats, drinks, or breathes it, depending on the type of poison.
        Ingestion poisons have no effect when touched or used to coat weapons.
        \parhead{Injury} An injury poison affects any creature that loses \glossterm{hit points} from something bearing the poison.
        Almost all injury poisons take liquid form, and are typically used to coat weapons.

    \subsection{Poison Forms}\label{Poison Forms}

        There are four forms of poison.

        \parhead{Gas} Gaseous poisons are difficult to store, but easy to affect foes with.
        Unless otherwise noted, a gas poison can be thrown within \shortrange, and affects a \tinyarea radius.
        \parhead{Liquid} Liquid poisons are the most common type of poison.
        Liquid poisons can be used to coat weapons, slipped into food, or simply thrown at foes.
        A dose of a liquid poison is usually about one ounce of the poison.
        \parhead{Pellet} Some rare poisons come in small, solid pellets or cubes.
        Typically, these pellets contain a powerful liquid poison that becomes inert quickly after being exposed.
        Pellet poisons are typically applied by being slipped into food.
        \parhead{Powder} Poison in powder form cannot be used to coat weapons, but can be slipped into food or thrown at foes.
        Unless otherwise noted, a powder poison functions as a \glossterm{thrown weapon} with \glossterm{range limits} of 5/15.

    \subsection{Coating Weapons with Poison}\label{Coating Weapons with Poison}
        As a standard action, you can coat a weapon with a single dose of a liquid contact-based or injury-based poison.
        The next time a creature takes damage from a \glossterm{strike} using that weapon, the struck creature comes in contact with the poison.
        This removes one dose of the poison from the weapon.
        Coated poisons expire and lose their effectiveness after ten minutes.

        An injury-based poison has no effect if the strike does not cause the struck creature to lose \glossterm{hit points}, but the dose is still removed from the weapon.
        For this reason, injury-based poisons are typically applied to secondary weapons that can be used after the subject is already weakened.

        A weapon can hold up to three poison doses of the same poison.
        Mixing different poison types on the same weapon is ineffective, as each poison dilutes the others.
        Only the highest rank poison on the weapon has any effect.

    \subsection{Creating Poisons}\label{Creating Poisons}

        You can use the Craft (poison) skill to create poisons.
        To create a poison, you must make a Craft (poison) check against a \glossterm{difficulty value} equal to 10 \add the poison's base accuracy.
        For every 2 points by which you beat this \glossterm{difficulty value}, the created poison's accuracy gains a \plus1 bonus, up to a maximum bonus of \plus10 more than the poison's base accuracy.

        Creating a poison requires special materials.
        The type of materials required, and how those materials can be acquired, depend on the type of poison.

        \begin{itemize}
            \itemhead{Plant} Plant-based poisons can typically be harvested by making a Survival check to search in appropriate terrain.
                The \glossterm{difficulty value} of this check is usually equal to 10 \add twice the poison's \glossterm{rank}.
            \itemhead{Venom} Venom requires an appropriate body part from a creature -- often, poison it naturally produces.
            \itemhead{Alchemical} Alchemical poisons require alchemical materials.
                These normally can't be found in nature.
                In unusual circumstances, these components can be synthesized from natural chemicals or magical materials with a Craft (alchemy) check equal to 10 \add the base accuracy of the poison.
        \end{itemize}


\chapter{Races}

Each character has a race.

\section{Racial Traits}

\subsection{Racial Bonus Feats}
Each race grants a bonus feat at 1st level. Most races can only choose from a small group of feats, listed in the description of the race. A character must meet any prerequisites for these bonus feats, as normal.

\subsection{Favored Weapons}
The names of some exotic weapons, such as the orcish double axe, include the name of a race. Members of the named race can treat those weapons as if they were martial weapons rather than exotic weapons.

\subsection{Race and Languages}
All characters know how to speak Common. A dwarf, elf, gnome, half-elf, half-orc, or halfling also speaks a racial language, as appropriate. A character who has an Intelligence bonus at 1st level speaks other languages as well, one extra language per point of Intelligence bonus as a starting character.

\parhead{Literacy} Any character except a barbarian can read and write all the languages he or she speaks.

\parhead{Class-Related Languages} Clerics, druids, and wizards can choose certain languages as bonus languages even if they're not on the lists found in the race descriptions. These class-related languages are as follows:
\subparhead{Cleric} Abyssal, Celestial, Infernal.
\subparhead{Druid} Sylvan.
\subparhead{Wizard} Draconic.

\subsection{Small Characters}
A Small character gets a \plus1 bonus to Armor Class, a \plus1 bonus on attack rolls, and a \plus2 bonus on Stealth checks. A Small character's carrying capacity is three-quarters of that of a Medium character.

A Small character generally moves about two-thirds as fast as a Medium character.

A Small character must use smaller weapons than a Medium character.

\section{Race Descriptions}

\subsection{Humans}
\begin{itemize*}
\item Medium: As Medium creatures, humans have no special bonuses or penalties due to their size.
\item Human base land speed is 30 feet.
\item Humans can choose any feat for their racial bonus feat.
\item 2 extra skill points at 1st level.
\item Automatic Language: Common. Bonus Languages: Any (other than secret languages, such as Druidic). See the Speak Language skill.
\end{itemize*}

\subsection{Dwarves}
\begin{itemize*}
\item \plus1 Constitution, \minus1 Dexterity.
\item Medium: As Medium creatures, dwarves have no special bonuses or penalties due to their size.
\item Dwarf base land speed is 20 feet.
\item Dwarves do not move slower in heavy armor, and can run normally in both medium and heavy armor.
\item Darkvision: Dwarves can see in the dark clearly up to 60 feet.   Beyond that, they can see dimly, treating areas of darkness as shadowy illumination. Darkvision does not function if a dwarf is in a brightly lit area or is dazzled, and does not resume functioning until 1 round after the dwarf leaves the brightly lit area or stops being dazzled.
\par Darkvision is black and white only, but it is otherwise like normal sight, and dwarves can function just fine with no light at all.
\item Stability: A dwarf gains a \plus2 bonus to Maneuver Class to resist being bull rushed, overrun, or tripped when standing on the ground (but not when climbing, flying, riding, or otherwise not standing firmly on the ground).
\item Dwarves can choose any of the following feats for their racial bonus feat: Armor Proficiency (any), Diehard, Dwarven Resilience, Endurance, Giantfighter, Great Fortitude, Perfect Health, Stonecunning, Toughness, Weapon Proficiency (axes)
\item Automatic Languages: Common and Dwarven. Bonus Languages: Giant, Gnome, Goblin, Orc, Terran, and Undercommon.
\end{itemize*}

\subsection{Elves}
\begin{itemize*}
\item \plus1 Dexterity, \minus1 Constitution.
\item Medium: As Medium creatures, elves have no special bonuses or penalties due to their size.
\item Elf base land speed is 30 feet.
\item Immunity to sleep effects.
\item Low-light Vision: An elf can see twice as far as a human in starlight, moonlight, torchlight, and similar conditions of poor illumination. She retains the ability to distinguish color and detail under these conditions.
 \item Trance: Elves that trance for 4 hours gain the same benefit as humans do from 8 hours of sleep. An elf in trance may make Listen checks at a \minus5 penalty.
\item Keen Senses: \plus2 bonus on Perception checks.
\item Elves can choose any of the following feats for their racial bonus feat: Dilettante, Focused Mind, Improved Initiative, Lightning Reflexes, Swift, Weapon Proficiency (bows, heavy blades, or light blades)
\item Automatic Languages: Common and Elven. Bonus Languages: Draconic, Gnoll, Gnome, Goblin, Orc, and Sylvan.
\end{itemize*}

\subsection{Gnomes}
\begin{itemize*}
\item \plus1 Constitution, \minus1 Strength.
\item Small: As a Small creature, a gnome gains a \plus1 bonus to Armor Class, a \plus1 bonus on attack rolls, and a \plus2 bonus on Stealth checks. However, he takes a \minus4 penalty to combat maneuver attack and defense, he uses smaller weapons than humans use, and his lifting and carrying limits are three-quarters of those of a Medium character.
\item Gnome base land speed is 20 feet.
\item Low-light Vision: A gnome can see twice as far as a human in starlight, moonlight, torchlight, and similar conditions of poor illumination. He retains the ability to distinguish color and detail under these conditions.
\item Gnomish Tricks: A gnome with a Charisma score of at least 0 gains \spell{create sound} and \spell{dancing lights} as spell-like abilities. The gnomish tricks ability can be used a number of times per day equal to 1 \add half the gnome's Charisma. The gnome's caster level with these abilities is equal to the gnome's character level, and the save DC is equal to 10 \add character level \add Charisma.

\item Gnomes can choose any magic feat, spellgift feat, or gnomish racial feat for their racial bonus feat.
\item Automatic Languages: Common and Gnome. Bonus Languages: Draconic, Dwarven, Elven, Giant, Goblin, and Orc.
\end{itemize*}

\subsection{Half-Elves}
\begin{itemize*}
\item Medium: As Medium creatures, half-elves have no special bonuses or penalties due to their size.
\item Half-elf base land speed is 30 feet.
\item Immunity to sleep effects.
\item Low-light Vision: A half-elf can see twice as far as a human in starlight, moonlight, torchlight, and similar conditions of poor illumination. She retains the ability to distinguish color and detail under these conditions.
 \item Skill Affinity: Half-elves can master skills with particular ease. If a half-elf has a skill as a class skill from any class, it is treated as a class skill for all of his classes.
\item Elven Blood: For all effects related to race, a half-elf is considered both a human and an elf.
\item Half-elves can choose any skill feat or any elven or human racial feat for their racial bonus feat.
\item Automatic Languages: Common and Elven. Bonus Languages: Any (other than secret languages, such as Druidic).
\end{itemize*}

\subsection{Half-Orcs}
\begin{itemize*}
\item \plus1 Strength, \minus1 Intelligence, \minus1 Wisdom.
\item Medium: As Medium creatures, half-orcs have no special bonuses or penalties due to their size.
\item Half-orc base land speed is 30 feet.
\item Darkvision: Half-orcs (and orcs) can see clearly in the dark up to 60 feet.  Beyond that, they can see dimly, treating areas of darkness as shadowy illumination. Darkvision does not function if an orc is in a brightly lit area or is dazzled, and does not resume functioning until 1 round after the orc leaves the brightly lit area or stops being dazzled.
\par Darkvision is black and white only, but it is otherwise like normal
sight, and half-orcs can function just fine with no light at all.
\item \plus2 bonus on Intimidate checks, but a \minus2 penalty on Persuasion checks.
\item Orc Blood: For all effects related to race, a half-orc is considered both a human and an orc.
\item Half-orcs can choose any combat feat or any orc or human racial feat for their racial bonus feat.
\item Automatic Languages: Common and Orc. Bonus Languages: Draconic, Giant, Gnoll, Goblin, and Abyssal.
\end{itemize*}

\subsection{Halflings}
\begin{itemize*}
\item \plus1 Dexterity, \minus1 Strength.
\item Small: As a Small creature, a halfling gains a \plus1 bonus to Armor Class, a \plus1 bonus on attack rolls, and a \plus2 bonus on Stealth checks. However, she takes a \minus4 penalty to combat maneuver attack and defense, she uses smaller weapons than humans use, and her lifting and carrying limits are three-quarters of those of a Medium character.
\item Halfling base land speed is 20 feet.
\item \plus1 bonus on all saving throws.
\item Halflings can choose any of the following feats for their racial bonus feat: Athletic, Giantfighter, Great Fortitude, Lightning Reflexes, Iron Will, Swift, Weapon Proficiency (thrown).
\item Automatic Languages: Common and Halfling. Bonus Languages: Dwarven, Elven, Gnome, Goblin, and Orc.
\end{itemize*}

\section{Backgrounds}
In addition to a race, each character also has at least one background. Each background describes what a character has done before the start of the story. At character creation, each character gains two bonus skill points that can only be spent on skills associated with one of their backgrounds. A character may have any number of backgrounds, but may only select skills from two of his backgrounds.

The backgrounds listed here are merely suggestions. You may choose to create a new background. A new background created in this way may be associated with up to two skills of your choice, provided that they make sense for the background, as determined by the DM.

%The backgrounds do not all need to have an equal number of associated skills. Backgrounds which are unlikely to become adventurers can have fewer skills. Max of 3 skills from a single background. NPCs have one background.

\subsection{Civilized Backgrounds}

\subsubsection{Bodyguard}
\parhead{Skills} Perception, Sense Motive.

\subsubsection{Commoner}
\parhead{Skill} Profession (any one).

\subsubsection{Linguist}
\parhead{Skills} Linguistics, Knowledge (local).

\subsubsection{Jester}
\parhead{Skills} Acrobatics, Perform (comedy), Sleight of Hand.

\subsubsection{Mage's Apprentice}
\parhead{Skills} Knowledge (arcana), Spellcraft.

\subsubsection{Merchant}
\parhead{Skills} Persuasion, Knowledge (local).

\subsubsection{Nobility}
\parhead{Skills} Bluff, Knowledge (local), Sense Motive.

\subsubsection{Priest}
\parhead{Skill} Heal, Knowledge (religion), Perform (oratory).

\subsubsection{Scholar}
\parhead{Skill} Knowledge (all).

\subsubsection{Scribe}
\parhead{Skill} Craft (manuscript), Linguistics.

\subsubsection{Smith}
\parhead{Skill} Craft (all).

\subsubsection{Spy}
\parhead{Skills} Bluff, Disguise.

\subsubsection{Watchman}
\parhead{Skills} Knowledge (local), Perception.

\subsection{Military Backgrounds}

\subsubsection{Border Guard}
\parhead{Skill} Knowledge (geography, nature), Survival.

\subsubsection{Cavalry}
\parhead{Skill} Creature Handling, Ride.

\subsubsection{Combat Engineer}
\parhead{Skill} Craft (any one), Knowledge (engineering).

\subsubsection{Diplomat}
\parhead{Skills} Bluff, Persuasion, Sense Motive.

\subsubsection{Infiltrator}
\parhead{Skills} Disguise, Stealth.

\subsubsection{Officer}
\parhead{Skills} Intimidate, Persuasion.

\subsubsection{Saboteur}
\parhead{Skills} Devices, Stealth.

\subsubsection{Scout}
\parhead{Skills} Perception, Stealth.

\subsection{Uncivilized Backgrounds}

\subsubsection{Bandit}
\parhead{Skills} Intimidate, Stealth.

\subsubsection{Explorer}
\parhead{Skills} Knowledge (geography), Survival.

\subsubsection{Hermit}
\parhead{Skill} Knowledge (nature), Survival.

\subsubsection{Minstrel}
\parhead{Skill} Perform (all).

\subsubsection{Primitive}
\parhead{Skill} Survival.

\subsubsection{Thief}
\parhead{Skills} Sleight of Hand, Stealth.


\chapter{Classes}\label{Classes}

Your character's class represents the things your character has chosen to train in.
This choice determines a great deal about your character's abilities.

\section{How Classes Work}
    When you first create a character, you choose a class.
    Your character has one level in that class.
    This grants your character all of the abilities your chosen class grants at 1st level, as given in the class description.
    Each time your character gains a level, you can choose to increase your level in your original class or gain a level in a new class.
    This grants your character all of the abilities your chosen class grants at the level your character just gained in it.

    \subsection{Archetypes}
        Each class has three \glossterm{archetypes}.
        An archetype is a collection of thematically related class abilities.
        For examples, barbarians have the Battlerager archetype, which grants abilities related to flying into a rage in combat.
        Normally, a member of a class has all three archetypes associated with that class.
        Characters with the Class Versatility feat can gain achetypes from two different classes (see \featpcref{Class Versatility}).

\section{Class Introductions}

    There are eleven classes in Rise.
    \begin{itemize}
        \item Barbarians are mighty warriors who can enter a deadly battlerage.
        \item Clerics are divine spellcasters who draw power from their veneration of a deity or ideal.
        \item Druids are nature spellcasters who draw power from their veneration of the natural world.
        \item Fighters are highly disciplined warriors who excel in physical combat of any variety.
        \item Mages are arcane spellcasters who wield the mystic forces of magic to create almost any effect.
        \item Monks are agile masters of ``\ki'' who hone their personal abilities to strike down foes and perform supernatural feats.
        \item Paladins are divinely empowered warriors whose devotion to an alignment grants them the ability to discern and smite their foes.
        \item Rangers are skilled hunters who bridge the divide between nature and civilization.
        \item Rogues are exceptionally skillful characters known for their ability to strike at their foe's weak points in combat.
        % \item Spellwarped wield a unique blend of martial skill and narrowly focused magical abilities.
    \end{itemize}

    \subsection{Class Description Format}
        Each class is described from the perspective of a member of that class, using ``you'' in the description.

        \parhead{Class Table}
        The class's table describes the special abilities they get at each level.

        \parhead{Alignment}
        Some classes require specific alignments (see \pcref{Alignment}).
        Most classes allow characters of any alignment.

        \parhead{Skill Points}
        This is the number of skill points that members of this class get.

        \parhead{Class Skills}
        These are skills that members of this class are typically good at (see \pcref{Skills}).

        \parhead{Defenses}
        Each class grants bonuses to specific defenses.

        \parhead{Weapon and Armor Proficiencies}
        These are the types of equipment that members of this class are trained in using.

        \parhead{Other Special Abilities}
        Some classes have abilities shared by all members of the class that are not part of an archetype, such as a cleric's \textit{divine power}.

        \parhead{Archetypes}
        The abilities associated with each of the three archetypes the class has.

\section{Barbarian}\label{Barbarian}
    \begin{dtable}
        \lcaption{Barbarian Progression}
        \begin{dtabularx}{\columnwidth}{>{\ccol}p{\levelcol} >{\lcol}X}
            \tb{Level} & \tb{Abilities} \\\bottomrule
            \nth{1}     & Athletic prowess, rage
            \\ \nth{2}  & Battle-scarred
            \\ \nth{3}  & Primal exertion
            \\ \nth{4}  & Blood frenzy
            \\ \nth{5}  & Uncanny dodge
            \\ \nth{6}  & Athletic prowess
            \\ \nth{7}  & Unstoppable rage
            \\ \nth{8}  & Deep scars
            \\ \nth{9}  & Primal exertion
            \\ \nth{10} & Unstoppable rage
            \\ \nth{11} & Greater uncanny dodge
            \\ \nth{12} & Primal prowess
            \\ \nth{13} & Frenzied assault
            \\ \nth{14} & Rapid recovery
            \\ \nth{15} & Primal exertion
            \\ \nth{16} & Mindless rage
            \\ \nth{17} & Uncanny whirlwind
            \\ \nth{18} & Primal supremacy
            \\ \nth{19} & Blood frenzy
            \\ \nth{20} & Deathless rage, primal soul, soulscarred
        \end{dtabularx}
    \end{dtable}

    \classbasics{Alignment} Any nonlawful.

    \subsection{Class Abilities}
        If you are a barbarian, you gain the following abilities.

        \classbasics{Skill Points} 6.

        \classbasics{Class Skills}
        \begin{itemize}
            \item \subparhead{Strength} Climb, Jump, Sprint, Swim.
            \item \subparhead{Dexterity} Acrobatics, Ride.
            \item \subparhead{Perception} Awareness, Creature Handling, Survival.
            \item \subparhead{Other} Bluff, Intimidate, Persuasion.
        \end{itemize}

        \classbasics{Defenses} \plus3 Fortitude, \plus2 Reflex, \plus1 Mental.

        \classbasics{Weapon and Armor Proficiency}
        You are proficient with simple weapons, any four other weapon groups, light armor, medium armor, and shields.

        \subsubsection{Battlerager}\label{Rage}

            \cf{Bbn}{Rage} 
            As a \glossterm{free action}, you can spend an \glossterm{action point} to use this ability.
            \begin{ability}
                \begin{spelleffects}
                    \spelleffect You have the following benefits and drawbacks:
                    \begin{itemize}
                        \item You gain a \plus1d bonus to \glossterm{strike damage}.
                        \item You are unable to take any action that requires patience or concentration, such as casting spells.
                        \item At the end of each round, if you did not attack a creature or object, you take \glossterm{nonlethal damage} equal to your level.
                            This damage ignores damage reduction and any similar abilities.
                    \end{itemize}
                    \spellspecial When this ability ends, you become \fatigued and unable to use it again until you take a \glossterm{short rest}.
                    \spelldur Attunement
                \end{spelleffects}
            \end{ability}

            \cf{Bbn}[4]{Blood Frenzy}
            You are immune to being \glossterm{bloodied} while raging

            \cf{Bbn}[7]{Senseless Rage}
            You are immune to being \glossterm{staggered} while raging.

            \cf{Bbn}[10]{Unstoppable Rage}
            You are immune to \glossterm{Mind} \glossterm{conditions} while raging.

            \cf{Bbn}[13]{Frenzied Assault}
            You gain a \plus1 bonus to \glossterm{accuracy} with \glossterm{strikes} while raging.

            \cf{Bbn}[16]{Mindless Rage} 
            You are immune to all hostile \glossterm{Mind} effects while raging.

            \cf{Bbn}[19]{Deathless Rage} 
            While raging, the you ignore all penalties from \glossterm{vital damage}.
            However, if your vital damage exceeds your maximum hit points, you immediately die.

        \subsubsection{Primal Warrior}
            \cf{Bbn}{Athletic Prowess} You gain two additional skill points that must be spent on Strength or Dexterity-based barbarian class skills.

            \cf{Bbn}[3]{Primal Exertion}
            You gain the ability to channel your primal energy into ferocious attacks.
            Choose a single \textit{primal exertion} from the list below.
            As a standard action, you can spend an \glossterm{action point} to use a \textit{primal exertion} ability.
            {
                \subcf{Battle Cry} You and all allies within an \arealarge radius burst from you gain \glossterm{temporary hit points} equal to twice your Willpower.
                This lasts as long as you \glossterm{sustain} this ability as a \glossterm{swift action}.

                \subcf{Demoralizing Shout}
                Make a Willpower vs. Mental attack against all enemies within an \arealarge radius burst from you.
                Success against a target means it is \shaken by you as a \glossterm{condition}.

                \subcf{Gut Punch} Make a \glossterm{strike}.
                If your attack result beats the target's Fortitude defense, the target is \sickened as a condition.

                \subcf{Leaping Strike} You make a Jump check to leap and move as normal for the leap, up to a maximum distance equal to your land speed (see \pcref{Leap}).
                If you use this ability during the \glossterm{action phase}, you can also make a \glossterm{strike} from your new location during the \glossterm{delayed action phase}.

                \subcf{Primal Maneuver} You can use a \glossterm{combat maneuver} with a \plus2 bonus to accuracy.

                \subcf{Sweeping Strike} Make a melee \glossterm{strike} against each of up to three adjacent creatures.
                You take a \minus1d penalty to \glossterm{strike damage} with each strike.

                \subcf{Whirlwind Spin} Make a melee \glossterm{strike} against all creatures you threaten.
                You take a \minus3d penalty to \glossterm{strike damage} with each strike.
            }

            \cf{Bbn}[6]{Athletic Prowess} You gain two additional skill points that must be spent on Strength or Dexterity-based barbarian class skills.

            \cf{Bbn}[9]{Primal Exertion}
            You learn an additional \textit{primal exertion} from the list below.
            As a standard action, you can spend an \glossterm{action point} to use a \textit{primal exertion} ability.
            {
                \subcf{Certain Strike} You can make a \glossterm{strike} with a \plus2 bonus to accuracy.

                \subcf{Ground Pound} Make a Strength vs. Reflex attack against all enemies standing on solid ground within an \areamed radius burst from you.
                Success against a target means it is knocked \prone.
                You can only use this ability while standing on solid ground.
                If you use this ability during the \glossterm{action phase}, you can also make a \glossterm{strike} during the \glossterm{delayed action phase}.

                \subcf{Mighty Smash} Make a \glossterm{strike} with a \plus2d bonus to damage.

                \subcf{Penetrating Strike} Make a \glossterm{strike}.
                The attack is made against the target's Reflex defense instead of its Armor defense.

                \subcf{Sickening Sweep} This ability functions like the \textit{sweeping strike} \textit{primal exertion}, except that if your attack result beats a target's Fortitude defense, the target is \sickened as a condition.

                \subcf{Threatening Shout} 
                Make a Willpower vs. Mental attack against all enemies within an \arealarge radius burst from you.
                Success against a target means it is \frightened by you as a \glossterm{condition}.

                \subcf{War Cry} You and all allies within an \arealarge radius burst from you gain \glossterm{damage reduction} against all damage equal to your Willpower.
                This lasts as long as you \glossterm{sustain} this ability as a \glossterm{swift action}.
            }

            \cf{Bbn}[12]{Primal Prowess}
            You gain a \plus2 bonus to Strength-based and Dexterity-based checks.
            In addition, you may replace your 3rd-level \textit{primal exertion} with a 9th-level \textit{primal exertion}.

            \cf{Bbn}[15]{Primal Exertion}
            You learn an additional \textit{primal exertion} from the list below.
            As a standard action, you can spend an \glossterm{action point} to use a \textit{primal exertion} ability.
            {
                \subcf{Earthshattering Stomp} Make a Strength vs. Reflex attack against all enemies standing on solid ground within an \areamed radius burst from you.
                Success against a target means it is \stunned as a \glossterm{condition}.
                You can only use this ability while standing on solid ground.

                \subcf{Fated Strike} Make a \glossterm{strike} with a \plus3 bonus to accuracy.

                \subcf{Meteor Slam} You make a \glossterm{Jump} to leap and move as normal for the leap, up to a maximum distance equal to your land speed (see \pcref{Leap}).
                If you use this ability during the \glossterm{action phase}, you can also make a \glossterm{strike} during the \glossterm{delayed action phase}, using your downward momentum to slam into your foe.
                % TODO: wording
                You gain a \plus1d bonus to damage on this strike for every 10 feet of height you travelled downward towards your foe during the leap, up to a maximum of \plus4d.

                \subcf{Stunning Blow} Make a \glossterm{strike}.
                If your attack result beats the target's Fortitude defense, it is \stunned as a condition.

                \subcf{Titanic Smash} Make a \glossterm{strike} with a \plus3d bonus to damage.
            }

            \cf{Bbn}[18]{Primal Supremacy}
            You gain a \plus1 bonus to your choice of Strength, Dexterity, or Constitution.
            In addition, you may replace a 9th-level \textit{primal exertion} with a 15th-level \textit{primal exertion}.

            \cf{Bbn}[20]{Primal Soul} You do not need to spend an \glossterm{action point} to use your \textit{primal exertion} abilities.
            In addition, you may replace a 9th-level \textit{primal exertion} with a 15th-level \textit{primal exertion}.

        \subsubsection{Battleforged Resilience}
            \cf{Bbn}[2]{Battle-Scarred} You gain \glossterm{damage reduction} against physical damage equal to your level.

            \cf{Bbn}[5]{Uncanny Dodge} You can react to danger before your senses would normally allow you to do so.
            You reduce your \glossterm{overwhelm penalties} by 1.
            If your overwhelm penalty is reduced to 0, you are not considered to be overwhelmed.
            In addition, you are not \unaware when attacked by surprise.

            \cf{Bbn}[8]{Deep Scars} Your \glossterm{damage reduction} from your \textit{battle-scarred} ability applies against all damage, not just physical damage.

            \cf{Bbn}[11]{Greater Uncanny Dodge}
            Your reduction of \glossterm{overwhelm penalties} from the \textit{uncanny dodge} ability increases to 2.

            \cf{Bbn}[14]{Rapid Recovery}
            At the end of each round, you heal hit points equal to your level.

            \cf{Bbn}[17]{Uncanny Whirlwind}
            Your reduction of overwhelm penalties from the \textit{uncanny dodge} ability increases to 4.

            \cf{Bbn}[20]{Soulscarred}
            Your \glossterm{damage reduction} from your \textit{battle-scarred} ability increases to twice your level.

        \subsubsection{Ex-Barbarians}
            If you become lawful, you cannot use your \textit{rage} ability.
            You retain all of your other class abilities.
            If you stop being lawful, you can use your \textit{rage} ability once more.

\section{Cleric}\label{Cleric}
    \begin{dtable}
        \lcaption{Cleric Progression}
        \begin{dtabularx}{\columnwidth}{>{\ccol}p{2em} c c >{\lcol}X}
            \tb{Level} & \tb{Spells} & \tb{Subspells} & \tb{Abilities} \\\bottomrule
            \nth{1}     & 2 & \tdash   & Domain gift, rituals, spells
            \\ \nth{2}  & 2 & \tdash   & Spell point
            \\ \nth{3}  & 3 & \tdash   & Domain gift, spell knowledge
            \\ \nth{4}  & 3 & 2        & \tdash
            \\ \nth{5}  & 3 & 2        & Domain aspect
            \\ \nth{6}  & 3 & 3        & Augment
            \\ \nth{7}  & 3 & 3        & Domain aspect
            \\ \nth{8}  & 4 & 4        & \tdash
            \\ \nth{9}  & 4 & 4        & Cleansing prayer
            \\ \nth{10} & 4 & 5        & Augment
            \\ \nth{11} & 4 & 5        & Domain essence
            \\ \nth{12} & 4 & 6        & \tdash
            \\ \nth{13} & 4 & 6        & Domain essence
            \\ \nth{14} & 4 & 7        & Augment
            \\ \nth{15} & 4 & 7        & Domain mastery
            \\ \nth{16} & 4 & 8        & Spell point
            \\ \nth{17} & 4 & 8        & Domain mastery
            \\ \nth{18} & 4 & 9        & Augment
            \\ \nth{19} & 4 & 9        & Greater cleansing prayer
            \\ \nth{20} & 4 & 10       & Miracle
        \end{dtabularx}
    \end{dtable}

    \classbasics{Alignment} Your alignment must be within one step of your deity's (that is, it may be one step away on either the lawful-chaotic axis or the good-evil axis, but not both).

    \subsection{Class Abilities}
        If you are a cleric, you gain the following abilities.

        \classbasics{Skill Points} 4.

        \classbasics{Class Skills}
        \subparhead{Intelligence} Heal, Knowledge (arcana, local, religion, the planes), Linguistics.
        \subparhead{Perception} Awareness, Sense Motive, Spellcraft.
        \subparhead{Other} Bluff, Intimidate, Persuasion.

        \classbasics{Defenses} \plus3 Mental, \plus2 Fortitude, \plus1 Reflex.

        \cf{Clr}{Weapon and Armor Proficiency}
        You are proficient with simple weapons, any two other weapon groups, light and medium armor, and shields.

        \cf{Clr}{Divine Power}
        The \glossterm{power} of many cleric spells and abilities is determined by your \textit{divine power}.
        Your \textit{divine power} is equal to your level or your Willpower, whichever is higher.

        \cf{Clr}{Deity}
        You must worship a specific deity to be a cleric.
        Deities and their associated \glossterm{domains} are listed in \trefnp{Deities}.

        \begin{dtable!*}
            \lcaption{Deities}
            \begin{dtabularx}{\textwidth}{X l X}
                \tb{Deity} & \tb{Alignment} & \tb{Domains} \\
                \bottomrule
                Guftas, horse god of justice          & Lawful good     & Good, Law, Strength, Travel         \\
                Lucied, paladin god of justice        & Lawful good     & Destruction, Good, Protection, War  \\
                Simor, fighter god of protection      & Lawful good     & Good, Protection, Strength, War     \\
                %Pabst, dwarf god of drink             & Neutral good & Good, Life, Strength, Wild \\
                Rucks, monk god of pragmatism         & Neutral good    & Good, Law, Protection, Travel       \\
                Vanya, centaur god of nature          & Neutral good    & Good, Strength, Travel, Wild        \\
                Brushtwig, pixie god of creativity    & Chaotic good    & Chaos, Good, Trickery, Wild         \\
                Chavi, god of stories                 & Chaotic good    & Chaos, Knowledge, Trickery          \\
                Ivan Ivanovitch, bear god of strength & Chaotic good    & Chaos, Strength, War, Wild          \\
                Krunch, barbarian god of destruction  & Chaotic good    & Destruction, Good, Strength, War    \\
                Sir Cakes, dwarf god of freedom       & Chaotic good    & Chaos, Good, Strength               \\
                Raphael, monk god of retribution      & Lawful neutral  & Death, Law, Protection, Travel      \\
                Declan, god of fire                   & True neutral    & Destruction, Fire, Knowledge, Magic \\
                Kurai, shaman god of nature           & True neutral    & Air, Earth, Fire, Water             \\
                %Amanita, druid god of decay           & Chaotic neutral & Chaos, Destruction, Life, Wild \\
                %Antimony, elf god of necromancy       & Chaotic neutral & Death, Knowledge, Life, Magic \\
                Clockwork, elf god of time            & Chaotic neutral & Chaos, Magic, Trickery, Travel      \\
                %Lord Khallus, fighter god of pride    & Chaotic neutral & Chaos, Strength, War \\
                %Celeano, sorcerer god of deception    & Chaotic neutral & Chaos, Magic, Protection, Trickery \\
                Murdoc, god of mercenaries            & Chaotic neutral & Destruction, Knowledge, Travel, War \\
                Ribo, halfling god of trickery        & Chaotic neutral & Chaos, Trickery, Water              \\
                Tak, orc god of war                   & Lawful evil     & Law, Strength, Trickery, War        \\
                Theodolus, sorcerer god of ambition   & Neutral evil    & Evil, Knowledge, Magic, Trickery    \\
                Daeghul, demon god of slaughter       & Chaotic evil    & Destruction, Evil, Magic, War       \\
            \end{dtabularx}
        \end{dtable!*}

        \subsubsection{Spellcasting}

            \cf{Clr}{Divine Spells} 
            Your deity grants you the ability to cast divine spells.
            You learn two divine spells from the divine \glossterm{spell list} (see \pcref{Divine Spells}).
            Your \glossterm{spellpower} with divine spells is equal to your \glossterm{divine power}.

            To cast a spell, you must normally spend an \glossterm{action point}.
            Every spell can also be cast as a cantrip.
            Cantrips are weaker, but do not require action points to cast.

            You can't cast spells of an alignment opposed to your deity's.
            Spells associated with particular alignments are indicated by the \glossterm{Chaos}, \glossterm{Good}, \glossterm{Evil}, and \glossterm{Law} tags in their spell descriptions.

            \cf{Clr}{Rituals} 
            You can perform divine rituals to create unique magical effects (see \pcref{Rituals}).
            You have a ritual book containing one divine ritual of your choice (see \pcref{Divine Rituals}).

            \cf{Clr}[6]{Augment}
            Choose an \glossterm{augment} (see \pcref{Augments}).
            You can apply that augment to divine spells you cast and divine rituals you perform.
            At 10th level, and every four levels thereafter, you learn an additional augment.

            \cf{Clr}[8]{Spell Knowledge}
            You learn an additional divine spell (see \pcref{Divine Spells}).

            \cf{Clr}[16]{Spell Point} 
            You gain a spell point.
            A spell point can be spent to cast spells in place of an action point.
            You recover all spent spell points after a \glossterm{short rest}.

        \subsubsection{Domain Influence}
            All \textit{domain influence} abilities are \glossterm{Magical}.

            \cf{Clr}{Domains}
            You choose two domains which represent your personal spiritual inclinations.
            You must choose your domains from among those your deity offers.
            The domains are listed below.

            \begin{itemize}
                \item{Air}
                \item{Chaos}
                \item{Death}
                \item{Destruction}
                \item{Earth}
                \item{Evil}
                \item{Fire}
                \item{Good}
                \item{Knowledge}
                \item{Law}
                \item{Life}
                \item{Magic}
                \item{Protection}
                \item{Strength}
                \item{Travel}
                \item{Trickery}
                \item{War}
                \item{Water}
                \item{Wild}
            \end{itemize}

            \cf{Clr}{Domain Gift}
            Each domain has a corresponding \textit{domain gift}.
            You gain the \textit{domain gift} for one of your domains (see \pcref{Domain Gifts}).

            \cf{Clr}[3]{Domain Gift}
            You gain the \textit{domain gift} for another one of your domains.

            \cf{Clr}[5]{Domain Aspect}
            Each domain has a corresponding \textit{domain aspect}.
            You gain the \textit{domain aspect} for one of your domains (see \pcref{Domain Gifts}).

            \cf{Clr}[7]{Domain Aspect} 
            You gain the \textit{domain aspect} for another one of your domains.

            \cf{Clr}[9]{Cleansing Prayer}
            When you use the \textit{recover} ability, you heal \plus1d hit points.
            In addition, instead of removing a condition, you can remove any \glossterm{sustained} effect on you.

            \cf{Clr}[11]{Domain Essence}
            Each domain has a corresponding \textit{domain essence}.
            You gain the \textit{domain essence} for one of your domains (see \pcref{Domain Gifts}).

            \cf{Clr}[13]{Domain Essence} 
            You gain the \textit{domain essence} for another one of your domains.

            \cf{Clr}[15]{Domain Mastery}
            Each domain has a corresponding \textit{domain mastery}.
            You gain the \textit{domain mastery} for one of your domains (see \pcref{Domain Gifts}).

            \cf{Clr}[17]{Domain Mastery} 
            You gain the \textit{domain mastery} for another one of your domains.

            \cf{Clr}[19]{Greater Cleansing Prayer} 
            The bonus to hit point recovery from the \textit{cleansing prayer} ability increases to \plus2d.
            In addition, when you use the \textit{recover} ability, you can remove any number of \glossterm{sustained} effects on you instead of removing a condition.

            \cf{Clr}[20]{Miracle}
            Once per week, you can request a miracle as a standard action.
            You mentally specify your request, and your deity fulfills that request in the manner it sees fit.
            This can emulate the effects of any spell or ritual, or have any other effect of a similar power level.
            If the deity has a direct interest in your situation, the miracle may be of even greater power.

            If you perform an extraordinary service for your deity, you can gain the ability to request an additional miracle that week.

        \subsubsection{Divine Spell Mastery}
            You must have the ability to cast divine spells to gain these abilities.

            \cf{Clr}[2]{Spell Point}
            You gain a spell point.
            A spell point can be spent to cast spells in place of an action point.
            You recover all spent spell points after a \glossterm{short rest}.

            \cf{Clr}[3]{Spell Knowledge}
            You learn an additional divine spell (see \pcref{Divine Spells}).

            \cf{Clr}[4]{Subspells}
            Choose two \glossterm{subspells} for divine spells you know.
            You can use those subspell when you cast those spell (see \pcref{Subspells}).
            At 6th level, and every two levels thereafter, you learn an additional subspell for a divine spell you know.

    \subsection{Cleric Domain Abilities}
        All cleric domain abilities are \glossterm{magical} unless otherwise specified.

        \subsubsection{Air}
            \parhead{Gift} You add the Jump skill to your class skill list and gain a \plus5 bonus to Jump checks (see \pcref{Jump}).
            \parhead{Aspect} You gain a \glossterm{glide speed} equal to your land speed (see \pcref{Gliding}).
            \parhead{Essence} As a standard action, you can spend an \glossterm{action point} to use this ability.
            \begin{ability}
                \begin{spelltargetinginfo}
                    \spellrng{\rnglong}
                \end{spelltargetinginfo}
                \begin{spelleffects}
                    \spelleffect You can speak with and command air within range.
                    You can ask the air simple questions and understand its responses.
                    If you command the air to perform a task, it will do so do the best of its ability until this effect ends.
                    You cannot compel the air to move faster than 50 mph.
                    \spelldur Sustain (swift)
                    \spellspecial After you use this ability on a particular area of air, you cannot use it again on that same area for 24 hours.
                \end{spelleffects}
            \end{ability}
            \parhead{Mastery} You gain a \glossterm{fly speed} with \glossterm{good maneuverability} equal to your land speed (see \pcref{Flying}).

        \subsubsection{Chaos}
            \parhead{Gift} Whenever you roll a 10 on a check on your first attempt, you gain a \plus5 bonus to the check.
            \parhead{Aspect} If you roll a 1 on an attack roll, it explodes (see \pcref{Exploding Attacks}).
            This does not affect additional dice rolled if the attack roll explodes.
            \parhead{Essence} As a standard action, you can spend an \glossterm{action point} to use this ability.
            \begin{ability}
                \begin{spelltargetinginfo}
                    \spellrng{\rnglong}
                \end{spelltargetinginfo}
                \begin{spelleffects}
                    \spelleffect An improbable event occurs.
                    You can specify in general terms what you want to happen, such as ``Make the bartender leave the bar''.
                    You cannot control the exact nature of the event, though it always beneficial for you in some way.
                    \spellspecial After using this ability, you cannot use it again for an hour.
                \end{spelleffects}
            \end{ability}
            \parhead{Mastery} Whenever you make an attack roll, if it misses, you can reroll.
            You must accept the second result.

        \subsection{Death}
            \parhead{Gift} Whenever you deal damage to a creature with no hit points remaining, it immediately dies.
            This is a \glossterm{Death} effect.
            % Needs wording to clarify interaction with simultaneous damage
            \parhead{Aspect} Whenever you deal damage to a creature, any of your damage in excess of that creature's hit points is dealt as \glossterm{vital damage}.
            In addition, you are immune to \glossterm{Death} effects.
            \parhead{Essence} As a standard action, you can spend an \glossterm{action point} to use this ability.
            \begin{ability}
                \begin{spelltargetinginfo}
                    \spellquicktargeting{One living creature}{\rngmed}
                    \spellspecial When you use this ability, you choose whether to summon the essence of Death or banish it.
                \end{spelltargetinginfo}
                \begin{spelleffects}
                    \spelleffect If you summoned the essence of Death, the target draws closer to the brink.
                    If it takes \glossterm{vital damage}, it immediately dies.

                    If you banished the essence of Death, the target draws closer to life.
                    It is immune to \glossterm{Death} effects and gains a bonus to \glossterm{stabilization rolls} equal to your \textit{divine power}.
                    \spelldur Attunement
                    \spelltags{\glossterm{Death}}
                \end{spelleffects}
            \end{ability}
            \parhead{Mastery} You constantly radiate an emanation of death, described below.
            \begin{ability}
                \begin{spelltargetinginfo}
                    \spellarea{\areahuge radius emanation from you}
                    \spelltgts{All living enemies in the area}
                \end{spelltargetinginfo}
                \begin{spelleffects}
                    \spelleffect If the target takes \glossterm{vital damage}, it immediately dies.
                \end{spelleffects}
            \end{ability}

        \subsubsection{Destruction}
            \parhead{Gift} Your attacks ignore an amount of \glossterm{hardness} and \glossterm{damage reduction} equal to your \textit{divine power}.
            \parhead{Aspect} You gain a \plus1d bonus to \glossterm{strike damage}.
            \parhead{Essence} As a standard action, you can spend an \glossterm{action point} to use this ability.
            \begin{ability}
                \begin{spelltargetinginfo}
                    \spellarea{\arealarge radius burst}
                    \spelltgts{All unattended objects in the area}
                    \spellspecial You may freely exclude any number of 5-ft\. cubes from the area, as long as the resulting area is still contiguous.
                \end{spelltargetinginfo}
                \begin{spelleffects}
                    \begin{spellattack}{Divine power vs. Fortitude}
                        \spellspecial Nonmagical objects do not have a Fortitude defense, so this attack automatically succeeds against such objects.
                        \spellsuccess The target crumbles into a fine powder and is irreparably \glossterm{broken}.
                    \end{spellattack}
                \end{spelleffects}
            \end{ability}
            \parhead{Mastery} Whenever you deal damage to a creature or object, its damage reduction and hardness (if any) are reduced by an amount equal to your \textit{divine power}.
            In addition, Fortitude defense is reduced by 2.
            This is a \glossterm{condition}, and lasts until it is removed.
            This effect stacks with itself, but can only be applied to a target once per round.
            % how do you remove a condition from an object???

        \subsubsection{Earth}
            \parhead{Gift} You gain a \plus2 bonus to Fortitude defense.
            \parhead{Aspect} You gain \glossterm{damage reduction} against physical damage equal to your \textit{divine power}.
            \parhead{Essence} As a standard action, you can spend an \glossterm{action point} to use this ability.
            \begin{ability}
                \begin{spelltargetinginfo}
                    \spellrng{\rnglong}
                \end{spelltargetinginfo}
                \begin{spelleffects}
                    \spelleffect You can speak with and command earth within range.
                    You can ask the earth simple questions and understand its responses.
                    If you command the earth to perform a task, it will do so do the best of its ability until this effect ends.
                    You cannot compel the earth to move faster than 10 feet per round.
                    \spelldur Sustain (swift)
                    \spellspecial After you use this ability on a particular area of earth, you cannot use it again on that same area for 24 hours.
                \end{spelleffects}
            \end{ability}
            \parhead{Mastery} You gain the \glossterm{earth glide} ability, as an earth elemental.
            This allows you to glide through stone, dirt, or almost any other sort of earth as if it were air.
            You can walk or climb at any angle in the earth.
            However, you cannot breathe, speak, or hear while gliding in this way.
            While gliding, you can remain partially within the earth, granting you cover.

        \subsubsection{Evil}
            \parhead{Gift} Whenever you take damage, you may choose an adjacent willing creature.
            If you do, that creature takes half of that damage (rounded down) instead of you.
            Any abilities it has that would make the attack miss or fail have no effect, but its abilities that allow it to reduce or ignore the attack's effects work normally.
            You take the remaining half of the damage, and suffer any non-damaging effects of the attack normally.
            \par You may learn which attacks hit you in the current phase before deciding which attacks to redirect damage from, but not how much damage they would do (or any other effects they might have).
            \parhead{Aspect} You can use this domain's domain gift to redirect damage to any willing creature within \rngclose range.
            \parhead{Essence} As a standard action, you can spend an \glossterm{action point} to use this ability.
            \begin{ability}
                \begin{spelltargetinginfo}
                    \spellquicktargeting{One creature}{\rngmed}
                \end{spelltargetinginfo}
                \begin{spelleffects}
                    \begin{spellattack}{Divine power vs. Mental}
                        \spellspecial Creatures who have strict codes prohibiting them from taking evil actions, such as paladins devoted to Good, are immune to this attack.
                        \spellsuccess The target takes an evil action as soon as it can.
                        You have no control over the act the creature takes, but circumstances can make the target more likely to take an action you desire.
                    \end{spellattack}
                    \spelltags{\glossterm{Compulsion}, \glossterm{Mind}}
                \end{spelleffects}
            \end{ability}
            \parhead{Mastery} Whenever you take damage, you may use this ability as an \glossterm{immediate action}.
            If you do, you can use this domain's domain gift to redirect damage to an unwilling creature within \rngmed range.

        \subsubsection{Fire}
            \parhead{Gift} All of your \glossterm{Fire} spells and abilities do not deal damage to your allies.
            \parhead{Aspect} Whenever you would take fire damage, you heal that many hit points instead.
            This applies damage reduction, damage immunity, and similar effects.
            \parhead{Essence} As a standard action, you can spend an \glossterm{action point} to use this ability.
            \begin{ability}
                \begin{spelltargetinginfo}
                    \spellrng{\rnglong}
                \end{spelltargetinginfo}
                \begin{spelleffects}
                    \spelleffect You can speak with and command fire within range.
                    You can ask the fire simple questions and understand its responses.
                    If you command the fire to perform a task, it will do so do the best of its ability until this effect ends.
                    You cannot compel the fire to move farther than 30 feet in a single round.
                    Fire that ends the round on non-combustable materials usually goes out, depending on the circumstances.
                    \spelldur Sustain (swift)
                    \spellspecial After you use this ability on a particular area of fire, you cannot use it again on that same area for 24 hours.
                    % TODO: What does an ``area of fire'' mean?
                \end{spelleffects}
            \end{ability}
            \parhead{Mastery} Whenever you deal fire damage to a creature, that creature becomes \ignited.
            This is a \glossterm{condition}, and lasts until removed.

        \subsubsection{Good}
            \parhead{Gift} Whenever an adjacent creature takes damage, you may use this ability.
            If you do, you take half of that damage (rounded down) instead of the creature.
            Any abilities you have that would make the attack miss or fail have no effect, but your abilities that allow you to reduce or ignore its effects work normally.
            The protected creature takes the remaining half of the damage, and suffers any non-damaging effects of the attack normally.
            \par You may learn which attacks hit you and your allies in the current phase before deciding which attacks to redirect damage from, but not how much damage they would do (or any other effects they might have).
            \parhead{Aspect} You can use this domain's domain gift to redirect damage from any creature within \rngclose range.
            \parhead{Essence} As a standard action, you can spend an \glossterm{action point} to use this ability.
            \begin{ability}
                \begin{spelltargetinginfo}
                    \spellquicktargeting{One creature}{\rngmed}
                \end{spelltargetinginfo}
                \begin{spelleffects}
                    \begin{spellattack}{Divine power vs. Mental}
                        \spellspecial Creatures who have strict codes prohibiting them from taking good actions, such as paladins devoted to Evil, are immune to this attack.
                        \spellsuccess The target takes an good action as soon as it can.
                        You have no control over the act the creature takes, but circumstances can make the target more likely to take an action you desire.
                    \end{spellattack}
                    \spelltags{\glossterm{Compulsion}, \glossterm{Mind}}
                \end{spelleffects}
            \end{ability}
            \parhead{Mastery} Whenever you redirect damage with this domain's domain gift, you can redirect all effects of the attack to you instead of only half the damage.

        \subsubsection{Knowledge}
            \parhead{Gift} You add all Knowledge skills to your cleric class skill list.
            In addition, you gain two skill points which must be spent on Knowledge skills.
            \parhead{Aspect} Your extensive knowledge of all methods of attack and defense grants you a \plus1 bonus to all defenses.
            \parhead{Essence} As a standard action, you can spend an \glossterm{action point} to use this ability.
            \begin{ability}
                \begin{spelltargetinginfo}
                    \spellarea{\arealarge radius burst}
                    \spelltgts{All willing creatures in the area}
                \end{spelltargetinginfo}
                \begin{spelleffects}
                    \spelleffect You make a Knowledge check of any kind.
                    You gain a bonus to the Knowledge check equal to your \textit{divine power}.
                    The target also learns the results of your check.
                    It believes the information gained in this way to be true as if it had seen it with its own eyes.
                    \par You cannot alter the knowledge you gain with this check in any way, such as by adding or withholding information.
                \end{spelleffects}
            \end{ability}
            \parhead{Mastery} You gain a \plus1 bonus to accuracy with all attacks.

        \subsubsection{Law}
            \parhead{Gift} You gain a \plus2 bonus to Mental defense.
            % Clarify - does this apply to exploding dice?
            \parhead{Aspect} Whenever you roll a 1 on an \glossterm{attack roll}, it is treated as if you had rolled a 6.
            \parhead{Essence} As a standard action, you can spend an \glossterm{action point} to use this ability.
            \begin{ability}
                \begin{spelltargetinginfo}
                    \spellarea{\arealarge radius burst}
                    \spelltgts{All creatures in the area}
                \end{spelltargetinginfo}
                \begin{spelleffects}
                    \begin{spellattack}{Divine power vs. Mental}
                        \spellspecial This attack automatically succeeds against you.
                        \spellsuccess The target is unable to break the laws that apply in the area, and any attempt to do so simply fails.
                        The laws which are applied are those which are most appropriate for the area, regardless of whether the cleric or any other creature know those laws.
                        In areas under ambiguous or nonexistent government, this ability may have unexpected effects, or it may have no effect at all.
                    \end{spellattack}
                    \spelldur Condition
                    \spelltags{\glossterm{Compulsion}, \glossterm{Mind}}
                \end{spelleffects}
            \end{ability}
            \parhead{Mastery} Whenever you roll less than a 5 on an \glossterm{attack roll}, it is treated as if you had rolled a 5.

        \subsubsection{Life}
            \parhead{Gift} You gain additional hit points equal to your \glossterm{divine power}.
            \parhead{Aspect} All of your healing spells and abilities can cure \glossterm{vital damage} as easily as they cure hit points.
            \parhead{Essence} As a standard action, you can spend all of your remaining \glossterm{action points} (minimum 1) to use this ability.
            \begin{ability}
                \begin{spelltargetinginfo}
                    \spellquicktargeting{One dead creature}{Adjacent}
                \end{spelltargetinginfo}
                \begin{spelleffects}
                    \spelleffect If the target was dead for no more than 5 minutes, it is restored to life, as the \ritual{resurrection} ritual.
                \end{spelleffects}
            \end{ability}
            % Need special text to clarify that it is affected by the Aspect?
            \parhead{Mastery} At the end of each round, you heal hit points equal to your \glossterm{divine power}.

        \subsubsection{Magic}
            \parhead{Gift} You gain a \plus1 bonus to spellpower with divine spells.
            \parhead{Aspect} The maximum spell level you can cast is increased by 1.
            \parhead{Essence} You gain \glossterm{magic resistance} equal to 5 \add your \textit{divine power}.
            If you already have \glossterm{magic resistance}, you can instead increase it by 2.
            \parhead{Mastery} You gain a \plus2 bonus to your \glossterm{magic resistance}.
            If you resist an ability with your magic resistance, you heal hit points equal to your \textit{divine power}.

        \subsubsection{Strength}
            \parhead{Gift} You add Climb, Jump, Sprint, and Swim to your cleric class skill list.
            In addition, you gain two skill points which must be spent on any combination of those skills.
            \parhead{Aspect} You may use your Strength to determine your \textit{divine power} in place of your Willpower.
            \parhead{Essence} You gain a \plus5 bonus to Strength for the purpose of checks and determining your carrying capacity.
            \parhead{Mastery} You gain a \plus1 bonus to your starting Strength.
            If this would increase your starting Strength above 4, you simply gain a \plus1 bous to your Strength.

        \subsubsection{Travel}
            \parhead{Gift} You add Knowledge (geography), Sprint, and Survival to your cleric class skill list.
            In addition, you gain two skill points which must be spent on any combination of those skills.
            \parhead{Aspect} You gain a \plus30 foot bonus to your speed in all movement modes, up to a maximum of double your normal speed.
            \parhead{Essence} As a standard action, you can spend an \glossterm{action point} to use this ability.
            \begin{ability}
                \begin{spelleffects}
                    \spelleffect You teleport yourself up to 1 mile in any direction.
                    You do not need \glossterm{line of sight} or \glossterm{line of effect} to your destination, but you must be able to clearly visualize it.
                \end{spelleffects}
            \end{ability}
            \parhead{Mastery} Whenever you move, you can teleport the same distance instead.
            This does not change the total distance you can move, but you can teleport in any direction, including vertically.

            You can even attempt to move to locations outside of \glossterm{line of sight} and \glossterm{line of effect}, up to the limit of your remaining movement speed.
            If your intended destination is invalid, the distance you tried to teleport is taken from your remaining movement, but you suffer no other ill effects.

        \subsubsection{Trickery}
            \parhead{Gift} You add Bluff, Disguise, and Stealth to your cleric class skill list.
            In addition, you gain two skill points which must be spent on any combination of those skills.
            \parhead{Aspect} You gain a \plus2 bonus to Bluff, Disguise, and Stealth.
            \parhead{Essence} As a standard action, you can spend an \glossterm{action point} to use this ability.
            \begin{ability}
                \begin{spelltargetinginfo}
                    \spellquicktargeting{One creature}{\rngmed}
                    \spellspecial Choose a belief the target has.
                    The belief may be a lie that you told it, or even a simple misunderstanding (such as believing a hidden creature is not present in a room).
                    If the creature does not already hold the chosen belief, this ability automatically fails.
                \end{spelltargetinginfo}
                \begin{spelleffects}
                    \begin{spellattack}{Divine power vs. Mental}
                        \spellsuccess The target continues to maintain the chosen belief, regardless of any evidence to the contrary.
                        It will interpret any evidence that the falsehood is incorrect to be somehow wrong -- an illusion, a conspiracy to decieve it, or any other reason it can think of to continue believing the falsehood.
                        At the end of the effect, the creature can decide whether it believes the falsehood or not, as normal.
                    \end{spellattack}
                    \spelldur Sustain (Swift)
                    \spelltags{\glossterm{Delusion}, \glossterm{Mind}}
                \end{spelleffects}
            \end{ability}
            \parhead{Mastery} You are undetectable by Divination spells and effects.
            They cannot detect your presence, sounds you make, or any actions you take.

        \subsubsection{War}
            \parhead{Gift} You gain proficiency with heavy armor, tower shields, and an additional weapon group of your choice.
            \parhead{Aspect} You gain a \plus1 bonus to \glossterm{accuracy} with \glossterm{strikes}.
            \parhead{Essence} Whenever you cast a spell, you can use this ability as an \glossterm{immediate action} by spending an \glossterm{action point}.
            If you do, the spell gains one of the following effects:
            \begin{itemize}
                \item Legion: If the spell would normally affect five or more specific targets, its range is doubled and it instead affects five times that many targets.
                \item Selective: If the spell has an area, it has no effect on your allies in the area.
                \item Widened: If the spell has an area, the size of the area is doubled.
            \end{itemize}
            \parhead{Mastery} You and all allies within a \arealarge radius emanation of you gain a \plus1 bonus to \glossterm{accuracy} with \glossterm{strikes}.

        \subsubsection{Water}
            \parhead{Gift} You add Swim to your cleric class skill list and gain a \plus5 bonus to Swim checks.
            \parhead{Aspect} You can breathe water as easily as a human breathes air, preventing you from drowning or suffocating underwater.
            You also gain a \glossterm{swim speed} equal to your land speed.
            \parhead{Essence} As a standard action, you can spend an \glossterm{action point} to use this ability.
            \begin{ability}
                \begin{spelltargetinginfo}
                    \spellrng{\rnglong}
                \end{spelltargetinginfo}
                \begin{spelleffects}
                    \spelleffect You can speak with and command water within range.
                    You can ask the water simple questions and understand its responses.
                    If you command the water to perform a task, it will do so do the best of its ability until this effect ends.
                    You cannot compel the water to move faster than 30 feet per round.
                    \spelldur Sustain (swift)
                    \spellspecial After you use this ability on a particular area of water, you cannot use it again on that same area for 24 hours.
                \end{spelleffects}
            \end{ability}
            \parhead{Mastery}
            Whenever you move, you can transform yourself into a rushing flow of water with a volume roughly equal to your normal volume until your movement is complete
            In this form, you may move wherever water could go, you cannot take other actions, such as jumping, attacking, or casting spells.
            You may move through squares occupied by creatures or threatened by blocking enemies without penalty.
            \par Your speed is halved when moving uphill and doubled when moving downhill.
            Unusually steep inclines may cause greater movement differences while in this form.
            \par If the water is split, you may reform from anywhere the water has reached, to as little as a single ounce of water.
            If not even an ounce of water exists contiguously, your body reforms from all of the largest available sections of water, cut into pieces of appropriate size.
            This usually causes you to die.

        \subsubsection{Wild}
            \parhead{Gift} You add Creature Handling, Knowledge (nature), and Survival to your cleric class skill list.
            In addition, you gain two skill points which must be spent on any combination of those skills.
            % Does this even make sense?
            \parhead{Aspect} When you gain this ability, you choose one wild aspect ability, as if you were a druid of a level equal to your cleric level (see Wild Aspect, \pref{Drd:Wild Aspect}).
            As a standard action, you can spend an \glossterm{action point} to embody that wild aspect for 1 hour.

        \subsubsection{Ex-Clerics}
            If you grossly violate the code of conduct required by your deity, you lose all spells and magical cleric class abilities.
            You cannot regain those abilities until you atone for your transgressions to your deity.

\section{Druid}\label{Druid}
    \begin{dtable}
        \lcaption{Druid Progression}
        \begin{dtabularx}{\columnwidth}{>{\ccol}p{\levelcol} >{\ccol}p{3.5em} c >{\lcol}X}
            \tb{Level} & \tb{Spells} & \tb{Subspells} & \tb{Abilities} \\\bottomrule
            \nth{1}     & 2 & \tdash   & Rituals, spells, wild speech
            \\ \nth{2}  & 2 & \tdash   & Natural lore, spell point
            \\ \nth{3}  & 2 & \tdash   & Spell knowledge, wild aspect
            \\ \nth{4}  & 2 & 2        & \tdash
            \\ \nth{5}  & 3 & 2        & Natural vigor
            \\ \nth{6}  & 3 & 3        & Augment
            \\ \nth{7}  & 3 & 3        & Wild aspect
            \\ \nth{8}  & 4 & 4        & \tdash
            \\ \nth{9}  & 4 & 4        & Natural lore
            \\ \nth{10} & 4 & 5        & Augment
            \\ \nth{11} & 4 & 5        & Wild aspect
            \\ \nth{12} & 4 & 6        & \tdash
            \\ \nth{13} & 4 & 6        & Natural vigor
            \\ \nth{14} & 4 & 7        & Augment
            \\ \nth{15} & 4 & 7        & Wild aspect
            \\ \nth{16} & 4 & 8        & Spell point
            \\ \nth{17} & 4 & 8        & Nature's champion
            \\ \nth{18} & 4 & 9        & Augment
            \\ \nth{19} & 4 & 9        & Wild aspect
            \\ \nth{20} & 4 & 10       & Avatar of nature
        \end{dtabularx}
    \end{dtable}

    \classbasics{Alignment} Neutral good, lawful neutral, neutral, chaotic neutral, or neutral evil.

    \subsection{Class Abilities}
        If you are a druid, you gain the following abilities.

        \classbasics{Skill Points} 6.

        \classbasics{Class Skills}
        \subparhead{Strength} Climb, Jump, Sprint, Swim.
        \subparhead{Dexterity} Acrobatics, Ride, Stealth.
        \subparhead{Intelligence} Heal, Knowledge (geography, nature).
        \subparhead{Perception} Awareness, Creature Handling, Survival.
        \subparhead{Other} Bluff, Intimidate, Persuasion.

        \classbasics{Defenses} \plus3 Fortitude, \plus2 Mental, \plus1 Reflex.

        \cf{Drd}{Weapon and Armor Proficiency}
        Druids are proficient with simple weapons, any one other weapon group, scimitars, sickles, and slings.
        In addition, druids are proficient with light armor, medium armor, and shields.
        However, a druid cannot use metal armor; see the Metal Abhorrence ability, below.

        \cf{Drd}{Druidic Language}
        You know Druidic, a secret language known only to druids, in addition to your normal languages.
        Druids are forbidden to teach this language to nondruids.
        Druidic has its own alphabet.

        \cf{Drd}{Metal Abhorrence}
        The oaths that you swear as part of your druidic initiation prohibit you from wearing armor made of metal.
        If you wear prohibited armor or carry a prohibited shield, you are unable to cast druid spells or use any of your \glossterm{magical} druid abilities while doing so and for 24 hours thereafter.
        
        You can avoid this penalty by using armor made of wood altered with the \ritual{ironwood} ritual.
        Such wood is as strong as steel.

        \cf{Drd}{Nature Power}
        The \glossterm{power} of many druid spells and abilities is determined by your \textit{nature power}.
        Your \textit{nature power} is equal to your level or your Perception, whichever is higher.

        \subsubsection{Spellcasting}

            \cf{Drd}{Nature Spells} 
            Your worship of nature grants you the ability to cast nature spells.
            You learn two nature spells from the nature \glossterm{spell list} (see \pcref{Nature Spells}).
            Your \glossterm{spellpower} with nature spells is equal to your \glossterm{nature power}.

            To cast a spell, you must normally spend an \glossterm{action point}.
            Every spell can also be cast as a cantrip.
            Cantrips are weaker, but do not require action points to cast.

            You can't cast spells of an alignment opposed to your own or your deity's.
            Spells associated with particular alignments are indicated by the \glossterm{Chaos}, \glossterm{Good}, \glossterm{Evil}, and \glossterm{Law} tags in their spell descriptions.

            \cf{Drd}{Rituals} 
            You can perform nature rituals to create unique magical effects (see \pcref{Rituals}).
            You have a ritual book containing one nature ritual of your choice (see \pcref{Nature Rituals}).

            \cf{Drd}[6]{Augment}
            Choose an \glossterm{augment} (see \pcref{Augments}).
            You can apply that augment to nature spells you cast and nature rituals you perform.
            At 10th level, and every four levels thereafter, you learn an additional augment.

            \cf{Drd}[8]{Spell Knowledge}
            You learn an additional nature spell (see \pcref{Nature Spells}).

            \cf{Drd}[16]{Spell Point} 
            You gain a spell point.
            A spell point can be spent to cast spells in place of an action point.
            You recover all spent spell points after a \glossterm{short rest}.

        \subsubsection{Natural Influence}

            \cf{Drd}{Wild Speech}[Magical] As a standard action, you can spend an \glossterm{action point} to use this ability.
            \begin{ability}
                \begin{spelltargetinginfo}
                    \spellquicktargeting{One animal}{\rngmed}
                \end{spelltargetinginfo}
                \begin{spelleffects}
                    \spelleffect You can speak to and understand the speech of the target animal, and any other animals of the same species.
                    This ability doesn't make the target any more friendly or cooperative than normal.
                    Wary and cunning animals are likely to be terse and evasive, while stupid ones tend to make inane comments and are unlikely to say or understand anything of use.
                    \spelldur Sustain (swift)
                \end{spelleffects}
            \end{ability}

            \cf{Drd}[2]{Natural Lore}
            You gain two extra skill points which must be spent on the Creature Handling, Heal, Knowledge (geography), Knowledge (nature), Ride, or Survival skills.

            \cf{Drd}[3]{Wild Aspect}[Magical]
            You gain the ability to embody an aspect of an animal or of nature itself.
            Choose a single wild aspect from the list below.
            Many wild aspects have a minimum level prerequisite, as indicated in the title of the ability.
            That ability is normally active.
            You may suppress or resume the effects of any number of \textit{wild aspects} you have as a \glossterm{swift action}.

            The abilities in the list below describe the effects of the aspect.
            Your appearance also changes to match the aspect's effects, but the nature of this change is not described.
            Different druids change in different ways.
            For example, one druid might gain unusually large eyes when embodying the low-light vision aspect, while another might change their irises into slits, like a cat, when embodying the same aspect.
            You choose how your appearance changes when you gain a wild aspect.
            This change cannot be used to gain an additional substantive benefit beyond the effects given in the description of the aspect.

            Many wild aspects grant natural weapons.
            See \pcref{Natural Weapons}, for details about natural weapons.
            At 7th level, and every four levels thereafter, you gain an additional wild aspect.

            \subcf{Animal Affinity}
            You gain a \plus2 bonus to Creature Handling and Ride checks.
            \subcf{Armaments of the Bear}
            Your mouth and hands transform, allowing you to perform bite and claw attacks.
            The bite attack deals 1d8 damage for a Medium creature, and the claws deal 1d6 damage.
            \subcf{Gore}
            Your head transforms, allowing you to perform a gore attack.
            The attack deals 1d8 damage for a Medium druid.
            In addition, you gain a \plus2 bonus to accuracy with shove attacks (see \pcref{Shove}).
            \subcf{Monkey Climb}
            You gain a \glossterm{climb speed} equal to your land speed.
            \subcf{Senses}
            You gain low-light vision.
            You treat sources of light as if they had double their normal illumination range.
            If you already have low-light vision, you double its benefit, allowing you to treat sources of light as if they had four times their normal illumination range.
            In addition, you gain \glossterm{darkvision} out to 50 feet, allowing you to see in complete darkness.
            If you already have darkvision, you increase its range by 50 feet.
            \subcf{Woodland Stride}
            You may move through any sort of undergrowth (such as natural thorns, briars, overgrown areas, and similar terrain) at your normal speed and without taking damage or suffering any other impairment.
            The plants bend of their own volition to allow the druid to pass.
            However, plants magically manipulated to impede motion still affect your.

            \subcf{7th -- A Thousand Faces}
            You may use your spellpower in place of your Disguise skill when making Disguise checks to alter your own appearance.
            \subcf{7th -- Constrict}
            Your body transforms, improving your grappling abilities.
            You gain a \plus2 bonus to accuracy with grapple attacks (see \pcref{Grapple}).
            In addition, you gain a constrict attack.
            This attack deals 1d10 damage for a Medium druid, but it can only be used against a foe you aregrappling with.
            \subcf{7th -- Hawk}
            You grow wings, granting your a glide speed equal to your land speed.
            See \pcref{Gliding}, for more details.
            In addition, your feet transform, allowing you to perform a talon attack.
            The attack deals 1d6 damage for a Medium creature.
            \subcf{7th -- Lope}
            You gain the ability to move on all four limbs.
            When doing so, you gain a \plus30 foot bonus to your land speed, up to a maximum of double your original speed.
            When not using your hands to move, your ability to use your hands is unchanged.
            Descending to four legs and rising up to stand on two legs again does not take an action.
            \subcf{7th -- Scent}
            You gain the \glossterm{scent} ability.
            \subcf{7th -- Shrink}
            You shrink by one size category (see \pcref{Size in Combat}).
            This is a \glossterm{Sizing} effect.
            \subcf{7th -- Slither}
            You gain a \glossterm{climb speed} equal to your land speed.
            You do not need to use your hands to climb in this way.
            In addition, you gain a bite attack that deals 1d8 damage for a Medium druid.
            % \subcf{7th -- Spikes}
            % Whenever a creature adjacent to the druid makes a physical attack against you, the attacking creature takes 1d4 piercing damage \plus1d per two nature power.
            % A creature can only be dealt damage by this effect once per round.

            % Is this how poison actually works?

            % \subcf{11th -- Elemental Retribution}
            % Whenever a creature within \rngmed range of you attacks you, the attacking creature takes 1d6 damage per two nature power of either cold, electricity, or fire damage.
            % You may choose the damage type independently for each attacking creature.
            % A creature can only be dealt damage by this effect once per round.
            \subcf{11th -- Beetle's Carapace} You gain a \plus1 bonus to Armor defense.
            \subcf{11th -- Fluid Motion} You are immune to effects that restrict your mobility, and you suffer no penalties for acting underwater.
            In addition, you gain a \plus10 bonus to Reflex defense against grapple attacks, as well as on grapple attacks or Escape Artist checks made to escape a grapple or a pin.
            \subcf{11th -- Grow}
            You increase in size by one size category (see \pcref{Size in Combat}).
            This is a \glossterm{Sizing} effect.
            \subcf{11th -- Natural Grab}
            If you hit with a natural attack, you may attempt to grapple your foe as an immediate action.
            \subcf{11th -- Natural Trip}
            If you hit with a natural attack, you may attempt to trip your foe as an immediate action.
            \subcf{11th -- Wolfpack}
            Overwhelmed foes you threaten increase their \glossterm{overwhelm penalties} by 1.
            \subcf{11th -- Venom}
            If you hit with a natural attack, you may inject poison into your foe as an immediate action.
            This is a \glossterm{condition}, and lasts until removed.
            At the end of each round, you make a \textit{nature power} vs. Fortitude attack against all creatures you have poisoned.
            The effects of the poison are described below.
            \begin{itemize}
                \item First success: the target is \sickened until the effect ends.
                \item Second success: the target is \staggered until the end of the next round.
                \item Third success: the target is \nauseated until the effect ends.
                \item Third failure: the target is no longer poisoned, and any lingering effects from the poison end.
            \end{itemize}
            % TODO poison stacking rules
            \par In addition, you gains a bite attack that deals 1d8 damage for a Medium druid.

            \subcf{15th -- Earth Glide}
            You gain the earth glide ability, as an earth elemental.
            This allows you to glide through stone, dirt, or almost any other sort of earth as if it were air.
            You can walk or climb at any angle in the earth.
            However, you cannot breathe, speak, or hear while gliding in this way.
            While gliding, you can remain partially within the earth, granting you cover.
            \subcf{15th -- Natural Renewal}
            At the end of each round, you heal hit points equal to your \textit{nature power}.
            \subcf{15th -- Wings}
            You grow wings, granting you a \glossterm{fly speed} equal to your land speed.
            While \unencumbered, you can fly (see \pcref{Flying}).
            % TODO: standard flight limitation

            \subcf{19th -- Solar Radiance}
            The druid continuously radiates bright light out to a 500 foot radius (and shadowy illumination for an additional 500 feet).
            The illumination is so bright that you become hard to look at.
            Any creature attacking your from within the radius of bright light becomes \partiallyblinded for 2 rounds after the attack.

            \cf{Drd}[5]{Natural Vigor}
            At the end of each round, you heal hit points equal to half your \textit{nature power}.

            \cf{Drd}[9]{Natural Lore}
            You gain two extra skill points which must be spent on the Creature Handling, Heal, Knowledge (geography), Knowledge (nature), Ride, or Survival skills.

            \cf{Drd}[13]{Greater Natural Vigor}
            Your healing from the \textit{natural vigor} ability increases to be equal to your \textit{nature power}.

            \cf{Drd}[17]{Nature's Champion}
            You gain a \plus2 bonus to the Creature Handling, Heal, Knowledge (geography), Knowledge (nature), Ride, and Survival skills.

            \cf{Drd}[20]{Avatar of Nature}[Magical]
            If you die, except if by old age, you may choose to have your body and soul become an instrument of nature's will.
            Your body immediately decomposes or otherwise disappears, and your soul does not travel to an afterlife.
            You has no physical form, and cannot use any of your normal abilities.
            Instead, you have a fly speed of 100 feet, with special maneuverability.
            As a standard action, you can temporarily possess any living plants or animals within a 10 mile radius of the place of your death.

            While possessing a living plant or animal, you can see through its senses and control its actions completely.
            In addition, you may cast spells, and the spells take effect as if the plant or animal had cast them.
            You use the plant or animal's position to determine range, visible targets, and so on.
            You do not require verbal or somatic components to cast your spells in this form.
            % TODO: are there any spells or rituals that have material components or focus objects?
            % but are unable to cast spells or perform rituals that require material components or focus objects.

            While not possessing a plant or animal, you can rest, or you can focus on reincarnating your physical form.
            Creating a new body in this way takes 12 consecutive hours of concentration.
            At the end of that time, you are reincarnated in a new body in your location, as the effect of the \ritual{reincarnation} ritual, except that you can choose your race from among the races listed (not including the ``Other'' race).

            While you are an avatar of nature, you do not age and you cannot die of old age.
            You can continue to exist in this form indefinitely.

        \subsubsection{Nature Spell Mastery}
            You must have the ability to cast nature spells to gain these abilities.

            \cf{Drd}[2]{Spell Point}
            You gain a spell point.
            A spell point can be spent to cast spells in place of an action point.
            You recover all spent spell points after a \glossterm{short rest}.

            \cf{Drd}[3]{Spell Knowledge} 
            You learn an additional nature spell (see \pcref{Nature Spells}).

            \cf{Drd}[4]{Subspells}
            Choose two \glossterm{subspells} for nature spells you know.
            You can use those subspells when you cast those spells (see \pcref{Subspells}).
            At 6th level, and every two levels thereafter, you learn an additional subspell for a nature spell you know.

    % \subcf{15th -- Air Mantle}
    % You are surrounded by a mantle of air.
    % Thrown and projectile weapons have a 50\% chance to miss your while this effect is active.
    % Unusually large weapons, such as a giant's boulders, may suffer a decreased miss chance as appropriate to their size.
    % \subcf{15th -- Aqueous Step}
    % Wherever the druid moves, you leave a path of animated water that can grab creatures.
    % Whenever a creature crosses the path, the druid makes a Reflex attack to trip the creature, causing it to fall prone and waste the rest of its movement.
    % Your accuracy is equal to your druid level \add your Constitution.
    % \subcf{15th -- Flaming Step}
    % Wherever the druid moves, you leave a path of burning flame behind your that lasts for 1 round.
    % Whenever a creature crosses the path, the druid makes a Reflex attack to deal damage to the creature.
    % The attack deals 1d8 points of fire damage per two druid levels.
    % Your accuracy is equal to your druid level \add your Constitution.
    % A failed attack deals half damage.
    % \subcf{15th -- Lifegiving Step}
    % Wherever the druid moves, you leave a path of small, living plants that entangle foes for 1 round.
    % Whenever a creature crosses the path, the druid makes a Reflex attack to entangle the creature, causing it to waste the rest of its movement.
    % Your accuracy is equal to your druid level \add your Constitution.
    % The plants appear on any surface, and will continue to grow if they can survive, though they may die quickly if they appear on inhospitable terrain.
    % \subcf{17th -- Flaming Soul}
    % You gain the fire subtype, making your immune to fire but giving your a 50\% vulnerability to cold damage.
    % In addition, whenever you deal fire damage to a creature, the creature is \ignited for 5 rounds.
    % \subcf{17th -- Sunblessed Rejuvenation}
    % You gain fast healing equal to your druid level as long as you remain in sunlight or touches a plant of your size or larger.
    % \subcf{17th -- Sunscour}
    % This aspect functions like the heart of the sun natural aspect, except that it also suppresses shadow effects and the visual components of illusions within the area of bright light.

    % \subcf{17th -- Water's Flow}
    % As a swift action, the druid can transform herself into a rushing flow of water with a volume roughly equal to your normal volume until the end of your turn.
    % In this form, she may move wherever water could go, but she cannot take other actions, such as jumping, attacking, or casting spells.
    % Your speed is halved when moving uphill and doubled when moving downhill.
    % She may move through squares occupied by creatures or threatened by blocking enemies without penalty.
    % She may return to your normal form as a free action.
    % \par If the water is split, she may reform from anywhere the water has reached, to as little as a single ounce of water.
    % If not even an ounce of water exists contiguously, your body reforms from the largest available parts of water, cut into pieces of appropriate size.
    % This usually causes the druid to die.

    \subsection{Ex-Druids}
        A druid who ceases to revere nature, changes to a prohibited alignment, or teaches the Druidic language to a nondruid loses all spells and magical druid class abilities.
        She cannot thereafter gain levels as a druid until you atone for your transgressions.

    % \subsection{Variant Druids}

    %     \subsubsection{Blighter}

    %         Blighters draw power from nature, as do other druids. However, while other druids revere nature and draw power from it gently, blighters steal power from nature forcefully. Wherever a blighter goes, destruction and death surely follows.

    %         \altcf{Blight} Instead of meditating to regain spell slots, a blighter draws power from your environment forcefully.
    %         This affects a \areahuge radius zone centered on your, and the process takes 1 minute of concentration.
    %         At the end of every round, every living thing in the area other than the blighter takes damage equal to your nature power.
    %         All inanimate plants of Huge size or smaller immediately wither and die.
    %         The earth becomes cracked and infertile, and any nutrients from the soil are destroyed.
    %         This ability has no effect on artificial environments or materials, such as metal or worked stone.
    %         At the end of the minute, the blighter regains your spent nature spell slots.

    %         A blighter can only blight your surroundings in this way once per hour.
    %         If your surroundings are already blighted or are not natural terrain, she cannot use this ability to regain your spells.
    %         Instead, she must meditate for 8 hours to slowly draw power from your surroundings, as a normal druid.

    %         \altcf{Spells} As normal, except that a blighter adds all Vivimancy arcane spells to your spell list.

    %         \altcf[2]{Wild Speech} As normal, except that a blighter gains a \plus5 bonus to Intimidate against your wild speech targets, and a \minus5 penalty to Persuasion.

    %         \altcf[10]{Blightcasting}

    %         \altcf[20]{Improved Blightcasting}

        % \subsubsection{Rotbringer}

        %     While most druids seek to emulate and interact with animals, rotbringers focus on the power of fungi, decay, and regeneration.

        %     \altcf{Invoke Rot} Instead of meditating to regain spell slots, a rotbringer accelerates the natural forces of decomposition and decay on your environment.
        %     This affects a \areahuge radius zone centered on your, and the process takes 1 minute of concentration.
        %     All organic objects of Huge size or smaller, such as plants and corpses, decompose.
        %     This decomposition kills inanimate, living plants.
        %     All organic objects, regardless of size, are covered with various fungi.
        %     This ability has no effect on artificial environments or materials, such as metal or worked stone.
        %     At the end of the minute, the rotbringer regains your spent nature spell slots.

        %     If the rotbringer decomposes a Huge object with this ability, or a combination of smaller objects equivalent in size to a Huge object, you gain an bonus nature spell slot of your highest available spell level.
        %     This extra spell slot lasts until it is used, or until you regain your spell slots again.

        %     A rotbringer can only invoke rot on your surroundings in this way once per hour.
        %     If your surroundings are already decomposed or are not natural terrain, she cannot use this ability to regain your spells.
        %     Instead, she must meditate for 8 hours to slowly draw power from your surroundings, as a normal druid.

        %     \altcf[2]{Wild Speech} The rotbringer gains the ability to speak with plants at 2nd level.
        %     You gain the ability to speak with animals at 6th level, instead of at 2nd level.

        %     \altcf[3rd]{Wild Aspect} The rotbringer does not gain this ability.

        %     \altcf[3rd]{Rot Spell} The druid learns an additional spell slot and spell known.
        %     The spell must be taken from the following list of spells.
        %     The spell's level cannot exceed half your druid level.
        %     If she already knows a spell from the list at every spell level you ha access to, she may instead learn any nature spell (see \pcref{Nature Spells}).

        %     At 5th level, and every odd level, the druid may learn a new spell.

        %     \begin{dtable}
        %         \begin{dtabularx}{\columnwidth}{l X}
        %             \tb{Spell level} & \tb{Rotbringer Spells} \\
        %             1st & \spell{excrete slime}, \spell{lesser regeneration} \\
        %             2nd & \spell{fungal growth} \\
        %             3rd & \spell{rotburst} \\
        %             4th & \spell{poison} \\
        %             6th & \spell{regeneration} \\
        %             7th & \spell{greater rotburst} \\
        %         \end{dtabularx}
        %     \end{dtable}

        %     \altcf[7]{Fungal Armor} The rotbringer becomes covered in fungus that protects your from attacks. You gain a \plus1 bonus to Armor and Fortitude defense.

        %     This bonus increases by 1 at your 7th druid level, and every 4 druid levels thereafter.

\section{Fighter}\label{Fighter}
    \begin{dtable}
        \lcaption{Fighter Progression}
        \begin{dtabularx}{\columnwidth}{>{\ccol}p{\levelcol} >{\lcol}X}
            \tb{Level} & \tb{Abilities} \\\bottomrule
            \nth{1}     & Consummate warrior, weapon adaptation
            \\ \nth{2}  & Armored agility
            \\ \nth{3}  & Discipline
            \\ \nth{4}  & Daunting strike
            \\ \nth{5}  & Weapon focus
            \\ \nth{6}  & Focused recovery
            \\ \nth{7}  & Twinstrike
            \\ \nth{8}  & Greater armored agility
            \\ \nth{9}  & Swift warrior
            \\ \nth{10} & Superior strike
            \\ \nth{11} & Weapon master
            \\ \nth{12} & Greater discipline
            \\ \nth{13} & Heartseeking strike
            \\ \nth{14} & Supreme armored agility
            \\ \nth{15} & Greater focused recovery
            \\ \nth{16} & Greater daunting strike
            \\ \nth{17} & Greater weapon focus
            \\ \nth{18} & Supreme discipline
            \\ \nth{19} & Warrior of legend
            \\ \nth{20} & Armored juggernaut, legendary discipline
        \end{dtabularx}
    \end{dtable}

    \classbasics{Alignment} Any.

    \subsection{Class Abilities}
        If you are a fighter, you gain the following abilities.

        \classbasics{Skill Points} 6.

        \classbasics{Class Skills}
        \subparhead{Strength} Climb, Jump, Sprint, Swim.
        \subparhead{Dexterity} Acrobatics, Escape Artist, Ride.
        \subparhead{Perception} Awareness.
        \subparhead{Other} Bluff, Intimidate, Persuasion.

        \classbasics{Defenses} \plus3 Fortitude, \plus2 Mental, \plus1 Reflex.

        \cf{Ftr}{Weapon and Armor Proficiency}
        A fighter is proficient with simple weapons, any four other weapon groups,  all armor (heavy, medium, and light), and shields.


        \subsubsection{Martial Supremacy}
            \cf{Ftr}{Consummate Warrior} You gain a \plus1 bonus to \glossterm{accuracy} with \glossterm{strikes}.

            \cf{Ftr}[4]{Daunting Strike} As a standard action, you can use this ability.
            \begin{ability}
                \begin{spelleffects}
                    \spelleffect You make a \glossterm{strike} against one creature.
                    If that strike deals damage, the target suffers a \minus2 penalty to \glossterm{accuracy} on strikes against you.
                    This effect does not stack with itself.
                    \spelldur Condition
                \end{spelleffects}
            \end{ability}

            \cf{Ftr}[7]{Superior Strike} As a standard action, you can use this ability.
            \begin{ability}
                \begin{spelleffects}
                    \spelleffect You make a \glossterm{strike} against one creature.
                    You gain a \plus1 bonus to \glossterm{accuracy} with the strike.
                \end{spelleffects}
            \end{ability}

            \cf{Ftr}[10]{Twinstrike} As a standard action, you can use this ability.
            \begin{ability}
                \begin{spelleffects}
                    \spelleffect You make a \glossterm{strike} against up to two different creatures.
                    Both strikes must be made with the same weapon or weapons.
                \end{spelleffects}
            \end{ability}

            \cf{Ftr}[13]{Heartseeking Strike} As a standard action, you can use this ability.
            \begin{ability}
                \begin{spelleffects}
                    \spelleffect You make a \glossterm{strike} against one creature.
                    Your attack roll \glossterm{explodes} on an 8, 9, or 10 (see \pcref{Exploding Attacks}).
                    This does not affect additional dice rolled if the attack roll explodes.
                \end{spelleffects}
            \end{ability}

            \cf{Ftr}[16]{Greater Daunting Strike}
            When you deal damage to a creature with your \textit{daunting strike} ability, the target also suffers a \minus2 penalty to defenses against \glossterm{strikes} you make.

            \cf{Ftr}[19]{Warrior of Legend}
            The accuracy bonus from your \textit{consummate warrior} ability increases to \plus2.

        \subsubsection{Equipment Training}
            \cf{Ftr}{Weapon Adaptation}
            If you spend an hour training with a weapon, you become proficient with that weapon's weapon group.
            You can only be proficient with one additional weapon group in this way at a time.

            \cf{Ftr}[2]{Armored Agility}
            You treat body armor you wear as less encumbering.
            You reduce its \glossterm{encumbrance penalty} by 2, and its arcane spell failure by 10\%.

            \cf{Ftr}[5]{Weapon Focus} 
            When you use your \textit{weapon adaptation} ability to gain proficiency with a weapon group, you gain a \plus1 bonus to \glossterm{accuracy} with attacks using weapons from that group.
            You can train with a weapon from a weapon group you are already proficient with to apply this ability to weapons from that group.

            \cf{Ftr}[8]{Greater Armored Agility}
            The reduction of body armor's encumbrance penalty from your \textit{armored agility} ability increases to 4, and your reduction of its arcane spell failure increases to 20\%.
            In addition, you treat it were one encumbrance category lighter than normal whenever doing so would be beneficial for you.

            \cf{Ftr}[11]{Weapon Master} 
            You gain a \plus1 bonus to \glossterm{accuracy} with \glossterm{strikes}.

            \cf{Ftr}[14]{Supreme Armored Expertise}
            The reduction of body armor's encumbrance penalty from your \textit{armored agility} ability increases to 6, and your reduction of its arcane spell failure increases to 30\%.
            In addition, you treat it as if it were an additional encumbrance category lighter than normal whenever doing so would be beneficical for you.

            \cf{Ftr}[17]{Greater Weapon Focus} 
            Your bonus to accuracy from your \textit{weapon focus} ability increases to \plus2.

            \cf{Ftr}[20]{Armored Juggernaut}
            You gain a \plus1 bonus to all defenses while you are wearing body armor.

        \subsubsection{Combat Discipline}

            \cf{Ftr}[3]{Discipline} As a \glossterm{swift action}, you can spend an \glossterm{action point} to use this ability.
            \begin{ability}
                \begin{spelleffects}
                    \spelleffect Remove one \glossterm{condition} affecting you.
                \end{spelleffects}
            \end{ability}

            \cf{Ftr}[6]{Focused Recovery}
            When you use the \textit{recover} action, you heal \plus1d hit points.
            In addition, instead of removing a condition, you can remove any \glossterm{sustained} effect on you.

            \cf{Ftr}[9]{Swift Warrior}
            If you use only \glossterm{mundane} abilities in a given round, you can take an additional \glossterm{swift action} or \glossterm{immediate action} that round.
            After using this ability in a round, you are unable to use any \glossterm{magical} abilities until the next round.

            \cf{Ftr}[12]{Greater Discipline}
            When you use the \textit{discipline} ability, you can remove any number of conditions.

            \cf{Ftr}[15]{Greater Focused Recovery}
            The bonus to hit point recovery from the \textit{focused recovery} ability increases to \plus2d.
            In addition, when you use the \textit{recover} ability, you can remove any number of \glossterm{sustained} effects on you instead of removing a condition.

            \cf{Ftr}[18]{Endless Discipline}
            You do not need to spend an action point to use the \textit{discipline} ability.

            \cf{Ftr}[20]{Legendary Discipline} 
            You are immune to all \glossterm{conditions}.

\section{Mage}\label{Mage}
    \begin{dtable}
        \lcaption{Mage Progression}
        \begin{dtabularx}{\columnwidth}{>{\ccol}p{2em} c c >{\lcol}X}
            \tb{Level} & \tb{Spells} & \tb{Subspells} & \tb{Abilities} \\\bottomrule
            \nth{1}     & 2 & \tdash   & Arcane essence, rituals, spells
            \\ \nth{2}  & 3 & \tdash   & Spell knowledge, spell point
            \\ \nth{3}  & 4 & \tdash   & Arcane insight, spell knowledge
            \\ \nth{4}  & 4 & 2        & \tdash
            \\ \nth{5}  & 4 & 2        & Lesser essence lore
            \\ \nth{6}  & 4 & 3        & Augment
            \\ \nth{7}  & 4 & 3        & Arcane insight
            \\ \nth{8}  & 4 & 4        & \tdash
            \\ \nth{9}  & 5 & 4        & Spell knowledge
            \\ \nth{10} & 5 & 5        & Augment
            \\ \nth{11} & 5 & 5        & Arcane insight
            \\ \nth{12} & 5 & 6        & \tdash
            \\ \nth{13} & 5 & 6        & Essence lore
            \\ \nth{14} & 5 & 7        & Augment
            \\ \nth{15} & 5 & 7        & Arcane insight
            \\ \nth{16} & 5 & 8        & Spell point
            \\ \nth{17} & 5 & 8        & Greater essence lore
            \\ \nth{18} & 5 & 9        & Augment
            \\ \nth{19} & 5 & 9        & Arcane insight
            \\ \nth{20} & 5 & 10       & Archmage
        \end{dtabularx}
    \end{dtable}

    \classbasics{Alignment} Any.

    \subsection{Class Abilities}
        If you are a mage, you gain the following abilities.

        \classbasics{Skill Points} 4.

        \classbasics{Class Skills}
        \subparhead{Intelligence} Knowledge (all kinds, taken individually), Linguistics.
        \subparhead{Perception} Awareness, Spellcraft.
        \subparhead{Other} Bluff, Intimidate, Persuasion.

        \classbasics{Defenses} \plus3 Mental, \plus2 Reflex, \plus1 Fortitude.

        \cf{Mge}{Weapon and Armor Proficiency}
        Mages are proficient with simple weapons and one other weapon group.
        They are not proficient with any type of armor or shield.
        Armor of any type interferes with a mage's arcane gestures, which can cause your spells with somatic components to fail.

        \cf{Mge}{Arcane Essence}
        All mages have access to great arcane power.
        However, not all mages acquired this power in the same way.
        You choose an arcane essence.
        Many mage abilities have special effects based on whether you are a sorcerer or a mage.
        \parhead{Sorcerer} Sorcerers have an intuitive connection to magic that allows them to cast spells without preparation or training.
        \parhead{Wizard} Wizards studied arcane mysteries for years to learn the secret ways of magic.
        A wizard casts spells with your Intelligence.

        \subsubsection{Spellcasting}

            \cf{Mge}{Arcane Spells} 
            You can cast arcane spells.
            You learn two arcane spells from the arcane \glossterm{spell list} (see \pcref{Arcane Spells}).
            \subparhead{Sorcerer} Your spellpower with arcane spells is equal to your character level or your Willpower, whichever is higher.
            The maximum spell level you can cast is equal to half your mage level (minimum 1).
            \subparhead{Wizard} Your spellpower with arcane spells is equal to your character level or your Intelligence, whichever is higher.
            The maximum spell level you can cast is equal to half your mage level (minimum 1) or your Intelligence, whichever is lower.

            To cast a spell, you must normally spend an \glossterm{action point}.
            Every spell can also be cast as a cantrip.
            Cantrips are weaker, but do not require action points to cast.

            \cf{Mge}{Rituals}
            \subparhead{Sorcerer} You cannot perform arcane rituals.
            \subparhead{Wizard} You can perform arcane rituals to create unique magical effects (see \pcref{Rituals}).
            You have a ritual book containing one arcane ritual of your choice (see \pcref{Arcane Rituals}).

            \cf{Mge}[6]{Augment}
            Choose an \glossterm{augments} (see \pcref{Augments}).
            You can apply that augment to arcane spells you cast and arcane rituals you perform.
            At 10th level, and every four levels thereafter, you learn an additional augment.

            \cf{Mge}[16]{Spell Point}
            You gain a spell point.
            A spell point can be spent to cast spells in place of an action point.
            You recover all spent spell points after a \glossterm{short rest}.

        \subsubsection{Arcane Lore}
            You must have the ability to cast arcane spells to gain these abilities.

            \cf{Mge}[2]{Spell Knowledge} 
            You learn an additional spell from the arcane \glossterm{spell list} (see \pcref{Arcane Spells}).

            \cf{Mge}[3]{Arcane Insight} 
            You gain a greater understanding of magic.
            You choose one of the following insights.
            Each insight can be chosen multiple times.
            At 7th level, and every four levels thereafter, you gain an additional arcane insight.
            \begin{itemize}
                \item Innate Spell: Choose a spell you know.
                    You no longer need verbal or somatic components to cast that spell.
                    \par If you choose this insight multiple times, she must choose a different spell each time.
                \item Personal Spell: Choose a spell you know.
                    You cannot miscast that spell (see \pcref{Miscasting}).
                    If you would miscast it, the spell simply fails without effect.
                    In addition, you automatically succeeds at all Concentration checks you make to cast the spell.
                    \par If you choose this insight multiple times, you must choose a different spell each time.
                \item Specialization: Choose a school of magic.
                    You learn an additional spell from that school of magic.
                    In exchange, you must ban two other schools of magic.
                    You can never learn or cast spells or rituals from your banned schools.
                    If you know spells from a banned school, you must immediately learn different spells from unbanned schools in their place.
                    \par If you choose this insight multiple times, you must choose to specialize in the same school each time.
            \end{itemize}

            \subparhead{Sorcerer} You may also choose the Expanded Spell Knowledge insight.
            You choose a single spell from the divine spell list or nature spell list and add it to your arcane spell list.
            This does not grant you the spell as a spell known, but you may exchange one of your spells known to learn that new spell.
            \par If you choose this insight multiple times, you must choose a different spell each time.

            \subparhead{Wizard} You may also choose the Ritual Spell insight.
            You scribe an arcane spell you know into your ritual book.
            The spell is treated as a ritual, and you can perform a one minute ritual to cause the spell's effect.
            Other creatures may not participate in this ritual.
            You can apply \glossterm{augments} or \glossterm{subspells} to the spell normally, increasing the ritual's level appropriately.

            \cf{Mge}[5]{Lesser Essence Lore}[Magical]
            You gain an ability based on your choice of arcane essence.
            \subparhead{Sorcerer} You gain bonus hit points equal to your spellpower with arcane spells.
            \subparhead{Wizard} You gains a \plus2 bonus to all Knowledge skills.

            \cf{Mge}[9]{Spell Knowledge}
            You learn an additional spell from the arcane \glossterm{spell list} (see \pcref{Arcane Spells}).

            \cf{Mge}[13]{Essence Lore}[Magical]
            You gain an ability based on your choice of arcane essence.
            \subparhead{Sorcerer} You gain \glossterm{magic resistance} equal to 5 \add your spellpower with arcane spells.
            \subparhead{Wizard} You learn the Contingency augment, allowing you to prepare a spell so it takes effect automatically if specific circumstances arise.
            The Contingency augment adds two levels to a spell's level.
            You can apply this augment to any arcane spell with a casting time of a single standard action.

            Casting a spell with the Contingency augment takes 5 minutes.
            When the casting is complete, the spell has no immediate effect.
            Instead, it automatically takes effect when some specific circumstances arise.
            During this casting time, you specify what circumstances cause the spell to take effect.

            The spell can be set to trigger in response to any circumstances that a typical human observing you and your situation could detect.
            For example, you could specify ``when I fall at least 50 feet'' or ``when I become bloodied'', but not ``when there is an invisible creature within 50 feet of me'' or ``when I am at 17 hit points or fewer.''
            The more specific the required circumstances, the better -- vague requirements, such as ``when I am in danger'', may cause the spell to trigger unexpectedly or fail to trigger at all.
            If you attempt to specify multiple separate triggering conditions, such as ``when I take damage or when an enemy is adjacent to me'', the spell will randomly ignore all but one of the conditions.

            If the spell needs to be targeted, the trigger condition can specify a simple rule for identifying how to target the spell, such as ``the closest enemy''.
            If the rule is poorly worded or imprecise, the spell may target incorrectly or fail to activate at all.
            Any spells which require decisions, such as the \spell{dimension door} spell, must have those decisions made at the time it is cast.
            You cannot alter those decisions when the contingency takes effect.

            You can have only one spell with this augment active at a time.
            If you use the augment again with a different spell, the new spell.

            \cf{Mge}[17]{Greater Essence Lore}[Magical]
            You gain an ability based on your choice of arcane essence.

            \subparhead{Sorcerer} Whenever you resist a spell with your \glossterm{magic resistance}, you gain the ability to cast that spell once.
            The spell retains all augments, effects from feats and other abilities, and similar modifications from the original caster, and you cannot choose any other augments or apply effects from your own abilities.
            However, you make all other decisions required to cast the spell, and uses your spellpower to determine the spell's effects.
            Once you cast the spell, you expend the absorbed energy, and you cannot cast it again.

            If you resist multiple spells simultaneously, or if you resist another spell with your magic resistance before casting the previous spell you resisted, you choose which spell you gain the ability to cast.

            \subparhead{Wizard} You may have two spells active with the Contingency augment, rather than only one.
            Whenever you cast a new spell with the Contingency augment, you choose which existing contingency to replace.

            Only one contingency can trigger in a given round.
            If both would trigger simultaneously, only the first spell cast triggers.
            The second spell cast does not trigger that round.

            \cf{Mge}[20]{Archmage}
            You no longer need to spend action points to cast spells.
            If you have any spell points, you lose those spell points and gain the same number of legend points instead.

        \subsubsection{Arcane Spell Mastery}
            You must have the ability to cast arcane spells to gain these abilities.

            \cf{Mge}[2]{Spell Point}
            You gain a spell point.
            A spell point can be spent to cast spells in place of an action point.
            You recover all spent spell points after a \glossterm{short rest}.

            \cf{Mge}[3]{Spell Knowledge} 
            You learn an additional spell from the arcane \glossterm{spell list} (see \pcref{Arcane Spells}).

            \cf{Mge}[4]{Subspells}
            Choose two \glossterm{subspells} for arcane spells you know.
            You can use those subspells when you cast those spells (see \pcref{Subspells}).
            At 6th level, and every two levels thereafter, you learn an additional subspell for an arcane spell you know.


\section{Monk}\label{Monk}
    \begin{dtable}
        \lcaption{Monk Progression}
        \begin{dtabularx}{\columnwidth}{>{\ccol}p{\levelcol} >{\lcol}X}
            \tb{Level} & \tb{Abilities} \\\bottomrule
            \nth{1}     & Manifest ki, Unarmed warrior, unfettered defense
            \\ \nth{2}  & Flurry of blows
            \\ \nth{3}  & Mental reserves
            \\ \nth{4}  & Manifest ki
            \\ \nth{5}  & Intuitive reaction
            \\ \nth{6}  & Transcend frailty
            \\ \nth{7}  & Manifest ki
            \\ \nth{8}  & Stunning fist
            \\ \nth{9}  & Serenity
            \\ \nth{10} & Manifest ki
            \\ \nth{11} & Greater intuitive reaction
            \\ \nth{12} & Transcend flesh
            \\ \nth{13} & Manifest ki
            \\ \nth{14} & Greater flurry of blows
            \\ \nth{15} & Inner peace
            \\ \nth{16} & Manifest ki
            \\ \nth{17} & Greater unfettered defense
            \\ \nth{18} & Transcend limits
            \\ \nth{19} & Manifest ki
            \\ \nth{20} & Transcend mortality
        \end{dtabularx}
    \end{dtable}

    \classbasics{Alignment} Any nonchaotic.

    \subsection{Class Abilities}
        If you are a monk, you gain the following abilities.

        \classbasics{Skill Points} 6.

        \classbasics{Class Skills}
        \subparhead{Strength} Climb, Jump, Sprint, Swim.
        \subparhead{Dexterity} Acrobatics, Escape Artist, Ride, Stealth.
        \subparhead{Intelligence} Heal.
        \subparhead{Perception} Awareness, Spellcraft, Survival.
        \subparhead{Other} Bluff, Intimidate, Perform, Persuasion.

        \classbasics{Defenses} \plus3 Reflex, \plus2 Mental, \plus1 Fortitude.

        \cf{Mnk}{Weapon and Armor Proficiency}
        Monks are proficient with simple weapons, monk weapons, and any one other weapon group.
        Monks are not proficient with any armor or shields.
        When wearing armor, using a shield, or carrying a medium or heavy load, a monk loses the benefit of your enlightened defense, fast movement, and \ki abilities.


        \subsubsection{Ki}
            \cf{Mnk}{Ki Power}
            The \glossterm{power} of your ki abilities is determined by your \textit{ki power}.
            Your \textit{ki power} is equal to your level or your Willpower, whichever is higher.

            % TODO: change this to choose two for the initial, have alternating non-choice abilities
            \cf{Mnk}{Manifest Ki} 
            You gain the ability to channel your ki to temporarily enhance your abilities.
            Choose a single ki manifestation from the list below.
            Many ki manifestations have a minimum level prerequisite, as indicated in the title of the ability.
            At the start of each round, you can spend an \glossterm{action point} to use a ki manifestation.
            This does not take an action, but you can only use one ki manifestation per round.

            At 4th level, and every three levels thereafter, you gain an additional ki manifestation.

            \subcf{Elegant Whirl of Fluid Motion}
            You gain a \plus10 bonus to Acrobatics checks until the end of the round.
            \subcf{Leap of the Heavens}
            You gain a \plus10 bonus to Jump checks until the end of the round.
            \subcf{Scale the Highest Tower}
            You gain a \plus10 bonus to Climb checks until the end of the round.

            \subcf{4th -- Dance of Falling Feathers}
            If you are in free-fall, your fall is dramatically slowed.
            You fall only 60 feet this round, and take no falling damage if you hit the ground.
            \subcf{4th -- Fists of Distant Force}
            You empower your unarmed attacks with \ki, allowing you to attack distant foes.
            Until the end of the round, you gain an additional ten feet of \glossterm{reach} with your unarmed attacks, extending your \textit{threatened area}.

            \subcf{7th -- Burst of Blinding Speed}
            You gain a \plus30 foot bonus to your land speed, up to a maximum of double your original speed, until the end of the round.
            In addition, you cannot be followed until the end of the round (see \pcref{Follow}).
            \subcf{7th -- See the Flow of Life}
            You gain the ability to see the \ki of living creatures until the end of the round.
            You can ``see'' any living creatures and their equipment within 50 feet perfectly, regardless of lighting conditions, invisibility, or any other means of concealment.
            This cannot detect living creatures through solid walls, however.
            \subcf{7th -- Surpass the Mortal Limits}
            Until the end of the round, you may use your \textit{ki power} in place of your Strength, Dexterity, and Constitution when making checks.

            % \subcf{11th -- Diamond Fist}
            % You gain a \plus2d bonus to damage with your unarmed attacks until the end of the round.
            % In addition, you treat your unarmed attack as if it were an adamantine weapon for the purpose of overcoming damage reduction and hardness.
            \subcf{11th -- Flash Step}
            Until the end of the round, whenever you move, you teleport directly to your destination instead.
            This does not change the total distance you can move, but you can teleport in any direction, even vertically.
            If your \glossterm{line of effect} to your destination is blocked, or if this would somehow place you inside a solid object, your movement is cancelled and you remain where you are.
            You can teleport in multiple steps within the same movement to get around obstacles you see.

            % \subcf{14th -- Awaken the Pacifist Heart}
            % As an immediate action, when you hit with a melee strike, she can make a \Ki power vs. Mental attack against the struck creature.
            % Success means the target is unable to take violent actions, such as attacking, for 2 rounds.
            % If the target takes damage after the current round, the effect is broken.

            \subcf{17th -- Flash Burst}
            This ability functions like this \textit{flash step} ability, except that your movement speed is also increased tenfold.

        \subsubsection{Unfettered Warrior}
            \cf{Mnk}{Unarmed Warrior}
            You are \glossterm{proficient} with your \glossterm{unarmed attack}.
            In addition, you gain a \plus2d bonus to damage with your unarmed attack.
            For details about how to fight while unarmed, see \pcref{Unarmed Combat}.

            \cf{Mnk}{Unfettered Defense}[Magical]
            When \monkunencumbered, you gain a \plus2 bonus to Armor defense.
            You lose this bonus when you are \helpless.

            \cf{Mnk}[2]{Flurry of Blows} As a standard action, you can use this ability.
            \begin{ability}
                \begin{spelleffects}
                    \spelleffect Make an unarmed \glossterm{strike}.
                    You may roll the attack roll twice and take the result you prefer.
                \end{spelleffects}
            \end{ability}

            \cf{Mnk}[5]{Intuitive Reaction} You can react to danger before your senses would normally allow you to do so.
            You reduce your \glossterm{overwhelm penalties} by 1.
            If your overwhelm penalty is reduced to 0, you are not considered to be overwhelmed.
            In addition, you are not \unaware when attacked by surprise.

            \cf{Mnk}[8]{Stunning Fist} As a standard action, you can spend an \glossterm{action point} to use this ability.
            \begin{ability}
                \begin{spelleffects}
                    % TODO this wording is ugly
                    \spelleffect Make an unarmed \glossterm{strike} against a creature.
                    If you deal damage to the creature, it is \dazed.
                    If you deal damage with a \glossterm{critical hit}, the target is \stunned instead.
                    \spelldur Condition
                \end{spelleffects}
            \end{ability}

            \cf{Mnk}[11]{Greater Intuitive Reaction}
            Your reduction of \glossterm{overwhelm penalties} from the \textit{intuitive reaction} ability increases to 2.

            \cf{Mnk}[14]{Greater Flurry of Blows}
            When you use your \textit{flurry of blows} ability, you may make the \glossterm{strike} with any melee weapon you wield, rather than only your unarmed attack.

            \cf{Mnk}[17]{Greater Unfettered Defense}
            Your bonus to Armor defense from your \textit{unfettered defense} ability improves to \plus3.

        \subsubsection{Transcendent Sage}
            \cf{Mnk}[3]{Mental Reserves} You gain an additional \glossterm{action point}.
            As long as you have at least one action point remaining, you gain a \plus2 bonus to Mental defense.

            \cf{Mnk}[6]{Transcend Frailty}
            You are immune to being \glossterm{deafened}, \glossterm{fatigued}, and \glossterm{sickened}.

            \cf{Mnk}[9]{Serenity}[Magical]
            You gain a \plus2 bonus to Mental defense.

            \cf{Mnk}[12]{Transcend Flesh}
            You are immune to being \glossterm{blinded}, \glossterm{exhausted}, and \glossterm{nauseated}.
            % TODO: do you still take aging penalties?
            In addition, you no longer take penalties to your attributes for aging, and cannot be magically aged.
            You still die of old age when your time is up.

            \cf{Mnk}[15]{Inner Peace}
            The bonus to Mental defense from the \textit{serenity} ability increases to \plus4.

            \cf{Mnk}[18]{Transcend Limits}
            Whenever you spend your last \glossterm{action point}, you regain a spent action point at the end of the next round.

            \cf{Mnk}[20]{Transcend Mortality}[Magical]
            If you die, you may choose to retain control of your body and soul through sheer force of will.
            Your body immediately disappears, and your soul does not travel to an afterlife.
            Instead, your body reforms with no trace of its injuries 24 hours later.
            The reformed body is in perfect health and can be any age you choose, to a minimum of the age of adulthood for your race.
            You can reform your body at the place where you died, or in any place on the same plane that is deeply familiar to you.

            After each time you reform herself this way, it takes 24 additional hours to reform the next time she ``dies''.
            A monk with this ability can only be permanently killed by the direct intervention of a deity.

        \subsubsection{Ex-Monks}
            % TODO: this wording is repetitive
            If you become chaotic, you lose all of your \glossterm{magical} monk abilities.
            If you stop being chaotic, you regain your magical monk abilities.

\section{Paladin}\label{Paladin}
    \begin{dtable}
        \lcaption{Paladin Progression}
        \begin{dtabularx}{\columnwidth}{>{\ccol}p{\levelcol} c >{\lcol}X}
            \tb{Level} & \tb{Spells} & \tb{Abilities} \\
            \bottomrule
            \nth{1}     & 2 & Smite, spells
            \\ \nth{2}  & 2 & Spell point
            \\ \nth{3}  & 2 & Enduring smite
            \\ \nth{4}  & 2 & Lay on hands
            \\ \nth{5}  & 2 & Aligned aura
            \\ \nth{6}  & 2 & Augment
            \\ \nth{7}  & 2 & Unfaltering zeal
            \\ \nth{8}  & 3 & Unbending devotion
            \\ \nth{9}  & 3 & Expanded aura
            \\ \nth{10} & 3 & Augment
            \\ \nth{11} & 3 & Pass judgment
            \\ \nth{12} & 3 & Greater lay on hands
            \\ \nth{13} & 3 & Greater aligned aura
            \\ \nth{14} & 3 & Augment
            \\ \nth{15} & 3 & Greater enduring smite
            \\ \nth{16} & 3 & Spell point
            \\ \nth{17} & 3 & Greater unbending devotion
            \\ \nth{18} & 3 & Augment
            \\ \nth{19} & 3 & Greater unfaltering zeal
            \\ \nth{20} & 3 & Aligned soul
        \end{dtabularx}
    \end{dtable}

    \classbasics{Alignment} Any other than true neutral.

    \subsection{Class Abilities}
        If you are a paladin, you gain the following abilities.

        \classbasics{Skill Points} 4.

        \classbasics{Class Skills}
        \subparhead{Dexterity} Ride.
        \subparhead{Intelligence} Heal, Knowledge (local, religion).
        \subparhead{Perception} Awareness, Intimidate, Sense Motive.
        \subparhead{Other} Bluff, Intimidate, Persuasion.

        \classbasics{Defenses} \plus3 Fortitude, \plus2 Mental, \plus1 Reflex.

        \cf{Pal}{Weapon and Armor Proficiency}
        Paladins are proficient with simple weapons, any three other weapon groups, all types of armor (heavy, medium, and light), and shields.

        \cf{Pal}{Devoted Alignment} 
        You are devoted to a specific alignment.
        You must choose one of your alignment components: good, evil, lawful, or chaotic.
        The alignment you choose is your devoted alignment.
        Your paladin abilities are affected by this choice.
        % seems unnecessary
        % You excel at slaying creatures with alignments opposed to your devoted alignment.
        Your alignment cannot be changed without extraordinary repurcussions.

        \cf{Pal}{Devotion Power}
        The \glossterm{power} of many paladin spells and abilities is determined by your \textit{devotion power}.
        Your \textit{devotion power} is equal to your level or your Willpower, whichever is higher.

        \subsubsection{Devoted Paragon}

            \cf{Pal}[2]{Spell Point}
            You gain a spell point.
            A spell point can be spent to cast spells in place of an action point.
            You recover all spent spell points after a \glossterm{short rest}.

            \cf{Pal}[4]{Lay on Hands}[Magical] As a standard action, you can spend an \glossterm{action point} to use this ability.
            \begin{ability}
                \begin{spelltargetinginfo}
                    \spellquicktargeting{One willing creature}{Adjacent}
                \end{spelltargetinginfo}
                \begin{spelleffects}
                    \spelleffect The target is healed for 1d8 damage \add 1d per two \textit{devotion power}.
                    In addition, you may remove one \glossterm{condition} from the target.
                \end{spelleffects}
            \end{ability}

            \cf{Pal}[5]{Aligned Aura}[Magical]
            Your devotion to your alignment affects the world around you, bringing it closer to your ideals.
            You constantly radiate an aura in an \areamed radius \glossterm{emanation} from you.
            The effect of the aura depends on your devoted alignment, as described below.
            You can suppress or resume the aura as a \glossterm{swift action}.

            \subparhead{Chaos} Whenever you or an ally in the area rolls a 1 on an attack roll when making a \glossterm{strike}, the attack roll explodes (see \pcref{Exploding Attacks}).
            This does not affect additional dice rolled if the attack roll explodes.
            \subparhead{Evil} All other creatures in the area suffer a \minus1 penalty to all defenses.
            \subparhead{Good} Whenever a creature in the area takes damage, you may take half that damage (rounded down) instead.
            Any abilities you have that would make the attack miss or fail have no effect, but your abilities that allow you to reduce or ignore its effects work normally.
            The protected creature takes the remaining half of the damage, and suffers any non-damaging effects of the attack normally.
            \subparhead{Law} Whenever you or an ally in the area rolls a 1 on an attack roll when making a \glossterm{strike}, the attack roll is treated as a 6.

            \cf{Pal}[8]{Unbending Devotion}[Magical]
            You are immune to \glossterm{Mind} \glossterm{conditions}.

            \cf{Pal}[9]{Expanded Aura}
            The area of your \textit{aligned aura} becomes a \arealarge radius \glossterm{emanation} from you.

            \cf{Pal}[12]{Greater Lay on Hands} 
            You gain a \plus1d bonus to the healing from your \textit{lay on hands} ability.
            In addition, you can remove any number of conditions with that ability, rather than only one.

            \cf{Pal}[13]{Greater Aligned Aura}[Magical]
            The effect of your \textit{aligned aura} becomes stronger based on your devoted alignment.

            \subparhead{Chaos} Whenever an enemy in the area rolls a 10 on an attack roll when making a \glossterm{strike}, it is forced to rereroll the attack roll and take the second result.
            \subparhead{Evil} The penalty imposed by the aura increases to \minus2.
            \subparhead{Good} When you redirect damage from an ally with this aura, you can redirect all effects of the attack to you instead of only half the damage.
            \subparhead{Law} Whenever an enemy in the area rolls a 10 on an attack roll when making a \glossterm{strike}, the attack roll is treated as a 6.

            \cf{Pal}[17]{Greater Unbending Devotion}
            You are immune to all hostile \glossterm{Mind} effects.

            \cf{Pal}[20]{Aligned Soul}[Magical]
            While you are dead, you may approach the deity or governing figure of your afterlife and request to be returned to life to continue your mission.
            Travelling to the relevant figure and making the request takes 12 hours.
            Unless there are extenuating circumstances, this request is almost always granted, and you are resurrected in a new body at a location of the entity's choice.
            This functions like the \ritual{resurrection} ritual, except that no part of the body is required, and a new body is created by the entity.
            You can be resurrected in this way regardless of the condition of your body, but not if your soul has been trapped or otherwise prevented from going to the correct afterlife.

        % TODO: clarify interaction between this ability and cleric spellcasting;
        % you shouldn't have both, but they technically use different spellpowers
        \subsubsection{Spellcasting}

            \cf{Pal}{Divine Spells}
            Your devotion to your alignment grants you the ability to cast divine spells.
            You learn two divine spells from the divine \glossterm{spell list} (see \pcref{Divine Spells}).
            Your \glossterm{spellpower} with divine spells is equal to your \glossterm{devotion power}.

            To cast a spell, you must normally spend an \glossterm{action point}.
            Every spell can also be cast as a cantrip.
            Cantrips are weaker, but do not require action points to cast.

            You can't cast spells of an alignment opposed to your own.
            Spells associated with particular alignments are indicated by the \glossterm{Chaos}, \glossterm{Good}, \glossterm{Evil}, and \glossterm{Law} tags in their spell descriptions.

            \cf{Pal}{Rituals}
            You can perform divine rituals to create unique magical effects (see \pcref{Rituals}).
            You have a ritual book containing one divine ritual of your choice (see \pcref{Divine Rituals}).

            \cf{Pal}[6]{Augment}
            Choose an \glossterm{augment} (see \pcref{Augments}).
            You can apply that augment to divine spells you cast and divine rituals you perform.
            At 10th level, and every four levels thereafter, you learn an additional augment.

            \cf{Pal}[8]{Spell Knowledge}
            You learn an additional divine spell (see \pcref{Divine Spells}).

            \cf{Pal}[16]{Spell Point} 
            You gain a spell point.
            A spell point can be spent to cast spells in place of an action point.
            You recover all spent spell points after a \glossterm{short rest}.

        \subsubsection{Zealous Warrior}
            \cf{Pal}{Smite}[Magical] As a standard action, you can spend an \glossterm{action point} to use this ability.
            \begin{ability}
                \begin{spelleffects}
                    \spelleffect You make a \glossterm{strike}.
                    If your target shares your devoted alignment, the strike deals no damage.
                    Otherwise, the strike gains a \plus1d bonus to \glossterm{strike damage}, and you regain the action point spent to use this ability.
                \end{spelleffects}
            \end{ability}

            \cf{Pal}[3]{Zealous Offense}[Magical]
            If you deal damage to a creature with your \textit{smite} ability, you gain a \plus1 bonus to \glossterm{accuracy} with \glossterm{strikes} against that creature.
            This effect lasts until you take a \glossterm{short rest}.

            \cf{Pal}[7]{Unfaltering Zeal}
            You gain a \plus1 bonus to Fortitude, Reflex, and Mental defense.

            \cf{Pal}[11]{Pass Judgment}[Magical] As a \glossterm{swift action}, you can spend an \glossterm{action point} to use this ability.
            \begin{ability}
                \begin{spelltargetinginfo}
                    \spellquicktargeting{One creature}{\rngmed}
                \end{spelltargetinginfo}
                \begin{spelleffects}
                    \spelleffect For the purpose of all spells and effects, the target is treated as if it had the alignment opposed to your devoted alignment.
                    This only affects its alignment along the alignment axis your devoted alignment is on.
                    For example, if your devoted alignment was evil, a chaotic neutral target would be treated as chaotic good.
                    In addition, the target is treated as if you had smited it for the purpose of the \textit{zealous offense} ability and similar effects.

                    You can use this ability to do battle against foes who share your alignment, but you should exercise caution in doing so.
                    Persecution of allies can lead you to fall and become an ex-paladin.
                    \spelldur{Attunement}
                \end{spelleffects}
            \end{ability}

            \cf{Pal}[15]{Greater Zealous Offense}[Magical]
            The accuracy bonus from your \textit{zealous offense} ability increases to \plus2.

            \cf{Pal}[19]{Greater Unfaltering Zeal}
            The bonus to defenses from your \textit{unfaltering zeal} ability increases to \plus2.

            % TODO: Doesn't work with new spell system
            % \cf{Pal}[20]{Martyr's Retribution}[Mag]
            % If you die in the service of your devoted alignment, you may choose to have your fallen body erupt in an immense burst of divine energy.
            % If you do, your body is almost completely consumed, preventing you from being raised with \ritual{resurrection} and similar effects that require an intact body.
            % This burst has two effects.
            % First, a \spell{sunburst} spell immediately takes effect over the area where you died.
            % Second, a \spell{storm of vengeance} spell begins to take effect, centered on the same area.
            % The spell lasts for 10 rounds, and the lightning strikes target the paladin's enemies.
            % Both of these effects harm only the paladin's foes, and do not harm your allies.
            % However, your allies' vision is still impeded by the \spell{storm of vengeance}.

        \subsubsection{Ex-Paladins}
            If you cease to follow your devoted alignment, you lose all \glossterm{magical} paladin class abilities.
            If your atone for your misdeeds and resume the service of your devoted alignment, you can regain your abilities.

\section{Ranger}\label{Ranger}
    \begin{dtable}
        \lcaption{Ranger Progression}
        \begin{dtabularx}{\columnwidth}{>{\ccol}p{\levelcol} >{\lcol}X}
            \tb{Level} & \tb{Abilities} \\\bottomrule
            \nth{1}     & Keen vision, quarry
            \\ \nth{2}  & Learned perception, tracker
            \\ \nth{3}  & Wilderness lore
            \\ \nth{4}  & Hunting style
            \\ \nth{5}  & Blindsense
            \\ \nth{6}  & Survival of the fittest
            \\ \nth{7}  & Learned pursuit
            \\ \nth{8}  & Farsight
            \\ \nth{9}  & Wilderness lore
            \\ \nth{10} & Hunting style
            \\ \nth{11} & Blindsight
            \\ \nth{12} & 
            \\ \nth{13} & Lethal quarry
            \\ \nth{14} & Greater farsight
            \\ \nth{15} & 
            \\ \nth{16} & Hunting style
            \\ \nth{17} & Truesight
            \\ \nth{18} & 
            \\ \nth{19} & 
            \\ \nth{20} & 
        \end{dtabularx}
    \end{dtable}

    \classbasics{Alignment} Any.

    \subsection{Class Abilities}
        If you are a ranger, you gain the following abilities.

        \classbasics{Skill Points} 8.

        \classbasics{Class Skills}
        \subparhead{Strength} Climb, Jump, Sprint, Swim.
        \subparhead{Dexterity} Acrobatics, Escape Artist, Ride, Stealth.
        \subparhead{Intelligence} Heal, Knowledge (dungeoneering, geography, nature).
        \subparhead{Perception} Awareness, Creature Handling, Survival.
        \subparhead{Other} Bluff, Intimidate, Persuasion.

        \classbasics{Defenses} \plus3 Reflex, \plus2 Fortitude, \plus1 Mental.

        \cf{Rgr}{Weapon and Armor Proficiency}
        A ranger is proficient with simple weapons, any two weapon groups, light and medium armor, and shields.
        You are also proficient with your choice of bows, crossbows, or thrown weapons.

        \subsubsection{Keen Senses}
            \cf{Rgr}{Keen Vision}
            Your sight improves, allowing you to see more easily.
            You gain \glossterm{low-light vision}, allowing you to treat sources of light as if they had double their normal illumination range.
            If you already have low-light vision, he double its benefit, allowing you to treat sources of light as if they had four times their normal illumination range.

            In addition, you gain \glossterm{darkvision} out to 50 feet, allowing you to see in complete darkness.
            If you already have darkvision, you increase its range by 50 feet.

            \cf{Rgr}[2]{Learned Perception} You gain two skill points that must be spent on Perception-based ranger class skills.

            \cf{Rgr}[5]{Blindsense}
            Your perceptions are so finely honed that you can sense your enemies without seeing them.
            You gain the \glossterm{blindsense} ability out to 50 feet.
            This ability allows you to sense the presence and location of objects and foes within 50 feet without seeing them.
            If you already have the blindsense ability, you increase its range by 50 feet.

            \cf{Rgr}[8]{Farsight}
            You increase the range of your \glossterm{darkvision} by 150 feet, and your \glossterm{blindsense} by 50 feet.
            In addition, you reduce your \glossterm{range increment penalties} for attacking at long range by 2.

            \cf{Rgr}[11]{Blindsight}
            You gain the \glossterm{blindsight} ability, allowing you to ``see'' perfectly without your eyes in a 50 foot radius around you.
            With this ability, you can fight just as well with your eyes closed as with them open.

            \cf{Rgr}[14]{Greater Farsight}
            You increase the range of your \glossterm{darkvision} by 500 feet, your \glossterm{blindsense} by 200 feet, and your \glossterm{blindsight} by 50 feet.
            In addition, the penalty reduction for \glossterm{range increment penalties} from your \textit{farsight} ability increases to 5.

            \cf{Rgr}[17]{Truesight} 
            Your perceptions are accurate enough to defeat even powerful magic.
            You can see through normal and magical darkness, see the truth behind visual figments and glamers, and see the true form of creatures and objects affected by \glossterm{Shaping} abilities.
            This ability works at any range.

        \subsubsection{Wilderness Warrior}
            \cf{Rgr}[3]{Wilderness Lore} You gain two extra skill points which must be spent on the Creature Handling, Heal, Knowledge (geography), Knowledge (nature), Ride, or Survival skills.

            \cf{Rgr}[6]{Survival of the Fittest}
            You gain a \plus1 bonus to \glossterm{accuracy} with \glossterm{strikes}.

            \cf{Rgr}[9]{Wilderness Lore} You gain two extra skill points which must be spent on the Creature Handling, Heal, Knowledge (geography), Knowledge (nature), Ride, or Survival skills.

            \cf{Rgr}[12]{} 

        \subsubsection{Focused Hunter}

            \cf{Rgr}{Quarry}
            As a \glossterm{swift action}, you can use this ability.
            \begin{ability}
                \begin{spelltargetinginfo}
                    \spellquicktargeting{One creature}{\rnglong}
                \end{spelltargetinginfo}
                \begin{spelleffects}
                    \spelleffect You gain a \plus5 bonus to checks made to follow the target's tracks, and a \plus1 bonus to \glossterm{accuracy} with \glossterm{strikes} against the target.
                    This ability lasts the target is \glossterm{defeated}, or until you use this ability again.
                \end{spelleffects}
            \end{ability}

            \cf{Rgr}[2]{Tracker}
            You gain a \plus2 bonus to checks made to follow tracks.
            In addition, you may use your level in place of the Survival skill to follow tracks (see \pcref{Survival}).

            \cf{Rgr}[4]{Hunting Style}
            You learn specific hunting styles to defeat particular quarries.
            Choose two hunting styles from the list below.
            Many hunting styles have a minimum level prerequisite, as indicated in the title of the ability.
            Whenever you use your \textit{quarry} ability, you can also gain the benefit of one hunting style you know.

            \subcf{Executioner}
            % TODO: clarify timing
            Whenever you deal damage to your quarry, any damage in excess of its remaining hit points is dealt as \glossterm{vital damage}.

            \subcf{Fearsome}
            Your quarry is \shaken by you as long as it remains your quarry.
            This is a \glossterm{Delusion}, \glossterm{Mind} effect.

            \subcf{Goading}
            Your quarry is \goaded by you as long as it remains your quarry.
            This is a \glossterm{Delusion}, \glossterm{Mind} effect.

            \subcf{Inescapable}
            Whenever you deal damage to your quarry, its movement speed is halved until the end of the next round.

            \subcf{Persistent}
            Your quarry remains your quarry even after you use the \textit{quarry} ability again.
            It remains your quarry until you use this ability again.

            \subcf{10th -- Anchored}[Magical]
            Whenever you deal damage to your quarry, it cannot travel extradimensionally until the end of the next round.
            This blocks teleportation and all planar travel abilities except planar rifts.

            \subcf{10th -- Leeching}
            Whenever you deal damage to your quarry, you heal hit points equal to your level.

            \subcf{10th -- Punishing}[Magical]
            At the end of each round, if your quarry is within \rnglong range of you, it takes life damage equal to your level.

            \subcf{10th -- Unerring}
            You ignore all miss chances and failure chances that would affect attacks and checks you make against your quarry.

            \subcf{10th -- Wolfpack}
            If your quarry is \glossterm{overwhelmed}, it increases its \glossterm{overwhelm penalties} by 1.

            \subcf{16th -- Greater Anchored}[Magical]
            Your quarry cannot travel extradimensionally.
            This blocks teleportation and all planar travel abilities except planar rifts.

            \subcf{16th -- Master of the Hunt}
            Your quarry takes a \minus2 penalty to defenses against attacks from creatures other than you.

            \subcf{16th -- Taunting}
            Your quarry is \taunted by you as long as it remains your quarry.
            This is a \glossterm{Delusion}, \glossterm{Mind} effect.

            \subcf{16th -- Terrifying}
            Your quarry is \frightened by you as long as it remains your quarry.
            This is a \glossterm{Delusion}, \glossterm{Mind} effect.

            \cf{Rgr}[7]{Learned Pursuit} You gain two extra skill points which must be spent on Strength or Dexterity-based ranger class skills.

            \cf{Rgr}[10]{Hunting Style}
            You learn an additional \textit{hunting style}.

            \cf{Rgr}[13]{Lethal Quarry}
            Your accuracy bonus from the \textit{quarry} ability increases to \plus2.

            \cf{Rgr}[16]{Hunting Style}
            You learn an additional \textit{hunting style}.

\section{Rogue}\label{Rogue}
    \begin{dtable}
        \lcaption{Rogue Progression}
        \begin{dtabularx}{\columnwidth}{>{\ccol}p{\levelcol} >{\lcol}X}
            \tb{Level} & \tb{Abilities}
            \\\bottomrule
               \nth{1}  & Skill lore, sneak attack
            \\ \nth{2}  & Stealth lore
            \\ \nth{3}  & Combat trick
            \\ \nth{4}  & Skill exemplar
            \\ \nth{5}  & Uncanny dodge
            \\ \nth{6}  & Ambush attack
            \\ \nth{7}  & Skill lore
            \\ \nth{8}  & Assassinate
            \\ \nth{9}  & Lucky slip
            \\ \nth{10} & Greater skill exemplar
            \\ \nth{11} & Greater uncanny dodge
            \\ \nth{12} & Combat trick
            \\ \nth{13} & Lucky break
            \\ \nth{14} & Greater sneak attack
            \\ \nth{15} & Twist of fate
            \\ \nth{16} & Supreme skill exemplar
            \\ \nth{17} & 
            \\ \nth{18} & Greater ambush attack
            \\ \nth{19} &
            \\ \nth{20} &
        \end{dtabularx}
    \end{dtable}

    \classbasics{Alignment} Any.

    \subsection{Class Abilities}
        If you are a rogue, you gain the following abilities.

        \classbasics{Skill Points} 8.

        \classbasics{Class Skills}
        \subparhead{Strength} Climb, Jump, Sprint, Swim.
        \subparhead{Dexterity} Acrobatics, Escape Artist, Sleight of Hand, Stealth.
        \subparhead{Intelligence} Devices, Disguise, Knowledge (dungeoneering, local), Linguistics.
        \subparhead{Perception} Awareness, Sense Motive.
        \subparhead{Other} Bluff, Intimidate, Perform, Persuasion.

        \classbasics{Defenses} \plus3 Reflex, \plus2 Mental, \plus1 Fortitude.

        \cf{Rog}{Weapon and Armor Proficiency}
        Rogues are proficient with simple weapons, any two weapon groups, light armor, and bucklers.
        They are also proficient with saps.

        \subsubsection{Assassin}
            \cf{Rog}{Sneak Attack} You gain a \plus1d bonus to \glossterm{strike damage} against creatures who are unable to defend themselves effectively.
            This applies against creatures who are \unaware, \defenseless, or \glossterm{overwhelmed}.

            You must be within \rngclose range of a creature to gain this damage bonus.
            In addition, you do not gain this damage bonus against creatures who are immune to \glossterm{critical hits} or who lack a discernible body structure, such as oozes.

            \cf{Rog}[2]{Stealth Lore} You gain two extra skill points which must be spent on the Acrobatics, Awareness, Disguise, Sleight of Hand, or Stealth skills.

            \cf{Rog}[5]{Uncanny Dodge} You can react to danger before your senses would normally allow you to do so.
            You reduce your \glossterm{overwhelm penalties} by 1.
            If your overwhelm penalty is reduced to 0, you are not considered to be overwhelmed.
            In addition, you are not \unaware when attacked by surprise.

            \cf{Rog}[8]{Assassinate} As a standard action, you can use this ability.
            \begin{ability}
                \begin{spelltargetinginfo}
                    \spellquicktargeting{One creature}{\rngmed}
                \end{spelltargetinginfo}
                \begin{spelleffects}
                    \spelleffect You study the target, finding weak points you can take advantage of.
                    % TODO is ``make a sneak attack'' clear enough wording?
                    Until the end of the next round, if you make a melee \textit{sneak attack} against the target while it is \unaware, your attack deals maximum damage.
                \end{spelleffects}
            \end{ability}

            \cf{Rog}[11]{Greater Uncanny Dodge}
            Your reduction of \glossterm{overwhelm penalties} from the \textit{uncanny dodge} ability increases to 2.

            \cf{Rog}[14]{Greater Sneak Attack}
            The damage bonus from your \textit{sneak attack} ability increases to \plus2d.

            \cf{Rog}[17]{} 

        \subsubsection{Jack of All Trades}

            \cf{Rog}{Skill Lore} You gain three extra skill points which must be spent on rogue class skills.

            \cf{Rog}[4]{Skill Exemplar} You gain a \plus1 bonus to all skills.

            \cf{Rog}[7]{Skill Lore} You gain three extra skill points which must be spent on rogue class skills.

            \cf{Rog}[10]{Greater Skill Exemplar} The skill bonus from your \textit{skill exemplar} ability increases to \plus2.

            \cf{Rog}[13]{Lucky Break} Once per round, when you make a \glossterm{check}, you can spend an \glossterm{action point} to use this ability.
            If you do, you treat your roll as a 10.

            \cf{Rog}[16]{Supreme Skill Exemplar} The skill bonus from your \textit{skill exemplar} ability increases to \plus3.

        \subsubsection{Scoundrel}

            \cf{Rog}[3]{Combat Trick}
            You learn how to confuse and confound your foes in combat.
            Choose a two combat tricks from the list below.
            Many combat tricks have a minimum level prerequisite, as indicated in the title of the ability.

            \subparhead{Bewildering Blow} As a standard action, you can spend an \glossterm{action point} to use this ability.
            \begin{ability}
                \begin{spelleffects}
                    % TODO this wording is ugly
                    \spelleffect Make a \glossterm{strike} against a creature.
                    If you deal damage to the creature, it is \disoriented.
                    If you deal damage with a \glossterm{critical hit}, the creature is \stunned instead.
                    \spelldur Condition
                \end{spelleffects}
            \end{ability}

            \subparhead{Distant Precision} If you have the \textit{sneak attack} ability, the maximum range at which you can use that ability increases to \rnglong.

            \subparhead{Distracting Blow} As a standard action, you can use this ability.
            \begin{ability}
                \begin{spelleffects}
                    % TODO this wording is ugly
                    \spelleffect Make a \glossterm{strike} against a creature.
                    If you deal damage to the creature, it automatically fails any Concentration checks it makes until the end of the round (see \pcref{Concentration}).
                \end{spelleffects}
            \end{ability}

            \subparhead{Hamstring} As a standard action, you can use this ability.
            \begin{ability}
                \begin{spelleffects}
                    % TODO this wording is ugly
                    \spelleffect Make a \glossterm{strike} against a creature.
                    If you deal damage to the creature, it moves at half speed.
                    If you deal damage with a \glossterm{critical hit}, the creature is \immobilized instead.
                    \spelldur Condition
                \end{spelleffects}
            \end{ability}

            \subparhead{Merciful} You take no penalties when using weapons to deal \glossterm{nonlethal damage}.

            %TODO: make poison work
            %\subcf{Swift Poisoner}
            %The rogue can apply poison to a weapon you are holding as a swift action.

            \subparhead{Tricky Maneuvers} You gain a \plus1 bonus to \glossterm{accuracy} with \glossterm{combat maneuvers}.

            \cf{Rog}[6]{Ambush Attack}
            You gain a \plus1 bonus to \glossterm{accuracy} with \glossterm{strikes} against creatures who did not \glossterm{threaten} you at the start of the round.

            \cf{Rog}[9]{Lucky Slip} Once per round, when you are hit by a \glossterm{strike}, you can spend an \glossterm{action point} to use this ability.
            If you do, the attacking creature rerolls the attack roll.

            \cf{Rog}[12]{Combat Trick}
            You learn an additional \textit{combat trick}.

            \cf{Rog}[15]{Twist of Fate} You can use your \textit{lucky slip} ability against any successful attack, not just a \glossterm{strike}.

            \cf{Rog}[18]{Greater Ambush Attack}
            The accuracy bonus from your \textit{ambush attack} ability increases to \plus2.


\chapter{Skills}\label{Skills}

Skills represent the myriad of talents that people can have, such as cooking or swimming.
This chapter describes each skill, including common uses for those skills.

\section{Skill Overview}
  This section describes how you learn and use skills.
  Most skills are used to make \glossterm{checks}.
  For details about how checks are made, see \pcref{Checks}.

  \subsection{Skill Modifiers}
    You are either trained or untrained with each skill.
    If you are untrained in a skill, your bonus with that skill is equal to its associated attribute (if any).
    If you are trained in a skill, your bonus with that skill is equal to 3 \add half your level \add its associated attribute (if any).
    Many abilities can increase or decrease your bonus with particular skills.

  \subsection{Class Skills}\label{Class Skills}
    Each \glossterm{class} has an associated set of skills that members of that class often know.
    These are called class skills.
    Your \glossterm{base class} automatically grants you training with a specific number of skills from among your class skills.
    The class skills from each class are summarized in \trefnp{Class Skills}.

    \begin{dtable!*}
      \lcaption{Class Skills}
      \begin{dtabularx}{\textwidth}{l *{11}{>{\ccol}X} >{\ccol}p{4em}}
        \tb{Skill}        & \tb{Bbn} & \tb{Clr} & \tb{Drd} & \tb{Ftr} & \tb{Mnk} & \tb{Pal} & \tb{Rgr} & \tb{Rog} & \tb{Sor} & \tb{War} & \tb{Wiz} & \tb{Key Ability} \tableheaderrule
        Climb             & C        & \tdash   & C        & C        & C        & \tdash   & C        & C        & \tdash   & \tdash   & \tdash   & Str          \\
        Jump              & C        & \tdash   & C        & C        & C        & \tdash   & C        & C        & \tdash   & \tdash   & \tdash   & Str          \\
        Swim              & C        & \tdash   & C        & C        & C        & \tdash   & C        & C        & \tdash   & \tdash   & \tdash   & Str          \\
        Balance           & C        & \tdash   & C        & C        & C        & \tdash   & C        & C        & \tdash   & \tdash   & \tdash   & Dex          \\
        Flexibility       & C        & \tdash   & \tdash   & C        & C        & \tdash   & \tdash   & C        & \tdash   & \tdash   & \tdash   & Dex          \\
        Perform           & \tdash   & \tdash   & \tdash   & \tdash   & C        & \tdash   & \tdash   & C        & \tdash   & \tdash   & \tdash   & Dex          \\
        Ride              & C        & \tdash   & \tdash   & C        & \tdash   & C        & \tdash   & \tdash   & \tdash   & C        & \tdash   & Dex          \\
        Sleight of Hand   & \tdash   & \tdash   & \tdash   & \tdash   & \tdash   & \tdash   & \tdash   & C        & \tdash   & \tdash   & \tdash   & Dex          \\
        Stealth           & \tdash   & \tdash   & \tdash   & \tdash   & C        & \tdash   & C        & C        & \tdash   & \tdash   & \tdash   & Dex          \\
        Endurance         & C        & \tdash   & C        & C        & C        & C        & C        & \tdash   & \tdash   & \tdash   & \tdash   & Con          \\
        Craft             & C        & C        & C        & C        & C        & C        & C        & C        & C        & C        & C        & Int          \\
        Deduction         & \tdash   & C        & C        & \tdash   & C        & C        & C        & C        & C        & C        & C        & Int          \\
        Devices           & \tdash   & \tdash   & \tdash   & \tdash   & \tdash   & \tdash   & \tdash   & C        & \tdash   & \tdash   & \tdash   & Int          \\
        Disguise          & \tdash   & \tdash   & \tdash   & \tdash   & \tdash   & \tdash   & \tdash   & C        & \tdash   & \tdash   & \tdash   & Int          \\
        Knowledge         & \tdash   & C        & \tdash   & \tdash   & C        & \tdash   & \tdash   & \tdash   & C        & C        & C        & Int          \\
        Medicine          & C        & C        & C        & C        & C        & C        & C        & \tdash   & \tdash   & \tdash   & \tdash   & Int          \\
        Awareness         & C        & C        & C        & C        & C        & C        & C        & C        & C        & C        & C        & Per          \\
        Creature Handling & C        & \tdash   & C        & \tdash   & C        & C        & C        & \tdash   & \tdash   & \tdash   & \tdash   & Per          \\
        Deception         & C        & C        & C        & C        & C        & C        & C        & C        & C        & C        & C        & Per          \\
        Persuasion        & C        & C        & C        & C        & C        & C        & C        & C        & C        & C        & C        & Per          \\
        Social Insight    & \tdash   & C        & \tdash   & \tdash   & C        & C        & \tdash   & C        & \tdash   & C        & \tdash   & Per          \\
        Survival          & C        & \tdash   & C        & \tdash   & C        & \tdash   & C        & \tdash   & \tdash   & \tdash   & \tdash   & Per          \\
        Intimidate        & C        & C        & C        & C        & C        & C        & C        & C        & C        & C        & C        & Varies\fn{1} \\
        Profession        & C        & C        & C        & C        & C        & C        & C        & C        & C        & C        & C        & Varies\fn{1} \\
      \end{dtabularx}
      C: class skill. \\
      1. Any attribute could apply depending on how the skill is used. \\
    \end{dtable!*}

  \subsection{Tasks}\label{Tasks}
    Each skill contains a brief description of how the skill is usually used.
    This description is followed by a series of specific ways in which the skills can be used.
    These tasks are simply examples, and do not list everything the skill can be used for.
    You should be creative with your skills, rather than only using the tasks explicitly listed.

  \subsection{Learning Languages}
    You can replace skill training from your class or Intelligence with learning languages.
    For each trained skill you forgo, you learn two \glossterm{common languages} or one \glossterm{rare language} (see \pcref{Communication and Languages}).

  \subsection{Improvising}
    Unlike maneuvers or spells, skills have a broad and ambiguous purview.
    They are not just a set of specific actions with precise rules.
    As a player, you should feel free to improvise actions that sound related to skills you have.
    However, you generally shouldn't use a skill to try to duplicate the effect of another ability, especially a combat ability.
    Skills are primarily intended to be useful out of combat, not to replace the power of manuevers and spells.

\newpage
\skill{Awareness}{Per}
  The Awareness skill represents your ability to observe things which you might otherwise fail to notice.
  It can be used to observe hidden creatures and traps, as well as to identify fleeting or subtle sensations.

  The Awareness skill governs the result regardless of the specific sense or senses used.
  It is most commonly used with sight and hearing, though other senses can be used, such as smell or touch.
  Whenever you make an Awareness check, you roll only once, and most creatures have the same Awareness modifier with all of their senses.
  However, the \glossterm{difficulty value} of the check, and the information granted by success, can be very different between senses.
  For example, it is impossible to see through walls or without light, but that does not make hearing impossible.

  \subsection{Common Awareness Tasks}

    \parhead{Identify Disguise} If you succeed at an opposed Awareness vs. Disguise check, you can determine whether a creature or object is disguised.
    \parhead{Identify Forgery} If you succeed at an opposed Awareness vs. Craft check, you can determine whether an object is a forgery.
    \parhead{Notice Hidden Creature} If you succeed at an opposed Awareness vs. Stealth check, you can notice a hidden creature (see \pcref{Stealth}).
    Success with a sight-based Awareness check means you can see the creature perfectly.
    Success with any other sense just means you know its exact location, and are still \partiallyunaware of it.
    \parhead{Notice Hidden Object} If you succeed at an opposed Awareness vs. Craft or Devices check, you can notice hidden objects such as traps and secret doors.
    \parhead{Notice Magic Trap} If you succeed at an Awareness check, you can notice hidden magical effects such as traps.
    The \glossterm{difficulty value} to is equal to 15 \add twice the \glossterm{rank} of the ability.
    \parhead{Notice Sleight of Hand} If you succeed at an opposed Awareness vs. Sleight of Hand check, you can notice a creature's attempt to use the Sleight of Hand skill.
    \parhead{Notice Subtle Effect} If you succeed at an Awareness check, you can notice the general effects of a \abilitytag{Subtle} ability affecting you.
    The \glossterm{difficulty value} to notice the effect when it is first applied to you is 15 \add twice the \glossterm{rank} of the ability.
    In addition, you can make another check when the ability ends to notice that you feel normal again.
    \parhead{Read Lips} When you see a creature speaking, you can make an sight-based Awareness check to read its lips.
    The \glossterm{difficulty value} is 10 for ordinary conversation, or up to 20 if the speaking creature makes an effort to avoid moving its lips.
    You must be able to understand the language spoken.
    Success means you can understand what is being said.

  \subsection{Common Awareness Modifiers}\label{Common Awareness Modifiers}
    While sleeping, you take a \minus10 penalty to the Awareness skill.
    You gain a \plus20 bonus to notice the presence of creatures and events that directly touch or damage you, such as a creature shoving you or making a \glossterm{strike} against you.

    There are three common circumstances that can make Awareness checks more difficult: obstructions, distance, and similar background sensations.
    Minor obstructions, short distances, and slightly similar backgrounds increase the \glossterm{difficulty value} by 2.
    Significant obstructions, long distances, and very similar backgrounds increase the DV by 5 or more.
    If a sensation is difficult to detect for multiple reasons, the difficulty modifiers stack.

    The \glossterm{difficulty value} of non-opposed checks changes depending on the size of the sensation.
    The difficulty value increases by 5 for each size category larger than Medium, and decreases by 5 for each size category smaller than Medium.
    Multiple sensations of the same type can also be treated as a single larger sensation, which makes them easier to detect.
    Non-visual sensations may not have a literal size category to rely on, so the GM can decide how this modifier applies.

\newpage
\skill{Balance}{Dex}
  % TODO: should "unsteady terrain" be a standard glossterm?
  The Balance skill represents your ability to maintain your balance and poise on unsteady terrain.
  It also represents your ability to precisely position your movements, such as to avoid touching traps that you are aware of.
  Generally, creatures only have to roll Balance checks if the terrain is unsteady for some reason.

  \subsection{Common Balance Tasks}
    \parhead{Agile Charge}\nonsectionlabel{Agile Charge} You can make a \glossterm{difficulty value} 10 Balance check while using the \ability{charge} ability to change directions while charging (see \pcref{Charge}).
    Success means you can make a single turn of up to 90 degrees during the movement.
    \parhead{Creature Balance}\nonsectionlabel{Creature Balance} As part of movement, you can make a Balance vs. Reflex attack against a creature you touch.
    The target must be two or more size categories larger than you.
    On a hit, you can balance on the target's body, allowing you to walk on or jump from its body.
    You must repeat this attack at the end of each subsequent round to stay balancing on the creature.
    For each consecutive round that you balance on a non-ally in this way, you take a cumulative \minus2 penalty to this attack.
    \parhead{Maintain Balance} If you take damage while on an unsteady surface, you must make a Balance \glossterm{reactive check} based on the surface.
    Failure means that you fall \prone.
    You only need to make this check once per round, even if you take damage from more than one source.
    \parhead{Rapid Stand}\nonsectionlabel{Rapid Stand} You make a \glossterm{difficulty value} 15 Balance check as a \glossterm{move action} to stand up from a prone position.
    Success means that you stand up so quickly that you can immediately make another movement.
    Failure simply means that you stand up.
    \parhead{Walk While Balancing} When you move using your \glossterm{walk speed} on an unsteady surface, you must make an Balance check based on the surface.
    If you choose to move at half speed, you gain a \plus5 bonus to the check.
    Success means you move along the surface.
    Failure means you stop moving and the rest of your movement is wasted.

  \subsection{Common Balance Modifiers}

    The base \glossterm{difficulty value} to balance on a normal terrain is 0.
    There are four common circumstances that make Balance checks more difficult: slippery, mobile, narrow, and uneven surfaces.
    Slightly impaired surfaces increase the \glossterm{difficulty value} by 2.
    Significantly impaired surfaces increase the DV by 5 or more.
    If a surface is impaired for multiple reasons, add all relevant modifiers.
    Some specific examples are listed below.

    \begin{columntable}
      \begin{dtabularx}{\columnwidth}{l X}
        \tb{Ice}                                  & \tb{DV} \tableheaderrule
        Rough, hardpacked ice, like a frozen lake & \plus2  \\
        Typical ice                               & \plus5  \\
        Recently frozen or ultra-smooth ice       & \plus10 \\
        \tb{Liquid}                               & \tb{DV} \tableheaderrule
        Water-covered ground, such as from rain   & \plus2  \\
        Ankle-deep moving stream                  & \plus5  \\
        Knee-deep static water                    & \plus5  \\
        Oil-coated ground                         & \plus5  \\
        Knee-deep moving stream                   & \plus10 \\
        \tb{Narrow Surface}                       & \tb{DV} \tableheaderrule
        About two feet wide                       & \plus2  \\
        About one foot wide                       & \plus5  \\
        About six inches wide                     & \plus10 \\
        About two inches wide                     & \plus15 \\
        Less than than two inches wide            & \plus20 \\
        \tb{Sand}                                 & \tb{DV} \tableheaderrule
        Water-logged beach sand                   & \plus2  \\
        Hard-packed desert sand                   & \plus2  \\
        Typical beach or desert sand              & \plus5  \\
        Quicksand                                 & \plus10 \\
        Unusually smooth, wind-tossed desert sand & \plus10 \\
        \tb{Uneven Ground}                        & \tb{DV} \tableheaderrule
        Infrequent ankle-high bumps and dips      & \plus2  \\
        Constant ankle-high bumps and dips        & \plus5  \\
        Infrequent knee-high bumps and dips       & \plus5  \\
        Constant knee-high bumps and dips         & \plus10 \\
      \end{dtabularx}
    \end{columntable}

\newpage
\skill{Climb}{Str}
  The Climb skill represents your ability to climb obstacles.
  A creature that is climbing without a \glossterm{climb speed} takes a \minus4 penalty to its \glossterm{accuracy} and Armor and Reflex defenses.
  Moving with the Climb skill requires using the \ability{climb forward} ability (see \pcref{Movement Abilities}).

  \subsection{Common Climb Tasks}
    \parhead{Creature Climb}\nonsectionlabel{Creature Climb} As a standard action, you can make a Climb vs. Reflex attack against a creature you touch.
    This requires one \glossterm{free hand}, and the target must be two or more size categories larger than you.
    On a hit, you latch onto the target and can climb on it as if it was a surface with a \glossterm{difficulty value} equal to its Reflex defense.
    The creature takes a \minus2 accuracy penalty with \glossterm{strikes} against you.
    You must repeat this attack at the end of each subsequent round to stay climbing on the creature.
    For each consecutive round that you climb on a non-ally in this way, you take a cumulative \minus2 penalty to this attack.
    \parhead{Grab Surface} You can make a Climb check as part of movement to grab a surface that you are passing by.
    The \glossterm{difficulty value} is 10 higher than normal for the surface if you were moving for reasons out of your control (such as if you are falling).
    Success means you grab onto the wall and interrupt your movement.
    This does not prevent you from taking \glossterm{falling damage} appropriate for the distance you fell.
    \parhead{Maintain Hold} Whenever you take damage while climbing on a surface, suddenly acquire significantly more weight (such as by catching a falling character), or otherwise are significantly distracted, you must make a Climb check based on the surface.
    Failure means you fall off of the surface, and are \prone when you land.
    \parhead{Wall Jump} You can make a \glossterm{difficulty value} 10 Climb check as part of movement to jump off of a wall you are adjacent to.
    This difficulty value increases by 5 for each time you have used this ability since landing on solid ground.
    Success means you can jump off of the wall (see \pcref{Jumping}).
    Failure means your jump fails, and your movement ends, which typically makes you fall to the ground.
    \parhead{Wallrun} You can make a Climb check as part of your movement while you are touching a solid surface.
    The \glossterm{difficulty value} is 10 higher than normal for the surface.
    Success means you can move using your \glossterm{walk speed} along the wall during the current phase.
    You move at half speed while going up.
    Failure means you fall.
    For every phase in which you use this ability on the same wall without reaching a stable stopping point, the DV increases by 5.

  \subsection{Common Climb Modifiers}
    Slippery and mobile surfaces make Climb checks more difficult.
    If you can brace against multiple surfaces, such as in a corner or between two opposed walls, climbing can be significantly easier.

  \subsection{Climb Difficulty Value Examples}
    \begin{columntable}
      \begin{dtabularx}{\columnwidth}{l X X}
        \tb{DV} & \tb{Surface or Activity}                     & \tb{Example} \tableheaderrule
        0       & Steep slope                                  & A hill too steep to walk up                                          \\
        5       & Large, sturdy hand and foot holds            & Knotted rope, heavily damaged stone wall, ship's rigging             \\
        10      & Small, sturdy hand and foot holds            & Surface with pitons or carved holes, weathered stone wall            \\
        10      & Inconsistent or unsteady hand and foot holds & Unknotted rope, unweathered natural rock, shoddy brick or brick wall \\
        10      & Only large hand holds                        & Tree limbs, pulling yourself up by your hands while dangling         \\
        15      & Rough surface with few holds                 & Weathered natural rock face, quality wood or brick wall              \\
        20      & Rough surface without holds                  & Quality stone wall                                                   \\
        25      & Smooth surface without holds                 & Window                                                               \\
      \end{dtabularx}
    \end{columntable}

  \subsection{Climb Speed}\label{Climb Speed}
    Some creatures have a listed climb speed.
    A creature with a climb speed must still make a Climb check to climb on surfaces.
    However, it can climb with its \glossterm{movement speed} as part of normal movement abilities rather than relying on the \ability{climb forward} ability.

\newpage
\skill{Craft}{Int}
  The Craft skills represent your ability to construct objects from raw materials.
  Like Knowledge, Perform, and Profession, Craft is actually a number of separate skills.
  You could have several Craft skills, each with a separate degree of training.
  Common Craft skills are listed below, with additional description for some skills.
  Other Craft skills exist in the universe, but are less generally useful for adventurers.

  % TODO: note which skills often have magic items?
  \begin{itemize}
    \item Alchemy (Alchemist's fire, tanglefoot bags, potions)
    \item Bone
    \item Ceramics (Glass, pottery)
    \item Leather
    \item Manuscripts (Books, official documents, scrolls)
    \item Metal
    \item Poison
    \item Stone
    \item Textiles (Cloth, fabric)
    \item Wood
  \end{itemize}

  A Craft skill is specifically focused on physical objects. If nothing can be created by an endeavor, it probably falls under the heading of a Profession skill. Complex structures, such as buildings or siege engines, may require Knowledge (engineering) in addition to an appropriate Craft skill.

  \subsection{Common Craft Tasks}\label{Common Craft Tasks}
    \parhead{Create Item} You can make a Craft check to create an item.
    For details, see \pcref{Crafting Items}.
    \parhead{Create Disguised Item} You can craft an item that superficially appears to function like a similar, but different, item.
    This functions like creating the item normally, except that you treat the item's \glossterm{rank} as being one higher than it actually is.
    A creature studying the item with the Identify Item task only identifies the item's false purpose unless they get a \glossterm{critical success} on the check.
    \parhead{Create Forgery} You can make a Craft check to create a false or defective version of an item.
    This functions like creating the item normally, except that you treat the item's \glossterm{rank} as being one lower than it actually is (to a minimum of 0).
    Forgeries which have a function, such as a weapon, are always defective in some way which makes them unsuitable for sustained usage.
    However, a forgery may function once or twice to pass cursory inspection.
    \parhead{Identify Forgery} If you succeed at an opposed Craft vs. Craft check, you can determine whether an object is a forgery.
    \parhead{Identify Item} You can make a Craft check to identify any unusual properties or functions of a magic item or esoteric mundane item.
    The \glossterm{difficulty value} is equal to 5 \add twice the item's \glossterm{rank}.
    Items that are particularly common in a particular setting may be easier to identify, which can reduce the \glossterm{difficulty value} by 2 or more.
    Success means that you know the item's general purpose, and how to activate its functions, including any magical effects.
    You also know the item's rank, which lets you estimate its value.
    \parhead{Rebuild Item} You can make a Craft check to repair a \glossterm{destroyed} item. This functions like creating the item normally, except that you treat the item's \glossterm{rank} as being one lower than it actually is (to a minimum of 0).
    Success means the item is restored to full hit points and functionality.
    \parhead{Repair Item} You can make a Craft check to repair a \glossterm{broken} or damaged item. This takes as much time as creating an item two ranks lower than the item (to a minimum of 0), and does not require any raw materials other than the broken item.
    Success means the item is restored to full hit points and functionality.

  \subsection{Crafting Items}\label{Crafting Items}
    You can use the Craft skill to create an item by expending time and material components.
    Creating an item generally requires multiple consecutive Craft checks.
    Success on a check means you make progress on completing the item.
    Failure means you failed to make progress, but can try again without penalty.

      % * Level 1 crafter making a rank 1 item: +5 vs DV 7, assuming 2 Int.
      %   * That means 10% failure, 50% 1x speed, 40% 2x speed, or 1.3x progress per craft.
      % * Level 10 crafter making a rank 4 item: +11 vs DV 13, assuming 3 Int.
      %   * That means 1.3x progress per craft again.
      % * Level 21 crafter making a rank 7 item: +17 vs DV 19, assuming 4 Int.
    The \glossterm{difficulty value} to craft an item is normally equal to 5 \add twice its rank.

    \subsubsection{Crafting Time}
      The time required to craft an item depends on the \glossterm{weight category} of the constructed item:
      \begin{raggeditemize}
        \item Fine--Tiny: 4 hours per rank.
        \item Small: 8 hours per rank.
        \item Medium: 24 hours per rank.
        \item Large: 40 hours (one week) per rank.
        \item Huge: 120 hours (three weeks) per rank.
        \item Gargantuan: 320 hours (two months) per rank.
        \item Colossal: 960 hours (six months) per rank.
      \end{raggeditemize}

      % TODO: clarify how thinly you can slice crafting?
      When you make a Craft check to craft an item, you make progress on the item based on how long you spend crafting.
      For every 5 points by which you beat the Craft check, you accomplish twice as much work in the same amount of time.
      Once your total effective working time meets or exceeds the time required to craft the item, you have finished the item.

      Individual items may be more or less time-consuming to craft than normal.
      Consumable items, such as potions and arrows, typically take half the normal time to craft.
      At the GM's discretion, individual items may be more time-consuming to craft.
      For example, mail armor was historically quite time-consuming to make, but item-specific modifiers like that are beyond the scope of Rise's rules.

      Multiple crafters can work on the same item, as long as it is physically possible to work on separate components of the item and then combine them together.
      For example, potions and alchemical items typically can't be split between crafters in this way, but multiple people can work on the same set of body armor.

    \subsubsection{Crafting Materials}
      Crafting an item requires the expenditure of raw materials.
      Typically, raw materials of a given item rank can be used to make two items of that rank.
      Note that raw materials for some items, particularly alchemical items, may be hard to come by in less civilized areas.

      You can attempt to craft items from inferior or ad-hoc materials.
      The materials do not have to be well-suited to the item's construction, but they must be physically capable of performing any necessary actions.
      For example, you could construct a simple arrow-throwing trap from bent sticks or creatively strung rope, but not from sand.
      Generally, using ill-suited materials increases the \glossterm{difficulty value} of the Craft check by at least 5, and it may negatively impact the item's function or longevity.

    \subsubsection{Crafting Tools}
      Creating an item requires artisan's tools to give the best chance of success.
      If improvised tools are used, the check may be made with a penalty of \minus5 or greater, or it may be impossible, depending on the tools available and the item to be crafted.
      For example, crafting a bow with improvised woodworking tools would impose a \minus5 penalty, but cutting a diamond without specialized tools is impossible.

    % TODO: clarify how "army of animal friends" works
\newpage
\skill{Creature Handling}{Per}
  % TODO: is non-sapient the right word?
  The Creature Handling skill represents your ability to influence non-sapient creatures.
  With it, you can convince them to do what you want or train them to follow commands.
  This skill has no effect on creatures with an Intelligence of \minus5 or higher.

  \subsection{Common Creature Handling Tasks}

    \begin{sustainability}{Command}{\abilitytag{Auditory}, \abilitytag{Compulsion}, \abilitytag{Sustain} (standard)}
      \abilityusagetime Standard action.
      \rankline
      Make a Creature Handling vs. Mental attack against a creature within \rngmed range.
      In addition, choose and state an action that the creature could take.
      \hit The target is unable to take any actions except to use the \textit{total defense} ability (see \pcref{Universal Abilities}).
      \crit The target performs the chosen action if it is physically capable of performing it.

      You take a \minus10 accuracy penalty against an actively hostile target.
      You take a \minus5 penalty to accuracy with this attack if the target is not an animal, as normal for Creature Handling attacks and checks.
      If the target is damaged or feels that it is in danger, this effect is automatically ended.
    \end{sustainability}
    \parhead{Perform Trained Action} You can make a \glossterm{difficulty value} 5 Creature Handling check to convince an \glossterm{ally} to perform an action it is already trained to perform.
    \parhead{Rear a Wild Creature} You can make a Creature Handling check to raise a wild creature from infancy so that it becomes domesticated.
    The time required depends on how long it takes the creature in question to reach adulthood.
    The \glossterm{difficulty value} for this check is equal to 5 \add twice the creature's level in its adult form.
    This check must be repeated once per year during the process of raising the creature, and when that process is complete.
    Failure means that an additional year of training is required.
    A successfully domesticated creature can be taught tricks at the same time it's being raised, or it can be taught as a domesticated creature later.
    \parhead{Teach Trick}\nonsectionlabel{Teach Trick} You can make a Creature Handling check to teach an \glossterm{ally} a trick.
    A trick is a specific behavior that generally requires a single-word command, like ``fetch'' or ``stay''.
    A creature can learn a maximum of two tricks per point of Intelligence it has above \minus10.
    Teaching a trick generally takes at least a week of intermittent training.
    Simple tricks have a \glossterm{difficulty value} of 5, while complex tricks have a \glossterm{difficulty value} of 10 or more.

  \subsection{Common Creature Handling Modifiers}
    Animals are easier to handle than other kinds of creatures.
    You take a \minus5 penalty to your Creature Handling skill when using it to affect non-animals.

  \subsection{Teaching Tricks}
    Generally speaking, teaching a creature a new trick requires spending at least four hours a day in training over the course of a week.
    It is not generally possible to accelerate the process by spending more time each day; the creature must take time to learn the new behavior.
    If a creature is taught more tricks than its Intelligence allows it to retain, it will forget one of its old tricks during the course of learning the new trick.
    The trainer can choose which old trick will be replaced in this way.

    A list of specific tricks that creatures can be taught is given below.
    Of course, players should feel free to define new tricks to accomplish more specific goals.
    However, complicated tricks are probably more difficult for an animal to learn, so the difficulty value to teach a custom trick might be 15 or higher.

    \parhead{Attack (DV 10)} The creature attacks apparent enemies. You may point to a particular creature that you wish the creature to attack, and it will comply if able. This trick includes teaching the creature how to stop attacking if you give it a command to relent.
    \parhead{Come (DV 5)} The creature comes to you.
    \parhead{Defend (DV 10)} The creature defends you (or is ready to defend you if no threat is present), even without any command being given. Alternatively, you can command the creature to defend a specific other character.
    \parhead{Down (DV 5)} The creature breaks off from combat or otherwise backs down. A creature that doesn't know this trick continues to fight until it must flee (due to injury, a fear effect, or the like) or its opponent is defeated.
    \parhead{Fetch (DV 5)} The creature goes and gets something. If you do not point out a specific item, the creature fetches some random object.
    \parhead{Guard (DV 10)} The creature stays in place and prevents others from approaching.
    \parhead{Heel (DV 5)} The creature follows you closely, even to places where it normally wouldn't go.
    \parhead{Messenger (DV 15)} The creature carries a small item to a destination.
    Once it arrives, it waits for up to 24 hours for someone to take the item from it.
    The destination must be known to the creature.
    \par When you instruct the creature to deliver the item, you must communicate the destination to the creature.
    This normally requires a DV 20 Creature Handling check as a standard action.
    The DV of this check is lowered to 15 for locations the creature is extremely familiar with, such as its home.
    If you have other means of communicating the destination to the creature, such as the \textit{animal speech} druid ability (see Animal Speech, page \pref{Drd:Animal Speech}), that check is unnecessary.
    \parhead{Perform (DV 10)} The creature performs a variety of simple tricks, such as sitting up, rolling over, roaring or barking, and so on.
    \parhead{Seek (DV 5)} The creature moves into an area and looks around for anything that is obviously alive or animate.
    \parhead{Stay (DV 5)} The creature stays in place, waiting for you to return. It does not challenge other creatures that come by, though it still defends itself if it needs to.
    \parhead{Track (DV 10)} The creature tracks the scent presented to it. (This requires the creature to have the \trait{scent} ability)
    \parhead{Work (DV 5)} The creature pulls or pushes a medium or heavy load.

    \subsubsection{Bonus Tricks}\label{Bonus Tricks}
      Some trainers can teach creatures bonus tricks in addition to their normal maximum number of tricks known.
      Once a creature has learned a bonus trick, that trick may not be retrained into a different trick by a trainer who does not have the same ability to grant bonus tricks.

  \subsection{Training Non-Domesticated Creatures}
    Although non-domesticated creatures of any type can be taught tricks with patience and training, they are not naturally obedient.
    When outside their trainer's influence, or in stressful situations, they tend to revert to their natural behaviors.
    In general, even the most skilled trainers can only control one non-domesticated creature in battle.

    In rare circumstances, a skilled trainer may temporarily gain the service of an elite monster.
    While acting at the behest of another creature using Creature Handling, elite monsters can only take one standard action or one elite action during each action phase, not both.

\newpage
\skill{Deception}{Per}
  The Deception skill represents your ability to lie or otherwise mislead people without being caught.
  Using a Deception check is part of conversation or other actions, so it requires no special action to perform.

  \subsection{Common Deception Tasks}
    \parhead{Blend In} If you succeed at an opposed Deception vs. Awareness check, you can avoid notice among a crowd of similar creatures.
    If you act or look significantly different from the creatures around you, observers gain a bonus to their Awareness check to notice you.
    \parhead{Convey Hidden Message} If you succeed at an opposed Deception vs. Social Insight check, you can convey a hidden message in the guise of an ordinary conversation.
    Failure means that an observer recognizes that a hidden message is being conveyed, and may even recognize what that message is.
    In general, you must already have a pre-established code or understanding with the intended recipients of your hidden message so they can grasp its true meaning more easily than outside observers.
    \parhead{Distract} If you succeed at an opposed Deception vs. Social Insight check, you can distract a creature you are talking with.
    This generally makes the distracted creature \glossterm{briefly} \partiallyunaware of you, which can allow you to hide or backstab them.
    Normally, the creature realizes that you tricked them once the distraction ends, which prevents them from being distracted again and may influence their behavior.
    \parhead{Fascinate} If you succeed at an opposed Deception vs. Social Insight check, you can keep attention focused on you during a conversation.
    This generally gives distracted creatures a \minus5 penalty to the Awareness and Social Insight skills to notice anything other than you.
    You repeat this check once per minute, with a cumulative \minus5 penalty for each minute that the distraction has lasted.
    \parhead{Impersonate} If you succeed at an opposed Deception vs. Social Insight check, you can impersonate another creature's mannerisms and speech patterns.
    If you are unable to replicate important aspects of the impersonation, such as the beautiful singing voice of a famous bard, you may suffer a penalty to the Deception check.
    This does not allow you to mimic a creature's appearance, which requires the Disguise skill.
    \parhead{Lie} If you succeed at an opposed Deception vs. Social Insight check, you can lie without giving any indication that you are lying.
    Failure means that the observer recognizes that you are intentionally lying.
    Even if you succeed at this check, you still need the Persuasion skill to believe or take actions based on your lies.
    This check only prevents a creature from recognizing the lie based on your body language and behavior.

    When your overall intention is to mislead or conceal information in a conversation, you may need to make this check even if everything you are saying is technically true.
    Generally, using half-truths and similar trickery instead of bald-faced lies gives you a bonus to your Deception check, but a skilled observer can still see through your ruse.


\newpage
\skill{Deduction}{Int}
  The Deduction skill represents your ability to make logical deductions based on evidence.
  It includes both determining which facts and observations are relevant to use as evidence, and reaching conclusions based on that evidence.
  However, this skill cannot protect you from coming to inaccurate conclusions if you rely on inaccurate or incomplete facts and observations.

  \subsection{Common Deduction Tasks}
    \parhead{Identify Surroundings} You can make a Deduction check as a standard action to understand what aspects of your environment are important and why.
    This may require a successful Awareness check to locate hidden objects or subtle clues.
    \parhead{Reach Conclusion} You can combine information that you know to reach a specific conclusion.
    This may require other checks, such as Knowledge or Awareness checks, to ensure that you have enough information to work with.
    The time required to reach a conclusion can vary dramatically depending on how much evidence you have to work with and how easy the conclusion is to reach.
    You can reach simple conclusions immediately after learning all of the relevant information, but complicated scenarios might require days of study and analysis to eliminate all possibilities.
    In general, sifting through a mixture of helpful and misleading evidence increases the difficulty of the Deduction check and the time required to complete it.

\newpage
\skill{Devices}{Int}
  The Devices skill represents your ability to to manipulate mechanical devices such as locks, traps, and other contraptions.
  With enough skill, you can even manipulate magical devices.
  Each device has a base \glossterm{difficulty value} based on its complexity.
  Some tasks are much easier than others, and modify the difficulty value accordingly.

  Many Devices checks require the use of thieves' tools, which contains items like lockpicks and precision cutting implements.
  If you do not have a proper set of tools, you may be able to improvise from your surroundings.
  Generally, this imposes a penalty of at least \minus5 to the Devices check.

  \begin{dtable}
    \lcaption{Devices Difficulty Values}
    \begin{dtabularx}{\columnwidth}{>{\lcol}X c}
      \tb{Device Type}                          & \tb{Difficulty Value} \tableheaderrule
      Simple device (wagon wheel, typical knot) & 5                              \\
      Average device (door hinge, complex knot) & 10                             \\
      Difficult device (typical lock)           & 15                             \\
      Extraordinary device (expert lock)        & 20                             \\
      Impossible device (magically sealed lock) & 25                             \\
      Mundane trap                              & 10 \add twice \glossterm{rank} \\
      Magic trap                                & 15 \add twice \glossterm{rank} \\
    \end{dtabularx}
  \end{dtable}

  \subsection{Common Devices Tasks}
    \parhead{Activate Device} You can make a Devices check using thieves' tools to make a device perform a function it was designed to do, even if you lack the normal requirements.
    For example, you could tie or untie a knot, lock or unlock a lock without its key, or activate a trap without using its normal triggering mechanism (hopefully without being in its line of fire).
    \parhead{Analyze Device} You can make a Devices check to study a device and understand how it functions.
    The \glossterm{difficulty value} to analyze a device is 5 lower than the device's base difficulty value.
    \parhead{Break Device} You can make a Devices check using thieves' tools to break a device.
    The device ceases to function in its intended way, and the sabotage is obvious to an observer.
    For example, you could jam a lock so it becomes unlocked and can never be locked again.
    Breaking a trap generally triggers the trap in an unpredictable way, which may be dangerous.
    The \glossterm{difficulty value} to break a device is 2 lower than the device's base difficulty value.
    \parhead{Create Bindings} You can make a Devices check with a \plus5 bonus to create bindings from rope or similar materials.
    Binding a helpless foe in this way generally requires a minute of work, though typing up very large creatures may take longer.
    The Flexibility \glossterm{difficulty value} to escape the binding is equal to your check result.
    \parhead{Improvise}\nonsectionlabel{Improvise} You can make a Devices check to construct ad-hoc devices.
    This functions like creating the item with a Craft check (see \pcref{Crafting Items}), with two exceptions.
    First, the item is flimsy, and it breaks after being used once or twice.
    Second, the time requirement is dramatically reduced.
    It takes five minutes to make a device of up to Tiny size.
    You can make a Small device in the time required to make four Tiny devices, a Medium device in the time required to make four Small devices, and so on.
    \parhead{Remove Device} You can make a Devices check using thieves' tools to fully disable a device and remove it if possible.
    This can allow you to bypass traps without ever triggering them, and even take them with you if they are small and portable.
    Magical traps and large-scale physical traps, such as pit traps, are generally not portable.
    The \glossterm{difficulty value} to remove a device is 5 higher than the device's base difficulty value.

\newpage
\skill{Disguise}{Int}
  The Disguise skill represents your ability to create disguises to conceal the appearance of creatures or objects.
  This skill does not help you act appropriately while disguised, which generally the Deception skill and may also require Social Insight.

  Many Disguise checks require the use of a disguise kit, which contains items like makeup and false beards.
  If you do not have a proper kit, you may be able to improvise from your surroundings.
  Generally, this imposes a penalty of at least \minus5 to the Disguise check.

  \subsection{Common Disguise Tasks}
    \parhead{Camouflage} You can make an opposed Disguise vs. Awareness check to blend into your surroundings and avoid being noticed.
    This generally takes at least a minute of work to prepare your disguise to match your exact surroundings.
    It only protects you from visual observation, so you would generally need the Stealth skill to avoid being heard while moving (see \pcref{Stealth}).
    \parhead{Change Appearance}\nonsectionlabel{Change Appearance}
    You can make an opposed Disguise vs. Awareness check using a disguise kit to change a creature's appearance.
    Generally, applying a disguise takes at least a minute, though complex makeup applications or clothing changes can take much longer.
    You take a penalty to the Disguise check based on how radical your changes are, especially to the creature's basic proportions.
    \parhead{Emulate Appearance} You can make a creature look like a different specific creature.
    This functions like the \textit{change appearance} task, except that the result of your Disguise check can't exceed the result of an Awareness check you or someone helping you made to observe the creature you are trying to emulate.

  \subsection{Common Disguise Modifiers}
    It is generally easier to enlarge a creature or add new features than it is to shrink a creature or remove existing features.
    You can use the table below a guide, and the GM can improvise penalties for more unusual disguises if necessary.
    If you make multiple alterations, the penalties stack.

    \begin{columntable}
      \begin{dtabularx}{\columnwidth}{l X}
        \tb{Age Change}                              & \tb{Disguise Penalty} \\
        Per age category of difference & \minus2 \\
        \tb{Body Shape Change}                              & \tb{Disguise Penalty} \\
        To a different gender                               & \minus2               \\
        Per removed arm                                     & \minus2               \\
        Per removed leg                                     & \minus5               \\
        Per additional arm or leg                           & \minus5               \\
        \tb{Species Change}                                 & \tb{Disguise Penalty} \tableheaderrule
        To a noticeably larger species (halfling to human)  & \minus5               \\
        To a noticeably smaller species (human to halfling) & \minus15              \\
        To a larger size category (human to ogre)           & \minus15              \\
        To a smaller size category (ogre to human)          & \minus30              \\
      \end{dtabularx}
    \end{columntable}

\newpage
\skill{Endurance}{Con}
  The Endurance skill represents your ability to persevere through physical trials.

  \subsection{Common Endurance Tasks}\label{Common Endurance Tasks}
    \parhead{Hold Breath}\nonsectionlabel{Hold Breath}
    You can make an Endurance check to hold your breath.
    While holding your breath, you must make an Endurance check at the end of every 5 rounds that you spend without taking any actions, or at the end of any round in which you take an action.
    The \glossterm{difficulty value} starts at 0, and increases by 1 for each subsequent check until you breathe in air.
    Failure means that you try to breathe in air, and you gain a \glossterm{vital wound} if there is no air available to breathe.

    Essentially, you can fight while holding your breath for a number of rounds equal to your Endurance modifier with no risk of failure.
    If you stay still, you can hold your breath for a number of minutes equal to half your Endurance modifier with no risk of failure.
    \parhead{Maintain Exertion}\nonsectionlabel{Maintain Exertion}
    Some activities are difficult to maintain indefinitely.
    For example, sprinting typically exhausts a creature in a minute or less.
    Even walking at a steady pace can become exhausting after hours without rest.
    You can make an Endurance check to continue performing a strenuous activity without rest.
    Most active abilities, such as casting spells and making \glossterm{strikes}, are considered strenuous activity.
    However, performing \glossterm{rituals} is not strenuous.

    Generally, this requires a \glossterm{difficulty value} 10 Endurance \glossterm{extended check} when you have performed the task for as long as a normal human can do it without rest.
    Each time you succeed at this check, you can maintain the exertion for half that length of time.
    At the end of that time, you must repeat this check to continue the activity, but the difficulty value increases by 5.
    If you fail the check, you must rest.
    The details of how long you have to rest depends on the activity you were performing, at the GM's discretion.

    For example, a human can normally sprint for 10 rounds without rest.
    If they pass a DV 10 Endurance \glossterm{extended check}, they could sprint for an additional 5 rounds.
    After that time, they would need to make a DV 15 Endurance \glossterm{extended check} for the next 5 rounds, then DV 20, and so on.
    \parhead{Overland Exertion} You can make an Endurance check while travelling overland to cover more ground (see \pcref{Overland Movement}).
    This works in the same way as maintaining exertion, described above.
    There are two ways that you can exert yourself: hustling, which doubles your distance travelled during a given hour, and making a forced march, which allows you to travel for an extra hour beyond the normal travel time.
    Making a forced march only increases the \glossterm{difficulty value} of the check by 2 for each additional hour, instead of the normal 5.

    \parhead{Stay Awake}\nonsectionlabel{Stay Awake} You can make an Endurance \glossterm{extended check} to stay awake beyond healthy limits.
    A typical creature needs a minimum of 6 hours of sleep for every 18 hours spent awake, and a minimum of 50 hours of sleep every week.
    The \glossterm{difficulty value} starts at 5, and increases by 5 for each subsequent check until you catch up on your missed sleep.
    Failure means you increase your \glossterm{fatigue level} by three.
    You must make another \glossterm{extended check} every 8 hours as long as you are still beyond your normal sleep limits.

    % This is thematically appropriate, but it breaks the rule about not introducing ``shadow'' resources
    % \subsection{Ignore Fatigue}\label{Ignore Fatigue}
    %     You can temporarily ignore your fatigue to try to accomplish your objectives.
    %     Ignoring your fatigue requires a \glossterm{minor action}.
    %     The \glossterm{difficulty value} starts at 10.
    %     If you succeed, you reduce your \glossterm{fatigue penalty} by 1 until the end of the round.
    %     For every 10 points by which you succeed, you reduce your fatigue penalty by an additional 1.
    %     Ignoring your fatigue is a \abilitytag{Swift} ability.
    %     You can only use this ability once per \glossterm{short rest}.

\newpage
\skill{Flexibility}{Dex}
  The Flexibility skill represents your ability to escape bindings and move through small areas by contorting your body.

  \begin{dtable}
    \lcaption{Flexibility Difficulty Values}
    \begin{dtabularx}{\columnwidth}{>{\lcol}X l}
      \tb{Restraint}         & \tb{Difficulty Value} \tableheaderrule
      Net                    & 8  \\
      Common manacles        & 15 \\
      High-quality manacles  & 20 \\
      Extraordinary manacles & 25 \\
      \tb{Tight Space}       & \tb{Difficulty Value} \tableheaderrule
      Can fit with outstretched elbows & 5 \\
      Can fit one outstretched elbow & 10 \\
      Can fit head and shoulders only & 15 \\
      Can fit head only & 25 \\
    \end{dtabularx}
  \end{dtable}

  \subsection{Common Flexibility Tasks}

    \parhead{Escape Bindings} As a standard action, you can make an Flexibility check to escape physical bindings.
    For simple restraints like nets and manacles, the \glossterm{difficulty value} generally depends on the quality of the restraint.
    For complex restraints like carefully tied rope bindings, make an opposed Flexibility vs. Devices check against the creature that created the restraint.
    \parhead{Escape Grapple} As a standard action, you can make a Flexibility check to escape a grapple.
    For details, see \pcref{Escape Grapple}.
    \parhead{Tight Squeeze} You can make a Flexibility check to squeeze into spaces too small to normally fit you (see \pcref{Squeezing}).

\newpage
\skill{Intimidate}{Varies}
  The Intimidate skill represents your ability to intimidate and coerce people into doing what you want.

  \parhead{Choosing an Attribute}
  No attribute is a key attribute for Intimidate.
  However, depending on how you are trying to intimidate creatures, you can add any attribute's base value to your Intimidate check.
  For example, if you intimidate a creature by smashing a table and threatening to smash its head in, you can add your Strength to the Intimidate check.
  On the other hand, if you intimidate a creature by staring it down in a cold fury, you can add your Willpower to the Intimidate check.

  \subsection{Common Intimidate Tasks}
    \parhead{Coerce} You can make an Intimidate check to convince a creature to do what you want. This functions like a Persuasion check to form an agreement with a group, except that you do not apply a relationship modifier (see \pcref{Persuasion}).
    Generally, people dislike being coerced.

    \parhead{Demoralize}\nonsectionlabel{Demoralize}
    \begin{activeability}{Demoralize}
      \abilityusagetime Standard action.
      \rankline
      Make an Intimidate vs. Mental attack against up to two creatures within \shortrange.
      % TODO: this is a little weird as a universal ability. It's r-1 power level.
      \hit If the target has no remaining \glossterm{damage resistance}, it becomes \frightened by you as a \glossterm{condition}.
    \end{activeability}

\newpage
\skill{Jump}{Str}
  The Jump skill represents your ability to jump.
  Unlike most skills, you have no numeric modifier to Jump, and you never make a Jump check.
  Instead, the distance you can jump changes depending on whether you are trained with the Jump skill.
  For details, see (see \pcref{Jumping}).

\newpage
\skill{Knowledge}{Int}
  The Knowledge skills represent your understanding of particular aspects of the world.
  Like the Craft, Profession, and Perform skills, Knowledge actually encompasses a number of separate skills.
  Knowledge represents a study of some body of lore, such an academic or even scientific discipline.
  Typical fields of study are listed below, but the GM may create additional fields or decide that some fields are irrelevant in a particular setting.
  \begin{itemize}
    \item Arcana (arcane spells, dragons, magical beasts)
    \item Dungeoneering (aberrations, caverns, oozes, spelunking, subterranean monsters)
    \item Engineering (architecture, buildings, bridges, fortifications, siege weapons)
    \item Items (magic items, artifacts, constructs)
    \item Local (myths and legends, laws and customs, history, nobility and royalty, nearby monsters)
    \item Nature (nature spells, animals, fae, monstrous humanoids, plants, terrain and climate)
    \item Planes (pact spells, the Primal Planes, the Aligned Planes, the Astral Plane,
      planeforged, magic related to the planes, extraplanar monsters)
    \item Religion (divine spells, undead, deities, mythic history, religious traditions, holy symbols)
  \end{itemize}

  \subsection{Common Knowledge Tasks}
    \parhead{Identify Monster} You can make a Knowledge check to identify a monster and recall its special powers or vulnerabilities.
    Each monster notes in its description the specific information that you learn from a successful Knowledge check.
    In general, the \glossterm{difficulty value} for basic information is equal to 5 \add the monster's level.
    Legendary monsters such as dragons can be much easier to recognize.
    \parhead{Identify Item} You can make a Knowledge check to identify any unusual properties or functions of a magic item or esoteric mundane item.
    The \glossterm{difficulty value} is equal to 5 \add twice the item's \glossterm{rank}.
    Items that are particularly common in a particular setting may be easier to identify, which can reduce the \glossterm{difficulty value} by 2 or more.
    Success means that you know the item's general purpose, and how to activate its functions, including any magical effects.
    You also know the item's rank, which lets you estimate its value.
    \parhead{Recall Information} You can make a Knowledge check to remember information related to your field of study.
    The \glossterm{difficulty value} varies depending on the difficulty of the question (see \pcref{Standard Difficulty Values}).
    \parhead{Identify Magical Effect}\nonsectionlabel{Identify Magical Effect} You can make a Knowledge check to identify the general nature of a magical effect that you observe.
    The \glossterm{difficulty value} is generally equal to 10 \add twice the effect's \glossterm{rank}.
    Unusually obscure or obvious magical effects can have higher or lower difficulty values.

    You must use a Knowledge skill relevant to the magical effect.
    Arcane effects require Knowledge (arcana), divine effects require Knowledge (nature), nature effects require Knowledge (nature), and pact effects require Knowledge (planes).
    In some circumstances, other Knowledge skills could be used if they are directly relevant to the magical effect.
    For example, Knowledge (dungeoneering) could be used to identify many spells from the \sphere{terramancy} \glossterm{mystic sphere}.

\newpage
\skill{Medicine}{Int}
  The Medicine skill represents your practical understanding of how to tend to the wounds of living creatures.
  In order to use this skill to aid a creature, you must be able to see and touch it, and the creature must be alive.

  Many Medicine checks require the use of a medical kit, which contains items like bandages and salves.
  If you do not have a proper kit, you may be able to improvise from your surroundings.
  Generally, this imposes a penalty of at least \minus5 to the Medicine check.

  \subsection{Common Medicine Tasks}
    \parhead{Accelerate Recovery}\nonsectionlabel{Accelerate Recovery}
    You can make a \glossterm{difficulty value} 15 Medicine \glossterm{extended check} using a medical kit to accelerate the recovery of up to four creatures from among yourself and your \glossterm{allies} during a \glossterm{long rest}.
    Success means that each creature removes an additional vital wound (see \pcref{Removing Vital Wounds}).
    For every 10 points by which you succeed, each creature removes an additional vital wound.
    \parhead{First Aid}\nonsectionlabel{First Aid}
    As a standard action, you can make a Medicine check using a medical kit to prevent a creature from dying from a \glossterm{vital wound} with a negative \glossterm{vital roll}.
    The \glossterm{difficulty value} is equal to 10 for a vital roll of 0.
    The DV increases by 5 for each point by which the vital roll is below 0.
    Success means that the target treats the \glossterm{vital roll} as a 1 instead of its original value.
    This changes the effect of the vital wound, generally preventing the target from dying.
    For details, see \pcref{Vital Wounds}.
    \parhead{Identify Affliction}\nonsectionlabel{Identify Affliction}
    You can make a Medicine check to identify a poison, disease, or similar affliction currently affecting a creature.
    The \glossterm{difficulty value} is equal to 5 \add twice the \glossterm{rank} of the poison or disease.
    \parhead{Treat Disease} With five minutes of work, you can make a Medicine check to treat a creature that is currently diseased.
    The next time it is attacked by its current disease, it can use your Medicine check or its Fortitude defense, whichever is higher.
    \parhead{Treat Poison} As a standard action, you can make a Medicine check to treat a creature that is currently poisoned.
    The next time it is attacked by its current poison, it can use your Medicine check or its Fortitude defense, whichever is higher.

\newpage
\skill{Perform}{Dex}
  The Perform skills represent your ability to create particular forms of entertainment.
  Like Craft, Knowledge, and Profession, Perform is actually a number of separate skills.
  You could have several Perform skills, each with a separate degree of training.
  Each of the nine categories of the Perform skill includes a variety of methods, instruments, or techniques, a small list of which is provided for each category below.

  \begin{itemize}
    \item Acting (drama, impersonation, mime)
    \item Comedy (buffoonery, limericks, joke-telling)
    \item Dance (ballet, waltz, jig)
    \item Keyboard instruments (harpsichord, piano, pipe organ)
    \item Oratory (epic, ode, storytelling)
    \item Percussion instruments (bells, chimes, drums, gong)
    \item Singing (ballad, chant, melody)
    \item String instruments (fiddle, harp, lute, mandolin)
    \item Wind instruments (flute, pan pipes, recorder, shawm, trumpet)
  \end{itemize}

  \subsection{Performance Types}\label{Performance Types}
    There are four types of performances: dance, instrumental, manipulation, and vocal.
    \begin{raggeditemize}
      \item Dance: You use your body to dance or act. This limits your ability to defend yourself, giving you a \minus2 penalty to your Armor and Reflex defenses as a \atSwift effect. Dance performances have the \atVisual tag.
      \item Instrumental: You use an instrument to make music. This requires at least one \glossterm{free hand} to use the instrument. Instrumental performances have the \atAuditory tag.
      \item Manipulation: You use objects or gestures to perform, such as juggling or puppetry. This requires at least one \glossterm{free hand} to use the objects. Manipulation performances have the \atVisual tag.
      \item Vocal: You use your voice to orate or sing. This prevents you from talking or using other abilities with \glossterm{verbal components}. Vocal performances have the \atAuditory tag.
    \end{raggeditemize}

    % TODO: how exactly does this interact with bardic performances?
  \subsection{Limitations while Performing}
    It takes a \glossterm{minor action} to initiate and sustain a performance.
    While you are performing, you take a \minus5 penalty to the Perform skill for any other performances.
    This penalty stacks, and applies separately for each simultaneous performance.
    For example, if you were playing a lyre, singing, and juggling balls with your feet, you would take a \minus10 penalty to all three performances.
    These limitations are in addition to any restrictions imposed by your method of performing, such as your hands being occupied playing an instrument.

    You can otherwise act normally while performing, including attacking in combat, if doing so is physically possible.

    \parhead{Performance Time}
    In general, you can maintain a performance for up to an hour.
    After that time, you must finish a \glossterm{short rest} before performing again.

  \subsection{Common Perform Tasks}
    \parhead{Distract} If you succeed at an opposed Perform vs. Social Insight check, you can distract a creature observing your performance.
    This generally makes the distracted creature \glossterm{briefly} \partiallyunaware of you, which can allow you to hide or backstab them.
    Normally, the creature realizes that you tricked them once the distraction ends, which prevents them from being distracted again and may influence their behavior.
    \parhead{Entertain} You can make a Perform check to provide entertainment or to show off your skills.
    % In theory, this should have nonlinear scaling, but it's probably not worth the effort.
    \parhead{Earn Income} You can make a Perform check to practice your trade and make a decent living.
    You earn about half your Perform check result in silver pieces per week of dedicated performance.
    \parhead{Fascinate} If you succeed at an opposed Perform vs. Social Insight check, you can keep attention focused on you while you perform.
    This generally gives distracted creatures a \minus5 penalty to the Awareness and Social Insight skills to notice anything other than you.
    You repeat this check once per minute, with a cumulative \minus5 penalty for each minute that the distraction has lasted.

\newpage
\skill{Persuasion}{Per}
  The Persuasion skill represents your ability to convince people to think what you want them to.
  Depending on how it is used, it represents a combination of verbal acuity, tact, argumentative ability, grace, etiquette, and personal magnetism.
  Using a Persuasion check usually takes at least a minute of sustained conversation.

  Not all social interactions require Persuasion checks. Much of the time, being extraordinarily persuasive is unnecessary, and creatures can be convinced with normal, inartful conversation and good reasoning. Persuasion checks should only be used when your personal persuasiveness matters.

  \subsection{Common Persuasion Tasks}
    \parhead{Compel Belief} As part of conversation, you can make a Persuasion check to cause creatures to believe something you say to be true.
    If you are lying, you must also make a Deception check to avoid revealing the lie.
    The base \glossterm{difficulty value} is equal to each creature's Mental defense.
    It is generally easier to convince creatures of things that are highly plausible or beneficial to them.
    Similarly, it is generally harder to convince creatures of things that are highly unlikely or detrimental to them.
    \parhead{Form Agreement} As part of conversation, you can make a Persuasion check to cause creatures to accept a deal or arrangement you propose.
    The base \glossterm{difficulty value} is equal to each creature's Mental defense.
    It is generally easier to convince creatures if the deal is good for them, and harder if it is bad for them.
    \parhead{Gather Information} You can make a Persuasion check to gather information from people around you.
    The \glossterm{difficulty value} is 5 for basic information, 10 for information that most people wouldn't know, and even higher for secrets or intentionally concealed information.
    This generally requires spending a few hours to meet a variety of people and learn what they know.

  \subsection{Common Persuasion Modifiers}
    The difficulty value for all Persuasion checks is modified based on the relationship between characters in a conversation, as listed in \tref{Persuasion Relationship Modifiers}.
    Regardless of what you are saying, you are more likely to succeed when talking to a close friend than a sworn enemy.

    \begin{dtable}
      \lcaption{Persuasion Relationship Modifiers}
      \begin{dtabularx}{\columnwidth}{>{\lcol}X r}
        \tb{Relationship}                                                                                                                                                                 & \tb{Difficulty Modifier} \tableheaderrule
        Intimate: Someone who with whom you have an implicit trust.
        Example: A lover or spouse.                                                                                                                                                       & \minus15 \\
        Friend: Someone with whom you have a regularly positive personal relationship.
        Example: A long-time buddy or a sibling.                                                                                                                                          & \minus10 \\
        Ally: Someone on the same team, but with whom you have no personal relationship.
        Example: A cleric of the same religion or a knight serving the same king.                                                                                                         & \minus5  \\
        Acquaintance (Positive): Someone you have met several times with no particularly negative experiences. Example: The blacksmith that buys your looted equipment regularly.         & \minus2  \\
        Just Met: No relationship whatsoever.
        Example: A guard at a castle or a traveler on a road.                                                                                                                             & \plus0   \\
        Acquaintance (Negative): Someone you have met several times with no particularly positive experiences. Example: A town guard that has arrested you for drunkenness once or twice. & \plus2   \\
        Opposition: Someone who is part of a group that consistently works against your interests, with whom you have no personal relationship.
        Example: An outlaw (to a law-abiding person), a paladin of law (to an outlaw), or a soldier who fights for a country at war with your country.                                                                                      & \plus5   \\
        Enemy: Someone with whom you have a specifically antagonistic relationship.
        Example: An evil warlord whom you are attempting to thwart, a bounty hunter who is tracking you down for your crimes, or a bandit currently robbing you.                                                          & \plus10  \\
        Nemesis: Someone who has sworn to do you, personally, harm, or vice versa. Example: The brother of a man you murdered in cold blood, or the person who murdered your brother in cold blood.                                                             & \plus15  \\
      \end{dtabularx}
    \end{dtable}

\newpage
\skill{Profession}{Varies}
  The Profession skills represent your practical understanding of a particular profession.
  Like Craft, Knowledge, and Perform, Profession is actually a number of separate skills.
  You could have several Profession skills, each with a separate degree of training.
  While a Craft skill represents ability in creating or making an item, a Profession skill represents an aptitude in a vocation requiring a broader range of less specific knowledge.
  Most commoners have some training in a Profession skill.

  \parhead{Choosing an Attribute}
  No attribute is a key attribute for Profession.
  However, depending on how you are using your Profession, you can add any attribute's base value to your Profession check.
  For example, if you use your experience as a farmer to harrow a field, you can add your Strength to the Profession check.
  On the other hand, if you use your experience as a sailor to determine the right angle for sails in the current wind, you can add your Perception to the Profession check.

  \subsection{Common Profession Tasks}
    % Average skilled level 1 character would have about +4 to Profession.
    % That's a 70% chance to make 1 gp, so 0.7gp/week.
    \parhead{Earn Income} You can make a Profession check to practice your trade and make a decent living.
    If you make a \glossterm{difficulty value} 8 Profession check, you earn 1gp with five days of work.
    For every 5 points by which you beat that difficulty value, you make twice as much.
    \parhead{Identify Item} You can make a Profession check to identify any unusual properties or functions of a magic item or esoteric mundane item.
    The \glossterm{difficulty value} is equal to 5 \add twice the item's \glossterm{rank}.
    Items that are particularly common in a particular setting may be easier to identify, which can reduce the \glossterm{difficulty value} by 2 or more.
    Success means that you know the item's general purpose, and how to activate its functions, including any magical effects.
    You also know the item's rank, which lets you estimate its value.
    \parhead{Perform Task} You can make a Profession check to perform some tasks related to your profession.
    This allows you to use Profession in place of other skills when it is appropriate.
    For example, a sailor could use Profession to tie common knots in place of Devices or Survival, or a farmer could use Profession to identify common animals and plants in place of Knowledge (nature).
    The \glossterm{difficulty value} when using Profession may be higher than it would be to use the normal skill for the task, depending on the relevance of the Profession skill.

\newpage
\skill{Ride}{Dex}
  The Ride skill represents your ability to ride and control horses and other mounts.
  Typical riding actions don't require checks. You can saddle, mount, ride, and dismount from a mount without a problem. However, some special actions require Ride checks.

  Unless an ability says otherwise, you can only use this skill to ride \glossterm{allies} that are exactly one size category larger than you.

  \subsection{Common Ride Modifiers}
    If a creature is not trained as a mount, the DV to ride it increases by 5.
    If it lacks a saddle and other riding gear, the DV to ride it increases by 5.
    If it takes a standard action other than movement, such as attacking, the DV to ride it that round increases by 5.
    If it uses a \glossterm{movement mode} other than a walk speed, the DV to ride it that round increases by 10.

  \subsection{Common Ride Tasks}
    \parhead{Guide Mount} When riding on a creature, you can make a Ride check to direct your mount's movement.
    While travelling, this check is only necessary when giving the mount directions.
    In battle, this check must be repeated at the start of each round.
    If the mount is trained for battle, the \glossterm{difficulty value} of this check is 0.
    Otherwise, the DV is 5.
    Success means the mount understands your direction, and will obey if it is willing and able.
    Failure means the mount does not understand your direction, and acts of its own volition.

    If you can communicate with your mount in other ways, such as by speaking with it, this check may be unnecessary.
    \parhead{Maintain Ride} Whenever you take damage or your mount makes a sudden motion, you must make a DV 5 Ride check to continue riding the creature.
    Sudden motions include jumping, attacking, and moving at more than half speed.
    Failure means you fall off of your mount.
    \parhead{Take Cover} You can make a DV 15 Ride check as a \glossterm{move action} to drop low and take \glossterm{cover} behind your mount.
    This requires the use of a \glossterm{free hand}.
    Failure means you can't get low enough and gain no benefit from the action.

\newpage
\skill{Sleight of Hand}{Dex}
  The Sleight of Hand skill represents your ability to pick pockets, palm objects, and perform other feats of legerdemain.

  \subsection{Common Sleight of Hand Modifiers}
    All Sleight of Hand checks apply a special modifier based on the size of the action taken or object affected, as shown on \trefnp{Sleight of Hand Difficulty Modifiers}.

    \begin{dtable}
      \lcaption{Sleight of Hand Difficulty Modifiers}
      \begin{dtabularx}{\columnwidth}{X l}
        \tb{Size}   & {Difficulty Modifier} \tableheaderrule
        Fine        & \minus10   \\
        Diminutive & \minus5   \\
        Tiny        & \plus0   \\
        Small       & \plus5  \\
        Medium      & \plus10  \\
        Large       & \plus15 \\
        Huge        & \plus20 \\
        Gargantuan  & \plus25 \\
        Colossal    & \plus30 \\
      \end{dtabularx}
    \end{dtable}

  \subsection{Common Sleight of Hand Tasks}
    \parhead{Conceal Object} You can make an opposed Sleight of Hand vs. Awareness check to conceal an \glossterm{ally} or \glossterm{unattended} object on your person.
    The target must be at least one size category smaller than you are.
    \parhead{Conceal Action} You can make an opposed Sleight of Hand vs. Awareness check to conceal an action that you take.
    The space required to perform the action is the size of the action, and applies a size-based bonus or penalty appropriately.
    The action must be at least one size category smaller than you are.
    For example, throwing a dagger is a Small-sized movement, so you take a \minus5 penalty to conceal the action.
    If you successfully conceal an attack, the defender is at least \partiallyunaware of it (see \pcref{Awareness and Surprise}).
    \parhead{Pickpocket} You can make an Sleight of Hand check to steal an object from another creature.
    The object must be loose and accessible, such as in a pocket.
    All observers, including the creature you are stealing from, can make an Awareness check against your Sleight of Hand check result to observe your attempt.
    If your check result beats the target's Reflex defense, you steal the object.

\newpage
\skill{Social Insight}{Per}
  The Social Insight skill represents your ability to read body language and emotion.

  \subsection{Common Social Insight Tasks}
    \parhead{Discern Enchantment} You can make a Social Insight check to notice whether a creature is affected by any behavior-altering effects.
    Noticing a \abilitytag{Compulsion} effect is \glossterm{difficulty value} 5, and noticing an \abilitytag{Emotion} or \ability{Subtle} effect is difficulty value 20.
    You can use this task to notice effects on yourself in addition to other creatures.
    \parhead{Discern Hidden Message} You can make an opposed Social Insight vs. Deception check to recognize when a hidden message is being conveyed in a conversation.
    \parhead{Discern Lies} You can make an opposed Social Insight vs. Deception check to recognize when a creature is intentionally lying or concealing the truth.
    \parhead{Notice Subtle Effect} If you succeed at a Social Insight check, you can notice the general effects of a \abilitytag{Subtle} ability affecting a creature you are talking with.
    The \glossterm{difficulty value} to notice the effect is 20 \add twice the \glossterm{rank} of the ability.
    \parhead{Social Assessment}\nonsectionlabel{Social Assessment}
    You can make a Social Insight check to get a general assessment of a social situation after a minute of observation.
    The base \glossterm{difficulty value} is equal to 10.
    Simple and familiar social situations are easier to understand, while complex and unfamiliar social situations can be much harder to understand.
    Success means you learn relevant information about the situation, such as a general understanding of expected behaviors or a rough understanding of the social hierarchy.

\newpage
\skill{Stealth}{Dex}
  The Stealth skill represents your ability to escape detection while moving or taking large-scale actions.
  All Stealth checks are made as part of movement or other actions, so they require no special action to perform.
  If you have been noticed by a creature, you automatically fail all Stealth checks against that creature until you can escape its notice, such as by disappearing out of sight.

  \subsection{Common Stealth Tasks}
    \parhead{Avoid Notice} You can make an opposed Stealth vs. Awareness check to prevent creatures from noticing you.
    Success means that the observer's awareness of you, such as unaware or partially unaware, does not change.
    Failure means that the observer can observe you using any senses they detected you with.
    Generally, success with sight-based senses causes creatures to become fully aware of you, while success with other senses causes creatures to be \partiallyunaware of you.
    You must repeat this check whenever you take an action that you want to conceal, such as moving, or your circumstances otherwise meaningfully change in a way that would make you easier to observe.
    \parhead{Hide} You can make an opposed Stealth vs. Awareness check to make creatures that are aware of you lose track of your position.
    In order to use this ability, you must move in a way that makes observers lose sight of you for at least ten feet of your motion.
    In addition, you must have \glossterm{cover} or \glossterm{concealment} for the entire duration of your movement.
    This can be achieved by moving through total darkness, moving out of \glossterm{line of sight}, teleporting at least ten feet, or similar activities.
    Success against an observer means that they become \partiallyunaware of you instead of fully aware of you.

  \subsection{Common Stealth Modifiers}\label{Common Stealth Modifiers}

    A creature smaller than Medium size gains a \plus5 bonus to the Stealth skill for each size category by which it is smaller than Medium.
    Similarly, a creature larger than Medium size takes a \minus5 bonus to the Stealth skill for each size category by which it is smaller than Medium.
    These effects are summarized below.
    \begin{itemize}
      \item Fine: \plus20
      \item Diminutive: \plus15
      \item Tiny: \plus10
      \item Small: \plus5
      \item Medium: \plus0
      \item Large: \minus5
      \item Huge: \minus10
      \item Gargantuan: \minus15
      \item Colossal: \minus20
    \end{itemize}

    Stealth checks generally require \glossterm{cover} or \glossterm{concealment} (see \pcref{Cover} and \pcref{Concealment}).
    For this purpose, do not consider any cover that would be hidden as a result of a successful check, such as an object you hold in front of you.
    You take a \minus20 penalty to Stealth checks against creatures who can observe you without any interfering cover or concealment.
    This includes creatures who can ignore concealment with abilities like \trait{blindsight}.

    You take a \minus10 penalty to Stealth checks against creatures who can know your location with a special ability like \trait{blindsense}.
    This does not stack with the penalty for not having cover or concealment.

    Moving stealthily is more difficult than hiding in place.
    If you use a movement speed to move, you take a penalty to your Stealth check to conceal that movement.
    This is a \minus5 penalty if you move at no more than half your speed.
    If you use the \textit{sprint} ability or move faster than half your speed, this penalty increases to \minus10.

    Making a \glossterm{strike}, using \glossterm{somatic components}, and taking other similar large-scale actions imposes a \minus10 penalty to the Stealth check.
    If you make a strike with a \weapontag{Heavy} weapon, this penalty increases to \minus20.
    This is separate from and stacks with the \plus20 bonus that a creature gets to notice you if you hit it with a \glossterm{strike} (see \pcref{Common Awareness Modifiers}).

\newpage
\skill{Survival}{Per}
  The Survival skill represents your ability to take care of yourself and others in the wilderness, including the ability to follow tracks.

  \subsection{Common Survival Tasks}
    \parhead{Find Sustenance} You can make a Survival check to hunt or forage for food and water.
    This generally takes a few hours of work to find enough sustenance for you and a small group of allies.
    The \glossterm{difficulty value} and details of what you find depend on the terrain.
    \parhead{Follow Tracks} You can make a Survival check to follow tracks at up to half your normal movement speed.
    You can move at full speed if you accept a \minus5 penalty to the check.
    The \glossterm{difficulty value} depends on how easy the tracks are to notice.
    You must repeat this check whenever the trail changes significantly, such as when it crosses other tracks or passes through a different environment.
    \parhead{Navigate Wilderness} You can make a Survival check while moving overland to avoid natural terrain hazards and getting lost.
    The \glossterm{difficulty value} and consequences of failure depend on the terrain.
    Overland travel often follows standard roads or paths, which may make this check unnecessary depending on the quality of the road.
    \parhead{Predict Weather} You can make a \glossterm{difficulty value} 10 Survival check to predict the local weather for the next day.

  \subsection{Terrain Difficulty Values}
    These are general guidelines, not exact rules.
    The GM can tell you more about the specific landscape you are traversing.
    \begin{columntable}
      \begin{dtabularx}{\columnwidth}{l X X}
        \tb{Terrain} & \tb{Navigation Difficulty Value} & \tb{Sustenance Difficulty Value} \tableheaderrule
        Desert       & 10                                & 20 \\
        Forest       & 10                                & 10 \\
        Jungle       & 15                                & 5 \\
        Mountains    & 10                                & 15 \\
        Hills        & 5                                 & 10 \\
        Plains       & 5                                 & 10 \\
        Swamp        & 15                                & 15 \\
      \end{dtabularx}
    \end{columntable}

  \subsection{Tracking}\label{Tracking}
    One of the key uses for the Survival skill is to follow tracks left by creatures.
    You can use the Awareness skill to notice signs of passage, but the Survival skill is necessary to follow tracks for any distance.
    Some suggestions for determining the difficulty of following a trail can be found in \trefnp{Example Tracking Difficulty Values} and \trefnp{Example Tracking Difficulty Modifiers}.
    The GM may also apply other circumstantial modifiers not listed here.

    \begin{dtable}
      \lcaption{Example Tracking Difficulty Values}
      \begin{dtabularx}{\columnwidth}{l >{\lcol}X l}
        \tb{Surface}     & \tb{Description}                                                                                                                                                     & \tb{Difficulty Value} \tableheaderrule
        Very soft ground & Any surface (fresh snow, thick dust, wet mud) that holds deep, clear impressions of footprints.                                                                      & 0                                 \\
        Soft ground      & Any surface soft enough to yield to pressure, but firmer than wet mud or fresh snow, in which a creature leaves frequent but shallow footprints                      & 5                                                                                  \\
        Firm ground      & Most normal outdoor surfaces (such as lawns, fields, woods, and the like) or exceptionally soft or dirty indoor surfaces (thick rugs and very dirty or dusty floors) & 10                                                                                          \\
        Hard ground      & Any surface that doesn't hold footprints at all, such as bare rock or a streambed                                                                                    & 15                    \\
        Scent            & Tracking using the \trait{scent} ability instead of vision                                                                                                           & 5 \\
      \end{dtabularx}
    \end{dtable}

    \begin{dtable}
      \lcaption{Example Tracking Difficulty Modifiers}
      \begin{dtabularx}{\columnwidth}{>{\lcol}X l}
        \tb{Condition}                                      & \tb{Difficulty Modifier} \tableheaderrule
        Every three creatures in the group being tracked    & \minus1      \\
        Size of creature or creatures being tracked:\fn{1}  &              \\
        Fine                                                & \plus20      \\
        Diminutive                                          & \plus15      \\
        Tiny                                                & \plus10       \\
        Small                                               & \plus5       \\
        Medium                                              & \plus0       \\
        Large                                               & \minus5      \\
        Huge                                                & \minus10      \\
        Gargantuan                                          & \minus15     \\
        Colossal                                            & \minus20     \\
        Every 24 hours since the trail was made             & \plus1\fn{3} \\
        Every hour of rain since the trail was made         & \plus1       \\
        Fresh snow cover since the trail was made           & \plus10      \\
        \tb{Poor visibility:\fn{2}}                         &              \\
        Overcast or moonless night                          & \plus6       \\
        Moonlight                                           & \plus3       \\
        Fog or precipitation                                & \plus3       \\
        Tracked party hides trail (and moves at half speed) & \plus5
      \end{dtabularx}
      1 For a group of mixed sizes, apply only the modifier for the largest size category. \\
      2 Apply only the largest modifier from this category. \\
      3 With scent-based tracking, apply this modifier per hour since the trail was made. High winds can increase this modifier even more quickly. \\
    \end{dtable}

\newpage
\skill{Swim}{Str}
  The Swim skill represents your ability to swim.
  Moving with the Swim skill requires using the \ability{swim forward} ability (see \pcref{Movement Abilities}).
  A creature that is in water without a \glossterm{swim speed} takes a \minus2 penalty to its \glossterm{accuracy} and Armor and Reflex defenses, even if it makes a successful Swim check.
  For details, see \pcref{Fighting In Water}.

  Creatures that are native to water, such as fish and monsters with a swim speed but no walk speed, gain a \plus10 bonus to Swim checks.

  % \subsection{Common Swim Tasks}

  \subsection{Swim Speed}\label{Swim Speed}
    Some creatures have a listed swim speed.
    % TODO: when? how exactly? what are the consequences of failure?
    A creature with a passive swim speed must still make a Swim check to move in liquid.
    However, it can swim with its \glossterm{movement speed} as part of normal movement abilities rather than relying on the \ability{swim forward} ability.

  \subsection{Swim Difficulty Values}
    \begin{columntable}
      \begin{dtabularx}{\columnwidth}{>{\lcol}X >{\lcol}X}
        \tb{Liquid}                                                       & \tb{Difficulty Value} \tableheaderrule
        Calm water                                                        & 5  \\
        Rough water                                                       & 10 \\
        Viscous liquid, like a muddy swamp                                & 10 \\
        Stormy water                                                      & 15 \\
        Extremely stormy water                                            & 20 \\
      \end{dtabularx}
    \end{columntable}


\chapter{Feats}\label{Feats}

Feats are special abilities that every character has.
Feats can be used to specialize your character particular area, to grant your character new abilities, or to change the way your character does certain things.

\section{Gaining Feats}
    Your character gains a feat every odd level: 1st, 3rd, 5th, and so on.
    Classes can sometimes grant bonus feats as well, which are in addition to these feats which every character gets.

    \subsection{Prerequisites}
        Some feats have prerequisites.
        Your character must have the indicated attribute score, ability, feat, skill, combat prowess, or other quality designated in order to select or use that feat.
        A character can gain a feat at the same level at which he or she gains the prerequisite.

        A character can't use a feat if he or she has lost a prerequisite.

\section{Types Of Feats}
    All feats belong to one of four broad categories.

    \parhead{General} General feats can have a wide variety of effects.
    They often grant new abilities or improve your defenses.

    \parhead{Combat} Combat feats improve your combat capabilities.
    They can increase the damage you deal, grant you new combat abilities, or improve your defenses.

    \parhead{Spell} Spell feats improve your spellcasting abilities.
    All Spell feats except for the Ritual Caster feat are useless to characters who cannot cast spells (see \featpcref{Ritual Caster}).

    \parhead{Skill} Skill feats improve your skills.
    They can make you more likely to succeed with skill checks and grant you new abilities based on your skills.
    % generic skill feat rank requirements:
    % 1st = 4, 5th = 6, 9th = 8, 13th = 10, 17th = 12
    %    because should be able to get them 1 feat behind if trained
    % mastery feats:
    % 1st = 6, 5th = 10, 9th = 14, 13th = 18, 17th = 23
    %    because should need to have mastered the skill

    \subsection{Feat Tags}

        All feats are tagged according to their category.
        In addition, some feats have more specific tags that describe what the feat does.
        The tags are described below.

        \parhead{Bloodline Feats}

        Some characters have traces of monstrous blood running in their veins.
        Most of those will never understand the full potential of their unusual heritage.
        Bloodline feats allow characters to explore those posibilities by gaining abilites related to their ancestry.
        Each bloodline feat belongs to a specific type of monster, such as ``dragon''.

        You can only have one type of bloodline feat.
        Each type of bloodline has a single feat with ``Heritage'' in the name, which all other feats in the bloodline have as a prerequisite.

        \parhead{Magical Feats}
        Magical feats are \glossterm{magical} in nature.
        Many feats are not entirely magical, but have specific effects that are magical.
        These effects are indicated by the \magical tag.

        \parhead{Performance Feats}\label{Performance Feats}
        Performance feats allow a character to use the Perform skill to create magical effects.
        All Performance feats are also Skill feats.

        Every Performance feat grants one or more abilities which can be used by making a successful Perform check.
        Unless otherwise noted, if the Perform check fails, the ability has no effect.

        Each performance has an effect when it is used.
        Most performances continue their effects if you continue performing.

        All performance feats are \glossterm{Auditory}, \glossterm{Speech}, or \glossterm{Visual} effects, depending on the nature of the performance used to activate the feat.

        You can use any combination of performance feats you possess a number of times per day equal to half the number of ranks you have in your highest Perform skill.
        If you fail the Perform check, the use of the ability is wasted.

        \parhead{Style Feats}\label{Style Feats}
        Style feats grant a character the ability to fight or cast spells in a particular style, granting them bonuses while in that style.
        A character can only be in one style at once.
        Once per round, a character can initiate a style, change to a different style, or stop using a style as a \glossterm{free action}.

        Most style feats have requirements.
        If a style requires specific equipment, such as a melee weapon, you must meet the requirements to activate the style.
        If you fail to meet a style's requirements during a round, you leave the style at the end of the round.

\section{Feat Tables}
    \onecolumn

% Feat names must follow ``have'', ``are (a)'', or ``can''.
        \begin{longtabuwrapper}
            \begin{longtabu}{>{\lcol}p{10em} >{\lcol}p{15em} >{\lcol}X >{\lcol}p{8em} >{\lcol}p{3em}}
                \lcaption{Feats}\\
                \tb{General Feats}\label{General Feats} & \tb{Prerequisites} & \tb{Benefit} & \tb{Feat Types} & \tb{Page} \\
                \featref{Celestial Heritage} & Non-evil & Gain holy abilities & Bloodline & \featpref{Celestial Heritage} \\
                \tind \featref{Celestial Apotheosis} & 9th level, Celestial Heritage, non-evil & Gain wings and holy judgment & Bloodline & \featpref{Celestial Apotheosis} \\
                \featref{Celestial Spell Conduit} & 5th level, 2nd level spells, Celestial Heritage, non-evil & Heal allies when you cast spells & Bloodline & \featpref{Celestial Spell Conduit} \\
                \featref{Draconic Heritage} & \tdash & Gain draconic abilities & Bloodline & \featpref{Draconic Heritage} \\
                \tind \featref{Draconic Apotheosis} & 9th level, Draconic Heritage & Gain wings, improved draconic abilities & Bloodline & \featpref{Draconic Apotheosis} \\
                \featref{Endurance} & Con 3 & Resist fatigue and exhaustion & \tdash & \featpref{Endurance} \\
                \featref{Iron Will} & Wil 3 & Improve Mental defenses & \tdash & \featpref{Iron Will} \\
                \featref{Item Conduit} & 5th level & Improve magic item usage & \tdash & \featpref{Item Conduit} \\
                \featref{Lightning Reflexes} & Dex 3 & Improve Reflex defense, initiative & \tdash & \featpref{Lightning Reflexes} \\
                \featref{Rapid Recovery} & Con 3 & Heal damage very quickly & \tdash & \featpref{Rapid Recovery} \\
                \featref{Spellbreaker} & 9th level, Wil 9 & Gain magic resistance & \tdash & \featpref{Spellbreaker} \\
                \featref{Spellgift} & 5 non-spellcaster levels, Wil 6 & Gain spell-like abilities & \tdash & \featpref{Spellgift} \\
                \featref{Swift} & \tdash & Gain speed bonus & \tdash & \featpref{Swift} \\
                \featref{Toughness} & Con 3 & Improve Fortitude defenses & \tdash & \featpref{Toughness} \\

                \tb{Class Feats}\label{Class Feats} & \tb{Prerequisites} & \tb{Benefit} & \tb{Feat Types} & \tb{Page} \\
                \featref{All Energy Becomes One} & Monk 5, Con 6, manifest \ki & Instantly absorb energy attacks & \tdash & \featpref{All Energy Becomes One} \\
                \featref{Arcane Resilience} & Sorcerer 1 & Reduce damage from spells & \tdash & \featpref{Arcane Resilience} \\
                \featref{Chaotic Rage} & Paladin 5 & Gain ability to rage & \tdash & \featpref{Chaotic Rage} \\
                \featref{Combat Leader} & Fighter 9, Int 9 & Grant feats to allies & \tdash & \featpref{Combat Leader} \\
                \featref{Controlling Wild Speech} & Druid or ranger 5, wild speech & Control animals and objects with wild speech & \tdash & \featpref{Controlling Wild Speech} \\
                \featref{Extra Domain} & Cleric 5 & Gain additional domain & \tdash & \featpref{Extra Domain} \\
                \featref{Style Fusion} & Fighter 9, Int 9 & Use multiple style feats at once & \tdash & \featpref{Style Fusion} \\
                \featref{Versatile Wild Speech} & Druid or ranger 1, wild speech & Improve communication with wild speech & \tdash & \featpref{Versatile Wild Speech} \\
                \featref{Wild Control} & Sorcerer 1, wild magic & Improve control over wild magic & \tdash & \featpref{Wild Control} \\

                \tb{Spell Feats}\label{Spell Feats} & \tb{Prerequisites} & \tb{Benefit} & \tb{Feat Types} & \tb{Page} \\
                \featref{Abjurer} & 1st level Abjuration spell & Improve Abjuration spells you cast & \tdash & \featpref{Abjurer} \\
                \featref{Battlecaster} & 1st level spells & Improve spellcasting in combat & \tdash & \featpref{Battlecaster} \\
                \featref{Conjurer} & 1st level Conjuration spell & Improve Conjuration spells you cast & \tdash & \featpref{Conjurer} \\
                \featref{Counterspell} & 5th level, 2nd level spells, Spellcraft 10 & Force foes to miscast spells & \tdash & \featpref{Counterspell} \\
                \featref{Devastating Magic} & 13th level, 6th level spells & Spells deal more damage & \tdash & \featpref{Devastating Magic} \\
                \featref{Distant Magic} & 5th level, 2nd level spells & Cast spells at greater distances & \tdash & \featpref{Distant Magic} \\
                \featref{Diviner} & 1st level Divination spell & Improve Divination spells you cast & \tdash & \featpref{Diviner} \\
                \featref{Energetic Magic} & 9th level, 4th level \glossterm{energy} spell & Improve energy spells you cast & \tdash & \featpref{Energetic Magic} \\
                \featref{Empowered Magic} & 13th level, 6th level spells & Empower effects of your spells & \tdash & \featpref{Empowered Magic} \\
                \featref{Enchant Item} & 1st level spells & Gain ability to create magic items & \tdash & \featpref{Enchant Item} \\
                \featref{Enchanter} & 1st level Enchantment spell & Improve Enchantment spells you cast & \tdash & \featpref{Enchanter} \\
                \featref{Evoker} & 1st level Evocation spell & Improve Evocation spells you cast & \tdash & \featpref{Evoker} \\
                \featref{Hidden Magic} & 5th level, 2nd level spells & Cast spells without components & \tdash & \featpref{Hidden Magic} \\
                \featref{Illusionist} & 1st level Illusion spell & Improve Illusion spells you cast & \tdash & \featpref{Illusionist} \\
                \featref{Miscaster} & 1st level spells & Improve your miscast spells & \tdash & \featpref{Miscaster} \\
                \featref{Quickened Magic} & 9th level, 4th level spells & Cast spells quickly & \tdash & \featpref{Quickened Magic} \\
                \featref{Ritual Caster} & Int 3 & Gain ability to perform rituals & \tdash & \featpref{Ritual Caster} \\
                \featref{Shaped Magic} & 5th level, 2nd level spells & Control area of spells & \tdash & \featpref{Shaped Magic} \\
                \featref{Spellwoven Performance} & 5th level, 1st level spells, any Performance feat & Blend spells and performances & Skill & \featpref{Spellwoven Performance} \\
                \featref{Somatic Strike} & 9th level, 4th level spells & Attack in place of somatic components & \tdash & \featpref{Somatic Strike} \\
                \featref{Spellcasting Versatility} & 5th level, 1st level spells & Improve spellcasting when multiclassing & \tdash & \featpref{Spellcasting Versatility} \\
                \featref{Spellstrike} & 5th level, 2nd level spells & Channel spell through physical attack  & \tdash & \featpref{Spellstrike} \\
                \featref{Transmuter} & 1st level Transmutation spell & Improve Transmutation spells you cast & \tdash & \featpref{Transmuter} \\
                \featref{Vivimancer} & 1st level Vivimancy spell & Improve Vivimancy spells you cast & \tdash & \featpref{Vivimancer} \\

                \tb{Skill Feats}\label{Skill Feats} & \tb{Prerequisites} & \tb{Benefit} & \tb{Feat Types} & \tb{Page} \\
                \featref{Acrobatics Mastery} & Acrobatics 6 & Improve Acrobatics checks & \tdash & \featpref{Acrobatics Mastery} \\
                \featref{Awareness Mastery} & Awareness 6 & Improve Awareness checks & \tdash & \featpref{Awareness Mastery} \\
                \featref{Bluff Mastery} & Bluff 6 & Improve Bluff checks & \tdash & \featpref{Bluff Mastery} \\
                \featref{Climb Mastery} & Climb 6 & Improve Climb checks & \tdash & \featpref{Climb Mastery} \\
                \featref{Craft Magic Item} & Craft (any) 4 & Gain ability to craft magic items & \tdash & \featpref{Craft Magic Item} \\
                \featref{Craft Mastery} & Craft 6 & Improve Craft checks & \tdash & \featpref{Craft Mastery} \\
                \featref{Creature Handling Mastery} & Creature Handling 6 & Improve Creature Handling checks & \tdash & \featpref{Creature Handling Mastery} \\
                \featref{Device Mastery} & Devices 6 & Improve Devices checks & \tdash & \featpref{Device Mastery} \\
                \featref{Disguise Mastery} & Disguise 6 & Improve Disguise checks & \tdash & \featpref{Disguise Mastery} \\
                \featref{Escape Artist Mastery} & Escape Artist 6 & Improve Escape Artist checks & \tdash & \featpref{Escape Artist Mastery} \\
                \featref{Heal Mastery} & Heal 6 & Improve Heal checks & \tdash & \featpref{Heal Mastery} \\
                \featref{Inspiring Performer} & Perform 4 & Perform to improve abilities of allies & Performance & \featpref{Inspiring Performer} \\
                \featref{Intimidate Mastery} & Intimidate 6 & Improve Intimidate checks & \tdash & \featpref{Intimidate Mastery} \\
                \featref{Jump Mastery} & Jump 6 & Improve Jump checks & \tdash & \featpref{Jump Mastery} \\
                \featref{Knowledge Mastery} & Knowledge (any) 6 & Improve Knowledge checks & \tdash & \featpref{Knowledge Mastery} \\
                \featref{Linguistic Mastery} & Linguistics 6 & Improve Linguistics checks & \tdash & \featpref{Linguistic Mastery} \\
                \featref{Mesmerizing Performer} & Perform (any) 4 & Perform to influence and distract foes & Performance & \featpref{Mesmerizing Performer} \\
                \featref{Mocking Performer} & Perform (any) 4 & Perform to impair foes & Performance & \featpref{Mocking Performer} \\
                \featref{Persuasion Mastery} & Persuasion 6 & Improve Persuasion checks & \tdash & \featpref{Persuasion Mastery} \\
                \featref{Perform Mastery} & Perform 6 & Improve Perform checks & \tdash & \featpref{Perform Mastery} \\
                \featref{Ride Mastery} & Ride 6 & Improve Ride checks & \tdash & \featpref{Ride Mastery} \\
                \featref{Sense Motive Mastery} & Sense Motive 6 & Improve Sense Motive checks & \tdash & \featpref{Sense Motive Mastery} \\
                \featref{Skill Savant} & \tdash & Gain skill points & \tdash & \featpref{Skill Savant} \\
                \featref{Sleight of Hand Mastery} & Sleight of Hand 6 & Improve Sleight of Hand checks & \tdash & \featpref{Sleight of Hand Mastery} \\
                \featref{Spellcraft Mastery} & Spellcraft 6 & Improve Spellcraft checks & \tdash & \featpref{Spellcraft Mastery} \\
                \featref{Sprint Mastery} & Sprint 6 & Improve Sprint checks & \tdash & \featpref{Sprint Mastery} \\
                \featref{Stealth Mastery} & Stealth 6 & Improve Stealth checks & \tdash & \featpref{Stealth Mastery} \\
                \featref{Sprint Mastery} & Sprint 6 & Improve Sprint checks & \tdash & \featpref{Sprint Mastery} \\
                \featref{Supreme Inspiration} & 9th level, Perform 9 & Perform to greatly inspire allies & \tdash & \featpref{Supreme Inspiration} \\
                \featref{Survival Mastery} & Survival 6 & Improve Survival checks & \tdash & \featpref{Survival Mastery} \\
                \featref{Swim Mastery} & Swim 6 & Improve Swim checks & \tdash & \featpref{Swim Mastery} \\
                \featref{Trapfinder} & Awareness 4 & Improve ability to notice traps & \tdash & \featpref{Trapfinder} \\

                \tb{Combat Feats}\label{Combat Feats} & \tb{Prerequisites} & \tb{Benefit} & \tb{Feat Types} & \tb{Page} \\
                \featref{Battlerage} & 5th level, Wil 6 & Gain ability to rage & \tdash & \featpref{Battlerage} \\
                \featref{Blindfighter} & Per 3 & Fight better while unable to see & \tdash & \featpref{Blindfighter} \\
                \featref{Close-Quarters Fighting} & Dex 3 & Fight better while squeezing and grappling & \tdash & \featpref{Close-Quarters Fighting} \\
                \featref{Combat Mobility} & Dex 3 & Move through creatures, even while attacking & \tdash & \featpref{Combat Mobility} \\
                \featref{Counterattack} & 5th level, Dex 6 & Make free attacks when missed & Style & \featpref{Counterattack} \\
                \featref{Covering Fire} & Per 3 & Impair foes with ranged attacks & Style & \featpref{Covering Fire} \\
                \featref{Deadly Aim} & 5th level, Per 6 & Bonus damage with ranged attacks & Style & \featpref{Deadly Aim} \\
                \featref{Defensive Fighting} & \tdash & Trade damage for defense bonus & Style & \featpref{Defensive Fighting} \\
                \featref{Deflect Arrows} & Dex 3 & Instantly deflect ranged attacks & \tdash & \featpref{Deflect Arrows} \\
                \featref{Destructive} & Str 3 & Ignore damage reduction and hardness & \tdash & \featpref{Destructive} \\
                \featref{Executioner} & 5th level & Make free attacks against weakened foes & \tdash & \featpref{Executioner} \\
                \featref{Eye of the Storm} & Dex 3 & Reduce overwhelm penalties & Style & \featpref{Eye of the Storm} \\
                \featref{Far Shot} & Str 3 & Fight better at long range & \tdash & \featpref{Far Shot} \\
                \featref{Fearsome} & 5th level, Intimidate 10 & Intimidate struck foes & Skill & \featpref{Fearsome} \\
                \featref{Guardian} & \tdash & Reduce overwhelm penalties of allies & \tdash & \featpref{Guardian} \\
                \featref{Heavy Weapon Fighting} & Str 3 & Bonus damage with two-handed attacks & \tdash & \featpref{Heavy Weapon Fighting} \\
                \featref{Improvised Fighting} & \tdash & Fight better with improvised weapons & \tdash & \featpref{Improvised Fighting} \\
                \featref{Inescapable} & 9th level & Immobilize struck foes & Style & \featpref{Inescapable} \\
                \featref{Infuriating} & 5th level & Goad struck foes into attacking you & Style & \featpref{Infuriating} \\
                \featref{Legendary Speed} & 13th level, Dex 12 & Make free attacks & Style & \featpref{Legendary Speed} \\
                \featref{Mage Slayer} & 5th level & Fight better against spellcasters & Style & \featpref{Mage Slayer} \\
                \featref{Maneuver Focus} & \tdash & Improve chosen combat maneuver & \tdash & \featpref{Maneuver Focus} \\
                \featref{Martial Training} & \tdash & Gain armor, weapon proficiencies & \tdash & \featpref{Martial Training} \\
                \featref{Mounted Combat} & Ride 4 & Fight better while mounted & \tdash & \featpref{Mounted Combat} \\
                \featref{Overwhelming Fire} & \tdash & Overwhelm foes with ranged weapons & Style & \featpref{Overwhelming Fire} \\
                \featref{Parry} & Dex 3 & Deflect strikes with martial skill & Style & \featpref{Parry} \\
                \featref{Point Blank Shot} & \tdash & Fight better with ranged weapons at close range & \tdash & \featpref{Point Blank Shot} \\
                \featref{Power Attack} & 5th level, Str 6 & Gain bonus to damage & Style & \featpref{Power Attack} \\
                \featref{Precise Attack} & Per 3 & Trade damage for accuracy bonus & Style & \featpref{Precise Attack} \\
                \featref{Precise Shot} & 5th level, Per 6 & Ignore cover, concealment, miss chances & \tdash & \featpref{Precise Shot} \\
                \featref{Precision Strikes} & 5th level, Per 6 & Gain precision-based special attacks & \tdash & \featpref{Precision Strikes} \\
                \featref{Quick Draw} & \tdash & Draw and stow items more quickly & \tdash & \featpref{Quick Draw} \\
                \featref{Reflexive Dodge} & 13th level, Dex 12 & Automatically dodge strikes & \tdash & \featpref{Reflexive Dodge} \\
                \featref{Shielded Fighting} & Shield proficiency & Gain defense bonuses with shields & \tdash & \featpref{Shielded Fighting} \\
                \featref{Tactical Prediction} & 5th level, Int 6 & Predict foe's next action & \tdash & \featpref{Tactical Prediction} \\
                \featref{Two-Weapon Fighting} & Dex 3 & Bonus accuracy, damage while dual wielding & \tdash & \featpref{Two-Weapon Fighting} \\
                \featref{Unarmed Fighting} & \tdash & Fight better with unarmed attacks & \tdash & \featpref{Unarmed Fighting} \\
                \featref{Weapon Focus} & \tdash & Fight better with chosen weapons & \tdash & \featpref{Weapon Focus} \\
            \end{longtabu}
            1. You can gain this feat multiple times. Each time you do, it has a different effect. \\
            2. You can gain this feat multiple times. Its effects stack. \\
        \end{longtabuwrapper}
        \twocolumn

\section{Feat Descriptions}
    Here is the format for feat descriptions.

    \ssecfake{Feat Name [Type of Feat]}
    \featpre A minimum attribute score, another feat or feats, a minimum combat prowess, a minimum number of ranks in one or more skills, or a class level that a character must have in order to acquire this feat.
    This entry is absent if a feat has no prerequisite.
    A feat may have more than one prerequisite.
    \featben What the feat enables the character (``you'' in the feat description) to do.
    If a character has the same feat more than once, its benefits do not stack unless indicated otherwise in the description.
    \par In general, characters cannot gain the same feat twice.
    \parhead{Normal}
    What a character who does not have this feat is limited to or restricted from doing.
    If not having the feat causes no particular drawback, this entry is absent.
    \parhead{Special}
    Additional facts about the feat that may be helpful when you decide whether to acquire the feat.

    \feat{Abjurer}{Magical, Spell}
    \spelldesc{You have great talent with Abjuration spells.}
    \featpre 1st level or higher \glossterm{Abjuration} spell known.
    \featben Your damaging spells grant you sympathetic resistance to attacks.
    Whenever you cast a spell that deals non-physical damage, you gain damage reduction equal to your spellpower against damage of that type for 2 rounds.

    At 7th level, if you know a 2nd level or higher Abjuration spell, your Abjuration spells are more powerful.
    Whenever you cast an Abjuration spell, you can reduce the cost to apply one augment to the spell by one spell level.

    At 11th level, you gain a \plus4 bonus to spellpower when dispelling effects with \spell{dispel magic} and similar spells.

    At 15th level, if you know a 6th level or higher Abjuration spell, you gain a \plus1 bonus to all defenses.

    \feat{Acrobatics Mastery}{Skill}
    \featpre Acrobatics 6 ranks.
    \featben Whenever you make a Acrobatics check, you may roll twice and take the higher result.

    At 3rd level, using Acrobatics to move along narrow surfaces does not reduce your speed.

    At 7th level, if you have 10 ranks in Acrobatics, you gain a \plus5 bonus to Acrobatics checks.

    At 11th level, if you have 14 ranks in Acrobatics, you can balance on surfaces that cannot support your weight.
    The DR is 30 for liquids such as water, 40 for dense gases and raw energy, and 50 for ordinary air.
    While balancing in this way, you must take a move action each round to continue moving; you cannot remain in the same place in consecutive rounds, or you will fall.
    The DR increases by 2 for each consecutive round that you spend balancing in this way.
    You gain a \plus4 bonus on this check per size category you are smaller than Medium, or a \minus4 penalty per size category larger than Medium.
    \magical

    \feat{All Energy Becomes One}{Class, Magical}
    \featpres Monk level 5, Constitution 6, manifest \ki ability.
    \featben Whenever you take \glossterm{energy damage}, you can spend a \glossterm{immediate action} to channel the energy into your body.
    You gain damage reduction against that attack equal to your \ki power.
    If the damage reduced in this way exceeds your level, you recover one spent \ki point.

    At monk level 7, you can use this ability to reduce any non-physical damage you take.

    At monk level 11, if you have 9 Constitution, you can use this ability once per round without spending an action.
    You cannot use it twice to affect the same attack.

    At monk level 15, if you have 12 Constitution, the damage reduction becomes equal to twice your \ki power.

    \feat{Arcane Resilience}{Class, Magical}
    \featpre Sorcerer level 1.
    \featben You gain damage reduction against arcane spells equal to your sorcerer level or Constitution, whichever is higher.
    This damage reduction applies against your arcane spells, including miscast effects.

    At sorcerer level 3, this damage reduction applies against all spells.

    At sorcerer level 7, you take half damage from your own spell and miscast effects (before applying damage reduction).

    At sorcerer level 11, this damage reduction doubles.

    \feat{Awareness Mastery}{Skill}
    \featpre Awareness 6 ranks.
    \featben Whenever you make an Awareness check, you may roll twice and take the higher result.

    At 3rd level, you gain one of the following senses: \glossterm{blindsense} (50 ft.), \glossterm{darkvision} (100 ft.), \glossterm{scent}, or \glossterm{tremorsense} (50 ft.).

    At 7th level, if you have 10 ranks in Awareness, you gain a \plus5 bonus to Awareness checks.

    At 11th level, you gain one of the following senses: \glossterm{blindsense} (200 ft.), \glossterm{blindsight} (50 ft.), \glossterm{darkvision} (500 ft.), \glossterm{tremorsense} (200 ft.), or \glossterm{tremorsight} (50 ft.).

    \feat{Battlecaster}{Spell}
    \featpre 1st level spells.
    \featben You gain a \plus3 bonus to Concentration checks made to cast spells.

    At 3rd level, you reduce your chance of arcane spell failure from wearing armor by 10\%.

    At 7th level, if you can cast 2nd level spells, the Concentration check bonus increases to \plus10.

    At 11th level, if you can cast 4th level spells, the Concentration check bonus increases to \plus20.

    At 15th level, if you can cast 6th level spells, you reduce your chance of arcane spell failure from wearing armor by a total of 20\%.

    \feat{Battlerage}{General}
    \featpres 5th level, Willpower 6.
    \featben You gain the rage ability, allowing you to fly into a rage as a free action.
    While raging, you have the following benefits and drawbacks:
    \begin{itemize}
        \item \plus2 bonus to damage with physical attacks.
        \item \plus2 bonus to Fortitude and Mental defense.
        \item 2 temporary hit points per Willpower.
            These extra hit points gained from raging are lost before any other hit points (see \pcref{Temporary Hit Points}).
        \item \minus2 to \glossterm{physical defenses}.
        \item Unable to take any action that requires patience or concentration, such as casting spells.
        \item If you does not spend a swift round to sustain the rage, it ends at the end of the round.
        \item At the end of each round, if you did not attack a creature or object, you take nonlethal damage equal to your level.
    \end{itemize}

    A rage lasts for up to 5 rounds.
    At the end of the rage, you take nonlethal damage equal to his level.
    If you have any temporary hit points remaining at the end of your rage, the nonlethal damage is dealt to those hit points before they go away.
    In addition, you become \fatigued and unable to rage until you rest for 5 minutes.

    The bonuses to physical damage, Fortitude, and Mental defense granted by your rage increase with level.
    This is called your \glossterm{rage bonus}.
    You may rage a number of times per day equal to your rage bonus.

    At 7th level, your rage bonus increases to \plus3.
    In addition, the number of hit points gained per Willpower increases to 3.

    At 11th level, if you have 9 Willpower, your rage bonus increases to \plus4.
    In addition, the number of hit points gained per Willpower increases to 4.

    At 15th level, if you have 12 Willpower, your rage bonus increases to \plus5.
    In addition, the number of hit points gained per Willpower increases to 5.

    \feat{Blindfighter}{Combat}
    \featpre Perception 3.
    \featben Whenever you miss a melee attack because of a miss chance caused by being unable to see your opponent, you can reroll your miss chance one time to see if you actually hit.

    At 3rd level, you are not \defenseless against foes you cannot see if you know their location.

    At 7th level, if you have 6 Perception, you gain \glossterm{blindsense} (50 ft.\ radius).

    At 11th level, if you have 9 Perception, you gain \glossterm{blindsight} (20 ft.\ radius).

    At 15th level, if you have 12 Perception, the radius of your blindsight improves to 50 feet.
    \parhead{Normal} You have a 50\% chance to miss opponents you can't see, and you are \defenseless against them.

    \feat{Bluff Mastery}{Skill}
    \featpre Bluff 6 ranks.
    \featben Whenever you make a Bluff check, you may roll twice and take the higher result.

    At 3rd level, \tdash.

    At 7th level, if you have 10 ranks in Bluff, you gain a \plus5 bonus to Bluff checks.

    At 11th level, if you have 14 ranks in Bluff, your lies can change how creatures perceive reality.
    As an immediate action, when you are telling a lie, you can make a Bluff vs. Mental attack against a creature within \rngmed range of you.
    Success means the target's sight, smell, hearing, and sense of temperature are altered so it perceives the world as you described it.
    This cannot remove things that do exist, but it can create new sensations where none existed.
    You can use this ability three times per day.
    \magical

    \feat{Celestial Apotheosis}{Bloodline, Magical}
    \featpres 9th level, non-evil alignment, Celestial Heritage.
    \featben Your celestial heritage has developed into its full potential.
    You gain feathery wings that sprout from your back.
    You can use these wings to glide at a rate equal to your land speed (see \pcref{Gliding}).
    The wings themselves are physical, but the ability to glide and fly with them is \glossterm{magical}.
    In addition, you gain the holy judgment ability.
    \parhead{Holy Judgment} As a standard action, you can inflict divine judgment on a non-good creature within \rnglong range. You make a Celestial power vs. Mental attack against the target. Success deals 1d6 divine damage per celestial power, and causes the target to be \dazed for two rounds. Critical success deals double damage. Failure deals half damage, and has no additional effects.

    At 11th level, your wings improve, granting you a fly speed equal to your land speed with average maneuverability.
    See \pcref{Flying}, for details.
    You can only fly for a number of rounds equal to half your celestial power.
    After that limit is reached, you must rest for 5 minutes before flying again.

    At 15th level, your holy judgment ability dazes the target even if the attack fails.

    At 19th level, you can fly for a number of minutes equal to half your celestial power before resting.

    \feat{Celestial Heritage}{Bloodline, Magical}
    \featpres Non-evil alignment.
    \featben You have the blood of a celestial creature in your veins, granting you celestial power.
    You gain two abilities you can use by spending a celestial point.
    Your celestial power is equal to your Willpower or your level, whichever is higher.
    You have a maximum number of celestial points equal to half your celestial power (minimum 1).
    Each day, you recover all spent celestial points.
    \parhead{Holy Blessing} As a standard action, you can bless a willing ally within \rngmed range. The target gains a general \glossterm{legend point} that lasts for 5 rounds, or until it is used.
    \parhead{Holy Protection} As a standard action, you can shield a willing ally within \rngclose range from evil. The target gains damage reduction equal to your celestial power against evil effects and physical attacks made by evil creatures.

    At 11th level, your \textit{holy blessing} ability affects up to five willing allies.

    At 15th level, you permanently gain the benefit of your \textit{holy protection} ability.

    \feat{Celestial Spell Conduit}{Bloodline, Magical}
    \featpres 5th level, 2nd level spells, non-evil alignment, Celestial Heritage.
    \featben Whenever you cast a spell, you can heal an ally within \rngclose range of you for 1d6 hit points per two celestial power.
    At 7th level, the range of this healing improves to \rngmed range.

    At 11th level, you gain \glossterm{magic resistance} against evil spells and spells cast by evil creatures.
    Your magic resistance is equal to 10 \add your celestial power.

    At 15th level, your healing with this ability increases to 1d8 hit points per two celestial power.

    \feat{Chaotic Rage}{Class}
    \featpres Paladin level 5, Chaos devoted alignment.
    \featben You gain the rage ability, as the Battlerage feat (see Battlerage, page \featpref{Battlerage}).
    However, your rage bonus increases based on your paladin level, rather than your level and Willpower.

    At paladin level 7, your rage bonus increases to \plus3.
    In addition, the number of hit points gained per Willpower increases to 3.

    At paladin level 11, your rage bonus increases to \plus4.
    In addition, the number of hit points gained per Willpower increases to 4.

    At paladin level 15, your rage bonus increases to \plus5.
    In addition, the number of hit points gained per Willpower increases to 5.

    \feat{Climb Mastery}{Skill}
    \featpre Climb 6 ranks.
    \featben Whenever you make a Climb check, you may roll twice and take the higher result.

    At 3rd level, you can attempt to climb other creatures more easily.
    As a standard action, you can make a Climb vs. Reflex attack against a creature adjacent to you.
    The creature must be three or more size categories larger than you.
    Success means you can climb the creature as if it were a solid object with a Climb DR equal to its Reflex defense.
    The creature takes a \minus4 penalty to accuracy on physical attacks against you, but is not otherwise encumbered by your presence unless it cannot support your weight.
    It can attempt to remove you by attacking you, or with an appropriate \glossterm{combat maneuver}, such as grappling or shoving.

    At 7th level, if you have 6 ranks in Climb, you gain a \glossterm{climb speed} equal to your land speed.
    This grants several benefits.
    \begin{itemize}
        \item A successful Climb check to move allows you to move a distance equal to your climb speed.
        \item You gain a \plus10 bonus to Climb checks.
    \end{itemize}

    At 11th level, if you have 14 ranks in Climb, you can climb surfaces that are perfectly smooth.
    The DR is 30 for perfectly smooth vertical surfaces, and 40 to climb on a perfectly smooth ceiling.
    You can also wallrun on ceilings.

    \feat{Close-Quarters Fighting}{Combat}
    \featpre Dexterity 3.
    \featben You reduce your penalties for \glossterm{squeezing} by 2. In addition, you reduce your penalty for attacking with non-light weapons in a grapple by 2.

    At 3rd level, your movement speed is not reduced while squeezing.

    At 7th level, if you have 6 Dexterity, you suffer no penalties for squeezing or attacking with non-light weapons in a grapple.

    At 11th level, if you have 9 Dexterity, you can choose to occupy space as if you were a non-squeezing creature one size category smaller than your actual size while squeezing.
    This does not affect the minimum physical size you can squeeze through when using the Escape Artist skill (see \pcref{Escape Artist}).

    \feat{Combat Leader}{Class, Combat}
    \featpres Fighter level 9, Intelligence 9.
    \featben As a swift action, you may grant the use of one of your combat feats to a willing creature within \rngmed range of you who can see and hear you.
    The target must meet level prerequisites for the granted feat, including class level prerequisites, but it can ignore all other prerequisites.
    The effect lasts as long as you spend a swift action to maintain it, to a maximum of 5 rounds.
    After using this ability, you cannot use it again for 5 minutes.

    At fighter level 11, you can grant that feat to two willing creatures of your choice.

    At fighter level 15, if you have 12 Intelligence, you can grant that feat to up to five willing creatures of your choice.

    \feat{Combat Mobility}{Combat, Mobility}
    \featpres Dexterity 3.
    \featben At the start of each phase, if you are \glossterm{unemcumbered}, you may choose a creature you can see.
    You can move through that creature's space this phase, treating it as difficult terrain.

    At 7th level, if you have 6 Dexterity, you gain the ability to move and attack simultaneously while unencumbered.
    As a standard action, you can move up to your speed while making a \glossterm{standard attack}.
    If you move more than half your speed without making an additional strike, you lose all remaining strikes you would make.
    As with other movement during the action phase, this does not affect which location you are in when other creatures declare the targets for their actions, allowing them to hit you even if you move away.

    At 11th level, if you have 9 Dexterity, you can choose up to two creatures to move through at the start of each phase.

    At 15th level, if you have 12 Dexterity, you do not treat space occupied by creatures as difficult terrain when moving through them with this feat.

    \feat{Conjurer}{Magical, Spell}
    \featpres 1st level or higher \glossterm{Conjuration} spell known.
    \featben Objects you create and creatures you summon with Conjuration spells have extra hit points equal to your spellpower.
    Permanent physical objects you create are not affected by this ability.

    At 7th level, if you know a 2nd level or higher Conjuration spell, your Conjuration spells are more powerful.
    Whenever you cast a Conjuration spell, you can reduce the cost to apply one augment to the spell by one spell level.

    At 11th level, if you know a 4th level or higher Conjuration spell, your connection to the Astral Plane is strengthened.
    Whenever you teleport, you drift between your plane and the Astral Plane for 1 round.
    During this time, all attacks against you 20\% failure chance.

    At 15th level, you drift between planes after teleporting for 2 rounds.
    In addition, objects you create and creatures you summon have damage reduction against all damage equal to twice your spellpower.

    \feat{Controlling Wild Speech}{Class, Magical}
    \featpres Druid or ranger level 5, wild speech ability.
    \featben  As a standard action, you can make a Nature power vs. Mental attack against one animal or object you are communicating with using your wild speech ability.
    This consumes a use of your wild speech ability.
    Success means the target is \charmed by you for the duration of the conversation, and for 5 rounds thereafter.
    Failure has no effect.
    This is a \glossterm{Subtle} effect, which means the target is unlikely to notice the attempt at mental influence.
    It is also not a \glossterm{Mind} effect, and can affect objects and elemental forces that the druid can use her wild speech to converse with.
    The attack automatically critically succeeds against non-intelligent objects.

    At druid or ranger level 11, critical success means the target is charmed permanently.
    This effect can only be broken by another druid or ranger using this ability.
    If they charm the target in this way, the target stops being charmed by you.

    At druid or ranger level 15, you can choose for the target to be \dominated by you instead of charmed.
    This is not a Subtle effect, and critical success has no effect if you choose this option.

    At druid or ranger level 19, if you choose for the target to be dominated and get a critical success, it is dominated by you permanently.
    As with charming, if another druid or ranger dominates the target with this ability, it stops being dominated by you.

    \feat{Counterattack}{Combat, Style}
    \featpres 5th level, Dexterity 6.
    \featben Whenever a creature misses you with a physical melee attack, you can make a \glossterm{strike} against that creature as an \glossterm{immediate action}.

    At 7th level, you gain a \plus2 bonus to damage with strikes made using this feat.

    At 11th level, if you have Dexterity 9, you can make one strike per round with this feat without spending an action.

    At 15th level, if you have Dexterity 12, you can also use this ability if a creature hits you with a physical melee attack.

    \stylereq Wield a melee weapon.
    \parhead{Special} This is a Style feat, and all of its effects only apply while you are in this style.
    For details, see \pcref{Style Feats}.

    \feat{Counterspell}{Magical, Spell}
    \featpres 5th level, 2nd level spells, Spellcraft 6 ranks.
    \featben As a standard action, you can prepare to counterspell a creature within \rngmed range of you.
    If that creature casts a spell during the same phase, you can make a Spellcraft check to identify the spell as normal (see \pcref{Spellcraft}).
    After attempting to identify the spell, you may cast any spell you know as a counterspell.
    You do not have to identify the spell successfully to counterspell it.
    The spell you cast does not have its normal effect.

    If you cast the same spell as your target, regardless of any augments applied, the target miscasts its spell.
    If you cast a different spell, but one from the same spell school or with all \glossterm{ability tags} that the target's spell has, and your spell is of the same spell level or higher, you and your target make opposed spellpower checks.
    If you win, the target miscasts its spell.
    If you fail the spellpower check, if your spell is of a different spell school, or if your spell is lower level, the target's spell takes effect normally.

    At 7th level, you can choose to completely negate the effects of spells you counter rather than causing them to be miscast.

    At 11th level, if you can cast 4th level spells and have 8 ranks in Spellcraft, your spell does not need to have the same school or ability tags as the target's spell to force a spellpower check.

    At 15th level, if you can cast 6th level spells and have 10 ranks in Spellcraft, you may counter up to five creatures within \rngmed range of you at once.
    You may still cast only one spell, but it is used to counter the spells cast by all of your targets.
    You make a single roll to determine your spellpower check result, and each target makes their own independent roll to oppose yours.

    \parhead{Special}
    Spell augments affect the spell level of the spell being cast as normal, making augmented spells slightly more difficult to counterspell.
    The \spell{dispel magic} spell can be used to counter any spell, even higher level spells.

    \feat{Covering Fire}{Combat, Style}
    \featpres Perception 3.
    \featben While in this style, if you hit a creature with a physical ranged attack, it is \impaired with physical attacks for 2 rounds.

    At 7th level, if you have 6 Perception, the target is \impaired with all attacks and checks for 2 rounds, not just physical attacks.

    At 11th level, if you have 9 Perception, all creatures you attack are impaired in this way, not just creatures you hit.
    \stylereq Make a physical ranged attack each round.
    \parhead{Special} This is a Style feat, and all of its effects only apply while you are in this style.

    \feat{Craft Magic Item}{Magical, Skill}
    \featpre Craft (any) 4 ranks.
    \featben You can imbue items with magic using your crafting skill.
    Imbuing an item with magic takes material components, as described in \pcref{Magic Item Creation}.
    It takes you one hour per 2 gp of material components to create a item.

    When you gain this feat, you choose three ability tags (see \pcref{Ability Tags}).
    You know how to craft items with those ability tags.

    You can also mend a broken magic item if it is one that you could make.
    Doing so costs a tenth of the raw materials and a quarter of the time it would take to craft that item in the first place.
    You cannot mend a destroyed magic item.

    At 3rd level, it takes you one hour per 10 gp of material components to create a magic item. % 125-600 gp materials, so 12.5-60 hours to craft

    At 7th level, if you have 6 ranks in any Craft skill, it takes you one hour per 100 gp of material components to create a magic item. % 900-3250 gp mats, 9-32 hrs

    At 11th level, if you have 8 ranks in any Craft skill, it takes you one hour per 500 gp of material components to create a magic item. % 5000-18500 gp, 10-37 hrs

    At 15th level, if you have 10 ranks in any Craft skill, it takes you one hour per 2,500 gp of material components to create a magic item. % 27500-200000 gp, 11-80 hrs

    \feat{Craft Mastery}{Skill}
    \featpre Craft (any) 6 ranks.
    \featben Whenever you make a Craft check, you may roll twice and take the higher result.

    At 3rd level, as long as you have mastered one Craft skill, you treat all Craft skills you were trained in as if you had mastered them.
    This grants you additional skill ranks and increases your check modifier as normal for mastering a skill.
    However, you cannot use these additional ranks to qualify for any feats or abilities.

    At 7th level, if you have 10 ranks in Craft, you gain a \plus5 bonus to Craft checks.

    At 11th level, if you have 14 ranks in Craft, \tdash.

    \feat{Creature Handling Mastery}{Skill}
    \featpre Creature Handling 6 ranks.
    \featben Whenever you make a Creature Handling check, you may roll twice and take the higher result.

    At 3rd level, you no longer take a penalty when using Creature Handling on non-animals.

    At 7th level, if you have 10 ranks in Creature Handling, you gain a \plus5 bonus to Creature Handling checks.

    At 11th level, if you have 14 ranks in Creature Handling, you can pacify and push creatures as a swift action.
    In addition, when training a creature, if your check result is 30 or higher, you can train it in only a hour.
    The creature remembers the trick just as if you had spent the full amount of time training it.

    \feat{Deadly Aim}{Combat, Style}
    \featpres 5th level, Perception 6.
    \featben You gain a \plus2 bonus to damage with physical ranged attacks.

    At 7th level, this bonus increases to \plus3.

    At 11th level, if you have 9 Perception, this bonus increases to \plus4.

    At 15th level, if you have 12 Perception, this bonus increases to \plus5.

    At 19th level, if you have 15 Perception, this bonus increases to \plus6.
    \stylereq Wield a ranged weapon.
    \parhead{Special} This is a Style feat, and all of its effects only apply while you are in this style.
    For details, see \pcref{Style Feats}.

    \feat{Defensive Fighting}{Combat, Style}
    \featben While in this style, you gain a \plus2 bonus to your \glossterm{physical defenses}.
    In exchange, you take a \minus2 penalty to damage with physical attacks (to a minimum of 1).

    At 3rd level, the defense bonus increases to \plus1.

    At 7th level, the damage penalty is reduced to \minus1.

    At 11th level, the damage penalty is removed.
    \stylereq Wield a melee weapon. You must make a physical melee attack or take the total defense action each round.
    \parhead{Special} This is a Style feat, and all of its effects only apply while you are in this style.
    For details, see \pcref{Style Feats}.

    \feat{Deflect Arrows}{Combat}
    \featpres Dexterity 3.
    \featben Whenever you are hit by a ranged \glossterm{strike}, you may deflect it as a \glossterm{immediate action}.
    If you do, the attack automatically misses.
    You must have a free hand and be aware of the attack.
    Objects of your size or larger cannot be deflected in this way.

    At 3rd level, you can choose to catch the object in your hand instead of deflecting it.
    If you do, you can throw the object back as part of the same action.
    You make a thrown weapon strike, using any normal modifiers that would apply to a thrown attack with the object.

    At 7th level, if you have 6 Dexterity, you can automatically catch or deflect one strike per round without spending an action.
    You can still spend an immediate action to catch or deflect an additional object, if you still have a free hand.
    You cannot throw an object back as part of this free deflection.

    At 11th level, if you have 9 Dexterity, you can automatically catch or deflect two objects per round without spending an action.
    In addition, if you have 9 Strength, you can catch or deflect objects of your size category with this feat.

    \feat{Destructive}{Combat}
    \featpres Strength 3.
    \featben Your physical attacks ignore an amount of hardness and damage reduction equal to half your Strength.

    At 3rd level, you gain a \plus1 bonus to damage with physical attacks.

    At 7th level, if you have 6 Strength, the hardness and damage reduction ignored by your physical attacks increases to be equal to your Strength.

    At 11th level, if you have 9 Strength, your bonus to damage increases to \plus2.

    At 15th level, if you have 12 Strength, you ignore all hardness and damage reduction with physical attacks.

    \feat{Devastating Magic}{Magical, Spell}
    \featpre 13th level, 6th level spells.
    \featben You gain a bonus to damage and healing with all spells equal to the number of dice you would roll for that spell's damage or healing.
    If no dice would be rolled for a spell, you instead gain a \plus2 bonus to damage or healing.
    This does not affect your accuracy or any other aspects of the spell.

    At 19th level, if you have 8th level spells, this bonus increases to be equal to twice the number of dice you would roll.
    If no dice would be rolled for a spell, the bonus instead increases to \plus4.

    \feat{Device Mastery}{Skill}
    \featpre Devices 6 ranks.
    \featben Whenever you make a Devices check, you may roll twice and take the higher result.

    At 3rd level, \tdash.

    At 7th level, if you have 10 ranks in Devices, you gain a \plus5 bonus to Devices checks.

    At 11th level, if you have 14 ranks in Devices, you can disable spell effects on objects or areas as if they were merely complex devices.
    You can make a Devices check against an active spell effect within your reach.
    The DR to dispel the effect is equal to 20 \add the spell's spellpower, and the result must be at least 30.
    Success means the spell is dispelled.
    \magical

    You must be aware of a spell to disable it, either through the Spellcraft skill or because the effect is noticeable.
    You cannot disable spell effects on creatures.

    \feat{Disguise Mastery}{Skill}
    \featpre Disguise 6 ranks.
    \featben Whenever you make a Disguise check, you may roll twice and take the higher result.

    At 3rd level, your disguises can change how Divination spells perceive a creature.
    When you make a Disguise check, you can decide how that creature and any items on the creature appear when examined by divination spells.
    For example, you could cause all of its equipment to appear nonmagical, or you could cause it to have a strong aura of good when examined with \spell{detect good}.
    You cannot create an aura of overwhelming strength with this skill.
    Anyone using divination magic on the creature must make a spellpower check with a DR equal to your Disguise check result in order to perceive the truth.
    Regardless of the result of the check, the caster is not aware that the check was made.
    \magical

    At 7th level, if you have 10 ranks in Disguise, you gain a \plus5 bonus to Disguise checks.

    At 11th level, you can disguise yourself in seemingly impossible ways.
    You can disguise yourself as larger or smaller size categories.
    The DR of the Disguise check increases by 20 per size category different from your own.
    This does not actually change the space you occupy, your \glossterm{reach}, or the size of space you can squeeze through, but creatures fooled by the disguise treat you as if your space and reach were appropriate to your disguised size.
    \magical

    \feat{Distant Magic}{Magical, Spell}
    \featpre 5th level, 2nd level spells.
    \featben If you cast a ranged spell using a spell slot one level higher than normal, its range is doubled.
    This can only be applied to spells with a range of \rngclose, \rngmed, \rnglong, or \rngext.

    At 7th level, you can cast a spell with a 5 foot range, or a range of Touch, using a spell slot two levels higher than normal.
    If you do, the spell's range becomes 30 feet.
    If the spell had a range of Touch, you must succeed at a Spellpower vs. Reflex attack against the spell's targets.
    This attack is used in place of a touch attack to determine the spell's effects.

    At 11th level, you double the range of all ranged spells you cast.
    If you use a spell slot one level higher than normal, the range is tripled, not quadrupled.

    \feat{Diviner}{Magical, Spell}
    \spelldesc{You have great talent with Divination spells.}
    \featpre 1st level or higher \glossterm{Divination} spell known.
    \featben You gain a \plus2 bonus to Awareness, Knowledge, and Sense Motive.

    At 7th level, if you know a 2nd level or higher Divination spell, your Divination spells are more powerful.
    Whenever you cast a Divination spell, you can reduce the cost to apply one augment to the spell by one spell level.

    At 11th level, if you know a 4th level or higher Divination spell, you have a precognitive sense that warns you of attacks.
    You are aware of all attacks against you, even those you cannot see, as long as you are conscious.
    This allows you to use some certain abilities to defend yourself, and prevents you from being \unaware.

    At 15th level, if you know a 6th level or higher Divination spell, the bonus to skills increases to \plus4.

    \feat{Draconic Apotheosis}{Bloodline}
    \featpres 9th level, Draconic Heritage.
    \featben Your draconic heritage has developed into its full potential.
    You gain scaly wings that sprout from your back.
    You can use these wings to glide at a rate equal to your land speed (see \pcref{Gliding}).
    The wings themselves are physical, but the ability to glide and fly with them is \glossterm{magical}.
    In addition, your breath weapon no longer has a delay before it can be used again.

    At 11th level, the size of your breath weapon increases.
    A line breath weapon becomes a \areahuge, 10 ft.\ wide line, and a cone breath weapon becomes a \arealarge cone.
    In addition, your wings improve, granting you a fly speed equal to your land speed with average maneuverability.
    See \pcref{Flying}, for details.
    You can only fly for a number of rounds equal to half your draconic power.
    After that limit is reached, you must rest for 5 minutes before flying again.

    At 15th level, your natural weapons gain an automatic enhancement bonus equal to one quarter of your draconic power. \magical

    At 19th level, you can fly for a number of minutes equal to half your celestial power before resting. \magical

    \feat{Draconic Heritage}{Bloodline}
    \featben You have the blood of a dragon in your veins.
    When you take this feat, choose a type of dragon.
    You gain damage reduction equal to twice your draconic power against the damage type that dragon's breath weapon deals.
    Your draconic power is equal to your level, your spellpower with arcane spells, or your Constitution, whichever is higher.
    A list of dragons and their associated damage type is given in \trefnp{Dragon Types}.

    In addition, your eyes begin to match the keen senses of dragons, granting you \glossterm{low-light vision}.
    If you already have low-light vision, you double the benefit, allowing you to quadruple the illumination range of light sources.
    Visible draconic scales, colored appropriately for your chosen dragon, grant you a \plus1 bonus to Armor defense.

    At 3rd level, your voice deepens and becomes more draconic, granting you a \plus2 bonus to Intimidate and Persuasion checks.
    Your eyes continue to adapt, granting you \glossterm{darkvision} (50 ft.), or increasing the range of your existing darkvision by 50 feet.
    In addition, you gain bite and claw natural weapons as your extremities shape into a more draconic form (see \pcref{Natural Weapons}).

    At 7th level, you gain a breath weapon based on your chosen type of dragon.
    The shape of the breath weapon is given on \trefnp{Dragon Types}: a burst in either a line or a cone.
    A line breath weapon is an \arealarge, 5 ft.\ wide line, and a cone breath weapon is a \areamed cone.

    When you use your breath weapon, you make a Draconic power vs. Reflex attack against everything in the area.
    Success deals 1d8 damage per two draconic power.
    Failure deals half damage.
    After using your breath weapon, you must wait 1d4 rounds before you can use it again.

    At 11th level, your voice becomes still more draconic, and the bonus to Intimidate and Persuasion checks increases to \plus4.
    In addition, your ancestral mastery of magic grants you a \plus1 bonus to spellpower with all spells.
    The bonus to spellpower is \glossterm{magical}.

    \begin{dtable}
        \lcaption{Dragon Types}
        \begin{dtabularx}{\columnwidth}{>{\lcol}X >{\lcol}X >{\lcol}X}
            \tb{Dragon} & \tb{Energy Type} & \tb{Breath Weapon} \\
            \hline
            Black & Acid & Line \\
            Blue & Electricity & Line \\
            Brass & Fire & Line \\
            Bronze & Electricity & Line \\
            Copper & Acid & Line \\
            Gold & Fire & Cone \\
            Green & Acid & Cone \\
            Red & Fire & Cone \\
            Silver & Cold & Cone \\
            White & Cold & Cone \\
        \end{dtabularx}
    \end{dtable}

    \feat{Empowered Magic}{Magical, Spell}
    \featpres 13th level, 6th level spells.
    \featben You reduce the cost to apply the Empowered augment to all of your spells by one spell level.
    Many spells have an Empowered augment which makes their effects stronger.

    At 19th level, if you have 8th level spells, you can empower spells to reach their maximum potential.
    If you cast a spell using a spell slot seven levels higher than normal, you do not roll dice to determine the amount of damage or healing provided by the spell.
    Instead, treat all dice the spell would have rolled as if they had rolled their maximum value.

    \feat{Enchant Item}{Magical, Spell}
    \featpre 1st level spells.
    \featben You can imbue items with magic using your spells.
    Imbuing an item with magic takes material components, as described in \pcref{Magic Item Creation}.
    The items you can craft are limited by the schools and ability tags on spells you know.
    It takes you one hour per 2 gp of material components to create a item.

    You can also mend a broken magic item if it is one that you could make.
    Doing so costs a tenth of the raw materials and a quarter of the time it would take to craft that item in the first place.
    You cannot mend a destroyed magic item.

    At 3rd level, it takes you one hour per 10 gp of material components to create a magic item. % 125-600 gp materials, so 12.5-60 hours to craft

    At 7th level, if you know 2nd level spells, it takes you one hour per 100 gp of material components to create a magic item. % 900-3250 gp mats, 9-32 hrs
    In addition, you can combine the schools and tags from two different spells you know to create magic items.
    This can allow you to meet item crafting prerequisites that you could not meet with a single spell.

    At 11th level, if you know 4th level spells, it takes you one hour per 500 gp of material components to create a magic item. % 5000-18500 gp, 10-37 hrs

    At 15th level, if you know 6th level spells, it takes you one hour per 2,500 gp of material components to create a magic item. % 27500-200000 gp, 11-80 hrs

    \feat{Enchanter}{Magical, Spell}
    \featpres 1st level or higher \glossterm{Enchantment} spell known.
    \featben you gain a \plus2 bonus to Bluff, Intimidate, and Persuasion checks.

    At 7th level, if you know a 2nd level or higher Enchantment spell, your Enchantment spells are more powerful.
    Whenever you cast an Enchantment spell, you can reduce the cost to apply one augment to the spell by one spell level.

    At 11th level, if you know a 4th level or higher Enchantment spell, you no longer need verbal components to cast Enchantment spells (see \pcref{Components}).

    At 15th level, if you know a 6th level or higher Enchantment spell, the bonus to skill checks increases to \plus4.

    \feat{Endurance}{General}
    \featpre Constitution 3.
    \featben You ignore effects which would make you \fatigued.
    This allows you to sleep in medium or heavy armor without penalty.
    In addition, if an effect would make you \exhausted, it makes you fatigued instead.
    This ability does not allow you to ignore this fatigue.

    At 3rd level, you treat your Constitution as if it were 5 higher for the purpose of abilities whose duration is limited by your Constitution, such as sprinting (see \pcref{Sprint}).
    In addition, you are immune to effects which would force you to sleep.

    At 7th level, if you have 6 Constitution, you are immune to all effects which would make you fatigued or exhausted.
    In addition, you need half the normal amount of rest and sleep each day to function normally.
    For example, a human would only need four hours of sleep per night.

    At 11th level, if you have 9 Constitution, you need need one quarter the normal amount of rest and sleep each day to function normally.

    \feat{Energetic Magic}{Magical, Spell}
    \featpre 9th level, 4th level or higher \glossterm{energy} spell.
    \featben You gain a \plus1 bonus to spellpower for the purpose of determining the damage you deal with energy spells.
    This does not affect your accuracy or any other aspects of the spells.

    At 11th level, if you cast an energy spell using a spell slot one level higher than normal, you can change its energy type to any other energy type.
    This changes the spell's tags and any energy damage the spell deals appropriately.

    At 15th level, if you know a 6th level or higher energy spell, the effective spellpower bonus increases to \plus2.

    \feat{Escape Artist Mastery}{Skill}
    \featpre Escape Artist 6 ranks.
    \featben Whenever you make a Escape Artist check, you may roll twice and take the higher result.

    At 3rd level, you can squeeze and escape bindings as a move action, rather than as a standard action.

    At 7th level, if you have 10 ranks in Escape Artist, you gain a \plus5 bonus to Escape Artist checks.

    At 11th level, if you have 14 ranks in Escape Artist, you can attempt to escape from magic itself, slipping hostile spells off of your body so they dissipate harmlessly.
    As a standard action, you can make an Escape Artist check to throw off magical effects on you.
    The DR to dispel an effect is equal to 20 \add its spellpower, and the result must be at least 30.
    Success means the effect is dispelled, if it is an effect that \spell{dispel magic} could dispel.
    \magical

    You must be aware of an effect to escape it, either through the Spellcraft skill or because the effect is noticeable.
    You can only dispel effects which target you directly, not area effects which include you.
    If an ability targets multiple creatures, you can only remove its effects on you.

    \feat{Evoker}{Magical, Spell}
    \featpre 1st level or higher Evocation spell known.
    \featben You gain \glossterm{damage reduction} against \glossterm{energy damage} equal to your spellpower with the source of magic used to qualify for this feat.

    At 3rd level, you gain a \plus1 bonus to spellpower for the purpose of determining the damage you deal with Evocation spells.
    This does not affect your accuracy or any other aspects of the spells.

    At 7th level, if you know a 2nd level or higher Evocation spell, your Evocation spells are more powerful.
    Whenever you cast a Evocation spell, you can reduce the cost to apply one augment to the spell by one spell level.

    At 11th level, if you know a 4th level or higher Evocation spell, the effective spellpower bonus increases to \plus2.
    In addition, your damage reduction from this feat increases to twice your spellpower.

    \feat{Executioner}{Combat}
    \featpres 5th level.
    \featben At the start of every round, you can make a free melee \glossterm{strike} against all creatures you threaten that have no hit points remaining.
    This attack happens before the movement phase begins, and can kill a creature or knock it unconscious before it can act in the movement phase.

    At 7th level, you gain a \plus5 bonus to accuracy with physical attacks against creatures with no hit points remaining.

    At 11th level, you also gain a free strike against creatures that are \helpless or \unaware.
    You can still only make one strike with this feat against any individual creature.

    At 15th level, you also gain a free strike against all creatures that are \bloodied.
    You can still only make one strike with this feat against any individual creature.

    \feat{Extra Domain}{Class}
    \featpre Cleric level 5.
    \featben When you gain this feat, choose one domain from among those offered by your deity.
    You gain that domain in addition to your other domains.
    You may learn spells from its spell list, and you may choose its domain abilities for your class abilities.
    In addition, you automatically gain its domain gift.

    \feat{Eye of the Storm}{Combat, Style}
    \featpres Dexterity 3.
    \featben You reduce your \glossterm{overwhelm penalties} by 2.
    If your overwhelm penalty is reduced to 0, you are not considered to be overwhelmed.

    At 7th level, if you have 6 Dexterity, this penalty reduction increases to 3.

    At 11th level, if you have 9 Dexterity, this penalty reduction increases to 4.

    At 15th level, if you have 12 Dexterity, this penalty reduction increases to 5.
    \stylereq Wield a melee weapon. You must make a physical melee attack or take the total defense action each round.
    \parhead{Special} This is a Style feat, and all of its effects only apply while you are in this style.
    For details, see \pcref{Style Feats}.

    \feat{Far Shot}{Combat}
    \featpre Strength 3.
    \featben When you use a projectile weapon, such as a bow, its range increment increases by one-half (multiply by 1-1/2).
    When you use a thrown weapon, its range increment is doubled.

    At 3rd level, you gain a \plus2 bonus to damage with physical ranged attacks against targets more than 50 feet away from you.

    At 7th level, if you have 6 Strength, the range increment of projectile weapons is doubled, and the range increment of thrown weapons is tripled.

    At 11th level, if you have 9 Strength, the damage bonus increases to \plus3.

    \feat{Fearsome}{Combat, Magical, Skill}
    \spelldesc{You inspire fear in your foes in combat.}
    \featpres 5th level, Intimidate 6 ranks.
    \featben When you damage a creature with a physical melee attack, you can take an immediate action to inspire fear in the struck creature.
    You make an Intimidate vs. Mental attack against the struck creature.
    Success means the struck creature is \shaken by you for 2 rounds.
    Critical success means the struck creature is \frightened by you for 2 rounds instead.
    This is a \glossterm{Delusion}, \glossterm{Mind} effect.

    At 11th level, if you have 8 ranks in Intimidate, you can inspire fear in this way once per round without spending an action.

    At 15th level, \tdash.

    \feat{Guardian}{Combat}
    \featben Allies adjacent to you reduce their \glossterm{overwhelm penalties} by 1.
    If this effect reduces an ally's overwhelm penalty to 0, the ally is not considered to be overwhelmed.

    At 3rd, 7th, 11th, and 15th level, this penalty reduction increases by 1.

    \feat{Heal Mastery}{Skill}
    \featpre Heal 6 ranks.
    \featben Whenever you make a Heal check, you may roll twice and take the higher result.

    At 3rd level, you gain a \plus10 bonus to Heal checks to stabilize dying creatures (see \pcref{Dying}).

    At 7th level, if you have 10 ranks in Heal, you gain a \plus5 bonus to Heal checks.

    At 11th level, if you have 14 ranks in Heal, you can heal wounds with incredible speed.
    As a standard action, you can make a Heal check on a creature you touch to heal its wounds.
    The target heals hit points equal to your check result.
    For every five hit points you would restore, you can instead cure one point of critical damage.
    Once you have used this ability on a creature, you cannot use it again on that creature until it rests for half an hour.

    \feat{Heavy Weapon Fighting}{Combat}
    \featpre Strength 3.
    \featben When making a melee attack with a weapon held in two hands, you increase the damage die of the weapon by one increment (see \pcref{Weapon Size}).

    At 3rd level, the die size modifier increases to two increments.

    At 7th level, if you have 6 Strength, the die size modifier increases to three increments.

    At 11th level, if you have 9 Strength, the die size modifier increases to four increments.

    \feat{Hidden Magic}{Magical, Spell}
    \featpre 5th level, 2nd level spells.
    \featben If you cast a spell using a spell slot one level higher than normal, you can omit the verbal or somatic components of the spell (see \pcref{Components}).

    At 11th level, if you know 4th level spells, you can cast a spell using a spell slot two levels higher than normal to omit both verbal and somatic components.

    At 15th level, you can omit either the verbal or somatic components of all spells you cast.
    In addition, the spell slot cost to omit both verbal and somatic components with this feat is reduced to one spell level.

    \feat{Illusionist}{Magical, Spell}
    \featpre 1st level or higher Illusion spell known.
    \featben You gain a \plus2 bonus to Bluff, Disguise, and Stealth checks.

    At 7th level, if you know a 2nd level or higher Illusion spell, your Illusion spells are more powerful.
    Whenever you cast a Illusion spell, you can reduce the cost to apply one augment to the spell by one spell level.

    At 11th level, if you know a 4th level or higher Illusion spell, you no longer need verbal components to cast Illusion spells (see \pcref{Components}).

    At 15th level, if you know a 6th level or higher Illusion spell, the bonus to skills increases to \plus4.

    \feat{Inspiring Performer}{Magical, Performance, Skill}
    \featpre Perform 4 ranks.
    \featben You can create inspiring performances, starting with the ability to inspire competence.
    You can use any combination of performance abilities a number of times per day equal to half the ranks you have in your highest Perform skill.
    For details on how performances work, see \pcref{Performance Feats}.
    \parhead{Inspire Competence}
    As a standard action, you can make a DR 12 Perform check to inspire competence in another willing creature within \rngmed range of you.
    Success means the target gains a bonus offensive \glossterm{legend point} which lasts as long as you sustain the performance.
    Failure means the use of this ability is wasted.
    If the legend point has not been used when you end the performance, it is wasted.
    This is a \glossterm{Mind} effect.

    At 3rd level, you learn how to inspire courage.
    \parhead{Inspire Courage} As a standard action, you can make a DR 14 Perform check to inspire courage in another willing creature within \rngmed range of you.
    Success means the target gains temporary hit points equal to twice your Perform ranks.
    Failure means the use of this ability is wasted.
    This is a \glossterm{Mind} effect.

    At 7th level, you can affect a number of creatures with \textit{inspire competence} equal to one fifth of your Perform check result.
    In addition, you learn how to inspire resilience.
    \parhead{Inspire Resilience} As a standard action, you can make a DR 18 Perform check to inspire resilience in another willing creature within \rngmed range of you.
    Success means the target gains damage reduction against all damage equal to your Perform ranks.

    At 11th level, you can affect a number of creatures with \textit{inspire courage} equal to one fifth of your Perform check result.

    At 15th level, you can affect a number of creatures with \textit{inspire resilience} equal to one fifth of your Perform check result.

    \feat{Improvised Fighting}{Combat}
    \featben When making a physical attack with an improvised weapon, you gain a \plus2 bonus to accuracy and you increase the damage die of the weapon by one die size (see \pcref{Weapon Size}).

    At 3rd level, the die size modifier increases to two increments.

    At 7th level, the accuracy bonus increases to \plus4.

    At 11th level, the die size modifier increases to three increments.

    \feat{Inescapable}{Combat, Style}
    \featpre 9th level.
    \featben While in this style, whenever you hit with a physical melee attack against a creature, you can take an \glossterm{immediate action} to impede its movement.
    If you do, it is \immobilized for 2 rounds.

    At 11th level, your enemies must pay four times the normal movement cost to move out of squares you threaten.
    This replaces the normal penalties for moving through threatened squares (see \pcref{Moving Near Foes}).

    At 15th level, you can immobilize one creature each round without spending an action.

    At 19th level, all enemies within spaces you threaten are immobilized until you stop threatening them.
    \stylereq Wield a melee weapon.
    You must make a melee attack each round.
    \parhead{Special} This is a Style feat, and all of its effects only apply while you are in this style.
    For details, see \pcref{Style Feats}.

    \feat{Infuriating}{Combat, Style}
    \spelldesc{You can goad your foes into attacking you with rude gestures, insults, and mocking behavior.}
    \featpre 5th level.
    \featben Whenever you hit with a physical attack against a creature, you can take an \glossterm{immediate action} to make it angry.
    If you do, it is \goaded by you for 2 rounds.

    At 11th level, you can goad one creature per round without spending an action.

    At 15th level, the struck creature is \taunted instead of goaded.
    \parhead{Special} This is a Style feat, and all of its effects only apply while you are in this style.
    For details, see \pcref{Style Feats}.

    \feat{Iron Will}{General}
    \featpre Willpower 3.
    \featben You gain a \plus2 bonus to your Mental defense.
    This bonus can increase your hit points (see \pcref{Hit Points}).

    At 3rd level, you become immune to hostile \glossterm{Compulsion} effects.

    At 7th level, if you have 6 Willpower, the defense bonus increases to \plus4.

    At 11th level, if you have 9 Willpower, you become immune to hostile \glossterm{Mind} effects.

    \feat{Item Conduit}{General, Magical}
    \featpre 5th level.
    \featben You gain two additional daily item uses.

    At 7th level, you gain a third additional daily item use.

    At 11th level, you recover one spent item use per hour.

    At 15th level, you gain a fourth additional daily item use.

    \feat{Intimidate Mastery}{Skill}
    \featpre Intimidate 6 ranks.
    \featben Whenever you make a Intimidate check, you may roll twice and take the higher result.

    At 3rd level, critical success when you demoralize a foe means the target is \frightened by you for 2 rounds instead of being shaken.

    At 7th level, if you have 10 ranks in Intimidate, you gain a \plus5 bonus to Intimidate checks.

    At 11th level, if you have 14 ranks in Intimidate, you can demoralize creatures as a \glossterm{swift action}.

    \feat{Jump Mastery}{Skill}
    \featpre Jump 6 ranks.
    \featben Whenever you make a Jump check, you may roll twice and take the higher result.

    At 3rd level, you are always treated as if you had a running start when jumping.

    At 7th level, if you have 10 ranks in Jump, you gain a \plus5 bonus to Jump checks.

    At 11th level, if you have 14 ranks in Jump, your maximum horizontal distance when leaping is equal to twice your check result, rather than being equal to your check result.
    In addition, your maximum height is equal to your check result, rather than a quarter of your check result.

    \feat{Knowledge Mastery}{Skill}
    \featpre Knowledge (any) 6 ranks.
    \featben Whenever you make a Knowledge check, you may roll twice and take the higher result.

    At 3rd level, as long as you have mastered one Knowledge skill, you treat all Knowledge skills you were trained in as if you had mastered them.
    This grants you additional skill ranks and increases your check modifier as normal for mastering a skill.
    However, you cannot use these additional ranks to qualify for any feats or abilities.

    At 7th level, if you have 10 ranks in any Knowledge skill, you gain a \plus5 bonus to all Knowledge checks.

    At 11th level, if you have 14 ranks in any Knowledge skill, you gain a \plus2 bonus on accuracy, checks, and defenses against non-humanoid creatures you identify with a successful Knowledge check.

    \feat{Legendary Speed}{Combat, Style}
    \featpres 13th level, Dexterity 12.
    \featben You can make an additional \glossterm{strike} whenever you make a \glossterm{standard attack} .
    This does not stack with other effects that grant extra strikes.

    At 19th level, if you have 15 Dexterity, you can make a strike as a \glossterm{swift action}.
    \stylereq None.
    \parhead{Special} This is a Style feat, and all of its effects only apply while you are in this style.
    For details, see \pcref{Style Feats}.

    \feat{Lightning Reflexes}{General}
    \featpre Dexterity 3.
    \featben You gain a \plus2 bonus to your Reflex defense.

    At 3rd level, you gain a \plus5 bonus to initiative checks.

    At 7th level, if you have 6 Dexterity, the defense bonus increases to \plus4.

    At 11th level, if you have 9 Dexterity, the initiative check bonus increases to \plus10.

    \feat{Linguistic Mastery}{Skill}
    \featpre Linguistics 6 ranks.
    \featben Whenever you make a Linguistics check, you may roll twice and take the higher result.

    At 3rd level, \tdash.

    At 7th level, if you have 10 ranks in Linguistics, you gain a \plus5 bonus to Linguistics checks.

    At 11th level, if you have 14 ranks in Linguistics, you can speak, read, and understand all languages.
    This does not allow you to speak with creatures that lack a language.
    Certain extremely obscure languages may be beyond your knowledge.

    \feat{Mage Slayer}{Combat, Style}
    \featpre 5th level.
    \featben While in this style, if you hit a creature with a physical melee attack, that creature automatically fails Concentration checks it makes that round.

    At 7th level, all foes you threaten take a \minus4 penalty to Concentration checks.

    At 11th level, you gain \glossterm{magic resistance} equal to 10 \add your level.

    At 15th level, if you attack a creature with a physical melee attack, that creature automatically fails Concentration checks it makes that round.
    \stylereq Wield a melee weapon. You must make a physical melee attack each round.

    \feat{Maneuver Focus}{Combat}
    Choose one \glossterm{combat maneuver}.
    \featben You gain an ability based on the maneuver chosen.
    \begin{itemize}
        \item Dirty trick: When you successfully perform a dirty trick, the target is \impaired for 1d4 rounds, rather than 1 round.
        \item Disarm: When you successfully perform a disarm, you can make the disarmed item land up to 15 feet away in a random direction.
        \item Feint: When you critically succeed at a feint, you deal damage with your weapon normally.
        \item Grapple: Grappling does not cause you to be \defenseless.
        \item Shove: When you successfully perform a shove, you can move the target the full distance without needing to move with it.
        \item Trip: When you successfully perform a trip, you can make a free \glossterm{strike} against the target. The target does not suffer prone penalties for the trip against the free attack.
    \end{itemize}

    At 3rd level, you a \plus1 bonus to accuracy with the chosen maneuver.

    At 7th level, the accuracy bonus increases to \plus2.

    At 11th level, whenever you perform the chosen maneuver, if your attack result also beats the target's Armor defense, you deal normal damage to the target with the weapon used to perform the maneuver.
    This damage is in addition to the maneuver's normal effects.
    If the maneuver was performed with a free hand, you deal damage with your unarmed attack (see \pcref{Unarmed Combat}).
    \parhead{Special} This feat can be taken multiple times.
    Each time, you choose a different combat maneuver.

    \feat{Martial Training}{General}
    \featben You are proficient in light and medium body armor, as well as shields.
    In addition, you become proficient in one additional weapon group of your choice.

    At 3rd level, you become proficient in heavy body armor.

    At 7th level, you become proficient in an additional weapon group of your choice.
    \parhead{Normal}
    A character who is wearing armor with which she is not proficient applies its \glossterm{encumbrance penalty} to accuracy with physical attacks.
    The character also suffers double the normal arcane spell failure chance for wearing the armor.

    \feat{Mesmerizing Performer}{Magical, Performance, Skill}
    \featpre Perform (any) 4 ranks.
    \featben You can create mesmerizing performances, starting with the ability to fascinate observers.
    You can use any combination of performance abilities a number of times per day equal to half the ranks you have in your highest Perform skill.
    For details on how performances work, see \pcref{Performance Feats}.
    \parhead{Fascinating Performance}
    \featben As a standard action, you can make a DR 12 Perform check to fascinate another creature within \rngmed range of you.
    If the check succeeds, treat the roll as an attack against the target's Mental defense.
    Success means the target is \fascinated by you as long as you sustain the performance.
    Otherwise, it is unaffected.
    This is a \glossterm{Compulsion}, \glossterm{Mind} effect.

    At 7th level, you learn how to suggest actions to creatures you have fascinated.
    \parhead{Suggestive Performance}
    As a standard action, you can make a DR 18 Perform check to suggest an course of action to a creature within \rngmed range of you.
    The target must be already fascinated by you using the \textit{fascinating performance} ability.
    If the check succeeds, treat the roll as an attack against the target's Mental defense.
    Success means the target thinks your suggestion is a good idea and will try to follow it as long as you sustain the performance.
    Otherwise, it is unaffected, though it remains fascinated.
    Using this ability is considered to continue your performance for the \textit{fascinating performance} ability, and does not free creatures you have fascinated from that effect.
    This is a \glossterm{Delusion}, \glossterm{Mind}, \glossterm{Speech} effect.

    At 11th level, you can affect a number of creatures with \textit{fascinating performance} equal to one fifth of your Perform check result.

    \feat{Miscaster}{Magical, Spell}
    \featpre 1st level spells.
    \featben Your explosive miscasts do not affect your allies.

    At 3rd level, your explosive miscasts no longer hurt you.
    Instead, they target all enemies in a \areasmall radius burst centered on you.

    At 7th level, if you know 2nd level spells, when you target a random creature with a spell's miscast effect, roll twice to determine which creature is affected.
    You choose which result is used.
    In addition, your explosive miscasts target all enemies in a \areamed radius burst centered on you.

    At 11th level, if you know 4th level spells, the area affected by your localized miscasts becomes a \areamed radius centered on you.
    In addition, you may choose to make any number of creatures in the area that you are aware of immune to the effect of the miscast.

    At 15th level, if you know 6th level spells, you are immune to the effects of all miscast spells, both by you and by other spellcasters.
    They are unable to damage you or affect you in any way.
    You may still choose to be affected by miscast effects, if desired.

    \feat{Mocking Performer}{Magical, Performance, Skill}
    \featpre Perform (any) 4 ranks.
    \featben You can create performances to mock your foes, starting with the ability to impair a foe.
    You can use any combination of performance abilities a number of times per day equal to half the ranks you have in your highest Perform skill.
    For details on how performances work, see \pcref{Performance Feats}.
    \parhead{Mocking Performance} As a standard action, you can make a DR 12 Perform check to make fun of a creature within \rngmed range of you.
    If the check succeeds, treat the roll as an attack against the target's Mental defense.
    Success means the target is \impaired with attacks and checks as long as you sustain the performance.
    Otherwise, it is unaffected.
    This is a \glossterm{Delusion}, \glossterm{Mind} effect.

    At 3rd level, you learn how to belittle a foe and make it feel as if it is already doomed to fail.
    \parhead{Demoralizing Performance} As a standard action, you can make a DR 14 Perform check to demoralize a creature within \rngmed range of you.
    If the check succeeds, treat the roll as an attack against the target's Mental defense.
    Success means the target is treated as \bloodied as long as you sustain the performance.
    Otherwise, it is unaffected.
    This is a \glossterm{Delusion}, \glossterm{Mind} effect.

    At 7th level, you can affect a number of creatures with \textit{mocking performance} equal to one fifth of your Perform check result.
    In addition, you learn how to taunt your foes to attack a creature of your choice.
    \parhead{Taunting Performance} As a standard action, you can make a DR 18 Perform check to taunt a creature within \rngmed range of you on behalf of one of your allies.
    If the check succeeds, treat the roll as an attack against the target's Mental defense.
    Success means the target is \taunted by a willing creature of your choice within \rngmed range of you.
    Otherwise, it is unaffected.
    This is a \glossterm{Delusion}, \glossterm{Mind} effect.

    At 11th level, you can affect a number of creatures with \textit{demoralizing performance} equal to one fifth of your Perform check result.

    At 15th level, you can affect a number of creatures with \textit{taunting performance} equal to one fifth of your Perform check result.
    All targets must be taunted by a single willing ally of your choice.

    \feat{Mounted Combat}{Combat}
    \featpre Ride 4 ranks.
    \featben Whenever your mount is hit by a physical attack, you can take an \glossterm{immediate action} to made a Ride check.
    Your mount can use your check result in place of its physical defenses against the attack, potentially causing the attack to miss.

    At 3rd level, the penalty you take when using a ranged weapon while mounted is decreased by 4: \minus0 instead of \minus4 if your mount moves during the action phase, and \minus4 instead of \minus8 if your mount is sprinting.
    In addition, whenever you charge a creature while mounted, you increase your weapon damage die by one size increment (see \pcref{Weapon Size}).

    At 7th level, you can make a Ride check to negate one attack per round without spending an action.

    At 11th level, you take no penalties for firing a ranged weapon while mounted.
    In addition, the die size modifier for charging a creature while mounted increases to two increments.

    \feat{Overwhelming Fire}{Combat, Style}
    \featben While wielding a ranged weapon, you contribute to overwhelm penalties against all creatures within a single range increment of you.
    You do not contribute to overwhelm penalties against creatures with cover from you.

    At 7th level, overwhelmed creatures you are overwhelming with this feat take an additional \minus1 penalty to physical defenses.
    In addition, you can help overwhelm creatures up to two range increments away from you.

    At 11th level, the additional defense penalty increases to \minus2.
    In addition, you can help overwhelm creatures up to three range increments away from you.
    \stylereq Wield a ranged weapon.
    You must make a physical ranged attack each round.
    \parhead{Special} This is a Style feat, and all of its effects only apply while you are in this style.
    For details, see \pcref{Style Feats}.

    \feat{Parry}{Combat, Style}
    \featpre Dexterity 3.
    \featben While in this style, whenever you are hit by a \glossterm{strike}, you may attempt to parry it as an \glossterm{immediate action}.
    If you do, you make an attack roll with a weapon you wield.
    You may use your attack result in place of your physical defenses against the attack.
    This can cause the attack to miss.

    At 3rd level, if you have a free hand and are not wielding a shield, you gain a \plus2 bonus to your parry attacks.
    If you are wielding a shield, you can add your shield's defense bonus to the attack you make to parry.

    At 7th level, if you have 6 Dexterity, you can parry one strike per round without spending an action.

    At 11th level, if you have 9 Dexterity, you can counterattack more effectively after parrying.
    When you successfully parry an attack from a foe you threaten, if your parry attempt exceeds your foe's attack roll by 10 or more, it is \defenseless against you for 1 round.

    At 15th level, if you have 12 Dexterity, you can parry two strikes per round without spending an action.

    \stylereq Wield a melee weapon.
    You must make a melee attack or take the total defense action each round.
    \parhead{Special} This is a Style feat, and all of its effects only apply while you are in this style.
    For details, see \pcref{Style Feats}.

    \feat{Persuasion Mastery}{Skill}
    \featpre Persuasion 6 ranks.
    \featben Whenever you make a Persuasion check, you may roll twice and take the higher result.

    At 3rd level, \tdash.

    At 7th level, if you have 10 ranks in Persuasion, you gain a \plus5 bonus to Persuasion checks.

    At 11th level, if you have 14 ranks in Persuasion, you can compel creatures to obey your suggestions.
    As an immediate action, when you speak a suggestion aloud, you can make a Persuasion vs. Mental attack against a creature within \rngmed range of you.
    Success means the target it is compelled to obey your suggestion, as the \spell{suggestion} spell.
    You can use this ability three times per day.
    \magical

    \feat{Perform Mastery}{Skill}
    \featpre Perform (any) 6 ranks.
    \featben Whenever you make a Perform check, you may roll twice and take the higher result.

    At 3rd level, as long as you have mastered one Perform skill, you treat all Perform skills you were trained in as if you had mastered them.
    This grants you additional skill ranks and increases your check modifier as normal for mastering a skill.
    However, you cannot use these additional ranks to qualify for any feats or abilities.

    At 7th level, if you have 10 ranks in any Perform skill, you gain a \plus5 bonus to Disguise checks.

    At 11th level, you can use two Performance abilities as part of the same performance (see \pcref{Performance Feats}).
    You can activate them both as part of the same action (if they require the same action to activate), and sustain both feats with a single performance.

    \feat{Point Blank Shot}{Combat}
    \featben You gain a \plus2 bonus to damage with physical ranged attacks against targets within 50 feet of you.

    At 3rd level, you reduce your penalty for firing medium and large ranged weapons at adjacent creatures to \minus2.

    At 7th level, you suffer no penalty for firing ranged weapons at adjacent creatures.

    At 11th level, you are not treated as \defenseless while wielding a ranged weapon you are proficient with.
    In addition, the damage bonus increases to \plus3.
    \stylereq Wield a ranged weapon.

    \feat{Power Attack}{Combat, Style}
    \featpres 5th level, Strength 6.
    \featben You gain a \plus2 bonus to damage with physical melee attacks.

    At 7th level, this bonus increases to \plus3.

    At 11th level, if you have 9 Strength, this bonus increases to \plus4.

    At 15th level, if you have 12 Strength, this bonus increases to \plus5.

    At 19th level, if you have 15 Strength, this bonus increases to \plus6.
    \stylereq Wield a ranged weapon.
    \parhead{Special} This is a Style feat, and all of its effects only apply while you are in this style.
    For details, see \pcref{Style Feats}.

    \feat{Precise Attack}{Combat, Style}
    \featpres Perception 3.
    \featben You gain a \plus2 bonus to accuracy with \glossterm{physical attacks}.
    In exchange, you take a \minus2 penalty to damage with physical attacks (to a minimum of 1).

    At 3rd level, the accuracy bonus increases to \plus3.

    At 7th level, if you have 6 Perception, the damage penalty is reduced to \minus1.

    At 11th level, if you have 9 Perception, the damage penalty is removed.
    \parhead{Special} This is a Style feat, and all of its effects only apply while you are in this style.
    For details, see \pcref{Style Feats}.

    \feat{Precise Shot}{Combat}
    \featpres 5th level, Perception 6.
    \featben Your ranged attacks ignore cover and concealment, except total cover and total concealment.

    At 7th level, when you attack a grappling opponent with a physical ranged attack, you do not have a chance to attack the wrong creature in the grapple.

    At 11th level, if you have 9 Perception, you ignore any effects which would give you a 20\% failure chance with your physical ranged attacks.
    This includes being \impaired.
    \stylereq Wield a ranged weapon.

    \feat{Precision Strikes}{Combat}
    \featpre 5th level, Perception 6.
    \featben As a standard action, you can make a single physical attack against a creature's Armor and Reflex defenses.
    Success against both defenses means the target takes damage from your weapon.
    This damage is doubled for each strike you can make beyond the first.
    If you fail to hit the target's Armor defense, but hit its Reflex defense, it takes half damage.

    At 7th level, if you have Perception 9, you can attempt to disable a foe's wings.
    As a standard action, you can make a single physical attack against a creature's Armor and Fortitude defenses.
    Success against Armor defense means the target takes damage from your weapon.
    This damage is doubled for each strike you can make beyond the first.
    If you also beat the target's Fortitude defense, it loses the ability to fly for 2 rounds.
    You must be able to hit the target's wings, which is usually only possible with melee attacks if it is no more than one size category larger than you.
    This only affects creatures who use wings or other physical means to fly, and has no effect on creatures with magical flight.

    At 11th level, if you have Perception 12, you can deliver a killing blow.
    As a standard action, you can make a single physical attack against a creature.
    Success against Armor defense means the target takes damage from your weapon.
    This damage is doubled for each strike you can make beyond the first.
    Any damage in excess of the target's remaining hit points is dealt as \glossterm{critical damage}.

    \feat{Quick Draw}{Combat}
    \featben You can draw light and medium weapons as a \glossterm{free action}.
    In addition, you can draw heavy weapons and hidden weapons of any type (see the Sleight of Hand skill) as a move action.
    This allows you to throw light weapons at your full normal rate of attacks (much like a character with a bow).

    At 3rd level, you can draw any item of similar size to a light or medium weapon as a free action.

    At 7th level, you can also draw any item of similar size to a heavy weapon as a free action.
    In addition, you can sheathe or similarly put away items of similar size to light and medium weapons as a free action.

    At 11th level, you can also sheathe or similarly put away items of similar size to heavy weapons as a free action.

    \feat{Quickened Magic}{Magical, Spell}
    \featpre 9th level, 4th level spells.
    \featben As a \glossterm{swift action}, you can cast a spell using a spell slot two levels higher than normal.
    The spell must have a casting time of one standard action.
    This is mentally draining, and if you cast a spell in this way, you cannot act during the next action phase.
    You can only cast one spell in this way per round, even if you can take multiple swift actions.

    At 15th level, if you know 6th level spells, you are less draining by casting a quickened spell.
    When you cast a spell with this feat, you cannot cast spells in the next action phase, but you can take other actions normally.

    \feat{Rapid Recovery}{General}
    \featpre Con 3.
    \featben You naturally heal in half the normal time, allowing you to recover a quarter of your hit points with fifteen minutes of rest.
    Likewise, you heal \glossterm{critical damage} after four hours, rather than eight.
    This stacks with the benefits of accelerating recovery with the Heal skill (see \pcref{Accelerate Recovery}).

    At 3rd level, you only need one successfully resisted \glossterm{stabilization roll} to stabilize while dying.

    At 7th level, if you have 6 Constitution, magical healing is more effective on you.
    Whenever you receive magical healing, you increase the healing by an amount equal to your Constitution, up to a maximum of the original healing provided.

    At 11th level, if you have 9 Constitution, you heal hit points equal to your Constitution at the end of every round.
    In addition, you heal \glossterm{critical damage} at a rate of one critical damage per 5 minutes.
    \magical

    \feat{Reflexive Dodge}{Combat}
    \featpres 13th level, Dexterity 12.
    \featben Whenever you are hit by a \glossterm{strike}, you can take an \glossterm{immediate action} to dodge it.
    If you do, the attack misses.
    You must be aware of an attack to dodge it in this way.

    At 19th level, if you have 15 Dexterity, you can dodge one attack per round in this way without spending an action.

    \feat{Ride Mastery}{Skill}
    \featpre Ride 6 ranks.
    \featben Whenever you make a Ride check, you may roll twice and take the higher result.

    At 3rd level, \tdash.

    At 7th level, if you have 10 ranks in Ride, you gain a \plus5 bonus to Ride checks.

    At 11th level, if you have 14 ranks in Ride, you can attempt to ride unwilling creatures.
    As a standard action, you can make a Ride vs. Fortitude and Reflex attack against a creature adjacent to you.
    The creature must be at least one size category larger than you.
    Success means you ride the creature.
    Riding a creature is like grappling the creature, with the following changes.
    \begin{itemize}
        \item You share space with the creature you ride, just like riding a normal mount.
            If your mount breaks the grapple, you move to an adjacent unoccupied square of your choice.
            If there are no adjacent unoccupied squares, you in the same space as your mount, squeezing as necessary.
        \item You make Ride checks instead of grapple attacks to remain in the grapple and take actions in the grapple.
        \item You cannot pin the creature.
    \end{itemize}

    \feat{Ritual Caster}{Spell}
    \featpre Intelligence 3.
    \featben You can learn and perform rituals as if you were an arcane caster with a spellpower equal to your level.
    The maximum level of ritual that you can learn or perform is equal to half your level or your Intelligence, whichever is lower.

    \feat{Sense Motive Mastery}{Skill}
    \featpre Sense Motive 6 ranks.
    \featben Whenever you make a Sense Motive check, you may roll twice and take the higher result.

    At 3rd level, \tdash.

    At 7th level, if you have 10 ranks in Sense Motive, you gain a \plus5 bonus to Sense Motive checks.

    At 11th level, if you have 14 ranks in Sense Motive, you can read the minds of creatures.
    As a standard action, you can make a Sense Motive vs. Mental attack against a creature within \rngmed range of you.
    Success means you read the target's mind, as the \spell{read mind} spell.
    You can use this ability three times per day.
    \magical

    \feat{Shaped Magic}{Magical, Spell}
    \featpre 5th level, 2nd level spells.
    \featben If you cast a spell with an area using a spell slot one level higher than normal, you can change its area.
    The spell's area be a \areasmall, \areamed, \arealarge, or \areahuge \glossterm{burst}.
    You can change a cone to a 5 ft.\ wide line of the same size, or a radius one size smaller.
    You can change a radius to a 5 ft.\ wide line one size larger, or a cone of the same size.
    You can change a line to a cone one size smaller, or a radius two sizes smaller.

    At 11th level, you can cast a spell with an area using a spell slot two levels higher than normal.
    If you do, you can exclude any number of 5-foot cubes within the spell's area from its effect.
    This allows you to prevent the spell from affecting your allies, while still allowing it to affect your enemies.
    The area affected by the spell, ignoring all removed cubes, must be contiguous.

    At 15th level, you can change the area of all area spells you cast without spending a spell slot of a higher level.

    \feat{Shielded Fighting}{Combat}
    \featpre Proficiency with shields.
    \featben While wielding a shield, you gain a \plus1 bonus to \glossterm{physical defenses}.

    At 3rd level, you gain a \plus5 bonus to defenses against physical ranged attacks while wielding a shield.

    At 7th level, the bonus to physical defenses increases to \plus2.

    At 11th level, when you are hit by a ranged \glossterm{strike} while wielding a shield, you can make the strike miss automatically as an \glossterm{immediate action}.
    You must be aware of an attack to negate it in this way.

    \feat{Skill Savant}{Skill}
    \featben You gain two skill points.
    You may spend these skill points immediately.

    At 7th level, you gain a third additional skill point.

    At 11th level, you gain a fourth additional skill point.

    \feat{Sleight of Hand Mastery}{Skill}
    \featpre Sleight of Hand 6 ranks.
    \featben Whenever you make a Sleight of Hand check, you may roll twice and take the higher result.

    At 3rd level, \tdash.

    At 7th level, if you have 10 ranks in Sleight of Hand, you gain a \plus5 bonus to Sleight of Hand checks.

    At 11th level, if you have 14 ranks in Sleight of Hand, you can hide objects in impossible ways.
    Whenever you make a Sleight of Hand check to conceal or pickpocket an object, if the result is 30 or higher, you can hide the object into a pocket dimension.
    You can retrieve the item later as a move action.
    You may only have up to three items hidden in this way, none of which can be larger than one size category smaller than you.
    \magical

    \feat{Somatic Strike}{Magical, Spell}
    \featpres 9th level, 4th level spells.
    \featben When you cast spells, you can make a single \glossterm{strike} with a melee weapon in place of the somatic components for the spell.
    The target of the strike does not matter, and you can even attack thin air.
    The spell is otherwise cast as normal, regardless of whether the strike hits or misses.

    At 11th level, you gain a bonus to damage with this strike equal to the spell level of the spell being cast.

    At 15th level, if you know 6th level spells, you can make two strikes instead of one.
    The damage bonus applies to both strikes.

    At 19th level, the bonus to damage increases to be equal to half your spellpower with the spell being cast.

    \feat{Spellbreaker}{General, Magical}
    \featpres 9th level, Willpower 9.
    \featben You gain \glossterm{magic resistance} equal to 10 \add your Willpower.

    At 15th level, if you have 12 Willpower, you can reflect spells.
    When you resist a \glossterm{targeted spell} with this magic resistance, you can choose for the spell to be reflected back at the caster.
    The spell otherise functions normally, including effects on other targets of the spell, except that the caster is treated as a target of the spell instead of you.
    A spell reflected in this way cannot be reflected back at you.

    At 19th level, if you have 15 Willpower, you can completely negate spells cast on you.
    When you resist a spell with this magic resistance, you can choose to negate the spell's effects entirely.
    The spell has no effect, even if it was an area spell or a spell that targeted other creatures.
    If the spell had a duration, it is immediately dispelled, even if resisting it would not normally end or negate the spell's effects.
    This happens before any other magic resistance effects, preventing other creatures from resisting or reflecting it.

    \feat{Spellcasting Versatility}{Spell}
    Choose a spellcasting class you have.
    \featpres 5th level, 1st level spells from the chosen class.
    \featben You gain a bonus when determining your class level for the purpose of spells per day (if any), spells known, and maximum available spell level in your chosen class.
    The bonus is equal to half your class level in your chosen class.
    This cannot increase your effective class level above your character level, and does not improve the power of class abilities or grant you additional class abilities from your chosen class.

    For example, a character with 4 levels in wizard and 4 levels in fighter would have the spells per day and spells known of a 6th level wizard, and she would be able to cast 3rd level spells.
    \parhead{Special} This feat can be taken multiple times.
    Each time, you choose a different spellcasting class you possess.

    \feat{Spellcraft Mastery}{Skill}
    \featpre Spellcraft 6 ranks.
    \featben Whenever you make a Spellcraft check, you may roll twice and take the higher result.

    At 3rd level, \tdash.

    At 7th level, if you have 10 ranks in Spellcraft, you gain a \plus5 bonus to Spellcraft checks.

    At 11th level, if you have 14 ranks in Spellcraft, you gain a \plus2 bonus on accuracy, checks, and defenses against spells and \glossterm{magical} effects you identify with a successful Spellcraft check.

    \feat{Spellgift}{General, Magical}
    \featpres 5 levels in classes without spellcasting, Willpower 6.
    \featben You have inherent magic in your body, granting you magical power.
    When you gain this feat, you choose a source of magic: arcane, divine, or nature.
    You also choose a 1st level spell from that source of magic.
    You can spend a spellgift point to use that spell as a magical ability.
    You have a maximum number of spellgift points equal half your Willpower.
    Your spellpower with spellgifts is equal to your Willpower or your level, whichever is higher.

    At 7 levels in non-spellcasting classes, you can choose a spell of up to 2nd level from the same source. You can also spend a spellgift point to use that spell.

    At 11 levels in non-spellcasting classes, if you have 9 Willpower, you can choose a spell of up to 4th level from the same source. You can also spend a spellgift point to use that spell.

    At 15 levels in non-spellcasting classes, if you have 12 Willpower, you can choose a spell of up to 6th level from the same source. You can also spend a spellgift point to use that spell.

    \feat{Spellstrike}{Magical, Spell}
    \featpre 5th level, 2nd level spells.
    \featben If you cast a \glossterm{targeted spell} using a spell slot one level higher than normal, you can make a single \glossterm{strike} with a weapon you wield as part of casting the spell.
    The spell must have a casting time of a standard action.
    The strike is made at the same time as other physical attacks in the action phase.
    If the strike hits a creature or object, you can spend an \glossterm{immediate action} to activate the spell.
    If you do, the target suffers the effects of the spell.

    If the strike misses, the spell is imbued into the weapon used to make the strike for 2 rounds.
    If the weapon hits a creature or object during that time, you can spend an immediate action to activate the spell, just as if the initial strike had succeeded.
    If you activate the spell, the duration runs out, or if you cast another spell, the spell in the weapon dissipates without effect.

    At 7th level, the imbuement lasts for 5 minutes before fading away.
    This can give you enough time to imbue a spell into a weapon and give the weapon to another creature.

    At 11th level, if you know 4th level spells, casting spells does not cause the imbuement to fade.
    You can still only have one spell imbued into a weapon with this feat at once.

    At 15th level, you can spend a spell slot of the spell's level to gain the benefits of this feat, rather than a spell slot one level higher than normal.

    \feat{Spellwoven Performance}{Magical, Skill, Spell}
    \featpres 5th level, 1st level spells, any Performance feat.
    \featben You can cast spells while sustaining a performance, including for performance abilities (see \pcref{Performance Feats}).
    This does not remove the need for verbal and somatic components, though you may attempt to incorporate those into your performance (see \pcref{Components}).
    The Spellcraft DR to identify those spells as they are cast increases by 10, as the performance disguises the magic.

    At 7th level, you can spend a spell slot to use a performance ability in place of a use of your performance abilities.

    At 11th level, you can spend a use of your performance abilities in place of a spell slot.
    The spell slot must be of a spell level you have access to, and its level cannot exceed half your Perform ranks.

    \feat{Sprint Mastery}{Skill}
    \featpre Sprint 6 ranks.
    \featben Whenever you make a Sprint check, you may roll twice and take the higher result.

    At 3rd level, you gain a \plus10 foot bonus to speed in all your movement modes.

    At 7th level, if you have 10 ranks in Sprint, you gain a \plus5 bonus to Sprint checks.

    At 11th level, if you have 14 ranks in Sprint, you can sprint for a number of minutes equal to twice your Constitution (minimum 2).
    After that time, you must rest for 5 minutes before you can sprint again.

    \feat{Stealth Mastery}{Skill}
    \featpre Stealth 6 ranks.
    \featben Whenever you make a Stealth check, you may roll twice and take the higher result.

    At 3rd level, your penalties for moving while hiding are reduced by 5.
    This means you can move at up to half speed with no penalty, and you can move at full speed with a \minus5 penalty.

    At 7th level, if you have 10 ranks in Stealth, you gain a \plus5 bonus to Stealth checks.

    At 11th level, if you have 14 ranks in Stealth, you can use the Stealth skill to hide even while being observed.
    You take take a \minus10 penalty to the Stealth check when hiding in this way, and you still need passive cover or concealment to hide.

    At 15th level, if you have 11 ranks in Stealth, the penalty for hiding while observed is reduced to \minus5.

    \feat{Style Fusion}{Class, Combat}
    \featpre Fighter level 9, Intelligence 9.
    \featben You can use two styles at once, gaining the benefits of both styles (see \pcref{Style Feats}).
    This allows you to use a single \glossterm{free action} to initiate two styles at once, or change which two styles you are using.
    You may only do this for a number of rounds per day equal to your Intelligence.

    At 15th level, if you have 12 Intelligence, there is no limit on the number of rounds you can use this feat each day.

    \feat{Supreme Inspiration}{Magical, Performance, Skill}
    \featpres 9th level, Perform 9 ranks.
    \featben You can create performances with staggering inspirational power, starting with the ability to inspire serenity.
    You can use any combination of performance abilities a number of times per day equal to half the ranks you have in your highest Perform skill.
    For details on how performances work, see \pcref{Performance Feats}.
    \parhead{Inspire Serenity} As a standard action, you can make a DR 20 Perform check to clear the mind of a willing creature within \rngmed range of you.
    The target is immune to all hostile \glossterm{Mind} effects as long as you sustain the performance.
    This is a Mind effect.

    At 15th level, you learn how to inspire mastery.
    \parhead{Inspire Mastery} As a standard action, you can make a DR 22 Perform check to inspire mastery in another willing creature within \rngmed range of you.
    As long as you sustain the performance, at the start of each round, the target gains a bonus general \glossterm{legend point} which it can spend during that round.
    If the target's legend point has not yet been used when the round ends, it is wasted.

    At 19th level, you can affect a number of creatures with \textit{inspire serenity} equal to one fifth of your Perform check result.
    In addition, you learn how to inspire perfection.
    \parhead{Inspire Perfection} As a standard action, you can make a DR 34 Perform check to inspire perfection in another willing creature within \rngmed range of you.
    Once during the performance's duration, the target can spend a general legend point to treat any attack or check it attempts as if it had rolled a 20, ignoring its actual roll.
    This decision is made in the same way as the creature would spend a normal legend point.
    This performance is extraordinarily draining, and you can only perform it once per day.

    \feat{Survival Mastery}{Skill}
    \featpre Survival 6 ranks.
    \featben Whenever you make a Survival check, you may roll twice and take the higher result.

    At 3rd level, you ignore difficult terrain and harmful natural terrain of any kind.
    If a skill check, such as Climb or Swim, would normally be required to move through the terrain, this ability does not help.
    In addition, you can choose to leave no trace of your passage as you move.
    If you do, tracking you is impossible by any physical means.

    At 7th level, if you have 10 ranks in Survival, you gain a \plus5 bonus to Survival checks.

    At 11th level, if you have 14 ranks in Survival, you are immune to harmful planar effects.
    In addition, as a standard action, you can find your way to any location, as the \spell{find the path} ritual.
    You may use this ability once per day.
    Finding locations in this way is a \glossterm{magical} ability.

    \feat{Swift}{General}
    \featben You increase your land speed by 5 feet.

    At 3rd level, the speed bonus increases to 10 feet.

    At 7th level, the speed bonus increases to 20 feet.

    At 11th level, the speed bonus increases to 30 feet.

    \feat{Swim Mastery}{Skill}
    \featpre Swim 6 ranks.
    \featben Whenever you make a Swim check, you may roll twice and take the higher result.

    At 3rd level, you gain a \glossterm{swim speed} equal to your land speed.
    This grants several benefits.
    \begin{itemize}
        \item A successful Swim check to move allows you to move a distance equal to your swim speed.
        \item You gain a \plus10 bonus to Swim checks.
    \end{itemize}

    At 7th level, if you have 10 ranks in Swim, you gain a \plus5 bonus to Swim checks.

    At 11th level, if you have 14 ranks in Swim, you do not suffer any penalties to physical melee attacks, checks, or physical defenses for being underwater.
    You still suffer the normal penalty with underwater ranged attacks.

    \feat{Tactical Prediction}{Combat}
    \featpres 5th level, Intelligence 6.
    \featben As a swift action, you can make an Intelligence vs. Mental attack against a creature within \rngmed range of you.
    Success means you learn in general terms what the creature is planning to do during the next phase.
    Of course, it can change its plans, particularly if it hears you tell your allies what it will do.

    At 11th level, if you have 9 Intelligence, you gain a \plus2 bonus to this attack per round you have seen the creature fight.

    At 15th level, if you have 12 Intelligence, once per round you can attempt to predict a creature's actions as a \glossterm{free action}.

    \feat{Transmuter}{Magical, Spell}
    \featpre 1st level or higher Transmutation spell known.
    \featben You gain \glossterm{damage reduction} against \glossterm{physical damage} equal to half your spellpower with the source of magic used to qualify for this feat.

    At 7th level, if you know a 2nd level or higher Transmutation spell, your Transmutation spells are more powerful.
    Whenever you cast a Transmutation spell, you can reduce the cost to apply one augment to the spell by one spell level.

    At 11th level, if you know a 4th level or higher Transmutation spell, your damage reduction increases to be equal to your spellpower.

    \feat{Toughness}{General}
    \featpre Constitution 3.
    \featben You gain a \plus2 bonus to your Fortitude defense.
    This bonus can increase your hit points (see \pcref{Hit Points}).

    At 3rd level, you gain bonus hit points equal to your level.
    In addition, you become immune to disease.

    At 7th level, if you have 6 Constitution, the defense bonus increases to \plus4.
    In addition, you halve the penalties you take from having \glossterm{critical damage}.

    At 11th level, if you have 9 Constitution, you become immune to poison.

    \feat{Trapfinder}{Skill}
    \spelldesc{You have honed your senses to recognize traps quickly and effectively.}
    \featpre Awareness 4 ranks.
    \featben As a full-round action, you can move up to 10 feet while searching every square within 10 feet of you for traps with the Awareness skill (see \pcref{Awareness}).
    If you detect a trap partway through your movement, you may immediately stop moving.

    At 3rd level, you gain a \plus5 bonus to Awareness checks to find traps.

    At 7th level, whenever you come within 10 feet of a trap, you can immediately make an Awareness check to notice the trap.
    This check should be made secretly, so you do not know whether you failed to notice a trap.
    In addition, when you take a full-round action to search for traps with this feat, the range at which you can notice traps increases to 50 feet.

    At 11th level, the bonus to Awareness checks to find traps increases to \plus10.
    In addition, you can immediately make an Awareness check to notice any traps within 20 feet of you.

    \feat{Two-Weapon Fighting}{Combat}
    \featpre Dexterity 3.
    \featben When making a \glossterm{dual attack} with two weapons at once, you gain a \plus1 bonus to accuracy with physical attacks.

    At 3rd level, you gain a \plus1 bonus to damage while making a dual attack with two weapons at once.

    At 7th level, if you have 6 Dexterity, the accuracy bonus increases to \plus2.

    At 11th level, if you have 9 Dexterity, the damage bonus increases to \plus2.

    \stylereq Wield two weapons at once.
    \parhead{Special} This is a Style feat, and all of its effects only apply while you are in this style.
    For details, see \pcref{Style Feats}.

    \feat{Unarmed Fighting}{Combat}
    \featben You gain proficiency with your unarmed attack.
    This grants you a \plus4 bonus to accuracy with the weapon and allows you to defend yourself with it, just as if you were using another melee weapon you are proficient with.
    In addition, your unarmed attacks can deal lethal or nonlethal damage as you choose.

    At 3rd level, you can attack with multiple parts of your body simultaneously.
    This allows you to make dual attacks with your unarmed strike (see \pcref{Dual Attacking}).

    At 7th level, you gain a \plus1 bonus to damage with unarmed attacks.

    At 11th level, the damage bonus increases to \plus2.

    \feat{Versatile Wild Speech}{Class, Magical}
    \featpres Druid or ranger level 1, wild speech ability.
    \featben You can use your wild speech ability to communicate with animates, such as oozes and animated plants, and ordinary plants. A regular plant's sense of its surroundings is limited, so it won't be able to give (or recognize) detailed descriptions of creatures or answer questions about events outside its immediate vicinity.

    At druid or ranger level 7, you can use your wild speech ability to communicate with magical beasts.

    At druid or ranger level 11, you can use your wild speech ability to communicate with one element of the natural world: air, earth, fire or water.
    You choose which element when you gain this ability.
    A element's sense of its surroundings is limited like a plant's.
    This also allows you to communicate with extraplanar elementals of your chosen element, which are much more intelligent.

    At druid or ranger level 15, you can use your wild speech to communicate with any living creature, even with creatures that do not have languages.

    \feat{Vivimancer}{Magical, Spell}
    \featpres 1st level or higher \glossterm{Vivimancy} spell known.
    \featben You are immune to hostile \glossterm{Life} and \glossterm{Death} effects.

    At 7th level, if you know a 2nd level or higher Vivimancy spell, your Vivimancy spells are more powerful.
    Whenever you cast an Vivimancy spell, you can reduce the cost to apply one augment to the spell by one spell level.

    At 11th level, if you know a 4th level or higher Vivimancy spell, you heal hit points equal to your spellpower with Vivimancy spells at the end of every round.

    \feat{Weapon Focus}{Combat}
    Choose one weapon group.
    \featpre Proficiency with selected weapon group.
    \featben You gain an ability based on the weapon group chosen.
    \begin{itemize}
        \item Armor weapons: When you perform a shield bash, you still benefit from the shield's defense bonus.
            In addition, armor spikes no longer impose a penalty to your physical defenses.
        \item Axes: You increase your \glossterm{critical multiplier} by 1 when attacking with axes.
        \item Blades, heavy: You increase your \glossterm{critical range} by 1 when attacking with heavy blades.
        \item Blades, light: You increase your critical range by 1 when attacking with light blades.
        \item Blunt weapons: When you deal damage to a creature with a blunt weapon, it takes a \minus2 penalty to Mental defense for 2 rounds.
            This penalty is not cumulative with itself.
        \item Bows: You can arc your shots with bows, allowing you to ignore cover and treat total cover as total concealment (50\% failure chance) if you can shoot above the obstacle to reach your target.
        \item Crossbows: The time required for you to reload crossbows is reduced to a free action (for a hand or light crossbow) or a move action (for a heavy crossbow).
        \item Flexible weapons: You gain a \plus2 accuracy bonus with \glossterm{combat maneuvers} that you perform with flexible weapons.
        \item Headed weapons: You increase your critical multiplier by 1 when attacking with headed weapons.
        \item Monk weapons: You gain a \plus2 accuracy bonus with combat maneuvers that you perform with monk weapons.
        \item Polearms: You can switch grips to short haft or stop short hafting a polearm as a swift action, and you take no penalty while short hafting it.
        \item Spears: When you are charged by a creature, you can brace a spear you wield as an \glossterm{immediate action}. The spear must have the Bracing property (see \pcref{Weapon Properties}).
        \item Thrown weapons: You can defend yourself with thrown weapons as you throw them, preventing you from being \defenseless (see \pcref{Thrown Weapons in Melee}).
    \end{itemize}

    At 3rd level, you gain proficiency with exotic weapons from your chosen weapon group.

    At 7th level, you gain a \plus1 bonus to damage on physical attacks with weapons from your chosen weapon group.

    At 11th level, the damage bonus increases to \plus2.

    \parhead{Special} This feat can be taken multiple times.
    Its effects do not stack.
    Each time, you choose a different weapon group.

    \feat{Wild Control}{Class, Magical, Spell}
    \spelldesc{You become more adept at controlling your wild magic.}
    \featpres Sorcerer level 1, wild magic ability.
    \featben When you fail a wild magic roll, you may choose to suppress the magical energy released.
    If you do, neither the spell nor its miscast effect occurs.
    In addition, the time required to regain the ability to cast spells of the same level as the suppressed spell is halved.

    At 3rd level, you gain a \plus1 bonus to wild magic rolls.

    At 7th level, when you fail a wild magic roll, you may treat the spell as if it were one level lower (to a minimum of 1st level) for the purpose of determining which spell level you lose the ability to cast.
    If you are already unable to cast that level of spells, you must treat the spell as if it were its normal level.

    At 11th level, the bonus to wild magic rolls increases to \plus2.

\section{Other Feat Rules}

    \subsection{Bonus Feats}
        Some abilities grant a character bonus feats.
        Unless otherwise specified, the character must still meet any prerequisites for the feat.
        If the character does not meet the prerequisites at the time the bonus feat is granted, the character does not gain the feat.
        If the character later meets the prerequisites, the character immediately gains the benefit of the bonus feat.

        If a character gains a feat as a bonus feat that he or she has already acquired through other means, the character may select instead any other feat for which she qualifies.

    \subsection{Retraining Feats}
        At every level, your character can choose to retrain an old feat in exchange for a new feat.
        You can only retrain feats for other feats you could have acquired at the time you took the original feat.
        For example, a 6th level fighter can retrain his 2nd level fighter bonus feat for any other combat feat that he qualified for at his 2nd fighter level.
        This also means you cannot retrain feats gained through class abilities which give you a specific feat, since there were no other feats you could have taken.


\chapter{Description}

\section{Alignment}\label{Alignment}
    A creature's general moral and personal attitudes are represented by its alignment: lawful good, neutral good, chaotic good, lawful neutral, neutral, chaotic neutral, lawful evil, neutral evil, or chaotic evil.

    Alignment is a tool for developing your identity.
    It is not a straitjacket for restricting your actions.
    Each alignment represents a broad range of personality types or personal philosophies, so two characters of the same alignment can still be quite different from each other.
    In addition, few people are completely consistent.

    \subsection{Good vs. Evil}
        Good characters and creatures protect innocent life. Evil characters and creatures debase or destroy innocent life, whether for fun or profit.

        ``Good'' implies altruism, respect for life, and a concern for the dignity of sentient beings. Good characters make personal sacrifices to help others.

        ``Evil'' implies selfishness and a willingness to hurt or kill others. Some evil creatures simply have no compassion for others and kill without qualms if doing so is convenient. Others actively pursue evil, killing for sport or out of duty to some evil deity or master.

        People who are neutral with respect to good and evil have compunctions against killing the innocent but lack the commitment to make sacrifices to protect or help others. Neutral people are committed to others by personal relationships.

        Being good or evil can be a conscious choice. For most people, though, being good or evil is an attitude that one recognizes but does not choose. Being neutral on the good-evil axis usually represents a lack of commitment one way or the other, but for some it represents a positive commitment to a balanced view. While acknowledging that good and evil are objective states, not just opinions, these folk maintain that a balance between the two is the proper place for people, or at least for them.

        Animals and other creatures incapable of moral action are neutral rather than good or evil. Even deadly vipers and tigers that eat people are neutral because they lack the capacity for morally right or wrong behavior.

    \subsection{Law vs. Chaos}
        Lawful characters tell the truth, keep their word, respect authority, honor tradition, and judge those who fall short of their duties.

        Chaotic characters follow their consciences, resent being told what to do, favor new ideas over tradition, and do what they promise if they feel like it.

        ``Law'' implies honor, trustworthiness, obedience to authority, and reliability. On the downside, lawfulness can include close-mindedness, reactionary adherence to tradition, judgmentalness, and a lack of adaptability. Those who consciously promote lawfulness say that only lawful behavior creates a society in which people can depend on each other and make the right decisions in full confidence that others will act as they should.

        ``Chaos'' implies freedom, adaptability, and flexibility. On the downside, chaos can include recklessness, resentment toward legitimate authority, arbitrary actions, and irresponsibility. Those who promote chaotic behavior say that only unfettered personal freedom allows people to express themselves fully and lets society benefit from the potential that its individuals have within them.

        Someone who is neutral with respect to law and chaos has a normal respect for authority and feels neither a compulsion to obey nor a compulsion to rebel. She is honest but can be tempted into lying or deceiving others.

        Devotion to law or chaos may be a conscious choice, but more often it is a personality trait that is recognized rather than being chosen. Neutrality on the lawful-chaotic axis is usually simply a middle state, a state of not feeling compelled toward one side or the other. Some few such neutrals, however, espouse neutrality as superior to law or chaos, regarding each as an extreme with its own blind spots and drawbacks.

        Animals and other creatures incapable of moral action are neutral. Dogs may be obedient and cats free-spirited, but they do not have the moral capacity to be truly lawful or chaotic.

    \subsection{The Nine Alignments}
        Nine distinct alignments define all the possible combinations of the lawful-chaotic axis with the good-evil axis. Each alignment description below depicts a typical character of that alignment. Remember that individuals vary from this norm, and that a given character may act more or less in accord with his or her alignment from day to day. Use these descriptions as guidelines, not as scripts.

        The first six alignments, lawful good through chaotic neutral, are the standard alignments for player characters. The three evil alignments are for monsters and villains.

        \parhead{Lawful Good, ``Crusader''} A lawful good character acts as a good person is expected or required to act. She combines a commitment to oppose evil with the discipline to fight relentlessly. She tells the truth, keeps her word, helps those in need, and speaks out against injustice. A lawful good character hates to see the guilty go unpunished.

        Lawful good is the best alignment you can be because it combines honor and compassion.

        \parhead{Neutral Good, ``Benefactor''} A neutral good character does the best that a good person can do. He is devoted to helping others. He works with kings and magistrates but does not feel beholden to them.

        Neutral good is the best alignment you can be because it means doing what is good without bias for or against order.

        \parhead{Chaotic Good, ``Rebel''} A chaotic good character acts as his conscience directs him with little regard for what others expect of him. He makes his own way, but he's kind and benevolent. He believes in goodness and right but has little use for laws and regulations. He hates it when people try to intimidate others and tell them what to do. He follows his own moral compass, which, although good, may not agree with that of society.

        Chaotic good is the best alignment you can be because it combines a good heart with a free spirit.

        \parhead{Lawful Neutral, ``Judge''} A lawful neutral character acts as law, tradition, or a personal code directs her. Order and organization are paramount to her. She may believe in personal order and live by a code or standard, or she may believe in order for all and favor a strong, organized government.

        Lawful neutral is the best alignment you can be because it means you are reliable and honorable without being a zealot.

        \parhead{Neutral, ``Undecided''} A neutral character does what seems to be a good idea. She doesn't feel strongly one way or the other when it comes to good vs.\ evil or law vs.\ chaos. Most neutral characters exhibit a lack of conviction or bias rather than a commitment to neutrality. Such a character thinks of good as better than evil -- after all, she would rather have good neighbors and rulers than evil ones. Still, she's not personally committed to upholding good in any abstract or universal way.

        Some neutral characters, on the other hand, commit themselves philosophically to neutrality. They see good, evil, law, and chaos as prejudices and dangerous extremes. They advocate the middle way of neutrality as the best, most balanced road in the long run.

        Neutral is the best alignment you can be because it means you act naturally, without prejudice or compulsion.

        \parhead{Chaotic Neutral, ``Free Spirit''} A chaotic neutral character follows his whims. He is an individualist first and last. He values his own liberty but doesn't strive to protect others' freedom. He avoids authority, resents restrictions, and challenges traditions. A chaotic neutral character does not intentionally disrupt organizations as part of a campaign of anarchy. To do so, he would have to be motivated either by good (and a desire to liberate others) or evil (and a desire to make those different from himself suffer). A chaotic neutral character may be unpredictable, but his behavior is not totally random. He is not as likely to jump off a bridge as to cross it.

        Chaotic neutral is the best alignment you can be because it represents true freedom from both society's restrictions and a do-gooder's zeal.

        \parhead{Lawful Evil, ``Dominator''} A lawful evil villain methodically takes what he wants within the limits of his code of conduct without regard for whom it hurts. He cares about tradition, loyalty, and order but not about freedom, dignity, or life. He plays by the rules but without mercy or compassion. He is comfortable in a hierarchy and would like to rule, but is willing to serve. He condemns others not according to their actions but according to species, religion, homeland, or social rank. He is loath to break laws or promises.

        This reluctance comes partly from his nature and partly because he depends on order to protect himself from those who oppose him on moral grounds. Some lawful evil villains have particular taboos, such as not killing in cold blood (but having underlings do it) or not letting children come to harm (if it can be helped). They imagine that these compunctions put them above unprincipled villains.

        Some lawful evil people and creatures commit themselves to evil with a zeal like that of a crusader committed to good. Beyond being willing to hurt others for their own ends, they take pleasure in spreading evil as an end unto itself. They may also see doing evil as part of a duty to an evil deity or master.

        Lawful evil is sometimes called ``diabolical,'' because devils are the epitome of lawful evil.

        Lawful evil is the most dangerous alignment because it represents methodical, intentional, and frequently successful evil.

        \parhead{Neutral Evil, ``Malefactor''} A neutral evil villain does whatever she can get away with. She is out for herself, pure and simple. She sheds no tears for those she kills, whether for profit, sport, or convenience. She has no love of order and holds no illusion that following laws, traditions, or codes would make her any better or more noble. On the other hand, she doesn't have the restless nature or love of conflict that a chaotic evil villain has.

        Some neutral evil villains hold up evil as an ideal, committing evil for its own sake. Most often, such villains are devoted to evil deities or secret societies.

        Neutral evil is the most dangerous alignment because it represents pure evil without honor and without variation.

        \parhead{Chaotic Evil, ``Destroyer''} A chaotic evil character does whatever his greed, hatred, and lust for destruction drive him to do. He is hot-tempered, vicious, arbitrarily violent, and unpredictable. If he is simply out for whatever he can get, he is ruthless and brutal. If he is committed to the spread of evil and chaos, he is even worse. Thankfully, his plans are haphazard, and any groups he joins or forms are poorly organized. Typically, chaotic evil people can be made to work together only by force, and their leader lasts only as long as he can thwart attempts to topple or assassinate him.

        Chaotic evil is sometimes called ``demonic'' because demons are the epitome of chaotic evil.

        Chaotic evil is the most dangerous alignment because it represents the destruction not only of beauty and life but also of the order on which beauty and life depend.

\section{Vital Statistics}

    \subsection{Age}
        You can choose or randomly generate your age.
        If you choose it, it must be at least the minimum age for your species and class (see \trefnp{Random Starting Ages}). Your minimum starting age is the adulthood age of your species plus the number of dice indicated in the entry corresponding to the character's species and class on \trefnp{Random Starting Ages}.

        Alternatively, refer to \trefnp{Random Starting Ages} and roll dice to determine how old you are.

        \begin{dtable}
            \lcaption{Random Starting Ages}
            \begin{dtabularx}{\columnwidth}{l c *{3}{>{\ccol}X}}
                % TODO: add spellwarped back once they are implemented
                \tb{Species} & \tb{Adulthood} & \tb{Barbarian Rogue} & \tb{Fighter Mage Paladin Ranger} & \tb{Cleric Druid Monk} \\
                \bottomrule
                Human    & 15 years  & \plus1d4 & \plus1d6 & \plus2d6  \\
                Dwarf    & 40 years  & \plus3d6 & \plus5d6 & \plus7d6  \\
                Elf      & 110 years & \plus4d6 & \plus6d6 & \plus10d6 \\
                Gnome    & 40 years  & \plus4d6 & \plus6d6 & \plus9d6  \\
                Half-elf & 20 years  & \plus1d6 & \plus2d6 & \plus3d6  \\
                Half-orc & 14 years  & \plus1d4 & \plus1d6 & \plus2d6  \\
                Halfling & 20 years  & \plus2d4 & \plus3d6 & \plus4d6  \\
            \end{dtabularx}
        \end{dtable}

        With age, your \glossterm{checks} based on physical attributes decrease and your checks based on mental attributes increase (see \trefnp{Aging Effects}).

        When you reach venerable age, the GM secretly rolls your maximum age, which is the number from the Venerable column on \trefnp{Aging Effects} plus the result of the dice roll indicated on the Maximum Age column on that table.
        They record the result.
        If you reach your maximum age, you die of old age at some time during the following year.

        The maximum ages are for player characters. Most people in the world at large die from pestilence, accidents, infections, or violence before getting to venerable age.

        \begin{dtable}
            \lcaption{Aging Effects}
            \begin{dtabularx}{\columnwidth}{l *{4}{>{\ccol}X}}
                \tb{Species}  & \tb{Middle Age\fn{1}} & \tb{Old\fn{2}} & \tb{Venerable\fn{3}} & \tb{Maximum Age} \\
                \bottomrule
                Human    & 35 years  & 53 years  & 70 years  & \plus4d10 years \\
                Dwarf    & 125 years & 188 years & 250 years & \plus2d\% years \\
                Elf      & 175 years & 263 years & 350 years & \plus4d\% years \\
                Gnome    & 100 years & 150 years & 200 years & \plus3d\% years \\
                Half-elf & 62 years  & 93 years  & 125 years & \plus6d10 years \\
                Half-orc & 30 years  & 45 years  & 60 years  & \plus2d10 years \\
                Halfling & 50 years  & 75 years  & 100 years & \plus1d\% years \\
            \end{dtabularx}
            1 At middle age, \minus1 to \glossterm{checks} based on Str, Dex, and Con; \plus1 to \glossterm{checks} based on Int, Per, and Wil. \\
            2 At old age, the aging modifiers change to \minus2 and \plus2.
            2 At venerable age, the aging modifiers change to \minus3 and \plus3.
        \end{dtable}

    \subsection{Height and Weight}
        The dice roll given in the Height Modifier column determines the character's extra height beyond the base height. That same number multiplied by the dice roll or quantity given in the Weight Modifier column determines the character's extra weight beyond the base weight.

        \begin{dtable}
            \lcaption{Random Height and Weight}
            \begin{dtabularx}{\columnwidth}{l *{3}{>{\lcol}X} >{\lcol}p{5em}}
                \tb{Species} & \tb{Base Height} & \tb{Height Modifier} & \tb{Base Weight} & \tb{Weight Modifier} \\
                \bottomrule
                Human, male      & 4' 10'' & \plus2d10 & 120 lb. & \x (2d4) lb. \\
                Human, female    & 4' 5''  & \plus2d10 & 85 lb.  & \x (2d4) lb. \\
                Dwarf, male      & 3' 9''  & \plus2d4  & 130 lb. & \x (2d6) lb. \\
                Dwarf, female    & 3' 7''  & \plus2d4  & 100 lb. & \x (2d6) lb. \\
                Elf, male        & 4' 5''  & \plus2d6  & 85 lb.  & \x (1d6) lb. \\
                Elf, female      & 4' 5''  & \plus2d6  & 80 lb.  & \x (1d6) lb. \\
                Gnome, male      & 3' 0''  & \plus2d4  & 40 lb.  & \x 1 lb.     \\
                Gnome, female    & 2' 10'' & \plus2d4  & 35 lb.  & \x 1 lb.     \\
                Half-elf, male   & 4' 7''  & \plus2d8  & 100 lb. & \x (2d4) lb. \\
                Half-elf, female & 4' 5''  & \plus2d8  & 80 lb.  & \x (2d4) lb. \\
                Half-orc, male   & 5' 0''  & \plus2d10 & 150 lb. & \x (2d6) lb. \\
                Half-orc, female & 4' 8''  & \plus2d10 & 110 lb. & \x (2d6) lb. \\
                Halfling, male   & 2' 8''  & \plus2d4  & 30 lb.  & \x 1 lb.     \\
                Halfling, female & 2' 6''  & \plus2d4  & 25 lb.  & \x 1 lb.
            \end{dtabularx}
        \end{dtable}

\section{Languages}\label{Languages}

    \parhead{Literacy}
    All characters with an Intelligence of \minus2 or higher are presumed to be literate, allowing them to read and write any language they speak. Each language has an alphabet, though sometimes several spoken languages share a single alphabet.

    \parhead{Language Rarity}
    Some languages are widely spoken in the world, while others are only encountered in unusual circumstances.
    Common languages are summarized on \trefnp{Common Languages}, below.
    Rare languages are summarized on \trefnp{Rare Languages}, below.
    Rare languages are more difficult to learn (see \pcref{Learning Languages}).

    \begin{dtable}
        \lcaption{Common Languages}
        \begin{dtabularx}{\columnwidth}{l >{\lcol}X l}
            \tb{Language}  & \tb{Typical Speakers}  & \tb{Alphabet} \\
            \bottomrule
            Common   & Civilized creatures & Common   \\
            Draconic & Dragons, kobolds    & Draconic \\
            Dwarven  & Dwarves             & Dwarven  \\
            Elven    & Elves               & Elven    \\
            Giant    & Ogres, giants       & Dwarven  \\
            Gnoll    & Gnolls              & Common   \\
            Gnome    & Gnomes              & Dwarven  \\
            Goblin   & Goblins, hobgoblins & Dwarven  \\
            Halfling & Halflings           & Common   \\
            Orc      & Orcs                & Dwarven  \\
        \end{dtabularx}
    \end{dtable}

    \begin{dtable}
        \lcaption{Rare Languages}
        \begin{dtabularx}{\columnwidth}{l >{\lcol}X l}
            \tb{Language}  & \tb{Typical Speakers}  & \tb{Alphabet} \\
            \bottomrule
            Abyssal     & Demons, chaotic evil outsiders & Infernal  \\
            Aquan       & Water-based creatures          & Elemental \\
            Auran       & Air-based creatures            & Elemental \\
            Celestial   & Good outsiders                 & Celestial \\
            Ignan       & Fire-based creatures           & Elemental \\
            Infernal    & Devils, lawful evil outsiders  & Infernal  \\
            Sylvan      & Dryads, faeries                & Elven     \\
            Terran      & Earth-based creatures          & Elemental \\
            Undercommon & Drow                           & Elven
        \end{dtabularx}
    \end{dtable}

% Is this the right location?
\section{Planes}\label{Planes}
    The universe of Rise is divided into \glossterm{planes}.
    A plane is a distinct realm of existence.
    Except for the connections between planes through \glossterm{planar rifts}, each plane is effectively an isolated universe, and different planes can obey different fundamental laws.
    For example, the Material Plane has gravity that exerts a consistent acceleration in a single absolute direction.
    However, the Astral Plane has subjective gravity, where each creature on the plane chooses the direction that gravity pulls it in, if any.

    \subsection{General Cosmology}
        The planes of Rise are divided up into groups.

        \parhead{Inner Planes} These six planes are manifestations of the basic building blocks of the universe.
        Each plane in this group is predominantly composed of a single element or type of energy.

        \parhead{Outer Planes} These nine planes are manifestations of the nine alignments that define the morality of the universe.
        Each plane in this group is strongly associated with a particular alignment.
        The souls of creatures with the corresponding alignment often spend their afterlife in the Outer Planes.

        \parhead{Nexus Planes} These three planes are composite planes with a number of distinct environments and filled with creatures of myriad alignments.
        These planes comprise the majority of civilization across all planes.

        \parhead{Demiplanes} These planes are small, fragmentary realms that are greatly limited in their scope.
        There is no specific list of demiplanes, and they share few common properties.
        Most demiplanes were created for particular purposes by beings of great power, though some simply came into existence through unknown means.

    \subsection{Planar Rifts}
        Normally, there are boundaries between different planes that prevent direct passage between them.
        However, \glossterm{planar rifts} are places where these boundaries have weakened, making interplanar travel easier.
        A planar rift joins a specific location on one plane to a specific location on a different plane.
        Most planar rifts lead to and from the Astral Plane, which is the space between the other planes (see \pcref{The Astral Plane}).

        Most planar rifts still require the use of magic, such as the \spell{plane shift} ritual, to actually cross between planes.
        Some especially large rifts enable physical travel between planes without the use of any magic.

    \subsection{Planar Traits}
        \subsubsection{Gravity Direction}
            The direction of gravity on a plane can take one of the following forms:
            \begin{itemize}
                \item Fixed Gravity: Gravity points in a fixed direction and with a fixed strength at all locations on the plane.
                \item Absolute Directional Gravity: Gravity points in a consistent direction according to a rule that applies equally to everything on the plane, but which is not in a fixed direction.
                    For example, a plane filled with floating spheres where gravity always points towards the closest sphere has absolute directional gravity.
                \item Subjective Gravity: Each creature on the plane chooses the direction of gravity for that creature.
                    The plane has no gravity for unattended objects and nonsentient creatures.
                    A creature on the plane can make use the \textit{control gravity} ability as a \glossterm{minor action}.
                    \begin{freeability}{Control Gravity}
                        Make a DR 10 Willpower check.
                        Success means that you choose the direction of gravity that applies to you on the current plane.
                        Alternately, you can choose for gravity to not apply to you.

                        Failure means you gain a \plus2 bonus to the next \textit{control gravity} ability you use on this plane.
                        This bonus stacks with itself and lasts until you succeed at a \textit{control gravity} ability on this plane.
                    \end{freeability}
            \end{itemize}

        \subsubsection{Gravity Strength} The strength of gravity on a plane can take one of the following forms:
            \begin{itemize}
                \item Normal Gravity: Gravity is about the strength of Earth.
                \item No Gravity: There is no gravity on the plane.
                    The \glossterm{range increment} of ranged weapons is tripled.
                    % TODO: what additional effects are there?
                \item Light Gravity: Gravity is about half the strength of Earth.
                    % Should there be check penalties?
                    The weight of all items is halved.
                    The \glossterm{range increment} of ranged weapons is doubled.
                \item Heavy Gravity: Gravity is about twice the strength of Earth.
                    Creatures take a \minus2 penalty to Strength and Dexterity-based checks.
                    The weight of all items is doubled.
                    The \glossterm{range increment} of ranged weapons is halved, to a minimum of 5 feet.
                    % TODO: falling damage
                \item Extreme Gravity: Gravity is about four times the strength of Earth.
                    Creatures take a \minus4 penalty to Strength and Dexterity-based checks.
                    The weight of all items is quadrupled.
                    The \glossterm{range increment} of ranged weapons is one quarter of the normal value, to a minimum of 5 feet.
                    % TODO: falling damage
            \end{itemize}

        \subsubsection{Planar Connectivity}
            Different planes have different degrees of connection to other planes.
            \begin{itemize}
                \item Isolated: The plane is difficult to reach or leave.
                    It has no permanent \glossterm{planar rifts}, and temporary rifts are rare or nonexistent.
                \item Stable Connected: The plane has multiple permanent \glossterm{planar rifts}.
                    However, temporary rifts are rare.
                \item Unstable Connected: The plane has no permanent \glossterm{planar rifts}, but temporary rifts are common.
                \item Conduit: The plane has a large number of permanent \glossterm{planar rifts}, and temporary rifts are common.
            \end{itemize}

            % TODO: size, shape, morphicness, alignment, magic

    \subsection{Plane Descriptions}
        \subsubsection{The Material Plane}
            The Material Plane is the plane that most Rise adventures begin on.
            It is the most familiar to most humanoid creatures.
            It has the following planar traits:
            \begin{itemize}
                \item Gravity direction: Fixed
                \item Gravity strength: Normal
                \item Planar connectivity: Isolated
            \end{itemize}

        \subsubsection{The Astral Plane}\label{The Astral Plane}
            The Astral Plane is the space between the other planes.
            It is a necessary intermediate destination on most planar journeys, as the vast majority of \glossterm{planar rifts} lead to and from the Astral Plane.
            It has the following planar traits:
            \begin{itemize}
                \item Directional gravity: Subjective
                \item Gravity strength: Normal
                \item Planar connectivity: Conduit
            \end{itemize}

            Most activity on the Astral Plane occurs in a space called the Inner Astral Plane, a massive region where almost all planar rifts on the plane appear.
            % Are there any other infinite planes?
            However, unlike all other planes, the Astral Plane has no known limits to its extent, and may in fact be infinite.
            The rest of the plane is known as the Deep Astral Plane, and few venture into those sparsely populated realms.
            % This should be more clearly defined
            The Deep Astral Plane has magical turbulence that interferes with long-range communication and transportation magic, making exploration difficult.


\chapter{Items and Equipment}

This chapter defines the items and equipment that exist in the universe of Rise, including both magical and nonmagical items.

\section{Wealth and Item Ranks}\label{Wealth}\label{Wealth and Item Ranks}

  The worth of an item can be measured with money, or with the more abstract concept of an item's rank.
  Both measurements are closely connected.
  In general, gold pieces are a more useful concept at low levels, and item rank are more useful at high levels.
  However, both concepts function at any level, so you can use whichever makes more sense in a particular game.

  \subsection{Coins}
    The most common coin is the gold piece (gp). A gold piece is worth 10 silver pieces. Each silver piece is worth 10 copper pieces (cp).
    In addition to copper, silver, and gold coins, there are also platinum pieces (pp), which are each worth 10 gp.

    The standard coin weighs about a third of an ounce (fifty to the pound).

    \begin{dtable}
      \lcaption{Coin Exchange Values}
      \begin{dtabularx}{\columnwidth}{l c *{4}{>{\ccol}X}}
        &   & \tb{CP} & \tb{SP} & \tb{GP} & \tb{PP} \tableheaderrule
        Copper piece (cp)   & = & 1       & 1/10    & 1/100   & 1/1,000 \\
        Silver piece (sp)   & = & 10      & 1       & 1/10    & 1/100   \\
        Gold piece (gp)     & = & 100     & 10      & 1       & 1/10    \\
        Platinum piece (pp) & = & 1,000   & 100     & 10      & 1
      \end{dtabularx}
    \end{dtable}

  \subsection{Item Ranks}\label{Item Ranks}

    Each item has a rank associated with it.
    An item's rank is generally correlated with the item's effectiveness, rarity, and value.
    A magic item's \glossterm{power} is equal to twice its rank.
    These effects are summarized in \trefnp{Item Ranks}.
    In general, five items of a given rank are worth the same as a single item that is one rank higher.

    \parhead{Permanent Items and Consumables}
    Long-term items that are expected to be worn or otherwise used repeatedly are more expensive than items that are destroyed immediately after being used.
    Although consumable items are cheaper, they still use their full rank for all other purposes, such as their power and how difficult they are to buy.
    In general, five consumable items of a given rank are worth as much as a single non-consumable item of that rank.

    \begin{columntable}
      \begin{dtabularx}{\columnwidth}{l l X X}
        \tb{Rank} & \tb{Power} & \tb{Typical Permanent Item Price} & \tb{Typical Consumable Price} \tableheaderrule
        0         & 0          & 10 gp or less                     & 2 gp or less \\
        1         & 2          & 40 gp                             & 10 gp        \\
        2         & 4          & 200 gp                            & 40 gp        \\
        3         & 6          & 1,000 gp                          & 200 gp       \\
        4         & 8          & 5,000 gp                          & 1,000 gp     \\
        5         & 10         & 25,000 gp                         & 5,000 gp     \\
        6         & 12         & 125,000 gp                        & 25,000 gp    \\
        7         & 14         & 625,000 gp                        & 125,000 gp   \\
      \end{dtabularx}
    \end{columntable}

    Items with a rank of 0 or 1 may be found among common folk, though few commoners would have more than one magic item of any value.
    Items with a rank of 2 or higher are usually only owned or used by nobility, wealthy merchants, and adventurers.

  \subsection{Buying and Selling Items}
    Items of any rank can be exchanged for other items based on their rank.
    In general, items with an rank of 3 or less can be bought or sold in exchange for gold pieces.
    Items with an rank of 4 or higher are exceptionally rare.
    The monetary value of such items is so exorbitant that they are almost never purchased or sold with gold pieces.
    Instead, they are typically exchanged for similarly rare magic items or gems.

    Wandering adventurers typically have a limited time frame to sell their items.
    It can be difficult to find a buyer for valuable items on short notice, so they must accept lower prices than merchants can charge.
    When selling for gold pieces, you can expect to receive a fifth of the item's typical value according to its rank.
    When selling for another item, you can expect to receive an item or trade good of one rank lower in exchange.
    These ratios can be negotiated, and favorably disposed merchants or nobles may give better deals.

    \subsubsection{Trade Goods}
      Some items are considered trade goods.
      Trade goods have a widely agreed upon value, but no intrinsic use.
      Gold pieces and gems are examples of trade goods.
      Trade goods differ from other items in that even adventurers can typically receive their full value when selling them to established merchants.
      Some common trade goods are detailed in \tref{Trade Goods}.

      \begin{dtable}
        \lcaption{Trade Goods}
        \begin{dtabularx}{\columnwidth}{l >{\lcol}X}
          \tb{Cost} & \tb{Item} \tableheaderrule
          1 cp & One pound of wheat \\
          2 cp & One pound of flour \\
          1 sp & One pound of iron, or one chicken \\
          5 sp & One pound of tobacco or copper \\
          1 gp & One pound of cinnamon, or one goat \\
          2 gp & One pound of ginger or pepper, or one sheep \\
          3 gp & One pig \\
          4 gp & One square yard of linen \\
          5 gp & One pound of salt or silver \\
          10 gp & One square yard of silk, or one cow \\
          15 gp & One pound of saffron or cloves, or one ox \\
          50 gp & One pound of gold \\
          500 gp & One pound of platinum
        \end{dtabularx}
      \end{dtable}

  \subsection{Typical Wealth Acquisition}
    A typical character finds one non-consumable item appropriate for them per level.
    That item would have a rank equal to that character's highest rank at the time.
    For example, a typical 5th level character would have five items: two rank 2 items, and 3 rank 1 items.
    In addition, characters typically find several consumable items per level that are appropriate to their rank.

    Over time, lower rank items stop being useful, so most characters use no more than five or six different items at a time, plus various consumables.
    A typical 20th level character does not carry around 20 different items.
    However, the lower level items are essentially irrelevant from the perspective of calculating wealth, so the ``one item per level'' guideline is still useful.

    This is a drastic simplification of the sometimes messy process of accumulating wealth and magic items over the course of a typical campaign.
    Characters will often find additional items that they have no immediate use for.
    Players may go several sessions without acquiring any particular items until they complete their current quest, which may reward them with a large number of items at once.
    This is all fine, and the GM should not feel compelled to keep item acquisition perfectly on rails.
    Rise is only loosely balanced around this general pace of item acquisition, and it is not hard - or always necessary - to adjust encounters to deal with unusually wealthy or poor characters.

\section{Using Items}

  \subsection{Manipulating Objects}\label{Manipulating Objects}
    There are two ways to determine how difficult an object is to interact with: its weight, and how it is contained.
    Use the slowest action type from among the both methods.
    You can only manipulate objects as a free action once per round.
    However, you can use a minor action or a standard action in place of a free action for this purpose, allowing you to manipulate up to three objects per round.

    These object manipulation rules are intended to cover interactions that take some amount of time and effort, such as drawing a weapon or opening a door.
    Simply dropping an object in your hands is so trivial that it does not count against your one free action object manipulation per round.
    Other similarly simple interactions may not count against that limit at the GM's discretion.

    Some objects have special rules that indicate how much time they take to interact with, such as \weapontag{Heavy} weapons.
    Those rules override these general guidelines.

    \begin{raggeditemize}
      \itemhead{Containment} If the item is freely accessible, interacting with that item is a \glossterm{free action}.
        Drawing or sheathing an item from a dedicated container for that item, such as a weapon sheath, is normally a \glossterm{minor action}.
        Withdrawing an item from a disorganized heap, such as a bag or backpack, requires a \glossterm{standard action}.
      \itemhead{Weight} If an item's weight category is no heavier than your \glossterm{carrying capacity}, interacting with that item is a \glossterm{free action} (see \pcref{Weight Limits}).
        Pushing or dragging something that exceeds your carrying capacity requires a \glossterm{standard action}.
    \end{raggeditemize}

  \subsection{Moving Items Between Hands}\label{Moving Items Between Hands}
    In general, you can move weapons and similar handheld objects between your hands freely, as long as you are strong enough to hold their combined weight in one hand.
    For example, if you were holding two short swords, you could quickly hold them both in one hand to open a door or cast a spell, and then return to holding one in each hand.
    Likewise, you can reload a longbow without worrying about when you are holding the longbow (and string) with both hands or holding it in one hand while retrieving an arrow from a quiver with the other hand.
    Since shields are more cumbersome to don and remove, this does not allow you to use your shield hand as a free hand in the same way.

  \subsection{Storing Items}\label{Storing Items}
    A character can only store a limited number of weapons in locations that are easy to access in combat.
    Generally, a humanoid creature can carry no more than five ordinary weapons or shields on their body.
    For each additional weapon or similar item stored in a convenient location, you increase your \glossterm{encumbrance} by 1.
    Items carried outside of easy reach in combat, such as in a backpack, are ignored for this purpose.

    A Heavy weapon takes up twice the space of an ordinary weapon, and a Light weapon takes up half that space.
    Compact weapons do not take up a meaningful amount of space, and can be ignored for this purpose.
    Ammunition is typically stored in a quiver or pouch.
    Treat ammunition storage as a separate Light item that can hold an unlimited quantity of any ammunition other than lances.

    For example, you could carry up to ten Light weapons, four normal weapons and a standard shield, or two Heavy weapons plus a normal weapon.

  \subsection{Magic Item Activation}

    Some magic items have to be explicitly activated to have unusual effects.
    For example, the \mitem{seven league boots} can be activated to teleport you across great distances.
    Other magic items constantly have magical effects.
    For example, a \mitem{ring of protection} passively grants you a defense bonus.

    The description of a magic item effect will specify what mechanical actions must be taken, if any, to activate the effects of the item.
    For example, a belt of healing requires a \glossterm{standard action} to activate.
    However, the item description will not specify the exact nature of the action.
    Different items, even if they have the same effect, can have different physical actions that are required to activate the item.
    These activation actions can come in one of the following forms:
    \begin{itemize}
      \item Command word: You must speak a specific word that the item will hear and react to.
        For example, you may need to say the word ``healing'' in Elven to activate an item that heals you.
      \item Mental command: You must mentally direct the item to activate, such as by visualizing the item or thinking a particular word.
        % TODO: does this item exist
        For example, you may need to imagine a warm blanket around you to activate an item that protects you from cold damage or environmental effects.
      \item Physical motion: You must perform a specific physical motion, usually involving the item in some way.
        For example, you may need to rapidly stomp one foot on the ground to activate an item that allows you to move faster.
    \end{itemize}

    % TODO: table of random item activations?

  \subsection{Magic Item Limitations}

    There are three restrictions on your ability to use magic items.
    First, you cannot equip two apparel items that take up the same physical location on your body.
    For example, you cannot equip two different gauntlet sets and gain the effects of both, but you could equip several amulets or up to ten rings.

    Second, all magic items require you to attune to them to gain their effect unless they indicate otherwise in their description.
    You can attune to a magic item with the \textit{item attunement} ability, below.

    Third, you cannot attune to two items with the same name, or if one is simply an upgrade of another one.

    \subsubsection{Item Attunement}\label{Item Attunement}

      As a standard action, you can use the \textit{item attunement} ability to attune to items.
      This is a \magical ability.

      \begin{attuneability}{Item Attunement}{\abilitytag{Attune}}
        \abilityusagetime Standard action.
        \rankline
        Choose a magic item you are touching.
        Any abilities the target has that require attunement become active, allowing you to use its full potential.
      \end{attuneability}

      \parhead{Shared Item Attunement} Multiple creatures can attune to the same item simultaneously.
      Since most items only function while worn or wielded, this does not usually allow multiple creatures to gain the benefits of the item.
      However, the creatures can swap the item between them without having to reattune to it each time.

    \subsubsection{Magic Item Power}\label{Magic Item Power}
      The \glossterm{power} of an item is equal to twice its rank.
      An item's power also affects its defenses.
      Its Fortitude and Mental defenses are equal to 5 \add its \glossterm{power}.
      Its Armor defense and Reflex defense are not affected by its \glossterm{power}, and are solely determined by its size and shape.

  \subsection{Removing Magic Items}
    Unless otherwise noted, magic items that have effects on the creature using the item must continue to be worn or held as long as the effect lasts.
    If a magic item has an ability with a duration, removing the item also ends the ability.
    Items which are consumed when used or which do not affect their user are unaffected by this rule.

\newpage
\sectiongraphic*{Weapons}{width=\columnwidth}{equipment/weapons}

  Each weapon has a \glossterm{weapon group} and any number of \glossterm{weapon tags}.
  In addition, each weapon has a particular \glossterm{accuracy} modifier and defines a base \glossterm{dice pool} for attacks using that weapon.
  This section explains each of those concepts and defines the statistics for weapons in Rise.
  You gain a bonus to your \glossterm{weapon damage} equal to half your \glossterm{power} (see \pcref{Weapon Damage}).

  Unless otherwise specified, a weapon must be held in a single \glossterm{free hand}.
  You can use two hands to hold a weapon if you want, but that provides no special benefit unless that weapon has the \weapontag{Versatile Grip} tag (see \pcref{Weapon Tags}).

  \subsection{Weapon Groups}\label{Weapon Groups}
    Weapons are organized into thematically related categories called weapon groups. They are described in \trefnp{Weapon Groups}. For example, all axes belong to the ``axes'' weapon group. Some weapons can be found in multiple weapon groups. For example, a dagger is a simple weapon, a blade, and a thrown weapon.

    \parhead{Exotic Weapons}\nonsectionlabel{Exotic Weapons} Some weapons are rare and have unusual fighting styles.
    These weapons are called exotic weapons.
    Although many characters are proficient with non-exotic weapons, proficiency with exotic weapons is rare.
    Some specific class abilities grant proficiency with exotic weapons.

    \begin{dtable!*}
      \lcaption{Weapon Groups}
      \begin{dtabularx}{\textwidth}{l >{\lcol}X >{\lcol}X}
        \tb{Group}         & \tb{Weapons}                                                                & \tb{Exotic Weapons} \tableheaderrule
        Armor weapons      & Standard shield, spiked shield                                              & Armblade, spiked knee                                            \\
        Axes               & Battleaxe, greataxe, handaxe, poleaxe, shepherd's axe, throwing axe         & Dwarven throwing axe, dwarven waraxe, orcish greataxe            \\
        Blades             & Broadsword, dagger, greatsword, rapier, scimitar, smallsword                & Boot dagger, falchion, katana, kukri                             \\
        Bows               & Longbow, shortbow                                                           & Flatbow, heartseeker arrows, recurve bow, takedown bow, titanbow \\
        Club-like weapons  & Club, greatmace, mace, morning star, sap, torch                             & Culacula, gnomish trick mace, knobkerrie, totokia                \\
        Crossbows          & Heavy crossbow, light crossbow, pellet crossbow                             & Arbalest, pistol crossbow, repeating crossbow   \\
        Flexible Weapons   & Flail, heavy flail, nunchaku, two-section staff, whip                       & Chain whip, meteor hammer, three-section staff                   \\
        Headed weapons     & Light hammer, longhammer, pick, sickle, sledgehammer, warhammer             & Dwarven longhammer, dwarven shorthammer, heavy pick, obuch       \\
        Improvised weapons & \tdash                                                                      & \tdash                                                           \\
        Monk weapons       & Jitte, kama, kunai, nunchaku, quarterstaff, shuriken, two-section staff     & Hook sword, sai, three-section staff, war fan                    \\
        Polearms           & Bardiche, glaive, halberd, longhammer, poleaxe, quarterstaff, scythe        & Fauchard, war scythe                                             \\
        Simple weapons     & Club, dagger, mace, quarterstaff                                            & \tdash                                                           \\
        Spears             & Greatspear, javelin, lance, spear, spontoon                                 & Gnomish smallspear, partisan, pike                               \\
        Thrown weapons     & Dagger, dart, handaxe, javelin, light hammer, shuriken, sling, throwing axe & Dwarven throwing axe, dwarven waraxe, net                 \\
      \end{dtabularx}
    \end{dtable!*}

    \subsubsection{Weapon Proficiency}\label{Weapon Proficiency}
      You take a \minus2 accuracy penalty with weapons you are not proficient with.

    \subsubsection{Improvised Weapons}\label{Improvised Weapons}
      Sometimes objects not crafted to be weapons nonetheless see use in combat.
      In general, treat improvised weapons as being equivalent to the non-exotic manufactured weapon that seems most similar in shape and composition.
      However, since you are not proficient with the improvised weapon, you take a \minus2 accuracy penalty with it.
      If you become proficient with improvised weapons, this accuracy penalty is removed.

    \subsubsection{Natural Weapons}\label{Natural Weapons}
      Natural weapons are weapons that are part of a creature's body instead of being manufactured and wielded.
      Many monsters have natural weapons, like claws or a bite attack.
      Natural weapons do not normally require a \glossterm{free hand} to use.
      All bipedal creatures also have two punch/kick natural weapons.

      Common natural weapons are listed in \tref{Natural Weapons}.
      In addition, some monsters and effects have unique natural weapons, such as the \spell{stonefist} spell.

      \parhead{Physical Size} Like all objects and creatures, weapons have a size category that represents how physically large they are. Most weapons are one size category smaller than their wielder, but \weapontag{Heavy} weapons are the same size category as the wielder.
      All weapons are \glossterm{lightweight} unless otherwise noted.

      \parhead{Inappropriately Sized Weapons}\nonsectionlabel{Inappropriately Sized Weapons} You can use weapons that are sized for creatures that are one size category larger or smaller than you.
      However, you take a \minus2 accuracy penalty on attacks using an inappropriately sized weapon.

  \subsection{Weapon Range Limits}\label{Weapon Range Limits}
    Ranged weapon attacks become less accurate if the target is far away.
    Ranged weapons have two \glossterm{range limits} listed, with a slash between them, such as 90/270.
    The first number indicates the maximum range for a weapon's \glossterm{close range}.
    The second number indicates the maximum range for a weapon's \glossterm{long range}.
    You cannot attack a target that is beyond a weapon's long range limit.

    Attacks at close range have no penalty.
    Attacks at long range have a \minus4 accuracy penalty.
    This is called a \glossterm{longshot penalty}, and some abilities can reduce this penalty.

  \subsection{Weapon Tags}\label{Weapon Tags}
    Some weapons found on \trefnp{Weapons} have tags that indicate that they have special abilities. The list of abilities that weapons can have is given below.
    \weapontagdef{Ammunition} When you hit with a strike using this weapon, it becomes \glossterm{broken}.
    It cannot have magic weapon properties.
    However, whenever you buy or craft this weapon, you receive multiple copies, as indicated in the table below.
    \weapontagdef{Bow} This weapon is a bow used to fire arrows.
    You need both hands to fire a bow.
    Drawing an arrow from a quiver and notching it into a bow requires one \glossterm{free hand} while holding the bow in another hand.
    This is not considered an independent action, so you can fire a bow any number of times per round.
    \weapontagdef{Clinch} You gain a \plus2 bonus to \glossterm{accuracy} on \glossterm{melee} \glossterm{strikes} using this weapon against creatures who are \grappled.
    \weapontagdef{Compact} This weapon is unusually small.
    It is one size category smaller than normal (that is, two size categories smaller than the creature it is intended for).
    This makes it easier to conceal (see \pcref{Sleight of Hand}).
    In addition, you only take a \minus5 penalty to Stealth when trying to conceal strikes with a Compact weapon instead of the normal \minus10 or \minus20 penalty for concealing a strike (see \pcref{Stealth}).
    \weapontagdef{Heavy} Heavy weapons are larger and heavier than other weapons.
    You normally need two hands to use a Heavy weapon, whether you are throwing it or using it in melee.
    Drawing a Heavy item from a sheath, or returning it to a sheath, requires a \glossterm{standard action}.

    If you have a Strength of 3 or higher, you draw or sheathe a Heavy item at the same speed as non-Heavy items.
    In addition, you can wield a Heavy weapon in one hand, but you take a \minus1 accuracy penalty with the weapon while doing so.
    This cannot be used if the weapon always requires two hands to fire, such as a titanbow.
    If your Strength is 6 or higher, this accuracy penalty is removed.

    While holding a Heavy weapon in two hands, you gain a \plus1 bonus per 3 \glossterm{power} to your \glossterm{weapon damage} with the weapon.
    This bonus is in addition to the normal bonus that you gain from your power with weapons (see \pcref{Weapon Damage}).
    \weapontagdef{Impact} You get a \glossterm{glancing blow} with this weapon if you would miss by 5 or less (see \pcref{Glancing Blows}).
    This does not allow you to get glancing blows if that would normally be impossible for you, such as if you are making a \glossterm{dual strike}.
    % Assume you have a 70% chance to hit (+0 vs AD 4).
    % +3 accuracy takes you to 1x hit damage per round.
    % +6 accuracy for crits means you crit on 8/9/10.
    % That's a 40% chance for a regular hit, a 20% chance for a glancing blow, and a 30% chance for a crit, so 1.1x hit damage per round.
    % Therefore, +2 accuracy with crits is worth about +1 global accuracy.
    \weapontagdef{Keen} You gain a \plus2 bonus to \glossterm{accuracy} with \glossterm{strikes} using this weapon for the purpose of determining whether you get a \glossterm{critical hit}.
    \weapontagdef{Light} Light weapons are smaller and easier to handle than other weapons.
    Drawing a Light item from a sheath, or returning it to a sheath, is a \glossterm{free action} rather than a minor action (see \pcref{Manipulating Objects}).
    Attacking with two weapons at once is more accurate with Light weapons (see \pcref{Dual Strikes}).
    \weapontagdef{Long}\label{Long Weapon} This weapon can be used to make melee \glossterm{strikes} against targets up to 10 feet away from you.
    If you use an ability with more specific targets than simply making a melee strike, such as affecting ``all enemies adjacent to you'', this weapon tag does not increase your range with that ability.
    \weapontagdef{Maneuverable} Whenever you use a \atBrawling ability that would normally require a free hand, you can use this weapon instead.
    When you do, you apply any accuracy bonuses that you would normally apply to \glossterm{strikes} with that weapon to the ability.
    The ability is still not a \glossterm{strike}, and this does not cause you to deal damage with the weapon if you hit with that ability.
    \weapontagdef{Massive} This weapon hits everything in a cube-shaped area.
    Attacks with it are not \glossterm{targeted}, so they are not affected by \glossterm{miss chances}.
    Unlike most area attacks, a miss with a Massive weapon does not deal half damage.

    Massive weapons have a measurement that indicates the length of each side of the cube in feet.
    For example, a net is a \abilitytag{Massive} (10) weapon, so it affects everything in a 10 foot wide cube.
    If a weapon has the Massive tag, that replaces any Sweeping tag it would otherwise have.

    Creatures that are Huge or larger automatically gain the Massive tag on all of their weapons.
    For details, see \tref{Size Categories}.
    \weapontagdef{Mounted}\label{Mounted Weapon} If you are mounted, and your mount moves in the same phase that you make a \glossterm{strike} with a Mounted weapon, you gain a \plus2 \glossterm{accuracy} bonus with the strike.
    \weapontagdef{Parrying} If a creature attacks you with a \glossterm{melee} \glossterm{strike} while you wield this weapon, you \glossterm{briefly} gain a \plus2 bonus to \glossterm{accuracy} with strikes using this weapon against that creature.
    \weapontagdef{Projectile} This weapon fires ammunition at range to deal damage.
    The ammunition generally breaks when used.
    Projectile weapons have two \glossterm{range limits} listed in their description (see \pcref{Weapon Range Limits}).
    They must be reloaded after being fired.
    The time required to reload a projectile weapon is given in the weapon description.
    You take a \minus4 accuracy penalty with Projectile weapons against creatures adjacent to you.
    While riding a moving mount, you take a \minus4 accuracy penalty with Projectile weapons, and your range limits are halved.
    \weapontagdef{Resonating} Whenever you hit and damage a creature with a strike using this weapon, it \glossterm{briefly} takes a -2 penalty to its Armor defense against your attacks.
    \weapontagdef{Subdual} This weapon deals \glossterm{subdual damage} (see \pcref{Subdual Damage}).
    % (30/60) is one improvement, (60/120) is two.
    \weapontagdef{Sweeping}\label{Sweeping} When you make a \glossterm{melee} \glossterm{strike} with this weapon, you may also target one or more secondary creatures or objects adjacent to you.
    If the weapon also has the \weapontag{Long} weapon tag, each secondary target may instead be within 10 feet of you.
    Each secondary target must be within 10 feet of a primary target, and must not already be a target of the strike.
    The strike affects each secondary target in the same way as the primary targets.
    Sweeping weapons have a number that indicates the number of secondary targets you can affect.
    \weapontagdef{Thrown} This weapon is designed to be thrown to deal damage at range.
    Thrown weapons have two \glossterm{range limits} listed in their description (see \pcref{Weapon Range Limits}).
    Unless otherwise noted in a weapon's description, a throwing weapon can be used to attack in melee without penalty.
    % Doesn't cost an improvement slot; very situational
    \weapontagdef{Versatile Grip} This weapon is designed to be held in either one hand or two hands.
    While holding a Versatile Grip weapon in two hands, you gain a \plus1 bonus per 3 \glossterm{power} to your \glossterm{weapon damage} with the weapon, with a minimum bonus of \plus1.
    This bonus is in addition to the normal bonus that you gain from your power with weapons (see \pcref{Weapon Damage}).

  \subsection{Weapon Table}
    Here is the format for weapon entries in the Weapons table, below.

    \parhead{Accuracy} This number modifies your \glossterm{accuracy} with \glossterm{strikes} using the weapon.

    \parhead{Damage} This \glossterm{dice pool} indicates the damage dealt by the weapon on a hit.

    \parhead{Item Rank (Cost)} The first value indicates the \glossterm{item rank} of the item (see \pcref{Item Ranks}).
    The second value in parentheses indicates the average cost to buy the item.
    Items crafted for unusually large or small creatures are more expensive.
    For each size category larger or smaller than Medium, the item's rank increases by one, which increases its price.

    \parhead{Weapon Tags} Some weapons have special properties. See \pcref{Weapon Tags} for details.

    % All weapons should have two improvements over a ``baseline'' +0a/1d6 weapon with no properties.
    % The Light weapon tag comes with an invisible +1a/-1d modifier, so +1a/1d4 with one extra improvement.
    % The Heavy weapon tag does not cost a tag slot and comes with an invisible +1d modifier, so +0a/1d8 with two extra improvements.
    \begin{longcolumn}
      \begin{longtablewrapper}
        \tablebookmark{Weapons}{weapons}
        \RaggedRight
        \begin{longtable}{p{12em} l l l >{\lcol}p{24em}}
          \lcaption{Weapons}                                                                                                                                         \\
          \tb{Name}                         & \tb{Accuracy} & \tb{Damage} & \tb{Item Rank (Cost)}\fn{1} & \tb{Weapon Tags}                          \tableheaderrule
          Armor weapons                     &               &             &                             &                                                            \\
          \tind Standard shield             & \plus0        & 1d6         & 0 (10 gp)                   & \tdash                                                     \\
          \tind Spiked standard shield      & \plus0        & 1d8         & 1 (40 gp)                   & \tdash                                                     \\

          Axes                              &               &             &                             &                                                            \\
          \tind Battleaxe                   & \plus0        & 1d8         & 0 (10 gp)                   & Sweeping (1), Versatile Grip                               \\
          \tind Greataxe                    & \plus0        & 1d10        & 0 (10 gp)                   & Heavy, Sweeping (1)                                        \\
          \tind Handaxe                     & \plus1        & 1d4         & 0 (10 gp)                   & Light, Thrown (30/60)                                      \\
          \tind Poleaxe                     & \plus1        & 1d8        & 0 (10 gp)                   & Heavy, Maneuverable                                        \\
          \tind Shepherd's axe              & \plus0        & 1d8         & 0 (10 gp)                   & Long, Versatile Grip                                       \\
          \tind Throwing axe                & \plus0        & 1d8         & 0 (10 gp)                   & Thrown (30/60)                                             \\

          Blades                            &               &             &                             &                                                            \\
          \tind Broadsword                  & \plus1        & 1d6         & 0 (10 gp)                   & Sweeping (1), Versatile Grip                               \\
          % -1d for an extra tag
          \tind Dagger                      & \plus1        & 1d3         & 0 (10 gp)                   & Compact, Light, Thrown (30/60)                             \\
          \tind Greatsword                  & \plus0        & 1d8         & 0 (10 gp)                   & Heavy, Sweeping (2)                                        \\
          \tind Rapier                      & \plus2        & 1d4         & 0 (10 gp)                   & Light                                                      \\
          \tind Scimitar                    & \plus1        & 1d6         & 0 (10 gp)                   & Mounted                                                    \\
          \tind Smallsword                  & \plus1        & 1d4         & 0 (10 gp)                   & Keen, Light                                                \\

          Bows                              &               &             &                             &                                                            \\
          \tind Longbow\fn{2}               & \plus0        & 1d6         & 1 (40 gp)                   & Bow, Projectile (90/270)                                   \\
          \tind Shortbow\fn{2}              & \plus0        & 1d4         & 0 (10 gp)                   & Bow, Projectile (60/180)                                   \\
          \tind Arrows (20)                 & \plus0        & \tdash      & 0 (2 gp)                    & Ammunition                                                 \\
          \tind Blunted arrows (20)         & \minus1       & \tdash      & 0 (2 gp)                    & Ammunition, Subdual                                        \\
          \tind Fire arrows (20)\fn{2}      & \minus1       & \tdash      & 2 (40 gp)                   & Ammunition                                                 \\
          \tind Lightning arrows (20)\fn{2} & \minus1       & \tdash      & 3 (200 gp)                  & Ammunition                                                 \\

          Club-like weapons                 &               &             &                             &                                                            \\
          \tind Cavalry mace                & \plus0        & 1d8         & 0 (10 gp)                   & Mounted, Versatile Grip                                    \\
          % only 1 improvement
          \tind Club                        & \plus0        & 1d8         & \tdash                      & Versatile Grip                                             \\
          \tind Greatmace                   & \plus0        & 1d10        & 0 (10 gp)                   & Heavy, Impact                                              \\
          % only 1 improvement
          \tind Mace                        & \plus0        & 1d6         & 0 (10 gp)                   & Impact                                                     \\
          \tind Morning star                & \plus0        & 1d10        & 0 (10 gp)                   & Versatile Grip                                             \\
          % -1d for 3 tags
          \tind Sap                         & \plus1        & 1d3         & 0 (10 gp)                   & Compact, Light, Subdual                                    \\
          % no improvements, but fire damage
          \tind Torch\fn{2}                 & \plus0        & 1d6         & \tdash                      & \tdash                                                \\

          Crossbows                         &               &             &                             &                                                            \\
          \tind Heavy crossbow\fn{2}        & \plus0        & 1d10        & 1 (40 gp)                   & Heavy, Projectile (90/270)                                 \\
          \tind Light crossbow\fn{2}        & \plus0        & 1d8         & 1 (40 gp)                   & Projectile (60/180)                                        \\
          \tind Pellet crossbow\fn{2}       & \plus2        & 1d4         & 2 (200 gp)                  & Projectile (60/180), Subdual                               \\
          \tind Bolts (20)                  & \plus0        & \tdash      & 0 (2 gp)                    & Ammunition                                                 \\
          \tind Blunted bolts (20)          & \minus1       & \tdash      & 0 (2 gp)                    & Ammunition, Subdual                                        \\
          \tind Round bullets (20)          & \tdash        & \tdash      & 0 (2 gp)                    & Ammunition                                                 \\

          Flexible weapons                  &               &             &                             &                                                            \\
          \tind Flail                       & \plus0        & 1d8         & 0 (10 gp)                   & Maneuverable, Versatile Grip                               \\
          \tind Heavy flail                 & \plus0        & 1d10        & 0 (10 gp)                   & Heavy, Maneuverable                                        \\
          \tind Two-section staff           & \plus2        & 1d6         & 0 (10 gp)                   & Heavy, Long                                                \\
          \tind Nunchaku                    & \plus1        & 1d4         & 0 (10 gp)                   & Light, Maneuverable                                        \\
          \tind Whip\fn{2}                  & \plus1        & 1d4         & 0 (10 gp)                   & Long                                                       \\

          Headed weapons                    &               &             &                             &                                                            \\
          \tind Light hammer                & \plus0        & 1d6         & 0 (10 gp)                   & Light, Thrown (30/60)                                      \\
          \tind Longhammer                  & \plus0        & 1d8         & 0 (10 gp)                   & Heavy, Long, Resonating                                    \\
          \tind Pick                        & \plus0        & 1d8         & 0 (10 gp)                   & Impact, Versatile Grip                                     \\
          \tind Sickle                      & \plus1        & 1d4         & 0 (10 gp)                   & Light, Sweeping (1)                                        \\
          \tind Sledgehammer                & \minus1       & 2d6         & 0 (10 gp)                   & Heavy, Resonating                                          \\
          \tind Warhammer                   & \plus0        & 1d8         & 0 (10 gp)                   & Resonating, Versatile Grip                                 \\

          Monk weapons                      &               &             &                             &                                                            \\
          \tind Jitte                       & \plus2        & 1d4         & 0 (10 gp)                   & Light                                                      \\
          \tind Kama                        & \plus1        & 1d4         & 0 (10 gp)                   & Light, Sweeping (1)                                        \\
          % -1d for extra throwing range
          \tind Kunai                       & \plus1        & 1d3         & 0 (10 gp)                   & Light, Thrown (60/120)                                     \\
          \tind Nunchaku                    & \plus1        & 1d4         & 0 (10 gp)                   & Light, Maneuverable                                        \\
          % -1d
          \tind Quarterstaff                & \plus1        & 1d6         & \tdash                      & Heavy, Long                                                \\
          \tind Shuriken (5)                & \plus2        & 1d3         & 0 (10 gp)                   & Ammunition, Compact, Light, Thrown (30/60)                 \\
          \tind Two-section staff           & \plus2        & 1d6         & 0 (10 gp)                   & Heavy, Long                                                \\

          Polearms                          &               &             &                             &                                                            \\
          \tind Bardiche                    & \plus0        & 1d10        & 0 (10 gp)                   & Heavy, Sweeping (1)                                        \\
          \tind Glaive                      & \plus0        & 1d8         & 0 (10 gp)                   & Heavy, Long, Sweeping (1)                                  \\
          \tind Halberd                     & \plus0        & 1d10         & 0 (10 gp)                   & Heavy, Long                                                \\
          \tind Longhammer                  & \plus0        & 1d8         & 0 (10 gp)                   & Heavy, Long, Resonating                                    \\
          \tind Poleaxe                     & \plus1        & 1d8        & 0 (10 gp)                   & Heavy, Maneuverable                                        \\
          % -1d relative to normal Heavy
          \tind Quarterstaff                & \plus1        & 1d6         & \tdash                      & Heavy, Long                                                \\
          \tind Scythe                      & \plus1        & 1d6         & 0 (10 gp)                   & Heavy, Sweeping (2)                                        \\

          Simple weapons                    &               &             &                             &                                                            \\
          \tind Claw sheath\fn{2}           & \tdash        & \tdash      & 1 (40 gp)                   & \tdash                                                     \\
          % only 1 improvement
          \tind Club                        & \plus0        & 1d8         & \tdash                      & \tdash                                                     \\
          % -1d for an extra tag
          \tind Dagger                      & \plus1        & 1d3         & 0 (10 gp)                   & Compact, Light, Thrown (30/60)                             \\
          % only 1 improvement
          \tind Mace                        & \plus0        & 1d6         & 0 (10 gp)                   & Impact                                                     \\
          % -1d
          \tind Quarterstaff                & \plus1        & 1d6         & \tdash                      & Heavy, Long                                                \\

          Spears                            &               &             &                             &                                                            \\
          \tind Greatspear                  & \plus0        & 1d10        & 0 (10 gp)                   & Heavy, Long                                                \\
          \tind Javelin                     & \plus0        & 1d6         & 0 (10 gp)                   & Thrown (60/120)                                            \\
          \tind Lance (5)\fn{2}             & \plus0        & 1d10        & 0 (10 gp)                   & Ammunition, Long, Mounted                                  \\
          \tind Spear                       & \plus1        & 1d6         & 0 (10 gp)                   & Thrown (30/60), Versatile Grip                             \\
          \tind Spontoon                    & \plus1        & 1d8         & 0 (10 gp)                   & Versatile Grip                                             \\

          Thrown weapons                    &               &             &                             &                                                            \\
          % -1d for an extra tag
          \tind Dagger                      & \plus1        & 1d3         & 0 (10 gp)                   & Compact, Light, Thrown (30/60)                             \\
          \tind Dart (5)                    & \plus1        & 1d3         & 0 (2 gp)                    & Ammunition, Compact, Light, Thrown (60/120)                \\
          \tind Handaxe                     & \plus1        & 1d4         & 0 (10 gp)                   & Light, Thrown (30/60)                                      \\
          \tind Light hammer                & \plus0        & 1d6         & 0 (10 gp)                   & Thrown (30/60)                                             \\
          \tind Javelin                     & \plus0        & 1d6         & 0 (10 gp)                   & Thrown (60/120)                                            \\
          \tind Shuriken (5)                & \plus2        & 1d3         & 0 (2 gp)                    & Ammunition, Compact, Light, Thrown (30/60)                 \\
          \tind Sling\fn{2}                 & \plus0        & 1d6         & 0 (10 gp)                   & Projectile (90/270)                                        \\
          \tind Round bullets (20)          & \tdash        & \tdash      & 0 (2 gp)                    & Ammunition                                                 \\
          \tind Throwing axe                & \plus0        & 1d8         & 0 (10 gp)                   & Thrown (30/60)                                             \\
        \end{longtable}
        1 See \pcref{Item Ranks}. \\
        2 This weapon has special rules. \\
      \end{longtablewrapper}
    \end{longcolumn}

    \begin{longcolumn}
      \begin{longtablewrapper}
        \tablebookmark{Exotic Weapons}{exoticweapons}
        \RaggedRight
        \begin{longtable}{p{12em} l l l >{\lcol}p{24em}}
          \lcaption{Exotic Weapons}                                                                                                        \\
          \tb{Exotic Weapons}
          \label{cap:Exotic Weapons}     & \tb{Accuracy} & \tb{Damage} & \tb{Item Rank (Cost)}\fn{1} & \tb{Weapon Tags} \tableheaderrule
          Armor                          &               &             &                             &                                     \\
          \tind Armblade\fn{2}           & \plus1        & 1d4         & 1 (40 gp)                   & Clinch, Keen, Light                 \\
          \tind Spiked knee\fn{2}        & \plus0        & 1d4         & 1 (40 gp)                   & Clinch, Impact, Light               \\
          \tind Tower shield             & \plus0        & 1d8         & 1 (40 gp)                   & Heavy, Resonating                   \\
          \tind Spiked tower shield      & \plus0        & 1d10        & 1 (40 gp)                   & Heavy, Resonating                   \\
          Axes                           &               &             &                             &                                     \\
          \tind Dwarven throwing axe     & \plus0        & 1d8         & 0 (10 gp)                   & Thrown (60/120)                     \\
          \tind Dwarven waraxe           & \plus0        & 1d10        & 1 (40 gp)                   & Thrown (30/60), Versatile Grip      \\
          \tind Orcish greataxe          & \minus1       & 2d6         & 1 (40 gp)                   & Heavy, Impact, Sweeping (1)         \\
          Blades                         &               &             &                             &                                     \\
          \tind Boot dagger\fn{2}        & \plus1        & 1d4         & 0 (10 gp)                   & Compact, Light                      \\
          \tind Falchion                 & \plus1        & 1d6         & 1 (40 gp)                   & Sweeping (2), Versatile Grip        \\
          \tind Katana                   & \plus1        & 1d8         & 1 (40 gp)                   & Heavy, Keen, Sweeping (1)           \\
          \tind Kukri                    & \plus2        & 1d4         & 0 (10 gp)                   & Light, Sweeping (1)                 \\
          \tind Parrying dagger          & \plus2        & 1d4         & 0 (10 gp)                   & Parrying, Light                     \\
          Bows                           &               &             &                             &                                     \\
          \tind Flatbow\fn{2}            & \plus1        & 1d6         & 2 (200 gp)                  & Bow, Projectile (90/270)            \\
          \tind Heartseeker arrows (20)  & \plus0        & \tdash      & 2 (40 gp)                   & Ammunition, Impact                  \\
          \tind Recurve bow\fn{2}        & \plus0        & 1d8         & 2 (200 gp)                  & Bow, Projectile (90/270)            \\
          \tind Takedown bow\fn{2}       & \plus0        & 1d6/1d4     & 2 (200 gp)                  & Bow, Projectile (90/270 or 60/180)  \\
          \tind Titanbow\fn{2}           & \minus1       & 1d8         & 2 (200 gp)                  & Bow, Heavy, Projectile (60/180)     \\
          Club-like weapons              &               &             &                             &                                     \\
          \tind Culacula                 & \plus0        & 1d10        & 0 (10 gp)                   & Heavy, Impact, Parrying             \\
          \tind Gnomish trick mace       & \plus2        & 1d4         & 0 (10 gp)                   & Light, Maneuverable                 \\
          \tind Knobkerrie               & \plus1        & 1d6         & 0 (10 gp)                   & Impact, Throwing (30/60)            \\
          \tind Totokia                  & \plus0        & 1d10        & 0 (10 gp)                   & Impact, Versatile Grip              \\
          Crossbows                      &               &             &                             &                                     \\
          \tind Arbalest\fn{2}           & \plus1        & 1d10        & 2 (200 gp)                  & Heavy, Impact, Projectile (90/270)  \\
          \tind Pistol crossbow\fn{2}    & \plus1        & 1d4         & 1 (40 gp)                   & Compact, Light, Projectile (30/90)  \\
          \tind Repeating crossbow\fn{2} & \plus0        & 1d8         & 2 (200 gp)                  & Projectile (90/270)                 \\
          \tind Repeating bolts (5)      & \plus0        & \tdash      & 1 (10 gp)                   & Ammunition                          \\
          Flexible weapons               &               &             &                             &                                     \\
          \tind Bladed whip\fn{2}        & \plus0        & 1d6         & 1 (40 gp)                   & Long, Sweeping (1)                  \\
          \tind Chain whip               & \plus1        & 1d6         & 1 (40 gp)                   & Long, Maneuverable                  \\
          \tind Meteor hammer            & \plus0        & 2d6         & 1 (40 gp)                   & Heavy, Long           \\
          \tind Three-section staff      & \plus2        & 1d6         & 1 (40 gp)                   & Heavy, Long, Maneuverable \\
          Headed weapons                 &               &             &                             &                                     \\
          \tind Dwarven longhammer       & \plus0        & 1d10        & 1 (40 gp)                   & Heavy, Long, Resonating             \\
          \tind Dwarven shorthammer      & \plus0        & 1d6         & 1 (40 gp)                   & Light, Thrown (60/120)              \\
          \tind Heavy pick               & \plus0        & 1d10        & 1 (40 gp)                   & Heavy, Keen, Impact                 \\
          \tind Obuch                    & \plus0        & 1d8         & 1 (40 gp)                   & Long, Maneuverable, Versatile Grip  \\
          Monk weapons                   &               &             &                             &                                     \\
          \tind Hook sword               & \plus1        & 1d6         & 1 (40 gp)                   & Light, Maneuverable                 \\
          \tind Sai                      & \plus1        & 1d4         & 0 (10 gp)                   & Clinch, Light, Parrying             \\
          \tind Three-section staff      & \plus1        & 1d6         & 1 (40 gp)                   & Heavy, Long, Maneuverable, Parrying \\
          \tind War fan\fn{2}            & \plus1        & 1d4         & 1 (40 gp)                   & Light, Parrying                     \\
          Polearms                       &               &             &                             &                                     \\
          \tind Fauchard                 & \minus1       & 2d6         & 1 (40 gp)                   & Heavy, Long, Sweeping (1)           \\
          \tind War scythe               & \plus0        & 1d10        & 1 (40 gp)                   & Heavy, Sweeping (2)                 \\
          Simple weapons                 &               &             &                             &                                     \\
          Spear                          &               &             &                             &                                     \\
          \tind Gnomish smallspear       & \plus2        & 1d4         & 0 (10 gp)                   & Light, Long                         \\
          \tind Partisan                 & \plus1        & 1d8         & 1 (40 gp)                   & Heavy, Long, Parrying               \\
          \tind Pike\fn{2}               & \plus0        & 1d10        & 0 (10 gp)                   & Heavy, Long                         \\
          Thrown weapons                 &               &             &                             &                                     \\
          % It's currently not clear how bolas could be used to trip from range without solving weird range issues and likely balance issues
          % \tind Bolas                    & \plus1        & 1d3         & 0 (10 gp)                   & Light, Maneuverable, Thrown (30/60)    \\
          \tind Dwarven throwing axe     & \plus0        & 1d8         & 0 (10 gp)                   & Thrown (60/120)                     \\
          \tind Dwarven waraxe           & \plus0        & 1d10        & 1 (40 gp)                   & Thrown (30/60), Versatile Grip      \\
          \tind Net\fn{2}                & \plus0        & \tdash      & 0 (10 gp)                   & Massive (10), Thrown (5/15)         \\
        \end{longtable}
        1 See \pcref{Item Ranks}. \\
        2 This weapon has special rules. \\
      \end{longtablewrapper}
    \end{longcolumn}

    \begin{dtable}
      % Every natural weapon that requires a free hand should be on par with a manufactured weapon.
      % Every natural weapon that can be used with both hands occupied should be one upgrade behind an equivalent manufactured weapon.
      \tablebookmark{Natural Weapons}{naturalweapons}
      \lcaption{Natural Weapons}
      \begin{dtabularx}{\columnwidth}{p{12em} c c >{\ccol}X}
        \tb{Natural Weapons}    & \tb{Accuracy} & \tb{Damage} & \tb{Weapon Tags} \tableheaderrule
        % Bite is weird since it does physical damage instead of any subtype, so it can't be used with many maneuvers.
        % Also, many creatures with a bite attack are not good at grappling.
        % For those reasons, it gets an extra +1d.
        Bite                    & \plus0        & 1d8         & Clinch \\
        Claw\fn{1}              & \plus2        & 1d4         & Light     \\
        Horn                    & \plus0        & 1d6         & Impact    \\
        Punch/kick\fn{1} \fn{2} & \plus0        & 1d3         & Subdual   \\
        Ram                     & \plus0        & 1d6         & Resonating  \\
        Stinger                 & \plus1        & 1d6         & \tdash    \\
        Talon\fn{2}             & \plus2        & 1d4         & Light     \\
        Tentacle                & \plus0        & 1d6         & Long    \\
      \end{dtabularx}
      1 This natural weapon must normally be used with a \glossterm{free hand}. \\
      2 This weapon has special rules. \\
    \end{dtable}

  \subsection{Individual Weapon Descriptions}
    Some weapons in \trefnp{Weapons} have additional abilities which are described below.
    \parhead{Arbalest} You draw an arbalest back by turning a small winch. Reloading an arbalest requires two standard actions.
    Each standard action requires one \glossterm{free hand} while holding the arbalest in another hand.
    \parhead{Armblade} This weapon is not held in a hand.
    Instead, it is affixed to the arm of body armor with a medium or heavy \glossterm{usage class}.
    When you attack with an armblade, you cannot use the arm it is attached to for any other combat purpose in the same phase.
    You can use that arm to hold items, but not to maintain a grapple or perform similar actions.
    If you are not proficient with this weapon, you increase your \glossterm{encumbrance} by 2 when wearing armor with an armblade.
    \parhead{Bladed Whip} A bladed whip can be used to attack targets within 15 feet instead of the normal 10 feet for a Long weapon.
    \parhead{Boot Dagger} A boot dagger is a modified boot or boot sole which contains a hidden dagger.
    The dagger is normally concealed, and requires an Awareness check with a \glossterm{difficulty value} of 15 to find.
    Attacking with a boot dagger does not require a \glossterm{free hand}, but you must make a Balance check with a \glossterm{difficulty value} of 10 during whenever you attack with it.
    If you fail this check, you fall \prone after the attack.

    After you attack with a boot dagger, the dagger remains plainly visible.
    Concealing the dagger again requires a standard action.
    \parhead{Claw Sheath} A claw sheath is not a weapon on its own.
    Instead, it wraps around one of your natural weapons.
    This gives no intrinsic benefit, but claw sheaths can be imbued with magic weapon properties that apply to the wrapped natural weapon.
    Claw sheaths are made for claws, but an equivalent can be made for any natural weapon that requires a free hand to use, such as a slam.
    % TODO: split up punch/kick into two natural weapons to make this more intuitive?
    A claw sheath for a punch/kick natural weapon is made for one hand and only affects punches with that hand.
    \parhead{Flatbow} A flatbow is too unwieldy to use while you are mounted.
    Unlike a longbow, a flatbow is flat when not under tension and has approximately rectangular limbs.
    This spreads stress more evenly over the bow's structure, allowing more precise shots, though the firing technique is different and less commonly known.
    A flatbow is the same size category as the creature it is sized for.
    \parhead{Fire Arrows} Attacks with these arrows have the \atFire tag.
    These arrows are treated with alchemist's fire so they can be ignited before being shot.
    The process requires thickening the arrow shaft, reducing the precision of the arrow.
    If you have access to an active flame that is at least as intense as a torch, you can ignite a fire arrow as part of drawing it from a quiver.
    \parhead{Heavy Crossbow} You draw a heavy crossbow back by turning a small winch.
    Loading a heavy crossbow is a standard action that requires one \glossterm{free hand} while holding the crossbow in another hand.
    \parhead{Lance} A lance can only be used effectively while mounted.
    Using a lance on foot imposes a \minus2 accuracy penalty for not being \glossterm{proficient} with the weapon.
    \parhead{Light Crossbow} You draw a light crossbow back by turning a small winch.
    Loading a light crossbow is a \glossterm{minor action} that requires one \glossterm{free hand} while holding the crossbow in another hand.
    \parhead{Lightning Arrows} Attacks with these arrows have the \atElectricity tag.
    These arrows are treated with a reactive alchemical substance so they deal electricity damage on contact.
    The process requires thickening the arrow shaft, reducing the precision of the arrow.
    \parhead{Longbow} A longbow is the same size category as the creature it is sized for.
    \parhead{Net} A net is used to entangle enemies. When you throw a net, you make an attack vs. Reflex against your target. If you hit, the target is \slowed.
    \par A netted creature can escape with a \glossterm{difficulty value} 8 Flexibility check (normally a standard action). The net has 8 hit points and can be burst with a \glossterm{difficulty value} 8 Strength check as a standard action.
    \par A net has no effect on creatures that are Tiny or smaller, or Huge or larger. It must be folded to be thrown effectively, which takes a minute of work. You take a \minus4 accuracy penalty with an unfolded net.
    \parhead{Pellet Crossbow} You need both hands to fire a pellet crossbow.
    You draw a pellet crossbow back by turning a small winch.
    Loading a pellet crossbow is a \glossterm{minor action} that requires one \glossterm{free hand} while holding the crossbow in another hand.
    Unlike most crossbows, a pellet crossbow uses round bullets as ammunition instead of bolts.
    \parhead{Pike} A pike can be used to attack targets within 15 feet instead of the normal 10 feet for a Long weapon.
    However, you cannot use it to attack targets adjacent to you.
    \parhead{Pistol Crossbow} You can draw a pistol crossbow back by hand.
    Loading a pistol crossbow is a \glossterm{minor action} that requires one \glossterm{free hand} while holding the crossbow in another hand.
    You can hold two pistol crossbows in one hand for reloading purposes, and reload them both with a single minor action.
    This makes them easy to make \glossterm{dual strikes} with (see \pcref{Moving Items Between Hands}).
    \parhead{Punch/Kick} All bipedal creatures have access to the punch/kick \glossterm{natural weapon}.
    Normally, this represents a punch, which requires a \glossterm{free hand}.
    If you are trained in the Balance skill, have a Dexterity of at least 3, or are currently flying or gliding, you can make it a kick instead.
    A kick does not require a free hand, but is not otherwise more powerful than a punch.
    \parhead{Recurve Bow} A recurve bow is the same size category as the creature it is sized for.
    \parhead{Repeating Crossbow} The repeating crossbow holds 5 crossbow bolts.
    As long as it holds bolts, you can fire it without reloading, allowing you to use it entirely one-handed.
    Loading a new case of 5 bolts is a \glossterm{standard action} that requires one \glossterm{free hand} while holding the crossbow in another hand.
    \parhead{Shortbow} Unlike most \weapontag{Projectile} weapons, your range limits are not penalized when firing a shortbow while riding a moving mount, and the accuracy penalty is reduced to \minus2 instead of \minus4.
    \parhead{Sling} Loading a sling is a \glossterm{minor action} that requires one \glossterm{free hand} while holding the sling in another hand.
    \par You can hurl ordinary stones with a sling, but stones are not as dense or as round as bullets. You take a \minus1 accuracy penalty with ordinary stones.
    \parhead{Spiked Knee} This weapon is not held in a hand.
    Instead, it is affixed to the leg of body armor with a medium or heavy \glossterm{usage class} (see \pcref{Armor Usage Classes}).
    If you are not proficient with this weapon, you increase your \glossterm{encumbrance} by 2 when wearing armor with a spiked knee.
    \parhead{Takedown Bow} A takedown bow is a bow assembled from multiple independent components that can be reconfigured into two different combinations.
    In its longbow configuration, it functions like a longbow, and in its shortbow configuration, it functions like a shortbow.
    In addition, when it is fully disassembled, it takes up space equivalent to a Compact weapon, making it easier to transport and conceal.
    \parhead{Talon} A talon is always attached to a foot.
    In order to attack with a talon, you must be trained in the Balance skill, have a Dexterity of at least 3, or be currently flying or gliding.
    \parhead{Titanbow} A titanbow is too unwieldy to use while you are mounted.
    It is the same size category as the creature it is sized for.
    \parhead{Torch} Attacks with a torch have the \atFire tag.
    \parhead{Tower Shield} Although you can hold a tower shield in one hand to defend yourself, you need to support it with your other hand to effectively smash it into an enemy.
    That makes it a \weapontag{Heavy} weapon.
    \parhead{War Fan} A war fan grants you a \plus1 bonus to Armor defense while you wield it.
    If you wield two war fans at once, the bonus increases to \plus2.
    This bonus is treated as coming from a shield, and it does not stack with the benefits of using any other shield.
    \parhead{Whip} A whip can be used to attack targets within 15 feet instead of the normal 10 feet for a Long weapon.

  \subsection{Weapon Special Materials}\label{Weapon Special Materials}
    Nonmagical weapons can be made from special materials that can alter the properties of the item.
    These special materials are described in \trefnp{Weapon Special Materials}.
    Depending on the construction of the weapon, it may be entirely composed of the special material, or it may only have its striking surface altered.
    For example, a dragonfang spear may have a wooden haft and still gain the full benefits of being a dragonfang weapon.
    An adamantine club would only have a thin layer of adamantine around the outside, rather than being entirely forged from adamantine, because the weight and cost would otherwise be absurd.

    A weapon that is made from a special material cannot have any magic item properties, and cannot be chosen as a \glossterm{legacy item}.
    \weapontag{Projectile} weapons cannot be made from special materials.
    However, the ammunition fired by Projectile weapons can be made from special materials.
    The benefits of ammunition made from special materials apply even when fired from magical Projectile weapons.

    Any individual weapon can only ever gain the combat benefits of a single special material, even if it contains multiple special materials in its construction.
    That special material is chosen at the time the weapon is crafted and cannot be altered without recrafting it.

    \subsubsection{Weapon Special Material Prices}
      Weapon special materials are listed as having a single rank and price in \trefnp{Armor Special Materials}.
      This is the price for an amount of the special material sufficient to forge a typical weapon, when combined with other normal weaponsmithing materials.

      \begin{dtable!*}
        \lcaption{Weapon Special Materials}
        \begin{dtabularx}{\textwidth}{l >{\lcol}X l}
          \tb{Material}             & \tb{Special Effect}                                               & \tb{Item Rank (Cost)}              \tableheaderrule
          % bonus weapon damage, object damage
          \tind Adamantine          & \plus1d4 damage, double damage to objects, extra weight             & 5 (25,000 gp)  \\
          \tind Adamantine, pure    & \plus1d6 damage, double damage to objects, extra weight           & 7 (625,000 gp) \\
          % just vulnerabilities
          \tind Cold iron           & Common vulnerabilities                                            & 2 (200 gp)     \\
          % crit accuracy
          \tind Diamondsteel        & \plus3 accuracy with critical hits                                & 3 (1,000 gp)   \\
          \tind Diamondsteel, pure  & \plus5 accuracy with critical hits                                & 6 (125,000 gp)  \\
          % energy damage
          \tind Dragonfang          & \plus1d4 \glossterm{extra damage}, deals energy damage                                               & 3 (1,000 gp)   \\
          \tind Dragonfang, ancient & \plus1d8 \glossterm{extra damage}, deals energy damage, grants breath attack                         & 5 (25,000 gp)  \\
          % trade damage for accuracy
          \tind Mithral             & Lighter                                           & 3 (1,000 gp)   \\
          \tind Mithral, pure       & \plus1 accuracy, lighter                          & 6 (125,000 gp)  \\
          % just vulnerabilities  
          \tind Silvered            & Common vulnerabilities                                            & 2 (200 gp)     \\
          % extra damage
          \tind Starmetal           & \plus1d4 \glossterm{extra damage}, \minus1 accuracy, extra weight   & 2 (200 gp)     \\
          \tind Starmetal, pure     & \plus1d8 \glossterm{extra damage}, \minus1 accuracy, extra weight & 4 (5,000 gp)   \\
        \end{dtabularx}
      \end{dtable!*}

      \parhead{Adamantine} An adamantine weapon deals \plus1d4 \glossterm{weapon damage}.
      In addition, \glossterm{strikes} with it deal double damage to objects that are not made of pure or ordinary adamantine.
      Unlike other weapons, adamantine weapons are not \glossterm{lightweight}, so \abilitytag{Heavy} adamantine weapons typically require a minimum Strength of 2.

      \parhead{Adamantine, Pure} A pure adamantine weapon deals \plus1d8 \glossterm{weapon damage}.
      In addition, \glossterm{strikes} with it deal double damage to objects that are not made of pure adamantine.
      Unlike other weapons, pure adamantine weapons are not \glossterm{lightweight}, so \abilitytag{Heavy} adamantine weapons typically require a minimum Strength of 2.

      \parhead{Cold Iron} Many fey creatures and some demons are \glossterm{vulnerable} to cold iron weapons.

      \parhead{Diamondsteel} A diamondsteel weapon grants you a \plus3 bonus to \glossterm{accuracy} with \glossterm{strikes} using it for the purpose of determining whether you get a \glossterm{critical hit}.

      \parhead{Diamondsteel, Pure} A pure diamondsteel weapon grants you a \plus5 bonus to \glossterm{accuracy} with \glossterm{strikes} using it for the purpose of determining whether you get a \glossterm{critical hit}.

      \parhead{Dragonfang} Strikes with a dragonfang weapon deal 1d4 \glossterm{extra damage} and have that dragon's associated tag (see \tref{Dragon Types}).

      \parhead{Dragonfang, Ancient} Strikes with an ancient dragonfang weapon deal 1d8 \glossterm{extra damage} and have that dragon's associated tag (see \tref{Dragon Types}).
      You also gain the \ability{dragonfang breath} ability while wielding an ancient dragonfang weapon.
      \begin{activeability}{Dragonfang Breath}
        \abilityusagetime Standard action.
        \abilitycost You \glossterm{briefly} cannot use this ability with this weapon again.
        \rankline
        % TODO: math to make sure this is right? It's probably right, especially with the brief cooldown.
        Make a \glossterm{strike} with double \glossterm{weapon damage} using one ancient dragonfang weapon.
        It cannot be a \glossterm{dual strike}.
        If the dragon's breath weapon is normally a line, the strike targets everything in a \arealarge, 10 ft. wide line from you.
        Otherwise, the strike targets everything in a \areamed cone from you.
        On a miss, you still deal half damage.
      \end{activeability}

      % Battleaxe 1d8 damage, Sweeping (1)
      % 1H mithral greataxe: 1d10 damage, Sweeping (1). So you're basically paying a special material cost for +1 weapon damage.
      % More generally, this is basically adding +1 weapon upgrade to a given weapon.
      \parhead{Mithral} A non-\weapontag{Heavy} mithral weapon gains the \weapontag{Light} weapon tag.
      A Heavy mithral weapon loses that tag and instead gains the \weapontag{Versatile Grip} weapon tag.

      \parhead{Mithral, Pure} A pure mithral weapon has a \plus1 accuracy bonus.
      In addition, a non-\weapontag{Heavy} mithral weapon gains the \weapontag{Light} weapon tag.
      A Heavy mithral weapon loses that tag and instead gains the \weapontag{Versatile Grip} weapon tag.

      \parhead{Silvered} Lycanthropes and some undead are \glossterm{vulnerable} to silvered weapons.

      \parhead{Starmetal} A starmetal weapon deals \plus1d4 \glossterm{extra damage}.
      However, it also has a \minus1 accuracy penalty.
      Unlike other weapons, starmetal weapons are not \glossterm{lightweight}, so they typically require a minimum Strength of 2 to wield.

      \parhead{Pure Starmetal} A pure starmetal weapon deals \plus1d8 \glossterm{extra damage}.
      However, it also has a \minus1 accuracy penalty.
      Unlike other weapons, pure starmetal weapons are not \glossterm{lightweight}, so they typically require a minimum Strength of 2 to wield.

      % Magic weapons are a highly limited slot.
      % They have the same power level as self-attune spells.
      \begin{longcolumn}
        \section{Magic Weapons}
          \begin{longtablepreface}

            Magic weapons improve a character's combat abilities.
            They must be wielded to gain their effects.

            \parhead{Ranged Weapons and Ammunition} Any magical properties of a \weapontag{Projectile} weapon also apply to all ammunition fired from that weapon.

            \parhead{Craft Skills} The craft skills used to create and repair items are listed in parentheses before the item's description.
            All magic weapons simply use the same materials as the original, nonmagical weapon.
          \end{longtablepreface}

          \input{generated/magic_weapons_table.tex}

      \end{longcolumn}

      \input{generated/magic_weapons.tex}

\newpage
\sectiongraphic*{Armor}{width=\columnwidth}{equipment/armor}

  Most characters use armor to protect themselves. There are two kinds of armor: \glossterm{body armor}, such as full plate armor, and \glossterm{shields}.
  Body armor is worn on your body.
  You can only benefit from one body armor at a time.
  If you somehow wear multiple layers of body armor, the penalties stack and the benefits do not stack.
  A shield requires a free hand instead of being worn on the body.

  \subsection{Armor Mechanics}

    \subsubsection{Armor Usage Classes}\label{Armor Usage Classes}
      An armor's \glossterm{usage class} is a measure of how the armor is used, and how much effort is required to use it.
      It indicates whether armor, when used by a creature the armor is sized for, is considered light armor, medium armor, or heavy armor.

    \subsubsection{Armor Proficiency}\label{Armor Proficiency}
      Proficiency with armor is defined by the armor's usage class.
      If you wear or use armor you are not proficient with, it provides half its normal defense bonus.
      In addition, you apply that armor's \glossterm{encumbrance} as a penalty to your \glossterm{accuracy}.
      Since standard shields and padded body armor have no \glossterm{encumbrance}, you can use them without penalizing your attacks.

    \subsubsection{Getting Into And Out Of Armor}
      The time required to don armor depends on its type; see \trefnp{Donning Armor}. Donning and removing body armor and shields takes both hands.
      \parhead{Don} This column tells how long it takes a character to put the armor on. (One minute is 10 rounds.)
      \parhead{Remove} This column tells how long it takes to get the armor off.

      \begin{dtable}
        \lcaption{Donning Armor}
        \begin{dtabularx}{\columnwidth}{>{\lcol}X c c}
          \tb{Armor Type}   & \tb{Don}          & \tb{Remove} \tableheaderrule
          Bucklet           & 1 standard action & 1 standard action \\
          Standard shield   & 1 standard action & 1 standard action \\
          Tower shield      & 1 standard action & 1 standard action \\
          Light body armor  & 1 minute          & 1 minute          \\
          Medium body armor & 5 minutes         & 1 minute          \\
          Heavy body armor  & 5 minutes         & 5 minutes         \\
        \end{dtabularx}
      \end{dtable}

    \subsubsection{Weight and Size}
      The size category of body armor is the same as the size category of the creature it is sized for.
      Bucklers and standard shields are one size category smaller than the creature they are sized for, while tower shields are the same size category as the creature they are sized for.
      All armor and shields except for heavy body armor are \glossterm{lightweight} objects.
      In general, heavy body armor weighs so much that only creatures with a Strength of at least 2 can wear it (see \pcref{Weight Limits}).

    \subsubsection{Barding}\label{Barding}
      Armor is normally designed for creatures with two arms and two legs, matching the normal humanoid shape.
      Creatures with more esoteric shapes can wear armor if they are proficient, but it is not as effective.
      This is called barding.

      The Armor defense bonus provided by barding is reduced by 2.
      This penalty also applies to magical effects that mimic armor, such as \ability{mage armor}, if those bonuses do not stack with wearing regular armor.
      Barding must also be custom made for the creature's body type, so unusual creatures cannot simply wear armor designed for humanoid creatures.

  \subsection{Armor Table}
    \par Here is the format for armor entries (given as column headings on \trefnp{Armor and Shields}, below).

    \parhead{Defense} This value indicates how much the armor increases your Armor defense.

    \parhead{Damage Resistance} This value indicates how much the armor increases your \glossterm{damage resistance} (see \pcref{Damage Resistance}).

    \parhead{Vital Rolls} This value indicates how much the armor increases your \glossterm{vital rolls} (see \pcref{Vital Rolls}).

    \parhead{Encumbrance} This value indicates how much the armor increases your \glossterm{encumbrance}.
    You apply your encumbrance as a penalty to all Strength and Dexterity-based checks and skills.
    For details, see \pcref{Encumbrance}.

    \parhead{Speed} This penalty applies to speed with all of your \glossterm{movement modes} while wearing the armor.

    \parhead{Dex Bonus} This multiplier affects the contribution of your Dexterity to your Armor defense.
    It does not change any other effects that Dexterity has.
    If you use multiple armor pieces that modify this bonus, use the lowest value from any single piece rather than multiplying them.
    For example, a creature using brigandine and a standard shield would still add half their Dexterity to their Armor defense.

    \parhead{Item Rank (Cost)} The first value indicates the \glossterm{item rank} of the item (see \pcref{Item Ranks}).
    The second value in parentheses indicates the average cost to buy the item.
    Items crafted for unusually large or small creatures are more expensive.
    For each size category larger than Medium or smaller than Small, the item's rank increases by one, which increases its price.

    \begin{dtable!*}
      \tablebookmark{Armor and Shields}{armor}
      \lcaption{Armor and Shields}
      \begin{dtabularx}{\textwidth}{l c c c c c c >{\lcol}X c}
        \tb{Armor}             & \tb{Defense} & \tb{Damage Resistance} & \tb{Vital Rolls} & \tb{Encumbrance} & \tb{Speed}   & \tb{Dex Bonus} & \tb{Material} & \tb{Item Rank (Cost)}  \tableheaderrule
        Light armor            &              &                        &                  &                  &              &                &               &              \\
        \tind Padded           & \plus2       & \plus3                 & \tdash           & \tdash           & \tdash       & \tdash         & Leather       & 0 (10 gp)    \\
        \tind Buff leather     & \plus2       & \plus4                 & \tdash           & \plus1           & \tdash       & \tdash         & Leather       & 1 (40 gp)    \\
        \tind Mail shirt       & \plus2       & \plus4                 & \tdash           & \plus1           & \tdash       & \tdash         & Metal         & 1 (40 gp)    \\
        \tind Buckler          & \plus1       & \tdash                 & \tdash           & \tdash           & \tdash       & \tdash         & Metal or wood & 0 (10 gp)    \\
        Medium armor           &              &                        &                  &                  &              &                &               &              \\
        \tind Leather lamellar & \plus4       & \plus4                 & \plus1           & \plus2           & \tdash       & \mult1/2       & Leather       & 1 (40 gp)    \\
        \tind Scale            & \plus4       & \plus5                 & \plus1           & \plus4           & \tdash       & \mult1/2       & Metal         & 1 (40 gp)    \\
        \tind Brigandine       & \plus4       & \plus6                 & \plus1           & \plus4           & \tdash       & \mult1/2       & Metal         & 2 (200 gp)   \\
        \tind Standard shield  & \plus2       & \tdash                 & \tdash           & \tdash\fn{1}     & \tdash       & \mult1/2       & Metal or wood & 0 (10 gp)    \\
        Heavy armor            &              &                        &                  &                  &              &                &               &              \\
        \tind Layered hide     & \plus5       & \plus8                 & \plus2           & \plus4           & \minus10 ft. & \mult1/2       & Leather       & 1 (40 gp)    \\
        \tind Half plate       & \plus5       & \plus10                & \plus2           & \plus6           & \minus10 ft. & \mult1/2       & Metal         & 2 (200 gp)   \\
        \tind Full plate       & \plus5       & \plus12                & \plus2           & \plus6           & \minus10 ft. & \mult1/2       & Metal         & 3 (1,000 gp) \\
        \tind Tower shield     & \plus3\fn{2} & \tdash                 & \tdash           & \plus2\fn{1}     & \tdash       & \mult1/2       & Metal or wood & 1 (40 gp)    \\
        Extras                 &              &                        &                  &                  &              &                &               &              \\
        \tind Shield spikes    & \tdash       & \tdash                 & \tdash           & \plus1           & \tdash       & \tdash         & Metal         & 1 (40 gp)    \\
      \end{dtabularx}
      1 The hand holding the shield is not free, which may limit your actions. \\
      2 Tower shields improve your ability to use the \textit{total defense} ability. See the description.
    \end{dtable!*}

  \subsection{Individual Armor Descriptions}
    Any special benefits or accessories to the types of armor found on \trefnp{Armor and Shields} are described below.
    \parhead{Buckler} This small metal shield is worn strapped to your forearm.
    At the start of each phase, you choose whether you treat the hand using a buckler as a \glossterm{free hand}.
    If you do, you can wield weapons or otherwise take actions using the arm bearing the buckler, but do not gain the buckler's defensive bonus during that phase.
    \par You can't make \glossterm{strikes} with a buckler.
    \parhead{Full Plate} Each suit of full plate must be individually fitted to its owner by an armorsmith.
    A captured suit can be resized to fit a new owner with a day of work and a \glossterm{difficulty value} 10 Craft (metalworking) check.
    The new owner must still be of the same size category as the size category and general body shape, such as humanoid, that the suit was originally designed for.
    \parhead{Shield, Standard, Wooden or Steel} You strap a shield to your forearm and grip it with your hand.
    A standard shield is so cumbersome that you can't use your shield hand for anything else.
    You can use a standard shield as a weapon (see \tref{Weapons}).
    However, you cannot make \glossterm{dual strikes} with a shield.
    \parhead{Shield, Tower} This massive shield is nearly as tall as an average human.
    When you use the \textit{total defense} ability while wielding a tower shield, you gain a \plus2 bonus to Armor defense in addition to the normal bonuses from taking that action (see \pcref{Total Defense}).
    A tower shield is so cumbersome that you can't use your shield hand for anything else.
    You can use a tower shield as an \glossterm{exotic weapon} (see \tref{Exotic Weapons}).
    However, you cannot make \glossterm{dual strikes} with a shield.

    While wielding a tower shield, you take a \minus1 penalty to \glossterm{accuracy} because of the shield's unwieldy nature.
    \parhead{Shield Spikes} These spikes improve the effectiveness of a standard shield or tower shield when used as a weapon.
    For details, see \tref{Weapons}.
    You can't put spikes on a buckler.
    \parhead{Studded Leather} The studs on studded leather are made of metal, but this amount of metal is not generally enough to make the item count as being made of metal.
    Studded leather armor made with studs from special materials does not grant the wearer the properties of the special material.

  \subsectiongraphic*{Armor Special Materials}{width=\columnwidth}{equipment/armor special materials}
    Nonmagical body armor can be made from special materials that can alter the properties of the item.
    These special materials are described in \trefnp{Armor Special Materials}.

    The benefits of special materials only apply to body armor that is fully made from the given special material.
    If you combine multiple special materials in any way, such as by wearing deepforged leggings with a mithral breastplate, you do not gain any benefits for having special materials.

    Body armor that is made from a special material cannot have any magic item properties, and cannot be chosen as a \glossterm{legacy item}.
    However, you can combine magic items that are not body armor with any set of body armor.
    For example, a set of full plate armor typically comes with gauntlets and heavy boots.
    However, you could wear magic boots and magic gauntlets without sacrificing the benefits of the body armor.

    \subsubsection{Armor Special Material Prices}
      Armor special materials are listed as having a single price in \trefnp{Armor Special Materials}.
      This is the price for an amount of the special material sufficient to forge light or medium armor, when combined with other normal armorsmithing materials.
      Heavy armor requires twice as much of the special material.

      \begin{dtable!*}
        \lcaption{Armor Special Materials}
        \begin{dtabularx}{\textwidth}{l l l l >{\ccol}X l}
          \tb{Material}              & \tb{Damage Resistance} & \tb{Encumbrance} & \tb{Special Effect}                     & \tb{Material} & \tb{Item Rank (Cost)} \tableheaderrule
          % metal high resistance, encumbrance
          Adamantine\fn{1}           & \mult6                 & \plus2           & Very heavy                              & Metal         & 5 (25,000 gp)  \\
          Adamantine, pure\fn{1}     & \mult12                & \plus2           & Very heavy                              & Metal         & 7 (625,000 gp) \\
          % magic defense                                                                                                                  
          Cold iron\fn{1}            & \tdash                 & \tdash           & \minus1 Armor, \plus1 defenses vs magic & Metal         & 2 (200 gp)     \\
          Cold iron, pure\fn{1}      & \mult2                 & \tdash           & \minus1 Armor, \plus2 defenses vs magic & Metal         & 4 (5,000 gp)   \\
          % metal crit resistance                                                  
          Diamondsteel\fn{1}         & \mult2                 & \tdash           & \plus2 defenses vs strike crits         & Metal         & 3 (1,000 gp)   \\
          Diamondsteel, pure\fn{1}   & \mult4                 & \tdash           & \plus2 defenses vs crits                & Metal         & 5 (25,000 gp)  \\
          % offset leather resistance, energy defense bonus                        
          Dragonhide\fn{1}           & \mult3                 & \tdash           & Impervious to specific energy type      & Leather       & 4 (5,000 gp)   \\
          Dragonhide, ancient\fn{1}  & \mult6                 & \tdash           & Immune to specific energy type          & Leather       & 6 (125,000 gp) \\
          % offset metal resistance, energy defense bonus                          
          Dragonscale\fn{1}          & \mult3                 & \tdash           & Impervious to specific energy type      & Metal         & 4 (5,000 gp)   \\
          Dragonscale, ancient\fn{1} & \mult6                 & \tdash           & Immune to specific energy type          & Metal         & 6 (125,000 gp) \\
          % leather encumbrance reduction                                                             
          Elvenweave                 & \mult2                 & \minus1          & \tdash                                  & Leather       & 3 (1,000 gp)   \\
          Elvenweave, pure           & \mult4                 & \minus2          & \tdash                                  & Leather       & 5 (25,000 gp)  \\
          % metal encumbrance reduction                                            
          Mithral                    & \mult2                 & \minus1          & \tdash                                  & Metal         & 3 (1,000 gp)   \\
          Mithral, pure              & \mult4                 & \minus2          & \tdash                                  & Metal         & 5 (25,000 gp)  \\
          % stealth leather
          Shadowweave                & \mult3                 & \tdash          & \tdash                                  & Leather       & 4 (5,000 gp)   \\
          Shadowweave, umbral        & \mult6                 & \tdash          & \tdash                                  & Leather       & 6 (125,000 gp) \\
          % early metal resistance, encumbrance                                                                                                               
          Starmetal\fn{1}            & \mult2                 & \plus2           & Very heavy                              & Metal         & 2 (200 gp)     \\
          Starmetal, pure\fn{1}      & \mult4                 & \plus2           & Very heavy                              & Metal         & 4 (5,000 gp)   \\
          % early leather resistance, fire vulnerability
          Vineweave                  & \mult2                 & \tdash          & Fire vulnerability                      & Leather       & 2 (200 gp)   \\
          Vineweave, braided         & \mult4                 & \tdash          & Fire vulnerability                      & Leather       & 4 (5,000 gp) \\
        \end{dtabularx}
        1. This armor has special rules explained below. \\
      \end{dtable!*}

      \parhead{Adamantine} Adamantine body armor is heavier than other armor.
      Light and medium body armor weighs as much as a normal item of its size category.
      Heavy body armor is a \glossterm{heavyweight} item of its size category.
      For Medium creatures, light and medium adamantine armor requires a minimum Strength of 2 to wear, and heavy adamantine requires a minimum Strength of 5.
      In addition, it provides \mult6 \glossterm{damage resistance} and increases your \glossterm{encumbrance} by 2.

      \parhead{Adamantine, Pure} Pure adamantine body armor is heavier than other armor, as described by adamantine armor.
      In addition, it provides \mult12 \glossterm{damage resistance} and increases your \glossterm{encumbrance} by 2.

      \parhead{Cold Iron} Cold iron armor grants you a \plus1 bonus to your defenses against \magical abilities.
      However, its Armor defense bonus is reduced by 1.

      \parhead{Cold Iron, Pure} Pure cold iron armor grants you a \plus2 bonus to your defenses against \magical abilities, and provides \mult2 damage resistance.
      However, its Armor defense bonus is reduced by 1.

      \parhead{Diamondsteel} Diamondsteel body armor grants you a \plus2 bonus to your defenses when determining whether a \glossterm{strike} gets a \glossterm{critical hit} against you instead of a normal hit.
      In addition, it provides \mult2 \glossterm{damage resistance}.

      \parhead{Diamondsteel, Pure} Pure diamondsteel body armor grants you a \plus2 bonus to your defenses when determining whether any attack gets a \glossterm{critical hit} against you instead of a normal hit.
      In addition, it provides \mult4 \glossterm{damage resistance}.

      \parhead{Dragonhide} Each dragonhide body armor is made from the hide of a particular type of dragon.
      You are \trait{impervious} to attacks with that dragon's ability tag (see \tref{Dragon Types}).
      In addition, it provides \mult3 \glossterm{damage resistance}.

      \parhead{Dragonhide, Ancient} Each ancient dragonhide body armor is made from the hide of a particular type of dragon.
      You are \trait{immune} to attacks with that dragon's ability tag (see \tref{Dragon Types}).
      In addition, it provides \mult6 \glossterm{damage resistance}.

      \parhead{Dragonscale} Each dragonscale body armor is made from the scales of a particular type of dragon.
      You are \trait{impervious} to attacks with that dragon's ability tag (see \tref{Dragon Types}).
      In addition, it provides \mult3 \glossterm{damage resistance}.
      Dragonscale is not considered to be metal, which may affect abilities like the \spell{heat metal} spell.

      \parhead{Dragonscale, Ancient} Each ancient dragonscale body armor is made from the scales of a particular type of dragon.
      You are \trait{immune} to attacks with that dragon's ability tag (see \tref{Dragon Types}).
      In addition, it provides \mult6 \glossterm{damage resistance}.
      Dragonscale is not considered to be metal, which may affect abilities like the \spell{heat metal} spell.

      \parhead{Shadowweave} Shadowweave body armor is exceptionally stealthy.
      You gain a \plus5 \glossterm{enhancement bonus} to the Stealth skill while wearing shadowweave armor.
      In addition, it provides \mult3 \glossterm{damage resistance}.

      \parhead{Shadowweave, Umbral} Umbral shadowweave body armor is exceptionally stealthy.
      You gain a \plus5 \glossterm{enhancement bonus} to the Stealth skill while wearing umbral shadowweave armor.
      In addition, it provides \mult6 \glossterm{damage resistance}.

      \parhead{Starmetal} Starmetal body armor is heavier than other armor, as described by adamantine armor.
      In addition, it provides \mult2 \glossterm{damage resistance} and increases your \glossterm{encumbrance} by 2.

      \parhead{Starmetal, Pure} Pure starmetal body armor is heavier than other armor, as described by adamantine armor.
      In addition, it provides \mult4 \glossterm{damage resistance} and increases your \glossterm{encumbrance} by 2.

      \parhead{Vineweave} Vineweave body armor provides \mult2 \glossterm{damage resistance}.
      However, it makes you \vulnerable to \atFire attacks.

      \parhead{Vineweave, Braided} Braided vineweave body armor provides \mult4 \glossterm{damage resistance}.
      However, it makes you \vulnerable to \atFire attacks.

      % Magic armor is a highly limited slot.
      % They have the same power level as self-attune spells.
      \begin{longcolumn}
        \section{Magic Armor}
          \begin{longtablepreface}
            Magic body armor must be worn to gain its effects, while magic shields must be wielded.
            You cannot imbue magic body armor effects on ordinary clothing, even if that clothing is worn on the body instead of armor.

            \subsection{Magic Armor Damage Resistance}\label{Magic Armor Damage Resistance}
              While you are attuned to magical body armor, that armor gains a multiplier to the damage resistance it provides.
              This multiplier does not apply to any special properties the armor might have, such as an enhancement bonus to your damage resistance.
              It only applies to the normal damage resistance normally provided by body armor of that type.
              The magnitude of the multiplier is based on the magic item's rank, as listed below.

              \begin{itemize}
                \item Rank 0--4: \mult1
                \item Rank 3: \mult2
                \item Rank 4: \mult3
                \item Rank 5: \mult4
                \item Rank 6: \mult6
                \item Rank 7: \mult8
              \end{itemize}
          \end{longtablepreface}

          \input{generated/magic_armor_table.tex}

      \end{longcolumn}

      \input{generated/magic_armor.tex}

      % Magic apparel is a loosely limited slot.
      % They are one rank behind self-attune spells.
      \begin{longcolumn}
        \section{Magic Apparel}
          \begin{longtablepreface}
            \includegraphics[width=\columnwidth]{equipment/magic apparel}
            Magic apparel items must be worn to gain their effects.

            \subsection{Body Slots}
              The main limiting factor on how many items you can have equipped is your attunement points, not the physical location of your items on your body.
              However, there are limits to how many items you can wear of the same type, as described below.
              For item types not listed here, use reasonable judgment about what would be plausible.
              \begin{itemize}
                \item Amulet: Up to 2
                \item Belt: Up to 2
                \item Boots: Up to 1
                \item Circlet: Up to 2
                \item Cloak: Up to 2
                \item Gauntlets: Up to 1 (separate from gloves)
                \item Gloves: Up to 1 (separate from gauntlets)
                \item Rings: Up to 5 per hand
              \end{itemize}
          \end{longtablepreface}

          
\begin{longtabuwrapper}
\begin{longtabu}{l l X l}
\lcaption{Apparel Items} \\
\tb{Name} & \tb{Level} & \tb{Description} & \tb{Page} \\
\bottomrule
Belt of Healing & \nth{1} & Grants healing & \pageref{item:Belt of Healing} \\
Bracers of Archery & \nth{1} & Grants bow proficiency & \pageref{item:Bracers of Archery} \\
Amulet of Health & \nth{2} & Increases your hit points & \pageref{item:Amulet of Health} \\
Boots of the Winterlands & \nth{2} & Eases travel in cold areas & \pageref{item:Boots of the Winterlands} \\
Bracers of Armor & \nth{2} & Grants invisible armor & \pageref{item:Bracers of Armor} \\
Gauntlet of the Ram & \nth{2} & Shoves foe when used to strike & \pageref{item:Gauntlet of the Ram} \\
Gauntlets of Improvisation & \nth{2} & Grants \plus1d damage with improvised weapons & \pageref{item:Gauntlets of Improvisation} \\
Lifekeeping Belt & \nth{2} & Reduces vital damage penalties by 2 & \pageref{item:Lifekeeping Belt} \\
Ring of Elemental Endurance & \nth{2} & Grants tolerance of temperature extremes & \pageref{item:Ring of Elemental Endurance} \\
Shield of Arrow Deflection & \nth{2} & Can block small projectiles & \pageref{item:Shield of Arrow Deflection} \\
Shield of Bashing & \nth{2} & Deals \plus1d damage & \pageref{item:Shield of Bashing} \\
Torchlight Gloves & \nth{2} & Sheds light as a torch & \pageref{item:Torchlight Gloves} \\
Boots of Freedom & \nth{3} & Grants immunity to magical mobility restrictions & \pageref{item:Boots of Freedom} \\
Ocular Circlet & \nth{3} & Can allow you to see at a distance & \pageref{item:Ocular Circlet} \\
Ring of Nourishment & \nth{3} & Provides food and water & \pageref{item:Ring of Nourishment} \\
Armor of Energy Resistance & \nth{4} & Reduces energy damage & \pageref{item:Armor of Energy Resistance} \\
Boots of Earth's Embrace & \nth{4} & Grants immunity to forced movement & \pageref{item:Boots of Earth's Embrace} \\
Boots of Elvenkind & \nth{4} & Grants \plus2 Stealth & \pageref{item:Boots of Elvenkind} \\
Bracers of Repulsion & \nth{4} & Can shove nearby creatures back & \pageref{item:Bracers of Repulsion} \\
Circlet of Blasting & \nth{4} & Can blast foe with fire & \pageref{item:Circlet of Blasting} \\
Circlet of Persuasion & \nth{4} & Grants \plus2 Persuasion & \pageref{item:Circlet of Persuasion} \\
Featherlight Armor & \nth{4} & Reduces encumbrance by 1 & \pageref{item:Featherlight Armor} \\
Hidden Armor & \nth{4} & Can look like normal clothing & \pageref{item:Hidden Armor} \\
Mask of Water Breathing & \nth{4} & Allows breathing water like air & \pageref{item:Mask of Water Breathing} \\
Throwing Gloves & \nth{4} & Allows throwing any item accurately & \pageref{item:Throwing Gloves} \\
Crown of Flame & \nth{5} & Grants nearby allies immunity to fire damage & \pageref{item:Crown of Flame} \\
Ring of Energy Resistance & \nth{5} & Reduces energy damage & \pageref{item:Ring of Energy Resistance} \\
Shield of Arrow Catching & \nth{5} & Redirects small nearby projectiles to hit you & \pageref{item:Shield of Arrow Catching} \\
Amulet of Mighty Fists & \nth{6} & Grants \plus1d damage with your body & \pageref{item:Amulet of Mighty Fists} \\
Amulet of Nondetection & \nth{6} & Grants \plus4 to defenses against detection & \pageref{item:Amulet of Nondetection} \\
Boots of Speed & \nth{6} & Increases speed by ten feet & \pageref{item:Boots of Speed} \\
Shield of Boulder Deflection & \nth{6} & Can block large projectiles & \pageref{item:Shield of Boulder Deflection} \\
Armor of Fortification & \nth{7} & Reduces critical hits from strikes & \pageref{item:Armor of Fortification} \\
Assassin's Cloak & \nth{7} & Grants invisibility while inactive & \pageref{item:Assassin's Cloak} \\
Belt of Healing, Greater & \nth{7} & Grants more healing & \pageref{item:Belt of Healing, Greater} \\
Boots of Water Walking & \nth{7} & Allows walking on liquids & \pageref{item:Boots of Water Walking} \\
Boots of the Skydancer & \nth{7} & Can walk on air & \pageref{item:Boots of the Skydancer} \\
Bracers of Archery, Greater & \nth{7} & Grants bow proficiency, \plus1 ranged accuracy & \pageref{item:Bracers of Archery, Greater} \\
Crown of Lightning & \nth{7} & Continuously damages nearby enemies & \pageref{item:Crown of Lightning} \\
Gauntlet of the Ram, Greater & \nth{7} & Shoves foe hard when use to strike & \pageref{item:Gauntlet of the Ram, Greater} \\
Gauntlets of Improvisation, Greater & \nth{7} & Grants \plus2d damage with improvised weapons & \pageref{item:Gauntlets of Improvisation, Greater} \\
Lifekeeping Belt, Greater & \nth{7} & Reduces vital damage penalties by 4 & \pageref{item:Lifekeeping Belt, Greater} \\
Ring of Sustenance & \nth{7} & Provides food, water, and rest & \pageref{item:Ring of Sustenance} \\
Amulet of Health, Greater & \nth{8} & Greatly increases your hit points & \pageref{item:Amulet of Health, Greater} \\
Armor of Invulnerability & \nth{8} & Reduces damage from physical attacks & \pageref{item:Armor of Invulnerability} \\
Boots of Gravitation & \nth{8} & Redirects personal gravity & \pageref{item:Boots of Gravitation} \\
Bracers of Repulsion, Greater & \nth{8} & Can shove foes back & \pageref{item:Bracers of Repulsion, Greater} \\
Cloak of Mist & \nth{8} & Fills nearby area with fog & \pageref{item:Cloak of Mist} \\
Ring of Protection & \nth{8} & Grants \plus1 to Armor and Reflex defenses & \pageref{item:Ring of Protection} \\
Shield of Arrow Deflection, Greater & \nth{8} & Blocks small projectiles & \pageref{item:Shield of Arrow Deflection, Greater} \\
Shield of Boulder Catching & \nth{8} & Redirects large nearby projectiles to hit you & \pageref{item:Shield of Boulder Catching} \\
Vanishing Cloak & \nth{8} & Can teleport a short distance and grant invisibility & \pageref{item:Vanishing Cloak} \\
Boots of Freedom, Greater & \nth{9} & Grants immunity to almost all mobility restrictions & \pageref{item:Boots of Freedom, Greater} \\
Crown of Thunder & \nth{9} & Continously deafens nearby enemies & \pageref{item:Crown of Thunder} \\
Greatreach Bracers & \nth{9} & Increases reach by five feet & \pageref{item:Greatreach Bracers} \\
Hidden Armor, Greater & \nth{9} & Can look and sound like normal clothing & \pageref{item:Hidden Armor, Greater} \\
Mask of Air & \nth{9} & Allows breathing in any environment & \pageref{item:Mask of Air} \\
Ocular Circlet, Greater & \nth{9} & description & \pageref{item:Ocular Circlet, Greater} \\
Boots of Speed, Greater & \nth{10} & Increases speed by twenty feet & \pageref{item:Boots of Speed, Greater} \\
Circlet of Blasting, Greater & \nth{10} & Can blast foe with intense fire & \pageref{item:Circlet of Blasting, Greater} \\
Crater Boots & \nth{10} & Deals your falling damage to enemies & \pageref{item:Crater Boots} \\
Featherlight Armor, Greater & \nth{10} & Reduces encumbrance by 2 & \pageref{item:Featherlight Armor, Greater} \\
Gloves of Spell Investment & \nth{10} & description & \pageref{item:Gloves of Spell Investment} \\
Shield of Arrow Catching, Greater & \nth{10} & Selectively redirects small nearby projectiles to hit you & \pageref{item:Shield of Arrow Catching, Greater} \\
Winged Boots & \nth{10} & Grants limited flight & \pageref{item:Winged Boots} \\
Crown of Frost & \nth{11} & Continuously damages and fatigues nearby enemies & \pageref{item:Crown of Frost} \\
Hexward Amulet & \nth{11} & Grants \plus4 defenses against targeted magical attacks & \pageref{item:Hexward Amulet} \\
Ring of Regeneration & \nth{11} & Grants fast healing & \pageref{item:Ring of Regeneration} \\
Amulet of the Planes & \nth{12} & Aids travel with \ritual{plane shift} & \pageref{item:Amulet of the Planes} \\
Armor of Energy Resistance, Greater & \nth{12} & Drastically reduces energy damage & \pageref{item:Armor of Energy Resistance, Greater} \\
Armor of Fortification, Mystic & \nth{12} & Reduces critical hits from all attacks & \pageref{item:Armor of Fortification, Mystic} \\
Seven League Boots & \nth{12} & Teleport seven leages with a step & \pageref{item:Seven League Boots} \\
Shield of Bashing, Greater & \nth{12} & Deals \plus2d damage & \pageref{item:Shield of Bashing, Greater} \\
Shield of Boulder Deflection, Greater & \nth{12} & Blocks large projectiles & \pageref{item:Shield of Boulder Deflection, Greater} \\
Shield of Mystic Reflection & \nth{12} & React to reflect magical attacks & \pageref{item:Shield of Mystic Reflection} \\
Lifekeeping Belt, Supreme & \nth{13} & Reduces vital damage penalties by 8 & \pageref{item:Lifekeeping Belt, Supreme} \\
Ring of Energy Resistance, Greater & \nth{13} & Drastically reduces energy damage & \pageref{item:Ring of Energy Resistance, Greater} \\
Titan Gauntlets & \nth{13} & Grants \plus1d damage with strikes & \pageref{item:Titan Gauntlets} \\
Amulet of Mighty Fists, Greater & \nth{14} & Grants \plus2d damage with your body & \pageref{item:Amulet of Mighty Fists, Greater} \\
Amulet of Nondetection, Greater & \nth{14} & Grants \plus8 to defenses against detection & \pageref{item:Amulet of Nondetection, Greater} \\
Boots of Speed, Supreme & \nth{14} & Increases speed by thirty feet & \pageref{item:Boots of Speed, Supreme} \\
Armor of Fortification, Greater & \nth{15} & Drastically reduces critical hits from strikes & \pageref{item:Armor of Fortification, Greater} \\
Armor of Invulnerability, Greater & \nth{16} & Drastically reduces damage from physical attacks & \pageref{item:Armor of Invulnerability, Greater} \\
Astral Boots & \nth{16} & Allows teleporting instead of moving & \pageref{item:Astral Boots} \\
Circlet of Blasting, Supreme & \nth{16} & Can blast foe with supremely intense fire & \pageref{item:Circlet of Blasting, Supreme} \\
Cloak of Mist, Greater & \nth{16} & Fills nearby area with thick fog & \pageref{item:Cloak of Mist, Greater} \\
Ring of Protection, Greater & \nth{16} & Grants \plus2 to Armor and Reflex defenses & \pageref{item:Ring of Protection, Greater} \\
Assassin's Cloak, Greater & \nth{17} & Grants invisibility while not attacking & \pageref{item:Assassin's Cloak, Greater} \\
Greatreach Bracers, Greater & \nth{17} & Increases reach by ten feet & \pageref{item:Greatreach Bracers, Greater} \\
Hexproof Amulet, Greater & \nth{17} & Grants \plus6 defenses against targeted magical attacks & \pageref{item:Hexproof Amulet, Greater} \\
Gloves of Spell Investment, Greater & \nth{18} & description & \pageref{item:Gloves of Spell Investment, Greater} \\
Amulet of the Planes, Greater & \nth{19} & Aid travel with \ritual{plane shift} subrituals & \pageref{item:Amulet of the Planes, Greater} \\
\end{longtabu}
\end{longtabuwrapper}


      \end{longcolumn}
\section{Magic Apparel Descriptions}

  
\lowercase{\hypertarget{item:Amulet of Health}{}}\label{item:Amulet of Health}
\hypertarget{item:Amulet of Health}{\subsubsection{Amulet of Health\hfill\nth{2} (125 gp)}}

You gain a \glossterm{magic bonus} equal to this item's \glossterm{power} to your \glossterm{wound threshold}.



\vspace{0.25em}
\spelltwocol{\textbf{Type}: Amulet}{}
\textbf{Materials}: Jewelry


\lowercase{\hypertarget{item:Amulet of Health, Greater}{}}\label{item:Amulet of Health, Greater}
\hypertarget{item:Amulet of Health, Greater}{\subsubsection{Amulet of Health, Greater\hfill\nth{8} (2,750 gp)}}

You gain a \glossterm{magic bonus} equal to twice this item's \glossterm{power} to your \glossterm{wound threshold}.



\vspace{0.25em}
\spelltwocol{\textbf{Type}: Amulet}{}
\textbf{Materials}: Jewelry


\lowercase{\hypertarget{item:Amulet of Mighty Fists}{}}\label{item:Amulet of Mighty Fists}
\hypertarget{item:Amulet of Mighty Fists}{\subsubsection{Amulet of Mighty Fists\hfill\nth{8} (2,750 gp)}}

You gain a \plus2 \glossterm{magic bonus} to \glossterm{power} with \glossterm{unarmed attacks} and natural weapons.



\vspace{0.25em}
\spelltwocol{\textbf{Type}: Amulet}{}
\textbf{Materials}: Jewelry


\lowercase{\hypertarget{item:Amulet of Mighty Fists, Greater}{}}\label{item:Amulet of Mighty Fists, Greater}
\hypertarget{item:Amulet of Mighty Fists, Greater}{\subsubsection{Amulet of Mighty Fists, Greater\hfill\nth{16} (85,000 gp)}}

You gain a \plus4 \glossterm{magic bonus} to \glossterm{power} with \glossterm{unarmed attacks} and natural weapons.



\vspace{0.25em}
\spelltwocol{\textbf{Type}: Amulet}{}
\textbf{Materials}: Jewelry


\lowercase{\hypertarget{item:Amulet of Nondetection}{}}\label{item:Amulet of Nondetection}
\hypertarget{item:Amulet of Nondetection}{\subsubsection{Amulet of Nondetection\hfill\nth{6} (1,200 gp)}}

You gain a \plus4 bonus to defenses against abilities with the \glossterm{Detection} or \glossterm{Scrying} tags.



\vspace{0.25em}
\spelltwocol{\textbf{Type}: Amulet}{}
\textbf{Materials}: Jewelry


\lowercase{\hypertarget{item:Amulet of Nondetection, Greater}{}}\label{item:Amulet of Nondetection, Greater}
\hypertarget{item:Amulet of Nondetection, Greater}{\subsubsection{Amulet of Nondetection, Greater\hfill\nth{14} (37,000 gp)}}

You gain a \plus8 bonus to defenses against abilities with the \glossterm{Detection} or \glossterm{Scrying} tags.



\vspace{0.25em}
\spelltwocol{\textbf{Type}: Amulet}{}
\textbf{Materials}: Jewelry


\lowercase{\hypertarget{item:Amulet of the Planes}{}}\label{item:Amulet of the Planes}
\hypertarget{item:Amulet of the Planes}{\subsubsection{Amulet of the Planes\hfill\nth{12} (16,000 gp)}}

When you perform the \ritual{plane shift} ritual, this amulet provides all action points required.
This does not grant you the ability to perform the \ritual{plane shift} ritual if you could not already.



\vspace{0.25em}
\spelltwocol{\textbf{Type}: Amulet}{}
\textbf{Materials}: Jewelry


\lowercase{\hypertarget{item:Armor of Energy Resistance}{}}\label{item:Armor of Energy Resistance}
\hypertarget{item:Armor of Energy Resistance}{\subsubsection{Armor of Energy Resistance\hfill\nth{5} (800 gp)}}

You gain a \glossterm{magic bonus} equal to half the item's \glossterm{power} to \glossterm{resistances} against \glossterm{energy damage}.
When you resist energy damage, it sheds light as a torch until the end of the next round.
The color of the light depends on the energy damage resisted: green for acid, blue for cold, yellow for electricity, and red for fire.



\vspace{0.25em}
\spelltwocol{\textbf{Type}: Body armor}{}
\textbf{Materials}: Bone, metal


\lowercase{\hypertarget{item:Armor of Energy Resistance, Greater}{}}\label{item:Armor of Energy Resistance, Greater}
\hypertarget{item:Armor of Energy Resistance, Greater}{\subsubsection{Armor of Energy Resistance, Greater\hfill\nth{14} (37,000 gp)}}

This item functions like the \mitem{armor of energy resistance} item, except that the bonus is equal to the item's \glossterm{power}.



\vspace{0.25em}
\spelltwocol{\textbf{Type}: Body armor}{}
\textbf{Materials}: Bone, metal


\lowercase{\hypertarget{item:Armor of Fortification}{}}\label{item:Armor of Fortification}
\hypertarget{item:Armor of Fortification}{\subsubsection{Armor of Fortification\hfill\nth{7} (1,800 gp)}}

You gain a \plus5 bonus to defenses when determining whether a \glossterm{strike} gets a \glossterm{critical hit} against you instead of a normal hit.



\vspace{0.25em}
\spelltwocol{\textbf{Type}: Body armor}{}
\textbf{Materials}: Bone, metal


\lowercase{\hypertarget{item:Armor of Fortification, Greater}{}}\label{item:Armor of Fortification, Greater}
\hypertarget{item:Armor of Fortification, Greater}{\subsubsection{Armor of Fortification, Greater\hfill\nth{15} (55,000 gp)}}

This item functions like the \mitem{armor of fortification} item, except that the bonus increases to \plus10.



\vspace{0.25em}
\spelltwocol{\textbf{Type}: Body armor}{}
\textbf{Materials}: Bone, metal


\lowercase{\hypertarget{item:Armor of Fortification, Mystic}{}}\label{item:Armor of Fortification, Mystic}
\hypertarget{item:Armor of Fortification, Mystic}{\subsubsection{Armor of Fortification, Mystic\hfill\nth{12} (16,000 gp)}}

This item functions like the \mitem{armor of fortification} item, except that it applies against all attacks instead of only against; \glossterm{strikes}.



\vspace{0.25em}
\spelltwocol{\textbf{Type}: Body armor}{}
\textbf{Materials}: Bone, metal


\lowercase{\hypertarget{item:Armor of Invulnerability}{}}\label{item:Armor of Invulnerability}
\hypertarget{item:Armor of Invulnerability}{\subsubsection{Armor of Invulnerability\hfill\nth{8} (2,750 gp)}}

The armor's bonus to \glossterm{resistances} based on its armor type is doubled.



\vspace{0.25em}
\spelltwocol{\textbf{Type}: Body armor}{}
\textbf{Materials}: Bone, metal


\lowercase{\hypertarget{item:Armor of Invulnerability, Greater}{}}\label{item:Armor of Invulnerability, Greater}
\hypertarget{item:Armor of Invulnerability, Greater}{\subsubsection{Armor of Invulnerability, Greater\hfill\nth{14} (37,000 gp)}}

The armor's bonus to \glossterm{resistances} based on its armor type is tripled.



\vspace{0.25em}
\spelltwocol{\textbf{Type}: Body armor}{}
\textbf{Materials}: Bone, metal


\lowercase{\hypertarget{item:Armor of Invulnerability, Supreme}{}}\label{item:Armor of Invulnerability, Supreme}
\hypertarget{item:Armor of Invulnerability, Supreme}{\subsubsection{Armor of Invulnerability, Supreme\hfill\nth{20} (400,000 gp)}}

The armor's bonus to \glossterm{resistances} based on its armor type is quadrupled.



\vspace{0.25em}
\spelltwocol{\textbf{Type}: Body armor}{}
\textbf{Materials}: Bone, metal


\lowercase{\hypertarget{item:Assassin's Cloak}{}}\label{item:Assassin's Cloak}
\hypertarget{item:Assassin's Cloak}{\subsubsection{Assassin's Cloak\hfill\nth{7} (1,800 gp)}}

At the end of each round, if you took no actions that round, you become \glossterm{invisible} until the end of the next round.



\vspace{0.25em}
\spelltwocol{\textbf{Type}: Cloak}{\parhead*{Tags} \glossterm{Sensation}}
\textbf{Materials}: Textiles


\lowercase{\hypertarget{item:Assassin's Cloak, Greater}{}}\label{item:Assassin's Cloak, Greater}
\hypertarget{item:Assassin's Cloak, Greater}{\subsubsection{Assassin's Cloak, Greater\hfill\nth{17} (125,000 gp)}}

At the end of each round, if you did not attack a creature that round, you become \glossterm{invisible} until the end of the next round.



\vspace{0.25em}
\spelltwocol{\textbf{Type}: Cloak}{\parhead*{Tags} \glossterm{Sensation}}
\textbf{Materials}: Textiles


\lowercase{\hypertarget{item:Astral Boots}{}}\label{item:Astral Boots}
\hypertarget{item:Astral Boots}{\subsubsection{Astral Boots\hfill\nth{16} (85,000 gp)}}

When you move, you can teleport the same distance instead.
This does not change the total distance you can move, but you can teleport in any direction, even vertically.
You cannot teleport to locations you do not have \glossterm{line of sight} and \glossterm{line of effect} to.



\vspace{0.25em}
\spelltwocol{\textbf{Type}: Boots}{}
\textbf{Materials}: Bone, leather, metal


\lowercase{\hypertarget{item:Belt of Healing}{}}\label{item:Belt of Healing}
\hypertarget{item:Belt of Healing}{\subsubsection{Belt of Healing\hfill\nth{6} (1,200 gp)}}

As a standard action, you can use this belt to regain a \glossterm{hit point}.
You can only use this item once between \glossterm{short rests}.



\vspace{0.25em}
\spelltwocol{\textbf{Type}: Belt}{}
\textbf{Materials}: Leather, textiles


\lowercase{\hypertarget{item:Belt of Healing, Greater}{}}\label{item:Belt of Healing, Greater}
\hypertarget{item:Belt of Healing, Greater}{\subsubsection{Belt of Healing, Greater\hfill\nth{14} (37,000 gp)}}

This item functions like the \textit{belt of healing}, except that you can regain two \glossterm{hit points} instead of one.



\vspace{0.25em}
\spelltwocol{\textbf{Type}: Belt}{}
\textbf{Materials}: Leather, textiles


\lowercase{\hypertarget{item:Boots of Earth's Embrace}{}}\label{item:Boots of Earth's Embrace}
\hypertarget{item:Boots of Earth's Embrace}{\subsubsection{Boots of Earth's Embrace\hfill\nth{4} (500 gp)}}

While you are standing on solid ground, you are immune to effects that would force you to move.
This does not protect you from other effects of those attacks, such as damage.



\vspace{0.25em}
\spelltwocol{\textbf{Type}: Boots}{}
\textbf{Materials}: Bone, leather, metal


\lowercase{\hypertarget{item:Boots of Elvenkind}{}}\label{item:Boots of Elvenkind}
\hypertarget{item:Boots of Elvenkind}{\subsubsection{Boots of Elvenkind\hfill\nth{4} (500 gp)}}

You gain a \plus2 \glossterm{magic bonus} to the Stealth skill (see \pcref{Stealth}).



\vspace{0.25em}
\spelltwocol{\textbf{Type}: Boots}{}
\textbf{Materials}: Bone, leather, metal


\lowercase{\hypertarget{item:Boots of Freedom}{}}\label{item:Boots of Freedom}
\hypertarget{item:Boots of Freedom}{\subsubsection{Boots of Freedom\hfill\nth{3} (250 gp)}}

You are immune to magical effects that restrict your mobility.
This does not prevent physical obstacles from affecting you, such as \glossterm{difficult terrain}.



\vspace{0.25em}
\spelltwocol{\textbf{Type}: Boots}{}
\textbf{Materials}: Bone, leather, metal


\lowercase{\hypertarget{item:Boots of Freedom, Greater}{}}\label{item:Boots of Freedom, Greater}
\hypertarget{item:Boots of Freedom, Greater}{\subsubsection{Boots of Freedom, Greater\hfill\nth{9} (4,000 gp)}}

You are immune to all effects that restrict your mobility, including nonmagical effects such as \glossterm{difficult terrain}.
This removes all penalties you would suffer for acting underwater, except for those relating to using ranged weapons.
This does not prevent you from being \grappled, but you gain a \plus10 bonus to defenses against the \textit{grapple} ability (see \pcref{Grapple}).



\vspace{0.25em}
\spelltwocol{\textbf{Type}: Boots}{}
\textbf{Materials}: Bone, leather, metal


\lowercase{\hypertarget{item:Boots of Gravitation}{}}\label{item:Boots of Gravitation}
\hypertarget{item:Boots of Gravitation}{\subsubsection{Boots of Gravitation\hfill\nth{8} (2,750 gp)}}

While these boots are within 5 feet of a solid surface, gravity pulls you towards the solid surface closest to your boots rather than in the normal direction.
This can allow you to walk easily on walls or even ceilings.



\vspace{0.25em}
\spelltwocol{\textbf{Type}: Boots}{}
\textbf{Materials}: Bone, leather, metal


\lowercase{\hypertarget{item:Boots of Speed}{}}\label{item:Boots of Speed}
\hypertarget{item:Boots of Speed}{\subsubsection{Boots of Speed\hfill\nth{6} (1,200 gp)}}

You gain a \plus10 foot \glossterm{magic bonus} to your land speed, up to a maximum of double your normal speed.



\vspace{0.25em}
\spelltwocol{\textbf{Type}: Boots}{}
\textbf{Materials}: Bone, leather, metal


\lowercase{\hypertarget{item:Boots of Speed, Greater}{}}\label{item:Boots of Speed, Greater}
\hypertarget{item:Boots of Speed, Greater}{\subsubsection{Boots of Speed, Greater\hfill\nth{10} (6,500 gp)}}

You gain a \plus20 foot \glossterm{magic bonus} to your land speed, up to a maximum of double your normal speed.



\vspace{0.25em}
\spelltwocol{\textbf{Type}: Boots}{}
\textbf{Materials}: Bone, leather, metal


\lowercase{\hypertarget{item:Boots of Speed, Supreme}{}}\label{item:Boots of Speed, Supreme}
\hypertarget{item:Boots of Speed, Supreme}{\subsubsection{Boots of Speed, Supreme\hfill\nth{14} (37,000 gp)}}

You gain a \plus30 foot \glossterm{magic bonus} to your land speed, up to a maximum of double your normal speed.



\vspace{0.25em}
\spelltwocol{\textbf{Type}: Boots}{}
\textbf{Materials}: Bone, leather, metal


\lowercase{\hypertarget{item:Boots of Water Walking}{}}\label{item:Boots of Water Walking}
\hypertarget{item:Boots of Water Walking}{\subsubsection{Boots of Water Walking\hfill\nth{7} (1,800 gp)}}

You treat the surface of all liquids as if they were firm ground.
Your feet hover about an inch above the liquid's surface, allowing you to traverse dangerous liquids without harm as long as the surface is calm.

If you are below the surface of the liquid, you rise towards the surface at a rate of 60 feet per round.
Thick liquids, such as mud and lava, may cause you to rise more slowly.



\vspace{0.25em}
\spelltwocol{\textbf{Type}: Boots}{}
\textbf{Materials}: Bone, leather, metal


\lowercase{\hypertarget{item:Boots of the Skydancer}{}}\label{item:Boots of the Skydancer}
\hypertarget{item:Boots of the Skydancer}{\subsubsection{Boots of the Skydancer\hfill\nth{7} (1,800 gp)}}

As a \glossterm{free action}, you can activate these boots.
When you do, you may treat air as if it were solid ground to your feet for the rest of the current phase.
You may selectively choose when to treat the air as solid ground, allowing you to walk or jump on air freely.
After using this ability, you cannot use it again until these boots touch the ground.



\vspace{0.25em}
\spelltwocol{\textbf{Type}: Boots}{\parhead*{Tags} \glossterm{Swift}}
\textbf{Materials}: Bone, leather, metal


\lowercase{\hypertarget{item:Boots of the Skydancer, Greater}{}}\label{item:Boots of the Skydancer, Greater}
\hypertarget{item:Boots of the Skydancer, Greater}{\subsubsection{Boots of the Skydancer, Greater\hfill\nth{13} (25,000 gp)}}

This item functions like the \magicitem{boots of the skydancer}, except that the ability lasts until the end of the round.
In addition, you can use this item twice before the boots touch the ground.



\vspace{0.25em}
\spelltwocol{\textbf{Type}: Boots}{\parhead*{Tags} \glossterm{Swift}}
\textbf{Materials}: Bone, leather, metal


\lowercase{\hypertarget{item:Boots of the Winterlands}{}}\label{item:Boots of the Winterlands}
\hypertarget{item:Boots of the Winterlands}{\subsubsection{Boots of the Winterlands\hfill\nth{2} (125 gp)}}

You can travel across snow and ice without slipping or suffering movement penalties for the terrain.
% TODO: degree symbol?
In addition, the boots keep you warn, protecting you in environments as cold as \minus50 Fahrenheit.



\vspace{0.25em}
\spelltwocol{\textbf{Type}: Boots}{}
\textbf{Materials}: Bone, leather, metal


\lowercase{\hypertarget{item:Bracers of Archery}{}}\label{item:Bracers of Archery}
\hypertarget{item:Bracers of Archery}{\subsubsection{Bracers of Archery\hfill\nth{1} (50 gp)}}

You are proficient with bows.



\vspace{0.25em}
\spelltwocol{\textbf{Type}: Bracers}{}
\textbf{Materials}: Bone, leather, metal, wood


\lowercase{\hypertarget{item:Bracers of Archery, Greater}{}}\label{item:Bracers of Archery, Greater}
\hypertarget{item:Bracers of Archery, Greater}{\subsubsection{Bracers of Archery, Greater\hfill\nth{7} (1,800 gp)}}

You are proficient with bows.
In addition, you gain a \plus1 \glossterm{magic bonus} to \glossterm{accuracy} with ranged \glossterm{strikes}.



\vspace{0.25em}
\spelltwocol{\textbf{Type}: Bracers}{}
\textbf{Materials}: Bone, leather, metal, wood


\lowercase{\hypertarget{item:Bracers of Armor}{}}\label{item:Bracers of Armor}
\hypertarget{item:Bracers of Armor}{\subsubsection{Bracers of Armor\hfill\nth{2} (125 gp)}}

You gain a \plus2 bonus to Armor defense.
The protection from these bracers is treated as body armor, and it does not stack with any other body armor you wear.



\vspace{0.25em}
\spelltwocol{\textbf{Type}: Bracers}{}
\textbf{Materials}: Bone, leather, metal, wood


\lowercase{\hypertarget{item:Bracers of Repulsion}{}}\label{item:Bracers of Repulsion}
\hypertarget{item:Bracers of Repulsion}{\subsubsection{Bracers of Repulsion\hfill\nth{7} (1,800 gp)}}

As a standard action, you can activate these bracers.
When you do, they emit a telekinetic burst of force.
Make an attack vs. Fortitude against everything within a \areasmall radius burst from you.
If you use this item during the \glossterm{delayed action phase},
you gain a \plus4 bonus to \glossterm{accuracy} with this attack against any creature that attacked you during the \glossterm{action phase}.
On a hit, you \glossterm{knockback} each target up to 20 feet.



\vspace{0.25em}
\spelltwocol{\textbf{Type}: Bracers}{}
\textbf{Materials}: Bone, leather, metal, wood


\lowercase{\hypertarget{item:Bracers of Repulsion, Greater}{}}\label{item:Bracers of Repulsion, Greater}
\hypertarget{item:Bracers of Repulsion, Greater}{\subsubsection{Bracers of Repulsion, Greater\hfill\nth{15} (55,000 gp)}}

This item functions like the \mitem{bracers of repulsion} item, except that it targets everything within a \arealarge radius burst.



\vspace{0.25em}
\spelltwocol{\textbf{Type}: Bracers}{}
\textbf{Materials}: Bone, leather, metal, wood


\lowercase{\hypertarget{item:Circlet of Blasting}{}}\label{item:Circlet of Blasting}
\hypertarget{item:Circlet of Blasting}{\subsubsection{Circlet of Blasting\hfill\nth{5} (800 gp)}}

As a standard action, you can activate this circlet.
If you do, make an attack vs. Armor against a creature or object within \rngmed range.
\hit The target takes fire \glossterm{standard damage}.



\vspace{0.25em}
\spelltwocol{\textbf{Type}: Circlet}{}
\textbf{Materials}: Bone, metal


\lowercase{\hypertarget{item:Circlet of Blasting, Greater}{}}\label{item:Circlet of Blasting, Greater}
\hypertarget{item:Circlet of Blasting, Greater}{\subsubsection{Circlet of Blasting, Greater\hfill\nth{10} (6,500 gp)}}

This item functions like the \textit{circlet of blasting}, except that it gains a \plus1d bonus to damage.



\vspace{0.25em}
\spelltwocol{\textbf{Type}: Circlet}{}
\textbf{Materials}: Bone, metal


\lowercase{\hypertarget{item:Circlet of Blasting, Supreme}{}}\label{item:Circlet of Blasting, Supreme}
\hypertarget{item:Circlet of Blasting, Supreme}{\subsubsection{Circlet of Blasting, Supreme\hfill\nth{16} (85,000 gp)}}

This item functions like the \textit{circlet of blasting}, except that it gains a \plus2d bonus to damage.



\vspace{0.25em}
\spelltwocol{\textbf{Type}: Circlet}{}
\textbf{Materials}: Bone, metal


\lowercase{\hypertarget{item:Circlet of Persuasion}{}}\label{item:Circlet of Persuasion}
\hypertarget{item:Circlet of Persuasion}{\subsubsection{Circlet of Persuasion\hfill\nth{4} (500 gp)}}

You gain a \plus2 \glossterm{magic bonus} to the Persuasion skill (see \pcref{Persuasion}).



\vspace{0.25em}
\spelltwocol{\textbf{Type}: Circlet}{}
\textbf{Materials}: Bone, metal


\lowercase{\hypertarget{item:Cloak of Mist}{}}\label{item:Cloak of Mist}
\hypertarget{item:Cloak of Mist}{\subsubsection{Cloak of Mist\hfill\nth{8} (2,750 gp)}}

Fog constantly fills a \areamed radius emanation from you.
This fog does not fully block sight, but it provides \concealment.

If a 5-foot square of fog takes fire damage equal to half this item's \glossterm{power}, the fog disappears from that area until the end of the next round.



\vspace{0.25em}
\spelltwocol{\textbf{Type}: Cloak}{\parhead*{Tags} \glossterm{Manifestation}}
\textbf{Materials}: Textiles


\lowercase{\hypertarget{item:Cloak of Mist, Greater}{}}\label{item:Cloak of Mist, Greater}
\hypertarget{item:Cloak of Mist, Greater}{\subsubsection{Cloak of Mist, Greater\hfill\nth{16} (85,000 gp)}}

A thick fog constantly fills a \areamed radius emanation from you.
This fog completely blocks sight beyond 10 feet.
Within that range, it still provides \concealment.

If a 5-foot square of fog takes fire damage equal to this item's \glossterm{power}, the fog disappears from that area until the end of the next round.



\vspace{0.25em}
\spelltwocol{\textbf{Type}: Cloak}{\parhead*{Tags} \glossterm{Manifestation}}
\textbf{Materials}: Textiles


\lowercase{\hypertarget{item:Crater Boots}{}}\label{item:Crater Boots}
\hypertarget{item:Crater Boots}{\subsubsection{Crater Boots\hfill\nth{10} (6,500 gp)}}

% This only works if you only take falling damage during the movement phase, which seems possible?
When you take \glossterm{falling damage}, make an attack vs Reflex against everything within a \areasmall radius from you.
\hit Each target takes damage as if they had fallen the same distance that you fell.
This roll is made separately from the damage roll to determine your falling damage.
\crit As above, and each target is knocked \glossterm{prone}.
This does not deal double damage on a critical hit.



\vspace{0.25em}
\spelltwocol{\textbf{Type}: Boots}{}
\textbf{Materials}: Bone, leather, metal


\lowercase{\hypertarget{item:Crown of Flame}{}}\label{item:Crown of Flame}
\hypertarget{item:Crown of Flame}{\subsubsection{Crown of Flame\hfill\nth{9} (4,000 gp)}}

This crown is continuously on fire.
The flame sheds light as a torch.

You and your \glossterm{allies} within a \arealarge radius emanation from you
gain a \glossterm{magic bonus} equal to this item's \glossterm{power} to \glossterm{resistances} against fire damage.



\vspace{0.25em}
\spelltwocol{\textbf{Type}: Crown}{}
\textbf{Materials}: Bone, metal


\lowercase{\hypertarget{item:Crown of Frost}{}}\label{item:Crown of Frost}
\hypertarget{item:Crown of Frost}{\subsubsection{Crown of Frost\hfill\nth{13} (25,000 gp)}}

At the end of each \glossterm{action phase}, you make an attack vs. Fortitude against all enemies within a \areamed radius emanation from you.
At hit deals cold \glossterm{standard damage} \minus2d.



\vspace{0.25em}
\spelltwocol{\textbf{Type}: Crown}{}
\textbf{Materials}: Bone, metal


\lowercase{\hypertarget{item:Crown of Lightning}{}}\label{item:Crown of Lightning}
\hypertarget{item:Crown of Lightning}{\subsubsection{Crown of Lightning\hfill\nth{7} (1,800 gp)}}

This crown continuously crackles with electricity.
The constant sparks shed light as a torch.

At the end of each \glossterm{action phase}, you make an attack vs. Fortitude against all enemies within a \areamed radius emanation from you.
A hit deals electricity \glossterm{standard damage} \minus3d.



\vspace{0.25em}
\spelltwocol{\textbf{Type}: Crown}{}
\textbf{Materials}: Bone, metal


\lowercase{\hypertarget{item:Crown of Thunder}{}}\label{item:Crown of Thunder}
\hypertarget{item:Crown of Thunder}{\subsubsection{Crown of Thunder\hfill\nth{11} (10,000 gp)}}

The crown constantly emits a low-pitched rumbling.
To you and your \glossterm{allies}, the sound is barely perceptible.
However, all other creatures within a \arealarge radius emanation from you hear the sound as a deafening, continuous roll of thunder.
The noise blocks out all other sounds quieter than thunder, causing them to be \deafened while they remain in the area.



\vspace{0.25em}
\spelltwocol{\textbf{Type}: Crown}{}
\textbf{Materials}: Bone, metal


\lowercase{\hypertarget{item:Featherlight Armor}{}}\label{item:Featherlight Armor}
\hypertarget{item:Featherlight Armor}{\subsubsection{Featherlight Armor\hfill\nth{4} (500 gp)}}

This armor's \glossterm{encumbrance} is reduced by 1.



\vspace{0.25em}
\spelltwocol{\textbf{Type}: Body armor}{}
\textbf{Materials}: Bone, metal


\lowercase{\hypertarget{item:Featherlight Armor, Greater}{}}\label{item:Featherlight Armor, Greater}
\hypertarget{item:Featherlight Armor, Greater}{\subsubsection{Featherlight Armor, Greater\hfill\nth{10} (6,500 gp)}}

This armor's \glossterm{encumbrance} is reduced by 2.



\vspace{0.25em}
\spelltwocol{\textbf{Type}: Body armor}{}
\textbf{Materials}: Bone, metal


\lowercase{\hypertarget{item:Gauntlet of the Ram}{}}\label{item:Gauntlet of the Ram}
\hypertarget{item:Gauntlet of the Ram}{\subsubsection{Gauntlet of the Ram\hfill\nth{2} (125 gp)}}

When you make a \glossterm{strike} with this gauntlet, you also compare the attack result to the target's Fortitude defense.
On a hit, you \glossterm{knockback} the target up to 10 feet.
Making a strike with this gauntlet is equivalent to an \glossterm{unarmed attack}.



\vspace{0.25em}
\spelltwocol{\textbf{Type}: Gauntlet}{}
\textbf{Materials}: Bone, metal, wood


\lowercase{\hypertarget{item:Gauntlet of the Ram, Greater}{}}\label{item:Gauntlet of the Ram, Greater}
\hypertarget{item:Gauntlet of the Ram, Greater}{\subsubsection{Gauntlet of the Ram, Greater\hfill\nth{7} (1,800 gp)}}

This item functions like the \mitem{gauntlet of the ram}, except that you \glossterm{knockback} the target up to 30 feet.



\vspace{0.25em}
\spelltwocol{\textbf{Type}: Gauntlet}{}
\textbf{Materials}: Bone, metal, wood


\lowercase{\hypertarget{item:Gauntlets of Improvisation}{}}\label{item:Gauntlets of Improvisation}
\hypertarget{item:Gauntlets of Improvisation}{\subsubsection{Gauntlets of Improvisation\hfill\nth{2} (125 gp)}}

You gain a \plus1d \glossterm{magic bonus} to damage with \glossterm{improvised weapons}.



\vspace{0.25em}
\spelltwocol{\textbf{Type}: Gauntlet}{}
\textbf{Materials}: Bone, metal, wood


\lowercase{\hypertarget{item:Gauntlets of Improvisation, Greater}{}}\label{item:Gauntlets of Improvisation, Greater}
\hypertarget{item:Gauntlets of Improvisation, Greater}{\subsubsection{Gauntlets of Improvisation, Greater\hfill\nth{7} (1,800 gp)}}

This item functions like the \mitem{gauntlets of improvisation}, except that the damage bonus is increased to \plus2d.



\vspace{0.25em}
\spelltwocol{\textbf{Type}: Gauntlet}{}
\textbf{Materials}: Bone, metal, wood


\lowercase{\hypertarget{item:Gloves of Spell Investment}{}}\label{item:Gloves of Spell Investment}
\hypertarget{item:Gloves of Spell Investment}{\subsubsection{Gloves of Spell Investment\hfill\nth{7} (1,800 gp)}}

When you cast a spell that does not have the \glossterm{AP}, \glossterm{Attune}, \glossterm{Sustain} tags,
you can invest the magic of the spell in these gloves.
If you do, the spell does not have its normal effect.

As a standard action, you can activate these gloves.
When you do, you cause the effect of the last spell invested in the gloves.
This does not require \glossterm{concentration} or \glossterm{somatic components}.
After you use a spell in this way, the energy in the gloves is spent, and you must invest a new spell to activate the gloves again.

If you remove either glove from your hand, the magic of the spell invested in the gloves is lost.



\vspace{0.25em}
\spelltwocol{\textbf{Type}: Gloves}{}
\textbf{Materials}: Leather


\lowercase{\hypertarget{item:Gloves of Spell Investment, Greater}{}}\label{item:Gloves of Spell Investment, Greater}
\hypertarget{item:Gloves of Spell Investment, Greater}{\subsubsection{Gloves of Spell Investment, Greater\hfill\nth{13} (25,000 gp)}}

This item functions like the \mitem{gloves of spell investment}, except that you can store up to two spells in the gloves.
When you activate the gauntlets, you choose which spell to use.



\vspace{0.25em}
\spelltwocol{\textbf{Type}: Gloves}{}
\textbf{Materials}: Leather


\lowercase{\hypertarget{item:Greatreach Bracers}{}}\label{item:Greatreach Bracers}
\hypertarget{item:Greatreach Bracers}{\subsubsection{Greatreach Bracers\hfill\nth{9} (4,000 gp)}}

Your \glossterm{reach} is increased by 5 feet.



\vspace{0.25em}
\spelltwocol{\textbf{Type}: Bracers}{}
\textbf{Materials}: Bone, leather, metal, wood


\lowercase{\hypertarget{item:Greatreach Bracers, Greater}{}}\label{item:Greatreach Bracers, Greater}
\hypertarget{item:Greatreach Bracers, Greater}{\subsubsection{Greatreach Bracers, Greater\hfill\nth{17} (125,000 gp)}}

Your \glossterm{reach} is increased by 10 feet.



\vspace{0.25em}
\spelltwocol{\textbf{Type}: Bracers}{}
\textbf{Materials}: Bone, leather, metal, wood


\lowercase{\hypertarget{item:Hexproof Amulet, Greater}{}}\label{item:Hexproof Amulet, Greater}
\hypertarget{item:Hexproof Amulet, Greater}{\subsubsection{Hexproof Amulet, Greater\hfill\nth{15} (55,000 gp)}}

You gain a \plus4 bonus to defenses against \glossterm{magical} abilities that target you directly.
This does not protect you from abilities that affect an area.



\vspace{0.25em}
\spelltwocol{\textbf{Type}: Amulet}{}
\textbf{Materials}: Jewelry


\lowercase{\hypertarget{item:Hexward Amulet}{}}\label{item:Hexward Amulet}
\hypertarget{item:Hexward Amulet}{\subsubsection{Hexward Amulet\hfill\nth{9} (4,000 gp)}}

You gain a \plus2 bonus to defenses against \glossterm{magical} abilities that target you directly.
This does not protect you from abilities that affect an area.



\vspace{0.25em}
\spelltwocol{\textbf{Type}: Amulet}{}
\textbf{Materials}: Jewelry


\lowercase{\hypertarget{item:Hidden Armor}{}}\label{item:Hidden Armor}
\hypertarget{item:Hidden Armor}{\subsubsection{Hidden Armor\hfill\nth{4} (500 gp)}}

As a standard action, you can use this item.
If you do, it appears to change shape and form to assume the shape of a normal set of clothing.
You may choose the design of the clothing.
The item retains all of its properties, including weight and sound, while disguised in this way.
Only its visual appearance is altered.

Alternately, you may return the armor to its original appearance.



\vspace{0.25em}
\spelltwocol{\textbf{Type}: Body armor}{\parhead*{Tags} \glossterm{Sensation}}
\textbf{Materials}: Bone, metal


\lowercase{\hypertarget{item:Hidden Armor, Greater}{}}\label{item:Hidden Armor, Greater}
\hypertarget{item:Hidden Armor, Greater}{\subsubsection{Hidden Armor, Greater\hfill\nth{9} (4,000 gp)}}

This item functions like the \mitem{hidden armor} item, except that the item also makes sound appropriate to its disguised form while disguised.



\vspace{0.25em}
\spelltwocol{\textbf{Type}: Body armor}{\parhead*{Tags} \glossterm{Sensation}}
\textbf{Materials}: Bone, metal


\lowercase{\hypertarget{item:Lifekeeping Belt}{}}\label{item:Lifekeeping Belt}
\hypertarget{item:Lifekeeping Belt}{\subsubsection{Lifekeeping Belt\hfill\nth{7} (1,800 gp)}}

You gain a \plus1 \glossterm{magic bonus} to \glossterm{wound rolls}.



\vspace{0.25em}
\spelltwocol{\textbf{Type}: Belt}{}
\textbf{Materials}: Leather, textiles


\lowercase{\hypertarget{item:Lifekeeping Belt, Greater}{}}\label{item:Lifekeeping Belt, Greater}
\hypertarget{item:Lifekeeping Belt, Greater}{\subsubsection{Lifekeeping Belt, Greater\hfill\nth{13} (25,000 gp)}}

You gain a \plus2 \glossterm{magic bonus} to \glossterm{wound rolls}.



\vspace{0.25em}
\spelltwocol{\textbf{Type}: Belt}{}
\textbf{Materials}: Leather, textiles


\lowercase{\hypertarget{item:Lifekeeping Belt, Supreme}{}}\label{item:Lifekeeping Belt, Supreme}
\hypertarget{item:Lifekeeping Belt, Supreme}{\subsubsection{Lifekeeping Belt, Supreme\hfill\nth{19} (280,000 gp)}}

You gain a \plus3 \glossterm{magic bonus} to \glossterm{wound rolls}.



\vspace{0.25em}
\spelltwocol{\textbf{Type}: Belt}{}
\textbf{Materials}: Leather, textiles


\lowercase{\hypertarget{item:Mask of Air}{}}\label{item:Mask of Air}
\hypertarget{item:Mask of Air}{\subsubsection{Mask of Air\hfill\nth{9} (4,000 gp)}}

If you breathe through this mask, you breathe in clean, fresh air, regardless of your environment.
This can protect you from inhaled poisons and similar effects.



\vspace{0.25em}
\spelltwocol{\textbf{Type}: Mask}{}
\textbf{Materials}: Textiles


\lowercase{\hypertarget{item:Mask of Water Breathing}{}}\label{item:Mask of Water Breathing}
\hypertarget{item:Mask of Water Breathing}{\subsubsection{Mask of Water Breathing\hfill\nth{4} (500 gp)}}

You can breathe water through this mask as easily as a human breaths air.
This does not grant you the ability to breathe other liquids.



\vspace{0.25em}
\spelltwocol{\textbf{Type}: Mask}{}
\textbf{Materials}: Textiles


\lowercase{\hypertarget{item:Ocular Circlet}{}}\label{item:Ocular Circlet}
\hypertarget{item:Ocular Circlet}{\subsubsection{Ocular Circlet\hfill\nth{3} (250 gp)}}

As a \glossterm{standard action}, you can concentrate to use this item.
If you do, a \glossterm{scrying sensor} appears floating in the air in an unoccupied square within \rngclose range.
As long as you \glossterm{sustain} the effect as a standard action, you see through the sensor instead of from your body.

While viewing through the sensor, your visual acuity is the same as your normal body,
except that it does not share the benefits of any \glossterm{magical} effects that improve your vision.
You otherwise act normally, though you may have difficulty moving or taking actions if the sensor cannot see your body or your intended targets, effectively making you \blinded.



\vspace{0.25em}
\spelltwocol{\textbf{Type}: Circlet}{\parhead*{Tags} \glossterm{Scrying}}
\textbf{Materials}: Bone, metal


\lowercase{\hypertarget{item:Ocular Circlet, Greater}{}}\label{item:Ocular Circlet, Greater}
\hypertarget{item:Ocular Circlet, Greater}{\subsubsection{Ocular Circlet, Greater\hfill\nth{9} (4,000 gp)}}

This item functions like the \mitem{ocular circlet}, except that it only takes a \glossterm{minor action} to activate and sustain the item's effect.
In addition, the sensor appears anywhere within \rngmed range.



\vspace{0.25em}
\spelltwocol{\textbf{Type}: Circlet}{\parhead*{Tags} \glossterm{Scrying}}
\textbf{Materials}: Bone, metal


\lowercase{\hypertarget{item:Protective Armor}{}}\label{item:Protective Armor}
\hypertarget{item:Protective Armor}{\subsubsection{Protective Armor\hfill\nth{7} (1,800 gp)}}

You gain a \plus1 \glossterm{magic bonus} to Armor defense.



\vspace{0.25em}
\spelltwocol{\textbf{Type}: Body armor}{}
\textbf{Materials}: Bone, metal


\lowercase{\hypertarget{item:Protective Shield}{}}\label{item:Protective Shield}
\hypertarget{item:Protective Shield}{\subsubsection{Protective Shield\hfill\nth{7} (1,800 gp)}}

You gain a \plus1 \glossterm{magic bonus} to Armor defense.



\vspace{0.25em}
\spelltwocol{\textbf{Type}: Shield}{}
\textbf{Materials}: Bone, metal, wood


\lowercase{\hypertarget{item:Ring of Angel's Grace}{}}\label{item:Ring of Angel's Grace}
\hypertarget{item:Ring of Angel's Grace}{\subsubsection{Ring of Angel's Grace\hfill\nth{9} (4,000 gp)}}

You gain \plus2 \glossterm{magic bonus} to Mental defense.
In addition, if you fall at least 20 feet, ephemeral angel wings spring from your back.
The wings slow your fall to a rate of 60 feet per round, preventing you from taking \glossterm{falling damage}.



\vspace{0.25em}
\spelltwocol{\textbf{Type}: Ring}{}
\textbf{Materials}: Bone, jewelry, metal, wood


\lowercase{\hypertarget{item:Ring of Elemental Endurance}{}}\label{item:Ring of Elemental Endurance}
\hypertarget{item:Ring of Elemental Endurance}{\subsubsection{Ring of Elemental Endurance\hfill\nth{2} (125 gp)}}

You can exist comfortably in conditions between \minus50 and 140 degrees Fahrenheit without any ill effects.
You suffer the normal penalties in temperatures outside of that range.



\vspace{0.25em}
\spelltwocol{\textbf{Type}: Ring}{}
\textbf{Materials}: Bone, jewelry, metal, wood


\lowercase{\hypertarget{item:Ring of Energy Resistance}{}}\label{item:Ring of Energy Resistance}
\hypertarget{item:Ring of Energy Resistance}{\subsubsection{Ring of Energy Resistance\hfill\nth{6} (1,200 gp)}}

You gain a \glossterm{magic bonus} equal to half this item's \glossterm{power} to \glossterm{resistances} against \glossterm{energy damage}.
When you resist energy with this ability, the ring sheds light as a torch until the end of the next round.
The color of the light depends on the energy damage resisted: green for acid, blue for cold, yellow for electricity, and red for fire.



\vspace{0.25em}
\spelltwocol{\textbf{Type}: Ring}{}
\textbf{Materials}: Bone, jewelry, metal, wood


\lowercase{\hypertarget{item:Ring of Energy Resistance, Greater}{}}\label{item:Ring of Energy Resistance, Greater}
\hypertarget{item:Ring of Energy Resistance, Greater}{\subsubsection{Ring of Energy Resistance, Greater\hfill\nth{15} (55,000 gp)}}

This item functions like the \mitem{ring of energy resistance}, except that the bonus is equal to the item's \glossterm{power}.



\vspace{0.25em}
\spelltwocol{\textbf{Type}: Ring}{}
\textbf{Materials}: Bone, jewelry, metal, wood


\lowercase{\hypertarget{item:Ring of Nourishment}{}}\label{item:Ring of Nourishment}
\hypertarget{item:Ring of Nourishment}{\subsubsection{Ring of Nourishment\hfill\nth{3} (250 gp)}}

You continuously gain nourishment, and no longer need to eat or drink.
This ring must be worn for 24 hours before it begins to work.



\vspace{0.25em}
\spelltwocol{\textbf{Type}: Ring}{\parhead*{Tags} \glossterm{Creation}}
\textbf{Materials}: Bone, jewelry, metal, wood


\lowercase{\hypertarget{item:Ring of Protection}{}}\label{item:Ring of Protection}
\hypertarget{item:Ring of Protection}{\subsubsection{Ring of Protection\hfill\nth{8} (2,750 gp)}}

This ring creates a transluscent shield-like barrier that floats in front of you, deflecting enemy attacks.
You gain a \plus1 \glossterm{magic bonus} to Armor and Reflex defenses.
This does not stack with the defense bonus from any shields you use.



\vspace{0.25em}
\spelltwocol{\textbf{Type}: Ring}{}
\textbf{Materials}: Bone, jewelry, metal, wood


\lowercase{\hypertarget{item:Ring of Protection, Greater}{}}\label{item:Ring of Protection, Greater}
\hypertarget{item:Ring of Protection, Greater}{\subsubsection{Ring of Protection, Greater\hfill\nth{16} (85,000 gp)}}

This item functions like the \magicitem{ring of protection}, except that the bonus increases to \plus2.



\vspace{0.25em}
\spelltwocol{\textbf{Type}: Ring}{}
\textbf{Materials}: Bone, jewelry, metal, wood


\lowercase{\hypertarget{item:Ring of Regeneration}{}}\label{item:Ring of Regeneration}
\hypertarget{item:Ring of Regeneration}{\subsubsection{Ring of Regeneration\hfill\nth{8} (2,750 gp)}}

As a standard action, you can use this ring to enhance your healing.
When you do, you gain a \plus1 bonus to the \glossterm{wound roll} of your most recent \glossterm{vital wound}.



\vspace{0.25em}
\spelltwocol{\textbf{Type}: Ring}{}
\textbf{Materials}: Bone, jewelry, metal, wood


\lowercase{\hypertarget{item:Ring of Regeneration, Greater}{}}\label{item:Ring of Regeneration, Greater}
\hypertarget{item:Ring of Regeneration, Greater}{\subsubsection{Ring of Regeneration, Greater\hfill\nth{12} (16,000 gp)}}

At the end of each round, you gain a \plus1 bonus to the \glossterm{wound roll} of your most recent \glossterm{vital wound}.



\vspace{0.25em}
\spelltwocol{\textbf{Type}: Ring}{}
\textbf{Materials}: Bone, jewelry, metal, wood


\lowercase{\hypertarget{item:Ring of Sustenance}{}}\label{item:Ring of Sustenance}
\hypertarget{item:Ring of Sustenance}{\subsubsection{Ring of Sustenance\hfill\nth{7} (1,800 gp)}}

You continuously gain nourishment, and no longer need to eat or drink.
In addition, you need only one-quarter your normal amount of sleep (or similar activity, such as elven trance) each day.

The ring must be worn for 24 hours before it begins to work.



\vspace{0.25em}
\spelltwocol{\textbf{Type}: Ring}{\parhead*{Tags} \glossterm{Creation}}
\textbf{Materials}: Bone, jewelry, metal, wood


\lowercase{\hypertarget{item:Seven League Boots}{}}\label{item:Seven League Boots}
\hypertarget{item:Seven League Boots}{\subsubsection{Seven League Boots\hfill\nth{12} (16,000 gp)}}

As a standard action, you can spend an \glossterm{action point} to activate these boots.
If you do, you teleport exactly 25 miles in a direction you specify.
If this would place you within a solid object or otherwise impossible space, the boots will shunt you up to 1,000 feet in any direction to the closest available space.
If there is no available space within 1,000 feet of your intended destination, the effect fails and you take \glossterm{standard damage} \minus1d.



\vspace{0.25em}
\spelltwocol{\textbf{Type}: Boots}{}
\textbf{Materials}: Bone, leather, metal


\lowercase{\hypertarget{item:Shield of Arrow Catching}{}}\label{item:Shield of Arrow Catching}
\hypertarget{item:Shield of Arrow Catching}{\subsubsection{Shield of Arrow Catching\hfill\nth{5} (800 gp)}}

When a creature within a \areamed radius emanation from you would be attacked by a ranged weapon, the attack is redirected to target you instead.
Resolve the attack as if it had initially targeted you, except that the attack is not affected by cover or concealment.
This item can only affect projectiles and thrown objects that are Small or smaller.



\vspace{0.25em}
\spelltwocol{\textbf{Type}: Shield}{}
\textbf{Materials}: Bone, metal, wood


\lowercase{\hypertarget{item:Shield of Arrow Catching, Greater}{}}\label{item:Shield of Arrow Catching, Greater}
\hypertarget{item:Shield of Arrow Catching, Greater}{\subsubsection{Shield of Arrow Catching, Greater\hfill\nth{10} (6,500 gp)}}

This item functions like the \mitem{shield of arrow catching} item, except that it affects a \arealarge radius from you.
In addition, you may choose to exclude creature from this item's effect, allowing projectiles to target nearby foes normally.



\vspace{0.25em}
\spelltwocol{\textbf{Type}: Shield}{}
\textbf{Materials}: Bone, metal, wood


\lowercase{\hypertarget{item:Shield of Arrow Deflection}{}}\label{item:Shield of Arrow Deflection}
\hypertarget{item:Shield of Arrow Deflection}{\subsubsection{Shield of Arrow Deflection\hfill\nth{2} (125 gp)}}

As a \glossterm{minor action}, you can activate this shield.
If you do, you gain a \plus5 \glossterm{magic bonus} to Armor defense against ranged \glossterm{physical attacks} from weapons or projectiles that are Small or smaller.
This is a \glossterm{Swift} ability, and it lasts until the end of the round.



\vspace{0.25em}
\spelltwocol{\textbf{Type}: Shield}{}
\textbf{Materials}: Bone, metal, wood


\lowercase{\hypertarget{item:Shield of Arrow Deflection, Greater}{}}\label{item:Shield of Arrow Deflection, Greater}
\hypertarget{item:Shield of Arrow Deflection, Greater}{\subsubsection{Shield of Arrow Deflection, Greater\hfill\nth{8} (2,750 gp)}}

You gain a \plus5 \glossterm{magic bonus} to Armor defense against ranged \glossterm{physical attacks} from weapons or projectiles that are Small or smaller.



\vspace{0.25em}
\spelltwocol{\textbf{Type}: Shield}{}
\textbf{Materials}: Bone, metal, wood


\lowercase{\hypertarget{item:Shield of Bashing}{}}\label{item:Shield of Bashing}
\hypertarget{item:Shield of Bashing}{\subsubsection{Shield of Bashing\hfill\nth{2} (125 gp)}}

You gain a \plus1d \glossterm{magic bonus} to damage with \glossterm{strikes} using this shield.



\vspace{0.25em}
\spelltwocol{\textbf{Type}: Shield}{}
\textbf{Materials}: Bone, metal, wood


\lowercase{\hypertarget{item:Shield of Bashing, Greater}{}}\label{item:Shield of Bashing, Greater}
\hypertarget{item:Shield of Bashing, Greater}{\subsubsection{Shield of Bashing, Greater\hfill\nth{12} (16,000 gp)}}

You gain a \plus2d \glossterm{magic bonus} to damage with \glossterm{strikes} using this shield.



\vspace{0.25em}
\spelltwocol{\textbf{Type}: Shield}{}
\textbf{Materials}: Bone, metal, wood


\lowercase{\hypertarget{item:Shield of Boulder Catching}{}}\label{item:Shield of Boulder Catching}
\hypertarget{item:Shield of Boulder Catching}{\subsubsection{Shield of Boulder Catching\hfill\nth{8} (2,750 gp)}}

This item functions like the \mitem{shield of arrow catching} item, except that it can affect projectile and thrown objects of up to Large size.



\vspace{0.25em}
\spelltwocol{\textbf{Type}: Shield}{}
\textbf{Materials}: Bone, metal, wood


\lowercase{\hypertarget{item:Shield of Boulder Deflection}{}}\label{item:Shield of Boulder Deflection}
\hypertarget{item:Shield of Boulder Deflection}{\subsubsection{Shield of Boulder Deflection\hfill\nth{6} (1,200 gp)}}

This item functions like the \mitem{shield of arrow deflection} item, except that it can affect weapons and projectiles of up to Large size.



\vspace{0.25em}
\spelltwocol{\textbf{Type}: Shield}{}
\textbf{Materials}: Bone, metal, wood


\lowercase{\hypertarget{item:Shield of Boulder Deflection, Greater}{}}\label{item:Shield of Boulder Deflection, Greater}
\hypertarget{item:Shield of Boulder Deflection, Greater}{\subsubsection{Shield of Boulder Deflection, Greater\hfill\nth{12} (16,000 gp)}}

This item functions like the \mitem{greater shield of arrow deflection} item, except that it can affect weapons and projectiles of up to Large size.



\vspace{0.25em}
\spelltwocol{\textbf{Type}: Shield}{}
\textbf{Materials}: Bone, metal, wood


\lowercase{\hypertarget{item:Shield of Mystic Reflection}{}}\label{item:Shield of Mystic Reflection}
\hypertarget{item:Shield of Mystic Reflection}{\subsubsection{Shield of Mystic Reflection\hfill\nth{12} (16,000 gp)}}

As a standard action, you can activate this shield.
When you do, any \glossterm{targeted} \glossterm{magical} abilities that would target you this round are redirected to target the creature using that ability instead of you.
Any other targets of the ability are affected normally.
This is a \glossterm{Swift} ability, so it affects any abilities targeting you in the phase you activate the item.



\vspace{0.25em}
\spelltwocol{\textbf{Type}: Shield}{}
\textbf{Materials}: Bone, metal, wood


\lowercase{\hypertarget{item:Throwing Gloves}{}}\label{item:Throwing Gloves}
\hypertarget{item:Throwing Gloves}{\subsubsection{Throwing Gloves\hfill\nth{4} (500 gp)}}

% TODO: reference basic "not designed to be thrown" mechanics?
You can throw any item as if it was designed to be thrown.
This does not improve your ability to throw items designed to be thrown, such as darts.



\vspace{0.25em}
\spelltwocol{\textbf{Type}: Gloves}{}
\textbf{Materials}: Leather


\lowercase{\hypertarget{item:Titan Gauntlets}{}}\label{item:Titan Gauntlets}
\hypertarget{item:Titan Gauntlets}{\subsubsection{Titan Gauntlets\hfill\nth{13} (25,000 gp)}}

You gain a \plus1d \glossterm{magic bonus} to damage with \glossterm{strikes}.



\vspace{0.25em}
\spelltwocol{\textbf{Type}: Gauntlet}{}
\textbf{Materials}: Bone, metal, wood


\lowercase{\hypertarget{item:Torchlight Gloves}{}}\label{item:Torchlight Gloves}
\hypertarget{item:Torchlight Gloves}{\subsubsection{Torchlight Gloves\hfill\nth{2} (125 gp)}}

These gloves shed light as a torch.
As a \glossterm{standard action}, you may snap your fingers to suppress or resume the light from either or both gloves.



\vspace{0.25em}
\spelltwocol{\textbf{Type}: Gloves}{}
\textbf{Materials}: Leather


\lowercase{\hypertarget{item:Vanishing Cloak}{}}\label{item:Vanishing Cloak}
\hypertarget{item:Vanishing Cloak}{\subsubsection{Vanishing Cloak\hfill\nth{13} (25,000 gp)}}

As a standard action, you can activate this cloak.
When you do, you teleport to an unoccupied location within \rngmed range of your original location.
In addition, you become \glossterm{invisible} until the end of the next round.

If your intended destination is invalid, or if your teleportation otherwise fails, you still become invisible.



\vspace{0.25em}
\spelltwocol{\textbf{Type}: Cloak}{\parhead*{Tags} \glossterm{Sensation}}
\textbf{Materials}: Textiles


\lowercase{\hypertarget{item:Winged Boots}{}}\label{item:Winged Boots}
\hypertarget{item:Winged Boots}{\subsubsection{Winged Boots\hfill\nth{10} (6,500 gp)}}

You gain a \glossterm{fly speed} equal to your \glossterm{base speed}.
However, the boots are not strong enough to keep you aloft indefinitely.
At the end of each round, if you are not standing on solid ground, the magic of the boots fails and you fall normally.
The boots begin working again at the end of the next round, even if you have not yet hit the ground.



\vspace{0.25em}
\spelltwocol{\textbf{Type}: Boots}{}
\textbf{Materials}: Bone, leather, metal


  % Magic implements are a highly limited slot.
  % They have the same power level as self-attune spells.
  % This has a lot of text, so we need two columns
\newpage
\sectiongraphic*{Magic Implements}{width=\columnwidth}{equipment/magic implements}

  Like magic weapons, magic implements must be wielded to gain their effects.
  However, while weapons are used to deal damage to enemies, implements are used to grant or enhance magical abilities.

  There are three types of implements: staffs, rods, and wands.
  Staffs improve your existing magical abilities.
  Rods grant new magical abilities, even to those who cannot cast spells.
  Wands grant spellcasters the knowledge of specific spells.

  Staffs are long and thin, with even short staffs measuring no less than four feet long.
  Rods are about three feet long, but sturdily constructed.
  Wands are only about a foot long and very thin.

  \parhead{Somatic Components} While wielding an implement, you may gesture with it to perform \glossterm{somatic components}.
  This means you do not need a separate \glossterm{free hand} to perform those components.

  \parhead{Staff Types}
  There are two types of staffs that you can find.
  Long staffs function like a quarterstaff weapon, but they require two hands to wield, even when used to cast spells.
  Short staffs only require one hand, but they are not suitable for combat.

  \begin{longcolumn}
    
\begin{longtablewrapper}
\begin{longtable}{p{15em} p{3em} p{6em} p{25em} p{3em}}

\lcaption{Implement Items} \\
\tb{Name} & \tb{Level} & \tb{Typical Price} & \tb{Description} & \tb{Page} \tableheaderrule
Spell Wand, 1st & \nth{5} & 800 gp & Grants knowledge of a rank 1 spell & \pageref{item:Spell Wand, 1st} \\
Staff of Focus & \nth{5} & 800 gp & Reduces \glossterm{focus penalty} by 1 & \pageref{item:Staff of Focus} \\
Staff of Giants & \nth{6} & 1,200 gp & Increases maximum size category of abilities & \pageref{item:Staff of Giants} \\
Staff of Transit & \nth{6} & 1,200 gp & Doubles your teleportation distance & \pageref{item:Staff of Transit} \\
Protective Staff & \nth{7} & 1,800 gp & Grants \plus1 Armor defense & \pageref{item:Protective Staff} \\
Cryptic Staff & \nth{8} & 2,750 gp & Makes spells hard to identify & \pageref{item:Cryptic Staff} \\
Spell Wand, 2nd & \nth{8} & 2,750 gp & Grants knowledge of a rank 2 spell & \pageref{item:Spell Wand, 2nd} \\
Staff of Power & \nth{8} & 2,750 gp & Grants \plus2 \glossterm{magical} power & \pageref{item:Staff of Power} \\
Staff of Precision & \nth{8} & 2,750 gp & Grants \plus1 accuracy & \pageref{item:Staff of Precision} \\
Extending Staff & \nth{9} & 4,000 gp & Doubles range & \pageref{item:Extending Staff} \\
Selective Staff & \nth{9} & 4,000 gp & Allows excluding areas & \pageref{item:Selective Staff} \\
Staff of Silence & \nth{9} & 4,000 gp & Allows casting spells without verbal components & \pageref{item:Staff of Silence} \\
Staff of Stillness & \nth{9} & 4,000 gp & Allows casting spells without somatic components & \pageref{item:Staff of Stillness} \\
Spell Wand, 3rd & \nth{11} & 10,000 gp & Grants knowledge of a rank 3 spell & \pageref{item:Spell Wand, 3rd} \\
Reaching Staff & \nth{12} & 16,000 gp & Allows ability use from a short distance away & \pageref{item:Reaching Staff} \\
Staff of Giants, Greater & \nth{12} & 16,000 gp & Significantly increaases maximum size category of abilities & \pageref{item:Staff of Giants, Greater} \\
Staff of Transit, Greater & \nth{12} & 16,000 gp & Triples your teleportation distance & \pageref{item:Staff of Transit, Greater} \\
Widening Staff & \nth{12} & 16,000 gp & Doubles area size & \pageref{item:Widening Staff} \\
Protective Staff, Greater & \nth{13} & 25,000 gp & Grants \plus2 Armor defense & \pageref{item:Protective Staff, Greater} \\
Staff of the Archmagi & \nth{13} & 25,000 gp & Grants \plus1 accuracy, \plus2 \glossterm{magical} power & \pageref{item:Staff of the Archmagi} \\
Spell Wand, 4th & \nth{14} & 37,000 gp & Grants knowledge of a rank 4 spell & \pageref{item:Spell Wand, 4th} \\
Staff of Power, Greater & \nth{14} & 37,000 gp & Grants \plus4 \glossterm{magical} power & \pageref{item:Staff of Power, Greater} \\
Staff of Precision, Greater & \nth{14} & 37,000 gp & Grants \plus2 accuracy & \pageref{item:Staff of Precision, Greater} \\
Extending Staff, Greater & \nth{15} & 55,000 gp & Triples range & \pageref{item:Extending Staff, Greater} \\
Selective Staff, Greater & \nth{15} & 55,000 gp & Allows excluding and splitting areas & \pageref{item:Selective Staff, Greater} \\
Staff of Tranquility & \nth{15} & 55,000 gp & Allows casting spells without components & \pageref{item:Staff of Tranquility} \\
Spell Wand, 5th & \nth{17} & 125,000 gp & Grants knowledge of a rank 5 spell & \pageref{item:Spell Wand, 5th} \\
Reaching Staff, Greater & \nth{18} & 190,000 gp & Allows ability use from a distance away & \pageref{item:Reaching Staff, Greater} \\
Staff of Giants, Supreme & \nth{18} & 190,000 gp & Drastically increaases maximum size category of abilities & \pageref{item:Staff of Giants, Supreme} \\
Staff of Transit, Supreme & \nth{18} & 190,000 gp & Quadruples your teleportation distance & \pageref{item:Staff of Transit, Supreme} \\
Widening Staff, Greater & \nth{18} & 190,000 gp & Triples area size & \pageref{item:Widening Staff, Greater} \\
Protective Staff, Supreme & \nth{19} & 280,000 gp & Grants \plus3 Armor defense & \pageref{item:Protective Staff, Supreme} \\
Staff of the Archmagi, Greater & \nth{19} & 280,000 gp & Grants \plus2 accuracy, \plus4 \glossterm{magical} power & \pageref{item:Staff of the Archmagi, Greater} \\
Staff of Power, Supreme & \nth{20} & 400,000 gp & Grants \plus6 \glossterm{magical} power & \pageref{item:Staff of Power, Supreme} \\
Staff of Precision, Supreme & \nth{20} & 400,000 gp & Grants \plus3 accuracy & \pageref{item:Staff of Precision, Supreme} \\

\end{longtable}
\end{longtablewrapper}

  \end{longcolumn}

  
\lowercase{\hypertarget{item:Extending Staff}{}}\label{item:Extending Staff}
\hypertarget{item:Extending Staff}{\subsubsection{Extending Staff\hfill\nth{10} (6,500 gp)}}

You double the range of your \glossterm{magical} abilities.



\vspace{0.25em}
\spelltwocol{\textbf{Type}: Staff}{}
\textbf{Materials}: Bone, wood


\lowercase{\hypertarget{item:Extending Staff, Greater}{}}\label{item:Extending Staff, Greater}
\hypertarget{item:Extending Staff, Greater}{\subsubsection{Extending Staff, Greater\hfill\nth{19} (280,000 gp)}}

You triple the range of your \glossterm{magical} abilities.



\vspace{0.25em}
\spelltwocol{\textbf{Type}: Staff}{}
\textbf{Materials}: Bone, wood


\lowercase{\hypertarget{item:Protective Staff}{}}\label{item:Protective Staff}
\hypertarget{item:Protective Staff}{\subsubsection{Protective Staff\hfill\nth{5} (800 gp)}}

You gain a \plus1 \glossterm{magic bonus} to Armor defense.



\vspace{0.25em}
\spelltwocol{\textbf{Type}: Staff}{}
\textbf{Materials}: Bone, wood


\lowercase{\hypertarget{item:Protective Staff, Greater}{}}\label{item:Protective Staff, Greater}
\hypertarget{item:Protective Staff, Greater}{\subsubsection{Protective Staff, Greater\hfill\nth{14} (37,000 gp)}}

You gain a \plus2 \glossterm{magic bonus} to Armor defense.



\vspace{0.25em}
\spelltwocol{\textbf{Type}: Staff}{}
\textbf{Materials}: Bone, wood


\lowercase{\hypertarget{item:Reaching Staff}{}}\label{item:Reaching Staff}
\hypertarget{item:Reaching Staff}{\subsubsection{Reaching Staff\hfill\nth{12} (16,000 gp)}}

Spells you cast with this staff automatically have the benefits of the Reach augment, if applicable (see \pcref{Augment Descriptions}).



\vspace{0.25em}
\spelltwocol{\textbf{Type}: Staff}{}
\textbf{Materials}: Bone, wood


\lowercase{\hypertarget{item:Spell Wand, 1st}{}}\label{item:Spell Wand, 1st}
\hypertarget{item:Spell Wand, 1st}{\subsubsection{Spell Wand, 1st\hfill\nth{5} (800 gp)}}

This wand grants you knowledge of a single 1st level spell.
You must have access to the \glossterm{mystic sphere} that spell belongs to.



\vspace{0.25em}
\spelltwocol{\textbf{Type}: Wand}{}
\textbf{Materials}: Bone, wood


\lowercase{\hypertarget{item:Spell Wand, 2nd}{}}\label{item:Spell Wand, 2nd}
\hypertarget{item:Spell Wand, 2nd}{\subsubsection{Spell Wand, 2nd\hfill\nth{9} (4,000 gp)}}

This item functions like a \mitem{spell wand}, except that it grants knowledge of a single 2nd level spell.



\vspace{0.25em}
\spelltwocol{\textbf{Type}: Wand}{}
\textbf{Materials}: Bone, wood


\lowercase{\hypertarget{item:Spell Wand, 3rd}{}}\label{item:Spell Wand, 3rd}
\hypertarget{item:Spell Wand, 3rd}{\subsubsection{Spell Wand, 3rd\hfill\nth{13} (25,000 gp)}}

This item functions like a \mitem{spell wand}, except that it grants knowledge of a single 3rd level spell.



\vspace{0.25em}
\spelltwocol{\textbf{Type}: Wand}{}
\textbf{Materials}: Bone, wood


\lowercase{\hypertarget{item:Spell Wand, 4th}{}}\label{item:Spell Wand, 4th}
\hypertarget{item:Spell Wand, 4th}{\subsubsection{Spell Wand, 4th\hfill\nth{17} (125,000 gp)}}

This item functions like a \mitem{spell wand}, except that it grants knowledge of a single 4th level spell.



\vspace{0.25em}
\spelltwocol{\textbf{Type}: Wand}{}
\textbf{Materials}: Bone, wood


\lowercase{\hypertarget{item:Staff of Expansion}{}}\label{item:Staff of Expansion}
\hypertarget{item:Staff of Expansion}{\subsubsection{Staff of Expansion\hfill\nth{7} (1,800 gp)}}

When you use a \glossterm{magical} ability that creates a \glossterm{zone} or \glossterm{emanation}, you can increase the size of the area by one size category, up to a maximum of \areahuge.
You can only increase the area of one ability at a time in this way.
If you increase the area of another ability or lose this staff, the area of the original ability returns to its normal size.



\vspace{0.25em}
\spelltwocol{\textbf{Type}: Staff}{}
\textbf{Materials}: Bone, wood


\lowercase{\hypertarget{item:Staff of Expansion, Greater}{}}\label{item:Staff of Expansion, Greater}
\hypertarget{item:Staff of Expansion, Greater}{\subsubsection{Staff of Expansion, Greater\hfill\nth{16} (85,000 gp)}}

This item functions like a \textit{staff of expansion}, except that it increases the area by two size categories.
In addition, the maximum area is a 200 foot radius, which is one size category larger than \areahuge.



\vspace{0.25em}
\spelltwocol{\textbf{Type}: Staff}{}
\textbf{Materials}: Bone, wood


\lowercase{\hypertarget{item:Staff of Focus}{}}\label{item:Staff of Focus}
\hypertarget{item:Staff of Focus}{\subsubsection{Staff of Focus\hfill\nth{6} (1,200 gp)}}

You reduce your \glossterm{focus penalty} by 1.



\vspace{0.25em}
\spelltwocol{\textbf{Type}: Staff}{}
\textbf{Materials}: Bone, wood


\lowercase{\hypertarget{item:Staff of Focus, Greater}{}}\label{item:Staff of Focus, Greater}
\hypertarget{item:Staff of Focus, Greater}{\subsubsection{Staff of Focus, Greater\hfill\nth{15} (55,000 gp)}}

You reduce your \glossterm{focus penalty} by 2.



\vspace{0.25em}
\spelltwocol{\textbf{Type}: Staff}{}
\textbf{Materials}: Bone, wood


\lowercase{\hypertarget{item:Staff of Power}{}}\label{item:Staff of Power}
\hypertarget{item:Staff of Power}{\subsubsection{Staff of Power\hfill\nth{8} (2,750 gp)}}

You gain a \plus2 \glossterm{magic bonus} to \glossterm{power} with \glossterm{magical} abilities.



\vspace{0.25em}
\spelltwocol{\textbf{Type}: Staff}{}
\textbf{Materials}: Bone, wood


\lowercase{\hypertarget{item:Staff of Power, Greater}{}}\label{item:Staff of Power, Greater}
\hypertarget{item:Staff of Power, Greater}{\subsubsection{Staff of Power, Greater\hfill\nth{17} (125,000 gp)}}

You gain a \plus4 \glossterm{magic bonus} to \glossterm{power} with \glossterm{magical} abilities.



\vspace{0.25em}
\spelltwocol{\textbf{Type}: Staff}{}
\textbf{Materials}: Bone, wood


\lowercase{\hypertarget{item:Staff of Precision}{}}\label{item:Staff of Precision}
\hypertarget{item:Staff of Precision}{\subsubsection{Staff of Precision\hfill\nth{8} (2,750 gp)}}

You gain a \plus1 \glossterm{magic bonus} to \glossterm{accuracy}.



\vspace{0.25em}
\spelltwocol{\textbf{Type}: Staff}{}
\textbf{Materials}: Bone, wood


\lowercase{\hypertarget{item:Staff of Precision, Greater}{}}\label{item:Staff of Precision, Greater}
\hypertarget{item:Staff of Precision, Greater}{\subsubsection{Staff of Precision, Greater\hfill\nth{17} (125,000 gp)}}

You gain a \plus2 \glossterm{magic bonus} to \glossterm{accuracy}.



\vspace{0.25em}
\spelltwocol{\textbf{Type}: Staff}{}
\textbf{Materials}: Bone, wood


\lowercase{\hypertarget{item:Staff of Transit}{}}\label{item:Staff of Transit}
\hypertarget{item:Staff of Transit}{\subsubsection{Staff of Transit\hfill\nth{6} (1,200 gp)}}

Your \glossterm{magical} abilities have the maximum distance they can \glossterm{teleport} targets doubled.



\vspace{0.25em}
\spelltwocol{\textbf{Type}: Staff}{}
\textbf{Materials}: Bone, wood


  % Magic consumables are a loosely limited slot, but being consumable is a big downside.
  % They have the same power level as spells.
  \begin{longcolumn}
    \section{Consumables}\label{Consumables}

      \subsection{Potions}
        A potion is a magical liquid that is typically contained in a Fine vial.
        Drinking a potion, or administering a potion to an unconscious creature, requires a standard action.
        Potions cannot be safely mixed together without diluting their magic, so you cannot consume two potions with the same action.

        \input{generated/consumable_tools_table.tex}
  \end{longcolumn}

  \input{generated/consumable_tools.tex}

  % Permanent tools like this generally have no clear relationship to spells.
  \begin{longcolumn}
    \section{Tools, Goods, and Mounts}
      \begin{longtablepreface}
        % \includegraphics[width=\columnwidth]{equipment/tools goods and mounts}

        The world of Rise has a wide range of minor items like backpacks, blankets, and ten-foot poles.
        In general, the cost of those items is so insignificant from the perspective of an adventuring party that it's not worth the effort to track their cost in detail.
        A subset of particularly expensive items is included in \tref{Permanent Tools, Goods, and Mounts}.

        \subsection{Standard Adventuring Kit}
          % Technically 15.2 gp and 50.5 pounds
          A standard adventuring kit costs 10 gp, weighs 50 pounds, and contains the following items:
          \begin{itemize}
            \item Backpack
            \item Bedroll
            \item Flint and steel
            \item Rations, trail (8 days)
            \item Rope, hempen (60 ft.)
            \item Sack (empty)
            \item Tent
            \item Torch
            \item Waterskin
          \end{itemize}
      \end{longtablepreface}

      \input{generated/permanent_tools_table.tex}
  \end{longcolumn}

  \input{generated/permanent_tools.tex}


\chapter{Adventuring}

\section{Carrying Capacity}\label{Carrying Capacity}

    \begin{dtable}
        \lcaption{Carrying Capacity by Strength}
        \setlength{\tabcolsep}{4pt}
        \begin{dtabularx}{\columnwidth}{X l l l l}
            \tb{Strength} & \tb{Light} & \tb{Maximum} & \tb{Overloaded} & \tb{Push/Drag} \tableheaderrule
            -9 & 6 lb. & 12 lb. & 18 lb. & 60 lb. \\
            -8 & 7     & 14     & 21     & 70     \\
            -7 & 9     & 18     & 27     & 90     \\
            -6 & 12    & 24     & 36     & 120    \\
            -5 & 15    & 30     & 45     & 150    \\
            -4 & 20    & 40     & 60     & 200    \\
            -3 & 25    & 50     & 75     & 250    \\
            -2 & 30    & 60     & 90     & 300    \\
            -1 & 40    & 80     & 120    & 400    \\
            0  & 50    & 100    & 150    & 500    \\
            1  & 60    & 120    & 180    & 600    \\
            2  & 80    & 160    & 240    & 800    \\
            3  & 100   & 200    & 300    & 1,000  \\
            4  & 120   & 240    & 360    & 1,200  \\
            5  & 160   & 320    & 480    & 1,600  \\
            6  & 200   & 400    & 600    & 2,000  \\
            7  & 250   & 500    & 750    & 2,500  \\
            8  & 320   & 640    & 960    & 3,200  \\
            9  & 400   & 800    & 1,200  & 4,000  \\
            10 & 500   & 1,000  & 1,500  & 5,000  \\
            11 & 630   & 1,260  & 1,890  & 6,300  \\
            12 & 800   & 1,600  & 2,400  & 8,000  \\
            13 & 1,000 & 2,000  & 3,000  & 10,000 \\
            14 & 1,300 & 2,600  & 3,900  & 13,000 \\
            15 & 1,600 & 3,200  & 4,800  & 16,000 \\
            16 & 2,000 & 4,000  & 6,000  & 20,000 \\
            17 & 2,500 & 5,000  & 7,500  & 25,000 \\
            18 & 3,200 & 6,400  & 9,600  & 32,000 \\
            19 & 4,000 & 8,000  & 12,000 & 40,000 \\
            20 & 5,000 & 10,000 & 15,000 & 50,000 \\
            21\plus\fn{1} & \tdash & \tdash & \tdash & \tdash \\
        \end{dtabularx}
        1 To calculate the carrying capacity for a creature with epic Strength, double its carrying capacity every 3 Strength.
    \end{dtable}

    A creature's Strength determines how much weight it can carry, as shown in \trefnp{Carrying Capacity by Strength}.
    A creature can carry weight up to its light carrying capacity without any penalty.
    If it carries more than that, but less than its maximum carrying capacity, it increases its \glossterm{encumbrance} by 4.
    This stacks with the encumbrance from any armor the creature wears.

    \parhead{Lifting and Dragging} You can lift as much as your maximum carrying capacity over your head.

    You can lift as much as 1-1/2 your maximum carrying capacity off the ground (the sum of your light and maximum weight limits).
    While overloaded in this way, you increase your \glossterm{encumbrance} by 10, you take a \minus10 penalty to \glossterm{accuracy} with \glossterm{mundane} attacks, and you can only move by spending a \glossterm{standard action} to move 5 feet.
    This replaces the encumbrance from carrying more than your light carrying capacity.

    You can generally push or drag along the ground as much as five times your maximum carrying capacity.

    \parhead{Multi-Legged Creatures} The figures on \trefnp{Carrying Capacity by Strength} are for bipedal creatures. A creature with four or more legs can carry 50\% more weight than a bipedal creature of the same Strength.

    \parhead{Tremendous Strength} For Strength scores not shown on \trefnp{Carrying Capacity by Strength}, subtract 3 from its Strength until you find a Strength value shown on the chart. For each time you subtracted in this way, double the weight limits listed on the chart.

\section{Movement}

    \begin{dtable}
        \lcaption{Movement and Distance}
        \begin{dtabularx}{\columnwidth}{>{\lcol}X c c c c}
            & \multicolumn{4}{c}{\tdash\tdash\tdash Speed \tdash\tdash\tdash} \tableheaderrule
            & 15 feet & 20 feet & 30 feet & 40 feet \\
            One Round (Tactical) &  &  &  &  \\
            Walk & 15 ft. & 20 ft. & 30 ft. & 40 ft. \\
            Hustle & 30 ft. & 40 ft. & 60 ft. & 80 ft. \\
            One Minute (Local) &  &  &  &  \\
            Walk & 150 ft. & 200 ft. & 300 ft. & 400 ft. \\
            Hustle & 300 ft. & 400 ft. & 600 ft. & 800 ft. \\
            One Hour (Overland) &  &  &  &  \\
            Walk & 3/4 mile & 1 mile & 1-1/2 miles & 2 miles \\
            Hustle & 1-1/2 miles & 2 miles & 3 miles & 4 miles \\
            One Day (Overland) &  &  &  &  \\
            Walk & 7-1/2 miles & 10 miles & 15 miles & 20 miles \\
            Hustle & \tdash & \tdash & \tdash & \tdash \\
        \end{dtabularx}
    \end{dtable}

    \begin{dtable}
        \lcaption{Hampered Movement}
        \begin{dtabularx}{\columnwidth}{l >{\lcol}X >{\ccol}p{8em}}
            Condition & Example Extra Movement Cost \tableheaderrule
            Difficult terrain & Rubble, undergrowth, steep slope, ice, cracked and pitted surface, uneven floor & \mult2 \\
            Obstacle\fn{1} & Low wall, deadfall, broken pillar & \mult2 \\
            Poor visibility & Darkness or fog & \mult2 \\
            Impassable & Floor-to-ceiling wall, closed door, blocked passage & \tdash \\
        \end{dtabularx}
        1 May require a skill check
    \end{dtable}

    There are three movement scales in the game, as follows.
    \begin{itemize}
        \item Tactical, for combat, measured in feet (or squares) per round.
        \item Local, for exploring an area, measured in feet per minute.
        \item Overland, for getting from place to place, measured in miles per
            hour or miles per day.
    \end{itemize}

    \subsection{Tactical Movement}
        Use tactical movement for combat.

        \parhead{Minimum Movement} In some situations, your movement may be so hampered that you don't have sufficient speed even to move 5 feet (1 square). In such a case, you may use a standard action to move 5 feet (1 square) in any direction, even diagonally. (You can't take advantage of this rule to move through impassable terrain or to move when all movement is prohibited to you, such as while paralyzed.)

    \subsection{Local Movement}
        Characters exploring an area use local movement, measured in feet per minute.
        \parhead{Walk} A character can walk without a problem on the local scale.
        \parhead{Hustle} A character can hustle without a problem on the local scale. See \trefnp{Terrain and Overland Movement}, below, for movement measured in miles per hour.

    \subsection{Overland Movement}\label{Overland Movement}

        \begin{dtable}
            \lcaption{Terrain and Overland Movement}
            \begin{dtabularx}{\columnwidth}{>{\lcol}X c c c}
                \tb{Terrain}   & \tb{Highway} & \tb{Road or Trail} & \tb{Trackless} \tableheaderrule
                Desert, sandy  & \mult1       & \mult1/2           & \mult1/2 \\
                Forest         & \mult1       & \mult1             & \mult1/2 \\
                Hills          & \mult1       & \mult3/4           & \mult1/2 \\
                Jungle         & \mult1       & \mult3/4           & \mult1/4 \\
                Moor           & \mult1       & \mult1             & \mult3/4 \\
                Mountains      & \mult3/4     & \mult3/4           & \mult1/2 \\
                Plains         & \mult1-1/2   & \mult1             & \mult3/4 \\
                Swamp          & \mult1       & \mult3/4           & \mult1/2 \\
                Tundra, frozen & \mult1       & \mult3/4           & \mult3/4
            \end{dtabularx}
        \end{dtable}

        \begin{dtable}
            \lcaption{Mounts and Vehicles}
            \begin{dtabularx}{\columnwidth}{>{\lcol}X l l}
                \tb{Mount/Vehicle} & \tb{Per Hour} & \tb{Per Day} \tableheaderrule
                Mount (carrying load) &  &  \\
                \tind Light horse or light warhorse & 6 miles & 60 miles \\
                \tind Light horse (151-450 lb.)\fn{1} & 4 miles & 40 miles \\
                \tind Light warhorse (231-690 lb.)\fn{1} & 4 miles & 40 miles \\
                \tind Heavy horse or heavy warhorse & 5 miles & 50 miles \\
                \tind Heavy horse (201-600 lb.)\fn{1} & 3-1/2 miles & 35 miles \\
                \tind Heavy warhorse (301-900 lb.)\fn{1} & 3-1/2 miles & 35 miles \\
                \tind Pony or warpony & 4 miles & 40 miles \\
                \tind Pony (76-225 lb.)\fn{1} & 3 miles & 30 miles \\
                \tind Warpony (101-300 lb.)\fn{1} & 3 miles & 30 miles \\
                \tind Donkey or mule & 3 miles & 30 miles \\
                \tind Donkey (51-150 lb.)\fn{1} & 2 miles & 20 miles \\
                \tind Mule (231-690 lb.)\fn{1} & 2 miles & 20 miles \\
                \tind Dog, riding & 4 miles & 40 miles \\
                \tind Dog, riding (101-300 lb.)\fn{1} & 3 miles & 30 miles \\
                \tind Cart or wagon & 2 miles & 20 miles \\
                \tb{Ship} &  &  \\
                \tind Raft or barge (poled or towed)\fn{2} & 1/2 mile & 5 miles \\
                \tind Keelboat (rowed)\fn{2} & 1 mile & 10 miles \\
                \tind Rowboat (rowed)\fn{2} & 1-1/2 miles & 15 miles \\
                \tind Sailing ship (sailed) & 2 miles & 48 miles \\
                \tind Warship (sailed and rowed) & 2-1/2 miles & 60 miles \\
                \tind Longship (sailed and rowed) & 3 miles & 72 miles \\
                \tind Galley (rowed and sailed) & 4 miles & 96 miles \\
            \end{dtabularx}
            1 Quadrupeds, such as horses, can carry heavier loads than characters can. See \tref{Carrying Capacity by Strength}, for more information. \\
            2 Rafts, barges, keelboats, and rowboats are used on lakes and rivers.
            If going downstream, add the speed of the current (typically 3 miles per hour) to the speed of the vehicle. In addition to 10 hours of being rowed, the vehicle can also float an additional 14 hours, if someone can guide it, so add an additional 42 miles to the daily distance traveled. These vehicles can't be rowed against any significant current, but they can be pulled upstream by draft animals on the shores.
        \end{dtable}

        Characters covering long distances cross-country use overland movement. Overland movement is measured in miles per hour or miles per day. A day represents 10 hours of actual travel time. For rowed watercraft, a day represents 10 hours of rowing. For a sailing ship, it represents 24 hours.

        \parhead{Walk} A character can walk 10 hours in a day of travel without a problem. Walking for longer than that, or hustling faster than that, requires an Endurance check (see \pcref{Overland Exertion}).
        \parhead{Terrain} The terrain through which a character travels affects how much distance they can cover in an hour or a day (see \trefnp{Terrain and Overland Movement}).
        A highway is a straight, major, paved road.
        A road is typically a dirt track.
        A trail is like a road, except that it allows only single-file travel and does not benefit a party traveling with vehicles.
        Trackless terrain is a wild area with no significant paths.
        \parhead{Mounted Movement} A mount bearing a rider can move at a hustle. The damage it takes when doing so, however, is not subdual damage. The creature can also be ridden in a forced march, but its Constitution checks automatically fail, and, again, the damage it takes is lethal damage. Mounts also become fatigued when they take any damage from hustling or forced marches.

        See \trefnp{Mounts and Vehicles} for mounted speeds and speeds for vehicles pulled by draft animals.

        \parhead{Waterborne Movement} See \trefnp{Mounts and Vehicles} for speeds for water vehicles.

\section{Vision and Light}\label{Vision and Light}
    Some creatures have \glossterm{darkvision}, but most creatures need light to see by. 
    In an area of \glossterm{bright illumination}, all characters can see clearly.
    A creature can't hide in an area with bright illumination unless it is invisible or has cover.

    In an area with shadowy illumination, creatures can see dimly.
    Creatures within this area have \concealment, which can allow them to make Stealth checks to hide (see \pcref{Stealth}).

    In areas of darkness, creatures without \glossterm{darkvision} or some other form of supernatural vision are \blinded.

    Characters with low-light vision (elves, gnomes, and half-elves) treat sources of light as if they had double their normal illumination range.

    Characters with \glossterm{darkvision} can see lit areas normally as well as dark areas within a radius defined by the ability -- usually, 50 feet. A creature can't hide within that range of a character using darkvision unless it is invisible or has cover. Darkvision does not function if the character is in \glossterm{bright illumination}, and does not resume functioning until the end of the next round after the character leaves the area of bright illumination.

    \subsection{Attacking Unseen Foes}
        You can make attacks against creatures and objects you cannot see.
        To do so, you choose a 5-foot square and make the attack against that square.
        You have a 50\% chance to hit nothing at all with the attack and a 50\% chance to hit a random valid target in that square with your attack.

\section{Breaking Objects}
    There are two main ways of breaking objects.
    You can deal damage to objects with attacks, similarly to how you can deal damage to creatures.
    Alternately, you can attempt to sunder the object with sheer strength.

    \subsection{Damaging Objects}
        An object's size and primary material determines the number of \glossterm{vital wounds} it can suffer before being destroyed.
        The primary material it is constructed from determines its \glossterm{resistances}, and can modify the number of vital wounds it can take.
        An object breaks it takes \glossterm{vital wounds} in excess of its maximum.
        Objects do not have \glossterm{hit points} or a \glossterm{wound resistance}, cannot be \glossterm{bloodied}, and do not make \glossterm{vital rolls}.
        Objects are also not normally subject to \glossterm{critical hits}.

    \subsection{Sundering Objects}
        As a standard action, you can attempt to sunder an object you can touch.
        This requires two hands.
        An object's size and primary material determines the \glossterm{difficulty rating} of the check.
        Success means that the object breaks.
        Failure by 5 or less means you inflict a \glossterm{vital wound} on the object, but it does not break.
        Failure by 6 or more means nothing happens.

        \begin{dtable}
            \lcaption{Object Defenses by Size}
            \begin{dtabularx}{\textwidth}{l X X}
                \tb{Size}  & \tb{Max Vital Wounds} & \tb{Sunder Difficulty Rating} \tableheaderrule
                Fine       & 1                     & 1\fn{1} \\
                Diminutive & 1                     & 2       \\
                Tiny       & 2                     & 5       \\
                Small      & 4                     & 10      \\
                Medium     & 8                     & 15      \\
                Large      & 16                    & 20      \\
                Huge       & 32                    & 25      \\
                Gargantuan & 64                    & 30      \\
                Colossal   & 128                   & 35      \\
            \end{dtabularx}
            1. Extremely small objects may be difficult to grip effectively, which can significantly increase the difficulty to sunder them.
        \end{dtable}

        \begin{dtable}
            \lcaption{Object Defenses by Material}
            \begin{dtabularx}{\textwidth}{l X X X}
                \tb{Material}   & \tb{Vital Resistance} & \tb{Max Vital Wounds} & \tb{Sunder Modifier}  & \tableheaderrule
                Adamantine      & 30                    & \mult3                & \plus20              \\
                Glass           & 5                     & \mult1/2              & \tdash               \\
                Ice             & 1                     & \mult1/2              & \minus5              \\
                Iron or steel   & 12                    & \mult2                & \plus10              \\
                Leather or hide & 3                     & \tdash                & \tdash               \\
                Mithral         & 15                    & \mult2                & \plus10              \\
                Paper or cloth  & 1                     & \mult1/2              & \minus5              \\
                Rope            & 2                     & \tdash                & \tdash               \\
                Stone           & 8                     & \mult2                & \plus5               \\
                Wood            & 5                     & \tdash                & \tdash               \\
            \end{dtabularx}
        \end{dtable}

    \subsection{Broken and Destroyed Objects}
        An object that reaches its maximum vital wounds or is sundered becomes \glossterm{broken}.
        You can destroy an object by inflicting ten times its maximum \glossterm{vital wounds}, or by succeeding at a check to sunder the object by 20.

        \parhead{Broken Objects}\label{Broken Objects}
        Broken objects cannot be used for their intended purpose, but still retain enough of their original form to be repaired without too much work.
        For example, a broken wall lies in pieces on the ground and no longer blocks passage, but can be repaired with far less effort than would be required to create a wall from scratch.
        Magic items that are broken retain their magical properties once fixed.
        Broken (but not destroyed) objects can be repaired with the Craft skill for a cost equal to 10\% of their value (see \pcref{Craft}).

        \parhead{Destroyed Objects}\label{Destroyed Objects}
        Destroyed object have been damaged beyond hope of any sort of repair short of crafting the object again from raw materials.
        For example, a destroyed wall is reduced to dust or small, useless chunks of rubble.
        Magic items that are destroyed irrevocably lose their magical properties.
        The remains of a destroyed object generally occupy a space one size category smaller than the original object.

    \subsection{Relative Vital Resistances}\label{Relative Vital Resistances}
        When an object would take damage, if the \glossterm{vital resistance} of the attacking object or creature is lower than the \glossterm{vital resistance} of the defender, the attacking object or creature takes the damage instead.
        For example, if you try to break a stone wall with a wooden club, the club will break instead of the wall.
        % TODO: define hardness for creatures and their natural weapons; natural weapons should generally have higher hardness than creatures to avoid hardness reflection being common

\section{Wealth And Money}

    \subsection{Coins}
        The most common coin is the gold piece (gp). A gold piece is worth 10 silver pieces. Each silver piece is worth 10 copper pieces (cp). In addition to copper, silver, and gold coins, there are also platinum pieces (pp), which are each worth 10 gp.

        The standard coin weighs about a third of an ounce (fifty to the pound).

        \begin{dtable}
            \lcaption{Coin Exchange Values}
            \begin{dtabularx}{\columnwidth}{l c *{4}{>{\ccol}X}}
                & & \tb{CP} & \tb{SP} & \tb{GP} & \tb{PP} \tableheaderrule
                Copper piece (cp) & = & 1 & 1/10 & 1/100 & 1/1,000 \\
                Silver piece (sp) & = & 10 & 1 & 1/10 & 1/100 \\
                Gold piece (gp) & = & 100 & 10 & 1 & 1/10 \\
                Platinum piece (pp) & = & 1,000 & 100 & 10 & 1
            \end{dtabularx}
        \end{dtable}

    \subsection{Wealth Other Than Coins}
        Merchants commonly exchange trade goods without using currency. As a means of comparison, some trade goods are detailed below.

        \begin{dtable}
            \lcaption{Trade Goods}
            \begin{dtabularx}{\columnwidth}{l >{\lcol}X}
                \tb{Cost} & \tb{Item} \tableheaderrule
                1 cp & One pound of wheat \\
                2 cp & One pound of flour \\
                1 sp & One pound of iron, or one chicken \\
                5 sp & One pound of tobacco or copper \\
                1 gp & One pound of cinnamon, or one goat \\
                2 gp & One pound of ginger or pepper, or one sheep \\
                3 gp & One pig \\
                4 gp & One square yard of linen \\
                5 gp & One pound of salt or silver \\
                10 gp & One square yard of silk, or one cow \\
                15 gp & One pound of saffron or cloves, or one ox \\
                50 gp & One pound of gold \\
                500 gp & One pound of platinum
            \end{dtabularx}
        \end{dtable}

    \subsection{Selling Items}
        In general, a character can sell something for a quarter its listed price.

        Trade goods, such as gems, are the exception to this rule and can be sold for their full value.
        A trade good, in this sense, is a valuable good that can be easily exchanged almost as if it were cash itself.


\chapter{Magic}\label{Magic}

Magic comes in many forms, but it is most commonly wielded with spells.
A spell is a one-time magical effect.
There are three types of spells: arcane (cast by mages), divine (cast by clerics), and nature (cast by druids). Cutting across these categories are the nine schools of magic.
Each of the nine schools represents a different type of mastery over the world, based on fundamentally distinct principles.

\section{Casting Spells}\label{Casting Spells}
    Whether a spell is arcane, divine, or natural, casting a spell works the same way.

    \subsection{Casting Process}

        \begin{itemize}
            \itemhead{Choose spell}: You must choose which spell to cast from among the spells you know.
                If a spell has \glossterm{subspells}, you must choose which subspell to use when you cast it.
            \itemhead{Choose augments}: If you know any \glossterm{augments}, you can apply any number of augments to the spell.
                If you apply an augment, you increase the spell's level by an amount equal to that augment's level.
                For details, see \pcref{Augments}.
            \itemhead{Pay action point}: If necessary, you must expend an action point to cast the spell.
                If you do not have an action point to spend, your attempt to cast the spell fails.
                Effects that replace action point costs also happen at this time.
            \itemhead{Perform spell components}: All spells have verbal and somatic components unless their description indicates otherwise (see \pcref{Components}).
            \itemhead{Choose effects}: You make choices about the spell's effects as you finish casting the spell.
                This includes deciding which creatures to target, where the spell takes effect, and so on.
        \end{itemize}

        All of the above steps take place at the start of the action phase, at the same time that other actions are decided.
        However, spells take time to cast, and their effects do not resolve during the \glossterm{action phase}.
        Instead, the targeting and effects of spells are resolved simultaneously with other actions during the \glossterm{delayed action phase}.
        If you take damage or are otherwise distracted during a phase in which you attempt to cast a spell, you may \glossterm{miscast} the spell (see \pcref{Concentration}).

    \subsection{Focusing and Concentration}\label{Concentration}\label{Focus}\label{Focusing and Concentration}

        Some actions require focusing, such as casting spells.
        If you are damaged or distracted while taking an action that requires focus, your concentration may be broken.

        \parhead{Concentration Checks}\label{Concentration Checks}

        To make a concentration check, roll d10 \add your level or Willpower \sub \glossterm{overwhelm penalties}.
        The DR is equal to 5 \add (twice the level of the spell you are casting).
        If the total damage you took in the current round exceeds your Mental defense, you take a \minus5 penalty to this check.
        If the damage exceeds the defense by 10, you take a \minus10 penalty instead.

        Success means you cast the spell successfully.
        Failure means you miscast the spell (see \pcref{Miscasting}).

        \parhead{Casting a Spell} You must concentrate to cast spells.
        When you finish casting a spell, if you took any damage while casting it, make a Concentration check (see \pcref{Concentration Checks}). Failure means you miscast the spell (see \pcref{Miscasting}), but you still lose the spell slot used to cast it.

        \parhead{Focusing on Existing Spells} Many spells allow you to spend a standard action focusing to extend their effects.
        At the end of every round you focus, if you took any damage, make a Concentration check.
        Failure means your focus ends, but the spell may continue to have effects, as indicated in the spell description.
        Most spells do not allow you to resume focusing on them after your concentration is broken.

        \parhead{Performing Rituals} You must focus to perform rituals.
        At the end of every round, if you took any damage, make a Concentration check.
        Failure means the ritual fails and has no effect.

        \parhead{Distracting Circumstances} In some circumstances, you need to Concentration make a concentration check to cast spells or take other actions even if you haven't taken damage.
        Examples include being on a galloping horse, in a storm-tossed ship, or in an earthquake.

        \parhead{Focus Limits} Focusing on a spell is mentally tiring.
        You can focus on a spell for up to 5 minutes without penalty.
        After 5 minutes, and every minute thereafter, you must make a Concentration check even if you haven't taken damage.
        If you fail, you lose your focus on the spell and become fatigued.
        The difficulty of the test increases by 2 for every additional minute of focus.

    \subsection{Miscasting}\label{Miscasting}

        If you start casting a spell and fail to complete it, such as if your concentration is broken or your armor interferes with your spellcasting, you miscast the spell.
        When you miscast a spell, the spell does not have its normal effect.
        Instead, a wave of magical energy causes a \glossterm{miscast backlash}.
        When a mystic backlash occurs, make an attack against the Mental defense of yourself and all creatures within a 5 foot radius.
        Your accuracy and \glossterm{power} with this ability is equal to your \glossterm{spellpower} with the spell you tried to cast.
        On a hit, each target takes \glossterm{standard damage} \minus1d.

        \parhead{Voluntary Miscasting} At the start of each phase while you are casting a spell, you can choose to stop casting the spell, causing you to \glossterm{miscast} it instead.

    \subsection{Subspells}\label{Subspells}
        % TODO: some of this wording is weird
        All spells have a number of \glossterm{subspells}.
        Each subspell has a name, a level, and an effect.
        Whenever you cast a spell, you can choose to apply a single subspell you know from that spell.
        If you do, the spell's level becomes equal to the subspell's level.
        In exchange, the spell gains the effects of the subspell.
        You cannot learn or cast subspells whose spell level exceeds your maximum spell level.

        Some subspells simply add additional properties to a spell's normal effect.
        Others change the targets or effects of the spell significantly.
        After choosing whether to cast a subspell, you can apply any number of \glossterm{augments}, described below.

    \subsection{Augments}\label{Augments}
        There are a number of \glossterm{augments} that can be applied to spells and rituals to increase their power.
        Each augment has a name, a level, and an effect.
        Whenever you cast a spell or perform a ritual, you can choose to apply any number of augments you know to the spell or ritual.
        For each augment you apply, you increase the spell or ritual's level by an amount equal to the augment's level.
        In exchange, the ability gains the effects of that augment.
        If an augment would increase the spell or ritual's level beyond the maximum level you can cast, you cannot apply the augment to that ability.

        \parhead{Augments and Subspells}
        If a spell or ritual changes its properties with a subspell or subritual, it may become eligible for different augments.
        % TODO: fix example
        For example, if you apply the Fireball subspell to the \spell{pyromancy} spell, it changes to affect an area.
        You would then be able to apply the Widened augment to increase its area.

        \subsubsection{Augment Descriptions}\label{Augment Descriptions}

            \augment{1}{Giant} The ability can affect a target one size category larger.
            This augment can be applied multiple times.
            Its effects stack.
            \par This augment can be applied to any spell or ritual that that has a maximum size category of targets that it can affect.

            \augment{2}{Cryptic} The spell's visual effects and magical aura changes to mimic a different spell of your choice.
            You may choose any combination of spell or \glossterm{subspell} you know, along with any other augments, that result in a spell of the same level or lower as the spell you are casting.
            This affects inspection of the spell itself by any means, such as with the Spellcraft skill (see \pcref{Spellcraft}).
            However, it does not alter the mechanical effects of the spell in any way.
            If the spell's effects depend on visual components, the spell may fail to work if you alter the spell's visuals too much. 

            \augment{2}{Extended} The ability's range increases by one step, to a maximum of \rngext.
            The steps are, in order: \rngtouch, \rngclose, \rngmed, \rnglong, and \rngext.
            This augment can be applied multiple times.
            Each time, the ability's range increases by an additional step.
            \par This augment can be applied to any spell or ritual with a range that is one of the above ranges.

            \augment{2}{Quickened} You can cast the spell as a \glossterm{minor action}.
            In exchange, you cannot take any actions during the \glossterm{action phase} or \glossterm{delayed action phase} of the next round.
            \par This augment can be applied to any spell.

            \augment{2}{Selective} You may freely exclude any areas from the spell's effect.
            However, all squares in the final area of the spell must be contiguous.
            You cannot create split a spell's area into multiple completely separate areas.
            \par This augment can be applied to any spell or ritual that affects an area.

            \augment{2}{Silent} You do not need to use \glossterm{verbal components} to cast the spell.
            \par This augment can be applied to any spell.

            \augment{2}{Stilled} You do not need to use \glossterm{somatic components} to cast the spell.
            \par This augment can be applied to any spell.

            \augment{2}{Widened} The ability's area increases by one step, to a maximum of \areahuge.
            The steps are, in order: \areasmall, \areamed, \arealarge, and \areahuge.
            This augment cannot affect abilities with other areas.
            Normally, a Small or Medium line is 5 ft.\ wide, while a Large or Huge line is 10 ft.\ wide.
            A line used to define a wall does not have a width.
            This augment can be applied multiple times.
            Each time, the ability's area increases by an additional step.
            \par This augment can be applied to any spell or ritual with an area that is one of the above areas.

            \augment{3}{Dual} The spell targets an additional creature within range.
            This augment can be applied multiple times.
            Its effects stack.
            \par This augment can be applied to any spell that has a range and affects a single target of the caster's choice.
            It cannot be applied to spells that affect a single specific target, such as the caster.

            \augment{3}{Intensified} The ability deals \plus1d damage.
            This augment can be applied multiple times.
            Its effects stack.
            \par This augment can be applied to any spell or ritual that deals a dice pool of damage.

            \augment{3}{Phasing} When determining whether you have \glossterm{line of sight} and \glossterm{line of effect} to a particular location with the spell, you can ignore a single solid obstacle up to five feet thick.
            This can allow you to cast spells through solid walls, though it does not grant you the ability to see through the wall.
            \par This augment can be applied to any spell with a range.

            \augment{3}{Precise} You gain a \plus1 bonus to accuracy with the ability.
            This augment can be applied multiple times.
            Its effects stack.
            \par This augment can be applied to any spell or ritual that has an attack roll.

            \augment{4}{Accelerated} The ritual takes half the normal amount of time to perform.
            \par This augment can be applied to any ritual.

            \augment{6}{Echoing} During the \glossterm{delayed action phase} of the next round, the spell's effect occurs again.
            All choices you made for the original casting of the spell are made identically for the repeat casting.
            It affects the same area, targets, and so on.

            If the spell is now invalid, such as if all of its targets are out of range, the additional casting has no effect.
            This augment does not allow you to \glossterm{attune} to the same spell more than once.

    \subsection{Dismissing Spells}

        As a \glossterm{minor action}, you can dismiss any spells you cast that have lasting effects.
        This requires the same casting components (verbal and somatic) as casting the spell normally.
        The effects of a dismissed spell immediately end.

    \subsection{Impossible Spell Effects}
        If you try to cast a spell in circumstances that make the spell's effect impossible, the spell fails and has no effect.
        You still lose the spell slot used to cast it.

\section{Determining Spell Effects}

    \subsection{Spellpower}

        Both the accuracy and power of your spells is determined by your spellpower.
        Normally, your spellpower is equal to your level or an \glossterm{attribute}, whichever is higher.
        Effects that increase spellpower never increase spells per day or spells known.
        Only your class levels affect those values.

        \parhead{Multiple Spell Sources} If you the ability to cast spells from more than one separate ability, use the spellpower appropriate to the ability that you are casting the spell with.

        \parhead{Reducing Spellpower} You can voluntarily reduce the power of the spells you cast by using a lower spellpower.
        However, you cannot use a spellpower lower than the minimum spellpower required to cast the spell, which is equal to twice the spell's level.

        \parhead{Magic Resistance} Some creatures have magic resistance, which is an ability which allows them to resist \glossterm{magical} effects such as spells.
        You can overcome magic resistance by making an attack with an accuracy equal to your spellpower.
        See \pcref{Magic Resistance}, for details.

    \subsection{Magical Attacks}

        To affect an unwilling creature with a spell, you must make a magical attack.
        Your accuracy is normally equal to your spellpower.

        \subsubsection{Resisting a Spell} A creature that successfully resists a spell that has no obvious physical effects feels a hostile force or a tingle, but cannot deduce the exact nature of the attack without the use of the Spellcraft skill (see \pcref{Spellcraft}).

        \subsubsection{Not Resisting a Spell} A creature can voluntarily forego its defenses and willingly accept a spell's result.
            However, a character with a special immunity to specific magical effects cannot suppress that quality.

    \subsection{Line of Effect}\label{Line of Effect}

        Almost all abilities must have a \glossterm{line of effect} to function.
        Unless otherwise noted in an ability's description, you cannot target a creature you do not have line of effect to.
        In addition, spells that affect an area do not affect targets that the spell does not have line of effect to.

        A line of effect is a straight, unblocked path that indicates what a spell can affect.
        A line of effect is canceled by a solid barrier.
        It's like line of sight for ranged weapons, except that it's not blocked by fog, darkness, and other factors that limit normal sight.

        You must have a clear line of effect to any target that you cast a spell on or to any space in which you wish to create an effect.
        You must have a clear line of effect to the point of origin of any spell you cast.

        A burst, cone, cylinder, or emanation spell affects only an area, creatures, or objects to which it has line of effect from its origin (a spherical burst's center point, a cone-shaped burst's starting point, a cylinder's circle, or an emanation's point of origin).

        An otherwise solid barrier with a hole of at least 1 square foot through it does not block a spell's line of effect.
        Such an opening means that the 5-foot length of wall containing the hole is no longer considered a barrier for purposes of a spell's line of effect.

        \subsubsection{Destroying Barriers}\label{Destroying Barriers}
            Some abilities, such as the \spell{fireball} spell, deal damage to both creatures and objects.
            If a physical barrier is destroyed by an ability, that barrier does not affect the ability's line of effect.
            For example, a thin curtain of silk normally blocks line of effect.
            However, a spell that destroyed the curtain would have its full effect on everything behind the curtain.

    \subsection{Targeting Spells}

        \parhead{Midair Locations} A creature or object brought into being or transported to your location by a spell cannot appear floating in an empty space.
        It must arrive in an open location on a surface capable of supporting it.

        \parhead{Targeting Inside Creatures} Creatures block line of effect to the inside of their own bodies.
        As a result, you cannot cast a spell that takes effect inside a creature unless you are also inside the creature.
        This restriction applies even if there is no physical barrier to the inside of the creature; you cannot detonate a \spell{fireball} inside a creature's mouth, even if it has its mouth open at the time.

    \subsection{Special Spell Effects}

        \subsubsection{Attacks}
            Some spell descriptions refer to attacking.
            All abilities that affect any unwilling creatures, even if they don't deal damage, are considered attacks.
            If all creatures affected by a spell are \glossterm{willing}, the spell is not considered an attack.
            Spells that damage objects or summon allies are not attacks because the spells themselves don't harm anyone.

        \subsubsection{Resurrecting the Dead}\label{Resurrecting the Dead}

            Several spells have the power to restore slain characters to life.

            When a living creature dies, its soul departs its body, leaves the Material Plane, travels through the Astral Plane, and goes to abide on the plane where the creature's deity resides.
            If the creature did not worship a deity, its soul departs to the plane corresponding to its alignment.
            Bringing someone back from the dead means retrieving his or her soul and returning it to his or her body.

            \subparhead{Preventing Revivification} Enemies can take steps to make it more difficult for a character to be returned from the dead.
            Except for \spell{true resurrection}, every ritual to raise the dead requires a body, so keeping or destroying the body is an effective deterrent.
            The \spell{soul bind} ritual prevents any sort of revivification unless the soul is first released.

            \subparhead{Revivification against One's Will} A soul cannot be returned to life if it does not wish to be.
            A soul infallibly knows the name, alignment, and patron deity (if any) of the character attempting to revive it and may refuse to return on that basis.

    \subsection{Combining Effects}
        Spells or magical effects usually work as described, no matter how many other spells or magical effects happen to be operating in the same area or on the same recipient.
        Except in special cases, a spell does not affect the way another spell operates.
        Whenever a spell has a specific effect on other spells, the spell description explains that effect.

        However, spells, feats, and other abilities that have very similar effects may not both help their target.
        A character can only be increased so far beyond his or her normal limits; even layered with powerful magical effects, a commoner is no serious threat to a giant.
        The limitations on these effects are provided by the stacking rules described below.

        \subsubsection{Stacking Effects}
            Spells that provide bonuses or penalties usually do not stack with themselves.
            More generally, two enhancement bonuses don't stack even if they come from different spells; see \pcref{Stacking Rules}, for more details.

            \parhead{Same Effect More than Once in Different Strengths} In cases when two or more identical spells are operating in the same area or on the same target, but at different strengths, only the best one applies.
            This is called overlapping.

            \parhead{Same Effect with Differing Results} The same spell can sometimes produce varying effects if applied to the same recipient more than once.
            Usually, the last spell in the series trumps the others.
            None of the previous spells are actually removed or \glossterm{dismissed}, but their effects become irrelevant while the final spell in the series lasts.

            \parhead{One Effect Makes Another Irrelevant} Sometimes, one spell can render a later spell irrelevant.
            Both spells are still active, but one has rendered the other useless in some fashion.

            \parhead{Multiple Mind Control Effects} Sometimes magical effects that affect a creature's mind render each other irrelevant, such as a spell that removes the target's ability to act.
            Mental controls that don't remove the recipient's ability to act usually do not interfere with each other.
            If a creature is under the mental control of two or more creatures, it tends to obey each to the best of its ability, and to the extent of the control each effect allows.
            If the controlled creature receives conflicting orders simultaneously, the competing controllers must make opposed spellpower checks to determine which one the creature obeys.

            \parhead{Spells with Opposite Effects} Spells with opposite effects apply normally, with all bonuses, penalties, or changes accruing in the order that they apply.

            \parhead{Instantaneous Effects} Two or more spells with instantaneous effects work cumulatively when they affect the same target.

        % TODO: pull these out into generic descriptions of suppressing/dismissing spells and abilities
        \subsubsection{Suppressing Spells}\label{Suppressing Spells}
            Spells can be \glossterm{suppressed} by effects such as the \spell{antimagic} spell.
            While a spell is suppressed, it has no effect.
            However, if it stops being suppressed, its effects continue as if they had not been interrupted.

        \subsubsection{Dismissing Spells}\label{Dismissing Spells}
            When a spell is \glossterm{dismissed}, all of its effects end.
            Unless otherwise specified, any spell with a \glossterm{duration} can be dismissed.

            If a spell affects multiple targets, it must be dismissed individually on each target.
            Dismissing the effect on one target does not affect the other targets of the spell.
            When you voluntarily dismiss a spell, you can choose to dismiss it for any number of targets with no more effort than dismissing it for a single target.

% TODO: make this whole section refer to abilities instead of spells
\section{Spell Descriptions}
    The description of each spell is presented in a standard format.
    Each category of information is explained and defined below.

    \subsection{Name}
        The first line of every spell description gives the name by which the spell is generally known.

    \subsection{Description}
        Beneath the spell name is a brief description of the spell's effect.
        This description has no mechanical significance, and simply describes how the spell usually appears or is used.

    \subsection{Schools of Magic}\label{Schools of Magic}
        The next line describes the schools of magic that the spell belongs to.
        Almost every spell belongs to at least one of nine schools of magic.
        A school of magic is a group of related spells that work in similar ways.
        They are described below.

        Some spells belong to more than one school of magic.
        Treat these spells for all purposes as if they were a member of both schools simultaneously.
        If you are prohibited from casting spells from a certain school, you cannot cast a spell which belongs to that school, even if it also belongs to another school.
        Likewise, any benefits which apply to casting spells from a specific school apply normally.
        If you have abilities which apply when casting spells from both schools that make up a spell, the abilities do not stack.

        A small number of spells (\spell{limited wish}, \spell{permanency}, \spell{prestidigitation}, and \spell{wish}) are universal, belonging to no school.

        \subsubsection{Abjuration}
            Abjuration spells reduce or negate damage, magic, and other effects.
            They can be used to protect allies and remove harmful magic.

        \subsubsection{Channeling}
            Channeling spells call upon the power of deities or other supernatural entities.
            They can be used to do anything those entities could do.
            Arcane spellcasters do not have access to Channeling spells.

        \subsubsection{Conjuration}
            Conjuration spells create and transport objects and creatures.
            They can be used to summon allies, transport creatures, and create objects from thin air.

        \subsubsection{Divination}
            Divination spells grant knowledge.
            They can be used to reveal hidden truths, predict the future, or communicate at great distances.

        \subsubsection{Enchantment}
            Enchantment spells alter the minds of creatures.
            They can be used to influence, control, or debilitate creatures.
            Almost all enchantment spells are \glossterm{Mind} spells, and many are \glossterm{Subtle} as well.

        \subsubsection{Evocation}
            Evocation spells create and manipulate energy.
            They can be used to inflict damage with energy blasts or manipulate the environment.

        \subsubsection{Illusion}
            Illusion spells create or manipulate sensory impressions.
            They can be used to create or remove light, conceal things that exist, or cause creatures to perceive things that do not exist.

        \subsubsection{Transmutation}
            Transmutation spells change the properties of creatures and objects.
            They can be used to grant new abilities, enhance existing abilities, change a target's form, or even alter the flow of time itself.

        \subsubsection{Vivimancy}
            Vivimancy spells manipulate the power of life and death, as well as souls.
            They can be used to heal or inflict wounds, resurrect the dead, create undead monsters, and cripple the bodies of creatures.

    \subsection{[Tags]}
        Appearing on the same line as the school, when applicable, are tags which further categorize the spell in some way.
        Ability tags are described at \pcref{Ability Tags}.

    \subsection{Components}\label{Components}
        A spell's components are what you must do or possess to cast it.
        All spells have verbal and somatic components unless the spell description says otherwise.
        The Components entry in a spell description includes abbreviations that tell you what type of components it has.
        Specifics for material components and focuses are given at the end of the descriptive text.

        \parhead{Verbal (V)} A verbal component is a spoken incantation.
        To provide a verbal component, you speak in a strong voice with a volume at least as loud as ordinary conversation.

        A gag spoils the incantation (and thus the spell). A spellcaster who has been deafened has a 20\% chance to spoil any spell with a verbal component that he or she tries to cast.
        Likewise, a \spell{silence} spell imposes a 20\% chance of failure.

        \parhead{Somatic (S)} A somatic component is a measured and precise movement of at least one hand.
        While casting a spell with somatic components, one hand is used to cast the spell, and cannot be used to defend yourself or take other actions.
        % Touch range spells often include the act of touching the spell recipient as part of the somatic component.

        \parhead{Material (M)} A material component is one or more physical substances or objects that are annihilated by the spell energies in the casting process.

    \subsection{Casting Time}
        All spells have a casting time of one standard action unless otherwise specified in the spell description.
        Some spells and subspells require only a \glossterm{minor action} to cast.
        If a spell can be cast as a minor action, any of its subspells that require a standard action to cast state that explicitly.

        You make all pertinent decisions about a spell (range, target, area, effect, version, and so forth) when you finish casting the spell, not when you start casting.

    \subsection{Range}
        A spell's range indicates how far from you it can reach, as defined in the Range entry of the spell description.
        It indicates the maximum distance at which you can designate the spell's point of origin.
        The effect of a spell can extend beyond that range if it affects an area.
        A spell without a range simply affects the area specified in the spell's description; if it becomes relevant, you are considered to be its point of origin.
        Standard ranges include the following.
        % TODO: are there any touch spells left?
        \parhead{Touch} You must touch a creature or object to affect it.
        Touching a creature requires a successful attack against its Reflex defense.
        Some touch spells allow you to touch multiple targets.
        You can touch as many willing targets as you can reach as part of the casting, but all targets of the spell must be touched in the same round that you finish casting the spell.
        Normally, you can touch no more than six targets per round of casting.

        If you have the ability to make multiple touch attacks, such as from the \spell{chill touch} spell, and you can make multiple attacks in a round, you can make a touch attack on each of those attacks.

        \parhead{Close} The spell reaches as far as 30 feet.
        \parhead{Medium} The spell reaches as far as 100 feet.
        \parhead{Far} The spell reaches as far as 300 feet.
        \parhead{Unlimited} The spell reaches anywhere on the same plane of existence.
        \parhead{Arbitrary} Some spells have no standard range category, just a range expressed in feet.

        \subsubsection{Unrestricted Ranges}

            Some spells have an unrestricted range, as denoted by \rngunrestricted.
            A spell with an unrestricted range does not require line of sight or line of effect.

    \subsection{Area}\label{Spell Area}

        Some spells affect an area.
        Sometimes a spell description specifies a specially defined area, but usually an area falls into one of the categories defined below.

        When casting an area spell, you select the point where the spell originates.
        The point of origin of a spell is always a grid intersection.
        When determining whether a given creature is within the area of a spell, count out the distance from the point of origin in squares just as you do when moving a character or when determining the range for a ranged attack.
        The only difference is that instead of counting from the center of one square to the center of the next, you count from intersection to intersection.

        You can freely decrease a spell's area, provided that you decrease it uniformly across all of the spell's dimensions.
        For example, you can cast a \spell{fireball} that affects a 5 foot radius if you choose to do so, but you can't cast a \spell{fireball} with any shape other than a sphere.

        You can count diagonally across a square, but remember that every second diagonal counts as 2 squares of distance.
        If the far edge of a square is within the spell's area, anything within that square is within the spell's area.
        If the spell's area only touches the near edge of a square, however, anything within that square is unaffected by the spell.

        \subsubsection{Area Types}\label{Area Types}

            \parhead{Burst} A burst spell has an immediate effect on all valid targets within an area.

            \parhead{Emanation} An emanation spell has effects within an area for the duration of the spell.
            It emanates from a specific creature or object, rather than a location.
            If that creature or object moves, the emanation moves with it.

            \parhead{Zone} A zone spell has effects within an area for the duration of the spell.
            Unless otherwise noted, it does not move after being created.

        \subsubsection{Area Shapes}

            \parhead{Cone} A cone extends from the point of origin in a quarter-sphere, up to the given length.

            \parhead{Cylinder} A cylinder extends out from the point of origin in a circle, up to the given radius.
            Cylinders also have a specific height.
            Unless otherwise specified, a cylinder's height is the same as its radius.
            Cylinders ignore obstacles that partially block line of effect, as long as there is a path around the obstacle that lies entirely within the spell's area.

            \parhead{Line} A line extends from the point of origin in a straight line, up to the given length.
            Lines also have a specific width and height.
            Unless otherwise specified, a line-shaped spell affects an area 5 feet wide and 5 feet high.
            The affected squares are chosen such that they stay close to the chosen line as possible.
            All squares affected by a line must be contiguous, so every square is adjacent to another affected square, disregarding diagonals.

            \parhead{Sphere} A sphere extends from the point of origin in all directions.
            Any spell which only specifies a radius for its area is sphere-shaped.

            \parhead{Wall} A wall is like a line, except that it has no width.
            Instead, it affects the boundary between squares.
            Walls can also be shapeable.

            Walls can normally be created within occupied squares, but not within solid objects.
            Some walls are called solid walls, and cannot be created within occupied squares.

            \parhead{Specific Shapes} Some spells specify a series of volumes that make up the area of the spell.
            Most commonly, the volumes are cubes.
            You may arrange the volumes as you want, with the restriction that each volume in the spell's area must be adjacent to one other volume in the spell's area.

        \subsubsection{Area Sizes}

            The area affected by many spells falls into one of three sizes.
            Each size defines the extent to which the spell extends out from its origin, whether as a radius or as a length.
            Some spells have specific sizes, as given in the spell description.

            \parhead{Small} Small spells extend 10 feet from their point of origin.
            \parhead{Medium} Medium spells extend 20 feet from their point of origin.
            \parhead{Large} Large spells extend 50 feet from their point of origin.

    \subsection{Targets}
        Some spells have a target or targets.
        You cast these spells on creatures or objects, as defined by the spell itself.
        You must be able to see or touch the target, and you must specifically choose that target.
        You do not have to select your target until you finish casting the spell.

        \subparhead{Multiple Targets} Most spells which have multiple targets also specify an area that the targets must reside in.
        If the spell says ``all creatures'', you do not have the ability to choose which creatures it affects; otherwise, you may pick and choose creatures within the area.

        \subparhead{Redirecting a Spell} Some spells allow you to redirect the effect to new targets or areas after you cast the spell.
        Redirecting a spell is a minor action.

        \subparhead{Targeting Restrictions} Many spells affect ``living creatures'', which means all creatures other than constructs and undead.
        Creatures in the spell's area that are not of the appropriate type do not count against the creatures affected.

        \subparhead{Willing Targets} Some spells restrict you to willing targets only.
        You can choose to be a willing target at any time.
        Unconscious creatures and objects are automatically considered willing, but a character who is conscious but immobile or helpless (such as one who is bound or paralyzed) is not automatically willing.

        \subparhead{Invalid Targets} You can always attempt to cast a spell on an invalid target.
        If the target is still invalid when the spell resolves, the spell is automatically \glossterm{miscast}.
        For example, you could attempt to cast the \spell{finger of death} spell, which only targets living creatures, on a creature that is secretly undead.
        The spell would automatically be miscast, which may reveal the target's true nature.

    \subsection{Duration}

        Many abilities have lingering effects that last for some \glossterm{duration}.

        \parhead{Sustain}\label{Sustain} Some abilities last as long as you take an action to sustain them.
        The type of action required is always specified in the ability.
        At the end of each round, the ability is dismissed unless you used the ability that round or took the action to sustain the ability that round.
        Sustaining spells does not take concentration, and cannot be disrupted in the same way that casting spells can.

        Taking an action to sustain an ability only allows you to sustain a single use of that ability.
        However, you can sustain multiple abilities at once if you have available actions.

        You can only sustain an ability for up to 5 minutes.
        After that time, the ability's effect is dismissed.

        \subparhead{Shared Sustain} Some abilities last as long as ``you and the target'' or ``you and each target'' sustain the ability.
        If you are a target, you only need to sustain the ability once, not twice.

        \parhead{Attunement} The ability lasts as long as you \glossterm{attune} to it (see \pcref{Attunement}).

        \subparhead{Multiple Attunement} Normally, you can only attune to one copy of a given ability at once.
        Some abilities allow you to attune to them multiple times.

        \subparhead{Shared Attunement} Some abilities last as long as ``you and the target'' or ``you and each target'' attune to the ability.
        If you are a target, you only need to attune to the ability once, not twice.

        \parhead{Condition} The ability lasts until its target removes it, such as by taking the Recover action (see \pcref{Recover}).
        Only abilities that affect creatures can have the Condition duration.

        \parhead{Permanent} The ability lasts until it is somehow cancelled or removed, such as with the \spell{antimagic} spell.

        \parhead{Instantaneous} Abilities without a listed duration are instantaneous.

        \parhead{Targets, Effects, and Areas} If an ability affects creatures directly, the effects travel with the targets for the ability's duration.
        If an ability creates or summons objects or creatures, they last for the duration, and are capable of moving outside the ability's initial range.
        Such effects can sometimes be destroyed prior to when their duration ends.

        % This redundant with the definition of emanations elsewhere
        % If an ability creates an emanation, then the spell stays with that area for its duration.
        % Creatures become subject to the spell when they enter the emanation and are no longer subject to it when they leave.

        % \parhead{Discharge} Occasionally a spells lasts for a set duration or until triggered or discharged.

    \subsection{Magic Resistance}\label{Magic Resistance}
        Magic resistance is an additional defense against \glossterm{magical} abilities such as spells.
        To affect a magic resistant creature with a magical ability, you must make an additional magical attack against the creature's magic resistance value.
        Your accuracy is equal to your \glossterm{power} with the ability you using, such as your spellpower with spells.
        If your attack result beats the creature's magic resistance, the ability works normally.
        Otherwise, the ability has no effect on the creature.

        Magic resistance does not prevent a magical ability from having its normal effect on other creatures or objects.
        Magical abilities which do not directly affect targets, such as the \spell{summon monster} or \spell{create image} spells, do not allow magic resistance.
        In addition, Thaumaturgy and Physical abilities do not allow magic resistance (see \pcref{Ability Tags}).

        Normally, creatures with magic resistance can choose to allow spells through their resistance.
        Some creatures cannot control their magic resistance, so an attack is always necessary to affect them.
        This is specified in the description of the creature's magic resistance.

    \subsection{Effect}
        This portion of a spell description details what the spell does and how it works.
        If one of the previous entries in the description included ``see text'', this is where the explanation is found.
        There are several key parts of a spell which are also contained here.

        \subsubsection{Damage}
            This is the amount of damage the spell deals.
            Typically, the effect will specify who takes the damage.
            If no effect is specified, the spell damages all of its targets, or all creatures (but not objects) in the area.
            A spell with this entry is considered a damaging spell.
            A spell without this entry is not, even if it could be used to deal damage.

            Spells can inflict many kinds of damage.
            Damage types are described in \tref{Damage Types}.

            \parhead{Damaging Items} Unless the descriptive text for the spell specifies otherwise, all items carried or worn by a creature are assumed to survive a magical attack.
            If an item is not carried or worn and is not magical, it does not get any defenses.
            It simply is dealt the appropriate damage.

        \subsubsection{Healing}
            This is the amount of damage the spell heals.
            Typically, the effect will specify who receives the healing.
            If no effect is specified, the spell heals all of its targets, or all creatures (but not objects) in the area.

\section{Ability Tags}\label{Ability Tags}

    Many spells and other abilities have tags that describe the ability's nature.
    Many of these tags have no game effect by themselves, but they govern how the ability interacts with spells, other abilities, unusual creatures, and so on.
    They are described below.

    \parhead{Acid} Acid abilities use corrosive acid.
    They do not function underwater.

    \parhead{Air} Air abilities control the surrounding air.
    They do not function in environments without air.

    \parhead{Auditory} Auditory abilities use sound to cause their effects.
    Creatures and objects that cannot hear the effect are immune to it.

    \parhead{Cold} Cold abilities use cold \glossterm{energy}. It is possible to freeze liquids and perform similar feats with cold abilities.

    \parhead{Compulsion} Compulsion abilities forcibly alter a creature's actions, but do not necessarily affect its opinions or personality.
    All Compulsion abilities are also \glossterm{Mind} abilities.

    \parhead{Creation} Creation abilities create permanent physical objects.
    Objects created with Creation abilities are identical to objects created through more mundane means.
    Unless otherwise specified, magical Creation abilities do not allow \glossterm{magic resistance}.

    \parhead{Curse} Curse abilities lay supernatural curses on their targets.
    They cannot be \glossterm{dismissed}, but can be removed with the \spell{break enchantment} or \spell{remove curse} spells.

    \parhead{Death} Death abilities only affect living creatures.
    A creature killed by a death effect cannot be returned to life by \spell{resurrection} or similar abilities that depend on an intact corpse.

    \parhead{Emotion} Emotion abilities alter a creature's opinons or personality, but do not necessarily affect their actions.
    All Emotion abilities are also Mind abilities.

    \parhead{Detection} Detection abilities reveal magical auras or information within an area.
    They can penetrate up to 1 foot of stone, 1 inch of common metal, a thin sheet of lead, or 3 feet of wood or dirt.
    For its ability to penetrate other materials, use the most similar substance from the list above.

    \parhead{Earth} Earth abilities manipulate the ground or other forms of dirt.
    They do not function if no earth is accessible.

    \parhead{Electricity} Electricity abilities use electrical \glossterm{energy}.

    \parhead{Enhancement} Enhancement abilities enhance the existing abilities of their targets.

    \parhead{Figment} Figment abilities create light, sound, or other sensations.
    Figments cannot remove real sensations present in their area, but they can add additional sensations.
    You can only create figments of sensations you understand; for example, you cannot create a figment which speaks in a language you do not understand.
    \par A figment's physical defenses are equal to 0.

    \parhead{Fire} Fire abilities use fire \glossterm{energy}. They do not function underwater.
    \par Fire abilities provide light equivalent to a torch for their duration.
    Abilities without a duration create a brief burst of torchlight.

    \parhead{Flesh} Flesh abilities manipulate the physical flesh of creatures.
    They have no effect on creatures without flesh, such as ghosts or oozes.

    \parhead{Glamer} Glamer abilities alter sensations present in an area or on a target.
    They can be used to change how something real appears, or to remove it from perception entirely.

    \parhead{Imbuement} Imbuement abilities imbue their targets with magic, granting them new abilities.

    \parhead{Life} Life abilities attack, restore, or manipulate the life force of creatures.
    They have no effect on objects and creatures that are not alive.
    \par Undead creatures are affected in a special way by Life abilities.
    In addition to any differences given in the effect's description, life damage instead heals undead creatures, and healing instead deals life damage.

    \parhead{Light} Light abilities create visible light.
    Their area is blocked by barriers that prevent sight, even if the barriers would not otherwise block effect areas.
    Similarly, their area of effect is not blocked by barriers which do not prevent sight, even if the barriers would normally block effect areas.

    \parhead{Manifestation} Manifestation abilities create temporary constructs formed from raw magical energy.
    Objects and creatures created with manifestation abilities seem real on the surface, but they have no internal structure.
    When an object or creature created by a Manifestation ability is destroyed or killed, or when the duration of the ability that created it ends, it disappears without a trace.
    Unlike \glossterm{Creation} abilities, magical Manifestation abilities allow \glossterm{magic resistance}.

    \parhead{Mind} Mind abilities manipulate the minds of creatures.
    They have no effect on objects or creatures without minds.

    \parhead{Physical} Physical abilities manipulate physical objects rather than having a direct magical effect on their targets.
    They do not allow magic resistance.
    Some abilities are not themselves Physical, but have Physical effects, such as \spell{mighty throw}.

    \parhead{Planar} Planar abilities transport matter or information between planes.

    \parhead{Poison} Poison abilities use substances to weaken the foe's body.

    \parhead{Scrying} Scrying abilities create one or more invisible magical sensors that send you information.
    Unless otherwise noted, the sensor created has the same powers of sensory acuity that you possess.
    This includes the effect of any abilities which target you personally, such as spells to increase your visual acuity, but not abilities which affect an area around you.
    However, the sensor is treated as a separate, independent sensory organ, and it functions normally even if you have been blinded, deafened, or otherwise suffered sensory impairment.
    \par Any creature trained in Spellcraft can notice the sensor by making a DR 20 Spellcraft check.
    The sensor can be dismissed as if it were an active spell.
    You cannot create a sensor in a location with lead sheeting between you and the location, and you sense that the effect is blocked in this way.
    \\

    \parhead{Shaping} Shaping abilities change the shape or structure of their targets.

    \parhead{Shielding} Shielding abilities improve the defenses of their targets.

    \parhead{Sizing} Sizing abilities alter the size of their targets.
    Unless otherwise stated, multiple effects which increase or decrease size do not stack.
    Opposing size modifications cancel each other out on a one for one basis, and any remaining effects occur normally.

    \parhead{Sonic} Sonic abilities use sonic \glossterm{energy}.

    \parhead{Speech} Speech abilities use words to achieve their ends.
    You must specify a language when using a Speech effect, and the language must be one you know (or have memorized the correct words to say). They have no effect on objects or creatures that do not understand the chosen language.

    \parhead{Subtle} Subtle abilities have no visual or otherwise perceivable manifestation.
    Creatures affected by Subtle abilities do not generally know that they are being magically influenced.
    Subtle spells can still be identified with the Spellcraft skill (see \pcref{Spellcraft}), but the DR is 10 higher than normal.

    \parhead{Swift} Swift abilities take effect before other abilities used during the same phase.
    For details, see \pcref{Swift Abilities}.

    \parhead{Telekinesis} Telekinesis abilities use telekinesis, the power of the mind.
    Many telekinesis abilities create fields of solid telekinetic force.

    \parhead{Teleportation} Teleportation abilities move creature or objects through the Astral Plane to a distant destination.
    A teleported creature can bring along equipment and held objects as long as their weight does not exceed the creature's maximum load (see \tref{Weight Limits}). Any excess items are left behind, in order of their distance from the creature's body.

    \parhead{Temporal} Temporal abilities alter the flow of time.

    \parhead{Thaumaturgy} Thaumaturgy abilities alter or destroy magic itself.
    They do not allow \glossterm{magic resistance}.

    \parhead{Trap} Trap abilities do not have their full effect immediately.
    All Trap abilities specify a condition or circumstance, such as opening a door, which triggers the full effect of the ability.
    \par Unless otherwise noted, active Trap effects can be detected with the Awareness skill and disabled with the Devices skill before their effect triggers (see \pcref{Awareness}, and \pcref{Devices}).
    The DR to detect and disable the effect is equal to 20 \add the \glossterm{power} of the effect.
    \par No more than one Trap ability can be placed on the same object or in the same area.
    Only the first trap placed has any effect.
    It must be dismissed before any new traps can be placed.

    \parhead{Visual} Visual abilities use visible objects or forces to cause their effects.
    Creatures and objects that cannot see the effect are immune to it.

    \parhead{Water} Water abilities use water to cause their effects.

    % TODO: different section?
    \subsection{Damage Types}\label{Damage Types}
        Abilities can deal many kinds of damage.
        The damage types are listed below, along with any special properties that type of damage has.

        \begin{dtable}
            \lcaption{Damage Types}
            \begin{dtabularx}{\columnwidth}{l X}
                \tb{Name} & \tb{Special Effects} \\
                \bottomrule
                Acid & Effective against many objects \\
                Bludgeoning & A type of physical damage \\
                Cold & \\
                Divine & \\
                Electricity & \\
                Fire & \\
                Life & Heals undead creatures instead of damaging them \\
                Physical & \\
                Piercing & A type of physical damage \\
                Slashing & A type of physical damage \\
                Solar & \\  % Does this need to exist?
                Sonic & Effective against many objects \\
            \end{dtabularx}
        \end{dtable}

\section{Cantrips}\label{Cantrips}
    Cantrips are special spells that arcane casters can use at will.
    Like other spells, they have verbal and somatic components.
    All cantrips take a standard action to cast unless specified otherwise in the description.
    Cantrips are considered to be 0th level for the purpose of spells and abilities which reference spell level.
    They are described at the end of Chapter 12.

\section{Rituals}\label{Rituals}
    Rituals are ceremonies that create magical effects.
    Spellcasting characters can learn and perform rituals.
    You don't memorize a ritual as you would a normal spell; rituals are too complex for all but the most knowledgeable sages to commit to memory.
    To perform a ritual, you need to read from a book or a scroll containing it.
    Rituals are similar to spells, but they are not considered spells.
    \subsection{Ritual Descriptions}
        \par Like a spell, each ritual has a school, a level, and a magical effect.
        % TODO: proper chapter references
        Rituals are described in Chapter 13. The description of each ritual follows the same format as the description of spells in Chapter 12, except that every ritual has a level, like \glossterm{subspells} do.

        \subparhead{Ritual Sources}
        A ritual always matches the magic source of the person performing the ritual.
        For example, \spell{scrying} is an arcane ritual when performed by a wizard, but a divine ritual when performed by a cleric.
    \subsection{Ritual Requirements}
        In order to learn and perform a ritual, you must be able to cast at least one spell of the same level as the ritual.
    \subsection{Ritual Books}
        A ritual book contains one or more rituals that you can use as frequently as you want, as long as you can spend the time and \glossterm{action point} to perform the ritual.
        Scribing a ritual in a ritual book costs an amount of precious inks.
    \subsection{Ritual Costs}\label{Ritual Costs}
        The costs to scribe rituals are described on \trefnp{Ritual Costs}.
        \begin{dtable}
            \lcaption{Ritual Costs}
            \begin{dtabularx}{\columnwidth}{X l l}
                \tb{Ritual Level} & \tb{Cost to Scribe} & \tb{Item Level} \\
                \bottomrule
                1st-Level & 50 gp & 1st \\
                2nd-Level & 200 gp & 3rd \\
                3rd-Level & 500 gp & 4th \\
                4th-Level & 1,250 gp & 7th \\
                5th-Level & 3,000 gp & 9th \\
                6th-Level & 7,500 gp & 11th \\
                7th-Level & 15,000 gp & 12th \\
                8th-Level & 35,000 gp & 14th \\
                9th-Level & 75,000 gp & 16th \\
            \end{dtabularx}
        \end{dtable}

    \subsection{Subrituals}\label{Subrituals}
        Many rituals have \glossterm{subrituals}, just like many spells have \glossterm{subspells} (see \pcref{Subspells}).
        Subrituals work in the same way as subspells, except that they are applied to rituals instead of spells.

    \subsection{Performing Rituals}
        To perform a ritual, you must have a ritual book containing the ritual and the material components required for the ritual.
        Unless otherwise specified, performing a ritual requires spending a single \glossterm{action point}.
        Some rituals require multiple action points to complete.
        Other creatures can supply action points to help you perform rituals; see Ritual Participants, below.

        If you are distracted during the ritual, you must make a Concentration check, just as if you were casting a spell of the ritual's level.
        If you fail, the ritual is ruined and you must start from the beginning.
        % TODO: can we remove the ``casting a ritual'' wording?
        \par Performing a ritual and casting a ritual mean the same thing.

        \subsubsection{Ritual Participants}
            Creatures can assist in the performance of rituals even if they are unable to perform rituals themselves.
            A creature that helps perform a ritual is called a ritual participant, and the creature performing the ritual is called the ritual leader.
            A ritual participant may spend an action point in place of or in addition to the action point spent by the creature performing the ritual.
            It may also \glossterm{attune} to the effect of the ritual in place of the creature performing the ritual.
            Only one creature may attune to the ritual's effect in this way.
            If multiple creatures are willing to spend action points or attune to effects, the ritual leader decides which creatures spend action points or attune to the ritual's effects.

            The steps required to participate in rituals can be complex.
            Ritual participants must be given specific instructions for the actions they must perform during a ritual by a creature who knows how to perform the ritual.
            This instruction generally takes half the time required to perform the ritual.
            A creature cannot participate in rituals unless it has an Intelligence of at least 0, can speak at least one language, and has the fine motor control required to perform the somatic components of spells.

            Normally, a ritual participant can only contribute one action point.
            If the participant can cast spells from the same source as the ritual, they can contribute any number of action points.

            \parhead{Changing Ritual Participation}
            Rituals are deeply complex magic, and they cannot be abandoned or paused partway through.
            If the number of ritual participants in a ritual decreases below its initial value, the ritual fails at the end of the next round if the number of participants is not restored.
            However, ritual participants can transfer their participation to other creatures without disrupting the ritual.

            In order to transfer ritual participation, the new creature must be able to participate in the ritual, and must immediately spend the same number of action points as the creature that it is taking over from.
            Similarly, the ritual leader can transfer their leadership to another creature.
            In addition to the requirements for transferring ritual participation, the new leader must know the ritual and be able to perform it themselves.

            Changing ritual participation and leadership is usually done when performing extraordinarily long or demanding rituals.

    \subsection{Magical Writings}
        To record a spell in written form, a character uses complex notation that describes the magical forces involved in the spell.
        The notation constitutes a universal language that spellcasters have discovered, not invented.
        Each writer uses this universal system regardless of their native language or culture.
        However, each character uses the system in their own way.
        Another person's magical writing remains incomprehensible to even the most powerful spellcaster until they take the time to study and decipher it.

        To decipher an magical writing (such as a single spell in written form on a scroll), you must make a Spellcraft check (DR 10 \add the spell's level). If the skill check fails, you cannot attempt to read that particular spell again until the next day.
        A \ritual{read magic} ritual automatically deciphers a magical writing without a skill check.
        If the person who created the magical writing is on hand to help the reader, success is also automatic.

        Once a character deciphers a particular magical writing, they do not need to decipher it again.
        Deciphering a magical writing allows the reader to identify the spell and gives some idea of its effects (as explained in the spell or ritual description).

\section{Types of Abilities}

    There are two types of abilities: magical abilities and physical abilities.

    \parhead{Magical Abilities}\label{Magical Abilities} A magical ability is an ability that has no physical explanation.
    Examples include spells, a medusa's petrifying gaze, and a cleric's domain invocations.
    Magical attacks often target Fortitude and Mental defenses, and can be resisted by \glossterm{magic resistance}.
    Abilities that are magical in nature are indicated with a [Mag] tag.
    Abilities that are not magical are \glossterm{mundane}.

    Many abilities which fundamentally concern magical effects are not themselves magical in nature.
    This is most commmon with abilities that represent choices the character makes or knowledge the character has.
    For example, although all spells are magical abilities, the ability to cast spells is not itself a magical ability.
    It is simply knowledge that the creature possesses.
    Of course, that knowledge would be useless if the creature had no access to magic.

    \parhead{Physical Abilities}\label{Physical Abilities} A physical ability has a tangible component and some form of natural explanation.
    Examples include weapon attacks, a dragon's breath weapon, and a barbarian's rage.
    Physical attacks often target Armor and Reflex defenses.
    Unless otherwise indicated, all abilities are physical in nature.
    Abilities that are not physical are \glossterm{magical}.


\chapter{Spells}\label{Spells}

An \M{} or \F{} appearing at the end of a spell's name in the spell lists denotes a spell with a material or focus component, respectively, that is not normally included in a spell component pouch.

\parhead{Order of Presentation} In the spell lists and the spell descriptions that follow them, the spells are presented in alphabetical order by name except for those belonging to certain spell chains.

\parhead{Targets, Subjects, Creatures, and Characters} All of these words refer to whatever is affected by the spell. ``Targets'' and ``Subjects'' are used for spells that target individual creatures or objects. Spells with instantaneous duration typically refer to ``targets'', while spells that have a duration typically refer to ``subjects''. ``Creatures'' and ``characters'' are used interchangeably, and are typically used with spells that do not affect specific targets.


\small
\subsection{Arcane Magic}\label{Arcane Magic}
\subsubsection{Arcane Spells}\label{Arcane Spells}
\begin{spelllist}
\spellhead{Astromancy} Transport creatures and objects instantly through space.
\spellhead{Barrier} Shield allies from hostile forces.
\spellhead{Chronomancy} Manipulate the passage of time to inhibit foes and aid allies.
\spellhead{Compel} Bend creatures to your will by controlling their actions.
\spellhead{Corruption} Weaken the life force of foes, reducing their combat prowess.
\spellhead{Cryomancy} Drain heat to injure and freeze foes.
\spellhead{Delusion} Instill false emotions to influence creatures.
\spellhead{Electromancy} Create electricity to injure and stun foes.
\spellhead{Fabrication} Create objects to damage and impair foes.
\spellhead{Glamer} Change how creatures and objects are perceived.
\spellhead{Photomancy} Create bright light to blind foes and illuminate your surroundings.
\spellhead{Polymorph} Change the physical forms of objects and creatures.
\spellhead{Pyromancy} Create fire to incinerate foes.
\spellhead{Revelation} Share visions of the present and future, granting insight or combat prowess.
\spellhead{Scry} See and hear at great distances.
\spellhead{Summon} Summon creatures to fight with you.
\spellhead{Telekinesis} Manipulate creatures and objects at a distance.
\spellhead{Thaumaturgy} Suppress and manipulate magical effects.
\spellhead{Weaponcraft} Create and manipulate weapons to attack foes.
\end{spelllist}

\subsubsection{Arcane Rituals}\label{Arcane Rituals}
\begin{spelllist}
\spellhead[1]{Create Water} TODO
\spellhead[1]{Endure Elements} TODO
\spellhead[1]{Fortify} TODO
\spellhead[1]{Light} TODO
\spellhead[1]{Magic Mouth} TODO
\spellhead[1]{Purify Sustenance} TODO
\spellhead[1]{Read Magic} TODO
\spellhead[2]{Animate Dead} TODO
\spellhead[2]{Create Object} TODO
\spellhead[2]{Create Sustenance} TODO
\spellhead[2]{Gentle Repose} TODO
\spellhead[2]{Mount} TODO
\spellhead[2]{Mystic Lock} TODO
\spellhead[2]{Purge Curse} TODO
\spellhead[2]{Scryward} TODO
\spellhead[2]{Seek Legacy} TODO
\spellhead[2]{Water Breathing} TODO
\spellhead[3]{Explosive Runes} TODO
\spellhead[3]{Plane Shift} TODO
\spellhead[3]{Retrieve Legacy} TODO
\spellhead[3]{Sending} TODO
\spellhead[3]{Telepathic Bond} TODO
\spellhead[4]{Discern Location} TODO
\spellhead[4]{Overland Teleportation} TODO
\spellhead[4]{Private Sanctum} TODO
\spellhead[4]{Scry Creature} TODO
\spellhead[5]{Soul Bind} TODO
\spellhead[7]{Gate} TODO
\end{spelllist}



\small
\subsection{Divine Magic}\label{Divine Magic}
\subsubsection{Divine Spells}\label{Divine Spells}
\begin{spelllist}
\spellhead{Bless} Grant divine blessings to aid allies and improve combat prowess.
\spellhead{Compel} Bend creatures to your will by controlling their actions.
\spellhead{Corruption} Weaken the life force of foes, reducing their combat prowess.
\spellhead{Delusion} Instill false emotions to influence creatures.
\spellhead{Divine Judgment} Smite foes with divine power.
\spellhead{Photomancy} Create bright light to blind foes and illuminate your surroundings.
\spellhead{Revelation} Share visions of the present and future, granting insight or combat prowess.
\spellhead{Scry} See and hear at great distances.
\spellhead{Summon} Summon creatures to fight with you.
\spellhead{Thaumaturgy} Suppress and manipulate magical effects.
\spellhead{Vital Surge} Alter life energy to cure or inflict wounds.
\spellhead{Weaponcraft} Create and manipulate weapons to attack foes.
\end{spelllist}

\subsubsection{Divine Rituals}\label{Divine Rituals}
\begin{spelllist}
\spellhead[1]{Bless Water} TODO
\spellhead[1]{Create Water} TODO
\spellhead[1]{Curse Water} TODO
\spellhead[1]{Endure Elements} TODO
\spellhead[1]{Fortify} TODO
\spellhead[1]{Light} TODO
\spellhead[1]{Purify Sustenance} TODO
\spellhead[1]{Read Magic} TODO
\spellhead[2]{Animate Dead} TODO
\spellhead[2]{Create Object} TODO
\spellhead[2]{Create Sustenance} TODO
\spellhead[2]{Gentle Repose} TODO
\spellhead[2]{Mystic Lock} TODO
\spellhead[2]{Purge Curse} TODO
\spellhead[2]{Scryward} TODO
\spellhead[2]{Seek Legacy} TODO
\spellhead[2]{Water Breathing} TODO
\spellhead[3]{Plane Shift} TODO
\spellhead[3]{Restoration} TODO
\spellhead[3]{Retrieve Legacy} TODO
\spellhead[3]{Sending} TODO
\spellhead[4]{Blessed Transit} TODO
\spellhead[4]{Discern Location} TODO
\spellhead[4]{Scry Creature} TODO
\spellhead[5]{Soul Bind} TODO
\spellhead[7]{Gate} TODO
\end{spelllist}



\small
\subsection{Nature Magic}\label{Nature Magic}
\subsubsection{Nature Spells}\label{Nature Spells}
\begin{spelllist}
\spellhead{Aeromancy} Command air to protect allies and blast foes.
\spellhead{Aquamancy} Command water to crush and drown foes.
\spellhead{Corruption} Weaken the life force of foes, reducing their combat prowess.
\spellhead{Cryomancy} Drain heat to injure and freeze foes.
\spellhead{Electromancy} Create electricity to injure and stun foes.
\spellhead{Photomancy} Create bright light to blind foes and illuminate your surroundings.
\spellhead{Polymorph} Change the physical forms of objects and creatures.
\spellhead{Pyromancy} Create fire to incinerate foes.
\spellhead{Revelation} Share visions of the present and future, granting insight or combat prowess.
\spellhead{Scry} See and hear at great distances.
\spellhead{Summon} Summon creatures to fight with you.
\spellhead{Thaumaturgy} Suppress and manipulate magical effects.
\spellhead{Vital Surge} Alter life energy to cure or inflict wounds.
\end{spelllist}

\subsubsection{Nature Rituals}\label{Nature Rituals}
\begin{spelllist}
\spellhead[1]{Create Water} TODO
\spellhead[1]{Endure Elements} TODO
\spellhead[1]{Fortify} TODO
\spellhead[1]{Light} TODO
\spellhead[1]{Purify Sustenance} TODO
\spellhead[1]{Read Magic} TODO
\spellhead[2]{Create Object} TODO
\spellhead[2]{Create Sustenance} TODO
\spellhead[2]{Fertility} TODO
\spellhead[2]{Gentle Repose} TODO
\spellhead[2]{Infertility} TODO
\spellhead[2]{Mystic Lock} TODO
\spellhead[2]{Purge Curse} TODO
\spellhead[2]{Scryward} TODO
\spellhead[2]{Seek Legacy} TODO
\spellhead[2]{Water Breathing} TODO
\spellhead[3]{Ironwood} TODO
\spellhead[3]{Plane Shift} TODO
\spellhead[3]{Restoration} TODO
\spellhead[3]{Resurrection} TODO
\spellhead[3]{Retrieve Legacy} TODO
\spellhead[3]{Sending} TODO
\spellhead[4]{Discern Location} TODO
\spellhead[4]{Lifeweb Transit} TODO
\spellhead[4]{Reincarnation} TODO
\spellhead[4]{Scry Creature} TODO
\spellhead[5]{Awaken} TODO
\spellhead[7]{Gate} TODO
\end{spelllist}



\small
\subsection{Pact Magic}\label{Pact Magic}
\subsubsection{Pact Spells}\label{Pact Spells}
\begin{spelllist}
\spellhead{Astromancy} Transport creatures and objects instantly through space.
\spellhead{Chronomancy} Manipulate the passage of time to inhibit foes and aid allies.
\spellhead{Compel} Bend creatures to your will by controlling their actions.
\spellhead{Corruption} Weaken the life force of foes, reducing their combat prowess.
\spellhead{Cryomancy} Drain heat to injure and freeze foes.
\spellhead{Delusion} Instill false emotions to influence creatures.
\spellhead{Electromancy} Create electricity to injure and stun foes.
\spellhead{Fabrication} Create objects to damage and impair foes.
\spellhead{Photomancy} Create bright light to blind foes and illuminate your surroundings.
\spellhead{Polymorph} Change the physical forms of objects and creatures.
\spellhead{Pyromancy} Create fire to incinerate foes.
\spellhead{Telekinesis} Manipulate creatures and objects at a distance.
\spellhead{Weaponcraft} Create and manipulate weapons to attack foes.
\end{spelllist}


\section{Spell Descriptions}
\begin{spellsection}{Agony}
\begin{spellheader}
\spelldesc{You inflict debilitating pain on your foe}
\end{spellheader}
\begin{spellcontent}
\begin{spelltargetinginfo}
\spelltwocol{\spelltgt{One creature}}{\spellrng{\rngclose}}
\end{spelltargetinginfo}
\begin{spelleffects}
\begin{spellattack}{Spellpower vs. Mental}
\spellsuccess Physical damage dealt to the target is increased by \plus2d.
\spellcritical Physical damage dealt to the target is increased by \plus4d.
\end{spellattack}
\spelldur Condition
\spelltags{\glossterm{Delusion}, \glossterm{Mind}}
\end{spelleffects}
\end{spellcontent}
\begin{spellfooter}
\spellinfo{Enchantment}{Arcane, Divine}
\parhead*{Augments} Extended, Mass, Quickened, Silent, Stilled
\spellnotes This damage increase applies before other effects that modify the total damage dealt, such as \glossterm{damage reduction}.
\end{spellfooter}
\begin{spellsubcontent}
\begin{spellcantrip}
The spell's duration becomes Sustain (swift).
Its effect is still a \glossterm{condition}, and can be removed by abilites that remove conditions.
\end{spellcantrip}
\end{spellsubcontent}
\end{spellsection}
\subsubsection{Subspells}
\augment{3}{Complete}
The damage increase applies to all damage, not just physical damage.
\begin{spellsection}{Antimagic}
\begin{spellcontent}
\begin{spelltargetinginfo}
\spelltwocol{\spelltgt{One creature, object, or active magical effect}}{\spellrng{\rngmed}}
\end{spelltargetinginfo}
\begin{spelleffects}
\begin{spellattack}{Spellpower vs. Special}
\spellspecial
The attack result is applied to every \glossterm{magical} effect on the target.
The DR for each effect is equal to 10 + the \glossterm{power} of that effect.
\spellsuccess
Success against a magical effect causes that effect to be \glossterm{suppressed}.
\end{spellattack}
\spelldur Sustain (swift)
\spelltags{\glossterm{Thaumaturgy}}
\end{spelleffects}
\end{spellcontent}
\begin{spellfooter}
\spellinfo{Abjuration}{Arcane, Divine, Magic, Nature}
\parhead*{Augments} Extended, Mass, Quickened, Silent, Stilled
\end{spellfooter}
\begin{spellsubcontent}
\begin{spellcantrip}
The spell's duration becomes Sustain (standard).
\end{spellcantrip}
\end{spellsubcontent}
\end{spellsection}
\subsubsection{Subspells}
\augment{2}{Alter Magic Aura}
Replace the spell's targets and effects with the following:
\begin{spellcontent}
\begin{augmenttargetinginfo}
\spelltwocol{\spelltgt{One magical object (Large or smaller)}}{\spellrng{\rngmed}}
\end{augmenttargetinginfo}
\begin{augmenteffects}
\begin{spellattack}{Spellpower vs. Mental}
\spellsuccess
One of the target's magic auras is altered (see \pcref{Spellcraft}).
You can change the school and descriptors of the aura.
In addition, you can decrease the spellpower of the aura by up to half your spellpower, or increase the spellpower of the aura up to a maximum of your spellpower.
\end{spellattack}
\spelldur Attunement
\spelltags{\glossterm{Thaumaturgy}}
\end{augmenteffects}
\end{spellcontent}
\augment{2}{Dimensional Anchor}
Replace the spell's effects with the following:
\begin{spellcontent}
\begin{augmenteffects}
\begin{spellattack}{Spellpower vs. Mental}
\spellsuccess
The target cannot travel extradimensionally.
This prevents all \glossterm{Manifestation}, \glossterm{Planar}, and \glossterm{Translocation} effects.
\end{spellattack}
\spelldur Condition
\spelltags{\glossterm{T}, \glossterm{a}, \glossterm{a}, \glossterm{g}, \glossterm{h}, \glossterm{m}, \glossterm{r}, \glossterm{t}, \glossterm{u}, \glossterm{u}, \glossterm{y}}
\end{augmenteffects}
\end{spellcontent}
\augment{2}{Suppress Item}
Replace the spell's targets and effects with the following:
\begin{spellcontent}
\begin{augmenttargetinginfo}
\spelltwocol{\spelltgt{One object}}{\spellrng{\rngmed}}
\end{augmenttargetinginfo}
\begin{augmenteffects}
\begin{spellattack}{Spellpower vs. Special}
\spellspecial
The DR is equal to 10 + the target's spellpower.
\spellsuccess
The target object is \glossterm{suppressed}.
\end{spellattack}
\spelldur Sustain (swift)
\spelltags{\glossterm{Thaumaturgy}}
\end{augmenteffects}
\end{spellcontent}
\augment{3}{Banishing}
Replace the spell's effects with the following:
\begin{spellcontent}
\begin{augmenteffects}
\begin{spellattack}{Spellpower vs. Special}
\spellspecial
If the target is an effect of an ongoing \glossterm{magical} ability, such as a summoned monster, the DR is equal to 10 + the target's spellpower.
Otherwise, this ability has no effect.
\spellsuccess
The target is treated as if the spell that created it was \glossterm{dispelled}.
This usually causes the target to disappear.
\end{spellattack}
\spelltags{\glossterm{Thaumaturgy}}
\end{augmenteffects}
\end{spellcontent}
\augment{5}{Dimensional Lock}
Replace the spell's targets and effects with the following:
\begin{spellcontent}
\begin{augmenttargetinginfo}
\spelltwocol{\spellburst{\arealarge radius}}{\spellrng{\rngmed}}
\spelltgts{Everything in the area}
\end{augmenttargetinginfo}
\begin{augmenteffects}
\spelleffect
Extradimensional travel into or out of the spell's area is impossible.
This prevents all \glossterm{Manifestation}, \glossterm{Planar}, and \glossterm{Translocation} effects.
\spelldur Attunement
\spelltags{\glossterm{T}, \glossterm{a}, \glossterm{a}, \glossterm{g}, \glossterm{h}, \glossterm{m}, \glossterm{r}, \glossterm{t}, \glossterm{u}, \glossterm{u}, \glossterm{y}}
\end{augmenteffects}
\end{spellcontent}
\augment{7}{Antimagic Field}
Replace the spell's targets and effects with the following:
\begin{spellcontent}
\begin{augmenttargetinginfo}
\spellspecial This emanation always includes you in its area
\spellemanation{\areasmall radius centered on you}
\end{augmenttargetinginfo}
\begin{augmenteffects}
\spelleffect
All magical abilities and objects are \glossterm{suppressed} in the area.
In addition, magical abilities and objects cannot be activated within the area.
\par Creatures within the area cannot concentrate on or dismiss spells. However, you can concentrate on and dismiss your own \spell{antimagic field}.
\spelldur Sustain (swift)
\spelltags{\glossterm{Thaumaturgy}}
\end{augmenteffects}
\end{spellcontent}
\begin{spellsection}{Barkskin}
\begin{spellheader}
\spelldesc{You toughen a creature's skin, giving it the appearance of tree bark.}
\end{spellheader}
\begin{spellcontent}
\begin{spelltargetinginfo}
\spelltwocol{\spelltgt{One living creature}}{\spellrng{\rngclose}}
\end{spelltargetinginfo}
\begin{spelleffects}
\spelleffect
The target gains \glossterm{damage reduction} against physical damage equal to your spellpower.
In addition, it is \glossterm{vulnerable} to fire damage.
\spelldur Attunement
\spelltags{\glossterm{Enhancement}}
\end{spelleffects}
\end{spellcontent}
\begin{spellfooter}
\spellinfo{Transmutation}{Nature}
\parhead*{Augments} Extended, Quickened, Silent, Stilled
\end{spellfooter}
\begin{spellsubcontent}
\begin{spellcantrip}
The spell's duration becomes Sustain (swift).
\end{spellcantrip}
\end{spellsubcontent}
\end{spellsection}
\subsubsection{Subspells}
\augment{3}{Stoneskin}
The spell does not make the target vulnerable to fire damage.
Instead, it makes the target \glossterm{vulnerable} to damage from adamantine weapons.
\augment{5}{Empowered}
The damage reduction granted by this spell increases by an amount equal to your spellpower.
\begin{spellsection}{Barrier}
\begin{spellcontent}
\begin{spelltargetinginfo}
\spellzone{\areamed radius centered on you}
\end{spelltargetinginfo}
\begin{spelleffects}
\spelleffect
Whenever a creature makes physical contact with the spell's area for the first time, you make a Spellpower vs. Mental attack against it.
Success means the creature is unable to enter the spell's area with any part of its body.
The rest of its movement in the current phase is cancelled.
Failure means the creature can enter the area unimpeded.
Creatures in the area at the time that the spell is cast are unaffected by the spell.
\spelldur Sustain (swift)
\end{spelleffects}
\end{spellcontent}
\begin{spellfooter}
\spellinfo{Abjuration}{Divine, Nature}
\parhead*{Augments} Quickened, Silent, Stilled, Widened
\end{spellfooter}
\begin{spellsubcontent}
\begin{spellcantrip}
The spell's duration becomes Sustain (standard)
\end{spellcantrip}
\end{spellsubcontent}
\end{spellsection}
\subsubsection{Subspells}
\augment{4}{Selective}
Whenever a creature attempts to pass through the barrier for the first time, you can allow it to pass through unimpeded.
You must be aware of a creature attempting to pass through the barrier to allow it through.
\augment{7}{Antilife Shell}
The spell only affects living creatures.
However, it affects them automatically, without requiring an attack.
\par
This is a \glossterm{Life} effect from the \glossterm{Vivimancy} school.
\begin{spellsection}{Bless}
\begin{spellheader}
\spelldesc{You invoke a divine blessing to aid your ally.}
\end{spellheader}
\begin{spellcontent}
\begin{spelltargetinginfo}
\spelltwocol{\spelltgt{One creature}}{\spellrng{\rngclose}}
\end{spelltargetinginfo}
\begin{spelleffects}
\spelleffect The target gains a \plus2d bonus to damage with all attacks.
\spelldur Attunement
\end{spelleffects}
\end{spellcontent}
\begin{spellfooter}
\spellinfo{Channeling}{Divine}
\parhead*{Augments} Extended, Quickened, Silent, Stilled
\end{spellfooter}
\begin{spellsubcontent}
\begin{spellcantrip}
The spell's duration becomes Sustain (swift).
\end{spellcantrip}
\end{spellsubcontent}
\end{spellsection}
\subsubsection{Subspells}
\augment{6}{Protection}
The target gains \glossterm{damage reduction} against all damage equal to your spellpower.
\begin{spellsection}{Boon of Mastery}
\begin{spellheader}
\spelldesc{You grant your ally great mastery over a particular domain.}
\end{spellheader}
\begin{spellcontent}
\begin{spelltargetinginfo}
\spelltwocol{\spelltgt{One willing creature}}{\spellrng{\rngclose}}
\end{spelltargetinginfo}
\begin{spelleffects}
\spellspecial
When you cast this spell, choose a skill.
You must have mastered the chosen skill.
\spelleffect
The target gains a \plus5 bonus to the chosen skill.
\spelldur Attunement
\spelltags{\glossterm{Enhancement}}
\end{spelleffects}
\end{spellcontent}
\begin{spellfooter}
\spellinfo{Transmutation}{Arcane, Divine, Nature}
\parhead*{Augments} Extended, Quickened, Silent, Stilled
\end{spellfooter}
\begin{spellsubcontent}
\begin{spellcantrip}
The spell's duration becomes Sustain (swift).
\end{spellcantrip}
\end{spellsubcontent}
\end{spellsection}
\subsubsection{Subspells}
\augment{4}{Myriad}
You may choose an additional skill that you have mastered as you cast the spell.
The target gains the same bonus to all chosen skills.
\begin{spellsection}{Charm Person}
\begin{spellheader}
\spelldesc{You manipulate a person's mind so they think of you as a trusted friend and ally.}
\end{spellheader}
\begin{spellcontent}
\begin{spelltargetinginfo}
\spelltwocol{\spelltgt{One humanoid creature}}{\spellrng{\rngclose}}
\end{spelltargetinginfo}
\begin{spelleffects}
\begin{spellattack}{Spellpower vs. Mental}
\spellspecial If the target thinks that you or your allies are threatening it, you take a \minus5 penalty to accuracy on the attack.
\spellsuccess
The target is \charmed by you.
Any act by you or your apparent allies that threatens or damages the \spell{charmed} person breaks the effect.
\spellcritical As above, but the effect's duration becomes permanent.
\end{spellattack}
\spelldur Sustain (swift)
\spelltags{\glossterm{Delusion}, \glossterm{Mind}, \glossterm{Subtle}}
\end{spelleffects}
\end{spellcontent}
\begin{spellfooter}
\spellinfo{Enchantment}{Arcane}
\parhead*{Augments} Extended, Mass, Quickened, Silent, Stilled
\end{spellfooter}
\begin{spellsubcontent}
\begin{spellcantrip}
The spell has no additional effects on a critical hit.
In addition, its duration becomes Sustain (standard).
\end{spellcantrip}
\end{spellsubcontent}
\end{spellsection}
\subsubsection{Subspells}
\augment{2}{Silent}
The spell does not require verbal components to cast.
\augment{3}{Monstrous}
The spell can target creatures of any creature type.
\augment{4}{Attuned}
The spell's duration becomes Attunement.
A critical sucess still makes the effect permanent.
\augment{5}{Amnesia}
When the spell ends, the target forgets all events that transpired during the spell's duration.
It becomes aware of its surroundings as if waking up from a daydream.
It is not directly aware of any magical influence on its mind, though unusually paranoid or perceptive creatures may deduce that their minds were affected.
\augment{5}{Dominating}
Replace the spell's effects with the following:
\begin{spellcontent}
\begin{augmenteffects}
\begin{spellattack}{Spellpower vs. Mental}
\spellsuccess The target is \confused for 2 rounds.
\spellcritical
The target is \dominated for 2 rounds.
If the target was already dominated by you, this effect lasts for 24 hours instead.
\end{spellattack}
\spelltags{\glossterm{Compulsion, Mind}}
\end{augmenteffects}
\end{spellcontent}
\begin{spellsection}{Cone of Cold}
\begin{spellheader}
\spelldesc{You drain the heat from an area, creating a field of extreme cold.}
\end{spellheader}
\begin{spellcontent}
\begin{spelltargetinginfo}
\spellburst{\areamed cone}
\spelltgts{Everything in the area}
\end{spelltargetinginfo}
\begin{spelleffects}
\begin{spellattack}{Spellpower vs. Fortitude}
\spellsuccess
\spelldamage{cold}[1d4].
In addition, the target is \fatigued as a condition.
\spellcritical As above, but double damage.
\end{spellattack}
\spelltags{\glossterm{Cold}}
\end{spelleffects}
\end{spellcontent}
\begin{spellfooter}
\spellinfo{Evocation}{Arcane, Nature}
\parhead*{Augments} Intensified, Quickened, Silent, Stilled, Widened
\end{spellfooter}
\begin{spellsubcontent}
\begin{spellcantrip}
The spell deals no damage.
\end{spellcantrip}
\end{spellsubcontent}
\end{spellsection}
\begin{spellsection}{Control Air}
\begin{spellheader}
\spelldesc{You shield your ally with a barrier of wind, protecting them from harm.}
\end{spellheader}
\begin{spellcontent}
\begin{spelltargetinginfo}
\spelltwocol{\spelltgt{One willing creature (Medium or smaller)}}{\spellrng{\rngclose}}
\end{spelltargetinginfo}
\begin{spelleffects}
\spelleffect
The target gains a \plus2 bonus to \glossterm{physical defenses}.
This bonus is increased to \plus5 against ranged \glossterm{strikes} from weapons or projectiles that are Small or smaller.
Any effect which increases the size of creature this spell can affect also increases the size of ranged weapon it defends against by the same amount.
\spelldur Attunement
\spelltags{\glossterm{Air}, \glossterm{Imbuement}}
\end{spelleffects}
\end{spellcontent}
\begin{spellfooter}
\spellinfo{Transmutation}{Air, Nature}
\parhead*{Augments} Extended, Quickened, Silent, Stilled
\end{spellfooter}
\begin{spellsubcontent}
\begin{spellcantrip}
The spell's duration becomes Sustain (swift).
\end{spellcantrip}
\end{spellsubcontent}
\end{spellsection}
\subsubsection{Subspells}
\augment{2}{Gentle Descent}
The target gains a 30 foot glide speed.
A creature with a glide speed can glide through the air at the indicated speed (see \pcref{Gliding}).
\augment{2}{Windstrike}
Replace the spell's targets and effects with the following:
\begin{spellcontent}
\begin{augmenttargetinginfo}
\spelltwocol{\spelltgt{One creature or object}}{\spellrng{\rngmed}}
\end{augmenttargetinginfo}
\begin{augmenteffects}
\begin{spellattack}{Spellpower vs. Fortitude}
\spellsuccess \spelldamage{bludgeoning}.
\spellcritical As above, but double damage.
\end{spellattack}
\spelltags{\glossterm{Air}}
\end{augmenteffects}
\end{spellcontent}
\augment{3}{Accelerated}
The glide speed granted by this spell increases to 60 feet.
\augment{3}{Gust of Wind}
Replace the spell's targets and effects with the following:
\begin{spellcontent}
\begin{augmenttargetinginfo}
\spellburst{\arealarge line, 10 ft. wide}
\spelltgts{Everything in the area}
\end{augmenttargetinginfo}
\begin{augmenteffects}
\begin{spellattack}{Spellpower vs. Fortitude}
\spellsuccess \spelldamage{bludgeoning}[1d4].
\spellcritical As above, but double damage.
\end{spellattack}
\spelltags{\glossterm{Air}}
\end{augmenteffects}
\end{spellcontent}
\augment{4}{Air Walk}
The target can walk on air as if it were solid ground.
The magic only affects the target's legs and feet.
By choosing when to treat the air as solid, it can traverse the air with ease.
\augment{4}{Wind Screen}
The miss chance for ranged strikes against the target increases to 50\%.
\augment{5}{Stormlord}
Whenever a creature within \rngclose range of the target attacks it, wind strikes the attacking creature.
The wind deals 1d4 bludgeoning damage \add 1d per two spellpower.
Any individual creature can only be dealt damage in this way once per round.
\par Any effect which increases this spell's range increases the range of this effect by the same amount.
\par
This is a \glossterm{Shielding} effect from the \glossterm{Evocation} school.
\augment{7}{Control Weather}
Replace the spell's targets and effects with the following:
\begin{spellcontent}
\begin{augmenttargetinginfo}
\spellzone{2 mile radius cylinder from your location}
\end{augmenttargetinginfo}
\begin{augmenteffects}
\spelleffect
When you cast this spell, you choose a new weather pattern.
You can only choose weather which would be possible in the climate and season of the area you are in.
For example, you can normally create a thunderstorm, but not if you are in a desert.
The weather begins to take effect in the area when you complete the spell.
After five minutes, your chosen weather pattern fully takes effect.
You can control the general tendencies of the weather, such as the direction and intensity of the wind.
You cannot control specific applications of the weather -- where lightning strikes, for example, or the exact path of a tornado.
Contradictory weather conditions are not possible simultaneously.
After the spell's duration ends, the weather continues on its natural course, which may cause your chosen weather pattern to end.
% TODO: This should be redundant with generic spell mechanics
If another ability would magically manipulate the weather in the same area, the most recently used ability takes precedence.
\spelldur Attunement
\spelltags{\glossterm{Air}}
\end{augmenteffects}
\end{spellcontent}
\begin{spellsection}{Corruption}
\begin{spellheader}
\spelldesc{You corrupt your foe's life force, weakening them.}
\end{spellheader}
\begin{spellcontent}
\begin{spelltargetinginfo}
\spelltwocol{\spelltgt{One living creature}}{\spellrng{\rngclose}}
\end{spelltargetinginfo}
\begin{spelleffects}
\begin{spellattack}{Spellpower vs. Fortitude}
\spellsuccess
The target takes a \minus2 penalty to \glossterm{accuracy}, \glossterm{checks}, and \glossterm{defenses}.
\spellcritical
As above, but the penalty is increased by 2.
\end{spellattack}
\spelldur Condition
\spelltags{\glossterm{Life}}
\end{spelleffects}
\end{spellcontent}
\begin{spellfooter}
\spellinfo{Vivimancy}{Arcane, Divine, Nature}
\parhead*{Augments} Extended, Mass, Quickened, Silent, Stilled
\end{spellfooter}
\begin{spellsubcontent}
\begin{spellcantrip}
The spell's duration becomes Sustain (swift).
Its effect is still a condition, and can be removed by abilites that remove conditions.
\end{spellcantrip}
\end{spellsubcontent}
\end{spellsection}
\subsubsection{Subspells}
\augment{3}{Eyebite}
If the spell's attack succeeds, the target is also \partiallyblinded. If it critically hits, the target is \blinded instead of partially blinded.
\augment{3}{Finger of Death}
If the spell's attack critically hits, the target immediately dies.
\par
This is a \glossterm{Death} effect.
\augment{4}{Empowered}
The penalty increases by 1.
\augment{5}{Corruption of Blood and Bone}
If the spell's attack succeeds, at the end of each round, the target takes life damage equal to your spellpower.
The target's maximum hit points are reduced by the amount of damage it takes in this way.
When the spell ends, the target's maximum hit points are restored.
\augment{6}{Corrupting Curse}
The spell's attack is made against Mental defense instead of Fortitude defense.
In addition, if it critically hits, the spell's effect becomes a permanent curse.
It is no longer a condition, and cannot be removed by abilities that remove conditions.
This is a \glossterm{Curse} effect.
\begin{spellsection}{Create Acid}
\begin{spellheader}
\spelldesc{You create a magical orb of acid in your hand that speeds to its target.}
\end{spellheader}
\begin{spellcontent}
\begin{spelltargetinginfo}
\spelltwocol{\spelltgt{One creature or object}}{\spellrng{\rngmed}}
\end{spelltargetinginfo}
\begin{spelleffects}
\begin{spellattack}{Spellpower vs. Reflex}
\spellsuccess \spelldamage{acid}.
\spellcritical As above, but double damage.
\end{spellattack}
\spelltags{\glossterm{Acid}, \glossterm{Manifestation}}
\end{spelleffects}
\end{spellcontent}
\begin{spellfooter}
\spellinfo{Conjuration}{Arcane}
\parhead*{Augments} Extended, Intensified, Mass, Quickened, Silent, Stilled
\end{spellfooter}
\begin{spellsubcontent}
\begin{spellcantrip}
The spell's range becomes \rngclose, and it deals \minus1d damage.
\end{spellcantrip}
\end{spellsubcontent}
\end{spellsection}
\subsubsection{Subspells}
\augment{3}{Corrosive}
The spell deals double damage to objects.
\augment{4}{Lingering}
The acid deals half damage on initial impact.
However, it deals damage to the target again at the end of each round for 2 rounds, including the initial round.
\begin{spellsection}{Cure Wounds}
\begin{spellcontent}
\begin{spelltargetinginfo}
\spelltwocol{\spelltgt{One creature}}{\spellrng{\rngmed}}
\end{spelltargetinginfo}
\begin{spelleffects}
\begin{spellattack}{Spellpower vs. Fortitude}
\spellsuccess The target is healed for \spelldamage{}.
\end{spellattack}
\spelltags{\glossterm{Life}}
\end{spelleffects}
\end{spellcontent}
\begin{spellfooter}
\spellinfo{Vivimancy}{Divine, Life, Nature}
\parhead*{Augments} Extended, Intensified, Mass, Quickened, Silent, Stilled
\end{spellfooter}
\begin{spellsubcontent}
\begin{spellcantrip}
Instead of healing, the spell grants \glossterm{temporary hit points} equal to twice your spellpower.
The duration of the temporary hit points is Sustain (swift).
\end{spellcantrip}
\end{spellsubcontent}
\end{spellsection}
\subsubsection{Subspells}
\augment{2}{Moderate Wounds}
For every 5 points of healing, this spell can instead cure 1 vital damage.
\augment{2}{Restore Senses}
Replace the spell's effects with the following:
\begin{spellcontent}
\begin{augmenteffects}
\spelleffect
One of the target's physical senses, such as sight or hearing, is restored to full capacity.
This can heal both magical and mundane conditions, but it cannot completely replace missing body parts required for a sense to function (such as missing eyes).
\spelltags{\glossterm{Flesh}}
\end{augmenteffects}
\end{spellcontent}
\augment{2}{Undead Bane}
If the target is undead, the spell gains a \plus2 bonus to accuracy and deals double damage on a critical hit.
\augment{3}{Remove Disease}
Replace the spell's effects with the following:
\begin{spellcontent}
\begin{augmenteffects}
\spelleffect
All diseases affecting the target are removed.
\spelltags{\glossterm{Flesh}}
\end{augmenteffects}
\end{spellcontent}
\augment{3}{Serious Wounds}
For every 2 points of healing, this spell can instead cure 1 vital damage.
\augment{4}{Critical Wounds}
For every point of healing, this spell can instead cure 1 vital damage.
\begin{spellsection}{Distort Image}
\begin{spellcontent}
\begin{spelltargetinginfo}
\spelltwocol{\spelltgt{One willing creature}}{\spellrng{\rngmed}}
\end{spelltargetinginfo}
\begin{spelleffects}
\spelleffect
The target's physical outline is distorted so it appears blurred, shifting, and wavering.
Targeted physical attacks against the target have a 20\% miss chance.
Spells and other non-physical attacks suffer no miss chance.
\spelldur Attunement
\spelltags{\glossterm{Glamer}, \glossterm{Visual}}
\end{spelleffects}
\end{spellcontent}
\begin{spellfooter}
\spellinfo{Illusion}{Arcane}
\parhead*{Augments} Extended, Quickened, Silent, Stilled
\end{spellfooter}
\begin{spellsubcontent}
\begin{spellcantrip}
The spell's duration becomes Sustain (swift).
\end{spellcantrip}
\end{spellsubcontent}
\end{spellsection}
\subsubsection{Subspells}
\augment{2}{Distort Light}
Replace the spell's targets and effects with the following:
\begin{spellcontent}
\begin{augmenttargetinginfo}
\spelltwocol{\spellemanation{\areamed radius from the target}}{\spellrng{\rngclose}}
\spelltgt{One object (Small or smaller)}
\end{augmenttargetinginfo}
\begin{augmenteffects}
\spelleffect
Light within or passing through the area is dimmed to be no brighter than shadowy illumination.
Any effect or object which blocks light also blocks this spell's emanation.
\spelldur Attunement (multiple)
\spelltags{\glossterm{Glamer}, \glossterm{Light}}
\end{augmenteffects}
\end{spellcontent}
\augment{3}{Disguise Image}
Replace the spell's effects with the following:
\begin{spellcontent}
\begin{augmenteffects}
\spelleffect
You make a Disguise check to alter the target's appearance (see \pcref{Disguise Creature}).
You gain a \plus5 bonus on the check, and you can freely alter the appearance of the target's clothes and equipment, regardless of their original form.
However, this effect is unable to alter the sound, smell, texture, or temperature of the target or its clothes and equipment.
\spelldur Attunement
\spelltags{\glossterm{Glamer}, \glossterm{Visual}}
\end{augmenteffects}
\end{spellcontent}
\augment{3}{Mirror Image}
Replace the spell's effects with the following:
\begin{spellcontent}
\begin{augmenteffects}
\spelleffect
Four illusory duplicates appear around the target that mirror its every move.
The duplicates shift chaotically in its space, making it difficult to identify the real creature.
All targeted attacks against the target have a 50% miss chance.
Whenever an attack misses in this way, it affects an image, destroying it.
\spelldur Sustain (swift)
\spelltags{\glossterm{Figment}, \glossterm{Visual}}
\end{augmenteffects}
\end{spellcontent}
\augment{4}{Shadow Mantle}
The spell's deceptive nature extends beyond merely altering light to affect the nature of reality itself.
The spell's miss chance changes to a failure chance, and applies to non-physical attacks as well as physical attacks.
In addition, it loses the \glossterm{Visual} tag, allowing it to affect creatures who do not rely on sight to affect the target.
\augment{5}{Displacement}
The target's image is futher distorted, and appears to be two to three feet from its real location.
The spell's miss chance increases to 50\%.
\begin{spellsection}{Elemental Blade}
\begin{spellheader}
\spelldesc{You transform the active part of a weapon into air, increasing its reach.}
\end{spellheader}
\begin{spellcontent}
\begin{spelltargetinginfo}
\spelltwocol{\spelltgt{One unattended weapon}}{\spellrng{\rngclose}}
\end{spelltargetinginfo}
\begin{spelleffects}
\spelleffect
The target weapon gains an additional five feet of reach, extending the wielder's threatened area.
This has no effect on ranged attacks with the weapon.
\spelldur Attunement
\spelltags{\glossterm{Air}, \glossterm{Shaping}}
\end{spelleffects}
\end{spellcontent}
\begin{spellfooter}
\spellinfo{Transmutation}{Arcane, Nature, War, Water}
\parhead*{Augments} Extended, Quickened, Silent, Stilled
\end{spellfooter}
\begin{spellsubcontent}
\begin{spellcantrip}
The spell's duration becomes Sustain (swift).
\end{spellcantrip}
\end{spellsubcontent}
\end{spellsection}
\subsubsection{Subspells}
\augment{2}{Aqueous Blade}
Replace the spell's effects with the following:
\begin{spellcontent}
\begin{augmenteffects}
\spelleffect
\glossterm{Strikes} with the affected weapon are made against Reflex defense instead of Armor defense.
However, damage with the weapon is halved, including any bonuses to damage.
\spelldur Attunement
\spelltags{\glossterm{Shaping}, \glossterm{Water}}
\end{augmenteffects}
\end{spellcontent}
\augment{4}{Zephyr Blade}
The weapon's reach is increased by ten feet instead of five feet.
\augment{7}{Greater Aqueous Blade}
Replace the spell's effects with the following:
\begin{spellcontent}
\begin{augmenteffects}
\spelleffect
\glossterm{Strikes} with the affected weapon are made against Reflex defense instead of Armor defense.
\spelldur Attunement
\spelltags{\glossterm{Shaping}, \glossterm{Water}}
\end{augmenteffects}
\end{spellcontent}
\begin{spellsection}{Fear}
\begin{spellheader}
\spelldesc{You terrify your foe.}
\end{spellheader}
\begin{spellcontent}
\begin{spelltargetinginfo}
\spelltwocol{\spelltgt{One creature}}{\spellrng{\rngmed}}
\end{spelltargetinginfo}
\begin{spelleffects}
\begin{spellattack}{Spellpower vs. Mental}
\spellsuccess The target is \frightened by you.
\spellcritical The target is \panicked by you.
\spellfailure The target is \shaken by you.
\end{spellattack}
\spelldur Condition
\spelltags{\glossterm{Delusion}, \glossterm{Mind}}
\end{spelleffects}
\end{spellcontent}
\begin{spellfooter}
\spellinfo{Enchantment}{Arcane}
\parhead*{Augments} Extended, Mass, Quickened, Silent, Stilled
\end{spellfooter}
\begin{spellsubcontent}
\begin{spellcantrip}
The spell's duration becomes Sustain (swift).
\end{spellcantrip}
\end{spellsubcontent}
\end{spellsection}
\subsubsection{Subspells}
\augment{2}{Redirected}
The target is afraid of a willing ally within the spell's range instead of being afraid of you.
\begin{spellsection}{Fireball}
\begin{spellheader}
\spelldesc{You create a small burst of flame.}
\end{spellheader}
\begin{spellcontent}
\begin{spelltargetinginfo}
\spelltwocol{\spellburst{\areasmall radius}}{\spellrng{\rngclose}}
\spelltgts{Everything in the area}
\end{spelltargetinginfo}
\begin{spelleffects}
\begin{spellattack}{Spellpower vs. Reflex}
\spellsuccess \spelldamage{fire}[1d4].
\spellcritical As above, but double damage.
\end{spellattack}
\spelltags{\glossterm{Fire}}
\end{spelleffects}
\end{spellcontent}
\begin{spellfooter}
\spellinfo{Evocation}{Arcane, Fire, Nature}
\parhead*{Augments} Extended, Intensified, Quickened, Silent, Stilled, Widened
\end{spellfooter}
\begin{spellsubcontent}
\begin{spellcantrip}
The spell affects a 5 foot radius, and it deals \minus1d damage.
\end{spellcantrip}
\end{spellsubcontent}
\end{spellsection}
\subsubsection{Subspells}
\augment{2}{Burning Hands}
Replace the spell's targets with the following:
\begin{spellcontent}
\begin{augmenttargetinginfo}
\spellburst{\arealarge cone}
\spelltgts{Everything in the area}
\end{augmenttargetinginfo}
\end{spellcontent}
\augment{2}{Flame Blade}
Replace the spell's targets and effects with the following:
\begin{spellcontent}
\begin{augmenttargetinginfo}
\spelltwocol{\spelltgt{One unattended weapon}}{\spellrng{\rngclose}}
\end{augmenttargetinginfo}
\begin{augmenteffects}
\spelleffect
The target weapon deals \plus2d damage with \glossterm{strikes}.
In addition, all damage dealt with the weapon with strikes becomes fire damage in addition to its normal damage types.
\spelldur Attunement
\spelltags{\glossterm{Fire}}
\end{augmenteffects}
\end{spellcontent}
\augment{3}{Fire Trap}
Replace the spell's targets and effects with the following:
\begin{spellcontent}
\begin{augmenttargetinginfo}
\spelltwocol{\spelltgt{One openable object (Large or smaller)}}{\spellrng{\rngclose}}
\end{augmenttargetinginfo}
\begin{augmenteffects}
\spelleffect
If a creature opens the target object, it explodes.
You make an attack against everything within an \areamed radius burst centered on the target.
After the object explodes in this way, the spell ends.
\begin{spellattack}{Spellpower vs. Reflex}
\spellsuccess \spelldamage{fire}[1d4].
\spellcritical As above, but double damage.
\end{spellattack}
\spelldur Attunement
\spelltags{\glossterm{Fire}, \glossterm{Trap}}
\end{augmenteffects}
\end{spellcontent}
\begin{spellsection}{Flare}
\begin{spellcontent}
\begin{spelltargetinginfo}
\spelltwocol{\spellburst{5 foot radius}}{\spellrng{\rngmed}}
\spelltgts{All creatures in the area}
\end{spelltargetinginfo}
\begin{spelleffects}
\spelleffect
A brilliant light appears in the area until the end of the round.
It illuminates a 100 foot radius around the area with bright light.
\begin{spellattack}{Spellpower vs. Reflex}
\spellsuccess
The target is \partiallyblinded.
\spellcritical
The target is \blinded instead.
\end{spellattack}
\spelldur Condition
\spelltags{\glossterm{Figment}, \glossterm{Light}, \glossterm{Visual}}
\end{spelleffects}
\end{spellcontent}
\begin{spellfooter}
\spellinfo{Illusion}{Arcane, Divine, Nature}
\parhead*{Augments} Extended, Quickened, Silent, Stilled, Widened
\end{spellfooter}
\begin{spellsubcontent}
\begin{spellcantrip}
The spell affects a single creature, rather than an area.
\end{spellcantrip}
\end{spellsubcontent}
\end{spellsection}
\subsubsection{Subspells}
\augment{2}{Dancing Lights}
Replace the spell's effects with the following:
\begin{spellcontent}
\begin{augmenteffects}
\spelleffect
Up to four glowing lights appear in the area.
The lights resemble lanterns or torches, and shed bright light in the same 20 foot radius.
However, you can freely choose the color of the lights when you cast the spell.
During each movement phase, you can move the lights up to 100 feet in any direction.
If one of the lights ever goes out of range from you, it immediately winks out.
\spelldur Sustain (swift)
\spelltags{\glossterm{Figment}, \glossterm{Light}, \glossterm{Visual}}
\end{augmenteffects}
\end{spellcontent}
\augment{2}{Expanded}
The spell's area increases to \areasmall.
This allows the standard Widened augment to be used to expand the spell's area further.
\augment{3}{Faerie Fire}
Each target is surrounded with a pale glow made of hundreds of ephemeral points of lights, causing it to bright light in a 5 foot radius as a candle.
The lights impose a \minus10 penalty to Stealth checks.
In addition, they reveal the outline of the creatures if they become \glossterm{invisible}.
This allows observers to see their location, though not to see them perfectly.
\augment{3}{Illuminating}
The brilliant light persists as long as you spend a \glossterm{swift action} each round to sustain it.
The light has no additional effects on creatures in the area.
\augment{4}{Flashbang}
An intense sound accompanies the flash of light caused by the spell.
If the spell's attack is successful, the target is also \deafened as a condition.
This is an \glossterm{Auditory}, \glossterm{Figment} effect.
\augment{5}{Blinding}
The spell's critical effect makes the target \blinded as a condition, rather than just for one round.
In addition, the blindness replaces the spell's normal success effect, rather than being applied in addition to it.
\augment{5}{Universal}
The light radiates from every point in the area simultaneously, making it impossible to avoid.
The spell's attack is made against Fortitude instead of Reflex.
\begin{spellsection}{Foresight}
\begin{spellheader}
\spelldesc{You grant a creature the ability to see fractions of a second into the future.}
\end{spellheader}
\begin{spellcontent}
\begin{spelltargetinginfo}
\spelltwocol{\spelltgt{One willing creature}}{\spellrng{\rngclose}}
\end{spelltargetinginfo}
\begin{spelleffects}
\spelleffect
The target gains a \plus2 bonus to \glossterm{accuracy} with physical attacks.
\spelldur Attunement
\spelltags{\glossterm{Enhancement}}
\end{spelleffects}
\end{spellcontent}
\begin{spellfooter}
\spellinfo{Divination}{Arcane, Divine, Nature}
\parhead*{Augments} Extended, Quickened, Silent, Stilled
\end{spellfooter}
\begin{spellsubcontent}
\begin{spellcantrip}
The spell's duration becomes Sustain (swift).
\end{spellcantrip}
\end{spellsubcontent}
\end{spellsection}
\subsubsection{Subspells}
\augment{2}{Augury}
Replace the spell's effects with the following:
\begin{spellcontent}
\begin{augmenteffects}
\spelleffect
Choose an action that the target could take.
You learn whether the stated action is likely to bring good or bad results for it within the next hour.
This spell provides one of four results:
\begin{itemize}
\item Weal (if the action will probably bring good results).
\item Woe (for bad results).
\item Weal and woe (for both).
\item No response (for actions that don't have especially good or bad results).
\end{itemize}
This spell does not describe the future with certainty.
It describes which result is most probable.
The more unambiguous the action's effects, the more likely the spell is to be correct.
% TODO: inconsistent wording; this spell vs. this subspell
After using this subspell, you cannot cast it again until the hour affected by the previous casting is over, regardless of whether the action was taken.
\end{augmenteffects}
\end{spellcontent}
\augment{7}{Foresee Actions}
The target can learn what actions all creatures it can observe intend to take during each phase before it decides its actions for that phase.
It learns this information in the instant before it acts, and normally does not have time to communicate it to other creatures.
\begin{spellsection}{Inertial Shield}
\begin{spellheader}
\spelldesc{You create a barrier around your ally that resists physical intrusion.}
\end{spellheader}
\begin{spellcontent}
\begin{spelltargetinginfo}
\spelltwocol{\spelltgt{One creature}}{\spellrng{\rngclose}}
\end{spelltargetinginfo}
\begin{spelleffects}
\spelleffect
The target gains \glossterm{damage reduction} against \glossterm{physical damage} equal to your spellpower.
In addition, it is \glossterm{vulnerable} to arcane damage.
\spelldur Attunement
\spelltags{\glossterm{Shielding}}
\end{spelleffects}
\end{spellcontent}
\begin{spellfooter}
\spellinfo{Abjuration}{Arcane}
\parhead*{Augments} Extended, Quickened, Silent, Stilled
\end{spellfooter}
\begin{spellsubcontent}
\begin{spellcantrip}
The spell's duration becomes Sustain (swift).
\end{spellcantrip}
\end{spellsubcontent}
\end{spellsection}
\subsubsection{Subspells}
\augment{3}{Complete}
The damage reduction applies against all damage, not just physical damage.
\augment{4}{Immunity}
Replace the spell's effects with the following:
\begin{spellcontent}
\begin{augmenteffects}
\spelleffect
Choose a type of damage.
The target becomes immune to damage of the chosen type.
Attacks that deal damage of multiple types still inflict damage normally unless the target is immune to all types of damage dealt.
\end{augmenteffects}
\end{spellcontent}
\augment{4}{Retributive}
Damage resisted by this spell is reflected back to the attacker as life damage.
If the attacker is beyond \rngclose range of the target, this reflection fails.
\par Any effect which increases this spell's range increases the range of this effect by the same amount.
\par
This is a \glossterm{Life} effect from the \glossterm{Vivimancy} school.
\augment{5}{Empowered}
The damage reduction increases by an amount equal to your spellpower.
\begin{spellsection}{Inflict Wounds}
\begin{spellcontent}
\begin{spelltargetinginfo}
\spelltwocol{\spelltgt{One creature}}{\spellrng{\rngmed}}
\end{spelltargetinginfo}
\begin{spelleffects}
\begin{spellattack}{Spellpower vs. Fortitude}
\spellsuccess \spelldamage{life}.
\spellcritical As above, but double damage.
\end{spellattack}
\spelltags{\glossterm{Life}}
\end{spelleffects}
\end{spellcontent}
\begin{spellfooter}
\spellinfo{Vivimancy}{Arcane, Divine, Nature}
\parhead*{Augments} Extended, Intensified, Mass, Quickened, Silent, Stilled
\end{spellfooter}
\begin{spellsubcontent}
\begin{spellcantrip}
The spell's range becomes \rngclose, and it deals \minus1d damage.
\end{spellcantrip}
\end{spellsubcontent}
\end{spellsection}
\subsubsection{Subspells}
\augment{3}{Drain Life}
You gain temporary hit points equal to half the damage you deal with this spell.
\augment{4}{Death Knell}
If the spell's attack succeeds, the target suffers a death knell.
At the end of each round, if the target has 0 hit points, it immediately dies.
This effect lasts until the target removes this condition.
\par
This is a \glossterm{Death} effect.
\begin{spellsection}{Lightning Bolt}
\begin{spellheader}
\spelldesc{You create a bolt of electricity that fries your foes.}
\end{spellheader}
\begin{spellcontent}
\begin{spelltargetinginfo}
\spellburst{\arealarge line, 10 ft\. wide}
\spelltgts{Everything in the area}
\end{spelltargetinginfo}
\begin{spelleffects}
\begin{spellattack}{Spellpower vs. Reflex}
\spellspecial You gain a \plus2 bonus to accuracy against creatures wearing metal armor or otherwise carrying a significant amount of metal.
\spellsuccess
\spelldamage{electricity}[1d4].
\spellcritical As above, but double damage.
\end{spellattack}
\spelltags{\glossterm{Electricity}}
\end{spelleffects}
\end{spellcontent}
\begin{spellfooter}
\spellinfo{Evocation}{Arcane, Nature}
\parhead*{Augments} Intensified, Quickened, Silent, Stilled, Widened
\end{spellfooter}
\begin{spellsubcontent}
\begin{spellcantrip}
The spell's area becomes a 5 ft\. wide \areamed line.
\end{spellcantrip}
\end{spellsubcontent}
\end{spellsection}
\subsubsection{Subspells}
\augment{4}{Instantaneous}
The lightning bolt created by the spell is faster, but less penetrating.
The spell's attack is made against Fortitude defense instead of Reflex defense.
\begin{spellsection}{Planar Disruption}
\begin{spellheader}
\spelldesc{You disrupt a creature's body by partially thrusting it into another plane.}
\end{spellheader}
\begin{spellcontent}
\begin{spelltargetinginfo}
\spelltwocol{\spelltgt{One creature}}{\spellrng{\rngmed}}
\end{spelltargetinginfo}
\begin{spelleffects}
\begin{spellattack}{Spellpower vs. Mental}
\spellsuccess \spelldamage{physical}.
\spellcritical
As above, but double damage.
In addition, if the creature is an \glossterm{outsider} native to another plane, it is sent back to its home plane.
\end{spellattack}
\spelltags{\glossterm{Planar}, \glossterm{Teleportation}}
\end{spelleffects}
\end{spellcontent}
\begin{spellfooter}
\spellinfo{Conjuration}{Arcane, Divine}
\parhead*{Augments} Extended, Intensified, Mass, Quickened, Silent, Stilled
\end{spellfooter}
\begin{spellsubcontent}
\begin{spellcantrip}
The spell's range becomes \rngclose, and it deals \minus1d damage.
\end{spellcantrip}
\end{spellsubcontent}
\end{spellsection}
\begin{spellsection}{Poison}
\begin{spellheader}
\spelldesc{You weaken your foe with a potent poison.}
\end{spellheader}
\begin{spellcontent}
\begin{spelltargetinginfo}
\spelltwocol{\spelltgt{One living creature}}{\spellrng{\rngclose}}
\end{spelltargetinginfo}
\begin{spelleffects}
\spelleffect
At the end of each round, you make a Spellpower vs. Fortitude attack against the target.
Success means the target takes poison damage equal to your spellpower.
If this is the second successful attack, the target takes a \minus2 penalty to \glossterm{accuracy}, \glossterm{checks}, and \glossterm{defenses}.
If this is the third successful attack, the penalty increases to -5.
\spelldur Condition
\spelltags{\glossterm{Poison}}
\end{spelleffects}
\end{spellcontent}
\begin{spellfooter}
\spellinfo{Transmutation}{Destruction, Divine, Nature}
\parhead*{Augments} Extended, Mass, Quickened, Silent, Stilled
\end{spellfooter}
\begin{spellsubcontent}
\begin{spellcantrip}
The spell does not have additional effects other than damage.
\end{spellcantrip}
\end{spellsubcontent}
\end{spellsection}
\begin{spellsection}{Polymorph}
\begin{spellheader}
\spelldesc{You change the target's physical form.}
\end{spellheader}
\begin{spellcontent}
\begin{spelltargetinginfo}
\spelltwocol{\spelltgt{One willing creature}}{\spellrng{\rngmed}}
\end{spelltargetinginfo}
\begin{spelleffects}
\spelleffect
You increase or decrease the target's size by one size category.
\spelldur Attunement
\spelltags{\glossterm{Shaping}, \glossterm{Sizing}}
\end{spelleffects}
\end{spellcontent}
\begin{spellfooter}
\spellinfo{Transmutation}{Arcane, Nature}
\parhead*{Augments} Extended, Quickened, Silent, Stilled
\end{spellfooter}
\begin{spellsubcontent}
\begin{spellcantrip}
The spell's duration becomes Sustain (swift).
\end{spellcantrip}
\end{spellsubcontent}
\end{spellsection}
\subsubsection{Subspells}
\augment{3}{Alter Appearance}
You can also make a Disguise check to alter the target's appearance (see \pcref{Disguise Creature}).
You gain a \plus5 bonus on the check, and you ignore penalties for changing the target's gender, race, subtype, or age.
However, this effect is unable to alter the target's clothes or equipment in any way.
\augment{4}{Fabricate}
Replace the spell's targets and effects with the following:
\begin{spellcontent}
\begin{augmenttargetinginfo}
\spelltwocol{\spelltgts{One or more unattended, nonmagical objects (Large or smaller); see text}}{\spellrng{\rngclose}}
\end{augmenttargetinginfo}
\begin{augmenteffects}
\spelleffect
You make a Craft check to transform the targets into a new item (or items) made of the same materials.
You require none of the tools or time expenditure that would normally be necessary.
The total size of all targets combined must be Large size or smaller.
\spelltags{\glossterm{Shaping}}
\end{augmenteffects}
\end{spellcontent}
\begin{spellsection}{Protection from Alignment}
\begin{spellcontent}
\begin{spelltargetinginfo}
\spelltwocol{\spelltgt{One creature}}{\spellrng{\rngclose}}
\end{spelltargetinginfo}
\begin{spelleffects}
\spellspecial
Choose an alignment other than neutral (chaotic, good, evil, lawful).
This spell gains the tag for that alignment's \glossterm{opposed alignment}.
\spelleffect
The target gains damage reduction equal to your spellpower against physical effects that have the chosen alignment, and physical attacks made by creatures with the chosen alignment.
\spelltags{\glossterm{Shielding}}
\end{spelleffects}
\end{spellcontent}
\begin{spellfooter}
\spellinfo{Abjuration}{Arcane, Chaos, Divine, Evil, Good, Law}
\parhead*{Augments} Extended, Quickened, Silent, Stilled
\end{spellfooter}
\begin{spellsubcontent}
\begin{spellcantrip}
The spell's duration becomes Sustain (swift).
\end{spellcantrip}
\end{spellsubcontent}
\end{spellsection}
\subsubsection{Subspells}
\augment{3}{Complete}
The damage reduction also applies against non-physical effects.
\augment{4}{Retributive}
Whenever a creature with the chosen alignment makes a physical melee attack against the target, you make a Spellpower vs. Mental attack against the attacking creature.
Success means the attacker takes \spelldamage{divine}[d4].
\begin{spellsection}{Scry}
\begin{spellheader}
\spelldesc{You create a scrying sensor that allows you to see at a distance.}
\end{spellheader}
\begin{spellcontent}
\begin{spelltargetinginfo}
\spelltwocol{\spelltgt{One square}}{\spellrng{\rngmed}}
\end{spelltargetinginfo}
\begin{spelleffects}
\spelleffect
A Fine object appears floating in the air in the target space.
It resembles a human eye in size and shape, though it is \glossterm{invisible}.
At the start of each round, you choose whether you see from this sensor or from your body.
The sensor's visual acuity is the same as your own, except that it does not share the benefits of any \glossterm{magical} effects that improve your vision.
You otherwise act normally, though you may have difficulty moving or taking actions if the sensor cannot see your body or your intended targets, effectively making you \blinded.
If undisturbed, the sensor floats in the air in its position.
As a standard action, you can concentrate to move the sensor up to 30 feet in any direction, even vertically.
You can only have one casting of this spell active at once.
If you cast it again, any previous castings of the spell are dismissed.
\spelldur Attunement
\spelltags{\glossterm{Scrying}}
\end{spelleffects}
\end{spellcontent}
\begin{spellfooter}
\spellinfo{Divination}{Arcane, Divine, Nature}
\parhead*{Augments} Extended, Mass, Quickened, Silent, Stilled
\end{spellfooter}
\begin{spellsubcontent}
\begin{spellcantrip}
The sensor cannot be moved after it is originally created, and the spell's duration becomes Sustain (swift).
\end{spellcantrip}
\end{spellsubcontent}
\end{spellsection}
\subsubsection{Subspells}
\augment{2}{Alarm}
The sensor continues to observe its surroundings while you are not sensing through it.
If it sees a creature or object of Tiny size or larger moving within 50 feet of it, it will trigger a mental "ping" that only you can notice.
You must be within 1 mile of the sensor to receive this mental alarm.
This mental sensation is strong enough to wake you from normal sleep, but does not otherwise disturb concentration.
\augment{2}{Auditory}
At the start of each round, you can choose whether you hear from the sensor or from your body.
This choice is made independently from your sight.
The sensor's auditory acuity is the same as your own, except that it does not share the benefits of any \glossterm{magical} effects that improve your hearing.
\augment{3}{Accelerated}
When you move the sensor, you can move it up to 100 feet, instead of up to 30 feet.
\augment{3}{Dual}
You create an additional sensor in the same location.
You must move and see through each sensor individually.
\augment{3}{Penetrating}
The spell's range becomes \rngunrestricted, allowing you to cast it into areas where you do not have \glossterm{line of sight} or \glossterm{line of effect}.
\augment{4}{Reverse Scrying}
% TODO: wording
The sensor created by this spell appears at the location of the source of the ability that created the target sensor.
Replace the spell's targets with the following:
\begin{spellcontent}
\begin{augmenttargetinginfo}
\spelltwocol{\spelltgt{One magical sensor}}{\spellrng{\rngmed}}
\end{augmenttargetinginfo}
\end{spellcontent}
\augment{4}{Semi-Autonomous}
You can move the sensor as a \glossterm{swift action} rather than as a standard action.
\augment{5}{Scry Creature}
You must make a Spellpower vs. Mental attack against the target.
Success means the sensor appears in the target's space.
Failure means the sensor does not appear at all.
Replace the spell's targets with the following:
\begin{spellcontent}
\begin{augmenttargetinginfo}
\spellspecial
You must specify your target with a precise mental image of its appearance.
The image does not have to be perfect, but it must unambiguously identify the target.
If you specify its appearance incorrectly, or if the target has changed its appearance, you may accidentally target a different creature, or the spell may simply fail.
\spelltwocol{\spelltgt{One creature}}{\spellrng{Same plane (Unrestricted)}}
\end{augmenttargetinginfo}
\end{spellcontent}
\augment{6}{Split Senses}
You do not have to choose whether to sense from the perspective of the sensor or from the perspective of your own body.
You constantly receive sensory input from both your body and the sensor.
\begin{spellsection}{Smite}
\begin{spellheader}
\spelldesc{You smite a foe with holy (or unholy) power.}
\end{spellheader}
\begin{spellcontent}
\begin{spelltargetinginfo}
\spelltwocol{\spelltgt{One creature}}{\spellrng{\rngmed}}
\end{spelltargetinginfo}
\begin{spelleffects}
\begin{spellattack}{Spellpower vs. Mental}
\spellsuccess \spelldamage{divine}.
\spellcritical As above, but double damage.
\end{spellattack}
\end{spelleffects}
\end{spellcontent}
\begin{spellfooter}
\spellinfo{Channeling}{Divine}
\parhead*{Augments} Extended, Intensified, Mass, Quickened, Silent, Stilled
\end{spellfooter}
\begin{spellsubcontent}
\begin{spellcantrip}
The spell's range becomes \rngclose, and it deals \minus1d damage.
\end{spellcantrip}
\end{spellsubcontent}
\end{spellsection}
\begin{spellsection}{Summon Monster}
\begin{spellheader}
\spelldesc{You summon a creature to fight by your side.}
\end{spellheader}
\begin{spellcontent}
\begin{spelltargetinginfo}
\spelltwocol{\spelltgt{One unoccupied square}}{\spellrng{\rngmed}}
\end{spelltargetinginfo}
\begin{spelleffects}
\spelleffect
A creature appears in the target location.
It visually appears to be a common Small or Medium animal of your choice, though in reality it is a manifestation of magical energy.
Regardless of the appearance and size chosen, the creature has hit points equal to twice your spellpower.
All of its defenses are equal to your 5 \add your spellpower, and its land speed is equal to 30 feet.
Each round, you choose the creature's actions.
There are only two actions it can take.
As a move action, it can move as you direct.
As a standard action, it can make a melee \glossterm{strike} against a creature it threatens.
Its accuracy is equal to your spellpower.
If it hits, it deals 1d3 damage \plus1d per two spellpower.
The type of damage dealt by this attack depends on the creature's appearance.
Most animals bite or claw their foes, which deals bludgeoning and slashing damage.
\spelldur Sustain (swift)
\spelltags{\glossterm{Manifestation}}
\end{spelleffects}
\end{spellcontent}
\begin{spellfooter}
\spellinfo{Conjuration}{Arcane, Divine, Nature}
\parhead*{Augments} Extended, Quickened, Silent, Stilled
\end{spellfooter}
\begin{spellsubcontent}
\begin{spellcantrip}
The spell's duration becomes Sustain (standard).
\end{spellcantrip}
\end{spellsubcontent}
\end{spellsection}
\subsubsection{Subspells}
\augment{2}{Summon Bear}
The creature appears to be a Medium bear.
As a standard action, it can make a \glossterm{grapple} attack against a creature it threatens.
Its accuracy is the same as its accuracy with strikes.
While grappling, the manifested creature can either make a strike or attempt to escape the grapple.
This augment replaces the effects of any other augments that change the appearance of the creature.
\begin{spellsection}{Telekinesis}
\begin{spellcontent}
\begin{spelltargetinginfo}
\spelltwocol{\spelltgt{One Medium or smaller creature or object}}{\spellrng{\rngclose}}
\end{spelltargetinginfo}
\begin{spelleffects}
\begin{spellattack}{Spellpower vs. Mental}
\spellsuccess
You move the target up five feet per spellpower. Moving the target upwards costs twice the normal movement cost.
\spellcritical
As above, but you move the target ten feet per spellpower instead of five feet per spellpower.
\end{spellattack}
\spelltags{\glossterm{Telekinesis}}
\end{spelleffects}
\end{spellcontent}
\begin{spellfooter}
\spellinfo{Evocation}{Arcane}
\parhead*{Augments} Extended, Mass, Quickened, Silent, Stilled
\end{spellfooter}
\begin{spellsubcontent}
\begin{spellcantrip}
If your attack succeeds, you move the target one foot per spellpower. In addition, the spell has no additional effects on a critical hit.
\end{spellcantrip}
\end{spellsubcontent}
\end{spellsection}
\subsubsection{Subspells}
\augment{2}{Mending}
Replace the spell's targets and effects with the following:
\begin{spellcontent}
\begin{augmenttargetinginfo}
\spelltwocol{\spelltgt{One unattended object}}{\spellrng{\rngclose}}
\end{augmenttargetinginfo}
\begin{augmenteffects}
\spelleffect
The target is healed for \spelldamage{}.
\end{augmenteffects}
\end{spellcontent}
\augment{2}{Precise}
Replace the spell's effects with the following:
\begin{spellcontent}
\begin{augmenteffects}
\begin{spellattack}{Spellpower vs. Mental}
\spellsuccess
You move the target up to five feet in any direction.
In addition, you can make a check to manipulate the target as if you were using your hands.
The check's result has a maximum equal to your attack result.
\end{spellattack}
\spelltags{\glossterm{Telekinesis}}
\end{augmenteffects}
\end{spellcontent}
\augment{3}{Binding}
If your attack roll beat both the target's Fortitude and Mental defenses, it is \immobilized after the forced movement is finished.
This is a \glossterm{condition}, and lasts until removed.
\augment{3}{Levitate}
Replace the spell's targets and effects with the following:
\begin{spellcontent}
\begin{augmenttargetinginfo}
\spelltwocol{\spelltgt{One unattended object or willing creature (Medium or smaller)}}{\spellrng{\rngclose}}
\end{augmenttargetinginfo}
\begin{augmenteffects}
\spelleffect
The target floats in midair, unaffected by gravity.
During the movement phase, you can move the target up to ten feet in any direction.
\spelldur Sustain (swift)
\end{augmenteffects}
\end{spellcontent}
\begin{spellsection}{Water Mastery}
\begin{spellheader}
\spelldesc{You create a wave of water to crush your foes.}
\end{spellheader}
\begin{spellcontent}
\begin{spelltargetinginfo}
\spellburst{\arealarge line, 10 ft\. wide}
\spelltgts{Everything in the area}
\end{spelltargetinginfo}
\begin{spelleffects}
\begin{spellattack}{Spellpower vs. Fortitude}
\spellsuccess \spelldamage{bludgeoning}[1d4].
\spellcritical As above, but double damage.
\end{spellattack}
\spelltags{\glossterm{Manifestation}, \glossterm{Water}}
\end{spelleffects}
\end{spellcontent}
\begin{spellfooter}
\spellinfo{Conjuration}{Nature, Water}
\parhead*{Augments} Intensified, Quickened, Silent, Stilled, Widened
\end{spellfooter}
\begin{spellsubcontent}
\begin{spellcantrip}
The spell's area becomes a 5 ft.\ wide, \areamed line.
\end{spellcantrip}
\end{spellsubcontent}
\end{spellsection}
\subsubsection{Subspells}
\augment{2}{Aqueuous Sphere}
Replace the spell's targets with the following:
\begin{spellcontent}
\begin{augmenttargetinginfo}
\spelltwocol{\spellburst{\areasmall radius}}{\spellrng{\rngclose}}
\spelltgts{Everything in the area}
\end{augmenttargetinginfo}
\end{spellcontent}
\augment{2}{Create Water}
Replace the spell's targets and effects with the following:
\begin{spellcontent}
\begin{augmenttargetinginfo}
\spellrng{\rngclose}
\end{augmenttargetinginfo}
\begin{augmenteffects}
\spelleffect
You create up to one gallon of wholesome, drinkable water.
The water can be created at multiple locations within the ritual's range, allowing you to fill multiple small water containers.
\spelltags{\glossterm{Creation}, \glossterm{Water}}
\end{augmenteffects}
\end{spellcontent}
\augment{4}{Sustained}
The area affected by this spell becomes completely filled with water.
You can sustain the water as a \glossterm{swift action}.
Creatures in this \glossterm{zone} suffer penalties appropriate for fighting underwater, and may be unable to breathe.
\begin{spellsection}{Web}
\begin{spellheader}
\spelldesc{
You create a many-layered mass of strong, stricky strands that trap creatures caught within them.
The strands are similar to spider webs, but larger and tougher.
}
\end{spellheader}
\begin{spellcontent}
\begin{spelltargetinginfo}
\spelltwocol{\spellzone{\areasmall radius}}{\spellrng{\rngclose}}
\spelltgts{Everything in the area}
\end{spelltargetinginfo}
\begin{spelleffects}
\spelleffect
The area becomes filled with webs, making it \glossterm{difficult terrain}.
Each 5-ft.\ square of webbing has hit points equal to your spellpower, and is \glossterm{vulnerable} to fire.
\begin{spellattack}{Spellpower vs. Reflex}
\spellsuccess The target is \immobilized as long as it has webbing from this spell in its space.
\end{spellattack}
\spelldur Sustain (swift)
\spelltags{\glossterm{Manifestation}}
\end{spelleffects}
\end{spellcontent}
\begin{spellfooter}
\spellinfo{Conjuration}{Arcane, Nature}
\parhead*{Augments} Extended, Quickened, Silent, Stilled, Widened
\end{spellfooter}
\begin{spellsubcontent}
\begin{spellcantrip}
The spell's duration becomes Sustain (standard).
\end{spellcantrip}
\end{spellsubcontent}
\end{spellsection}
\subsubsection{Subspells}
\augment{3}{Reinforced}
Each 5-ft.\ square of webbing gains additional hit points equal to your spellpower.
In addition, the webs are no longer vulnerable to fire.
\begin{spellsection}{Word of Faith}
\begin{spellheader}
\spelldesc{You speak an utterance that rebukes those who do not share your faith.}
\end{spellheader}
\begin{spellcontent}
\begin{spelltargetinginfo}
\spellburst{\areamed radius from you}
\spelltgts{Creatures in the area that do not worship your deity}
\end{spelltargetinginfo}
\begin{spelleffects}
\begin{spellattack}{Spellpower vs. Mental}
\spellsuccess \spelldamage{divine}.
\spellcritical As above, but double damage.
\end{spellattack}
\end{spelleffects}
\end{spellcontent}
\begin{spellfooter}
\spellinfo{Channeling}{Divine}
\parhead*{Augments} Intensified, Quickened, Silent, Stilled, Widened
\end{spellfooter}
\begin{spellsubcontent}
\begin{spellcantrip}
The spell's area becomes an \areasmall radius.
\end{spellcantrip}
\end{spellsubcontent}
\end{spellsection}
\subsubsection{Subspells}
\augment{4}{Bolstering}
Creatures in the spell's area that worship your deity heal 1d4 damage \plus1d per two spellpower.

\begin{spellsection}{Animate Dead}[3]

\begin{spellheader}
\spelldesc{You bind a fragment of a dead creature's soul to its corpse, reanimating it as an undead skeleton or zombie.}
\end{spellheader}


\begin{ability}{Animate Dead}[\glossterm{Attune} (multiple)]

Choose any number of corpses within \rngclose range.
The combined levels of all targets cannot exceed your spellpower.
The target becomes an undead creature that obeys your spoken commands.
You choose whether to create a skeleton or a zombie.
Creating a zombie require a mostly intact corpse, including most of the flesh.
Creating a skeleton only requires a mostly intact skeleton.
If a skeleton is made from an intact corpse, the flesh quickly falls off the animated bones.

This ritual takes one hour to perform.

\end{ability}




\parhead{Schools} Vivimancy

\parhead{Spell Lists} Arcane, Divine
\end{spellsection}


\begin{spellsection}{Awaken}[7]


\begin{ability}{Awaken}

Choose a Large or smaller willing animal within \rngclose range.
The target becomes sentient.
Its Intelligence becomes 1d6 \sub 5.
Its type changes from animal to magical beast.
It gains the ability to speak and understand one language that you know of your choice.
This effect is permanent.

This ritual takes 24 hours to perform, and requires 49 action points from its participants.

\end{ability}




\parhead{Schools} Transmutation

\parhead{Spell Lists} Nature
\end{spellsection}


\begin{spellsection}{Binding}[3]


\begin{ability}{Binding}[\glossterm{Attune}]

This ritual creates a \areasmall radius zone.
The outlines of the zone are denoted by a magic circle physically inscribed on the ground during the ritual.
The circle is obvious, but a DR 16 Perception or Spellcraft check is required to verify that the circle belongs to a \ritual{binding} ritual.
If the circle's perimeter is broken, the ritual's effects end immediately.
Whenever a creature enters the area, you make a Spellpower vs. Mental attack against it.
\hit The target is unable to escape the ritual's area physically or alter the circle in any way.
It treats the edge of the area as an impassable barrier, preventing the effects of any of its abilities from extending outside that area.

This ritual takes one hour to perform.

\end{ability}




\parhead{Schools} Abjuration

\parhead{Spell Lists} Arcane, Divine
\end{spellsection}


\subsubsection{Subrituals}


\begin{ability}[\nth{5}]{Dimension Lock}
This subritual functions like the \ritual{binding} ritual, except that a struck creature also cannot travel extradimensionally.
This prevents all \glossterm{Manifestation}, \glossterm{Planar}, and \glossterm{Teleportation} effects.
\end{ability}
\vspace{0.25em}


\begin{spellsection}{Bless Water}[1]


\begin{ability}{Bless Water}[\glossterm{Attune} (multiple)]

Choose one pint of unattended, nonmagical water within \rngclose range.
The target becomes holy water.
Holy water can be can be thrown as a splash weapon, dealing 1d8 points of damage to a struck undead creature or an evil outsider.

This ritual takes one minute to perform.

\end{ability}




\parhead{Schools} Channeling

\parhead{Spell Lists} Divine
\end{spellsection}


\begin{spellsection}{Blessed Transit}[5]


\begin{ability}{Blessed Transit}[\glossterm{Teleportation}]

This ritual functions like the \ritual{overland teleporation} ritual, except that the destination must be a temple or equivalent holy site to your deity.

\end{ability}




\parhead{Schools} Conjuration

\parhead{Spell Lists} divine
\end{spellsection}


\begin{spellsection}{Create Object}[3]


\begin{ability}{Create Object}[\glossterm{Attune} (multiple), \glossterm{Manifestation}]

Make a Craft check to create an object of no greater than Small size.
The object appears out of thin air in an unoccupied square within \rngclose range.
% TODO: add ability to create objects of other sizes/materials
It must be made of nonliving, nonreactive vegetable matter, such as wood or cloth.

This ritual takes one hour to perform.

\end{ability}




\parhead{Schools} Conjuration

\parhead{Spell Lists} Arcane, Divine, Nature
\end{spellsection}


\begin{spellsection}{Create Sustenance}[3]


\begin{ability}{Create Sustenance}[\glossterm{Creation}]

Choose an unoccupied square within \rngclose range.
This ritual creates food and drink in that square that is sufficient to sustain two Medium creatures per spellpower for 24 hours.
The food that this ritual creates is simple fare of your choice -- highly nourishing, if rather bland.

This ritual takes one hour to perform.

\end{ability}




\parhead{Schools} Conjuration

\parhead{Spell Lists} Arcane, Divine, Nature
\end{spellsection}


\begin{spellsection}{Curse Water}[1]


\begin{ability}{Curse Water}[\glossterm{Attune} (multiple)]

Choose one pint of unattended, nonmagical water within \rngclose range.
The target becomes unholy water.
Unholy water can be can be thrown as a splash weapon, dealing 1d8 points of damage to a struck good outsider.

This ritual takes one minute to perform.

\end{ability}




\parhead{Schools} Channeling

\parhead{Spell Lists} Divine
\end{spellsection}


\begin{spellsection}{Discern Location}[5]


\begin{ability}{Discern Location}

Choose a creature or object on the same plane as you.
You do not need \glossterm{line of sight} or \glossterm{line of effect} to the target.
However, you must specify your target with a precise mental image of its appearance.
The image does not have to be perfect, but it must unambiguously identify the target.
You learn the location (place, name, business name, or the like), community, country, and continent where the target lies.

This ritual takes 24 hours to perform, and it requires 25 action points from its participants.

\end{ability}




\parhead{Schools} Divination

\parhead{Spell Lists} Arcane, Divine, Nature
\end{spellsection}


\subsubsection{Subrituals}


\begin{ability}[\nth{7}]{Interplanar}
This subritual functions like the \ritual{discern location} ritual, except that the target does not have to be on the same plane as you.
It gains the \glossterm{Planar} tag in addition to the tags from the \ritual{discern location} ritual.

This ritual takes 24 hours to perform, and it requires 49 action points from its participants.
\end{ability}
\vspace{0.25em}


\begin{spellsection}{Endure Elements}[1]


\begin{ability}{Endure Elements}[\glossterm{Attune} (multiple)]

Choose a willing creature or unattended object within \rngclose range.
The target suffers no harm from being in a hot or cold environment.
It can exist comfortably in conditions between \minus50 and 140 degrees Fahrenheit.
Its equipment, if any, is also protected.
This does not protect the target from fire or cold damage.

This ritual takes one minute to perform.

\end{ability}




\parhead{Schools} Abjuration

\parhead{Spell Lists} Arcane, Divine, Nature
\end{spellsection}


\begin{spellsection}{Explosive Runes}[3]


\begin{ability}{Explosive Runes}[\glossterm{Attune} (multiple), \glossterm{Trap}]

Choose a Small or smaller unattended object with writing on it within \rngclose range.
In addition, choose a type of \glossterm{energy damage} (cold, electricity, fire, or sonic).
This ritual gains the tag appropriate to the chosen energy type.
If a creature reads the target, the target explodes.
You make a Spellpower vs. Reflex attack against everything within an \areamed radius from the target.
\hit Each target takes \glossterm{standard damage} \minus1d of the damage type chosen.

After the target explodes in this way, the ritual is \glossterm{dismissed}.
If the target object is destroyed or rendered illegible, the ritual is dismissed without exploding.
This ritual takes one hour to perform.

\end{ability}




\parhead{Schools} Evocation

\parhead{Spell Lists} Arcane
\end{spellsection}


\begin{spellsection}{Fertility}[3]


\begin{ability}{Fertility}

This ritual creates an area of bountiful growth in a one mile radius zone from your location.
Normal plants within the area become twice as productive as normal for the next year.
This ritual does not stack with itself.
If the \ritual{infertility} ritual is also applied to the same area, the most recently performed ritual takes precedence.

This ritual takes 24 hours to perform, and requires 9 action points from its participants.

\end{ability}




\parhead{Schools} Transmutation

\parhead{Spell Lists} Nature
\end{spellsection}


\begin{spellsection}{Gate}[9]


\begin{ability}{Gate}[\glossterm{Planar}, \glossterm{Sustain} (standard), \glossterm{Teleportation}]

Choose a plane that connects to your current plane, and a location within that plane.
This ritual creates an interdimensional connection between your current plane and the location you choose, allowing travel between those two planes in either direction.
The gate takes the form of an \areasmall radius circular disk, oriented a direction you choose (typically vertical).
It is a two-dimensional window looking into the plane you specified when casting the spell, and anyone or anything that moves through it is shunted instantly to the other location.
The gate cannot be \glossterm{sustained} for more than 5 rounds, and is automatically dismissed at the end of that time.

You must specify the gate's destination with a precise mental image of its appearance.
The image does not have to be perfect, but it must unambiguously identify the location.
Incomplete or incorrect mental images may result in the ritual leading to an unintended destination within the same plane, or simply failing entirely.

% TODO: Is this planar cosmology correct?
The Astral Plane connects to every plane, but transit from other planes is usually more limited.
From the Material Plane, you can only reach the Astral Plane.

This ritual takes one week to perform, and requires 81 action points from its participants.

\end{ability}




\parhead{Schools} Conjuration

\parhead{Spell Lists} Arcane, Divine, Nature
\end{spellsection}


\begin{spellsection}{Gentle Repose}[2]


\begin{ability}{Gentle Repose}[\glossterm{Attune} (multiple), \glossterm{Temporal}]

Choose an unattended, nonmagical object within \rngclose range.
Time does not pass for the target, preventing it from decaying or spoiling.
This can extend the time a poison or similar item lasts before becoming inert.
If used on a corpse, this effectively extends the time limit on raising that creature from the dead (see \ritual{resurrection}) and similar effects that require a fresh body.
Additionally, this can make transporting a fallen comrade more pleasant.

This ritual takes one minute to perform.

\end{ability}




\parhead{Schools} Transmutation

\parhead{Spell Lists} Arcane, Divine, Nature
\end{spellsection}


\begin{spellsection}{Infertility}[3]


\begin{ability}{Fertility}

This ritual creates an area of death and decay in a one mile radius zone from your location.
Normal plants within the area become half as productive as normal for the next year.
This ritual does not stack with itself.
If the \ritual{fertility} ritual is also applied to the same area, the most recently performed ritual takes precedence.

This ritual takes 24 hours to perform, and requires 9 action points from its participants.

\end{ability}




\parhead{Schools} Transmutation

\parhead{Spell Lists} Nature
\end{spellsection}


\begin{spellsection}{Ironwood}[3]


\begin{ability}{Ironwood}[\glossterm{Shaping}]

Choose a Small or smaller unattended, nonmagical wooden object within \rngclose range.
The target is transformed into ironwood.
While remaining natural wood in almost every way, ironwood is as strong, heavy, and resistant to fire as iron.
Metallic armor and weapons, such as full plate, can be crafted from ironwood.

% Should this have an action point cost? May be too rare...
This ritual takes 24 hours to perform.

\end{ability}




\parhead{Schools} Transmutation

\parhead{Spell Lists} Nature
\end{spellsection}


\begin{spellsection}{Lifeweb Transit}[5]


\begin{ability}{Lifeweb Transit}[\glossterm{Teleportation}]

This ritual functions like the \ritual{overland teleporation} ritual, except that both the starting and ending points must be living plants.
Both plants must be larger than the largest creature being teleported in this way.

\end{ability}




\parhead{Schools} Conjuration

\parhead{Spell Lists} Nature
\end{spellsection}


\begin{spellsection}{Light}[1]


\begin{ability}{Light}[\glossterm{Attune} (multiple), \glossterm{Figment}, \glossterm{Light}]

Choose a Medium or smaller willing creature or unattended object within \rngclose range.
The target glows like a torch, shedding bright light in an \areamed radius (and dim light for an additional 20 feet).

This ritual takes one minute to perform.

\end{ability}




\parhead{Schools} Illusion

\parhead{Spell Lists} Arcane, Divine, Nature
\end{spellsection}


\begin{spellsection}{Magic Mouth}[1]


\begin{ability}{Magic Mouth}[\glossterm{Attune} (multiple), \glossterm{Figment}]

Choose a Large or smaller willing creature or unattended object within \rngclose range.
In addition, choose a triggering condition and a message of twenty-five words or less.
The condition must be something that a typical human in the target's place could detect.

When the triggering condition occurs, the target appears to grow a magically animated mouth.
The mouth speaks the chosen message aloud.
After the message is spoken, this effect is \glossterm{dismissed}.

This ritual takes 24 hours to perform.

\end{ability}




\parhead{Schools} Illusion

\parhead{Spell Lists} Arcane
\end{spellsection}


\begin{spellsection}{Mount}[3]


\begin{ability}{Mount}[\glossterm{Attune} (multiple), \glossterm{Manifestation}]

This ritual creates your choice of a light horse or a pony to serve as a mount.
The creature appears in an unoccupied location within \rngclose range.
It comes with a bit and bridle and a riding saddle, and will readily accept any creature as a rider.

\end{ability}




\parhead{Schools} Conjuration

\parhead{Spell Lists} Arcane
\end{spellsection}


\begin{spellsection}{Mystic Lock}[2]


\begin{ability}{Mystic Lock}[\glossterm{Attune} (multiple)]

Choose a Large or smaller closable, nonmagical object within \rngclose range, such as a door or box.
The target object becomes magically locked.
It can be unlocked with a Devices check against a DR equal to 20 \add your spellpower.
The DR to break it open forcibly increases by 10.

You can freely pass your own \ritual{arcane lock} as if the object were not locked.
This effect lasts as long as you \glossterm{attune} to it.
If you use this ability multiple times, you can attune to it each time.

This ritual takes one minute to perform.

\end{ability}




\parhead{Schools} Transmutation

\parhead{Spell Lists} Arcane, Divine, Nature
\end{spellsection}


\subsubsection{Subrituals}


\begin{ability}[\nth{5}]{Resilient}
This subritual functions like the \ritual{mystic lock} ritual, except that the DR to unlock the target with a Devices check is instead equal to 30 + your spellpower.
In addition, the DR to break it open increases by 20 instead of by 10.
\end{ability}
\vspace{0.25em}


\begin{spellsection}{Overland Teleportation}[5]


\begin{ability}{Overland Teleportation}[\glossterm{Teleportation}]

Choose up to five willing, Medium or smaller ritual participants.
Choose a destination up to 100 miles away from you on your current plane.
Each target is teleproted to the chosen destination.

You must specify the destination with a precise mental image of its appearance.
The image does not have to be perfect, but it must unambiguously identify the destination.
If you specify its appearance incorrectly, or if the area has changed its appearance, the destination may be a different area than you intended.
The new destination will be one that more closely resembles your mental image.
If no such area exists, the ritual simply fails.
% TODO: does this need more clarity about what teleportation works?

This ritual takes 24 hours to perform and requires 25 action points from its ritual participants.

\end{ability}




\parhead{Schools} Conjuration

\parhead{Spell Lists} Arcane
\end{spellsection}


\begin{spellsection}{Plane Shift}[5]


\begin{ability}{Plane Shift}[\glossterm{Planar}, \glossterm{Teleportation}]

Choose up to five Medium or smaller willing ritual participants.
In addition, choose a plane that connects to your current plane and a location within that plane.
The targets teleport to a random location on that plane 1d100 miles away from the intended destination.

You must specify the destination with a precise mental image of its appearance.
The image does not have to be perfect, but it must unambiguously identify the location.
Incomplete or incorrect mental images may result in the ritual leading to an unintended destination within the same plane, or simply failing entirely.

% TODO: Is this planar cosmology correct?
The Astral Plane connects to every plane, but transit from other planes is usually more limited.
From the Material Plane, you can only reach the Astral Plane.

This ritual takes 24 hours to perform, and requires 25 action points from its participants.

\end{ability}




\parhead{Schools} Conjuration

\parhead{Spell Lists} Arcane, Divine, Nature
\end{spellsection}


\subsubsection{Subrituals}


\begin{ability}[\nth{8}]{Precise}
This subritual functions like the \ritual{plane shift} ritual, except that the actual destination is the same as the intended destination, rather than being a random distance away.
This ritual takes 24 hours to perform, and requires 64 action points from its participants.
\end{ability}
\vspace{0.25em}


\begin{spellsection}{Private Sanctum}[5]


\begin{ability}{Private Sanctum}[\glossterm{Mystic}]

This ritual creates a ward against any external perception in an \arealarge radius zone centered on your location.
This effect is permanent.
Everything in the area is completely imperceptible from outside the area.
Anyone observing the area from outside sees only a dark, silent void, regardless of darkvision and similar abilities.
In addition, all \glossterm{Scrying} effects fail to function in the area.
Creatures inside the area can see within the area and outside of it without any difficulty.

This ritual takes 24 hours to perform, and requires 25 action points from its participants.

\end{ability}




\parhead{Schools} Abjuration

\parhead{Spell Lists} Arcane
\end{spellsection}


\begin{spellsection}{Purge Curse}[3]


\begin{ability}{Purge Curse}[\glossterm{Mystic}]

Choose a willing creature within \rngclose range.
All curses affecting the target are removed.
This ritual cannot remove a curse that is part of the effect of an item the target has equipped.
However, it can allow the target to remove any cursed items it has equipped.

This ritual takes 24 hours to perform, and requires 9 action points from its participants.

\end{ability}




\parhead{Schools} Abjuration

\parhead{Spell Lists} Arcane, Divine, Nature
\end{spellsection}


\begin{spellsection}{Purify Sustenance}[1]


\begin{ability}{Purify Sustenance}[\glossterm{Shaping}]

All food and water in a single square within \rngclose range is purified.
Spoiled, rotten, poisonous, or otherwise contaminated food and water becomes pure and suitable for eating and drinking.
This does not prevent subsequent natural decay or spoiling.

This ritual takes one hour to perform.

\end{ability}




\parhead{Schools} Transmutation

\parhead{Spell Lists} Arcane, Divine, Nature
\end{spellsection}


\begin{spellsection}{Read Magic}[1]


\begin{ability}{Read Magic}[\glossterm{Attune}]

You gain the ability to decipher magical inscriptions that would otherwise be unintelligible.
This can allow you to read ritual books and similar objects created by other creatures.
After you have read an inscription in this way, you are able to read that particular writing without the use of this ritual.

This ritual takes one minute to perform.

\end{ability}




\parhead{Schools} Divination

\parhead{Spell Lists} Arcane, Divine, Nature
\end{spellsection}


\begin{spellsection}{Regeneration}[4]


\begin{ability}{Regeneration}[\glossterm{Flesh}]

Choose a willing creature within \rngclose range.
All of the target's hit points, \glossterm{subdual damage}, and \glossterm{vital damage} are healed.
In addition, any of the target's severed body parts or missing organs grow back by the end of the next round.

This ritual takes 24 hours to perform, and requires 16 action points from its participants.

\end{ability}




\parhead{Schools} Vivimancy

\parhead{Spell Lists} Divine, Nature
\end{spellsection}


\begin{spellsection}{Reincarnation}[5]


\begin{ability}{Reincarnation}[\glossterm{Creation}, \glossterm{Flesh}, \glossterm{Life}]

Choose one Diminuitive or larger piece of a humanoid corpse.
The target must have been part of the original creature's body at the time of death.
The creature the target corpse belongs to returns to life in a new body.
It must not have died due to old age.

This ritual creates an entirely new body for the creature's soul to inhabit from the natural elements at hand.
During the ritual, the body ages to match the age of the original creature at the time it died.
The creature has 0 hit points when it returns to life.

A reincarnated creature is identical to the original creature in all respects, except for its race.
The creature's race is replaced with a random race from \tref{Humanoid Reincarnations}.
Its appearance changes as necessary to match its new race, though it retains the general shape and distinguishing features of its original appearance.
The creature loses all attribute modifiers and abilities from its old race, and gains those of its new race.
However, its racial bonus feat and languages are unchanged.

A creature's soul naturally rejects being placed into a different body than its original home.
Until the creature is restored to its initial race, its maximum action points are reduced by 1.
This penalty does not stack if the creature is reincarnated multiple times.

Coming back from the dead is an ordeal.
All of the creature's action points and other daily abilities are expended when it returns to life.
In addition, its maximum action points are reduced by 1.
This penalty lasts for thirty days, or until the creature gains a level.
If this would reduce a creature's maximum action points below 0, the creature cannot be resurrected.

This ritual takes 24 hours to perform, and requires 25 action points from its participants.

\end{ability}




\parhead{Schools} Conjuration, Vivimancy

\parhead{Spell Lists} Nature
\end{spellsection}


\subsubsection{Subrituals}


\begin{ability}[\nth{7}]{Fated}
This subritual functions like the \ritual{reincarnation} ritual, except that the target is reincarnated as its original race instead of as a random race.
This ritual takes 24 hours to perform, and requires 49 action points from its participants.
\end{ability}
\vspace{0.25em}


\begin{dtable}
\lcaption{Humanoid Reincarnations}
\begin{dtabularx}{\columnwidth}{l X}
d\% & Incarnation \\
\bottomrule
01 & Bugbear \\
02-\minus13 & Dwarf \\
14-\minus25 & Elf \\
26 & Gnoll \\
27-\minus38 & Gnome \\
39-\minus42 & Goblin \\
43-\minus52 & Half-elf \\
53-\minus62 & Half-orc \\
63-\minus74 & Halfling \\
75-\minus89 & Human \\
90-\minus93 & Kobold \\
94 & Lizardfolk \\
95-\minus99 & Orc \\
100 & Other
\end{dtabularx}
\end{dtable}


\begin{spellsection}{Resurrection}[4]


\begin{ability}{Resurrection}[\glossterm{Flesh}, \glossterm{Life}]

Choose one intact humanoid corpse within \rngclose range.
The target returns to life.
It must not have died due to old age.

The creature has 0 hit points when it returns to life.
It is cured of all \glossterm{vital damage} and other negative effects, but the body's shape is unchanged.
Any missing or irreparably damaged limbs or organs remain missing or damaged.
The creature may therefore die shortly after being resurrected if its body is excessively damaged.

Coming back from the dead is an ordeal.
All of the creature's action points and other daily abilities are expended when it returns to life.
In addition, its maximum action points are reduced by 1.
This penalty lasts for thirty days, or until the creature gains a level.
If this would reduce a creature's maximum action points below 0, the creature cannot be resurrected.

This ritual takes 24 hours to perform, and requires 16 action points from its participants.

\end{ability}




\parhead{Schools} Conjuration, Vivimancy

\parhead{Spell Lists} Nature
\end{spellsection}


\subsubsection{Subrituals}


\begin{ability}[\nth{7}]{Complete}
This subritual functions like the \ritual{resurrection} ritual, except that it does not have to target a fully intact corpse.
Instead, it targets a Diminuitive or larger piece of a humanoid corpse.
The target must have been part of the original creature's body at the time of death.
The resurrected creature's body is fully restored to its healthy state before dying, including regenerating all missing or damaged body parts.

This ritual takes 24 hours to perform, and requires 49 action points from its participants.
\end{ability}
\vspace{0.25em}


\begin{spellsection}{Retrieve Legacy}[4]


\begin{ability}{Retrieve Legacy}[\glossterm{Teleportation}]

Choose a willing creature within \rngclose range.
If the target's \glossterm{legacy item} is on the same plane and \glossterm{unattended}, it is teleported into the target's hand.

This ritual takes 24 hours to perform, and requires 16 action points from its ritual participants.

\end{ability}




\parhead{Schools} Conjuration, Divination

\parhead{Spell Lists} Arcane, Divine, Nature
\end{spellsection}


\begin{spellsection}{Scryward}[3]


\begin{ability}{Scryward}[\glossterm{Mystic}]

This ritual creates a ward against scrying in an \arealarge radius zone centered on your location.
All \glossterm{Scrying} effects fail to function in the area.
This effect is permanent.

This ritual takes 24 hour to perform, and requires 9 action points from its participants.

\end{ability}




\parhead{Schools} Abjuration

\parhead{Spell Lists} Arcane, Divine, Nature
\end{spellsection}


\begin{spellsection}{Seek Legacy}[2]


\begin{ability}{Seek Legacy}

Choose a willing creature within \rngclose range.
The target learns the precise distance and direction to their \glossterm{legacy item}, if it is on the same plane.

This ritual takes 24 hours to perform.

\end{ability}




\parhead{Schools} Divination

\parhead{Spell Lists} Arcane, Divine, Nature
\end{spellsection}


\begin{spellsection}{Sending}[4]


\begin{ability}{Sending}[\glossterm{Sustain} (standard)]

Choose a creature on the same plane as you.
You do not need \glossterm{line of sight} or \glossterm{line of effect} to the target.
However,  must specify your target with a precise mental image of its appearance.
The image does not have to be perfect, but it must unambiguously identify the target.
If you specify its appearance incorrectly, or if the target has changed its appearance, you may accidentally target a different creature, or the spell may simply fail.

You send the target a short verbal message.
The message must be twenty-five words or less, and speaking the message must not take longer than five rounds.

After the the target receives the message, it may reply with a message of the same length as long as the ritual's effect continues.
Once it speaks twenty-five words, or you stop sustaining the effect, the ritual is \glossterm{dismissed}.

This ritual takes one hour to perform.

\end{ability}




\parhead{Schools} Divination

\parhead{Spell Lists} Arcane, Divine, Nature
\end{spellsection}


\subsubsection{Subrituals}


\begin{ability}[\nth{6}]{Interplanar}
This subritual functions like the \ritual{sending} ritual, except that the target does not have to be on the same plane as you.
It gains the \glossterm{Planar} tag in addition to the tags from the \ritual{sending} ritual.
\end{ability}
\vspace{0.25em}


\begin{spellsection}{Soul Bind}[8]


\begin{ability}{Soul Bind}[\glossterm{Life}]

Choose one intact corpse within \rngclose range.
In addition, choose a gem you hold that is worth at least 1,000 gp.
The soul of the creature that the target corpse belongs to is imprisoned in the chosen gem.
This prevents the creature from being resurrected, and prevents the corpse from being used to create undead creatures, as long as the gem is intact.
A creature holding the gem may still resurrect or reanimate the creature.

This ritual takes one hour to perform.

\end{ability}




\parhead{Schools} Vivimancy

\parhead{Spell Lists} Arcane, Divine
\end{spellsection}


\begin{spellsection}{Telepathic Bond}[4]


\begin{ability}{Telepathic Bond}[\glossterm{Attune} (shared)]

Choose up to five willing ritual participants.
Each target can communicate mentally through telepathy with each other target.
This communication is instantaneous across any distance, but cannot reach across planes.

% Is this grammatically correct?
This effect lasts as long as you and each target \glossterm{attune} to it.
Each target must attune to this ritual independently.
If a target breaks its attunement, it stops being able to send and receive mental messages with other targets.
However, the effect continues as long as you attune to it.
If you stop attuning to it, the ritual is \glossterm{dismissed} as usual.

This ritual takes 24 hours to perform.

\end{ability}




\parhead{Schools} Divination

\parhead{Spell Lists} Arcane
\end{spellsection}


\subsubsection{Subrituals}


\begin{ability}[\nth{8}]{Interplanar}
This subritual functions like the \ritual{telepathic bond} ritual, except that each target can communicate telepathically even across different planes.
It gains the \glossterm{Planar} tag in addition to the tags from the \ritual{telepathic bond} ritual.
\end{ability}
\vspace{0.25em}


\begin{spellsection}{Water Breathing}[2]


\begin{ability}{Water Breathing}[\glossterm{Attune} (multiple)]

Choose a Medium or smaller willing creature within \rngclose range.
The target can breathe water as easily as a human breathes air, preventing it from drowning or suffocating underwater.
This effect lasts as long as you \glossterm{attune} to it.
If you use this ability multiple times, you can attune to it each time.

This ritual takes one minute to perform.

\end{ability}




\parhead{Schools} Transmutation

\parhead{Spell Lists} Arcane, Divine, Nature
\end{spellsection}


\normalsize
\setlength\parindent{1em}
\appendix

\chapter{Glossary}\label{Glossary}

\glossdef{ability} An ability is a generic term for any unusual property a creature has or any special actions it can take to cause particular effects.
Spells, feats, and the benefits from class \glossterm{archetypes} can all be called abilities.

\glossdef{ability tag} An ability tag describes the effects of an ability.
For details, see \pcref{Ability Tags}.

\glossdef{acid}[Acid] Acid is a type of damage is very effective against most objects.
For the Acid ability tag, see \pcref{Ability Tags}.

\glossdef{action phase} The action phase is the second of two \glossterm{phases} in a combat \glossterm{round}.
During the action phase, creatures can \glossterm{attack}, cast \glossterm{spells}, and take other major combat actions.

\glossdef{action point} Action points allow you to perform special abilities that you have access to.
There are two types of action points: \glossterm{recovery action points} and \glossterm{reserve action points}.
For details, see \pcref{Action Points}.

\glossdef{active cover} Active cover is a type of \glossterm{cover} provided by mobile obstacles that can block \glossterm{physical attacks}.
Physical attacks against creatures and objects with active cover suffer a 20\% miss chance.
For details, see \pcref{Active Cover}.

\glossdef{accuracy} The bonus added to an \glossterm{attack roll}.

\glossdef{Air} See \pcref{Ability Tags}.

\glossdef{alignment} Your alignment represents your general morality in broad terms.
For details, see \pcref{Alignment}.

\glossdef{ally}[allies] Many abilities affect allies.
An ally is any creature you designate who is willing to be considered an ally, not including yourself.

\glossdef{AP} An ability with the AP tag costs an \glossterm{action point} to use.
For details, see \pcref{Ability Tags}.

\glossdef{archetype}[archetypes] An archetype is a collection of related abilities from a particular class.
Each class has three archetypes that members of that class normally have.
For details, see \pcref{Archetypes}.

\glossdef{archetype rank} Each ability from an \glossterm{archetype} has a minimum rank required to gain the ability.
For details, see \pcref{Archetype Ranks}.

\glossdef{armor} Armor is a form of equipment that protects your body from harm.
There are two kinds of armor: \glossterm{body armor}, which you wear on your body, and \glossterm{shields}, which you wield in a hand.
For details, see \pcref{Armor}.

\glossdef{attack}[attacks] Anything that affects another creature in a potentially harmful way. There are two kinds of attacks: \glossterm{physical attacks} and \glossterm{magical} attacks.

\glossdef{attack result} An attack result is the total you get on an \glossterm{attack roll}, after taking to account any bonuses or penalties that apply to the roll.

\glossdef{attack roll}[attack rolls] A roll required to succeed with an attack.
To make an attack roll, roll 1d10 \add your \glossterm{accuracy} with the attack.
If the result of the attack roll equals or exceeds the target's \glossterm{defense}, the attack succeeds.
Some attacks, especially magical attacks, have effects even if the attack roll fails.

\glossdef{attended} An attended item is an item currently being held or carried by a creature.
Some abilities can only affect \glossterm{unattended} items.

\glossdef{attribute} A core representation of a character's capacity in a wide range of areas. There are six attributes: \glossterm{Strength}, \glossterm{Dexterity}, \glossterm{Constitution}, \glossterm{Intelligence}, \glossterm{Perception}, and \glossterm{Willpower}.

% TODO: also link ``attunement''
\glossdef{attune}[attunement] Some abilities last as long as you attune to them.
Attuning to an ability costs an \glossterm{action point} that you cannot recover as long as you maintain your attunement to that ability.
For details, see \pcref{Attunement}.

\glossdef{Attune} An ability with this \glossterm{ability tag} lasts as long as a creature attunes to it.
For details, see \pcref{Attunement}.

\glossdef{attuned} If you are attuned to an ability, you have invested an action point in it to maintain its effect.
For details, see \pcref{Attunement}.

\glossdef{Auditory} See \pcref{Ability Tags}.

\glossdef{augment}[augments] Many spells have augments.
Each augment on a spell has a level and an effect.
When casting a spell, you add the augment's level to the spell's level.
If you do, the spell gains the effect of the augment.
You can apply any number of augments to a spell in this way, increasing the spell's level for each augment.

\glossdef{Barrier} See \pcref{Ability Tags}.

\glossdef{base speed} Your base speed is the distance that you can usually move.
For details, see \pcref{Base Speed}.

\glossdef{blinded} A blinded creature cannot see. It automatically fails at actions which depend on vision, including simply seeing the locations of other objects and creatures (but see \pcref{Awareness}). It has a 50\% miss chance with \glossterm{strikes} and vision-related checks, even if it knows the location of its target. Finally, it is \defenseless.

\glossdef{blindsense} A creature with blindsense can sense its surroundings without any light, regardless of concealment or invisiblity.
It knows the location of everything around it, but it still takes normal \glossterm{miss chances} for concealment, invisibility, and so on.
It still needs line of effect to see its surroundings.
Blindsense always has a range, and grants no benefits beyond that range.

\glossdef{blindsight} A creature with blindsight can ``see'' its surroundings perfectly without any light, regardless of concealment or invisibility.
It still needs line of effect to see its surroundings.
Blindsight always has a range, and grants no benefits beyond that range.

\glossdef{block}[blocking] The \textit{block} ability allows you to try to prevent other creatures from entering an area.
For details, see \pcref{Block}.

\glossdef{bloodied} At or below half hit points. Bloodied creatures take a \minus4 penalty to Fortitude and Mental defense.

\glossdef{body armor} Body armor is a form of \glossterm{armor} that you wear on your body.
For details, see \pcref{Armor}.

\glossdef{broken} A broken object is damaged and unsuitable for use, though it retains its general structure and can be repaired.
For details, see \pcref{Broken Objects}.

An object that reaches 0 hit points is broken. If an object takes additional damage equal to its maximum hit points, it is \glossterm{destroyed}. A destroyed object cannot be repaired by any means.

\glossdef{burst} A burst is a type of area that an ability can have (see \pcref{Area Types}).
A burst ability has an immediate effect on all valid targets within an area.

\glossdef{cantrip}[cantrips] Every \glossterm{spell} can be cast as a cantrip.
A cantrip is a weaker version of the spell that does not cost an \glossterm{action point} to use.
For details, see \pcref{Cantrips}.

\glossdef{Chaos} See \pcref{Ability Tags}.

\glossdef{challenge rating} The challenge rating of a monster indicates its approximate strength within its level.
For details, see \pcref{Challenge Rating}.

\glossdef{charge}[charging] You can move up to a foe and attack it with the \textit{charge} ability.
For details, see \pcref{Charge}.

\glossdef{Charm} See \pcref{Ability Tags}.

\glossdef{character level} Your character level is your total level, including levels from all of your classes.
Whenever text refers to your ``level'', without specifying a particular kind of level, it means your character level.

\glossdef{charmed} A charmed creature is mentally influenced to like another creature.
It always sees the words and actions of the creature that charmed it in the most favorable way, as a close friend or trusted ally.
A charmed creature cannot be controlled like an automaton, but can be persuaded to take particular actions with the Persuasion skill (see \pcref{Persuasion}).
It treats the creature that charmed it as a friend (a \plus10 relationship modifier) for the purpose of Persuasion checks.

\glossdef{check}[checks] A check is a d10 roll required to accomplish an action that has a chance of failure that is not an attack.
If the result of your roll, including your \glossterm{check modifier}, is high enough, you succeed.
Otherwise, you fail.
For details, see \pcref{Checks}.

\glossdef{check modifier}[check modifiers] A check modifier is a number that you add to or subtract from your d10 roll when you make a \glossterm{check}.
For details, see \pcref{Checks}.

\glossdef{class}[classes] Your class represents your fundamental source of power and the type of abilities you have.
For example, barbarians draw power from the primal energy found deep within all living things, while clerics draw power from their worship of mighty deities.
For details, see \pcref{Classes}.

\glossdef{class skill}[class skills] A class skill is a skill that a class is particularly good at using.
Each class has a specific set of class skills given in its description.
Normally, it costs 3 \glossterm{skill points} to make a skill \glossterm{mastered}.
It only costs 2 skill points to make a class skill \glossterm{mastered}.
For details, see \pcref{Skill Training}.

\glossdef{climb speed} A creature with a climb speed can climb as easily as a human walks on land.
The effects of a climb speed are described at \pcref{Climb Speed}.

\glossdef{cold}[Cold] A kind of \glossterm{energy}. For the Cold ability tag, see \pcref{Ability Tags}.

\glossdef{common language}[common languages] Common languages are languages that are widely spoken.
They are described in \tref{Common Languages}.

\glossdef{Compulsion}[Compulsions] See \pcref{Ability Tags}.

\glossdef{concealment} Concealment represents effects which make a target harder to see, such as shadowy lighting.
A creature or object with concealment from you gains a \plus2 bonus to Armor defense.
For details, see \pcref{Concealment}.

\glossdef{concentration} Some abilities, such as spells, require concentration to use successfully.
If your concentration is disrupted, the ability may fail.
Your bonus with concentration checks is normally equal to the higher of your level and your Willpower.
For details, see \pcref{Concentration}.

\glossdef{condition}[conditions] A condition is a negative effect on a creature.
Conditions last until they are removed, such as by the \textit{recover} ability (see \pcref{Recover}).

\glossdef{confused} A confused creature is unable to independently control its actions. \confusionexplanation

\glossdef{cover} Cover represents any obstacle that physically prevents you from striking your target, such as a tree or intervening creature.
There are three kinds of cover: \glossterm{active cover}, \glossterm{passive cover}, and \glossterm{total cover}.
For details, see \pcref{Cover}.

\glossdef{Creation} See \pcref{Ability Tags}.

\glossdef{crouching} A crouching creature is ducking down instead of standing normally.
\glossterm{Melee attacks} against it gain a \plus2 bonus to \glossterm{accuracy}, while physical ranged attacks against it take a \minus2 penalty to accuracy.
In addition, it takes a \minus2 penalty to accuracy with melee attacks and moves at half speed.

\glossdef{Curse} See \pcref{Ability Tags}.

\glossdef{critical failure} When you make a check, if your result failed to beat the DR by 10 or more, you get a critical failure.
Some abilities have special effects on critical failures.

\glossdef{critical hit}[critical hits] When you make an attack, if your result beat the target's defense by 10 or more, you get a critical hit.
Unless otherwise noted, damaging attacks deal double damage on a critical hit.
Some abilities have special effects on critical hits.

\glossdef{critical success} When you make a check, if your result beat the DR by 10 or more, you get a critical success.
Some abilities have special effects on critical successes.

\glossdef{damage} Some attacks deal damage to you when they hit.
When you take damage, you reduce your \glossterm{hit points} by that amount.
If you have no hit points remaining, you may take that damage as \glossterm{vital damage} instead, which represents potentially life-threatening injuries.
For details, see \pcref{Damage}.

\glossdef{damage reduction} Damage reduction allows you to ignore a certain amount of incoming damage.
If you have damage reduction, you ignore the first points of damage you would take each \glossterm{round}, up to a maximum equal to the value of your damage reduction.
Any additional damage is dealt normally.
% TODO: clarify how to tell whether a particular attack dealt damage to you if you have damage reduction
Once it reduces that much damage, it stops functioning until the start of the next round.

Most sources of damage reduction only apply against a specific type of attack.
For example, a barbarian's damage reduction only applies against damage dealt by \glossterm{physical attacks}.
If an attack deals multiple types of damage, you must have damage reduction against every type of damage dealt.
For example, damage reduction against piercing damage would not help if you are struck by a morningstar, since it deals both bludgeoning and piercing damage.

\glossdef{darkvision} A creature with darkvision can see in the dark clearly up to a given range.
Beyond that, it can see dimly, treating areas of darkness as \glossterm{shadowy illumination}.
Darkvision does not function if a creature is in a brightly lit area, and does not resume functioning until the end of the next round after the creature leaves the brightly lit area.

\glossdef{dazed} A dazed creature cannot act during the \glossterm{action phase}.
It can take its normal actions during the \glossterm{delayed action phase}.
In addition, the creature takes a \minus2 penalty to \glossterm{defenses}.

\glossdef{dazzled} A dazzled creature has difficulty seeing.
It loses any special vision abilities it has, such as \glossterm{darkvision} or \glossterm{low-light vision}.
In addition, it takes a \minus2 penalty to \glossterm{accuracy} and visual Awareness checks (see \pcref{Awareness}).

\glossdef{dead} A dead creature's soul leaves its body. Dead creatures cannot benefit from normal or magical healing, but they can be restored to life via magic (see \pcref{Resurrecting the Dead}). A dead body decays normally unless magically preserved.

\glossdef{deafened} A deafened creature cannot hear. It automatically fails at actions which depend on hearing. In addition, it has a 20\% failure chance when casting any spell with verbal components.

\glossdef{decelerated} A decelerated creature moves at one quarter speed and takes a \minus4 penalty to Reflex defense.

\glossdef{defenseless} A defenseless creature is unable to defend itself in melee combat.
\glossterm{Melee attacks} against it gain a \plus2 bonus to accuracy.
Any creature not capable of using a weapon or shield to defend itself, such as most unarmed creatures, is defenseless.

\glossdef{Death} See \pcref{Ability Tags}.

\glossdef{defeated} A creature is defeated if it dies, surrenders or is incapacitated for an extended period of time (such as by being knocked unconscious).
Some abilities, such as a ranger's \textit{quarry} ability (see \pcref{Quarry}), last until their target is defeated.
If there is ambiguity about whether a surrendering or seemingly incapacitated enemy still poses a threat, you choose whether you consider the enemy to be defeated.

\glossdef{defense}[defenses] A defense is a static number which represents how difficult you are to affect with attacks.
There are four defenses: Armor, Fortitude, Reflex, and Mental.
For details, see \pcref{Defenses}.

\glossdef{defensive casting} When you use a spell or ritual, you can choose to balance concentrating on the ability and defending yourself.
This is called defensive casting.
If you do, there is a chance that you may \glossterm{miscast} the spell.
For details, see \pcref{Focused and Defensive Casting}.

\glossdef{delayed action phase} The delayed action phase is a \glossterm{phase} that occurs after the \glossterm{action phase}.
It is not always necessary, because most actions are not delayed.
For details, see \pcref{The Delayed Action Phase}.

\glossdef{destroyed} A destroyed object has been damaged to the point where it is completely beyond repair.
For details, see \pcref{Destroyed Objects}.

\glossdef{Detection} See \pcref{Ability Tags}.

\glossdef{die increment}[die increments] A die increment is a single increase or decrease of the die size of a pool of dice.
For example, a 1d8 that is increased by one die increment becomes a 1d10 die.
Similarly, a 2d6 dice pool that is decreased by one die increment also becomes a 1d10 die.
For details, see \pcref{Die Increments}.

\glossdef{difficult terrain} Difficult terrain costs double the normal movement cost to move out of.
For details, see \pcref{Difficult Terrain}.

\glossdef{difficulty rating} The difficulty rating of a \glossterm{check} is the check result required to succeed.
In general, attacks are rolled to beat \glossterm{defenses}, and checks are rolled to beat difficulty ratings.

\glossdef{DR} A shorthand for \glossterm{difficulty rating}.

\glossdef{dirty trick} You can use the \textit{dirty trick} ability to impair a foe by using your environment.
For details, see \pcref{Dirty Trick}.

\glossdef{disarm} You can use the \textit{disarm} ability to strike items held or worn by a creature.
For details, see \pcref{Disarm}.

\glossdef{disease} An affliction of the body, causing a steady deterioration over time. For the Disease ability tag, see \pcref{Ability Tags}.

\glossdef{dismiss}[dismissed] When you dismiss an ability, it ends, and all of its lingering effects are removed.
Unless otherwise noted, all abilities with a \glossterm{duration} can be dismissed.

\glossdef{disoriented} During each movement phase, a disoriented creature is compelled to move its full speed in a random direction.
It moves as far as it can, but will not sprint or take similar strenuous actions to increase its speed.

\glossdef{dominated} A dominated creature is mentally compelled to obey another creature.
It obeys the commands of the creature of the dominated it unquestioningly, as an automaton.
If it does not understand the language of the creature that dominated it, it still attempts to obey as much as possible, and simple commmands (such as ``attack'' or ``follow'') can usually be communicated successfully.

% Does this need more description?
\glossdef{duration} An ability's duration determines how long that ability lasts.

\glossdef{dying} A dying creature is unconscious and near death. See \pcref{Dying}.

\glossdef{Earth} See \pcref{Ability Tags}.

\glossdef{effect} The result of using an \glossterm{ability}.

\glossdef{electricity}[Electricity] A kind of \glossterm{energy}. For the Electricity ability tag, see \pcref{Ability Tags}.

\glossdef{emanation} An emanation is a type of area that an ability can have (see \pcref{Area Types}).
An emanation ability has effects within an area for the \glossterm{duration} of the ability.
It emanates from a specific creature or object, rather than a location.
If that creature or object moves, the emanation moves with it.

\glossdef{Emotion} See \pcref{Ability Tags}.

\glossdef{encumbrance} Your encumbrance is a value that represents how much you are burdened by armor and weight.
For details, see \pcref{Encumbrance}.

\glossdef{energy}[energy damage] There are four types of energy: cold, electricity, fire, and sonic. Energy effects often deal damage.

% \glossdef{Enchantment} Enchantment spells affect the minds of others, influencing or controlling their behavior or mental capabilities. Almost all enchantment spells are \glossterm{Mind} spells, and many of them are \glossterm{Subtle} as well.

\glossdef{exhausted} An exhausted creature moves at half speed and takes a \minus4 penalty to \glossterm{defenses}
This does not stack with the \glossterm{fatigued} effect.

\glossdef{exotic weapon}[exotic weapons] A rare few weapons are considered exotic weapons.
They are unusually difficult to wield, and even being \glossterm{proficient} with the associated \glossterm{weapon group} does not grant you the ability to use an exotic ewapon.
To use an exotic weapon, you must take the Martial Training feat (see \featpref{Martial Training}).

\glossdef{explode}[explodes] When you roll a 10 on an \glossterm{attack roll}, the die can explode.
If it does, you roll it again and add the two results together to determine the total.
For details, see \pcref{Exploding Attacks}.

\glossdef{falling damage} For every 10 feet you fall, you take 1d6 bludgeoning damage, to a maximum of 20d6 damage.
If you control your fall with a successful Acrobatics or Jump check, you can reduce the falling damage you take (see \pcref{Acrobatics}, and \pcref{Jump}).

\glossdef{fascinated} A fascinated creature can take no actions. It remains in place, giving its total attention to some object, creature, or effect. It takes a \minus4 penalty to skill checks made as reactions, such as Awareness checks.
If the creature notices any threat against it, such as an approaching enemy, it is no longer fascinated.

\glossdef{fast healing} A creature with fast healing automatically heals hit points at the end of every \glossterm{action phase}.

\glossdef{fatigued} A fatigued creature moves at half speed and takes a \minus2 penalty to \glossterm{defenses}.
This does not stack with the \glossterm{exhausted} effect.

\glossdef{Fear} See \pcref{Ability Tags}.

\glossdef{feint} You can use the \textit{feint} ability to trick a creature into lowering its defenses.
For details, see \pcref{Feint}.

\glossdef{fire}[Fire] A kind of \glossterm{energy}. For the Fire ability tag, see \pcref{Ability Tags}.

\glossdef{Flesh} See \pcref{Ability Tags}.

\glossdef{fly speed} A creature with a fly speed has the ability to fly through the air.
Its speed is the distance it covers in a single \glossterm{move action}.
For details, see \pcref{Flying}.

\glossdef{focused casting} When you use a spell or ritual, you can choose to focus exclusively on completing the effect.
This is called focused casting.
While you do, you suffer a \minus4 penalty to defenses.
For details, see \pcref{Focused and Defensive Casting}.

\glossdef{Fog} See \pcref{Ability Tags}.

\glossdef{follow} The \textit{follow} ability allows you to follow another creature to match their movements during the \glossterm{movement phase}.
For details, see \pcref{Follow}.

\glossdef{Force} See \pcref{Ability Tags}.

\glossdef{free action}[free actions] Each round, you can any number of free actions.
Free actions can be taken in any phase.
For details, see \pcref{Free Actions}.

\glossdef{frightened} A frightened creature takes a \minus4 penalty to \glossterm{defenses} as long as it is within \rngmed range of the source of its fear.
This does not stack with the \glossterm{shaken} effect.

If the source of a frightened creature's fear is a creature and is \glossterm{defeated}, this effect is broken.

\glossdef{glide speed} A creature with a glide speed can glide through the air.
It cannot fly upwards, but it can travel forward while it descends, and it descends at a significantly reduced rate.
For details, see \pcref{Gliding}.

\glossdef{good}[Good] One of the four \glossterm{alignment} components. For the Good ability tag, see \pcref{Ability Tags}.

\glossdef{grapple} You can use the \textit{grapple} ability to physically restrain a creature.
For details, see \pcref{Grapple}.

\glossdef{grappled} A grappled creature is wrestling or in some other form of hand-to-hand struggle with at least one other creature.
While grappled, you suffer certain penalties and restrictions, as described below.
\begin{itemize}
    \item You must use a free hand (or equivalent limbs) to grapple, preventing you from taking any actions which would require having two free hands.
        If you cannot free a hand, you suffer a \minus10 penalty to accuracy on all \glossterm{physical attacks} until you have a free hand.
    \item You are \defenseless against creatures who are not grappled by you.
    \item You take a \minus4 penalty to \glossterm{accuracy} with weapons that are not light, since they are too large and cumbersome to be used effectively in a grapple.
    \item Spellcasting is extremely difficult. You cannot cast spells with \glossterm{somatic components}.
        Casting a spell without somatic components requires a \glossterm{concentration} check with a DR equal to 20 \add double spell level.
    \item You cannot normally move from your location. 
\end{itemize}

Other than the restrictions listed above, you can act normally. You can also try to move the grapple, escape the grapple, or pin your opponent. For details, see \pcref{Grapple Actions}.

\glossdef{hardness} An object's hardness indicates how durable it is.
When a creature or object with hardness would take damage, if the hardness of the attacking object or creature lower than the hardness of the defender, the attacking object or creature takes the damage instead.
For details, see \pcref{Hardness}.

\glossdef{heavy undergrowth} A space overrun with thick bushes, vines, and similar natural obstacles has heavy undergrowth.
Heavy undergrowth quadruples the movement cost required to move out of each square and provides \glossterm{concealment}.
In addition, using the \textit{charge} and \textit{sprint} actions is impossible in heavy undergrowth (see \pcref{Movement Abilities}).

\glossdef{heavy weapon} A heavy weapon is a type of \glossterm{weapon} that requires two hands to wield properly.
For details, see \pcref{Weapon Usage Classes}.

\glossdef{helpless} A helpless creature is completely at an opponent's mercy.
Its Dexterity is treated as \minus10.
Paralyzed, bound, and unconscious creatures are helpless.
Any \glossterm{physical attack} against a helpless creature automatically \glossterm{explodes} on the first die.

\glossdef{hidden task}[hidden tasks] Any checks for a hidden \glossterm{task} should be rolled secretly by the GM.\@
You should not know the result of your check, or even that a check was made.
For details, see \pcref{Hidden Tasks}.

\glossdef{hit point}[hit points] Your hit points measure how hard you are to kill.
When you take damage, you subtract that damage from your hit points.
For details, see \pcref{Hit Points}.

\glossdef{hunting party} A hunting party is the group of allies affected by a ranger's \textit{quarry} ability (see \pcref{Quarry}).

\glossdef{ignited} An ignited creature has been set on fire.
It takes 1d6 fire damage at the end of each \glossterm{action phase} and takes a \minus2 penalty to \glossterm{defenses}.

\glossdef{immobilized} An immobilized creature can't move out of the space it was in when it became immobilized.
Immobilized flying creatures that have the ability to hover can maintain their initial altitude.
All other flying creatures subjected to this condition descend at a rate of 20 feet per round until they reach the ground, taking no falling damage.

\glossdef{improvised weapon}[improvised weapons] An improvised weapon is an object which could conceivably be used as a weapon, but which was not designed for that purpose.
Common examples include doors and wine bottles.
For details, see \pcref{Improvised Weapons}.

\glossdef{incorporeal} An incorporeal creature does not have a body.
It has no Strength or Constitution attributes.
It cannot take any action that requires having a body, and is immune to all such effects.
This includes suffering critical hits, moving objects, grappling, setting off pressure traps, and so on.

An incorporeal creature is immune to all nonmagical effects.
Even magical effects, including spells and attacks with magic weapons, have a 50\% chance to fail.

An incorporeal creature can enter or pass through solid objects, but it must remain adjacent to the object's exterior at all times.
If it is completely inside an object, it cannot see out or attack.
It can fight while partially inside an object, which grants it passive \glossterm{cover} and allows it to attack and see normallly.

\glossdef{initiative} When multiple creatures take mutually impossible actions simultaneously, such as racing to be the first one to a door, they must roll initiative checks to determine who completes the action first.
For details, see \pcref{Initiative}.

\glossdef{insight points} Insight points can be spent to learn additional abilities.
You gain access to insight points at 2nd level.
For details, see \pcref{Insight Points}.

\glossdef{invisible} An invisible creature or object cannot be seen. Creatures unable to see an invisible creature are \defenseless against its attacks. Attackers suffer a 50\% miss chance even if they know the location of the invisible creature. See \pcref{Awareness}, and \pcref{Stealth}, for how to identify invisible creatures.

\glossdef{item slot}[item slots] Item slots are a resource that you can use to \glossterm{attune} to items in place of \glossterm{action points}.
If you use an item slot to attune to an item, you gain its effects without reducing your available action points.
For details, see \pcref{Item Slots}.

\glossdef{key attribute} The key attribute for a skill is the attribute associated with that skill.
For example, Climb is a Strength-based skill.
Some skills, such as Persuasion, do not have a key attribute.

\glossdef{legacy item} A legacy item is an item magically bonded to its bearer.
As its bearer gains levels, it increases in power as well.
For details, see \pcref{Legacy Items}.

\glossdef{legend point}[legend points] Legend points can be used to achieve extraordinary results on a roll you make, or to force a foe to accept a terrible failure.
You gain access to legend points at 3rd level.
For details, see \pcref{Legend Points}.

\glossdef{Life} See \pcref{Ability Tags}.

\glossdef{Light} See \pcref{Ability Tags}.

\glossdef{light undergrowth} A space with passable bushes, vines, and similar natural obstacles has light undergrowth.
Light undergrowth is \glossterm{difficult terrain} and provides \glossterm{concealment}.

\glossdef{light weapon}[light weapons] A light weapon is a type of \glossterm{weapon} that is relatively small and easy to use.
You can use your Dexterity to determine your \glossterm{accuracy} when making \glossterm{physical attacks} with light weapons.
For details, see \pcref{Weapon Usage Classes}.

\glossdef{line of effect} You cannot target something that you do not have line of effect to.
Line of sight is blocked by solid obstacles.
For details, see \pcref{Line of Effect}.

\glossdef{line of sight} You cannot target something that you do not have line of sight to.
Line of sight is blocked by any obstacle that blocks sight.
For details, see \pcref{Line of Sight}.

\glossdef{long rest} A long rest represents eight hours of relaxation or sleep.
It allows you to recover all of your spent \glossterm{action points} and \glossterm{legend points}.
For details, see \pcref{Long Rest}.

\glossdef{low-light vision} A creature with low-light vision can see more clearly in conditions of dim light.
It treats sources of light as if they had double their normal illumination range.
In addition, the creature treats environments with ambient dim light, such as a moonlit night, as if they were brightly lit when doing so is beneficial for it.

\glossdef{magic bonus} Some abilities provide a magic bonus instead of a regular bonus.
Magic bonuses function like normal bonuses except that they do not stack with each other, even if the magic bonuses come from different sources.
For details, see \pcref{Stacking Rules}.

\glossdef{magic source}[magic sources] A magic source defines where a creature's spells and rituals come from.
There are four magic sources: arcane, divine, nature, and pact.
Mages cast arcane spells, clerics cast divine spells, druids cast nature spells, and warlocks cast pact spells.

\glossdef{magical} A magical ability is an ability that has no physical explanation.
Examples include spells, a medusa's petrifying gaze, and a cleric's domain invocations.
For details, see \pcref{Magical Abilities}.

\glossdef{maneuverability} While flying, your maneuverability determines how easily you can change directions and perform aerial feats.
There are three types of manueverability: good, normal, and poor.
For details, see \pcref{Maneuverability}.

\glossdef{Manifestation} See \pcref{Ability Tags}.

\glossdef{mastered} If you have \glossterm{mastered} a skill, you have learned to use it to its maximum potential.
Your modifier with a mastered skill is equal to 3 \add either the skill's key attribute (if any) or your level, whichever is higher.
For details, see \pcref{Skill Training}.

% TODO: ``medium'' is an annoying name
\glossdef{medium weapon} A medium weapon is a type of \glossterm{weapon} that can be wielded in either one or two hands.
For details, see \pcref{Weapon Usage Classes}.

\glossdef{melee attack}[Melee attack] A melee attack is a \glossterm{physical attack} using your body or a weapon that does not leave your grasp.
You can only make melee attacks against targets within your \glossterm{reach}.

\glossdef{melee weapon}[Melee weapon] A melee weapon is a weapon designed for \glossterm{melee attacks}.

\glossdef{Mind} See \pcref{Ability Tags}.

\glossdef{minor action}[minor actions] Each round, you can take a single minor action in addition to your other actions that round.
Minor actions can be taken in either the \glossterm{action phase} or the \glossterm{delayed action phase}.
They are declared and resolved simultaneously with any other actions you take during that phase.
For details, see \pcref{Minor Actions}.

\glossdef{miscast}[miscasting] If your \glossterm{concentration} is disrupted while casting a \glossterm{spell}, you miscast the spell instead.
The spell does not have its normal effect.
Instead, a damaging \glossterm{miscast backlash} occurs.

\glossdef{miscast backlash} When you \glossterm{miscast} a spell, you deal damage to yourself and creatures around you.
For details, see \pcref{Miscasting}.

\glossdef{miss chance}[miss chances] If you have a miss chance with an \glossterm{attack}, you have a random chance to miss with the attack regardless of the result of your attack roll.
If you have multiple miss chances, only the highest one applies.

\glossdef{move} When you move, you usually travel a distance equal to your speed.
See \pcref{Movement and Positioning}, for details.
For specific actions that involve movement, see \glossterm{move action}.

\glossdef{move action}[move actions] A move action is a minor action that requires motion, such as drawing a sword.
You can take move actions during the \glossterm{movement phase}.
For the act of moving from one place to another, see \glossterm{move}.

\glossdef{movement mode}[movement modes] A movement mode is a method of moving from one location to another.
The most common mode is a land speed.
For details, see \pcref{Movement Modes}.

\glossdef{movement phase} The movement phase is the first of two \glossterm{phases} in a combat \glossterm{round}.
During the movement phase, creatures can \glossterm{move} and take \glossterm{move actions}.
The movement phase is followed by the \glossterm{action phase}.

\glossdef{mundane} Most abilities are considered mundane abilities.
Mundane abilities have a tangible component and some form of natural explanation.
Examples include \glossterm{strikes}, a dragon's breath weapon, and a barbarian's rage.
Unless otherwise indicated, all abilities are mundane in nature.

\glossdef{Mystic} See \pcref{Ability Tags}.

\glossdef{mystic sphere} A mystic sphere is a collection of thematically related magical effects that includes both \glossterm{spells} and \glossterm{rituals}.
For details, see \pcref{Mystic Spheres}.

\glossdef{natural weapon} A natural weapon is a \glossterm{weapon} that is part of a creature's body.
For details, see \pcref{Natural Weapons}.

\glossdef{nauseated} A nauseated creature takes a \minus4 penalty to \glossterm{accuracy} and Fortitude defense.
This does not stack with the \glossterm{sickened} effect.

\glossdef{opposed alignment} Each \glossterm{alignment} has an opposed alignment that is antethical to its principles and goals.
Good and Evil are opposed alignments, and Chaos and Law are opposed alignments.
For details, see \pcref{Alignment}.

\glossdef{overkill damage} If you take damage in excess of your \glossterm{bloodied} hit point total in a single round, the excess damage is dealt as \glossterm{vital damage}.
This excess damage is called overkill damage.
For details, see \pcref{Overkill Damage}.

\glossdef{overrun} An overrun is a special movement that allows you to move directly through creatures.
For details, see \pcref{Overrun}.

\glossdef{overwhelm penalties} An \glossterm{overwhelmed} creature suffers overwhelm penalties equal to half the combined \glossterm{overwhelm value} of all creatures threatening it.
Overwhelm penalties apply to Armor and Reflex defenses.
For details, see \pcref{Overwhelm}.

\glossdef{overwhelm resistance} A creature with \glossterm{overwhelm resistance} reduces the effective \glossterm{overwhelm value} of creatures threatening it.
This can reduce or remove their \glossterm{overwhelm penalties}.
For details, see \pcref{Overwhelm Resistance}.

\glossdef{overwhelm value}[overwhelm values] Your overwhelm value determines how much you contribute to \glossterm{overwhelm penalties} against creature you \glossterm{threaten}.
Most Small and Medium creatures have an overwhelm value of 1.
For details, see \pcref{Overwhelm Value}.

\glossdef{overwhelmed} A creature is overwhelmed if the combined \glossterm{overwhelm value} of all creatures that \glossterm{threaten} it is at least 2.
An overwhelmed creature suffers \glossterm{overwhelm penalties}.
For details, see \pcref{Overwhelm}.

\glossdef{outsider} An outsider is a type of creature.
Outsiders are composed of planar material from a plane other than the Material Plane.

\glossdef{panicked} A panicked creature must flee from the source of its fear by any means necessary if it is within \rngmed range of the source of its fear.
If unable to flee, it must do nothing other than use the \textit{total defense} ability every round (see \pcref{Total Defense}).

If the source of a panicked creature's fear is a creature and is \glossterm{defeated}, this effect is broken.

\glossdef{paralyzed} A paralyzed creature is unable to take physical actions. It has effective Dexterity and Strength scores of \minus10 and is \helpless, but can take purely mental actions. This can cause flying creatures to crash, swimming creatures to drown, and so on. A creature can move through a space occupied by a paralyzed creature -- ally or otherwise. Each square occupied by a paralyzed creature, however, counts as 2 squares.

\glossdef{passive cover} Passive cover is a type of \glossterm{cover} provided by immobile obstacles that can block \glossterm{physical attacks}.
Creatures and objects with passive cover from you gain a \plus2 bonus to Armor defense.
For details, see \pcref{Passive Cover}.

\glossdef{petrified} A petrified creature has been turned to stone. It is neither alive nor dead, but is unconscious and unable to take actions, and its body is an inanimate statue. If the statue is broken or damaged before the creature is restored to its original state, the creature has equivalent damage or deformities.

\glossdef{phase}[phases] A phase is part of the combat \glossterm{round}.
There are two phases: the \glossterm{movement phase} and the \glossterm{action phase}.
A phase does not represent a fixed span of time.
It is an abstract concept designed to represent a variety of actions that all take place nearly simultaneously.

\glossdef{Physical} See \pcref{Ability Tags}.

\glossdef{physical attack} A physical attack is an \glossterm{attack} made with a creature's body.
The most common type of physical attack is a \glossterm{strike}, but there are other physical attacks, such as the \textit{trip} ability (see \pcref{Trip}).
Most physical attacks target Armor defense.

\glossdef{pinned} A pinned creature is held completely immobile in a grapple.
The only physical actions it can make are to escape the grapple (see \pcref{Grappling}).
Like a \glossterm{helpless} creature, its Dexterity is treated as \minus10.

\glossdef{Planar} See \pcref{Ability Tags}.

\glossdef{planar rift}[planar rifts] A planar rift is a location where the boundaries between planes are unusually thin.
Planar rifts can be used to travel between planes using the appropriate rituals.
For details, see \label{Planar Rifts}.

\glossdef{plane} A plane is a distinct realm of existence.
Except for the connections between planes through \glossterm{planar rifts}, each plane is effectively an isolated universe, and different planes can obey different fundamental laws.
For details, see \pcref{Planes}.

\glossdef{point of origin}[points of origin] A point of origin is the grid intersection, creature, or object that an area originates from.
For details, see \pcref{Area}.

% TODO: better short description
\glossdef{poison}[poisons] For a description of poisons and how they work, see \pcref{Poisons}.

\glossdef{Poison} See \pcref{Ability Tags}.

\glossdef{Positive} See \pcref{Ability Tags}.

% This seems unlikely to remain accurate once poisons are fixed
\glossdef{potency} The potency of a poison, disease, or similar effect determines its \glossterm{accuracy}.

\glossdef{power} The power of an \glossterm{ability} represents how strong the ability is.
This determines the ability's \glossterm{standard damage}, and may determine other effects of the ability.
Your power with an ability depends on whether the ability is \glossterm{magical} or \glossterm{mundane}.
Your power with magical abilities is normally equal to the higher of your level and Willpower, and your power with mundane abilities is normally equal to the higher of your level and Strength.
For details, see \pcref{Power}.

\glossdef{proficient}[proficiency] A creature can be proficient with weapons and armor.
If you try to attack with a weapon you are not proficient with, you take a \minus2 penalty to accuracy (see \pcref{Weapon Proficiency}).
If you try to use armor you are not proficient with, it is less effective and your \glossterm{accuracy} with \glossterm{physical attacks} is reduced (see \pcref{Armor Proficiency}).

\glossdef{projectile} A projectile is an object fired from a weapon at a target.
Arrows and bolts are projectiles.

\glossdef{projectile weapon} A projectile weapon is a weapon designed to fire \glossterm{projectiles}.
For details about how to attack with projectile weapons, see \pcref{Projectile Strike}.

\glossdef{prone} A prone creature is lying on the ground, rather than standing normally.
Melee \glossterm{strikes} against it gain a \plus2 bonus to \glossterm{accuracy}, while ranged \glossterm{strikes} against it take a \minus2 penalty to accuracy.
In addition, it takes a \minus2 penalty to accuracy with melee \glossterm{strikes} and is unable to move until it stands up.
A creature can stand up from being prone during the movement phase.
This generally requires one free hand.

\glossdef{range} The range of an ability determines how far away it can be used.
You can't use abilities on a target outside of the ability's range.

\glossdef{range increment}[range increments] Physical ranged attacks often have a specific range increment.
A range increment is always measured in feet.
You take a \minus1 penalty to accuracy with the ranged attack for each full range increment between you and your target.

\glossdef{rare language}[rare languages] Rare languages are languages that are only spoken by rare or distant creatures or cultures.
They are described in \tref{Rare Languages}.

\glossdef{reach} Your reach is how far away from your body you can make melee attacks.
A typical Medium creature has a five-foot reach.

\glossdef{recovery action point} A recovery action point is a type of \glossterm{action point}.
You can regain spent recovery action points with a \glossterm{short rest}.
Recovery action points are also required to \glossterm{attune} to abilities.
For details, see \pcref{Action Points}, and \pcref{Attunement}.

\glossdef{reserve action point} A reserve action point is a type of \glossterm{action point}.
You can regain spent reserve action points with a \glossterm{long rest}.
For detais, see \pcref{Action Points}.

\glossdef{ritual}[rituals] A ritual is a discrete \glossterm{magical} ability with esoteric effects.
For details, see \pcref{Rituals}.

\glossdef{round}[rounds] Combat takes place in a series of rounds, which represent about six seconds of action.
Rounds are divided into two \glossterm{phases}: the \glossterm{movement phase}, and the \glossterm{action phase}.

\glossdef{scent} A creature with the scent ability has an unusually good sense of smell.
It gains a \plus10 bonus to scent-based Awareness checks (see \pcref{Senses}).

\glossdef{Scrying} See \pcref{Ability Tags}.

\glossdef{scrying sensor} A scrying sensor is a magical construct created by some magical abilities.
Scrying sensors are Fine objects resembling a human eye in size and shape, though they are \\glossterm<invisible>.
Scrying sensors typically float in a fixed position in the air.
They cannot normally be moved by external forces without destroying the sensor.
Unless otherwise specified, a scrying sensor's visual acuity is the same as that of a normal human, giving it a \plus0 bonus to Awareness and similar checks.

\glossdef{Sensation} See \pcref{Ability Tags}.

\glossdef{shadowy illumination} In an area with shadowy illumination, creatures can see dimly.
Creatures within this area have \concealment, which can allow them to make Stealth checks to hide (see \pcref{Stealth}).

\glossdef{shaken} A shaken creature takes a \minus2 penalty to \glossterm{defenses} as it is within \rngclose range of the source of its fear.
This does not stack with the \glossterm{frightened} effect.

If the source of a shaken creature's fear is a creature and is \glossterm{defeated}, this effect is broken.

\glossdef{Shaping} See \pcref{Ability Tags}.

\glossdef{shield}[shields] Shields are a form of \glossterm{armor} that you wield in a hand to protect you from harm.
For details, see \pcref{Armor}.

\glossdef{Shielding} See \pcref{Ability Tags}.

\glossdef{short rest}[short rests] A short rest represents five minutes of relaxation.
It allows you to recover a small amount of \glossterm{hit points} and some of your spent \glossterm{action points}.
For details, see \pcref{Short Rest}.

\glossdef{shove} You can use the \textit{shove} ability to forcibly move a creature.
For details, see \pcref{Shove}.

\glossdef{sickened} A sickened creature takes a \minus2 penalty to \glossterm{accuracy} and Fortitude defense.
This does not stack with the \glossterm{nauseated} effect.

\glossdef{size category} A creature's size category indicates how large it is.
There are nine size categories, from smallest to largest: Fine, Diminuitive, Tiny, Small, Medium, Large, Huge, Gargantuan, Colossal.
For details, see \pcref{Size in Combat}.

\glossdef{Sizing} See \pcref{Ability Tags}.

\glossdef{skill}[skills] A skill represents your degree of talent with a particular non-combat aspect of the world.
For example, the Climb skill represents how skilled you are at climbing.
For details, see \pcref{Skills}.

\glossdef{skill point}[skill points] You can spend skill points to gain training in skills (see \pcref{Skill Training}).
You gain skill points from your class, from having a high Intelligence, and from taking penalties to your starting attributes (see \pcref{Impaired Attributes}).
For details, see \pcref{Skill Points}.

\glossdef{slowed} A slowed creature moves at half speed and takes a \minus2 penalty to Reflex defense.

\glossdef{somatic components}[somatic] Somatic components are hand motions required to cast arcane and pact spells.
For details, see \pcref{Casting Components}.

\glossdef{Sonic} See \pcref{Ability Tags}.

\glossdef{space} Your space is the area that your physical body occupies.
For convenience, your space is measured in five-foot \glossterm{squares}.
Small and Medium creatures occupy space equal to a single five-foot square.

\glossdef{Speech} See \pcref{Ability Tags}.

\glossdef{speed} Your speed represents the number of feet you can move with a single movement (see \pcref{The Movement Phase}).

\glossdef{spell}[spells] A spell is a disrete \glossterm{magical} ability with combat-relevant effects.
For details, see \pcref{Spells}.

\glossdef{spell list} The list of spells you can cast from a particular \glossterm{spell source}.
Each spell source has a specific spell list which is described at \pcref{Spells}.
Most characters with the same spell sources have the same spell lists.
However, some effects, such as a cleric's domains, can add spells to a character's individual spell list.

\glossdef{sprint} You can use the \textit{sprint} ability to move faster for a short period of time.
For details, see \pcref{Sprint}.

\glossdef{square}[squares] A square represents a single 5-ft.\ by 5-ft.\ space.
Many areas are measured in squares for convenience.

\glossdef{squeezing} A squeezing creature is trying to move though an area too small for it to fight in normally.
While squeezing, a creature moves at half speed and takes a \minus2 penalty to \glossterm{accuracy} with \glossterm{strikes} and Armor and Reflex defenses.
For details, see \pcref{Squeezing}.

\glossdef{stabilization check} A \glossterm{check} made when a creature is \glossterm{dying} to see if it stabilizes or dies.
For details, see \pcref{Injury, Death, and Healing}.

\glossdef{staggered} A staggered creature is temporarily overwhelmed by physical trauma.
It takes a \minus4 penalty to \glossterm{accuracy} and \glossterm{checks}.
Becoming \glossterm{bloodied} cause you to become staggered (see \pcref{Staggered}).

\glossdef{standard action}[standard actions] You can use a standard action to attack with a weapon, cast a spell, drink a potion, and do most other things that take concentration and effort.

\glossdef{standard damage} A common damage value for abilities.
For details, see \pcref{Standard Damage}.

\glossdef{strike}[strikes] A strike is a single physical attack with a weapon.
It is the most common type of attack.
You can make a strike as a \glossterm{standard action} in the \glossterm{action phase}.
For details, see \pcref{Strikes}.

\glossdef{stunned} A stunned creature cannot take any actions during the \glossterm{action phase} or \glossterm{delayed action phase} except the \textit{recover} and \textit{desperate recovery} actions (see \pcref{Recover}, and \pcref{Desperate Recovery}).
In addition, it takes a \minus2 penalty to \glossterm{accuracy}, \glossterm{checks}, and \glossterm{defenses}.

\glossdef{subdual damage} Subdual damage is a special kind of damage that can't kill you.
If you take subdual damage in excess of your hit points, you fall unconscious.
For details, see \pcref{Subdual Damage}.

\glossdef{Subtle} See \pcref{Ability Tags}.

\glossdef{suppressed} A suppressed ability has temporarily ceased to function.
It has no effect for as long as it remains suppressed.
Time spent while suppressed counts against the ability's \glossterm{duration}, and it may expire while suppressed if it lasts for a specific amount of time.
Only \glossterm{magical} abilities can be suppressed.
Mundane results of magical abilities that have already occured, such as the water created by a \ritual{create water} ritual, cannot themselves be suppressed, and do not disappear if they enter an area that suppresses magical abilities.

\glossdef{sustain}[sustained] Some abilities last as long as you sustain them.
Each ability specifies a particular action that is required to sustain the ability, such as a \glossterm{minor action}
At the end of each round, the ability is dismissed unless you used the ability that round or took the action to sustain the ability that round.
For details, see \pcref{Sustained Abilities}.

\glossdef{Sustain} An ability with this \glossterm{ability tag} lasts as long as you sustain it each round.
The tag includes an action type, such as (minor), which indicates the type of action required to sustain the ability.
For details, see \pcref{Sustained Abilities}.

\glossdef{Swift} An ability with this \glossterm{ability tag} resolves its effects before other actions in the same phase.
For details, see \pcref{Swift Abilities}.

\glossdef{swim speed} A creature with a swim speed can swim as easily as a human walks on land.
The effects of a swim speed are described at \pcref{Swim Speed}.

\glossdef{take 10}[taking 10] If you have plenty of time to accomplish a task that requires a \glossterm{check}, and there are no meaningful consequences for failure, you can take 10 to accomplish the task.
If you do, the task takes ten times as long, but you treat your roll for the check as if you had rolled a 10.
For details, see \pcref{Taking 10}.

\glossdef{target} A target is a creature or object directly affected by an ability.
Many abilities only affect a single target, and some affect a specific number of targets.

\glossdef{target square} A target square is a particular \glossterm{square} that an attack is made against.
A target square is chosen to determine \glossterm{cover} and \glossterm{concealment} (see \pcref{Cover}).

\glossdef{targeted}[Targeted] A \glossterm{targeted} ability is an ability that allows you to directly choose which targets the ability affects.
A spell that affects an area is not a targeted ability, because you choose the area affected instead of choosing the targets directly.

\glossdef{task} A task is a particular way to use a \glossterm{skill}.
For example, balancing on slippery ground is a task that you can use the Acrobatics skill for (see \pcref{Balance}).
For details, see \pcref{Tasks}.

\glossdef{Teleportation} See \pcref{Ability Tags}.

\glossdef{Temporal} See \pcref{Ability Tags}.

\glossdef{threat} A creature's threat represents how threatening it is.
Many monsters choose the targets of their attacks based on the threat of their foes.
For details, see \pcref{Threat}.

\glossdef{threaten}[threatened] If you are using a \glossterm{melee} weapon, all enemies within your \glossterm{reach} with that weapon are \glossterm{threatened}.
A threatened creature may suffer \glossterm{overwhelm penalties} if there are multiple creatures threatening it.

\glossdef{thrown weapon}[thrown weapons] A thrown weapon is a weapon designed to be thrown at a target.
For details about attacking with thrown weapons, see \pcref{Thrown Strike}.

\glossdef{total cover} Total cover is a type of \glossterm{cover}.
If a creature is completely behind a physical object that blocks sight, it has \glossterm{total cover} from attacks.
A creature with total cover cannot be targeted by any attacks.
For details, see \pcref{Total Cover}.

\glossdef{trained} If you are trained in a skill, you have learned how to use it well, but you have not \glossterm{mastered} it.
Your modifier with a trained skill is equal to either half your level \add 1 or the skill's \glossterm{key attribute} (if any), whichever is higher.
For details, see \pcref{Skill Training}.

\glossdef{Trap} See \pcref{Ability Tags}.

\glossdef{trip} You can knock a foe off its feet with the \textit{trip} ability.
For details, see \pcref{Trip}.

\glossdef{tremorsight} A creature with tremorsight can ``see'' its surroundings perfectly without any light, regardless of concealment or invisibility.
It needs an uninterrupted path through solid objects to sense its surroundings, but does not require line of effect.
Tremorsight always has a range, and grants no benefits beyond that range.

\glossdef{tremorsense} A creature with tremorsense can sense its surroundings without any light, regardless of concealment or invisiblity.
It knows the location of everything around it, but it still takes normal failure chances for concealment, invisibility, and so on.
It needs an uninterrupted path through solid objects to sense its surroundings, but does not require line of effect.
Tremorsense always has a range, and grants no benefits beyond that range.

\glossdef{truesight} A creature with truesight can ignore all \glossterm{Sensation} effects within a given range.
Despite the name of the ability, it affects all senses, not merely sight.

\glossdef{unarmed attack}[unarmed attacks] Every corporeal creature is capable of making an attack using its bare fists (or similar appendages).
For details, see \pcref{Unarmed Combat}.

% TODO: expand definition to include ``attended by a willing creature''
% Or just expand ``willing'' to include objects
\glossdef{unattended} An unattended item is an item not being held or carried by a creature.
Some abilities can only affect unattended items.

\glossdef{unaware} An unaware creature does not know that it is being attacked.
Any \glossterm{physical attack} against an unaware creature automatically \glossterm{explodes} on the first die.
After being attacked, an unaware creature typically stops being unaware of future attacks, even if it cannot see or identify its attacker.

\glossdef{undergrowth} The presence of a significant amount of roots, bushes, and similar plants that can obstruct movement is called undergrowth.
There are two kinds of undergrowth: \glossterm{light undergrowth} and \glossterm{heavy undergrowth}.
For details, see \pcref{Undergrowth}.

\glossdef{usage class} The \glossterm{usage class} of a weapon or armor is a measure of how much effort it takes to use the item.
For details, see \pcref{Weapon Usage Classes} and \pcref{Armor Usage Classes}.

\glossdef{verbal components}[verbal] Verbal components are words required to cast most spells.
For details, see \pcref{Casting Components}.

\glossdef{Visual} See \pcref{Ability Tags}.

\glossdef{vital damage} If you take damage in excess of your \glossterm{hit points}, that damage is dealt as vital damage.
Vital damage inflicts debilitating \glossterm{vital damage penalties}.
For details, see \pcref{Vital Damage}.

\glossdef{vital damage penalties} For every 4 points of \glossterm{vital damage} you have, you take a \minus1 penalty to \glossterm{accuracy}, \glossterm{checks}, and \glossterm{defenses}.
For details, see \pcref{Vital Damage}.

\glossdef{vulnerable} A creature can be vulnerable to a type of damage or a special weapon material.
It takes double damage from sources it is vulnerable to.
If it takes damage from a damage source with multiple types or multiple materials, it takes double damage if it is vulnerable to any of those types or materials.
Vulnerability is calculated before applying \glossterm{damage reduction}.
\par If a creature would become vulnerable to the same thing multiple times, it still only takes double damage from damage of that type.

\glossdef{Water} See \pcref{Ability Tags}.

\glossdef{weapon}[weapons] A weapon is an object used to inflict damage.
Some creatures can treat parts of their body as weapons.
For details, see \pcref{Weapons}.

\glossdef{weapon group} A weapon group is a category of \glossterm{weapons} with a similar design and fighting style.
You have proficiency with some number of weapon groups based on your \glossterm{class}.
For details, see \pcref{Weapon Groups}.

\glossdef{willing} Some abilities can only affect willing targets. You can choose to be a willing target at any time. Unconscious creatures and objects are automatically considered willing, but a character who is conscious but immobile or helpless (such as one who is bound or paralyzed) is not automatically willing.

\glossdef{withdraw} The \textit{withdraw} ability allows you to stay away from a creature, preventing it from coming too close to you.
For details, see \pcref{Withdraw}.

\glossdef{zone} A zone is a type of area that an ability can have (see \pcref{Area Types}).
A zone ability has effects within an area for the \glossterm{duration} of the ability.
Unless otherwise noted, it does not move after being created.


\chapter{Wealth}

\section{Wealth By Level}
Characters can generally expect to have a certain amount of total wealth, gained through the course of their adventures. The below chart summarizes the amount of wealth a character can expect to have. This may take the form of currency, precious gems, magic items, land, or anything else of significant value. For the purpose of character wealth, magic items are considered to be worth their market price, regardless of how they were acquired.

\begin{dtable}
\lcaption{Character Wealth}
\begin{tabularx}{\columnwidth}{c >{\ccol}X >{\ccol}X}
\thead{Level} & \thead{Total wealth} & \thead{Wealth gained at level} \\
1 & 100 gp & 100 gp \\
2 & 500 gp & 400 gp \\
3 & 1,500 gp & 1,000 gp \\
4 & 3,000 gp & 1,500 gp \\
5 & 5,500 gp & 2,500 gp \\
6 & 9,000 gp & 3,500 gp \\
7 & 14,000 gp & 5,000 gp \\
8 & 20,500 gp & 6,500 gp \\
9 & 28,500 gp & 8,000 gp \\
10 & 38,500 gp & 10,000 gp \\
11 & 50,500 gp & 12,000 gp \\
12 & 65,000 gp & 14,500 gp \\
13 & 82,000 gp & 17,000 gp \\
14 & 101,500 gp & 19,500 gp \\
15 & 124,000 gp & 22,500 gp \\
16 & 149,500 gp & 25,500 gp \\
17 & 178,500 gp & 29,000 gp \\
18 & 211,000 gp & 32,500 gp \\
19 & 247,000 gp & 36,000 gp \\
20 & 287,000 gp & 40,000 gp
\end{tabularx}
\end{dtable}

\section{Item Levels}

Each item has a level associated with it. This level is different from its caster level, and has no in-game significance; instead, it represents the level of character for which the item is appropriate. Item levels are based on the price of the item, using the table below.

\subsection{Using Item Levels}

You can equip a character using by using a number of items of appropriate levels instead of by individually spending all of the wealth allotted to the character. To do so, give the character one item of each level, starting with the character's level and ending five levels lower or at 1st level, for a total of six items (or fewer if the character is less than 6th level).

If you want more items, you can trade an item of one level for two items of a lower level. You can also trade two items of a lower level for an item of a higher level, but this should not be used to gain an item of a level higher than the character's level. Items can be traded according to the table below.

\begin{dtable}
\lcaption{Item Levels}
\begin{tabularx}{\columnwidth}{c c >{\ccol}X}
\thead{Item Level} & \thead{Market Price Range} & \thead{Worth two items of this level}\\
1/2 & 1 gp - 50 gp & \x\\
1 & 51 gp - 150 gp & 1/2 \\
2 & 151 gp - 500 gp & 1 \\
3 & 501 gp - 1,000 gp & 2 \\
4 & 1,001 gp - 1,750 gp & 3 \\
5 & 1,751 gp - 2,750 gp & 3 \\
6 & 2,751 gp - 3,750 gp & 4 \\
7 & 3,751 gp - 5,500 gp & 5 \\
8 & 5,501 gp - 7,500 gp & 6 \\
9 & 7,501 gp - 9,500 gp & 7 \\
10 & 9,501 gp - 12,000 gp & 7 \\
11 & 12,001 gp - 15,000 gp & 8 \\
12 & 15,001 gp - 19,000 gp & 9 \\
13 & 19,001 gp - 23,000 gp & 10 \\
14 & 23,001 gp - 28,000 gp & 11 \\
15 & 28,001 gp - 33,000 gp & 11 \\
16 & 33,001 gp - 38,000 gp & 12 \\
17 & 38,001 gp - 45,000 gp & 13 \\
18 & 45,001 gp - 52,000 gp & 14 \\
19 & 52,001 gp - 60,000 gp & 14 \\
20 & 60,001 gp - 70,000 gp & 15
\end{tabularx}
\end{dtable}

\chapter{Magic Items}

Magic items are objects that have been imbued with magical energy. They can take almost any form, and their potential uses are only as limited as the magic that created them.

\section{Magic Item Types}
    Magic items are divided into four broad categories:
    \begin{itemize}
        \item Weapons are used to make physical attacks. They provide access to their abilities when wielded.
            A \mitem{flaming longsword} and a \mitem{vampiric scythe} are weapons.
        \item Implements are used to cast spells. They provide access to their abilities when wielded.
            A \mitem{staff of fire} and a \mitem{staff of time} are implements.
        \item Apparel items are usually not used individually. They provide access to their abilities when worn.
            A \mitem{shield of arrow deflection} and a \mitem{ring of protection} are apparel items.
        \item Tools provide access to their abilities when used in some way.
            A \mitem{bag of carrying} is a tool.
    \end{itemize}

\section{Using Magic Items}

    \subsection{Item Activation}

        Some magic items have to be explicitly activated to have unusual effects.
        For example, the \mitem{seven league boots} can be activated to teleport you across great distances.
        Other magic items constantly have magical effects.
        For example, a \mitem{ring of protection} passively grants you a defense bonus.

        The description of a magic item effect will specify what mechanical actions must be taken, if any, to activate the effects of the item.
        For example, a belt of healing requires taking a \glossterm{standard action}.
        However, the item description will not specify the exact nature of the action.
        Different items, even if they have the same effect, can have different physical actions that are required to activate the item.
        These activation actions can come in one of the following forms:
        \begin{itemize}
           \item Command word: You must speak a specific word that the item will hear and react to.
                For example, you may need to say the word ``healing'' in Elven to activate an item that heals you.
            \item Mental command: You must mentally direct the item to activate, such as by visualizing the item or thinking a particular word.
                % TODO: does this item exist
                For example, you may need to imagine a warm blanket around you to activate an item that protects you from cold damage or environmental effects.
            \item Physical motion: You must perform a specific physical motion, usually involving the item in some way.
                For example, you may need to rapidly stomp one foot on the ground to activate an item that allows you to move faster.
        \end{itemize}

        % TODO: table of random item activations?

    \subsection{Item Limitations}

        There are three restrictions on your ability to use magic items.
        First, you cannot equip two apparel items that take up the same physical location on your body.
        For example, you cannot equip two different gauntlet sets and gain the effects of both, but you could equip several amulets or up to ten rings.

        Second, all magic items require you to attune to them to gain their effect unless they indicate otherwise in their description.
        You can attune to a magic item with the \textit{item attunement} ability, below.

        Third, you cannot attune to two items with the same name, or if one is simply a Greater or Lesser version of the other.

        \subsubsection{Item Attunement}\label{Item Attunement}

            As a standard action, you can use the \textit{item attunement} ability to attune to items.

            \begin{attuneability}{Item Attunement}
                \glossterm{Attune} (self)
                \rankline
                Choose a magic item you are touching.
                Any abilities the target has that require attunement become active, allowing you to use its full potential.
            \end{attuneability}

            \parhead{Shared Item Attunement} Multiple creatures can attune to the same item simultaneously.
            Since most items only function while worn or wielded, this does not usually allow multiple creatures to gain the benefits of the item.
            However, the creatures can swap the item between them without having to reattune to it each time.

    \subsubsection{Item Power}\label{Item Power}
        The \glossterm{power} of an item depends on its level.
        If the item is not being attuned to by a creature, its power is equal to its level.
        If a creature is attuning to the item, its power is equal to its level or the level of the attuning creature, whichever is higher.

        An item's \glossterm{power} also affects its defenses.
        Its Fortitude and Mental defenses are equal to 5 \add its \glossterm{power}.
        Its Armor defense and Reflex defense are not affected by its \glossterm{power}, and are solely determined by its size and shape.

    \subsection{Removing Magic Items}
        Unless otherwise noted, magic items that have effects on the creature using the item must continue to be worn or held as long as the effect lasts.
        If a magic item has an ability with a duration, removing the item also ends the ability.
        Items which are consumed when used or which do not affect their user are unaffected by this rule.

\section{Item Description Format}
    TODO

\section{Apparel}
    Apparel items must be worn to gain their effects.

    \subsection{Body Slots}
        The main limiting factor on how many items you can have equipped is your attunement points, not the physical location of your items on your body.
        However, there are limits to how many items you can wear of the same type, as described below.
        For item types not listed here, use reasonable judgment about what would be plausible.
        \begin{itemize}
            \item Amulet: Up to 2
            \item Belt: Up to 2
            \item Boots: Up to 1
            \item Circlet: Up to 1
            \item Gauntlets: Up to 1 (separate from gloves)
            \item Gloves: Up to 1 (separate from gauntlets)
            \item Rings: Up to 5 per hand
        \end{itemize}

    
\begin{longtabuwrapper}
\begin{longtabu}{l l X l}
\lcaption{Apparel Items} \\
\tb{Name} & \tb{Level} & \tb{Description} & \tb{Page} \\
\bottomrule
Belt of Healing & \nth{1} & Grants healing & \pageref{item:Belt of Healing} \\
Bracers of Archery & \nth{1} & Grants bow proficiency & \pageref{item:Bracers of Archery} \\
Amulet of Health & \nth{2} & Increases your hit points & \pageref{item:Amulet of Health} \\
Boots of the Winterlands & \nth{2} & Eases travel in cold areas & \pageref{item:Boots of the Winterlands} \\
Bracers of Armor & \nth{2} & Grants invisible armor & \pageref{item:Bracers of Armor} \\
Gauntlet of the Ram & \nth{2} & Shoves foe when used to strike & \pageref{item:Gauntlet of the Ram} \\
Gauntlets of Improvisation & \nth{2} & Grants \plus1d damage with improvised weapons & \pageref{item:Gauntlets of Improvisation} \\
Lifekeeping Belt & \nth{2} & Reduces vital damage penalties by 2 & \pageref{item:Lifekeeping Belt} \\
Ring of Elemental Endurance & \nth{2} & Grants tolerance of temperature extremes & \pageref{item:Ring of Elemental Endurance} \\
Shield of Arrow Deflection & \nth{2} & Can block small projectiles & \pageref{item:Shield of Arrow Deflection} \\
Shield of Bashing & \nth{2} & Deals \plus1d damage & \pageref{item:Shield of Bashing} \\
Torchlight Gloves & \nth{2} & Sheds light as a torch & \pageref{item:Torchlight Gloves} \\
Boots of Freedom & \nth{3} & Grants immunity to magical mobility restrictions & \pageref{item:Boots of Freedom} \\
Ocular Circlet & \nth{3} & Can allow you to see at a distance & \pageref{item:Ocular Circlet} \\
Ring of Nourishment & \nth{3} & Provides food and water & \pageref{item:Ring of Nourishment} \\
Armor of Energy Resistance & \nth{4} & Reduces energy damage & \pageref{item:Armor of Energy Resistance} \\
Boots of Earth's Embrace & \nth{4} & Grants immunity to forced movement & \pageref{item:Boots of Earth's Embrace} \\
Boots of Elvenkind & \nth{4} & Grants \plus2 Stealth & \pageref{item:Boots of Elvenkind} \\
Bracers of Repulsion & \nth{4} & Can shove nearby creatures back & \pageref{item:Bracers of Repulsion} \\
Circlet of Blasting & \nth{4} & Can blast foe with fire & \pageref{item:Circlet of Blasting} \\
Circlet of Persuasion & \nth{4} & Grants \plus2 Persuasion & \pageref{item:Circlet of Persuasion} \\
Featherlight Armor & \nth{4} & Reduces encumbrance by 1 & \pageref{item:Featherlight Armor} \\
Hidden Armor & \nth{4} & Can look like normal clothing & \pageref{item:Hidden Armor} \\
Mask of Water Breathing & \nth{4} & Allows breathing water like air & \pageref{item:Mask of Water Breathing} \\
Throwing Gloves & \nth{4} & Allows throwing any item accurately & \pageref{item:Throwing Gloves} \\
Crown of Flame & \nth{5} & Grants nearby allies immunity to fire damage & \pageref{item:Crown of Flame} \\
Ring of Energy Resistance & \nth{5} & Reduces energy damage & \pageref{item:Ring of Energy Resistance} \\
Shield of Arrow Catching & \nth{5} & Redirects small nearby projectiles to hit you & \pageref{item:Shield of Arrow Catching} \\
Amulet of Mighty Fists & \nth{6} & Grants \plus1d damage with your body & \pageref{item:Amulet of Mighty Fists} \\
Amulet of Nondetection & \nth{6} & Grants \plus4 to defenses against detection & \pageref{item:Amulet of Nondetection} \\
Boots of Speed & \nth{6} & Increases speed by ten feet & \pageref{item:Boots of Speed} \\
Shield of Boulder Deflection & \nth{6} & Can block large projectiles & \pageref{item:Shield of Boulder Deflection} \\
Armor of Fortification & \nth{7} & Reduces critical hits from strikes & \pageref{item:Armor of Fortification} \\
Assassin's Cloak & \nth{7} & Grants invisibility while inactive & \pageref{item:Assassin's Cloak} \\
Belt of Healing, Greater & \nth{7} & Grants more healing & \pageref{item:Belt of Healing, Greater} \\
Boots of Water Walking & \nth{7} & Allows walking on liquids & \pageref{item:Boots of Water Walking} \\
Boots of the Skydancer & \nth{7} & Can walk on air & \pageref{item:Boots of the Skydancer} \\
Bracers of Archery, Greater & \nth{7} & Grants bow proficiency, \plus1 ranged accuracy & \pageref{item:Bracers of Archery, Greater} \\
Crown of Lightning & \nth{7} & Continuously damages nearby enemies & \pageref{item:Crown of Lightning} \\
Gauntlet of the Ram, Greater & \nth{7} & Shoves foe hard when use to strike & \pageref{item:Gauntlet of the Ram, Greater} \\
Gauntlets of Improvisation, Greater & \nth{7} & Grants \plus2d damage with improvised weapons & \pageref{item:Gauntlets of Improvisation, Greater} \\
Lifekeeping Belt, Greater & \nth{7} & Reduces vital damage penalties by 4 & \pageref{item:Lifekeeping Belt, Greater} \\
Ring of Sustenance & \nth{7} & Provides food, water, and rest & \pageref{item:Ring of Sustenance} \\
Amulet of Health, Greater & \nth{8} & Greatly increases your hit points & \pageref{item:Amulet of Health, Greater} \\
Armor of Invulnerability & \nth{8} & Reduces damage from physical attacks & \pageref{item:Armor of Invulnerability} \\
Boots of Gravitation & \nth{8} & Redirects personal gravity & \pageref{item:Boots of Gravitation} \\
Bracers of Repulsion, Greater & \nth{8} & Can shove foes back & \pageref{item:Bracers of Repulsion, Greater} \\
Cloak of Mist & \nth{8} & Fills nearby area with fog & \pageref{item:Cloak of Mist} \\
Ring of Protection & \nth{8} & Grants \plus1 to Armor and Reflex defenses & \pageref{item:Ring of Protection} \\
Shield of Arrow Deflection, Greater & \nth{8} & Blocks small projectiles & \pageref{item:Shield of Arrow Deflection, Greater} \\
Shield of Boulder Catching & \nth{8} & Redirects large nearby projectiles to hit you & \pageref{item:Shield of Boulder Catching} \\
Vanishing Cloak & \nth{8} & Can teleport a short distance and grant invisibility & \pageref{item:Vanishing Cloak} \\
Boots of Freedom, Greater & \nth{9} & Grants immunity to almost all mobility restrictions & \pageref{item:Boots of Freedom, Greater} \\
Crown of Thunder & \nth{9} & Continously deafens nearby enemies & \pageref{item:Crown of Thunder} \\
Greatreach Bracers & \nth{9} & Increases reach by five feet & \pageref{item:Greatreach Bracers} \\
Hidden Armor, Greater & \nth{9} & Can look and sound like normal clothing & \pageref{item:Hidden Armor, Greater} \\
Mask of Air & \nth{9} & Allows breathing in any environment & \pageref{item:Mask of Air} \\
Ocular Circlet, Greater & \nth{9} & description & \pageref{item:Ocular Circlet, Greater} \\
Boots of Speed, Greater & \nth{10} & Increases speed by twenty feet & \pageref{item:Boots of Speed, Greater} \\
Circlet of Blasting, Greater & \nth{10} & Can blast foe with intense fire & \pageref{item:Circlet of Blasting, Greater} \\
Crater Boots & \nth{10} & Deals your falling damage to enemies & \pageref{item:Crater Boots} \\
Featherlight Armor, Greater & \nth{10} & Reduces encumbrance by 2 & \pageref{item:Featherlight Armor, Greater} \\
Gloves of Spell Investment & \nth{10} & description & \pageref{item:Gloves of Spell Investment} \\
Shield of Arrow Catching, Greater & \nth{10} & Selectively redirects small nearby projectiles to hit you & \pageref{item:Shield of Arrow Catching, Greater} \\
Winged Boots & \nth{10} & Grants limited flight & \pageref{item:Winged Boots} \\
Crown of Frost & \nth{11} & Continuously damages and fatigues nearby enemies & \pageref{item:Crown of Frost} \\
Hexward Amulet & \nth{11} & Grants \plus4 defenses against targeted magical attacks & \pageref{item:Hexward Amulet} \\
Ring of Regeneration & \nth{11} & Grants fast healing & \pageref{item:Ring of Regeneration} \\
Amulet of the Planes & \nth{12} & Aids travel with \ritual{plane shift} & \pageref{item:Amulet of the Planes} \\
Armor of Energy Resistance, Greater & \nth{12} & Drastically reduces energy damage & \pageref{item:Armor of Energy Resistance, Greater} \\
Armor of Fortification, Mystic & \nth{12} & Reduces critical hits from all attacks & \pageref{item:Armor of Fortification, Mystic} \\
Seven League Boots & \nth{12} & Teleport seven leages with a step & \pageref{item:Seven League Boots} \\
Shield of Bashing, Greater & \nth{12} & Deals \plus2d damage & \pageref{item:Shield of Bashing, Greater} \\
Shield of Boulder Deflection, Greater & \nth{12} & Blocks large projectiles & \pageref{item:Shield of Boulder Deflection, Greater} \\
Shield of Mystic Reflection & \nth{12} & React to reflect magical attacks & \pageref{item:Shield of Mystic Reflection} \\
Lifekeeping Belt, Supreme & \nth{13} & Reduces vital damage penalties by 8 & \pageref{item:Lifekeeping Belt, Supreme} \\
Ring of Energy Resistance, Greater & \nth{13} & Drastically reduces energy damage & \pageref{item:Ring of Energy Resistance, Greater} \\
Titan Gauntlets & \nth{13} & Grants \plus1d damage with strikes & \pageref{item:Titan Gauntlets} \\
Amulet of Mighty Fists, Greater & \nth{14} & Grants \plus2d damage with your body & \pageref{item:Amulet of Mighty Fists, Greater} \\
Amulet of Nondetection, Greater & \nth{14} & Grants \plus8 to defenses against detection & \pageref{item:Amulet of Nondetection, Greater} \\
Boots of Speed, Supreme & \nth{14} & Increases speed by thirty feet & \pageref{item:Boots of Speed, Supreme} \\
Armor of Fortification, Greater & \nth{15} & Drastically reduces critical hits from strikes & \pageref{item:Armor of Fortification, Greater} \\
Armor of Invulnerability, Greater & \nth{16} & Drastically reduces damage from physical attacks & \pageref{item:Armor of Invulnerability, Greater} \\
Astral Boots & \nth{16} & Allows teleporting instead of moving & \pageref{item:Astral Boots} \\
Circlet of Blasting, Supreme & \nth{16} & Can blast foe with supremely intense fire & \pageref{item:Circlet of Blasting, Supreme} \\
Cloak of Mist, Greater & \nth{16} & Fills nearby area with thick fog & \pageref{item:Cloak of Mist, Greater} \\
Ring of Protection, Greater & \nth{16} & Grants \plus2 to Armor and Reflex defenses & \pageref{item:Ring of Protection, Greater} \\
Assassin's Cloak, Greater & \nth{17} & Grants invisibility while not attacking & \pageref{item:Assassin's Cloak, Greater} \\
Greatreach Bracers, Greater & \nth{17} & Increases reach by ten feet & \pageref{item:Greatreach Bracers, Greater} \\
Hexproof Amulet, Greater & \nth{17} & Grants \plus6 defenses against targeted magical attacks & \pageref{item:Hexproof Amulet, Greater} \\
Gloves of Spell Investment, Greater & \nth{18} & description & \pageref{item:Gloves of Spell Investment, Greater} \\
Amulet of the Planes, Greater & \nth{19} & Aid travel with \ritual{plane shift} subrituals & \pageref{item:Amulet of the Planes, Greater} \\
\end{longtabu}
\end{longtabuwrapper}


    
\lowercase{\hypertarget{item:Amulet of Health}{}}\label{item:Amulet of Health}
\hypertarget{item:Amulet of Health}{\subsubsection{Amulet of Health\hfill\nth{2} (125 gp)}}

You gain a \glossterm{magic bonus} equal to this item's \glossterm{power} to your \glossterm{wound threshold}.



\vspace{0.25em}
\spelltwocol{\textbf{Type}: Amulet}{}
\textbf{Materials}: Jewelry


\lowercase{\hypertarget{item:Amulet of Health, Greater}{}}\label{item:Amulet of Health, Greater}
\hypertarget{item:Amulet of Health, Greater}{\subsubsection{Amulet of Health, Greater\hfill\nth{8} (2,750 gp)}}

You gain a \glossterm{magic bonus} equal to twice this item's \glossterm{power} to your \glossterm{wound threshold}.



\vspace{0.25em}
\spelltwocol{\textbf{Type}: Amulet}{}
\textbf{Materials}: Jewelry


\lowercase{\hypertarget{item:Amulet of Mighty Fists}{}}\label{item:Amulet of Mighty Fists}
\hypertarget{item:Amulet of Mighty Fists}{\subsubsection{Amulet of Mighty Fists\hfill\nth{8} (2,750 gp)}}

You gain a \plus2 \glossterm{magic bonus} to \glossterm{power} with \glossterm{unarmed attacks} and natural weapons.



\vspace{0.25em}
\spelltwocol{\textbf{Type}: Amulet}{}
\textbf{Materials}: Jewelry


\lowercase{\hypertarget{item:Amulet of Mighty Fists, Greater}{}}\label{item:Amulet of Mighty Fists, Greater}
\hypertarget{item:Amulet of Mighty Fists, Greater}{\subsubsection{Amulet of Mighty Fists, Greater\hfill\nth{16} (85,000 gp)}}

You gain a \plus4 \glossterm{magic bonus} to \glossterm{power} with \glossterm{unarmed attacks} and natural weapons.



\vspace{0.25em}
\spelltwocol{\textbf{Type}: Amulet}{}
\textbf{Materials}: Jewelry


\lowercase{\hypertarget{item:Amulet of Nondetection}{}}\label{item:Amulet of Nondetection}
\hypertarget{item:Amulet of Nondetection}{\subsubsection{Amulet of Nondetection\hfill\nth{6} (1,200 gp)}}

You gain a \plus4 bonus to defenses against abilities with the \glossterm{Detection} or \glossterm{Scrying} tags.



\vspace{0.25em}
\spelltwocol{\textbf{Type}: Amulet}{}
\textbf{Materials}: Jewelry


\lowercase{\hypertarget{item:Amulet of Nondetection, Greater}{}}\label{item:Amulet of Nondetection, Greater}
\hypertarget{item:Amulet of Nondetection, Greater}{\subsubsection{Amulet of Nondetection, Greater\hfill\nth{14} (37,000 gp)}}

You gain a \plus8 bonus to defenses against abilities with the \glossterm{Detection} or \glossterm{Scrying} tags.



\vspace{0.25em}
\spelltwocol{\textbf{Type}: Amulet}{}
\textbf{Materials}: Jewelry


\lowercase{\hypertarget{item:Amulet of the Planes}{}}\label{item:Amulet of the Planes}
\hypertarget{item:Amulet of the Planes}{\subsubsection{Amulet of the Planes\hfill\nth{12} (16,000 gp)}}

When you perform the \ritual{plane shift} ritual, this amulet provides all action points required.
This does not grant you the ability to perform the \ritual{plane shift} ritual if you could not already.



\vspace{0.25em}
\spelltwocol{\textbf{Type}: Amulet}{}
\textbf{Materials}: Jewelry


\lowercase{\hypertarget{item:Armor of Energy Resistance}{}}\label{item:Armor of Energy Resistance}
\hypertarget{item:Armor of Energy Resistance}{\subsubsection{Armor of Energy Resistance\hfill\nth{5} (800 gp)}}

You gain a \glossterm{magic bonus} equal to half the item's \glossterm{power} to \glossterm{resistances} against \glossterm{energy damage}.
When you resist energy damage, it sheds light as a torch until the end of the next round.
The color of the light depends on the energy damage resisted: green for acid, blue for cold, yellow for electricity, and red for fire.



\vspace{0.25em}
\spelltwocol{\textbf{Type}: Body armor}{}
\textbf{Materials}: Bone, metal


\lowercase{\hypertarget{item:Armor of Energy Resistance, Greater}{}}\label{item:Armor of Energy Resistance, Greater}
\hypertarget{item:Armor of Energy Resistance, Greater}{\subsubsection{Armor of Energy Resistance, Greater\hfill\nth{14} (37,000 gp)}}

This item functions like the \mitem{armor of energy resistance} item, except that the bonus is equal to the item's \glossterm{power}.



\vspace{0.25em}
\spelltwocol{\textbf{Type}: Body armor}{}
\textbf{Materials}: Bone, metal


\lowercase{\hypertarget{item:Armor of Fortification}{}}\label{item:Armor of Fortification}
\hypertarget{item:Armor of Fortification}{\subsubsection{Armor of Fortification\hfill\nth{7} (1,800 gp)}}

You gain a \plus5 bonus to defenses when determining whether a \glossterm{strike} gets a \glossterm{critical hit} against you instead of a normal hit.



\vspace{0.25em}
\spelltwocol{\textbf{Type}: Body armor}{}
\textbf{Materials}: Bone, metal


\lowercase{\hypertarget{item:Armor of Fortification, Greater}{}}\label{item:Armor of Fortification, Greater}
\hypertarget{item:Armor of Fortification, Greater}{\subsubsection{Armor of Fortification, Greater\hfill\nth{15} (55,000 gp)}}

This item functions like the \mitem{armor of fortification} item, except that the bonus increases to \plus10.



\vspace{0.25em}
\spelltwocol{\textbf{Type}: Body armor}{}
\textbf{Materials}: Bone, metal


\lowercase{\hypertarget{item:Armor of Fortification, Mystic}{}}\label{item:Armor of Fortification, Mystic}
\hypertarget{item:Armor of Fortification, Mystic}{\subsubsection{Armor of Fortification, Mystic\hfill\nth{12} (16,000 gp)}}

This item functions like the \mitem{armor of fortification} item, except that it applies against all attacks instead of only against; \glossterm{strikes}.



\vspace{0.25em}
\spelltwocol{\textbf{Type}: Body armor}{}
\textbf{Materials}: Bone, metal


\lowercase{\hypertarget{item:Armor of Invulnerability}{}}\label{item:Armor of Invulnerability}
\hypertarget{item:Armor of Invulnerability}{\subsubsection{Armor of Invulnerability\hfill\nth{8} (2,750 gp)}}

The armor's bonus to \glossterm{resistances} based on its armor type is doubled.



\vspace{0.25em}
\spelltwocol{\textbf{Type}: Body armor}{}
\textbf{Materials}: Bone, metal


\lowercase{\hypertarget{item:Armor of Invulnerability, Greater}{}}\label{item:Armor of Invulnerability, Greater}
\hypertarget{item:Armor of Invulnerability, Greater}{\subsubsection{Armor of Invulnerability, Greater\hfill\nth{14} (37,000 gp)}}

The armor's bonus to \glossterm{resistances} based on its armor type is tripled.



\vspace{0.25em}
\spelltwocol{\textbf{Type}: Body armor}{}
\textbf{Materials}: Bone, metal


\lowercase{\hypertarget{item:Armor of Invulnerability, Supreme}{}}\label{item:Armor of Invulnerability, Supreme}
\hypertarget{item:Armor of Invulnerability, Supreme}{\subsubsection{Armor of Invulnerability, Supreme\hfill\nth{20} (400,000 gp)}}

The armor's bonus to \glossterm{resistances} based on its armor type is quadrupled.



\vspace{0.25em}
\spelltwocol{\textbf{Type}: Body armor}{}
\textbf{Materials}: Bone, metal


\lowercase{\hypertarget{item:Assassin's Cloak}{}}\label{item:Assassin's Cloak}
\hypertarget{item:Assassin's Cloak}{\subsubsection{Assassin's Cloak\hfill\nth{7} (1,800 gp)}}

At the end of each round, if you took no actions that round, you become \glossterm{invisible} until the end of the next round.



\vspace{0.25em}
\spelltwocol{\textbf{Type}: Cloak}{\parhead*{Tags} \glossterm{Sensation}}
\textbf{Materials}: Textiles


\lowercase{\hypertarget{item:Assassin's Cloak, Greater}{}}\label{item:Assassin's Cloak, Greater}
\hypertarget{item:Assassin's Cloak, Greater}{\subsubsection{Assassin's Cloak, Greater\hfill\nth{17} (125,000 gp)}}

At the end of each round, if you did not attack a creature that round, you become \glossterm{invisible} until the end of the next round.



\vspace{0.25em}
\spelltwocol{\textbf{Type}: Cloak}{\parhead*{Tags} \glossterm{Sensation}}
\textbf{Materials}: Textiles


\lowercase{\hypertarget{item:Astral Boots}{}}\label{item:Astral Boots}
\hypertarget{item:Astral Boots}{\subsubsection{Astral Boots\hfill\nth{16} (85,000 gp)}}

When you move, you can teleport the same distance instead.
This does not change the total distance you can move, but you can teleport in any direction, even vertically.
You cannot teleport to locations you do not have \glossterm{line of sight} and \glossterm{line of effect} to.



\vspace{0.25em}
\spelltwocol{\textbf{Type}: Boots}{}
\textbf{Materials}: Bone, leather, metal


\lowercase{\hypertarget{item:Belt of Healing}{}}\label{item:Belt of Healing}
\hypertarget{item:Belt of Healing}{\subsubsection{Belt of Healing\hfill\nth{6} (1,200 gp)}}

As a standard action, you can use this belt to regain a \glossterm{hit point}.
You can only use this item once between \glossterm{short rests}.



\vspace{0.25em}
\spelltwocol{\textbf{Type}: Belt}{}
\textbf{Materials}: Leather, textiles


\lowercase{\hypertarget{item:Belt of Healing, Greater}{}}\label{item:Belt of Healing, Greater}
\hypertarget{item:Belt of Healing, Greater}{\subsubsection{Belt of Healing, Greater\hfill\nth{14} (37,000 gp)}}

This item functions like the \textit{belt of healing}, except that you can regain two \glossterm{hit points} instead of one.



\vspace{0.25em}
\spelltwocol{\textbf{Type}: Belt}{}
\textbf{Materials}: Leather, textiles


\lowercase{\hypertarget{item:Boots of Earth's Embrace}{}}\label{item:Boots of Earth's Embrace}
\hypertarget{item:Boots of Earth's Embrace}{\subsubsection{Boots of Earth's Embrace\hfill\nth{4} (500 gp)}}

While you are standing on solid ground, you are immune to effects that would force you to move.
This does not protect you from other effects of those attacks, such as damage.



\vspace{0.25em}
\spelltwocol{\textbf{Type}: Boots}{}
\textbf{Materials}: Bone, leather, metal


\lowercase{\hypertarget{item:Boots of Elvenkind}{}}\label{item:Boots of Elvenkind}
\hypertarget{item:Boots of Elvenkind}{\subsubsection{Boots of Elvenkind\hfill\nth{4} (500 gp)}}

You gain a \plus2 \glossterm{magic bonus} to the Stealth skill (see \pcref{Stealth}).



\vspace{0.25em}
\spelltwocol{\textbf{Type}: Boots}{}
\textbf{Materials}: Bone, leather, metal


\lowercase{\hypertarget{item:Boots of Freedom}{}}\label{item:Boots of Freedom}
\hypertarget{item:Boots of Freedom}{\subsubsection{Boots of Freedom\hfill\nth{3} (250 gp)}}

You are immune to magical effects that restrict your mobility.
This does not prevent physical obstacles from affecting you, such as \glossterm{difficult terrain}.



\vspace{0.25em}
\spelltwocol{\textbf{Type}: Boots}{}
\textbf{Materials}: Bone, leather, metal


\lowercase{\hypertarget{item:Boots of Freedom, Greater}{}}\label{item:Boots of Freedom, Greater}
\hypertarget{item:Boots of Freedom, Greater}{\subsubsection{Boots of Freedom, Greater\hfill\nth{9} (4,000 gp)}}

You are immune to all effects that restrict your mobility, including nonmagical effects such as \glossterm{difficult terrain}.
This removes all penalties you would suffer for acting underwater, except for those relating to using ranged weapons.
This does not prevent you from being \grappled, but you gain a \plus10 bonus to defenses against the \textit{grapple} ability (see \pcref{Grapple}).



\vspace{0.25em}
\spelltwocol{\textbf{Type}: Boots}{}
\textbf{Materials}: Bone, leather, metal


\lowercase{\hypertarget{item:Boots of Gravitation}{}}\label{item:Boots of Gravitation}
\hypertarget{item:Boots of Gravitation}{\subsubsection{Boots of Gravitation\hfill\nth{8} (2,750 gp)}}

While these boots are within 5 feet of a solid surface, gravity pulls you towards the solid surface closest to your boots rather than in the normal direction.
This can allow you to walk easily on walls or even ceilings.



\vspace{0.25em}
\spelltwocol{\textbf{Type}: Boots}{}
\textbf{Materials}: Bone, leather, metal


\lowercase{\hypertarget{item:Boots of Speed}{}}\label{item:Boots of Speed}
\hypertarget{item:Boots of Speed}{\subsubsection{Boots of Speed\hfill\nth{6} (1,200 gp)}}

You gain a \plus10 foot \glossterm{magic bonus} to your land speed, up to a maximum of double your normal speed.



\vspace{0.25em}
\spelltwocol{\textbf{Type}: Boots}{}
\textbf{Materials}: Bone, leather, metal


\lowercase{\hypertarget{item:Boots of Speed, Greater}{}}\label{item:Boots of Speed, Greater}
\hypertarget{item:Boots of Speed, Greater}{\subsubsection{Boots of Speed, Greater\hfill\nth{10} (6,500 gp)}}

You gain a \plus20 foot \glossterm{magic bonus} to your land speed, up to a maximum of double your normal speed.



\vspace{0.25em}
\spelltwocol{\textbf{Type}: Boots}{}
\textbf{Materials}: Bone, leather, metal


\lowercase{\hypertarget{item:Boots of Speed, Supreme}{}}\label{item:Boots of Speed, Supreme}
\hypertarget{item:Boots of Speed, Supreme}{\subsubsection{Boots of Speed, Supreme\hfill\nth{14} (37,000 gp)}}

You gain a \plus30 foot \glossterm{magic bonus} to your land speed, up to a maximum of double your normal speed.



\vspace{0.25em}
\spelltwocol{\textbf{Type}: Boots}{}
\textbf{Materials}: Bone, leather, metal


\lowercase{\hypertarget{item:Boots of Water Walking}{}}\label{item:Boots of Water Walking}
\hypertarget{item:Boots of Water Walking}{\subsubsection{Boots of Water Walking\hfill\nth{7} (1,800 gp)}}

You treat the surface of all liquids as if they were firm ground.
Your feet hover about an inch above the liquid's surface, allowing you to traverse dangerous liquids without harm as long as the surface is calm.

If you are below the surface of the liquid, you rise towards the surface at a rate of 60 feet per round.
Thick liquids, such as mud and lava, may cause you to rise more slowly.



\vspace{0.25em}
\spelltwocol{\textbf{Type}: Boots}{}
\textbf{Materials}: Bone, leather, metal


\lowercase{\hypertarget{item:Boots of the Skydancer}{}}\label{item:Boots of the Skydancer}
\hypertarget{item:Boots of the Skydancer}{\subsubsection{Boots of the Skydancer\hfill\nth{7} (1,800 gp)}}

As a \glossterm{free action}, you can activate these boots.
When you do, you may treat air as if it were solid ground to your feet for the rest of the current phase.
You may selectively choose when to treat the air as solid ground, allowing you to walk or jump on air freely.
After using this ability, you cannot use it again until these boots touch the ground.



\vspace{0.25em}
\spelltwocol{\textbf{Type}: Boots}{\parhead*{Tags} \glossterm{Swift}}
\textbf{Materials}: Bone, leather, metal


\lowercase{\hypertarget{item:Boots of the Skydancer, Greater}{}}\label{item:Boots of the Skydancer, Greater}
\hypertarget{item:Boots of the Skydancer, Greater}{\subsubsection{Boots of the Skydancer, Greater\hfill\nth{13} (25,000 gp)}}

This item functions like the \magicitem{boots of the skydancer}, except that the ability lasts until the end of the round.
In addition, you can use this item twice before the boots touch the ground.



\vspace{0.25em}
\spelltwocol{\textbf{Type}: Boots}{\parhead*{Tags} \glossterm{Swift}}
\textbf{Materials}: Bone, leather, metal


\lowercase{\hypertarget{item:Boots of the Winterlands}{}}\label{item:Boots of the Winterlands}
\hypertarget{item:Boots of the Winterlands}{\subsubsection{Boots of the Winterlands\hfill\nth{2} (125 gp)}}

You can travel across snow and ice without slipping or suffering movement penalties for the terrain.
% TODO: degree symbol?
In addition, the boots keep you warn, protecting you in environments as cold as \minus50 Fahrenheit.



\vspace{0.25em}
\spelltwocol{\textbf{Type}: Boots}{}
\textbf{Materials}: Bone, leather, metal


\lowercase{\hypertarget{item:Bracers of Archery}{}}\label{item:Bracers of Archery}
\hypertarget{item:Bracers of Archery}{\subsubsection{Bracers of Archery\hfill\nth{1} (50 gp)}}

You are proficient with bows.



\vspace{0.25em}
\spelltwocol{\textbf{Type}: Bracers}{}
\textbf{Materials}: Bone, leather, metal, wood


\lowercase{\hypertarget{item:Bracers of Archery, Greater}{}}\label{item:Bracers of Archery, Greater}
\hypertarget{item:Bracers of Archery, Greater}{\subsubsection{Bracers of Archery, Greater\hfill\nth{7} (1,800 gp)}}

You are proficient with bows.
In addition, you gain a \plus1 \glossterm{magic bonus} to \glossterm{accuracy} with ranged \glossterm{strikes}.



\vspace{0.25em}
\spelltwocol{\textbf{Type}: Bracers}{}
\textbf{Materials}: Bone, leather, metal, wood


\lowercase{\hypertarget{item:Bracers of Armor}{}}\label{item:Bracers of Armor}
\hypertarget{item:Bracers of Armor}{\subsubsection{Bracers of Armor\hfill\nth{2} (125 gp)}}

You gain a \plus2 bonus to Armor defense.
The protection from these bracers is treated as body armor, and it does not stack with any other body armor you wear.



\vspace{0.25em}
\spelltwocol{\textbf{Type}: Bracers}{}
\textbf{Materials}: Bone, leather, metal, wood


\lowercase{\hypertarget{item:Bracers of Repulsion}{}}\label{item:Bracers of Repulsion}
\hypertarget{item:Bracers of Repulsion}{\subsubsection{Bracers of Repulsion\hfill\nth{7} (1,800 gp)}}

As a standard action, you can activate these bracers.
When you do, they emit a telekinetic burst of force.
Make an attack vs. Fortitude against everything within a \areasmall radius burst from you.
If you use this item during the \glossterm{delayed action phase},
you gain a \plus4 bonus to \glossterm{accuracy} with this attack against any creature that attacked you during the \glossterm{action phase}.
On a hit, you \glossterm{knockback} each target up to 20 feet.



\vspace{0.25em}
\spelltwocol{\textbf{Type}: Bracers}{}
\textbf{Materials}: Bone, leather, metal, wood


\lowercase{\hypertarget{item:Bracers of Repulsion, Greater}{}}\label{item:Bracers of Repulsion, Greater}
\hypertarget{item:Bracers of Repulsion, Greater}{\subsubsection{Bracers of Repulsion, Greater\hfill\nth{15} (55,000 gp)}}

This item functions like the \mitem{bracers of repulsion} item, except that it targets everything within a \arealarge radius burst.



\vspace{0.25em}
\spelltwocol{\textbf{Type}: Bracers}{}
\textbf{Materials}: Bone, leather, metal, wood


\lowercase{\hypertarget{item:Circlet of Blasting}{}}\label{item:Circlet of Blasting}
\hypertarget{item:Circlet of Blasting}{\subsubsection{Circlet of Blasting\hfill\nth{5} (800 gp)}}

As a standard action, you can activate this circlet.
If you do, make an attack vs. Armor against a creature or object within \rngmed range.
\hit The target takes fire \glossterm{standard damage}.



\vspace{0.25em}
\spelltwocol{\textbf{Type}: Circlet}{}
\textbf{Materials}: Bone, metal


\lowercase{\hypertarget{item:Circlet of Blasting, Greater}{}}\label{item:Circlet of Blasting, Greater}
\hypertarget{item:Circlet of Blasting, Greater}{\subsubsection{Circlet of Blasting, Greater\hfill\nth{10} (6,500 gp)}}

This item functions like the \textit{circlet of blasting}, except that it gains a \plus1d bonus to damage.



\vspace{0.25em}
\spelltwocol{\textbf{Type}: Circlet}{}
\textbf{Materials}: Bone, metal


\lowercase{\hypertarget{item:Circlet of Blasting, Supreme}{}}\label{item:Circlet of Blasting, Supreme}
\hypertarget{item:Circlet of Blasting, Supreme}{\subsubsection{Circlet of Blasting, Supreme\hfill\nth{16} (85,000 gp)}}

This item functions like the \textit{circlet of blasting}, except that it gains a \plus2d bonus to damage.



\vspace{0.25em}
\spelltwocol{\textbf{Type}: Circlet}{}
\textbf{Materials}: Bone, metal


\lowercase{\hypertarget{item:Circlet of Persuasion}{}}\label{item:Circlet of Persuasion}
\hypertarget{item:Circlet of Persuasion}{\subsubsection{Circlet of Persuasion\hfill\nth{4} (500 gp)}}

You gain a \plus2 \glossterm{magic bonus} to the Persuasion skill (see \pcref{Persuasion}).



\vspace{0.25em}
\spelltwocol{\textbf{Type}: Circlet}{}
\textbf{Materials}: Bone, metal


\lowercase{\hypertarget{item:Cloak of Mist}{}}\label{item:Cloak of Mist}
\hypertarget{item:Cloak of Mist}{\subsubsection{Cloak of Mist\hfill\nth{8} (2,750 gp)}}

Fog constantly fills a \areamed radius emanation from you.
This fog does not fully block sight, but it provides \concealment.

If a 5-foot square of fog takes fire damage equal to half this item's \glossterm{power}, the fog disappears from that area until the end of the next round.



\vspace{0.25em}
\spelltwocol{\textbf{Type}: Cloak}{\parhead*{Tags} \glossterm{Manifestation}}
\textbf{Materials}: Textiles


\lowercase{\hypertarget{item:Cloak of Mist, Greater}{}}\label{item:Cloak of Mist, Greater}
\hypertarget{item:Cloak of Mist, Greater}{\subsubsection{Cloak of Mist, Greater\hfill\nth{16} (85,000 gp)}}

A thick fog constantly fills a \areamed radius emanation from you.
This fog completely blocks sight beyond 10 feet.
Within that range, it still provides \concealment.

If a 5-foot square of fog takes fire damage equal to this item's \glossterm{power}, the fog disappears from that area until the end of the next round.



\vspace{0.25em}
\spelltwocol{\textbf{Type}: Cloak}{\parhead*{Tags} \glossterm{Manifestation}}
\textbf{Materials}: Textiles


\lowercase{\hypertarget{item:Crater Boots}{}}\label{item:Crater Boots}
\hypertarget{item:Crater Boots}{\subsubsection{Crater Boots\hfill\nth{10} (6,500 gp)}}

% This only works if you only take falling damage during the movement phase, which seems possible?
When you take \glossterm{falling damage}, make an attack vs Reflex against everything within a \areasmall radius from you.
\hit Each target takes damage as if they had fallen the same distance that you fell.
This roll is made separately from the damage roll to determine your falling damage.
\crit As above, and each target is knocked \glossterm{prone}.
This does not deal double damage on a critical hit.



\vspace{0.25em}
\spelltwocol{\textbf{Type}: Boots}{}
\textbf{Materials}: Bone, leather, metal


\lowercase{\hypertarget{item:Crown of Flame}{}}\label{item:Crown of Flame}
\hypertarget{item:Crown of Flame}{\subsubsection{Crown of Flame\hfill\nth{9} (4,000 gp)}}

This crown is continuously on fire.
The flame sheds light as a torch.

You and your \glossterm{allies} within a \arealarge radius emanation from you
gain a \glossterm{magic bonus} equal to this item's \glossterm{power} to \glossterm{resistances} against fire damage.



\vspace{0.25em}
\spelltwocol{\textbf{Type}: Crown}{}
\textbf{Materials}: Bone, metal


\lowercase{\hypertarget{item:Crown of Frost}{}}\label{item:Crown of Frost}
\hypertarget{item:Crown of Frost}{\subsubsection{Crown of Frost\hfill\nth{13} (25,000 gp)}}

At the end of each \glossterm{action phase}, you make an attack vs. Fortitude against all enemies within a \areamed radius emanation from you.
At hit deals cold \glossterm{standard damage} \minus2d.



\vspace{0.25em}
\spelltwocol{\textbf{Type}: Crown}{}
\textbf{Materials}: Bone, metal


\lowercase{\hypertarget{item:Crown of Lightning}{}}\label{item:Crown of Lightning}
\hypertarget{item:Crown of Lightning}{\subsubsection{Crown of Lightning\hfill\nth{7} (1,800 gp)}}

This crown continuously crackles with electricity.
The constant sparks shed light as a torch.

At the end of each \glossterm{action phase}, you make an attack vs. Fortitude against all enemies within a \areamed radius emanation from you.
A hit deals electricity \glossterm{standard damage} \minus3d.



\vspace{0.25em}
\spelltwocol{\textbf{Type}: Crown}{}
\textbf{Materials}: Bone, metal


\lowercase{\hypertarget{item:Crown of Thunder}{}}\label{item:Crown of Thunder}
\hypertarget{item:Crown of Thunder}{\subsubsection{Crown of Thunder\hfill\nth{11} (10,000 gp)}}

The crown constantly emits a low-pitched rumbling.
To you and your \glossterm{allies}, the sound is barely perceptible.
However, all other creatures within a \arealarge radius emanation from you hear the sound as a deafening, continuous roll of thunder.
The noise blocks out all other sounds quieter than thunder, causing them to be \deafened while they remain in the area.



\vspace{0.25em}
\spelltwocol{\textbf{Type}: Crown}{}
\textbf{Materials}: Bone, metal


\lowercase{\hypertarget{item:Featherlight Armor}{}}\label{item:Featherlight Armor}
\hypertarget{item:Featherlight Armor}{\subsubsection{Featherlight Armor\hfill\nth{4} (500 gp)}}

This armor's \glossterm{encumbrance} is reduced by 1.



\vspace{0.25em}
\spelltwocol{\textbf{Type}: Body armor}{}
\textbf{Materials}: Bone, metal


\lowercase{\hypertarget{item:Featherlight Armor, Greater}{}}\label{item:Featherlight Armor, Greater}
\hypertarget{item:Featherlight Armor, Greater}{\subsubsection{Featherlight Armor, Greater\hfill\nth{10} (6,500 gp)}}

This armor's \glossterm{encumbrance} is reduced by 2.



\vspace{0.25em}
\spelltwocol{\textbf{Type}: Body armor}{}
\textbf{Materials}: Bone, metal


\lowercase{\hypertarget{item:Gauntlet of the Ram}{}}\label{item:Gauntlet of the Ram}
\hypertarget{item:Gauntlet of the Ram}{\subsubsection{Gauntlet of the Ram\hfill\nth{2} (125 gp)}}

When you make a \glossterm{strike} with this gauntlet, you also compare the attack result to the target's Fortitude defense.
On a hit, you \glossterm{knockback} the target up to 10 feet.
Making a strike with this gauntlet is equivalent to an \glossterm{unarmed attack}.



\vspace{0.25em}
\spelltwocol{\textbf{Type}: Gauntlet}{}
\textbf{Materials}: Bone, metal, wood


\lowercase{\hypertarget{item:Gauntlet of the Ram, Greater}{}}\label{item:Gauntlet of the Ram, Greater}
\hypertarget{item:Gauntlet of the Ram, Greater}{\subsubsection{Gauntlet of the Ram, Greater\hfill\nth{7} (1,800 gp)}}

This item functions like the \mitem{gauntlet of the ram}, except that you \glossterm{knockback} the target up to 30 feet.



\vspace{0.25em}
\spelltwocol{\textbf{Type}: Gauntlet}{}
\textbf{Materials}: Bone, metal, wood


\lowercase{\hypertarget{item:Gauntlets of Improvisation}{}}\label{item:Gauntlets of Improvisation}
\hypertarget{item:Gauntlets of Improvisation}{\subsubsection{Gauntlets of Improvisation\hfill\nth{2} (125 gp)}}

You gain a \plus1d \glossterm{magic bonus} to damage with \glossterm{improvised weapons}.



\vspace{0.25em}
\spelltwocol{\textbf{Type}: Gauntlet}{}
\textbf{Materials}: Bone, metal, wood


\lowercase{\hypertarget{item:Gauntlets of Improvisation, Greater}{}}\label{item:Gauntlets of Improvisation, Greater}
\hypertarget{item:Gauntlets of Improvisation, Greater}{\subsubsection{Gauntlets of Improvisation, Greater\hfill\nth{7} (1,800 gp)}}

This item functions like the \mitem{gauntlets of improvisation}, except that the damage bonus is increased to \plus2d.



\vspace{0.25em}
\spelltwocol{\textbf{Type}: Gauntlet}{}
\textbf{Materials}: Bone, metal, wood


\lowercase{\hypertarget{item:Gloves of Spell Investment}{}}\label{item:Gloves of Spell Investment}
\hypertarget{item:Gloves of Spell Investment}{\subsubsection{Gloves of Spell Investment\hfill\nth{7} (1,800 gp)}}

When you cast a spell that does not have the \glossterm{AP}, \glossterm{Attune}, \glossterm{Sustain} tags,
you can invest the magic of the spell in these gloves.
If you do, the spell does not have its normal effect.

As a standard action, you can activate these gloves.
When you do, you cause the effect of the last spell invested in the gloves.
This does not require \glossterm{concentration} or \glossterm{somatic components}.
After you use a spell in this way, the energy in the gloves is spent, and you must invest a new spell to activate the gloves again.

If you remove either glove from your hand, the magic of the spell invested in the gloves is lost.



\vspace{0.25em}
\spelltwocol{\textbf{Type}: Gloves}{}
\textbf{Materials}: Leather


\lowercase{\hypertarget{item:Gloves of Spell Investment, Greater}{}}\label{item:Gloves of Spell Investment, Greater}
\hypertarget{item:Gloves of Spell Investment, Greater}{\subsubsection{Gloves of Spell Investment, Greater\hfill\nth{13} (25,000 gp)}}

This item functions like the \mitem{gloves of spell investment}, except that you can store up to two spells in the gloves.
When you activate the gauntlets, you choose which spell to use.



\vspace{0.25em}
\spelltwocol{\textbf{Type}: Gloves}{}
\textbf{Materials}: Leather


\lowercase{\hypertarget{item:Greatreach Bracers}{}}\label{item:Greatreach Bracers}
\hypertarget{item:Greatreach Bracers}{\subsubsection{Greatreach Bracers\hfill\nth{9} (4,000 gp)}}

Your \glossterm{reach} is increased by 5 feet.



\vspace{0.25em}
\spelltwocol{\textbf{Type}: Bracers}{}
\textbf{Materials}: Bone, leather, metal, wood


\lowercase{\hypertarget{item:Greatreach Bracers, Greater}{}}\label{item:Greatreach Bracers, Greater}
\hypertarget{item:Greatreach Bracers, Greater}{\subsubsection{Greatreach Bracers, Greater\hfill\nth{17} (125,000 gp)}}

Your \glossterm{reach} is increased by 10 feet.



\vspace{0.25em}
\spelltwocol{\textbf{Type}: Bracers}{}
\textbf{Materials}: Bone, leather, metal, wood


\lowercase{\hypertarget{item:Hexproof Amulet, Greater}{}}\label{item:Hexproof Amulet, Greater}
\hypertarget{item:Hexproof Amulet, Greater}{\subsubsection{Hexproof Amulet, Greater\hfill\nth{15} (55,000 gp)}}

You gain a \plus4 bonus to defenses against \glossterm{magical} abilities that target you directly.
This does not protect you from abilities that affect an area.



\vspace{0.25em}
\spelltwocol{\textbf{Type}: Amulet}{}
\textbf{Materials}: Jewelry


\lowercase{\hypertarget{item:Hexward Amulet}{}}\label{item:Hexward Amulet}
\hypertarget{item:Hexward Amulet}{\subsubsection{Hexward Amulet\hfill\nth{9} (4,000 gp)}}

You gain a \plus2 bonus to defenses against \glossterm{magical} abilities that target you directly.
This does not protect you from abilities that affect an area.



\vspace{0.25em}
\spelltwocol{\textbf{Type}: Amulet}{}
\textbf{Materials}: Jewelry


\lowercase{\hypertarget{item:Hidden Armor}{}}\label{item:Hidden Armor}
\hypertarget{item:Hidden Armor}{\subsubsection{Hidden Armor\hfill\nth{4} (500 gp)}}

As a standard action, you can use this item.
If you do, it appears to change shape and form to assume the shape of a normal set of clothing.
You may choose the design of the clothing.
The item retains all of its properties, including weight and sound, while disguised in this way.
Only its visual appearance is altered.

Alternately, you may return the armor to its original appearance.



\vspace{0.25em}
\spelltwocol{\textbf{Type}: Body armor}{\parhead*{Tags} \glossterm{Sensation}}
\textbf{Materials}: Bone, metal


\lowercase{\hypertarget{item:Hidden Armor, Greater}{}}\label{item:Hidden Armor, Greater}
\hypertarget{item:Hidden Armor, Greater}{\subsubsection{Hidden Armor, Greater\hfill\nth{9} (4,000 gp)}}

This item functions like the \mitem{hidden armor} item, except that the item also makes sound appropriate to its disguised form while disguised.



\vspace{0.25em}
\spelltwocol{\textbf{Type}: Body armor}{\parhead*{Tags} \glossterm{Sensation}}
\textbf{Materials}: Bone, metal


\lowercase{\hypertarget{item:Lifekeeping Belt}{}}\label{item:Lifekeeping Belt}
\hypertarget{item:Lifekeeping Belt}{\subsubsection{Lifekeeping Belt\hfill\nth{7} (1,800 gp)}}

You gain a \plus1 \glossterm{magic bonus} to \glossterm{wound rolls}.



\vspace{0.25em}
\spelltwocol{\textbf{Type}: Belt}{}
\textbf{Materials}: Leather, textiles


\lowercase{\hypertarget{item:Lifekeeping Belt, Greater}{}}\label{item:Lifekeeping Belt, Greater}
\hypertarget{item:Lifekeeping Belt, Greater}{\subsubsection{Lifekeeping Belt, Greater\hfill\nth{13} (25,000 gp)}}

You gain a \plus2 \glossterm{magic bonus} to \glossterm{wound rolls}.



\vspace{0.25em}
\spelltwocol{\textbf{Type}: Belt}{}
\textbf{Materials}: Leather, textiles


\lowercase{\hypertarget{item:Lifekeeping Belt, Supreme}{}}\label{item:Lifekeeping Belt, Supreme}
\hypertarget{item:Lifekeeping Belt, Supreme}{\subsubsection{Lifekeeping Belt, Supreme\hfill\nth{19} (280,000 gp)}}

You gain a \plus3 \glossterm{magic bonus} to \glossterm{wound rolls}.



\vspace{0.25em}
\spelltwocol{\textbf{Type}: Belt}{}
\textbf{Materials}: Leather, textiles


\lowercase{\hypertarget{item:Mask of Air}{}}\label{item:Mask of Air}
\hypertarget{item:Mask of Air}{\subsubsection{Mask of Air\hfill\nth{9} (4,000 gp)}}

If you breathe through this mask, you breathe in clean, fresh air, regardless of your environment.
This can protect you from inhaled poisons and similar effects.



\vspace{0.25em}
\spelltwocol{\textbf{Type}: Mask}{}
\textbf{Materials}: Textiles


\lowercase{\hypertarget{item:Mask of Water Breathing}{}}\label{item:Mask of Water Breathing}
\hypertarget{item:Mask of Water Breathing}{\subsubsection{Mask of Water Breathing\hfill\nth{4} (500 gp)}}

You can breathe water through this mask as easily as a human breaths air.
This does not grant you the ability to breathe other liquids.



\vspace{0.25em}
\spelltwocol{\textbf{Type}: Mask}{}
\textbf{Materials}: Textiles


\lowercase{\hypertarget{item:Ocular Circlet}{}}\label{item:Ocular Circlet}
\hypertarget{item:Ocular Circlet}{\subsubsection{Ocular Circlet\hfill\nth{3} (250 gp)}}

As a \glossterm{standard action}, you can concentrate to use this item.
If you do, a \glossterm{scrying sensor} appears floating in the air in an unoccupied square within \rngclose range.
As long as you \glossterm{sustain} the effect as a standard action, you see through the sensor instead of from your body.

While viewing through the sensor, your visual acuity is the same as your normal body,
except that it does not share the benefits of any \glossterm{magical} effects that improve your vision.
You otherwise act normally, though you may have difficulty moving or taking actions if the sensor cannot see your body or your intended targets, effectively making you \blinded.



\vspace{0.25em}
\spelltwocol{\textbf{Type}: Circlet}{\parhead*{Tags} \glossterm{Scrying}}
\textbf{Materials}: Bone, metal


\lowercase{\hypertarget{item:Ocular Circlet, Greater}{}}\label{item:Ocular Circlet, Greater}
\hypertarget{item:Ocular Circlet, Greater}{\subsubsection{Ocular Circlet, Greater\hfill\nth{9} (4,000 gp)}}

This item functions like the \mitem{ocular circlet}, except that it only takes a \glossterm{minor action} to activate and sustain the item's effect.
In addition, the sensor appears anywhere within \rngmed range.



\vspace{0.25em}
\spelltwocol{\textbf{Type}: Circlet}{\parhead*{Tags} \glossterm{Scrying}}
\textbf{Materials}: Bone, metal


\lowercase{\hypertarget{item:Protective Armor}{}}\label{item:Protective Armor}
\hypertarget{item:Protective Armor}{\subsubsection{Protective Armor\hfill\nth{7} (1,800 gp)}}

You gain a \plus1 \glossterm{magic bonus} to Armor defense.



\vspace{0.25em}
\spelltwocol{\textbf{Type}: Body armor}{}
\textbf{Materials}: Bone, metal


\lowercase{\hypertarget{item:Protective Shield}{}}\label{item:Protective Shield}
\hypertarget{item:Protective Shield}{\subsubsection{Protective Shield\hfill\nth{7} (1,800 gp)}}

You gain a \plus1 \glossterm{magic bonus} to Armor defense.



\vspace{0.25em}
\spelltwocol{\textbf{Type}: Shield}{}
\textbf{Materials}: Bone, metal, wood


\lowercase{\hypertarget{item:Ring of Angel's Grace}{}}\label{item:Ring of Angel's Grace}
\hypertarget{item:Ring of Angel's Grace}{\subsubsection{Ring of Angel's Grace\hfill\nth{9} (4,000 gp)}}

You gain \plus2 \glossterm{magic bonus} to Mental defense.
In addition, if you fall at least 20 feet, ephemeral angel wings spring from your back.
The wings slow your fall to a rate of 60 feet per round, preventing you from taking \glossterm{falling damage}.



\vspace{0.25em}
\spelltwocol{\textbf{Type}: Ring}{}
\textbf{Materials}: Bone, jewelry, metal, wood


\lowercase{\hypertarget{item:Ring of Elemental Endurance}{}}\label{item:Ring of Elemental Endurance}
\hypertarget{item:Ring of Elemental Endurance}{\subsubsection{Ring of Elemental Endurance\hfill\nth{2} (125 gp)}}

You can exist comfortably in conditions between \minus50 and 140 degrees Fahrenheit without any ill effects.
You suffer the normal penalties in temperatures outside of that range.



\vspace{0.25em}
\spelltwocol{\textbf{Type}: Ring}{}
\textbf{Materials}: Bone, jewelry, metal, wood


\lowercase{\hypertarget{item:Ring of Energy Resistance}{}}\label{item:Ring of Energy Resistance}
\hypertarget{item:Ring of Energy Resistance}{\subsubsection{Ring of Energy Resistance\hfill\nth{6} (1,200 gp)}}

You gain a \glossterm{magic bonus} equal to half this item's \glossterm{power} to \glossterm{resistances} against \glossterm{energy damage}.
When you resist energy with this ability, the ring sheds light as a torch until the end of the next round.
The color of the light depends on the energy damage resisted: green for acid, blue for cold, yellow for electricity, and red for fire.



\vspace{0.25em}
\spelltwocol{\textbf{Type}: Ring}{}
\textbf{Materials}: Bone, jewelry, metal, wood


\lowercase{\hypertarget{item:Ring of Energy Resistance, Greater}{}}\label{item:Ring of Energy Resistance, Greater}
\hypertarget{item:Ring of Energy Resistance, Greater}{\subsubsection{Ring of Energy Resistance, Greater\hfill\nth{15} (55,000 gp)}}

This item functions like the \mitem{ring of energy resistance}, except that the bonus is equal to the item's \glossterm{power}.



\vspace{0.25em}
\spelltwocol{\textbf{Type}: Ring}{}
\textbf{Materials}: Bone, jewelry, metal, wood


\lowercase{\hypertarget{item:Ring of Nourishment}{}}\label{item:Ring of Nourishment}
\hypertarget{item:Ring of Nourishment}{\subsubsection{Ring of Nourishment\hfill\nth{3} (250 gp)}}

You continuously gain nourishment, and no longer need to eat or drink.
This ring must be worn for 24 hours before it begins to work.



\vspace{0.25em}
\spelltwocol{\textbf{Type}: Ring}{\parhead*{Tags} \glossterm{Creation}}
\textbf{Materials}: Bone, jewelry, metal, wood


\lowercase{\hypertarget{item:Ring of Protection}{}}\label{item:Ring of Protection}
\hypertarget{item:Ring of Protection}{\subsubsection{Ring of Protection\hfill\nth{8} (2,750 gp)}}

This ring creates a transluscent shield-like barrier that floats in front of you, deflecting enemy attacks.
You gain a \plus1 \glossterm{magic bonus} to Armor and Reflex defenses.
This does not stack with the defense bonus from any shields you use.



\vspace{0.25em}
\spelltwocol{\textbf{Type}: Ring}{}
\textbf{Materials}: Bone, jewelry, metal, wood


\lowercase{\hypertarget{item:Ring of Protection, Greater}{}}\label{item:Ring of Protection, Greater}
\hypertarget{item:Ring of Protection, Greater}{\subsubsection{Ring of Protection, Greater\hfill\nth{16} (85,000 gp)}}

This item functions like the \magicitem{ring of protection}, except that the bonus increases to \plus2.



\vspace{0.25em}
\spelltwocol{\textbf{Type}: Ring}{}
\textbf{Materials}: Bone, jewelry, metal, wood


\lowercase{\hypertarget{item:Ring of Regeneration}{}}\label{item:Ring of Regeneration}
\hypertarget{item:Ring of Regeneration}{\subsubsection{Ring of Regeneration\hfill\nth{8} (2,750 gp)}}

As a standard action, you can use this ring to enhance your healing.
When you do, you gain a \plus1 bonus to the \glossterm{wound roll} of your most recent \glossterm{vital wound}.



\vspace{0.25em}
\spelltwocol{\textbf{Type}: Ring}{}
\textbf{Materials}: Bone, jewelry, metal, wood


\lowercase{\hypertarget{item:Ring of Regeneration, Greater}{}}\label{item:Ring of Regeneration, Greater}
\hypertarget{item:Ring of Regeneration, Greater}{\subsubsection{Ring of Regeneration, Greater\hfill\nth{12} (16,000 gp)}}

At the end of each round, you gain a \plus1 bonus to the \glossterm{wound roll} of your most recent \glossterm{vital wound}.



\vspace{0.25em}
\spelltwocol{\textbf{Type}: Ring}{}
\textbf{Materials}: Bone, jewelry, metal, wood


\lowercase{\hypertarget{item:Ring of Sustenance}{}}\label{item:Ring of Sustenance}
\hypertarget{item:Ring of Sustenance}{\subsubsection{Ring of Sustenance\hfill\nth{7} (1,800 gp)}}

You continuously gain nourishment, and no longer need to eat or drink.
In addition, you need only one-quarter your normal amount of sleep (or similar activity, such as elven trance) each day.

The ring must be worn for 24 hours before it begins to work.



\vspace{0.25em}
\spelltwocol{\textbf{Type}: Ring}{\parhead*{Tags} \glossterm{Creation}}
\textbf{Materials}: Bone, jewelry, metal, wood


\lowercase{\hypertarget{item:Seven League Boots}{}}\label{item:Seven League Boots}
\hypertarget{item:Seven League Boots}{\subsubsection{Seven League Boots\hfill\nth{12} (16,000 gp)}}

As a standard action, you can spend an \glossterm{action point} to activate these boots.
If you do, you teleport exactly 25 miles in a direction you specify.
If this would place you within a solid object or otherwise impossible space, the boots will shunt you up to 1,000 feet in any direction to the closest available space.
If there is no available space within 1,000 feet of your intended destination, the effect fails and you take \glossterm{standard damage} \minus1d.



\vspace{0.25em}
\spelltwocol{\textbf{Type}: Boots}{}
\textbf{Materials}: Bone, leather, metal


\lowercase{\hypertarget{item:Shield of Arrow Catching}{}}\label{item:Shield of Arrow Catching}
\hypertarget{item:Shield of Arrow Catching}{\subsubsection{Shield of Arrow Catching\hfill\nth{5} (800 gp)}}

When a creature within a \areamed radius emanation from you would be attacked by a ranged weapon, the attack is redirected to target you instead.
Resolve the attack as if it had initially targeted you, except that the attack is not affected by cover or concealment.
This item can only affect projectiles and thrown objects that are Small or smaller.



\vspace{0.25em}
\spelltwocol{\textbf{Type}: Shield}{}
\textbf{Materials}: Bone, metal, wood


\lowercase{\hypertarget{item:Shield of Arrow Catching, Greater}{}}\label{item:Shield of Arrow Catching, Greater}
\hypertarget{item:Shield of Arrow Catching, Greater}{\subsubsection{Shield of Arrow Catching, Greater\hfill\nth{10} (6,500 gp)}}

This item functions like the \mitem{shield of arrow catching} item, except that it affects a \arealarge radius from you.
In addition, you may choose to exclude creature from this item's effect, allowing projectiles to target nearby foes normally.



\vspace{0.25em}
\spelltwocol{\textbf{Type}: Shield}{}
\textbf{Materials}: Bone, metal, wood


\lowercase{\hypertarget{item:Shield of Arrow Deflection}{}}\label{item:Shield of Arrow Deflection}
\hypertarget{item:Shield of Arrow Deflection}{\subsubsection{Shield of Arrow Deflection\hfill\nth{2} (125 gp)}}

As a \glossterm{minor action}, you can activate this shield.
If you do, you gain a \plus5 \glossterm{magic bonus} to Armor defense against ranged \glossterm{physical attacks} from weapons or projectiles that are Small or smaller.
This is a \glossterm{Swift} ability, and it lasts until the end of the round.



\vspace{0.25em}
\spelltwocol{\textbf{Type}: Shield}{}
\textbf{Materials}: Bone, metal, wood


\lowercase{\hypertarget{item:Shield of Arrow Deflection, Greater}{}}\label{item:Shield of Arrow Deflection, Greater}
\hypertarget{item:Shield of Arrow Deflection, Greater}{\subsubsection{Shield of Arrow Deflection, Greater\hfill\nth{8} (2,750 gp)}}

You gain a \plus5 \glossterm{magic bonus} to Armor defense against ranged \glossterm{physical attacks} from weapons or projectiles that are Small or smaller.



\vspace{0.25em}
\spelltwocol{\textbf{Type}: Shield}{}
\textbf{Materials}: Bone, metal, wood


\lowercase{\hypertarget{item:Shield of Bashing}{}}\label{item:Shield of Bashing}
\hypertarget{item:Shield of Bashing}{\subsubsection{Shield of Bashing\hfill\nth{2} (125 gp)}}

You gain a \plus1d \glossterm{magic bonus} to damage with \glossterm{strikes} using this shield.



\vspace{0.25em}
\spelltwocol{\textbf{Type}: Shield}{}
\textbf{Materials}: Bone, metal, wood


\lowercase{\hypertarget{item:Shield of Bashing, Greater}{}}\label{item:Shield of Bashing, Greater}
\hypertarget{item:Shield of Bashing, Greater}{\subsubsection{Shield of Bashing, Greater\hfill\nth{12} (16,000 gp)}}

You gain a \plus2d \glossterm{magic bonus} to damage with \glossterm{strikes} using this shield.



\vspace{0.25em}
\spelltwocol{\textbf{Type}: Shield}{}
\textbf{Materials}: Bone, metal, wood


\lowercase{\hypertarget{item:Shield of Boulder Catching}{}}\label{item:Shield of Boulder Catching}
\hypertarget{item:Shield of Boulder Catching}{\subsubsection{Shield of Boulder Catching\hfill\nth{8} (2,750 gp)}}

This item functions like the \mitem{shield of arrow catching} item, except that it can affect projectile and thrown objects of up to Large size.



\vspace{0.25em}
\spelltwocol{\textbf{Type}: Shield}{}
\textbf{Materials}: Bone, metal, wood


\lowercase{\hypertarget{item:Shield of Boulder Deflection}{}}\label{item:Shield of Boulder Deflection}
\hypertarget{item:Shield of Boulder Deflection}{\subsubsection{Shield of Boulder Deflection\hfill\nth{6} (1,200 gp)}}

This item functions like the \mitem{shield of arrow deflection} item, except that it can affect weapons and projectiles of up to Large size.



\vspace{0.25em}
\spelltwocol{\textbf{Type}: Shield}{}
\textbf{Materials}: Bone, metal, wood


\lowercase{\hypertarget{item:Shield of Boulder Deflection, Greater}{}}\label{item:Shield of Boulder Deflection, Greater}
\hypertarget{item:Shield of Boulder Deflection, Greater}{\subsubsection{Shield of Boulder Deflection, Greater\hfill\nth{12} (16,000 gp)}}

This item functions like the \mitem{greater shield of arrow deflection} item, except that it can affect weapons and projectiles of up to Large size.



\vspace{0.25em}
\spelltwocol{\textbf{Type}: Shield}{}
\textbf{Materials}: Bone, metal, wood


\lowercase{\hypertarget{item:Shield of Mystic Reflection}{}}\label{item:Shield of Mystic Reflection}
\hypertarget{item:Shield of Mystic Reflection}{\subsubsection{Shield of Mystic Reflection\hfill\nth{12} (16,000 gp)}}

As a standard action, you can activate this shield.
When you do, any \glossterm{targeted} \glossterm{magical} abilities that would target you this round are redirected to target the creature using that ability instead of you.
Any other targets of the ability are affected normally.
This is a \glossterm{Swift} ability, so it affects any abilities targeting you in the phase you activate the item.



\vspace{0.25em}
\spelltwocol{\textbf{Type}: Shield}{}
\textbf{Materials}: Bone, metal, wood


\lowercase{\hypertarget{item:Throwing Gloves}{}}\label{item:Throwing Gloves}
\hypertarget{item:Throwing Gloves}{\subsubsection{Throwing Gloves\hfill\nth{4} (500 gp)}}

% TODO: reference basic "not designed to be thrown" mechanics?
You can throw any item as if it was designed to be thrown.
This does not improve your ability to throw items designed to be thrown, such as darts.



\vspace{0.25em}
\spelltwocol{\textbf{Type}: Gloves}{}
\textbf{Materials}: Leather


\lowercase{\hypertarget{item:Titan Gauntlets}{}}\label{item:Titan Gauntlets}
\hypertarget{item:Titan Gauntlets}{\subsubsection{Titan Gauntlets\hfill\nth{13} (25,000 gp)}}

You gain a \plus1d \glossterm{magic bonus} to damage with \glossterm{strikes}.



\vspace{0.25em}
\spelltwocol{\textbf{Type}: Gauntlet}{}
\textbf{Materials}: Bone, metal, wood


\lowercase{\hypertarget{item:Torchlight Gloves}{}}\label{item:Torchlight Gloves}
\hypertarget{item:Torchlight Gloves}{\subsubsection{Torchlight Gloves\hfill\nth{2} (125 gp)}}

These gloves shed light as a torch.
As a \glossterm{standard action}, you may snap your fingers to suppress or resume the light from either or both gloves.



\vspace{0.25em}
\spelltwocol{\textbf{Type}: Gloves}{}
\textbf{Materials}: Leather


\lowercase{\hypertarget{item:Vanishing Cloak}{}}\label{item:Vanishing Cloak}
\hypertarget{item:Vanishing Cloak}{\subsubsection{Vanishing Cloak\hfill\nth{13} (25,000 gp)}}

As a standard action, you can activate this cloak.
When you do, you teleport to an unoccupied location within \rngmed range of your original location.
In addition, you become \glossterm{invisible} until the end of the next round.

If your intended destination is invalid, or if your teleportation otherwise fails, you still become invisible.



\vspace{0.25em}
\spelltwocol{\textbf{Type}: Cloak}{\parhead*{Tags} \glossterm{Sensation}}
\textbf{Materials}: Textiles


\lowercase{\hypertarget{item:Winged Boots}{}}\label{item:Winged Boots}
\hypertarget{item:Winged Boots}{\subsubsection{Winged Boots\hfill\nth{10} (6,500 gp)}}

You gain a \glossterm{fly speed} equal to your \glossterm{base speed}.
However, the boots are not strong enough to keep you aloft indefinitely.
At the end of each round, if you are not standing on solid ground, the magic of the boots fails and you fall normally.
The boots begin working again at the end of the next round, even if you have not yet hit the ground.



\vspace{0.25em}
\spelltwocol{\textbf{Type}: Boots}{}
\textbf{Materials}: Bone, leather, metal


\section{Weapons} % or implements?
    Magic weapons improve a character's combat abilities.
    They must be wielded to gain their effects.

    \parhead{Ranged Weapons and Ammunition} Any magical properties of a projectile weapon also apply to all ammunition fired from that weapon.

    \subsection{Weapon Description}

        
\begin{longtablewrapper}
\begin{longtable}{p{15em} p{3em} p{6em} p{25em} p{3em}}

\lcaption{Weapon Items} \\
\tb{Name} & \tb{Level} & \tb{Typical Price} & \tb{Description} & \tb{Page} \tableheaderrule
Longshot & \nth{2} & 125 gp & Ignores one range increment & \pageref{item:Longshot} \\
Morphing & \nth{2} & 125 gp & Can change into similar weapon & \pageref{item:Morphing} \\
Merciful & \nth{3} & 250 gp & Deals subdual damage & \pageref{item:Merciful} \\
Returning & \nth{3} & 250 gp & Teleports back to you after being thrown & \pageref{item:Returning} \\
Forceful & \nth{4} & 500 gp & Can shove struck foes & \pageref{item:Forceful} \\
Concussive & \nth{6} & 1,200 gp & Can daze a foe & \pageref{item:Concussive} \\
Flaming & \nth{6} & 1,200 gp & Deals fire damage & \pageref{item:Flaming} \\
Freezing & \nth{6} & 1,200 gp & Deals cold damage & \pageref{item:Freezing} \\
Shocking & \nth{6} & 1,200 gp & Deals electicity damage & \pageref{item:Shocking} \\
Protective & \nth{7} & 1,800 gp & Grants \plus1 Armor defense & \pageref{item:Protective} \\
Seeking & \nth{7} & 1,800 gp & Reduces miss chances & \pageref{item:Seeking} \\
Thieving & \nth{7} & 1,800 gp & Can absorb small items & \pageref{item:Thieving} \\
Longshot, Greater & \nth{8} & 2,750 gp & Ignores two range increments & \pageref{item:Longshot, Greater} \\
Morphing, Greater & \nth{8} & 2,750 gp & Can change into any weapon & \pageref{item:Morphing, Greater} \\
Phasing & \nth{8} & 2,750 gp & Can ignore obstacles when attacking & \pageref{item:Phasing} \\
Potency & \nth{8} & 2,750 gp & Grants \plus2 \glossterm{mundane} power & \pageref{item:Potency} \\
Surestrike & \nth{8} & 2,750 gp & Grants \plus1 accuracy bonus & \pageref{item:Surestrike} \\
Boomerang & \nth{9} & 4,000 gp & Can be thrown to strike multiple foes & \pageref{item:Boomerang} \\
Soulreaving & \nth{9} & 4,000 gp & Deals delayed damage & \pageref{item:Soulreaving} \\
Disorienting & \nth{10} & 6,500 gp & Can disorient struck foes & \pageref{item:Disorienting} \\
Flaming, Greater & \nth{12} & 16,000 gp & Deals fire damage, can ignite foes & \pageref{item:Flaming, Greater} \\
Freezing, Greater & \nth{12} & 16,000 gp & Deals cold damage, can chill & \pageref{item:Freezing, Greater} \\
Shocking, Greater & \nth{12} & 16,000 gp & Deals electricity damage, can daze foes & \pageref{item:Shocking, Greater} \\
Forceful, Greater & \nth{13} & 25,000 gp & Shoves struck foes & \pageref{item:Forceful, Greater} \\
Protective, Greater & \nth{13} & 25,000 gp & Grants \plus2 Armor defense & \pageref{item:Protective, Greater} \\
Thieving, Greater & \nth{13} & 25,000 gp & Can absorb large items & \pageref{item:Thieving, Greater} \\
Longshot, Supreme & \nth{14} & 37,000 gp & Ignores three range increments & \pageref{item:Longshot, Supreme} \\
Phasing, Greater & \nth{14} & 37,000 gp & Can ignore many obstacles when attacking & \pageref{item:Phasing, Greater} \\
Potency, Greater & \nth{14} & 37,000 gp & Grants \plus4 \glossterm{mundane} power & \pageref{item:Potency, Greater} \\
Surestrike, Greater & \nth{14} & 37,000 gp & Grants \plus2 accuracy bonus & \pageref{item:Surestrike, Greater} \\
Fixating & \nth{15} & 55,000 gp & Grants accuracy bonus against struck foe & \pageref{item:Fixating} \\
Soulreaving, Greater & \nth{15} & 55,000 gp & Deals delayed damage that can be quickly converted & \pageref{item:Soulreaving, Greater} \\
Disorienting, Greater & \nth{16} & 85,000 gp & Disorients struck foes & \pageref{item:Disorienting, Greater} \\
Flaming, Supreme & \nth{18} & 190,000 gp & Deals fire damage, ignites foes & \pageref{item:Flaming, Supreme} \\
Freezing, Supreme & \nth{18} & 190,000 gp & Deals cold damage, chills foes & \pageref{item:Freezing, Supreme} \\
Shocking, Supreme & \nth{18} & 190,000 gp & Deals electicity damage, dazes foes & \pageref{item:Shocking, Supreme} \\
Protective, Supreme & \nth{19} & 280,000 gp & Grants \plus3 Armor defense & \pageref{item:Protective, Supreme} \\
Heartseeker & \nth{20} & 400,000 gp & Rolls attacks twice & \pageref{item:Heartseeker} \\
Potency, Supreme & \nth{20} & 400,000 gp & Grants \plus6 \glossterm{mundane} power & \pageref{item:Potency, Supreme} \\
Surestrike, Supreme & \nth{20} & 400,000 gp & Grants \plus3 accuracy bonus & \pageref{item:Surestrike, Supreme} \\
Vorpal & \nth{20} & 400,000 gp & Inflicts lethal critical hits & \pageref{item:Vorpal} \\

\end{longtable}
\end{longtablewrapper}


        
\lowercase{\hypertarget{item:Concussive}{}}\label{item:Concussive}
\hypertarget{item:Concussive}{\subsubsection{Concussive\hfill\nth{4}}}

As a standard action, you can infuse this weapon with concussive force.
The next time you make a \glossterm{strike} with this weapon, if your attack result beats the target's Fortitude defense, it is \glossterm{dazed} as a \glossterm{condition}.



\parhead*{Materials} As weapon


\lowercase{\hypertarget{item:Cutthroat}{}}\label{item:Cutthroat}
\hypertarget{item:Cutthroat}{\subsubsection{Cutthroat\hfill\nth{4}}}

As a standard action, you can make a \glossterm{strike} with this weapon.
In addition to the normal effects of the strike, if your attack result beats the target's Fortitude defense, it is \glossterm{muted} as a \glossterm{condition}.



\parhead*{Materials} As weapon


\lowercase{\hypertarget{item:Defending}{}}\label{item:Defending}
\hypertarget{item:Defending}{\subsubsection{Defending\hfill\nth{9}}}

You gain a \plus1 \glossterm{magic bonus} to Armor defense.



\parhead*{Tags} \glossterm{Shielding}


\parhead*{Materials} As weapon


\lowercase{\hypertarget{item:Disorienting}{}}\label{item:Disorienting}
\hypertarget{item:Disorienting}{\subsubsection{Disorienting\hfill\nth{9}}}

This weapon shimmers with a chaotic pattern of colors.
As a \glossterm{minor action}, you can intensify the shimmering.
If you do, when you make a \glossterm{strike}  with this weapon and your attack result beats the target's Mental defense, it is \disoriented as a \glossterm{condition}.
This is a \glossterm{Swift} ability, and it lasts until the end of the round.



\parhead*{Tags} \glossterm{Compulsion}, \glossterm{Mind}


\parhead*{Materials} As weapon


\lowercase{\hypertarget{item:Disorienting, Greater}{}}\label{item:Disorienting, Greater}
\hypertarget{item:Disorienting, Greater}{\subsubsection{Disorienting, Greater\hfill\nth{15}}}

This weapon shimmers with a chaotic pattern of colors.
When you make a \glossterm{strike} with this weapon and your attack result beats the target's Mental defense, it is \disoriented as a \glossterm{condition}.



\parhead*{Tags} \glossterm{Compulsion}, \glossterm{Mind}


\parhead*{Materials} As weapon


\lowercase{\hypertarget{item:Fixating}{}}\label{item:Fixating}
\hypertarget{item:Fixating}{\subsubsection{Fixating\hfill\nth{13}}}

When you make a \glossterm{strike} with this weapon, you gain a \plus1 bonus to accuracy against the target.
This bonus lasts until you make a strike with this weapon against a different target.
This bonus can stack with itself, up to a maximum of \plus5.



\parhead*{Materials} As weapon


\lowercase{\hypertarget{item:Flaming}{}}\label{item:Flaming}
\hypertarget{item:Flaming}{\subsubsection{Flaming\hfill\nth{5}}}

This weapon is on fire.
It sheds light as a torch, and all damage dealt with it is fire damage in addition to its other types.
As a \glossterm{minor action}, you can kindle the flames.
If you do, you gain a \plus1d \glossterm{magic bonus} to \glossterm{damage} with \glossterm{strikes} using this weapon.
This is a \glossterm{Swift} ability, and it lasts until the end of the round.



\parhead*{Tags} \glossterm{Fire}


\parhead*{Materials} As weapon


\lowercase{\hypertarget{item:Flaming, Greater}{}}\label{item:Flaming, Greater}
\hypertarget{item:Flaming, Greater}{\subsubsection{Flaming, Greater\hfill\nth{11}}}

This weapon is on fire.
It sheds light as a torch, and all damage dealt with it is fire damage in addition to its other types.
You gain a \plus1d \glossterm{magic bonus} to \glossterm{damage} with \glossterm{damage} using this weapon.



\parhead*{Tags} \glossterm{Fire}


\parhead*{Materials} As weapon


\lowercase{\hypertarget{item:Forceful}{}}\label{item:Forceful}
\hypertarget{item:Forceful}{\subsubsection{Forceful\hfill\nth{6}}}

This weapon feels heavy in the hand.
As a \glossterm{minor action}, you can intensify the weapon's heft.
If you do, when you make a \glossterm{strike} with this weapon, you can also use your attack result as a \glossterm{shove} attack agsint the target.
You do not need to move with your foe to move it the full distance of the shove.
This is a \glossterm{Swift} ability, and it lasts until the end of the round.



\parhead*{Materials} As weapon


\lowercase{\hypertarget{item:Forceful, Greater}{}}\label{item:Forceful, Greater}
\hypertarget{item:Forceful, Greater}{\subsubsection{Forceful, Greater\hfill\nth{12}}}

This weapon feels heavy in the hand.
When you make a \glossterm{strike} with this weapon, you can also use your attack result as a \glossterm{shove} attack agsint the target.
You do not need to move with your foe to move it the full distance of the shove.



\parhead*{Materials} As weapon


\lowercase{\hypertarget{item:Freezing}{}}\label{item:Freezing}
\hypertarget{item:Freezing}{\subsubsection{Freezing\hfill\nth{4}}}

This weapon is bitterly cold, and all damage dealt with it is cold damage in addition to its other types.
As a \glossterm{minor action}, you can intensify the cold.
If you do, when you make a \glossterm{strike} with this weapon and your attack result beats the target's Fortitude defense, the target is \fatigued as a \glossterm{condition}.
This is a \glossterm{Swift} ability, and it lasts until the end of the round.



\parhead*{Tags} \glossterm{Cold}


\parhead*{Materials} As weapon


\lowercase{\hypertarget{item:Freezing, Greater}{}}\label{item:Freezing, Greater}
\hypertarget{item:Freezing, Greater}{\subsubsection{Freezing, Greater\hfill\nth{10}}}

This weapon is bitterly cold, and all damage dealt with it is cold damage in addition to its other types.
When you make a \glossterm{strike} with this weapon, if your attack result beats the target's Fortitude defense, the target is \fatigued as a \glossterm{condition}.



\parhead*{Tags} \glossterm{Cold}


\parhead*{Materials} As weapon


\lowercase{\hypertarget{item:Heartseeker}{}}\label{item:Heartseeker}
\hypertarget{item:Heartseeker}{\subsubsection{Heartseeker\hfill\nth{17}}}

When you make a \glossterm{strike} with this weapon, you can roll twice and take the higher result.



\parhead*{Tags} \glossterm{Knowledge}


\parhead*{Materials} As weapon


\lowercase{\hypertarget{item:Longshot}{}}\label{item:Longshot}
\hypertarget{item:Longshot}{\subsubsection{Longshot\hfill\nth{4}}}

Ranged attacks with this weapon have twice the normal \glossterm{range increment}.



\parhead*{Materials} As weapon


\lowercase{\hypertarget{item:Longshot, Greater}{}}\label{item:Longshot, Greater}
\hypertarget{item:Longshot, Greater}{\subsubsection{Longshot, Greater\hfill\nth{10}}}

Ranged attacks with this weapon have three times the normal \glossterm{range increment}.



\parhead*{Materials} As weapon


\lowercase{\hypertarget{item:Merciful}{}}\label{item:Merciful}
\hypertarget{item:Merciful}{\subsubsection{Merciful\hfill\nth{3}}}

This weapon deals \glossterm{subdual damage} instead of lethal damage.



\parhead*{Materials} As weapon


\lowercase{\hypertarget{item:Morphing}{}}\label{item:Morphing}
\hypertarget{item:Morphing}{\subsubsection{Morphing\hfill\nth{2}}}

As a standard action, you can spend an \glossterm{action point} to activate this item.
If you do, it changes shape into a new weapon of your choice from the same weapon group.



\parhead*{Tags} \glossterm{Shaping}


\parhead*{Materials} As weapon


\lowercase{\hypertarget{item:Morphing, Greater}{}}\label{item:Morphing, Greater}
\hypertarget{item:Morphing, Greater}{\subsubsection{Morphing, Greater\hfill\nth{6}}}

As a standard action, you can spend an \glossterm{action point} to activate this item.
If you do, it changes shape into a new weapon of your choice that you are proficient with.
This can only change into existing manufactured weapons, not improvised weapons (see \pcref{Weapons}).



\parhead*{Tags} \glossterm{Shaping}


\parhead*{Materials} As weapon


\lowercase{\hypertarget{item:Phasing}{}}\label{item:Phasing}
\hypertarget{item:Phasing}{\subsubsection{Phasing\hfill\nth{9}}}

\glossterm{Strikes} with this weapon can pass through a single solid obstacle of up to five feet thick on the way to their target.
This can allow you to ignore \glossterm{cover}, or even attack through solid walls.
It does not allow you to ignore armor, shields, or or similar items used by the target of your attacks.



\parhead*{Tags} \glossterm{Planar}


\parhead*{Materials} As weapon


\lowercase{\hypertarget{item:Returning}{}}\label{item:Returning}
\hypertarget{item:Returning}{\subsubsection{Returning\hfill\nth{3}}}

After being thrown, this weapon teleports back into your hand at the end of the current phase.
Catching a rebounding weapon when it comes back is a free action.
If you can't catch it, the weapon drops to the ground in the square from which it was thrown.



\parhead*{Tags} \glossterm{Teleportation}


\parhead*{Materials} As weapon


\lowercase{\hypertarget{item:Seeking}{}}\label{item:Seeking}
\hypertarget{item:Seeking}{\subsubsection{Seeking\hfill\nth{7}}}

This weapon automatically veers towards its intended target.
\glossterm{Strikes} with this weapon that would suffer a 50\% miss chance instead suffer a 20\% miss chance.
In addition, attacks that would otherwise suffer a 20\% miss chance instead suffer no miss chance.



\parhead*{Tags} \glossterm{Knowledge}


\parhead*{Materials} As weapon


\lowercase{\hypertarget{item:Shocking}{}}\label{item:Shocking}
\hypertarget{item:Shocking}{\subsubsection{Shocking\hfill\nth{7}}}

This weapon continuously crackles with electricity.
The constant sparks shed light as a torch, and all damage dealt with it is electricity damage in addition to its other types.
As a \glossterm{minor action}, you can intensify the electricity.
If you do, when you make a \glossterm{strike} with this weapon and your attack result beats the target's Fortitude defense, the target is \dazed as a \glossterm{condition}.
This is a \glossterm{Swift} ability, and it lasts until the end of the round.



\parhead*{Tags} \glossterm{Electricity}


\parhead*{Materials} As weapon


\lowercase{\hypertarget{item:Shocking, Greater}{}}\label{item:Shocking, Greater}
\hypertarget{item:Shocking, Greater}{\subsubsection{Shocking, Greater\hfill\nth{13}}}

This weapon continuously crackles with electricity.
The constant sparks shed light as a torch, and all damage dealt with it is electricity damage in addition to its other types.
When you make a \glossterm{strike} with this weapon, if your attack result beats the target's Fortitude defense, it is \dazed as a \glossterm{condition}.



\parhead*{Tags} \glossterm{Electricity}


\parhead*{Materials} As weapon


\lowercase{\hypertarget{item:Soulreaving}{}}\label{item:Soulreaving}
\hypertarget{item:Soulreaving}{\subsubsection{Soulreaving\hfill\nth{13}}}

This weapon is transluscent and has no physical presence for anyone except you.
It has no effect on objects or constructs, and creatures do not feel any pain or even notice attacks from it.
Attacks with this weapon ignore all damage reduction and hardness, but the damage is delayed instead of being dealt immediately.
Damage that would be dealt by the weapon can be delayed indefinitely.
While the damage is delayed, it cannot be removed by any means short of the destruction of this weapon or the creature's death.

As a \glossterm{minor action}, you can cut yourself with this weapon to activate it.
This deals no damage to you.
If you do, all delayed damage dealt by this weapon is converted into real damage.
Any such damage dealt in excess of a creature's hit points is dealt immediately as \glossterm{vital damage}.



\parhead*{Materials} As weapon


\lowercase{\hypertarget{item:Surestrike}{}}\label{item:Surestrike}
\hypertarget{item:Surestrike}{\subsubsection{Surestrike\hfill\nth{9}}}

You gain a \plus1 \glossterm{magic bonus} to accuracy with \glossterm{strikes} with this weapon.



\parhead*{Tags} \glossterm{Knowledge}


\parhead*{Materials} As weapon


\lowercase{\hypertarget{item:Thieving}{}}\label{item:Thieving}
\hypertarget{item:Thieving}{\subsubsection{Thieving\hfill\nth{7}}}

As a \glossterm{standard action}, you can spend an \glossterm{action point} to activate this weapon.
If you do, make a \glossterm{strike} or a \glossterm{disarm} attack.
If your disarm succeeds, or if your strike hit an unattended object, this weapon can absorb the struck object.
The object must be at least one size category smaller than the weapon.
An absorbed object leaves no trace that it ever existed.

This weapon can hold no more than three objects at once.
If you attempt to absorb an object while the weapon is full, the attempt fails.

As a standard action, you can retrieve the last item absorbed by the weapon.
The item appears in your hand, or falls to the ground if your hand is occupied.



\parhead*{Tags} \glossterm{Shaping}


\parhead*{Materials} As weapon


\lowercase{\hypertarget{item:Thieving, Greater}{}}\label{item:Thieving, Greater}
\hypertarget{item:Thieving, Greater}{\subsubsection{Thieving, Greater\hfill\nth{13}}}

This item functions like the \mitem{thieving} item, except that the maximum size category of object it can absorb is one size category larger than the weapon.



\parhead*{Tags} \glossterm{Shaping}


\parhead*{Materials} As weapon


\lowercase{\hypertarget{item:Thundering}{}}\label{item:Thundering}
\hypertarget{item:Thundering}{\subsubsection{Thundering\hfill\nth{5}}}

This weapon constantly emits a low-pitched rumbling noise and vibrates slightly in your hand.
All damage dealt with it is sonic damage in addition to its other types.
As a \glossterm{minor action}, you can intensify the vibration.
If you do, when you make a \glossterm{strike} with this weapon and your attack result beats the target's Fortitude defense, the target is \deafened as a \glossterm{condition}.
This is a \glossterm{Swift} ability, and it lasts until the end of the round.



\parhead*{Tags} \glossterm{Sonic}


\parhead*{Materials} As weapon


\lowercase{\hypertarget{item:Thundering, Greater}{}}\label{item:Thundering, Greater}
\hypertarget{item:Thundering, Greater}{\subsubsection{Thundering, Greater\hfill\nth{11}}}

This weapon constantly emits a low-pitched rumbling noise and vibrates slightly in your hand.
All damage dealt with it is sonic damage in addition to its other types.
When you make a \glossterm{strike} with this weapon and your attack result beats the target's Fortitude defense, the target is \deafened as a \glossterm{condition}.



\parhead*{Tags} \glossterm{Sonic}


\parhead*{Materials} As weapon


\lowercase{\hypertarget{item:Vampiric}{}}\label{item:Vampiric}
\hypertarget{item:Vampiric}{\subsubsection{Vampiric\hfill\nth{6}}}

When you deal damage to a living creature with a \glossterm{strike} with this weapon, you heal hit points equal to your level.



\parhead*{Tags} \glossterm{Life}


\parhead*{Materials} As weapon


\lowercase{\hypertarget{item:Vampiric, Greater}{}}\label{item:Vampiric, Greater}
\hypertarget{item:Vampiric, Greater}{\subsubsection{Vampiric, Greater\hfill\nth{14}}}

When you deal damage to a living creature with a \glossterm{strike} with this weapon, you heal hit points equal to twice your level.



\parhead*{Tags} \glossterm{Life}


\parhead*{Materials} As weapon


\lowercase{\hypertarget{item:Vorpal}{}}\label{item:Vorpal}
\hypertarget{item:Vorpal}{\subsubsection{Vorpal\hfill\nth{12}}}

Critical hits on \glossterm{strikes} with this weapon deal maximum damage.



\parhead*{Materials} As weapon


\section{Implements}\label{Implements}

    Implements can take many forms: staffs, wands, holy symbols, and more.
    Like magic weapons, magic implements must be wielded to gain their effects.
    However, while weapons are used to deal damage to enemies, implements are used to cast spells.

    \parhead{Somatic Components} While wielding an implement, you may gesture with it and channel magic through it.
    These qualify as somatic components for the purpose of casting spells.
    This does not remove the possibility of \glossterm{somatic component failure}.

    \subsection{Implement Types}

        \subsubsection{Holy Symbols}

            \parhead{Physical Description} A typical holy symbol is a no larger than 4 inches in each dimension and can be easily held in the palm of a hand.
            Most holy symbols are metal, but they can be made from wood, bone, or even more exotic materials, depending on the deity they symbolize.

            \parhead{Special Rules} All holy symbols are implements for divine spells.
            Most holy symbols are designed to be worn as an amulet in addition to being held in the hand.
            A magical holy symbol grants its magical abilities if it is either worn as an amulet or held in the hand.

        \subsubsection{Staffs}

            \parhead{Physical Description} A typical staff is 3 feet to 7 feet long and 2 inches to 3 inches thick, weighing about 5 pounds.
            Most staffs are wood, but a rare few are bone, metal, or even glass.
            Staffs often have a gem or some device at their tip or are shod in metal at one or both ends.

            Staffs are often decorated with carvings or runes.
            Long staffs are quarterstaffs.
            They must be held in two hands, and can be used to attack like any other quarterstaff.
            Short staffs resemble thin clubs.
            They can be held in one hand, but are not suitable for combat and are treated as \glossterm{improvised weapons} if used to attack.
            A typical staff has 20 \glossterm{hit points} and a sunder \glossterm{difficulty rating} of 10.

        \subsubsection{Wands}

            \parhead{Physical Description} A typical wand is 6 inches to 12 inches long and about 1/4 inch thick, and usually weighs no more than 1 ounce.
            Most wands are wood, but some are bone.
            A rare few are metal, glass, or even ceramic, but these are quite exotic.
            Occasionally, a wand has a gem or some device at its tip, and most are decorated with carvings or runes.
            A typical wand has 5 \glossterm{hit points} and a sunder \glossterm{difficulty rating} of 5.

    \subsection{Implement Descriptions}

        
\begin{longtablewrapper}
\begin{longtable}{p{15em} p{3em} p{6em} p{25em} p{3em}}

\lcaption{Implement Items} \\
\tb{Name} & \tb{Level} & \tb{Typical Price} & \tb{Description} & \tb{Page} \tableheaderrule
Spell Wand, 1st & \nth{5} & 800 gp & Grants knowledge of a rank 1 spell & \pageref{item:Spell Wand, 1st} \\
Staff of Focus & \nth{5} & 800 gp & Reduces \glossterm{focus penalty} by 1 & \pageref{item:Staff of Focus} \\
Staff of Giants & \nth{6} & 1,200 gp & Increases maximum size category of abilities & \pageref{item:Staff of Giants} \\
Staff of Transit & \nth{6} & 1,200 gp & Doubles your teleportation distance & \pageref{item:Staff of Transit} \\
Protective Staff & \nth{7} & 1,800 gp & Grants \plus1 Armor defense & \pageref{item:Protective Staff} \\
Cryptic Staff & \nth{8} & 2,750 gp & Makes spells hard to identify & \pageref{item:Cryptic Staff} \\
Spell Wand, 2nd & \nth{8} & 2,750 gp & Grants knowledge of a rank 2 spell & \pageref{item:Spell Wand, 2nd} \\
Staff of Power & \nth{8} & 2,750 gp & Grants \plus2 \glossterm{magical} power & \pageref{item:Staff of Power} \\
Staff of Precision & \nth{8} & 2,750 gp & Grants \plus1 accuracy & \pageref{item:Staff of Precision} \\
Extending Staff & \nth{9} & 4,000 gp & Doubles range & \pageref{item:Extending Staff} \\
Selective Staff & \nth{9} & 4,000 gp & Allows excluding areas & \pageref{item:Selective Staff} \\
Staff of Silence & \nth{9} & 4,000 gp & Allows casting spells without verbal components & \pageref{item:Staff of Silence} \\
Staff of Stillness & \nth{9} & 4,000 gp & Allows casting spells without somatic components & \pageref{item:Staff of Stillness} \\
Spell Wand, 3rd & \nth{11} & 10,000 gp & Grants knowledge of a rank 3 spell & \pageref{item:Spell Wand, 3rd} \\
Reaching Staff & \nth{12} & 16,000 gp & Allows ability use from a short distance away & \pageref{item:Reaching Staff} \\
Staff of Giants, Greater & \nth{12} & 16,000 gp & Significantly increaases maximum size category of abilities & \pageref{item:Staff of Giants, Greater} \\
Staff of Transit, Greater & \nth{12} & 16,000 gp & Triples your teleportation distance & \pageref{item:Staff of Transit, Greater} \\
Widening Staff & \nth{12} & 16,000 gp & Doubles area size & \pageref{item:Widening Staff} \\
Protective Staff, Greater & \nth{13} & 25,000 gp & Grants \plus2 Armor defense & \pageref{item:Protective Staff, Greater} \\
Staff of the Archmagi & \nth{13} & 25,000 gp & Grants \plus1 accuracy, \plus2 \glossterm{magical} power & \pageref{item:Staff of the Archmagi} \\
Spell Wand, 4th & \nth{14} & 37,000 gp & Grants knowledge of a rank 4 spell & \pageref{item:Spell Wand, 4th} \\
Staff of Power, Greater & \nth{14} & 37,000 gp & Grants \plus4 \glossterm{magical} power & \pageref{item:Staff of Power, Greater} \\
Staff of Precision, Greater & \nth{14} & 37,000 gp & Grants \plus2 accuracy & \pageref{item:Staff of Precision, Greater} \\
Extending Staff, Greater & \nth{15} & 55,000 gp & Triples range & \pageref{item:Extending Staff, Greater} \\
Selective Staff, Greater & \nth{15} & 55,000 gp & Allows excluding and splitting areas & \pageref{item:Selective Staff, Greater} \\
Staff of Tranquility & \nth{15} & 55,000 gp & Allows casting spells without components & \pageref{item:Staff of Tranquility} \\
Spell Wand, 5th & \nth{17} & 125,000 gp & Grants knowledge of a rank 5 spell & \pageref{item:Spell Wand, 5th} \\
Reaching Staff, Greater & \nth{18} & 190,000 gp & Allows ability use from a distance away & \pageref{item:Reaching Staff, Greater} \\
Staff of Giants, Supreme & \nth{18} & 190,000 gp & Drastically increaases maximum size category of abilities & \pageref{item:Staff of Giants, Supreme} \\
Staff of Transit, Supreme & \nth{18} & 190,000 gp & Quadruples your teleportation distance & \pageref{item:Staff of Transit, Supreme} \\
Widening Staff, Greater & \nth{18} & 190,000 gp & Triples area size & \pageref{item:Widening Staff, Greater} \\
Protective Staff, Supreme & \nth{19} & 280,000 gp & Grants \plus3 Armor defense & \pageref{item:Protective Staff, Supreme} \\
Staff of the Archmagi, Greater & \nth{19} & 280,000 gp & Grants \plus2 accuracy, \plus4 \glossterm{magical} power & \pageref{item:Staff of the Archmagi, Greater} \\
Staff of Power, Supreme & \nth{20} & 400,000 gp & Grants \plus6 \glossterm{magical} power & \pageref{item:Staff of Power, Supreme} \\
Staff of Precision, Supreme & \nth{20} & 400,000 gp & Grants \plus3 accuracy & \pageref{item:Staff of Precision, Supreme} \\

\end{longtable}
\end{longtablewrapper}


        
\lowercase{\hypertarget{item:Extending Staff}{}}\label{item:Extending Staff}
\hypertarget{item:Extending Staff}{\subsubsection{Extending Staff\hfill\nth{10} (6,500 gp)}}

You double the range of your \glossterm{magical} abilities.



\vspace{0.25em}
\spelltwocol{\textbf{Type}: Staff}{}
\textbf{Materials}: Bone, wood


\lowercase{\hypertarget{item:Extending Staff, Greater}{}}\label{item:Extending Staff, Greater}
\hypertarget{item:Extending Staff, Greater}{\subsubsection{Extending Staff, Greater\hfill\nth{19} (280,000 gp)}}

You triple the range of your \glossterm{magical} abilities.



\vspace{0.25em}
\spelltwocol{\textbf{Type}: Staff}{}
\textbf{Materials}: Bone, wood


\lowercase{\hypertarget{item:Protective Staff}{}}\label{item:Protective Staff}
\hypertarget{item:Protective Staff}{\subsubsection{Protective Staff\hfill\nth{5} (800 gp)}}

You gain a \plus1 \glossterm{magic bonus} to Armor defense.



\vspace{0.25em}
\spelltwocol{\textbf{Type}: Staff}{}
\textbf{Materials}: Bone, wood


\lowercase{\hypertarget{item:Protective Staff, Greater}{}}\label{item:Protective Staff, Greater}
\hypertarget{item:Protective Staff, Greater}{\subsubsection{Protective Staff, Greater\hfill\nth{14} (37,000 gp)}}

You gain a \plus2 \glossterm{magic bonus} to Armor defense.



\vspace{0.25em}
\spelltwocol{\textbf{Type}: Staff}{}
\textbf{Materials}: Bone, wood


\lowercase{\hypertarget{item:Reaching Staff}{}}\label{item:Reaching Staff}
\hypertarget{item:Reaching Staff}{\subsubsection{Reaching Staff\hfill\nth{12} (16,000 gp)}}

Spells you cast with this staff automatically have the benefits of the Reach augment, if applicable (see \pcref{Augment Descriptions}).



\vspace{0.25em}
\spelltwocol{\textbf{Type}: Staff}{}
\textbf{Materials}: Bone, wood


\lowercase{\hypertarget{item:Spell Wand, 1st}{}}\label{item:Spell Wand, 1st}
\hypertarget{item:Spell Wand, 1st}{\subsubsection{Spell Wand, 1st\hfill\nth{5} (800 gp)}}

This wand grants you knowledge of a single 1st level spell.
You must have access to the \glossterm{mystic sphere} that spell belongs to.



\vspace{0.25em}
\spelltwocol{\textbf{Type}: Wand}{}
\textbf{Materials}: Bone, wood


\lowercase{\hypertarget{item:Spell Wand, 2nd}{}}\label{item:Spell Wand, 2nd}
\hypertarget{item:Spell Wand, 2nd}{\subsubsection{Spell Wand, 2nd\hfill\nth{9} (4,000 gp)}}

This item functions like a \mitem{spell wand}, except that it grants knowledge of a single 2nd level spell.



\vspace{0.25em}
\spelltwocol{\textbf{Type}: Wand}{}
\textbf{Materials}: Bone, wood


\lowercase{\hypertarget{item:Spell Wand, 3rd}{}}\label{item:Spell Wand, 3rd}
\hypertarget{item:Spell Wand, 3rd}{\subsubsection{Spell Wand, 3rd\hfill\nth{13} (25,000 gp)}}

This item functions like a \mitem{spell wand}, except that it grants knowledge of a single 3rd level spell.



\vspace{0.25em}
\spelltwocol{\textbf{Type}: Wand}{}
\textbf{Materials}: Bone, wood


\lowercase{\hypertarget{item:Spell Wand, 4th}{}}\label{item:Spell Wand, 4th}
\hypertarget{item:Spell Wand, 4th}{\subsubsection{Spell Wand, 4th\hfill\nth{17} (125,000 gp)}}

This item functions like a \mitem{spell wand}, except that it grants knowledge of a single 4th level spell.



\vspace{0.25em}
\spelltwocol{\textbf{Type}: Wand}{}
\textbf{Materials}: Bone, wood


\lowercase{\hypertarget{item:Staff of Expansion}{}}\label{item:Staff of Expansion}
\hypertarget{item:Staff of Expansion}{\subsubsection{Staff of Expansion\hfill\nth{7} (1,800 gp)}}

When you use a \glossterm{magical} ability that creates a \glossterm{zone} or \glossterm{emanation}, you can increase the size of the area by one size category, up to a maximum of \areahuge.
You can only increase the area of one ability at a time in this way.
If you increase the area of another ability or lose this staff, the area of the original ability returns to its normal size.



\vspace{0.25em}
\spelltwocol{\textbf{Type}: Staff}{}
\textbf{Materials}: Bone, wood


\lowercase{\hypertarget{item:Staff of Expansion, Greater}{}}\label{item:Staff of Expansion, Greater}
\hypertarget{item:Staff of Expansion, Greater}{\subsubsection{Staff of Expansion, Greater\hfill\nth{16} (85,000 gp)}}

This item functions like a \textit{staff of expansion}, except that it increases the area by two size categories.
In addition, the maximum area is a 200 foot radius, which is one size category larger than \areahuge.



\vspace{0.25em}
\spelltwocol{\textbf{Type}: Staff}{}
\textbf{Materials}: Bone, wood


\lowercase{\hypertarget{item:Staff of Focus}{}}\label{item:Staff of Focus}
\hypertarget{item:Staff of Focus}{\subsubsection{Staff of Focus\hfill\nth{6} (1,200 gp)}}

You reduce your \glossterm{focus penalty} by 1.



\vspace{0.25em}
\spelltwocol{\textbf{Type}: Staff}{}
\textbf{Materials}: Bone, wood


\lowercase{\hypertarget{item:Staff of Focus, Greater}{}}\label{item:Staff of Focus, Greater}
\hypertarget{item:Staff of Focus, Greater}{\subsubsection{Staff of Focus, Greater\hfill\nth{15} (55,000 gp)}}

You reduce your \glossterm{focus penalty} by 2.



\vspace{0.25em}
\spelltwocol{\textbf{Type}: Staff}{}
\textbf{Materials}: Bone, wood


\lowercase{\hypertarget{item:Staff of Power}{}}\label{item:Staff of Power}
\hypertarget{item:Staff of Power}{\subsubsection{Staff of Power\hfill\nth{8} (2,750 gp)}}

You gain a \plus2 \glossterm{magic bonus} to \glossterm{power} with \glossterm{magical} abilities.



\vspace{0.25em}
\spelltwocol{\textbf{Type}: Staff}{}
\textbf{Materials}: Bone, wood


\lowercase{\hypertarget{item:Staff of Power, Greater}{}}\label{item:Staff of Power, Greater}
\hypertarget{item:Staff of Power, Greater}{\subsubsection{Staff of Power, Greater\hfill\nth{17} (125,000 gp)}}

You gain a \plus4 \glossterm{magic bonus} to \glossterm{power} with \glossterm{magical} abilities.



\vspace{0.25em}
\spelltwocol{\textbf{Type}: Staff}{}
\textbf{Materials}: Bone, wood


\lowercase{\hypertarget{item:Staff of Precision}{}}\label{item:Staff of Precision}
\hypertarget{item:Staff of Precision}{\subsubsection{Staff of Precision\hfill\nth{8} (2,750 gp)}}

You gain a \plus1 \glossterm{magic bonus} to \glossterm{accuracy}.



\vspace{0.25em}
\spelltwocol{\textbf{Type}: Staff}{}
\textbf{Materials}: Bone, wood


\lowercase{\hypertarget{item:Staff of Precision, Greater}{}}\label{item:Staff of Precision, Greater}
\hypertarget{item:Staff of Precision, Greater}{\subsubsection{Staff of Precision, Greater\hfill\nth{17} (125,000 gp)}}

You gain a \plus2 \glossterm{magic bonus} to \glossterm{accuracy}.



\vspace{0.25em}
\spelltwocol{\textbf{Type}: Staff}{}
\textbf{Materials}: Bone, wood


\lowercase{\hypertarget{item:Staff of Transit}{}}\label{item:Staff of Transit}
\hypertarget{item:Staff of Transit}{\subsubsection{Staff of Transit\hfill\nth{6} (1,200 gp)}}

Your \glossterm{magical} abilities have the maximum distance they can \glossterm{teleport} targets doubled.



\vspace{0.25em}
\spelltwocol{\textbf{Type}: Staff}{}
\textbf{Materials}: Bone, wood


\section{Tools}

    
\begin{longtabuwrapper}
\begin{longtabu}{l l l X l}
\lcaption{Tool Items} \\
\tb{Name} & \tb{Level} & \tb{Typical Price} & \tb{Description} & \tb{Page} \tableheaderrule
Acid Flask & 1/2 & 10 gp & Throw to deal acid damage & \pageref{item:Acid Flask} \\
Alchemist's Fire & 1/2 & 10 gp & Throw to deal fire damage & \pageref{item:Alchemist's Fire} \\
Flash Powder & 1/2 & 10 gp & Emits burst of bright light & \pageref{item:Flash Powder} \\
Tindertwig & 1/2 & 10 gp & Quickly activated flame & \pageref{item:Tindertwig} \\
Potion of Wound Closure & \nth{1} & 50 gp & Grants \plus1 bonus to a \glossterm{wound roll} & \pageref{item:Potion of Wound Closure} \\
Smokestick & \nth{1} & 50 gp & Creates a cloud of smoke & \pageref{item:Smokestick} \\
Everburning Torch & \nth{3} & 250 gp & Emits light like a torch for a week & \pageref{item:Everburning Torch} \\
Potion of Healing & \nth{3} & 250 gp & Restores one hit point & \pageref{item:Potion of Healing} \\
Snowball & \nth{3} & 250 gp & Throw to deal cold damage & \pageref{item:Snowball} \\
Sunrod & \nth{3} & 250 gp & Emits bright light continuously & \pageref{item:Sunrod} \\
Tanglefoot Bag & \nth{3} & 250 gp & Slows a foe & \pageref{item:Tanglefoot Bag} \\
Thunderstone & \nth{3} & 250 gp & Deafens a foe & \pageref{item:Thunderstone} \\
Antitoxin Elixir & \nth{4} & 500 gp & Resists poisons & \pageref{item:Antitoxin Elixir} \\
Enduring Sunrod & \nth{6} & 1,200 gp & Emits bright light continuously & \pageref{item:Enduring Sunrod} \\
Enduring Antitoxin Elixir & \nth{7} & 1,800 gp & Resists poisons for 8 hours & \pageref{item:Enduring Antitoxin Elixir} \\
Potion of Wound Closure, Greater & \nth{7} & 1,800 gp & Grants \plus2 bonus to a \glossterm{wound roll} & \pageref{item:Potion of Wound Closure, Greater} \\
Potion of Healing, Greater & \nth{9} & 4,000 gp & Restores two hit points & \pageref{item:Potion of Healing, Greater} \\
Cleansing Potion & \nth{11} & 10,000 gp & Removes a condition & \pageref{item:Cleansing Potion} \\
Potion of Wound Closure, Supreme & \nth{13} & 25,000 gp & Grants \plus3 bonus to a \glossterm{wound roll} & \pageref{item:Potion of Wound Closure, Supreme} \\
Potion of Healing, Supreme & \nth{15} & 55,000 gp & Restores three hit points & \pageref{item:Potion of Healing, Supreme} \\
Cleansing Potion, Greater & \nth{17} & 125,000 gp & Removes two conditions & \pageref{item:Cleansing Potion, Greater} \\
\end{longtabu}
\end{longtabuwrapper}


    
\lowercase{\hypertarget{item:Cleansing Potion}{}}\label{item:Cleansing Potion}
\hypertarget{item:Cleansing Potion}{\subsubsection{Cleansing Potion\hfill\nth{11} (10,000 gp)}}

When you drink this \glossterm{potion}, you remove your most recent \glossterm{condition}.
This cannot remove a condition applied during the current round.



\vspace{0.25em}
\spelltwocol{\textbf{Type}: Potion}{}
\textbf{Materials}: Alchemy


\lowercase{\hypertarget{item:Cleansing Potion, Greater}{}}\label{item:Cleansing Potion, Greater}
\hypertarget{item:Cleansing Potion, Greater}{\subsubsection{Cleansing Potion, Greater\hfill\nth{17} (125,000 gp)}}

When you drink this \glossterm{potion}, you remove your two most recent \glossterm{conditions}.
This cannot remove a condition applied during the current round.



\vspace{0.25em}
\spelltwocol{\textbf{Type}: Potion}{}
\textbf{Materials}: Alchemy


\lowercase{\hypertarget{item:Potion of Healing}{}}\label{item:Potion of Healing}
\hypertarget{item:Potion of Healing}{\subsubsection{Potion of Healing\hfill\nth{3} (250 gp)}}

When you drink this \glossterm{potion}, you heal one \glossterm{hit point}.



\vspace{0.25em}
\spelltwocol{\textbf{Type}: Potion}{}
\textbf{Materials}: Alchemy


\lowercase{\hypertarget{item:Potion of Healing, Greater}{}}\label{item:Potion of Healing, Greater}
\hypertarget{item:Potion of Healing, Greater}{\subsubsection{Potion of Healing, Greater\hfill\nth{9} (4,000 gp)}}

When you drink this \glossterm{potion}, you heal two \glossterm{hit points}.



\vspace{0.25em}
\spelltwocol{\textbf{Type}: Potion}{}
\textbf{Materials}: Alchemy


\lowercase{\hypertarget{item:Potion of Healing, Supreme}{}}\label{item:Potion of Healing, Supreme}
\hypertarget{item:Potion of Healing, Supreme}{\subsubsection{Potion of Healing, Supreme\hfill\nth{15} (55,000 gp)}}

When you drink this \glossterm{potion}, you heal three \glossterm{hit point}.



\vspace{0.25em}
\spelltwocol{\textbf{Type}: Potion}{}
\textbf{Materials}: Alchemy


\lowercase{\hypertarget{item:Potion of Wound Closure}{}}\label{item:Potion of Wound Closure}
\hypertarget{item:Potion of Wound Closure}{\subsubsection{Potion of Wound Closure\hfill\nth{1} (50 gp)}}

When you drink this \glossterm{potion}, you gain a \plus1 bonus to the \glossterm{wound roll} of your most recent \glossterm{vital wound}.
The \glossterm{wound roll} for that \glossterm{vital wound} cannot be modified again.



\vspace{0.25em}
\spelltwocol{\textbf{Type}: Potion}{}
\textbf{Materials}: Alchemy


\lowercase{\hypertarget{item:Potion of Wound Closure, Greater}{}}\label{item:Potion of Wound Closure, Greater}
\hypertarget{item:Potion of Wound Closure, Greater}{\subsubsection{Potion of Wound Closure, Greater\hfill\nth{7} (1,800 gp)}}

When you drink this \glossterm{potion}, you gain a \plus2 bonus to the \glossterm{wound roll} of your most recent \glossterm{vital wound}.
The \glossterm{wound roll} for that \glossterm{vital wound} cannot be modified again.



\vspace{0.25em}
\spelltwocol{\textbf{Type}: Potion}{}
\textbf{Materials}: Alchemy


\lowercase{\hypertarget{item:Potion of Wound Closure, Supreme}{}}\label{item:Potion of Wound Closure, Supreme}
\hypertarget{item:Potion of Wound Closure, Supreme}{\subsubsection{Potion of Wound Closure, Supreme\hfill\nth{13} (25,000 gp)}}

When you drink this \glossterm{potion}, you gain a \plus3 bonus to the \glossterm{wound roll} of your most recent \glossterm{vital wound}.
The \glossterm{wound roll} for that \glossterm{vital wound} cannot be modified again.



\vspace{0.25em}
\spelltwocol{\textbf{Type}: Potion}{}
\textbf{Materials}: Alchemy


\section{Legacy Items}\label{Legacy Items}

    Over time, items associated with places and people of great power gain magical properties.
    This process takes place for you as you gain levels in addition to in the world as a whole.

    At 3rd level, you choose a nonmagical weapon, piece of armor, apparel item, or implement you own.
    That item becomes a \glossterm{legacy item}, and gains a magic item ability you choose.
    You do not have to \glossterm{attune} to your legacy item to gain its benefits.
    The ability's level must be no greater than 5th level, and it must be appropriate for the category of item you chose: weapon, armor, apparel, or implement.
    You do not have to precisely match the location of an apparel item.
    For example, you can choose an amulet as your legacy item and give it the effect of the \mitem{boots of translocation}.

    At 9th, 15th, and 21st level, your legacy item increases in power again.
    You choose an ability of the appropriate type with a level no greater than two levels higher than your level when you choose the ability.
    You can choose a modified version of an existing ability on the item, such as the \textit{greater armor of invulnerability} ability if your legacy item already has the \textit{armor of invulnerability} ability.
    However, if you do so, you must change the lower level ability to be a different magic item ability.
    The new ability must meet the same maximum level requirement that it had when you first chose it.

    If you lose your legacy item, you must retrieve it to regain its power.
    There are rituals to facilitate this retrieval such as \ritual{seek legacy} and \ritual{retrieve legacy}.
    If your legacy item is \glossterm{destroyed}, you can designate a new item of the same type to be your legacy item, causing it to gain all of your legacy item abilities.
    Designating a new item in this way requires taking a \glossterm{long rest} while holding or wearing the replacement item.

    \parhead{Unique Legacy Items}
        Legacy items are fundamentally a reflection of the character who wields them.
        Their effects can be more unusual and complex than abilities on normal magic items, and they can have a larger effect on the way that character interacts with the world.
        As a player, you can work with your GM to create custom magical effects of an appropriate power that are a better reflection of your character's personality and powers than the magic item abilities that exist.

\section{Magic Item Creation}\label{Magic Item Creation}

    TODO


\chapter{Optional Rules}

\section{Attributes}

    \subsection{Other Methods of Attribute Generation}
        Point buy offers the fairest and most customizable system for determining attribute scores, ensuring that players can be almost any character they want to be. However, some groups may wish to determine attribute scores differently. Other options are provided below.

        \subsubsection{Simple Random Point Buy}
            With this method, you have only a small degree of control over your attribute scores, but all characters generated in this way are equally powerful.
            As with the point buy method, all your attribute scores start at 0, and you get 15 points to distribute among your attribute scores.
            However, you do not have full control over how to distribute those points.

            For each attribute, starting with the attributes you care about most, roll 1d8.
            You spend that many points on that attribute, ignoring any extra points that can't be spent
            For example, if you roll a 4, you spend 3 points on the attribute, causing you to start with a 2.
            If you do not have enough points remaining to spend the amount indicated by the die roll, spend as many as you can and move on to the next attribute.

            If you have points remaining after rolling all of your attribute scores, you may distribute the points freely among your abilities, using the normal point buy rules.
            You cannot increase the starting value of any individual attribute by more than 1 during this stage.
            If any of your attributes start as a 0, you may choose to lower them to gain the normal benefits from having low attributes (see \pcref{Attribute Penalties}).

            To further limit your character creation options, you may choose to randomize the order in which you roll your attributes instead of rolling them in an order of your choice.

        \subsubsection{Smoothed Random Point Buy}
            This method functions like the Simple Random Point Buy method, except that the resulting attribute values have a smoother distribution, and you can randomly end up with attribute penalties.

            For each attribute, starting with the attributes you care about most, roll 4d6.
            Then, remove any one of the rolls after seeing the results.
            Sum the results of the remaining three dice and spend the appropriate number of attribute points as indicated in \trefnp{Smoothed Random Point Buy Results}.
            If you do not have enough points remaining to spend the amount indicated by the die roll, spend as many as you can and move on to the next ability.

            If you have points remaining after rolling all of your attribute scores, you may distribute the points freely among your abilities, using the normal point buy rules.
            You cannot increase the starting value of any individual attribute by more than 1 during this stage.

            To further limit your character creation options, you may choose to randomize the order in which you roll your attributes instead of rolling them in an order of your choice.

            \begin{dtable}
                \lcaption{Smoothed Random Point Buy Results}
                \begin{dtabularx}{\columnwidth}{X X X}
                    \tb{Roll} & \tb{Base Attribute} & \tb{Point Cost} \tableheaderrule
                    3-4       & \minus2             & 0\fn{1} \\
                    5-6       & \minus1             & 0\fn{2} \\
                    7-8       & 0                   & 0       \\
                    9-10      & 1                   & 1       \\
                    11-12     & 2                   & 3       \\
                    13-15     & 3                   & 5       \\
                    16-18     & 4                   & 8       \\
                \end{dtabularx}
                1 You gain one \glossterm{insight point}. \\
                2 You gain an additional \glossterm{trained skill}. \\
            \end{dtable}

        \subsubsection{Classic Hardcore}

            This method is completely random and can generate very overpowered or underpowered characters.
            It represents the unfairness of the world, where some people are just better or worse than others.
            For each attribute, roll 2d6, take the average (rounded down), and subtract 2.
            If you roll a 1 on both dice, treat the average as a 0.
            The result is your base value for that attribute.

\section{Epic Fates}
    After 21st level, characters no longer gain levels normally.
    However, they can still increase their personal power as they make progress towards their ultimate fate.

    When you reach 21st level, you may choose an epic fate that you qualify for, or you may delay choosing until you meet the prerequisites for your desired fate.
    You do not start with any ranks in you chosen epic fate.
    Each epic fate specifies ways that you can make progress towards that epic fate.
    Whenever you make dramatic progress towards your epic fate, your rank in that epic fate may increase, at the discretion of the Game Master.

    None of the epic fate abilities have a tag to indicate that they are \magical abilities.
    Many of them are not fundamentally \glossterm{mundane} in nature, but they are beyond normal magic, and effects like an \spell{antimagic field} cannot interact with or suppress them.

    \subsection{Artificial Immortality}
        You have sought out strange magical power in search of a way to artificially prolong your life.
        As your power grows, you become increasingly able to resist death and return from it.
        Eventually, you will transcend death entirely.

        \parhead{Prerequisites} You must perform a series of rituals to prepare yourself for immortality, at least one of which must be rank 7 or higher. There are many kinds of immortality that you can pursue with this epic fate, and the exact nature of the rituals will change depending on the type of immortality you pursue.

        \parhead{Progression} You must discover powerful new magic rituals that support your particular form of immortality. This generally requires exploring sites of ancient magic, gaining favor with powerful creatures who have relevant knowledge or abilities, and independent experimentation based on your findings.

        \subsubsection{Artifical Immortality Ranks}

            \parhead{Rank 1 -- Life After Death} If you die from any cause other than old age, you resurrect according to nature of your chosen path to immortality.
            For example, you can have a phylactery regenerate a new body for you like a lich, or you can create clones of yourself or golems that you inhabit if your first body dies.
            You must always return in a new body of some sort.

            Your specific form of immortality determines where you return, such as at the site of your death or at your personal sanctum.
            However, it cannot cannot be based on the location or state of your old corpse, since that corpse is no longer ``you''.
            The timing of your resurrection may also differ based on your immortality, but you cannot complete your resurrection sooner than one day after the time of your death. After you resurrect in this way, this ability does not function for one week, allowing you to be killed normally.

            This immortality may change your base species, such as if you become a lich or move your body into a flesh golem. If it does, you retain all benefits and modifiers from your original species other than size, and you gain the effects of the new species in addition.

            \parhead{Rank 2 -- Death Familiarity} You become so familiar with the trauma of injury and death that your mind and body adapt to it.
            You gain a \plus10 bonus to vital rolls.
            In addition, the time of your vulnerability to true death after resurrection is reduced to 48 hours.

            \parhead{Rank 3 -- Artificial Life} Whenever you resurrect with your \textit{life after death} ability, your new body gains a \plus2 bonus to two random attributes. The attributes are randomized differently for each new body. In addition, that resurrection functions even if the cause of your death was old age, and you can control the physical age of your new body.

            \parhead{Rank 4 -- Deathcaller} You are deeply familiar with death, and know how to most effectively inflict it on others.
            Whenever you cause a living creature to lose at least half its hit points in a single round, you may kill that creature outright.
            In addition, your \textit{artificial life} ability grants a bonus to three random attributes instead of two random attributes.

            \parhead{Rank 5 -- True Immortality} You become fully immortal. There is no time limit after the resurrection from your \textit{life after death} ability where you become vulnerable to a true death. In addition, the resurrection can complete as quickly as one minute after your death. If a physical component limits your immortality, such as a phylactery, it can no longer be damaged or destroyed without the direct intervention of a rank 5 Slayer.

    \subsection{Ascendant}
        You have begun to see through the weave of the world and glimpse the higher truths beyond.
        As your insight into the true nature of reality grows, you begin to transcend the physical realm.
        Eventually, you become a being of pure energy.

        \parhead{Prerequisites} You must have spent at least a week living in each of the following planes: Air, Astral, Earth, Fire, Material, and Water.
        In addition, you must have an Intelligence or Perception of at least 2.

        \parhead{Progression} You must discover and spend time in exotic environments with unusual properties, especially with energy-related phenomena, to discern the underlying structure of the universe revealed in extremes.
        This involves a mix of meditation, observation, and potentially dangerous personal experience.
        Discovering potentially valuable locations may require extensive research.
        In order to reach the highest ranks, you must journey into forbidden realms of powerful magic, like the inner sanctums of major deities or the horrific depths of the Far Realm.

        \subsubsection{Ascendant Ranks}

            \parhead{Rank 1 -- Energetic Soul} Whenever you deal damage, you can treat at as energy damage in addition to its other types.
            In addition, whenever you take or deal energy damage while not \trait{incorporeal}, you and your equipment \glossterm{briefly} become incorporeal.
            While you are incorporeal in this way, you gain a \glossterm{fly speed} equal to your base speed with a \glossterm{height limit} of 60 feet (see \pcref{Flight}).
            Your \glossterm{maneuverability} with this fly speed is perfect (see \pcref{Flying Maneuverability}).

            \parhead{Rank 2 -- See Through the Weave} You can see everything within 120 feet of you perfectly, regardless of obstacles of any kind or light levels.
            This is similar to \trait{blindsight}, except that it also ignores solid obstacles of any kind, allowing you to have \glossterm{line of sight} through walls.
            You can perceive the presence of obstacles just as well as you can see what lies behind them.

            \parhead{Rank 3 -- Reach Through the Weave} When you use any of your abilities, you can treat yourself as being up to 60 feet away from your true location.
            You do not need \glossterm{line of effect} to your chosen location.
            For example, this allows you to make melee attacks against creatures up to 60 feet away.
            This changes your \glossterm{line of effect}, but does not change your \glossterm{line of sight}.

            % Does this need to have an explicit Recover reminder? Seems like that just works as intended
            \parhead{Rank 4 -- Become Energy} You become permanently \trait{incorporeal}, along with any equipment you carry.
            The fly speed from your \textit{energetic soul} ability also becomes permanent.
            In addition, you no longer age and no longer have hit points.
            Instead, you gain a bonus to your \glossterm{damage resistance} equal to the number of hit points you would normally have from your level, base class, and Constitution.
            Other effects that would increase or decrease your maximum hit points have no effect on you.
            You gain vital wounds based on taking damage in excess of your damage resistance rather than in excess of your hit points, including the extra vital wounds for taking massive damage.

            \parhead{Rank 5 -- Ascension} You cannot be killed, only dissipated.
            When you die, you automatically reform at a random location within a mile of your death after 10 minutes.
            Reforming in this way returns you to full damage resistance and removes all conditions and vital wounds, but your \glossterm{fatigue} and other effects remain the same.

    \subsection{Deity}
        People have begun to worship you, putting you on the path to become a deity.
        As your followers grow, you become capable of ever greater miraculous acts, and you can grant your followers some of your power.
        Eventually, you ascend into the pantheon of gods.

        \parhead{Prerequisites} You must have at least a hundred worshippers with souls to choose this epic fate.
        In addition, you must not have any cleric archetypes.

        \parhead{Progression} To progress towards this epic fate, you must gain a significant number of additional worshippers.
        In general, you must at least double your worshippers to progress towards each new rank of this fate, though this can vary widely.
        Having worshippers among many different places is more valuable than converting an isolated group to worship you, though both are helpful.

        \subsubsection{Deity Ranks}
            \parhead{Rank 1 -- Domain Influence} Choose a cleric domain.
            You gain all abilities from that domain except for its mastery ability.
            In addition, your worshippers become eligible to gain cleric archetypes, though they cannot exceed a maximum rank in those archetypes of twice your rank in this epic fate (to a maximum of 8).
            This does not grant additional archetypes to worshippers who have already chosen their three archetypes, and is usually only relevant to NPC worshippers.

            \parhead{Rank 2 -- Prayers} You hear all prayers directed to you.
            Once per week, you can teleport yourself and up to ten \glossterm{allies} any distance within the same plane as a \glossterm{standard action}.
            Your destination must either be a worshipper actively praying to you or a holy place dedicated to you.
            In addition, choose a second cleric domain.
            You gain all abilities from that domain except for its mastery ability.

            \parhead{Rank 3 -- Domain Mastery} Choose a third cleric domain.
            You gain all abilities from that domain.
            In addition, you gain the mastery ability from the domains you chose with your \textit{domain influence} and \textit{prayers} abilities.

            \parhead{Rank 4 -- Demigod} You become a demigod.
            You no longer age normally, and you cannot die from old age.
            You become a planeforged native to an Aligned Plane matching your alignment.
            For details about the aligned planes, see the Tome of Guidance.
            While you are on that plane, you can teleport to any plane with your \textit{prayers} ability from this epic fate.
            In addition, you can use that teleportation ability once per hour instead of once per week.

            \parhead{Rank 5 -- Deification} You become a deity.
            You are transported to an Aligned Plane matching your alignment, and you gain divine dominion over an amount of territory in that plane.
            While you are in your territory, you can can freely reshape your territory with a thought to match your desires, and you are immune to all damage and \glossterm{conditions}.

            Regardless of which plane you are on, you can teleport to anywhere within your home plane as a \glossterm{standard action}.
            In addition, there is no limit on the number of times you can teleport with your \textit{prayers} ability from this epic fate.

    \subsection{Hero of Legend}
        You are widely known as a hero, rescuing those in need.
        As your deeds of heroism spread, you gain abilities to help you protect others.
        Although you will eventually die, your legend will live on, inspiring others to save people as you did.

        \parhead{Prerequisites} You must be publicly known to be involved with saving at least one major country or similarly large group of people from some sort of disaster to choose this epic fate.

        \parhead{Progression} To progress towards this epic fate, you must publicly contribute to saving large numbers of people from death or other major disasters in a way that builds your reputation.
        Reaching the higher ranks typically requires saving a significant fraction of a major plane from some sort of catastrophe.

        \subsubsection{Hero of Legend Ranks}
            \parhead{Rank 1 -- Worthy Hero} You and all \glossterm{allies} who can see or hear you are immune to being frightened and panicked.
            In addition, you gain a \plus100 bonus to your \glossterm{hit points}.

            \parhead{Rank 2 -- Heroic Intervention} As a \glossterm{free action}, you may choose any number of \glossterm{allies} within \shortrange of you.
            Whenever a chosen creature would be attacked that round, that attack is made against you instead.
            If the attack would have targeted both you and that ally, the attack only targets you once, not twice.
            This ability has the \abilitytag{Swift} tag.

            \parhead{Rank 3 -- Invincible Hero} You gain a \plus4 bonus to all defenses.
            In addition, you cannot be \vulnerable for any reason.

            \parhead{Rank 4 -- Answer the Call} You gain an intuitive sense for when people need your aid.
            Whenever someone on the same plane as you is in danger, you are aware of the existence of that danger.
            You can sense the general category of danger (fire, combat, drowning, etc.) and a very approximate direction and distance.
            This generally allows you to sense if a large number of people are in danger from the same thing.
            As a \glossterm{standard action}, you can teleport any distance within that plane to reach a person in danger.

            \parhead{Rank 5 -- Heroic Legacy} If you die, your legend lives on.
            You may choose a worthy successor, either before your death or after your death from your afterlife plane.
            When you die, or as soon as you choose a successor while dead, your successor immediately gains the Rank 1 benefit from this archetype.
            As long as your successor lives and remains worthy, you cannot choose to be resurrected from your afterlife.
            As a player, you may choose to play as your successor instead of your original character if the GM allows it.

    \subsection{Mutant}
        Your body has been altered by battle scars and strange experiments.
        As your mutations grow ever more extreme, you become more powerful - and more monstrous.
        Eventually, you can regenerate from death itself, though some scars never fade.

        \parhead{Prerequisites} Dangerous experiments to mutate your body in extreme ways must have been performed on you.
        You can choose the nature of these experiments, such as alchemical or magical.
        Some mutants do this to themselves, while others find willing collaborators.
        In addition, you must have a Constitution of at least 2.

        \parhead{Progression} To progress towards this epic fate, you must continue ever more radical forms of experimentation.
        As you become inured to ordinary alterations to your body, you must travel and research to find unique substances of immense power to fuel the experiments.
        To reach the higher ranks, you must undergo experiments that kill you far more often than they succeed, so you will need to be resurrected multiple times to continue down this path.

        \subsubsection{Mutant Ranks}
            \parhead{Rank 1 -- Unnatural Arsenal} You grow an extra functioning arm and hand.
            In addition, you gain a wide variety of natural weapons (see \tref{Natural Weapons}).
            You gain a bite, horn, ram, stinger, and tentacle.
            In addition, two of your hands become claws.
            You gain these weapons in addition to any natural weapons you already have, no matter how biologically implausible that may be.
            % TODO: mutants will often have Massive, so Sweeping is oddly redundant.
            In addition, each of your natural weapons has the \weapontag{Sweeping} (3) weapon tag in addition to its other weapon tags.

            \parhead{Rank 2 -- Regeneration} At the end of each round, you regain hit points equal to a quarter of your maximum hit points.
            In addition, you may increase the result of one of your \glossterm{vital wounds} by 1, to a maximum of 10.
            If you are unconscious, this automatically applies to your most severe vital wounds first.

            \parhead{Rank 3 -- Monstrous Form} Your size increases by one size category.
            In addition, you gain a wide variety of movement modes.

            You gain a \glossterm{climb speed} and \glossterm{swim speed} equal to your base speed, and a 10 foot \glossterm{burrow speed}.
            For each of those movement modes, if you already have it, you gain a \plus10 foot bonus to your speed with that movement mode instead.
            Wings also grow from your back, granting you a \glossterm{fly speed} equal to your base speed with a 60 foot \glossterm{height limit} (see \pcref{Flight}).

            \parhead{Rank 4 -- Two Heads Are Better Than One} You grow a second head.
            Whenever you gain a \glossterm{condition}, you choose which head gains the condition, with the restriction that the chosen head must not already have more conditions than the other head.
            At the start of each round, you can choose which heads are active during that round.
            You are only subject to the effects of the conditions affecting active heads.
            If you choose for both heads to be active, you can use an extra \glossterm{minor action} during the \glossterm{action phase}.

            \parhead{Rank 5 -- Regenerative Immortality} You can regenerate from any wounds, even lethal ones.
            When you die, your \textit{unnatural regeneration} ability continues functioning.
            Once that ability improves your vital wounds so they are all above 0, you return to life.
            However, each time you die, you gain a new scar that your regeneration always recreates.

            If your corpse is mutilated, burned, immersed in acid, or fully destroyed, this process can take much longer to complete, but it cannot be fully stopped.
            Some drop of blood, flake of skin, or other remnant of your corpse will always persist and regenerate eventually.
            If your body is separated into pieces while you are dead, each piece will attempt to regenerate individually.
            Your soul will automatically return to the first piece that regenerates completely, at which point the remaining fragments will wither and die.

    \subsection{Paradox}
        You exist partly outside of the ordinary flow of time.
        Your very existence wreaks havoc on prophecies and the orderly sequence of events.
        As your alterations to the natural timeline of the universe grow in scope, your ability to bend time to your whims grows in turn.
        Eventually, you become a fixed point across all of time.

        \parhead{Prerequisites} You must have been directly involved with an action that resulted in a significant change to at least one major country or similarly significant entity.
        Any type of change is acceptable, as long as it would be historically important would not have happened without your intervention.

        \parhead{Progression} To progress towards this epic fate, you must be alter the course of other major events that will be remembered to history.
        Reaching the higher ranks typically requires changing the fate of major planes, or creatures of similar importance.

        \subsubsection{Paradox Ranks}
            \parhead{Rank 1 -- Temporal Aberration} Your actions, and events involving you, cannot be observed in any effect that sees or predicts the future.
            This applies against both magical abilities and abilities that rely on direct observation.
            Prophecies are only able to describe how events would happen without your intervention, and are blind to any changes you might cause.

            In addition, whenever you would make a \glossterm{movement}, you can make two different movements and then decide which one was the movement you actually made.
            You can make this decision after observing how other creatures react to your movement, but before taking any other actions.

            The other movement never happened, and had no effect.
            Only the movement itself is reverted in this way.
            Any other abilities you used during the resolution of that movement, such as the \ability{sprint} ability, still happened, so you would still gain fatigue and resolve other effects.

            \parhead{Rank 2 -- A Fork In Time's Road} Whenever you would take a standard action, you can take two different standard actions and then decide which one was the action you actually took.
            You can make this decision after seeing all die results and observing all effects of both actions, but before taking any other actions.

            The other action never happened, and has no effect.
            Only the standard action itself is reverted in this way.
            Any other abilities you used during the resolution of that action, such as the \ability{desperate exertion} ability, still happened, so you would still gain fatigue and resolve other effects.

            \parhead{Rank 3 -- Choose Fate} Once per \glossterm{short rest}, whenever you or any other creature you are aware of rolls an attack or check, you can choose the result of that die.
            You can choose to use this ability after learning the result of the action using that die roll, including whether it succeeded or failed and the result of any damage dice based on the attack.
            However, you must use it before any other actions resolve.
            If you use this to make an attack \glossterm{explode}, subsequent dice after the die you modify in this way are rolled normally.
            Using this to affect an enemy's action may change the actions taken by other enemies in that enemy's \glossterm{allied group}, which the GM should resolve.

            \parhead{Rank 4 -- Paradoxical Defense} Whenever a creature attacks you, it must roll twice and take the lower result.
            This does not protect any other targets of the attack.

            \parhead{Rank 5 -- Fixed Point} You become an immutable fact across all of time.
            You no longer age.
            Whenever you die, history is rewritten so you retroactively never died instead.
            The changes are as subtle and believable as possible, but even extraordinary coincidences can occur to save you from death.
            You typically still end up unconscious from vital wounds, and are always removed from combat or otherwise unable to usefully act for at least ten minutes, but you survive.

    \subsection{Slayer}
        You are a killer of legendary skill.
        As your body count increases, you gain abilities to help you track down and kill increasingly powerful foes.
        Eventually, your powers threaten the gods themselves, allowing you a unique ability to transcend death.

        \parhead{Prerequisites} You must be directly involved with slaying at least one \glossterm{elite} creature with a level of at least 21.

        \parhead{Progression} To progress towards this epic fate, you must publicly contribute to slaying increasingly dangerous and fearsome foes.
        To reach the higher ranks, you must kill creatures of singular power whose influence is felt across multiple planes.
        This might include demon princes, supreme dragons beyond even the power of wyrms, or the nightmarish aberration progenitors in the Far Realm.

        \subsubsection{Slayer Ranks}
            \parhead{Rank 1 -- Lethality} You gain a \plus4 bonus to your accuracy for the purpose of determining whether you get a \glossterm{critical hit}.
            This bonus stacks with other abilities with the same effect, such as \weapontag{Keen} weapons.

            \parhead{Rank 2 -- Precision Killer} You gain a \plus4 bonus to your \glossterm{accuracy}.
            In addition, you can inflict \glossterm{critical hits} on any creature, regardless of its size, body structure, or other abilities.

            \parhead{Rank 3 -- Mark of the Slayer} As a \glossterm{minor action}, you can choose to mark any creature you can unambiguously identify.
            This includes any creature you can see, as well as any creature you know the name of and can differentiate from other similar creatures.
            You can only mark one creature at a time, and applying a new mark replaces any previous mark.
            You cannot use this ability to replace a mark that is less than a week old if the recipient of the previous mark still lives.

            This mark is visible on the creature's body with a design that is recognizably yours.
            It appears on top of any clothing or other attempt to conceal it, even if the creature is invisible.
            Anyone can recognize the significance of the mark with a \glossterm{difficulty value} 15 Knowledge (arcana or local) check, and creatures that understand the significance of the mark may refuse to give your target aid of any kind to avoid risking your wrath.

            You know the exact distance and direction to any creature you have marked with this ability that is on the same plane as you.
            As a \glossterm{standard action}, you can create a \glossterm{scrying sensor} adjacent to them that you can see and hear through.
            The sensor lasts as long as you \glossterm{sustain} it as a \glossterm{free action}.
            It moves to stay adjacent to the target, regardless of its speed.

            \parhead{Rank 4 -- Slayer's Journey} As a \glossterm{standard action}, you can \glossterm{teleport} yourself and up to ten \glossterm{allies} any distance within the same plane to the location of a creature affected by your \textit{mark of the slayer} ability from this epic fate.
            You cannot precisely choose the destination of this ability, and it does not leave you immediately adjacent to the marked creature.
            Generally, it leaves you just outside any sort of fortress or defenses the marked creature has constructed.
            After you use this ability, you cannot use it to travel to the same creature for a day.
            This does not limit your ability to travel to a different creature if you mark a different creature.

            \parhead{Rank 5 -- Godslayer}
            Damage from your attacks ignores all forms of invulnerability and immunity.
            You can deal fire damage to fire elementals, physical damage to ghosts, and so on.
            In addition, you can destroy artifacts and even inflict damage on deities in their divine dominion.
            As a result, even deities fear to interfere with you directly.
            If you ever die, you can generally threaten or fight your way past any planar guardians to leave your afterlife whenever you want.
            After you do this once, you become a planeforged native to your afterlife plane, since your new body is formed from the raw material of that plane.

\section{Uncommon Species}\label{Uncommon Species}

    \subsection{Animal Hybrid}
        Animal hybrids are humanoid creatures that are a combination of humans and animals.
        The abilities of an animal hybrid depend on the type of animal it is based on.

        \parhead{Size} Medium.
        \parhead{Attributes} The attributes of an awakened animal depend on its size.
        \parhead{Special Abilities} As the original animal.
        \parhead{Automatic Languages} Common and any one \glossterm{common language} (see \tref{Common Languages}).

        \subsubsection{Sample Animal Hybrids}

            \parhead{Hybrid Bee}

            \subparhead{Special Abilities}
            \parhead{Attribute} \plus1 Dexterity, \minus1 Constitution.
            \begin{itemize}
                \itemhead{Low-light Vision} A hybrid bee has \trait{low-light vision}, allowing it to see clearly in \glossterm{shadowy illumination} (see \pcref{Low-light Vision}).
                \itemhead{Stinger} A hybrid bee has a stinger natural weapon (see \pcref{Natural Weapons}).
                    Whenever it causes a creature to lose \glossterm{hit points} with that natural weapon, the struck creature is poisoned by giant wasp venom (see \pcref{Poison}).
                    Its stage 1 effect makes the target \slowed while the poison lasts.
                    Its stage 3 effect makes the target \immobilized while the poison lasts.
                \itemhead{Winged Agility} A hybrid bee has wings that are not strong enough to help it fly.
                    However, the wings still help it stabilize its movements.
                    It gains a \plus3 bonus to the Balance and Jump skills, and it gains a \plus5 foot bonus to its maximum horizontal jump distance (see \pcref{Jumping}).
                    This increases its maximum vertical jump distance normally.
            \end{itemize}

            \parhead{Hybrid Shark}

            \subparhead{Special Abilities}
            \begin{itemize}
                \itemhead{Bloodscent} A hybrid shark has the scent ability (see \pcref{Scent}).
                    In addition, it gains a \plus10 bonus to Awareness checks to detect blood.
                \itemhead{Bite} A hybrid shark's mouth is elongated, which it can use as a bite attack (see \pcref{Natural Weapons}).
                    A hybrid shark's bite deals 1d6 damage.
                \itemhead{Gills} You can breathe water as easily as a human breathes air, preventing you from drowning or suffocating underwater.
                \itemhead{Swim Speed} A hybrid shark has a swim speed equal to the base speed for its size.
            \end{itemize}

            \parhead{Hybrid Wolf}

            \subparhead{Special Abilities}
            \begin{itemize}
                \itemhead{Scent} A hybrid wolf has the scent ability (see \pcref{Scent}).
                \itemhead{Bite} A hybrid wolf's mouth is elongated, which it can use as a bite attack (see \pcref{Natural Weapons}).
                    A hybrid wolf's bite deals 1d6 damage.
                \itemhead{Low-light Vision} A hybrid wolf has \trait{low-light vision}, allowing it to see clearly in \glossterm{shadowy illumination} (see \pcref{Low-light Vision}).
            \end{itemize}

    \subsection{Automaton}
        An automaton appears to be a humanoid construct, like a golem.
        Its body is made from some combination of stone, wood, and metal.
        However, its artificial body is inhabited by a true soul, making it an indwelt (see \pcref{Indwelt}).

        \parhead{Size} Medium.
        \parhead{Attributes} \plus1 Constitution or Intelligence, \minus1 Dexterity.
        \parhead{Special Abilities}
        \begin{itemize}
            \itemhead{Artificial Life} Automatons are not alive. They ignore abilities which only affect living creatures, including poisons and most healing abilities. In addition, they do not need to eat, drink, or sleep.
            \itemhead{Automaton Archetype} Automatons only gain two class archetypes instead of three.
                Instead, they treat the Automaton archetype as one of their archetypes, and they gain ranks in it just like they gain ranks in class archetypes.
            \itemhead{Manual Repair} A Craft skill relevant to the automaton's body can be used to achieve the same effects that the Medicine skill would have on a living creature.
            \itemhead{Mechanical Body} Automatons are considered both objects and creatures, and are affected by abilities which affect either.
                They are always considered to be \glossterm{attended} by themselves, so they are never affected by abilities that only affect unattended objects, even while unconscious.
            \itemhead{Mechanical Intuition} Automatons gain a \plus2 bonus to the Devices skill and one Craft skill of their choice.
        \end{itemize}

        \subsubsection{Automaton Archetype}

            \cf{Aut}[1]{Modular Carapace} You can adjust the density and layering of your hardened exterior to augment your defenses.
            Changing your configuration in this way requires 10 minutes of work, and spare armor parts that you generally keep with you.
            You can choose to treat your carapace as being light, medium, or heavy armor.
            The benefits from this ability are considered to come from body armor, and do not stack with actual body armor.

            \begin{itemize}
                \item Light armor: You gain a \plus3 bonus to your Armor defense, and a damage resistance bonus equal to three times your rank in this archetype.
                    You can wear body armor on top of this carapace.
                    Although the benefits of that armor do not stack with the carapace, you can use the higher Armor defense value and damage resistance bonus from either armor.
                \item Medium armor: You gain a \plus4 bonus to your Armor defense, and a damage resistance bonus equal to four times your rank in this archetype.
                    However, your Dexterity bonus to your Armor defense is halved, and you cannot wear body armor.
                \item Heavy armor: You gain a \plus5 bonus to your Armor defense, and a damage resistance bonus equal to five times your rank in this archetype.
                    However, your Dexterity bonus to your Armor defense is removed, you take a \minus10 foot penalty to your speed with all movement modes, and you cannot wear body armor.
            \end{itemize}

            If you lose your original armor parts, you can create or buy new parts that are suited to your body.
            These parts are considered a Rank 1 (40 gp) item.

            \cf{Aut}[2]{Modular Weaponry} You can customize your arms to augment your combat prowess.
            Changing your configuration in this way requires 10 minutes of work, and spare arm and weapon parts that you generally keep with you.
            \begin{itemize}
                % Two improvements over baseline
                \item Blade: You convert one of your arms into a blade natural weapon.
                    It has a \plus1 accuracy bonus, deals 1d8 slashing damage, and has the \weapontag{Keen} and \weapontag{Sweeping} (1) weapon tags (see \pcref{Weapon Tags}).
                    However, that arm no longer has a \glossterm{free hand}.
                % Two improvements over baseline
                \item Bulk: You convert one of your arms into a bulky, overburdened slam natural weapon.
                    It deals 1d10 bludgeoning damage and has the \weapontag{Impact} and \weapontag{Resonating} weapon tags.
                    However, that arm no longer has a \glossterm{free hand}.
                \item Fortified: You add additional protective plating to your arms.
                    You gain a \plus1 bonus to your Armor defense.
                    This does not require a \glossterm{free hand}, but it is still considered to come from a shield, and it does not stack with the benefit from using a shield.
                \item Slim: You trim away the excess protection from your arms to make their movements more precise.
                    You gain a \plus1 accuracy bonus with any attack that uses your hands, including \glossterm{strikes} and abilities like \ability{grapple}.
                    This does not apply to spells that simply have \glossterm{somatic components}, though it does apply to touch spells like \spell{freezing grasp}.
                    However, you take a \minus1 penalty to your Armor defense.
            \end{itemize}

            If you lose your original arm and weapon parts, you can create or buy new parts that are suited to your body.
            These parts are considered a Rank 1 (40 gp) item.

            \magicalcf{Aut}[3]{Imbued Carapace} When you gain this ability, choose one magic body armor property for each of your three \ability{modular carapace} types (see \tref{Magic Armor}).
            The property must be rank 3 or lower.
            Whenever you use a carapace type, you can gain the benefit of its magic armor property, though you must still \glossterm{attune} to the property.
            Each carapace can have a unique property, or you can choose the same property for multiple carapace types.
            If you are using a light carapace, you can gain the benefit of its magic armor property in addition to any magic armor you wear on top of the carapace.

            Whenever you gain a rank in this archetype, you can choose a new property for each of your carapace types.
            Each property's rank must be no higher than your rank in this archetype.
            Magic armor of rank 3 or higher does not multiply the damage resistance provided by your carapace, as magic armor normally does (see \pcref{Magic Armor Damage Resistance}).

            \magicalcf{Aut}[4]{Embedded Armory} Whenever you use your \ability{modular weaponry} ability to change your configuration, you can also embed one magic weapon, implement, glove, gauntlet, or bracer into one of your arms.
            You can attune to that item without having it physically present on your arm.
            If you attune to a weapon embedded this way, you apply its magical properties to any nonmagical weapons you use with that arm, including natural weapons and weapons made from special materials.
            You cannot embed a \glossterm{legacy item} in this way.

            \cf{Aut}[5]{Modular Carapace+} The damage resistance from your \textit{modular carapace} armor increases.
            Light carapace increases to four times your rank in this archetype, medium carapace increases to five times your rank, and heavy carapace increases to seven times your rank.

            \cf{Aut}[5]{Reassembly} You can recover from vital wounds more easily by simply replacing broken parts.
            You can remove a vital wound with ten minutes of work.
            This increases your \glossterm{fatigue level} by three, and it requires replacement parts that you generally keep with you.
            The parts are considered a consumable Rank 3 (200 gp) item.

            This can even save you from death, though that is more difficult and requires more advanced parts.
            A creature can spend eight hours replacing broken parts of your corpse to \glossterm{resurrect} you (see \pcref{Resurrection}).
            % Baseline trained bonus at rank 5 is +9.
            This requires a \glossterm{difficulty value} 15 Craft check appropriate to the composition of your body.
            The parts required to perform this feat are considered a consumable Rank 5 (5,000 gp) item.

            \cf{Aut}[6]{Artificial Mind} You become immune to \abilitytag{Compulsion} and \abilitytag{Emotion} attacks.

            \cf{Aut}[7]{Hypercognition} You can take an additional \glossterm{minor action} each round, as long as that minor action is purely mental.
            As normal, you cannot use the same ability twice in the same round.

            Alternately, you can take an additional purely mental standard action instead of a minor action.
            When you do, you increase your \glossterm{fatigue level} by one and you \glossterm{briefly} cannot take any additional actions with this ability.

        \subsubsection{Base Class Abilities}
            If you choose automaton as your base class, you gain the following abilities.

            \cf{Aut}{Defenses}
            You gain the following bonuses to your \glossterm{defenses}: \plus4 Fortitude, \plus2 Reflex, \plus3 Mental.

            \cf{Aut}{Hit Points}
                You have 8 hit points \add twice your Constitution, plus 2 hit points per level beyond 1.
                This increases as your level increases, as indicated below.
                \begin{itemize}
                    \itemhead{Level 7} 20 hit points \add three times your Constitution, plus 3 hit points per level beyond 7.
                    \itemhead{Level 13} 40 hit points \add six times your Constitution, plus 6 hit points per level beyond 13.
                    \itemhead{Level 19} 80 hit points \add twelve times your Constitution, plus 12 hit points per level beyond 19.
                \end{itemize}

            \cf{Aut}{Resources} You have the following \glossterm{resources}:
            \begin{itemize}
                \item Three \glossterm{attunement points}, which you can use to attune to items and abilities that affect you (see \pcref{Attunement Points}).
                \item A \glossterm{fatigue tolerance} equal to 4 \add your Constitution.
                    Your fatigue tolerance makes it easier for you to use powerful abilities that fatigue you (see \pcref{Fatigue}).
                \item A number of \glossterm{insight points} equal to 1 \add your Intelligence.
                    You can spend insight points to gain additional abilities (see \pcref{Insight Points}).
                \item Three \glossterm{trained skills} from among your \glossterm{class skills}, plus additional trained skills equal to your Intelligence (see \pcref{Skills}).
            \end{itemize}

            \cf{Aut}{Weapon Proficiencies} 
            You are proficient with simple weapons and one weapon group of your choice.

            \cf{Aut}{Armor Proficiencies} 
            You are not proficient with armor.

            \cf{Aut}{Skills}
            You have the following \glossterm{class skills}:
            \begin{itemize}
                \itemhead{Strength} Climb.
                \itemhead{Dexterity} Balance, Flexibility.
                \itemhead{Constitution} Endurance.
                \itemhead{Intelligence} Craft (any), Deduction, Devices, Disguise.
                \itemhead{Perception} Awareness.
            \end{itemize}

    \subsection{Awakened Animal}

        Awakened animals are animals that have been granted sentience by the \spell{awaken} ritual.
        The abilities of an awakened animal depend on the type of animal it is.

        \parhead{Size} Small or Medium, as original animal.
        \parhead{Attributes} The attributes of an awakened animal depend on its size.
        \subparhead{Medium} No change.
        \subparhead{Small} \minus2 Strength, \plus1 Dexterity.
        \parhead{Special Abilities} As the original animal.
        \parhead{Automatic Languages} Common.

        \subsubsection{Sample Awakened Animals}

            \parhead{Cat}

            \subparhead{Size} Small. This gives a cat a 20 foot \glossterm{base speed} and a \plus5 bonus to the Stealth skill, among other effects (see \pcref{Size Categories}).
            \subparhead{Attributes} \minus2 Strength, \plus1 Dexterity
            \subparhead{Special Abilities}
            \begin{itemize}
                \itemhead{Claws} A cat's paws end in claws, which it can use to attack (see \pcref{Natural Weapons}). A cat's claws have a \plus2 accuracy bonus and deal 1d4 damage.
                \itemhead{Low-light Vision} A cat has \trait{low-light vision}, allowing it to see clearly in \glossterm{shadowy illumination} (see \pcref{Low-light Vision}).
                \itemhead{Multipedal} A cat is \trait{multipedal}, which gives it a \plus10 foot bonus to its \glossterm{land speed} and a \plus5 bonus to Balance.
                \itemhead{Scent} A cat has the scent ability (see \pcref{Scent}).
            \end{itemize}

    \subsection{Changeling}

        \parhead{Size} Medium.
        \parhead{Attributes} No change.
        \parhead{Special Abilities}
        \begin{itemize}
            \itemhead{Alter Shape} A changeling can alter its physical form in minor ways. As a standard action, a changeling can make a Disguise check with a \plus10 bonus to alter its body. This ability does not alter the changeling's equipment, which may give away its identity unless disguised normally.

            This is a \magical ability.
        \end{itemize}
        \parhead{Bonus Languages} Any.
        \parhead{Automatic Languages} Common, any two \glossterm{common languages}.

    \subsection{Dragon}
        Ancient dragons are magical creatures of immense power and wisdom, and are far more powerful than any ordinary character of the same level.
        However, young dragons can be played as characters, though their unique abilities do pose unique challenges.

        \parhead{Creature Type} Unlike most other playable species, dragons are magical beasts instead of humanoids.
        \parhead{Size} Small. This gives a dragon a 20 foot \glossterm{base speed} and a \plus5 bonus to the Stealth skill, among other effects (see \pcref{Size Categories}).
        \parhead{Attributes} \minus2 Strength, \plus1 Dexterity.
        \parhead{Special Abilities}
        \begin{itemize}
            \itemhead{Dragon Archetype} Dragons only gain two class archetypes instead of three.
                Instead, they treat the Dragon archetype as one of their archetypes, and they gain ranks in it just like they gain ranks in class archetypes.
            \itemhead{Draconic Senses} Dragons have \trait{darkvision} with a 60 foot range, allowing them to see in complete darkness (see \pcref{Darkvision}).
                In addition, dragons gain \trait{low-light vision}, allowing them to see clearly in \glossterm{shadowy illumination} (see \pcref{Low-light Vision}).
            \itemhead{Draconic Scales} Dragons gain a \plus2 bonus to their Armor defense.
            \itemhead{Draconic Weapons} Dragons have a bite natural weapon and two claw natural weapons.
                For details, see \pcref{Natural Weapons}.
            \itemhead{Draconic Wings} Dragons have scaly wings that sprout from their backs.
                These wings grant them a glide speed equal to the \glossterm{base speed} for their size (see \pcref{Gliding}).
                The wings themselves are \glossterm{mundane}, but the ability to fly and glide with them is \magical.
            \itemhead{Dragon Type} Each dragon has a single type from among the dragon types on \trefnp{Dragon Types}.
                They are immune to the damage type dealt by their type's breath weapon.
            \itemhead{Limited Equipment} A dragon's claws are not able to effectively wield shields or manufactured weapons.
                They can wear armor, but it is treated as \glossterm{barding} instead of normal armor, reducing its effectiveness (see \pcref{Barding}).
        \end{itemize}
        \parhead{Automatic Languages} Common, Draconic, any one \glossterm{common language}.

        \begin{dtable}
            % Don't use lcaption because there is already a Dragon Types table with a label
            \caption[]{Dragon Types}
            \begin{dtabularx}{\columnwidth}{l >{\lcol}X >{\lcol}X}
                \tb{Dragon} & \tb{Damage Type} & \tb{Breath Weapon} \tableheaderrule
                Black       & Acid             & \areamed, 5 ft. wide line \\
                Blue        & Electricity      & \areamed, 5 ft. wide line \\
                Brass       & Fire             & \areamed, 5 ft. wide line \\
                Bronze      & Electricity      & \areamed, 5 ft. wide line \\
                Copper      & Acid             & \areamed, 5 ft. wide line \\
                Gold        & Fire             & \areasmall cone           \\
                Green       & Acid             & \areasmall cone           \\
                Red         & Fire             & \areasmall cone           \\
                Silver      & Cold             & \areasmall cone           \\
                White       & Cold             & \areasmall cone           \\
            \end{dtabularx}
        \end{dtable}

        \subsubsection{Dragon Archetype}

            \cf{Dgn}[1]{Draconic Breath} You can use the \textit{breath weapon} ability as a \glossterm{standard action}.
            % +1r damage to compensate for cooldown... ish. Kind of awkward scaling right now.
            \begin{activeability}{Breath Weapon}
                \rankline
                Make an attack vs. Reflex against everything in the area defined by your dragon type (see Dragon Types, above).
                After you use this ability, you \glossterm{briefly} cannot use it again.
                \hit \damageranktwo{}.
                The damage type is defined by your dragon type.
                \miss Half damage.

                \rankline
                % T2 area
                \rank{2} The area increases.
                    A line breath weapon becomes a \arealarge, 5 ft.\ wide line.
                    A cone breath weapon becomes a \areamed cone.
                \rank{3} The damage increases to \damagerankthree{}.
                \rank{4} The damage increases to \damagerankfour{}.
                % T4 area
                \rank{5} The area increases.
                    A line breath weapon becomes a \areahuge, 10 ft.\ wide line.
                    A cone breath weapon becomes a \arealarge cone.
                \rank{6} The damage increases to \damageranksix{}.
                \rank{7} The damage increases to \damagerankseven{}.
            \end{activeability}

            \cf{Dgn}[2]{Draconic Body} You gain a \plus1 bonus to your Armor defense.

            \magicalcf{Dgn}[3]{Draconic Flight} Your wings grow larger, granting you a limited ability to fly.
            You gain a \glossterm{fly speed} equal to the \glossterm{base speed} for your size with a maximum height of 15 feet (see \pcref{Flight}).
            As a \glossterm{free action}, you can increase your \glossterm{fatigue level} by one to ignore this height limit until the end of the round.

            \cf{Dgn}[4]{Draconic Bulk} Your size category increases to Medium.
            This increases your \glossterm{base speed} to 30 feet.
            You reduce your Dexterity by 1 and increase your Strength by 2.
            In addition, you gain a \plus1 bonus to your \glossterm{power} with all abilities.

            \cf{Dgn}[5]{Draconic Body+} The Armor bonus from your \textit{draconic body} ability increases to \plus2.

            \magicalcf{Dgn}[6]{Draconic Flight+} The maximum height from your \textit{draconic flight} ability increases to 60 feet.
            In addition, you gain a \plus10 foot bonus to your fly speed with that ability.

            \cf{Dgn}[7]{Draconic Bulk+} Your size category increases to Large.
            This increases your \glossterm{base speed} to 40 feet.
            In addition, the attribute modifiers to Dexterity and Strength increase to \minus2 and \plus3 respectively, and the power bonus increases to \plus2.
            % Natural dragons treat their tail slam as heavy, but that may not work for a PC? They also treat their Bite as heavy, which these dragons don't do.
            You also gain a tail slam \glossterm{natural weapon}.
            It deals 1d10 bludgeoning damage and has the \weapontag{Impact} weapon tag (see \pcref{Weapon Tags}).
            % TODO: too fast?
            In addition, you gain a \plus20 foot bonus to your fly speed with your \textit{draconic flight} ability, but your maneuverability drops to poor maneuverability (see \pcref{Flying Maneuverability}).

        \subsubsection{Base Class Abilities}
            If you choose dragon as your base class, you gain the following abilities.

            \cf{Drg}{Defenses}
            You gain the following bonuses to your \glossterm{defenses}: \plus4 Fortitude, \plus2 Reflex, \plus3 Mental.

            \cf{Drg}{Hit Points}
                You have 8 hit points \add twice your Constitution, plus 2 hit points per level beyond 1.
                This increases as your level increases, as indicated below.
                \begin{itemize}
                    \itemhead{Level 7} 20 hit points \add three times your Constitution, plus 3 hit points per level beyond 7.
                    \itemhead{Level 13} 40 hit points \add six times your Constitution, plus 6 hit points per level beyond 13.
                    \itemhead{Level 19} 80 hit points \add twelve times your Constitution, plus 12 hit points per level beyond 19.
                \end{itemize}

            \cf{Drg}{Resources} You have the following \glossterm{resources}:
            \begin{itemize}
                \item Three \glossterm{attunement points}, which you can use to attune to items and abilities that affect you (see \pcref{Attunement Points}).
                \item A \glossterm{fatigue tolerance} equal to 3 \add your Constitution.
                    Your fatigue tolerance makes it easier for you to use powerful abilities that fatigue you (see \pcref{Fatigue}).
                \item A number of \glossterm{insight points} equal to 2 \add your Intelligence.
                    You can spend insight points to gain additional abilities (see \pcref{Insight Points}).
                \item Three \glossterm{trained skills} from among your \glossterm{class skills}, plus additional trained skills equal to your Intelligence (see \pcref{Skills}).
            \end{itemize}

            \cf{Drg}{Weapon Proficiencies} 
            You are not proficient with any weapon groups, even simple weapons.
            You are still proficient with your natural weapons.

            \cf{Drg}{Armor Proficiencies} 
            You are proficient with light and medium armor.
            Armor shaped appropriately for dragons can be hard to find, and may need to be crafted individually for the dragon.

            \cf{Drg}{Skills}
            You have the following \glossterm{class skills}:
            \begin{itemize}
                \item \subparhead{Strength} Climb, Swim.
                \item \subparhead{Dexterity} Balance, Stealth.
                \item \subparhead{Constitution} Endurance.
                \item \subparhead{Intelligence} Craft, Deduction, Knowledge (arcana), Medicine.
                \item \subparhead{Perception} Awareness, Creature Handling, Social Insight, Survival.
                \item \subparhead{Other} Deception, Intimidate, Persuasion.
            \end{itemize}

    \subsection{Drow}

        Drow are an offshoot group of elves that live deep underground.
        The deep caves are a far harsher environment than the surface world.
        Resources are scarce, and dangerous monsters are far more common.
        In order to survive, drow were forced to adopt a variety of practices condemned by surface civilizations.
        The most notorious are their frequent use of poison, their refusal to take prisoners, their willingness to eat any non-drow creatures they kill, even sentient creatures.
        In addition, drow society tends to reward selfishness and ambition more explicitly than surface civilizations, and the vast majority of drow are evil.

        When drow find opportunities to reach the surface world, they seek to conquer territory for themselves, usually with great violence.
        They have always been defeated and banished back to their caves, but surface civilizations still remember the danger that drow pose.
        Even more so than tieflings or orcs, who are already viewed with suspicion, drow are anathema in almost any civilized society.
        Drow who escape the deep caves are more likely to find a peaceful existence on other planes that do not fear an underground invasion.

        \parhead{Size} Medium.
        \parhead{Attributes} \minus1 Constitution, \plus1 Dexterity
        \parhead{Special Abilities}
        \begin{itemize}
            \itemhead{Darkvision} Drow have \trait{darkvision} with a 120 foot range, allowing them to see in complete darkness (see \pcref{Darkvision}).
            \itemhead{Deep Darkness}[\sparkle] A drow can use the \textit{deep darkness} ability as a \glossterm{standard action}.
                \begin{magicalsustainability}{Deep Darkness}{\abilitytag{Sustain} (minor)}
                    \rankline
                    \target{One \glossterm{zone} within \rngmed range}
                    You can choose this ability's radius, up to a maximum of a \areamed radius.
                    Light within or passing through the area is dimmed to be no brighter than \glossterm{shadowy illumination}
                    Any object or effect which blocks light also blocks this spell's effect.
                \end{magicalsustainability}
            \itemhead{Drow Prejudice} Almost all surface-dwellers have negative associations with drow.
                Drow have an Opposition relationship with most people that they meet, which influences people's behavior and makes Persuasion checks harder (see \pcref{Persuasion}).
                People in some locations, such as deep underground, do not have this attitude.
            \itemhead{Keen Senses} Drow gain a \plus2 bonus to the Awareness skill (see \pcref{Awareness}).
            \itemhead{Poison Tolerance} Drow are \trait{impervious} to poison.
            \itemhead{Sensitive Eyes} Drow take a \minus2 penalty to \glossterm{accuracy} while they are in \glossterm{bright illumination}.
                This penalty is doubled while they are in \glossterm{brilliant illumination}.
            \itemhead{Trance} Drow do not sleep, and are immune to \magical effects that would cause them to sleep.
                Instead of sleeping, drow can trance for 4 hours.
                An elf in trance may make Perception-based checks at a \minus5 penalty.
                Drow must still avoid strenuous activity for 8 hours to heal and gain other benefits of taking a \glossterm{long rest}.
        \end{itemize}
        \parhead{Automatic Languages} Common, Elven, Undercommon

    \subsection{Dryaidi}

        Dryaidi are humanoid creatures with plantlike characteristics.
        They might have leaves instead of hair, a green skin tone, or rough, barky skin.
        They are descended from dryads, and share some fey heritage and an affinity for trees.

        \parhead{Size} Medium.
        \parhead{Attributes} No change.
        \parhead{Special Abilities}
        \begin{itemize}
            \itemhead{Dryad Archetype} Dryaidi only gain two class archetypes instead of three.
                Instead, they treat the Dryad archetype as one of their archetypes, and they gain ranks in it just like they gain ranks in class archetypes.
            \itemhead{Enchanting Appearance} A dryaidi gains a \plus2 bonus to the Creature Handling, Perform, and Persuasion skills.
            \itemhead{Fey Vulnerability} Dryaidi are \vulnerable to cold iron weapons.
            \itemhead{Tree Bond} A dryaidi must be bonded with a specific tree.
            The tree must be at least a hundred years old, healthy, and intact.
            Forming a bond or severing a bond takes one week of meditation and ritual, periodically interrupted by rest.
            Forming a bond also requires asking permission from the tree through the ritual.
            Any individual tree can only be bonded to one dryad or dryaidi in this way.

            As long as the bonded tree remains healthy and intact, the dryaidi gains a \plus1 bonus to Mental defense and a \plus1 bonus to its \glossterm{fatigue tolerance}.
            If the bonded tree becomes unhealthy, is seriously damaged, or is killed, these bonuses are inverted into penalties until the dryaidi forms a bond with a new tree.
            A bonded dryaidi can passively observe the general health and status of the tree it bonded to.
            \itemhead{Verdant Flourishing} Dryaidi can use the \spell{fertile patch} and \spell{rapid growth} \glossterm{cantrips} from the \sphere{verdamancy} sphere.
            If they already have access to that sphere, they can cast each of those cantrips as a \glossterm{minor action}.
        \end{itemize}
        \parhead{Automatic Languages} Common, Sylvan.

        \subsubsection{Dryad Archetype}

            \cf{Dry}[1]{Tree Stride} You can walk into and through living trees.
            Moving through a tree does not impede your movement in any way, and you can end your movement inside a tree.
            When you do, you can choose to be partially melded or fully melded with the tree.
            While partially melded, the tree provides \glossterm{cover} against all attacks against you.
            While fully melded, the tree blocks \glossterm{line of sight} or \glossterm{line of effect} between you and the outside world as long as it remains intact.

            At the end of each round, if you are fully or partially melded with a tree that you are bonded with using your \textit{tree bond} ability, you regain hit points equal to half your maximum hit points.

            \cf{Dry}[2]{Natural Speech} You can speak with plants and animals as if they were capable of ordinary speech.
                This ability does not make them any more friendly or cooperative than normal.
                Wary and cunning animals are likely to be terse and evasive, while stupid ones tend to make inane comments and are unlikely to say or understand anything of use.
                Plants do not have complex thought processes, but can provide information about events that have happened near them.
                In general, plants can remember events that happened within the most recent quarter of their lifespan.

            \cf{Dry}[3]{Tree Stride+} You can \glossterm{teleport} between living trees instead of moving using your \glossterm{land speed}.
            Teleporting a given distance costs movement equal to half that distance.
            If this teleportation fails for any reason, you still expend that movement.

            \cf{Dry}[4]{Fey Charm} You can use the \ability{fey charm} ability as a \glossterm{minor action}.
            \begin{magicalsustainability}{Fey Charm}{\abilitytag{Emotion}, \abilitytag{Subtle}, \abilitytag{Sustain} (minor)}
                \rankline
                \noindent

                Make an attack vs. Mental against a creature within \medrange that is an animal, plant, or humanoid.
                You take a \minus10 penalty to \glossterm{accuracy} with this attack against creatures who have made an attack or been attacked since the start of the last round.
                \hit The target is \charmed by you.
                Any act by you or by creatures that appear to be your allies that threatens or harms the charmed person breaks the effect.
                Harming the target is not limited to dealing it damage, but also includes causing it significant subjective discomfort.
                An observant target may interpret overt threats to its allies as a threat to itself.

                \rankline

                \noindent The attack's \glossterm{accuracy} increases by \plus2 for each rank beyond 4.
            \end{magicalsustainability}

            \cf{Dry}[5]{Tree Bond+} You can bond to a grove of trees instead of a single tree.
            The cumulative age of all trees in the grove must be at least a thousand years, and the grove must fit within a 500 foot radius.
            While bonded to a grove, the bonuses from your \textit{tree bond} and \textit{enchanting appearance} abilities double.

            \cf{Dry}[6]{Tree Union} When you meld with a tree using your \textit{tree stride} ability, you can fully unite with it.
            When you do, you have \glossterm{line of sight} and \glossterm{line of effect} from all areas of the tree simultaneously, as if you were everywhere in the tree's body.
            Attacks against the tree simultaneously affect both you and the tree, but both you and the tree are \impervious to all damage except for fire damage and damage from cold iron weapons.
            You and the tree are instead \vulnerable to fire damage and damage from cold iron weapons.

            \cf{Dry}[7]{Acorns of Life} Whenever you visit a tree you are bonded to with your \textit{tree bond} ability, you can gather acorns of life.
            You can have up to ten acorns of life at once.
            As a \glossterm{minor action}, you can throw an acorn of life onto an unoccupied \glossterm{grounded} space within \medrange of you.
            The space must be made of dirt, earth, or stone.
            When the acorn lands, a tree immediately grows in that space.
            The tree has a five foot diameter trunk and grows vertically until it reaches a hundred feet tall or until it encounters a solid obstacle preventing its growth.

        \subsubsection{Base Class Abilities}
            If you choose dryad as your base class, you gain the following abilities.

            \cf{Dry}{Defenses}
            You gain the following bonuses to your \glossterm{defenses}: \plus2 Fortitude, \plus3 Reflex, \plus4 Mental.

            \cf{Dry}{Hit Points}
                You have 8 hit points \add  your Constitution, plus 1 hit points per level beyond 1.
                This increases as your level increases, as indicated below.
                \begin{itemize}
                    \itemhead{Level 7} 18 hit points \add twice your Constitution, plus 2 hit points per level beyond 7.
                    \itemhead{Level 13} 35 hit points \add five times your Constitution, plus 5 hit points per level beyond 13.
                    \itemhead{Level 19} 70 hit points \add ten times your Constitution, plus 10 hit points per level beyond 19.
                \end{itemize}

            \cf{Dry}{Resources} You have the following \glossterm{resources}:
            \begin{itemize}
                \item Three \glossterm{attunement points}, which you can use to attune to items and abilities that affect you (see \pcref{Attunement Points}).
                \item A \glossterm{fatigue tolerance} equal to 3 \add your Constitution.
                    Your fatigue tolerance makes it easier for you to use powerful abilities that fatigue you (see \pcref{Fatigue}).
                \item A number of \glossterm{insight points} equal to 2 \add your Intelligence.
                    You can spend insight points to gain additional abilities (see \pcref{Insight Points}).
                \item Five \glossterm{trained skills} from among your \glossterm{class skills}, plus additional trained skills equal to your Intelligence (see \pcref{Skills}).
            \end{itemize}

            \cf{Dry}{Weapon Proficiencies} 
            You are proficient with simple weapons and bows.

            \cf{Dry}{Armor Proficiencies} 
            You are proficient with light armor.

            \cf{Dry}{Skills}
            You have the following \glossterm{class skills}:
            \begin{itemize}
                \item \subparhead{Strength} Climb, Jump, Swim.
                \item \subparhead{Dexterity} Balance, Flexibility, Perform, Stealth.
                \item \subparhead{Intelligence} Craft (wood), Knowledge (arcana, nature), Medicine
                \item \subparhead{Perception} Awareness, Creature Handling, Deception, Persuasion, Social Insight, Survival.
                \item \subparhead{Other} Intimidate.
            \end{itemize}

    \subsection{Eladrin}

        \parhead{Size} Medium.
        \parhead{Attributes} \minus1 Constitution, either \plus1 Dexterity or \plus1 Willpower
        \parhead{Special Abilities}
        \begin{raggeditemize}
            \itemhead{Fae Step} As a standard action, you can use the \ability{fae step} ability.
            \begin{magicalactiveability}{Fae Step}
                \rankline
                You \glossterm{teleport} horizontally to a location within \shortrange.

                \rankline
                This ability improves based on your rank in your highest-rank archetype.
                \rank{3} The range increases to \medrange.
                \rank{5} The range increases to \longrange.
                \rank{7} The range increases to \distrange.
            \end{magicalactiveability}
            \itemhead{Fae Season} Eladrin respond strongly to their emotions, and change their abilities based on the season they currently represent.
                An eladrin must choose one of the following seasons when it finishes a \glossterm{short rest}.
                The chosen season lasts until it changes to a different season.
                \subcf{Spring} \plus1 bonus to Mental defense, \minus1 penalty to Fortitude defense.
                Eladrin expressing the spring season are filled with the joy of a new year.
                However, they are also visibly thinner and more frail, as if recovering from a long winter.
                \subcf{Summer} \plus1 bonus to Fortitude defense, \minus1 penalty to Reflex defense.
                Eladrin expressing the summer season are visibly hearty and a little more plump.
                However, they also move with all the alacrity of a long summer day.
                \subcf{Autumn} \plus1 bonus to all checks, \minus1 penalty to \glossterm{accuracy}.
                Eladrin expressing the autumn season embody the spirit of the harvest.
                They are filled with goodwill towards all creatures, and prefer finding peaceful solutions to problems.
                Their bodies tend to be firm and toned, reflecting the hard work required to prepare for the winter.
                \subcf{Winter} \plus1 bonus to \glossterm{vital rolls}, \minus1 penalty to Mental defense.
                Eladrin expressing the winter season are prepared for the worst.
                They tend to be dour and pessimistic, but they press on despite the certainty of doom.
            \itemhead{Low-light Vision} Eladrin have \trait{low-light vision}, allowing them to see clearly in \glossterm{shadowy illumination} (see \pcref{Low-light Vision}).
            \itemhead{Trance} Eladrin do not sleep, and are immune to \magical effects that would cause them to sleep.
                Instead of sleeping, eladrin can trance for 4 hours.
                An eladrin in trance may make Perception-based checks at a \minus5 penalty.
                Eladrin must still avoid strenuous activity for 8 hours to heal and gain other benefits of taking a \glossterm{long rest}.
        \end{raggeditemize}
        \parhead{Species Feat Options} 
        \parhead{Automatic Languages} Common, Sylvan, and any one \glossterm{common language} (see \tref{Common Languages}).

    \subsection{Harpy}
        Harpies are winged creatures with the upper body of a humanoid and the lower body of a bird.
        Most harpies are female, but male harpies do exist.

        \parhead{Creature Type} Unlike most other playable species, harpies are monstrous humanoids instead of humanoids.
        \parhead{Size} Medium.
        \parhead{Attributes} \minus1 Intelligence, \plus1 Dexterity.
        \parhead{Special Abilities}
        \begin{itemize}
            \itemhead{Harpy Archetype} Harpies only gain two class archetypes instead of three.
                Instead, they treat the Harpy archetype as one of their archetypes, and they gain ranks in it just like they gain ranks in class archetypes.
            \itemhead{Limited Equipment} Harpies can wear armor, but it is treated as \glossterm{barding} instead of normal armor, reducing its effectiveness (see \pcref{Barding}).
                Harpy talons are not able to effectively wield shields or manufactured weapons.
            \itemhead{Prehensile Talons} Harpies have a talon natural weapon on each foot (see \pcref{Natural Weapons}).
                In addition, they can use their feet as \glossterm{free hands}.
            \itemhead{Wings} Harpies have a feathered wings that sprout from their shoulders.
                These wings grant them a glide speed equal to the \glossterm{base speed} for their size (see \pcref{Gliding}).
                However, they have no arms or hands.
        \end{itemize}
        \parhead{Automatic Languages} Common.
        
        \subsubsection{Harpy Archetype}

            \cf{Hrp}[1]{Winged Agility} While you are able to use your wings, you gain a \plus2 bonus to Armor defense and a \plus4 bonus to the Balance and Jump skills.
            In addition, you gain a \plus10 foot bonus to your maximum horizontal jump distance (see \pcref{Jumping}).
            This increases your maximum vertical jump distance normally.

            \cf{Hrp}[1]{Winged Combat} As a \glossterm{free action}, you can make a short hop off of solid ground to hover in your space using your wings.
            This allows you to use both of your feet without standing on them during that phase, so you can attack with both talons at once.
            You must land at the end of the phase, and you can only use this ability once per round.
            This brief hovering does not cause you to suffer penalties for flying.

            \magicalcf{Hrp}[2]{Luring Song} You can use the \textit{luring song} ability as a standard action.
            \begin{magicalactiveability}{Luring Song}[\abilitytag{Auditory}, \abilitytag{Compulsion}]
                    \rankline
                    Make an attack vs. Mental against a creature within \longrange.
                    \hit As a \glossterm{condition}, the target must move towards you as best it can during each \glossterm{movement phase}.
                    In addition, it cannot move farther away from you at any time, except as necessary to get closer to you (such as to avoid an intervening obstacle).
                    It can otherwise act freely, and is still able to attack you and your allies.

                    If you attack the target with any ability other than this one, this effect is automatically broken.
                    When this effect ends, the target becomes immune to this effect until it finishes a \glossterm{short rest}.
                    \crit The condition must be removed twice before the effect ends.

                    \rankline
                    \rank{4} You can target an additional creature within range.
                    \rank{6} The maximum number of targets increases to 5.
                \end{magicalactiveability}

            \cf{Hrp}[3]{Flight} You gain a \glossterm{fly speed} equal to the \glossterm{base speed} for your size with a maximum height of 15 feet (see \pcref{Flight}).
            As a \glossterm{free action}, you can increase your \glossterm{fatigue level} by one to ignore this height limit until the end of the round.

            \cf{Hrp}[4]{Greater Winged Agility} You can use your wings to help you maneuver on the ground more effectively, such as by briefly hovering or gliding over obstacles.
            The Balance and Jump bonuses from your \textit{winged agility} ability increase to \plus8.
            In addition, you ignore \glossterm{difficult terrain}.

            \cf{Hrp}[4]{Greater Winged Combat} You can use your \textit{winged combat} any number of times in the same round.
            You must still land at the end of each phase.

            % TODO: define correct rank
            \magicalcf{Hrp}[5]{Siren Song} You can use the \textit{siren song} ability as a standard action.
            \begin{magicalsustainability}{Siren Song}{\abilitytag{Auditory}, \abilitytag{Emotion}, \abilitytag{Sustain} (minor)}
                \rankline
                Make an attack vs. Mental against all \glossterm{enemies} within a \medarea radius from you.
                \hit Each target is both \charmed by you and \stunned as long as it can still hear you sing.
                It remains stunned even if it stops being charmed, such as if you or your allies attack it.
                This ability does not have the \abilitytag{Subtle} tag, so an observant target may notice that it is being influenced.
                \rankline
                \rank{7} The area increases to a \largearea radius.
            \end{magicalsustainability}

            \cf{Hrp}[6]{Agile Flight} Your flight improves to have good maneuverability (see \pcref{Flying Maneuverability}).
            In addition, your maximum height increases to 30 feet.

            \magicalcf{Hrp}[7]{Mythic Siren} You gain a \plus5 \glossterm{accuracy} bonus with your \ability{luring song} and \ability{siren song} abilities.

        \subsubsection{Base Class Abilities}
            If you choose harpy as your base class, you gain the following abilities.

            \cf{Hrp}{Defenses}
            You gain the following bonuses to your \glossterm{defenses}: \plus2 Fortitude, \plus4 Reflex, \plus3 Mental.

            \cf{Hrp}{Hit Points}
                You have 8 hit points \add twice your Constitution, plus 2 hit points per level beyond 1.
                This increases as your level increases, as indicated below.
                \begin{itemize}
                    \itemhead{Level 7} 20 hit points \add three times your Constitution, plus 3 hit points per level beyond 7.
                    \itemhead{Level 13} 40 hit points \add six times your Constitution, plus 6 hit points per level beyond 13.
                    \itemhead{Level 19} 80 hit points \add twelve times your Constitution, plus 12 hit points per level beyond 19.
                \end{itemize}

            \cf{Hrp}{Resources} You have the following \glossterm{resources}:
            \begin{itemize}
                \item Two \glossterm{attunement points}, which you can use to attune to items and abilities that affect you (see \pcref{Attunement Points}).
                \item A \glossterm{fatigue tolerance} equal to 4 \add your Constitution.
                    Your fatigue tolerance makes it easier for you to use powerful abilities that fatigue you (see \pcref{Fatigue}).
                \item A number of \glossterm{insight points} equal to 2 \add your Intelligence.
                    You can spend insight points to gain additional abilities (see \pcref{Insight Points}).
                \item Five \glossterm{trained skills} from among your \glossterm{class skills}, plus additional trained skills equal to your Intelligence (see \pcref{Skills}).
            \end{itemize}

            \cf{Hrp}{Weapon Proficiencies} 
            You are proficient with simple weapons.

            \cf{Hrp}{Armor Proficiencies} 
            You are proficient with light armor.

            \cf{Hrp}{Skills}
            You have the following \glossterm{class skills}:
            \begin{itemize}
                \item \subparhead{Strength} Climb, Jump.
                \item \subparhead{Dexterity} Balance, Flexibility, Perform, Stealth.
                \item \subparhead{Perception} Awareness, Creature Handling, Deception, Persuasion, Survival.
                \item \subparhead{Other} Intimidate.
            \end{itemize}

    \subsection{Kit}

        Kit are humanoid creatures that have noticeable foxlike characteristics.
        They are descended from natural fox spirits.
        All kit have at least one tail, and some have multiple tails.
        Their tails are distinctly fluffy and fox-like, and most kit put effort into concealing their tails to avoid revealing their true nature.

        \parhead{Size} Medium.
        \parhead{Attributes} No change.
        \parhead{Special Abilities}
        \begin{itemize}
            \itemhead{Foxlike Agility} Kit gain a \plus2 bonus to the Balance and Stealth skills.
            \itemhead{Illusory Guise} As a standard action, a kit can magically disguise its physical appearance in minor ways.
                This functions like the \textit{change appearance} ability with a \plus4 bonus, except that a kit cannot change the appearance of its equipment, creature type, or number of limbs, including any tails it may have (see \pcref{Change Appearance}).
                This is a \magical ability.
                It lasts until the kit \glossterm{dismisses} it as a free action or uses this ability again.
            \itemhead{Instictive Trickster} Kit gain a \plus2 bonus to the Deception and Social Insight skills.
            \itemhead{Low-light Vision} Kit have \trait{low-light vision}, allowing them to see clearly in \glossterm{shadowy illumination} (see \pcref{Low-light Vision}).
        \end{itemize}
        \parhead{Automatic Languages} Common, any one \glossterm{common language}.

    \subsection{Naiadi}
        Naiadi are humanoid creatures descended from naiads.
        Most naiadi are unusually physically appealing, but show no other outward signs of their heritage.

        \parhead{Size} Medium.
        \parhead{Attributes} No change.
        \parhead{Special Abilities}
        \begin{itemize}
            \itemhead{Create Water} A naiadi can cast the \spell{create water} cantrip.
                When they do so, they do not require verbal or somatic \glossterm{casting components}, and their spellcasting rank is considered to be equal to their rank in their highest rank archetype.
                If they would already know that cantrip through the Aquamancy sphere, the volume of water created with the cantrip doubles.
            \itemhead{Enchanting Appearance} A naiadi gains a \plus2 bonus to the Creature Handling, Perform, and Persuasion skills.
            \itemhead{Low-light Vision} Naiadi have \trait{low-light vision}, allowing them to see clearly in \glossterm{shadowy illumination} (see \pcref{Low-light Vision}).
            \itemhead{Naiad Archetype} You may choose three class archetypes, as normal.
                However, you may choose the Naiad archetype in place of one of your class archetypes.
                If you do, you gain ranks in it just like you gain ranks in class archetypes.
                You cannot choose naiad as your base class.
            \itemhead{Water Affinity} A naiadi has a \glossterm{swim speed} equal to the \glossterm{base speed} for their size.
                In addition, they can breathe clean water like a human breathes air.
        \end{itemize}
        \parhead{Automatic Languages} Common, Sylvan, any one \glossterm{common language}.

        \subsubsection{Naiad Archetype}

            \magicalcf{Nai}[1]{Water Bond} You can form a bond with a fresh stream, lake, or other Gargantuan or larger body of fresh water (not salt water).
            Forming a bond or severing a bond takes one week of meditation and ritual, periodically interrupted by rest.
            Forming a bond also requires asking permission from the water.
            Any individual body of water can only be bonded to one naiad or naiadi in this way.

            As long as your bonded water remains clean, pure, and large enough to be a valid subject of bonding, you gain a \plus1 bonus to Mental defense and a bonus equal to twice your rank in this archetype to your \glossterm{hit points}.
            If your bonded water becomes contaminated or shrinks below the minimum size, these bonuses are inverted into penalties until you sever the bond.
            You can passively observe the general health and status of water you are bonded to, including knowing when significant pollutants enter the water and when the water grows or shrinks significantly.

            \magicalcf{Nai}[2]{Fluidseeker} You gain a \plus1 bonus to \glossterm{accuracy} against creatures significantly composed of water or watery fluids.
            This is true of almost all living creatures.

            \magicalcf{Nai}[2]{Freshwater Fountain} The volume of water you can create with the \spell{create water} cantrip increases by five times.

            \magicalcf{Nai}[3]{Aqueous Form} You can cast the \spell{aqueous form} spell.
            When you do, you do not require verbal or somatic \glossterm{casting components}, and you use your rank in this archetype as your your spellcasting rank.
            In addition, it has the Attune tag instead of the Attune (deep) tag.

            \magicalcf{Nai}[4]{Greater Water Bond} The bonuses from your \textit{water bond} ability increase to \plus2 Mental defense and three times your rank in this archetype to your hit points.

            \magicalcf{Nai}[5]{Greater Fluidseeker} The accuracy bonus from your \textit{fluidseeker} ability increases to \plus2.

            \magicalcf{Nai}[5]{Greater Freshwater Fountain} The multiplier from your \textit{freshwater fountain} ability increases to twenty times the normal volume of water.

            \magicalcf{Nai}[6]{Greater Aqueous Form} When you cast the \spell{aqueous form} spell, it does not have the \abilitytag{Attune} tag.
            Instead, it lasts until you \glossterm{dismiss} it as a \glossterm{free action}.

            \magicalcf{Nai}[7]{Supreme Water Bond} The bonuses from your \textit{water bond} ability increase to \plus3 Mental defense and three times your rank in this archetype to your hit points.

    \subsection{Oozeborn}
        \includegraphics[width=\columnwidth]{optional rules/oozeborn}
        Oozeborn are ooze creatures that have gained true sentience through a strange quirk of their birth.
        They are very rare to see in civilized lands, as most oozeborn lack the opportunity to discover more than the dark caves in which they were spawned.
        Since they often grow up without mentorship from any civilized creature, oozeborn tend to have odd mannerisms and a poor ability to mask their emotions, even after spending years in civilization.
        Old oozeborn may eventually adapt to societal norms and act perfectly natural, or they may abandon civilized company entirely.

        The body of an oozeborn is amorphous, and they lack any identifiable internal organs.
        Their natural color depends on the nature of the ooze that spawned them, so green and gray are the most common colors.
        Adventuring oozeborn typically assume a bipedal shape for both practical and social convenience, but their natural shape is a loosely spherical blob.
        Unconscious oozeborn revert to their default state automatically, though some learn to maintain a semblance of cohesion while asleep.

        \parhead{Creature Type} Unlike most other playable species, oozeborn are animates instead of humanoids.
        \parhead{Size} Medium.
        \parhead{Attributes} \minus1 Intelligence, \plus1 Constitution.
        \parhead{Special Abilities}
        \begin{itemize}
            \itemhead{Acidic Body} Ooozeborn are \trait{impervious} to acid damage and poisons.
            \itemhead{Amorphous Form} An oozeborn's natural form is a loosely spherical blob.
                They have a \minus10 foot penalty to their \glossterm{land speed}, but they gain a \plus5 bonus to the Flexibility skill (see \pcref{Flexibility}).
                They can use the \ability{mold body} ability as a standard action to adopt a particular shape.
                \begin{sustainability}{Mold Body}{\abilitytag{Sustain} (free)}
                    \rankline
                    You make a Disguise check to alter your appearance (see \pcref{Change Appearance}).
                    This physically changes your body to match the results of your disguise.
                    You gain a \plus4 bonus on the check, and you ignore penalties for changing your gender, species, subtype, age, and number of limbs (up to 4).
                    However, this effect is unable to alter your equipment in any way.

                    You cannot create more than two \glossterm{free hands} with this ability.
                    If you add at least two legs, you gain a \plus10 foot bonus to your land speed.
                    % TODO: awkwardly worded
                    This speed bonus does not stack with the bonus for becoming \trait{multipedal}, so the only benefit you gain from creating three or more legs is a \plus5 bonus to the Balance skill.
                    If you give yourself a standard humanoid shape, you can wear armor designed for humanoids without suffering the normal penalties for \glossterm{barding} (see \pcref{Barding}).

                    You can sustain this ability for any length of time without mental strain, ignoring the normal 5 minute limit.
                \end{sustainability}
            \itemhead{Compressible Body} Oozeborn can compress their head and shoulders down to a minimum of a one inch radius, allowing them to squeeze through very small areas.
                Their clothing or armor is not compressed, so they may limit their ability to move through extremely narrow spaces.
            \itemhead{Darkvision} Oozeborn have \trait{darkvision} with a 60 foot range, allowing them to see in complete darkness (see \pcref{Darkvision}).
            \itemhead{Oozeborn Archetype} Oozeborn only gain two class archetypes instead of three.
                Instead, they treat the Oozeborn archetype as one of their archetypes, and they gain ranks in it just like they gain ranks in class archetypes.
        \end{itemize}
        \parhead{Automatic Languages} Common.
        
        \subsubsection{Oozeborn Archetype}

            \cf{Ooz}[1]{Acidic Pseudopod} One of your arms becomes a pseudopod \glossterm{natural weapon}.
            It deals 1d10 bludgeoning and acid damage and has the \weapontag{Long} weapon tag (see \pcref{Weapon Tags}).
            You do not have a \glossterm{free hand} on that arm while using it as a weapon in this way.

            In addition, all damage you deal with natural weapons is acid damage in addition to its other types.
            This does not affect damage you deal with manufactured weapons.

            \cf{Ooz}[2]{Darkborn Senses} You gain \trait{blindsense} with a 60 foot range, allowing you to sense your surroundings without light (see \pcref{Blindsense}).
            If you already have the blindsense ability, you increase its range by 60 feet.
            In addition, you gain \trait{blindsight} with a 15 foot range, allowing you to see without light (see \pcref{Blindsight}).
            If you already have the blindsight ability, you increase its range by 15 feet.

            \cf{Ooz}[2]{Ingest Object} You can use the \textit{ingest object} ability as a standard action.
            This functions like the \spell{absorb object} spell, except that the maximum size of the object is equal to your size.
            Anything you absorb in this way takes a single point of \glossterm{environmental} acid damage during each of your actions while it remains absorbed.
            This damage is insufficient to hurt most objects made from wood, stone, or metal, but it can destroy more fragile objects like paper or complex mechanical traps.

            \cf{Ooz}[3]{Greater Amorphous Form} You gain a \plus4 bonus to your defenses when determining whether a \glossterm{strike} gets a \glossterm{critical hit} against you instead of a normal hit.
            In addition, your \ability{mold body} ability loses the \abilitytag{Sustain} (free) tag.
            Instead, it lasts until you choose to \glossterm{dismiss} it as a \glossterm{free action}.
            This allows you to maintain your shape while unconscious.

            \cf{Ooz}[3]{Greater Compressible Body} You reduce your penalties for \squeezing by 1.

            \cf{Ooz}[4]{Acidic Body+} You are \trait{immune} to acid damage and poisons.

            \cf{Ooz}[5]{Greater Darkborn Senses} The range of your \trait{blindsense} increases by 60 feet.
            In addition, the range of your \trait{blindsight} increases by 15 feet.

            \cf{Ooz}[5]{Greater Ingest Object} The maximum number of objects you can absorb with your \textit{ingest object} ability increases to 2.
            In addition, you may absorb \glossterm{allies} with that ability in addition to unattended objects.

            \cf{Ooz}[6]{Supreme Amorphous Form} The bonus from your \textit{greater amorphous form} ability increases to \plus8.

            \cf{Ooz}[6]{Supreme Compressible Body} You reduce your penalties for squeezing by 2, which means you take no penalties for squeezing unless you use the \ability{tight squeeze} ability (see \pcref{Flexibility}).

            \cf{Ooz}[7]{Third Arm} When you use your \ability{mold body} ability, you can create three arms instead of two.
            You can use all three hands as free hands.
            For example, this can allow you to use a \weapontag{Heavy} weapon and a shield simultaneously.

            In addition, your arms become stronger and more agile.
            You can use any of your arms as a pseudopod natural weapon, and your pseudopods gain the \weapontag{Light} weapon tag (see \pcref{Weapon Tags}).

        \subsubsection{Base Class Abilities}
            If you choose oozeborn as your base class, you gain the following abilities.

            \cf{Ooz}{Defenses}
            You gain the following bonuses to your \glossterm{defenses}: \plus4 Fortitude, \plus2 Reflex, \plus3 Mental.

            \cf{Ooz}{Hit Points}
                You have 10 hit points \add twice your Constitution, plus 2 hit points per level beyond 1.
                This increases as your level increases, as indicated below.
                \begin{itemize}
                    \itemhead{Level 7} 24 hit points \add four times your Constitution, plus 4 hit points per level beyond 7.
                    \itemhead{Level 13} 50 hit points \add eight times your Constitution, plus 8 hit points per level beyond 13.
                    \itemhead{Level 19} 100 hit points \add fifteen times your Constitution, plus 15 hit points per level beyond 19.
                \end{itemize}

            \cf{Ooz}{Resources} You have the following \glossterm{resources}:
            \begin{itemize}
                \item Two \glossterm{attunement points}, which you can use to attune to items and abilities that affect you (see \pcref{Attunement Points}).
                \item A \glossterm{fatigue tolerance} equal to 4 \add your Constitution.
                    Your fatigue tolerance makes it easier for you to use powerful abilities that fatigue you (see \pcref{Fatigue}).
                \item A number of \glossterm{insight points} equal to 1 \add your Intelligence.
                    You can spend insight points to gain additional abilities (see \pcref{Insight Points}).
                \item Four \glossterm{trained skills} from among your \glossterm{class skills}, plus additional trained skills equal to your Intelligence (see \pcref{Skills}).
            \end{itemize}

            \cf{Ooz}{Weapon Proficiencies} 
            You are proficient with simple weapons.

            \cf{Ooz}{Armor Proficiencies} 
            You are proficient with light and medium armor.
            Depending on whether you are sustaining your \ability{mold body} and the form you choose, you may need \glossterm{barding} instead of regular armor (see \pcref{Barding}).

            \cf{Ooz}{Skills}
            You have the following \glossterm{class skills}:
            \begin{itemize}
                \item \subparhead{Strength} Climb, Swim.
                \item \subparhead{Dexterity} Balance, Flexibility, Sleight of Hand, Stealth.
                \item \subparhead{Constitution} Endurance.
                \item \subparhead{Intelligence} Craft, Knowledge (dungeoneering).
                \item \subparhead{Perception} Awareness, Survival.
                \item \subparhead{Other} Intimidate.
            \end{itemize}

    \subsection{Sapling}

        Saplings are young treants that have left their forest home in search of adventure.
        They tend to be slow to think and act, but resilient once they have made up their mind.

        \parhead{Creature Type} Unlike most other playable species, saplings are considered animates instead of humanoids.
        \parhead{Size} Medium.
        \parhead{Attributes} \plus1 Constitution, \plus1 Willpower, \minus1 Dexterity, \minus1 Intelligence
        \parhead{Special Abilities}
        \begin{itemize}
            \itemhead{Barkskin} A sapling gains a \plus2 bonus to Armor defense.
            \itemhead{Ingrain} A sapling can use the \textit{ingrain} ability as a \glossterm{minor action} while it is \glossterm{grounded}.
                \begin{activeability}{Ingrain}
                    \rankline
                    The sapling's land speed becomes 5 feet, regardless of any modifiers that normally apply.
                    It cannot voluntarily stop being \glossterm{grounded} while this ability lasts.
                    It gains a \plus4 bonus to Fortitude defense and a \plus2 bonus to Armor defense.

                    If the sapling finishes a \glossterm{long rest} with this ability active for the duration of the rest, it acquires nutrients sufficient to replace a day's worth of food and water.
                    This ability lasts until the sapling ends it as a standard action, or until it stops being \glossterm{grounded}.
                \end{activeability}
            \itemhead{Limited Equipment} Saplings can wear armor, but it is treated as \glossterm{barding} instead of normal armor, reducing its effectiveness (see \pcref{Barding}).
            \itemhead{Made of Wood} Saplings are \vulnerable to fire damage. In addition, they are both creatures and plants.
            \itemhead{Treant Archetype} Saplings only gain two class archetypes instead of three.
                Instead, they treat the Treant archetype as one of their archetypes, and they gain ranks in it just like they gain ranks in class archetypes.
            \itemhead{Tree Appearance} When a sapling stays perfectly still, observers must make a DV 15 Awareness check to recognize that it is not an ordinary tree.
            Careful observers may still notice that the ordinary tree has appeared where no tree used to be, so they may be suspicious of a sapling even if they do not pass this check.
            \itemhead{Unhurried and Unfaltering} Saplings have a \minus10 penalty to speed with all \glossterm{movement modes}.
                However, saplings ignore all penalties to their land speed, such as from wearing heavy armor.
                Those penalties still apply to any other movement modes normally.
                This also does not change effects that completely replace the sapling's land speed, such as the \ability{ingrain} ability.
                In addition, saplings ignore \glossterm{difficult terrain} from inanimate natural sources, such as \glossterm{heavy undergrowth}.
        \end{itemize}
        \parhead{Automatic Languages} Common, Sylvan.

        \subsubsection{Treant Archetype}
            % Nourishing Ingrain:
            % * R1, 2 HP - 25% of fighter base HP (8)
            % * R2, 4 HP - 28.5% of fighter base HP (14)
            % * R3, 6 HP - 30% of fighter base HP (20)
            % * R4, 8 HP - 27.5% (29)
            % * R5, 15 HP - 37.5% (40)
            % * R6, 18 HP - 31% (58)
            % * R7, 21 HP - 26% (80)

            % This intentionally can bring you above half max
            \cf{Tre}[1]{Nourishing Ingrain} At the end of each round while you are \ability{ingrained}, you regain hit points equal to twice your rank in this archetype, and you may choose to remove a \glossterm{condition}.
            If you do, you increase your \glossterm{fatigue level} by one.

            \cf{Tre}[2]{Sturdy as the Mighty Oak} You gain a bonus equal to three times your rank in this archetype to your \glossterm{hit points} (see \pcref{Hit Points}).
            In addition, you gain a \plus1 bonus to your \glossterm{vital rolls} (see \pcref{Vital Wounds}).

            \cf{Tre}[3]{Animate Plants} You can use the \textit{animate plants} ability as a standard action.
            \begin{activeability}{Animate Plants}
                \spelltwocol{}{\abilitytag{Manifestation}}
                \rankline
                Make an attack vs. Reflex against one Large or smaller \glossterm{grounded} creature within \medrange.
                You gain a \plus2 accuracy bonus if the target is in \glossterm{undergrowth}.

                \hit The target is \slowed as a \glossterm{condition}.
                In addition, it takes 1d8 bludgeoning damage immediately, and during each of your subsequent actions while this condition lasts.

                This condition can be removed if the target makes a \glossterm{difficulty value} 10 Strength check as a \glossterm{movement} to break the plants.
                If the target makes this check as a standard action, it gains a \plus5 bonus.
                In addition, this condition is removed if the target takes fire damage.
                \crit The condition must be removed an additional time before the effect ends.
                \rankline
                For each rank beyond 3, the attack's \glossterm{accuracy} increases by \plus2 and the \glossterm{difficulty value} to break the plants increases by 2.
                In addition, the damage increases at each rank as described below.
                \rank{4} 1d10 bludgeoning damage.
                \rank{5} 2d6 bludgeoning damage.
                \rank{6} 2d8 bludgeoning damage.
                \rank{7} 2d10 bludgeoning damage.
            \end{activeability}

            \cf{Tre}[4]{Tall as the Noble Pine} Your size category increases to Large.
            This would normally increase your \glossterm{base speed} to 40 feet.
            However, you take a \minus10 foot penalty to your base speed, so it is still 30 feet.
            You also gain a \plus1 bonus to your Strength, and a \minus1 penalty to your Dexterity.

            \cf{Tre}[5]{Nourishing Ingrain+} The healing from your \textit{nourishing ingrain} ability increases to three times your rank in this archetype.
            In addition, removing a condition with that ability no longer increases your fatigue level.

            \cf{Tre}[6]{Sturdy as the Mighty Oak+} The hit point bonus increases to four times your rank in this archetype.

            \cf{Tre}[7]{Tall as the Noble Pine+} Your size category increases to Huge.
            This increases your \glossterm{base speed} to 40 feet.
            The modifiers to Strength and Dexterity increase to \plus2 and \minus2, respectively.

        \subsubsection{Base Class Abilities}
            If you choose treant as your base class, you gain the following abilities.

            \cf{Tre}{Defenses}
            You gain the following bonuses to your \glossterm{defenses}: \plus5 Fortitude, \plus3 Reflex, \plus4 Mental.

            \cf{Tre}{Hit Points}
                You have 10 hit points \add twice your Constitution, plus 2 hit points per level beyond 1.
                This increases as your level increases, as indicated below.
                \begin{itemize}
                    \itemhead{Level 7} 24 hit points \add four times your Constitution, plus 4 hit points per level beyond 7.
                    \itemhead{Level 13} 50 hit points \add eight times your Constitution, plus 8 hit points per level beyond 13.
                    \itemhead{Level 19} 100 hit points \add fifteen times your Constitution, plus 15 hit points per level beyond 19.
                \end{itemize}

            \cf{Tre}{Resources} You have the following \glossterm{resources}:
            \begin{itemize}
                \item Two \glossterm{attunement points}, which you can use to attune to items and abilities that affect you (see \pcref{Attunement Points}).
                \item A \glossterm{fatigue tolerance} equal to 4 \add your Constitution.
                    Your fatigue tolerance makes it easier for you to use powerful abilities that fatigue you (see \pcref{Fatigue}).
                \item A number of \glossterm{insight points} equal to 1 \add your Intelligence.
                    You can spend insight points to gain additional abilities (see \pcref{Insight Points}).
                \item Three \glossterm{trained skills} from among your \glossterm{class skills}, plus additional trained skills equal to your Intelligence (see \pcref{Skills}).
            \end{itemize}

            \cf{Tre}{Weapon Proficiencies} 
            You are proficient with simple weapons and club-like weapons.

            \cf{Tre}{Armor Proficiencies} 
            You are proficient with light, medium, and heavy armor.

            \cf{Tre}{Skills}
            You have the following \glossterm{class skills}:
            \begin{itemize}
                \item \subparhead{Dexterity} Balance.
                \item \subparhead{Constitution} Endurance.
                \item \subparhead{Intelligence} Knowledge (nature).
                \item \subparhead{Perception} Awareness, Creature Handling, Survival.
                \item \subparhead{Other} Intimidate.
            \end{itemize}

    \subsection{Tiefling}
        \includegraphics[width=\columnwidth]{optional rules/tiefling}

        Tieflings are humanoid creatures descended from fiends.
        They inherit a tendency towards evil from their ancestors, and are therefore viewed with great suspicion by most civilized societies.
        Good-aligned tieflings exist, but they may have difficulty using their natural talents for subterfuge and deceit for noble ends, and they often struggle with hidden vices.

        \parhead{Size} Medium.
        \parhead{Attributes} No change.
        \parhead{Special Abilities}
        \begin{itemize}
            \itemhead{Darkvision} Tieflings have \trait{darkvision} with a 60 foot range, allowing them to see in complete darkness (see \pcref{Darkvision}).
            \itemhead{Demonic Prejudice} Most people have negative associations with tieflings thanks to the malign influence that demons have on the world.
                Tieflings have an Opposition relationship with most people that they meet, which influences people's behavior and makes Persuasion checks harder (see \pcref{Persuasion}).
                People in some locations, such as the Abyss, do not have this attitude.
            \itemhead{Hellfire Tolerance} Tieflings are \trait{impervious} to fire damage.
            \itemhead{Infernal Presence} Tieflings gain a \plus2 bonus to the Deception and Intimidate skills.
            \itemhead{Tiefling Archetype} You may choose three class archetypes, as normal.
                However, you may choose the Tiefling archetype in place of one of your class archetypes.
                If you do, you gain ranks in it just like you gain ranks in class archetypes.
                You cannot choose tiefling as your base class.
        \end{itemize}
        \parhead{Automatic Languages} Abyssal, Common, any one \glossterm{common language}.

        \subsubsection{Tiefling Archetype}
            \magicalcf{Tif}[1]{Abyssal Hop} You can use the \ability{abyssal hop} ability as a standard action.
            \begin{magicalactiveability}{Abyssal Hop}
                \rankline
                You teleport horizontally into an unoccupied location within \shortrange on a stable surface that can support your weight.
                If the destination is invalid, this spell fails with no effect.
                In addition, make an attack vs. Reflex against each \glossterm{enemy} adjacent to your location after you arrive.
                \hit \damagerankzero{fire}.
                \miss Half damage.

                \rankline
                % Stay at 1 rank lower than par
                \rank{2} The damage bonus from your power increases to \plus1 per 2 power.
                \rank{3} The base damage increases to 1d8.
                \rank{4} The damage bonus increases to be equal to your power.
                \rank{5} The base damage increases to 1d10.
                \rank{6} The damage bonus increases to 1d8 per 3 power.
                \rank{7} The base damage increases to 2d8.
            \end{magicalactiveability}

            \cf{Tif}[1]{Infernal Resilience} You gain a bonus equal to twice your rank in this archetype to your \glossterm{damage resistance}.

            \magicalcf{Tif}[2]{Infernal Ancestry} You deepen your connection to a particular aspect of your demonic ancestry.
            Choose one of the following infernal ancestries: hellfire conduit, tempting allure, or unholy might.
            You gain a benefit based on your chosen ancestry.
            \begin{itemize}
                \item Infernal Rebuke: You can use the \textit{infernal rebuke} ability as a standard action.
                    \begin{magicalactiveability}{Infernal Rebuke}
                        \rankline
                        Make an attack vs. Fortitude against one creature within \shortrange.
                        You gain a \plus2 bonus to \glossterm{accuracy} with this attack if the target attacked you during the previous round.
                        \hit \damagerankthree{fire}.

                        \rankline
                        % Stay at rank+1 damage
                        \rank{3} The base damage increases to 1d10.
                        \rank{4} The damage bonus from your power increases to 1d8 per 3 power.
                        \rank{5} The base damage increases to 2d8.
                        \rank{6} The damage bonus from your power increases to 1d8 per 2 power.
                        \rank{7} The base damage increases to 4d8.
                    \end{magicalactiveability}
                \item Tempting Allure: You gain a \plus2 bonus to the Deception, Disguise, and Persuasion skills.
                    In addition, you can use the \ability{charming temptation} ability as a standard action.
                    \begin{magicalsustainability}{Charming Temptation}{\abilitytag{Emotion}, \abilitytag{Subtle}, \abilitytag{Sustain} (minor)}
                        \rankline
                        \noindent

                        Make an attack vs. Mental against a humanoid creature within \medrange.
                        You take a \minus10 penalty to \glossterm{accuracy} with this attack against creatures who have made an attack or been attacked since the start of the last round.
                        \vspace{0.25em}
                        \hit The target is \charmed by you.
                        Any act by you or by creatures that appear to be your allies that threatens or harms the charmed person breaks the effect.
                        Harming the target is not limited to dealing it damage, but also includes causing it significant subjective discomfort.
                        An observant target may interpret overt threats to its allies as a threat to itself.

                        \rankline

                        \noindent The attack's \glossterm{accuracy} increases by \plus2 for each rank beyond 2.
                        \vspace{0.1em}
                    \end{magicalsustainability}
                \item Unholy Might: You gain two claw natural weapons and one bite natural weapon (see \pcref{Natural Weapons}).
                    In addition, you gain a \plus1 bonus to your \glossterm{mundane power}.
            \end{itemize}

            \magicalcf{Tif}[3]{Abysswalker} You can use your \ability{abyssal hop} ability to teleport as a \glossterm{movement} instead of as a standard action.
            When you do, you do not deal fire damage at your destination, and you \glossterm{briefly} cannot use that ability as a movement again.

            \magicalcf{Tif}[4]{Greater Infernal Ancestry} The benefits of your \textit{infernal ancestry} ability improve.
            \begin{itemize}
                \item Infernal Conduit: You gain a \plus1 bonus to your \glossterm{magical power}.
                    In addition, the area affected by your \textit{abyssal hop} ability increases to a \smallarea radius from your destination.
                \item Tempting Allure: The skill bonuses from your \textit{infernal ancestry} ability increase to \plus3.
                    In addition, you can use the \ability{dominating temptation} ability as a standard action.
                    \begin{magicalactiveability}{Dominating Temptation}{\abilitytag{Emotion}}
                        \rankline
                        \noindent
                        Make an attack vs. Mental against a humanoid creature within \shortrange.%
                        \vspace{0.25em}
                        \hit The target is \stunned as a \glossterm{condition}.
                        \crit The target is \confused instead of stunned.
                        In addition, if the target is humanoid and was already confused from a previous use of this ability, you may \glossterm{attune} to this ability.
                        When you do, the target becomes \dominated by you for the duration of that attunement.

                        \rankline
                        \noindent The attack's \glossterm{accuracy} increases by \plus1 for each rank beyond 4.
                        \vspace{0.1em}
                    \end{magicalactiveability}
                \item Unholy Might: You can use the \ability{unholy strength} ability as a standard action.
                    \begin{magicalattuneability}{Unholy Strength}{\abilitytag{Attune}}
                        \rankline
                        You gain a \plus1 \glossterm{enhancement bonus} to your Strength.
                    \end{magicalattuneability}
            \end{itemize}

            \cf{Tif}[5]{Greater Hellfire Tolerance} You become \trait{immune} to fire damage.

            \cf{Tif}[5]{Greater Infernal Resilience}  The bonus from your \ability{infernal resilience} ability increases to three times your rank in this archetype.

            \magicalcf{Tif}[6]{Supreme Infernal Ancestry} The benefits of your \textit{infernal ancestry} ability reach their peak.
            \begin{itemize}
                \item Infernal Conduit: The magical power bonus from your \textit{greater infernal ancestry} ability increases to \plus2.
                    In addition, the area affected by your \textit{abyssal hop} ability increases to a \medarea radius from your destination.
                \item Tempting Allure: The skill bonuses from your \textit{infernal ancestry} ability increase to \plus4.
                    In addition, your \textit{tempting domination} ability can dominate non-humanoid creatures.
                \item Unholy Might: The power bonus from your \textit{infernal ancestry} ability increases to \plus2.
                    In addition, your \textit{unholy surge} ability loses the \abilitytag{Attune} tag.
                    Instead, it lasts until you \glossterm{dismiss} it as a \glossterm{free action}.
            \end{itemize}

            \magicalcf{Tif}[7]{Greater Abyssal Hop} When you use your \ability{abyssal hop} ability, you no longer require \glossterm{line of sight} or \glossterm{line of effect} to your destination.
            In addition, when you use it to teleport as a standard action, the range increases to \distrange.

        % \subsubsection{Base Class Abilities}
        %     If you choose tiefling as your base class, you gain the following abilities.

        %     \cf{Tif}{Defenses}
        %     You gain the following bonuses to your \glossterm{defenses}: \plus3 Fortitude, \plus5 Reflex, \plus7 Mental.

        %     \cf{Tif}{Resources} You have the following \glossterm{resources}:
        %     \begin{itemize}
        %         \item Four \glossterm{attunement points}, which you can use to attune to items and abilities that affect you (see \pcref{Attunement Points}).
        %         \item A \plus2 bonus to your \glossterm{fatigue tolerance}, which makes it easier for you to use powerful abilities that fatigue you (see \pcref{Fatigue}).
        %         \item Two \glossterm{insight points}, which you can spend to gain additional abilities or proficiencies (see \pcref{Insight Points}).
        %         \item Five \glossterm{trained skills}, which you can spend to learn skills (see \pcref{Trained Skills}).
        %     \end{itemize}

        %     \cf{Tif}{Weapon Proficiencies} 
        %     You are proficient with simple weapons.

        %     \cf{Tif}{Armor Proficiencies} 
        %     You are proficient with light and medium armor.

        %     \cf{Tif}{Skills}
        %     You have the following \glossterm{class skills}:
        %     \begin{itemize}
        %         \item \subparhead{Strength} Climb.
        %         \item \subparhead{Dexterity} Balance, Sleight of Hand, Stealth.
        %         \item \subparhead{Intelligence} Craft, Disguise, Knowledge (arcana, planes)
        %         \item \subparhead{Perception} Awareness, Social Insight.
        %         \item \subparhead{Other} Deception, Intimidate, Perform, Persuasion.
        %     \end{itemize}

    \subsection{Vampire}
        A vampire is an undead creature that must drink the blood of living creatures to survive.
        Unlike most undead creatures, vampires appear to be alive and human, allowing them to act normally in society.
        Vampires have great power, but also many dangerous weaknesses.

        \parhead{Creature Type} Unlike most other playable species, vampires are undead instead of humanoids.
        \parhead{Size} Medium.
        \parhead{Attributes} \plus1 Strength and Dexterity, \minus1 Constitution.
        \parhead{Special Abilities}
        \begin{itemize}
            \itemhead{Climb Speed} Vampires have a \glossterm{climb speed} 10 feet slower than their \glossterm{base speed}.
            \itemhead{Darkvision} Vampires have \trait{darkvision} with a 90 foot range, allowing them to see in complete darkness (see \pcref{Darkvision}).
            \itemhead{Fangs} Vampires have a bite natural weapon (see \tref{Natural Weapons}).
                These fangs retract when not in use, so vampires cannot be identified as non-human by their fangs unless they choose to expose them.
            \itemhead{Undead} Vampires are \trait{undead} instead of \glossterm{living}, and they take damage from most healing effects (see \pcref{Undead})).
            \itemhead{Unnatural Charm} Vampires gain a \plus3 bonus to the Persuasion skill.
            \itemhead{Vampire Archetype} Vampires only gain two class archetypes instead of three.
                Instead, they treat the Vampire archetype as one of their archetypes, and they gain ranks in it just like they gain ranks in class archetypes.
        \end{itemize}

        \parhead{Special Weaknesses}
            Vampires have a number of specific weaknesses.
        \begin{itemize}
            \itemhead{Blood Dependence} For every 24 hours that a vampire remains awake without ingesting at least one pint of blood from living creatures, its maximum hit points are reduced by 20.
            If its maximum hit points are reduced to 0 in this way, it dies and withers away into a pile of ash.
            This penalty is removed as soon as the vampire drinks a pint of blood.
            A vampire can can enter a torpor to survive indefinitely without blood.
            While in a torpor, it is unconscious until it smells blood nearby.
            \itemhead{Garlic} Whenever a vampire smells or touches garlic, it takes 10 energy damage and becomes \frightened by any creatures bearing garlic as a condition.
            This damage is repeated at the during each subsequent \glossterm{action phase} that the vampire spends exposed to garlic.
            \itemhead{Holy Water} Whenever a vampire takes damage from holy water, it becomes \stunned as a condition.
            \itemhead{Running Water} Whenever a vampire touches or flies over running water, it takes 10 energy damage and \glossterm{briefly} becomes \immobilized.
            This applies as long as the vampire is within 100 feet of the running water, even the water is underground or under a bridge.
            It can use the \ability{struggle} ability to move despite being immobilized, but only towards the closest shore.
            This damage is repeated at the during each subsequent \glossterm{action phase} that the vampire spends touching or flying over running water.
            \itemhead{Silver} Vampires are \vulnerable to strikes using silvered weapons.
            \itemhead{True Sunlight} Whenever a vampire is exposed to true sunlight, it takes 20 energy damage and becomes \blinded as a condition.
            If it suffers a \glossterm{vital wound} from this damage, it immediately dies and dissolves into a pile of ash.
            This damage is repeated at the during each subsequent \glossterm{action phase} that the vampire spends in true sunlight.
            \itemhead{Unmirrored} Vampires have no reflection in mirrors, including their clothes and equipment.
            This can allow careful observers to identify vampires.
            \itemhead{Wooden Stakes} If a vampire loses hit points from a critical strike using a wooden stake, the stake becomes impaled in its heart.
            The vampire becomes \paralyzed until the stake is removed.
            A wooden stake is a light improvised weapon that deals 1d4 piercing damage.
        \end{itemize}

        \subsubsection{Vampire Archetype}

            \cf{Vmp}[1]{Blood Drain} Whenever a living creature loses hit points from a \glossterm{strike} using your bite natural weapon, you can increase your \glossterm{fatigue level} by one.
            When you do, you regain \glossterm{damage resistance} and \glossterm{hit points} at the end of the round.
            The recovery is equal to the hit points the target lost from the attack, ignoring any extra damage from critical hits.

            \cf{Vmp}[2]{Gentle Fangs} Whenever you deal damage using your bite natural weapon, you can choose not to reduce the target's hit points below 0, or you can treat the damage as \glossterm{subdual damage}.
            In addition, damage dealt using your bite natural weapon does not wake sleeping creatures unless you inflict a vital wound.

            \magicalcf{Vmp}[2]{Reviving Coffin} You can designate a coffin as your home by resting in it for 24 consecutive hours.
            When you take a \glossterm{long rest} in your home coffin, you recover two \glossterm{vital wounds} instead of one.
            In addition, you can cross running water without penalty while in your home coffin.

            When you die, if your corpse is placed in your home coffin, you can be resurrected after 24 hours.
            You can only be resurrected in this way if you were killed by a vital wound with a \glossterm{vital roll} of \minus3 or higher.
            If you were killed by a more severe vital wound, or by some other effect, even your home coffin cannot save you.

            \magicalcf{Vmp}[3]{Charming Gaze} You can use the \ability{charming gaze} ability as a standard action.
            \begin{magicalsustainability}{Charming Gaze}{\abilitytag{Emotion}, \abilitytag{Subtle}, \abilitytag{Sustain} (minor), \abilitytag{Visual}}
                \rankline
                Make an attack vs. Mental against all humanoid creatures in a \largearea cone from you.
                You take a \minus10 penalty to \glossterm{accuracy} with this attack against creatures who have made an attack or been attacked since the start of the last round.
                \hit The target is \charmed by you.
                Any act by you or by creatures that appear to be your allies that threatens or harms the charmed person breaks the effect.
                Harming the target is not limited to dealing it damage, but also includes causing it significant subjective discomfort.
                An observant target may interpret overt threats to its allies as a threat to itself.

                \rankline

                \noindent The attack's \glossterm{accuracy} increases by \plus2 for each rank beyond 3.
            \end{magicalsustainability}

            \magicalcf{Vmp}[4]{Creature of the Night} You can use the \ability{creature of the night} ability as a standard action.
            \begin{magicalattuneability}{Creature of the Night}{\abilitytag{Attune}}
                \rankline
                You \glossterm{shapeshift} into the form of a Tiny bat, a Medium cloud of mist, or your normal humanoid form.
                While in your bat form, you gain \trait{blindsense} (120 ft.) and a 40 foot fly speed with a 60 ft. height limit.
                While in your mist form, you become \trait{incorporeal} and gain a 20 foot fly speed with a 30 ft. height limit and perfect maneuverability.

                In either non-humanoid form, you are unable to take any standard action other than \glossterm{movement}.
                You cannot use this ability while \paralyzed.
            \end{magicalattuneability}

            \magicalcf{Vmp}[4]{Reviving Coffin+} You can designate up to three home coffins, rather than only one.
            This can allow you to travel with one coffin while keeping others safe for emergencies.

            \cf{Vmp}[5]{Unholy Perfection} Choose any two attributes other than Constitution.
            You gain a \plus1 bonus to those two attributes.

            \magicalcf{Vmp}[6]{Dominating Gaze} You can use the \ability{dominating gaze} ability as a standard action.
            \begin{magicalactiveability}{Dominating Gaze}{\abilitytag{Emotion}, \abilitytag{Visual}}
                \rankline
                Make an attack vs. Mental against a humanoid creature within \shortrange.
                \hit The target is \stunned as a \glossterm{condition}.
                While it has no remaining \glossterm{damage resistance}, it is \confused instead of stunned.
                \crit If the target is humanoid and was already confused from a previous use of this ability, you may \glossterm{attune} to this ability.
                When you do, the target becomes \dominated by you for the duration of that attunement.

                \rankline
                The attack's \glossterm{accuracy} increases by \plus2 for each rank beyond 6.
            \end{magicalactiveability}

            \magicalcf{Vmp}[7]{Blood Drain+} Using your \ability{blood drain} ability does not increase your fatigue level.

            \magicalcf{Vmp}[7]{Eternal Undeath} Your \ability{reviving coffin} ability can revive you from any cause of death or severity of vital wound.
            As long as some part of your corpse, even just a pinch of ash, is placed inside one of your home coffins, you will resurrect after 24 hours.
            Only the destruction of all of your home coffins or the total annihilation of your corpse can prevent your return.

        \subsubsection{Base Class Abilities}
            If you choose vampire as your base class, you gain the following abilities.

            \cf{Vmp}{Defenses}
            You gain the following bonuses to your \glossterm{defenses}: \plus3 Fortitude, \plus4 Reflex, \plus5 Mental.

            \cf{Vmp}{Hit Points}
                You have 8 hit points \add twice your Constitution, plus 2 hit points per level beyond 1.
                This increases as your level increases, as indicated below.
                \begin{itemize}
                    \itemhead{Level 7} 20 hit points \add three times your Constitution, plus 3 hit points per level beyond 7.
                    \itemhead{Level 13} 40 hit points \add six times your Constitution, plus 6 hit points per level beyond 13.
                    \itemhead{Level 19} 80 hit points \add twelve times your Constitution, plus 12 hit points per level beyond 19.
                \end{itemize}

            \cf{Vmp}{Resources} You have the following \glossterm{resources}:
            \begin{itemize}
                \item Three \glossterm{attunement points}, which you can use to attune to items and abilities that affect you (see \pcref{Attunement Points}).
                \item A \glossterm{fatigue tolerance} equal to 3 \add your Constitution.
                    Your fatigue tolerance makes it easier for you to use powerful abilities that fatigue you (see \pcref{Fatigue}).
                \item A number of \glossterm{insight points} equal to 1 \add your Intelligence.
                    You can spend insight points to gain additional abilities (see \pcref{Insight Points}).
                \item Five \glossterm{trained skills} from among your \glossterm{class skills}, plus additional trained skills equal to your Intelligence (see \pcref{Skills}).
            \end{itemize}

            \cf{Vmp}{Weapon Proficiencies} 
            You are proficient with simple weapons and one weapon group of your choice.

            \cf{Vmp}{Armor Proficiencies} 
            You are proficient with light armor.

            \cf{Vmp}{Skills}
            You have the following \glossterm{class skills}:
            \begin{itemize}
                \itemhead{Strength} Climb, Jump.
                \itemhead{Dexterity} Balance, Stealth.
                \itemhead{Intelligence} Deduction, Disguise, Knowledge (dungeoneering and religion)
                \itemhead{Perception} Awareness, Creature Handling, Deception, Persuasion, Social Insight.
                \itemhead{Other} Intimidate, Persuasion
            \end{itemize}

\section{Classes}
    \subsection{Bard}
        A bard is a rogue with the ability to perform magical feats through music.
        It is unclear whether bards actually draw power from music in the same way that druids draw power from nature, or whether they simply channel their innate magical talent through music.
        The bard class functions like the rogue class, with the following exceptions:
        \begin{itemize}
            \item A bard cannot choose the \textit{assassin} archetype. However, the \textit{arcane magic} sorcerer archetype is considered to be part of their class, and they may choose that archetype without spending insight points to multiclass.
            \item A bard casts spells without \glossterm{somatic components}.
            \item A bard can only cast spells while sustaining a performance with the Perform skill. This performance can be either a mundane performance or a \textit{bardic performance} ability.
        \end{itemize}

    \subsection{Blighter}
        Blighter practice a strange inversion of druidic traditions.
        While druids venerate nature in all its forms, blighters dedicate their lives to the destruction of nature for its own sake.
        They rip power directly from the death of natural beings, using it to fuel their own warped version of nature magic.
        The blighter class functions like the druid class, with the following exceptions:
        \begin{itemize}
            \item Whenever a blighter rests, they automatically destroy nature and kill anything living around them.
                Plants wither and die, insects fall dead in the air, and so on.
                A ten minute rest destroys life in a radius equal to five feet times the blighter's highest rank in the blighter class (minimum 5 feet total).
                In general, Diminuitive or larger creatures and Medium or larger plants suffer no ill effects, though creatures may feel subtle pains.
                An eight hour rest destroys life in ten times that radius, and kills life one size category larger.
                Resting beyond that point does not increase the radius or severity of the effect.
                This destruction spreads out gradually throughout the resting period, and even a partially completed rest destroys some natural life.
            \item A blighter cannot choose the \textit{wildspeaker} archetype.
                However, the \textit{domain influence} cleric archetype is considered to be part of their class, and they may choose that archetype without spending insight points to multiclass.
                A blighter can only choose the Death, Destruction, and Evil domains.
            \item A blighter cannot gain access to the \sphere{verdamancy} mystic sphere by any means.
        \end{itemize}

    \subsection{Faebonder}
        A faebonder is a warlock who made their pact with a fae creature instead of a demon.
        The faebonder class functions like the warlock class, with the following exceptions:
        \begin{itemize}
            \item The magic source for the faebonder class is nature magic instead of pact magic.
                This changes the \glossterm{mystic spheres} a faebonder has access to and all other effects based on their source of magic.
                However, they still require both \glossterm{verbal components} and \glossterm{somatic components} to cast spells from the faebonder class (see \pcref{Casting Components}).
            \item A faebonder cannot choose the \textit{blessings of the abyss} archetype. However, the \textit{elementalist} druid archetype is considered to be part of their class, and they may choose that archetype without spending insight points to multiclass.
            \item Faebonders add Knowledge (nature) to their class skill list and remove Knowledge (planes).
        \end{itemize}

    \subsection{Favored Soul}
        A favored soul is a warlock who made their pact with a deity instead of a demon.
        This is an unusual arrangement, as deities would normally influence their clerics to achieve their aims.
        However, in special circumstances, a deity may want to empower a non-worshipper to influence mortal affairs.
        The favored soul class functions like the warlock class, with the following exceptions:
        \begin{itemize}
            \item The magic source for the favored soul class is divine magic instead of pact magic.
                This changes the \glossterm{mystic spheres} a favored soul has access to and all other effects based on their source of magic.
                However, they still require both \glossterm{verbal components} and \glossterm{somatic components} to cast spells from the favored soul class (see \pcref{Casting Components}).
            \item A favored soul cannot choose the \textit{blessings of the abyss} archetype. However, the \textit{domain influence} cleric archetype is considered to be part of their class, and they may choose that archetype without spending insight points to multiclass.
            \item Favored souls add Knowledge (religion) to their class skill list and remove Knowledge (planes).
        \end{itemize}

    \subsection{Shaman}
        A shaman, like a cleric, is a divine worshipper.
        However, while clerics worship powerful, well-established deities, shamans worship more primitive deities of lesser power.
        As a result, their divine powers are more limited and take different forms.
        Shamans are common among less civilized humanoid societies like bugbears.
        The shaman class functions like the cleric class, with the following exceptions:
        \begin{itemize}
            \item The magic source for the shaman class is nature magic instead of divine magic.
                This changes the \glossterm{mystic spheres} a shaman has access to and all other effects based on their source of magic.
            \item A shaman cannot choose the \textit{divine spell mastery} archetype. However, the \textit{elementalist} druid archetype is considered to be part of their class, and they may choose that archetype without spending insight points to multiclass.
            \item A shaman cannot gain access to more than two \textit{mystic spheres} from the magic source granted by the shaman class by any means.
            \item Shamans add Knowledge (nature) to their class skill list and remove Knowledge (planes).
        \end{itemize}

\section{Alternate Play Styles}

    \subsection{Being Surrounded}\label{Being Surrounded}
        Normally, exact positioning doesn't matter that much in combat.
        This makes it easier to play without a grid, or to just spend less time worrying about the details of everyone's positions on a grid.
        With this optional rule, you can make positioning more important in combat, increasing tactical depth for melee characters.
        This generally has the downside of making movement more complicated, however, as combatants try to surround others and avoid being surrounded themselves.

        If you play with this alternate rule, when you are being attacked by multiple foes at once, you are less able to defend yourself.
        If every space adjacent to you either contains an \glossterm{enemy} or is adjacent to an \glossterm{enemy}, you are surrounded.
        A creature that is surrounded takes a \minus2 penalty to its Armor and Reflex defenses.
        When determining whether you are surrounded, ignore any enemies that are sharing space with you, and ignore any enemies that are at least two size categories smaller than you.

        Any effect that makes a creature immune to being \partiallyunaware, such as the \spell{foresight} spell, also makes that creature immune to being surrounded.

    \subsection{Critical Failure}
        Normally, there is no explicit penalty for catastrophic failure built into the rules.
        Even if you fail at a check by a large amount, it doesn't leave you worse off than when you started.
        Sometimes, it may be narratively appropriate to punish significant failure more severely, at the GM's discretion.
        For example, attempting a difficult Persuasion check and completely botching the execution might leave the target feeling more hostile than if no Persuasion had been attempted at all.
        A good threshold for critical failure would generally be failing a check by 8 or more.
        For specific tasks, it may make more sense to have punishments for failure at lower thresholds as well.

        This is considered an optional rule because it generally makes trying silly ideas or extremely difficult tasks more dangerous, which isn't appropriate for every game.
        It also depends heavily on GM discretion.

    \subsection{Easy Magic Item Reforging}
        The Craft Specialization feat allows characters to transfer magic item properties between different items.
        For example, if the players find a magic meteor hammer that none of them could use, they could reforge that item as a magic battleaxe so they could use its property.
        With this optional rule, skilled item crafters capable of this action are assumed to be common in major cities or towns.
        The typical price to reforge an item in this way is two ranks lower than the item's rank, to a minimum rank of 1.

        The advantage of using this optional rule is that it makes magic items more likely to be useful to the party.
        Without this rule, you may be forced to have the party ``randomly'' only find magic items that they are coincidentally proficient with, or the party may frequently find magic items that they can't use.
        On the other hand, this rule assumes a more magical and highly developed civilization.
        It also may require the party to frequently return to town to reforge useless items into items that are useful for them.
        Either of those requirements may not match the intended tone of your campaign.

    \subsection{Expanded Insight Points}
        Normally, \glossterm{insight points} can only be used to learn new special abilities from your class, or from a small number of feats.
        This alternate rule allows you to spend insight points to gain a wide variety of other proficiencies and benefits.
        This makes character creation more complicated, but it also allows you to personalize your character much more precisely.

        If you play with this alternate rule, you can spend insight points in any of the following ways.
        \begin{raggeditemize}
            \item You can spend an \glossterm{insight point} to gain an additional \glossterm{trained skill}.
            \item You can spend an \glossterm{insight point} to gain proficiency in an additional \glossterm{usage class} of armor (light, medium, or heavy).
                You must be proficient with light armor to become proficient with medium armor, and you must be proficient with medium armor to become proficient with heavy armor.
            \item You can spend an \glossterm{insight point} to gain proficiency in an additional \glossterm{weapon group}.
            \item You can spend two \glossterm{insight points} to gain proficiency with \glossterm{exotic weapons} from a single \glossterm{weapon group} you are already proficient with.
            \item You can spend an \glossterm{insight point} to learn two \glossterm{common languages} or one \glossterm{rare language} (see \pcref{Communication and Languages}).
        \end{raggeditemize}

    \subsection{Longer Rests}\label{Longer Rests}
        Normally, characters can take a short rest in ten minutes and a long rest in eight hours.
        With this optional rule, a short rest instead requires eight hours of rest, and long rest requires a week.

        This dramatically slows down the narrative pacing of the world, and makes the world feel much more brutal and unforgiving.
        Characters will often be forced to start a combat while missing damage resistance or even hit points, and taking vital wounds can be crippling.

    \subsection{Obscure Magic Items}\label{Obscure Magic Items}
        The base rules of Rise make it fairly easy to identify magic items.
        This keeps the pace of the game up when players find magic items frequently.
        However, you may choose to treat magic items as being more rare and mysterious.
        If you do, make the following changes:
        \begin{itemize}
            \item The \ability{identify item} ability from the Craft and Knowledge skills provides no information about how to use a magic item's properties or what they might be.
                It can still be used to identify whether or not an item is magical.
            \item The Knowledge (items) Knowledge skill is removed entirely.
            \item Magic items are more rare, and therefore more valuable.
                Calculate the prices for all magic items as if they were one rank higher than they actually are.
                Rank 7 magic items cannot be bought for any price - they are simply too rare.
            \item All spells with the \abilitytag{Attune} tag require an additional \glossterm{attunement point} to attune to.
                If magic items are hard to find and use, spellcasters gain a powerful benefit, since their personal attunement spells are still reliably available.
                This change ensures that spellcasters still gain a benefit from their personal access to magic, but they are not drastically more powerful than characters who depend on finding useful magic items.
        \end{itemize}

        You may also want to add complex or unintuitive activation conditions to magic items.
        For example, \mitem{boots of speed} may only function while hopping on one foot, or while you are not wearing socks.
        This can encourage players to experiment more with magic items to figure out how to use them.

    \subsection{Rage Accuracy}
        Normally, a barbarian's \ability{rage} ability provides a +2 accuracy bonus.
        With this variant, raging barbarians instead gain no accuracy bonus, but roll 1d12 instead of 1d10 for their attack rolls while raging.
        The attack's \glossterm{explosion target} is reduced by 3.
        A barbarian's overall accuracy and damage output with this rule essentially equivalent to the normal rule, but they are more likely to get critical hits or completely whiff on important attacks.

        This variant can be more fun for people who like big hits and big misses, and for RPG veterans who naturally associate a barbarian with a d12.
        It is considered a variant rule because not everyone owns a 12-sided die, and you shouldn't need to buy one just to play a barbarian.

    \subsection{Restricted Archetype Order}
        Normally, when a character in Rise levels up, they can freely choose which of their class archetypes they want to rank up (as long as they don't exceed their maximum rank).
        However, this means that most levels require making a choice that may be confusing for newer players.
        The process of leveling up can be simplified if each player chooses an order for their archetypes.

        With this variant, each character has a primary archetype, a secondary archetype, and a tertiary archetype.
        This choice is made at character creation.
        Whenever they increase their maximum rank, they increase their rank in their primary archetype.
        In their next level up, they increase their rank in their secondary archetype, and then finally their tertiary archetype.

    \subsection{Sleeping While Encumbered}
        Normally, characters can sleep in their armor without any penalty.
        This is unrealistic, but it can be time-consuming to make everyone track how their sleeping statistics differ from their waking statistics.
        Being ambushed while sleeping is very rare in most games, so it's generally not worth the hassle.
        However, if you want a more realistic game with more punishing night ambushes, you can use this alternate rule.

        If you play with this alternate rule, resting in armor is difficult.
        If you take a \glossterm{long rest} while you have \glossterm{encumbrance}, you finish your rest with a \glossterm{fatigue level} equal to the value of your encumbrance.
        In addition, only half the time you spend sleeping while you have encumbrance counts as sleep for the purpose of determining your fatigue (see \pcref{Sleep and Fatigue}).

    \subsection{Tap Out}
        With this optional rule, whenever you gain a vital wound, you can ``tap out'' to guarantee that you survive while taking yourself out of the fight.
        If you tap out, you treat the result of the vital roll for that vital wound as a 10, regardless of any bonuses or penalties you would normally have to the vital roll.
        However, you fall unconscious immediately, and you cannot regain consciousness by any means until you finish a \glossterm{short rest}.

        This optional rule significantly reduces the likelihood of character death, and makes fights less likely to impose long-term consequences on characters.
        However, it also makes vital wounds more likely to entirely knock characters out of a fight, which can increase the risk that the entire party is defeated.


\chapter{Monsters}
Monsters are all of the various non-humanoid creatures that exist in the world of Rise.
Many of them are dangerous, and adventurers may need to fight them.
This chapter describes the rules for monsters, and the combat statistics for a variety of monsters.
\section{Aberrations}
\begin{monsection}{Aboleth}{12}[4]
\vspace{-1em}\spelltwocol{}{Huge aberration}\vspace{-1em}
\begin{spellcontent}
\begin{spelltargetinginfo}
\spelltwocol{
\textbf{HP} 416;
\textbf{Bloodied} 208;
\textbf{Recovery} 6d10
}
{\textbf{AP} 8}
\pari \textbf{Armor} 18;
\textbf{Fort} 19;
\textbf{Ref} 12;
\textbf{Ment} 23
\pari \textbf{Strike} Tentacle \plus12 (5d10)
\pari \textbf{Actions} Two in action phase, two in delayed action phase
\end{spelltargetinginfo}
\begin{spelleffects}
\pari
\spelltwocol{\textbf{Confusion}: \plus14 vs Mental}{One creature in Medium range}
\par
\par \textit{Hit}: The target is confused as a condition.
\vspace{0.5em}
\pari
\spelltwocol{\textbf{Enslave}: \plus14 vs Mental}{One creature in Medium range}
\par
This ability costs an action point to use.
\par \textit{Hit}: The target is stunned as a condition.
\par \textit{Critical}:
The target is dominated by the aboleth.
This effect lasts as long as the aboleth attunes to it.
\vspace{0.5em}
\pari
\spelltwocol{\textbf{Mind Crush}: \plus14 vs Mental}{One creature in Long range}
\par
\par \textit{Hit}: 6d10 psionic damage
\vspace{0.5em}
\pari
\spelltwocol{\textbf{Psionic Blast}: \plus14 vs Mental}{Enemies in Large cone}
\par
\par \textit{Hit}: 4d10 psionic damage
\end{spelleffects}
\end{spellcontent}
\begin{spellsubcontent}
\begin{spellfooter}
\pari \textbf{Awareness} \plus6
\pari \textbf{Speed} 50 ft.;
\textbf{Space} 15 ft.;
\textbf{Reach} 15 ft.
\pari \textbf{Attributes}
Str 13,
Dex 0,
Con 14,
Int 13,
Per 7,
Wil 14
\end{spellfooter}
\end{spellsubcontent}
\end{monsection}
\section{Animates}
\begin{monsection}[Ram]{Animus}{6}[4]
\vspace{-1em}\spelltwocol{}{Huge animate}\vspace{-1em}
\begin{spellcontent}
\begin{spelltargetinginfo}
\spelltwocol{
\textbf{HP} 196;
\textbf{Bloodied} 98;
\textbf{Recovery} 4d8
}
{\textbf{AP} 5}
\pari \textbf{Armor} 12;
\textbf{Fort} 16;
\textbf{Ref} 7;
\textbf{Ment} 9
\pari \textbf{Strike} Slam \plus8 (4d8) or hoof \plus8 (4d6)
\pari \textbf{Actions} Two in action phase, two in delayed action phase
\end{spelltargetinginfo}
\begin{spelleffects}
\pari
\textbf{Forceful Smash}:
The ram makes a slam strike.
Treat the attack result as a shove attack against the target in addition to the strike.
The ram does not have to move with the target to push it back.
\end{spelleffects}
\end{spellcontent}
\begin{spellsubcontent}
\begin{spellfooter}
\pari \textbf{Awareness} \plus6
\pari \textbf{Speed} 50 ft.;
\textbf{Space} 15 ft.;
\textbf{Reach} 15 ft.
\pari \textbf{Attributes}
Str 8,
Dex 0,
Con 7,
Int 0,
Per 8,
Wil 0
\end{spellfooter}
\end{spellsubcontent}
\end{monsection}
\section{Animals}
\begin{monsection}[Black]{Bear}{2}[2]
\vspace{-1em}\spelltwocol{}{Medium animal}\vspace{-1em}
\begin{spellcontent}
\begin{spelltargetinginfo}
\spelltwocol{
\textbf{HP} 42;
\textbf{Bloodied} 21;
\textbf{Recovery} 2d6
}
{\textbf{AP} 3}
\pari \textbf{Armor} 6;
\textbf{Fort} 12;
\textbf{Ref} 6;
\textbf{Ment} 5
\pari \textbf{Strike} Bite \plus2 (2d6) or claw \plus2 (1d10)
\pari \textbf{Immune} staggered
\pari \textbf{Actions} One in action phase, one in delayed action phase
\end{spelltargetinginfo}
\begin{spelleffects}
\pari
\textbf{Rend}:
The bear makes a claw strike against two targets within reach.
\end{spelleffects}
\end{spellcontent}
\begin{spellsubcontent}
\begin{spellfooter}
\pari \textbf{Awareness} \plus6
\pari \textbf{Speed} 30 ft.;
\textbf{Space} 5 ft.;
\textbf{Reach} 5 ft.
\pari \textbf{Attributes}
Str 4,
Dex 2,
Con 3,
Int \minus7,
Per 2,
Wil 0
\end{spellfooter}
\end{spellsubcontent}
\end{monsection}
\begin{monsection}[Brown]{Bear}{4}[2]
\vspace{-1em}\spelltwocol{}{Large animal}\vspace{-1em}
\begin{spellcontent}
\begin{spelltargetinginfo}
\spelltwocol{
\textbf{HP} 70;
\textbf{Bloodied} 35;
\textbf{Recovery} 2d8
}
{\textbf{AP} 3}
\pari \textbf{Armor} 8;
\textbf{Fort} 14;
\textbf{Ref} 6;
\textbf{Ment} 7
\pari \textbf{Strike} Bite \plus4 (2d10) or claw \plus4 (2d8)
\pari \textbf{Immune} staggered
\pari \textbf{Actions} One in action phase, one in delayed action phase
\end{spelltargetinginfo}
\begin{spelleffects}
\pari
\textbf{Rend}:
The bear makes a claw strike against two targets within reach.
\end{spelleffects}
\end{spellcontent}
\begin{spellsubcontent}
\begin{spellfooter}
\pari \textbf{Awareness} \plus6
\pari \textbf{Speed} 40 ft.;
\textbf{Space} 10 ft.;
\textbf{Reach} 10 ft.
\pari \textbf{Attributes}
Str 6,
Dex 3,
Con 5,
Int \minus7,
Per 3,
Wil 0
\end{spellfooter}
\end{spellsubcontent}
\end{monsection}
\begin{monsection}{Dire Wolf}{5}[2]
\vspace{-1em}\spelltwocol{}{Large animal}\vspace{-1em}
\begin{spellcontent}
\begin{spelltargetinginfo}
\spelltwocol{
\textbf{HP} 72;
\textbf{Bloodied} 36;
\textbf{Recovery} 2d8
}
{\textbf{AP} 3}
\pari \textbf{Armor} 11;
\textbf{Fort} 13;
\textbf{Ref} 12;
\textbf{Ment} 8
\pari \textbf{Strike} Bite \plus6 (2d10)
\pari \textbf{Actions} One in action phase, one in delayed action phase
\end{spelltargetinginfo}
\begin{spelleffects}
\pari
\textbf{Pounce}:
The dire wolf moves up to its movement speed.
If it uses this ability during the action phase, it can make a bite strike during the delayed action phase.
\end{spelleffects}
\end{spellcontent}
\begin{spellsubcontent}
\begin{spellfooter}
\pari \textbf{Awareness} \plus6
\pari \textbf{Speed} 40 ft.;
\textbf{Space} 10 ft.;
\textbf{Reach} 10 ft.
\pari \textbf{Attributes}
Str 7,
Dex 7,
Con 3,
Int \minus6,
Per 6,
Wil 0
\end{spellfooter}
\end{spellsubcontent}
\end{monsection}
\begin{monsection}{Ferret}{1}[1]
\vspace{-1em}\spelltwocol{}{Tiny animal}\vspace{-1em}
\begin{spellcontent}
\begin{spelltargetinginfo}
\spelltwocol{
\textbf{HP} 2;
\textbf{Bloodied} 1;
\textbf{Recovery} 1d8
}
{\textbf{AP} 1}
\pari \textbf{Armor} 3;
\textbf{Fort} 2;
\textbf{Ref} 10;
\textbf{Ment} 2
\pari \textbf{Strike} Bite \plus1 (1d4)
\end{spelltargetinginfo}
\end{spellcontent}
\begin{spellsubcontent}
\begin{spellfooter}
\pari \textbf{Awareness} \plus6
\pari \textbf{Speed} 20 ft.;
\textbf{Space} 2.5 ft.;
\textbf{Reach} 2.5 ft.
\pari \textbf{Attributes}
Str \minus6,
Dex 1,
Con \minus4,
Int \minus7,
Per 1,
Wil \minus2
\end{spellfooter}
\end{spellsubcontent}
\end{monsection}
\begin{monsection}{Raven}{1}[1]
\vspace{-1em}\spelltwocol{}{Tiny animal}\vspace{-1em}
\begin{spellcontent}
\begin{spelltargetinginfo}
\spelltwocol{
\textbf{HP} 2;
\textbf{Bloodied} 1;
\textbf{Recovery} 1d8
}
{\textbf{AP} 2}
\pari \textbf{Armor} 5;
\textbf{Fort} 2;
\textbf{Ref} 14;
\textbf{Ment} 4
\pari \textbf{Strike} Talon \plus3 (1d3)
\end{spelltargetinginfo}
\end{spellcontent}
\begin{spellsubcontent}
\begin{spellfooter}
\pari \textbf{Awareness} \plus6
\pari \textbf{Speed} 20 ft.;
\textbf{Space} 2.5 ft.;
\textbf{Reach} 2.5 ft.
\pari \textbf{Attributes}
Str \minus9,
Dex 3,
Con \minus4,
Int \minus6,
Per 2,
Wil 0
\end{spellfooter}
\end{spellsubcontent}
\end{monsection}
\begin{monsection}{Roc}{9}[4]
\vspace{-1em}\spelltwocol{}{Gargantuan animal}\vspace{-1em}
\begin{spellcontent}
\begin{spelltargetinginfo}
\spelltwocol{
\textbf{HP} 240;
\textbf{Bloodied} 120;
\textbf{Recovery} 4d10
}
{\textbf{AP} 5}
\pari \textbf{Armor} 16;
\textbf{Fort} 19;
\textbf{Ref} 9;
\textbf{Ment} 12
\pari \textbf{Strike} Talon \plus10 (5d10)
\pari \textbf{Actions} Two in action phase, two in delayed action phase
\end{spelltargetinginfo}
\begin{spelleffects}
\pari
\textbf{Flyby Attack}:
The roc flies up to its flying movement speed.
It can make a talon strike or grapple attack at any point during this movement.
\end{spelleffects}
\end{spellcontent}
\begin{spellsubcontent}
\begin{spellfooter}
\pari \textbf{Awareness} \plus6
\pari \textbf{Speed} 80 ft. fly ft.;
\textbf{Space} 20 ft.;
\textbf{Reach} 20 ft.
\pari \textbf{Attributes}
Str 12,
Dex 10,
Con 5,
Int \minus7,
Per 5,
Wil 0
\end{spellfooter}
\end{spellsubcontent}
\end{monsection}
\begin{monsection}{Wolf}{1}[1]
\vspace{-1em}\spelltwocol{}{Large animal}\vspace{-1em}
\begin{spellcontent}
\begin{spelltargetinginfo}
\spelltwocol{
\textbf{HP} 12;
\textbf{Bloodied} 6;
\textbf{Recovery} 1d8
}
{\textbf{AP} 2}
\pari \textbf{Armor} 7;
\textbf{Fort} 7;
\textbf{Ref} 8;
\textbf{Ment} 4
\pari \textbf{Strike} Bite \plus2 (1d10)
\end{spelltargetinginfo}
\end{spellcontent}
\begin{spellsubcontent}
\begin{spellfooter}
\pari \textbf{Awareness} \plus6
\pari \textbf{Speed} 40 ft.;
\textbf{Space} 10 ft.;
\textbf{Reach} 10 ft.
\pari \textbf{Attributes}
Str 1,
Dex 3,
Con 1,
Int \minus6,
Per 2,
Wil 0
\end{spellfooter}
\end{spellsubcontent}
\end{monsection}
\section{Humanoids}
\begin{monsection}{Cultist}{1}[1]
\vspace{-1em}\spelltwocol{}{Medium humanoid}\vspace{-1em}
\begin{spellcontent}
\begin{spelltargetinginfo}
\spelltwocol{
\textbf{HP} 10;
\textbf{Bloodied} 5;
\textbf{Recovery} 1d8
}
{\textbf{AP} 5}
\pari \textbf{Armor} 5;
\textbf{Fort} 4;
\textbf{Ref} 5;
\textbf{Ment} 11
\pari \textbf{Strike} Club \plus1 (1d6)
\end{spelltargetinginfo}
\begin{spelleffects}
\pari
\spelltwocol{\textbf{Hex}: \plus3 vs Fortitude}{One target in Medium range}
\par
\par \textit{Hit}: 1d10 life damage, and the target is sickened as a condition.
\end{spelleffects}
\end{spellcontent}
\begin{spellsubcontent}
\begin{spellfooter}
\pari \textbf{Awareness} \plus6
\pari \textbf{Speed} 30 ft.;
\textbf{Space} 5 ft.;
\textbf{Reach} 5 ft.
\pari \textbf{Attributes}
Str 0,
Dex 0,
Con 0,
Int \minus1,
Per \minus1,
Wil 3
\end{spellfooter}
\end{spellsubcontent}
\end{monsection}
\begin{monsection}{Goblin Shouter}{2}[2]
\vspace{-1em}\spelltwocol{}{Small humanoid}\vspace{-1em}
\begin{spellcontent}
\begin{spelltargetinginfo}
\spelltwocol{
\textbf{HP} 24;
\textbf{Bloodied} 12;
\textbf{Recovery} 2d6
}
{\textbf{AP} 4}
\pari \textbf{Armor} 6;
\textbf{Fort} 5;
\textbf{Ref} 11;
\textbf{Ment} 7
\pari \textbf{Strike} Club \plus3 (1d6) or sling \plus3 (1d6)
\pari \textbf{Actions} One in action phase, one in delayed action phase
\end{spelltargetinginfo}
\begin{spelleffects}
\pari
\textbf{Shout of Running}:
All other willing allies who can hear the shouter can use the sprint ability without spending action points.
This effect lasts as long as the shouter sustains it as a standard action.
\vspace{0.5em}
\pari
\textbf{Shout of Stabbing}:
All other willing allies who can hear the shouter gain a \plus1d bonus to strike damage.
This effect lasts as long as the shouter sustains it as a standard action.
\end{spelleffects}
\end{spellcontent}
\begin{spellsubcontent}
\begin{spellfooter}
\pari \textbf{Awareness} \plus6
\pari \textbf{Speed} 25 ft.;
\textbf{Space} 5 ft.;
\textbf{Reach} 5 ft.
\pari \textbf{Attributes}
Str 0,
Dex 3,
Con \minus1,
Int \minus2,
Per 3,
Wil 2
\end{spellfooter}
\end{spellsubcontent}
\end{monsection}
\begin{monsection}{Goblin Stabber}{1}[1]
\vspace{-1em}\spelltwocol{}{Small humanoid}\vspace{-1em}
\begin{spellcontent}
\begin{spelltargetinginfo}
\spelltwocol{
\textbf{HP} 8;
\textbf{Bloodied} 4;
\textbf{Recovery} 1d8
}
{\textbf{AP} 2}
\pari \textbf{Armor} 6;
\textbf{Fort} 4;
\textbf{Ref} 12;
\textbf{Ment} 5
\pari \textbf{Strike} Shortsword \plus3 (1d4) or sling \plus2 (1d4)
\end{spelltargetinginfo}
\begin{spelleffects}
\pari
\textbf{Sneeky Stab}:
The stabber makes a shortsword strike.
If the target is defenseless, overwhelmed, or unaware, the damage becomes 1d8.
\end{spelleffects}
\end{spellcontent}
\begin{spellsubcontent}
\begin{spellfooter}
\pari \textbf{Awareness} \plus6
\pari \textbf{Speed} 25 ft.;
\textbf{Space} 5 ft.;
\textbf{Reach} 5 ft.
\pari \textbf{Attributes}
Str 0,
Dex 3,
Con \minus1,
Int \minus2,
Per 2,
Wil 0
\end{spellfooter}
\end{spellsubcontent}
\end{monsection}
\begin{monsection}{Orc Chieftain}{5}[3]
\vspace{-1em}\spelltwocol{}{Medium humanoid}\vspace{-1em}
\begin{spellcontent}
\begin{spelltargetinginfo}
\spelltwocol{
\textbf{HP} 108;
\textbf{Bloodied} 54;
\textbf{Recovery} 2d10
}
{\textbf{AP} 6}
\pari \textbf{Armor} 11;
\textbf{Fort} 13;
\textbf{Ref} 10;
\textbf{Ment} 12
\pari \textbf{Strike} Greataxe \plus6 (4d6)
\pari \textbf{Actions} Two in action phase, one in delayed action phase
\end{spelltargetinginfo}
\begin{spelleffects}
\pari
\textbf{Hit Everyone Else}:
All other willing allies who can hear the chieftain gain a \plus2 bonus to accuracy with strikes.
This effect lasts as long as the chieftain sustains it as a standard action.
\vspace{0.5em}
\pari
\textbf{Hit Hardest}:
The chieftain makes a greataxe strike.
The strike deals 4d10 damage.
\vspace{0.5em}
\pari
\textbf{Hit Fast}:
The chieftain makes a greataxe strike.
Its accuracy is increased to 8.
\end{spelleffects}
\end{spellcontent}
\begin{spellsubcontent}
\begin{spellfooter}
\pari \textbf{Awareness} \plus6
\pari \textbf{Speed} 30 ft.;
\textbf{Space} 5 ft.;
\textbf{Reach} 5 ft.
\pari \textbf{Attributes}
Str 8,
Dex 0,
Con 3,
Int 0,
Per 6,
Wil 6
\end{spellfooter}
\end{spellsubcontent}
\end{monsection}
\begin{monsection}{Orc Grunt}{2}[1]
\vspace{-1em}\spelltwocol{}{Medium humanoid}\vspace{-1em}
\begin{spellcontent}
\begin{spelltargetinginfo}
\spelltwocol{
\textbf{HP} 18;
\textbf{Bloodied} 9;
\textbf{Recovery} 1d10
}
{\textbf{AP} 2}
\pari \textbf{Armor} 6;
\textbf{Fort} 9;
\textbf{Ref} 6;
\textbf{Ment} 6
\pari \textbf{Strike} Greataxe \plus2 (2d8)
\end{spelltargetinginfo}
\begin{spelleffects}
\pari
\textbf{Hit Harder}:
The grunt makes a greataxe strike.
Its accuracy is reduced to 0, but the strike deals 4d6 damage.
\end{spelleffects}
\end{spellcontent}
\begin{spellsubcontent}
\begin{spellfooter}
\pari \textbf{Awareness} \plus6
\pari \textbf{Speed} 30 ft.;
\textbf{Space} 5 ft.;
\textbf{Reach} 5 ft.
\pari \textbf{Attributes}
Str 4,
Dex 0,
Con 2,
Int \minus1,
Per 0,
Wil 0
\end{spellfooter}
\end{spellsubcontent}
\end{monsection}
\begin{monsection}{Orc Loudmouth}{3}[2]
\vspace{-1em}\spelltwocol{}{Medium humanoid}\vspace{-1em}
\begin{spellcontent}
\begin{spelltargetinginfo}
\spelltwocol{
\textbf{HP} 48;
\textbf{Bloodied} 24;
\textbf{Recovery} 2d6
}
{\textbf{AP} 5}
\pari \textbf{Armor} 7;
\textbf{Fort} 10;
\textbf{Ref} 7;
\textbf{Ment} 10
\pari \textbf{Strike} Greataxe \plus3 (2d8)
\pari \textbf{Actions} One in action phase, one in delayed action phase
\end{spelltargetinginfo}
\begin{spelleffects}
\pari
\textbf{Hit Harder}:
The loudmouth makes a greataxe strike.
Its accuracy is reduced to 1, but the strike deals 4d6 damage.
\vspace{0.5em}
\pari
\textbf{Hit That One Over There}:
All other willing allies who can hear the loudmouth gain a \plus2 bonus to accuracy with strikes against one creature within Long range.
This effect lasts as long as the loudmouth sustains it as a standard action.
\end{spelleffects}
\end{spellcontent}
\begin{spellsubcontent}
\begin{spellfooter}
\pari \textbf{Awareness} \plus6
\pari \textbf{Speed} 30 ft.;
\textbf{Space} 5 ft.;
\textbf{Reach} 5 ft.
\pari \textbf{Attributes}
Str 5,
Dex 0,
Con 2,
Int \minus1,
Per 0,
Wil 4
\end{spellfooter}
\end{spellsubcontent}
\end{monsection}
\begin{monsection}{Orc Shaman}{3}[2]
\vspace{-1em}\spelltwocol{}{Medium humanoid}\vspace{-1em}
\begin{spellcontent}
\begin{spelltargetinginfo}
\spelltwocol{
\textbf{HP} 48;
\textbf{Bloodied} 24;
\textbf{Recovery} 2d6
}
{\textbf{AP} 5}
\pari \textbf{Armor} 7;
\textbf{Fort} 8;
\textbf{Ref} 7;
\textbf{Ment} 11
\pari \textbf{Strike} Greatstaff \plus3 (2d6)
\pari \textbf{Actions} One in action phase, one in delayed action phase
\end{spelltargetinginfo}
\begin{spelleffects}
\pari
\spelltwocol{\textbf{Hit Worse}: \plus4 vs Mental}{One target in Close range}
\par
\par \textit{Hit}: As a condition, the target takes a \minus3 penalty to accuracy with strikes.
\par \textit{Critical}: As above, except that the penalty is increased to -6.
\vspace{0.5em}
\pari
\textbf{Hurt Less}:
One other willing creature in Close range heals 2d8 hit points
\end{spelleffects}
\end{spellcontent}
\begin{spellsubcontent}
\begin{spellfooter}
\pari \textbf{Awareness} \plus6
\pari \textbf{Speed} 30 ft.;
\textbf{Space} 5 ft.;
\textbf{Reach} 5 ft.
\pari \textbf{Attributes}
Str 4,
Dex 0,
Con 2,
Int \minus1,
Per 0,
Wil 4
\end{spellfooter}
\end{spellsubcontent}
\end{monsection}
\begin{monsection}{Orc Savage}{4}[1]
\vspace{-1em}\spelltwocol{}{Medium humanoid}\vspace{-1em}
\begin{spellcontent}
\begin{spelltargetinginfo}
\spelltwocol{
\textbf{HP} 30;
\textbf{Bloodied} 15;
\textbf{Recovery} 2d6
}
{\textbf{AP} 2}
\pari \textbf{Armor} 11;
\textbf{Fort} 12;
\textbf{Ref} 11;
\textbf{Ment} 8
\pari \textbf{Strike} Greataxe \plus4 (2d10)
\end{spelltargetinginfo}
\begin{spelleffects}
\pari
\textbf{Hit Fast}:
The savage makes a greataxe strike.
Its accuracy is 6.
\end{spelleffects}
\end{spellcontent}
\begin{spellsubcontent}
\begin{spellfooter}
\pari \textbf{Awareness} \plus6
\pari \textbf{Speed} 30 ft.;
\textbf{Space} 5 ft.;
\textbf{Reach} 5 ft.
\pari \textbf{Attributes}
Str 7,
Dex 5,
Con 3,
Int \minus1,
Per 0,
Wil 0
\end{spellfooter}
\end{spellsubcontent}
\end{monsection}
\section{Magical Beasts}
\begin{monsection}{Ankheg}{6}[2]
\vspace{-1em}\spelltwocol{}{Large magical beast}\vspace{-1em}
\begin{spellcontent}
\begin{spelltargetinginfo}
\spelltwocol{
\textbf{HP} 84;
\textbf{Bloodied} 42;
\textbf{Recovery} 2d10
}
{\textbf{AP} 3}
\pari \textbf{Armor} 13;
\textbf{Fort} 14;
\textbf{Ref} 11;
\textbf{Ment} 9
\pari \textbf{Strike} Bite \plus6 (4d6)
\pari \textbf{Actions} One in action phase, one in delayed action phase
\end{spelltargetinginfo}
\begin{spelleffects}
\pari
\textbf{Drag Prey}:
The ankheg makes a shove attack with an accuracy of \plus13.
It can move with the target up to a maximum distance equal to its land speed.
\vspace{0.5em}
\pari
\spelltwocol{\textbf{Spit Acid}: \plus6 vs Reflex}{Everything in 5 ft. wide Medium line}
\par
\par \textit{Hit}: 2d6 acid damage, and each target is sickened as a condition.
\end{spelleffects}
\end{spellcontent}
\begin{spellsubcontent}
\begin{spellfooter}
\pari \textbf{Awareness} \plus6
\pari \textbf{Speed} 40 ft.;
\textbf{Space} 10 ft.;
\textbf{Reach} 10 ft.
\pari \textbf{Attributes}
Str 8,
Dex 7,
Con 4,
Int \minus7,
Per 4,
Wil 0
\end{spellfooter}
\end{spellsubcontent}
\end{monsection}
\begin{monsection}{Aranea}{5}[1]
\vspace{-1em}\spelltwocol{}{Medium magical beast}\vspace{-1em}
\begin{spellcontent}
\begin{spelltargetinginfo}
\spelltwocol{
\textbf{HP} 30;
\textbf{Bloodied} 15;
\textbf{Recovery} 2d6
}
{\textbf{AP} 5}
\pari \textbf{Armor} 10;
\textbf{Fort} 9;
\textbf{Ref} 12;
\textbf{Ment} 14
\pari \textbf{Strike} Bite \plus5 (2d6)
\end{spelltargetinginfo}
\begin{spelleffects}
\pari
\textbf{Shapeshift}:
The aranea makes a Disguise check to change its appearance.
It ignores all penalties for differences between its natural appearance and its intended appearance.
\end{spelleffects}
\end{spellcontent}
\begin{spellsubcontent}
\begin{spellfooter}
\pari \textbf{Awareness} \plus6
\pari \textbf{Speed} 30 ft.;
\textbf{Space} 5 ft.;
\textbf{Reach} 5 ft.
\pari \textbf{Attributes}
Str 0,
Dex 6,
Con 0,
Int 6,
Per 3,
Wil 7
\end{spellfooter}
\end{spellsubcontent}
\end{monsection}
\begin{monsection}{Basilisk}{5}[2]
\vspace{-1em}\spelltwocol{}{Medium magical beast}\vspace{-1em}
\begin{spellcontent}
\begin{spelltargetinginfo}
\spelltwocol{
\textbf{HP} 84;
\textbf{Bloodied} 42;
\textbf{Recovery} 2d8
}
{\textbf{AP} 3}
\pari \textbf{Armor} 11;
\textbf{Fort} 14;
\textbf{Ref} 7;
\textbf{Ment} 8
\pari \textbf{Strike} Bite \plus5 (2d8)
\pari \textbf{Actions} One in action phase, one in delayed action phase
\end{spelltargetinginfo}
\begin{spelleffects}
\pari
\spelltwocol{\textbf{Petrifying Gaze}: \plus5 vs Fortitude}{One creature in Medium range}
\par
\par \textit{Hit}: The target is nauseated as a condition.
\par \textit{Critical}:
As above, and as an additional condition, the target takes 1d10 physical damage at the end of each action phase.
If it takes vital damage in this way, it is petrified permanently.
\end{spelleffects}
\end{spellcontent}
\begin{spellsubcontent}
\begin{spellfooter}
\pari \textbf{Awareness} \plus6
\pari \textbf{Speed} 30 ft.;
\textbf{Space} 5 ft.;
\textbf{Reach} 5 ft.
\pari \textbf{Attributes}
Str 6,
Dex \minus1,
Con 6,
Int \minus6,
Per 3,
Wil 0
\end{spellfooter}
\end{spellsubcontent}
\end{monsection}
\begin{monsection}{Centaur}{3}[1]
\vspace{-1em}\spelltwocol{}{Large magical beast}\vspace{-1em}
\begin{spellcontent}
\begin{spelltargetinginfo}
\spelltwocol{
\textbf{HP} 28;
\textbf{Bloodied} 14;
\textbf{Recovery} 1d10
}
{\textbf{AP} 2}
\pari \textbf{Armor} 9;
\textbf{Fort} 11;
\textbf{Ref} 8;
\textbf{Ment} 6
\pari \textbf{Strike} Longsword \plus4 (2d6) or longbow \plus4 (2d6) or hoof \plus4 (1d10)
\end{spelltargetinginfo}
\end{spellcontent}
\begin{spellsubcontent}
\begin{spellfooter}
\pari \textbf{Awareness} \plus6
\pari \textbf{Speed} 40 ft.;
\textbf{Space} 10 ft.;
\textbf{Reach} 10 ft.
\pari \textbf{Attributes}
Str 2,
Dex 4,
Con 4,
Int 0,
Per 4,
Wil 0
\end{spellfooter}
\end{spellsubcontent}
\end{monsection}
\begin{monsection}{Girallon}{5}[4]
\vspace{-1em}\spelltwocol{}{Large magical beast}\vspace{-1em}
\begin{spellcontent}
\begin{spelltargetinginfo}
\spelltwocol{
\textbf{HP} 120;
\textbf{Bloodied} 60;
\textbf{Recovery} 4d6
}
{\textbf{AP} 4}
\pari \textbf{Armor} 13;
\textbf{Fort} 12;
\textbf{Ref} 12;
\textbf{Ment} 7
\pari \textbf{Strike} Claw \plus7 (2d8)
\pari \textbf{Actions} Two in action phase, two in delayed action phase
\end{spelltargetinginfo}
\end{spellcontent}
\begin{spellsubcontent}
\begin{spellfooter}
\pari \textbf{Awareness} \plus6
\pari \textbf{Speed} 40 ft.;
\textbf{Space} 10 ft.;
\textbf{Reach} 10 ft.
\pari \textbf{Attributes}
Str 7,
Dex 7,
Con 0,
Int \minus8,
Per 6,
Wil \minus1
\end{spellfooter}
\end{spellsubcontent}
\end{monsection}
\begin{monsection}{Griffon}{4}[2]
\vspace{-1em}\spelltwocol{}{Large magical beast}\vspace{-1em}
\begin{spellcontent}
\begin{spelltargetinginfo}
\spelltwocol{
\textbf{HP} 70;
\textbf{Bloodied} 35;
\textbf{Recovery} 2d8
}
{\textbf{AP} 3}
\pari \textbf{Armor} 12;
\textbf{Fort} 12;
\textbf{Ref} 11;
\textbf{Ment} 7
\pari \textbf{Strike} Talon \plus6 (2d6)
\pari \textbf{Actions} One in action phase, one in delayed action phase
\end{spelltargetinginfo}
\begin{spelleffects}
\pari
\textbf{Flyby Attack}:
The griffin flies up to its flying movement speed.
It can make a talon strike at any point during this movement.
\end{spelleffects}
\end{spellcontent}
\begin{spellsubcontent}
\begin{spellfooter}
\pari \textbf{Awareness} \plus6
\pari \textbf{Speed} 80 ft. fly ft.;
\textbf{Space} 10 ft.;
\textbf{Reach} 10 ft.
\pari \textbf{Attributes}
Str 5,
Dex 6,
Con 5,
Int \minus4,
Per 3,
Wil 0
\end{spellfooter}
\end{spellsubcontent}
\end{monsection}
\begin{monsection}{Thaumavore}{3}[1]
\vspace{-1em}\spelltwocol{}{Small magical beast}\vspace{-1em}
\begin{spellcontent}
\begin{spelltargetinginfo}
\spelltwocol{
\textbf{HP} 20;
\textbf{Bloodied} 10;
\textbf{Recovery} 1d10
}
{\textbf{AP} 2}
\pari \textbf{Armor} 11;
\textbf{Fort} 9;
\textbf{Ref} 14;
\textbf{Ment} 6
\pari \textbf{Strike} Bite \plus3 (1d10)
\end{spelltargetinginfo}
\end{spellcontent}
\begin{spellsubcontent}
\begin{spellfooter}
\pari \textbf{Awareness} \plus6
\pari \textbf{Speed} 25 ft.;
\textbf{Space} 5 ft.;
\textbf{Reach} 5 ft.
\pari \textbf{Attributes}
Str 4,
Dex 5,
Con 0,
Int \minus7,
Per 2,
Wil 0
\end{spellfooter}
\end{spellsubcontent}
\end{monsection}
\begin{monsection}{Banehound}{5}[4]
\vspace{-1em}\spelltwocol{}{Huge magical beast}\vspace{-1em}
\begin{spellcontent}
\begin{spelltargetinginfo}
\spelltwocol{
\textbf{HP} 120;
\textbf{Bloodied} 60;
\textbf{Recovery} 4d6
}
{\textbf{AP} 5}
\pari \textbf{Armor} 13;
\textbf{Fort} 10;
\textbf{Ref} 10;
\textbf{Ment} 8
\pari \textbf{Strike} Bite \plus7 (2d10)
\pari \textbf{Actions} Two in action phase, two in delayed action phase
\end{spelltargetinginfo}
\end{spellcontent}
\begin{spellsubcontent}
\begin{spellfooter}
\pari \textbf{Awareness} \plus6
\pari \textbf{Speed} 50 ft.;
\textbf{Space} 15 ft.;
\textbf{Reach} 15 ft.
\pari \textbf{Attributes}
Str 3,
Dex 7,
Con 0,
Int 3,
Per 7,
Wil 0
\end{spellfooter}
\end{spellsubcontent}
\end{monsection}
\section{Monstrous Humanoids}
\begin{monsection}{Banshee}{3}[1]
\vspace{-1em}\spelltwocol{}{Medium monstrous humanoid}\vspace{-1em}
\begin{spellcontent}
\begin{spelltargetinginfo}
\spelltwocol{
\textbf{HP} 20;
\textbf{Bloodied} 10;
\textbf{Recovery} 1d10
}
{\textbf{AP} 4}
\pari \textbf{Armor} 8;
\textbf{Fort} 6;
\textbf{Ref} 10;
\textbf{Ment} 11
\pari \textbf{Strike} Claw \plus4 (1d8)
\end{spelltargetinginfo}
\begin{spelleffects}
\pari
\spelltwocol{\textbf{Wail}: \plus4 vs Fortitude}{Everything in a Large radius}
\par
\par \textit{Hit}: 1d10 sonic damage, and each target is sickened as a condition.
\end{spelleffects}
\end{spellcontent}
\begin{spellsubcontent}
\begin{spellfooter}
\pari \textbf{Awareness} \plus6
\pari \textbf{Speed} 30 ft.;
\textbf{Space} 5 ft.;
\textbf{Reach} 5 ft.
\pari \textbf{Attributes}
Str 2,
Dex 4,
Con 0,
Int 0,
Per 2,
Wil 4
\end{spellfooter}
\end{spellsubcontent}
\end{monsection}
\begin{monsection}[Hill]{Giant}{5}[1]
\vspace{-1em}\spelltwocol{}{Large monstrous humanoid}\vspace{-1em}
\begin{spellcontent}
\begin{spelltargetinginfo}
\spelltwocol{
\textbf{HP} 36;
\textbf{Bloodied} 18;
\textbf{Recovery} 2d6
}
{\textbf{AP} 2}
\pari \textbf{Armor} 13;
\textbf{Fort} 13;
\textbf{Ref} 4;
\textbf{Ment} 9
\pari \textbf{Strike} Greatclub \plus5 (4d6) or boulder \plus5 (2d10)
\end{spelltargetinginfo}
\begin{spelleffects}
\pari
\textbf{Boulder Toss}:
The giant makes a ranged boulder strike, treating it as a thrown weapon with a 100 ft.\ range increment.
\end{spelleffects}
\end{spellcontent}
\begin{spellsubcontent}
\begin{spellfooter}
\pari \textbf{Awareness} \plus6
\pari \textbf{Speed} 40 ft.;
\textbf{Space} 10 ft.;
\textbf{Reach} 10 ft.
\pari \textbf{Attributes}
Str 7,
Dex \minus2,
Con 3,
Int \minus2,
Per 0,
Wil 0
\end{spellfooter}
\end{spellsubcontent}
\end{monsection}
\begin{monsection}[Stone]{Giant}{9}[1]
\vspace{-1em}\spelltwocol{}{Huge monstrous humanoid}\vspace{-1em}
\begin{spellcontent}
\begin{spelltargetinginfo}
\spelltwocol{
\textbf{HP} 80;
\textbf{Bloodied} 40;
\textbf{Recovery} 2d10
}
{\textbf{AP} 2}
\pari \textbf{Armor} 19;
\textbf{Fort} 19;
\textbf{Ref} 7;
\textbf{Ment} 13
\pari \textbf{Strike} Greatclub \plus10 (5d10) or boulder \plus10 (4d10)
\end{spelltargetinginfo}
\begin{spelleffects}
\pari
\textbf{Boulder Toss}:
The giant makes a ranged boulder strike, treating it as a thrown weapon with a 100 ft.\ range increment.
\end{spelleffects}
\end{spellcontent}
\begin{spellsubcontent}
\begin{spellfooter}
\pari \textbf{Awareness} \plus6
\pari \textbf{Speed} 50 ft.;
\textbf{Space} 15 ft.;
\textbf{Reach} 15 ft.
\pari \textbf{Attributes}
Str 11,
Dex \minus2,
Con 11,
Int \minus1,
Per 10,
Wil 0
\end{spellfooter}
\end{spellsubcontent}
\end{monsection}
\begin{monsection}[Storm]{Giant}{15}[1]
\vspace{-1em}\spelltwocol{}{Gargantuan monstrous humanoid}\vspace{-1em}
\begin{spellcontent}
\begin{spelltargetinginfo}
\spelltwocol{
\textbf{HP} 96;
\textbf{Bloodied} 48;
\textbf{Recovery} 4d10
}
{\textbf{AP} 4}
\pari \textbf{Armor} 23;
\textbf{Fort} 22;
\textbf{Ref} 13;
\textbf{Ment} 22
\pari \textbf{Strike} Greatsword \plus16 (9d10)
\pari \textbf{Immune} deafened
\end{spelltargetinginfo}
\begin{spelleffects}
\pari
\spelltwocol{\textbf{Lightning Javelin}: \plus16 vs Reflex}{All in a 10 ft. wide Large line}
\par
\par \textit{Hit}: 5d10 electricity damage.
\vspace{0.5em}
\pari
\textbf{Thunderstrike}:
The storm giant makes a greatsword strike against a target.
If its attack result beats the target's Fortitude defense,
the target also takes 4d10 sonic damage
and is deafened as a condition.
\end{spelleffects}
\end{spellcontent}
\begin{spellsubcontent}
\begin{spellfooter}
\pari \textbf{Awareness} \plus6
\pari \textbf{Speed} 60 ft.;
\textbf{Space} 20 ft.;
\textbf{Reach} 20 ft.
\pari \textbf{Attributes}
Str 17,
Dex \minus1,
Con 8,
Int 8,
Per 16,
Wil 16
\end{spellfooter}
\end{spellsubcontent}
\end{monsection}
\begin{monsection}[Green]{Hag}{5}[2]
\vspace{-1em}\spelltwocol{}{Medium monstrous humanoid}\vspace{-1em}
\begin{spellcontent}
\begin{spelltargetinginfo}
\spelltwocol{
\textbf{HP} 60;
\textbf{Bloodied} 30;
\textbf{Recovery} 2d8
}
{\textbf{AP} 5}
\pari \textbf{Armor} 12;
\textbf{Fort} 8;
\textbf{Ref} 13;
\textbf{Ment} 13
\pari \textbf{Strike} Claw \plus7 (1d10)
\pari \textbf{Actions} One in action phase, one in delayed action phase
\end{spelltargetinginfo}
\begin{spelleffects}
\pari
\spelltwocol{\textbf{Vital Surge}: \plus7 vs Fortitude}{}
\par
\par \textit{Hit}: 2d10 life damage.
\vspace{0.5em}
\pari
\spelltwocol{\textbf{Green Hag's Curse}: \plus7 vs Mental}{}
\par
\par \textit{Hit}:
As a condition, the target is either dazed, fatigued, or sickened, as the hag chooses.
\par \textit{Critical}: As three separate conditions, the target is dazed, fatigued, and sickened.
\vspace{0.5em}
\pari
\textbf{Coven Rituals}:
Whenever three or more hags work together, they form a coven.
All members of the coven gain the ability to perform nature rituals as long as they work together.
Hags of any type can form a coven together.
\end{spelleffects}
\end{spellcontent}
\begin{spellsubcontent}
\begin{spellfooter}
\pari \textbf{Awareness} \plus6
\pari \textbf{Speed} 30 ft.;
\textbf{Space} 5 ft.;
\textbf{Reach} 5 ft.
\pari \textbf{Attributes}
Str 0,
Dex 6,
Con 0,
Int 6,
Per 7,
Wil 6
\end{spellfooter}
\end{spellsubcontent}
\end{monsection}
\begin{monsection}{Medusa}{7}[2]
\vspace{-1em}\spelltwocol{}{Medium monstrous humanoid}\vspace{-1em}
\begin{spellcontent}
\begin{spelltargetinginfo}
\spelltwocol{
\textbf{HP} 80;
\textbf{Bloodied} 40;
\textbf{Recovery} 2d10
}
{\textbf{AP} 5}
\pari \textbf{Armor} 11;
\textbf{Fort} 10;
\textbf{Ref} 13;
\textbf{Ment} 15
\pari \textbf{Strike} Longbow \plus8 (2d8) or snakes \plus8 (2d6)
\pari \textbf{Actions} One in action phase, one in delayed action phase
\end{spelltargetinginfo}
\begin{spelleffects}
\pari
\spelltwocol{\textbf{Petrifying Gaze}: \plus8 vs Fortitude}{One creature in Medium range}
\par
\par \textit{Hit}: The target is nauseated as a condition.
\par \textit{Critical}:
As above, and as an additional condition, the target takes 1d10 physical damage at the end of each action phase.
If it takes vital damage in this way, it is petrified permanently.
\end{spelleffects}
\end{spellcontent}
\begin{spellsubcontent}
\begin{spellfooter}
\pari \textbf{Awareness} \plus6
\pari \textbf{Speed} 30 ft.;
\textbf{Space} 5 ft.;
\textbf{Reach} 5 ft.
\pari \textbf{Attributes}
Str 0,
Dex 4,
Con 0,
Int 4,
Per 8,
Wil 8
\end{spellfooter}
\end{spellsubcontent}
\end{monsection}
\section{Outsiders}
\begin{monsection}[Astral Deva]{Angel}{14}[1]
\vspace{-1em}\spelltwocol{}{Medium outsider}\vspace{-1em}
\begin{spellcontent}
\begin{spelltargetinginfo}
\spelltwocol{
\textbf{HP} 105;
\textbf{Bloodied} 52;
\textbf{Recovery} 4d10
}
{\textbf{AP} 4}
\pari \textbf{Armor} 23;
\textbf{Fort} 19;
\textbf{Ref} 21;
\textbf{Ment} 23
\pari \textbf{Strike} Mace \plus15 (4d10)
\end{spelltargetinginfo}
\begin{spelleffects}
\pari
\textbf{Smite}:
The angel makes a mace strike.
If its target is evil, it gains a \plus2 bonus to accuracy and a \plus2d bonus to damage on the strike.
\vspace{0.5em}
\pari
\textbf{Angel's Grace}:
One willing creature within reach heals 5d10 hit points.
\end{spelleffects}
\end{spellcontent}
\begin{spellsubcontent}
\begin{spellfooter}
\pari \textbf{Awareness} \plus6
\pari \textbf{Speed} 30 ft.;
\textbf{Space} 5 ft.;
\textbf{Reach} 5 ft.
\pari \textbf{Attributes}
Str 15,
Dex 15,
Con 15,
Int 15,
Per 15,
Wil 15
\end{spellfooter}
\end{spellsubcontent}
\end{monsection}
\begin{monsection}{Arrowhawk}{3}[1]
\vspace{-1em}\spelltwocol{}{Medium outsider}\vspace{-1em}
\begin{spellcontent}
\begin{spelltargetinginfo}
\spelltwocol{
\textbf{HP} 16;
\textbf{Bloodied} 8;
\textbf{Recovery} 1d10
}
{\textbf{AP} 2}
\pari \textbf{Armor} 10;
\textbf{Fort} 5;
\textbf{Ref} 14;
\textbf{Ment} 8
\pari \textbf{Strike} Bite \plus5 (1d10)
\end{spelltargetinginfo}
\begin{spelleffects}
\pari
\spelltwocol{\textbf{Electrobolt}: \plus5 vs Reflex}{One target in Medium range}
\par
\par \textit{Hit}: 2d6 electricity damage.
\end{spelleffects}
\end{spellcontent}
\begin{spellsubcontent}
\begin{spellfooter}
\pari \textbf{Awareness} \plus6
\pari \textbf{Speed} 60 ft. fly (good) ft.;
\textbf{Space} 5 ft.;
\textbf{Reach} 5 ft.
\pari \textbf{Attributes}
Str 2,
Dex 6,
Con \minus1,
Int 0,
Per 5,
Wil 0
\end{spellfooter}
\end{spellsubcontent}
\end{monsection}
\begin{monsection}[Bebelith]{Demon}{11}[1]
\vspace{-1em}\spelltwocol{}{Huge outsider}\vspace{-1em}
\begin{spellcontent}
\begin{spelltargetinginfo}
\spelltwocol{
\textbf{HP} 84;
\textbf{Bloodied} 42;
\textbf{Recovery} 4d6
}
{\textbf{AP} 2}
\pari \textbf{Armor} 19;
\textbf{Fort} 17;
\textbf{Ref} 16;
\textbf{Ment} 16
\pari \textbf{Strike} Bite \plus11 (5d10)
\end{spelltargetinginfo}
\begin{spelleffects}
\pari
\textbf{Venomous Bite}:
The bebelith makes a bite strike.
If it hits, and the attack result beats the target's Fortitude defense, the target is also poisoned as a condition.
If the target is poisoned, it takes 4d6 poison damage at the end of each action phase after the first round.
\end{spelleffects}
\end{spellcontent}
\begin{spellsubcontent}
\begin{spellfooter}
\pari \textbf{Awareness} \plus6
\pari \textbf{Speed} 50 ft.;
\textbf{Space} 15 ft.;
\textbf{Reach} 15 ft.
\pari \textbf{Attributes}
Str 12,
Dex 13,
Con 12,
Int 0,
Per 6,
Wil 0
\end{spellfooter}
\end{spellsubcontent}
\end{monsection}
\begin{monsection}{Hell Hound}{4}[1]
\vspace{-1em}\spelltwocol{}{Medium outsider}\vspace{-1em}
\begin{spellcontent}
\begin{spelltargetinginfo}
\spelltwocol{
\textbf{HP} 25;
\textbf{Bloodied} 12;
\textbf{Recovery} 2d6
}
{\textbf{AP} 2}
\pari \textbf{Armor} 10;
\textbf{Fort} 7;
\textbf{Ref} 13;
\textbf{Ment} 9
\pari \textbf{Strike} Bite \plus5 (2d6)
\pari \textbf{Immune} fire damage
\end{spelltargetinginfo}
\begin{spelleffects}
\pari
\spelltwocol{\textbf{Fire Breath}: \plus5 vs Reflex}{}
\par
\par \textit{Hit}: 2d6 fire damage
\end{spelleffects}
\end{spellcontent}
\begin{spellsubcontent}
\begin{spellfooter}
\pari \textbf{Awareness} \plus6
\pari \textbf{Speed} 30 ft.;
\textbf{Space} 5 ft.;
\textbf{Reach} 5 ft.
\pari \textbf{Attributes}
Str 3,
Dex 6,
Con 0,
Int \minus3,
Per 5,
Wil 0
\end{spellfooter}
\end{spellsubcontent}
\end{monsection}
\begin{monsection}[Flamebrother]{Salamander}{4}[1]
\vspace{-1em}\spelltwocol{}{Medium outsider}\vspace{-1em}
\begin{spellcontent}
\begin{spelltargetinginfo}
\spelltwocol{
\textbf{HP} 25;
\textbf{Bloodied} 12;
\textbf{Recovery} 2d6
}
{\textbf{AP} 2}
\pari \textbf{Armor} 11;
\textbf{Fort} 10;
\textbf{Ref} 11;
\textbf{Ment} 9
\pari \textbf{Strike} Spear \plus4 (2d8) or tail slam \plus4 (2d8)
\pari \textbf{Immune} fire damage
\end{spelltargetinginfo}
\begin{spelleffects}
\pari
\textbf{Tail Grab}:
The salamander makes a tail slam strike.
If the attack result beats the target's Fortitude defense, it is grappled.
\vspace{0.5em}
\pari
\textbf{Flame Aura}:
The salamander intensifies its natural body heat, creating a burning aura around it.
At the end of each action phase, the salamander makes a 4 attack
against everything within a Medium radius emanation of it.
A hit deals 1d10 fire damage to each target.
This ability costs an action point to use.
It lasts as long as the salamander sustains it as a standard action.
\end{spelleffects}
\end{spellcontent}
\begin{spellsubcontent}
\begin{spellfooter}
\pari \textbf{Awareness} \plus6
\pari \textbf{Speed} 30 ft.;
\textbf{Space} 5 ft.;
\textbf{Reach} 5 ft.
\pari \textbf{Attributes}
Str 7,
Dex 5,
Con 0,
Int 3,
Per 3,
Wil 0
\end{spellfooter}
\end{spellsubcontent}
\end{monsection}
\begin{monsection}[Battlemaster]{Salamander}{5}[3]
\vspace{-1em}\spelltwocol{}{Medium outsider}\vspace{-1em}
\begin{spellcontent}
\begin{spelltargetinginfo}
\spelltwocol{
\textbf{HP} 90;
\textbf{Bloodied} 45;
\textbf{Recovery} 2d10
}
{\textbf{AP} 5}
\pari \textbf{Armor} 12;
\textbf{Fort} 11;
\textbf{Ref} 12;
\textbf{Ment} 11
\pari \textbf{Strike} Spear \plus6 (2d10) or tail slam \plus6 (2d10)
\pari \textbf{Actions} Two in action phase, one in delayed action phase
\end{spelltargetinginfo}
\begin{spelleffects}
\pari
\textbf{Tail Grab}:
The salamander makes a tail slam strike.
If the attack result beats the target's Fortitude defense, it is grappled.
\vspace{0.5em}
\pari
\textbf{Flame Aura}:
The salamander intensifies its natural body heat, creating a burning aura around it.
At the end of each action phase, the salamander makes a 5 attack
against everything within a Medium radius emanation of it.
A hit deals 1d10 fire damage to each target.
This ability costs an action point to use.
It lasts as long as the salamander sustains it as a standard action.
\end{spelleffects}
\end{spellcontent}
\begin{spellsubcontent}
\begin{spellfooter}
\pari \textbf{Awareness} \plus6
\pari \textbf{Speed} 30 ft.;
\textbf{Space} 5 ft.;
\textbf{Reach} 5 ft.
\pari \textbf{Attributes}
Str 8,
Dex 6,
Con 0,
Int 3,
Per 6,
Wil 3
\end{spellfooter}
\end{spellsubcontent}
\end{monsection}
\section{Undead}
\begin{monsection}{Allip}{4}[1]
\vspace{-1em}\spelltwocol{}{Medium undead}\vspace{-1em}
\begin{spellcontent}
\begin{spelltargetinginfo}
\spelltwocol{
\textbf{HP} 25;
\textbf{Bloodied} 12;
\textbf{Recovery} 2d6
}
{\textbf{AP} 5}
\pari \textbf{Armor} 10;
\textbf{Fort} 6;
\textbf{Ref} 13;
\textbf{Ment} 15
\pari \textbf{Strike} Draining touch \plus6 (1d10)
\end{spelltargetinginfo}
\end{spellcontent}
\begin{spellsubcontent}
\begin{spellfooter}
\pari \textbf{Awareness} \plus6
\pari \textbf{Speed} 30 ft.;
\textbf{Space} 5 ft.;
\textbf{Reach} 5 ft.
\pari \textbf{Attributes}
Str 0,
Dex 6,
Con 0,
Int 0,
Per 0,
Wil 6
\end{spellfooter}
\end{spellsubcontent}
\end{monsection}
\begin{monsection}{Dirgewalker}{4}[4]
\vspace{-1em}\spelltwocol{}{Medium undead}\vspace{-1em}
\begin{spellcontent}
\begin{spelltargetinginfo}
\spelltwocol{
\textbf{HP} 100;
\textbf{Bloodied} 50;
\textbf{Recovery} 4d6
}
{\textbf{AP} 7}
\pari \textbf{Armor} 12;
\textbf{Fort} 6;
\textbf{Ref} 13;
\textbf{Ment} 13
\pari \textbf{Strike} Claw \plus6 (1d10)
\pari \textbf{Actions} Two in action phase, two in delayed action phase
\end{spelltargetinginfo}
\begin{spelleffects}
\pari
\textbf{Animating Caper}:
One corpse within Close range is animated as a skeleton under the dirgewalker's control.
This ability costs an action point to use.
It lasts as long as the dirgewalker attunes to it.
\vspace{0.5em}
\pari
\spelltwocol{\textbf{Mournful Dirge}: \plus6 vs Mental}{All creatures in a Medium radius}
\par
\par \textit{Hit}: Each target is dazed as a condition.
\par \textit{Critical}: Each target is stunned as a condition.
\end{spelleffects}
\end{spellcontent}
\begin{spellsubcontent}
\begin{spellfooter}
\pari \textbf{Awareness} \plus6
\pari \textbf{Speed} 30 ft.;
\textbf{Space} 5 ft.;
\textbf{Reach} 5 ft.
\pari \textbf{Attributes}
Str 0,
Dex 6,
Con 0,
Int 3,
Per 6,
Wil 5
\end{spellfooter}
\end{spellsubcontent}
\end{monsection}
\begin{monsection}{Skeleton}{1}[1]
\vspace{-1em}\spelltwocol{}{Medium undead}\vspace{-1em}
\begin{spellcontent}
\begin{spelltargetinginfo}
\spelltwocol{
\textbf{HP} 10;
\textbf{Bloodied} 5;
\textbf{Recovery} 1d8
}
{\textbf{AP} 2}
\pari \textbf{Armor} 8;
\textbf{Fort} 5;
\textbf{Ref} 8;
\textbf{Ment} 6
\pari \textbf{Strike} Claw \plus2 (1d8)
\end{spelltargetinginfo}
\end{spellcontent}
\begin{spellsubcontent}
\begin{spellfooter}
\pari \textbf{Awareness} \plus6
\pari \textbf{Speed} 30 ft.;
\textbf{Space} 5 ft.;
\textbf{Reach} 5 ft.
\pari \textbf{Attributes}
Str 2,
Dex 2,
Con 0,
Int 0,
Per 0,
Wil 0
\end{spellfooter}
\end{spellsubcontent}
\end{monsection}
\begin{monsection}{Zombie}{1}[1]
\vspace{-1em}\spelltwocol{}{Medium undead}\vspace{-1em}
\begin{spellcontent}
\begin{spelltargetinginfo}
\spelltwocol{
\textbf{HP} 16;
\textbf{Bloodied} 8;
\textbf{Recovery} 1d8
}
{\textbf{AP} 2}
\pari \textbf{Armor} 7;
\textbf{Fort} 10;
\textbf{Ref} 4;
\textbf{Ment} 6
\pari \textbf{Strike} Slam \plus1 (1d10)
\end{spelltargetinginfo}
\end{spellcontent}
\begin{spellsubcontent}
\begin{spellfooter}
\pari \textbf{Awareness} \plus6
\pari \textbf{Speed} 30 ft.;
\textbf{Space} 5 ft.;
\textbf{Reach} 5 ft.
\pari \textbf{Attributes}
Str 2,
Dex 0,
Con 3,
Int 0,
Per 0,
Wil 0
\end{spellfooter}
\end{spellsubcontent}
\end{monsection}
\end{document}
