\chapter{Combat}\label{Combat}
    The world of Rise can be a harsh one, and not all disagreements can be resolved peacefully.
    At some point, you will be forced to enter combat.
    This chapter explains how combat works in Rise.

\section{Combat Time}\label{Combat Time}
    This section explains how time passes in combat.

    \subsection{Rounds}\label{Rounds}

        Combat takes place in a series of \glossterm{rounds}, which represent about six seconds of time.
        Each round of a combat is divided into three \glossterm{phases} (see \pcref{Phases}).
        After all phases are complete, the round ends and the next round begins.

    \subsection{Actions}\label{Actions}

        You can take actions in combat to defeat your foes.
        There are four types of actions: \glossterm{standard actions}, \glossterm{minor actions}, \glossterm{move actions}, and \glossterm{free actions}.
        % TODO: define conflicting action limits (drawing a sword and sheathing a shield?)

        \subsubsection{Standard Actions}\label{Standard Actions}
            Most common activities require a \glossterm{standard action}, such as attacking with a weapon, casting a \glossterm{spell}, and using many special abilities.
            Using a standard action generally takes about three seconds of time within the game, and it requires most of your attention during that time.

            You can take one standard action per round.

        \subsubsection{Minor Actions}\label{Minor Actions}
            Some special abilities require a \glossterm{minor action}.
            Using a minor action does not take much time or attention, and it can be done at the same time as any other actions.
            You cannot use a \glossterm{minor action} during the \glossterm{movement phase}.

            You can normally take one minor action per round.
            However, you can choose to take an additional minor action in place of a \glossterm{standard action}.

        \subsubsection{Move Actions}\label{Move Actions}
            Abilities that require a move action typically move you around the battlefield, and are usually used in the \glossterm{movement phase}.
            Using a move action generally takes about three seconds of time within the game, and it requires most of your attention during that time.

            You can normally take one move action per round.
            However, you can choose to take an additional move action in place of a \glossterm{standard action}.

        \subsubsection{Free Actions}\label{Free Actions}
            Many minor activities require a \glossterm{free action}, such as drawing or sheathing a weapon.
            Using a free action does not take much time or attention, and it can be done at the same time as any other actions.

            You can take any number of free actions per round.

    \subsection{Phases}\label{Phases}

        There are three \glossterm{phases} in each round: a \glossterm{movement phase}, an \glossterm{action phase}, and sometimes a \glossterm{delayed action phase}.
        Each phase specifies the types of actions that can be taken during that phase.
        As a special case, \glossterm{free actions} may be taken during any phase.

        \subsubsection{The Movement Phase}\label{The Movement Phase}
            During the \glossterm{movement phase}, you can take one \glossterm{move action}.
            The most common move action is the \textit{hustle} ability, which allows you to move a distance equal to your \glossterm{speed}.
            For details, see \pcref{Movement and Positioning}.

        \subsubsection{The Action Phase}\label{The Action Phase}
            During the \glossterm{action phase}, you can take one \glossterm{minor action} and one \glossterm{standard action}.
            Alternately, you can take a \glossterm{move action} or additional \glossterm{minor action} in place of your standard action.
            Most of the time, you will simply take a single standard action.

        \subsubsection{The Delayed Action Phase}\label{The Delayed Action Phase}
            During the \glossterm{delayed action phase}, you can take a \glossterm{minor action}, a \glossterm{standard action}, or both if you did not use the corresponding action in the \glossterm{action phase}.
            Alternately, you can take a \glossterm{move action} or additional \glossterm{minor action} in place of a standard action.
            In addition, some abilities have effects during the delayed action phase instead of or in addition to their effects in the action phase.
            For example, the \textit{spring attack} \glossterm{maneuver} allows you to move during the action phase and again during the delayed action phase (see Spring Attack, \pref{maneuver:Spring Attack}).

        \subsubsection{Triggered Ability Timing}\label{Triggered Ability Timing}
            Some abilities trigger at the start or end of particular phases, or at the start or end of the round.
            Here is the order in which these abilities trigger each round, and some notable events that occur during the round:
            \begin{itemize}
                \item Start of round
                \item Start of movement phase
                \item End of movement phase
                \item Start of action phase
                \item End of action phase
                \item Start of delayed action phase
                \item End of delayed action phase
                \item End of round
            \end{itemize}

    \subsection{Resolving Actions}\label{Resolving Actions}

        Within each phase, actions of all creatures are simultaneously resolved in the following order.
        All \glossterm{allies} with the ability to communicate can freely coordinate their actions with each other, within reasonable limits.

        \begin{enumerate*}
            \item Choose actions.
            \item Determine targets affected by actions.
            \item Apply the results of \abilitytag{Swift} abilities.
            \item Check action success.
                Example: Making attack rolls.
            \item Determine action results.
                Example: Making damage rolls.
            \item Apply action results.
                Examples: Adding \glossterm{vital wounds}, moving creature locations, and applying penalties.
        \end{enumerate*}

        In the vast majority of cases, there is no need to go through this order explicitly.
        Combats will run much faster if attack and damage rolls are generally made and announced at the same time as those actions are chosen, even before all characters have explicitly stated their actions.
        However, the order of resolution is important because it limits direct interaction between player actions and enemy actions.
        Even if a player is knocked unconscious or suffers a debilitating penalty from an enemy attack, they still get to act normally during that phase.

        \subsubsection{Swift Abilities}\label{Swift Abilities}
            Some abilities resolve before other actions in the same phase.
            These abilities have the \abilitytag{Swift} tag.
            They resolve after targets are determined, but before attack rolls are made.
            Swift abilities never require attack rolls, and almost always affect only the creature using the ability.

            For example, the \textit{total defense} ability is a swift ability.
            It increases your defenses against attacks made during the same phase (see \pcref{Total Defense}).

            Some abilities have only part of their effect resolve early.
            For example, the \textit{reckless attack} ability immediately reduces your defenses, which affects attacks made against you during the current phase, and makes an attack with the normal timing.

        % This may be too complicated
        \subsubsection{Conflicting Actions}\label{Conflicting Actions}
            Sometimes, actions that occur in the same phase can conflict with each other.
            In this case, each creature involved with conflicting actions in that phase rolls an \glossterm{initiative} check (see \pcref{Initiative}).
            Starting from the highest check result and continuing to the lowest, each creature immediately resolves its chosen action.
            Creatures that resolve their action afterward accomplish as much of their intended action as possible before being blocked or otherwise prevented.

            For example, if three different creatures use the \ability{hustle} ability to move into the same space, only the creature with the highest initiative check would actually enter that space.
            The other two creatures would take their intended path, but they would interrupt their movement when they cannot proceed farther, generally because they run into the space occupied by the first creature.

            In general, directly conflicting actions are rare.
            Most movements do not conflict - even reactive movements, such as when one creature attempts to follow a withdrawing creature.
            In that case, no initiative check is necessary - both creatures simply move as far as they can, and the creatures' relative movement speeds determine who is more successful.
            This does make it possible for creatures to be ``stranded'' out of melee range of any attackers.
            Player characters are normally allowed to break this symmetry by reactively using the \textit{sprint} ability, while monsters cannot sprint.
            This can help prevents melee characters from feeling stuck or useless.
            In addition, the \ability{charge} universal ability can be helpful in such cases.

\section{Movement and Positioning}\label{Movement and Positioning}
    This section describes how creatures move and position themselves on a battlefield.

    \subsection{Movement Modes}\label{Movement Modes}
        A \glossterm{movement mode} is a method of moving from one location to another.
        % TODO: terminology confusion mode vs. speed
        The most common movement mode is a land speed, which allows creatures to move across the ground.
        Unless otherwise noted, all creatures have a land speed equal to the base speed for their size (see \pcref{Size Categories}).
        In addition, some abilities grant creatures the ability to move in unusual ways.
        These forms of movement are described here.

        \parhead{Burrowing}
        A creature with a burrow speed can move through the ground at the indicated speed in any direction, even vertically. Unless otherwise noted, the creature can only burrow through dirt and loose earth, not rock or harder substances. It does not leave behind a usable tunnel for other creatures.

        \parhead{Climbing}
        A creature with a \glossterm{climb speed} can move a distance equal to its climb speed with a successful Climb check (see \pcref{Climb}).
        In addition, it gains a \plus10 bonus to any Climb checks it makes.

        \parhead{Flying}\label{Flying}
        A creature with a \glossterm{fly speed} can fly through the air at the indicated speed.
        Flying is more complicated than some other movement speeds.
        For details, see \pcref{Flying Mechanics}.

        \parhead{Gliding}\label{Gliding}
        A creature with a glide speed can glide through the air at the indicated speed
        It must not be carrying weight in excess of its maximum \glossterm{carrying capacity} (see \pcref{Weight Limits}).
        Whenever a creature glides, it takes a \minus2 penalty to Armor and Reflex defenses until it reaches solid ground.

        While in the air, a creature with a glide speed can control its fall as a \glossterm{move action}. This allows it to move up to its speed horizontally in a direction of its choice while moving only five feet down. If it desires, it can move half as far horizontally and fall down twice as fast. It takes no falling damage if it touches the ground while gliding.

        \parhead{Land}
        A creature with a land speed can move across the ground at the indicated speed.
        Most creatures have a land speed.

    \subsection{Flying Mechanics}\label{Flying Mechanics}
        A creature with a fly speed cannot fly while it is carrying weight in excess of its maximum \glossterm{carrying capacity} (see \pcref{Weight Limits}).
        In addition, it cannot fly while it has any \glossterm{encumbrance}.

        \parhead{Maximum Height} Some abilities that grant a fly speed also have a height limit for the maximum height you can reach with that fly speed.
        This height measures your maximum distance directly above an object at least two size categories larger than you that is free-standing and capable of supporting your weight.
        You can fly above surfaces like water as long as they are thick enough to support your weight.

        \parhead{Falling} If a flying creature loses control, usually by failing to maintain its minimum forward speed, or loses the ability to fly, it falls just like any other creature would in midair. As long as it still has the ability to fly, it can regain control of its fall as a standard action, causing it to resume flying normally.

        \subsubsection{Size-Based Abilities While Flying}
            You can reach a creature's weak spots more easily while flying than while trying to attack its feet from the ground.
            While flying, you are treated as one size category larger than normal for the purpose of determing the effects of your \abilitytag{Size-Based} abilities.
            This cannot be combined with other effects that increase your effective size for the purpose of Size-Based abilities, such as climbing on creatures (see \pcref{Creature Climb}).

        \subsubsection{Flying Maneuverability}\label{Flying Maneuverability}
        Each creature with a fly speed also has a maneuverability: good, average, or poor.
        Unless otherwise specified, a creature with a fly speed has average maneuverability.

            \parhead{Good Maneuverability}
            \begin{itemize}
                \item Minimum speed: The creature does not need to move horizontally to maintain its flight, allowing it to hover.
                \item Vertical movement: The creature can move up at the same speed as it moves horizontally, and it can fly down twice as fast.
            \end{itemize}

            \parhead{Average Maneuverability}
            \begin{itemize}
                \item Defense penalties: Whenever the creature flies, it takes a \minus2 penalty to Armor and Reflex defenses until it reaches solid ground.
                \item Minimum speed: The creature must move horizontally by at least a quarter of its fly speed each round. If it does not, it falls.
                \item Vertical movement: The creature can fly up at half speed, but can fly down twice as fast.
            \end{itemize}

            \parhead{Poor Maneuverability}
            \begin{itemize}
                \item Defense penalties: Whenever the creature flies, it takes a \minus4 penalty to Armor and Reflex defenses until it reaches solid ground.
                \item Minimum speed: The creature must move horizontally by at least half its fly speed each round. If it does not, it falls.
                \item Vertical movement: The creature can move up or down by only one square vertically per square traveled horizontally.
                    The creature can fly up at half speed, but can fly down twice as fast.
            \end{itemize}

    \subsection{Measuring Movement}

        For simplicity, all movement in combat is measured in five-foot increments.
        While it is possible to be more precise than that, it's generally not worth the complexity.

        \parhead{Squares}\label{Squares} Area is commonly measured in 5-ft.\ by 5-ft.\ spaces called \glossterm{squares}.
        A single square represents the area occupied by a single humanoid creature in combat.
        Sometimes, movement and distance are represented by the number of squares travelled.
        A 30-ft.\ movement is the same thing as moving six squares.

        \parhead{Diagonals}\label{Diagonals} When measuring distance, the first diagonal counts as five feet of movement, and the second counds as ten feet of movement.
        The third costs five feet, the fourth costs ten feet, and so on.
        You can move diagonally past corners and enemies.

    \subsection{Movement Abilities}\label{Movement Abilities}

        Almost all creatures can use these abilities to move around a battlefield.
        Many movement abilities are reactive, allowing you to move automatically in response to the movement of other creatures.
        For example, you can try to follow a creature wherever it goes that round.
        In all cases, if you run out of movement speed before accomplishing your intended task, you simply stop where you ran out of movement.

        The most common types of reactive movements are the \textit{block}, \textit{follow}, and \textit{withdraw} abilities, which are described below.
        However, you can can come up with other reactive movements.
        The main requirement is that a reactive movement must have a simple criteria for determining how you move based on easily observable events.
        Secondarily, reactive movements should be simple to resolve.
        If you find yourself rolling a lot of initiative checks to get through the movement phase, you're probably trying to make overly complicated movements.

        \parhead{Hustle} As a \glossterm{move action}, you can use the \textit{hustle} ability to move.
        This is the most common movement ability.

        \begin{instantability}{Hustle}
            \label{Hustle}
            Instant
            \rankline
            Choose a path that you want to travel.
            You travel that path, up to the limit of your relevant movement speed.
        \end{instantability}

        \parhead{Block} As a \glossterm{move action}, you can use the \textit{block} ability to prevent a creature from entering a particular area.

        \begin{instantability}{Block}
            \label{Block}
            Instant
            \rankline
            During the current phase, whenever a creature that you can see attempts to move from a space adjacent to you into another space adjacent to you, you can attempt to block its movement.
            This includes creatures whose path takes them through two consecutive spaces adjacent to you, even if neither the creature's location at the start of the phase nor its intended location at the end of the phase are adjacent to you.
            When you do, make an opposed \glossterm{initiative} check against the creature.
            If you beat it on the initiative check, it must spend additional movement equal to one of your relevant movement speeds to move from its space.
            If it cannot, it stops moving.
            This represents you automatically repositioning yourself to block its movement.

            You can only make this check once against any individual creature during a current phase.
            If the creature has the ability to move through your space, such as if it uses the \ability{overrun} ability, it can ignore this additional movement cost.
            If multiple creatures are able to block the same creature from moving, it must pay both additional movement costs, which generally keeps it stuck in place.
        \end{instantability}

        \parhead{Follow} As a \glossterm{move action}, you can use the \textit{follow} ability to follow a creature as it moves.

        \begin{instantability}{Follow}
            \label{Follow}
            Instant
            \rankline
            Choose a creature you can see, and the maximum distance you want to follow at.
            During the current phase, you automatically move such that your distance to the target is no greater than your desired follow distance, up to the limit of your relevant movement speed.

            If the target uses an ability that makes it impossible for you to follow its movement, such as teleporting or disappearing from your sight, it is harder for you to follow its movement.
            If you can see its destination, such as if it teleported to a different location within your \glossterm{line of sight}, you must beat the target on an opposed \glossterm{initiative} check.
            Success means that you can follow its movement normally.
            If you fail at the initiative check, or if you cannot tell where the target went, you complete your movement as if the creature was still at the location where it disappeared.
        \end{instantability}

        \parhead{React} As a \glossterm{free action}, you can use the \textit{react} ability to try to choose your movement after seeing what another creature is going to do.

        \begin{instantability}{React}
            \label{React}
            \spelltwocol{Instant}{\abilitytag{Swift}}
            \rankline
            Choose a creature that you can see.
            Make an opposed \glossterm{initiative} check against that creature.
            If you beat it on the initiative check, you learn whether it is going to take a \glossterm{move action} during the current phase, and if so, what that move action will be.
            This does not give you any information about actions other than move actions, so using this ability during the action phase is often pointless.
            If you fail, you learn nothing about that creature's movement, and that creature automatically beats you on any other opposed initiative checks during the current phase.
            This represents you wasting time trying to watch the creature, giving it extra time to beat you in any sort of opposed contest or race.
        \end{instantability}

        \parhead{Struggle} As a standard action, you can use the \textit{struggle} ability to move despite movement impediments.

        \begin{durationability}{Struggle}
            \label{Struggle}
            Duration
            \rankline
            Until the end of the current phase, your land speed becomes five feet, regardless of all other effects that would modify your land speed.
            In addition, you can move a distance up to your land speed.
            This does not allow you to pass obstacles unrelated to movement speed penalties, such as walls.
            It also does not help you if you are unable to use your movement speeds, such as if you are \immobilized.
        \end{durationability}

        \parhead{Withdraw} As a \glossterm{move action}, you can use the \textit{withdraw} ability to keep away from creatures as they move.

        \begin{instantability}{Withdraw}
            \label{Withdraw}
            Instant
            \rankline
            This ability functions like the \textit{follow} ability, except that you specify a minimum distance between you and the target instead of a maximum distance.
            In addition, you can specify multiple targets and try to keep away from all of them.
        \end{instantability}

    \subsection{Movement Impediments}

        \parhead{Difficult Terrain}\label{Difficult Terrain}
        Some terrain is hard to move through, like thick bushes or a swamp.
        If a square is \glossterm{difficult terrain}, it increases the movement cost required to move out of the square by 5 feet.

        If a square is considered difficult terrain for multiple reasons, the cost increases stack.
        For example, a square in a swamp that also has thick bushes blocking your passage would cost 10 extra feet of movement to leave.

        \parhead{Obstacles}
        An obstacle is anything that gets in your way. Enemies and large solid objects like walls completely block your movement. If you can get past an obstacle, like a low wall, that square is treated as difficult terrain. Some obstacles require a \glossterm{check} to bypass, such as an Balance check (see \pcref{Balance}).

        \parhead{Squeezing}\label{Squeezing}
        In some cases, you may have to squeeze into or through an area that isn't as wide as the space you take up.
        You can squeeze through or into a space that is at least half as wide as your normal space.
        While \squeezing, you move at half speed, and you take a \minus2 penalty to \glossterm{accuracy}, as well as Armor and Reflex defenses.
        You can squeeze into tighter spaces with the Flexibility skill.

        Creatures that take up multiple squares take up half their normal number of squares while squeezing. For example, a Large creature who normally takes up four spaces takes up two spaces while squeezing.

        \parhead{Accidentally Squeezing} Sometimes a character ends its movement while moving through a space where it's not normally allowed to stop. When that happens, the character is squeezing in the space until it can move. If squeezing is impossible, the creature immediately moves to the closest available space. Try not to do this.

        \parhead{Undergrowth}\label{Undergrowth} Vines, roots, bushes, and similar plants that can obstruct sight are common in forested areas.
        These small plants can impede movement in large quantities.
        There are two kinds of undergrowth: \glossterm{light undergrowth} and \glossterm{heavy undergrowth}.

        \subparhead{Light Undergrowth}\label{Light Undergrowth}
        Light undergrowth provides \glossterm{concealment}.

        \subparhead{Heavy Undergrowth}\label{Heavy Undergrowth}
        Heavy undergrowth provides \glossterm{concealment} and is \glossterm{difficult terrain}, which increases the movement cost required to move out of each square by 5 feet.

    \subsection{Forced Movement}\label{Forced Movement}
        Some abilities can physically move you against your will.
        Effects that limit movement speed, such as \glossterm{difficult terrain}, similarly limit the distance you can be moved by forced movement effects.
        There are two kinds of forced movement: \glossterm{push} effects and \glossterm{knockback} effects.
        Unless otherwise noted, all forced movement effects move the target in a single straight horizontal line.

        \subsubsection{Push Effects}\label{Push Effects}
            A creature affected by a \glossterm{push} effect is being pushed by a constant force.
            If it encounters another creature or a solid obstacle during the movement, the forced movement effect ends without causing additional harm to the creature or the obstacle.
            Similarly, if a creature being pushed stops being supported and would fall, it falls instead of being pushed further.
            This can allow creatures pushed off the edge of a cliff to grab the edge of the cliff.

        \subsubsection{Knockback Effects}\label{Knockback Effects}
            A creature affected by a \glossterm{knockback} effect is thrown backwards by a single point of impact.
            If it encounters another creature or a solid obstacle during the movement, it and the obstacle each take 1d6 damage per 10 feet of movement remaining.
            A creature moving as a result of a knockback effect does not have to be supported during the movement by solid ground.
            This can allow you to knockback creatures off of cliffs without allowing them to save themselves.

    \subsection{Overland Movement}\label{Overland Movement}

        Characters covering long distances cross-country use overland movement.
        Overland movement is measured in miles per hour or miles per day.
        A day normally represents 10 hours of actual travel time.
        However, sailing ships and other methods of travel that keep moving without requiring a rest are listed with a full 24 hours of travel time.

        \begin{dtable}
            \lcaption{Overland Travel Distances}
            \begin{dtabularx}{\columnwidth}{>{\lcol}X c c c c}
                & \multicolumn{4}{c}{\tdash\tdash\tdash Speed \tdash\tdash\tdash} \tableheaderrule
                                     & 15 feet     & 20 feet  & 30 feet     & 40 feet  \\
                One Hour (Overland)  &             &          &             &          \\
                Walk                 & 3/4 mile    & 1 mile   & 1-1/2 miles & 2 miles  \\
                Hustle               & 1-1/2 miles & 2 miles  & 3 miles     & 4 miles  \\
                One Day (Overland)   &             &          &             &          \\
                Walk                 & 7-1/2 miles & 10 miles & 15 miles    & 20 miles \\
                Hustle               & \tdash      & \tdash   & \tdash      & \tdash   \\
            \end{dtabularx}
        \end{dtable}

        \begin{dtable}
            \lcaption{Terrain and Overland Movement}
            \begin{dtabularx}{\columnwidth}{>{\lcol}X c c c}
                \tb{Terrain}   & \tb{Highway} & \tb{Road or Trail} & \tb{Trackless} \tableheaderrule
                Desert, sandy  & \mult1       & \mult1/2           & \mult1/2 \\
                Forest         & \mult1       & \mult1             & \mult1/2 \\
                Hills          & \mult1       & \mult3/4           & \mult1/2 \\
                Jungle         & \mult1       & \mult3/4           & \mult1/4 \\
                Moor           & \mult1       & \mult1             & \mult3/4 \\
                Mountains      & \mult3/4     & \mult3/4           & \mult1/2 \\
                Plains         & \mult1-1/2   & \mult1             & \mult3/4 \\
                Swamp          & \mult1       & \mult3/4           & \mult1/2 \\
                Tundra, frozen & \mult1       & \mult3/4           & \mult3/4
            \end{dtabularx}
        \end{dtable}

        \begin{dtable}
            \lcaption{Mounts and Vehicles}
            \begin{dtabularx}{\columnwidth}{>{\lcol}X l l}
                \tb{Mount/Vehicle}                         & \tb{Per Hour} & \tb{Per Day} \tableheaderrule
                Mount (carrying load)                      &               &          \\
                \tind Light horse or light warhorse        & 6 miles       & 60 miles \\
                \tind Light horse                          & 4 miles       & 40 miles \\
                \tind Light warhorse                       & 4 miles       & 40 miles \\
                \tind Heavy horse or heavy warhorse        & 5 miles       & 50 miles \\
                \tind Heavy horse                          & 3-1/2 miles   & 35 miles \\
                \tind Heavy warhorse                       & 3-1/2 miles   & 35 miles \\
                \tind Pony or warpony                      & 4 miles       & 40 miles \\
                \tind Pony                                 & 3 miles       & 30 miles \\
                \tind Warpony                              & 3 miles       & 30 miles \\
                \tind Donkey or mule                       & 3 miles       & 30 miles \\
                \tind Donkey                               & 2 miles       & 20 miles \\
                \tind Mule                                 & 2 miles       & 20 miles \\
                \tind Dog, riding                          & 4 miles       & 40 miles \\
                \tind Dog, riding                          & 3 miles       & 30 miles \\
                \tind Cart or wagon                        & 2 miles       & 20 miles \\
                \tb{Ship}                                  &               &          \\
                \tind Raft or barge (poled or towed)\fn{1} & 1/2 mile      & 5 miles  \\
                \tind Keelboat (rowed)\fn{1}               & 1 mile        & 10 miles \\
                \tind Rowboat (rowed)\fn{1}                & 1-1/2 miles   & 15 miles \\
                \tind Sailing ship (sailed)                & 2 miles       & 48 miles \\
                \tind Warship (sailed and rowed)           & 2-1/2 miles   & 60 miles \\
                \tind Longship (sailed and rowed)          & 3 miles       & 72 miles \\
                \tind Galley (rowed and sailed)            & 4 miles       & 96 miles \\
            \end{dtabularx}
            1 Rafts, barges, keelboats, and rowboats are used on lakes and rivers.
            If going downstream, add the speed of the current (typically 3 miles per hour) to the speed of the vehicle. In addition to 10 hours of being rowed, the vehicle can also float an additional 14 hours, if someone can guide it, so add an additional 42 miles to the daily distance traveled. These vehicles can't be rowed against any significant current, but they can be pulled upstream by draft animals on the shores.
        \end{dtable}

        \parhead{Walk} A character can walk 10 hours in a day of travel without a problem. Walking for longer than that, or hustling faster than that, requires an Endurance check (see \pcref{Overland Exertion}).
        \parhead{Terrain} The terrain through which a character travels affects how much distance they can cover in an hour or a day (see \trefnp{Terrain and Overland Movement}).
        A highway is a straight, major, paved road.
        A road is typically a dirt track.
        A trail is like a road, except that it allows only single-file travel and does not benefit a party traveling with vehicles.
        Trackless terrain is a wild area with no significant paths.

\section{Universal Abilities}\label{Universal Abilities}
    All creatures can use the following abilities.

    \subsection{Strikes}\label{Strikes}
        A \glossterm{strike} is the most common type of attack.
        There are three kinds of strikes: melee, projectile, and thrown.
        Many abilities allow you to make one or more strikes.
        Whenever you make a strike, you can choose which kind of strike to make.

        All strikes are \glossterm{mundane} abilities.
        Your \glossterm{accuracy} with a strike is the same as your accuracy with most other abilities (see \pcref{Accuracy}).
        Your \glossterm{damage} with a strike is determined by your Strength, your \glossterm{power}, and the damage dice for the weapon you hit with (see \pcref{Strike Damage}).

        Whenever you make a strike, you must choose one weapon to make the strike with.
        Wielding two weapons does not change anything about each strike you make.
        However, wielding two weapons can allow you to make an additional strike each round.
        For details, see \pcref{Offhand Strike}.

        \begin{instantability}{Melee Strike}
            \label{Melee Strike}
            Instant
            \rankline
            Choose one weapon you are wielding and are able to attack with.
            Make an attack vs. Armor with that weapon against anything within your \glossterm{reach} with that weapon.
            You must have \glossterm{line of effect} to the target.

            \hit The target takes damage from the weapon (see \pcref{Strike Damage}).
            \crit You double your damage dice with the attack, as normal for critical hits (see \pcref{Critical Hits}).
        \end{instantability}

        \begin{instantability}{Projectile Strike}
            \label{Projectile Strike}
            Instant
            \rankline
            Choose one weapon with the Projectile \glossterm{weapon tag} that you are wielding and are able to attack with (see \pcref{Weapon Tags}).
            Make an attack vs. Armor with that weapon against anything that you have \glossterm{line of effect} to.
            You suffer a \glossterm{longshot penalty} if the target is at \glossterm{long range} from you with that weapon (see \pcref{Weapon Range Limits}).

            \hit The target takes damage from the weapon (see \pcref{Strike Damage}).
            \crit You double your damage dice with the attack, as normal for critical hits (see \pcref{Critical Hits}).
        \end{instantability}

        \begin{instantability}{Thrown Strike}
            \label{Thrown Strike}
            Instant
            \rankline
            Choose one non-projectile weapon that you are wielding and are able to attack with.
            If the weapon does not have the Thrown \glossterm{weapon tag}, your \glossterm{range limits} with the attack are 10/30, and you are not treated as being \glossterm{proficient} with the weapon (see \pcref{Weapon Proficiency}, and \pcref{Weapon Proficiency}).
            Make an attack vs. Armor with that weapon against anything that you have \glossterm{line of effect} to.
            You suffer a \glossterm{longshot penalty} if the target is at \glossterm{long range} from you with that weapon (see \pcref{Weapon Range Limits}).

            \hit The target takes damage from the weapon (see \pcref{Strike Damage}).
            \crit You double your damage dice with the attack, as normal for critical hits (see \pcref{Critical Hits}).
        \end{instantability}

        \subsubsection{Strike Damage}\label{Strike Damage}
            When you deal damage with a strike, you roll your weapon's damage dice and add your \glossterm{power} with the strike to get the total damage.
            Almost all strikes are considered \glossterm{mundane} abilities, so you would normally use your Strength to determine their damage (see \pcref{Dice Bonuses From Attributes}).

            Weapon damage dice are defined in the Equipment chapter (see \pcref{Weapons}).
            Some abilities modify your weapon damage dice with \glossterm{dice increments}, such as by granting you a \plus1d bonus to your weapon's damage dice.
            For details about dice increments, see \pcref{Dice Increments}.

        \subsubsection{Secondary Strike Targets}\label{Secondary Strike Targets}
            Some abilities allow you to make strikes that affect secondary targets in addition to the primary target or targets.
            You make the same attack roll and damage roll against all targets of the strike.
            For example, weapons with the Sweeping weapon tag can make attacks against secondary targets adjacent to the primary target.
            If a strike has multiple primary targets, you must choose a single creature to be treated as the primary target for the purpose of all abilities that reference secondary targets.

            Multiple abilities that cause a strike to affect secondary targets stack normally unless noted otherwise.

    \subsection{Special Combat Abilities}\label{Special Combat Abilities}

        \begin{dtable}
            \lcaption{Special Combat Abilities}
            \begin{dtabularx}{\columnwidth}{>{\lcol}p{6em} l X}
                \tb{Ability}             & \tb{Defense}       & \tb{Brief Description} \tableheaderrule
                Charge                   & Armor              & Move and attack                      \\
                Desperate Exertion\fn{1} & \tdash             & Gain a bonus on a single roll        \\
                Dirty Trick              & Fort or Ref\fn{2}  & Impose penalty on a foe              \\
                Disarm                   & Ref\fn{2}          & Attack item, knocking it free        \\
                Grapple                  & Fort and Ref\fn{2} & Wrestle with a foe                   \\
                Offhand Strike           & Armor              & Make a strike with an offhand weapon \\
                Overrun\fn{1}            & Fort\fn{2}         & Move through foe's space             \\
                Recover\fn{1}            & \tdash             & Regain hit points, remove conditions \\
                Shove                    & Fort\fn{2}         & Move a foe                           \\
                Sprint\fn{1}             & \tdash             & Move at double speed                 \\
                Total Defense            & \tdash             & Gain \plus2 to defenses              \\
                Throw                    & \tdash             & Throw a held object                  \\
                Trip                     & Ref\fn{2}          & Trip a foe                           \\
            \end{dtabularx}
            1. This ability increases your \glossterm{fatigue level} when used. \\
            2. This ability is \ability{Size-Based} (see \pcref{Size-Based}). \\
        \end{dtable}

        \parhead{Charge}\label{Charge} You can use the \textit{charge} ability as a standard action.

        \begin{instantability}{Charge}[Instant]
            \rankline
            After you use this ability, you \glossterm{briefly} take a \minus2 penalty to all defenses.
            This ability does not have the \abilitytag{Swift} tag, so it does not affect attacks made against you during the current phase.

            Move up to your speed in a single straight line.
            At the end of your movement, you can make a melee \glossterm{strike} from your new location.
        \end{instantability}

        \parhead{Desperate Exertion}\label{Desperate Exertion} You can use the \textit{desperate exertion} ability to succeed at a critical moment when you would otherwise fail.
        Using this ability is not an action, and can be done at any time.
        You can decide to use this ability after you learn whether the original roll succeeded or failed.
        You can even use it after you learn what the effects of a successful attack or check would be, if that is information you could normally learn if it succeeded.
        However, you must use it before the phase is over.

        \begin{instantability}{Desperate Exertion}[Instant]
            \abilitytag{Swift}
            \rankline
            After you use this ability, you increase your \glossterm{fatigue level} by two (see \pcref{Fatigue}).

            You reroll any \glossterm{attack} or \glossterm{check} you just made and gain a \plus2 bonus.
            You must reroll the entire roll, not just one die from the roll (such as if the original roll \glossterm{explodes}).
            As normal for rerolls, if you already rerolled the attack or check because of another ability, you simply roll one additional time when you use this ability.

            You cannot use this to affect rolls that are not attacks or checks, such as \glossterm{vital rolls}.
            You cannot use this ability multiple times to affect the same roll.
        \end{instantability}

        \parhead{Dirty Trick}\label{Dirty Trick} As a standard action, you can use the \textit{dirty trick} ability to creatively impair a foe's ability to fight.

        \begin{durationability}{Dirty Trick}[Duration]
            \abilitytag{Size-Based}
            \rankline
            When you use this ability, you must describe the kind of dirty trick you are performing.
            For example, you can pull a creature's pants down, throw sand, or otherwise use your environment to attack.
            The same creature can be affected by multiple dirty tricks, but each must apply a different penalty.

            Make a melee attack with a free hand against the Fortitude or Reflex defense of one creature within your \glossterm{reach}.
            The target uses whichever defense is appropriate to the nature of the trick you describe.

            On a hit, the target \glossterm{briefly} suffers a \minus2 penalty to one defense of your choice: Armor, Fortitude, Reflex, or Mental.
            You choose the defense, which must be appropriate for the action you described.
            On a critical hit, the effect becomes a \glossterm{condition}.
        \end{durationability}

        \parhead{Disarm}\label{Disarm} As a standard action, you can use the \textit{disarm} ability to knock an item out of a foe's hands.

        \begin{instantability}{Disarm}[Instant]
            \abilitytag{Size-Based}
            \rankline
            Make a melee \glossterm{strike} against an object.
            Unlike most abilities, this ability can target specific items \glossterm{attended} by creatures.
            This attack must beat the target's Reflex defense.
            If the target is attended by a creature, the attack must also beat the attending creature's Reflex defense.

            \hit You choose whether the target object takes damage from the weapon you hit it with.
            In addition, if it is \glossterm{attended} and is not held in a hand or well secured, you can choose to knock it loose.
            Well secured objects include rings worn on fingers, equipped shields, and similarly affixed objects.
            If you do, it falls to the ground in the square occupied by the attending creature that is closest to you.

            \crit As above, except that you can deal double damage and you can also knock loose objects that are held in a single hand, but not objects that are held in two hands or well secured.
        \end{instantability}

        \parhead{Grapple}\label{Grapple} As a standard action, you can use the \textit{grapple} ability to physically grab and restrain a creature.

        \begin{durationability}{Grapple}[Duration]
            \abilitytag{Size-Based}
            \rankline
            Make a melee attack with a free hand against the Fortitude and Reflex defenses of one creature within your \glossterm{reach}.

            On a hit against both defenses, you and the target are \grappled by each other.
            For details, see \pcref{Grappling}.
        \end{durationability}

        \parhead{Offhand Strike}\label{Offhand Strike} As a \glossterm{minor action}, you can use the \textit{offhand strike} ability to quickly attack with an offhand weapon while you attack with a primary weapon.
        Your Dexterity must be at least 1 to use this ability.
        \begin{instantability}{Offhand Strike}[Instant]
            \rankline
            Make a \glossterm{strike}.
            The weapon must be held in a single \glossterm{free hand} or must not require a free hand to attack with, such as a bite natural weapon.
            You cannot use this ability unless you also make a \glossterm{strike} with a different weapon as part of a \glossterm{standard action} during the same phase.
            You take a \minus2 penalty to \glossterm{accuracy} with this strike, and you do not add your \glossterm{power} to damage with the strike.
            In addition, you take a \minus1 penalty to \glossterm{accuracy} with the strike for each non-light weapon you attack with this phase, including the weapon used to make this strike.
        \end{instantability}

        \parhead{Overrun}\label{Overrun} As a \glossterm{move action}, you can use the \textit{overrun} ability to move through creatures in your way.

        \begin{instantability}{Overrun}[Instant]
            \abilitytag{Size-Based}
            \rankline
            After you use this ability, you increase your \glossterm{fatigue level} by one.

            Move up to your movement speed in a straight line.
            You can try to move directly through creatures in your way during this movement.
            Each creature in your way can choose to avoid you, allowing you to pass through its square unhindered.
            If a creature does not attempt to avoid you, you make an attack vs. Fortitude against it.
            You use your full Strength in place of half your Perception to determine your \glossterm{accuracy} with this attack.
            If you move into a creature's space with this ability, but you do not move out of it, you and the creature are usually considered \squeezing as long as you continue sharing space (see \pcref{Squeezing}).

            On a hit, you can move through each target's space.
            On a critical hit, each target is also knocked \prone.
            On a miss, you end your movement immediately.
        \end{instantability}

        \parhead{Recover}\label{Recover} You can use the \textit{recover} ability as a standard action.
        \begin{instantability}{Recover}[Instant]
            \rankline
            After you use this ability, you increase your \glossterm{fatigue level} by two, and you cannot use it again until you take a short rest.

            You regain hit points equal to half your maximum \glossterm{hit points}.
            In addition, you remove all \glossterm{brief} effects and \glossterm{conditions} affecting you.
            This cannot remove effects applied during the current round.
            If you take damage in the same phase that you use this ability, the healing and damage offset, which can prevent you from gaining vital wounds from dropping below 0 hit points (see \pcref{Simultaneous Damage and Healing}).
        \end{instantability}

        \parhead{Shove}\label{Shove} As a standard action, you can use the \textit{shove} ability to physically move a creature.

        \begin{instantability}{Shove}[Instant]
            \abilitytag{Size-Based}
            \rankline
            Choose either one creature within your \glossterm{reach} or all creatures grappling you (see \pcref{Grappling}).

            Make a melee attack with a free hand against the Fortitude defense of each target.
            You use your full Strength in place of half your Perception to determine your \glossterm{accuracy} with this attack.
            If you are not able to use any of your movement speeds, such as if you are being carried by a flying creature, you automatically fail when you try to use this ability, and your defense is treated as 0 against this ability.

            On a hit, you can move up to half your movement speed in a straight line, \glossterm{pushing} each target as you move.
            On a critical hit, you can move up to your full movement speed instead.
        \end{instantability}

        % It's important that this can only be used with move actions
        % to prevent it from being abused with abilities like Shove.
        \parhead{Sprint}\label{Sprint} As a \glossterm{move action}, you can use the \textit{sprint} ability to move more quickly.

        \begin{instantability}{Sprint}[Instant]
            \abilitytag{Swift}
            \rankline
            After you use this ability, you increase your \glossterm{fatigue level} by one.
            You can use this ability in the middle of the movement phase after noticing that your movement is insufficient to keep up with an enemy's reactive movement (see \pcref{Movement Abilities}).

            You can immediately take another \glossterm{move action}.
            For the duration of that move action, you double your speed with all of your movement modes.
        \end{instantability}

        \parhead{Throw}\label{Throw} You can use the \textit{throw} ability to throw an object.
        You can use the ability as a standard action.
        Alternately, you can use it as a \glossterm{move action}.
        If you do, you take a \minus20 penalty to the check, and you cannot make an attack roll to hit with the thrown object.

        As long as you have a Strength of at least \minus2, you do not have to use this ability to throw weapons that are sized appropriately for you and which are designed to be thrown.
        Instead, you can simply use the listed \glossterm{range limits} for those weapons.

        \begin{instantability}{Throw}[Instant]
            \rankline
            Make a Strength check to throw an object you hold in at least one hand.
            The base \glossterm{difficulty value} of this check is 0.
            For each size category larger or smaller than the target that you are, you gain a \plus10 bonus or penalty to the check, to a maximum bonus of \plus20.
            You cannot throw an object whose weight exceeds your maximum \glossterm{carrying capacity} (see \pcref{Weight Limits}).

            If you succeed, you throw the object five feet.
            For every 5 points by which you succeed, you double the distance you throw the object.
            Unlike normal, this doubling uses real-world doubling rules: ten feet, then twenty feet, then forty feet, and so on.
            If you throw the object at a creature or object, you can make an attack roll to hit it with the thrown object, as the \textit{thrown strike} ability.
            That attack roll is rolled separately from the Strength check you make to use this ability.
        \end{instantability}

        \parhead{Total Defense}\label{Total Defense} As a standard action, you can use the \textit{total defense} ability to focus entirely on defending yourself.

        \begin{durationability}{Total Defense}[Duration]
            \abilitytag{Swift}
            \rankline
            You gain a \plus2 bonus to your \glossterm{defenses} until the end of the round.
            Because this ability has the \abilitytag{Swift} tag, this improves your defenses against attacks made against you during the current phase.
        \end{durationability}

        \parhead{Trip}\label{Trip} As a standard action, you can use the \textit{trip} ability to trip a creature.

        \begin{instantability}{Trip}[Instant]
            \abilitytag{Size-Based}
            \rankline
            Make a melee attack using a free hand or a weapon with the Tripping tag against a creature's Reflex defense (see \pcref{Weapon Tags}).
            If you attack with a weapon, you add the weapon's accuracy bonus, if any, to the attack.
            However, this is not a \glossterm{strike}, so abilities like the \glossterm{Sweeping} weapon tag have no effect on this attack.

            On a hit, the target becomes \prone.
            In addition, if you made the attack with a Tripping weapon, the target also takes damage as if you had hit it with a \glossterm{strike} using the weapon.
            You do not add your \glossterm{power} to this damage.
        \end{instantability}

    \subsection{Grappling}\label{Grappling}
        A grappled creature is physically struggling with at least one other creature.
        While grappled, you suffer certain penalties and restrictions, as described below.

        \subsubsection{Being In A Grapple}
            While grappling, you suffer certain penalties and restrictions, as described below. Other than these restrictions, you can act normally. You can also take certain actions in a grapple, as described in \pcref{Grapple Actions}
            \begin{itemize}
                \item One of your hands cannot be used for any purposes other than grappling.
                    This prevents humanoid creatures from taking any actions which would require having two free hands, such as attacking with heavy weapons.
                    This does not affect creatures without hands.
                \item You take a \minus2 penalty to Armor and Reflex defenses.
                \item Abilities that have \glossterm{somatic components} have a 25\% \glossterm{failure chance}.
                \item You cannot move unless you \glossterm{push} all creatures grappling you, such as with the \textit{shove} ability (see \pcref{Shove}).
            \end{itemize}

        \subsubsection{Grapple Actions}\label{Grapple Actions}
            While grappled, you can use two special abilities to try to affect the grapple.

            \parhead{Escape Grapple}\label{Escape Grapple} As a standard action, you can use the \textit{escape grapple} ability to try to stop being grappled.

            \begin{instantability}{Escape Grapple}[Instant]
                \rankline
                Make an attack against any number of creatures that you are grappled by.
                You may use either the Flexibility skill or half your level \add your Strength in place of your normal \glossterm{accuracy} with this attack (see \pcref{Flexibility}).
                The defense of each creature is equal to the result of the attack it made with its \textit{maintain grapple} ability, or 0 if it did not use that ability.
                For each size category by which a creature is larger than you, it gains a \plus4 bonus to its defense against this attack.
                % TODO: awkward wording
                For each target, if you hit that target with this attack, it stops being grappled by you and you stop being grappled by it.
            \end{instantability}

            \parhead{Maintain Grapple}\label{Maintain Grapple} As a \glossterm{free action}, you can use the \textit{maintain grapple} ability to maintain a grapple that you are part of.
            If you do not use this ability while you are in a grapple, then creatures can easily escape the grapple with the \textit{escape grapple} ability.
            \begin{instantability}{Maintain Grapple}[Duration]
                \abilitytag{Swift}
                \rankline
                Make an attack using a \glossterm{free hand}.
                You may use half your level \add your Strength in place of your normal \glossterm{accuracy} with this attack.
                This attack has no immediate effect.
                The attack result determines how difficult it is for a creature to escape the grapple during the current round using the \textit{escape grapple} ability.
            \end{instantability}

        \subsubsection{Asymmetric Grappling}\label{Asymmetric Grappling}
            Normally, when you use the \textit{grapple} ability, both you and the target become grappled by each other.
            Some abilities allow you to grapple other creatures without becoming grappled yourself.
            You can release a creature that you are not grappled by as a \glossterm{free action}.
            If you do, the creatures stops being grappled by you.

\section{Vision and Light}\label{Vision and Light}
    Some creatures have \trait{darkvision} or other extraordinary senses, but most creatures need light to see by. 
    In an area of \glossterm{bright illumination}, all characters can see clearly.

    Creatures can see only dimly into areas that have \glossterm{shadowy illumination}.
    Everything in the area has \glossterm{concealment}.
    This allows creatures in the area to make Stealth checks to hide even if they don't have \glossterm{cover} (see \pcref{Stealth}).

    In an area with \glossterm{brilliant illumination}, creatures can see clearly just like an area with bright illumination.
    In addition, no shadows exist within an an area of brilliant illumination.
    This makes many effects from the \sphere{umbramancy} mystic sphere difficult or impossible to use.

    In areas of total darkness, creatures without \trait{darkvision} or some other form of supernatural vision are \blinded.

    \subsection{Attacking Unseen Foes}
        You can make \glossterm{targeted} attacks against creatures and objects you cannot see.
        To do so, you choose a 5-foot square and make the attack against that square.
        You have a 50\% \glossterm{miss chance} with the attack.
        Otherwise, you hit a random valid target in that square with your attack, if one exists.

    \subsection{Concealment}\label{Concealment}
        Concealment represents anything which makes it more difficult to see your target, such as \glossterm{shadowy illumination}.
        All \glossterm{targeted} attacks against a creature or object with concealment from you have a 25\% \glossterm{miss chance}.
        Generally, this means that you roll 1d4, and the attack misses on a 1.
        Determining concealment works similarly to determining cover.
        You must use the same \glossterm{points of origin} and \glossterm{target square} when determining concealment that you would use to determine cover.

        \parhead{Determining Concealment} There are two things that can cause a creature to be concealed: poor lighting, and intervening obstacles that block sight.
        Determining concealment from obstacles that block sight works the same way as determining cover (see \pcref{Cover}).

        Determining concealment from lighting conditions is simpler, since it ignores lighting conditions between you and the target.
        If your \glossterm{target square} is in lighting that provides concealment, the target has concealment.
        Otherwise, it does not.

\section{Obstacles and Cover}\label{Obstacles and Cover}
    In a battle, you may not be able to perfectly see all of your opponents.
    When obstacles get in the way, they may make some attacks impossible.
    Almost all abilities, including \glossterm{strikes}, must have \glossterm{line of sight} and \glossterm{line of effect}.
    Smaller obstacles may simply provide \glossterm{cover} instead of making attacks impossible.
    This section explains how to deal with obstacles and related limitations.

    \subsection{Point of Origin}\label{Point of Origin}
        When you make an attack, you have to determine the \glossterm{point of origin}.
        For \glossterm{targeted} attacks, which are the most common, the point or origin is a grid intersection of your choice that is touching your \glossterm{space}.
        For area attacks, the point of origin depends on the shape of the area and whether it has a defined \glossterm{range}.

        If an area attack has a defined range, the point of origin is a single grid intersection of your choice within that range.
        Cones, lines, and walls without a range use a grid intersection of your choice that is touching your space, just like targeted attacks.
        Cylinders and spheres without a range are unusual, since they radiate from your whole body instead of a single point.
        When determining their total size, treat every grid intersection touching your space as a point of origin.
        When determining cover and similar effects, only use the grid intersection that is closest to the target.

    \subsection{Cover}\label{Cover}

        Cover represents any obstacle that physically prevents you from striking your target, such as a tree or intervening creature.
        A creature or object behind cover gains a \plus2 bonus to Armor and Reflex defenses.
        If an attack misses the defense of a creature or object behind cover by no more than the defense bonus provided by the cover,
            the attack is applied to the obstacle instead of to the intended target.
        This can protect creatures behind cover from \glossterm{glancing blows} (see \pcref{Glancing Blow}).
        In addition, a creature behind cover can hide (see \pcref{Stealth}).

        Cover is only relevant if the attacker has \glossterm{line of effect} to its target (see \pcref{Line of Effect}).
        If you don't have line of effect, you generally can't attack the target at all, so the defense bonuses from cover don't matter.

        \subsubsection{Measuring Cover}
            To measure cover for a particular attack, draw a cone from the attack's point of origin to the two closest corners of the target's space.
            Note that these must be corners where the target's space ends, not just grid intersections touching the target's space.
            The defender can choose between equally distant corners.
            If there are any obstacles in that cone, the target has cover.

            Obstacles only provide cover if the relevant part of the obstacle is no more than one size category smaller than the target.
            You should ignore any irrelevant parts of the obstacle that are outside of the cone.
            For example, although a tree might be Gargantuan or Colossal if you include all of its leaves and branches, most trees are only a Medium size obstacle at ground level, since only their trunk is relevant.
            The rules typically ignore the complexity of three-dimensional space, so you'll have to estimate what would provide reasonable cover in some cases.

            \subsubsection{Improved Cover}
            Cover of the same type generally doesn't stack; a creature behind two trees is not substantially more protected than a creature behind a single tree.
            However, exceptionally well covered creatures, such as a creature behind an arrow slit in a castle, may gain a greater than normal benefit to defenses from cover at the GM's discretion.

    \subsection{Line of Sight}\label{Line of Sight}
        Unless otherwise noted in an ability's description, you cannot target a creature, object, or location that you do not have line of sight to.
        Line of sight measures whether you can see things, not whether you can touch or reach them.

        A line of sight is a straight, unblocked path between an attacker and a target.
        To measure line of sight for a particular attack, draw a line between any grid intersection touching your \glossterm{space} and any grid intersection touching the target's space.
        If you're targeting a particular point, you would naturally draw the line to that point instead.
        If this line is not blocked by any obstacles that impede sight, you have line of sight to your target.

    \subsection{Line of Effect}\label{Line of Effect}
        Almost all abilities, including \glossterm{strikes}, must have a \glossterm{line of effect} to function.
        Line of effect measures whether physical passage is possible between two locations, regardless of any sight obstacles.
        For example, a pane of glass would block line of effect, but not line of sight.

        Unless otherwise noted in an ability's description, you cannot target a creature, object, or location that you do not have line of effect to.
        In addition, abilities that affect an area do not affect targets that the ability does not have line of effect to.

        A line of effect is a straight, unblocked path between an attacker and a target.
        To measure line of sight for a particular attack, draw a line between the attack's \glossterm{point of origin} and any grid intersection touching the target's space.
        If you're targeting a particular point, you would naturally draw the line to that point instead.
        If this line is not blocked by any obstacles that make physical passage impossible, you have line of effect to your target.

        \subsubsection{Destroying Barriers}\label{Destroying Barriers}
            Some abilities deal damage to both creatures and objects.
            If a physical barrier is \glossterm{broken} by an ability, that barrier does not affect the ability's line of effect.
            For example, a thin curtain of silk normally blocks line of effect.
            However, an ability that destroyed the curtain would have its full effect on everything behind the curtain.

        \subsubsection{Inside Creatures}
            Creatures block line of effect to the inside of their own bodies.
            As a result, you cannot use an ability that takes effect inside a creature unless you are also inside the creature.
            This restriction applies even if there is no physical barrier to the inside of the creature.
            For example, you cannot place the \glossterm{point of origin} for an area inside a creature's mouth, even if the creature has its mouth open at the time.

\section{Awareness and Surprise}\label{Awareness and Surprise}
    In combat, creatures are sometimes not fully aware of danger, which makes them less able to defend against it.
    A creature can be described as either aware, \unaware, or \partiallyunaware of an attack against it.
    Normally, creatures are aware of all attacks against them in combat.
    This causes no special bonuses or penalties.

    Sometimes, creatures are fully \unaware that they are in danger from attack.
    This typically happens as a result of stealth, but it can also happen as a result of sudden treachery.
    A creature takes a \minus5 penalty to Armor and Reflex defenses against attacks that it is unaware of.
    After being attacked, an unaware creature typically stops being fully unaware of future attacks.

    A creature that knows that it is in danger and is attempting to defend itself, but does not know the exact location or nature of its attackers, is \partiallyunaware.
    For example, a creature that is already in combat that is attacked by a previously unseen foe is partially unaware of the attack.
    Similarly, a creature that just barely fails to beat an opponent's Stealth check may hear an ominous sound that makes it partially aware of danger without knowing the exact location of any attackers.

    \subsection{Surprise Attacks}\label{Surprise Attacks}
        Sometimes, creatures are not aware that combat is taking place when the combat starts.
        This most commonly happens with ambushes.
        Any creature that is not aware of the combat continues taking whatever actions it would normally be taking until it becomes aware of the combat.
        It is usually \unaware of all until that point, though unusually vigilant or perceptive creatures may be \partiallyunaware.

        If a surprise attack begins a combat, the creatures who initiate the attack can choose which phase to start in.
        Generally, this should start in \glossterm{delayed action phase}, though sometimes the \glossterm{movement phase} is more advantageous for the attackers.
        Starting a surprise attack during the \glossterm{action phase} is generally a bad idea because the attacked creatures may be able to take actions during the delayed action phase.

% TODO: This is a bad name; organize these better
\section{Special Combat Rules}

    \subsection{Unusual Combat Situations}

        \subsubsection{Mounted Combat}\label{Mounted Combat}
            \parhead{Horses in Combat} Warhorses and warponies can serve readily as combat steeds. Light horses, ponies, and heavy horses, however, are frightened by combat.
            At the start of each round, you must make a \glossterm{difficulty value} 10 Ride check to control such a horse.
            Success means you can act normally that round, directing the horse's movements as if it was trained for combat.
            Failure means that the horse acts of its own volition that round, usually fleeing in panic.

            \parhead{Space} A horse (not a pony) is a Large creature, and thus takes up a space 10 feet (2 squares) across. While mounted, you share your mount's space completely. Anyone who is close enough to hit your mount can attack either you or your mount. However, your \glossterm{reach} is still that of a creature of your normal size. Thus, a Medium paladin would be able to attack all squares adjacent to their Large horse with a longsword, and all squares 10 feet away from their mount with a lance.

            In the case of abnormally large mounts (two or more size categories larger than you), you may not completely share space. Such situations should be handled on a case-by-case basis, depending on the nature of the mount.

            \parhead{Flying Mounts} Flying mounts are harder to ride and control than terrestrial mounts, especially mounts that can change directions rapidly.
            The \glossterm{difficulty value} for all Ride checks on a mount using a fly speed is increased by 10 if the mount has poor or average maneuverablity, or by 15 if it has perfect maneuverability.

            \parhead{Combat while Mounted} With a \glossterm{difficulty value} 5 Ride check, you can guide your mount with your knees so as to use both hands to attack or defend yourself. This is a free action.

            If your mount is moving in the current phase, you take a \minus2 accuracy penalty with ranged strikes.
            If your mount uses the \textit{sprint} ability, this penalty increases to \minus4 (see \pcref{Sprint}).

            \parhead{If Your Mount Falls in Battle} If your mount falls, you fall to the ground with it.

            \parhead{If You Are Dropped} If you are knocked unconscious, you fall from your mount to the ground, which may cause you to take \glossterm{falling damage}.
            If you have a military saddle, you stay on your mount instead.
            In either case, the mount acts according to its nature.
            Most mounts flee combat without a rider.

    \subsection{Allies and Enemies}\label{Allies and Enemies}
        Each creature you interact with in Rise is either an \glossterm{ally}, an \glossterm{enemy}, or a \glossterm{neutral party}.
        Some beneficial abilities only affect allies, and some offensive abilities only affect enemies.

        You can choose how you consider each creature at the start of each \glossterm{phase}.
        You cannot consider yourself an \glossterm{ally} or an \glossterm{enemy}.
        While you are \unconscious, you treat all creatures as \glossterm{allies}.

        \parhead{Allies} An ally is any creature you consider an ally who also considers you an ally.
        If you consider someone an ally, but they do not consider you an ally, you treat them as a neutral party for the purpose of your abilities.
        Allies can move through your \glossterm{space}.

        \parhead{Enemies} An enemy is any creature who you consider to be an enemy.
        Enemies cannot move through your \glossterm{space}.

        \parhead{Neutral Parties} A neutral party is any creature who is neither an ally nor an enemy.
        You treat all creatures you have not declared an opinion of as neutral parties.
        Neutral parties can move through your \glossterm{space}.

    \subsection{Sleep and Fatigue}\label{Sleep and Fatigue}
        A typical creature needs a minimum of 6 hours of sleep for every 18 hours spent awake, and a minimum of 50 hours of sleep every week.
        You can stay awake beyond those limits with the Endurance skill (see \pcref{Stay Awake}).

    \subsection{Teleportation}\label{Teleportation}
        Some abilities can \glossterm{teleport} creatures or objects.
        When you are teleported, you move through the Astral Plane and arrive at a new location.
        You can be teleported between two different locations on the same \glossterm{plane}, or between two different locations on different planes.
        If for some reason you cannot access the Astral Plane, you cannot be teleported.

        Unless an ability explicitly teleports to other planes or specifies otherwise, anything being teleported must have both \glossterm{line of sight} and \glossterm{line of effect} to its destination.
        Otherwise, the teleportation fails without effect.

        \subsubsection{Teleportation Noise}\label{Teleportation Noise}
            Creatures and objects that are teleported make a sound when they depart and arrive.
            This noise is caused by the displacement of air (or other substances) created by the teleportation.
            The base \glossterm{difficulty value} of an Awareness check to hear this sound for a Medium creature or object is 10.
            This difficulty value changes based on the size of the teleported creature or object:

            \begin{itemize}
                \item Fine: 30
                \item Diminutive: 25
                \item Tiny: 20
                \item Small: 15
                \item Medium: 10
                \item Large: 5
                \item Huge: 0
                \item Gargantuan: \minus5
                \item Colossal: \minus10
            \end{itemize}

        \subsubsection{Carrying Objects}
            When a creature is teleported, it can bring along equipment and held objects as long as two conditions are met.
            First, the combined weight of the objects cannot exceed the creature's maximum \glossterm{carrying capacity} (see \pcref{Weight Limits}).
            If a creature is teleported while carrying more than its maximum carrying capacity, all excess objects are left behind, starting with the heaviest object and proceeding in order of weight.

            Second, no object can extend more than two feet away from the creature's body.
            Any objects that extend beyond that distance are left behind.
            For example, a creature wearing handcuffs will arrive at its teleportation destination still wearing the handcuffs.
            However, a creature that is tied to a post by a long rope will arrive at its teleportation destination without the rope.

        \subsubsection{Horizontal Teleportation}
            Some planes have a curved primary surface.
            On those planes, ``horizontal'' teleportation isn't objectively horizontal.
            Instead, it is horizontal relative to the surface of the plane.

    \subsection{Resolving Ambiguity}\label{Resolving Ambiguity}
        When the rules are ambiguous about how they apply to you and no other creature, you decide how to resolve that ambiguity.
        For example, if an ability causes you to remove one of your \glossterm{vital wounds}, and you have more than one vital wound, you choose which vital wound is removed.
        When the rules are ambiguous in any other situation, the GM decides how to resolve that ambiguity.
        This includes situations where multiple creatures are relevant and situations where no particular creature is relevant.
