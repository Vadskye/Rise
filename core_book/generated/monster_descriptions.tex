
        \section{Aberrations}
      
  \begin{monsection}{Aboleth}{12}[4]
    \vspace{-1em}\spelltwocol{}{Huge aberration}\vspace{-1em}
    \vspace{0em}

    
    The aboleth is a revolting fishlike amphibian found primarily in subterranean lakes and rivers.
    It has a pink belly.
    Four pulsating blueblack orifices line the bottom of its body and secrete gray slime that smells like rancid grease.
    It uses its tail for propulsion in the water and drags itself along with its tentacles on land.
    An aboleth weighs about 6,500 pounds.
  

    \begin{spellcontent}
      \begin{spelltargetinginfo}
        \pari \textbf{HP} 10;
          \textbf{AD} 20;
          \textbf{Fort} 25;
          \textbf{Ref} 19;
          \textbf{Ment} 26
        \pari \textbf{DR} Physical 20; Energy 29
        \pari \textbf{WR} Physical 60; Energy 69
        \pari \textbf{Strike:}
            Tentacle \plus17 (6d10)
      \end{spelltargetinginfo}
    \end{spellcontent}
    \begin{monsterfooter}
      \pari \textbf{Speed} 50 ft.;
        \textbf{Space} 15 ft.;
        \textbf{Reach} 15 ft.
      \pari \textbf{Awareness} \plus3
      \pari \textbf{Attributes}:
        Str 15, Dex -1,
        Con 15, Int 14,
        Per 7, Wil 15
      \pari \textbf{Accuracy} 17;
        \textbf{Power} 18
    \end{monsterfooter}
  \end{monsection}
  \begin{freeability}{Mind Crush}
      \target{One creature within \rnglong range}
      The aboleth makes a \plus17 attack
        vs. Mental against the target.
    
    \hit The target takes 8d10  damage and is \glossterm{confused} as a \glossterm{condition}.
    \crit 
          The aboleth can spend an action point to attune to this ability.
          If it does, the target is dominated by the aboleth as long as the ability lasts.
          Otherwise, the target takes double the damage of a non-critical hit.
    \end{freeability}
  

    \begin{freeability}{Psionic Blast}
      \targets{Each enemy in a \arealarge cone from the aboleth}
      The aboleth makes a \plus17 attack
        vs. Mental against each target.
    
    \hit Each target takes 6d10  damage and is \glossterm{dazed} as a \glossterm{condition}.
    \end{freeability}
  
      \parhead{Psionic barrier} The aboleth gains a bonus equal to its level to \glossterm{resistances} against \glossterm{energy damage}.
    \parhead{Rituals} The aboleth can learn and perform arcane rituals of up to 5th level.
    \parhead{Slime} 
        Whenever a creature takes damage from the aboleth's tentacle,
          if the attack result beat the target's Fortitude defense,
          the damaged creature becomes \glossterm{poisoned}.
        The poison's primary effect makes the target \glossterm{nauseated},
          and the secondary effect inflicts a \glossterm{vital wound} with a special effect and ends the poison.

        Instead of making a \glossterm{wound roll} for the \glossterm{vital wound},
          the target's skin is transformed into a clear, slimy membrane.
        An afflicted creature must remain moistened with cool, fresh water every 10 minutes
          or it will lose a \glossterm{hit point}.
        This effect lasts until the \glossterm{vital wound} is removed.
  
        \section{Animals}
      
  \begin{monsection}{Black Bear}{3}[2]
    \vspace{-1em}\spelltwocol{}{Medium animal}\vspace{-1em}
    \vspace{0em}

    
      The black bear is a forest-dwelling omnivore that usually is not dangerous unless an interloper threatens its cubs or food supply.

      Black bears can be pure black, blond, or cinnamon in color and are rarely more than 5 feet long.
    

    \begin{spellcontent}
      \begin{spelltargetinginfo}
        \pari \textbf{HP} 9;
          \textbf{AD} 7;
          \textbf{Fort} 12;
          \textbf{Ref} 8;
          \textbf{Ment} 6
        \pari \textbf{DR} 4
        \pari \textbf{WR} 21
        \pari \textbf{Strike:}
            Bite \plus4 (2d8)
      \end{spelltargetinginfo}
    \end{spellcontent}
    \begin{monsterfooter}
      \pari \textbf{Speed} 30 ft.;
        \textbf{Space} 5 ft.;
        \textbf{Reach} 5 ft.
      \pari \textbf{Awareness} \plus0
      \pari \textbf{Attributes}:
        Str 5, Dex 0,
        Con 5, Int -8,
        Per 0, Wil -1
      \pari \textbf{Accuracy} 4;
        \textbf{Mundane Power} 6;
      \textbf{Magical Power} 4
    \end{monsterfooter}
  \end{monsection}
  
  
  \begin{monsection}{Brown Bear}{5}[2]
    \vspace{-1em}\spelltwocol{}{Large animal}\vspace{-1em}
    \vspace{0em}

    
      These massive omnivores weigh more than 1,800 pounds and stand nearly 9 feet tall when they rear up on their hind legs.
      They are bad-tempered and territorial.
      The brown bear’s statistics can be used for almost any big bear, including the grizzly.
    

    \begin{spellcontent}
      \begin{spelltargetinginfo}
        \pari \textbf{HP} 10;
          \textbf{AD} 10;
          \textbf{Fort} 16;
          \textbf{Ref} 11;
          \textbf{Ment} 9
        \pari \textbf{DR} 6
        \pari \textbf{WR} 27
        \pari \textbf{Strike:}
            Bite \plus7 (2d10)
      \end{spelltargetinginfo}
    \end{spellcontent}
    \begin{monsterfooter}
      \pari \textbf{Speed} 40 ft.;
        \textbf{Space} 10 ft.;
        \textbf{Reach} 10 ft.
      \pari \textbf{Awareness} \plus0
      \pari \textbf{Attributes}:
        Str 8, Dex 0,
        Con 8, Int -8,
        Per 0, Wil -1
      \pari \textbf{Accuracy} 7;
        \textbf{Mundane Power} 9;
      \textbf{Magical Power} 6
    \end{monsterfooter}
  \end{monsection}
  
  
  \begin{monsection}{Colossal Centipede}{13}[4]
    \vspace{-1em}\spelltwocol{}{Colossal animal}\vspace{-1em}
    \vspace{0em}

    
    Monstrous centipedes tend to attack anything that resembles food, biting with their jaws and injecting their poison.
  

    \begin{spellcontent}
      \begin{spelltargetinginfo}
        \pari \textbf{HP} 11;
          \textbf{AD} 21;
          \textbf{Fort} 28;
          \textbf{Ref} 21;
          \textbf{Ment} 21
        \pari \textbf{DR} Physical 22; Energy 19
        \pari \textbf{WR} Physical 67; Energy 64
        \pari \textbf{Strike:}
            Bite \plus18 (7d10)
      \end{spelltargetinginfo}
    \end{spellcontent}
    \begin{monsterfooter}
      \pari \textbf{Speed} 70 ft.;
        \textbf{Space} 30 ft.;
        \textbf{Reach} 30 ft.
      \pari \textbf{Awareness} \plus0
      \pari \textbf{Attributes}:
        Str 17, Dex -1,
        Con 17, Int -9,
        Per 0, Wil 0
      \pari \textbf{Accuracy} 18;
        \textbf{Mundane Power} 20;
      \textbf{Magical Power} 16
    \end{monsterfooter}
  \end{monsection}
  \parhead{Poison sting} 
    Whenever a creature takes damage from the colossal centipede's bite,
      if the attack result beat the target's Fortitude defense,
      the damaged creature becomes \glossterm{poisoned}.
    The poison's primary effect makes the target lose a \glossterm{hit point}, and the terminal effect makes it lose two \glossterm{hit points}
  
  \begin{monsection}{Dire Wolf}{5}[2]
    \vspace{-1em}\spelltwocol{}{Large animal}\vspace{-1em}
    \vspace{0em}

    
      Dire wolves are efficient pack hunters that will kill anything they can catch.

      Dire wolves are mottled gray or black, about 9 feet long and weighing some 800 pounds.
    

    \begin{spellcontent}
      \begin{spelltargetinginfo}
        \pari \textbf{HP} 8;
          \textbf{AD} 13;
          \textbf{Fort} 14;
          \textbf{Ref} 14;
          \textbf{Ment} 10
        \pari \textbf{DR} 6
        \pari \textbf{WR} 27
        \pari \textbf{Strike:}
            Bite \plus8 (2d10)
      \end{spelltargetinginfo}
    \end{spellcontent}
    \begin{monsterfooter}
      \pari \textbf{Speed} 40 ft.;
        \textbf{Space} 10 ft.;
        \textbf{Reach} 10 ft.
      \pari \textbf{Awareness} \plus3
      \pari \textbf{Attributes}:
        Str 8, Dex 7,
        Con 6, Int -7,
        Per 6, Wil 0
      \pari \textbf{Accuracy} 8;
        \textbf{Mundane Power} 9;
      \textbf{Magical Power} 6
    \end{monsterfooter}
  \end{monsection}
  
  
  \begin{monsection}{Gargantuan Centipede}{10}[4]
    \vspace{-1em}\spelltwocol{}{Gargantuan animal}\vspace{-1em}
    \vspace{0em}

    
    Monstrous centipedes tend to attack anything that resembles food, biting with their jaws and injecting their poison.
  

    \begin{spellcontent}
      \begin{spelltargetinginfo}
        \pari \textbf{HP} 10;
          \textbf{AD} 17;
          \textbf{Fort} 23;
          \textbf{Ref} 17;
          \textbf{Ment} 17
        \pari \textbf{DR} Physical 16; Energy 13
        \pari \textbf{WR} Physical 49; Energy 46
        \pari \textbf{Strike:}
            Bite \plus14 (5d10)
      \end{spelltargetinginfo}
    \end{spellcontent}
    \begin{monsterfooter}
      \pari \textbf{Speed} 60 ft.;
        \textbf{Space} 20 ft.;
        \textbf{Reach} 20 ft.
      \pari \textbf{Awareness} \plus0
      \pari \textbf{Attributes}:
        Str 13, Dex -1,
        Con 13, Int -9,
        Per 0, Wil 0
      \pari \textbf{Accuracy} 14;
        \textbf{Mundane Power} 16;
      \textbf{Magical Power} 13
    \end{monsterfooter}
  \end{monsection}
  \parhead{Poison sting} 
    Whenever a creature takes damage from the gargantuan centipede's bite,
      if the attack result beat the target's Fortitude defense,
      the damaged creature becomes \glossterm{poisoned}.
    The poison's primary effect makes the target lose a \glossterm{hit point}, and the terminal effect makes it lose two \glossterm{hit points}
  
  \begin{monsection}{Giant Bombardier Beetle}{7}[2]
    \vspace{-1em}\spelltwocol{}{Large animal}\vspace{-1em}
    \vspace{0em}

    
      These creatures feed primarily on carrion and offal, gathering heaps of the stuff in which to build nests and lay eggs.
      A giant bombardier beetle is about 6 feet long. Giant bombardier beetles normally attack only to defend themselves, their nests, or their eggs.
    

    \begin{spellcontent}
      \begin{spelltargetinginfo}
        \pari \textbf{HP} 10;
          \textbf{AD} 12;
          \textbf{Fort} 18;
          \textbf{Ref} 12;
          \textbf{Ment} 12
        \pari \textbf{DR} Physical 11; Energy 8
        \pari \textbf{WR} Physical 36; Energy 33
        \pari \textbf{Strike:}
            Bite \plus9 (4d6)
      \end{spelltargetinginfo}
    \end{spellcontent}
    \begin{monsterfooter}
      \pari \textbf{Speed} 40 ft.;
        \textbf{Space} 10 ft.;
        \textbf{Reach} 10 ft.
      \pari \textbf{Awareness} \plus0
      \pari \textbf{Attributes}:
        Str 10, Dex -1,
        Con 10, Int -9,
        Per 0, Wil 0
      \pari \textbf{Accuracy} 9;
        \textbf{Mundane Power} 11;
      \textbf{Magical Power} 8
    \end{monsterfooter}
  \end{monsection}
  \begin{freeability}{Acid Breath}
      \targets{Everything in a \areamed cone}
      The giant bombardier beetle makes a \plus9 attack
        vs. Armor against each target.
    
    \hit Each target takes 4d6 acid damage.
    \end{freeability}
  
  \begin{monsection}{Giant Wasp}{6}[2]
    \vspace{-1em}\spelltwocol{}{Large animal}\vspace{-1em}
    \vspace{0em}

    
      Giant wasps attack when hungry or threatened, stinging their prey to death.
      They take dead or incapacitated opponents back to their lairs as food for their unhatched young.
    

    \begin{spellcontent}
      \begin{spelltargetinginfo}
        \pari \textbf{HP} 7;
          \textbf{AD} 16;
          \textbf{Fort} 14;
          \textbf{Ref} 16;
          \textbf{Ment} 9
        \pari \textbf{DR} Physical 10; Energy 7
        \pari \textbf{WR} Physical 33; Energy 30
        \pari \textbf{Strike:}
            Stinger \plus9 (2d10)
      \end{spelltargetinginfo}
    \end{spellcontent}
    \begin{monsterfooter}
      \pari \textbf{Speed} 40 ft.;
        \textbf{Space} 10 ft.;
        \textbf{Reach} 10 ft.
      \pari \textbf{Awareness} \plus3
      \pari \textbf{Attributes}:
        Str 4, Dex 9,
        Con 4, Int -8,
        Per 7, Wil -2
      \pari \textbf{Accuracy} 9;
        \textbf{Power} 7
    \end{monsterfooter}
  \end{monsection}
  \parhead{Poison sting} 
    Whenever a creature takes damage from the giant wasp's stinger,
      if the attack result beat the target's Fortitude defense,
      the damaged creature becomes \glossterm{poisoned}.
    The poison's primary effect makes the target \glossterm{sickened}, and the terminal effect makes it \glossterm{paralyzed}
  
  \begin{monsection}{Huge Centipede}{7}[4]
    \vspace{-1em}\spelltwocol{}{Huge animal}\vspace{-1em}
    \vspace{0em}

    
    Monstrous centipedes tend to attack anything that resembles food, biting with their jaws and injecting their poison.
  

    \begin{spellcontent}
      \begin{spelltargetinginfo}
        \pari \textbf{HP} 9;
          \textbf{AD} 14;
          \textbf{Fort} 19;
          \textbf{Ref} 14;
          \textbf{Ment} 14
        \pari \textbf{DR} Physical 11; Energy 8
        \pari \textbf{WR} Physical 36; Energy 33
        \pari \textbf{Strike:}
            Bite \plus11 (4d8)
      \end{spelltargetinginfo}
    \end{spellcontent}
    \begin{monsterfooter}
      \pari \textbf{Speed} 50 ft.;
        \textbf{Space} 15 ft.;
        \textbf{Reach} 15 ft.
      \pari \textbf{Awareness} \plus0
      \pari \textbf{Attributes}:
        Str 9, Dex -1,
        Con 9, Int -9,
        Per 0, Wil 0
      \pari \textbf{Accuracy} 11;
        \textbf{Mundane Power} 12;
      \textbf{Magical Power} 10
    \end{monsterfooter}
  \end{monsection}
  \parhead{Poison sting} 
    Whenever a creature takes damage from the huge centipede's bite,
      if the attack result beat the target's Fortitude defense,
      the damaged creature becomes \glossterm{poisoned}.
    The poison's primary effect makes the target lose a \glossterm{hit point}, and the terminal effect makes it lose two \glossterm{hit points}
  
  \begin{monsection}{Large Centipede}{4}[4]
    \vspace{-1em}\spelltwocol{}{Large animal}\vspace{-1em}
    \vspace{0em}

    
    Monstrous centipedes tend to attack anything that resembles food, biting with their jaws and injecting their poison.
  

    \begin{spellcontent}
      \begin{spelltargetinginfo}
        \pari \textbf{HP} 8;
          \textbf{AD} 10;
          \textbf{Fort} 14;
          \textbf{Ref} 10;
          \textbf{Ment} 10
        \pari \textbf{DR} Physical 8; Energy 5
        \pari \textbf{WR} Physical 27; Energy 24
        \pari \textbf{Strike:}
            Bite \plus7 (2d10)
      \end{spelltargetinginfo}
    \end{spellcontent}
    \begin{monsterfooter}
      \pari \textbf{Speed} 40 ft.;
        \textbf{Space} 10 ft.;
        \textbf{Reach} 10 ft.
      \pari \textbf{Awareness} \plus0
      \pari \textbf{Attributes}:
        Str 5, Dex -1,
        Con 5, Int -9,
        Per 0, Wil 0
      \pari \textbf{Accuracy} 7;
        \textbf{Mundane Power} 8;
      \textbf{Magical Power} 7
    \end{monsterfooter}
  \end{monsection}
  \parhead{Poison sting} 
    Whenever a creature takes damage from the large centipede's bite,
      if the attack result beat the target's Fortitude defense,
      the damaged creature becomes \glossterm{poisoned}.
    The poison's primary effect makes the target lose a \glossterm{hit point}, and the terminal effect makes it lose two \glossterm{hit points}
  
  \begin{monsection}{Pony}{2}[1]
    \vspace{-1em}\spelltwocol{}{Medium animal}\vspace{-1em}
    \vspace{0em}

    
      The statistics presented here describe a small horse, under 5 feet tall at the shoulder.
      Ponies are similar to light horses and cannot fight while carrying a rider.
      A pony's maximum load is 120 pounds, and it can drag up to 800 pounds.
    

    \begin{spellcontent}
      \begin{spelltargetinginfo}
        \pari \textbf{HP} 4;
          \textbf{AD} 5;
          \textbf{Fort} 10;
          \textbf{Ref} 6;
          \textbf{Ment} 4
        \pari \textbf{DR} Physical 1; Energy 0
        \pari \textbf{WR} Physical 17; Energy 16
        \pari \textbf{Strike:}
            Bite \plus2 (1d10)
      \end{spelltargetinginfo}
    \end{spellcontent}
    \begin{monsterfooter}
      \pari \textbf{Speed} 30 ft.;
        \textbf{Space} 5 ft.;
        \textbf{Reach} 5 ft.
      \pari \textbf{Awareness} \plus0
      \pari \textbf{Attributes}:
        Str 2, Dex 0,
        Con 4, Int -7,
        Per 0, Wil -1
      \pari \textbf{Accuracy} 2;
        \textbf{Power} 2
    \end{monsterfooter}
  \end{monsection}
  
  
  \begin{monsection}{Roc}{9}[4]
    \vspace{-1em}\spelltwocol{}{Gargantuan animal}\vspace{-1em}
    \vspace{0em}

    
      Rocs are massive and incredibly strong birds with the ability to carry off horses.
      It is typically 30 feet long from the beak to the base of the tail, with a wingspan as wide as 80 feet.
      A roc weighs about 8,000 pounds.
      Its plumage is either dark brown or golden from head to tail.

      A roc attacks from the air, swooping earthward to snatch prey in its powerful talons and carry it off for itself and its young to devour.
      A solitary roc is typically hunting and will attack any Medium or larger creature that appears edible.
      A mated pair of rocs attack in concert, fighting to the death to defend their nests or hatchlings.
    

    \begin{spellcontent}
      \begin{spelltargetinginfo}
        \pari \textbf{HP} 9;
          \textbf{AD} 18;
          \textbf{Fort} 21;
          \textbf{Ref} 19;
          \textbf{Ment} 15
        \pari \textbf{DR} 8
        \pari \textbf{WR} 38
        \pari \textbf{Strike:}
            Bite \plus13 (4d10)
      \end{spelltargetinginfo}
    \end{spellcontent}
    \begin{monsterfooter}
      \pari \textbf{Speed} 60 ft.;
        \textbf{Space} 20 ft.;
        \textbf{Reach} 20 ft.
      \pari \textbf{Awareness} \plus0
      \pari \textbf{Attributes}:
        Str 12, Dex 10,
        Con 11, Int -7,
        Per 0, Wil -1
      \pari \textbf{Accuracy} 13;
        \textbf{Mundane Power} 15;
      \textbf{Magical Power} 12
    \end{monsterfooter}
  \end{monsection}
  
  
  \begin{monsection}{Spider}[Colossal]{12}[4]
    \vspace{-1em}\spelltwocol{}{Colossal animal}\vspace{-1em}
    \vspace{0em}

    
    All monstrous spiders are aggressive predators that use their poisonous bites to subdue or kill prey.
  

    \begin{spellcontent}
      \begin{spelltargetinginfo}
        \pari \textbf{HP} 6;
          \textbf{AD} 25;
          \textbf{Fort} 22;
          \textbf{Ref} 25;
          \textbf{Ment} 20
        \pari \textbf{DR} Physical 20; Energy 17
        \pari \textbf{WR} Physical 60; Energy 57
        \pari \textbf{Strike:}
            Bite \plus20 (6d10)
      \end{spelltargetinginfo}
    \end{spellcontent}
    \begin{monsterfooter}
      \pari \textbf{Speed} 70 ft.;
        \textbf{Space} 30 ft.;
        \textbf{Reach} 30 ft.
      \pari \textbf{Awareness} \plus7
      \pari \textbf{Attributes}:
        Str 15, Dex 15,
        Con 0, Int -9,
        Per 15, Wil 0
      \pari \textbf{Accuracy} 20;
        \textbf{Mundane Power} 18;
      \textbf{Magical Power} 15
    \end{monsterfooter}
  \end{monsection}
  \parhead{Poison sting} 
    Whenever a creature takes damage from the colossal spider's bite,
      if the attack result beat the target's Fortitude defense,
      the damaged creature becomes \glossterm{poisoned}.
    The poison's primary effect makes the target \glossterm{sickened}, and the terminal effect makes it \glossterm{paralyzed}
  
  \begin{monsection}{Spider}[Gargantuan]{9}[4]
    \vspace{-1em}\spelltwocol{}{Gargantuan animal}\vspace{-1em}
    \vspace{0em}

    
    All monstrous spiders are aggressive predators that use their poisonous bites to subdue or kill prey.
  

    \begin{spellcontent}
      \begin{spelltargetinginfo}
        \pari \textbf{HP} 6;
          \textbf{AD} 21;
          \textbf{Fort} 18;
          \textbf{Ref} 21;
          \textbf{Ment} 16
        \pari \textbf{DR} Physical 14; Energy 11
        \pari \textbf{WR} Physical 44; Energy 41
        \pari \textbf{Strike:}
            Bite \plus15 (4d10)
      \end{spelltargetinginfo}
    \end{spellcontent}
    \begin{monsterfooter}
      \pari \textbf{Speed} 60 ft.;
        \textbf{Space} 20 ft.;
        \textbf{Reach} 20 ft.
      \pari \textbf{Awareness} \plus5
      \pari \textbf{Attributes}:
        Str 11, Dex 12,
        Con 0, Int -9,
        Per 11, Wil 0
      \pari \textbf{Accuracy} 15;
        \textbf{Mundane Power} 14;
      \textbf{Magical Power} 12
    \end{monsterfooter}
  \end{monsection}
  \parhead{Poison sting} 
    Whenever a creature takes damage from the gargantuan spider's bite,
      if the attack result beat the target's Fortitude defense,
      the damaged creature becomes \glossterm{poisoned}.
    The poison's primary effect makes the target \glossterm{sickened}, and the terminal effect makes it \glossterm{paralyzed}
  
  \begin{monsection}{Spider}[Huge]{6}[4]
    \vspace{-1em}\spelltwocol{}{Huge animal}\vspace{-1em}
    \vspace{0em}

    
    All monstrous spiders are aggressive predators that use their poisonous bites to subdue or kill prey.
  

    \begin{spellcontent}
      \begin{spelltargetinginfo}
        \pari \textbf{HP} 6;
          \textbf{AD} 17;
          \textbf{Fort} 15;
          \textbf{Ref} 17;
          \textbf{Ment} 13
        \pari \textbf{DR} Physical 10; Energy 7
        \pari \textbf{WR} Physical 33; Energy 30
        \pari \textbf{Strike:}
            Bite \plus12 (4d6)
      \end{spelltargetinginfo}
    \end{spellcontent}
    \begin{monsterfooter}
      \pari \textbf{Speed} 50 ft.;
        \textbf{Space} 15 ft.;
        \textbf{Reach} 15 ft.
      \pari \textbf{Awareness} \plus4
      \pari \textbf{Attributes}:
        Str 7, Dex 8,
        Con 0, Int -9,
        Per 8, Wil 0
      \pari \textbf{Accuracy} 12;
        \textbf{Mundane Power} 10;
      \textbf{Magical Power} 9
    \end{monsterfooter}
  \end{monsection}
  \parhead{Poison sting} 
    Whenever a creature takes damage from the huge spider's bite,
      if the attack result beat the target's Fortitude defense,
      the damaged creature becomes \glossterm{poisoned}.
    The poison's primary effect makes the target \glossterm{sickened}, and the terminal effect makes it \glossterm{paralyzed}
  
  \begin{monsection}{Spider}[Large]{3}[4]
    \vspace{-1em}\spelltwocol{}{Large animal}\vspace{-1em}
    \vspace{0em}

    
    All monstrous spiders are aggressive predators that use their poisonous bites to subdue or kill prey.
  

    \begin{spellcontent}
      \begin{spelltargetinginfo}
        \pari \textbf{HP} 6;
          \textbf{AD} 13;
          \textbf{Fort} 11;
          \textbf{Ref} 13;
          \textbf{Ment} 9
        \pari \textbf{DR} Physical 7; Energy 4
        \pari \textbf{WR} Physical 24; Energy 21
        \pari \textbf{Strike:}
            Bite \plus7 (2d8)
      \end{spelltargetinginfo}
    \end{spellcontent}
    \begin{monsterfooter}
      \pari \textbf{Speed} 40 ft.;
        \textbf{Space} 10 ft.;
        \textbf{Reach} 10 ft.
      \pari \textbf{Awareness} \plus2
      \pari \textbf{Attributes}:
        Str 2, Dex 5,
        Con 0, Int -9,
        Per 4, Wil 0
      \pari \textbf{Accuracy} 7;
        \textbf{Power} 6
    \end{monsterfooter}
  \end{monsection}
  \parhead{Poison sting} 
    Whenever a creature takes damage from the large spider's bite,
      if the attack result beat the target's Fortitude defense,
      the damaged creature becomes \glossterm{poisoned}.
    The poison's primary effect makes the target \glossterm{sickened}, and the terminal effect makes it \glossterm{paralyzed}
  
  \begin{monsection}{Vampire Eel}{6}[2]
    \vspace{-1em}\spelltwocol{}{Large animal}\vspace{-1em}
    \vspace{0em}

    
      Vampire eels are large, slimy snakelike carnivores.
      They swim through murky water, looking for edible creatures.
    

    \begin{spellcontent}
      \begin{spelltargetinginfo}
        \pari \textbf{HP} 8;
          \textbf{AD} 15;
          \textbf{Fort} 15;
          \textbf{Ref} 15;
          \textbf{Ment} 10
        \pari \textbf{DR} Physical 8; Energy 6
        \pari \textbf{WR} Physical 31; Energy 29
        
      \end{spelltargetinginfo}
    \end{spellcontent}
    \begin{monsterfooter}
      \pari \textbf{Speed} 40 ft.;
        \textbf{Space} 10 ft.;
        \textbf{Reach} 10 ft.
      \pari \textbf{Awareness} \plus0
      \pari \textbf{Attributes}:
        Str 8, Dex 8,
        Con 7, Int -8,
        Per 0, Wil -1
      \pari \textbf{Accuracy} 8;
        \textbf{Mundane Power} 9;
      \textbf{Magical Power} 7
    \end{monsterfooter}
  \end{monsection}
  
  
  \begin{monsection}{Wolf}{2}[1]
    \vspace{-1em}\spelltwocol{}{Medium animal}\vspace{-1em}
    \vspace{0em}

    
      Wolves are pack hunters known for their persistence and cunning.
    

    \begin{spellcontent}
      \begin{spelltargetinginfo}
        \pari \textbf{HP} 3;
          \textbf{AD} 7;
          \textbf{Fort} 8;
          \textbf{Ref} 8;
          \textbf{Ment} 4
        \pari \textbf{DR} 3
        \pari \textbf{WR} 19
        \pari \textbf{Strike:}
            Bite \plus2 (1d10)
      \end{spelltargetinginfo}
    \end{spellcontent}
    \begin{monsterfooter}
      \pari \textbf{Speed} 30 ft.;
        \textbf{Space} 5 ft.;
        \textbf{Reach} 5 ft.
      \pari \textbf{Awareness} \plus0
      \pari \textbf{Attributes}:
        Str 2, Dex 3,
        Con 2, Int -7,
        Per 0, Wil -1
      \pari \textbf{Accuracy} 2;
        \textbf{Power} 2
    \end{monsterfooter}
  \end{monsection}
  
  
  \begin{monsection}{Raven}{1}[1]
    \vspace{-1em}\spelltwocol{}{Tiny animal}\vspace{-1em}
    \vspace{0em}

    
      These glossy black birds are about 2 feet long and have wingspans of about 4 feet.
      The statistics presented here can describe most nonpredatory birds of similar size.
    

    \begin{spellcontent}
      \begin{spelltargetinginfo}
        \pari \textbf{HP} 1;
          \textbf{AD} 7;
          \textbf{Fort} 2;
          \textbf{Ref} 8;
          \textbf{Ment} 4
        \pari \textbf{DR} 0
        \pari \textbf{WR} 14
        \pari \textbf{Strike:}
            Talon \plus2 (1d8)
      \end{spelltargetinginfo}
    \end{spellcontent}
    \begin{monsterfooter}
      \pari \textbf{Speed} 20 ft.;
        \textbf{Space} 2-1/2 ft.;
        \textbf{Reach} 0 ft.
      \pari \textbf{Awareness} \plus1
      \pari \textbf{Attributes}:
        Str -8, Dex 3,
        Con -4, Int -6,
        Per 2, Wil 0
      \pari \textbf{Accuracy} 2;
        \textbf{Power} 1
    \end{monsterfooter}
  \end{monsection}
  
  
        \section{Animates}
      
  \begin{monsection}{Air Elemental}[Elder]{11}[4]
    \vspace{-1em}\spelltwocol{}{Huge animate}\vspace{-1em}
    \vspace{0em}

    
    Air elementals are an embodiment of the natural element of air.
    Their rapid flying speed makes them useful on vast battlefields or in extended aerial combat.
  

    \begin{spellcontent}
      \begin{spelltargetinginfo}
        \pari \textbf{HP} 7;
          \textbf{AD} 24;
          \textbf{Fort} 21;
          \textbf{Ref} 25;
          \textbf{Ment} 20
        \pari \textbf{DR} Physical 13; Energy 12
        \pari \textbf{WR} Physical 49; Energy 48
        \pari \textbf{Strike:}
            Slam \plus18 (4d10)
      \end{spelltargetinginfo}
    \end{spellcontent}
    \begin{monsterfooter}
      \pari \textbf{Speed} 50 ft.;
        \textbf{Space} 15 ft.;
        \textbf{Reach} 15 ft.
      \pari \textbf{Awareness} \plus6
      \pari \textbf{Attributes}:
        Str 6, Dex 15,
        Con 6, Int 0,
        Per 13, Wil 0
      \pari \textbf{Accuracy} 18;
        \textbf{Power} 14
    \end{monsterfooter}
  \end{monsection}
  \begin{freeability}{Whirlwind}
      \targets{Each \glossterm{enemy} within reach}
      The air elemental makes a \plus18
        \glossterm{strike} vs. Armor
        with its slam against each target.
    
    \hit Each target takes 4d10  damage.
    \end{freeability}
  
  \begin{monsection}{Air Elemental}[Huge]{8}[2]
    \vspace{-1em}\spelltwocol{}{Huge animate}\vspace{-1em}
    \vspace{0em}

    
    Air elementals are an embodiment of the natural element of air.
    Their rapid flying speed makes them useful on vast battlefields or in extended aerial combat.
  

    \begin{spellcontent}
      \begin{spelltargetinginfo}
        \pari \textbf{HP} 7;
          \textbf{AD} 17;
          \textbf{Fort} 15;
          \textbf{Ref} 18;
          \textbf{Ment} 14
        \pari \textbf{DR} Physical 7; Energy 6
        \pari \textbf{WR} Physical 35; Energy 34
        \pari \textbf{Strike:}
            Slam \plus11 (2d10)
      \end{spelltargetinginfo}
    \end{spellcontent}
    \begin{monsterfooter}
      \pari \textbf{Speed} 50 ft.;
        \textbf{Space} 15 ft.;
        \textbf{Reach} 15 ft.
      \pari \textbf{Awareness} \plus4
      \pari \textbf{Attributes}:
        Str 5, Dex 11,
        Con 5, Int -2,
        Per 9, Wil 0
      \pari \textbf{Accuracy} 11;
        \textbf{Power} 9
    \end{monsterfooter}
  \end{monsection}
  \begin{freeability}{Whirlwind}
      \targets{Each \glossterm{enemy} within reach}
      The air elemental makes a \plus11
        \glossterm{strike} vs. Armor
        with its slam against each target.
    
    \hit Each target takes 2d10  damage.
    \end{freeability}
  
  \begin{monsection}{Air Elemental}[Large]{5}[2]
    \vspace{-1em}\spelltwocol{}{Large animate}\vspace{-1em}
    \vspace{0em}

    
    Air elementals are an embodiment of the natural element of air.
    Their rapid flying speed makes them useful on vast battlefields or in extended aerial combat.
  

    \begin{spellcontent}
      \begin{spelltargetinginfo}
        \pari \textbf{HP} 7;
          \textbf{AD} 14;
          \textbf{Fort} 12;
          \textbf{Ref} 15;
          \textbf{Ment} 11
        \pari \textbf{DR} Physical 4; Energy 3
        \pari \textbf{WR} Physical 25; Energy 24
        \pari \textbf{Strike:}
            Slam \plus8 (2d8)
      \end{spelltargetinginfo}
    \end{spellcontent}
    \begin{monsterfooter}
      \pari \textbf{Speed} 40 ft.;
        \textbf{Space} 10 ft.;
        \textbf{Reach} 10 ft.
      \pari \textbf{Awareness} \plus3
      \pari \textbf{Attributes}:
        Str 3, Dex 8,
        Con 3, Int -2,
        Per 6, Wil 0
      \pari \textbf{Accuracy} 8;
        \textbf{Power} 6
    \end{monsterfooter}
  \end{monsection}
  \begin{freeability}{Whirlwind}
      \targets{Each \glossterm{enemy} within reach}
      The air elemental makes a \plus8
        \glossterm{strike} vs. Armor
        with its slam against each target.
    
    \hit Each target takes 2d8  damage.
    \end{freeability}
  
        \section{Humanoids}
      
  \begin{monsection}{Cultist}{2}[1]
    \vspace{-1em}\spelltwocol{}{Medium humanoid}\vspace{-1em}
    \vspace{0em}

    
      Cultists may serve many masters.
      They are united in their generally malign intentions and their magical abilities.
    

    \begin{spellcontent}
      \begin{spelltargetinginfo}
        \pari \textbf{HP} 3;
          \textbf{AD} 2;
          \textbf{Fort} 6;
          \textbf{Ref} 6;
          \textbf{Ment} 8
        \pari \textbf{DR} 0
        \pari \textbf{WR} 16
        \pari \textbf{Strike:}
            Club \plus2 (1d10)
      \end{spelltargetinginfo}
    \end{spellcontent}
    \begin{monsterfooter}
      \pari \textbf{Speed} 30 ft.;
        \textbf{Space} 5 ft.;
        \textbf{Reach} 5 ft.
      \pari \textbf{Awareness} \plus0
      \pari \textbf{Attributes}:
        Str -1, Dex 0,
        Con 0, Int -1,
        Per 0, Wil 3
      \pari \textbf{Accuracy} 2;
        \textbf{Mundane Power} 2;
      \textbf{Magical Power} 3
    \end{monsterfooter}
  \end{monsection}
  \begin{freeability}{Drain Life}
      \target{One creature within \rngmed range}
      The cultist makes a \plus2 attack
        vs. Fortitude against the target.
    
    \hit The target loses a \glossterm{hit point}
    \end{freeability}
  
  \begin{monsection}{Lizardfolk}[Elite]{10}[2]
    \vspace{-1em}\spelltwocol{}{Medium humanoid}\vspace{-1em}
    \vspace{0em}

    
      Lizardfolk are usually 6 to 7 feet tall with green, gray, or brown scales.
      Their tail is used for balance and is 3 to 4 feet long.
      They can weigh from 200 to 250 pounds.

      Lizardfolk fight as unorganized individuals.
      They prefer frontal assaults and massed rushes, sometimes trying to force foes into the water, where the lizardfolk have an advantage.
      If outnumbered or if their territory is being invaded, they set snares, plan ambushes, and make raids to hinder enemy supplies.
      Advanced tribes use more sophisticated tactics and have better traps and ambushes.
    

    \begin{spellcontent}
      \begin{spelltargetinginfo}
        \pari \textbf{HP} 8;
          \textbf{AD} 20;
          \textbf{Fort} 18;
          \textbf{Ref} 16;
          \textbf{Ment} 16
        \pari \textbf{DR} Physical 19; Energy 10
        \pari \textbf{WR} Physical 52; Energy 43
        \pari \textbf{Strike:}
            Spear \plus12 (4d8)
      \end{spelltargetinginfo}
    \end{spellcontent}
    \begin{monsterfooter}
      \pari \textbf{Speed} 30 ft.;
        \textbf{Space} 5 ft.;
        \textbf{Reach} 5 ft.
      \pari \textbf{Awareness} \plus0
      \pari \textbf{Attributes}:
        Str 12, Dex 0,
        Con 11, Int 0,
        Per 0, Wil 0
      \pari \textbf{Accuracy} 12;
        \textbf{Mundane Power} 13;
      \textbf{Magical Power} 11
    \end{monsterfooter}
  \end{monsection}
  \parhead{Hold breath} A lizardfolk can hold its breath for ten times the normal length of time
  
  \begin{monsection}{Lizardfolk}[Grunt]{10}[1]
    \vspace{-1em}\spelltwocol{}{Medium humanoid}\vspace{-1em}
    \vspace{0em}

    
      Lizardfolk are usually 6 to 7 feet tall with green, gray, or brown scales.
      Their tail is used for balance and is 3 to 4 feet long.
      They can weigh from 200 to 250 pounds.
    

    \begin{spellcontent}
      \begin{spelltargetinginfo}
        \pari \textbf{HP} 4;
          \textbf{AD} 19;
          \textbf{Fort} 17;
          \textbf{Ref} 15;
          \textbf{Ment} 15
        \pari \textbf{DR} Physical 19; Energy 10
        \pari \textbf{WR} Physical 52; Energy 43
        \pari \textbf{Strike:}
            Spear \plus11 (4d6)
      \end{spelltargetinginfo}
    \end{spellcontent}
    \begin{monsterfooter}
      \pari \textbf{Speed} 30 ft.;
        \textbf{Space} 5 ft.;
        \textbf{Reach} 5 ft.
      \pari \textbf{Awareness} \plus0
      \pari \textbf{Attributes}:
        Str 11, Dex 0,
        Con 11, Int 0,
        Per 0, Wil 0
      \pari \textbf{Accuracy} 11;
        \textbf{Mundane Power} 11;
      \textbf{Magical Power} 10
    \end{monsterfooter}
  \end{monsection}
  \parhead{Hold breath} A lizardfolk can hold its breath for ten times the normal length of time
  
  \begin{monsection}{Orc}[Elite]{8}[1]
    \vspace{-1em}\spelltwocol{}{Medium humanoid}\vspace{-1em}
    \vspace{0em}

    
      Elite orcs are battle-hardened war veterans who are deadly at any range.
      They tend to prioritize raw strength over subtlety.
    

    \begin{spellcontent}
      \begin{spelltargetinginfo}
        \pari \textbf{HP} 4;
          \textbf{AD} 12;
          \textbf{Fort} 15;
          \textbf{Ref} 13;
          \textbf{Ment} 12
        \pari \textbf{DR} Physical 13; Energy 6
        \pari \textbf{WR} Physical 41; Energy 34
        \pari \textbf{Strikes:}
            Light Crossbow \plus9 (4d6); Greataxe \plus9 (4d8)
      \end{spelltargetinginfo}
    \end{spellcontent}
    \begin{monsterfooter}
      \pari \textbf{Speed} 30 ft.;
        \textbf{Space} 5 ft.;
        \textbf{Reach} 5 ft.
      \pari \textbf{Awareness} \plus0
      \pari \textbf{Attributes}:
        Str 11, Dex 0,
        Con 9, Int -2,
        Per 0, Wil -1
      \pari \textbf{Accuracy} 9;
        \textbf{Mundane Power} 11;
      \textbf{Magical Power} 8
    \end{monsterfooter}
  \end{monsection}
  \begin{freeability}{Power Smash}
      \target{One creature or object within \glossterm{reach}}
      The orc makes a \plus7
        \glossterm{strike} vs. Armor
        with its greataxe against the target.
    
    \hit The target takes 5d10  damage.
    \end{freeability}
  
  \begin{monsection}{Pyromancer}{5}[1]
    \vspace{-1em}\spelltwocol{}{Medium humanoid}\vspace{-1em}
    \vspace{0em}

    
      Pyromancers wield powerful fire magic to attack their foes.
    

    \begin{spellcontent}
      \begin{spelltargetinginfo}
        \pari \textbf{HP} 3;
          \textbf{AD} 6;
          \textbf{Fort} 11;
          \textbf{Ref} 10;
          \textbf{Ment} 12
        \pari \textbf{DR} 3
        \pari \textbf{WR} 24
        \pari \textbf{Strike:}
            Club \plus6 (2d6)
      \end{spelltargetinginfo}
    \end{spellcontent}
    \begin{monsterfooter}
      \pari \textbf{Speed} 30 ft.;
        \textbf{Space} 5 ft.;
        \textbf{Reach} 5 ft.
      \pari \textbf{Awareness} \plus0
      \pari \textbf{Attributes}:
        Str -1, Dex 0,
        Con 3, Int -1,
        Per 0, Wil 6
      \pari \textbf{Accuracy} 6;
        \textbf{Mundane Power} 5;
      \textbf{Magical Power} 6
    \end{monsterfooter}
  \end{monsection}
  \begin{freeability}{Combustion}
      \target{One creature or object within \rngmed range}
      The pyromancer makes a \plus6 attack
        vs. Reflex against the target.
    
    \hit The target takes 4d6 fire damage.
    \end{freeability}
  

    \begin{freeability}{Fireball}
      \targets{Everything in a \areasmall radius within \rngclose range}
      The pyromancer makes a \plus6 attack
        vs. Armor against each target.
    
    \hit Each target takes 2d8 fire damage.
    \end{freeability}
  
        \section{Monstrous Humanoids}
      
  \begin{monsection}{Giant}[Hill]{11}[2]
    \vspace{-1em}\spelltwocol{}{Huge monstrous humanoid}\vspace{-1em}
    \vspace{0em}

    
      Giants relish melee combat.
      They favor massive two-handed weapons and wield them with impressive skill.
      They have enough cunning to soften up a foe with ranged attacks first, if they can.
      A giant's favorite ranged weapon is a big rock.
    

    \begin{spellcontent}
      \begin{spelltargetinginfo}
        \pari \textbf{HP} 9;
          \textbf{AD} 15;
          \textbf{Fort} 21;
          \textbf{Ref} 16;
          \textbf{Ment} 16
        \pari \textbf{DR} Physical 20; Energy 12
        \pari \textbf{WR} Physical 56; Energy 48
        \pari \textbf{Strike:}
            Greatclub \plus12 (6d10)
      \end{spelltargetinginfo}
    \end{spellcontent}
    \begin{monsterfooter}
      \pari \textbf{Speed} 50 ft.;
        \textbf{Space} 15 ft.;
        \textbf{Reach} 15 ft.
      \pari \textbf{Awareness} \minus1
      \pari \textbf{Attributes}:
        Str 16, Dex -2,
        Con 13, Int 0,
        Per -2, Wil -2
      \pari \textbf{Accuracy} 12;
        \textbf{Mundane Power} 17;
      \textbf{Magical Power} 12
    \end{monsterfooter}
  \end{monsection}
  
  
        \section{Outsiders}
      
        \section{Undeads}
      
  \begin{monsection}{Allip}{3}[4]
    \vspace{-1em}\spelltwocol{}{Medium undead}\vspace{-1em}
    \vspace{0em}

    
      An allip is the spectral remains of someone driven to suicide by a madness that afflicted it in life.
      It craves only revenge and unrelentingly pursues those who tormented it in life and pushed it over the brink.

      An allip cannot speak intelligibly.
    

    \begin{spellcontent}
      \begin{spelltargetinginfo}
        \pari \textbf{HP} 6;
          \textbf{AD} 9;
          \textbf{Fort} 9;
          \textbf{Ref} 13;
          \textbf{Ment} 13
        \pari \textbf{DR} 1
        \pari \textbf{WR} 18
        
      \end{spelltargetinginfo}
    \end{spellcontent}
    \begin{monsterfooter}
      \pari \textbf{Speed} 30 ft.;
        \textbf{Space} 5 ft.;
        \textbf{Reach} 5 ft.
      \pari \textbf{Awareness} \plus7
      \pari \textbf{Attributes}:
        Str N/A, Dex 5,
        Con N/A, Int 2,
        Per 2, Wil 4
      \pari \textbf{Accuracy} 6;
        \textbf{Mundane Power} 6;
      \textbf{Magical Power} 7
    \end{monsterfooter}
  \end{monsection}
  \begin{freeability}{Draining Touch}
      \target{One creature the allip \glossterm{threatens}}
      The allip makes a \plus6 attack
        vs. Reflex against the target.
    
    \hit The target loses a \glossterm{hit point}.
    \crit The target loses two \glossterm{hit points}.
    \end{freeability}
  
      \parhead{Babble} 
          During each \glossterm{action phase}, the allip makes an attack vs. Mental against each creature
          within an \arealarge radius \glossterm{emanation} from it.
          It cannot make this attack more than once against any individual target between \glossterm{short rests}.
          \hit Each target is \glossterm{confused} as a \glossterm{condition}.
        
    \parhead{Incorporeal} 
      The allip has no physical body.
      It makes no sound while moving, and may be unaffected by other abilities that only affect corporeal creatures, such as \glossterm{tremorsense}.
      It is immune to \glossterm{physical} damage and all \glossterm{mundane} abilities that do not deal damage.
      Whenever it would take damage, it has a 50\% chance to take no damage instead.

      The allip can enter or pass through solid objects, but it must remain adjacent to the object's exterior at all times.
      While completely inside a solid object, the object provides \glossterm{total cover}, so it must emerge from the object to attack or be attacked.
  