\section{Cantrip Descriptions}

\begin{spellsection}{Acid Orb}
    \begin{spellheader}
        \spelldesc{You conjure a small orb of acid out of nothingness and propel it towards your foe.}
    \end{spellheader}
    \begin{spellcontent}
        \begin{spelltargetinginfo}
            \spelltwocol{\spelltgt{One creature or object}}{\spellrng{\rngclose}}
        \end{spelltargetinginfo}
        \begin{spelleffects}
            \begin{spellattack}{Spellpower vs. Reflex}
                \spellsuccess \spelldamage{acid}[d6].
                \spellfailure Half damage.
            \end{spellattack}
        \end{spelleffects}
    \end{spellcontent}
    \begin{spellfooter}
        \spellschool{Conjuration [Acid, Creation, Physical]}
        \physicalspellnotes
    \end{spellfooter}
\end{spellsection}

\begin{spellsection}{Augment Attack}
    \begin{spellheader}
        \spelldesc{You imbue an ally with magical energy, making its next attack more powerful.}
    \end{spellheader}
    \begin{spellcontent}
        \begin{spelltargetinginfo}
            \spelltwocol{\spelltgt{One willing creature}}{\spellrng{\rngclose}}
        \end{spelltargetinginfo}
        \begin{spelleffects}
            \spelleffect The next time the target makes a physical attack, the attack deals bonus damage if it hits.
            The bonus damage is equal to \spelldamage{}[d8].
            \spelldur 1 round.
        \end{spelleffects}
    \end{spellcontent}
    \begin{spellfooter}
        \spellschool{Transmutation [Augment]}
        \spellnotes The creature struck by the enhanced attack can apply spell resistance to avoid taking the bonus damage.
    \end{spellfooter}
\end{spellsection}

\begin{spellsection}{Combat Telekinesis}
    \begin{spellheader}
        \spelldesc{You telekinetically control a weapon and use it to attack.}
    \end{spellheader}
    \begin{spellcontent}
        \begin{spelltargetinginfo}
            \spelltgt{One unattended weapon (Tiny or smaller)}
            \spellrng{\rngclose}
            \spelltime{Swift action}
        \end{spelltargetinginfo}
        \begin{spelleffects}
            \spelleffect You can use the target weapon to attack from its location.
            This functions as if you were attacking with the weapon in your hands, except that you must use your spellpower to determine your attack and damage bonus.
            You cannot use your Strength, Dexterity, or combat prowess to attack with the target weapon.
            In addition, you cannot use any magical properties of the weapon.

            You contribute to overwhelm penalties and threaten enemies from both your location and the weapon's location.
            If you take a \glossterm{standard attack} action, you can attack with your own hands, with the weapon, or both, as you choose.
            The weapon's physical defenses are equal to 10 \add half your spellpower.

            During the movement phase, you can move the weapon up to 30 feet in any direction, including vertically.
            If the weapon goes outside of the spell's range, you lose control of it and it falls to the ground.

            \spelldur 1 round. You can use a swift action at the beginning of each round to maintain control of the weapon.
        \end{spelleffects}
    \end{spellcontent}
    \begin{spellfooter}
        \spellschool{Evocation [Telekinesis]}
    \end{spellfooter}
\end{spellsection}

\begin{spellsection}[Lesser]{Confusion}
    \begin{spellheader}
        \spelldesc{You compel a creature to act randomly, sowing confusion in your foes' ranks.}
    \end{spellheader}
    \begin{spellcontent}
        \begin{spelltargetinginfo}
            \spelltwocol{\spelltgt{One creature}}{\spellrng{\rngclose}}
        \end{spelltargetinginfo}
        \begin{spelleffects}
            \begin{spellattack}{Spellpower vs. Mental}
                \spellsuccess The target is \disoriented.
                \spellcritical The target is \confused.
            \end{spellattack}
            \spelldur 1 round.
        \end{spelleffects}
    \end{spellcontent}
    \begin{spellfooter}
        \spellschool{Enchantment [Compulsion, Mind]}
        \spellnotes \norepeatspellnotes
    \end{spellfooter}
\end{spellsection}

\begin{spellsection}{Conjure Projectile}
    \begin{spellheader}
        \spelldesc{You create a small arrow and fire it at your foe.}
    \end{spellheader}
    \begin{spellcontent}
        \begin{spelltargetinginfo}
            \spelltwocol{\spelltgt{One creature or object}}{\spellrng{\rngmed}}
        \end{spelltargetinginfo}
        \begin{spelleffects}
            \begin{spellattack}{Spellpower vs. Armor defense}
                \spellsuccess \spelldamage{piercing}[d8]
            \end{spellattack}
        \end{spelleffects}
    \end{spellcontent}
    \begin{spellfooter}
        \spellschool{Conjuration [Creation, Physical]}
        \spellnotes At the end of the spell's duration, the projectile conjured disappears without a trace.

        \physicalspellnotes
    \end{spellfooter}
\end{spellsection}

\begin{spellsection}[Lesser]{Displacement}
    \begin{spellheader}
        \spelldesc{You briefly shift your ally's image, causing it to appear to be about 1 foot away from its true location.}
    \end{spellheader}
    \begin{spellcontent}
        \begin{spelltargetinginfo}
            \spelltwocol{\spelltgt{One creature}}{\spellrng{\rngclose}}
        \end{spelltargetinginfo}
        \begin{spelleffects}
            \spelleffect Targeted physical attacks against the target have a 20\% miss chance. Spells and other special attacks suffer no miss chance.
            \spelldur 1 round
        \end{spelleffects}
    \end{spellcontent}
    \begin{spellfooter}
        \spellinfo{Illusion [Glamer]}{Arcane}
    \end{spellfooter}
\end{spellsection}

\begin{spellsection}{Draining Touch}
    \begin{spellheader}
    \end{spellheader}
    \begin{spellcontent}
        \begin{spelltargetinginfo}
            \spelltwocol{\spelltgt{One living creature}}{\spellrng{5 ft.}}
        \end{spelltargetinginfo}
        \begin{spelleffects}
            \spelleffect \spelldamage{life}[d6]. You gain temporary hit points equal to half the damage you deal. You can't gain more hit points than the target had.

            The temporary hit points disappear after 5 rounds. If you take life damage, you lose all temporary hit points provided by this spell before applying the damage.
        \end{spelleffects}
    \end{spellcontent}
    \begin{spellfooter}
        \spellschool{Vivimancy [Life]}
    \end{spellfooter}
\end{spellsection}

\begin{spellsection}{Exhaustion}
    \begin{spellheader}
        \spelldesc{You momentarily weaken your foe's body.}
    \end{spellheader}
    \begin{spellcontent}
        \begin{spelltargetinginfo}
            \spellquicktargeting{One creature}{\rngclose}
        \end{spelltargetinginfo}
        \begin{spelleffects}
            \begin{spellattack}{Spellpower vs. Fortitude}
                \spellsuccess The target is \fatigued for 5 rounds.
                \spellcritical The target is \exhausted for 5 rounds.
            \end{spellattack}
        \end{spelleffects}
    \end{spellcontent}
    \begin{spellfooter}
        \spellschool{Vivimancy [Flesh]}
    \end{spellfooter}
\end{spellsection}

\begin{spellsection}[Lesser]{Fear}
    \begin{spellheader}
        \spelldesc{You terrify your foe.}
    \end{spellheader}
    \begin{spellcontent}
        \begin{spelltargetinginfo}
            \spelltwocol{\spelltgt{One creature}}{\spellrng{\rngmed}}
        \end{spelltargetinginfo}
        \begin{spelleffects}
            \begin{spellattack}{Spellpower vs. Mental}
                \spellcritical The target is \frightened by you.
                \spellsuccess The target is \shaken by you.
            \end{spellattack}
            \spelldur \durshort \dismissable
        \end{spelleffects}
    \end{spellcontent}
    \begin{spellfooter}
        \spellinfo{Enchantment [Mind]}{Arcane}
    \end{spellfooter}
\end{spellsection}

\begin{spellsection}{Flare}
    \begin{spellheader}
        \spelldesc{You create a burst of bright light in a foe's eyes, impairing its vision.}
    \end{spellheader}
    \begin{spellcontent}
        \begin{spelltargetinginfo}
            \spelltwocol{\spelltgt{One creature}}{\spellrng{\rngclose}}
        \end{spelltargetinginfo}
        \begin{spelleffects}
            \begin{spellattack}{Spellpower vs. Reflex}
                \spellsuccess The target's vision is \impaired for 5 rounds. This affects all sight-related actions, including physical attacks and targeted spells.
            \end{spellattack}
        \end{spelleffects}
    \end{spellcontent}
    \begin{spellfooter}
        \spellschool{Illusion [Light]}
    \end{spellfooter}
\end{spellsection}

\begin{spellsection}{Glimpse the Future}[1]
    \begin{spellheader}
    \end{spellheader}
    \begin{spellcontent}
        \begin{spelltargetinginfo}
            \spellquicktargeting{One willing creature}{\rngclose}
        \end{spelltargetinginfo}
        \begin{spelleffects}
            \spelleffect The target gains an offensive legend point.
            \spelldur 1 round.
        \end{spelleffects}
    \end{spellcontent}
    \begin{spellfooter}
        \spellschool{Divination}
    \end{spellfooter}
\end{spellsection}

\begin{spellsection}{Magic Ray}
    \begin{spellheader}
        \spelldesc{You fire a ray of magical energy at your foe.}
    \end{spellheader}
    \begin{spellcontent}
        \begin{spelltargetinginfo}
            \spelltwocol{\spelltgt{One creature}}{\spellrng{\rngclose}}
        \end{spelltargetinginfo}
        \begin{spelleffects}
            \begin{spellattack}{Spellpower vs. Reflex}
                \spellsuccess \spelldamage{force}[d6].
                \spellfailure Half damage.
            \end{spellattack}
        \end{spelleffects}
    \end{spellcontent}
    \begin{spellfooter}
        \spellschool{Evocation [Force]}
        \spellnotes \forcespellnotes
    \end{spellfooter}
\end{spellsection}

\begin{spellsection}{Resist Damage}
    \begin{spellheader}
        \spelldesc{You surround an ally with a faint yellow barrier that partially shields it from incoming damage.}
    \end{spellheader}
    \begin{spellcontent}
        \begin{spelltargetinginfo}
            \spelltwocol{\spelltgt{One creature}}{\spellrng{\rngclose}}
        \end{spelltargetinginfo}
        \begin{spelleffects}
            \spellspecial Choose a physical damage type: slashing, piercing, or bludgeoning.
            \spelleffect The target gains damage reduction against the chosen type of physical damage equal to your spellpower. Physical damage of other types ignores this damage reduction and negates it for 1 round.
            \spelldur 1 round.
        \end{spelleffects}
    \end{spellcontent}
    \begin{spellfooter}
        \spellschool{Abjuration [Shielding]}
    \end{spellfooter}
\end{spellsection}

\begin{spellsection}[Lesser]{Slow}
    \begin{spellheader}
        \spelldesc{You briefly decelerate your enemy's motions, causing it to move and act more slowly than normal.}
    \end{spellheader}
    \begin{spellcontent}
        \begin{spelltargetinginfo}
            \spelltwocol{\spelltgt{One creature}}{\spellrng{\rngclose}}
        \end{spelltargetinginfo}
        \begin{spelleffects}
            \begin{spellattack}{Spellpower vs. Mental}
                \spellsuccess The target is \slowed.
                \spellfailure The target moves at half speed.
            \end{spellattack}
            \spelldur 1 round.
        \end{spelleffects}
    \end{spellcontent}
    \begin{spellfooter}
        \spellinfo{Transmutation [Temporal]}{Arcane}
    \end{spellfooter}
\end{spellsection}

\begin{spellsection}{Twist Fate}
    \begin{spellheader}
    \end{spellheader}
    \begin{spellcontent}
        \begin{spelltargetinginfo}
            \spelltwocol{\spelltgt{One creature}}{\spellrng{\rngclose}}
        \end{spelltargetinginfo}
        \begin{spelleffects}
            \begin{spellattack}{Spellpower vs. Mental}
                \spellsuccess You know what the subject is most likely going to do during the next round.
                \spellcritical As above, but after gaining that knowledge, you can impose a \minus4 penalty to the target's accuracy, defenses, or checks for 1 round.
            \end{spellattack}
        \end{spelleffects}
    \end{spellcontent}
    \begin{spellfooter}
        \spellinfo{Divination}{Arcane}
    \end{spellfooter}
\end{spellsection}
