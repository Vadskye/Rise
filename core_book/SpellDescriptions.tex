\section{Spell Descriptions}

\small

\pdfbookmark[2]{A}{SpellDescriptionsA}

\begin{spellsection}{Ablative Shield}[2]
    \begin{spellheader}
        \spelldesc{You instantly reduce the power of an incoming attack.}
    \end{spellheader}
    \begin{spellcontent}
        \begin{spelltargetinginfo}
            \spelltgt{You}
            \spelltwocol{\spelltime{Immediate action}}{\spellcmp{Verbal only}}
        \end{spelltargetinginfo}
        \begin{spelleffects}
            \spelleffect You gain damage reduction against physical damage equal to your spellpower.
            Arcane damage ignores this damage reduction and negates it for 1 round.
            \spelldur Until end of round
        \end{spelleffects}
    \end{spellcontent}
    \begin{spellfooter}
        \spellinfo{Abjuration [Shielding]}{Arcane}
        \spellnotes After casting this spell, you cannot cast it again for 5 rounds.
        You can cast this spell in response to an opponent attacking you, before the attack is rolled.
        \miscastexplode
    \end{spellfooter}
    \begin{spellaugments}
        \spellaugment{1}{Complete}{The damage reduction applies against all damage, not just physical damage.}
        % \spellaugment{4}{Empowered}{The damage reduction is equal to twice your spellpower.}
    \end{spellaugments}
\end{spellsection}

\begin{spellsection}{Acid Arrow}[3]
    \begin{spellheader}
        \spelldesc{You fire a magical arrow of acid from your hand that speeds to its target.}
    \end{spellheader}
    \begin{spellcontent}
        \begin{spelltargetinginfo}
            \spellquicktargeting{One creature or object}{\rnglong}
        \end{spelltargetinginfo}
        \begin{spelleffects}
            \begin{spellattack}{Spellpower vs. Reflex}
                \spellsuccess 1d8 acid damage per two spellpower immediately, and again at the end of the next round.
                \spellcritical As above, except that the acid persists for 5 rounds, dealing damage at the end of each round.
                In addition, the target is \sickened for 1 round each time it takes damage.
            \end{spellattack}
        \end{spelleffects}
    \end{spellcontent}
    \begin{spellfooter}
        \spellinfo{Conjuration [Acid, Creation]}{Arcane}
        \spellnotes If the target becomes submerged in water or is affected by a cold, fire, or water effect, it takes no secondary damage.

        \physicalspellnotes
        \miscastrandom
    \end{spellfooter}
    \begin{spellaugments}
        \spellaugment{3}{Staggering}{The target is \staggered for 1 round each time it takes damage.}
        \spellaugment{4}{Empowered}{The damage increases to 1d10 acid damage per two spellpower immediately, and an additional 1d10 damage per round.}
    \end{spellaugments}
\end{spellsection}

\begin{spellsection}{Acid Splash}[1]
    \begin{spellheader}
        \spelldesc{You throw a magical sphere of acid from your hand that speeds to its target.}
    \end{spellheader}
    \begin{spellcontent}
        \begin{spelltargetinginfo}
            \spellquicktargeting{One creature or object}{\rngclose}
        \end{spelltargetinginfo}
        \begin{spelleffects}
            \begin{spellattack}{Spellpower vs. Reflex}
                \spellsuccess \spelldamage{acid}.
                \spellcritical Double damage.
                \spellfailure Half damage.
            \end{spellattack}
        \end{spelleffects}
    \end{spellcontent}
    \begin{spellfooter}
        \spellinfo{Conjuration [Acid, Creation]}{Arcane}
        \spellnotes \physicalspellnotes
        \miscastrandom
    \end{spellfooter}
    \begin{spellaugments}
        \spellaugment{4}{Empowered}{The damage increases to \spelldamageemp{acid}}
    \end{spellaugments}
\end{spellsection}

\begin{spellsection}{Agony}[3]
    \begin{spellheader}
        \spelldesc{You inflict debilitating pain on your foe.}
    \end{spellheader}
    \begin{spellcontent}
        \begin{spelltargetinginfo}
            \spellquicktargeting{One creature}{\rngmed}
        \end{spelltargetinginfo}
        \begin{spelleffects}
            \begin{spellattack}{Spellpower vs. Mental}
                \spellsuccess Whenever the target takes physical damage, it takes additional damage equal to your spellpower.
                \spellcritical As above, and the target immediately takes \spelldamage{physical}. The target takes additional damage equal to your spellpower from this effect, just like other physical damage.
                \spellfailure The target is \sickened.
            \end{spellattack}
            \spelldur \durbrief
        \end{spelleffects}
    \end{spellcontent}
    \begin{spellfooter}
        \spellinfo{Enchantment [Delusion, Mind]}{Arcane, Divine}
        \miscastrandom
    \end{spellfooter}
    \begin{spellaugments}
        \spellaugment{2}{Complete}{The additional damage applies to all damage, not just physical damage.}
        \spellaugment{3}{Mass}{The spell can affect up to five targets.}
    \end{spellaugments}
\end{spellsection}

\begin{spellsection}{Aid}[1]
    \begin{spellheader}
        \spelldesc{You fill your ally with confidence, improving his resilience in combat.}
    \end{spellheader}
    \begin{spellcontent}
        \begin{spelltargetinginfo}
            \spellquicktargeting{One creature}{\rngclose}
        \end{spelltargetinginfo}
        \begin{spelleffects}
            \spelleffect The target gains temporary hit points equal to twice your spellpower.
            If the target takes life damage, it loses all temporary hit points provided by this spell before applying the damage.
            \spelldur \durpersonallong
        \end{spelleffects}
    \end{spellcontent}
    \begin{spellfooter}
        \spellinfo{Enchantment [Delusion, Mind]}{Divine, Good}
        \miscastexplode
    \end{spellfooter}
    \begin{spellaugments}
        \spellaugment{2}{Heroic}{The target also gains an offensive legend point.}
        \spellaugment{3}{Mass}{The spell can affect up to five targets.}
        \spellaugment{4}{Empowered}{The temporary hit points granted by this spell increase to three times your spellpower.}
    \end{spellaugments}
\end{spellsection}

\begin{spellsection}{Air Walk}[4]
    \begin{spellheader}
        \spelldesc{You imbue an ally with the ability to walk on nothing but air.}
    \end{spellheader}
    \begin{spellcontent}
        \begin{spelltargetinginfo}
            \spellquicktargeting{One creature (Gargantuan size or smaller)}{\rngtouch}
        \end{spelltargetinginfo}
        \begin{spelleffects}
            \spelleffect The target can walk on air as if it were solid ground. The magic only affects the target's legs, and does not grant the ability to climb vertically through the air.
            \par If the spell ends while the target is still aloft, the magic fails slowly. The target floats downward 60 feet per round for 1d6 rounds. If it reaches the ground in that amount of time, it lands safely. If not, it falls the rest of the distance, taking falling damage if appropriate.
            \spelldur \durshort
        \end{spelleffects}
    \end{spellcontent}
    \begin{spellfooter}
        \spellinfo{Transmutation [Air, Imbuement]}{Air, Divine, Nature, Travel}
        \miscastexplode
    \end{spellfooter}
\end{spellsection}

\begin{spellsection}{Antilife Shell}[7]
    \begin{spellheader}
        \spelldesc{You create an immobile, spherical energy field that hedges out living creatures.}
    \end{spellheader}
    \begin{spellcontent}
        \begin{spelltargetinginfo}
            \spellzone{\areasmall radius centered on you}
        \end{spelltargetinginfo}
        \begin{spelleffects}
            \spelleffect Living creatures cannot enter the spell's area. Nonliving creatures, such as constructs and undead, suffer no ill effect.
            \spelldur \durmed \dismissable
        \end{spelleffects}
    \end{spellcontent}
    \begin{spellfooter}
        \spellinfo{Vivimancy}{Divine, Nature}
        \spellnotes Creatures in the area at the time that the spell is cast are unaffected by the spell.
        \miscastexplode
    \end{spellfooter}
    \begin{spellaugments}
        \spellaugment{2}{Widened}{The spell's area becomes a \areamed radius.}
    \end{spellaugments}
\end{spellsection}

\begin{spellsection}{Antimagic Field}[7]
    \begin{spellheader}
        \spelldesc{You create a mobile, spherical energy field that suppresses magic.}
    \end{spellheader}
    \begin{spellcontent}
        \begin{spelltargetinginfo}
            \spellemanation{\areasmall radius from you}
        \end{spelltargetinginfo}
        \begin{spelleffects}
            \spelleffect All magical abilities and objects fail to function in the area. They cannot be activated from within the field, and any existing effects brought into or cast into the area are suppressed. Time spent within an \spell{antimagic field} counts against a suppressed ability's duration.
            \par Creatures within an \spell{antimagic field} cannot concentrate on or dismiss spells. However, you can concentrate on and dismiss your own \spell{antimagic field}.
            \spelldur \durshort
        \end{spelleffects}
    \end{spellcontent}
    \begin{spellfooter}
        \spellinfo{Abjuration [Thaumaturgy]}{Arcane, Divine, Magic}
        \spellnotes The effects of instantaneous conjurations, such as \spell{create water}, are not affected by this spell because the conjuration itself is no longer in effect, only its result.

        \par \spell{Dispel magic} and similar magic has no effect on an \spell{antimagic field}. Two or more \spellindirect{antimagicfield}{antimagic fields} sharing any of the same space have no effect on each other.
    %\par Any part of a creature that lies outside the field is unaffected by the field.
        \par Artifacts and deities are unaffected by mortal magic such as this.
        \miscastexplode
    \end{spellfooter}
    \begin{spellaugments}
        \spellaugment{2}{Widened}{The spell's area becomes a \areamed radius.}
    \end{spellaugments}
\end{spellsection}

\begin{spellsection}{Animate Corpse}[1]
    \begin{spellheader}
        \spelldesc{You infuse a recently deceased corse with life force, giving it a semblance of life.}
    \end{spellheader}
    \begin{spellcontent}
        \begin{spelltargetinginfo}
            \spellquicktargeting{One Large or smaller corpse}{\rngclose}
        \end{spelltargetinginfo}
        \begin{spelleffects}
            \spelleffect The target corpse animates and fights for you.
            It does not have any special abilities the creature had in life, with the exception of its natural weapons.
            The creature is only capable of making unarmed attacks, or attacking with its natural weapons.

            Most of the corpse's defenses and attributes are equal to the creature's original defenses and attributes.
            However, its Strength is equal to your spellpower, and it has no Intelligence or Willpower.
            In addition, its Mental defense is 0.
            The corpse's hit points are based on its defenses and level, as normal.

            At the start of each round, you must spend a swift action to control the animated corpse.
            If you do, you control its actions that round.
            You can mentally command it to attack your enemies, follow you, or stay in place.
            More complex commands are not possible.
            If you do not control the creature, it falls prone and is inanimate that round.

            \spelldur \durshort \dismissable
        \end{spelleffects}
    \end{spellcontent}
    \begin{spellfooter}
        \spellinfo{Vivimancy [Life, Soul]}{Arcane}
        \miscastrandom
    \end{spellfooter}
\end{spellsection}

\begin{spellsection}{Aquatic Sphere}[5]
    \begin{spellheader}
        \spelldesc{You surround yourself in a sphere of water, hindering your foes.}
    \end{spellheader}
    \begin{spellcontent}
        \begin{spelltargetinginfo}
            \spellemanation{\areamed radius from you}
        \end{spelltargetinginfo}
        \begin{spelleffects}
            \spelleffect A mobile sphere of water fills the area.
            Creatures in the area suffer penalties appropriate for fighting underwater, and may be unable to breathe.
            In addition, you suffer no penalties for fighting underwater within the area, and can breathe this water as if it was air.
            \spelldur \durshort
        \end{spelleffects}
    \end{spellcontent}
    \begin{spellfooter}
        \spellinfo{Conjuration [Creation, Water]}{Nature, Water}
        \spellnotes If a 5-foot section of the sphere is dealt 50 points of fire damage, it boils away for 1 round.
        This creates a cloud of fog, as the \spell{fog cloud} spell, in a 10 foot radius around the destroyed area that also lasts for 1 round.
        \miscastexplode
    \end{spellfooter}
    \begin{spellaugments}
        \spellaugment{1}{Fixed}{The spell's area becomes a zone rather than an emanation. The water stays in place once the spell is cast, and does not move with you.}
        \spellaugment{2}{Widened}{The spell's area becomes a \arealarge radius.}
    \end{spellaugments}
\end{spellsection}

\begin{spellsection}{Aqueous Blade}[3]
    \begin{spellheader}
        \spelldesc{You transform the active part of a weapon into water, weakening its blows but allowing it penetrate defenses more easily.}
    \end{spellheader}
    \begin{spellcontent}
        \begin{spelltargetinginfo}
            \spellquicktargeting{One weapon}{\rngclose}
        \end{spelltargetinginfo}
        \begin{spelleffects}
            \begin{spellattack}{Spellpower vs. Mental}
                \spellsuccess Attacks with the affected weapon are made against Reflex defense instead of Armor defense. However, damage with the weapon is halved, including any bonuses to damage.
            \end{spellattack}
            \spelldur \durshort
        \end{spelleffects}
    \end{spellcontent}
    \begin{spellfooter}
        \spellinfo{Transmutation [Shaping, Water]}{Nature, War, Water}
        \miscastrandom
    \end{spellfooter}
    \begin{spellaugments}
        \spellaugment{3}{Empowered}{Damage with the weapon is not halved.}
    \end{spellaugments}
\end{spellsection}

\begin{spellsection}{Assimilate}[9]
    \begin{spellheader}
        \spelldesc{Your pointing finger turns black as obsidian. You touch a creature and it dissolves into dust as you assimilate its form into your own body.}
    \end{spellheader}
    \begin{spellcontent}
        \begin{spelltargetinginfo}
            \spellquicktargeting{One living creature}{\rngclose}
        \end{spelltargetinginfo}
        \begin{spelleffects}
            \begin{spellattack}{Spellpower vs. Fortitude}
                \spellsuccess \spelldamage{life}. If the target has no hit points remaining, it immediately dies, and you are transformed to mimic its appearance for 12 hours. This grants you a \plus10 bonus on Disguise checks made to appear as that creature.
                \spellcritical As above, but double damage.
                \spellfailure Half damage, and no additional effects.
            \end{spellattack}
            \spelldur \durext; see text
        \end{spelleffects}
    \end{spellcontent}
    \begin{spellfooter}
        \spellinfo{Transmutation/Vivimancy [Death, Shaping]}{Arcane}
        \miscastrandom
    \end{spellfooter}
\end{spellsection}

\begin{spellsection}{Avatar of Blades}[9]
    \begin{spellheader}
        \spelldesc{You gain the ability to summon armies of ephemeral blades, slicing your foes to bloody shreds.}
    \end{spellheader}
    \begin{spellcontent}
        \begin{spelltargetinginfo}
            \spelltgt{You}
        \end{spelltargetinginfo}
        \begin{spelleffects}
            \spelleffect At any time during the spell's duration, you can concentrate as a standard action. If you do, blades attack a foe, as described below.
            \spelldur \durlong
        \end{spelleffects}
    \end{spellcontent}
    \begin{spellsubcontent}
        \begin{spelltargetinginfo}
            \spellquicktargeting{One creature or object}{\rngclose}
        \end{spelltargetinginfo}
        \begin{spelleffects}
            \begin{spellattack}{Spellpower vs. Fortitude}
                \spellsuccess \spelldamage{slashing}. In addition, all physical damage the target takes is doubled for 2 rounds.
                This does not apply to the initial damage dealt by this spell.
                If the target receives magical healing, the doubling of damage ends.
                \spellcritical As above, except that the doubling of physical damage also applies to the initial damage dealt by the spell.
                \spellfailure Half damage, and no additional effects.
            \end{spellattack}
        \end{spelleffects}
    \end{spellsubcontent}
    \begin{spellfooter}
        \spellinfo{Conjuration [Creation]}{Arcane, War}
        \miscastexplode
    \end{spellfooter}
\end{spellsection}

\begin{spellsection}{Avatar of Chaos}[9]
    \begin{spellheader}
        \spelldesc{You embody the essence of chaos, allowing you to smite your foes.}
    \end{spellheader}
    \begin{spellcontent}
        \begin{spelltargetinginfo}
            \spelltgt{You}
        \end{spelltargetinginfo}
        \begin{spelleffects}
            \spelleffect At any time during the spell's duration, you can concentrate as a standard action. If you do, you smite a foe, as described below.
            \spelldur \durlong
        \end{spelleffects}
    \end{spellcontent}
    \begin{spellsubcontent}
        \begin{spelltargetinginfo}
            \spellquicktargeting{One nonchaotic creature}{\rngmed}
        \end{spelltargetinginfo}
        \begin{spelleffects}
            \spelleffect The target takes 1d10 divine damage per two spellpower. In addition, it is \disoriented for 2 rounds.
        \end{spelleffects}
    \end{spellsubcontent}
    \begin{spellfooter}
        \spellinfo{Channeling [Chaos]}{Chaos, Divine}
        \miscastexplode
    \end{spellfooter}
\end{spellsection}

\begin{spellsection}{Avatar of Death}[8]
    \begin{spellheader}
        \spelldesc{You embody death, gaining the ability to kill your foes at will.}
    \end{spellheader}
    \begin{spellcontent}
        \begin{spelltargetinginfo}
            \spelltgt{You}
        \end{spelltargetinginfo}
        \begin{spelleffects}
            \spelleffect At any time during the spell's duration, you can concentrate as a standard action. If you do, you damage and possibly kill a foe, as described below.
            \spelldur \durlong
        \end{spelleffects}
    \end{spellcontent}
    \begin{spellsubcontent}
        \begin{spelltargetinginfo}
            \spellquicktargeting{One living creature}{\rngmed}
        \end{spelltargetinginfo}
        \begin{spelleffects}
            \begin{spellattack}{Spellpower vs. Fortitude}
                \spellsuccess \spelldamage{life}. If the target has no hit points remaining, it dies.
                \spellcritical As above, but double damage.
            \end{spellattack}
        \end{spelleffects}
    \end{spellsubcontent}
    \begin{spellfooter}
        \spellinfo{Vivimancy [Death]}{Arcane, Death}
        \miscastexplode
    \end{spellfooter}
    \begin{spellaugments}
        \spellaugment{1}{Persistent}{The spell's duration becomes \durext.}
    \end{spellaugments}
\end{spellsection}

\begin{spellsection}{Avatar of Fealty}[7]
    \begin{spellheader}
        \spelldesc{Your voice becomes smooth and enchanting, allowing you to bend others to your will with ease.}
    \end{spellheader}
    \begin{spellcontent}
        \begin{spelltargetinginfo}
            \spelltgt{You}
        \end{spelltargetinginfo}
        \begin{spelleffects}
            \spelleffect At any time during the spell's duration, you can concentrate as a standard action. If you do, you command a creature's actions, as described below.
            \spelldur \durlong
        \end{spelleffects}
    \end{spellcontent}
    \begin{spellsubcontent}
        \begin{spelltargetinginfo}
            \spellquicktargeting{One creature}{\rnglong}
        \end{spelltargetinginfo}
        \begin{spelleffects}
            \spellspecial You speak a command the target must obey, as the \spell{command} spell.
        \end{spelleffects}
    \end{spellsubcontent}
    \begin{spellfooter}
        \spellinfo{Enchantment [Auditory, Compulsion, Mind, Speech]}*{Arcane, Divine, Law}
        \miscastexplode
    \end{spellfooter}
\end{spellsection}

\begin{spellsection}{Avatar of Growth}[5]
    \begin{spellheader}
        \spelldesc{You channel the endless power of nature, granting you the ability to grow plants to ensnare your foes at will.}
    \end{spellheader}
    \begin{spellcontent}
        \begin{spelltargetinginfo}
            \spelltgt{You}
        \end{spelltargetinginfo}
        \begin{spelleffects}
            \spelleffect At any time during the spell's duration, you can concentrate as a standard action. If you do, plants entangle a foe, as described below.
            \spelldur \durlong
        \end{spelleffects}
    \end{spellcontent}
    \begin{spellsubcontent}
        \begin{spelltargetinginfo}
            \spellquicktargeting{One creature or object}{\rngmed}
        \end{spelltargetinginfo}
        \begin{spelleffects}
            \spelleffect The target is \entangled.
            It can break this effect with a grapple or Escape Artist check against a DR equal to 10 \add your spellpower.
            This is a \glossterm{Physical} effect, and does not allow \glossterm{magic resistance}.
            \spelldur \durbrief
        \end{spelleffects}
    \end{spellsubcontent}
    \begin{spellfooter}
        \spellinfo{Transmutation [Animation]}{Nature, Wild}
        \spellnotes The effects of the ability granted by this spell may be altered somewhat based on the nature of the plants near the target.
        \miscastexplode
    \end{spellfooter}
    \begin{spellaugments}
        \spellaugment{1}{Persistent}{The spell's duration becomes \durext.}
        \spellaugment{2}{Wild Growth}{The target of the ability does not need to be near plants.}
    \end{spellaugments}
\end{spellsection}

\begin{spellsection}{Avatar of Healing}[4]
    \begin{spellheader}
        \spelldesc{You become a wellspring of curative energy, allowing you to heal your allies freely.}
    \end{spellheader}
    \begin{spellcontent}
        \begin{spelltargetinginfo}
            \spelltgt{You}
        \end{spelltargetinginfo}
        \begin{spelleffects}
            \spelleffect At any time during the spell's duration, you can concentrate as a standard action. If you do, you heal an ally, as described below.
            \spelldur \durlong
        \end{spelleffects}
    \end{spellcontent}
    \begin{spellsubcontent}
        \begin{spelltargetinginfo}
            \spellquicktargeting{One living creature}{\rngtouch}
        \end{spelltargetinginfo}
        \begin{spelleffects}
            \spelleffect The target is healed for \spelldamage{}.
        \end{spelleffects}
    \end{spellsubcontent}
    \begin{spellfooter}
        \spellinfo{Vivimancy [Life]}{Divine, Good, Life}
        \miscastexplode
    \end{spellfooter}
    \begin{spellaugments}
        \spellaugment{1}{Persistent}{The spell's duration becomes \durext.}
        \spellaugment{4}{Empowered}{The healing increases to \spelldamageemp{}}
    \end{spellaugments}
\end{spellsection}

\begin{spellsection}{Avatar of Life}[9]
    \begin{spellheader}
        \spelldesc{You embody the power of life itself, gaining the ability to resurrect your allies at will.}
    \end{spellheader}
    \begin{spellcontent}
        \begin{spelltargetinginfo}
            \spelltgt{You}
        \end{spelltargetinginfo}
        \begin{spelleffects}
            \spelleffect At any time during the spell's duration, you can concentrate as a standard action. If you do, you resurrect an ally, as described below.
            \spelldur \durlong
        \end{spelleffects}
    \end{spellcontent}
    \begin{spellsubcontent}
        \begin{spelltargetinginfo}
            \spellquicktargeting{One dead creature}{\rngclose}
        \end{spelltargetinginfo}
        \begin{spelleffects}
            \spelleffect If the target has been dead for no longer than 5 rounds, it is restored to life, as the \ritual{resurrection} ritual.
            \spelldur \durlong
        \end{spelleffects}
    \end{spellsubcontent}
    \begin{spellfooter}
        \spellinfo{Vivimancy [Life]}{Divine, Life}
        \miscastexplode
    \end{spellfooter}
\end{spellsection}

\begin{spellsection}{Avatar of Might}[8]
    \begin{spellheader}
        \spelldesc{You become strength incarnate, tossing your foes aside at will.}
    \end{spellheader}
    \begin{spellcontent}
        \begin{spelltargetinginfo}
            \spelltgt{You}
        \end{spelltargetinginfo}
        \begin{spelleffects}
            \spelleffect At any time during the spell's duration, you can concentrate as a standard action. If you do, you attempt to throw all adjacent enemies, as the \spell{mighty throw} spell.
            \spelldur \durlong
        \end{spelleffects}
    \end{spellcontent}
    \begin{spellfooter}
        \spellinfo{Transmutation [Enhancement]}{Nature, Strength}
        \miscastexplode
    \end{spellfooter}
\end{spellsection}

\begin{spellsection}{Avatar of Missiles}[5]
    \begin{spellheader}
        \spelldesc{You can summon orbs of magical energy at will to destroy your foes.}
    \end{spellheader}
    \begin{spellcontent}
        \begin{spelltargetinginfo}
            \spelltgt{You}
        \end{spelltargetinginfo}
        \begin{spelleffects}
            \spelleffect At any time during the spell's duration, you can concentrate as a standard action. If you do, you fire a flurry of missiles, as described below.
            \spelldur \durlong
        \end{spelleffects}
    \end{spellcontent}
    \begin{spellsubcontent}
        \begin{spelltargetinginfo}
            \spellquicktargeting{One creature}{\rnglong}
        \end{spelltargetinginfo}
        \begin{spelleffects}
            \spelleffect You create a number of missiles equal to half your spellpower. Each missile strikes one target creature for 1d10 arcane damage. You can direct each missile to strike the same or different targets.
        \end{spelleffects}
    \end{spellsubcontent}
    \begin{spellfooter}
        \spellinfo{Evocation}{Arcane, Magic}
        \miscastexplode
    \end{spellfooter}
    \begin{spellaugments}
        \spellaugment{1}{Persistent}{The spell's duration becomes \durext.}
        \spellaugment{4}{Empowered}{Each missile deals 2d6 arcane damage.}
    \end{spellaugments}
\end{spellsection}

\begin{spellsection}{Avatar of Order}[9]
    \begin{spellheader}
        \spelldesc{You embody the essence of order, allowing you to smite your foes.}
    \end{spellheader}
    \begin{spellcontent}
        \begin{spelltargetinginfo}
            \spelltgt{You}
        \end{spelltargetinginfo}
        \begin{spelleffects}
            \spelleffect At any time during the spell's duration, you can concentrate as a standard action. If you do, you smite a foe, as described below.
            \spelldur \durlong
        \end{spelleffects}
    \end{spellcontent}
    \begin{spellsubcontent}
        \begin{spelltargetinginfo}
            \spellquicktargeting{One nonchaotic creature}{\rngmed}
        \end{spelltargetinginfo}
        \begin{spelleffects}
            \spelleffect The target takes 1d10 divine damage per two spellpower. In addition, it is \slowed for 2 rounds.
        \end{spelleffects}
    \end{spellsubcontent}
    \begin{spellfooter}
        \spellinfo{Channeling [Law]}{Divine, Law}
        \miscastexplode
    \end{spellfooter}
\end{spellsection}

\begin{spellsection}{Avatar of Shielding}[7]
    \begin{spellheader}
        \spelldesc{You can to shield an ally from harm at will.}
    \end{spellheader}
    \begin{spellcontent}
        \begin{spelltargetinginfo}
            \spelltgt{You}
        \end{spelltargetinginfo}
        \begin{spelleffects}
            \spelleffect At any time during the spell's duration, you can concentrate as a swift action. If you do, an ally gains damage reduction, as described below.
            \spelldur \durlong
        \end{spelleffects}
    \end{spellcontent}
    \begin{spellsubcontent}
        \begin{spelltargetinginfo}
            \spellquicktargeting{One willing creature}{\rngtouch}
        \end{spelltargetinginfo}
        \begin{spelleffects}
            \spelleffect The target gains damage reduction against all damage equal to your spellpower. This effect lasts until the spell ends, or until you shield a different creature.
        \end{spelleffects}
    \end{spellsubcontent}
    \begin{spellfooter}
        \spellinfo{Abjuration [Shielding]}{Arcane, Divine, Protection, War}
        \miscastexplode
    \end{spellfooter}
    \begin{spellaugments}
        \spellaugment{1}{Persistent}{The spell's duration becomes \durext.}
    \end{spellaugments}
\end{spellsection}

\begin{spellsection}{Avatar of Suffering}[8]
    \begin{spellheader}
        \spelldesc{You gain the ability to inflict exquisite pain on your foes at will.}
    \end{spellheader}
    \begin{spellcontent}
        \begin{spelltargetinginfo}
            \spelltgt{You}
        \end{spelltargetinginfo}
        \begin{spelleffects}
            \spelleffect At any time during the spell's duration, you can concentrate as a standard action. If you do, you inflict pain on a foe, as described below.
            \spelldur \durlong
        \end{spelleffects}
    \end{spellcontent}
    \begin{spellsubcontent}
        \begin{spelltargetinginfo}
            \spellquicktargeting{One creature}{\rngclose}
        \end{spelltargetinginfo}
        \begin{spelleffects}
            \spelleffect The target is \staggered and \impaired with attacks and checks for 2 rounds.
        \end{spelleffects}
    \end{spellsubcontent}
    \begin{spellfooter}
        \spellinfo{Enchantment [Compulsion, Mind]}{Arcane, Divine, Evil}
        \miscastexplode
    \end{spellfooter}
    \begin{spellaugments}
        \spellaugment{1}{Persistent}{The spell's duration becomes \durext.}
    \end{spellaugments}
\end{spellsection}

\begin{spellsection}{Avatar of Translocation}[7]
    \begin{spellheader}
        \spelldesc{You flit quickly between dimensions, teleporting yourself and your allies at will.}
    \end{spellheader}
    \begin{spellcontent}
        \begin{spelltargetinginfo}
            \spelltgt{You}
        \end{spelltargetinginfo}
        \begin{spelleffects}
            \spelleffect At any time during the spell's duration, you can concentrate as a standard action or move action. If you do, you teleport an ally, as described below.
            \spelldur \durlong
        \end{spelleffects}
    \end{spellcontent}
    \begin{spellsubcontent}
        \begin{spelltargetinginfo}
            \spellquicktargeting{One willing creature}{\rngmed}

            \spelltime Standard action to target an ally, or a move action to target yourself.
        \end{spelltargetinginfo}
        \begin{spelleffects}
            \spelleffect The target teleports to an unoccupied destination up to 100 feet away from its original location. If the destination is invalid, the teleportation fails.
        \end{spelleffects}
    \end{spellsubcontent}
    \begin{spellfooter}
        \spellinfo{Conjuration [Teleportation]}{Arcane, Travel}
        \miscastexplode
    \end{spellfooter}
    \begin{spellaugments}
        \spellaugment{1}{Persistent}{The spell's duration becomes \durext.}
    \end{spellaugments}
\end{spellsection}

\begin{spellsection}{Avatar of Wind}[7]
    \begin{spellheader}
        \spelldesc{You are constantly surrounded by a gentle, whirling wind, which you can command to strike your foes.}
    \end{spellheader}
    \begin{spellcontent}
        \begin{spelltargetinginfo}
            \spelltgt{You}
        \end{spelltargetinginfo}
        \begin{spelleffects}
            \spelleffect At any time during the spell's duration, you can concentrate as a standard action. If you do, wind attacks a foe, as described below.
            \spelldur \durlong
        \end{spelleffects}
    \end{spellcontent}
    \begin{spellsubcontent}
        \begin{spelltargetinginfo}
            \spellquicktargeting{One creature or object}{\rngext}
        \end{spelltargetinginfo}
        \begin{spelleffects}
            \begin{spellattack}{Spellpower vs. Fortitude defense}
                \spellsuccess \spelldamage{bludgeoning}. In addition, buffeting winds make the target move at half speed for 2 rounds.
                \spellcritical As above, but double damage.
                \spellfailure As above, but half damage.
            \end{spellattack}
        \end{spelleffects}
    \end{spellsubcontent}
    \begin{spellfooter}
        \spellinfo{Evocation [Air]}{Air, Arcane, Nature}
        \miscastexplode
    \end{spellfooter}
    \begin{spellaugments}
        \spellaugment{1}{Persistent}{The spell's duration becomes \durext.}
    \end{spellaugments}
\end{spellsection}

\begin{spellsection}{Aura of Immunity}[6]
    \begin{spellheader}
        \spelldesc{You grant your allies immunity to a particular kind of damage.}
    \end{spellheader}
    \begin{spellcontent}
        \begin{spelltargetinginfo}
            \spellemanation{\arealarge radius from you}
            \spelltgts{All willing allies in the area}
        \end{spelltargetinginfo}
        \begin{spelleffects}
            \spellspecial As you cast this spell, choose a type of damage.
            \spelleffect The target becomes immune to the chosen damage type. Attacks that deal damage of multiple types still inflict damage normally unless the target is immune to all types of damage dealt.
            % clarify interaction with immunity + DR against multiple damage types?
            \spelldur \durshort
        \end{spelleffects}
    \end{spellcontent}
    \begin{spellfooter}
        \spellinfo{Abjuration [Shielding]}{Arcane, Protection}
        \spellnotes A creature can only be protected by one \spell{aura of immunity} spell at a time, even if there are multiple auras active in the same area. The creature may choose which aura to benefit from.
        \miscastexplode
    \end{spellfooter}
\end{spellsection}

\begin{spellsection}{Aversion}[2]
    \begin{spellheader}
        \spelldesc{You make a creature want to avoid something.}
    \end{spellheader}
    \begin{spellcontent}
        \begin{spelltargetinginfo}
            \spellquicktargeting{One creature}{\rngmed}
        \end{spelltargetinginfo}
        \begin{spelleffects}
            \begin{spellattack}{Spellpower vs. Mental}
                \spellsuccess The target feels an aversion to a particular person or object for 2 rounds. If the object of the implanted aversion is an individual or a physical object, she will prefer not to approach within 30 feet of it. If it is a word, she will try not to utter it; if it is an action, she will not willingly attempt to perform it; and if it is an event, she will not willingly attend it. The target will take reasonable steps to avoid the object of its aversion, but will not put herself in jeopardy by doing so.
                \par If the target is unable to avoid the object of her aversion, she is \severelyimpaired with attacks and checks as long as she is close to it (or similarly engaged with the object of her aversion, if the aversion is not a location).
                \spellcritical As above, but the effect is permanent.
            \end{spellattack}
        \end{spelleffects}
    \end{spellcontent}
    \begin{spellfooter}
        \spellinfo{Enchantment [Mind]}{Arcane}
        \miscastrandom
    \end{spellfooter}
\end{spellsection}

\pdfbookmark[2]{B}{SpellDescriptionsB}

\begin{spellsection}{Bane}[1]
    \begin{spellheader}
    %\spelldesc{}
    \end{spellheader}
    \begin{spellcontent}
        \begin{spelltargetinginfo}
            \spellquicktargeting{One creature}{\rngclose}
        \end{spelltargetinginfo}
        \begin{spelleffects}
            \spelleffect The target is \impaired with attacks and checks.

            \spelldur 5 rounds
        \end{spelleffects}
    \end{spellcontent}
    \begin{spellfooter}
        \spellinfo{Enchantment [Compulsion, Mind]}{Divine, Evil}
        \miscastrandom
    \end{spellfooter}
    \begin{spellaugments}
        \spellaugment{3}{Mass}{The spell can affect up to five targets.}
    \end{spellaugments}
\end{spellsection}

\begin{spellsection}{Barkskin}[2]
    \begin{spellheader}
        \spelldesc{You toughen a creature's skin, giving it the appearance of tree bark.}
    \end{spellheader}
    \begin{spellcontent}
        \begin{spelltargetinginfo}
            \spellquicktargeting{One living creature}{\rngclose}
        \end{spelltargetinginfo}
        \begin{spelleffects}
            \spelleffect The target gains damage reduction against physical damage equal to your spellpower.
            Fire damage ignores this damage reduction and negates it for 1 round.
            \spelldur \durshort
        \end{spelleffects}
    \end{spellcontent}
    \begin{spellfooter}
        \spellinfo{Transmutation [Enhancement]}{Nature}
        \miscastexplode
    \end{spellfooter}
    \begin{spellaugments}
        \spellaugment{1}{Stoneskin}{Adamantine weapons ignore and negate the damage reduction, rather than fire damage.}[Earth]
        \spellaugment{2}{Critical Resistance}{The target is immune to critical hits unless the damage reduction is negated.}
        %\spellaugment{2}{Personal}{The spell targets you, and its duration becomes \durlong.}
    \end{spellaugments}
\end{spellsection}

\begin{spellsection}{Black Tentacles}[5]
    \begin{spellheader}
        \spelldesc{You create a field of rubbery black tentacles, each 5 feet long. These waving members seem to spring forth from the earth, floor, or whatever surface is underfoot -- including water. They grasp and entwine around creatures that enter the area, holding them fast and crushing them with great strength.}
    \end{spellheader}
    \begin{spellcontent}
        \begin{spelltargetinginfo}
            \spelltwocol{\spellzone{\areamed radius}}{\spellrng{\rngmed}}
        \end{spelltargetinginfo}
        \begin{spelleffects}
            \spelleffect Each 5-ft.\ square within the spell that is adjacent to a solid surface contains a single tentacle.
            A square with a tentacle is considered difficult terrain.
            At the end of every round, each tentacle attacks one random creature within its space.
            You make a Spellpower vs. Fortitude attack to determine the success of each tentacle's attack.
            Success against a target means it takes \spelldamage{bludgeoning}[d8].
            A critical success causes the target to be \immobilized for the duration of the spell, and the tentacle in that space stops attacking until the creature breaks free.

            The tentacles can be attacked and destroyed.
            Each tentacle has defenses equal to 10 \add your spellpower, and hit points equal to three times your spellpower.
            If a tentacle is destroyed, its space is no longer considered difficult terrain and can be traversed safely.
            \spelldur \durshort
        \end{spelleffects}
    \end{spellcontent}
    \begin{spellfooter}
        \spellinfo{Conjuration/Transmutation [Animation, Creation]}{Arcane}
        \miscastyou
    \end{spellfooter}
    \begin{spellaugments}
        \spellaugment{3}{Widened}{The spell's area becomes a \arealarge radius.}
        \spellaugment{4}{Empowered}{The tentacle attacks gain a \plus5 bonus to accuracy.}
    \end{spellaugments}
\end{spellsection}

\begin{spellsection}{Blade Barrier}[4]
    \begin{spellheader}
        \spelldesc{You create an immobile, vertical curtain of whirling blades.}
    \end{spellheader}
    \begin{spellcontent}
        \begin{spelltargetinginfo}
            \spelltwocol{\spellzone{20 ft.\ high wall: line up to 100 ft.\ long or 20 ft.\ radius}}{\spellrng{\rngmed}}
        \end{spelltargetinginfo}
        \begin{spelleffects}
            \spelleffect This spell creates a wall of whirling blades. The wall provides active cover (20\% miss chance) against attacks made through it. Attacks that miss in this way harmlessly strike the wall. Passing through the wall costs twice as much movement as normal.

            Whenever a creature passes through the wall, make a Spellpower vs. Reflex attack against it. Success means the creature takes \spelldamage{slashing}[d8]. Failure means it takes half damage.
            \spelldur \durshort \dismissable
        \end{spelleffects}
    \end{spellcontent}
    \begin{spellfooter}
        \spellinfo{Conjuration [Creation]}{Divine, War}
        \miscastexplode
    \end{spellfooter}
    \begin{spellaugments}
        \spellaugment{1}{Impeding}{Passing through the wall costs an extra ten feet of movement.}
        \spellaugment{2}{Contracting}{The wall's radius shrinks by 5 feet at the end of every round, dealing damage to all creatures it moves through. This augment can only be used if the wall is created in a radius.}
        \spellaugment{2}{Dual}{Two parallel walls appear, five feet apart. This augment can only be used if the wall is created in a line.}
        \spellaugment{3}{Widened}{The wall appears in a 300 ft.\ line or a 50 ft.\ radius.}
        \spellaugment{4}{Empowered}{The damage dealt to creatures passing through the wall increases to \spelldamage{slashing}[d10].}
    \end{spellaugments}
\end{spellsection}

\begin{spellsection}{Bladestorm}[8]
    \begin{spellheader}
        \spelldesc{You summon a titanic flurry of blades that cut your foes to pieces.}
    \end{spellheader}
    \begin{spellcontent}
        \begin{spelltargetinginfo}
            \spellburst{\areahuge radius}
            \spelltgts{All enemies in the area}
        \end{spelltargetinginfo}
        \begin{spelleffects}
            \begin{spellattack}{Spellpower vs. Reflex}
                \spellsuccess \spelldamage{slashing}[d8].
                \spellcritical Double damage.
                \spellfailure Half damage.
            \end{spellattack}
        \end{spelleffects}
    \end{spellcontent}
    \begin{spellfooter}
        \spellinfo{Conjuration [Creation]}{Arcane, War}
        \miscastexplode
    \end{spellfooter}
\end{spellsection}

\begin{spellsection}{Blasphemy}[5]
    \begin{spellheader}
        \spelldesc{You speak an unholy utterance of great power, afflicting all those nearby who do not share your allegiance to evil.}
    \end{spellheader}
    \begin{spellcontent}
        \begin{spelltargetinginfo}
            \spellburst{\arealarge radius centered on you}
            \spelltgts{All nonevil creatures in the area}
            \spellcmp{Verbal only}
        \end{spelltargetinginfo}
        \begin{spelleffects}
            \begin{spellattack}{Spellpower vs. Mental}
                \spellsuccess \spelldamage{divine}[d8], and the target is \staggered for 2 rounds.
                \spellcritical Double damage, and the target is \nauseated for 1 round instead of staggered.
                \spellfailure Half damage, and no additional effects.
            \end{spellattack}
        \end{spelleffects}
    \end{spellcontent}
    \begin{spellfooter}
        \spellinfo{Channeling [Evil]}{Divine, Evil}
        \miscastexplode
    \end{spellfooter}
    \begin{spellaugments}
        \spellaugment{4}{Empowered}{The damage increases to \spelldamage{divine}[d10]}
    \end{spellaugments}
\end{spellsection}

\begin{spellsection}{Bleed}[5]
    \begin{spellheader}
        \spelldesc{You carve a wound into your foe's flesh.}
    \end{spellheader}
    \begin{spellcontent}
        \begin{spelltargetinginfo}
            \spellquicktargeting{One creature}{\rngclose}
        \end{spelltargetinginfo}
        \begin{spelleffects}
            \begin{spellattack}{Spellpower vs. Fortitude}
                \spellsuccess \spelldamage{slashing}. In addition, the target takes one point of critical damage after this damage is dealt, regardless of its current hit points.
                \spellcritical As above, except the target takes critical damage equal to your spellpower.
                \spellfailure Half damage, and no additional effects.
            \end{spellattack}
        \end{spelleffects}
    \end{spellcontent}
    \begin{spellfooter}
        \spellinfo{Vivimancy [Flesh]}{Arcane, Death}
        \miscastrandom
    \end{spellfooter}
    \begin{spellaugments}
        \spellaugment{5}{Empowered}{The damage increases to \spelldamageemp{slashing}, and the automatic critical damage dealt is doubled.}
    \end{spellaugments}
\end{spellsection}

\begin{spellsection}{Bless}[1]
    \begin{spellheader}
        \spelldesc{You fill your ally with confidence, improving her prowess in combat.}
    \end{spellheader}
    \begin{spellcontent}
        \begin{spelltargetinginfo}
            \spellquicktargeting{One creature}{\rngmed}
        \end{spelltargetinginfo}
        \begin{spelleffects}
            \spelleffect The target gains a legend point.
            \spelldur \durshort or until expended
        \end{spelleffects}
    \end{spellcontent}
    \begin{spellfooter}
        \spellinfo{Enchantment [Delusion, Mind]}{Divine}
        \miscastrandom
    \end{spellfooter}
    \begin{spellaugments}
        \spellaugment{2}{Empowered}{The target gains an additional legend point.}
        \spellaugment{3}{Mass}{The spell can affect up to five targets.}
    \end{spellaugments}
\end{spellsection}

\begin{spellsection}{Blessed Blade}[6]
    \begin{spellheader}
    \end{spellheader}
    \begin{spellcontent}
        \begin{spelltargetinginfo}
            \spellquicktargeting{One unattended weapon}{\rngclose}
        \end{spelltargetinginfo}
        \begin{spelleffects}
            \spelleffect Attacks with the affected weapon are made against Mental defense instead of Armor defense.
            In addition, the weapon is considered to be of your alignment, which may allow attacks with it to overcome damage reduction or similar defenses.
        \end{spelleffects}
    \end{spellcontent}
    \begin{spellfooter}
        \spellinfo{Transmutation [Imbuement]}{Divine, War}
        \spellnotes A weapon wielded by a creature can be affected if its wielder is willing.
        \miscastexplode
    \end{spellfooter}
\end{spellsection}

\begin{spellsection}{Blink}[3]
    \begin{spellheader}
        \spelldesc{You rapidly blink in and out of reality, confounding your foes and protecting you from their attacks.}
    \end{spellheader}
    \begin{spellcontent}
        \begin{spelltargetinginfo}
            \spelltgt{You}
        \end{spelltargetinginfo}
        \begin{spelleffects}
            \spelleffect You spend half your time on the Astral Plane.
            All attacks and abilities used on you have a 50\% chance to fail.
            However, your attacks and abilities you use on anything other than yourself have a 20\% chance to fail.
            \spelldur \durshort \dismissable
        \end{spelleffects}
    \end{spellcontent}
    \begin{spellfooter}
        \spellinfo{Conjuration [Planar]}{Arcane}
        \spellnotes If you are on the Astral Plane when you cast this spell, it has no effect.
        \miscastexplode
    \end{spellfooter}
    \begin{spellaugments}
        \spellaugment{3}{Controlled}{Your attacks and abilities do not have a chance to fail.}
    \end{spellaugments}
\end{spellsection}

\begin{spellsection}{Blur}[2]
    \begin{spellheader}
        \spelldesc{You distort an ally's outline so it appears blurred, shifting, and wavering.}
    \end{spellheader}
    \begin{spellcontent}
        \begin{spelltargetinginfo}
            \spellquicktargeting{One creature}{\rngclose}
        \end{spelltargetinginfo}
        \begin{spelleffects}
            \spelleffect Targeted physical attacks against the target have a 20\% miss chance.
            Spells and other non-physical attacks suffer no miss chance.
            \spelldur \durshort
        \end{spelleffects}
    \end{spellcontent}
    \begin{spellfooter}
        \spellinfo{Illusion [Glamer]}{Arcane}
        \miscastrandom
    \end{spellfooter}
    \begin{spellaugments}
        \spellaugment{3}{Empowered}{The miss chance increases to 50\%.}
    \end{spellaugments}
\end{spellsection}

\begin{spellsection}{Boon of Knowledge}[2]
    \begin{spellheader}
        \spelldesc{Your ally becomes deeply knowledgeable.}
    \end{spellheader}
    \begin{spellcontent}
        \begin{spelltargetinginfo}
            \spellquicktargeting{One willing creature}{\rngclose}
        \end{spelltargetinginfo}
        \begin{spelleffects}
            \spelleffect The target gains a \plus5 bonus to Knowledge checks.
            In addition, it is treated as being trained in all Knowledge skills, allowing it to make any Knowledge check.
            \spelldur \durshort
        \end{spelleffects}
    \end{spellcontent}
    \begin{spellfooter}
        \spellinfo{Divination [Knowledge]}{Arcane, Knowledge}
        \spellnotes A creature who has already tried to use a Knowledge skill about a topic may retry the attempt under the effects of this spell, because its Knowledge modifier changed.
        However, repeated castings of this spell do not grant additional attempts to recall information about a topic.
        \miscastrandom
    \end{spellfooter}
    \begin{spellaugments}
        \spellaugment{3}{Empowered}{The bonus increases to \plus10.}
    \end{spellaugments}
\end{spellsection}

\begin{spellsection}{Boon of Many Eyes}[4]
    \begin{spellheader}
        \spelldesc{Your ally becomes able to fight foes on all sides with equal skill.}
    \end{spellheader}
    \begin{spellcontent}
        \begin{spelltargetinginfo}
            \spellquicktargeting{One willing creature}{\rngclose}
        \end{spelltargetinginfo}
        \begin{spelleffects}
            \spelleffect Whenever the target makes Awareness checks, it rolls twice and takes the higher result.
            In addition, it reduces its \glossterm{overwhelm penalties} by 5 (to a minimum of 0).
            If this effect reduces the target's overwhelm penalty to 0, it is not considered to be overwhelmed.
            \spelldur \durshort
        \end{spelleffects}
    \end{spellcontent}
    \begin{spellfooter}
        \spellinfo{Transmutation [Enhancement]}{Arcane, Knowledge}
        \miscastrandom
    \end{spellfooter}
\end{spellsection}

\begin{spellsection}{Boon of Mastery}[3]
    \begin{spellheader}
        \spelldesc{Your ally becomes skilled in all things.}
    \end{spellheader}
    \begin{spellcontent}
        \begin{spelltargetinginfo}
            \spellquicktargeting{One willing creature}{\rngclose}
        \end{spelltargetinginfo}
        \begin{spelleffects}
            \spelleffect Whenever the target makes a skill check, it rolls twice and takes the higher result.
            \spelldur \durshort
        \end{spelleffects}
    \end{spellcontent}
    \begin{spellfooter}
        \spellinfo{Divination [Knowledge]}{Arcane, Knowledge}
        \miscastrandom
    \end{spellfooter}
    \begin{spellaugments}
        \spellaugment{4}{Empowered}{The target also gains a \plus5 bonus to all skills.}
    \end{spellaugments}
\end{spellsection}

\begin{spellsection}{Boon of Perception}[1]
    \begin{spellheader}
        \spelldesc{Your ally becomes acutely aware of their surroundings.}
    \end{spellheader}
    \begin{spellcontent}
        \begin{spelltargetinginfo}
            \spellquicktargeting{One willing creature}{\rngclose}
        \end{spelltargetinginfo}
        \begin{spelleffects}
            \spelleffect Whenever the target makes Awareness, Sense Motive, and Spellcraft checks, it rolls twice and takes the higher result.
            \spelldur \durshort
        \end{spelleffects}
    \end{spellcontent}
    \begin{spellfooter}
        \spellinfo{Divination}{Arcane, Knowledge}
        \miscastrandom
    \end{spellfooter}
    \begin{spellaugments}
        \spellaugment{3}{Empowered}{The target also gains a \plus5 bonus to Awareness, Sense Motive, and Spellcraft checks.}
    \end{spellaugments}
\end{spellsection}

\begin{spellsection}{Boon of Precision}[5]
    \begin{spellheader}
        \spelldesc{Your ally gains the ability to see the weak points of creatures she fights.}
    \end{spellheader}
    \begin{spellcontent}
        \begin{spelltargetinginfo}
            \spellquicktargeting{One willing creature}{\rngclose}
        \end{spelltargetinginfo}
        \begin{spelleffects}
            \spelleffect The target increases its critical range and critical multiplier with physical attacks by 1.

            If it scores a number of critical hits equal to one quarter of your spellpower, the spell is expended.
            \spelldur \durshort or until expended
        \end{spelleffects}
    \end{spellcontent}
    \begin{spellfooter}
        \spellinfo{Divination [Awareness]}{Arcane, Knowledge}
        \miscastrandom
    \end{spellfooter}
\end{spellsection}

\begin{spellsection}{Boulder Drop}[4]
    \begin{spellheader}
        \spelldesc{You create a massive boulder over the heads of your foes and drop it on them.}
    \end{spellheader}
    \begin{spellcontent}
        \begin{spelltargetinginfo}
            \spelltwocol{\spellburst{\areasmall radius}}{\spellrng{\rngmed}}
            \spelltgts{Everything in the area}
        \end{spelltargetinginfo}
        \begin{spelleffects}
            \spelleffect A boulder appears in midair above the target area and falls, crushing everything in the area before disappearing. You can create the boulder up to 50 feet above the area. The height of the boulder does not affect the damage dealt, but it must not share space with any creature creature in the area.
            \begin{spellattack}{Spellpower vs. Reflex}
                \spellsuccess \spelldamage{bludgeoning}[d8].
                \spellcritical Double damage.
                \spellfailure Half damage.
            \end{spellattack}
        \end{spelleffects}
    \end{spellcontent}
    \begin{spellfooter}
        \spellinfo{Conjuration [Creation]}{Arcane, Nature}
        \miscastexplode
    \end{spellfooter}
    \begin{spellaugments}
        \spellaugment{2}{Widened}{You summon a larger boulder, causing the spell's area becomes a \areamed radius.}
        \spellaugment{4}{Empowered}{The damage increases to \spelldamage{bludgeoning}[d10]}
    \end{spellaugments}
\end{spellsection}

\begin{spellsection}{Burning Hands}[1]
    \begin{spellheader}
        \spelldesc{You expel a cone of searing flame shoots from your fingertips, searing creatures in front of you.}
    \end{spellheader}
    \begin{spellcontent}
        \begin{spelltargetinginfo}
            \spelltwocol{\spellburst{\areamed cone}}{\spelltgts{Everything in the area}}
        \end{spelltargetinginfo}
        \begin{spelleffects}
            \begin{spellattack}{Spellpower vs. Reflex}
                \spellsuccess \spelldamage{fire}[d8]
                \spellfailure Half damage.
            \end{spellattack}
        \end{spelleffects}
    \end{spellcontent}
    \begin{spellfooter}
        \spellinfo{Evocation [Fire]}{Arcane, Nature, Fire}
        \miscastexplode
    \end{spellfooter}
    \begin{spellaugments}
        \spellaugment{3}{Widened}{The spell's area becomes a \arealarge cone.}
        \spellaugment{4}{Empowered}{The damage increases to \spelldamage{fire}[d10]}
    \end{spellaugments}
\end{spellsection}

\pdfbookmark[2]{C}{SpellDescriptionsC}

\begin{spellsection}{Cacaphonic Word}[5]
    \begin{spellheader}
        \spelldesc{You utter an incoherent burst of noise, disorienting your foes.}
    \end{spellheader}
    \begin{spellcontent}
        \begin{spelltargetinginfo}
            \spellburst{\arealarge radius centered on you}
            \spelltgts{All nonchaotic creatures in the area}
            \spellcmp{Verbal only}
        \end{spelltargetinginfo}
        \begin{spelleffects}
            \begin{spellattack}{Spellpower vs. Mental}
                \spellsuccess \spelldamage{divine}[d8], and the target is \disoriented for 2 rounds.
                \spellcritical Double damage, and the target is \confused for 2 rounds instead of disoriented.
                \spellfailure Half damage, and no additional effects.
            \end{spellattack}
        \end{spelleffects}
    \end{spellcontent}
    \begin{spellfooter}
        \spellinfo{Channeling [Chaotic]}{Chaos, Divine}
        \miscastexplode
    \end{spellfooter}
    \begin{spellaugments}
        \spellaugment{4}{Empowered}{The damage increases to \spelldamage{divine}[d10]}
    \end{spellaugments}
\end{spellsection}

\begin{spellsection}{Call Lightning}[3]
    \begin{spellheader}
        \spelldesc{You repeatedly call bolts of lightning that flash down from thin air to smite your foes.}
    \end{spellheader}
    \begin{spellcontent}
        \begin{spelltargetinginfo}
            \spelltwocol{\spellburst{\arealarge vertical line, 5 ft.\ wide}}{\spellrng{\rngmed}}
            \spelltgts{Everything in the area}
        \end{spelltargetinginfo}
        \begin{spelleffects}
            \begin{spellattack}{Spellpower vs. Reflex}
                \spellsuccess \spelldamage{electricity}[d10].
                If you are outdoors in cloudy or stormy weather, each bolt instead deals \spelldamage{electricity}.
                \spellfailure Half damage.
            \end{spellattack}
            \spelleffect You can concentrate as a standard action to call down another bolt of lightning. You may call a total number of bolts equal to your spellpower before the spell is discharged.
            \spelldur \durmed or until discharged
        \end{spelleffects}
    \end{spellcontent}
    \begin{spellfooter}
        \spellinfo{Evocation [Electricity]}{Air, Nature}
        \spellnotes This spell functions indoors or underground, but not underwater.
        \miscastexplode
    \end{spellfooter}
    \begin{spellaugments}
        \spellaugment{1}{Seeking}{If no creatures or objects lie in the area of a bolt, the lightning strikes elsewhere instead. It strikes the occupied square within the spell's range that lies closest to its original destination. If multiple occupied squares are equally close, it strikes the largest target. The lightning can unerringly identify invisible and concealed creatures, but it does not differentiate between friend, foe, and inanimate object.}
        \spellaugment{3}{Staggering}{A successful attack also makes a target \staggered for 2 rounds.}
        \spellaugment{4}{Empowered}{The damage increases to \spelldamage{electricity}. If you are outdoors in cloudy or storm weather, the damage instead increases to \spelldamageemp{electricity}.}
    \end{spellaugments}
\end{spellsection}

\begin{spellsection}{Calm Emotions}[3]
    \begin{spellheader}
        \spelldesc{You calm a group of creatures, preventing the situation from getting out of hand.}
    \end{spellheader}
    \begin{spellcontent}
        \begin{spelltargetinginfo}
            \spelltwocol{\spellburst{\arealarge radius}}{\spellrng{\rngmed}}
            \spelltgts{All creatures in the area}
        \end{spelltargetinginfo}
        \begin{spelleffects}
            \begin{spellattack}{Spellpower vs. Mental}
                \spellsuccess The target has its emotions calmed. It cannot take violent actions (although it can defend itself) or do anything destructive.
                If an aggressive action is taken against a nearby creature, this effect is broken.
                \spellcritical As above, except that nearby violence does not break the effect, and the effect lasts for 5 rounds after you stop concentrating on the spell.
            \end{spellattack}
            \spelldur Focus
        \end{spelleffects}
    \end{spellcontent}
    \begin{spellfooter}
        \spellinfo{Enchantment [Mind, Subtle]}{Arcane, Divine}
        \spellnotes This spell automatically suppresses (but does not dispel) any effects of spells or abilities that affect or require emotions, including all other enchantment (emotion) spells.
        \miscastyou
    \end{spellfooter}
\end{spellsection}

\begin{spellsection}{Chain Lightning}[6]
    \begin{spellheader}
        \spelldesc{You create a stroke of lightning which strikes a single foe before arcing to hit a number of other foes of your choice.}
    \end{spellheader}
    \begin{spellcontent}
        \begin{spelltargetinginfo}
            \spelltwocol{\spelltgt{One creature or object}[Primary]}{\spellrng{\rngmed}}
            \spelltgts{Up to five creatures or objects}[Secondary]
        \end{spelltargetinginfo}
        \begin{spelleffects}
            \begin{spellattack}{Spellpower vs. Reflex}
                \spellsuccess \spelldamage{electricity}[d8].
                \spellfailure Half damage.
                \spellspecial This attack automatically succeeds against the primary target.
            \end{spellattack}
        \end{spelleffects}
    \end{spellcontent}
    \begin{spellfooter}
        \spellinfo{Evocation [Electricity]}{Air, Arcane, Nature}
        \miscastexplode
    \end{spellfooter}
    \begin{spellaugments}
        \spellaugment{4}{Empowered}{The damage increases to \spelldamage{electricity}[d10]}
    \end{spellaugments}
\end{spellsection}

\begin{spellsection}{Chaos Hammer}[3]
    \begin{spellheader}
        \spelldesc{You unleash a multicolored explosion of leaping, ricocheting energy to smite your foe.}
    \end{spellheader}
    \begin{spellcontent}
        \begin{spelltargetinginfo}
            \spellquicktargeting{One nonchaotic creature}{\rngmed}
        \end{spelltargetinginfo}
        \begin{spelleffects}
            \begin{spellattack}{Spellpower vs. Mental}
                \spellsuccess \spelldamage{divine}, and the target is \disoriented for 2 rounds.
                \spellcritical As above, except that the target is \confused for 2 rounds instead of disoriented.
                \spellfailure Half damage.
            \end{spellattack}
        %\spelldur 5 rounds
        \end{spelleffects}
    \end{spellcontent}
    \begin{spellfooter}
        \spellinfo{Channeling [Chaotic]}{Chaos}
        \miscastrandom
    \end{spellfooter}
    \begin{spellaugments}
        \spellaugment{3}{Mass}{The spell can affect up to five targets. Its damage becomes \spelldamage{divine}[d8]}
        \spellaugment{4}{Empowered}{The damage increases to \spelldamageemp{divine}. If the Mass augment is applied, the damage instead increases to \spelldamage{divine}[d10].}
    \end{spellaugments}
\end{spellsection}

\begin{spellsection}{Charm Person}[2]
    \begin{spellheader}
        \spelldesc{You manipulate a person's mind so he thinks of you as a trusted friend and ally.}
    \end{spellheader}
    \begin{spellcontent}
        \begin{spelltargetinginfo}
            \spellquicktargeting{One humanoid creature}{\rngclose}
            \spellcmp{Somatic only}
        \end{spelltargetinginfo}
        \begin{spelleffects}
            \begin{spellattack}{Spellpower vs. Mental}
                \spellspecial If the target thinks that you or your allies are threatening it, you take a \minus5 penalty to accuracy on the attack.
                \spellsuccess The target is \charmed by you.
                \spellcritical As above, but the effect is permanent.
            \end{spellattack}
            \spelldur \durlong
        \end{spelleffects}
    \end{spellcontent}
    \begin{spellfooter}
        \spellinfo{Enchantment [Delusion, Mind, Subtle]}{Arcane}
        \spellnotes Any act by you or your apparent allies that threatens or damages the \spell{charmed} person breaks the spell.

        \subtlespellnotes

        \norepeatspellnotes
        \miscastrandom
    \end{spellfooter}
    \begin{spellaugments}
        \spellaugment{1}{Charm Monster}{The spell can affect creatures of any type.}
        \spellaugment{3}{Mass}{The spell can affect up to five targets.}
        \spellaugment{3}{Persistent}{The spell lasts for thirty days.}
    \end{spellaugments}
\end{spellsection}

\begin{spellsection}{Circle of Death}[5]
    \begin{spellheader}
        \spelldesc{You channel powerful necrotic magic to damage nearby foes.}
    \end{spellheader}
    \begin{spellcontent}
        \begin{spelltargetinginfo}
            \spellemanation{\areahuge radius from you}
            \spelltgts{All living enemies in the area}
        \end{spelltargetinginfo}
        \begin{spelleffects}
            \spelleffect At the end of every round, the target takes life damage equal to your spellpower.
            \spelldur 5 rounds
        \end{spelleffects}
    \end{spellcontent}
    \begin{spellfooter}
        \spellinfo{Vivimancy [Death]}{Divine}
        \miscastexplode
    \end{spellfooter}
\end{spellsection}

\begin{spellsection}{Circle of Healing}[5]
    \begin{spellheader}
        \spelldesc{You channel powerful healing magic to heal nearby allies.}
    \end{spellheader}
    \begin{spellcontent}
        \begin{spelltargetinginfo}
            \spellemanation{\arealarge radius from you}
            \spelltgts{All allies in the area}
        \end{spelltargetinginfo}
        \begin{spelleffects}
            \spelleffect At the end of every round, the target heals hit points equal to your spellpower.
            \spelldur 5 rounds
        \end{spelleffects}
    \end{spellcontent}
    \begin{spellfooter}
        \spellinfo{Vivimancy [Life]}{Divine, Nature}
        \miscastexplode
    \end{spellfooter}
\end{spellsection}

\begin{spellsection}{Color Spray}[1]
    \begin{spellheader}
    \end{spellheader}
    \begin{spellcontent}
        \begin{spelltargetinginfo}
            \spellburst{\areamed cone}
            \spelltgts{All creatures in the area}
        \end{spelltargetinginfo}
        \begin{spelleffects}
            \spelleffect The target is \impaired with sight-related attacks and checks.
            \spelldur 2 rounds
        \end{spelleffects}
    \end{spellcontent}
    \begin{spellfooter}
        \spellinfo{Illusion [Figment, Light, Visual]}{Arcane}
        \spelldesc{You project a vivid cone of clashing colors from your outstretched hand, striking creatures in front of you.}
        \spellnotes Creatures who cannot see the light are not affected by this spell. Merely closing one's eyes is insufficient protection, however.
        \miscastexplode
    \end{spellfooter}
    \begin{spellaugments}
        \spellaugment{1}{Focused}{The spell lasts for 5 rounds on one target creature in the area.}
        \spellaugment{3}{Widened}{The spell's area becomes a \arealarge cone.}
    \end{spellaugments}
\end{spellsection}

\begin{spellsection}{Command}[2]
    \begin{spellheader}
        \spelldesc{You compel a foe to obey a single command of your choice.}
    \end{spellheader}
    \begin{spellcontent}
        \begin{spelltargetinginfo}
            \spellquicktargeting{One creature}{\rngclose}
            \spellcmp{Verbal only}
        \end{spelltargetinginfo}
        \begin{spelleffects}
            \spellspecial When you cast this spell, you speak a command.
            The command must be a single word or short, simple phrase.
            You must command the creature to perform a movement that can be completed during the movement phase.
            For example, you could command a creature to ``flee'' or ``grovel'', but not to ``attack'' or ``betray allies''.
            \begin{spellattack}{Spellpower vs. Mental}
                \spellsuccess The target must obey the command during the next movement phase.
                It can take no other actions during that time, but it can defend itself normally.
                During the action phase, it can act normally.
                \spellcritical As above, except that it must also obey the command during the next action phase.
                If the action was completed during the movement phase, the creature tries to complete the action again if possible, or simply remains still otherwise.
                \spellfailure The target must obey the command or be \impaired with attacks and checks.
            \end{spellattack}
            \spelldur \durbrief
        \end{spelleffects}
    \end{spellcontent}
    \begin{spellfooter}
        \spellinfo{Enchantment [Auditory, Compulsion, Mind, Speech]}*{Arcane, Divine}
        \spellnotes If the target can't understand your command, the spell automatically fails. The target must obey the literal meaning of the commmand given, potentially allowing intelligent targets to subvert your intentions.
        \miscastrandom
    \end{spellfooter}
    \begin{spellaugments}
        \spellaugment{3}{Mass}{The spell can affect up to five targets.}
    \end{spellaugments}
\end{spellsection}

\begin{spellsection}{Cone of Cold}[2]
    \begin{spellheader}
        \spelldesc{You create an area of extreme cold that drains heat from creatures in the area, diminishing their ability to move.}
    \end{spellheader}
    \begin{spellcontent}
        \begin{spelltargetinginfo}
            \spellburst{\areamed cone}
            \spelltgts{Everything in the area}
        \end{spelltargetinginfo}
        \begin{spelleffects}
            \begin{spellattack}{Spellpower vs. Reflex}
                \spellsuccess \spelldamage{cold}[d8]. In addition, the target moves at half speed for 2 rounds.
                \spellfailure As above, but half damage.
            \end{spellattack}
        \end{spelleffects}
    \end{spellcontent}
    \begin{spellfooter}
        \spellinfo{Evocation [Cold]}{Arcane, Nature}
        \miscastexplode
    \end{spellfooter}
    \begin{spellaugments}
        \spellaugment{3}{Freezing}{If the attack succeeds against a creature, it is \immobilized for 2 rounds.}
        \spellaugment{3}{Widened}{The spell's area becomes a \arealarge cone.}
        \spellaugment{4}{Empowered}{The damage increases to \spelldamage{cold}[d10]}
    \end{spellaugments}
\end{spellsection}

\begin{spellsection}{Confusion}[4]
    \begin{spellheader}
        \spelldesc{You compel a group of creatures to act randomly, sowing confusion in your foes' ranks.}
    \end{spellheader}
    \begin{spellcontent}
        \begin{spelltargetinginfo}
            \spellquicktargeting*{Up to five creatures}{\rnglong}
        \end{spelltargetinginfo}
        \begin{spelleffects}
            \begin{spellattack}{Spellpower vs. Mental}
                \spellsuccess The target is \disoriented.
                \spellcritical The target is \confused.
            \end{spellattack}
            \spelldur \durbrief
        \end{spelleffects}
    \end{spellcontent}
    \begin{spellfooter}
        \spellinfo{Enchantment [Compulsion, Mind]}{Arcane, Chaos, Divine}
        \miscastrandom
    \end{spellfooter}
    \begin{spellaugments}
        \spellaugment{2}{Focused}{One target is \disoriented even if the attack fails against it.}
    \end{spellaugments}
\end{spellsection}

\begin{spellsection}{Create Ballista}[3]
    \begin{spellheader}
        \spelldesc{You create a ballista which fires at your foes.}
    \end{spellheader}
    \begin{spellcontent}
        \begin{spelltargetinginfo}
            \spellquicktargeting{Location}{\rngclose}
        \end{spelltargetinginfo}
        \begin{spelleffects}
            \spelleffect This spell creates a fully functional Large ballista.
            Immediately after being created, the ballista fires at a foe of your choice within \rnglong range.
            It automatically reloads itself during the movement phase.

            At the beginning of each round, you may spend a swift action to control the ballista.
            If you do, it fires at a target you designate during the action phase.
            Otherwise, another creature may spend a standard action action to manually fire the ballista.

            When the ballista fires, you make a Spellpower \add 4 vs. Armor attack against the target.
            Success means the target takes \spelldamage{piercing}[d8].
            Failure means the bolt misses, and the target takes no damage.
            This is a \glossterm{Physical} effect, and does not allow \glossterm{magic resistance}.

            The ballista has hit points equal to three times your spellpower.
            In all other respects, it is treated as an ordinary ballista.
            \spelldur \durshort
        \end{spelleffects}
    \end{spellcontent}
    \begin{spellfooter}
        \spellinfo{Conjuration/Transmutation [Creation]}{Arcane}
        \spellnotes The ballista must be created on solid, stable ground, or the spell automatically fails.

        You can learn and cast spell without the Transmutation school.
        If you do, the ballista does not fire automatically or reload itself.
        However, it can still be loaded and fired manually, like an ordinary ballista.
        If it is fired manually, you still use your spellpower to determine the accuracy and damage of the ballista bolt.
        \miscastexplode
    \end{spellfooter}
    \begin{spellaugments}
        \spellaugment{2}{Dual Load}{The ballista is created with two separate bolt tracks, allowing it to fire at two different targets in the same round. It cannot fire at the same target twice.}
        \spellaugment{3}{Persistent}{The spell's duration becomes \durlong. If you use this augment again during that time, the previous effect immediately ends.}
        \spellaugment{4}{Empowered}{The damage dealt by each ballista bolt increases to \spelldamage{piercing}[d10]}
    \end{spellaugments}
\end{spellsection}

\begin{spellsection}{Create Image}[2]
    \begin{spellheader}
    \end{spellheader}
    \begin{spellcontent}
        \begin{spelltargetinginfo}
            \spellrng{\rngmed}
        \end{spelltargetinginfo}
        \begin{spelleffects}
            \spelleffect This spell creates the visual illusion of an Large or smaller object, creature, or force, as determined by you. The figment does not create sound, smell, texture, or temperature. If you concentrate as a swift action, you can move the figment anywhere within the range and alter its form for the rest of the round. For example, you could concentrate to make an illusion of human guards walk realistically across a room. If you do not concentrate on the image, it is static.

            When you cast this spell, you make a check with a bonus equal to your spellpower \add 10. Creatures can recognize the figment is created by illusory magic by interacting with it physically, or by making an Awareness check against a DR equal to your check result when casting the spell. A creature gets a \plus10 bonus on this Awareness check when using senses which should be present in the figment, but which are missing.
            \spelldur \durshort
        \end{spelleffects}
    \end{spellcontent}
    \begin{spellfooter}
        \spellinfo{Illusion [Figment]}{Arcane, Trickery}
        \spellnotes A creature that recognizes the created figment as illusory still perceives the figment normally.
        \miscastexplode
    \end{spellfooter}
    \begin{spellaugments}
        \spellaugment{1}{Sensory}{The illusion affects an additional sense: sound, smell, texture, or temperature. This augment can be used multiple times, affecting a different sense each time.}
        \spellaugment{1}{Giant}{The spell can create a figment one size category larger. This augment can be used multiple times.}
        \spellaugment{2}{Persistent}{The spell's duration becomes \durlong.}
        \spellaugment{2}{Scripted}{When you cast the spell, you set a simple script for the figment to follow. It follows that script automatically. As a swift action, you can concentrate to change the script for the remainder of the spell.}
        \spellaugment{5}{False Reality}{The spell's area becomes a 1 mile radius zone, centered on you.}
    \end{spellaugments}
\end{spellsection}

\begin{spellsection}{Create Sound}[1]
    \begin{spellheader}
        \spelldesc{You create false sounds from nowhere.}
    \end{spellheader}
    \begin{spellcontent}
        \begin{spelltargetinginfo}
            \spellrng{\rngmed}
        \end{spelltargetinginfo}
        \begin{spelleffects}
            \spelleffect You create sound from a location within range. The sound can be of any kind, but can be no louder than the sound that could be created by one human per spellpower. You can create understandable speech, but the sound is not precise enough to trigger magical effects activated by command words.
            \spelldur \durshort \dismissable
        \end{spelleffects}
    \end{spellcontent}
    \begin{spellfooter}
        \spellinfo{Illusion [Figment]}{Arcane}
        \miscastexplode
    \end{spellfooter}
\end{spellsection}

\begin{spellsection}{Cripple}[4]
    \begin{spellheader}
        \spelldesc{You render your foe's limbs useless.}
    \end{spellheader}
    \begin{spellcontent}
        \begin{spelltargetinginfo}
            \spellquicktargeting{One creature}{\rngmed}
        \end{spelltargetinginfo}
        \begin{spelleffects}
            \begin{spellattack}{Spellpower vs. Fortitude}
                \spellsuccess \spelldamage{life}. In addition, the target is \staggered for 2 rounds.

                \spellcritical As above, but instead of being staggered, the target is unable to move its limbs, including any wings. Generally, that means it is \paralyzed, except that it can move its head and mouth.

                \spellfailure Half damage, and no additional effects.
            \end{spellattack}
        \end{spelleffects}
    \end{spellcontent}
    \begin{spellfooter}
        \spellinfo{Vivimancy [Flesh]}{Arcane}
        \miscastrandom
    \end{spellfooter}
    \begin{spellaugments}
        \spellaugment{3}{Mass}{The spell can affect up to five targets. Its damage becomes \spelldamage{life}[d8].}
        \spellaugment{4}{Empowered}{The damage increases to \spelldamageemp{life}}
    \end{spellaugments}
\end{spellsection}

\begin{spellsection}{Cure Wounds}[1]
    \begin{spellheader}
        \spelldesc{You lay your hand on a creature and channel life force into it, healing some of its wounds.}
    \end{spellheader}
    \begin{spellcontent}
        \begin{spelltargetinginfo}
            \spellquicktargeting{One creature}{\rngmed}
        \end{spelltargetinginfo}
        \begin{spelleffects}
            \spelleffect The target is healed for \spelldamage{}.
        \end{spelleffects}
    \end{spellcontent}
    \begin{spellfooter}
        \spellinfo{Vivimancy [Life]}{Divine, Life, Nature}
        \miscastrandom
    \end{spellfooter}
    \begin{spellaugments}
        \spellaugment{1}{Moderate Wounds}{For every 10 points of healing, this spell can instead cure 1 critical damage.}
        \spellaugment{2}{Serious Wounds}{For every 5 points of healing, this spell can instead cure 1 critical damage.}
        \spellaugment{3}{Critical Wounds}{For every 2 points of healing, this spell can instead cure 1 critical damage.}
        \spellaugment{3}{Mass}{The spell can affect up to five targets. Its damage becomes \spelldamage{}[d8]}
        \spellaugment{4}{Empowered}{The damage increases to \spelldamageemp{}. If the Mass augment is applied, the damage instead increases to \spelldamage{}[d10].}
    \end{spellaugments}
\end{spellsection}

\begin{spellsection}{Curse of Blood and Bone}[2]
    \begin{spellheader}
        \spelldesc{You curse your foe's body, leaving it vulnerable to attacks.}
    \end{spellheader}
    \begin{spellcontent}
        \begin{spelltargetinginfo}
            \spellquicktargeting{One creature}{\rngmed}
        \end{spelltargetinginfo}
        \begin{spelleffects}
            \begin{spellattack}{Spellpower vs. Mental}
                \spellsuccess \spelldamage{life}. In addition, the target's maximum hit points are reduced by the amount of damage it takes from this effect, to a minimum of 1 hit point, for 5 rounds.
                \spellcritical As above, but double damage and the hit point reduction is permanent.
                \spellfailure Half damage, and no additional effects.
            \end{spellattack}
        \end{spelleffects}
    \end{spellcontent}
    \begin{spellfooter}
        \spellinfo{Vivimancy [Curse, Flesh]}{Arcane, Death, Divine, Evil}
        \spellnotes \cursespellnotes
        \miscastrandom
    \end{spellfooter}
    \begin{spellaugments}
        \spellaugment{4}{Empowered}{The damage increases to \spelldamageemp{life}}
    \end{spellaugments}
\end{spellsection}

\begin{spellsection}{Curse of the Wayward Mind}[5]
    \begin{spellheader}
    \end{spellheader}
    \begin{spellcontent}
        \begin{spelltargetinginfo}
            \spellquicktargeting{One creature}{\rngmed}
        \end{spelltargetinginfo}
        \begin{spelleffects}
            \begin{spellattack}{Spellpower vs. Mental}
                \spellsuccess The target is \disoriented for one hour.
                \spellcritical The target is \disoriented permanently.
                \spellfailure The target is \disoriented for 2 rounds.
            \end{spellattack}
        \end{spelleffects}
    \end{spellcontent}
    \begin{spellfooter}
        \spellinfo{Vivimancy [Curse]}{Arcane, Divine}
        \spellnotes \cursespellnotes
        \miscastrandom
    \end{spellfooter}
\end{spellsection}

\pdfbookmark[2]{D}{SpellDescriptionsD}

\begin{spellsection}{Dancing Lights}[1]
    \begin{spellheader}
        \spelldesc{You create floating lights to guide your way.}
    \end{spellheader}
    \begin{spellcontent}
        \begin{spelltargetinginfo}
            \spellrng{\rngmed}
        \end{spelltargetinginfo}
        \begin{spelleffects}
            \spelleffect This spell creates mobile sources of light. You can create up to four lights which resemble lanterns or torches, up to four glowing spheres of light, or a single glowing, vaguely humanoid shape. Regardless of their form, each light creates bright illumination in a \areamed radius, as a torch.

            As a swift action, you can move the lights as you desire through the air. They can move up to 100 feet per round, but they must always stay within range of you. Any light which goes beyond that limit winks out.
            \spelldur \durshort \dismissable
        \end{spelleffects}
    \end{spellcontent}
    \begin{spellfooter}
        \spellinfo{Illusion [Figment, Light]}{Arcane}
        \miscastexplode
    \end{spellfooter}
\end{spellsection}

\begin{spellsection}{Darkvision}[1]
    \begin{spellheader}
        \spelldesc{You grant an ally the ability to see in complete darkness.}
    \end{spellheader}
    \begin{spellcontent}
        \begin{spelltargetinginfo}
            \spellquicktargeting{One creature}{\rngtouch}
        \end{spelltargetinginfo}
        \begin{spelleffects}
            \spelleffect The target gains the ability to see 50 feet even in total darkness. Beyond 60 feet, the target can see dimly, treating areas of darkness as shadowy illumination. Darkvision does not function if a creature is in an area of bright light or is dazzled. Darkvision is black and white only, but otherwise like normal sight.
            \spelldur \durlong
        \end{spelleffects}
    \end{spellcontent}
    \begin{spellfooter}
        \spellinfo{Transmutation [Imbuement]}{Arcane}
        \spellnotes This spell does not grant the ability to see in magical darkness.
        \miscastexplode
    \end{spellfooter}
    \begin{spellaugments}
        \spellaugment{3}{Mass}{The spell can affect up to five targets.}
    \end{spellaugments}
\end{spellsection}

\begin{spellsection}{Death Knell}[3]
    \begin{spellheader}
        \spelldesc{You draw forth the ebbing life force of a creature and use it to fuel your own power.}
    \end{spellheader}
    \begin{spellcontent}
        \begin{spelltargetinginfo}
            \spellquicktargeting{One Living creature}{\rngclose}
        \end{spelltargetinginfo}
        \begin{spelleffects}
            \begin{spellattack}{Spellpower vs. Fortitude}
                \spellsuccess \spelldamage{life}. In addition, for 2 rounds, the target automatically dies if it takes critical damage while it has no hit points remaining.

                If the target dies in this way, you gain temporary hit points equal to twice your spellpower. These temporary hit points last for 1 round per level the target had.

                \spellcritical As above, except that the temporary hit points are also granted to all allies within a \arealarge radius of the dead creature.

                \spellfailure Half damage, and no additional effects.
            \end{spellattack}
        \end{spelleffects}
    \end{spellcontent}
    \begin{spellfooter}
        \spellinfo{Vivimancy [Death]}{Arcane, Death, Divine}
        \spellnotes If you take life damage, you lose all temporary hit points provided by this spell before applying the damage.
        \miscastrandom
    \end{spellfooter}
    \begin{spellaugments}
        \spellaugment{4}{Empowered}{The damage increases to \spelldamageemp{life}.}
    \end{spellaugments}
\end{spellsection}

\begin{spellsection}{Life Ward}[4]
    \begin{spellheader}
        \spelldesc{You shield your allies from spells and abilities that affect their life force.}
    \end{spellheader}
    \begin{spellcontent}
        \begin{spelltargetinginfo}
            \spellquicktargeting*{Up to five creatures}{\rngmed}
        \end{spelltargetinginfo}
        \begin{spelleffects}
            \spelleffect The target is immune to hostile \glossterm{Life} effects.
            \spelldur \durshort
        \end{spelleffects}
    \end{spellcontent}
    \begin{spellfooter}
        \spellinfo{Vivimancy [Life, Shielding]}*{Divine, Protection}
        \miscastexplode
    \end{spellfooter}
\end{spellsection}

\begin{spellsection}{Deep Slumber}[5]
    \begin{spellheader}
        \spelldesc{You fill your foes with an overpowering urge to sleep, rendering them comatose.}
    \end{spellheader}
    \begin{spellcontent}
        \begin{spelltargetinginfo}
            \spellquicktargeting{One creature}{\rngmed}
        \end{spelltargetinginfo}
        \begin{spelleffects}
            \begin{spellattack}{Spellpower vs. Mental}
                \spellsuccess The target is \blinded for 2 rounds.
                \spellcritical The target falls asleep. It cannot be awakened by any means for 2 rounds. After that time, it can be awoken by other creatures, but if left undisturbed, it will sleep until it dies.
                \spellfailure The target is \dazed for 2 rounds.
            \end{spellattack}
        \end{spelleffects}
    \end{spellcontent}
    \begin{spellfooter}
        \spellinfo{Enchantment [Compulsion, Mind]}{Arcane}
        \spellnotes Creatures that are unable to sleep, such as elves, are immune to all effects of this spell.
        \miscastrandom
    \end{spellfooter}
    \begin{spellaugments}
        \spellaugment{3}{Mass}{The spell can affect up to five targets.}
    \end{spellaugments}
\end{spellsection}

\begin{spellsection}{Deflection}[5]
    \begin{spellheader}
        \spelldesc{You shield your ally from enemy attacks, causing harmful blows to deflect away from them.}
    \end{spellheader}
    \begin{spellcontent}
        \begin{spelltargetinginfo}
            \spellquicktargeting{One willing creature}{\rngclose}
        \end{spelltargetinginfo}
        \begin{spelleffects}
            \spelleffect Physical attacks against the target have a 50\% miss chance.
            Spells and other targeted attacks are unaffected.
            \spelldur \durshort
        \end{spelleffects}
    \end{spellcontent}
    \begin{spellfooter}
        \spellinfo{Evocation [Shielding]}{Arcane, Protection}
        \miscastexplode
    \end{spellfooter}
\end{spellsection}

\begin{spellsection}{Delay Damage}[3]
    \begin{spellheader}
        \spelldesc{You partially shift yourself into the future, delaying the impact of attacks against you.}
    \end{spellheader}
    \begin{spellcontent}
        \begin{spelltargetinginfo}
            \spelltgt{You}
        \end{spelltargetinginfo}
        \begin{spelleffects}
            \spelleffect Whenever you take damage, half of the damage (rounded down) is not dealt to you immediately. This damage is tracked separately. At the end of the spell's duration, you take all of the delayed damage at once. When this happens, any damage in excess of your hit points is dealt as critical damage.
            \spelldur \durmed
        \end{spelleffects}
        \begin{spellfooter}
            \spellinfo{Abjuration/Transmutation [Shielding, Temporal]}{Divine, Nature}
            \spellnotes If this spell is dispelled or otherwise prematurely ended, you immediately take all of the delayed damage.
            \miscastexplode
        \end{spellfooter}
    \end{spellcontent}
\end{spellsection}

\begin{spellsection}{Divine Judgment}[7]
    \begin{spellheader}
    \end{spellheader}
    \begin{spellcontent}
        \begin{spelltargetinginfo}
            \spellquicktargeting{One creature}{\rnglong}
        \end{spelltargetinginfo}
        \begin{spelleffects}
            \begin{spellattack}{Spellpower vs. Mental}
                \spellsuccess \spelldamage{divine}. In addition, the target is \staggered for 5 rounds.

                \spellcritical The target dies, and divine fire utterly consumes its body. Its equipment is unaffected.

                \spellfailure Half damage, and no additional effects.
            \end{spellattack}
        \end{spelleffects}
    \end{spellcontent}
    \begin{spellfooter}
        \spellinfo{Channeling}{Divine, Good}
        \miscastrandom
    \end{spellfooter}
    \begin{spellaugments}
        \spellaugment{1}{Final}{The target's body cannot be used to restore it to life, such as with the \spell{resurrection} ritual. (See \pcref{Resurrecting the Dead}, for details).}
    \end{spellaugments}
\end{spellsection}

\begin{spellsection}{Divine Shield}[7]
    \begin{spellheader}
        \spelldesc{You shield your ally with a powerful force that protects it from danger and heals its wounds.}
    \end{spellheader}
    \begin{spellcontent}
        \begin{spelltargetinginfo}
            \spellquicktargeting{One willing creature}{\rngclose}
        \end{spelltargetinginfo}
        \begin{spelleffects}
            \spelleffect The target gains damage reduction against physical damage equal to your spellpower.
            In addition, at the end of each round, it heals hit points equal to your spellpower.
            \spelldur 5 rounds
        \end{spelleffects}
    \end{spellcontent}
    \begin{spellfooter}
        \spellinfo{Abjuration/Vivimancy [Life, Shielding]}{Divine, Life}
        \miscastexplode
    \end{spellfooter}
\end{spellsection}

\begin{spellsection}{Detect Alignment}[1]
    \begin{spellheader}
        \spelldesc{You sense the presence of creatures with a particular alignment.}
    \end{spellheader}
    \begin{spellcontent}
        \begin{spelltargetinginfo}
            \spellemanation{\arealarge cone from you}
        \end{spelltargetinginfo}
        \begin{spelleffects}
            \spelleffect As you cast this spell, you choose an alignment: good, evil, lawful, or chaotic. Anything within the spell's area that has the chosen alignment has a faint aura, visible only to you.

            As a swift action, you can concentrate on an aura to determine the strength of the aura. Most aligned creatures and magic items have a faint aura. Creatures that embody the alignment, such as outsiders with the appropriate subtype and undead,  have a moderate aura. Creatures that act directly on behalf of the alignment, such as paladins, have a strong aura. Extraordinary magical objects or effects, such as artifacts, can also have a strong aura.
            \spelldur \durshort
        \end{spelleffects}
    \end{spellcontent}
    \begin{spellfooter}
        \spellinfo{Divination [Detection]}{Divine}
        \spellnotes Each round, you can turn to detect objects in a new area. A detection spell can penetrate barriers, but 1 foot of stone, 1 inch of common metal, a thin sheet of lead, or 3 feet of wood or dirt blocks it.
        \miscastexplode
    \end{spellfooter}
    \begin{spellaugments}
        \spellaugment{2}{Empowered}{You can detect all alignments, rather than a single alignment.}
        \spellaugment{2}{Penetrating}{The spell's area penetrates all physical barriers except lead.}
        \spellaugment{2}{Widened}{The spell's area becomes a 100 ft.\ radius.}
    \end{spellaugments}
\end{spellsection}

\begin{spellsection}{Dictum}[5]
    \begin{spellheader}
        \spelldesc{You utter a powerful command, binding your foes in place.}
    \end{spellheader}
    \begin{spellcontent}
        \begin{spelltargetinginfo}
            \spellburst{\arealarge radius centered on you}
            \spelltgts{All nonlawful creatures in the area}
            \spellcmp{Verbal only}
        \end{spelltargetinginfo}
        \begin{spelleffects}
            \begin{spellattack}{Spellpower vs. Mental}
                \spellsuccess \spelldamage{divine}[d8], and the target is \immobilized for 2 rounds.
                \spellcritical As above, except that the target is \stunned for 1 round instead of slowed.
                \spellfailure Half damage, and no additional effects.
            \end{spellattack}
        \end{spelleffects}
    \end{spellcontent}
    \begin{spellfooter}
        \spellinfo{Channeling [Lawful]}{Divine, Law}
        \miscastexplode
    \end{spellfooter}
    \begin{spellaugments}
        \spellaugment{4}{Empowered}{The damage increases to \spelldamage{divine}[d10]}
    \end{spellaugments}
\end{spellsection}

\begin{spellsection}{Dimension Door}[4]
    \begin{spellheader}
    \end{spellheader}
    \begin{spellcontent}
        \begin{spelltargetinginfo}
            \spelltgt{You}
        \end{spelltargetinginfo}
        \begin{spelleffects}
            \spelleffect You teleport to a destination within 1,000 feet of you. You must clearly visualize the destination, but you do not need line of sight or line of effect. After arriving, you cannot act until the next action phase.

            If the destination is occupied, or dramatically different from how you visualized it, the spell fails.
        \end{spelleffects}
    \end{spellcontent}
    \begin{spellfooter}
        \spellinfo{Conjuration [Teleportation]}{Arcane}
        \miscastexplode
    \end{spellfooter}
    \begin{spellaugments}
        \spellaugment{1}{Companion}{You can touch a willing creature to bring them with you. This augment can be selected multiple times, allowing you to bring an additional creature each time.}
        \spellaugment{2}{Distant}{You can teleport to a destination within 10,000 feet.}
    \end{spellaugments}
\end{spellsection}

\begin{spellsection}{Dimension Slide}[2]
    \begin{spellheader}
    \end{spellheader}
    \begin{spellcontent}
        \begin{spelltargetinginfo}
            \spellquicktargeting{One willing creature}{\rngclose}
            \spelltime Standard action. If you cast this spell on yourself, you can cast it as a move action instead.
        \end{spelltargetinginfo}
        \begin{spelleffects}
            \spelleffect The target teleports to an unoccupied destination up to 100 feet away from its original location. If the destination is invalid, the spell fails.
        \end{spelleffects}
    \end{spellcontent}
    \begin{spellfooter}
        \spellinfo{Conjuration [Teleportation]}{Arcane, Travel}
        \miscastrandom
    \end{spellfooter}
    \begin{spellaugments}
        \spellaugment{1}{Airborne}{The destination does not have to be on stable ground.}
        \spellaugment{2}{Distant}{The destination can be up to 300 feet away.}
    \end{spellaugments}
\end{spellsection}

\begin{spellsection}{Dimensional Anchor}[2]
    \begin{spellheader}
        \spelldesc{You sever your foe's connection to the Astral Plane, trapping it where it is.}
    \end{spellheader}
    \begin{spellcontent}
        \begin{spelltargetinginfo}
            \spellquicktargeting{One creature}{\rngmed}
        \end{spelltargetinginfo}
        \begin{spelleffects}
            \begin{spellattack}{Spellpower vs. Mental}
                \spellsuccess  The target cannot travel extradimensionally for 5 rounds. This blocks teleportation and all planar travel abilities except planar rifts.
                \spellcritical As above, except that the effect lasts for 1 year.
                \spellfailure As above, except that the effect lasts for 1 round.
            \end{spellattack}
        \end{spelleffects}
    \end{spellcontent}
    \begin{spellfooter}
        \spellinfo{Abjuration [Thaumaturgy]}{Arcane, Divine, Magic}
        \spellnotes This spell does not interfere with the movement of creatures already on other planes when the spell is cast, nor does it block extradimensional perception or attack forms, such as summoning monsters. Also, it does not prevent summoned creatures from disappearing at the end of a summoning spell.
        \miscastrandom
    \end{spellfooter}
    \begin{spellaugments}
        \spellaugment{2}{Resilient}{The spell cannot be dispelled. It can only be removed by physically travelling to the Astral Plane, such as through a planar rift or a gate created by the \spell{gate} ritual.}
        \spellaugment{3}{Mass}{The spell can affect up to five targets.}
    \end{spellaugments}
\end{spellsection}

\begin{spellsection}{Dimensional Army}[9]
    \begin{spellheader}
        \spelldesc{You teleport your allies a great distance, ambushing your foes.}
    \end{spellheader}
    \begin{spellcontent}
        \begin{spelltargetinginfo}
            \spellquicktargeting*{Up to five willing creatures}{\rngmed}
        \end{spelltargetinginfo}
        \begin{spelleffects}
            \spelleffect The target teleports to a destination within \rngext range of you. You must clearly visualize the destination, but you do not need line of sight or line of effect. After arriving, the target cannot act until the next action phase.

            If the destination is occupied, or dramatically different from how you visualized it, the spell fails.
        \end{spelleffects}
    \end{spellcontent}
    \begin{spellfooter}
        \spellinfo{Conjuration [Teleporation]}{Arcane, Travel}
        \miscastexplode
    \end{spellfooter}
\end{spellsection}

\begin{spellsection}{Discern Lies}[2]
    \begin{spellheader}
        \spelldesc{You can discern subtle magical disturbances caused by lying.}
    \end{spellheader}
    \begin{spellcontent}
        \begin{spelltargetinginfo}
            \spellemanation{\arealarge cone from you}
        \end{spelltargetinginfo}
        \begin{spelleffects}
            \spelleffect You know when any creature in the area deliberately and knowingly speaks a lie. The spell does not reveal the truth, uncover unintentional inaccuracies, or necessarily reveal evasions.
            \spelldur \durshort
        \end{spelleffects}
    \end{spellcontent}
    \begin{spellfooter}
        \spellinfo{Divination [Detection]}{Arcane, Divine, Law}
        \spellnotes Each round, you can turn to discern lies in a new area. A detection spell can penetrate barriers, but 1 foot of stone, 1 inch of common metal, a thin sheet of lead, or 3 feet of wood or dirt blocks it.
        \miscastexplode
    \end{spellfooter}
    \begin{spellaugments}
        \spellaugment{2}{Penetrating}{The spell's area penetrates all physical barriers except lead.}
        \spellaugment{3}{Persistent}{The spell's duration becomes \durlong.}
    \end{spellaugments}
\end{spellsection}

\begin{spellsection}{Discern Vulnerability}[3]
    \begin{spellheader}
    \end{spellheader}
    \begin{spellcontent}
        \begin{spelltargetinginfo}
            \spellquicktargeting{One creature}{\rngmed}
            \spelltime{Swift action}
        \end{spelltargetinginfo}
        \begin{spelleffects}
            \spelleffect You instantly learn all of the target's weaknesses. This includes, but is not limited to, the following information:
            \begin{itemize}
                \item Which of the target's defenses is lowest
                \item If the target has any vulnerabilities to specific damage types
                \item How to overcome the target's damage reduction, regeneration, or other similar abilities
            \end{itemize}
        \end{spelleffects}
    \end{spellcontent}
    \begin{spellfooter}
        \spellinfo{Divination}{Arcane, Divine, Nature}
        \spellnotes This spell gives no information about a creature's strengths or abilities -- only its weaknesses.
        \miscastrandom
    \end{spellfooter}
    \begin{spellaugments}
        \spellaugment{3}{Mass}{The spell can affect up to five targets.}
    \end{spellaugments}
\end{spellsection}

\begin{spellsection}{Discordant Song}[7]
    \begin{spellheader}
        \spelldesc{Magical music fills the air, sowing confusion among your foes.}
    \end{spellheader}
    \begin{spellcontent}
        \begin{spelltargetinginfo}
            \spelltwocol{\spellzone{\areamed radius}}{\spellrng{\rngmed}}
            \spelltgts{All creatures in the area}
        \end{spelltargetinginfo}
        \begin{spelleffects}
            \spelleffect At the beginning of each round, make an attack against all targets.
            \begin{spellattack}{Spellpower vs. Mental}
                \spellsuccess The target is \disoriented for 2 rounds.
                \spellcritical The target is \confused for 2 rounds.
            \end{spellattack}
            \spelldur \durshort
        \end{spelleffects}
    \end{spellcontent}
    \begin{spellfooter}
        \spellinfo{Enchantment [Auditory, Compulsion, Mind]}{Arcane, Chaos}
        \miscastyou
    \end{spellfooter}
\end{spellsection}

\begin{spellsection}{Disintegrate}[6]
    \begin{spellheader}
        \spelldesc{You shoot a thin, green ray from your pointing finger that completely destroys whatever it hits.}
    \end{spellheader}
    \begin{spellcontent}
        \begin{spelltargetinginfo}
            \spellquicktargeting{One creature or object}{\rngmed}
        \end{spelltargetinginfo}
        \begin{spelleffects}
            \begin{spellattack}{Spellpower vs. Fortitude}
                \spellsuccess \spelldamage{physical}. If the target has no hit points remaining, it dies. Its body is completely disintegrated, leaving behind only a pinch of fine dust. Its equipment is unaffected.
                \spellcritical As above, but double damage.
                \spellfailure As above, but half damage.
            \end{spellattack}
            \spellspecial When used against an object, this spell simply disintegrates as much as one 10-foot cube of nonliving matter. Thus, the spell disintegrates only part of any very large object or structure targeted.
        \end{spelleffects}
    \end{spellcontent}
    \begin{spellfooter}
        \spellinfo{Transmutation [Shaping]}{Arcane, Destruction}
        \spellnotes This spell affects even objects constructed entirely of telekinetic force, such as \spell{wall of force}, but not magical effects such as an \spell{antimagic field}.
        \miscastrandom
    \end{spellfooter}
    \begin{spellaugments}
        \spellaugment{1}{Complete}{If the target is completely disintegrated, its equipment also takes damage from the spell. Equipment destroyed in this way is also disintegrated. This has no effect on artifacts.}
        \spellaugment{3}{Mass}{The spell can affect up to five targets. Its damage becomes \spelldamage{physical}[d8].}
        \spellaugment{4}{Empowered}{The damage increases to \spelldamageemp{physical}.}
    \end{spellaugments}
\end{spellsection}

\begin{spellsection}{Disjoin Magic}[9]
    \begin{spellheader}
    \end{spellheader}
    \begin{spellcontent}
        \begin{spelltargetinginfo}
            \spellspecial This spell has two versions: an area dispel, and a targeted destruction of a magic item. Its effects depend on which version is chosen.
        \end{spelltargetinginfo}
    \end{spellcontent}
    \begin{spellsubcontent}
        \begin{spelltargetinginfo}
            \spelltwocol{\spellburst{\areamed radius burst}}{\spellrng{\rngmed}}
        \end{spelltargetinginfo}
        \begin{spelleffects}
            \spelleffect All spells in the area are dispelled.
        \end{spelleffects}
    \end{spellsubcontent}
    \begin{spellsubcontent}
        \begin{spelltargetinginfo}
            \spelltgt{One magic item}
        \end{spelltargetinginfo}
        \begin{spelleffects}
            \begin{spellattack}{Spellpower vs. 10 \add the spellpower of the target object}
                \spellsuccess The target item is permanently rendered nonmagical.
                \spellfailure The target item is suppressed for 5 rounds. A suppressed object loses all its magical abilities, though it is still treated as being a magical object for the purpose of spells and effects.
                \spellspecial If the item is an artifact, there is only a 1\% chance per spellpower that the spell works. If you destroy an artifact in this way, you permanently lose the ability to cast this spell.
            \end{spellattack}
        \end{spelleffects}
    \end{spellsubcontent}
    \begin{spellfooter}
        \spellinfo{Abjuration [Thaumaturgy]}{Arcane, Magic}
        \spellnotes Destroying artifacts is dangerous, and it is likely to attract the attention of some powerful being who has an interest in or connection with the device.
        \miscastyou
    \end{spellfooter}
\end{spellsection}

\begin{spellsection}{Dispel Magic}[3]
    \begin{spellheader}
        \spelldesc{You destroy magical effects}.
    \end{spellheader}
    \begin{spellcontent}
        \begin{spelltargetinginfo}
            \spellquicktargeting{One creature, object, or location}{\rngmed}
        \end{spelltargetinginfo}
        \begin{spelleffects}
            \begin{spellattack}{Spellpower vs. Special}
                \spelleffect For every spell affecting the target, if the attack result beats a DR equal to 10 \add the spellpower of the spell, the spell is dispelled.

                If the target is an object, and the attack result beats a DR equal to 10 \add the spellpower of the object, the object is suppressed for 5 rounds. A suppressed object loses all its magical abilities, though it is still treated as being a magical object for the purpose of spells and effects.

                If the target is an effect of an ongoing spell (such as a summoned creature), and the attack result beats a DR equal to 10 \add the spellpower of the spell, the target is treated as if the spell that created it was dispelled. This usually causes the target to disappear.
            \end{spellattack}
        \end{spelleffects}
    \end{spellcontent}
    \begin{spellfooter}
        \spellinfo{Abjuration [Thaumaturgy]}{Arcane, Divine, Magic, Nature}
        \spellnotes When a spell is dispelled, all its effects with a duration end. Unless otherwise specified, any spell with a lasting effect can be dispelled.

        If a spell affects multiple targets, it must be dispelled individually on each target. Dispelling the effect on one target does not affect the other targets of the spell.

        You may choose to automatically succeed or fail on your attack against any spell that you cast yourself.

        Artifacts and deities are unaffected by mortal magic such as this.
        \spellmiscast{Retargeting} The spell targets a random active spell, or object or creature which is affected by at least one dispellable spell, within range (including yourself, if applicable).
    \end{spellfooter}
    \begin{spellaugments}
        \spellaugment{2}{Spelltheft}{You can choose to gain the effects of any spells you dispel as if they had been originally cast on you. The effects last for the remainder of their original durations or for 5 rounds, whichever is shorter. Spells that cannot be cast on you, such as spells which only affect the caster, are simply dispelled.}
        \spellaugment{3}{Mass}{The spell can affect up to five targets.}
    \end{spellaugments}
\end{spellsection}

\begin{spellsection}{Divine Favor}[3]
    \begin{spellheader}
        \spelldesc{You imbue yourself with divine fortune by calling on your patron.}
    \end{spellheader}
    \begin{spellcontent}
        \begin{spelltargetinginfo}
            \spelltgt{You}
        \end{spelltargetinginfo}
        \begin{spelleffects}
            \spelleffect You gain an legend point.
            This legend point can only be spent on physical actions, such as strikes with your weapon.
            If you spend it, you get another legend point 5 rounds later.
            \spelldur \durlong \dismissable
        \end{spelleffects}
    \end{spellcontent}
    \begin{spellfooter}
        \spellinfo{Transmutation [Enhancement]}{Divine, War}
        \miscastexplode
    \end{spellfooter}
    \begin{spellaugments}
        \spellaugment{3}{Empowered}{You gain an additional legend point. Both legend points regenerate independently after being spent.}
    \end{spellaugments}
\end{spellsection}

\begin{spellsection}{Divine Might}[5]
    \begin{spellheader}
    \end{spellheader}
    \begin{spellcontent}
        \begin{spelltargetinginfo}
            \spelltgt{You}
        \end{spelltargetinginfo}
        \begin{spelleffects}
            \spelleffect You become larger, as the \spell{enlarge} spell.
            In addition, you gain damage reduction against physical damage equal to your spellpower.
            Appropriately aligned damage ignores this damage reduction and negates it for 1 round.
            Evil attacks overcome your damage reduction if you are good or neutral, and good attacks overcome your damage reduction if you are evil.
            \spelldur \durshort \dismissable
        \end{spelleffects}
    \end{spellcontent}
    \begin{spellfooter}
        \spellinfo{Channeling/Transmutation [Shaping, Sizing]}*{Divine, Strength}
        \spellnotes \sizingspellnotes
        \miscastexplode
    \end{spellfooter}
\end{spellsection}

\begin{spellsection}{Dominate Person}[6]
    \begin{spellheader}
    \end{spellheader}
    \begin{spellcontent}
        \begin{spelltargetinginfo}
            \spellquicktargeting{One humanoid creature}{\rngclose}
            \spellcmp{Somatic only}
        \end{spelltargetinginfo}
        \begin{spelleffects}
            \begin{spellattack}{Spellpower vs. Mental}

                \spellsuccess The target is \confused for 2 rounds.

                \spellcritical The target is \dominated for 2 rounds.

                When this effect's duration ends, you must make another Spellpower vs. Mental attack against the target.
                Success means the target remains dominated for another 2 rounds, and this attack is repeated at the end of that time.
                Failure means the target breaks free of your control.
                Critical success means the target is dominated for an additional 24 hours, and no further attacks are made.

                \spellfailure The target is \dazed for 2 rounds.
            \end{spellattack}
        \end{spelleffects}
    \end{spellcontent}
    \begin{spellfooter}
        \spellinfo{Enchantment [Compulsion, Mind, Subtle]}{Arcane}
        \spellnotes This spell gives you no special ability to communicate with the target, except as noted above. Rituals such as \spell{telepathic bond} can be used to exert influence over a dominated creature from a distance.
        \miscastrandom
    \end{spellfooter}
    \begin{spellaugments}
        \spellaugment{1}{Dominate Monster}{The spell can affect creatures of any type.}
        \spellaugment{3}{Mass}{The spell can affect up to five targets.}
    \end{spellaugments}
\end{spellsection}

\begin{spellsection}{Drain Life}[3]
    \begin{spellheader}
    \end{spellheader}
    \begin{spellcontent}
        \begin{spelltargetinginfo}
            \spellquicktargeting{One living creature}{\rngmed}
        \end{spelltargetinginfo}
        \begin{spelleffects}
            \begin{spellattack}{Spellpower vs. Fortitude}
                \spellsuccess \spelldamage{life}. You gain temporary hit points equal to half the damage you deal. You can't gain more hit points than the target had.
                The temporary hit points disappear after 5 rounds. If you take life damage, you lose all temporary hit points provided by this spell before applying the damage.
                \spellcritical As above, but double damage.
                \spellfailure Half damage.
            \end{spellattack}
        \end{spelleffects}
    \end{spellcontent}
    \begin{spellfooter}
        \spellinfo{Vivimancy [Life]}{Arcane}
        \miscastrandom
    \end{spellfooter}
    \begin{spellaugments}
        \spellaugment{2}{Efficient}{You gain temporary hit points equal to the damage you deal.}
        \spellaugment{4}{Empowered}{The damage increases to \spelldamageemp{life}.}
    \end{spellaugments}
\end{spellsection}

\begin{spellsection}{Drown}[8]
    \begin{spellheader}
        \spelldesc{You fill a foe's lungs with water, causing it to begin drowning.}
    \end{spellheader}
    \begin{spellcontent}
        \begin{spelltargetinginfo}
            \spellquicktargeting{One creature}{\rngmed}
        \end{spelltargetinginfo}
        \begin{spelleffects}
            \begin{spellattack}{Spellpower vs. Fortitude}
                \spellsuccess The target is \severelyimpaired with attacks and checks.
                In addition, the target is unable to breathe.
                If it continues being unable to breathe for longer than it can hold its breath, it may die (see \pcref{Drowning}, for details).
                \spellcritical As above, except that the target is \nauseated instead of severely impaired.
                \spellfailure The target is \impaired with attacks and checks.
            \end{spellattack}
            \spelldur \durshort
        \end{spelleffects}
    \end{spellcontent}
    \begin{spellfooter}
        \spellinfo{Conjuration [Creation]}{Nature, Water}
        \spellnotes Unlike most \glossterm{Creation} spells, this spell allows \glossterm{magic resistance}.
        \miscastexplode
    \end{spellfooter}
\end{spellsection}

\pdfbookmark[2]{E}{SpellDescriptionsE}

\begin{spellsection}{Earth's Pull}[3]
    \begin{spellheader}
        \spelldesc{You intensify the pull of gravity on your foe, causing it to feel much heavier and making its movements sluggish.}
    \end{spellheader}
    \begin{spellcontent}
        \begin{spelltargetinginfo}
            \spellrng{\rngmed}
            \spelltgt{One Large or smaller creature within 10 feet of solid ground}
        \end{spelltargetinginfo}
        \begin{spelleffects}
            \spelleffect The target is \slowed.
            \spelldur \durbrief
        \end{spelleffects}
    \end{spellcontent}
    \begin{spellfooter}
        \spellinfo{Transmutation [Earth]}{Earth, Nature, Wild}
        \spellnotes If the target gets farther than 10 feet from the ground, the spell's effect is broken.
        \miscastrandom
    \end{spellfooter}
    \begin{spellaugments}
        \spellaugment{1}{Distant}{The target must be within 50 feet of solid ground, rather than within 10 feet.}
        \spellaugment{1}{Giant}{The spell can affect a target one size category larger. This augment can be used multiple times.}
        \spellaugment{3}{Mass}{The spell can affect up to five targets.}
    \end{spellaugments}
\end{spellsection}

\begin{spellsection}{Earthen Blade}[1]
    \begin{spellheader}
    \end{spellheader}
    \begin{spellcontent}
        \begin{spelltargetinginfo}
            \spellrng{Touch}
        \end{spelltargetinginfo}
        \begin{spelleffects}
            \spelleffect This spell creates a weapon sized for you from the ground. The weapon can be of any type you are proficient with.

            If earth enhanced with the \spell{enhance component} ritual is used, the item created will have an enhancement bonus.
            \spelldur \durlong \dismissable
        \end{spelleffects}
    \end{spellcontent}
    \begin{spellfooter}
        \spellinfo{Transmutation [Earth, Shaping]}{Earth, Nature, War}
        \miscastexplode
    \end{spellfooter}
    \begin{spellaugments}
        \spellaugment{1}{Tiny}{You can create a weapon sized for a creature up to two size categories smaller than you.}
        \spellaugment{2}{Arsenal}{You can create up to ten weapons of different types.}
        \spellaugment{2}{Giant}{You can create a weapon sized for a creature up to two size categories larger than you.}
    \end{spellaugments}
\end{spellsection}

\begin{spellsection}{Earth Glide}[4]
    \begin{spellheader}
    \end{spellheader}
    \begin{spellcontent}
        \begin{spelltargetinginfo}
            \spellquicktargeting{One creature}{Touch}
        \end{spelltargetinginfo}
        \begin{spelleffects}
            \spelleffect The target gains the earth glide ability, as an earth elemental. This allows it to glide through stone, dirt, or almost any other sort of earth as if it were air. The target can walk or climb at any angle in the earth. However, the target generally cannot breathe, speak, or hear while gliding. While gliding, the target can remain partially within the earth, granting it cover.
            \spelldur \durshort
        \end{spelleffects}
    \end{spellcontent}
    \begin{spellfooter}
        \spellinfo{Transmutation [Earth, Imbuement]}{Earth, Nature}
        \spellnotes The target's burrowing leaves behind no tunnel or hole, nor does it create any ripple or other signs of its presence.
        This spell does not grant the target the ability to breathe earth, and it may suffocate if it remains within earth for a prolonged period of time.

        If this spell ends while the target is inside solid earth, it is shunted to the nearest open space large enough to fit it, taking 1d6 physical damage per 5 feet moved.
        \miscastexplode
    \end{spellfooter}
    \begin{spellaugments}
        \spellaugment{1}{Accelerated}{The target's speed is doubled while gliding through earth.}
        \spellaugment{3}{Mass}{The spell can affect up to five targets.}
    \end{spellaugments}
\end{spellsection}

\begin{spellsection}{Earthquake}[9]
    \begin{spellheader}
        \spelldesc{An intense but highly localized tremor shakes the ground. The shock knocks creatures down, and rifts open in the earth to trap unwary creatures.}
    \end{spellheader}
    \begin{spellcontent}
        \begin{spelltargetinginfo}
            \spelltwocol{\spellburst{\arealarge radius}}{\spellrng{\rngmed}}
            \spelltgts{All enemies on the ground in the area}
        \end{spelltargetinginfo}
        \begin{spelleffects}
            \begin{spellattack}{Spellpower vs. Reflex}
                \spelleffect The target is knocked prone.
                \spellsuccess The target is trapped in a crack in the ground, causing it to be \immobilized. It can escape with a grapple or Escape Artist check against a DR equal to 10 \add your spellpower.
                \spellcritical As above, except that the target is \grappled by the earth instead of immobilized.
            \end{spellattack}
        \end{spelleffects}
    \end{spellcontent}
    \begin{spellfooter}
        \spellinfo{Transmutation [Earth, Physical]}*{Destruction, Divine, Earth, Nature, Strength, Wild}
        \spellnotes In terrain with unusual ground, such as rivers or swamps, this spell may have different effects.

        This is a \glossterm{Physical} effect, and does not allow \glossterm{magic resistance}.
        \miscastyou
    \end{spellfooter}
\end{spellsection}

\begin{spellsection}{Earthspike}[2]
    \begin{spellheader}
        \spelldesc{You create a spike from the ground that impales your foe, slowing its movement.}
    \end{spellheader}
    \begin{spellcontent}
        \begin{spelltargetinginfo}
            \spellrng{\rngmed}
            \spelltgt{One creature or object within 10 feet of natural earth or stone}
        \end{spelltargetinginfo}
        \begin{spelleffects}
            \begin{spellattack}{Spellpower vs. Reflex}
                \spellsuccess \spelldamage{piercing}. In addition, the target moves at half speed for 2 rounds.
                \spellcritical As above, but double damage.
                \spellfailure As above, but half damage.
            \end{spellattack}
        \end{spelleffects}
    \end{spellcontent}
    \begin{spellfooter}
        \spellinfo{Transmutation [Earth, Physical, Shaping]}{Earth, Nature}
        \spellnotes This is a \glossterm{Physical} effect, and does not allow \glossterm{magic resistance}.
        \miscastrandom
    \end{spellfooter}
    \begin{spellaugments}
        \spellaugment{2}{Impaling}{If the attack succeeds, the target is \immobilized for 2 rounds instead of moving at half speed.}
        \spellaugment{2}{Spike Trap}{The spell targets a 5-ft.\ square of ground, and lasts for 5 rounds. If an enemy steps into the area, it suffers the effects of the spell. You may only have one such trap active at once.}
        \spellaugment{4}{Empowered}{The damage increases to \spelldamageemp{piercing}.}
    \end{spellaugments}
\end{spellsection}

\begin{spellsection}{Enervation}[4]
    \begin{spellheader}
        \spelldesc{Your foe's body loses its color momentarily as you drain its life force away.}
    \end{spellheader}
    \begin{spellcontent}
        \begin{spelltargetinginfo}
            \spellquicktargeting{One creature}{\rngclose}
        \end{spelltargetinginfo}
        \begin{spelleffects}
            \spelleffect If the target is living, it gains two \conditionlink{negative levels}. This imposes a \minus2 penalty to the target's accuracy, defenses, and checks, and a penalty to its current and maximum hit points equal to twice its level.

            If the target is undead, it gains damage reduction against physical damage equal to your spellpower.
            Life damage ignores this damage reduction and negates it for 1 round.
        \end{spelleffects}
    \end{spellcontent}
    \begin{spellfooter}
        \spellinfo{Vivimancy [Life]}{Arcane, Death, Divine, Evil}
        \spellnotes These negative levels do not stack with other negative levels the target has, if any.
        \miscastrandom
    \end{spellfooter}
    \begin{spellaugments}
        \spellaugment{2}{Empowered}{If the target is living, it gains an additional negative level. This augment can be used multiple times.}
    \end{spellaugments}
\end{spellsection}

\begin{spellsection}{Enlarge}[2]
    \begin{spellheader}
    \end{spellheader}
    \begin{spellcontent}
        \begin{spelltargetinginfo}
            \spellquicktargeting{One creature (Large or smaller)}{\rngclose}
        \end{spelltargetinginfo}
        \begin{spelleffects}
            \begin{spellattack}{Spellpower vs. Fortitude}
                \spellsuccess The target and its equipment instantly grows, doubling its height and multiplying its weight by 8. This changes the creature's size category to the next larger one. This has several effects.
                \begin{itemize}
                    \item \plus4 bonus to Fortitude defense.
                    \item \minus1 penalty to other physical accuracy and defenses.
                    \item \minus4 penalty to Stealth checks.
                    \item Weapons increase damage die size (see \tref{Weapon Damage and Size}).
                    \item If the target's new size is Small or smaller, it gains a \plus5 ft.\ bonus to movement speed. Otherwise, it gains a \plus10 ft.\ bonus to movement speed.
                \end{itemize}
                \par If insufficient room is available for the desired growth, the creature attains the maximum possible size and may make a Strength check (using its increased Strength) to burst any enclosures in the process. If it fails, it is constrained without harm by the materials enclosing it -- the spell cannot be used to crush a creature by increasing its size.
                \par Equipment that leaves the target's possession returns to its original size.
                As a result, thrown and projectile weapons use the target's true size to determine their accuracy and damage, rather than its modified size.
            \end{spellattack}
            \spelldur \durshort \dismissable
        \end{spelleffects}
    \end{spellcontent}
    \begin{spellfooter}
        \spellinfo{Transmutation [Shaping, Sizing]}{Arcane, Nature, Strength}
        \spellnotes A typical humanoid creature whose size increases to Large has a space of 10 feet and a natural reach of 10 feet. \sizingspellnotes
        \miscastrandom
    \end{spellfooter}
    \begin{spellaugments}
        \spellaugment{3}{Mass}{The spell can affect up to five targets.}
    \end{spellaugments}
\end{spellsection}

\begin{spellsection}{Entangle}[1]
    \begin{spellheader}
        \spelldesc{Plants grow and ensnare your foe.}
    \end{spellheader}
    \begin{spellcontent}
        \begin{spelltargetinginfo}
            \spellquicktargeting{One creature within 5 feet of plants}{\rngclose}
        \end{spelltargetinginfo}
        \begin{spelleffects}
            \spelleffect The target is \entangled.
            It can break this effect with a grapple or Escape Artist check against a DR equal to 10 \add your spellpower.
            \spelldur \durbrief
        \end{spelleffects}
    \end{spellcontent}
    \begin{spellfooter}
        \spellinfo{Transmutation [Animation, Physical]}{Nature, Wild}
        \spellnotes The effects of this spell may be altered somewhat based on the nature of the plants near the target.
        This is a \glossterm{Physical} effect, and does not allow \glossterm{magic resistance}.
        \miscastrandom
    \end{spellfooter}
    \begin{spellaugments}
        \spellaugment{2}{Wild Growth}{The target does not need to be near plants. This is a Conjuration [ \glossterm{Creation}] effect.}
        \spellaugment{3}{Mass}{The spell can affect up to five targets.}
    \end{spellaugments}
\end{spellsection}

\begin{spellsection}{Entropic Shield}[1]
    \begin{spellheader}
        \spelldesc{You surround your ally with a magical field that glows with a chaotic blast of multicolored hues. This field deflects incoming ranged attacks, causing them to randomly swerve away from their intended target.}
    \end{spellheader}
    \begin{spellcontent}
        \begin{spelltargetinginfo}
            \spellquicktargeting{One creature}{\rngclose}
        \end{spelltargetinginfo}
        \begin{spelleffects}
            \spelleffect Each physical ranged attack directed at the target has a 50\% miss chance. Other attacks that simply work at a distance are not affected.
            \spelldur \durshort \dismissable
        \end{spelleffects}
    \end{spellcontent}
    \begin{spellfooter}
        \spellinfo{Evocation [Shielding]}{Chaos, Divine}
        \miscastrandom
    \end{spellfooter}
    \begin{spellaugments}
        \spellaugment{3}{Mass}{The spell can affect up to five targets.}
        \spellaugment{3}{Spellshield}{Ranged spells directed at the target also have a 50\% failure chance. This only affects directly targeted spells, not spells that include the target in their area.}
    \end{spellaugments}
\end{spellsection}

\begin{spellsection}{Excrete Slime}[1]
    \begin{spellheader}
        \spelldesc{You coat yourself in a sheen of acidic slime that damages your attackers.}
    \end{spellheader}
    \begin{spellcontent}
        \begin{spelltargetinginfo}
            \spelltgt{You}
        \end{spelltargetinginfo}
        \begin{spelleffects}
            \spelleffect You are coated in slime. Whenever a creature hits you with a melee attack, make a Spellpower vs. Reflex attack against the attacking creature. Success means the creature takes \spelldamage{acid}[d8].
            \spelldur \durshort
        \end{spelleffects}
    \end{spellcontent}
    \begin{spellfooter}
        \spellinfo{Transmutation [Imbuement]}{Nature}
        \miscastexplode
    \end{spellfooter}
    \begin{spellaugments}
        \spellaugment{4}{Empowered}{The damage increases to \spelldamage{acid}[d10].}
    \end{spellaugments}
\end{spellsection}

\begin{spellsection}{Eyebite}[2]
    \begin{spellcontent}
        \begin{spelltargetinginfo}
            \spellquicktargeting{One creature}{\rngmed}
        \end{spelltargetinginfo}
        \begin{spelleffects}
            \begin{spellattack}{Spellpower vs. Fortitude}
                \spellsuccess \spelldamage{life}. In addition, the target is \partiallyblinded for 2 rounds.
                \spellcritical As above, but double damage and the target is \blinded for 2 rounds instead of partially blinded.
                \spellfailure Half damage, and no additional effects.
            \end{spellattack}
        \end{spelleffects}
    \end{spellcontent}
    \begin{spellfooter}
        \spellinfo{Vivimancy [Flesh]}{Arcane}
        \spellnotes This spell has no effect on creatures without eyes.
        \miscastrandom
    \end{spellfooter}
    \begin{spellaugments}
        \spellaugment{3}{Earbite}{The target is also \deafened for 2 rounds.}
        \spellaugment{3}{Mass}{The spell can affect up to five targets. Its damage becomes \spelldamage{life}[d8].}
        \spellaugment{4}{Empowered}{The damage increases to \spelldamageemp{life}.}
    \end{spellaugments}
\end{spellsection}

\pdfbookmark[2]{F}{SpellDescriptionsF}

\begin{spellsection}{Faerie Fire}[1]
    \begin{spellheader}
    \end{spellheader}
    \begin{spellcontent}
        \begin{spelltargetinginfo}
            \spelltwocol{\spellburst{\areasmall radius}}{\spellrng{\rngmed}}
            \spelltgts{Everything in the area}
        \end{spelltargetinginfo}
        \begin{spelleffects}
            \spelleffect A pale glow surrounds and outlines the target, causing it to shed light as a candle. This imposes a \minus20 penalty to Stealth checks, and negates invisibility, concealment, and similar effects.
            \spelldur \durbrief
        \end{spelleffects}
    \end{spellcontent}
    \begin{spellfooter}
        \spellinfo{Illusion [Figment, Light]}{Nature}
        \spellnotes Illusory figments, such as those created by the \spell{create image} spell, are not outlined, which may reveal their false nature. The lights continue illuminating creatures after they leave the area.
        \miscastyou
    \end{spellfooter}
    \begin{spellaugments}
        \spellaugment{2}{Widened}{The spell's area becomes a \areamed radius.}
        \spellaugment{4}{Blinding}{Creatures in the area are also \partiallyblinded.}
    \end{spellaugments}
\end{spellsection}

\begin{spellsection}{Fear}[2]
    \begin{spellheader}
        \spelldesc{You terrify your foe.}
    \end{spellheader}
    \begin{spellcontent}
        \begin{spelltargetinginfo}
            \spellquicktargeting{One creature}{\rngclose}
        \end{spelltargetinginfo}
        \begin{spelleffects}
            \begin{spellattack}{Spellpower vs. Mental}
                \spellsuccess The target is \frightened by you.
                \spellcritical The target is \panicked by you.
                \spellfailure The target is \shaken by you.
            \end{spellattack}
            \spelldur \durbrief
        \end{spelleffects}
    \end{spellcontent}
    \begin{spellfooter}
        \spellinfo{Enchantment [Delusion, Mind]}{Arcane}
        \miscastrandom
    \end{spellfooter}
    \begin{spellaugments}
        \spellaugment{1}{Redirected}{The target is instead afraid of a willing ally within \rngmed range.}
        \spellaugment{3}{Mass}{The spell can affect up to five targets.}
    \end{spellaugments}
\end{spellsection}

\begin{spellsection}{Feather Fall}[1]
    \begin{spellheader}
    \end{spellheader}
    \begin{spellcontent}
        \begin{spelltargetinginfo}
            \spelltwocol{One freefalling object or willing creature (Medium or smaller)}{\spellrng{\rngmed}}
            \spelltwocol{\spelltime{Swift action}}{\spellcmp{Verbal only}}
        \end{spelltargetinginfo}
        \begin{spelleffects}
            \spelleffect The target falls at only 60 feet per round (equivalent to the end of a fall from a few feet). It takes no falling damage from falls of any length. If the object is heavy enough to deal falling damage to other creatures and objects, it deals half its normal falling damage, with no bonus for the height of the drop.
            \spelldur \durshort
        \end{spelleffects}
    \end{spellcontent}
    \begin{spellfooter}
        \spellinfo{Transmutation [Air]}{Arcane}
        \spellnotes This spell works only upon free-falling objects and creatures. It no special effect on ranged weapons or projectiles unless they are falling an extraordinary distance.
        \miscastrandom
    \end{spellfooter}
    \begin{spellaugments}
        \spellaugment{1}{Giant}{The spell can affect a target one size category larger. This augment can be used multiple times.}
        \spellaugment{3}{Mass}{The spell can affect up to five targets.}
    \end{spellaugments}
\end{spellsection}

\begin{spellsection}{Feeblemind}[7]
    \begin{spellheader}
    \end{spellheader}
    \begin{spellcontent}
        \begin{spelltargetinginfo}
            \spellquicktargeting{One creature}{\rngclose}
        \end{spelltargetinginfo}
        \begin{spelleffects}
            \begin{spellattack}{Spellpower vs. Mental}
                \spellsuccess The target's Intelligence drops to \minus9 for 5 rounds, giving it roughly the intellect of a lizard. It is unable to cast spells, understand language, or communicate coherently. Still, it knows who its friends are and can follow them and even protect them.
                \spellcritical As above, except that the effect is permanent.
                \spellfailure The target is \dazed for 5 rounds.
            \end{spellattack}
        \end{spelleffects}
    \end{spellcontent}
    \begin{spellfooter}
        \spellinfo{Enchantment [Delusion, Mind]}{Arcane}
        \miscastrandom
    \end{spellfooter}
\end{spellsection}

\begin{spellsection}{Finger of Death}[6]
    \begin{spellheader}
    \end{spellheader}
    \begin{spellcontent}
        \begin{spelltargetinginfo}
            \spellquicktargeting{One living creature}{\rngmed}
        \end{spelltargetinginfo}
        \begin{spelleffects}
            \begin{spellattack}{Spellpower vs. Fortitude}
                \spellsuccess \spelldamage{life}. In addition, the target is \staggered for 5 rounds.
                \spellcritical The target dies.
                \spellfailure Half damage, and no additional effects.
            \end{spellattack}
        \end{spelleffects}
    \end{spellcontent}
    \begin{spellfooter}
        \spellinfo{Vivimancy [Death]}{Arcane, Death}
        \miscastrandom
    \end{spellfooter}
    \begin{spellaugments}
        \spellaugment{1}{Final}{The target's body cannot be used to restore it to life, such as with the \spell{resurrection} ritual. (See \pcref{Resurrecting the Dead}, for details).}
        \spellaugment{3}{Mass}{The spell can affect up to five targets. Its damage becomes \spelldamage{life}[d8].}
        \spellaugment{4}{Empowered}{The damage increases to \spelldamageemp{life}.}
    \end{spellaugments}
\end{spellsection}

\begin{spellsection}{Fire Shield}[4]
    \begin{spellheader}
        \spelldesc{You appear to immolate yourself in a wreath of flame that lashes out at anyone who tries to harm you.}
    \end{spellheader}
    \begin{spellcontent}
        \begin{spelltargetinginfo}
            \spellquicktargeting{One creature}{\rngclose}
        \end{spelltargetinginfo}
        \begin{spelleffects}
            \spelleffect The target gains damage reduction against cold damage equal to twice your spellpower. In addition, it radiates light as a torch.

            When a creature within \rngclose range of the target attacks it, the attacking creature takes \spelldamage{fire}[d6].
            A creature can only be dealt damage by this spell once per round.
            \spelldur \durshort \dismissable
        \end{spelleffects}
    \end{spellcontent}
    \begin{spellfooter}
        \spellinfo{Evocation [Fire, Shielding]}{Arcane, Fire}
        \miscastexplode
    \end{spellfooter}
    \begin{spellaugments}
        \spellaugment{4}{Empowered}{The damage increases to \spelldamage{fire}[d10].}
    \end{spellaugments}
\end{spellsection}

\begin{spellsection}{Fire Storm}[8]
    \begin{spellheader}
        \spelldesc{You fill a massive area with sheets of roaring flame, burning everyone who opposes you.}
    \end{spellheader}
    \begin{spellcontent}
        \begin{spelltargetinginfo}
            \spelltwocol{\spellburst{\arealarge radius}}{\spellrng{\rngmed}}
            \spelltgts{Everything in the area, except allied creatures and plants}
        \end{spelltargetinginfo}
        \begin{spelleffects}
            \begin{spellattack}{Spellpower vs. Reflex}
                \spellsuccess \spelldamage{fire}[d8].
                \spellcritical Double damage.
                \spellfailure Half damage.
            \end{spellattack}
        \end{spelleffects}
    \end{spellcontent}
    \begin{spellfooter}
        \spellinfo{Evocation [Fire]}{Destruction, Fire, Nature, War, Wild}
        \miscastyou
    \end{spellfooter}
\end{spellsection}

\begin{spellsection}{Fireball}[3]
    \begin{spellheader}
        \spelldesc{You create an explosion of flame that detonates with a low roar, damaging nearby creatures and objects.}
    \end{spellheader}
    \begin{spellcontent}
        \begin{spelltargetinginfo}
            \spelltwocol{\spellburst{\areasmall radius}}{\spellrng{\rngmed}}
            \spelltgts{Everything in the area}
        \end{spelltargetinginfo}
        \begin{spelleffects}
            \begin{spellattack}{Spellpower vs. Reflex}
                \spellsuccess \spelldamage{fire}[d8].
                \spellcritical Double damage.
                \spellfailure Half damage.
            \end{spellattack}
        \end{spelleffects}
    \end{spellcontent}
    \begin{spellfooter}
        \spellinfo{Evocation [Fire]}{Arcane, Fire, Nature}
        \spellnotes \firespellnotes
        \miscastyou
    \end{spellfooter}
    \begin{spellaugments}
        \spellaugment{2}{Delayed}{The fireball's detonation is delayed. As a swift action, you can command it to detonate during the next action phase. If not commanded to detonate, it will automatically detonate after 5 rounds. While the spell is delayed, an intangible bead of fire sits at the point of origin, shedding light as a torch.}
        \spellaugment{2}{Incendiary}{A successful attack also makes a target \ignited for 5 rounds.}
        \spellaugment{2}{Widened}{The spell's area becomes a \areamed radius.}
        \spellaugment{4}{Empowered}{The damage increases to \spelldamage{fire}[d10].}
    \end{spellaugments}
\end{spellsection}

\begin{spellsection}{Fissure}[6]
    \begin{spellheader}
        \spelldesc{You open a rift in the earth beneath your opponent that slams shut with immense force.}
    \end{spellheader}
    \begin{spellcontent}
        \begin{spelltargetinginfo}
            \spellquicktargeting{One Huge or smaller creature within 10 feet of solid ground}{\rngmed}
        \end{spelltargetinginfo}
        \begin{spelleffects}
            \begin{spellattack}{Spellpower vs. Reflex}
                \spellsuccess \spelldamage{bludgeoning}. In addition, the target is \immobilized for 5 rounds.
                \spellcritical The target dies.
                \spellfailure Half damage, and no additional effects.
            \end{spellattack}
        \end{spelleffects}
    \end{spellcontent}
    \begin{spellfooter}
        \spellinfo{Transmutation [Earth, Physical]}{Earth, Nature}
        \spellnotes This is a \glossterm{Physical} effect, and does not allow \glossterm{magic resistance}.
        \miscastrandom
    \end{spellfooter}
    \begin{spellaugments}
        \spellaugment{1}{Giant}{The spell can affect a target one size category larger. This augment can be used multiple times.}
        \spellaugment{3}{Mass}{The spell can affect up to five targets. Its damage becomes \spelldamage{bludgeoning}[d8].}
        \spellaugment{4}{Empowered}{The damage increases to \spelldamageemp{bludgeoning}.}
    \end{spellaugments}
\end{spellsection}

\begin{spellsection}{Flame Blade}[2]
    \begin{spellheader}
        \spelldesc{You create a 3 foot long beam of red-hot fire to serve you as a weapon.}
    \end{spellheader}
    \begin{spellcontent}
        \begin{spelleffects}
            \spelleffect A scimitar-like weapon appears in your hand. You can attack with it as a light melee weapon, except that you use your spellpower in place of your Strength for damage, and it deals both fire and slashing damage.

            Alternately, you can hurl flames from the weapon up to \rngmed range as if it were a thrown weapon.
            \spelldur \durlong \dismissable
        \end{spelleffects}
    \end{spellcontent}
    \begin{spellfooter}
        \spellinfo{Evocation [Fire]}{Nature, War}
        \spellnotes \glossterm{Magic resistance} applies when a foe is struck by the weapon, but not when the blade is created.
        \firespellnotes
        \miscastexplode
    \end{spellfooter}
    \begin{spellaugments}
        \spellaugment{1}{Empowered}{The scimitar gains a bonus to damage equal to one quarter of your spellpower.}
        \spellaugment{2}{Persistent}{The spell's duration becomes \durext.}
    \end{spellaugments}
\end{spellsection}

\begin{spellsection}{Flame Strike}[5]
    \begin{spellheader}
        \spelldesc{You call a vertical column of divine fire that roars downward, consuming your unworthy foes.}
    \end{spellheader}
    \begin{spellcontent}
        \begin{spelltargetinginfo}
            \spelltwocol{\spellburst{\areamed radius cylinder, 40 ft.\ high}}{\spellrng{\rngmed}}
            \spelltgts{Everything in the area}
        \end{spelltargetinginfo}
        \begin{spelleffects}
            \begin{spellattack}{Spellpower vs. Reflex}
                \spellsuccess \spelldamage{fire and divine}[d8]. Allied creatures take half damage, and all of that damage is fire damage.
                \spellcritical As above, but double damage.
                \spellfailure As above, but half damage.
            \end{spellattack}
        \end{spelleffects}
    \end{spellcontent}
    \begin{spellfooter}
        \spellinfo{Evocation [Fire]}{Destruction, Divine, Fire}
        \spellnotes \firespellnotes
        \miscastyou
    \end{spellfooter}
    \begin{spellaugments}
        \spellaugment{2}{Widened}{The area increases to a \arealarge radius cylinder, 100 ft.\ high.}
        \spellaugment{4}{Empowered}{The damage increases to \spelldamage{fire and divine}[d10].}
    \end{spellaugments}
\end{spellsection}

\begin{spellsection}{Fly}[4]
    \begin{spellheader}
    \end{spellheader}
    \begin{spellcontent}
        \begin{spelltargetinginfo}
            \spellquicktargeting{One creature}{\rngtouch}
        \end{spelltargetinginfo}
        \begin{spelleffects}
            \spelleffect The target gains a 30 foot fly speed with good maneuverability.
            \spelldur \durshort
        \end{spelleffects}
    \end{spellcontent}
    \begin{spellfooter}
        \spellinfo{Transmutation [Imbuement]}{Arcane}
        \spellnotes An unencumbered creature with a fly speed can fly through the air. See \pcref{Flying}, for more details.
        \miscastexplode
    \end{spellfooter}
    \begin{spellaugments}
        \spellaugment{1}{Accelerated}{The spell's fly speed becomes 60 feet.}
        \spellaugment{1}{Maneuverable}{The spell's maneuverability becomes perfect.}
        \spellaugment{3}{Mass}{The spell can affect up to five targets.}
    \end{spellaugments}
\end{spellsection}

\begin{spellsection}{Fog Cloud}[2]
    \begin{spellheader}
        \spelldesc{You conjure a bank of fog, concealing those inside.}
    \end{spellheader}
    \begin{spellcontent}
        \begin{spelltargetinginfo}
            \spelltwocol{\spellzone{\areamed radius cylinder}}{\spellrng{\rngmed}}
        \end{spelltargetinginfo}
        \begin{spelleffects}
            \spelleffect Fog blocks sight in the area, causing all creatures within or looking through the area to treat everything they see as if it had \concealment.
            \spelldur \durshort
        \end{spelleffects}
    \end{spellcontent}
    \begin{spellfooter}
        \spellinfo{Conjuration [Creation]}{Arcane, Nature, Water}
        \spellnotes \fogspellnotes \fogwindspellnotes

        \physicalspellnotes
        \miscastyou
    \end{spellfooter}
    \begin{spellaugments}
        \spellaugment{1}{Sickening}{Creatures within the area are \sickened for as long as they remain within the cloud, and for 1 round after they leave.}
        %\spellaugment{3}{Shielding}{The spell's area becomes an emanation from you that moves with you. When you move, new fog does not form immediately. At the end of each round, the fog in your previous location disappears, and new fog forms around your current location.}[Abjuration]
        \spellaugment{3}{Solid}{All creatures in the area move at half speed and suffer penalties as if they were fighting underwater. Attacks entering or passing through the area are similarly penalized.}
        \spellaugment{4}{Acidic}{At the end of every round, everything in the fog takes 1d10 acid damage per four spellpower.}[Acid]
        %\spellaugment{5}{Sea of Fog}{The spell's area becomes a 50 ft.\ high, 500 ft.\ radius cylinder centered on you. A severe wind disperses the fog within 1 minute, a windstorm disperses it within 5 rounds, and a hurricane disperses it within a round.}
    \end{spellaugments}
\end{spellsection}

\begin{spellsection}{Forcecage}[8]
    \begin{spellheader}
    \end{spellheader}
    \begin{spellcontent}
        \begin{spelltargetinginfo}
            \spellquicktargeting{One creature or object (Large or smaller)}{\rngmed}
        \end{spelltargetinginfo}
        \begin{spelleffects}
            \spelleffect An immobile, invisible prison appears around the target. The prison can be a perfect sphere, a perfect cube, or a barred cage. The cage bars are an inch wide, with one inch gaps between them.
            \spelldur 5 rounds
        \end{spelleffects}
    \end{spellcontent}
    \begin{spellfooter}
        \spellinfo{Evocation [Telekinesis]}{Arcane}
        \spellnotes The walls and ceiling can be destroyed. A 5-foot square of wall has hit points equal to five times your spellpower, and hardness equal to your spellpower.
        \miscastrandom
    \end{spellfooter}
\end{spellsection}

\begin{spellsection}{Forget}[1]
    \begin{spellheader}
    \end{spellheader}
    \begin{spellcontent}
        \begin{spelltargetinginfo}
            \spellquicktargeting{One creature}{\rngmed}
        \end{spelltargetinginfo}
        \begin{spelleffects}
            \begin{spellattack}{Spellpower vs. Mental}
                \spellsuccess The target forgets something simple for 1 hour. You can't make it forget something important, such as its name. You must know what you want it to forget. The spell does not prevent the target from learning the information again, and it can remember the information normally after the spell's duration.
                \spellcritical As above, except that the effect is permanent.
            \end{spellattack}
            \spelldur \durlong
        \end{spelleffects}
    \end{spellcontent}
    \begin{spellfooter}
        \spellinfo{Enchantment [Delusion]}{Arcane}
        \miscastrandom
    \end{spellfooter}
    \begin{spellaugments}
        \spellaugment{3}{Empowered}{For the duration of the spell, if the target learns the information again, it immediately forgets at the end of the round.}
        \spellaugment{3}{Mass}{The spell can affect up to five targets. All targets must forget the same thing.}
    \end{spellaugments}
\end{spellsection}

\begin{spellsection}{Freedom}[2]
    \begin{spellheader}
    \end{spellheader}
    \begin{spellcontent}
        \begin{spelltargetinginfo}
            \spellquicktargeting{One creature}{\rngclose}
        \end{spelltargetinginfo}
        \begin{spelleffects}
            \spelleffect The target is immune to effects that restrict its mobility. It suffers no penalties for acting underwater. In addition, it gains a \plus20 bonus to Reflex defense against grapple attacks, as well as on grapple attacks or Escape Artist checks made to escape a grapple or a pin.
            \spelldur \durshort
        \end{spelleffects}
    \end{spellcontent}
    \begin{spellfooter}
        \spellinfo{Transmutation [Imbuement]}{Divine, Nature}
        \miscastrandom
    \end{spellfooter}
    \begin{spellaugments}
        \spellaugment{3}{Mass}{The spell can affect up to five targets.}
    \end{spellaugments}
\end{spellsection}

\begin{spellsection}{Fungal Growth}[3]
    \begin{spellheader}
        \spelldesc{You create fungus all over your foe's body.}
    \end{spellheader}
    \begin{spellcontent}
        \begin{spelltargetinginfo}
            \spellquicktargeting{One creature}{\rngclose}
        \end{spelltargetinginfo}
        \begin{spelleffects}
            \spelleffect The target becomes covered in living fungus. It is \sickened, and after every 2 rounds it takes physical damage equal to your spellpower.

            The fungus can be removed as a full-round action. The creature removing the fungus must make a Heal check against a DR equal to 10 \add your spellpower. If it fails, the target takes additional physical damage equal to your spellpower.

            If the target takes fire or cold damage equal to your spellpower, the fungus is destroyed.
        \end{spelleffects}
    \end{spellcontent}
    \begin{spellfooter}
        \spellinfo{Conjuration/Vivimancy [Creation, Flesh]}{Nature}
        \spellnotes Unlike most \glossterm{Creation} spells, this spell allows \glossterm{magic resistance}.
        \miscastexplode
    \end{spellfooter}
    \begin{spellaugments}
        \spellaugment{1}{Sickening}{The target is \sickened until the fungus is removed.}
        \spellaugment{2}{Deep Growth}{The DR to remove the fungus with a Heal check increases by 10.}
        \spellaugment{2}{Resilient}{The damage required to remove the fungus increases to five times your spellpower.}
    \end{spellaugments}
\end{spellsection}

\pdfbookmark[2]{G}{SpellDescriptionsG}

\begin{spellsection}{Gaseous Form}[3]
    \begin{spellheader}
        \spelldesc{The target and all its equipment becomes insubstantial, misty, and translucent.}
    \end{spellheader}
    \begin{spellcontent}
        \begin{spelltargetinginfo}
            \spellquicktargeting{One willing corporeal creature}{\rngtouch}
        \end{spelltargetinginfo}
        \begin{spelleffects}
            \spelleffect The target becomes a cloud of mist. All its equipment melds into its new form, though magical equipment retains its effects. Its Armor defense becomes 10, but it is immune to physical damage and critical hits.

            As a cloud of mist, the target cannot take any physical actions other than movement. It has a fly speed of 10 feet, with perfect maneuverability. It can pass through holes and openings as narrow as one quarter inch, but cannot enter water or similar liquids.
            \spelldur \durshort \dismissable
        \end{spelleffects}
    \end{spellcontent}
    \begin{spellfooter}
        \spellinfo{Transmutation [Shaping]}{Arcane, Travel}
        \miscastexplode
    \end{spellfooter}
    \begin{spellaugments}
        \spellaugment{2}{Accelerated}{The fly speed increases to 30 feet.}
    \end{spellaugments}
\end{spellsection}

\begin{spellsection}{Gentle Descent}[1]
    \begin{spellheader}
        \spelldesc{You grant your ally ephemeral wings which allow him to glide.}
    \end{spellheader}
    \begin{spellcontent}
        \begin{spelltargetinginfo}
            \spellquicktargeting{One creature}{\rngclose}
        \end{spelltargetinginfo}
        \begin{spelleffects}
            \spelleffect The target gains a 30 foot glide speed.
            \spelldur \durshort
        \end{spelleffects}
    \end{spellcontent}
    \begin{spellfooter}
        \spellinfo{Transmutation [Air, Imbuement]}{Air, Nature}
        \spellnotes A creature with a glide speed can glide through the air at the indicated speed (see \pcref{Gliding}).
        \miscastrandom
    \end{spellfooter}
    \begin{spellaugments}
        \spellaugment{1}{Accelerated}{The glide speed increases to 60 feet.}
        \spellaugment{3}{Mass}{The spell can affect up to five targets.}
    \end{spellaugments}
\end{spellsection}

\begin{spellsection}{Ghoul Touch}[4]
    \begin{spellheader}
        \spelldesc{Your foe feels the touch of a ghoul's undead hand against its flesh.}
    \end{spellheader}
    \begin{spellcontent}
        \begin{spelltargetinginfo}
            \spellquicktargeting{One living creature}{\rngmed}
        \end{spelltargetinginfo}
        \begin{spelleffects}
            \begin{spellattack}{Spellpower vs. Fortitude}
                \spellsuccess The target is \staggered and \impaired with attacks and checks.
                \spellcritical The target is \paralyzed.
                \spellfailure The target is \impaired with attacks and checks.
            \end{spellattack}
            \spelldur \durbrief
        \end{spelleffects}
    \end{spellcontent}
    \begin{spellfooter}
        \spellinfo{Vivimancy [Flesh]}{Arcane}
        \miscastrandom
    \end{spellfooter}
    \begin{spellaugments}
        \spellaugment{3}{Mass}{The spell can affect up to five targets.}
    \end{spellaugments}
\end{spellsection}

\begin{spellsection}{Glitterdust}[2]
    \begin{spellheader}
        \spelldesc{You create a flurry of glittering dust which reveals hidden foes and blinds creatures.}
    \end{spellheader}
    \begin{spellcontent}
        \begin{spelltargetinginfo}
            \spelltwocol{\spellburst{\areasmall radius}}{\spellrng{\rngmed}}
            \spelltgts{Everything in the area}
        \end{spelltargetinginfo}
        \begin{spelleffects}
            \spelleffect A flurry of golden particles swarm around the target. This imposes a \minus20 penalty to Stealth checks, and negates invisibility, concealment, and similar effects. Illusory figments, such as those created by the \spell{create image} spell, are not outlined, which may reveal their false nature. In addition, you make an attack.
            \begin{spellattack}{Spellpower vs. Reflex}
                \spellsuccess The target is \partiallyblinded.
                \spellcritical The target is \blinded.
            \end{spellattack}
            \spelldur \durbrief
        \end{spelleffects}
    \end{spellcontent}
    \begin{spellfooter}
        \spellinfo{Conjuration [Creation]}{Arcane}
        \miscastyou
    \end{spellfooter}
\end{spellsection}

\begin{spellsection}{Golem Heart}[7]
    \begin{spellheader}
        \spelldesc{Your skin becomes gray and metallic as you embody the power of golems.}
    \end{spellheader}
    \begin{spellcontent}
        \begin{spelltargetinginfo}
            \spelltgt{You}
        \end{spelltargetinginfo}
        \begin{spelleffects}
            \spelleffect This spell grants you multiple resistances and immunities.
            \begin{itemize}
                \item You gain damage reduction against physical damage equal to your spellpower. Adamantine weapons ignore this damage reduction and negate it for 1 round.
                \item You gain \glossterm{magic resistance} equal to 10 \add your spellpower.
                \item You become immune to poison and disease.
            \end{itemize}
            \spelldur \durshort
        \end{spelleffects}
    \end{spellcontent}
    \begin{spellfooter}
        \spellinfo{Abjuration/Transmutation [Enhancement, Shielding]}{Arcane, Earth, Strength}
        \miscastexplode
    \end{spellfooter}
\end{spellsection}

\begin{spellsection}{Grease}[1]
    \begin{spellheader}
        \spelldesc{You conjure a layer of slippery grease on the ground, tripping up your foes.}
    \end{spellheader}
    \begin{spellcontent}
        \begin{spelltargetinginfo}
            \spelltwocol{\spellzone{\areasmall radius}}{\spellrng{\rngclose}}
        \end{spelltargetinginfo}
        \begin{spelleffects}
            \spelleffect The ground in the area is covered in grease for 5 rounds, making it slippery. A DR 15 Acrobatics check is usually required to move on oily surfaces. See \pcref{Balance}, for more details.
        \end{spelleffects}
    \end{spellcontent}
    \begin{spellfooter}
        \spellinfo{Conjuration [Creation]}{Arcane}
        \miscastyou
    \end{spellfooter}
    \begin{spellaugments}
        \spellaugment{2}{Widened}{The spell's area becomes a \areamed radius.}
        \spellaugment{3}{Slick}{The DR of the Acrobatics check is equal to 10 \add your spellpower.}
    \end{spellaugments}
\end{spellsection}

\begin{spellsection}{Gust of Wind}[1]
    \begin{spellheader}
        \spelldesc{You create a severe blast of air that knocks your foes flying.}
    \end{spellheader}
    \begin{spellcontent}
        \begin{spelltargetinginfo}
            \spellburst{\arealarge line from you}
            \spelltgts{Everything in the area}
        \end{spelltargetinginfo}
        \begin{spelleffects}
            \begin{spellattack}{Spellpower vs. Fortitude Defense}
                \spellsuccess The target is affected by a shove attack, pushing it back by 5 feet \add 5 feet per 5 points by which your attack exceeded its defense. If it is pushed outside the spell's area, it is not pushed farther.
            \end{spellattack}
        \end{spelleffects}
    \end{spellcontent}
    \begin{spellfooter}
        \spellinfo{Evocation [Air]}{Air, Nature}
        \spellnotes In addition to the effect noted, a \spell{gust of wind} can do anything that a sudden blast of wind would be expected to do. It can extinguish open flames, create a stinging spray of sand or dust, fan a large fire, overturn delicate awnings or hangings, heel over a small boat, and blow gases or vapors to the edge of its range.
        \miscastexplode
    \end{spellfooter}
    \begin{spellaugments}
        \spellaugment{3}{Persistent}{The gust of wind continues in the same area for 5 rounds. At the end of each round, its effect happens again.}
    \end{spellaugments}
\end{spellsection}

\pdfbookmark[2]{H}{SpellDescriptionsH}

\begin{spellsection}{Harm}[6]
    \begin{spellheader}
        \spelldesc{You attack your foe's life force, weakening its body.}
    \end{spellheader}
    \begin{spellcontent}
        \begin{spelltargetinginfo}
            \spellquicktargeting{One creature}{\rngmed}
        \end{spelltargetinginfo}
        \begin{spelleffects}
            \begin{spellattack}
                \spellsuccess The target takes \spelldamage{life}. If it takes damage in this way, it is \impaired with attacks and checks for 2 rounds. All damage dealt in excess of the target's hit points is dealt as critical damage.
                \spellcritical As above, but double damage.
                \spellfailure As above, but half damage.
            \end{spellattack}
        \end{spelleffects}
    \end{spellcontent}
    \begin{spellfooter}
        \spellinfo{Vivimancy [Life]}{Arcane, Divine, Evil}
        \miscastrandom
    \end{spellfooter}
    \begin{spellaugments}
        \spellaugment{3}{Mass}{The spell can affect up to five targets. Its damage and healing become \spelldamage{life}[d8].}
        \spellaugment{4}{Empowered}{The damage increases to \spelldamageemp{life}.}
    \end{spellaugments}
\end{spellsection}

\begin{spellsection}{Haste}[1]
    \begin{spellheader}
    \end{spellheader}
    \begin{spellcontent}
        \begin{spelltargetinginfo}
            \spellquicktargeting{One creature}{\rngclose}
        \end{spelltargetinginfo}
        \begin{spelleffects}
            \spelleffect The target gains a \plus30 foot bonus to its speed in all its movement modes, up to a maximum of double its original speed.
            \spelldur \durshort \dismissable
        \end{spelleffects}
    \end{spellcontent}
    \begin{spellfooter}
        \spellinfo{Transmutation [Temporal]}{Arcane, Strength}
        \spellnotes As with any effect that increases your speed, this effect affects your ability to jump (see \pcref{Jump}).
        \miscastrandom
    \end{spellfooter}
    \begin{spellaugments}
        \spellaugment{3}{Mass}{The spell can affect up to five targets.}
        \spellaugment{4}{Empowered}{Whenever the target takes a standard attack action, it may make an additional strike with a \minus5 penalty to accuracy. This does not stack with any other effects which grant extra strikes.}
    \end{spellaugments}
\end{spellsection}

\begin{spellsection}{Heal}[6]
    \begin{spellheader}
        \spelldesc{You fill an ally with a massive influx of life force, restoring its body to perfect health.}
    \end{spellheader}
    \begin{spellcontent}
        \begin{spelltargetinginfo}
            \spellquicktargeting{One creature}{\rngclose}
        \end{spelltargetinginfo}
        \begin{spelleffects}
            \spelleffect The target is healed for \spelldamage{}.
            For every point of healing granted by this spell, it can instead cure 1 point of critical damage.
        \end{spelleffects}
    \end{spellcontent}
    \begin{spellfooter}
        \spellinfo{Vivimancy [Life]}{Divine, Good, Life, Nature}
        \miscastrandom
    \end{spellfooter}
    \begin{spellaugments}
        \spellaugment{3}{Mass}{The spell can affect up to five targets. Its healing becomes \spelldamage{}[d8].}
        \spellaugment{4}{Empowered}{The healing increases to \spelldamageemp{}.}
    \end{spellaugments}
\end{spellsection}

\begin{spellsection}{Hold Person}[1]
    \begin{spellheader}
    \end{spellheader}
    \begin{spellcontent}
        \begin{spelltargetinginfo}
            \spellquicktargeting{One humanoid creature}{\rngclose}
        \end{spelltargetinginfo}
        \begin{spelleffects}
            \spelleffect The target is \immobilized for 2 rounds.
        \end{spelleffects}
    \end{spellcontent}
    \begin{spellfooter}
        \spellinfo{Enchantment [Compulsion, Mind]}{Arcane, Divine, Law}
        \miscastrandom
    \end{spellfooter}
    \begin{spellaugments}
        \spellaugment{1}{Hold Monster}{The spell can affect creatures of any type.}
        \spellaugment{3}{Mass}{The spell can affect up to five targets.}
    \end{spellaugments}
\end{spellsection}

\begin{spellsection}{Holy Avatar}[9]
    \begin{spellheader}
        \spelldesc{You embody the essence of good, allowing you to smite your foes.}
    \end{spellheader}
    \begin{spellcontent}
        \begin{spelltargetinginfo}
            \spelltgt{You}
        \end{spelltargetinginfo}
        \begin{spelleffects}
            \spelleffect At any time during the spell's duration, you can concentrate as a standard action. If you do, you smite a foe, as described below.
            \spelldur \durlong
        \end{spelleffects}
    \end{spellcontent}
    \begin{spellsubcontent}
        \begin{spelltargetinginfo}
            \spellquicktargeting{One nongood creature}{\rngmed}
        \end{spelltargetinginfo}
        \begin{spelleffects}
            \spelleffect The target takes 1d10 divine damage per two spellpower. In addition, it is \dazed for 2 rounds.
        \end{spelleffects}
    \end{spellsubcontent}
    \begin{spellfooter}
        \spellinfo{Channeling [Good]}{Divine, Good}
        \miscastexplode
    \end{spellfooter}
\end{spellsection}

\begin{spellsection}{Holy Smite}[3]
    \begin{spellheader}
    \end{spellheader}
    \begin{spellcontent}
        \begin{spelltargetinginfo}
            \spellquicktargeting{One nongood creature}{\rngmed}
        \end{spelltargetinginfo}
        \begin{spelleffects}
            \begin{spellattack}{Spellpower vs. Mental}
                \spellsuccess \spelldamage{divine}, and the target is \dazed for 2 rounds.
                \spellcritical Double damage, and the target is \stunned instead of dazed.
                \spellfailure Half damage, and no additional effects.
            \end{spellattack}
        \end{spelleffects}
    \end{spellcontent}
    \begin{spellfooter}
        \spellinfo{Channeling [Good]}{Good}
        \miscastrandom
    \end{spellfooter}
    \begin{spellaugments}
        \spellaugment{3}{Mass}{The spell can affect up to five targets. Its damage becomes \spelldamage{divine}[d8].}
        \spellaugment{4}{Empowered}{The damage increases to \spelldamageemp{divine}. If the Mass augment is applied, the damage instead increases to \spelldamage{divine}[d10].}
    \end{spellaugments}
\end{spellsection}

\begin{spellsection}{Holy Word}[5]
    \begin{spellheader}
    \end{spellheader}
    \begin{spellcontent}
        \begin{spelltargetinginfo}
            \spellburst{\arealarge radius centered on you}
            \spelltgts{All nongood creatures in the area}
            \spellcmp{Verbal only}
        \end{spelltargetinginfo}
        \begin{spelleffects}
            \begin{spellattack}{Spellpower vs. Mental}
                \spellsuccess \spelldamage{divine}[d8], and the target is \dazed for 2 rounds.
                \spellcritical Double damage, and the target is \stunned for 1 round instead of dazed.
                \spellfailure Half damage, and no additional effects.
            \end{spellattack}
        \end{spelleffects}
    \end{spellcontent}
    \begin{spellfooter}
        \spellinfo{Channeling [Good]}{Good, Divine}
        \miscastexplode
    \end{spellfooter}
    \begin{spellaugments}
        \spellaugment{4}{Empowered}{The damage increases to \spelldamage{divine}[d10].}
    \end{spellaugments}
\end{spellsection}

\begin{spellsection}{Horrid Wilting}[9]
    \begin{spellheader}
        \spelldesc{You dessicate your foes from a great distance, shriveling their bodies.}
    \end{spellheader}
    \begin{spellcontent}
        \begin{spelltargetinginfo}
            \spelltwocol{\spellburst{\arealarge burst}}{\spellrng{\rnglong}}
            \spelltgts{All enemies in the area}
        \end{spelltargetinginfo}
        \begin{spelleffects}
            \begin{spellattack}{Spellpower vs. Fortitude}
                \spellspecial You gain a \plus5 bonus to accuracy against plants and creatures with the water subtype.
                \spellsuccess \spelldamage{physical}[d8]
                \spellcritical Double damage.
                \spellfailure Half damage.
            \end{spellattack}
        \end{spelleffects}
    \end{spellcontent}
    \begin{spellfooter}
        \spellinfo{Vivimancy [Flesh]}{Arcane}
        \miscastyou
    \end{spellfooter}
\end{spellsection}

\begin{spellsection}{Hypnotic Pattern}[4]
    \begin{spellheader}
        \spelldesc{You create a twisting pattern of subtle, shifting colors that weaves through the air, fascinating creatures within it.}
    \end{spellheader}
    \begin{spellcontent}
        \begin{spelltargetinginfo}
            \spelltwocol{\spellzone{\arealarge radius}}{\spellrng{\rngmed}}
        \end{spelltargetinginfo}
        \begin{spelleffects}
            \spelleffect Lights appear in the area, illuminating the surroundings like a torch.
            When the lights first appear, you make a Spellpower vs. Mental attack against all creatures in the area.
            Success means a target is \fascinated by the lights.
            At the end of each round, you make the same attack against any creatures that entered the area that round.
            \spelldur \durshort
        \end{spelleffects}
    \end{spellcontent}
    \begin{spellfooter}
        \spellinfo{Illusion [Figment, Light, Mind, Visual]}*{Arcane}
        \miscastyou
    \end{spellfooter}
    \begin{spellaugments}
        \spellaugment{4}{Mobile}{As a swift action, you can concentrate to move the lights up to 50 feet in any direction. Creatures fascinated by the lights will follow them to remain in the area to the best of their ability. If they are unable to remain in the area, they break free of their fascination at the end of the round.}
    \end{spellaugments}
\end{spellsection}

\pdfbookmark[2]{I}{SpellDescriptionsI}

\begin{spellsection}{Ice Spike}[5]
    \begin{spellheader}
        \spelldesc{You create a shard of ice that you hurl at a foe, freezing it and slowing its movement.}
    \end{spellheader}
    \begin{spellcontent}
        \begin{spelltargetinginfo}
            \spellquicktargeting{One creature or object}{\rnglong}
        \end{spelltargetinginfo}
        \begin{spelleffects}
            \begin{spellattack}{Spellpower vs. Fortitude}
                \spellsuccess \spelldamage{cold and piercing}, and the target moves at half speed for 5 rounds.
                \spellcritical As above, but double damage.
                \spellfailure As above, but half damage.
            \end{spellattack}
        \end{spelleffects}
    \end{spellcontent}
    \begin{spellfooter}
        \spellinfo{Conjuration/Evocation [Cold, Creation]}*{Arcane, Nature, Water}
        \miscastexplode
    \end{spellfooter}
    \begin{spellaugments}
        \spellaugment{2}{Freezing}{If the attack succeeds, the target is \immobilized instead of moving at half speed.}
        \spellaugment{4}{Empowered}{The damage increases to \spelldamageemp{cold}.}
    \end{spellaugments}
\end{spellsection}

\begin{spellsection}{Ice Storm}[7]
    \begin{spellheader}
        \spelldesc{You conjure magical hailstones that pound down, smashing and chilling creatures in their path.}
    \end{spellheader}
    \begin{spellcontent}
        \begin{spelltargetinginfo}
            \spelltwocol{\spellburst{\areamed radius cylinder, 20 ft.\ high}}{\spellrng{\rngmed}}
        \end{spelltargetinginfo}
        \begin{spelleffects}
            \spelleffect The ground in the area is covered in ice for 5 rounds, making it slippery. A DR 15 Acrobatics check is usually required to move on icy surfaces. See \pcref{Balance}, for more details.
        \end{spelleffects}
    \end{spellcontent}
    \begin{spellsubcontent}
        \begin{spelltargetinginfo}
            \spelltgts{Everything in the area}
        \end{spelltargetinginfo}
        \begin{spelleffects}
            \spelleffect \spelldamage{cold and bludgeoning}[d6]
        \end{spelleffects}
    \end{spellsubcontent}
    \begin{spellfooter}
        \spellinfo{Conjuration/Evocation [Cold, Creation]}*{Arcane, Destruction, Nature, Water}
        \miscastyou
    \end{spellfooter}
\end{spellsection}

\begin{spellsection}{Immolation}[7]
    \begin{spellheader}
        \spelldesc{You completely consume a foe in fire, destroying it utterly.}
    \end{spellheader}
    \begin{spellcontent}
        \begin{spelltargetinginfo}
            \spellquicktargeting{One creature or object}{\rngext}
        \end{spelltargetinginfo}
        \begin{spelleffects}
            \begin{spellattack}{Spellpower vs. Reflex}
                \spellsuccess \spelldamage{fire}. If the target has no hit points remaining, it dies. Its body is completely immolated, leaving behind only a handful of ashes. Its equipment is unaffected.
                \spellcritical As above, but double damage.
                \spellfailure As above, but half damage.
            \end{spellattack}
        \end{spelleffects}
    \end{spellcontent}
    \begin{spellfooter}
        \spellinfo{Evocation [Fire]}{Arcane, Fire, Nature}
        \miscastexplode
    \end{spellfooter}
    \begin{spellaugments}
        \spellaugment{1}{Complete}{If the target is completely immolated, its equipment also takes damage from the spell. Equipment destroyed in this way is also immolated. This has no effect on artifacts.}
        \spellaugment{3}{Mass}{The spell can affect up to five targets. Its damage becomes \spelldamage{fire}[d8].}
    \end{spellaugments}
\end{spellsection}

\begin{spellsection}{Implosion}[9]
    \begin{spellheader}
        \spelldesc{You create a destructive resonance in your foe's body that destroys it from the inside out.}
    \end{spellheader}
    \begin{spellcontent}
        \begin{spelltargetinginfo}
            \spelltgr{At the end of every round}
            \spellquicktargeting{One creature}{\rnglong}
            \spellspecial You cannot target the same creature more than once per casting of this spell.
            \spelldur Focus (maximum 5 rounds)
        \end{spelltargetinginfo}
        \begin{spelleffects}
            \begin{spellattack}{Spellpower vs. Fortitude}
                \spellsuccess \spelldamage{bludgeoning}. In addition, the target is \staggered for 5 rounds.
                \spellcritical The target dies.
                \spellfailure Half damage, and no additional effects.
            \end{spellattack}
        \end{spelleffects}
    \end{spellcontent}
    \begin{spellfooter}
        \spellinfo{Evocation [Sonic]}{Divine}
        \spellnotes This spell has no effect on creatures in \spell{gaseous form} or on incorporeal creatures.
        \miscastexplode
    \end{spellfooter}
\end{spellsection}

\begin{spellsection}{Imprisonment}[8]
    \begin{spellheader}
        \spelldesc{You teleport your foe deep beneath the earth, leaving it in stasis forever.}
    \end{spellheader}
    \begin{spellcontent}
        \begin{spelltargetinginfo}
            \spellquicktargeting{One creature}{\rngmed}
        \end{spelltargetinginfo}
        \begin{spelleffects}
            \begin{spellattack}{Spellpower vs. Mental}
                \spellsuccess \spelldamage{physical}. In addition, the target is \slowed for 5 rounds.
                \spellfailure As above, but half damage.
                \spellcritical The target becomes permanently entombed in a state of suspended animation (as the \spell{temporal stasis} spell) in a small sphere far beneath the surface of the earth. It remains there until an \spell{emancipation} spell is cast at the location where the imprisonment took place.
            \end{spellattack}
        \end{spelleffects}
    \end{spellcontent}
    \begin{spellfooter}
        \spellinfo{Conjuration/Transmutation [Teleportation, Temporal]}{Arcane, Earth, Law, Travel}
        \spellnotes If the target becomes imprisoned beneath the earth, it is very difficult to find. Magical search by a crystal ball, a \spell{locate creature} spell, or some other similar divination does not reveal the fact that a creature is imprisoned, but \spell{discern location} does. A \spell{wish} or \spell{miracle} spell will not free the recipient, but will reveal where it is entombed.

        On planes that have no earth to imprison the target, a critical success has the same effect as an ordinary success.
        \miscastrandom
    \end{spellfooter}
\end{spellsection}

\begin{spellsection}{Inertial Shield}[2]
    \begin{spellheader}
        \spelldesc{You create a barrier around your ally that resists physical intrusion.}
    \end{spellheader}
    \begin{spellcontent}
        \begin{spelltargetinginfo}
            \spellquicktargeting{One creature}{\rngtouch}
        \end{spelltargetinginfo}
        \begin{spelleffects}
            \spelleffect The target gains damage reduction against all damage equal to your spellpower.
            Arcane damage ignores this damage reduction and negates it for 1 round.
            \spelldur \durshort
        \end{spelleffects}
    \end{spellcontent}
    \begin{spellfooter}
        \spellinfo{Abjuration [Shielding]}{Arcane}
        \miscastexplode
    \end{spellfooter}
    \begin{spellaugments}
        \spellaugment{3}{Retributive}{Damage resisted by this spell is reflected back to the attacker as life damage. If the attacker is beyond \rngmed range of the target, this reflection fails.}[Vivimancy [Life]]
    \end{spellaugments}
\end{spellsection}

\begin{spellsection}{Inferno}[6]
    \begin{spellheader}
        \spelldesc{You are surrounded in a flaming inferno that damages your foes.}
    \end{spellheader}
    \begin{spellcontent}
        \begin{spelltargetinginfo}
            \spellarea{\arealarge radius emanation from you}
            \spelltgts{All enemies in the area}
        \end{spelltargetinginfo}
        \begin{spelleffects}
            \spelleffect At the end of each round, the target takes fire damage equal to your spellpower.
            \spelldur 5 rounds
        \end{spelleffects}
    \end{spellcontent}
    \begin{spellfooter}
        \spellinfo{Evocation [Fire]}{Arcane, Fire, Nature}
        \miscastexplode
    \end{spellfooter}
    \begin{spellaugments}
        \spellaugment{3}{Incendiary}{Targets that take damage from the spell are also \ignited for 2 rounds.}
    \end{spellaugments}
\end{spellsection}

\begin{spellsection}{Inflict Wounds}[1]
    \begin{spellheader}
    \end{spellheader}
    \begin{spellcontent}
        \begin{spelltargetinginfo}
            \spellquicktargeting{One creature}{\rngmed}
        \end{spelltargetinginfo}
        \begin{spelleffects}
            \begin{spellattack}
                \spellsuccess \spelldamage{life}.
                \spellcritical Double damage.
                \spellfailure Half damage.
            \end{spellattack}
        \end{spelleffects}
    \end{spellcontent}
    \begin{spellfooter}
        \spellinfo{Vivimancy [Life]}{Arcane, Death, Divine}
        \miscastrandom
    \end{spellfooter}
    \begin{spellaugments}
        \spellaugment{2}{Critical Wounds}{All damage dealt in excess of the target's hit points is dealt as critical damage.}
        \spellaugment{3}{Mass}{The spell can affect up to five targets. Its damage and healing become \spelldamage{}[d8].}
        \spellaugment{4}{Empowered}{The damage increases to \spelldamageemp{life}.}
    \end{spellaugments}
\end{spellsection}

\begin{spellsection}{Invisibility}[3]
    \begin{spellheader}
    \end{spellheader}
    \begin{spellcontent}
        \begin{spelltargetinginfo}
            \spellquicktargeting{One creature or object (Large or smaller)}{\rngclose}
        \end{spelltargetinginfo}
        \begin{spelleffects}
            \spelleffect The target and its equipment become invisible. An invisible creature cannot be seen, even by darkvision. Invisible creatures can be detected with the Awareness skill (see \pcref{Awareness}).

            If the target attacks any creature, such as by using any ability that affects an unwilling creature, it becomes visible.
            \spelldur \durshort \dismissable
        \end{spelleffects}
    \end{spellcontent}
    \begin{spellfooter}
        \spellinfo{Illusion [Glamer]}{Arcane, Trickery}
        \miscastrandom
    \end{spellfooter}
    \begin{spellaugments}
        \spellaugment{1}{Giant}{The spell can affect a target one size category larger. This augment can be used multiple times.}
        \spellaugment{1}{Sensory}{The target becomes undetectable by an additional sense: sound, smell, texture, or temperature. This augment can be used multiple times, affecting a different sense each time.}
        \spellaugment{3}{Mass}{The spell can affect up to five targets.}
        \spellaugment{3}{Recharging}{At the end of every round, if the target did not attack a creature that round, it becomes invisible again.}
    \end{spellaugments}
\end{spellsection}

\begin{spellsection}{Invulnerability}[4]
    \begin{spellheader}
        \spelldesc{You become nearly invulnerable to damage.}
    \end{spellheader}
    \begin{spellcontent}
        \begin{spelltargetinginfo}
            \spelltgt{You}
        \end{spelltargetinginfo}
        \begin{spelleffects}
            \spellspecial When you cast this spell, choose a type of \glossterm{energy} (cold, electricity, fire, or sonic).
            \spelleffect You gain damage reduction against all damage equal to your spellpower.
            Damage of the chosen energy type ignores this damage reduction and negates it for 1 round.
            \spelldur \durlong
        \end{spelleffects}
    \end{spellcontent}
    \begin{spellfooter}
        \spellinfo{Abjuration [Shielding]}{Arcane}
        \miscastexplode
    \end{spellfooter}
    \begin{spellaugments}
        \spellaugment{2}{Complete}{You do not need to choose an energy type when casting the spell, and the damage reduction cannot be overcome.}
        % \spellaugment{4}{Empowered}{The damage reduction is equal to twice your spellpower.}
        \spellaugment{5}{Spellshield}{You also gain \glossterm{magic resistance} equal to 10 \add your spellpower.}
    \end{spellaugments}
\end{spellsection}

\begin{spellsection}{Irresistible Dance}[9]
    \begin{spellheader}
        \spelldesc{You fill your enemy with an overpowering urge to dance and caper in place. Against its will, it begins doing so, complete with foot shuffling and tapping.}
    \end{spellheader}
    \begin{spellcontent}
        \begin{spelltargetinginfo}
            \spellquicktargeting{One creature}{\rngmed}
        \end{spelltargetinginfo}
        \begin{spelleffects}
            \begin{spellattack}{Spellpower vs. Mental}
                \spellsuccess The target must spend a standard action each round to do nothing but dance.
                \spellcritical As above, except that the effect lasts for one year.
                \spellfailure The target must spend a move action each round to dance.
                In addition, the struggle to resist dancing makes it \impaired with attacks and checks.
            \end{spellattack}
            \spelldur \durbrief or one year
        \end{spelleffects}
    \end{spellcontent}
    \begin{spellfooter}
        \spellinfo{Enchantment [Compulsion, Mind]}{Arcane}
        \miscastrandom
    \end{spellfooter}
\end{spellsection}

\pdfbookmark[2]{J-L}{SpellDescriptionsJ-L}

\begin{spellsection}{Knock}[2]
    \begin{spellheader}
    \end{spellheader}
    \begin{spellcontent}
        \begin{spelltargetinginfo}
            \spellquicktargeting{One object (Medium or smaller)}{\rngclose}
        \end{spelltargetinginfo}
        \begin{spelleffects}
            \spelleffect This spell telekinetically opens stuck, barred, locked, held, or arcane locked objects. If the target object is stuck or held, you can immediately make an Strength check to break it open, using your spellpower instead of your Strength. Others can aid you on this check as normal.

            In addition, if the target object is locked, you can immediately make a Disable Device check to open the lock as if you had rolled a 20 on the check. You get a bonus on the Disable Device check equal to half your spellpower.
        \end{spelleffects}
    \end{spellcontent}
    \begin{spellfooter}
        \spellinfo{Evocation [Telekinesis]}{Arcane}
        \spellnotes If this spell is cast on an \spellindirect{arcane lock}{arcane locked} door, make a spellpower check against a DR of 10 \add the spellpower of the \spell{arcane lock}. If you succeed, the \spell{arcane lock} is suppressed for 10 minutes. If you fail, you may still bypass the door with the checks above, if possible.
        \miscastrandom
    \end{spellfooter}
    \begin{spellaugments}
        \spellaugment{1}{Giant}{The spell can affect a target one size category larger. This augment can be used multiple times.}
        \spellaugment{2}{Silent}{Opening the target object makes no noise.}[Illusion [Glamer]]
    \end{spellaugments}
\end{spellsection}

\begin{spellsection}{Levitate}[3]
    \begin{spellheader}
    \end{spellheader}
    \begin{spellcontent}
        \begin{spelltargetinginfo}
            \spellrng{\rngclose}
            \spelltgt{One unattended object or willing creature (Large or smaller)}
        \end{spelltargetinginfo}
        \begin{spelleffects}
            \spelleffect As a swift action, you can mentally direct the target to move up or down as much as 30 feet each round. You cannot move the recipient horizontally, but the recipient could clamber along the face of a cliff, for example, or push against a ceiling to move laterally (generally at half its land speed).
            \spelldur \durshort \dismissable
        \end{spelleffects}
    \end{spellcontent}
    \begin{spellfooter}
        \spellinfo{Evocation [Telekinesis]}{Arcane}
        \miscastrandom
    \end{spellfooter}
    \spellaugment{2}{Flexible}{You can move the target in any direction, rather than just vertically.}
    \spellaugment{3}{Mass}{The spell can affect up to five targets.}
\end{spellsection}

\begin{spellsection}{Lifebound}[8]
    \begin{spellheader}
        \spelldesc{You bind the life force of one ally to another, preventing one from dying while the other lives.}
    \end{spellheader}
    \begin{spellcontent}
        \begin{spelltargetinginfo}
            \spellquicktargeting*{Two willing creatures}{\rngmed}
        \end{spelltargetinginfo}
        \begin{spelleffects}
            \spellspecial When you cast this spell, you choose which target will be protected.
            \spelleffect The protected creature cannot take critical damage. Any damage it takes while it has no hit points remaining, or critical damage it would take for other reasons, is simply ignored. It is still \disabled when it has no hit points remaining.

            If the targets become farther than 100 feet from each other, or if the unprotected target takes critical damage, the spell immediately ends.
            \spelldur \durshort
        \end{spelleffects}
    \end{spellcontent}
    \begin{spellfooter}
        \spellinfo{Vivimancy [Life, Shielding]}{Arcane, Life}
        \spellnotes A creature affected by this spell cannot be affected by other \spell{lifebound} spells.
        \miscastexplode
    \end{spellfooter}
\end{spellsection}

\begin{spellsection}{Lifegiving Roots}[6]
    \begin{spellheader}
        \spelldesc{You raise roots from the ground that give their energy to a creature.}
    \end{spellheader}
    \begin{spellcontent}
        \begin{spelltargetinginfo}
            \spellquicktargeting{One willing creature}{\rngmed}
        \end{spelltargetinginfo}
        \begin{spelleffects}
            \spellsuccess The target is \immobilized. In addition, it cannot be moved by any forced movement effects (such as shoving). At the end of every round, it heals hit points equal to twice your spellpower.
            \spelldur 5 rounds
        \end{spelleffects}
    \end{spellcontent}
    \begin{spellfooter}
        \spellinfo{Transmutation [Imbuement]}{Nature, Wild}
        \miscastexplode
    \end{spellfooter}
\end{spellsection}

\begin{spellsection}{Lightning Bolt}[3]
    \begin{spellheader}
    \end{spellheader}
    \begin{spellcontent}
        \begin{spelltargetinginfo}
            \spellburst{\arealarge line, 10 ft.\ wide}
            \spelltgts{Everything in the area}
        \end{spelltargetinginfo}
        \begin{spelleffects}
            \begin{spellattack}{Spellpower vs. Reflex}
                \spellsuccess \spelldamage{electricity}[d8]
                \spellcritical Double damage.
                \spellfailure Half damage.
            \end{spellattack}
        \end{spelleffects}
    \end{spellcontent}
    \begin{spellfooter}
        \spellinfo{Evocation [Electricity]}{Arcane, Nature}
        \miscastexplode
    \end{spellfooter}
    \begin{spellaugments}
        \spellaugment{3}{Staggering}{A successful attack also makes a target \staggered for 2 rounds.}
        \spellaugment{4}{Empowered}{The damage increases to \spelldamage{electricity}[d10].}
    \end{spellaugments}
\end{spellsection}

\begin{spellsection}{Living Missile}[4]
    \begin{spellheader}
        \spelldesc{You telekinetically throw an ally at a distant foe with great force.}
    \end{spellheader}
    \begin{spellcontent}
        \begin{spelltargetinginfo}
            \spellquicktargeting{One creature, object, or location}{\rngmed}
            \spellquicktargeting{One willing ally}{\rngtouch}
        \end{spelltargetinginfo}
        \begin{spelleffects}
            \spelleffect You throw a willing ally at the target. The ally gains damage reduction against physical damage equal to twice your spellpower for 2 rounds.
            \begin{spellattack}{Spellpower vs. Reflex}
                \spellsuccess The target takes \spelldamage{bludgeoning}. The ally takes half this damage, reduced by the damage reduction as appropriate.
                \spellcritical As above, but the target takes doubles damage. This does not increase the damage taken by the ally.
                \spellfailure As above, but the target takes half damage. This does not reduce the damage taken by the ally.
            \end{spellattack}
        \end{spelleffects}
    \end{spellcontent}
    \begin{spellfooter}
        \spellinfo{Abjuration/Evocation}{Arcane}
        \miscastexplode
    \end{spellfooter}
    \begin{spellaugments}
        \spellaugment{3}{Mass}{The spell can affect up to five willing allies. All allies must be thrown at the same target. The target only takes damage once, not from each ally.}
        \spellaugment{4}{Empowered}{The damage increases to \spelldamage{bludgeoning}[d10]. This also increases the damage taken by the ally.}
    \end{spellaugments}
\end{spellsection}

\begin{spellsection}{Longeye}[2]
    \begin{spellheader}
        \spelldesc{You grant your ally the ability to see distant foes clearly, allowing her to strike them accurately.}
    \end{spellheader}
    \begin{spellcontent}
        \begin{spelltargetinginfo}
            \spellquicktargeting{One willing creature}{\rngclose}
            \spelltime{Swift action}
        \end{spelltargetinginfo}
        \begin{spelleffects}
            \spelleffect The target reduces its \glossterm{range increment} penalties by an amount equal to your spellpower until the end of the round.
        \end{spelleffects}
    \end{spellcontent}
    \begin{spellfooter}
        \spellinfo{Transmutation [Enhancement]}{Arcane, Nature}
        \miscastexplode
    \end{spellfooter}
    \begin{spellaugments}
        \spellaugment{3}{Mass}{The spell can affect up to five targets.}
    \end{spellaugments}
\end{spellsection}

\begin{spellsection}{Longstrider}[1]
    \begin{spellheader}
    \end{spellheader}
    \begin{spellcontent}
        \begin{spelltargetinginfo}
            \spelltgt{You}
        \end{spelltargetinginfo}
        \begin{spelleffects}
            \spelleffect You gain a \plus10 foot bonus to your speed in all your movement modes.
            \spelldur \durlong \dismissable
        \end{spelleffects}
    \end{spellcontent}
    \begin{spellfooter}
        \spellinfo{Transmutation [Enhancement]}{Nature, Travel}
        \miscastexplode
    \end{spellfooter}
    \begin{spellaugments}
        \spellaugment{2}{Empowered}{The speed bonus increases to 30 feet.}
    \end{spellaugments}
\end{spellsection}

\pdfbookmark[2]{M}{SpellDescriptionsM}

\begin{spellsection}{Mage Armor}[1]
    \begin{spellheader}
        \spelldesc{You create an invisible but tangible field of force that shields you from attacks.}
    \end{spellheader}
    \begin{spellcontent}
        \begin{spelltargetinginfo}
            \spelltgt{You}
            \spellspecial When you cast this spell, you choose whether to create body armor or a shield.
        \end{spelltargetinginfo}
        \begin{spelleffects}
            \spelleffect You gain invisible body armor or a shield of the chosen kind, formed from telekinetic force. Body armor grants a \plus4 defense bonus, while a shield grants a \plus2 defense bonus.
            \par Unlike mundane armor, this armor has no \glossterm{encumbrance penalty}, arcane spell failure chance, or encumbrance. If you create a shield, it floats in front of you, and does not need to be wielded actively to grant its bonus.
            \spelldur \durlong
        \end{spelleffects}
    \end{spellcontent}
    \begin{spellfooter}
        \spellinfo{Evocation [Telekinesis]}{Arcane}
        \spellnotes If you cast this spell twice, you can gain both body armor and a shield. The armor created by this spell is treated as a separate piece or armor from any other armor the creature is wearing, so it does not stack with any existing bonuses.
        \miscastexplode
    \end{spellfooter}
    \begin{spellaugments}
        \spellaugment{1}{Dual}{You create both armor and a shield.}
    \end{spellaugments}
\end{spellsection}

\begin{spellsection}{Mage Hand}[1]
    \begin{spellheader}
    \end{spellheader}
    \begin{spellcontent}
        \begin{spelltargetinginfo}
            \spellrng{\rngclose}
        \end{spelltargetinginfo}
        \begin{spelleffects}
            \spelleffect By concentrating as a swift action, you can move an object within range up to 10 feet per round.

            Your effective Strength is \minus4, allowing you to hold and move objects up to 25 pounds. You cannot perform tasks requiring fine motor skills (with a DR higher than 0).
            \spelldur \durshort
        \end{spelleffects}
    \end{spellcontent}
    \begin{spellfooter}
        \spellinfo{Evocation [Telekinesis]}{Arcane}
        \miscastexplode
    \end{spellfooter}
\end{spellsection}

\begin{spellsection}{Magic Missile}[1]
    \begin{spellheader}
    \end{spellheader}
    \begin{spellcontent}
        \begin{spelltargetinginfo}
            \spellquicktargeting*{See text}{\rngmed}
        \end{spelltargetinginfo}
        \begin{spelleffects}
            \spelleffect You create a number of missiles equal to half your spellpower. Each missile strikes one target creature for 1d10 arcane damage. You can direct each missile to strike the same or different targets.
        \end{spelleffects}
    \end{spellcontent}
    \begin{spellfooter}
        \spellinfo{Evocation}{Arcane, Magic}
        \miscastrandom
    \end{spellfooter}
    \begin{spellaugments}
        \spellaugment{2}{Lifeseeking}{Any missiles you do not explicitly target will automatically strike a living creature within range. The missiles are able to unerringly strike creatures you cannot see or are not aware of, including invisible or concealed creatures. You can direct the missiles to avoid specific targets, allowing you to strike a hidden foe among your allies.}
        \spellaugment{4}{Empowered}{Each missile deals 2d6 arcane damage.}
    \end{spellaugments}
\end{spellsection}

\begin{spellsection}{Mark of Scrying}[4]
    \begin{spellheader}
        \spelldesc{You create a mark that allows you to scry on your target.}
    \end{spellheader}
    \begin{spellcontent}
        \begin{spelltargetinginfo}
            \spellquicktargeting{One creature or object}{\rngmed}
            \spellcmp{Somatic only}
        \end{spelltargetinginfo}
        \begin{spelleffects}
            \spelleffect The target gains an invisible mark on its forehead (or similarly prominent feature).
            As long as the mark remains, you can focus on the mark (a standard action) to see and hear as if you were where the target is.
            However, this perception is limited, and you can only see and hear within a 20 foot radius of the target.
            Abilities which improve your senses, such as darkvision, do not apply when scrying through the mark.

            While you are scrying through the mark, it becomes visible.
            A DR 10 Awareness check is sufficient to notice the mark once it is visible, though the target usually cannot see its own mark due to the mark's location.
            \spelldur \durlong
        \end{spelleffects}
    \end{spellcontent}
    \begin{spellfooter}
        \spellinfo{Divination [Scrying]}{Arcane, Divine, Law, Nature}
        \spellnotes The mark's shape is the Draconic word for ``sight''.
        Although it is invisible, the mark can be detected with \spell{see invisibility} or a DR 30 Awareness check, if the mark is not covered by armor or other clothing.

        The mark can be removed by scrubbing it away, which usually takes a minute of work, or by dispelling it with \spell{dispel magic} or similar effects.
        \miscastrandom
    \end{spellfooter}
    \begin{spellaugments}
        \spellaugment{2}{Hidden}{When you cast the spell, you can freely choose where the mark appears on the target's body.}
        \spellaugment{2}{Widened}{While scrying, you can see and hear within a 100 foot radius of the target.}
        \spellaugment{3}{Persistent}{The mark lasts for thirty days.}
    \end{spellaugments}
\end{spellsection}

\begin{spellsection}{Mark of Tracking}[2]
    \begin{spellheader}
        \spelldesc{You create an invisible mark which allows you to follow your target anywhere.}
    \end{spellheader}
    \begin{spellcontent}
        \begin{spelltargetinginfo}
            \spellquicktargeting{One creature or object}{\rngmed}
            \spellcmp{Somatic only}
        \end{spelltargetinginfo}
        \begin{spelleffects}
            \spelleffect The target gains an invisible mark on its forehead (or similarly prominent feature).
            As long as the mark remains, you know the approximate direction and distance to the target.
            If the target is farther than 10 miles away from you, or is on another plane, you do not gain the benefits of this spell.
            \spelldur \durlong
        \end{spelleffects}
    \end{spellcontent}
    \begin{spellfooter}
        \spellinfo{Divination [Knowledge]}{Arcane, Divine, Nature}
        \spellnotes The mark's shape is the Draconic word for ``tracking''.
        It appears on the target's forehead or other similarly prominent body feature.
        Although it is invisible, the mark can be detected with \spell{see invisibility} or a DR 30 Awareness check, if the mark is not covered by armor or other clothing.

        The mark can be removed by scrubbing it away, which usually takes a minute of work, or by dispelling it with \spell{dispel magic} or similar effects.
        \miscastrandom
    \end{spellfooter}
    \begin{spellaugments}
        \spellaugment{2}{Hidden}{When you cast the spell, you can freely choose where the mark appears on the target's body.}
        \spellaugment{3}{Persistent}{The mark lasts for thirty days.}
    \end{spellaugments}
\end{spellsection}

\begin{spellsection}{Martial Transformation}[5]
    \begin{spellheader}
        \spelldesc{You grant an ally incredible combat prowess.}
    \end{spellheader}
    \begin{spellcontent}
        \begin{spelltargetinginfo}
            \spelltwocol{One willing creature; see text}{\rngclose}
        \end{spelltargetinginfo}
        \begin{spelleffects}
            \spelleffect The target is proficient with all weapons, including improvised weapons.
            It may use your spellpower \add 4 in place of its normal accuracy with all physical attacks.
            This replaces all other bonuses and penalties to accuracy.

            If the target's level is lower than half your spellpower, it takes damage equal to your spellpower at the end of each round from the power of the transformation.
            \spelldur \durshort
        \end{spelleffects}
    \end{spellcontent}
    \begin{spellfooter}
        \spellinfo{Transmutation [Augment]}{Arcane, War}
        \spellnotes This spell does not affect the damage the target deals with physical attacks.
        \miscastexplode
    \end{spellfooter}
    \begin{spellaugments}
        \spellaugment{3}{Damaging Transformation}{The target may also use your spellpower in place of all other bonuses to physical damage. It still rolls damage with its weapon normally.}
    \end{spellaugments}
\end{spellsection}

\begin{spellsection}{Martyr's Gift}[8]
    \begin{spellheader}
        \spelldesc{You selflessly shield your allies from harm by sacrificing your own health.}
    \end{spellheader}
    \begin{spellcontent}
        \begin{spelltargetinginfo}
            \spellemanation{\arealarge emanation from you}
            \spelltgt{You}
        \end{spelltargetinginfo}
        \begin{spelleffects}
            \spelleffect Whenever an ally in the area takes damage, you may choose to take that damage instead. This damage ignores all forms of damage reduction and similar abilities. If you take damage in excess of your hit points in this way, the excess damage is dealt directly as critical damage.
        \end{spelleffects}
    \end{spellcontent}
    \begin{spellfooter}
        \spellinfo{Vivimancy [Life]}{Divine, Good, Protection}
        \miscastexplode
    \end{spellfooter}
\end{spellsection}

\begin{spellsection}{Mask of the Deceiver}[1]
    \begin{spellheader}
        \spelldesc{You create a trustworthy facade for your ally, making their words feel genuine and true.}
    \end{spellheader}
    \begin{spellcontent}
        \begin{spelltargetinginfo}
            \spellquicktargeting{One willing creature}{\rngclose}
            \spellcmp{Somatic only}
        \end{spelltargetinginfo}
        \begin{spelleffects}
            \spelleffect Whenever the target makes Bluff and Persuasion checks, it rolls twice and takes the higher result.
            \spelldur \durshort
        \end{spelleffects}
    \end{spellcontent}
    \begin{spellfooter}
        \spellinfo{Illusion}{Arcane, Trickery}
        \miscastexplode
    \end{spellfooter}
    \begin{spellaugments}
        \spellaugment{3}{Empowered}{The target also gains a \plus5 bonus to Bluff and Persuasion checks.}
        \spellaugment{3}{Persistent}{The spell's duration becomes \durlong. If you use this augment again during that time, the previous effect immediately ends.}
    \end{spellaugments}
\end{spellsection}

\begin{spellsection}{Maze}[9]
    \begin{spellheader}
    \end{spellheader}
    \begin{spellcontent}
        \begin{spelltargetinginfo}
            \spellquicktargeting{One creature}{\rngmed}
        \end{spelltargetinginfo}
        \begin{spelleffects}
            \begin{spellattack}{Spellpower vs. Mental}
                \spellsuccess The target is teleported into an extradimensional labyrinth of stone walls. Each round, as a full-round action, it may attempt a DR 20 Intelligence check to escape the labyrinth. If the target doesn't escape, the maze disappears after 5 minutes, forcing the target back to the location where it was originally banished.
                \spellfailure As above, but the DR of the Intelligence check to escape is 10.
            \end{spellattack}
        \end{spelleffects}
    \end{spellcontent}
    \begin{spellfooter}
        \spellinfo{Conjuration [Planar, Teleportation]}*{Arcane, Trickery}
        \spellnotes Spells and abilities that move a creature within a plane, such as \spell{teleport} and \spell{dimension door}, do not help a creature escape a \spell{maze} spell, although a \spell{plane shift} spell allows it to exit to whatever plane is designated in that spell. Minotaurs can escape the spell automatically.

        When leaving the maze, the target reappears where it had been when the maze spell was cast. If this location is filled with a solid object, the target appears in the nearest open space.

        In extraordinarily rare circumstances, it may be possible to meet other creatures trapped in the same maze.

        \norepeatspellnotes
        \miscastrandom
    \end{spellfooter}
\end{spellsection}

\begin{spellsection}{Meld into Stone}[2]
    \begin{spellheader}
    \end{spellheader}
    \begin{spellcontent}
        \begin{spelltargetinginfo}
            \spelltgt{One solid stone object of your size or larger}
        \end{spelltargetinginfo}
        \begin{spelleffects}
            \spelleffect You and your equipment meld into the target block of stone. While in the stone, you can move, breathe, and speak as if the stone was air, but you cannot see or hear out of the stone unless you move your head out of the stone. In addition, you are unable to move farther than 5 feet from your original entrance point.

            Minor physical damage to the stone does not harm you, but if its size is reduced to be smaller than yours, or if it is otherwise altered to be unsuitable for the spell (such as by \spell{transmute flesh and stone}), you are expelled and take 5d6 points of damage.

            If you leave the stone completely, the spell immediately ends.

            \spelldur \durlong
        \end{spelleffects}
    \end{spellcontent}
    \begin{spellfooter}
        \spellinfo{Transmutation [Earth, Shaping]}{Nature}
        \miscastexplode
    \end{spellfooter}
    \begin{spellaugments}
        \spellaugment{1}{Meld into Plants}{You can also meld into plants.}
        \spellaugment{2}{Compact}{You can meld into an object one size category smaller. This augment can be used multiple times.}
    \end{spellaugments}
\end{spellsection}

\begin{spellsection}{Message}[1]
    \begin{spellheader}
    \end{spellheader}
    \begin{spellcontent}
        \begin{spelltargetinginfo}
            \spellquicktargeting*{Up to five creatures}{\rngmed}
            \spellcmp{Somatic only}
        \end{spelltargetinginfo}
        \begin{spelleffects}
            \spelleffect Whenever you whisper, you may cause any or all of the targets to hear the message as if you were whispering in their ears.
            \spelldur \durshort
        \end{spelleffects}
    \end{spellcontent}
    \begin{spellfooter}
        \spellinfo{Divination}{Arcane}
        \spellnotes This is not telepathic communication, and observers can still read your lips. Very close observers may also hear the message.
        \miscastexplode
    \end{spellfooter}
    \begin{spellaugments}
        \spellaugment{2}{Mass}{The whispers of all targets can be shared in this way, rather than only yours.}
        \spellaugment{3}{Persistent}{The spell's duration becomes \durlong.}
    \end{spellaugments}
\end{spellsection}

\begin{spellsection}{Meteor Swarm}[9]
    \begin{spellheader}
        \spelldesc{You call a swarm of meteors that streak down from the heavens, leaving a fiery trail behind them. The meteors crash into your foes, driving flying creatures to the ground and knocking creatures off their feet.}
    \end{spellheader}
    \begin{spellcontent}
        \begin{spelltargetinginfo}
            \spellrng{\rngmed}
            \spellburst{\arealarge radius cylinder, 100 ft.\ high}
            \spelltgts{Everything in the area}
        \end{spelltargetinginfo}
        \begin{spelleffects}
            \begin{spellattack}{Spellpower vs. Reflex}
                \spellsuccess \spelldamage{fire}[d8].
                If the target is on the ground, it falls prone. If the target is in the air, and is Gargantuan or smaller, it is driven to the ground. It takes falling damage as appropriate for the distance descended.
                \spellcritical As above, but double damage.
                \spellfailure Half damage, and no additional effects.
            \end{spellattack}
        \end{spelleffects}
    \end{spellcontent}
    \begin{spellfooter}
        \spellinfo{Evocation [Fire]}{Arcane, Fire}
        \spellnotes \firespellnotes
        \miscastyou
    \end{spellfooter}
\end{spellsection}

\begin{spellsection}{Mighty Throw}[3]
    \begin{spellheader}
        \spelldesc{You momentarily become immensely strong, allowing you to throw your foe a great distance from you.}
    \end{spellheader}
    \begin{spellcontent}
        \begin{spelltargetinginfo}
            \spellquicktargeting{One creature or object}{\rngtouch}
        \end{spelltargetinginfo}
        \begin{spelleffects}
            \begin{spellattack}{Strength or Spellpower vs. Fortitude (shove)}
                \spellsuccess The target is thrown up to 20 feet per spellpower in any direction. You can throw it vertically, but it flies half as far vertically as it does horizontally. When the creature strikes a solid object, it suffers 1d6 bludgeoning damage per 20 feet of movement remaining.
                \par This is a \glossterm{Physical} effect, and does not allow \glossterm{magic resistance}.
                \spellcritical As above, but the target is thrown 40 feet per spellpower.
                \spellfailure The target takes \spelldamage{bludgeoning}[d6] as it resists the force of the throw.
            \end{spellattack}
        \end{spelleffects}
    \end{spellcontent}
    \begin{spellfooter}
        \spellinfo{Transmutation [Enhancement]}{Divine, Nature, Strength, Wild}
        \spellnotes The somatic components of this spell consist of throwing the target.
        \miscastexplode
    \end{spellfooter}
    \begin{spellaugments}
        \spellaugment{4}{Empowered}{The damage from striking a solid object increases to 1d8 bludgeoning damage per 20 feet of movement remaining.}
    \end{spellaugments}
\end{spellsection}

\begin{spellsection}{Mind Blank}[4]
    \begin{spellheader}
        \spelldesc{The target is protected from all effects that influence emotions or thoughts.}
    \end{spellheader}
    \begin{spellcontent}
        \begin{spelltargetinginfo}
            \spellquicktargeting{One willing creature}{\rngclose}
        \end{spelltargetinginfo}
        \begin{spelleffects}
            \spelleffect The target is immune to all Mind effects, even beneficial ones.
            \spelldur \durshort
        \end{spelleffects}
    \end{spellcontent}
    \begin{spellfooter}
        \spellinfo{Enchantment [Shielding]}{Arcane}
        \miscastrandom
    \end{spellfooter}
    \begin{spellaugments}
        \spellaugment{3}{Mass}{The spell can affect up to five targets.}
    \end{spellaugments}
\end{spellsection}

\begin{spellsection}{Mirror Image}[3]
    \begin{spellheader}
        \spelldesc{You create illusory duplicates of yourself that mirror your every move, making it difficult for enemies to know which image to attack.}
    \end{spellheader}
    \begin{spellcontent}
        \begin{spelltargetinginfo}
            \spelltgt{You}
        \end{spelltargetinginfo}
        \begin{spelleffects}
            \spelleffect You gain one image per two spellpower. As long as you have images remaining, targeted attacks against you have a 50\% miss chance. Whenever an attack misses in this way, it strikes an image, destroying it. If you run out of images, the spell is expended.

            Each image is considered a separate creature for the purpose of attacks and effects which can target multiple creatures. All images are considered to exist within your space for the purposes of targeting, though they visually drift into nearby spaces.
            \spelldur \durshort or until expended \dismissable
        \end{spelleffects}
    \end{spellcontent}
    \begin{spellfooter}
        \spellinfo{Illusion [Figment, Visual]}{Arcane}
        \spellnotes This spell offers no defense against creatures unable to see you or your images.
        \miscastexplode
    \end{spellfooter}
    \begin{spellaugments}
        \spellaugment{3}{Endless}{A new image is created whenever an image is destroyed, preventing you from running out of images. Attacks that target you and all existing images simultaneously will still hit you without a miss chance.}
    \end{spellaugments}
\end{spellsection}

\begin{spellsection}{Missile Storm}[4]
    \begin{spellheader}
        \spelldesc{You unleash a swarm of missiles which seek out and destroy your foes.}
    \end{spellheader}
    \begin{spellcontent}
        \begin{spelltargetinginfo}
            \spellquicktargeting*{Up to five creatures}{\rngmed}
        \end{spelltargetinginfo}
        \begin{spelleffects}
            \spelleffect \spelldamage{arcane}[d6]
        \end{spelleffects}
    \end{spellcontent}
    \begin{spellfooter}
        \spellinfo{Evocation}{Arcane, Magic}
        \miscastexplode
    \end{spellfooter}
    \begin{spellaugments}
        \spellaugment{3}{Myriad}{The spell targets any number of creatures within range.}
        \spellaugment{4}{Empowered}{The damage increases to \spelldamage{arcane}[d8].}
    \end{spellaugments}
\end{spellsection}

\begin{spellsection}{Moment of Prescience}[3]
    \begin{spellheader}
        \spelldesc{You extend your mind a fraction of a second into the future, allowing you to succeed where you would have failed.}
    \end{spellheader}
    \begin{spellcontent}
        \begin{spelltargetinginfo}
            \spelltgt{You}
            \spelltime{Immediate action}
            \spellspecial You can cast this spell any time you could use a legend point, even while casting another spell.
        \end{spelltargetinginfo}
        \begin{spelleffects}
            \spelleffect You gain a legend point.
            \spelldur Until the end of the round
        \end{spelleffects}
    \end{spellcontent}
    \begin{spellfooter}
        \spellinfo{Divination}{Divination}
        \spellnotes After using this spell, you cannot cast it again for 1 hour.
        \miscastexplode
    \end{spellfooter}
\end{spellsection}

\pdfbookmark[2]{O-P}{SpellDescriptionsO-P}

\begin{spellsection}{Order's Wrath}[3]
    \begin{spellheader}
    \end{spellheader}
    \begin{spellcontent}
        \begin{spelltargetinginfo}
            \spellquicktargeting{One nonlawful creature}{\rngmed}
        \end{spelltargetinginfo}
        \begin{spelleffects}
            \begin{spellattack}{Spellpower vs. Mental}
                \spellsuccess \spelldamage{divine}, and the target is \slowed for 2 rounds.
                \spellcritical Double damage, and the target is \stunned for 1 round instead of slowed.
                \spellfailure Half damage, and no additional effects.
            \end{spellattack}
        \end{spelleffects}
    \end{spellcontent}
    \begin{spellfooter}
        \spellinfo{Channeling [Lawful]}{Law}
        \miscastrandom
    \end{spellfooter}
    \begin{spellaugments}
        \spellaugment{3}{Mass}{The spell can affect up to five targets. Its damage becomes \spelldamage{divine}[d8]}
        \spellaugment{4}{Empowered}{The damage increases to \spelldamageemp{divine}. If the Mass augment is applied, the damage instead increases to \spelldamage{divine}[d10].}
    \end{spellaugments}
\end{spellsection}

\begin{spellsection}{Phantasmal Killer}[4]
    \begin{spellheader}
        \spelldesc{You make your foe believe it sees the most fearsome creature it can imagine.}
    \end{spellheader}
    \begin{spellcontent}
        \begin{spelltargetinginfo}
            \spellquicktargeting{One creature}{\rngclose}
        \end{spelltargetinginfo}
        \begin{spelleffects}
            \begin{spellattack}{Spellpower vs. Mental}
                \spellsuccess The target is \frightened for 5 rounds.
                \spellcritical Make a second Spellpower vs. Fortitude attack against the target. If that attack succeeds, the target dies.
                \spellfailure The target is \shaken for 5 rounds.
            \end{spellattack}
        \end{spelleffects}
    \end{spellcontent}
    \begin{spellfooter}
        \spellinfo{Enchantment [Delusion, Mind]}*{Arcane, Trickery}
        \miscastrandom
    \end{spellfooter}
    \begin{spellaugments}
        \spellaugment{3}{Mass}{The spell can affect up to five targets.}
    \end{spellaugments}
\end{spellsection}

\begin{spellsection}{Planar Disruption}[2]
    \begin{spellheader}
        \spelldesc{You disrupt a creature's body by partially thrusting it into another plane.}
    \end{spellheader}
    \begin{spellcontent}
        \begin{spelltargetinginfo}
            \spellquicktargeting{One creature}{\rngmed}
        \end{spelltargetinginfo}
        \begin{spelleffects}
            \begin{spellattack}{Spellpower vs. Mental}
                \spellsuccess \spelldamage{physical}.
                \spellcritical Double damage, and if the creature is an outsider native to another plane, it is sent back to its home plane.
                \spellfailure Half damage, and no additional effects.
            \end{spellattack}
        \end{spelleffects}
    \end{spellcontent}
    \begin{spellfooter}
        \spellinfo{Conjuration [Planar, Teleportation]}{Arcane, Divine}
        \miscastrandom
    \end{spellfooter}
    \begin{spellaugments}
        \spellaugment{3}{Mass}{The spell can affect up to five targets. Its damage becomes \spelldamage{physical}[d8]}
        \spellaugment{4}{Empowered}{The damage increases to \spelldamageemp{physical}. If the Mass augment is applied, the damage instead increases to \spelldamage{physical}[d10].}
    \end{spellaugments}
\end{spellsection}

\begin{spellsection}{Poison}[4]
    \begin{spellheader}
        \spelldesc{You create a deadly poison on your foe's skin.}
    \end{spellheader}
    \begin{spellcontent}
        \begin{spelltargetinginfo}
            \spellquicktargeting{One creature}{\rngclose}
        \end{spelltargetinginfo}
        \begin{spelleffects}
            \begin{spellattacktriggered}{At the end of every round, you make a Spellpower vs. Fortitude against the target.}
                \spellsuccess If this is the first successful attack, the target is \sickened. If this is the second successful attack, the target is \nauseated. If this is the third successful attack, the target is \paralyzed.
                \spellfailure If this is the third failed attack, the target resists the poison. No further attacks are made, though the effects of any previous attacks linger until the end of the spell.
            \end{spellattacktriggered}
            \spelldur 5 minutes
        \end{spelleffects}
    \end{spellcontent}
    \begin{spellfooter}
        \spellinfo{Conjuration [Creation]}{Destruction, Divine, Nature}
        \miscastrandom
    \end{spellfooter}
\end{spellsection}

\begin{spellsection}{Polar Ray}[6]
    \begin{spellheader}
        \spelldesc{You fire a blue-white ray of frigid air and ice, freezing your foe in place.}
    \end{spellheader}
    \begin{spellcontent}
        \begin{spelltargetinginfo}
            \spellquicktargeting{One creature or object}{\rngclose}
        \end{spelltargetinginfo}
        \begin{spelleffects}
            \begin{spellattack}{Spellpower vs. Reflex}
                \spellsuccess \spelldamage{cold}. In addition, the target is \slowed for 2 rounds.
                \spellcritical As above, but double damage.
                \spellfailure As above, but half damage.
            \end{spellattack}
        \end{spelleffects}
    \end{spellcontent}
    \begin{spellfooter}
        \spellinfo{Evocation [Cold]}{Arcane, Water}
        \miscastrandom
    \end{spellfooter}
    \begin{spellaugments}
        \spellaugment{3}{Mass}{The spell can affect up to five targets. Its damage becomes \spelldamage{cold}[d8].}
        \spellaugment{4}{Empowered}{The damage increases to \spelldamageemp{cold}. If the Mass augment is applied, the damage instead increases to \spelldamage{cold}[d10].}
    \end{spellaugments}
\end{spellsection}

\begin{spellsection}{Power Word Blind}[6]
    \begin{spellheader}
    \end{spellheader}
    \begin{spellcontent}
        \begin{spelltargetinginfo}
            \spellquicktargeting{One creature}{\rngclose}
            \spellcmp{Verbal only}
        \end{spelltargetinginfo}
        \begin{spelleffects}
            \begin{spellattack}{Spellpower vs. Fortitude}
                \spellsuccess The target is \blinded for 2 rounds.
                \spellcritical The target is \blinded for one year.
                \spellfailure The target is \partiallyblinded for 2 rounds.
            \end{spellattack}
            \spelldur \durbrief or one year
        \end{spelleffects}
    \end{spellcontent}
    \begin{spellfooter}
        \spellinfo{Vivimancy [Flesh]}{Arcane}
        \miscastrandom
    \end{spellfooter}
\end{spellsection}

\begin{spellsection}{Power Word Fear}[3]
    \begin{spellheader}
    \end{spellheader}
    \begin{spellcontent}
        \begin{spelltargetinginfo}
            \spellquicktargeting{One creature}{\rngclose}
            \spellcmp{Verbal only}
        \end{spelltargetinginfo}
        \begin{spelleffects}
            \begin{spellattack}{Spellpower vs. Mental}
                \spellsuccess The target is \frightened by you for 2 rounds.
                \spellcritical The target is \frightened by you for one year.
                \spellfailure The target is \shaken by you for 2 rounds.
            \end{spellattack}
            \spelldur \durbrief or one year
        \end{spelleffects}
    \end{spellcontent}
    \begin{spellfooter}
        \spellinfo{Enchantment [Delusion, Mind]}{Arcane}
        \miscastrandom
    \end{spellfooter}
\end{spellsection}

\begin{spellsection}{Power Word Stagger}[1]
    \begin{spellheader}
    \end{spellheader}
    \begin{spellcontent}
        \begin{spelltargetinginfo}
            \spellquicktargeting{One creature}{\rngclose}
            \spellcmp{Verbal only}
        \end{spelltargetinginfo}
        \begin{spelleffects}
            \begin{spellattack}{Spellpower vs. Fortitude}
                \spellsuccess The target is \staggered for 2 rounds.
                \spellcritical The target is \staggered for one year.
            \end{spellattack}
            \spelldur \durbrief or one year.
        \end{spelleffects}
    \end{spellcontent}
    \begin{spellfooter}
        \spellinfo{Vivimancy [Flesh]}{Arcane}
        \miscastrandom
    \end{spellfooter}
\end{spellsection}

\begin{spellsection}{Precognition}[1]
    \begin{spellheader}
    \end{spellheader}
    \begin{spellcontent}
        \begin{spelltargetinginfo}
            \spellquicktargeting{One creature}{\rngclose}
        \end{spelltargetinginfo}
        \begin{spelleffects}
            \spelleffect The target gains a legend point.
            \spelldur \durshort \dismissable
        \end{spelleffects}
    \end{spellcontent}
    \begin{spellfooter}
        \spellinfo{Divination}{Divination}
        \miscastexplode
    \end{spellfooter}
    \begin{spellaugments}
        \spellaugment{3}{Empowered}{The target gains an additional legend point. This augment can be used multiple times.}
    \end{spellaugments}
\end{spellsection}

\begin{spellsection}{Prismatic Beam}[3]
    \begin{spellheader}
    \end{spellheader}
    \begin{spellcontent}
        \begin{spelltargetinginfo}
            \spellquicktargeting{One creature}{\rngmed}
        \end{spelltargetinginfo}
        \begin{spelleffects}
            \spellspecial The target is struck by a randomly colored beam of light. The beam color determines the effect and the defense used, as shown on \tref{Prismatic Effects}. The damaging effects deal \spelldamage{}.
        \end{spelleffects}
    \end{spellcontent}
    \begin{spellfooter}
        \spellinfo{Universal [Light]}{Arcane}
        \miscastrandom
    \end{spellfooter}
    \begin{spellaugments}
        \spellaugment{4}{Empowered}{The damage of the damaging effects increases to \spelldamageemp{}.}
    \end{spellaugments}
\end{spellsection}
\begin{dtable*}
    \lcaption{Prismatic Effects}
    \begin{dtabularx}{\textwidth}{l >{\lcol}p{3.6em} l >{\lcol}X >{\lcol}X l}
        \tb{1d8} & \tb{Color of Beam} & \tb{Defense} & \tb{Success}\fn{1} & \tb{Critical Success} & \tb{Failure} \\
        \hline
        1 & Red      & Reflex    & Fire damage and ignited for 2 rounds                       & Double damage, ignited              & Half damage, ignited           \\
        2 & Orange   & Mental    & Frightened by you for 2 rounds                             & Panicked by you for 2 rounds        & Shaken by you for 2 rounds     \\
        3 & Yellow   & Reflex    & Electricity damage and partially blinded for 2 rounds      & Double damage, blinded for 2 rounds & Half damage, partially blinded \\
        4 & Green    & \tdash        & Staggered for 2 rounds                                     & Staggered for 5 minutes             & \tdash                             \\
        5 & Blue     & Fortitude & Cold damage and slowed for 2 rounds                        & Double damage, slowed               & Half damage, not slowed        \\
        6 & Indigo   & \tdash        & Disoriented for 2 rounds                                   & Confused for 2 rounds               & \tdash                             \\
        7 & Violet   & None      & Damage of all \glossterm{energy} types & Double damage                       & \tdash                             \\
        8 & Octarine & \tdash        & Struck by two beams; roll twice more, ignoring any ``8'' results.
    \end{dtabularx}
    1 See \pcref{Conditions} for a summary of the conditions imposed.
\end{dtable*}

\begin{spellsection}{Prismatic Storm}[9]
    \begin{spellheader}
    \end{spellheader}
    \begin{spellcontent}
        \begin{spelltargetinginfo}
            \spelltwocol{\spellburst{\arealarge radius}}{\spellrng{\rngmed}}
        \end{spelltargetinginfo}
        \begin{spelleffects}
            \begin{spellattack}{Spellpower vs. Special}
                \spellspecial The target is struck by a randomly colored beam of light. The beam color determines the effect and the defense used, as shown on \tref{Prismatic Effects}. The damaging effects deal \spelldamage{}[d8].
            \end{spellattack}
        \end{spelleffects}
    \end{spellcontent}
    \begin{spellfooter}
        \spellinfo{Universal [Light]}{Arcane}
        \miscastyou
    \end{spellfooter}
\end{spellsection}

\begin{spellsection}{Prismatic Spray}[6]
    \begin{spellheader}
        \spelldesc{This spell causes seven shimmering, intertwined, multicolored beams of light to spray from your hand.}
    \end{spellheader}
    \begin{spellcontent}
        \begin{spelltargetinginfo}
            \spellburst{\arealarge cone}
            \spelltgts{All creatures in the area}
        \end{spelltargetinginfo}
        \begin{spelleffects}
            \begin{spellattack}{Spellpower vs. Special}
                \spellspecial The target is struck by a randomly colored beam of light. The beam color determines the effect and the defense used, as shown on \tref{Prismatic Effects}. The damaging effects deal \spelldamage{}[d8].
            \end{spellattack}
        \end{spelleffects}
    \end{spellcontent}
    \begin{spellfooter}
        \spellinfo{Universal [Light]}{Arcane, Chaos}
        \miscastexplode
    \end{spellfooter}
\end{spellsection}

\begin{spellsection}{Prismatic Wall}[5]
    \begin{spellheader}
    \end{spellheader}
    \begin{spellcontent}
        \begin{spelltargetinginfo}
            \spelltwocol{\spellzone{\areahuge wall, 20 ft.\ high}}{\spellrng{\rngmed}}
        \end{spelltargetinginfo}
        \begin{spelleffects}
            \spelleffect This spell creates a shimmering, multicolored plane of light that blocks all sight.
            \spelldur \durshort \dismissable
        \end{spelleffects}
    \end{spellcontent}
    \begin{spellsubcontent}
        \begin{spelltargetinginfo}
            \spelltwocol{\spelltgr{A creature passes through the wall}}{\spelltgt{Triggering creature}}
        \end{spelltargetinginfo}
        \begin{spelleffects}
            \begin{spellattack}{Spellpower vs. Reflex}
                \spellspecial The target is struck by a randomly colored beam of light. The beam color determines the effect and the defense used, as shown on \tref{Prismatic Effects}. The damaging effects deal \spelldamage{}[d8].
            \end{spellattack}
        \end{spelleffects}
    \end{spellsubcontent}
    \begin{spellfooter}
        \spellinfo{Universal [Light]}{Arcane}
        \miscastexplode
    \end{spellfooter}
\end{spellsection}

\begin{spellsection}{Prohibition}[6]
    \begin{spellheader}
    \end{spellheader}
    \begin{spellcontent}
        \begin{spelltargetinginfo}
            \spellemanation{\arealarge radius fromyou}
        \end{spelltargetinginfo}
        \begin{spelleffects}
            \spelleffect You loudly declare a prohibition on a single, specific action which creatures must not take, such as ``Do not use ranged weapons'' or ``Do not lie''. You may choose any action that must be taken intentionally, but not involuntary actions or states of being, such as breathing or wearing armor. If the rule is too complicated, the spell fails. You also choose a type of \glossterm{energy damage}.

            Whenever a creature breaks the rule, that creature takes \spelldamage{}[d8] of the chosen energy type. You know a creature broke the rule, but not which creature.

            The spell grants all creatures that enter the area an understanding of the prohibition, even if they were unable to understand the rule as originally stated. If you break the rule, the spell ends -- after you suffer the consequences.
            \spelldur \durshort
        \end{spelleffects}
    \end{spellcontent}
    \begin{spellfooter}
        \spellinfo{Divination/Evocation}*{Arcane, Law}
        \spellnotes Mindless creatures are given no special insight into the rule. Any individual creature can only take damage for breaking the rule once per round.

        When cast, this spell has the tag appropriate to the energy type chosen.
        \miscastexplode
    \end{spellfooter}
    \begin{spellaugments}
        \spellaugment{4}{Empowered}{The damage increases to \spelldamage{}[d10] of the chosen energy type.}
    \end{spellaugments}
\end{spellsection}

\begin{spellsection}{Protection from Alignment}[1]
    \begin{spellheader}
    \end{spellheader}
    \begin{spellcontent}
        \begin{spelltargetinginfo}
            \spellquicktargeting{One creature}{\rngclose}
        \end{spelltargetinginfo}
        \begin{spelleffects}
            \spellspecial Choose an alignment other than neutral (chaotic, good, evil, lawful).
            \spelleffect The target gains damage reduction equal to your spellpower against effects that have the chosen alignment, and physical attacks made by creatures with the chosen alignment.
            \spelldur \durshort \dismissable
        \end{spelleffects}
    \end{spellcontent}
    \begin{spellfooter}
        \spellinfo{Abjuration [Shielding]}{Arcane, Chaos, Divine, Evil, Good, Law}
        \spellnotes This spell has the subtype of the alignment opposed to the chosen alignment.
        \miscastrandom
    \end{spellfooter}
    \begin{spellaugments}
        \spellaugment{3}{Mass}{The spell can affect up to five targets.}
        \spellaugment{3}{Protection from Spells}{The target gains \glossterm{magic resistance} against abilities with the chosen alignment, and abilities used by creatures with the chosen alignment.}
        \spellaugment{4}{Retribution}{Whenever a creature with the chosen alignment makes a physical melee attack against the target, you make a Spellpower vs. Mental attack against the attacking creature. Success means the creature takes \spelldamage{divine}[d8].}
    \end{spellaugments}
\end{spellsection}

\pdfbookmark[2]{Q-R}{SpellDescriptionsQR}

\begin{spellsection}{Rapid Reversal}[3]
    \begin{spellheader}
    \end{spellheader}
    \begin{spellcontent}
        \begin{spelltargetinginfo}
            \spellquicktargeting{One creature}{\rngmed}
            \spelltime{Swift action}
        \end{spelltargetinginfo}
        \begin{spelleffects}
            \begin{spellattack}{Spellpower vs. Mental}
                \spellsuccess The target teleports back to the location it occupied at the beginning of the round.
                If that location is out of range, or is currently occupied, this spell automatically fails.
            \end{spellattack}
            \spellspecial After casting this spell, you cannot cast it again for 5 rounds.
        \end{spelleffects}
    \end{spellcontent}
    \begin{spellfooter}
        \spellinfo{Conjuration [Teleportation]}{Arcane}
        \miscastrandom
    \end{spellfooter}
\end{spellsection}

\begin{spellsection}{Read Mind}[2]
    \begin{spellheader}
    \end{spellheader}
    \begin{spellcontent}
        \begin{spelltargetinginfo}
            \spellquicktargeting{One creature}{\rngmed}
        \end{spelltargetinginfo}
        \begin{spelleffects}
            \begin{spellattack}{Spellpower vs. Mental}
                \spellsuccess You can read the target's surface thoughts. You gain a \plus4 bonus to Bluff, Persuasion, and Intimidate checks against a creature whose mind you are reading.
            \end{spellattack}
            \spelldur Focus
        \end{spelleffects}
    \end{spellcontent}
    \begin{spellfooter}
        \spellinfo{Divination [Mind]}{Divination}
        \miscastrandom
    \end{spellfooter}
    \begin{spellaugments}
        \spellaugment{3}{Certain}{The spell takes effect automatically, without an attack.}
        \spellaugment{3}{Mass}{The spell can affect up to five targets.}
    \end{spellaugments}
\end{spellsection}

\begin{spellsection}{Regeneration}[1]
    \begin{spellheader}
        \spelldesc{You grant an ally's body the ability to heal itself rapidly.}
    \end{spellheader}
    \begin{spellcontent}
        \begin{spelltargetinginfo}
            \spellquicktargeting{One living creature}{Touch}
        \end{spelltargetinginfo}
        \begin{spelleffects}
            \spelleffect At the end of every round, the target heals hit points equal to your spellpower.
            \spelldur 5 rounds
        \end{spelleffects}
    \end{spellcontent}
    \begin{spellfooter}
        \spellinfo{Transmutation [Imbuement]}{Divine, Nature}
        \miscastexplode
    \end{spellfooter}
    \begin{spellaugments}
        \spellaugment{3}{Critical Wounds}{At the end of every round, you can heal the target for one critical damage per two spellpower instead of the normal healing.}
        \spellaugment{3}{Regrowth}{Instead of the normal effect, you can regrow lost portions of the target's body and reattach severed limbs or body parts. Both you and the target must do nothing but concentrate on regrowing the lost body part or reattaching the severed limb for the spell's duration.}
    \end{spellaugments}
\end{spellsection}

\begin{spellsection}{Repulsion}[6]
    \begin{spellheader}
        \spelldesc{An invisible, mobile field surrounds you and prevents creatures from approaching you.}
    \end{spellheader}
    \begin{spellcontent}
        \begin{spelltargetinginfo}
            \spellemanation{\areahuge radius from you}
        \end{spelltargetinginfo}
        \begin{spelleffects}
            Whenever a creature within the area tries to move towards you, you can make a Spellpower vs. Mental attack against it.
            \spellsuccess The creature is unable to move towards you for the remainder of the spell. It can stand still, or alter the direction of its movement to move parallel towards you or away from you.
            \spellcritical The creature's movement is cancelled, and it is shoved backwards to the edge of the area.
            If it encounters an obstacle, it stops moving. It cannot move towards you for the remainder of the spell.
            \spellfailure The creature's movement is unimpeded, and it is immune to the spell for the rest of its duration.
            \spelldur \durshort \dismissable
        \end{spelleffects}
    \end{spellcontent}
    \begin{spellfooter}
        \spellinfo{Evocation [Telekinesis]}{Arcane}
        \spellnotes If you move towards a creature held at bay by the barrier, the spell continues to prevent that creature from approaching you, but the creature suffers no other ill effect.
        \miscastexplode
    \end{spellfooter}
\end{spellsection}

\begin{spellsection}{Resist Energy}[1]
    \begin{spellheader}
    \end{spellheader}
    \begin{spellcontent}
        \begin{spelltargetinginfo}
            \spellquicktargeting{One creature}{\rngclose}
        \end{spelltargetinginfo}
        \begin{spelleffects}
            \spelleffect The target gains damage reduction against \glossterm{energy damage} equal to your spellpower.
            \spelldur \durshort
        \end{spelleffects}
    \end{spellcontent}
    \begin{spellfooter}
        \spellinfo{Abjuration [Shielding]}{Arcane, Divine, Nature}
        \miscastexplode
    \end{spellfooter}
    \begin{spellaugments}
        \spellaugment{3}{Empowered}{The damage reduction increases to twice your spellpower.}
    \end{spellaugments}
\end{spellsection}

\begin{spellsection}{Resist Magic}[3]
    \begin{spellheader}
    \end{spellheader}
    \begin{spellcontent}
        \begin{spelltargetinginfo}
            \spellquicktargeting{One creature}{\rngclose}
        \end{spelltargetinginfo}
        \begin{spelleffects}
            \spelleffect The target gains \glossterm{magic resistance} equal to 10 \add your spellpower.
            \spelldur \durshort
            \spellspecial This spell cannot be dispelled.
        \end{spelleffects}
    \end{spellcontent}
    \begin{spellfooter}
        \spellinfo{Abjuration [Shielding]}{Arcane, Protection}
        \miscastrandom
    \end{spellfooter}
    \begin{spellaugments}
        \spellaugment{3}{Mass}{The spell can affect up to five targets.}
        \spellaugment{4}{Reflective}{Targeted spells resisted by this spell are reflected back at their original caster.}
    \end{spellaugments}
\end{spellsection}

\begin{spellsection}{Resist Poison}[1]
    \begin{spellheader}
    \end{spellheader}
    \begin{spellcontent}
        \begin{spelltargetinginfo}
            \spellquicktargeting{One creature}{\rngclose}
            \spelltime{Swift action}
        \end{spelltargetinginfo}
        \begin{spelleffects}
            \spelleffect The target becomes temporarily unaffected by poisons. Poisons it is exposed to do not make attacks against it. This effect does not prevent the target from becoming poisoned, and any poisons in the target's system when the spell ends will continue their effects normally.
            \spelldur \durshort
        \end{spelleffects}
    \end{spellcontent}
    \begin{spellfooter}
        \spellinfo{Vivimancy [Flesh]}{Divine, Nature}
        \spellnotes This spell does not cure any damage that poison may have already done.
        \miscastrandom
    \end{spellfooter}
    \begin{spellaugments}
        \spellaugment{2}{Immunizing}{The target is also immune to being poisoned.}
        \spellaugment{3}{Mass}{The spell can affect up to five targets.}
    \end{spellaugments}
\end{spellsection}

\begin{spellsection}{Restoration}[3]
    \begin{spellheader}
        \spelldesc{You remove negative conditions from an ally, restoring them to health.}
    \end{spellheader}
    \begin{spellcontent}
        \begin{spelltargetinginfo}
            \spellquicktargeting{One willing creature}{\rngclose}
        \end{spelltargetinginfo}
        \begin{spelleffects}
            \spelleffect The target is cured of all simple conditions afflicting it. Special effects with unique properties and some unusual conditions cannot be removed in this way.
        \end{spelleffects}
    \end{spellcontent}
    \begin{spellfooter}
        \spellinfo{Vivimancy [Flesh]}{Divine, Life, Nature}
        \spellnotes The following conditions are considered simple conditions: blinded, confused, dazed, dazzled, deafened, goaded, exhausted, fascinated, fatigued, frightened, nauseated, panicked, shaken, sickened, stunned, and taunted.
        \miscastexplode
    \end{spellfooter}
\end{spellsection}

\begin{spellsection}{Retrieve Object}[1]
    \begin{spellheader}
        \spelldesc{You teleport an object into your hand.}
    \end{spellheader}
    \begin{spellcontent}
        \begin{spelltargetinginfo}
            \spellquicktargeting{One unattended object (Medium or smaller)}{\rngmed}
        \end{spelltargetinginfo}
        \begin{spelleffects}
            \spelleffect The target teleports into your hands.
        \end{spelleffects}
    \end{spellcontent}
    \begin{spellfooter}
        \spellinfo{Conjuration [Teleportation]}{Arcane}
        \spellnotes This spell has no effect on attended objects or intelligent items.
        \miscastrandom
    \end{spellfooter}
    \begin{spellaugments}
        \spellaugment{1}{Giant}{The spell can affect a target one size category larger. This augment can be used multiple times.}
        \spellaugment{3}{Forced}{If you make a successful Spellpower vs. Mental attack, you can also retrieve attended objects.}
    \end{spellaugments}
\end{spellsection}

\begin{spellsection}{Revelation}[9]
    \begin{spellheader}
        \spelldesc{You grant the target a powerful vision of a possible future.}
    \end{spellheader}
    \begin{spellcontent}
        \begin{spelltargetinginfo}
            \spellquicktargeting{One creature}{\rngmed}
        \end{spelltargetinginfo}
        \begin{spelleffects}
            \spellspecial This spell has three versions. Its effects depend on which version is chosen.
            \spelleffect[\subspell{Revelation of Destruction}] You inflict a vision of a terrible future upon the target. It is \severelyimpaired with attacks and checks as it struggles to avoid the certainty of its own doom.
            \spelleffect[\subspell{Revelation of Prowess}] You show the target a vision of its success in the combat to come. It gains the benefits of a \spell{precognition} spell, except that the target gains three legend points instead of one.
            \spelleffect[\subspell{Revelation of Truth}] You show the target the truth of the world around it. It gains the benefits of a \spell{true seeing} spell.
            \spelldur \durshort
        \end{spelleffects}
    \end{spellcontent}
    \begin{spellfooter}
        \spellinfo{Divination}{Arcane, Knowledge}
        \spellnotes Creatures without an Intelligence are not affected by this spell.
        \miscastrandom
    \end{spellfooter}
\end{spellsection}

\begin{spellsection}{Reverse Gravity}[8]
    \begin{spellheader}
        \spelldesc{You reverse gravity an area, causing everything within it to fall upwards.}
    \end{spellheader}
    \begin{spellcontent}
        \begin{spelltargetinginfo}
            \spelltwocol{\spellzone{\areamed radius cylinder, 50 ft.\ high}}{\spellrng{\rngmed}}
            \spelltgts{Everything in the area}
        \end{spelltargetinginfo}
        \begin{spelleffects}
            \spelleffect Gravity in the area is reversed for the target. It falls upwards, reaching the top of the area within 1 round. If it strikes a solid object, such as a ceiling, it is affected in the same way as it would be during a normal fall. Otherwise, it floats at the top of the area, oscillating slightly.
            \spelldur \durshort
        \end{spelleffects}
    \end{spellcontent}
    \begin{spellfooter}
        \spellinfo{Transmutation}{Arcane, Air, Chaos}
        \spellnotes Creatures who can fly or levitate can keep themselves from falling, though the shift in gravity can be disorienting. When the spell ends, everything still floating falls, potentially taking damage for the fall.
        \miscastyou
    \end{spellfooter}
\end{spellsection}

\begin{spellsection}{Revivify}[5]
    \begin{spellheader}
        \spelldesc{You reconnect a corpse's soul with its body before the soul has completely passed on.}
    \end{spellheader}
    \begin{spellcontent}
        \begin{spelltargetinginfo}
            \spellquicktargeting{One dead creature}{Touch}
            \spellcmp{Verbal, Somatic, and Material}
        \end{spelltargetinginfo}
        \begin{spelleffects}
            \spelleffect If the target has been dead for no longer than 5 rounds, it is restored to life, as the \ritual{resurrection} ritual.
        \end{spelleffects}
    \end{spellcontent}
    \begin{spellfooter}
        \spellinfo{Vivimancy [Life]}{Divine, Life}
        \spellmat{Diamonds worth at least 500 gp.}
        \miscastexplode
    \end{spellfooter}
    \begin{spellaugments}
        \spellaugment{2}{Extended}{You can affect a target that has been dead for up to one round per spellpower.}
    \end{spellaugments}
\end{spellsection}

\begin{spellsection}{Rock Blast}[2]
    \begin{spellheader}
        \spelldesc{You create a blast of rocks that damages everything in its path.}
    \end{spellheader}
    \begin{spellcontent}
        \begin{spelltargetinginfo}
            \spellburst{\areamed line, 10 ft.\ wide}
            \spelltgts{Everything in the area}
        \end{spelltargetinginfo}
        \begin{spelleffects}
            \begin{spellattack}{Spellpower vs. Fortitude}
                \spellsuccess \spelldamage{bludgeoning}[d8].
                \spellcritical Double damage.
                \spellfailure Half damage.
            \end{spellattack}
        \end{spelleffects}
    \end{spellcontent}
    \begin{spellfooter}
        \spellinfo{Conjuration [Creation, Earth]}{Nature, Wild}
        \miscastexplode
    \end{spellfooter}
    \begin{spellaugments}
        \spellaugment{1}{Precise}{The spell only targets enemies in the area.}
        \spellaugment{2}{Widened}{The spell's area becomes a \arealarge line, 10 ft.\ wide.}
        \spellaugment{4}{Empowered}{The damage increases to \spelldamage{bludgeoning}[d10].}
    \end{spellaugments}
\end{spellsection}

\begin{spellsection}{Rotburst}[3]
    \begin{spellheader}
        \spelldesc{You rot the flesh of nearby creatures.}
    \end{spellheader}
    \begin{spellcontent}
        \begin{spelltargetinginfo}
            \spelltwocol{\spellburst{\areasmall radius}}{\spellrng{\rngmed}}
            \spelltgts{Everything in the area}
        \end{spelltargetinginfo}
        \begin{spelleffects}
            \begin{spellattack}{Spellpower vs. Fortitude}
                \spellsuccess For the next 5 rounds, the target takes physical damage equal to your spellpower at the end of each round. This damage ignores the hardness of non-metallic objects.
                \spellcritical As above, but the target takes double damage each round.
            \end{spellattack}
        \end{spelleffects}
    \end{spellcontent}
    \begin{spellfooter}
        \spellinfo{Vivimancy [Flesh]}{Destruction, Nature}
        \miscastexplode
    \end{spellfooter}
    \begin{spellaugments}
        \spellaugment{2}{Widened}{The spell's area becomes a \areamed radius.}
    \end{spellaugments}
\end{spellsection}

\begin{spellsection}{Rotting Grasp}[1]
    \begin{spellheader}
        \spelldesc{You rot your foe's flesh with a touch.}
    \end{spellheader}
    \begin{spellcontent}
        \begin{spelltargetinginfo}
            \spellquicktargeting{One creature or object}{\rngmed}
        \end{spelltargetinginfo}
        \begin{spelleffects}
            \begin{spellattack}{Spellpower vs. Fortitude}
                \spellsuccess For the next 5 rounds, the target takes physical damage equal to your spellpower at the end of each round. This damage ignores the hardness of non-metallic objects.
                \spellcritical As above, but the target takes double damage each round.
                \spellfailure As above, except that the effect lasts for 2 rounds.
            \end{spellattack}
        \end{spelleffects}
    \end{spellcontent}
    \begin{spellfooter}
        \spellinfo{Vivimancy [Flesh]}{Destruction, Nature}
        \miscastexplode
    \end{spellfooter}
\end{spellsection}

\pdfbookmark[2]{S}{SpellDescriptionsS}

\begin{spellsection}{Sanctuary}[1]
    \begin{spellheader}
    \end{spellheader}
    \begin{spellcontent}
        \begin{spelltargetinginfo}
            \spellquicktargeting{One creature}{Touch}
        \end{spelltargetinginfo}
        \begin{spelleffects}
            \spelleffect The target is protected from attacks. If it takes any actions other than movement, this spell immediately ends.
            \spelldur \durshort
        \end{spelleffects}
    \end{spellcontent}
    \begin{spellsubcontent}
        \begin{spelltargetinginfo}
            \spelltgr{A creature attacks the target}
            \spelltgt{The attacking creature}
        \end{spelltargetinginfo}
        \begin{spelleffects}
            \begin{spellattack}{Spellpower vs. Mental}
                \spellsuccess The target's attack fails, and it is unable to attack the protected creature for the next 5 rounds.
            \end{spellattack}
        \end{spelleffects}
    \end{spellsubcontent}
    \begin{spellfooter}
        \spellinfo{Enchantment [Compulsion, Mind, Shielding]}{Arcane, Divine, Protection}
        \spellnotes This is considered a \glossterm{Mind} effect on any creature that attempts to attack the target. Creatures immune to Mind effects can attack the target freely.
        \miscastexplode
    \end{spellfooter}
    \begin{spellaugments}
        \spellaugment{3}{Mass}{The spell can affect up to five targets. Each target is protected individually. If a target attacks, it loses its protection, but other targets do not.}
    \end{spellaugments}
\end{spellsection}

\begin{spellsection}{Scorching Ray}[4]
    \begin{spellheader}
        \spelldesc{You blast your foe with a fiery ray.}
    \end{spellheader}
    \begin{spellcontent}
        \begin{spelltargetinginfo}
            \spellquicktargeting{One creature or object}{\rngmed}
        \end{spelltargetinginfo}
        \begin{spelleffects}
            \begin{spellattack}{Spellpower vs. Reflex}
                \spellsuccess \spelldamage{fire}. In addition, the target is \ignited for 2 rounds.
                \spellcritical As above, but double damage.
                \spellfailure As above, but half damage.
            \end{spellattack}
        \end{spelleffects}
    \end{spellcontent}
    \begin{spellfooter}
        \spellinfo{Evocation [Fire]}{Arcane, Destruction, Nature}
        \miscastrandom
    \end{spellfooter}
    \begin{spellaugments}
        \spellaugment{3}{Mass}{The spell can affect up to five targets. Its damage becomes \spelldamage{fire}[d8].}
        \spellaugment{4}{Empowered}{The damage increases to \spelldamageemp{fire}. If the Mass augment is applied, the damage instead increases to \spelldamage{fire}[d10].}
    \end{spellaugments}
\end{spellsection}

\begin{spellsection}{Scrybolt}[7]
    \begin{spellheader}
        \spelldesc{You attack your foe's life force from a great distance, beyond risk of reprisal.}
    \end{spellheader}
    \begin{spellcontent}
        \begin{spelltargetinginfo}
            \spellquicktargeting{One creature}{One mile \rngunrestricted}
        \end{spelltargetinginfo}
        \begin{spelleffects}
            \spellspecial You can target any creature that you can unambiguously identify, regardless of its location.
            \begin{spellattack}{Spellpower vs. Mental}
                \spellsuccess \spelldamage{life}.
                \spellcritical Double damage.
                \spellfailure Half damage.
            \end{spellattack}
        \end{spelleffects}
    \end{spellcontent}
    \begin{spellfooter}
        \spellinfo{Divination/Vivimancy [Life, Scrying]}{Arcane, Knowledge}
        \spellnotes This is a Scrying spell, and effects that block or inhibit scrying can also prevent this spell from dealing damage.
        \miscastexplode
    \end{spellfooter}
    \begin{spellaugments}
        \spellaugment{2}{Distant}{The spell's range increases to 10 miles.}
    \end{spellaugments}
\end{spellsection}

\begin{spellsection}{Searing Light}[2]
    \begin{spellheader}
        \spelldesc{You fire a blast of light that strikes your foe.}
    \end{spellheader}
    \begin{spellcontent}
        \begin{spelltargetinginfo}
            \spellquicktargeting{One creature or object}{\rngclose}
        \end{spelltargetinginfo}
        \begin{spelleffects}
            \begin{spellattack}{Spellpower vs. Reflex}
                \spellspecial You gain a \plus5 bonus to accuracy against creatures vulnerable to sunlight.
                \spellsuccess \spelldamage{solar}. In addition, the target is \partiallyblinded for 2 rounds.
                \spellcritical Double damage, and the target is \blinded instead of partially blinded.
                \spellfailure Half damage, and no additional effects.
            \end{spellattack}
        \end{spelleffects}
    \end{spellcontent}
    \begin{spellfooter}
        \spellinfo{Illusion [Figment, Light]}{Arcane, Nature}
        \miscastrandom
    \end{spellfooter}
    \begin{spellaugments}
        \spellaugment{3}{Mass}{The spell can affect up to five targets. Its damage becomes \spelldamage{solar}[d8].}
        \spellaugment{4}{Empowered}{The damage increases to \spelldamageemp{solar}. If the Mass augment is applied, the damage instead increases to \spelldamage{solar}[d10].}
    \end{spellaugments}
\end{spellsection}

\begin{spellsection}{See Invisibility}[1]
    \begin{spellheader}
    \end{spellheader}
    \begin{spellcontent}
        \begin{spelltargetinginfo}
            \spellquicktargeting{One creature}{\rngclose}
        \end{spelltargetinginfo}
        \begin{spelleffects}
            \spelleffect The target can see any objects or beings that are invisible within its range of vision as if they were normally visible. Such creatures are visible as translucent shapes, allowing the target to easily distinguish between visible and invisible creatures.
            \spelldur \durpersonallong
        \end{spelleffects}
    \end{spellcontent}
    \begin{spellfooter}
        \spellinfo{Divination [Imbuement]}{Arcane}
        \spellnotes The spell does not reveal the method used to obtain invisibility. It does not reveal illusions other than invisibility. It does not reveal creatures who are simply hiding, concealed, or otherwise hard to see.
        \miscastexplode
    \end{spellfooter}
    \begin{spellaugments}
        \spellaugment{3}{Mass}{The spell can affect up to five targets.}
    \end{spellaugments}
\end{spellsection}

\begin{spellsection}{Seismic Slam}[6]
    \begin{spellheader}
        \spelldesc{You slam your opponent into the ground with a mighty surge of strength, trapping them in the earth.}
    \end{spellheader}
    \begin{spellcontent}
        \begin{spelltargetinginfo}
            \spellquicktargeting{One creature or object on solid ground}{\rngtouch}
        \end{spelltargetinginfo}
        \begin{spelleffects}
            \begin{spellattack}{Strength or Spellpower vs. Fortitude (shove)}
                \spellsuccess \spelldamage{bludgeoning}, and the target is \immobilized for 2 rounds.
                This is a \glossterm{Physical} effect, and does not allow \glossterm{magic resistance}.
                \spellcritical As above, but double damage.
                \spellfailure As above, but half damage.
            \end{spellattack}
        \end{spelleffects}
    \end{spellcontent}
    \begin{spellfooter}
        \spellinfo{Transmutation [Enhancement]}{Strength}
        \spellnotes The somatic components of this spell consist of slamming the target into the ground. The ground beneath the target also takes damage from the spell, which may cause weak ground (such as a building floor) to break.
        \miscastexplode
    \end{spellfooter}
    \begin{spellaugments}
        \spellaugment{4}{Empowered}{The damage increases to \spelldamageemp{bludgeoning}.}
    \end{spellaugments}
\end{spellsection}

\begin{spellsection}{Sever Magic}[8]
    \begin{spellheader}
        \spelldesc{You disconnect your foe from the power of magic, preventing any magic from affecting it.}
    \end{spellheader}
    \begin{spellcontent}
        \begin{spelltargetinginfo}
            \spellquicktargeting{One creature or object}{\rngmed}
        \end{spelltargetinginfo}
        \begin{spelleffects}
            \begin{spellattack}
                \spellsuccess All magical abilities and magic items fail to function on the target for 2 rounds. The target cannot activate them or be affected by them, and any existing abilities the target has and and effects on the target are suppressed.
                \spellcritical As above, except that the effect lasts for thirty days.
                \spellfailure The target has a 20\% chance to fail when using magical abilities and magic items for 2 rounds. It is otherwise affected normally by magic.
            \end{spellattack}
        \end{spelleffects}
    \end{spellcontent}
    \begin{spellfooter}
        \spellinfo{Abjuration [Thaumaturgy]}{Arcane, Magic}
        \miscastexplode
    \end{spellfooter}
\end{spellsection}

\begin{spellsection}{Shadow Duplicate}[8]
    \begin{spellheader}
    \end{spellheader}
    \begin{spellcontent}
        \begin{spelltargetinginfo}
            \spellrng{\rngmed}
            \spellcmp{Somatic only}
        \end{spelltargetinginfo}
        \begin{spelleffects}
            \spelleffect You create an illusory duplicate of yourself, as the \spell{create image} spell. The duplicate looks, sounds, and smells like you, but is intangible. Normally, it mimics your actions perfectly, including speech.
            \par As a swift action, you can attune to the projected image. This has several effects.
            \begin{itemize}
                \item You see and hear from the image's location, rather from where your body is.
                \item You can control the image's actions independently from your own actions. Each round, it can move up to 100 feet in any direction, including vertically.
                \item If you have line of effect to the image, you may have any spells you cast originate from the image instead of from you.
                    This causes you to measure range, line of effect, and so on from the image's location, rather than from your location.
            \end{itemize}

            As a free action, you can stop attuning to the projected image, restoring your perceptions and spells to your original body.

            \spelldur \durlong \dismissable
        \end{spelleffects}
    \end{spellcontent}
    \begin{spellfooter}
        \spellinfo{Divination/Illusion [Figment, Scrying]}{Arcane}
        \spellnotes Since the image is not a creature, it is difficult to disrupt, and many spells have no effect on it.
        However, the image is treated as a scrying sensor for the purpose of abilities that affect scrying sensors.
        \miscastexplode
    \end{spellfooter}
\end{spellsection}

\begin{spellsection}{Shadow Wall}[6]
    \begin{spellheader}
        \spelldesc{You create a wall that blocks your foes and allows your allies through.}
    \end{spellheader}
    \begin{spellcontent}
        \begin{spelltargetinginfo}
            \spelltwocol{\spellarea{\arealarge solid wall, 10 ft.\ high}}{\spellrng{\rngmed}}
        \end{spelltargetinginfo}
        \begin{spelleffects}
            \spelleffect This spell creates an image of a wall or similar physical barricade.
            In the same position, this spell creates a barrier of telekinetic force that blocks the passage of most objects and creatures.
            However, your allies and their equipment can pass through the barrier unimpeded.
            The wall blocks all spell effects, including those of your allies.

            When you cast this spell, you make a check with a bonus equal to your spellpower \add 10. Creatures can recognize the wall is created by illusory magic by making an Awareness check against a DR equal to your check result when casting the spell. A creature gets a \plus10 bonus on this Awareness check when using senses which should be present in the figment, but which are missing.
            \spelldur \durshort
        \end{spelleffects}
    \end{spellcontent}
    \begin{spellfooter}
        \spellinfo{Evocation/Illusion [Figment, Telekinesis]}{Arcane, Trickery}
        \miscastexplode
    \end{spellfooter}
\end{spellsection}

\begin{spellsection}{Shadowbolt}[2]
    \begin{spellheader}
        \spelldesc{You fire an invisible bolt of energy at your foe.}
    \end{spellheader}
    \begin{spellcontent}
        \begin{spelltargetinginfo}
            \spellquicktargeting{One creature or object}{\rngmed}
            \spellcmp{Somatic only}
        \end{spelltargetinginfo}
        \begin{spelleffects}
            \spelleffect \spelldamage{cold}[d10].
        \end{spelleffects}
    \end{spellcontent}
    \begin{spellfooter}
        \spellinfo{Evocation/Illusion [Energy, Subtle]}{Arcane}
        \spellnotes \subtlespellnotes
        \miscastexplode
    \end{spellfooter}
    \begin{spellaugments}
        \spellaugment{3}{Mass}{The spell can affect up to five targets. Its damage becomes \spelldamage{cold}[d6].}
        \spellaugment{4}{Empowered}{The damage increases to \spelldamage{cold}. If the Mass augment is applied, the damage instead increases to \spelldamage{cold}[d8].}
    \end{spellaugments}
\end{spellsection}

\begin{spellsection}{Shadowshield}[9]
    \begin{spellheader}
        \spelldesc{You shield your ally with an invisible force that negates attacks without alerting foes to its presence.}
    \end{spellheader}
    \begin{spellcontent}
        \begin{spelltargetinginfo}
            \spellquicktargeting{One creature}{\rngclose}
        \end{spelltargetinginfo}
        \begin{spelleffects}
            \spelleffect All attacks that would affect the creature, including magical attacks, have a 50\% chance to fail. The shield is selective, and does not inhibit beneficial effects.

            Whenever the shield negates an attack, it alters the creature's appearance (including smell, sound, and other senses, as appropriate) with a glamer. This causes the creature to seem as if were affected by the attack. Outside observers have no way of knowing which attacks were absorbed by the umbra unless they can recognize the illusion. The spell does not attempt to mimic the effects of extraordinary attacks which cannot be disguised, such as attacks which would destroy the creature's body.

            When you cast this spell, you make a check with a bonus equal to your spellpower \add 10. Creatures can recognize the attack effects are created by illusory magic by interacting with the target physically, or by making an Awareness check against a DR equal to your check result when casting the spell.
            \spelldur \durshort
        \end{spelleffects}
    \end{spellcontent}
    \begin{spellfooter}
        \spellinfo{Abjuration/Illusion [Glamer, Shielding, Subtle]}*{Arcane, Protection, Trickery}
        \miscastrandom
    \end{spellfooter}
\end{spellsection}

\begin{spellsection}{Shadowstep}[5]
    \begin{spellheader}
    \end{spellheader}
    \begin{spellcontent}
        \begin{spelltargetinginfo}
            \spelltgt{You}
        \end{spelltargetinginfo}
        \begin{spelleffects}
            \spelleffect This spell has three simultaneous effects.
            First, you become invisible for 5 rounds, as \spell{invisibility}.
            Second, you teleport up to 100 feet, as \spell{dimension slide}.
            Third, an illusory duplicate of you appears in your original location for 5 rounds, as the \spell{create image} spell.
            The image appears superimposed over your original position, preventing onlookers from noticing your disappearance.

            You can control the image of yourself as you would control any other figment with \spell{create image}. If not directed, it remains stationary.
        \end{spelleffects}
    \end{spellcontent}
    \begin{spellfooter}
        \spellinfo{Conjuration/Illusion [Figment, Glamer, Teleportation]}*{Arcane, Travel, Trickery}
        \miscastexplode
    \end{spellfooter}
    \begin{spellaugments}
        \spellaugment{1}{Sensory}{The duplicate affects an additional sense: sound, smell, texture, or temperature. This augment can be used multiple times, affecting a different sense each time.}
        \spellaugment{2}{Scripted}{When you cast the spell, you set a simple script for your duplicate to follow. It follows that script automatically. As a swift action, you can concentrate to change the script for the remainder of the spell.}
    \end{spellaugments}
\end{spellsection}

\begin{spellsection}{Shadowstorm}[7]
    \begin{spellheader}
        \spelldesc{You create an invisible storm of energy that damages your foes.}
    \end{spellheader}
    \begin{spellcontent}
        \begin{spelltargetinginfo}
            \spelltwocol{\spellarea{\areamed radius zone}}{\spellrng{\rngmed}}
            \spellcmp{Somatic only}
        \end{spelltargetinginfo}
        \begin{spelleffects}
            \spelleffect At the end of every round, all enemies in the area take 1d10 cold damage per four spellpower.
            \spelldur 5 rounds
        \end{spelleffects}
    \end{spellcontent}
    \begin{spellfooter}
        \spellinfo{Evocation/Illusion [Energy, Subtle]}{Arcane, Trickery}
        \miscastexplode
    \end{spellfooter}
\end{spellsection}

\begin{spellsection}{Share Pain}[2]
    \begin{spellheader}
    \end{spellheader}
    \begin{spellcontent}
        \begin{spelltargetinginfo}
            \spellquicktargeting*{Two willing creatures}{\rngtouch}
        \end{spelltargetinginfo}
        \begin{spelleffects}
            \spellspecial When you cast this spell, you choose which target will be protected.
            \spelleffect When the protected creature would take hit point damage, it instead loses half that many hit points (rounded down), and the other target loses hit points equal to the other half of the damage (rounded up).

            If the targets get out of range of each other, the effect is suppressed until they return within range.
            \spelldur \durmed \dismissable
        \end{spelleffects}
    \end{spellcontent}
    \begin{spellfooter}
        \spellinfo{Abjuration/Vivimancy [Life, Shielding]}{Arcane, Good, Divine, Life, Protection}
        \spellnotes The loss of hit points caused by this spell is not damage, and is not affected by damage reduction or other abilities which affect damage.
        \miscastexplode
    \end{spellfooter}
    \begin{spellaugments}
        \spellaugment{1}{Flexible}{As a swift action, you can suppress or resume the spell's effects without dismissing the spell.}
        \spellaugment{2}{Lifebound}{The targets also share healing in the same way that they share damage.}
        \spellaugment{3}{Distant}{The targets are considered to be within range of each other as long as they are on the same plane, regardless of their distance from each other.}
    \end{spellaugments}
\end{spellsection}

\begin{spellsection}{Shining Beacon}[3]
    \begin{spellheader}
        \spelldesc{You become a beacon of light, illuminating your surroundings and blinding your foes.}
    \end{spellheader}
    \begin{spellcontent}
        \begin{spelleffects}
            \spelleffect You radiate bright light up to a 500 foot radius, and dim light an additional 500 feet beyond that. In addition, whenever a creature within \rngclose range of you attacks you, it is \partiallyblinded for 2 rounds.
            \spelldur \durshort
        \end{spelleffects}
    \end{spellcontent}
    \begin{spellfooter}
        \spellinfo{Illusion [Figment, Light, Visual]}{Arcane}
        \miscastexplode
    \end{spellfooter}
    \begin{spellaugments}
        \spellaugment{2}{Aura}{Allies within a \rngmed radius emanation from you also partially blind foes that attack them.}
        \spellaugment{2}{Distant}{The range of partial blinding is extended to include the entire radius of bright light.}
        \spellaugment{3}{Widened}{The radius of bright light is extended to 1 mile, with dim light for an additional 1 mile beyond that.}
    \end{spellaugments}
\end{spellsection}

\begin{spellsection}{Shocking Grasp}[1]
    \begin{spellheader}
        \spelldesc{You deliver a powerful electrical shock to your foe.}
    \end{spellheader}
    \begin{spellcontent}
        \begin{spelltargetinginfo}
            \spellquicktargeting{One creature or object}{Touch}
        \end{spelltargetinginfo}
        \begin{spelleffects}
            \begin{spellattack}{Spellpower vs. Reflex}
                \spellspecial You gain a \plus5 bonus to accuracy if the target is wearing metal armor or otherwise has a significant quantity of metal.
                \spellsuccess \spelldamage{electricity}.
                \spellcritical Double damage, and the target is \staggered for 5 rounds.
                \spellfailure Half damage, and no additional effects.
            \end{spellattack}
        \end{spelleffects}
    \end{spellcontent}
    \begin{spellfooter}
        \spellinfo{Evocation [Electricity]}{Arcane, Destruction, Nature}
        \miscastexplode
    \end{spellfooter}
    \begin{spellaugments}
        \spellaugment{4}{Empowered}{The damage increases to \spelldamageemp{electricity}.}
    \end{spellaugments}
\end{spellsection}

\begin{spellsection}{Shout}[4]
    \begin{spellheader}
        \spelldesc{You emit an ear-splitting yell that deafens and damages creatures in its path.}
    \end{spellheader}
    \begin{spellcontent}
        \begin{spelltargetinginfo}
            \spellburst{\areamed cone}
            \spelltgts{Everything in the area}
            \spellcmp{Verbal only}
        \end{spelltargetinginfo}
        \begin{spelleffects}
            \begin{spellattack}{Spellpower vs. Fortitude}
                \spellspecial You gain a \plus5 bonus to accuracy against brittle or crystalline objects and creatures.
                \spellsuccess \spelldamage{sonic}[d8]. In addition, the target is \deafened for 5 rounds.
                \spellfailure Half damage, and no additional effects.
            \end{spellattack}
        \end{spelleffects}
    \end{spellcontent}
    \begin{spellfooter}
        \spellinfo{Evocation [Sonic]}{Arcane, Destruction, Strength}
        \miscastexplode
    \end{spellfooter}
    \begin{spellaugments}
        \spellaugment{3}{Widened}{The spell's area becomes a \arealarge cone.}
        \spellaugment{4}{Empowered}{The damage increases to \spelldamage{sonic}[d10].}
    \end{spellaugments}
\end{spellsection}

\begin{spellsection}{Shrink}[1]
    \begin{spellheader}
    \end{spellheader}
    \begin{spellcontent}
        \begin{spelltargetinginfo}
            \spellquicktargeting{One creature (Small or larger)}{\rngclose}
        \end{spelltargetinginfo}
        \begin{spelleffects}
            \begin{spellattack}{Spellpower vs. Fortitude}
                \spellsuccess The target and its equipment instantly shrinks, halving its height and dividing its weight by 8. This changes the creature's size category to the next smaller one. This has several effects.
                \begin{itemize}
                    \item \minus4 penalty to Fortitude defense.
                    \item \plus1 bonus to other physical attacks and defenses.
                    \item \plus4 bonus to Stealth checks.
                    \item Weapons decrease damage die size (see \tref{Weapon Damage and Size}).
                    \item If the target's new size is Medium or larger, it takes a \minus10 ft.\ penalty to movement speed. Otherwise, it takes a \minus5 ft.\ penalty to movement speed.
                \end{itemize}
                \par Equipment that leaves the target's possession returns to its original size.
            \end{spellattack}
            \spelldur \durshort \dismissable
        \end{spelleffects}
    \end{spellcontent}
    \begin{spellfooter}
        \spellinfo{Transmutation [Shaping, Sizing]}{Arcane, Nature}
        \spellnotes A Small humanoid creature whose size decreases to Tiny has a space of 2-1/2 feet and a natural reach of 0 feet (meaning that it must enter an opponent's square to attack).

        \sizingspellnotes
        \miscastrandom
    \end{spellfooter}
    \begin{spellaugments}
        \spellaugment{1}{Giant}{The spell can affect a target one size category larger. This augment can be used multiple times.}
        \spellaugment{3}{Mass}{The spell can affect up to five targets.}
    \end{spellaugments}
\end{spellsection}

\begin{spellsection}{Shrink Item}[3]
    \begin{spellheader}
    \end{spellheader}
    \begin{spellcontent}
        \begin{spelltargetinginfo}
            \spellquicktargeting{One nonmagical object (Medium or smaller)}{\rngclose}
        \end{spelltargetinginfo}
        \begin{spelleffects}
            \spellspecial As you cast this spell, choose a command word.
            \begin{spellattack}{Spellpower vs. Mental}
                \spellsuccess The target shrinks to 1/16 of its normal size in each dimension (to about 1/4,000 the original volume and mass). This change effectively reduces its size by four size categories. If the target is physically unable to shrink, such as a ring on a finger, it shrinks as much as it can without causing harm to itself or the physical impediment.

                As a standard action, any creature can speak the command word to return the target to its original size. It must be resting on a stable surface. If the command word is spoken while the target is not stable, such as while it is in the air, it returns to its original size as soon as it finds a resting point. Restoring the target to its normal size ends the spell.
            \end{spellattack}
            \spelldur \durext or until discharged
        \end{spelleffects}
    \end{spellcontent}
    \begin{spellfooter}
        \spellinfo{Transmutation [Shaping, Sizing]}{Arcane}
        \spellnotes  If you recast this spell each day on an object, you can keep it at its small size indefinitely.
        \miscastrandom
    \end{spellfooter}
    \begin{spellaugments}
        \spellaugment{1}{Giant}{The spell can affect a target one size category larger. This augment can be used multiple times.}
        \spellaugment{2}{Repeatable}{After the command word is spoken, the ritual is not discharged. Instead, it is suppressed for 5 minutes, after which time the object shrinks again.}
    \end{spellaugments}
\end{spellsection}

\begin{spellsection}{Silence}[2]
    \begin{spellheader}
    \end{spellheader}
    \begin{spellcontent}
        \begin{spelltargetinginfo}
            \spellquicktargeting{One creature or object}{\rngmed}
        \end{spelltargetinginfo}
        \begin{spelleffects}
            \begin{spellattack}{Spellpower vs. Mental}
                \spellsuccess The target becomes unable to make noise.  Extraordinarily loud noises, such as the yell of a giant, are merely muffled by the spell rather than completely silenced. The DR to hear such sounds produced by the target is increased by 40. Sonic attacks function normally.

                Spellcasters can still cast spells with verbal components while silenced, but suffer a 20\% chance of spell failure.

                \spellcritical As above, except that the silence is absolute. Even extraordinarily loud noises are utterly silenced, and spellcasters are unable to cast spells with verbal components.
            \end{spellattack}
            \spelldur \durshort \dismissable
        \end{spelleffects}
    \end{spellcontent}
    \begin{spellfooter}
        \spellinfo{Illusion [Glamer]}{Divine}
        \miscastrandom
    \end{spellfooter}
    \begin{spellaugments}
        \spellaugment{1}{Subtle}{The target can still hear itself normally, potentially causing it to be unaware of the effect of the spell. In addition, the spell becomes a Subtle effect.}
        \spellaugment{3}{Mass}{The spell can affect up to five targets.}
    \end{spellaugments}
\end{spellsection}

\begin{spellsection}{Sleep}[1]
    \begin{spellheader}
    \end{spellheader}
    \begin{spellcontent}
        \begin{spelltargetinginfo}
            \spellquicktargeting{One creature}{\rngmed}
            \spellcmp{Somatic only}
        \end{spelltargetinginfo}
        \begin{spelleffects}
            \begin{spellattack}{Spellpower vs. Mental}
                \spellsuccess The target is \fatigued and attempts to go to sleep as soon as possible, though it will not stop fighting to do so. Awakening a creature put to sleep by this spell is difficult, and requires a standard action.
                \spellcritical As above, except that the target is \exhausted instead of \fatigued. In addition, if the creature goes to sleep, it cannot be awoken by nonmagical means during the spell's duration.
            \end{spellattack}
            \spelldur \durmed
        \end{spelleffects}
    \end{spellcontent}
    \begin{spellfooter}
        \spellinfo{Enchantment [Delusion, Mind, Sleep]}{Arcane}
        \miscastrandom
    \end{spellfooter}
    \begin{spellaugments}
        \spellaugment{3}{Mass}{The spell can affect up to five targets.}
    \end{spellaugments}
\end{spellsection}

\begin{spellsection}{Slow}[3]
    \begin{spellheader}
        \spelldesc{You decelerate your enemy's motions, causing it to move and act more slowly than normal.}
    \end{spellheader}
    \begin{spellcontent}
        \begin{spelltargetinginfo}
            \spellquicktargeting{One creature}{\rngclose}
        \end{spelltargetinginfo}
        \begin{spelleffects}
            \spelleffect The target is \slowed.
            \spelldur \durbrief
        \end{spelleffects}
    \end{spellcontent}
    \begin{spellfooter}
        \spellinfo{Transmutation [Temporal]}{Arcane}
        \miscastrandom
    \end{spellfooter}
    \begin{spellaugments}
        \spellaugment{3}{Mass}{The spell can affect up to five targets.}
    \end{spellaugments}
\end{spellsection}

\begin{spellsection}{Sound Burst}[2]
    \begin{spellheader}
        \spelldesc{You create a cacophony of sound.}
    \end{spellheader}
    \begin{spellcontent}
        \begin{spelltargetinginfo}
            \spelltwocol{\spellburst{\areasmall radius}}{\spellrng{\rngclose}}
            \spelltgts{Everything in the area}
        \end{spelltargetinginfo}
        \begin{spelleffects}
            \begin{spellattack}{Spellpower vs. Fortitude}
                \spellsuccess \spelldamage{sonic}[d8].
                \spellcritical Double damage.
                \spellfailure Half damage.
            \end{spellattack}
        \end{spelleffects}
    \end{spellcontent}
    \begin{spellfooter}
        \spellinfo{Evocation [Sonic]}{Arcane, Destruction}
        \miscastyou
    \end{spellfooter}
    \begin{spellaugments}
        \spellaugment{2}{Widened}{The spell's area becomes a \areamed radius.}
        \spellaugment{3}{Deafening}{Affected targets are also \deafened for 2 rounds.}
        \spellaugment{4}{Empowered}{The damage increases to \spelldamage{sonic}[d10].}
    \end{spellaugments}
\end{spellsection}

\begin{spellsection}{Spellsight}[6]
    \begin{spellheader}
        \spelldesc{You gain the ability to see magic perfectly.}
    \end{spellheader}
    \begin{spellcontent}
        \begin{spelltargetinginfo}
            \spelltgt{You}
        \end{spelltargetinginfo}
        \begin{spelleffects}
            \spelleffect You gain the ability to see and understand magic within 300 feet of you. This has three effects.
            \begin{itemize}
                \item You can automatically identify any active spells and spells cast as if you succeeded on a Spellcraft check. (See \pcref{Spellcraft}, for details.)
                \item You can ``see'' any creatures or objects affected by spells perfectly, regardless of concealment or invisibility.
                \item If you concentrate as a standard action, you can identify all properties of a magic item you touch, as the \ritual{identify} ritual.
            \end{itemize}
            \spelldur \durlong
        \end{spelleffects}
    \end{spellcontent}
    \begin{spellfooter}
        \spellinfo{schools}{lists}
        \miscastexplode
    \end{spellfooter}
    \begin{spellaugments}
        \spellaugment{1}{Persistent}{The spell's duration becomes \durext.}
        \spellaugment{2}{Penetrating}{The sight is not blocked by physical obstacles other than lead.}
        \spellaugment{3}{Distant}{The sight extends to a range of 1 mile.}
    \end{spellaugments}
\end{spellsection}

\begin{spellsection}{Spider Climb}[2]
    \begin{spellheader}
        \spelldesc{You grant your ally the ability to climb on walls and ceilings as well as a spider does.}
    \end{spellheader}
    \begin{spellcontent}
        \begin{spelltargetinginfo}
            \spellquicktargeting{One creature}{\rngclose}
        \end{spelltargetinginfo}
        \begin{spelleffects}
            \spelleffect The target gains a \glossterm{climb speed} of 20 feet. It must use at least one hand to climb in this manner.
            \spelldur \durmed
        \end{spelleffects}
    \end{spellcontent}
    \begin{spellfooter}
        \spellinfo{Transmutation [Imbuement]}{Arcane, Nature}
        \spellnotes See \pcref{Climb}, for details about how to climb.
        \miscastexplode
    \end{spellfooter}
    \begin{spellaugments}
        \spellaugment{1}{Accelerated}{The climb speed increases to 40 feet.}
        \spellaugment{3}{Mass}{The spell can affect up to five targets.}
    \end{spellaugments}
\end{spellsection}

\begin{spellsection}{Storm of Vengeance}[9]
    \begin{spellheader}
    \end{spellheader}
    \begin{spellcontent}
        \begin{spelltargetinginfo}
            \spelltwocol{\spellzone{500 ft.\ radius cylinder}}{\spellrng{\rnglong}}
        \end{spelltargetinginfo}
        \begin{spelleffects}
            \spelleffect An enormous storm cloud occupies the top 200 feet of the area, as \spell{fog cloud}. Within the area, lightning strikes and thunder rolls. Sunlight is blocked by the dark cloud. This may cause the area to have shadowy illumination, granting everything in it \concealment.

            At the end of every round, the storm has an additional effect, as shown on \tref{Storm of Vengeance Effects}. Damaging effects deal \spelldamage{}[d8].

            \spelldur Focus (maximum 10 rounds)
        \end{spelleffects}
    \end{spellcontent}
    \begin{spellfooter}
        \spellinfo{Conjuration/Evocation [Acid, Creation, Electricity]}*{Air, Divine, Nature, Water}
        \spellnotes When the storm has multiple effects in the same round, roll a single attack and compare the result to all relevant defenses.

        \physicalspellnotes
        \miscastyou
    \end{spellfooter}
\end{spellsection}
\begin{dtable*}
    \lcaption{Storm of Vengeance Effects}
    \begin{dtabularx}{\textwidth}{l l l >{\lcol}X l}
        \tb{Rounds} & \tb{Effect} & \tb{Defense} & \tb{Success} & \tb{Failure} \\
        \hline
        Odd (1, 3, 5, 7, 9)   & Lightning  & Reflex    & Electricity damage (enemies only) & Half damage \\
        Even (2, 4, 6, 8, 10) & Thunder    & None & Deafened for 5 rounds & \tdash \\
        2, 6, 10              & Hail       & Fortitude    & Bludgeoning damage & Half damage \\
        4, 8                  & Acid rain  & None      & Acid damage & \tdash \\
    \end{dtabularx}
\end{dtable*}

\begin{spellsection}{Stormlord}[5]
    \begin{spellheader}
        \spelldesc{You surround yourself in a whirlwind which deflects ranged attacks and batters your foes.}
    \end{spellheader}
    \begin{spellcontent}
        \begin{spelltargetinginfo}
            \spelltgt{You}
        \end{spelltargetinginfo}
        \begin{spelleffects}
            \spelleffect Physical ranged attacks against you have a 50\% miss chance. Other attacks that simply work at a distance are not affected.

            In addition, whenever a creature within \rnglong range of you makes a physical attack against you, the attacking creature takes \spelldamage{bludgeoning}[d6].
            A creature can only be dealt damage by this spell once per round.
            \spelldur \durshort
        \end{spelleffects}
    \end{spellcontent}
    \begin{spellfooter}
        \spellinfo{Evocation [Air, Electricity, Shielding]}{Air, Nature}
        \miscastexplode
    \end{spellfooter}
    \begin{spellaugments}
        \spellaugment{4}{Empowered}{The damage against creatures that attack you increases to \spelldamage{bludgeoning}[d8].}
    \end{spellaugments}
\end{spellsection}

\begin{spellsection}{Strip the Flesh}[7]
    \begin{spellheader}
        \spelldesc{You rend parts of your foe's skin off its body, inflicting grievous wounds and leaving it vulnerable.}
    \end{spellheader}
    \begin{spellcontent}
        \begin{spelltargetinginfo}
            \spellquicktargeting{One creature}{\rngmed}
        \end{spelltargetinginfo}
        \begin{spelleffects}
            \begin{spellattack}{Spellpower vs. Fortitude}
                \spellsuccess \spelldamage{slashing}. In addition, all damage the target takes is doubled for 2 rounds.
                This does not apply to the initial damage dealt by this spell.
                \spellcritical As above, except that the doubling of damage lasts for thirty days, and applies to the initial damage dealt by this spell.
                Effects which accelerate natural healing, such as the Heal skill, also reduce the duration of this effect.
                \spellfailure Half damage, and no additional effects.
            \end{spellattack}
        \end{spelleffects}
    \end{spellcontent}
    \begin{spellfooter}
        \spellinfo{Vivimancy [Flesh]}{Arcane, Death, Evil}
        \spellnotes \physicalspellnotes
        \miscastrandom
    \end{spellfooter}
\end{spellsection}

\begin{spellsection}{Suggestion}[4]
    \begin{spellheader}
    \end{spellheader}
    \begin{spellcontent}
        \begin{spelltargetinginfo}
            \spellquicktargeting{One creature}{\rngclose}
            \spellcmp{Verbal only. The only verbal component is the stated suggestion.}
        \end{spelltargetinginfo}
        \begin{spelleffects}
            \begin{spellattack}{Spellpower vs. Mental}
                \spellspecial You suggest a course of action that the target could take. The suggestion must not be longer than a couple of sentences. It must be worded in a way that makes the activity sound reasonable. Asking the creature to do some obviously harmful act makes the spell fail automatically.

                You take a \minus5 penalty to accuracy if the target thinks it is threatened. A very reasonable suggestion can grant a \plus2 or greater bonus to accuracy.
                \spellsuccess For 5 rounds, the target is compelled to obey your suggestion. If the suggested activity is completed during that time, the spell's effect ends.
                \spellcritical As above, except that the target will obey the suggestion indefinitely, until it completes its task.
            \end{spellattack}
        \end{spelleffects}
    \end{spellcontent}
    \begin{spellfooter}
        \spellinfo{Enchantment [Auditory, Delusion, Mind, Speech, Subtle]}*{Arcane}
        \norepeatspellnotes
        \miscastrandom
    \end{spellfooter}
    \begin{spellaugments}
        \spellaugment{3}{Mass}{The spell can affect up to five targets. All targets must receive the same suggestion.}
    \end{spellaugments}
\end{spellsection}

\begin{spellsection}{Summon Monster}[1]
    \begin{spellheader}
    \end{spellheader}
    \begin{spellcontent}
        \begin{spelltargetinginfo}
            \spellquicktargeting{Location}{\rngclose}
        \end{spelltargetinginfo}
        \begin{spelleffects}
            \spelleffect This spell summons a facsimile of an extraplanar creature.
            The creature appears at the target location and acts during the next round.

            At the start of each round, you must spend a swift action to control the summoned creature.
            If you do, you control the creature's actions that round.
            You can mentally command it to attack your enemies, follow you, or stay in place.
            Alternately, if you can communicate with the creature using other means, you can give it more complex commands.
            If you do not control the creature, it acts according to its nature.
            Most creatures will flee combat or attack indiscriminately.

            \spellspecial When you learn this spell, you choose two creatures from the 1st-level list on \tref{Summon Monster List}. You can only summon those creatures with this spell.
            \par A summoned monster cannot summon or otherwise conjure another creature, nor can it use any teleportation or planar travel abilities. Creatures cannot be summoned into an environment that cannot support them.
            \par When you use a summoning spell to summon an air, chaotic, earth, evil, fire, lawful, or water creature, it is a spell of that type.
            \spelldur \durshort \dismissable
        \end{spelleffects}
    \end{spellcontent}
    \begin{spellfooter}
        \spellinfo{Conjuration [Summoning, see text]}{Arcane, Divine}
        \spellnotes You can learn this spell multiple times. Each time, you learn how to summon two additional creatures. The creatures must be chosen from a list with a maximum level equal to the highest level of spells you know how to cast.
        \miscastexplode
    \end{spellfooter}
    \begin{spellaugments}
        \spellaugment{1}{Empowered}{You summon creatures from a higher level list. You can use this augment multiple times.}
        \spellaugment{1}{Multiple}{You summon 1d3 creatures of the same kind, rather than one creature.}
    \end{spellaugments}
\end{spellsection}

\begin{dtable!*}
    \lcaption{Summon Monster List}
    \begin{dtabularx}{\textwidth}{>{\lcol}X c >{\lcol}X c >{\lcol}X c}
        \tb{1st Level} &  & \tb{4th Level} &  & Fiendish monstrous spider, Huge & CE \\
        \hline
        Celestial dog & LG & Archon, lantern & LG & Fiendish snake, giant constrictor & CE \\
        Celestial owl & LG & Celestial giant owl & LG &  &  \\
        Celestial giant fire beetle & NG & Celestial giant eagle & CG & \tb{7th Level} &  \\
        Celestial porpoise\fn{1} & NG & Celestial lion & CG & Celestial elephant & LG \\
        Celestial badger & CG & Mephit (any)\fn{2} & N & Avoral (guardinal) & NG \\
        Celestial monkey & CG & Fiendish dire wolf & LE & Celestial baleen whale\fn{1} & NG \\
        Fiendish dire rat & LE & Fiendish giant wasp & LE & Djinni (genie) & CG \\
        Fiendish raven & LE & Fiendish giant praying mantis & NE & Elemental, Huge (any)\fn{2} & N \\
        Fiendish monstrous centipede, Medium & NE & Fiendish shark, Large\fn{1} & NE & Invisible stalker & N \\
        Fiendish monstrous scorpion, Small & NE & Yeth hound & NE & Devil, bone & LE \\
        Fiendish hawk & CE & Fiendish monstrous spider, Large & CE & Fiendish megaraptor & LE \\
        Fiendish monstrous spider, Small & CE & Fiendish snake, Huge viper & CE & Fiendish monstrous scorpion, Huge & \\ NE
        Fiendish octopus\fn{1} & CE & Howler & CE & Babau (demon) & CE \\
        Fiendish snake, Small viper & CE &  &  & Fiendish giant octopus\fn{1} & CE \\
        &  & \tb{5th Level} &  & Fiendish girallon & CE \\
        \tb{2nd Level} &  & Archon, hound & K &  &  \\
        Celestial giant bee & LG & Celestial brown bear & LG &  &  \\
        Celestial giant bombardier beetle & NG & Celestial giant stag beetle & LG & \tb{8th Level} &  \\
        Celestial riding dog & NG & Celestial sea cat\fn{1} & NG & Celestial dire bear & LG \\
        Celestial eagle & CG & Celestial griffon & NG & Celestial cachalot whale\fn{1} & NG \\
        Lemure (devil) & LE & Elemental, Medium (any)\fn{2} & CG & Celestial triceratops & NG \\
        Fiendish squid\fn{1} & LE & Achaierai & N & Lillend & CG \\
        Fiendish wolf & LE & Devil, bearded & LE & Elemental, greater (any)\fn{2} & N \\
        Fiendish monstrous centipede, Large & NE & Fiendish deinonychus & LE & Fiendish giant squid\fn{1} & LE \\
        Fiendish monstrous scorpion, Medium & NE & Fiendish dire ape & LE & Hellcat & LE \\
        Fiendish shark, Medium\fn{1} & NE & Fiendish dire boar & LE & Fiendish monstrous centipede, Colossal & NE \\
        Fiendish monstrous spider, Medium & CE & Fiendish shark, Huge & NE & Fiendish dire tiger & CE \\
        Fiendish snake, Medium viper & CE & Fiendish monstrous scorpion, Large & NE & Fiendish monstrous spider, Gargantuan & CE \\
        &  & Shadow mastiff & NE & Fiendish tyrannosaurus & CE \\
        \tb{3rd Level} &  & Fiendish dire wolverine & NE & Vrock (demon) & CE \\
        Celestial black bear & LG & Fiendish giant crocodile & CE &  &  \\
        Celestial bison & NG & Fiendish tiger & CE &  &  \\
        Celestial dire badger & CG &  &  & \tb{9th Level} &  \\
        Celestial hippogriff & CG & \tb{6th Level} &  & Couatl & LG \\
        Elemental, Small (any)\fn{2} & N & Celestial polar bear & LG & Leonal (guardinal) & NG \\
        Fiendish ape & LE & Celestial orca whale\fn{1} & NG & Celestial roc & CG \\
        Fiendish dire weasel & LE & Bralani (eladrin) & CG & Elemental, elder (any)\fn{2} & N \\
        Hell hound & LE & Celestial dire lion & CG & Devil, barbed & LE \\
        Fiendish snake, constrictor  & LE & Elemental, Large (any)\fn{2} & N & Fiendish dire shark\fn{1} & NE \\
        Fiendish boar & NE & Janni (genie) & N & Fiendish monstrous scorpion, Gargantuan & NE \\
        Fiendish dire bat & NE & Chaos beast & CN & Night hag & NE \\
        Fiendish monstrous centipede, Huge & NE & Devil, chain & LE & Bebilith (demon) & CE \\
        Fiendish crocodile & CE & Xill & LE & Fiendish monstrous spider, Colossal & CE \\
        Dretch (demon) & CE & Fiendish monstrous centipede, Gargantuan & NE & Hezrou (demon) & CE \\
        Fiendish snake, Large viper & CE & Fiendish rhinoceros & NE & & \\
        Fiendish wolverine & CE & Fiendish elasmosaurus\fn{1} & CE & &
    \end{dtabularx}
    1 May be summoned only into an aquatic or watery environment. \\
    2 Each variety must be learned individually.
\end{dtable!*}

\begin{spellsection}{Summon Nature's Ally}[1]
    \begin{spellheader}
    \end{spellheader}
    \begin{spellcontent}
        \begin{spelltargetinginfo}
            \spellrng{\rngclose}
        \end{spelltargetinginfo}
        \begin{spelleffects}
            \spelleffect This spell summons a natural creature. It appears where you designate and acts on your next turn. You must spend a swift action each round to control the creature summoned by this spell. If you do, it attacks your opponents to the best of its ability. You can direct the creature not to attack, to attack particular enemies, or to perform other actions if you can communicate with it. If you do not actively control the creature summoned by this spell, it acts according to its nature.
            \par When you learn this spell, you choose two creatures from the 1st-level list on \tref{Summon Nature's Ally List}. You can summon those creatures with this or any other \spell{summon nature's ally} spell.
            \par A summoned monster cannot summon or otherwise conjure another creature, nor can it use any teleportation or planar travel abilities. Creatures cannot be summoned into an environment that cannot support them.
            \par All the creatures on the table are neutral unless otherwise noted.
            \spelldur \durshort \dismissable
        \end{spelleffects}
    \end{spellcontent}
    \begin{spellfooter}
        \spellinfo{Conjuration [Summoning]}{Nature}
        \spellnotes You can learn this spell multiple times. Each time, you learn how to summon two additional creatures. The creatures must be chosen from a list with a maximum level equal to the highest level of spells you know how to cast.
        \miscastexplode
    \end{spellfooter}
    \begin{spellaugments}
        \spellaugment{1}{Empowered}{You summon creatures from a higher level list. You can use this augment multiple times.}
        \spellaugment{1}{Multiple}{You summon 1d3 creatures of the same kind, rather than one creature.}
    \end{spellaugments}
\end{spellsection}

\begin{dtable*}
    \lcaption{Summon Nature's Ally List}
    \begin{dtabularx}{\textwidth}{>{\lcol}X >{\lcol}X >{\lcol}X >{\lcol}X}
        \tb{1st Level} & Eagle, giant [NG] & \tb{5th Level} & \tb{7th Level} \\
        \hline
        Dire rat & Lion & Arrowhawk, adult & Arrowhawk, elder \\
        Eagle (animal) & Owl, giant [NG] & Bear, polar (animal) & Dire tiger \\
        Monkey (animal) & Satyr [CN;\ without pipes] & Dire lion & Elemental, greater (any)\fn{2} \\
        Octopus\fn{1} (animal) & Shark, Large\fn{1} (animal) & Elasmosaurus\fn{1} (dinosaur) & Djinni (genie) [NG] \\
        Owl (animal) & Snake, constrictor (animal) & Elemental, Large (any)\fn{2} & Invisible stalker \\
        Porpoise\fn{1} (animal) & Snake, Large viper (animal) & Griffon & Pixie\fn{3} (sprite) [NG;\ with sleep arrows] \\
        Snake, Small viper (animal) & Thoqqua & Janni (genie) & Squid, giant\fn{1} (animal) \\
        Wolf (animal) &  & Rhinoceros (animal) & Triceratops (dinosaur) \\
        & \tb{4th Level} & Satyr [CN;\ with pipes] & Tyrannosaurus (dinosaur) \\
        \tb{2nd Level} & Arrowhawk, juvenile & Snake, giant constrictor (animal) & Whale, cachalot\fn{1} (animal) \\
        Bear, black (animal) & Bear, brown (animal) & Nixie (sprite) & Xorn, elder \\
        Crocodile (animal) & Crocodile, giant (animal) & Tojanida, adult\fn{1} &  \\
        Dire badger & Deinonychus (dinosaur) & Whale, orca\fn{1} (animal) & \tb{8th Level} \\
        Dire bat & Dire ape &  & Dire shark\fn{1} \\
        Elemental, Small (any)\fn{2} & Dire boar & \tb{6th Level} & Roc \\
        Hippogriff & Dire wolverine & Dire bear & Salamander, noble [NE] \\
        Shark, Medium\fn{1} (animal) & Elemental, Medium (any)\fn{2} & Elemental, Huge (any)\fn{2} & Tojanida, elder \\
        Snake, Medium viper (animal) & Salamander, flamebrother [NE] & Elephant (animal) &  \\
        Squid\fn{1} (animal) & Sea cat\fn{1} & Girallon & \tb{9th Level} \\
        Wolverine (animal) & Shark, Huge\fn{1} (animal) & Megaraptor (dinosaur) & Elemental, elder \\
        & Snake, Huge viper (animalo) & Octopus, giant\fn{1} (animal) & Grig [NG;\ with fiddle] (sprite) \\
        \tb{3rd Level} & Tiger (animal) & Pixie\fn{3} (sprite) [NG;\ no special arrows] & Pixie\fn{4} (sprite) [NG;\ with sleep and memory loss arrows] \\
        Ape (animal) & Tojanida, juvenile\fn{1} & Salamander, average [NE] & Unicorn, celestial charger \\
        Dire weasel & Unicorn [CG] & Whale, baleen\fn{1} &  \\
        Dire wolf & Xorn, minor & Xorn, average &
    \end{dtabularx}
    1 May be summoned only into an aquatic or watery environment. \\
    2 Each variety must be learned individually. \\
    3 Can't cast irresistible dance \\
    4 Can cast irresistible dance \\
\end{dtable*}

\begin{spellsection}{Sunbeam}[4]
    \begin{spellheader}
        \spelldesc{You evoke a dazzling beam of intense light, blinding your foes with the power of the sun itself.}
    \end{spellheader}
    \begin{spellcontent}
        \begin{spelltargetinginfo}
            \spellburst{\arealarge line, 10 ft.\ wide}
            \spelltgts{Everything in the area}
        \end{spelltargetinginfo}
        \begin{spelleffects}
            \begin{spellattack}{Spellpower vs. Reflex}
                \spellspecial You gain a \plus5 bonus to accuracy against creatures vulnerable to sunlight.
                \spellsuccess \spelldamage{solar}[d8]. In addition, the target is \partiallyblinded for 2 rounds.
                \spellcritical Double damage, and the target is \blinded instead of partially blinded.
                \spellfailure Half damage, and no additional effects.
            \end{spellattack}
        \end{spelleffects}
    \end{spellcontent}
    \begin{spellfooter}
        \spellinfo{Illusion [Figment, Light]}{Arcane, Nature, Wild}
        \spellnotes This light is considered natural sunlight for the purpose of effects which depend on sunlight.
        \miscastexplode
    \end{spellfooter}
    \begin{spellaugments}
        \spellaugment{2}{Widened}{The spell's area becomes a \areahuge, 20 ft.\ wide line.}
        \spellaugment{4}{Empowered}{The damage increases to \spelldamage{solar}[d10].}
    \end{spellaugments}
\end{spellsection}

\begin{spellsection}{Sunburst}[7]
    \begin{spellheader}
        \spelldesc{You cause a globe of searing radiance to explode silently from a point you select.}
    \end{spellheader}
    \begin{spellcontent}
        \begin{spelltargetinginfo}
            \spellburst{\arealarge radius}
            \spelltgts{All enemies in the area}
        \end{spelltargetinginfo}
        \begin{spelleffects}
            \begin{spellattack}{Spellpower vs. Reflex}
                \spellspecial You gain a \plus5 bonus to accuracy against creatures vulnerable to sunlight.
                \spellsuccess \spelldamage{solar}[d8]. In addition, the target is \partiallyblinded for 2 rounds.
                \spellcritical Double damage, and the target is \blinded instead of partially blinded.
                \spellfailure Half damage, and no additional effects.
            \end{spellattack}
        \end{spelleffects}
    \end{spellcontent}
    \begin{spellfooter}
        \spellinfo{Illusion [Figment, Light]}{Arcane, Nature, Wild}
        \spellnotes This light is considered natural sunlight for the purpose of effects which depend on sunlight.
        \miscastyou
    \end{spellfooter}
    \begin{spellaugments}
        \spellaugment{2}{Widened}{The spell's area becomes a \areahuge radius.}
    \end{spellaugments}
\end{spellsection}

\begin{spellsection}{Swarm of Bats}[2]
    \begin{spellheader}
        \spelldesc{You summon a swarm of bats that attack the eyes of your foes.}
    \end{spellheader}
    \begin{spellcontent}
        \begin{spelltargetinginfo}
            \spellburst{\areamed cone}
            \spelltgts{All enemies in the area}
        \end{spelltargetinginfo}
        \begin{spelleffects}
            \spelleffect The target is \impaired with vision-related attacks and checks for 2 rounds.
        \end{spelleffects}
    \end{spellcontent}
    \begin{spellfooter}
        \spellinfo{Conjuration [Summoning]}{Arcane, Chaos, Nature}
        \spellnotes The bats disappear after 2 rounds.
        \miscastyou
    \end{spellfooter}
    \begin{spellaugments}
        \spellaugment{3}{Widened}{The spell's area becomes a \arealarge cone.}
    \end{spellaugments}
\end{spellsection}

\pdfbookmark[2]{T}{SpellDescriptionsT}

\begin{spellsection}{Telekinetic Blast}[5]
    \begin{spellheader}
        \spelldesc{You blast your foes away from you.}
    \end{spellheader}
    \begin{spellcontent}
        \begin{spelltargetinginfo}
            \spellburst{\areamed radius}
            \spelltgts{All enemies in the area}
        \end{spelltargetinginfo}
        \begin{spelleffects}
            \begin{spellattack}{Spellpower vs. Mental}
                \spellsuccess \spelldamage{bludgeoning}[d8].
                In addition, the target is forcibly moved away from you to the edge of the area.
                If it encounters an occupied space or other obstacle, it stops at the obstacle.
                \spellcritical As above, but double damage.
                \spellfailure Half damage.
            \end{spellattack}
        \end{spelleffects}
    \end{spellcontent}
    \begin{spellfooter}
        \spellinfo{Evocation [Telekinesis]}{Arcane}
        \miscastexplode
    \end{spellfooter}
    \begin{spellaugments}
        \spellaugment{2}{Widened}{The spell's area becomes a \arealarge radius.}
        \spellaugment{4}{Empowered}{The damage increases to \spelldamage{bludgeoning}[d10].}
    \end{spellaugments}
\end{spellsection}

\begin{spellsection}{Telekinetic Shove}[2]
    \begin{spellheader}
        \spelldesc{You push and pull your foes around the battlefield with your mind.}
    \end{spellheader}
    \begin{spellcontent}
        \begin{spelltargetinginfo}
            \spelltgt{You}
        \end{spelltargetinginfo}
        \begin{spelleffects}
            \spelleffect When you cast this spell, you can make a shove attack against a creature within \rngclose range.
            This functions like a normal shove attack, except that your accuracy is equal to your spellpower, and you do not need to move with the target to move it.
            However, you cannot move it beyond \rngclose range.

            At the beginning of each round, you may spend a swift action to focus your mind.
            If you do, you can use this spell to shove a target during the action phase.
            You can shove the same target, or a different target, as you choose.
            \spelldur \durshort
        \end{spelleffects}
    \end{spellcontent}
    \begin{spellfooter}
        \spellinfo{Evocation [Telekinesis]}{Arcane}
        \spellnotes You can shove yourself with this spell.
        \miscastexplode
    \end{spellfooter}
\end{spellsection}

\begin{spellsection}{Telepathy}[5]
    \begin{spellheader}
    \end{spellheader}
    \begin{spellcontent}
        \begin{spelltargetinginfo}
            \spelltgt{You}
        \end{spelltargetinginfo}
        \begin{spelleffects}
            \spelleffect You gain telepathy out to a range of 100 feet. This allows you to send mental messages to any creature within range that has a language. Non-telepathic creatures can reply mentally to your messages, but they cannot initiate a telepathic conversation with you.

            You can address multiple creatures at once with telepathy, but maintaining separate mental conversations is just as difficult as simultaneously speaking and listening to multiple creatures at the same time.
            \spelldur \durlong
        \end{spelleffects}
    \end{spellcontent}
    \begin{spellfooter}
        \spellinfo{Divination}{Divination}
        \spellnotes You can use telepathy to communicate with creatures you cannot see. However, you must be aware of the existence and approximate location of a creature to communicate with it. Telepathy is blocked by any physical obstacle that blocks line of effect.
        \miscastexplode
    \end{spellfooter}
    \begin{spellaugments}
        \spellaugment{1}{Distant}{The telepathy has a range of 1,000 feet.}
        \spellaugment{2}{Penetrating}{The telepathy is not blocked by physical obstacles other than lead.}
    \end{spellaugments}
\end{spellsection}

\begin{spellsection}{Temporal Stasis}[6]
    \begin{spellheader}
    \end{spellheader}
    \begin{spellcontent}
        \begin{spelltargetinginfo}
            \spellquicktargeting{One creature}{\rngmed}
        \end{spelltargetinginfo}
        \begin{spelleffects}
            \begin{spellattack}{Spellpower vs. Mental}
                \spellsuccess The target is placed in a state of suspended animation for 5 rounds. Time ceases to flow for it, and it cannot be altered or moved by any effect.
                \spellcritical As above, except that the effect is permanent.
                \spellfailure The targes moves at one-quarter speed for 5 rounds.
            \end{spellattack}
        \end{spelleffects}
    \end{spellcontent}
    \begin{spellfooter}
        \spellinfo{Transmutation [Temporal]}{Arcane}
        \miscastrandom
    \end{spellfooter}
    \begin{spellaugments}
        \spellaugment{3}{Mass}{The spell can affect up to five targets.}
    \end{spellaugments}
\end{spellsection}

\begin{spellsection}{Third Eye}[8]
    \begin{spellheader}
        \spelldesc{You gain a mystic third eye, allowing you to see beyond the limitations of your mortal senses.}
    \end{spellheader}
    \begin{spellcontent}
        \begin{spelltargetinginfo}
            \spelltgt{You}
        \end{spelltargetinginfo}
        \begin{spelleffects}
            \spelleffect You gain \glossterm{blindsight} out to a 100 foot range, allowing you to see perfectly without any light, regardless of concealment or invisibility. In addition, you can foresee events an instant before they occur, preventing you from being \unaware.
            \spelldur \durlong
        \end{spelleffects}
    \end{spellcontent}
    \begin{spellfooter}
        \spellinfo{Divination/Transmutation [Imbuement, Knowledge]}*{Arcane, Knowledge}
        \miscastexplode
    \end{spellfooter}
    \begin{spellaugments}
        \spellaugment{1}{Persistent}{The spell's duration becomes \durext.}
    \end{spellaugments}
\end{spellsection}

\begin{spellsection}{Time Stop}[9]
    \begin{spellheader}
        \spelldesc{This spell seems to make time cease to flow for everyone but you. In fact, you step into an alternate timestream, causing you to speed up so greatly that all other creatures seem frozen, though they are actually still moving at their normal speeds.}
    \end{spellheader}
    \begin{spellcontent}
        \begin{spelleffects}
            % exception for Temporal effects?
            \spelleffect You can take 1d3\plus1 rounds of actions immediately. During this time, all other creatures and objects are fixed in time, and cannot be moved or altered by any effect. You can still affect yourself and create areas or new effects, such with \spell{fog cloud} or \spell{summon monster}.

            You are still vulnerable to danger, such as from heat or dangerous gases. However, you cannot be detected by any means while you travel.
        \end{spelleffects}
    \end{spellcontent}
    \begin{spellfooter}
        \spellinfo{Transmutation [Temporal]}{Arcane}
        \spellnotes Spells active on you have their normal effects, including decreasing their remaining duration as appropriate, but spells active on other creatures have no effects and do not decrease in remaining duration.

        You cannot enter an area protected by an \spell{antimagic field} while under the effect of this spell.

        Most spellcasters use the additional time to improve their defenses or flee from combat.
        \miscastexplode
    \end{spellfooter}
\end{spellsection}

\begin{spellsection}{Transmute Any Object}[9]
    \begin{spellheader}
    \end{spellheader}
    \begin{spellcontent}
        \begin{spelltargetinginfo}
            \spellrng{\rngmed}
        \end{spelltargetinginfo}
        \begin{spelleffects}
            \spellspecial This spell can be used to duplicate the effects of \spell{fabricate}, \spell{passwall}, \spell{shape metal}, \spell{shape stone}, \spell{shape wood}, or \spell{transmute flesh and stone}. The object or creature to be transformed must meet any requirements of the spell being duplicated, other than range.
        \end{spelleffects}
    \end{spellcontent}
    \begin{spellfooter}
        \spellinfo{Transmutation [Shaping]}{Arcane}
        \miscastexplode
    \end{spellfooter}
\end{spellsection}

\begin{spellsection}{Transmute Flesh and Stone}[5]
    \begin{spellheader}
    \end{spellheader}
    \begin{spellcontent}
        \begin{spelltargetinginfo}
            \spellspecial This spell has two versions: transmuting flesh into stone, and transmuting stone into flesh. Its effects depend on which version is chosen.
        \end{spelltargetinginfo}
    \end{spellcontent}
    \begin{spellsubcontent}
        \begin{spelltargetinginfo}
            \spellquicktargeting{One creature (Large or smaller)}{\rngclose}
        \end{spelltargetinginfo}
        \begin{spelleffects}
            \spellspecial If the target is not made of flesh (such as a golem), it is unaffected.
            \begin{spellattack}{Spellpower vs. Fortitude}
                \spellsuccess \spelldamage{physical}. For the next 5 rounds, if the target has no hit points remaining at the end of the round, it becomes \petrified along with its equipment.
                \spellcritical Double damage, and the target is immediately \petrified along with its equipment.
                \spellfailure Half damage, and no additional effects.
            \end{spellattack}
        \end{spelleffects}
    \end{spellsubcontent}
    \begin{spellsubcontent}
        \begin{spelltargetinginfo}
            \spellquicktargeting{One creature (Large or smaller)}{\rngclose}
        \end{spelltargetinginfo}
        \begin{spelleffects}
            \spelleffect The target is restored to its normal state, including its equipment. Stone which was not originally a petrified creature is unaffected.
        \end{spelleffects}
    \end{spellsubcontent}
    \begin{spellfooter}
        \spellinfo{Transmutation [Shaping]}{Arcane, Earth}
        \miscastrandom
    \end{spellfooter}
    \begin{spellaugments}
        \spellaugment{1}{Giant}{The spell can affect a target one size category larger. This augment can be used multiple times.}
        \spellaugment{3}{Mass}{The spell can affect up to five targets. The same version must be used for all targets.}
        \spellaugment{4}{Empowered}{The damage dealt by the damaging version of this spell increases to \spelldamageemp{physical}.}
    \end{spellaugments}
\end{spellsection}

\begin{spellsection}{Tree Shape}[2]
    \begin{spellheader}
    \end{spellheader}
    \begin{spellcontent}
        \begin{spelltargetinginfo}
            \spelltgt{You}
        \end{spelltargetinginfo}
        \begin{spelleffects}
            \spelleffect You transform into a Large tree, shrub, or dead tree trunk. In this form, you are effectively \paralyzed, but you can see around you in any direction as if you were in your normal form.
            \spelldur \durext \dismissable
        \end{spelleffects}
    \end{spellcontent}
    \begin{spellfooter}
        \spellinfo{Transmutation [Shaping]}{Nature}
        \spellnotes You can sleep comfortably in this form.
        \miscastexplode
    \end{spellfooter}
\end{spellsection}

\begin{spellsection}{Tremorsense}[1]
    \begin{spellheader}
    \end{spellheader}
    \begin{spellcontent}
        \begin{spelltargetinginfo}
            \spellquicktargeting{One creature}{\rngclose}
        \end{spelltargetinginfo}
        \begin{spelleffects}
            \spelleffect The target gains the tremorsense ability with a range of 50 feet. If it is touching a surface, it can automatically pinpoint the location of anything within 50 feet that is in contact with the surface, including inanimate objects.
            \spelldur \durpersonallong
        \end{spelleffects}
    \end{spellcontent}
    \begin{spellfooter}
        \spellinfo{Transmutation [Imbuement]}{Nature}
        \spellnotes Tremorsense functions on surfaces of any kind, regardless of lighting conditions.
        \miscastexplode
    \end{spellfooter}
    \begin{spellaugments}
        \spellaugment{2}{Distant}{The tremorsense ability has a range of 200 feet.}
        \spellaugment{3}{Mass}{The spell can affect up to five targets.}
    \end{spellaugments}
\end{spellsection}

\begin{spellsection}{True Seeing}[6]
    \begin{spellheader}
        \spelldesc{You grant your ally the ability to see all things as they actually are.}
    \end{spellheader}
    \begin{spellcontent}
        \begin{spelltargetinginfo}
            \spellquicktargeting{One creature}{\rngtouch}
        \end{spelltargetinginfo}
        \begin{spelleffects}
            \spelleffect The target sees through normal and magical darkness, sees the truth behind visual figments and glamers, and sees the true form of polymorphed, changed, or transmuted things. The effect extends out to \rngmed range.
            \spelldur \durshort
        \end{spelleffects}
    \end{spellcontent}
    \begin{spellfooter}
        \spellinfo{Divination [Imbuement]}{Arcane, Divine, Knowledge}
        \spellnotes This spell does not negate concealment, including that caused by fog and the like. It does not help against mundane disguises or concealed objects or creatures. In addition, the spell's effects cannot be further enhanced with known magic, so the benefits do not apply when seeing through a scrying effect or similar vision enhancements.
        \miscastexplode
    \end{spellfooter}
\end{spellsection}

\begin{spellsection}{True Strike}[1]
    \begin{spellheader}
        \spelldesc{You grant your ally a temporary, intuitive insight into the immediate future.}
    \end{spellheader}
    \begin{spellcontent}
        \begin{spelltargetinginfo}
            \spellquicktargeting{One creature}{\rngmed}
        \end{spelltargetinginfo}
        \begin{spelleffects}
            \spelleffect The target gains an offensive \glossterm{legend point}. It is automatically lost at the end of the spell's duration if not used.
            \spelldur \durshort
        \end{spelleffects}
    \end{spellcontent}
    \begin{spellfooter}
        \spellinfo{Divination [Knowledge]}{Arcane}
        \miscastrandom
    \end{spellfooter}
    \begin{spellaugments}
        \spellaugment{3}{Mass}{The spell can affect up to five targets.}
    \end{spellaugments}
\end{spellsection}

\pdfbookmark[2]{U-Z}{SpellDescriptionsU-Z}

\begin{spellsection}{Unholy Avatar}[9]
    \begin{spellheader}
        \spelldesc{You embody the essence of evil, allowing you to smite your foes.}
    \end{spellheader}
    \begin{spellcontent}
        \begin{spelltargetinginfo}
            \spelltgt{You}
        \end{spelltargetinginfo}
        \begin{spelleffects}
            \spelleffect At any time during the spell's duration, you can concentrate as a standard action. If you do, you smite a foe, as described below.
            \spelldur \durlong
        \end{spelleffects}
    \end{spellcontent}
    \begin{spellsubcontent}
        \begin{spelltargetinginfo}
            \spellquicktargeting{One nonevil creature}{\rngmed}
        \end{spelltargetinginfo}
        \begin{spelleffects}
            \spelleffect The target takes 1d10 divine damage per two spellpower. In addition, it is \staggered for 2 rounds.
        \end{spelleffects}
    \end{spellsubcontent}
    \begin{spellfooter}
        \spellinfo{Channeling [Evil]}{Divine, Evil}
        \miscastexplode
    \end{spellfooter}
\end{spellsection}

\begin{spellsection}{Unholy Blight}[3]
    \begin{spellheader}
    \end{spellheader}
    \begin{spellcontent}
        \begin{spelltargetinginfo}
            \spellquicktargeting{One nonevil creature}{\rngmed}
        \end{spelltargetinginfo}
        \begin{spelleffects}
            \begin{spellattack}{Spellpower vs. Mental}
                \spellsuccess \spelldamage{divine}, and the target is \staggered for 2 rounds.
                \spellcritical Double damage, and the target is \nauseated for 1 round instead of staggered.
                \spellfailure Half damage, and no additional effects.
            \end{spellattack}
        \end{spelleffects}
    \end{spellcontent}
    \begin{spellfooter}
        \spellinfo{Channeling [Evil]}{Evil}
        \miscastrandom
    \end{spellfooter}
    \begin{spellaugments}
        \spellaugment{3}{Mass}{The spell can affect up to five targets. Its damage becomes \spelldamage{divine}[d8]}
        \spellaugment{4}{Empowered}{The damage increases to \spelldamageemp{divine}. If the Mass augment is applied, the damage instead increases to \spelldamage{divine}[d10].}
    \end{spellaugments}
\end{spellsection}

\begin{spellsection}{Unliving Eyes}[4]
    \begin{spellheader}
    \end{spellheader}
    \begin{spellcontent}
        \begin{spelltargetinginfo}
            \spellquicktargeting{One creature}{\rngclose}
        \end{spelltargetinginfo}
        \begin{spelleffects}
            \spelleffect The target gains the ability to ``see'' any living creatures and their equipment within 30 feet perfectly, regardless of lighting conditions, invisibility, or any other means of concealment. This cannot detect living creatures through solid walls, however.

            If the target is undead, the range of the vision is increased to 50 feet.
            \spelldur \durpersonallong
        \end{spelleffects}
    \end{spellcontent}
    \begin{spellfooter}
        \spellinfo{Vivimancy [Life]}{Arcane}
        \miscastexplode
    \end{spellfooter}
    \begin{spellaugments}
        \spellaugment{2}{Distant}{The range of the vision increases to 200 feet.}
        \spellaugment{2}{Penetrating}{The vision is not blocked by physical obstacles other than lead.}
    \end{spellaugments}
\end{spellsection}

\begin{spellsection}{Unliving Heart}[1]
    \begin{spellheader}
        \spelldesc{You harness the power of unlife to grant yourself a limited ability to avoid death.}
    \end{spellheader}
    \begin{spellcontent}
        \begin{spelltargetinginfo}
            \spelltgt{You}
        \end{spelltargetinginfo}
        \begin{spelleffects}
            \spelleffect You gain temporary hit points equal to twice your spellpower. If you take life damage, you lose all temporary hit points provided by this spell before applying the damage.

            In addition, you are treated as being undead for the purpose of spells or abilities which affect undead. This can cause some unintelligent undead, such as skeletons and zombies, to avoid attacking you.
            \spelldur \durlong
        \end{spelleffects}
    \end{spellcontent}
    \begin{spellfooter}
        \spellinfo{Vivimancy [Life]}{Arcane}
        \miscastexplode
    \end{spellfooter}
    \begin{spellaugments}
        \spellaugment{2}{Regenerating}{If you lose temporary hit points, you regain the lost hit points 5 minutes later.}
        \spellaugment{4}{Empowered}{The temporary hit points granted by this spell increase to three times your spellpower.}
    \end{spellaugments}
\end{spellsection}

\begin{spellsection}{Ventriloquism}[1]
    \begin{spellheader}
    \end{spellheader}
    \begin{spellcontent}
        \begin{spelltargetinginfo}
            \spellrng{\rngmed}
            \spellcmp{Somatic only}
        \end{spelltargetinginfo}
        \begin{spelleffects}
            \spelleffect Your voice (or any sound that you can normally make vocally) originates from another location within range. As a swift action, you can concentrate to change the apparent origin of your voice. If you move out of range of your designated location, the sound of your voice comes from your own mouth as normal.
            \spelldur \durshort \dismissable
        \end{spelleffects}
    \end{spellcontent}
    \begin{spellfooter}
        \spellinfo{Illusion [Figment]}{Arcane}
        \miscastexplode
    \end{spellfooter}
\end{spellsection}

\begin{spellsection}{Wail of the Banshee}[9]
    \begin{spellheader}
        \spelldesc{You emit a terrible scream that kills anyone that hears it.}
    \end{spellheader}
    \begin{spellcontent}
        \begin{spelltargetinginfo}
            \spellburst{\arealarge radius centered on you}
            \spelltgts{Everything in the area}
            \spellcmp{Verbal only}
        \end{spelltargetinginfo}
        \begin{spelleffects}
            \begin{spellattack}{Spellpower vs. Fortitude}
                \spellsuccess \spelldamage{sonic}[d8]. If the target is living and has no hit points remaining, it dies.
                \spellcritical As above, but double damage.
                \spellfailure As above, but half damage.
            \end{spellattack}
        \end{spelleffects}
    \end{spellcontent}
    \begin{spellfooter}
        \spellinfo{Vivimancy [Auditory, Death]}{Arcane, Death}
        \miscastexplode
    \end{spellfooter}
\end{spellsection}

\begin{spellsection}{Wall of Antimagic}[4]
    \begin{spellheader}
        \spelldesc{You create a shimmering field that hangs in the air, blocking the magical abilities of your foes.}
    \end{spellheader}
    \begin{spellcontent}
        \begin{spelltargetinginfo}
            \spelltwocol{\spellzone{50 ft.\ wall, 10 ft.\ high}}{\spellrng{\rngmed}}
        \end{spelltargetinginfo}
        \begin{spelleffects}
            \spelleffect This spell creates a translucent wall that selectively blocks abilities.
            The wall blocks line of effect for all \glossterm{magical} abilities, except for abilities that you allow to pass through the wall.
            Any creature may attempt to use abilities through the wall.
            If you block an ability, it fails if its range would be beyond the wall, and any effects of the ability stop at the edge of the wall.
            If you allow an ability through the wall, it functions normally.

            Deciding whether to allow an ability through the wall does not take an action, but you must be aware of the use of an ability to make that decision.
            Each round, you may choose whether the wall blocks abilities that you are not aware of or allows them through the wall.
        \end{spelleffects}
    \end{spellcontent}
    \begin{spellfooter}
        \spellinfo{Abjuration [Thaumaturgy]}{Arcane}
        \spellnotes You cannot partially block an ability, such as by allowing a \spell{fireball} spell to be cast through the wall, but blocking its effect at the edge of the well.
        An ability can either pass through the wall entirely or is blocked completely.
        \miscastexplode
    \end{spellfooter}
    \begin{spellaugments}
        \spellaugment{2}{Widened}{The wall appears in a 100 ft.\ line or a 20 ft.\ radius.}
    \end{spellaugments}
\end{spellsection}

\begin{spellsection}{Wall of Fire}[2]
    \begin{spellheader}
    \end{spellheader}
    \begin{spellcontent}
        \begin{spelltargetinginfo}
            \spelltwocol{\spellzone{20 ft.\ wall, 10 ft.\ high}}{\spellrng{\rngmed}}
        \end{spelltargetinginfo}
        \begin{spelleffects}
            \spelleffect This spell creates a wall made of fire.
            When the spell is cast, you choose which sides of the wall radiate heat.
            At the end of each round, all creatures within 10 feet of a side radiating heat take 1d10 fire damage per four spellpower.

            In addition, whenever a creature passes through the wall, make a spellpower vs. Reflex attack against it.
            Success deals \spelldamage{fire}[d8] to the creature.
            Failure deals half damage.
            \spelldur \durshort
        \end{spelleffects}
    \end{spellcontent}
    \begin{spellfooter}
        \spellinfo{Evocation [Fire]}{Arcane, Fire, Nature}
        \spellnotes Any part of the wall takes cold damage in excess of your spellpower in a single round is extinguished.
        \miscastexplode
    \end{spellfooter}
    \begin{spellaugments}
        \spellaugment{2}{Dual}{Two parallel walls appear, five feet apart. This augment can only be used if the wall is created in a line.}
        \spellaugment{2}{Widened}{The wall appears in a 50 ft.\ line or a 20 ft.\ radius.}
        \spellaugment{4}{Empowered}{The damage dealt to creatures near a side of the wall radiating heat increases to \spelldamage{fire}[d6]. In addition, the damage dealt to creatures passing through the wall increases to \spelldamage{fire}[d10].}
    \end{spellaugments}
\end{spellsection}

\begin{spellsection}{Wall of Force}[5]
    \begin{spellheader}
    \end{spellheader}
    \begin{spellcontent}
        \begin{spelltargetinginfo}
            \spelltwocol{\spellzone{100 ft.\ solid wall, 10 ft.\ high}}{\spellrng{\rngmed}}
        \end{spelltargetinginfo}
        \begin{spelleffects}
            \spelleffect This spell creates an invisible wall made of telekinetic force. Nothing can pass through the wall.
            \spelldur \durshort \dismissable
        \end{spelleffects}
    \end{spellcontent}
    \begin{spellfooter}
        \spellinfo{Evocation [Telekinesis]}{Arcane}
        \spellnotes The wall can be destroyed. A 5-foot square of wall has hit points equal to five times your spellpower, and hardness equal to your spellpower.

        \miscastexplode
    \end{spellfooter}
\end{spellsection}

\begin{spellsection}{Wall of Thorns}[2]
    \begin{spellheader}
    \end{spellheader}
    \begin{spellcontent}
        \begin{spelltargetinginfo}
            \spelltwocol{\spellzone{50 ft.\ line, 5 ft.\ wide, 5 ft.\ high}}{\spellrng{\rngmed}}
        \end{spelltargetinginfo}
        \begin{spelleffects}
            \spelleffect This spell creates a thicket of thorns in the area. Moving out of a square with thorns in it costs 20 feet of movement. The wall can be created where creatures are.


            The wall provides total cover against attacks through the wall. A creature in the wall has cover from attacks on either side of the wall.
            \spelldur \durshort
        \end{spelleffects}
    \end{spellcontent}
    \begin{spellsubcontent}
        \begin{spelltargetinginfo}
            \spelltwocol{\spelltgr{A creature exits a square in the area}}{\spelltgt{The moving creature}}
        \end{spelltargetinginfo}
        \begin{spelleffects}
            \spelleffect The target takes piercing damage equal to your spellpower.
        \end{spelleffects}
    \end{spellsubcontent}
    \begin{spellfooter}
        \spellinfo{Conjuration [Creation]}{Nature}
        \spellnotes A \spell{wall of thorns} can be destroyed. A 5-foot cube of wall has hit points equal to twice your spellpower, is vulnerable to fire damage, and ignores piercing damage.
        \physicalspellnotes
        \miscastexplode
    \end{spellfooter}
    \begin{spellaugments}
        \spellaugment{1}{Shapeable}{The wall becomes can be shaped (see \pcref{Shapeable}).}
        \spellaugment{3}{Widened}{The wall appears in a 100 ft.\ line.}
    \end{spellaugments}
\end{spellsection}

\begin{spellsection}{Water Walk}[1]
    \begin{spellheader}
    \end{spellheader}
    \begin{spellcontent}
        \begin{spelltargetinginfo}
            \spellquicktargeting{One creature or object}{\rngmed}
        \end{spelltargetinginfo}
        \begin{spelleffects}
            \spelleffect The target threats the surface of any liquid as if it were firm ground.
            Mud, oil, snow, quicksand, running water, ice, and even lava can be traversed easily, since the target's feet hover an inch or two above the surface.
            \par If the target is below the surface of a liquid, it rises toward the surface at 60 feet per round.
            Thick liquids, such as mud and lava, may cause the target to rise more slowly.
            \spelldur \durshort
        \end{spelleffects}
    \end{spellcontent}
    \begin{spellfooter}
        \spellinfo{Transmutation [Imbuement, Water]}{Nature, Water}
        \miscastexplode
    \end{spellfooter}
    \begin{spellaugments}
        \spellaugment{2}{Persistent}{The spell's duration becomes \durlong.}
        \spellaugment{3}{Mass}{The spell can affect up to five targets.}
    \end{spellaugments}
\end{spellsection}

\begin{spellsection}{Waves of Fatigue}[3]
    \begin{spellheader}
    \end{spellheader}
    \begin{spellcontent}
        \begin{spelltargetinginfo}
            \spellburst{\arealarge cone}
            \spelltgts{All creatures in the area}
        \end{spelltargetinginfo}
        \begin{spelleffects}
            \spelleffect The target is \fatigued for 2 rounds.
        \end{spelleffects}
    \end{spellcontent}
    \begin{spellfooter}
        \spellinfo{Vivimancy [Flesh]}{Arcane}
        \miscastexplode
    \end{spellfooter}
    \begin{spellaugments}
        \spellaugment{1}{Selective}{The spell only targets enemies in the area.}
        \spellaugment{3}{Exhaustion}{The targets are \exhausted instead of fatigued.}
    \end{spellaugments}
\end{spellsection}

\begin{spellsection}{Web}[5]
    \begin{spellheader}
        \spelldesc{You create a many-layered mass of strong, stricky strands that trap creatures caught within them. The strands are similar to spider webs, but larger and tougher.}
    \end{spellheader}
    \begin{spellcontent}
        \begin{spelltargetinginfo}
            \spelltwocol{\spellzone{40 foot cube}}{\spellrng{\rngmed}}
            \spellspecial The area must border two solid objects on opposing sides. For example, this spell can be cast between two opposing walls, or between a floor and a ceiling, but not in empty space.
        \end{spelltargetinginfo}
        \begin{spelleffects}
            \spelleffect The area is filled with webs, causing it to be treated as difficult terrain. The webs are thick and strong, but too widely spaced to significantly obscure sight. % info about how to destroy webs
            \spelldur \durshort \dismissable
        \end{spelleffects}
    \end{spellcontent}
    \begin{spellsubcontent}
        \begin{spelltargetinginfo}
            \spelltgts{Everything in the area}
        \end{spelltargetinginfo}
        \begin{spelleffects}
            \begin{spellattack}{Spellpower vs. Reflex}
                \spellsuccess The target is \immobilized. It can escape by destroying the webs in its space, or with a grapple or Escape Artist check against a DR equal to 10 \add your spellpower.
            \end{spellattack}
        \end{spelleffects}
    \end{spellsubcontent}
    \begin{spellfooter}
        \spellinfo{Conjuration [Creation]}{Arcane}
        \spellnotes The webs can be destroyed. A 5-foot cube of webs has hit points equal to twice your spellpower, is vulnerable to fire damage, and ignores piercing damage. Any fire can set the webs alight and burn away 5 square feet over the course of 2 rounds. All creatures within flaming webs are \ignited by the flames.

        \physicalspellnotes
        \miscastyou
    \end{spellfooter}
\end{spellsection}

\begin{spellsection}{Windstrike}[2]
    \begin{spellheader}
        \spelldesc{You command the air to bludgeon the target, sending it flying.}
    \end{spellheader}
    \begin{spellcontent}
        \begin{spelltargetinginfo}
            \spellquicktargeting{One creature or object}{\rngmed}
        \end{spelltargetinginfo}
        \begin{spelleffects}
            \begin{spellattack}{Spellpower vs. Fortitude defense (shove)}
                \spellsuccess[Fortitude] \spelldamage{bludgeoning}. In addition, you shove the target in any direction -- even vertically. Moving the target up takes twice as much movement as moving the target horizontally.
                \spellfailure[Fortitude] Half damage.
            \end{spellattack}
        \end{spelleffects}
    \end{spellcontent}
    \begin{spellfooter}
        \spellinfo{Evocation [Air]}{Air, Nature}
        \miscastrandom
    \end{spellfooter}
    \begin{spellaugments}
        \spellaugment{2}{Distant}{The spell's range becomes \rngext.}
        \spellaugment{2}{Forceful}{You gain a \plus10 bonus to accuracy on the shove attack.}
        \spellaugment{4}{Empowered}{The damage increases to \spelldamageemp{bludgeoning}.}
    \end{spellaugments}
\end{spellsection}

\begin{spellsection}{Word of Recall}[6]
    \begin{spellheader}
    \end{spellheader}
    \begin{spellcontent}
        \begin{spelltargetinginfo}
            \spellquicktargeting{You}{Unlimited \rngunrestricted}
            \spellcmp{Verbal only}
        \end{spelltargetinginfo}
        \begin{spelleffects}
            \spelleffect This spell teleports you instantly back to your sanctuary. You must designate the sanctuary when you ready the spell for the day, and it must be a very familiar place. The actual point of arrival is a designated area no larger than 10 feet by 10 feet. You can be transported any distance within a plane but cannot travel between planes. You can transport, in addition to yourself, any objects you carry, as long as their weight doesn't exceed your maximum load. Exceeding this limit causes the spell to fail.
        \end{spelleffects}
    \end{spellcontent}
    \begin{spellfooter}
        \spellinfo{Conjuration [Teleportation]}{Divine, Travel}
        \miscastexplode
    \end{spellfooter}
\end{spellsection}

\begin{spellsection}{Zephyr Blade}[2]
    \begin{spellheader}
        \spelldesc{You imbue a weapon with the power of the wind, allowing it to strike opponents with nothing but the air itself.}
    \end{spellheader}
    \begin{spellcontent}
        \begin{spelltargetinginfo}
            \spellquicktargeting{One melee weapon}{\rngclose}
        \end{spelltargetinginfo}
        \begin{spelleffects}
            \spelleffect The target weapon gains an additional five feet of reach, extending the wielder's threatened area.

            In addition, the weapon can also be used to attack as a ranged weapon by expelling blasts of wind. This functions like attacking with the weapon normally, using the wielder's normal attack and damage bonuses, except that the attack is a ranged attack against any creature within \rngclose range. All damage dealt when attacking in this way is bludgeoning damage instead of the attack's normal damage types. This effect does not increase the wielder's threatened area.
            \spelldur \durpersonallong
        \end{spelleffects}
    \end{spellcontent}
    \begin{spellfooter}
        \spellinfo{Evocation/Transmutation [Air, Imbuement]}{Nature}
        \spellnotes Despite the name of the spell, it can affect melee weapons of any type, even reach weapons. The weapon's extended reach is visible, and opponents can defend themselves normally against the attacks.
        \miscastexplode
    \end{spellfooter}
    \begin{spellaugments}
        \spellaugment{2}{Empowered}{The target gains ten feet of reach, rather than five.}
        \spellaugment{3}{Mass}{The spell can affect up to five targets.}
    \end{spellaugments}
\end{spellsection}
