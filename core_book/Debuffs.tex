\chapter{Debuffs}\label{Debuffs}

\debuffdef{blinded} A blinded creature cannot see.
It automatically fails at actions which depend on vision, including simply seeing the locations of other objects and creatures (but see \pcref{Awareness}).
It has a 50\% miss chance with all attacks and vision-related checks, even if it knows the location of its target.
It is at least \glossterm{partially unaware} of all attacks against it, and it can be fully \glossterm{unaware} as normal depending on its level of awareness.

\debuffdef{charmed} A charmed creature is mentally influenced to like another creature.
It always sees the words and actions of the creature that charmed it in the most favorable way, as a close friend or trusted ally.
A charmed creature cannot be controlled like an automaton, but can be persuaded to take particular actions with the Persuasion skill (see \pcref{Persuasion}).
It treats the creature that charmed it as a friend (a \plus10 relationship modifier) for the purpose of Persuasion checks.

\debuffdef{confused} A confused creature is unable to independently control its actions. \confusionexplanation

\debuffdef{deafened} A deafened creature cannot hear. It automatically fails at actions which depend on hearing. In addition, it has a 20\% failure chance when casting any spell with verbal components.

\debuffdef{decelerated} A decelerated creature moves at one quarter speed and takes a \minus4 penalty to Armor and Reflex defenses.
In addition, a decelerated creature cannot act during the \glossterm{action phase}.
It can take its normal actions during the \glossterm{delayed action phase}.
This does not stack with the \glossterm{slowed} effect.

\debuffdef{disoriented} A disoriented creature cannot control the direction of its movement.
If it tries to move, it instead moves the same distance in a random direction.
A disoriented creature can remain in the same location without penalty.
This does not affect abilities that change location without normal movement, such as \glossterm{teleportation}.

\debuffdef{dominated} A dominated creature is mentally compelled to obey another creature.
It obeys the commands of the creature of the dominated it unquestioningly, as an automaton.
If it does not understand the language of the creature that dominated it, it still attempts to obey as much as possible, and simple commmands (such as ``attack'' or ``follow'') can usually be communicated successfully.

\debuffdef{fascinated} A fascinated creature can take no actions. It remains in place, giving its total attention to some object, creature, or effect. It takes a \minus4 penalty to skill checks made as reactions, such as Awareness checks.
If the creature notices any threat against it, such as an approaching enemy, it is no longer fascinated.

\debuffdef{frightened} A frightened creature takes a \minus4 penalty to \glossterm{accuracy} and Mental defense while it is within \rngmed range of the source of its fear.
This does not stack with the \glossterm{shaken} or \glossterm{panicked} effects.
If the source of a frightened creature's fear is \glossterm{defeated}, this effect is broken.

\debuffdef{grappled} A grappled creature is wrestling or in some other form of hand-to-hand struggle with at least one other creature.
While grappled, you suffer certain penalties and restrictions, as described below.
\begin{itemize}
    \item You must use a free hand (or equivalent limb) to grapple, preventing you from taking any actions which would require having two free hands.
        If you do not have a free hand, you suffer a \minus10 penalty to accuracy with all \glossterm{mundane} attacks and grapple actions until you have a free hand.
    \item You take a \minus2 penalty to Armor and Reflex defenses.
    \item The creature you are grappling with provides \glossterm{cover} for you against non-\glossterm{melee} attacks (see \pcref{Cover}).
    \item You take a \minus4 penalty to accuracy with weapons that are not \glossterm{light weapons}, since they are too large and cumbersome to be used effectively in a grapple.
    \item Abilities that have \glossterm{somatic components}, such as spells or rituals, have a 50\% chance to fail with no effect when used.
    \item You cannot move unless you \glossterm{push} all creatures grappling you, such as with the \textit{shove} ability (see \pcref{Shove}).
\end{itemize}

Other than the restrictions listed above, you can act normally. You can also try to move the grapple, escape the grapple, or pin your opponent. For details, see \pcref{Grapple Actions}.

\debuffdef{immobilized} An immobilized creature can't use any of its movement speeds.
Immobilized flying creatures that have the ability to hover can maintain their initial altitude.
% TODO: Fix flying interaction; safest way to descend for non-good maneuverability
All other flying creatures subjected to this condition descend at a rate of 20 feet per round until they reach the ground, taking no falling damage.

\debuffdef{nauseated} A nauseated creature takes a \minus4 penalty to all defenses.
This does not stack with the \glossterm{sickened} effect.

\debuffdef{surrounded} An creature is surrounded if every space adjacent to it either contains an \glossterm{enemy} or is both empty and adjacent to an \glossterm{enemy}.
A surrounded creature suffers a \minus2 penalty to Armor and Reflex defenses.

\debuffdef{panicked} While a panicked creature is within \rngmed range of the source of its fear, it takes a \minus4 penalty to Mental defense and must flee from the source of its fear by any means necessary.
If unable to flee, it must do nothing other than use the \textit{total defense} ability every round (see \pcref{Total Defense}).
The penalty from this effect does not stack with the \glossterm{frightened} or \glossterm{panicked} effects.

If the source of a panicked creature's fear is \glossterm{defeated}, this effect is broken.

\debuffdef{paralyzed} A paralyzed creature is unable to take physical actions. It has effective Dexterity and Strength scores of \minus10 and is \helpless, but can take purely mental actions. This can cause flying creatures to crash, swimming creatures to drown, and so on. Any creature can move through a space occupied by a paralyzed creature, though its space is treated as \glossterm{difficult terrain}.

\debuffdef{partially unaware} An creature that is partially unaware of an attack knows that it is in danger, but is missing information about the exact location or nature of the attack.
Creatures take a \minus2 penalty to Armor and Reflex defenses against attacks that they are partially unaware of.
For details, see \pcref{Awareness and Surprise}.

\debuffdef{petrified} A petrified creature has been turned to stone. It is neither alive nor dead, but is unconscious and unable to take actions, and its body is an inanimate statue. If the statue is broken or damaged before the creature is restored to its original state, the creature has equivalent damage or deformities.

\debuffdef{pinned} A pinned creature is held completely immobile in a grapple.
The only physical actions it can make are to escape the grapple (see \pcref{Grappling}).
Like a \glossterm{helpless} creature, its Dexterity is treated as \minus10.

\debuffdef{prone} A prone creature is lying on the ground, rather than standing normally.
It takes a \minus2 penalty to Armor and Reflex defenses, though it gains a +4 bonus to all defenses against ranged \glossterm{strikes}.
In addition, it takes a \minus2 penalty to accuracy with melee \glossterm{strikes} and is unable to move until it stands up.
A creature can stand up from being prone during the movement phase.
This generally requires one free hand.

\debuffdef{shaken} A shaken creature takes a \minus2 penalty to \glossterm{accuracy} and Mental defense while it is within \rngmed range of the source of its fear.
This does not stack with the \glossterm{frightened} or \glossterm{panicked} effects.

If the source of a shaken creature's fear is \glossterm{defeated}, this effect is broken.

\debuffdef{sickened} A sickened creature takes a \minus2 penalty to all defenses.
This does not stack with the \glossterm{nauseated} effect.

\debuffdef{slowed} A slowed creature moves at half speed and takes a \minus2 penalty to Armor and Reflex defenses.
This does not stack with the \glossterm{decelerated} effect.

\debuffdef{squeezing} A squeezing creature is trying to move though an area too small for it to fight in normally.
While squeezing, a creature moves at half speed and takes a \minus2 penalty to \glossterm{accuracy}, as well as Armor and Reflex defenses.
For details, see \pcref{Squeezing}.

\debuffdef{stunned} A stunned creature takes a \minus4 penalty to all defenses.
This does not stack with the \glossterm{dazed} effect.

\debuffdef{unaware} An creature that is unaware of an attack makes no attempt to defend itself.
Creatures take a \minus5 penalty to Armor and Reflex defenses against attacks that they are unaware of.
For details, see \pcref{Awareness and Surprise}.

\debuffdef{unconscious} While you are unconscious, you are \glossterm{helpless} and completely unable to take any actions.
Some sensory abilities, such as the Awareness and Spellsense skills, can be used while you are asleep, but not while you are forcibly knocked unconscious.
