\chapter{The Universe}

\section{Planes}\label{Planes}
    The universe of Rise is divided into \glossterm{planes}.
    A plane is a distinct realm of existence.
    Except for the connections between planes through \glossterm{planar rifts}, each plane is effectively an isolated universe, and different planes can obey different fundamental laws.
    For example, the Material Plane has gravity that exerts a consistent acceleration in a single absolute direction.
    However, the Astral Plane has subjective gravity, where each creature on the plane chooses the direction that gravity pulls it in, if any.

    \subsection{General Cosmology}
        The planes of Rise are divided up into groups.

        \parhead{Primal Planes} The primal planes are manifestations of the basic building blocks of the universe.
        Each plane in this group is predominantly composed of a single element or type of energy.
        There are four primal planes: Air, Earth, Fire, and Water.

        \parhead{Aligned Planes} The four aligned planes are manifestations of the four alignments.
        The Celestial Heavens is good-aligned, the Abyss is evil-aligned, Ordus is law-aligned, and Discord is chaos-aligned.

        When mortal creatures die, their souls travel to an appropriate location on an aligned plane, where they gain new planeforged bodies and live again.
        If they pledged their soul to a deity in life, that deity can take ownership over their soul in death, and the soul is reborn within that deity's territory and under their protection.
        Otherwise, they appear on the aligned plane that most closely reflects their primary alignment in life.

        For details about aligned planes, see \pcref{Aligned Planes}.

        \parhead{Nexus Planes} The nexus planes are composite planes with a number of distinct environments and filled with creatures of myriad alignments.
        Nexus planes comprise the majority of civilization across all planes.
        They do not have their own unique planar essence, and no planar creatures are native to nexus planes.
        There are two nexus planes: the Material Plane and the Astral Plane.

        \parhead{Demiplanes} These planes are small, fragmentary realms that are greatly limited in their scope.
        There is no specific list of demiplanes, and they share few common properties.
        Most demiplanes were created for particular purposes by beings of great power, though some simply came into existence through unknown means.

    \subsection{Planar Rifts}\label{Planar Rifts}
        Normally, there are boundaries between different planes that prevent direct passage between them.
        However, \glossterm{planar rifts} are places where these boundaries have weakened, making interplanar travel easier.
        A planar rift joins a specific location on one plane to a specific location on a different plane.
        Most planar rifts lead to and from the Astral Plane, which is the space between the other planes (see \pcref{The Astral Plane}).

        Most planar rifts still require the use of magic, such as the \spell{plane shift} ritual, to actually cross between planes.
        Some especially large rifts enable physical travel between planes without the use of any magic.

    \subsection{Planar Traits}
        \subsubsection{Gravity Direction}
            The direction of gravity on a plane can take one of the following forms:
            \begin{itemize}
                \item Fixed Gravity: Gravity points in a fixed direction and with a fixed strength at all locations on the plane.
                    Almost all planes with a fixed gravity have a perfectly flat surface.
                \item Absolute Directional Gravity: Gravity points in a consistent direction according to a rule that applies equally to everything on the plane, but which is not in a fixed direction.
                    For example, a plane filled with floating spheres where gravity always points towards the closest sphere has absolute directional gravity.
                \item Subjective Gravity: Each creature on the plane chooses the direction of gravity for that creature.
                    The plane has no gravity for unattended objects and nonsentient creatures.
                    A creature on the plane can make use the \textit{control gravity} ability as a \glossterm{minor action}.
                    \begin{freeability}{Control Gravity}
                        Make a Willpower check with a \glossterm{difficulty value} of 10.
                        Success means that you choose the direction of gravity that applies to you on the current plane.
                        Alternately, you can choose for gravity to not apply to you.

                        Failure means you gain a \plus2 bonus to the next \textit{control gravity} ability you use on this plane.
                        This bonus stacks with itself and lasts until you succeed at a \textit{control gravity} ability on this plane.
                    \end{freeability}
            \end{itemize}

        \subsubsection{Gravity Strength} The strength of gravity on a plane can take one of the following forms:
            \begin{itemize}
                \item Normal Gravity: Gravity is about the strength of Earth.
                \item No Gravity: There is no gravity on the plane.
                    The \glossterm{range limits} of ranged weapons are quadrupled.
                    % TODO: what additional effects are there?
                \item Light Gravity: Gravity is about half the strength of Earth.
                    % Should there be check penalties?
                    The weight of all items is halved.
                    The \glossterm{range limits} of ranged weapons are doubled.
                \item Heavy Gravity: Gravity is about twice the strength of Earth.
                    Creatures take a \minus2 penalty to Strength and Dexterity-based checks.
                    The weight of all items is doubled.
                    The \glossterm{range limits} of ranged weapons are halved, to a minimum of 5 feet.
                    % TODO: falling damage
                \item Extreme Gravity: Gravity is about four times the strength of Earth.
                    Creatures take a \minus4 penalty to Strength and Dexterity-based checks.
                    The weight of all items is quadrupled.
                    The \glossterm{range limits} of ranged weapons are reduced to one quarter of the normal value, to a minimum of 5 feet.
                    % TODO: falling damage
            \end{itemize}

        \subsubsection{Light} Various planes are illuminated in different ways.
            \begin{itemize}
                \item Fixed Source: There is a single constant source of light on the plane.
                \item Mobile Source: There is a single source of light on the plane that moves around it, illuminating different parts of the plane at different times.
                \item None: There is no natural source of light on the plane.
                    Other sources of light, such as torches, function normal.
            \end{itemize}

        \subsubsection{Limits} The behavior of a plane at its limits can vary widely.
            Some planes have different behaviors at different limits depending on their shape.
            \begin{itemize}
                \item Astral Gate: If you reach the limits of the plane, you find a planar gate to the Astral Plane.
                \item Barrier: If you reach the limits of the plane, you find an impassable barrier.
                    The barrier takes the form of a substance relevant to the plane's nature.
                    It may be possible to dig tunnels into the barrier to some depth, but there is nothing behind the barrier.
                    As you progress past the limit, the barrier becomes increasingly difficult to break through, and eventually it becomes completely impenetrable.
                \item Looped: If you go beyond the limits of the plane, you wrap around to the opposite side of the plane.
                    There is no obvious transition point or perception of transportation when this occurs - the shape of the plane simply connects to itself.
                    On very small planes, this can allow you to see your own back, though looped planes of that size are rare.
                \item Infinite: The plane has no limits. This is extremely rare.
            \end{itemize}

        \subsubsection{Planar Connectivity}
            Different planes have different degrees of connection to other planes.
            \begin{itemize}
                \item Isolated: The plane is difficult to reach or leave.
                    It has no permanent \glossterm{planar rifts}, and temporary rifts are rare or nonexistent.
                \item Stable Connected: The plane has multiple permanent \glossterm{planar rifts}.
                    However, temporary rifts are rare.
                \item Unstable Connected: The plane has no permanent \glossterm{planar rifts}, but temporary rifts are common.
                \item Conduit: The plane has a large number of permanent \glossterm{planar rifts}, and temporary rifts are common.
            \end{itemize}

        \subsubsection{Shape} The shape of a plane defines the shape of its core surface, and what happens if you travel beyond that surface.

            \begin{itemize}
                \item Flat Surface: The plane consists of a flat surface generally made of earth or similar material.
                    Most activity and civilization on the plane happens on this surface.
                    It is usually possible to construct tunnels into a flat surface plane to some depth, depending on the size of the plane.
                \item Hollow Sphere: The plane consists of a hollow sphere with an outer boundary generally made of earth or similar material.
                    Most activity and civilization on the plane happens on the inner surface of the sphere or in the vast open space between.
                    Some hollow sphere planes have an outer surface that can also be accessed, but in most planes it is impossible to leave the interior of the sphere.
                \item Solid Sphere: The plane consists of a solid sphere generally made of earth or similar material.
                    Most activity and civilization on the plane happens on the surface of the sphere.
                    It is possible to construct tunnels into a solid sphere plane, but it may become increasingly difficult to traverse the plane as you approach the center of the sphere.
                    In general, the limit of a solid sphere plane is located at ten times the radius of the plane's primary sphere.
                \item Uniform: The plane has no well-defined surface or ground layer.
                    Some uniform planes have no ground or solid obstacles, while others are composed almost entirely of ground and firmament.
                    Uniform planes almost always still have limits of some kind.
            \end{itemize}

    \subsection{Planeforged Creatures}
        A planeforged is a type of creature that is fundamentally composed of the essence of one or more planes.
        The vast majority of planeforged creatures are composed of only a single plane.
        When a planeforged dies, its essence returns to its native plane or planes.
        Weak planeforged lose their independent identity and become part of the core composition of the plane once more.
        Strong planeforged can retain their identity and reform from that raw material given time, making them difficult or impossible to kill completely.
        In either case, planeforged cannot be resurrected by soul-based magic such as the \spell{resurrection} spell.

            % TODO: morphicness, alignment, magic

\section{Plane Descriptions}

    \subsection{Primal Planes}

        \subsubsection{The Plane of Air}
        The Plane of Air is a a soaring landscape unencumbered by gravity or ground.
        The vast expanses of empty air are littered with clouds and unpredictable winds.
        Any inhabitants of the plane must adapt to a highly mobile lifestyle.
        A number of towns and structures have been built in the plane using raw materials brought from other planes.
        They sail through the air at the whims of the wind, and are occasionally battered by intersections with other wind streams.

        The Plane of Air has the following planar traits:
        \begin{itemize}
            \item Gravity strength: No gravity
            \item Light: Fixed source, from a sun outside the limits of the plane
            \item Limits: Barrier, formed from wind currents which push back with such force that nothing can travel far.
            \item Planar connectivity: Unstable connected
            \item Shape: Hollow sphere with a radius of about 2,000 miles.
        \end{itemize}

        \subsubsection{The Plane of Earth}
        The Plane of Earth is a titanically large body of earth and stone.
        A labyrinthine series of mostly airless tunnels weave their way through the plane, connecting the few cities.
        The plane is a major source of valuable gems, diamonds, and rare metals like mithral, but the dense rock and airless environment make successful mining difficult.
        In addition, earthquakes periodically reshape the environment by collapsing old tunnel systems and constructing new ones.
        Some cities have been carved out in vast underground rooms reinforced to survive the earthquakes.

        The Plane of Earth has the following planar traits:
        \begin{itemize}
            \item Gravity direction: Fixed
            \item Gravity strength: Normal
            \item Light: None
            \item Limits: Barrier, formed from increasingly dense rock that eventually becomes so hard that no known material or magic can damage it.
            \item Planar connectivity: Stable connected
            \item Shape: Hollow sphere with a radius of about 500 miles.
        \end{itemize}

        \subsubsection{The Plane of Fire}
        The Plane of Fire is an endless searing inferno.
        The plane's essence is highly combustible, allowing fires to burn indefinitely without any obvious fuel.
        However, the intensity of flames on the plane are highly uneven, as the plane generates fuel in various locations that shift over time.
        Some pockets on the surface are devoid of natural fuel, allowing the allow the construction of trading hubs where the few inhabitants of the plane who are not naturally immune to fire can survive.
        A variety of large tunnels and magma flows run through the sphere, and the intensity of the heat generally increases as you approach the center.

        The Plane of Fire has the following planar traits:
        \begin{itemize}
            \item Gravity direction: Absolute directional, pointing to the center of the sphere
            \item Gravity strength: Normal
            \item Light: None, though the constant fires provide sufficient illumination in most locations on the plane
            \item Limits: Barrier, formed from fires which burn so fiercely that further travel becomes physically impossible, even for creatures immune to fire.
            \item Planar connectivity: Unstable connected
            \item Shape: Flat surface, in a disc with a radius of about 2,000 miles.
        \end{itemize}

        \subsubsection{The Plane of Water}
        The Plane of Water is an impossibly vast ocean.
        Powerful currents sweep through the ocean, but much of it is calm, and many forms of aquatic life abound in the water.
        The Plane of Water is the most densely populated Primal Plane, both by sentient creatures and monsters.
        Magnificant underwater cities are carved from huge rocks that float peacefully suspended in the water.
        Though there is no sun, simple creatures akin to plankton form the base of the food chain by feeding directly on the plane's essence.

        The Plane of Earth has the following planar traits:
        \begin{itemize}
            \item Gravity strength: No gravity
            \item Light: None, though bioluminescent creatures like plankton are extremely common, making many parts of the plane well-lit
            \item Limits: Barrier, formed from water currents which push back with such force that nothing can travel far.
            \item Planar connectivity: Stable connected
            \item Shape: Hollow sphere with a radius of about 1,000 miles.
        \end{itemize}

        \subsection{Aligned Planes}\label{Aligned Planes}

            \subsubsection{The Celestial Heavens}
                The Celestial Heavens are beautiful and majestic.
                Mountains rise dramatically out of misty clouds, trees are massive and laden with delicious fruit, and buildings surpass the wildest dreams of mortal architects.
                A serene blue sky gives way to a night so lit by stars that it is almost as bright as the day.

            \subsubsection{The Abyss}
                The Abyss is a hellscape of fire, brimstone, and distant screaming.
                With the exception of the great palaces of demon princes, the buildings that exist are designed for defense rather than aesthetics.
                The terrain is typically rocky and dull, and most of the color belongs to carcasses left behind after violent battles.

                All manner of nightmarish creatures stalk the Abyss.
                The best known of these creatures are demons and devils.
                Demons are formed when mortal souls are splintered by trauma.
                The soul splinters drift into the Astral Plane, and from there are guided to the Abyss by ancient astral currents.
                When they arrive in the Abyss, its planar essence envelops them in new planeforged body, much like dead souls gain new bodies in their proper afterlife.

                Newly formed demons, known as demonspawn, are barely functional creatures.
                They are driven entirely by the primal emotion that separated the soul splinter from its original soul, such as rage, grief, or pain.
                This makes them functionally insane, and they are almost always driven to lash out at everything around them.
                Rage-born demons violently attack anything they see, pain-born demons try to lessen their pain by sharing it, and so on.
                When they succeed in their attacks, they can feed on the trauma they inflict, strengthening their soul.
                Unfortunately, this does not generally make them more sane, since they only feed on the same urges that created them.

                Demonspawn instinctively avoid attacking other demonspawn, since they can find no gratification for their urges in attacking such small, broken souls.
                Instead, they hunt creatures with complete souls, which generally means attacking the afterlife bodies of evil-aligned creatures who went to the Abyss for their afterlife.
                The greatest feast, however, comes from attacking mortal souls, which are much easier to splinter.
                Demonic incursions into other planes are devastating but fortunately rare.

                Unlike demons, devils are native to the Abyss itself.
                They are far more intelligent and organized than demons, but also far less numerous.
                Devils rule vast territories within the Abyss, using demons as their foot soldiers to protect and enlarge their territorial claims.

                The only competition with devils for rulership of the Abyss comes from the evil deities and greater demons.
                Evil deities are fairly simple to deal with.
                They have absolute dominion over their own territory, so invading their lands is pointless.
                In addition, since their territorial limits come from their divine power rather than force of arms, they have little ability to expand or even exert significant influence outside of their own lands.
                As a result, devils and greater demons alike mostly ignore the deities.

                Greater demons are much more troublesome.
                On rare occasions, demonspawn are so successful in their attacks that they claim soul splinters outside the scope of their original urges.
                This typically happens when demons find and break mortal souls.
                When this happens, the demonspawn gains a more complete soul, and becomes a little more sane.
                Often, this simply entices other demonspawn to attack and destroy the wayward demon.
                However, if the demon survives the attacks from its allies and repeats this process, it can grow in power.

                Demons who have expanded their soul beyond a single soul splinter are called greater demons.
                Eventually, the demon can gain something resembling a complete soul from all of the splinters it has collected, making it a demon prince.
                Though more sane and functional than demonspawn, these more developed demons are no less evil.
                Both greater demons and demon princes have enough skill with splintering and manipulating souls to make pacts with warlocks.
                In addition, demon princes have the power to command armies of demonspawn and greater demons, allowing them to claim territory like devils do.

            \subsubsection{Ordus}
                Ordus is a masterpiece of logical organization.
                It is the most consistently civilized of the aligned planes, and the cities are exquisitely planned.
                However, laws are enforced with extreme severity.
                Outside of the cities, even the natural territories are cleanly and simply divided.
                A forest of evenly spaced trees might border a field in a sharp, clean transition along a perfectly straight line.

            \subsubsection{Discord}
                Discord is a wild maelstrom.
                Much of the plane can be freely reshaped with only minimal force of will.
                By working together, its inhabitants can create vast cities from thin air, though they can be destroyed with similar ease.
                Beyond the shaped spaces, the terrain is constantly changing.
                A field might grow trees that are consumed by a forest fire and then fall into chasms newly formed by an earthquake in a matter of minutes.

    \subsubsection{Nexus Planes}

        \parhead{The Material Plane}\label{The Material Plane}
        The Material Plane is the plane that most Rise adventures begin on.
        The surface of the plane is a massive sphere with a radius of about 4,000 miles.
        It is the most familiar to most humanoid creatures.

        The Material Plane has the following planar traits:
        \begin{itemize}
            \item Gravity direction: Absolute Directional, pointing to the center of the sphere
            \item Gravity strength: Normal
            \item Light: Mobile Source, from a sun and moon outside of the plane's limits
            \item Limits: Looped
            \item Planar connectivity: Isolated
            \item Shape: Solid Sphere, with a radius of about 4,000 miles.
        \end{itemize}

    \parhead{The Astral Plane}\label{The Astral Plane}
        The Astral Plane is the space between the other planes.
        It is a necessary intermediate destination for virtually all planar journeys, as all \glossterm{planar rifts} lead to and from the Astral Plane.
        Most activity on the Astral Plane occurs in a space called the Inner Astral Plane, a massive but finite region where all planar rifts on the Astral Plane appear.
        % Are there any other infinite planes?
        However, unlike all other planes, the Astral Plane has no known limits to its extent, and may in fact be infinite.
        The area outside the Inner Astral Plane is known as the Deep Astral Plane, and few venture into those sparsely populated realms.
        % This should be more clearly defined
        The Deep Astral Plane has magical turbulence that interferes with long-range communication and transportation magic, making exploration difficult.

        The Astral Plane has the following planar traits:
        \begin{itemize}
            \item Directional gravity: Subjective
            \item Gravity strength: Normal
            \item Light: Fixed Source, from the infinite reaches of the Deep Astral Plane
            \item Limits: Infinite
            \item Planar connectivity: Conduit
            \item Shape: Uniform
        \end{itemize}

\section{Creatures and Objects}
    In the world of Rise, creatures and objects are meaningfully different.
    Many abilities affect only creatures or only objects.
    The difference between a creature and an object is defined as being agency.
    Creatures have agency, and objects do not.
    This is not the same as sentience or life, which either creatures or objects may have.

    For example, skeletons are nonsapient, nonliving creatures.
    Conversely, trees are a nonsapient, living objects.
    Some rare magic items can be made intelligent by magic, making them sapient, nonliving objects.
    Some unintelligent animals and magical beasts like ants and giant spiders are nonsapient, living creatures.

    \subsection{Animates}
        One type of entity in the world is both an object and a creature.
        Animates are a type of creature that are made of nonsapient matter given a semblance of life and sentience by some form of magic.
        Fire elementals, clay golems, and plant creatures like treants are all animates.
        Animates are considered to be both creatures and objects, and are affected fully by abilities that affect both.
