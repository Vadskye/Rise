\chapter{The World}

Rise does not attempt to define a single geography with specific countries and locations that is shared between all games.
It is common for GMs to define their own setting when running a game, and that freedom is important.
However, the universe of Rise does differ in a number of important ways from the real world.
The fundamental assumptions that Rise makes about the its fantasy world are:

\begin{raggeditemize}
  \item Magic is commonly known but rarely understood.
  \item Elite people and monsters have vastly more physical and magical personal power than normal people.
  \item When you die, your soul goes to an afterlife where deities exist.
  \item Spiritually similar things are attracted to each other by a fundamental force called "affinity"
\end{raggeditemize}

These core elements are ambiguous about some details, and GMs are encouraged to fill in those details as they see fit.
Of course, a GM has absolute power, and can create a world that changes any number of these assumptions.
However, doing so can significantly change the tone of the game and create logical inconsistencies, so it should be done carefully.

\section{Magic}
  The world of Rise is a magical place.
  Many people are capable of using magic to perform feats that would be impossible in the real world.
  Not everyone is capable of magic, of course.
  It's reasonable to assume that about ten percent of the civilized people in the world have a magical ability of some kind.
  In some societies, such as a feudal human-dominated society with a large number of commoners and serfs, the percentage of people with magic can be lower.
  However, this is balanced by the existence of other societies that tend to be more magical, such as elven communes in deep forests.
  Even in low-magic societies, everyone knows that magic exists, and almost everyone has observed something magical at some point in their lives.

  People can have magical abilities for a wide variety of reasons.
  There are three main categories to explain why people can access magic: intrinsic magic, learned magic, and gifted magic.
  Each class with magical abilities belongs to one of these groups.
  Characters with magical feats are free to choose any of those three explanations for their feats.
  The explanation does not have to be the same as for any other magical abilities they possess.
  For example, a cleric may be gifted their magical cleric abilities because they worship a particular deity, but they may also be naturally telepathic.

  Some people are simply intrinsically magical.
  They may require training and experience to improve their natural magical talents, but they had magical capabilities before doing any training.
  This intrinsic magic can come from magical ancestry, unusual birth circumstances, magical experimentation, exposure to powerful magic, simple random chance, or any number of other sources.
  This is the standard explanation for sorcerers.
  In addition, this is the most common explanation for the magical abilities of monsters.

  Some people gain access to magic through personal training or research.
  These people find ways to tap into some pre-existing magical property of the universe and manipulate it at their command.
  This is the standard explanation for monks, rangers with the Beastmaster archetype, rogues with the Bardic Music archetype, and wizards.

  Some people are gifted magic by their association with powerful magical entities or forces.
  They offer worship, allegiance, or their souls, and are granted magical power in exchange.
  This is the standard explanation for clerics, druids, paladins, and votives.

\sectiongraphic*{Personal Power}{width=\columnwidth}{the universe/great deeds}
  The median person in the world of Rise is not particularly more or less capable than the median person in the real world.
  Training can help people improve their skills, but as in the real world, anyone who tries to improve themselves through training and practice eventually reaches an upper limit to their potential.
  However, unlike in the real world, people in Rise can reach beyond their ordinary limitations.
  By defeating powerful foes and performing great deeds that influence the world around them, people can gain levels, which allows them to reach new heights of power.
  At high levels, people can perform clearly superhuman feats that would be impossible for ordinary humans, even without the influence of magic.

  People in Rise wouldn't usually talk about "levels" as a discrete concept ranging from 1 to 21.
  They would perceive the world as a spectrum, and the specific divisions would be more subtle.
  However, they would be aware that some people are fundamentally stronger and more skilled than others.
  Individual scholars or scholastic groups may create their own concepts in-universe to categorize and explain the phenomenon of levels, since the growth of personal power over time is observable and studiable.
  However, those in-universe concepts would never exactly replicate the metagame concept of a level.

  It is common for people in positions of political power to also wield unusually large amounts of personal power.
  High level individuals can be savvier, wiser, and more persuasive than any ordinary human.
  They are more likely than low-level individuals to be able to gain political power through whatever means they see fit, and more likely to maintain their hold on that power.
  In addition, political power can grant further opportunities for performing great deeds, which helps those in power to gain levels and stay ahead of any competition.

  The fastest path to acquiring personal power does not come from pursuing political power.
  It comes from adventuring.
  Adventurers can defeat powerful monsters, help towns in need, and otherwise have a significant personal influence on the world.
  In the process of these adventures, they can amass personal power much more rapidly than ordinary people.
  Of course, adventuring also has an unusually high risk of death.
  Even worse, people who die while adventuring often leave their corpse in the middle of nowhere - in a monster's stomach - which prevents them from being resurrected without incredibly rare magic.
  Adventurers must constantly seek out new challenges to test their limits, or else they will stagnate and stop acquiring personal power, so it is never a sustainable long-term activity.
  There are many people in the world who were adventurers at some point in their past, and everyone is familiar with the concept, but active adventurers are still unusual.

\sectiongraphic*{Deities and Afterlifes}{width=\columnwidth}{the universe/deities and afterlifes}

  When a humanoid creature dies in Rise, they know beyond a shadow of a doubt that they will go to an afterlife.
  Most likely, they know exactly which afterlife they will go to, either as a result of their alignment or their worship of a particular deity.
  In that afterlife, they will live again for as long as they want, though they cannot leave without being magically resurrected.
  People are confident that this is true because deities have told them so, and deities are provably real.
  Also, rare and powerful magic can be used to communicate with people in their afterlife, or even to physically travel to an afterlife plane.

  It is an undisputed fact that Rise is filled with a wide variety of deities of varying power and influence.
  They divinely empower their clerics to act on their behalf.
  Many people know, though some chain of connections, someone who chose to become a cleric and was quickly rewarded with divine magic far beyond anything they could previously do on their own.
  Everyone has heard legends of deities intervening more directly in the world even without a cleric, though these stories are rare and few have experienced them firsthand.

  There are nine distinct afterlife planes, with one plane for each alignment combination.
  Each of those planes is divided into layers.
  Some of those layers are reserved for deities, with major deities claiming layers that are entirely their own and multiple minor deities sharing territory within a single layer.
  The remaining layers have no specific associated deity.
  People can travel between the layers, though the specific mechanisms for traversing layers are different for each afterlife plane.
  Most people do not know this level of detail about afterlife planes, and a commoner would simply be confident that they will go where they belong.

  It is well known that the afterlife planes for evildoers are much harsher than the other afterlife planes.
  The three evil afterlife planes are collectively referred to the Abyss.
  Demons stalk those planes, tormenting evildoers for their own sadistic reasons.
  One of the reasons that some people worship evil deities is to gain a promise of safety, since evil deities protect their worshippers from demonic torment in the afterlife.
  It is also said that demons only torment the weak-willed, and that those who escape demonic torments are free to live in hedonistic luxury.
  There is truth in this, though there are far more people who are confident that they would rule proudly in the Abyss than people who succeed.

  A list of specific well-known deities is given in \trefnp{Deities}.
  The many minor deities worshipped by monsters are not listed here.
  Of course, the GM may use their own custom pantheon.

  % Keep in sync with the cleric list in class.rs
  \begin{dtable!*}
  \lcaption{Deities}
  \begin{dtabularx}{\textwidth}{X l X}
    \tb{Deity} & \tb{Alignment} & \tb{Domains} \tableheaderrule
    Gregory, warrior god of mundanity     & Lawful good     & Law, Protection, Strength, War         \\
    Guftas, horse god of justice          & Lawful good     & Good, Law, Strength, Travel            \\
    Lucied, paladin god of justice        & Lawful good     & Destruction, Good, Protection, War     \\
    Simor, fighter god of protection      & Lawful good     & Good, Protection, Strength, War        \\
    Ayala, naiad god of water             & Neutral good    & Life, Magic, Water, Wild               \\
    Pabs Beerbeard, dwarf god of drink    & Neutral good    & Good, Life, Strength, Wild             \\
    Rucks, monk god of pragmatism         & Neutral good    & Good, Law, Protection, Travel          \\
    Vanya, centaur god of nature          & Neutral good    & Good, Strength, Travel, Wild           \\
    Brushtwig, pixie god of creativity    & Chaotic good    & Chaos, Good, Trickery, Wild            \\
    Camilla, tiefling god of fire         & Chaotic good    & Fire, Good, Magic, Protection          \\
    Chavi, wandering god of stories       & Chaotic good    & Chaos, Knowledge, Trickery             \\
    Chort, dwarf god of optimism          & Chaotic good    & Good, Life, Travel, Wild               \\
    Ivan Ivanovitch, bear god of strength & Chaotic good    & Chaos, Strength, War, Wild             \\
    Krunch, barbarian god of destruction  & Chaotic good    & Destruction, Good, Strength, War       \\
    Sir Cakes, dwarf god of freedom       & Chaotic good    & Chaos, Good, Strength                  \\
    Mikolash, scholar god of knowledge    & Lawful neutral  & Knowledge, Law, Magic, Protection      \\
    Raphael, monk god of retribution      & Lawful neutral  & Death, Law, Protection, Travel         \\
    Declan, god of fire                   & True neutral    & Destruction, Fire, Knowledge, Magic    \\
    Mammon, golem god of endurance        & True neutral    & Knowledge, Magic, Protection, Strength \\
    Kurai, shaman god of nature           & True neutral    & Air, Earth, Fire, Water                \\
    Amanita, druid god of decay           & Chaotic neutral & Chaos, Destruction, Life, Wild         \\
    Antimony, elf god of necromancy       & Chaotic neutral & Death, Knowledge, Life, Magic          \\
    Clockwork, elf god of time            & Chaotic neutral & Chaos, Magic, Trickery, Travel         \\
    Diplo, doll god of destruction        & Chaotic neutral & Chaos, Destruction, Strength, War      \\
    Lord Khallus, fighter god of pride    & Chaotic neutral & Chaos, Strength, War                   \\
    Celeano, sorcerer god of deception    & Chaotic neutral & Chaos, Magic, Protection, Trickery     \\
    Murdoc, god of mercenaries            & Chaotic neutral & Destruction, Knowledge, Travel, War    \\
    Ribo, halfling god of trickery        & Chaotic neutral & Chaos, Trickery, Water                 \\
    Tak, orc god of war                   & Lawful evil     & Law, Strength, Trickery, War           \\
    Theodolus, sorcerer god of ambition   & Neutral evil    & Evil, Knowledge, Magic, Trickery       \\
    Daeghul, demon god of slaughter       & Chaotic evil    & Destruction, Evil, Magic, War          \\
  \end{dtabularx}
  \end{dtable!*}

  \subsection{Affinity}\label{Affinity}
    Affinity works in essentially the same way as gravity, but it is more selective.
    Gravity attracts all matter to all other matter, and its strength is proportional to total mass.
    Affinity attracts similar souls to other similar souls, and its strength is proportional to the strength of the souls.
    It has no effect on things without souls, and it is generally weaker than gravity, but operates at a longer distance.

    While alive, creatures are affected by the pull of affinity, but it is so weak compared to gravity that it is essentially impossible to observe or measure.
    When a creature dies and a soul is severed from its body, the pull of affinity usually becomes the strongest force acting on the untethered soul.
    This pulls the soul towards the Spiritual Planes that it is most connected to.
    It passes through the Astral Expanse on the way, but since it has no physical substance, it is unharmed by the journey.

    The similarities that determine how affinity affects a soul are complex and multifaceted.
    Alignment is a factor, but personal choice is also significant, such as worshipping a particular deity in life.
    The most powerful form of affinity comes from formal, magically enforced ownership, such as a votive who has pledged their soul to a soulkeeper.

    The Spiritual Planes are forged from the souls of eons of living creatures.
    They are incredibly powerful sources of affinity, and they act most powerfully on their most powerful inhabitants - deities.
    In a sense, deities are actually prisoners of their home planes.
    They have virtually unlimited power within that plane, but affinity prevents them from using that power directly elsewhere.
    This is why deities must channel their power through other agents, like angels and clerics.
    The bond of affinity that clerics forge with their deity allows deities to send a small part of their power across the vast distance between the Spiritual Planes and the Material Plane.

\section{Cosmology}\label{Cosmology}
  The universe of Rise can be generally divided into three parts: the Material Plane, the Spiritual Planes, and the Astral Expanse.
  The Material Plane consists of a central sun and a number of planets orbiting that sun.
  Everything in the Material Plane is essentially ordinary matter, and it generally obeys real-world physics.

  The Spiritual Planes are made from \glossterm{essentia}, or soul residue.
  They are shaped by the deities that have lived there since the dawn of civilization.
  When mortal souls die, they travel to an appropriate Spiritual Plane for their afterlife.
  Deities often reside in highly unique territories that may have impossible physical properties such as inverted gravity.
  However, the Spiritual Planes generally function like the real world unless otherwise specified.

  The Astral Expanse is entirely different.
  Unlike the Material Plane and Spiritual Planes, the Astral Expanse has four spacial dimensions.
  The Material Plane and Spiritual Planes are three-dimensional projections within the four-dimensional space of the Astral Expanse.
  Anything entering them is compressed into three dimensions.
  That is why they are called "planes", like a plane is a two-dimensional object within a three-dimensional space.

  It is impossible for mortals to reach the Spiritual Planes by normal means.
  To do so, they would have to move through the fourth spacial dimension.
  Imagine that there are two pieces of paper resting flat on different shelves.
  You could make each paper as long and wide as you want, but they would never intersect each other, because they are parallel.
  To move from one piece of paper to the other, you would have to move in a third dimension: vertically.
  In the same way, you could fly in any direction for any length of time and never reach the Spiritual Plane.

  \subsectiongraphic*{Planar Rifts}{width=\columnwidth}{the universe/planar rifts}\label{Planar Rifts}
    Planar rifts are places where the fabric of a plane has "folded" to directly connect two distant locations within a single plane.
    The two locations on either side of the rift are physically next to each other, so travel through a planar rift is not magical.
    It just involves walking from one side of the rift to the other.

    On their own, planar rifts generally disappear within a few days.
    This time can be extended with the \ritual{stabilize planar rift} ritual.
    Some extraordinary planar rifts last for years without magical intervention.
    The \ritual{forge planar rift} ritual is the only way to create a planar rift, and the planar rifts created in that way are temporary.

  \subsection{Demiplanes}
    In addition to the main three categories, there may also exist smaller planes scattered throughout the Astral Expanse.
    These demiplanes may be compose of ordinary material, \glossterm{essentia}, or both, and they can have widely varied properties.
    Most demiplanes were created for particular purposes by beings of great power, though some simply came into existence through unknown means.

  % TODO: This seems like it belongs in a different section
  \subsection{Soulforged Creatures}
    The Spiritual Planes have native inhabitants, but they are not born in the same way as creatures of the Material Plane.
    % TODO: name lawful/chaotic equivalents
    Angels, demons, devils, and other such creatures are made from the essentia of their corresponding Spiritual Plane.
    They are called soulforged creatures because they are made of essentia, which comes from souls.
    Most such creatures do not have their own souls.
    They are agents of deities and other powerful figures, used to enforce their will.

    When a soulforged is destroyed, its essentia is absorbed by the plane it is on.
    This is one reason that soulforged creatures rarely leave their home plane, since dying on other planes is a net loss of essentia to their creators.
    If a soulforged has a soul and a strong will, the soul travels to its home plane like mortal souls do.
    It can generally make a new body for itself within a week if it died on another plane, or within a day if it died on its home plane.
    Soulforged that die without a soul, or with a soul too weak to reconstitute itself, do not reform.

\section{Realm Traits}

  Specific areas can have unusual traits.
  Effects that change how physics or magic work in a particular area are called realm traits.
  It is common for the divine dominion of deities to have special realm traits chosen by the deity.
  Unusual realms can also exist naturally in any Spiritual Plane.
  Esoteric realm traits are less common on the Material Plane, but some planets or magical areas have unusual traits.

  \subsection{Atmosphere}
    The atmosphere in a realm can take one of the following forms:
    \begin{raggeditemize}
      \item Air: The realm has breathable air suitable for humans.
        This is the default for all locations unless otherwise specified.
      \item Vacuum: The realm has no atmosphere at all.
        At the end of each round, everything in the realm takes 10 cold damage unless it is protected from the cold.
      \item Unbreathable: The realm has an atmosphere, but creatures that need air are unable to breathe it.
    \end{raggeditemize}

  \subsection{Gravity Direction}
    The direction of gravity in a realm can take one of the following forms:
    \begin{raggeditemize}
      \item Massive: Gravity pulls towards object based on their mass, as it does in the real world.
        This is the default for all locations unless otherwise specified.
      \item Fixed: Gravity points in a fixed direction and with a fixed strength at all locations in the realm.
      \item Absolute directional: Gravity points in a consistent direction according to a rule that applies equally to everything on the realm, but which is not in a fixed direction.
        For example, a realm filled with floating spheres where gravity always points towards the closest sphere, but the strength and direction of gravity does not depend on the mass of the spheres, has absolute directional gravity.
      \item Subjective: Each creature in the realm chooses the direction of gravity for that creature.
        The realm has no gravity for unattended objects and mindless creatures.
        You can control gravity in a subjective gravity realm using the \ability{control gravity} ability.
        \begin{activeability}{Control Gravity}{\glossterm{Minor action}}
          \rankline
          You can only use this ability while in a realm with subjective gravity.

          Make a Willpower check with a \glossterm{difficulty value} of 10.
          Success means that you choose the direction of gravity that applies to you in your current realm.
          Alternately, you can choose for gravity to not apply to you.

          Failure means you gain a \plus2 bonus to the next \textit{control gravity} ability you use in your current realm.
          This bonus stacks with itself and lasts until you succeed with this ability.
        \end{activeability}
    \end{raggeditemize}

  \subsection{Gravity Strength} The strength of gravity in a realm can take one of the following forms:
    \begin{raggeditemize}
      \item Normal Gravity: Gravity is about the strength of Earth.
          This is the default for all locations unless otherwise specified.
      \item None: There is no gravity in the realm.
        The weight of all objects is reduced by one \glossterm{weight category}.
        Even while weightless, objects still have mass, so heavy objects can still slow your movements.
        The \glossterm{range limits} of ranged weapons are quadrupled.
        Falling damage is impossible.
        A creature that jumps continues at the same rate indefinitely.
        % TODO: what additional effects are there?
      \item Light: Gravity is about half the strength of Earth.
        The weight of all objects is halved.
        The \glossterm{range limits} of ranged weapons are doubled.
        The horizontal and vertical jump distances for all creatures are doubled.
        Falling damage is halved.
      \item Heavy: Gravity is about twice the strength of Earth.
        The weight of all objects is doubled.
        The range limits of ranged weapons are halved, to a minimum of 5 feet.
        The horizontal and vertical jump distances for all creatures are halved.
        Falling damage is doubled.
        % TODO: falling damage
      \item Extreme: Gravity is about four times the strength of Earth.
        The weight of all objects is increased by one \glossterm{weight category}.
        The range limits of ranged weapons are reduced to one quarter of their normal value, to a minimum of 5 feet.
        The horizontal and vertical jump distances for all creatures are also reduced to one quarter.
        Falling damage is quadrupled.
    \end{raggeditemize}

  \subsection{Light} Various realms are illuminated in different ways.
  Unless otherwise specified, this only determines whether there is an ambient source of light.
  It does not affect the behavior of other sources of light, such as torches or a sun.
  A realm can have multiple simultaneous traits that change its light.
    \begin{raggeditemize}
      \item None: There is no ambient source of light in the realm.
        This is the default for all realms unless otherwise specified.
      \item Fixed Source: There is a single constant source of light in the realm.
      \item Mobile Source: There is a single source of light in the realm that moves around it, illuminating different parts of the realm at different times.
      \item Universal: All parts of the realm, even indoors, are constantly illuminated with either \glossterm{shadowy illumination} or \glossterm{bright illumination}.
        Total darkness is impossible without magic.
      \item Suppressed: All sources of light are downgraded to one step, to a minimum of either total darkness or \glossterm{shadowy illumination}.
        This means \glossterm{bright illumination} becomes shadowy, and \glossterm{brilliant illumination} becomes bright.
    \end{raggeditemize}

  \subsection{Magic} Some realms have strange effects on the use of magic.
  A realm can have multiple simultaneous traits that change its magic.
    \begin{raggeditemize}
      \item Normal: Magic functions normally.
        This is the default for all realms unless otherwise specified.
      \item Suppressed: Whenever a creature in the realm uses a magical ability as a \glossterm{standard action}, \glossterm{minor action}, or \glossterm{elite action}, it must roll 1d10 \add its \glossterm{rank}.
        If the result is less than 10, the ability fails with no effect.
        This does not affect other types of magical abilities, such as passive magical abilities.
      \item Siphoning: Creatures in the realm have no \glossterm{attunement points}, and they immediately deattune from all of their effects as soon as they enter the realm.
      \item Nullified: Creatures in the realm have no magical abilities of any kind.
        Magic items in the realm have no effect and cannot be attuned to.
    \end{raggeditemize}

  \subsection{Temperature} Some realms have unusual temperature effects.
  A realm can have multiple simultaneous traits that change its temperature.
    \begin{raggeditemize}
      \item Normal: Temperature is determined normally.
        This is the default for all realms unless otherwise specified.
      \item Warm: The realm has a minimum ambient temperature that it reaches if left alone, even indoors.
         Specific locations can still be cooled with effort.
      \item Siphoning: The realm siphons heat away.
        Sources of heat, such as fire, still have their normal temperature.
        However, that temperature is lost quickly as it radiates.
        It takes approximately four times the normal amount of heat to warm an area.
        Some realms may siphon heat more or less intensely.
    \end{raggeditemize}

\section{Planets of the Material Plane}\label{Planets of the Material Plane}
  The Material Plane consists of planets orbiting a central sun, named Sol.
  Each planet has its own composition and structure, and many are inhospitable for normal humans.
  It is possible to fly through space between the planets, though space is extremely dangerous, so magical transportation is much more common.
  Each planet is listed below in order of its distance from the sun.

  \subsection{Terra}
    The surface of Terra is almost devoid of air, burning hot during the day, and freezing at night.
    Its variable and harsh conditions make it inhospitable to almost everything.
    Instead, creatures on Terra live underground in its sturdy rock.
    A labyrinthine series of mostly airless tunnels weave their way through the planet.
    Some underground cities even contain air, which is carefully controlled to prevent it from escaping.

    Deep tunnels on the plane can lead to gems, diamonds, and rare metals like mithral.
    However, the dense rock and airless environment make successful mining difficult.
    In addition, earthquakes periodically reshape the environment by collapsing old tunnel systems and constructing new ones.
    The deepest tunnels are blisteringly hot, and terminate in vast oceans of magma.

    Terra is the natural home of earth elementals.

  \subsection{Ignis}
    Ignis is a burning planet with a thick atmosphere.
    It is hot enough to light ordinary wood on fire within minutes.
    Instead of rivers of water, it has rivers of lava, and firestorms frequently illuminate the sky.
    Unlike Terra, the planet's surface is generally a mixture of sand and glass, so tunneling to escape the hostile conditions is infeasible.
    There are a few enclaves that maintain slightly more liveable temperatures through magical means, but they are rare.

    Ignis is the natural home of fire elementals.

  \subsection{Nexus}
    Nexus is an Earth-like planet inhabited by humanoid creatures and covered in a mixture of biomes.
    Almost all fantasy adventures typically start on this world, and many never leave it.
    The name of this planet may change in a particular campaign setting, or there may be multiple separate planets that fit this general description.

    \subsubsection{Luna / Arcadia}
      Luna is a moon of Nexus.
      Although it appears barren and lifeless from Nexus, like the real life moon, that is an ancient magical illusion.
      In reality, Luna is a verdant paradise ruled by fae, who call it Arcadia.
      This is not common knowledge, but it is also not a closely guarded secret.
      A DV 15 Knowledge (local) check is enough to know Luna's true nature.
      Even people who do not know about the illusion generally know that fae are associated with Luna.

  \subsection{Ventus}
    Ventus is a gas giant with powerful winds and frequent storms.
    It does not have an ordinary surface, though the air becomes thicker as you descend into its depths in a way that eventually renders all travel impossible.
    The vast majority of creatures on Ventus can fly, though there are flying cities where landbound creatures from other worlds can reside.

    Ventus is the natural home of air elementals.

  \subsection{Aqua}
    Aqua is a water planet with an ocean that is up to a hundred miles deep.
    Its surface is covered in a thick layer of ice, and creatures adapted to the cold live on its surface.
    The deepest depths of the ocean terminate in a mysterious ice-like substance that seems to exist despite the warmth of the water.
    Towering underwater mountains break up the underwater landscape, though they rarely reach above the ice cap.

    Powerful currents sweep through the ocean, but much of it is calm, and many forms of aquatic life abound in the water.
    Magnificant underwater cities are carved from huge rocks that float peacefully suspended in the water, or are embedded into mountains.
    Though the sun is distant and diffused by the surface ice, the ocean is warmed by extensive underwater vents.
    Simple creatures akin to plankton feed on the energy from those vents and form the base of the ocean's food chain.
    Many of them are bioluminescent

    Aqua's ocean is the natural home of water elementals.

\section{Spiritual Planes}
    The five Spiritual Planes are manifestations of the five alignments.
    Elysium is good-aligned, the Abyss is evil-aligned, Concord is law-aligned, Monody is chaos-aligned, and Haven is neutral-aligned.
    They are made from \glossterm{essentia}, the substance left behind when souls decompose.
    This makes them makes them much more malleable than the Material Plane.
    Essentia responds to the will exerted by creatures with souls.
    This is how deities form their divine dominions, which can have arbitrary and impossible rules and physics.
    Even ordinary creatures, such as the souls of mortals living out their afterlife, can create smaller changes in the world around them.
    As a result, individual regions within each Spiritual Plane can have widely varied traits that do not match that Spiritual Plane as a whole.

    \subsection{Elysium}
      Elysium is a beautiful and majestic world.
      It is shaped like a shallow bowl.
      The concave side is significantly more populated, and it is what most people think of when they imagine Elysium.
      Mountains rise dramatically out of misty clouds, trees are massive and laden with delicious fruit, and buildings surpass the wildest dreams of mortal architects.
      A serene blue sky gives way to a night so lit by stars that it is almost as bright as the day.
      Because of the world's bowl-like structure, you can see the majestic landscape spread out in every direction even from flat ground, and the view from mountains is staggering.

      The underside of Elysium is wreathed in fog and an endless tranquil twilight.
      It is a place for contemplation, serenity, and isolation.

      Mortal souls inhabiting Elysium are guided and assisted by the choirs.
      The choirs refer collectively to all of the seraphim, cherubim, angels, archangels, and other benevolent creatures native to Elysium.

      Elysium has the following nonstandard realm traits:
      \begin{raggeditemize}
        \item Gravity: Absolute directional. At any location on the plane, gravity pulls towards the bowl's core, so creatures can walk normally on the bowl from any location. If you were exactly between the top and bottom of the bowl at any point along its breadth, you would not experience gravity. However, there are no natural tunnels that reach that deep below the surface.
        \item Light: Fixed and mobile sources. The "top" of the plane has a false sun that travels through the sky each day, creating day/night cycles like the Material Plane.
          At night, the stars so plentiful and bright that the world is almost as bright as the day.
          The bottom of the plane has enough stars to generate shadowy illumination.
      \end{raggeditemize}

    \subsection{The Abyss}

      The Abyss is a desolate and brutal plane.
      It consists of cliffs surrounding a deep central pit.
      There is a 20 mile elevation gap between the base of the pit and the peak of the cliffs.
      The highest point of the cliffs, and their back faces, extend out into the Astral Expanse.
      The cliffs are jagged and broken, with many flat areas between elevation changes where creatures survive.
      Cities exist in these flat areas, and are built into the cliffs to extend their usable area and to improve their defenses.

      New mortal souls generally arrive at the bottom of the pit.
      This gives them safety in numbers from demons and other threats, but conditions are crowded and unpleasant, so most mortals climb the cliffs to find better opportunities.
      Higher elevations are generally higher status, though some powerful creatures choose to live closer to the base of the pit.

      With the exception of the great palaces of the Lords of the Abyss, most buildings that exist are designed for defense rather than aesthetics.
      The terrain is typically rocky and dull, and most of the color belongs to carcasses left behind after violent battles.
      All manner of nightmarish creatures stalk the Abyss.
      The best known of these creatures are demons and devils.

      Demons are soulforged creatures that echo the emotions of mortal souls that enter the Abyss.
      They are chaotic creatures that are driven entirely by the primal emotion that birthed them.
      This makes them functionally insane, and they are almost always driven to lash out at everything around them.
      Rage-born demons violently attack anything they see, pain-born demons try to lessen their pain by sharing it, and so on.
      Some demons are naturally spawned in reaction to mortal souls, but others are intentionally created and directed by evil deities.

      Demons generally avoid attacking other demons, since they can find no gratification for their urges in attacking soulless creatures.
      Instead, they hunt creatures with complete souls, which generally means attacking the afterlife bodies of evil-aligned creatures who went to the Abyss for their afterlife.
      If they have access to ordinary mortal, they eagerly hunt them, as they are generally easier prey.
      Demonic incursions into other planes are devastating but fortunately rare.

      Unlike demons, devils are lawful creatures with a functioning society.
      Devil-run cities are isolated enclaves in the general chaos of the Abyss, and many mortal souls choose to live there.
      The laws there are not necessarily fair or ethical, but they are scrupulously followed.
      Devils are far less numerous than demons, since they cannot simply spawn accidentally like demons do.
      Instead, every devil must be intentionally created.

      Since most devils do not have souls, this requires the intervention of creatures with souls.
      This generally means that devils must make deals with mortals or deities.
      They do that when they can, since the more servants a devil has, the greater their power.
      The greatest prize for a devil is to gain mastery over a mortal soul.
      They can use its ability to manipulate essentia to create powerful servants or reshape their home in the Abyss to their liking.
      This makes powerful devils common soulkeepers, and they seek out votives wherever possible.

      Some powerful devils leave the devil-run cities to establish their own territory.
      This generally requires that they must already have control of multiple mortal souls, or their territory could be reshaped by any passing mortal.
      A devil that can establish its own territory gains the title of Lord of the Abyss, and it is a title that almost all devils desperately crave.

      The Abyss has the following nonstandard realm traits:
      \begin{raggeditemize}
        \item Gravity: Fixed directional. Gravity pulls towards the side of the plane with the deep pit.
        \item Light: Fixed. A blood-red orb hovers directly over the central pit above the highest cliffs, casting bright illumination throughout the plane.
      \end{raggeditemize}

    \subsection{Concord}
      Concord is a hollow cube that is divided into a myriad of different levels and rooms.
      The cube is made from hardened \glossterm{essentia} that functions like pure adamantine, except that it is self-healing when damaged and cannot be removed from the plane.
      Although it can bend to exceptionally strong-willed souls, it returns to its original shape and structure when not actively controlled.
      The boundaries of the cube are precisely aligned with the extent of the plane itself, so anything standing on the outside of the cube would be in the Astral Expanse.
      Each level and room is organized in a consistent way, making navigation easy once you understand the system.
      The center of each level is a transit hub containing stairs, mechanical lifts, and magical portals to make travel easier.

      Each level has its own rules, which are clearly explained in every language at that level's transit hub.
      Individual rooms often have their own additional rules, which may or may not be clearly communicated.
      They can be as large as countries or as small as closets.
      Large rooms often contain natural elements, such as forests or lakes, with ecosystems that are carefully kept in balance.
      Corridors run through the levels to facilitate travel between rooms, except for rooms which are intentionally isolated.
      Although Concord itself has no intrinsic light source, its corridors are generally well-lit by magical lights, and rooms often have their own lighting.

      Concord is administered and inhabited by the moirai, who are soulforged native to that plane.
      Each moirai is an impartial arbiter and embodiment of a specific concept, directive, or material fact.
      Lesser moirai represent extremely specific or obscure concepts, while greater moirai represent universal truths or fundamental forces.
      Constructs called telos enforce the decisions of moirai and the rules of Concord.

      Concord has the following nonstandard realm traits:
      \begin{raggeditemize}
        \item Gravity: Fixed directional. Gravity pulls towards the bottom of the cube.
      \end{raggeditemize}

  \subsection{Monody}
    Monody is a constantly shifting plane.
    Even its core shape contorts and changes over time.
    It could be a hollow sphere where everyone lives on the inside, a torus twisting around itself, a collection of small spheres, or anything else.

    The \glossterm{essentia} comprising all Spiritual Planes is malleable to some degree, but Monody is the easiest to control.
    Even relatively weak-willed mortals can still bend the landscape around them to suit their desires.
    The real challenge is in making your changes permanent.
    Beyond actively shaped spaces, the terrain is constantly changing.
    In a matter of hours, a field might grow trees that are consumed by a forest fire before vast chasms open up and consume their charred remains.

    The native inhabitants of Monody are called proteans.
    They are amorphous creatures that can freely reshape their bodies.
    Individual proteans tend to have forms they prefer to use.
    Proteans can feel mortals attempting to change them in the same way that mortals can shape inanimate essentia, though they can ignore those commands.
    Sometimes proteans choose to act like terrain for a while as part of their own games.

    Fae are not native to Monody, but it is not uncommon to find them there.
    They enjoy its malleable nature, and its constant change helps fight their tendency to become bored.

    Monody has the following nonstandard realm traits:
    \begin{raggeditemize}
      \item Gravity: Subjective. The difficulty value to change your gravity on Monody is 0 instead of the normal 10.
      \item Light: Mobile source. The number of lights, and their color and direction, are highly variable.
        They usually provide \glossterm{bright illumination}, and less commonly provide shadowy or brilliant illumination.
    \end{raggeditemize}

  \subsection{Haven}
    Haven is a stable spherical planet.
    Its basic structure resembles Nexus, though it is smaller and has no moon.
    It has a wide variety of biomes, and is generally peaceful and has a place for anyone.

    Unlike the other Spiritual Planes, Haven has no native inhabitants.
    However, it has more mortal souls than any other Spiritual Plane, so there are still cities for those who prefer the company of others.

    Monody has the following nonstandard realm traits:
    \begin{raggeditemize}
      \item Light: Mobile source. A glowing orb orbits the planet, mimicking the sun as viewed from Haven.
    \end{raggeditemize}

\section{The Astral Expanse}

  The true shape of the Astral Expanse is beyond a mortal mind's ability to comprehend.
  It has empty fields of space interspersed with objects and terrain of widely varying sizes, ranging from small rocks to planets.
  Some of the terrain seems to be natural, while other structures seem to be carved with skill.
  However, everything there has strange, unnatural angles that boggle the mind.
  A collection of separate sculptures may all move together, revealing that they are somehow the same object.
  Small objects can have immense mass, and large objects can be strangely light.
  Tiny objects floating in midair that are almost too small to see can be sharp and immovable enough to cut anything that passes into them.
  Objects can fold in on themselves, disappearing entirely, or rapidly grow out of nowhere.
  There is no safe place in the Astral Expanse for a three-dimensional creature, even ignoring the threat of intentional attack.

  There are four-dimensional creatures native to the Astral Expanse, but they are horrifying and utterly incomprehensible.
  They are called the precursors, because they predate the three-dimensional planes that humanoids inhabit.
  Precursors drift or fly around the Astral Expanse hunting for food, making travel through the Astral Expanse extremely dangerous.

  As four-dimensional creatures, the precursors are essentially invincible in the Astral Expanse.
  Fighting a precursor is like the drawings on a piece of paper trying to do battle against an artist holding a pencil.
  At most, the drawings could only ever interact with the tip of the pencil touching the paper.
  Even if they destroyed the tip, they could never understand or harm the artist themselves.
  Meanwhile, the pencil can freely draw \textit{inside} the drawings without "passing through" their outer edge.
  No matter how reinforced the drawing's armor might be, it is irrelevant since the pencil could tap directly on their eyes or heart.
  Deities are sufficiently strong and uniform in composition to survive such internal attacks, but even they can only survive, not effectively counterattack.

  The precursors are unable to directly enter or interact with the Material and Spiritual Planes.
  The flattening effect of the planes is lethal to their four-dimensional bodies, just like compressing a person into two dimensions would be lethal.
  As a result, they generally give the planes a wide berth.
  Nonliving objects from the Astral Expanse occasionally collide with the Material Plane.
  Typically, they appear in empty space as bizarre and misshapen asteroids, and they are either never seen or they burn up upon entering a planet's atmosphere, making this phenomenon essentially unnoticeable.
  On extremely rare occasions, they may arrive close to a planet's surface, creating inexplicable eldritch ruins on the ground or in caves.
  These become a hotbed of precursor cult activity if they are ever discovrered.

  Aberrations are the closest things to precursors on the Material Plane.
  They arrived fully formed on Nexus within the past thousand years, and their bodies seem to operate fundamentally differently than other life.
  Some fear that aberrations are an attempt by precursors to invade the Material Plane, though how this would be accomplished on a larger scale remains unclear.

  The Astral Expanse holds ancient monstrosities of cosmic size and power.
  Fortunately, they live far away from the Material and Spiritual Planes.
  In the same way that deities have difficulty leaving their Spiritual Plane, the cosmic force of affinity makes it difficult for precursor leviathans to leave their distant homes.
  Instead, they reside in a distant region of the Astral Expanse called the Eternal Void.
  They can still reach out through intermediaries and cultists, as deities do.
  On rare occasions, a precursor leviathan may try to more directly interfere with the Planes.
  That would be an interplanar crisis which even deities fear.
