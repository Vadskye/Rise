\chapter{The World}

Rise does not attempt to define a single geography with specific countries and locations that is shared between all games.
It is common for GMs to define their own setting when running a game, and that freedom is important.
However, the universe of Rise does differ in a number of important ways from the real world.
The fundamental assumptions that Rise makes about the its fantasy world are:

\begin{raggeditemize}
  \item Magic is commonly known but rarely understood.
  \item Elite people and monsters have vastly more physical and magical personal power than normal people.
  \item When you die, your soul goes to an afterlife where deities exist.
  \item Spiritually similar things are attracted to each other by a fundamental force called "affinity"
\end{raggeditemize}

These core elements are ambiguous about some details, and GMs are encouraged to fill in those details as they see fit.
Of course, a GM has absolute power, and can create a world that changes any number of these assumptions.
However, doing so can significantly change the tone of the game and create logical inconsistencies, so it should be done carefully.

\section{Magic}
  The world of Rise is a magical place.
  Many people are capable of using magic to perform feats that would be impossible in the real world.
  Not everyone is capable of magic, of course.
  It's reasonable to assume that about ten percent of the civilized people in the world have a magical ability of some kind.
  In some societies, such as a feudal human-dominated society with a large number of commoners and serfs, the percentage of people with magic can be lower.
  However, this is balanced by the existence of other societies that tend to be more magical, such as elven communes in deep forests.
  Even in low-magic societies, everyone knows that magic exists, and almost everyone has observed something magical at some point in their lives.

  People can have magical abilities for a wide variety of reasons.
  There are three main categories to explain why people can access magic: intrinsic magic, learned magic, and gifted magic.
  Each class with magical abilities belongs to one of these groups.
  Characters with magical feats are free to choose any of those three explanations for their feats.
  The explanation does not have to be the same as for any other magical abilities they possess.
  For example, a cleric may be gifted their magical cleric abilities because they worship a particular deity, but they may also be naturally telepathic.

  Some people are simply intrinsically magical.
  They may require training and experience to improve their natural magical talents, but they had magical capabilities before doing any training.
  This intrinsic magic can come from magical ancestry, unusual birth circumstances, magical experimentation, exposure to powerful magic, simple random chance, or any number of other sources.
  This is the standard explanation for sorcerers.
  In addition, this is the most common explanation for the magical abilities of monsters.

  Some people gain access to magic through personal training or research.
  These people find ways to tap into some pre-existing magical property of the universe and manipulate it at their command.
  This is the standard explanation for monks, rangers with the Beastmaster archetype, rogues with the Bardic Music archetype, and wizards.

  Some people are gifted magic by their association with powerful magical entities or forces.
  They offer worship, allegiance, or their souls, and are granted magical power in exchange.
  This is the standard explanation for clerics, druids, paladins, and votives.

\sectiongraphic*{Personal Power}{width=\columnwidth}{the universe/great deeds}
  The median person in the world of Rise is not particularly more or less capable than the median person in the real world.
  Training can help people improve their skills, but as in the real world, anyone who tries to improve themselves through training and practice eventually reaches an upper limit to their potential.
  However, unlike in the real world, people in Rise can reach beyond their ordinary limitations.
  By defeating powerful foes and performing great deeds that influence the world around them, people can gain levels, which allows them to reach new heights of power.
  At high levels, people can perform clearly superhuman feats that would be impossible for ordinary humans, even without the influence of magic.

  People in Rise wouldn't usually talk about "levels" as a discrete concept ranging from 1 to 21.
  They would perceive the world as a spectrum, and the specific divisions would be more subtle.
  However, they would be aware that some people are fundamentally stronger and more skilled than others.
  Individual scholars or scholastic groups may create their own concepts in-universe to categorize and explain the phenomenon of levels, since the growth of personal power over time is observable and studiable.
  However, those in-universe concepts would never exactly replicate the metagame concept of a level.

  It is common for people in positions of political power to also wield unusually large amounts of personal power.
  High level individuals can be savvier, wiser, and more persuasive than any ordinary human.
  They are more likely than low-level individuals to be able to gain political power through whatever means they see fit, and more likely to maintain their hold on that power.
  In addition, political power can grant further opportunities for performing great deeds, which helps those in power to gain levels and stay ahead of any competition.

  The fastest path to acquiring personal power does not come from pursuing political power.
  It comes from adventuring.
  Adventurers can defeat powerful monsters, help towns in need, and otherwise have a significant personal influence on the world.
  In the process of these adventures, they can amass personal power much more rapidly than ordinary people.
  Of course, adventuring also has an unusually high risk of death.
  Even worse, people who die while adventuring often leave their corpse in the middle of nowhere - in a monster's stomach - which prevents them from being resurrected without incredibly rare magic.
  Adventurers must constantly seek out new challenges to test their limits, or else they will stagnate and stop acquiring personal power, so it is never a sustainable long-term activity.
  There are many people in the world who were adventurers at some point in their past, and everyone is familiar with the concept, but active adventurers are still unusual.

\sectiongraphic*{Deities and Afterlifes}{width=\columnwidth}{the universe/deities and afterlifes}

  When a humanoid creature dies in Rise, they know beyond a shadow of a doubt that they will go to an afterlife.
  Most likely, they know exactly which afterlife they will go to, either as a result of their alignment or their worship of a particular deity.
  In that afterlife, they will live again for as long as they want, though they cannot leave without being magically resurrected.
  People are confident that this is true because deities have told them so, and deities are provably real.
  Also, rare and powerful magic can be used to communicate with people in their afterlife, or even to physically travel to an afterlife plane.

  It is an undisputed fact that Rise is filled with a wide variety of deities of varying power and influence.
  They divinely empower their clerics to act on their behalf.
  Many people know, though some chain of connections, someone who chose to become a cleric and was quickly rewarded with divine magic far beyond anything they could previously do on their own.
  Everyone has heard legends of deities intervening more directly in the world even without a cleric, though these stories are rare and few have experienced them firsthand.

  There are nine distinct afterlife planes, with one plane for each alignment combination.
  Each of those planes is divided into layers.
  Some of those layers are reserved for deities, with major deities claiming layers that are entirely their own and multiple minor deities sharing territory within a single layer.
  The remaining layers have no specific associated deity.
  People can travel between the layers, though the specific mechanisms for traversing layers are different for each afterlife plane.
  Most people do not know this level of detail about afterlife planes, and a commoner would simply be confident that they will go where they belong.

  It is well known that the afterlife planes for evildoers are much harsher than the other afterlife planes.
  The three evil afterlife planes are collectively referred to the Abyss.
  Demons stalk those planes, tormenting evildoers for their own sadistic reasons.
  One of the reasons that some people worship evil deities is to gain a promise of safety, since evil deities protect their worshippers from demonic torment in the afterlife.
  It is also said that demons only torment the weak-willed, and that those who escape demonic torments are free to live in hedonistic luxury.
  There is truth in this, though there are far more people who are confident that they would rule proudly in the Abyss than people who succeed.

  A list of specific well-known deities is given in \trefnp{Deities}.
  The many minor deities worshipped by monsters are not listed here.
  Of course, the GM may use their own custom pantheon.

  % Keep in sync with the cleric list in class.rs
  \begin{dtable!*}
  \lcaption{Deities}
  \begin{dtabularx}{\textwidth}{X l X}
    \tb{Deity} & \tb{Alignment} & \tb{Domains} \tableheaderrule
    Gregory, warrior god of mundanity     & Lawful good     & Law, Protection, Strength, War         \\
    Guftas, horse god of justice          & Lawful good     & Good, Law, Strength, Travel            \\
    Lucied, paladin god of justice        & Lawful good     & Destruction, Good, Protection, War     \\
    Simor, fighter god of protection      & Lawful good     & Good, Protection, Strength, War        \\
    Ayala, naiad god of water             & Neutral good    & Life, Magic, Water, Wild               \\
    Pabs Beerbeard, dwarf god of drink    & Neutral good    & Good, Life, Strength, Wild             \\
    Rucks, monk god of pragmatism         & Neutral good    & Good, Law, Protection, Travel          \\
    Vanya, centaur god of nature          & Neutral good    & Good, Strength, Travel, Wild           \\
    Brushtwig, pixie god of creativity    & Chaotic good    & Chaos, Good, Trickery, Wild            \\
    Camilla, tiefling god of fire         & Chaotic good    & Fire, Good, Magic, Protection          \\
    Chavi, wandering god of stories       & Chaotic good    & Chaos, Knowledge, Trickery             \\
    Chort, dwarf god of optimism          & Chaotic good    & Good, Life, Travel, Wild               \\
    Ivan Ivanovitch, bear god of strength & Chaotic good    & Chaos, Strength, War, Wild             \\
    Krunch, barbarian god of destruction  & Chaotic good    & Destruction, Good, Strength, War       \\
    Sir Cakes, dwarf god of freedom       & Chaotic good    & Chaos, Good, Strength                  \\
    Mikolash, scholar god of knowledge    & Lawful neutral  & Knowledge, Law, Magic, Protection      \\
    Raphael, monk god of retribution      & Lawful neutral  & Death, Law, Protection, Travel         \\
    Declan, god of fire                   & True neutral    & Destruction, Fire, Knowledge, Magic    \\
    Mammon, golem god of endurance        & True neutral    & Knowledge, Magic, Protection, Strength \\
    Kurai, shaman god of nature           & True neutral    & Air, Earth, Fire, Water                \\
    Amanita, druid god of decay           & Chaotic neutral & Chaos, Destruction, Life, Wild         \\
    Antimony, elf god of necromancy       & Chaotic neutral & Death, Knowledge, Life, Magic          \\
    Clockwork, elf god of time            & Chaotic neutral & Chaos, Magic, Trickery, Travel         \\
    Diplo, doll god of destruction        & Chaotic neutral & Chaos, Destruction, Strength, War      \\
    Lord Khallus, fighter god of pride    & Chaotic neutral & Chaos, Strength, War                   \\
    Celeano, sorcerer god of deception    & Chaotic neutral & Chaos, Magic, Protection, Trickery     \\
    Murdoc, god of mercenaries            & Chaotic neutral & Destruction, Knowledge, Travel, War    \\
    Ribo, halfling god of trickery        & Chaotic neutral & Chaos, Trickery, Water                 \\
    Tak, orc god of war                   & Lawful evil     & Law, Strength, Trickery, War           \\
    Theodolus, sorcerer god of ambition   & Neutral evil    & Evil, Knowledge, Magic, Trickery       \\
    Daeghul, demon god of slaughter       & Chaotic evil    & Destruction, Evil, Magic, War          \\
  \end{dtabularx}
  \end{dtable!*}

  \subsection{Affinity}\label{Affinity}
    Affinity works in essentially the same way as gravity, but it is more selective.
    Gravity attracts all matter to all other matter, and its strength is proportional to total mass.
    Affinity attracts similar souls to other similar souls, and its strength is proportional to the strength of the souls.
    It has no effect on things without souls, and it is generally weaker than gravity, but operates at a longer distance.

    While alive, creatures are affected by the pull of affinity, but it is so weak compared to gravity that it is essentially impossible to observe or measure.
    When a creature dies and a soul is severed from its body, the pull of affinity usually becomes the strongest force acting on the untethered soul.
    This pulls the soul towards the Spiritual Planes that it is most connected to.
    It passes through the Astral Expanse on the way, but since it has no physical substance, it is unharmed by the journey.

    The similarities that determine how affinity affects a soul are complex and multifaceted.
    Alignment is a factor, but personal choice is also significant, such as worshipping a particular deity in life.
    The most powerful form of affinity comes from formal, magically enforced ownership, such as a votive who has pledged their soul to a soulkeeper.

    The Spiritual Planes are forged from the souls of eons of living creatures.
    They are incredibly powerful sources of affinity, and they act most powerfully on their most powerful inhabitants - deities.
    In a sense, deities are actually prisoners of their home planes.
    They have virtually unlimited power within that plane, but affinity prevents them from using that power directly elsewhere.
    This is why deities must channel their power through other agents, like angels and clerics.
    The bond of affinity that clerics forge with their deity allows deities to send a small part of their power across the vast distance between the Spiritual Planes and the Material Plane.

\section{Cosmology}\label{Cosmology}
  The universe of Rise can be generally divided into three parts: the Material Plane, the Spiritual Planes, and the Astral Expanse.
  The Material Plane consists of a central sun and a number of planets orbiting that sun.
  Everything in the Material Plane is essentially ordinary matter, and it generally obeys real-world physics.

  The Spiritual Planes are made from \glossterm{essentia}, or soul residue.
  They are shaped by the deities that have lived there since the dawn of civilization.
  When mortal souls die, they travel to an appropriate Spiritual Plane for their afterlife.
  Deities often reside in highly unique territories that may have impossible physical properties such as inverted gravity.
  However, the Spiritual Planes generally function like the real world unless otherwise specified.

  The Astral Expanse is entirely different.
  Unlike the Material Plane and Spiritual Planes, the Astral Expanse has four spacial dimensions.
  The Material Plane and Spiritual Planes are three-dimensional projections within the four-dimensional space of the Astral Expanse.
  Anything entering them is compressed into three dimensions.
  That is why they are called "planes", like a plane is a two-dimensional object within a three-dimensional space.

  \subsectiongraphic*{Planar Rifts}{width=\columnwidth}{the universe/planar rifts}\label{Planar Rifts}
    Planar rifts are places where the fabric of a plane has "folded" to directly connect two distant locations within a single plane.
    The two locations on either side of the rift are physically next to each other, so travel through a planar rift is not magical.
    It just involves walking from one side of the rift to the other.

    On their own, planar rifts generally disappear within a few days.
    This time can be extended with the \ritual{stabilize planar rift} ritual.
    Some extraordinary planar rifts last for years without magical intervention.
    The \ritual{forge planar rift} ritual is the only way to create a planar rift, and the planar rifts created in that way are temporary.

  \subsection{Demiplanes}
    In addition to the main three categories, there may also exist smaller planes scattered throughout the Astral Expanse.
    These demiplanes may be compose of ordinary material, \glossterm{essentia}, or both, and they can have widely varied properties.
    Most demiplanes were created for particular purposes by beings of great power, though some simply came into existence through unknown means.

\section{Planets of the Material Plane}\label{Planets of the Material Plane}
  The Material Plane consists of planets orbiting a central sun, named Sol.
  Each planet has its own composition and structure, and many are inhospitable for normal humans.
  It is possible to fly through space between the planets, though space is extremely dangerous, so magical transportation is much more common.
  Each planet is listed below in order of its distance from the sun.

  \subsection{Ignis}
    Ignis is a burning planet with a thick atmosphere.
    It is hot enough to light ordinary wood on fire within minutes.
    Instead of rivers of water, it has rivers of lava, and firestorms frequently illuminate the sky.
    It is the natural home of fire elementals.

  \subsection{Terra}
    The surface of Terra is almost devoid of air, burning hot during the day, and freezing at night.
    Its variable and harsh conditions make it inhospitable to almost everything.
    Instead, creatures on Terra live underground in vast tunnels and caverns.
    It is the natural home of earth elementals.

  \subsection{Nexus}
    Nexus is an Earth-like planet inhabited by humanoid creatures and covered in a mixture of biomes.
    Almost all fantasy adventures typically start on this world, and many never leave it.
    The name of this planet may change in a particular campaign setting, or there may be multiple separate planets that fit this general description.

    \subsubsection{Luna / Arcadia}
      Luna is a moon of Nexus.
      Although it appears barren and lifeless from Nexus, like the real life moon, that is an ancient magical illusion.
      In reality, Luna is a verdant paradise ruled by fae, who call it Arcadia.
      This is not common knowledge, but it is not a closely guarded secret.
      A DV 15 Knowledge (local) check is enough to know the truth of Luna.
      Even people who do not know about the illusion generally know that fae are associated with Luna.

  \subsection{Ventus}
    Ventus is a gas giant with powerful winds and frequent storms.
    It does not have an ordinary surface, though the air becomes thicker as you descend into its depths in a way that eventually renders all travel impossible.
    There are flying cities where landbound creatures from other worlds can reside, but almost all creatures on Ventus can fly.
    It is the natural home of air elementals.

  \subsection{Aqua}
    Aqua is a water planet with an ocean that is a hundred miles deep.
    Its surface is covered in a thick layer of ice, and creatures adapted to the cold live on its surface.
    The bottom of the ocean is composed of a mysterious ice-like substance that seems to exist despite the warmth of the water.
    The ocean is the natural home of water elementals.

\section{Spiritual Planes}
    The five Spiritual Planes are manifestations of the five alignments.
    Elysium is good-aligned, the Abyss is evil-aligned, Ordus is law-aligned, Discord is chaos-aligned, and the Expanse is neutral-aligned.
    They are made from \glossterm{essentia}, the substance left behind when souls decompose.
    This makes them makes them much more malleable than the Material Plane.
    Essentia responds to the will exerted by creatures with souls.
    This is how deities form their divine dominions, which can have arbitrary and impossible rules and physics.
    Even ordinary creatures, such as the souls of mortals living out their afterlife, can create smaller changes in the world around them.

  \subsection{Soulforged Creatures}
    The Spiritual Planes have native inhabitants, but they are not born in the same way as creatures of the Material Plane.
    % TODO: name lawful/chaotic equivalents
    Angels, demons, devils, and other such creatures are made from the essentia of their corresponding Spiritual Plane.
    They are called soulforged creatures because they are made of essentia, which comes from souls.
    Most such creatures do not have their own souls.
    They are agents of deities and other powerful figures, used to enforce their will.

    When a soulforged is destroyed, its essentia is absorbed by the plane it is on.
    This is one reason that soulforged creatures rarely leave their home plane, since dying on other planes is a net loss of essentia to their creators.
    If a soulforged has a soul and a strong will, the soul travels to its home plane like mortal souls do.
    It can generally make a new body for itself within a week if it died on another plane, or within a day if it died on its home plane.
    Soulforged that die without a soul, or with a soul too weak to reconstitute itself, do not reform.

\section{The Astral Expanse}

  The true shape of the Astral Expanse is beyond a mortal mind's ability to comprehend.
  It has empty fields of space interspersed with objects and terrain of widely varying sizes, ranging from small rocks to planets.
  Some of the terrain seems to be natural, while other structures seem to be carved with skill.
  However, everything there has strange, unnatural angles that boggle the mind.
  A collection of separate sculptures may all move together, revealing that they are somehow the same object.
  Small objects can have immense mass, and large objects can be strangely light.
  Tiny objects floating in midair that are almost too small to see can be sharp and immovable enough to cut anything that passes into them.
  Objects can fold in on themselves, disappearing entirely, or rapidly grow out of nowhere.
  There is no safe place in the Astral Expanse for a three-dimensional creature, even ignoring the threat of intentional attack.

  There are four-dimensional creatures native to the Astral Expanse, but they are horrifying and utterly incomprehensible.
  They are called the precursors, because they predate the three-dimensional planes that humanoids inhabit.
  Precursors drift or fly around the Astral Expanse hunting for food, making travel through the Astral Expanse extremely dangerous.

  As four-dimensional creatures, the precursors are essentially invincible in the Astral Expanse.
  Fighting a precursor is like the drawings on a piece of paper trying to do battle against an artist holding a pencil.
  At most, the drawings could only ever interact with the tip of the pencil touching the paper.
  Even if they destroyed the tip, they could never understand or harm the artist themselves.
  Meanwhile, the pencil can freely draw \textit{inside} the drawings without "passing through" their outer edge.
  No matter how reinforced the drawing's armor might be, it is irrelevant since the pencil could tap directly on their eyes or heart.
  Deities are sufficiently strong and uniform in composition to survive such internal attacks, but even they can only survive, not effectively counterattack.

  The Astral Expanse holds ancient monstrosities of cosmic size and power.
  Fortunately, they live far away from the Material and Spiritual Planes.
  In the same way that deities have difficulty leaving their Spiritual Plane, the cosmic force of affinity makes it difficult for precursor leviathans to leave their distant homes.
  Instead, they reside in a distant region of the Astral Expanse called the Eternal Void.
  They can still reach out through intermediaries and cultists, as deities do.
  On rare occasions, a precursor leviathan may try to more directly interfere with the Planes.
  That would be an interplanar crisis which even deities fear.
