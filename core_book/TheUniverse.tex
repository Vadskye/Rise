\chapter{The World}

Rise does not attempt to define a single geography with specific countries and locations that is shared between all games.
It is common for GMs to define their own setting when running a game, and that freedom is important.
However, the universe of Rise does differ in a number of important ways from the real world.
The fundamental assumptions that Rise makes about the its fantasy world are:

\begin{itemize}
  \item Magic is commonly known but rarely understood
  \item Some people are vastly more powerful than others
  \item Souls, deities, and the afterlife are real
\end{itemize}

These core elements are ambiguous about some details, and GMs are encouraged to fill in those details as they see fit.
Of course, a GM has absolute power, and can create a world that changes any number of these assumptions.
However, doing so can significantly change the tone of the game and create logical inconsistencies, so it should be done carefully.

\section{Magic}
  The world of Rise is a magical place.
  Many people are capable of using magic to perform feats that would be impossible in the real world.
  Not everyone is capable of magic, of course.
  It's reasonable to assume that about ten percent of the civilized people in the world have a magical ability of some kind.
  In some societies, such as a feudal human-dominated society with a large number of commoners and serfs, the percentage of people with magic can be lower.
  However, this is balanced by the existence of other societies that tend to be more magical, such as societies ruled by gnomes and elves.
  Even in low-magic societies, everyone knows that magic exists, and almost everyone has observed or been personally affected by magic at some point in their lives.

  People can have magical abilities for a wide variety of reasons.
  There are three main categories to explain why people can access magic: intrinsic magic, learned magic, and gifted magic.
  Each class with magical abilities belongs to one of these groups.
  Characters with magical feats are free to choose any of those three explanations for their feats.
  The explanation does not have to be the same as for any other magical abilities they possess.
  For example, a cleric may be gifted their magical cleric abilities because they worship a particular deity, but they may also be naturally telepathic.

  Some people are simply intrinsically magical.
  They may require training and experience to improve their natural magical talents, but they had magical capabilities before doing any training.
  This intrinsic magic can come from magical ancestry, unusual birth circumstances, magical experimentation, exposure to powerful magic, simple random chance, or any number of other sources.
  This is the standard explanation for sorcerers.
  In addition, this is the most common explanation for the magical abilities of monsters.

  Some people gain access to magic through personal training or research.
  These people find ways to tap into some pre-existing magical property of the universe and manipulate it at their command.
  This is the standard explanation for monks, rangers with the Beastmaster archetype, rogues with the Bardic Music archetype, and wizards.

  Some people are gifted magic by their association with powerful magical entities or forces.
  They offer worship, allegiance, or their souls, and are granted magical power in exchange.
  This is the standard explanation for clerics, druids, paladins, and votives.

\sectiongraphic*{Personal Power}{width=\columnwidth}{the universe/great deeds}
  The median person in the world of Rise is not particularly more or less capable than the median person in the real world.
  Training can help people improve their skills, but as in the real world, anyone who tries to improve themselves through training and practice eventually reaches an upper limit to their potential.
  However, unlike in the real world, people in Rise can reach beyond their ordinary limitations.
  By defeating powerful foes and performing great deeds that influence the world around them, people can gain levels, which allows them to reach new heights of power.
  At high levels, people can perform clearly superhuman feats that would be impossible for ordinary humans, even without the influence of magic.

  People in Rise wouldn't usually talk about "levels" as a discrete concept ranging from 1 to 21.
  They would perceive the world as a spectrum, and the specific divisions would be more subtle.
  However, they would be aware that some people are fundamentally stronger and more skilled than others.
  Individual scholars or scholastic groups may create their own concepts in-universe to categorize and explain the phenomenon of levels, since the growth of personal power over time is observable and studiable.
  However, those in-universe concepts would never exactly replicate the metagame concept of a level.

  It is common for people in positions of political power to also wield unusually large amounts of personal power.
  High level individuals can be savvier, wiser, and more persuasive than any ordinary human.
  They are more likely than low-level individuals to be able to gain political power through whatever means they see fit, and more likely to maintain their hold on that power.
  In addition, political power can grant further opportunities for performing great deeds, which helps those in power to gain levels and stay ahead of any competition.

  The fastest path to acquiring personal power does not come from pursuing political power.
  It comes from adventuring.
  Adventurers can defeat powerful monsters, help towns in need, and otherwise have a significant personal influence on the world.
  In the process of these adventures, they can amass personal power much more rapidly than ordinary people.
  Of course, adventuring also has an unusually high risk of death.
  Even worse, people who die while adventuring often leave their corpse in the middle of nowhere - in a monster's stomach - which prevents them from being resurrected without incredibly rare magic.
  Adventurers must constantly seek out new challenges to test their limits, or else they will stagnate and stop acquiring personal power, so it is never a sustainable long-term activity.
  There are many people in the world who were adventurers at some point in their past, and everyone is familiar with the concept, but active adventurers are still unusual.

\sectiongraphic*{Deities and Afterlifes}{width=\columnwidth}{the universe/deities and afterlifes}

  When a humanoid creature dies in Rise, they know beyond a shadow of a doubt that they will go to an afterlife.
  Most likely, they know exactly which afterlife they will go to, either as a result of their alignment or their worship of a particular deity.
  In that afterlife, they will live again for as long as they want, though they cannot leave without being magically resurrected.
  People are confident that this is true because deities have told them so, and deities are provably real.
  Also, rare and powerful magic can be used to communicate with people in their afterlife, or even to physically travel to an afterlife plane.

  It is an undisputed fact that Rise is filled with a wide variety of deities of varying power and influence.
  They divinely empower their clerics to act on their behalf.
  Many people know, though some chain of connections, someone who chose to become a cleric and was quickly rewarded with divine magic far beyond anything they could previously do on their own.
  Everyone has heard legends of deities intervening more directly in the world even without a cleric, though these stories are rare and few have experienced them firsthand.

  There are nine distinct afterlife planes, with one plane for each alignment combination.
  Each of those planes is divided into layers.
  Some of those layers are reserved for deities, with major deities claiming layers that are entirely their own and multiple minor deities sharing territory within a single layer.
  The remaining layers have no specific associated deity.
  People can travel between the layers, though the specific mechanisms for traversing layers are different for each afterlife plane.
  Most people do not know this level of detail about afterlife planes, and a commoner would simply be confident that they will go where they belong.

  It is well known that the afterlife planes for evildoers are much harsher than the other afterlife planes.
  The three evil afterlife planes are collectively referred to the Abyss.
  Demons stalk those planes, tormenting evildoers for their own sadistic reasons.
  One of the reasons that some people worship evil deities is to gain a promise of safety, since evil deities protect their worshippers from demonic torment in the afterlife.
  It is also said that demons only torment the weak-willed, and that those who escape demonic torments are free to live in hedonistic luxury.
  There is truth in this, though there are far more people who are confident that they would rule proudly in the Abyss than people who succeed.

  A list of specific well-known deities is given in \trefnp{Deities}.
  The many minor deities worshipped by monsters are not listed here.
  Of course, the GM may use their own custom pantheon.

  % Keep in sync with the cleric list in class.rs
  \begin{dtable!*}
  \lcaption{Deities}
  \begin{dtabularx}{\textwidth}{X l X}
    \tb{Deity} & \tb{Alignment} & \tb{Domains} \tableheaderrule
    Gregory, warrior god of mundanity     & Lawful good     & Law, Protection, Strength, War         \\
    Guftas, horse god of justice          & Lawful good     & Good, Law, Strength, Travel            \\
    Lucied, paladin god of justice        & Lawful good     & Destruction, Good, Protection, War     \\
    Simor, fighter god of protection      & Lawful good     & Good, Protection, Strength, War        \\
    Ayala, naiad god of water             & Neutral good    & Life, Magic, Water, Wild               \\
    Pabs Beerbeard, dwarf god of drink    & Neutral good    & Good, Life, Strength, Wild             \\
    Rucks, monk god of pragmatism         & Neutral good    & Good, Law, Protection, Travel          \\
    Vanya, centaur god of nature          & Neutral good    & Good, Strength, Travel, Wild           \\
    Brushtwig, pixie god of creativity    & Chaotic good    & Chaos, Good, Trickery, Wild            \\
    Camilla, tiefling god of fire         & Chaotic good    & Fire, Good, Magic, Protection          \\
    Chavi, wandering god of stories       & Chaotic good    & Chaos, Knowledge, Trickery             \\
    Chort, dwarf god of optimism          & Chaotic good    & Good, Life, Travel, Wild               \\
    Ivan Ivanovitch, bear god of strength & Chaotic good    & Chaos, Strength, War, Wild             \\
    Krunch, barbarian god of destruction  & Chaotic good    & Destruction, Good, Strength, War       \\
    Sir Cakes, dwarf god of freedom       & Chaotic good    & Chaos, Good, Strength                  \\
    Mikolash, scholar god of knowledge    & Lawful neutral  & Knowledge, Law, Magic, Protection      \\
    Raphael, monk god of retribution      & Lawful neutral  & Death, Law, Protection, Travel         \\
    Declan, god of fire                   & True neutral    & Destruction, Fire, Knowledge, Magic    \\
    Mammon, golem god of endurance        & True neutral    & Knowledge, Magic, Protection, Strength \\
    Kurai, shaman god of nature           & True neutral    & Air, Earth, Fire, Water                \\
    Amanita, druid god of decay           & Chaotic neutral & Chaos, Destruction, Life, Wild         \\
    Antimony, elf god of necromancy       & Chaotic neutral & Death, Knowledge, Life, Magic          \\
    Clockwork, elf god of time            & Chaotic neutral & Chaos, Magic, Trickery, Travel         \\
    Diplo, doll god of destruction        & Chaotic neutral & Chaos, Destruction, Strength, War      \\
    Lord Khallus, fighter god of pride    & Chaotic neutral & Chaos, Strength, War                   \\
    Celeano, sorcerer god of deception    & Chaotic neutral & Chaos, Magic, Protection, Trickery     \\
    Murdoc, god of mercenaries            & Chaotic neutral & Destruction, Knowledge, Travel, War    \\
    Ribo, halfling god of trickery        & Chaotic neutral & Chaos, Trickery, Water                 \\
    Tak, orc god of war                   & Lawful evil     & Law, Strength, Trickery, War           \\
    Theodolus, sorcerer god of ambition   & Neutral evil    & Evil, Knowledge, Magic, Trickery       \\
    Daeghul, demon god of slaughter       & Chaotic evil    & Destruction, Evil, Magic, War          \\
  \end{dtabularx}
  \end{dtable!*}

\section{Planes}\label{Planes}
  The universe of Rise is divided into planes.
  A plane is a distinct realm of existence.
  Except for the connections between planes through planar rifts, each plane is effectively an isolated universe, and different planes can obey different fundamental laws.
  For example, the Material Plane has gravity that exerts a consistent acceleration in a single absolute direction.
  However, the Astral Plane has subjective gravity, where each creature on the plane chooses the direction that gravity pulls it in, if any.

  \subsection{General Cosmology}
    The planes of Rise are divided up into groups.

    \parhead{Primal Planes} The primal planes are manifestations of the basic building blocks of the universe.
    Each plane in this group is predominantly composed of a single element or type of energy.
    There are four primal planes: Air, Earth, Fire, and Water.

    \parhead{Aligned Planes} The five aligned planes are manifestations of the five alignments.
    Elysium is good-aligned, the Abyss is evil-aligned, Ordus is law-aligned, Discord is chaos-aligned, and the Expanse is neutral-aligned.

    When mortal creatures die, their souls travel to an appropriate location on an aligned plane, where they gain new planeforged bodies and live again.
    If they pledged their soul to a deity in life, that deity can take ownership over their soul in death, and the soul is reborn within that deity's territory and under their protection.
    Otherwise, they appear on the aligned plane that most closely reflects their primary alignment in life.

    \parhead{Nexus Planes} The nexus planes are composite planes with a number of distinct environments and filled with creatures of myriad alignments.
    Nexus planes comprise the majority of civilization across all planes.
    They do not have their own unique planar essence, and no planar creatures are native to nexus planes.
    There are two nexus planes: the Material Plane and the Astral Plane.

    \parhead{Demiplanes} These planes are small, fragmentary realms that are greatly limited in their scope.
    There is no specific list of demiplanes, and they share few common properties.
    Most demiplanes were created for particular purposes by beings of great power, though some simply came into existence through unknown means.

    \parhead{Eternal Void} The Eternal Void is the space beyond all other planes.
    It is occupied by aberrations and horrors that ruled the cosmos before the rise of mortals and deities.
    The precursors, ancient aberrations of unfathomable size and power, periodically try to reclaim what was once rightfully theirs.

  \subsectiongraphic*{Planar Rifts}{width=\columnwidth}{the universe/planar rifts}\label{Planar Rifts}
    Normally, there are boundaries between different planes that prevent direct passage between them.
    However, planar rifts are places where these boundaries have weakened, making interplanar travel easier.
    A planar rift joins a specific location on one plane to a specific location on a different plane.
    Most planar rifts lead to and from the Astral Plane, which is the space between the other planes.

    Most planar rifts still require the use of magic, such as the \spell{plane shift} ritual, to actually cross between planes.
    Some especially large rifts enable physical travel between planes without the use of any magic.

  \subsection{Planeforged Creatures}
    A planeforged is a type of creature that is fundamentally composed of the essence of one or more planes.
    The vast majority of planeforged creatures are composed of only a single plane.
    When a planeforged dies, its essence returns to its native plane or planes.
    Weak planeforged lose their independent identity and become part of the core composition of the plane once more.
    Strong planeforged can retain their identity and reform from that raw material given time, making them difficult or impossible to kill completely.
    In either case, planeforged cannot be resurrected by soul-based magic such as the \spell{resurrection} spell.

    % TODO: morphicness, alignment, magic
