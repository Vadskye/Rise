\chapter{Monsters}

Monsters are all of the various non-humanoid creatures that exist in the world of Rise.
Many of them are dangerous, and adventurers may need to fight them.
This chapter describes the rules for monsters, and the combat statistics for a variety of monsters.

\section{Monster Statistics}
    Like player characters, monsters have levels, attributes, and abilities.
    However, they do not have classes, legacy items, or many other elements of characters.
    This section defines how monsters function.

    \subsection{Level}
        Each monster has a level that indicates its approximate strength.
        This has all of the same effects as the level for a player character, except that monsters do not gain legacy items or attunement points as described in \tref{Character Advancement}.

    \subsection{Challenge Rating}\label{Challenge Rating}
        Each monster has a \glossterm{challenge rating} that indicates its approximate strength within its level, ranging from 1/2 to 4.
        A monster's challenge rating is a guideline to how many of that monster should typically be present in an encounter (see \pcref{Encounter Balancing}).
        This has several effects on the monster's statistics, as decribed in \trefnp{Challenge Rating Effects}.

        \begin{dtable}
            \lcaption{Challenge Rating Effects}
            \begin{dtabularx}{\textwidth}{l l l X l l l}
                \tb{CR} & \tb{HP}  & \tb{Resists} & \tb{Vital Wound Max} & \tb{AP} & \tb{Accuracy} & \tb{Defenses} \tableheaderrule
                1/2     & \mult1/2 & \mult0           & 1                    & 0       & \plus0        & \plus0 \\
                1       & \mult1   & \mult0           & 1                    & 1       & \plus0        & \plus0 \\
                2       & \mult1   & \mult1           & 2                    & 1       & \plus1        & \plus1 \\
                3       & \mult2   & \mult2           & 3                    & 2       & \plus2        & \plus2 \\
                4       & \mult2   & \mult2\fn{1}     & 4                    & 3       & \plus3        & \plus3 \\
            \end{dtabularx}
            1. CR 4 monsters also have a universal damage resistance equal to their HP (see \pcref{Monster Universal Resistance}). \\
        \end{dtable}

        \subsection{Monster Universal Resistance}
            CR 4 monsters have a special universal resistance to damage of all types equal to their maximum \glossterm{hit points}.
            This resistance is independent of their specific resistances to physical damage or energy damage, and is applied after their more specific resistances have been expended.
            A small number of monsters that are not CR 4 can also gain universal damage resistance as indicated in their descriptions.

            This resistance ensures that CR 4 monsters are appropriately durable in combat without requiring all damage they take be of any particular type.
            It should take multiple rounds of combat before they start losing hit points, since losing hit points makes them vulnerable to powerful detrimental effects.

        \subsubsection{Monster Vital Wounds}
            Monsters do not normally make \glossterm{vital rolls} like player characters do.
            Instead, once a monster takes at least as many \glossterm{vital wounds} as its \glossterm{challenge rating}, it is defeated.
            Defeated monsters may be unconscious, dead, or merely flee the fight at the discretion of the Game Master.

            Some monsters have specific mechanics for what happens if they gain \glossterm{vital wounds}.
            Unless otherwise specified on the monster's description, \glossterm{vital wounds} have no negative effects on monsters.

            Monsters do suffer vital wounds from taking at least half their maximum hit points in damage from a single attack.
            This generally makes CR 1/2 and CR 1 monsters easy to quickly dispatch with strong attacks.

        \subsubsection{Spending Resources}
            Unless otherwise noted in their description, all monsters are unable to use abilities that would cause them to gain \glossterm{fatigue points}, such as the \textit{desperate exertion} ability.

        \subsubsection{Recovering Hit Points and Conditions}
            Monsters cannot normally use the \textit{recover} ability.
            However, monsters with a high challenge rating can remove conditions automatically.
            At the end of each round, monsters with CR of 1 or higher roll 1d10 \add their CR.
            If the result is 11 or higher, the monster removes one condition of its choice.
            This effect cannot remove a condition applied during the current round.
            Monsters without a sufficient understanding of the conditions affecting them generally choose randomly.

    \subsection{Attributes}
        Each of a monster's base attributes can range from \minus9 to 3, as appropriate for the type of monster.
        A monster's attributes scale with level in the same way as character attributes.
        A monster can also have up to two attributes starting at 4 or 5.
        In general, a monster with higher starting attributes will be slightly stronger, but not all monsters need to start with the same starting attribute total.

    \subsection{Level Scaling}
        Monster level scaling is described in \tref{Monster Advancement}.
        This summarizes the following effects:
        \begin{itemize}
            \item A \plus1 bonus to all defenses at 3rd level and every 6 levels thereafter
            \item A \plus1 bonus to power with all abilities at 3rd level that gradually increases every 3 levels thereafter
            \item A \plus1d bonus to damage with strikes at 4th level and every 3 levels thereafter
            \item A \plus1 bonus to accuracy at 5th level and every 6 levels thereafter
        \end{itemize}
        These scaling values are designed to keep monsters at a similar power level to player characters without having to repetitively define specific bonuses for each monster.
        In general, monsters that are designed to have unusually high or low values for any particular statistic define that with their attributes, though some monsters have more specific abilities.

    \begin{dtable*}
        \lcaption{Monster Advancement}
        \begin{dtabularx}{\textwidth}{l >{\lcol}X >{\lcol}X >{\lcol}X >{\lcol}X >{\lcol}X >{\lcol}X}
            \tb{Level} & \tb{HP}   & \tb{Resistances} & \tb{Defenses} & \tb{Power} & \tb{Strike Damage} & \tb{Accuracy} \tableheaderrule
            1st        & 11        & 2                & \dash         & \dash      & \tdash             & \dash  \\
            2nd        & 12        & 3                & \dash         & \dash      & \tdash             & \dash  \\
            3rd        & 13        & 3                & \plus1        & \plus1     & \tdash             & \dash  \\
            4th        & 15        & 3                & \plus1        & \plus1     & \plus1d            & \dash  \\
            5th        & 17        & 4                & \plus1        & \plus1     & \plus1d            & \plus1 \\
            6th        & 19        & 4                & \plus1        & \plus2     & \plus1d            & \plus1 \\
            7th        & 22        & 5                & \plus1        & \plus2     & \plus2d            & \plus1 \\
            8th        & 25        & 6                & \plus1        & \plus2     & \plus2d            & \plus1 \\
            9th        & 28        & 7                & \plus2        & \plus3     & \plus2d            & \plus1 \\
            10th       & 31        & 8                & \plus2        & \plus3     & \plus3d            & \plus1 \\
            11th       & 35        & 9                & \plus2        & \plus3     & \plus3d            & \plus2 \\
            12th       & 39        & 10               & \plus2        & \plus4     & \plus3d            & \plus2 \\
            13th       & 44        & 11               & \plus2        & \plus4     & \plus4d            & \plus2 \\
            14th       & 50        & 12               & \plus2        & \plus4     & \plus4d            & \plus2 \\
            15th       & 56        & 14               & \plus3        & \plus6     & \plus4d            & \plus2 \\
            16th       & 63        & 15               & \plus3        & \plus6     & \plus5d            & \plus2 \\
            17th       & 70        & 17               & \plus3        & \plus6     & \plus5d            & \plus3 \\
            18th       & 78        & 19               & \plus3        & \plus8     & \plus5d            & \plus3 \\
            19th       & 88        & 22               & \plus3        & \plus8     & \plus6d            & \plus3 \\
            20th       & 100       & 25               & \plus3        & \plus8     & \plus6d            & \plus3 \\
            21st       & 115\fn{5} & 28               & \plus4        & \plus12    & \plus6d            & \plus3 \\
        \end{dtabularx}
        1. For each level beyond 21, a monster increases its maximum hit points by 15 and its maximum resistances by 4. \\
    \end{dtable*}

    \subsection{Monster Armor}
        In general, monsters gain Armor Defense bonuses from armor in the same way that player characters do.
        Of course, not all monsters wear physical armor.
        However, their statistics are modified in a similar way.
        The narrative justification is that the bonus is caused by their scales, an unusually tough hide, or similar effects.

        Monster resistances are generally calculated more simply than player resistances from armor, however.
        At higher levels, player resistances are heavily influenced by special materials, and calculating monster resistances in the same way would require annoying bookkeeping and lead to oddly large gaps between resistances at similar levels.
        Instead, many monsters have special abilities indicating that they double their base resistance bonus for physical damage, energy damage, or both, depending on the monster's narrative.

\section{Monster Combat Mechanics}

    \subsection{Monster Actions}
        All monsters are able to take \glossterm{free actions}.
        CR 1 or higher monsters can take \glossterm{minor actions}, though many monsters do not have any relevant minor actions to take.
        Some monsters, including many high CR monsters, can either take additional actions each round or use powerful abilities as minor actions.
        These special abilities are listed in their descriptions.

    \subsection{Encounter Balancing}\label{Encounter Balancing}
        In general, a group of PCs of a given level will have an appropriate challenge from fighting monsters of the same level with a combined challenge rating equal to the number of PCs.
        Fighting monsters of a lower level, or monsters whose combined challenge rating is less than the number of PCs, will yield an easier encounter.
        Fighting monsters of a higher level, or fighting monsters whose combined challenge rating is greater than the number of PCs, will yield an easier encounter.

        It is generally not a good idea for PCs to fight monsters more than three levels higher or lower than their own.
        They may find that their attacks always miss, or never miss, and other aspects of the encounter may be similarly imbalanced in ways that are not easily remedied by simply changing the number of enemies.

    \subsection{Monster Knowledge}
        You can remember relevant knowledge about monsters by making Knowledge checks (see \pcref{Knowledge}).
        Each monster has a set of associated information that you can learn with a knowledge check of the listed \glossterm{difficulty rating}.
        Most monsters have multiple tiers of information that you can learn depending on how good your Knowledge check is.
        All information recorded in these descriptions is accurate, though lower check results may not provide full context for a monster's behavior or nature.


\section{Aberrations}
\begin{monsection}{Aboleth}{12}[4]
\vspace{-1em}\spelltwocol{}{Huge aberration}\vspace{-1em}
\begin{spellcontent}
\begin{spelltargetinginfo}
\spelltwocol{
\textbf{HP} 384;
\textbf{Bloodied} 192;
}
{\textbf{AP} 5/5}

\pari \textbf{Armor} 21;
\textbf{Fort} 22;
\textbf{Ref} 15;
\textbf{Ment} 30
\pari \textbf{Strike} Tentacle \plus17 (4d10)


\pari \textbf{Actions} One in action phase, one in delayed action phase
\pari \textbf{Behavior} Attack highest threat
\end{spelltargetinginfo}
\end{spellcontent}

\begin{monsterfooter}
\pari \textbf{Awareness} \plus6
\pari \textbf{Speed} 50 ft. swim ft.;
\textbf{Space} 15 ft.;
\textbf{Reach} 15 ft.
\pari \textbf{Attributes}:
Str 17,
Dex 0,
Con 14,
Int 15,
Per 14,
Wil 16
\end{monsterfooter}
\end{monsection}


\subsubsection{Aboleth Abilities}

\begin{freeability}{Mind Crush}[\glossterm{Compulsion}]
The aboleth makes a \plus17 vs. Mental attack against a creature in \rnglong range.
\hit The target takes 10d10 psionic damage and is \glossterm{stunned} as a \glossterm{condition}.
\crit The aboleth can spend an \glossterm{action point} to \glossterm{attune} to this ability.
If it does, the target is \glossterm{dominated} by the aboleth as long as the ability lasts.
Otherwise, the target takes double the damage of a non-critical hit.
\end{freeability}

\vspace{0.5em}
\begin{freeability}{Psionic Blast}[\glossterm{Compulsion}]
The aboleth makes a \plus17 vs. Mental attack against enemies in a Large cone.
\hit Each target takes 8d10 psionic damage and is \glossterm{stunned} as a \glossterm{condition}.
\end{freeability}

\parhead{Rituals}
The aboleth can learn and perform arcane rituals of up to 6th level.

\section{Animates}
\begin{monsection}{Elemental}[Air]{10}[1]
\vspace{-1em}\spelltwocol{}{Large animate}\vspace{-1em}
\begin{spellcontent}
\begin{spelltargetinginfo}
\spelltwocol{
\textbf{HP} 70;
\textbf{Bloodied} 35;
}
{\textbf{AP} 2/0}

\pari \textbf{Armor} 19;
\textbf{Fort} 19;
\textbf{Ref} 23;
\textbf{Ment} 12
\pari \textbf{Strike} Slam \plus12 (4d6)



\pari \textbf{Behavior} Attack highest threat
\end{spelltargetinginfo}
\end{spellcontent}

\begin{monsterfooter}
\pari \textbf{Awareness} \plus6
\pari \textbf{Speed} 40 ft.;
\textbf{Space} 10 ft.;
\textbf{Reach} 10 ft.
\pari \textbf{Attributes}:
Str 13,
Dex 15,
Con 11,
Int 0,
Per 12,
Wil 0
\end{monsterfooter}
\end{monsection}

\begin{monsection}{Animus}[Ram]{6}[4]
\vspace{-1em}\spelltwocol{}{Huge animate}\vspace{-1em}
\begin{spellcontent}
\begin{spelltargetinginfo}
\spelltwocol{
\textbf{HP} 216;
\textbf{Bloodied} 108;
}
{\textbf{AP} 4/0}

\pari \textbf{Armor} 16;
\textbf{Fort} 22;
\textbf{Ref} 12;
\textbf{Ment} 11
\pari \textbf{Strike} Slam \plus10 (4d6) or hoof \plus10 (2d10)


\pari \textbf{Actions} One in action phase, one in delayed action phase
\pari \textbf{Behavior} Attack highest threat
\end{spelltargetinginfo}
\end{spellcontent}

\begin{monsterfooter}
\pari \textbf{Awareness} \plus6
\pari \textbf{Speed} 50 ft.;
\textbf{Space} 15 ft.;
\textbf{Reach} 15 ft.
\pari \textbf{Attributes}:
Str 13,
Dex 7,
Con 9,
Int 0,
Per 7,
Wil 0
\end{monsterfooter}
\end{monsection}


\subsubsection{Animus Abilities}

\begin{freeability}{Forceful Smash}
The ram makes a slam strike.
In addition to the strike's normal effects, compare the attack result against the target's Fortitude defense.
\hit The target moves up to 10 feet in a direction of the ram's choice, as the \textit{shove} ability (see \pcref{Shove}).
The ram does not have to move with the target to push it back.
\end{freeability}

\begin{monsection}{Earth Elemental}[Elder]{12}[3]
\vspace{-1em}\spelltwocol{}{Huge animate}\vspace{-1em}
\begin{spellcontent}
\begin{spelltargetinginfo}
\spelltwocol{
\textbf{HP} 360;
\textbf{Bloodied} 180;
}
{\textbf{AP} 4/3}

\pari \textbf{Armor} 20;
\textbf{Fort} 29;
\textbf{Ref} 14;
\textbf{Ment} 21
\pari \textbf{Strike} Slam \plus16 (5d10)


\pari \textbf{Actions} One in action phase, one in delayed action phase
\pari \textbf{Behavior} Attack highest threat
\end{spelltargetinginfo}
\end{spellcontent}

\begin{monsterfooter}
\pari \textbf{Awareness} \plus6
\pari \textbf{Speed} 50 ft.;
\textbf{Space} 15 ft.;
\textbf{Reach} 15 ft.
\pari \textbf{Attributes}:
Str 19,
Dex 0,
Con 16,
Int 7,
Per 14,
Wil 14
\end{monsterfooter}
\end{monsection}

\section{Animals}
\begin{monsection}{Bear}[Black]{2}[2]
\vspace{-1em}\spelltwocol{}{Medium animal}\vspace{-1em}
\begin{spellcontent}
\begin{spelltargetinginfo}
\spelltwocol{
\textbf{HP} 36;
\textbf{Bloodied} 18;
}
{\textbf{AP} 2/0}

\pari \textbf{Armor} 7;
\textbf{Fort} 16;
\textbf{Ref} 8;
\textbf{Ment} 5
\pari \textbf{Strike} Bite \plus3 (1d10) or claw \plus3 (1d8)
\pari \textbf{Immune} staggered


\pari \textbf{Behavior} Attack highest threat
\end{spelltargetinginfo}
\end{spellcontent}

\begin{monsterfooter}
\pari \textbf{Awareness} \plus6
\pari \textbf{Speed} 30 ft.;
\textbf{Space} 5 ft.;
\textbf{Reach} 5 ft.
\pari \textbf{Attributes}:
Str 5,
Dex 2,
Con 5,
Int \minus7,
Per 2,
Wil 0
\end{monsterfooter}
\end{monsection}


\subsubsection{Bear Abilities}

\begin{freeability}{Rend}
The bear makes a claw strike against two targets within reach.
\end{freeability}

\begin{monsection}{Bear}[Brown]{4}[2]
\vspace{-1em}\spelltwocol{}{Large animal}\vspace{-1em}
\begin{spellcontent}
\begin{spelltargetinginfo}
\spelltwocol{
\textbf{HP} 72;
\textbf{Bloodied} 36;
}
{\textbf{AP} 2/0}

\pari \textbf{Armor} 9;
\textbf{Fort} 18;
\textbf{Ref} 8;
\textbf{Ment} 7
\pari \textbf{Strike} Bite \plus5 (2d8) or claw \plus5 (2d6)
\pari \textbf{Immune} staggered


\pari \textbf{Behavior} Attack highest threat
\end{spelltargetinginfo}
\end{spellcontent}

\begin{monsterfooter}
\pari \textbf{Awareness} \plus6
\pari \textbf{Speed} 40 ft.;
\textbf{Space} 10 ft.;
\textbf{Reach} 10 ft.
\pari \textbf{Attributes}:
Str 9,
Dex 3,
Con 7,
Int \minus7,
Per 3,
Wil 0
\end{monsterfooter}
\end{monsection}


\subsubsection{Bear Abilities}

\begin{freeability}{Rend}
The bear makes a claw strike against two targets within reach.
\end{freeability}

\begin{monsection}{Beetle}[Dire]{7}[2]
\vspace{-1em}\spelltwocol{}{Large animal}\vspace{-1em}
\begin{spellcontent}
\begin{spelltargetinginfo}
\spelltwocol{
\textbf{HP} 126;
\textbf{Bloodied} 63;
}
{\textbf{AP} 3/0}

\pari \textbf{Armor} 14;
\textbf{Fort} 21;
\textbf{Ref} 10;
\textbf{Ment} 10
\pari \textbf{Strike} Bite \plus9 (4d6)



\pari \textbf{Behavior} Attack highest threat
\end{spelltargetinginfo}
\end{spellcontent}

\begin{monsterfooter}
\pari \textbf{Awareness} \plus6
\pari \textbf{Speed} 40 ft.;
\textbf{Space} 10 ft.;
\textbf{Reach} 10 ft.
\pari \textbf{Attributes}:
Str 12,
Dex 0,
Con 10,
Int \minus9,
Per 8,
Wil 0
\end{monsterfooter}
\end{monsection}

\begin{monsection}{Centipede}[Huge]{8}[3]
\vspace{-1em}\spelltwocol{}{Huge animal}\vspace{-1em}
\begin{spellcontent}
\begin{spelltargetinginfo}
\spelltwocol{
\textbf{HP} 240;
\textbf{Bloodied} 120;
}
{\textbf{AP} 4/0}

\pari \textbf{Armor} 16;
\textbf{Fort} 25;
\textbf{Ref} 11;
\textbf{Ment} 12
\pari \textbf{Strike} Bite \plus11 (4d8)


\pari \textbf{Actions} One in action phase, one in delayed action phase
\pari \textbf{Behavior} Attack highest threat
\end{spelltargetinginfo}
\end{spellcontent}

\begin{monsterfooter}
\pari \textbf{Awareness} \plus6
\pari \textbf{Speed} 50 ft.;
\textbf{Space} 15 ft.;
\textbf{Reach} 15 ft.
\pari \textbf{Attributes}:
Str 15,
Dex 5,
Con 12,
Int \minus9,
Per 9,
Wil 0
\end{monsterfooter}
\end{monsection}

\begin{monsection}{Dire Wolf}{5}[2]
\vspace{-1em}\spelltwocol{}{Large animal}\vspace{-1em}
\begin{spellcontent}
\begin{spelltargetinginfo}
\spelltwocol{
\textbf{HP} 70;
\textbf{Bloodied} 35;
}
{\textbf{AP} 2/0}

\pari \textbf{Armor} 12;
\textbf{Fort} 15;
\textbf{Ref} 13;
\textbf{Ment} 8
\pari \textbf{Strike} Bite \plus8 (2d8)



\pari \textbf{Behavior} Attack highest threat
\end{spelltargetinginfo}
\end{spellcontent}

\begin{monsterfooter}
\pari \textbf{Awareness} \plus6
\pari \textbf{Speed} 40 ft.;
\textbf{Space} 10 ft.;
\textbf{Reach} 10 ft.
\pari \textbf{Attributes}:
Str 9,
Dex 7,
Con 6,
Int \minus6,
Per 7,
Wil 0
\end{monsterfooter}
\end{monsection}


\subsubsection{Dire Wolf Abilities}

\begin{freeability}{Pounce}
The dire wolf moves up to its speed in a single straight line and makes a bite \glossterm{strike} from its new location.
\end{freeability}

\begin{monsection}{Eel}{6}[2]
\vspace{-1em}\spelltwocol{}{Large animal}\vspace{-1em}
\begin{spellcontent}
\begin{spelltargetinginfo}
\spelltwocol{
\textbf{HP} 84;
\textbf{Bloodied} 42;
}
{\textbf{AP} 2/0}

\pari \textbf{Armor} 15;
\textbf{Fort} 16;
\textbf{Ref} 14;
\textbf{Ment} 9
\pari \textbf{Strike} Bite \plus9 (2d8)



\pari \textbf{Behavior} Attack highest threat
\end{spelltargetinginfo}
\end{spellcontent}

\begin{monsterfooter}
\pari \textbf{Awareness} \plus6
\pari \textbf{Speed} 40 ft.;
\textbf{Space} 10 ft.;
\textbf{Reach} 10 ft.
\pari \textbf{Attributes}:
Str 9,
Dex 8,
Con 7,
Int \minus8,
Per 8,
Wil 0
\end{monsterfooter}
\end{monsection}

\begin{monsection}{Ferret}{1}[1]
\vspace{-1em}\spelltwocol{}{Tiny animal}\vspace{-1em}
\begin{spellcontent}
\begin{spelltargetinginfo}
\spelltwocol{
\textbf{HP} 1;
\textbf{Bloodied} 0;
}
{\textbf{AP} 1/\minus2}

\pari \textbf{Armor} 6;
\textbf{Fort} 3;
\textbf{Ref} 16;
\textbf{Ment} 1
\pari \textbf{Strike} Bite \plus1 (1d6)



\pari \textbf{Behavior} Attack highest threat
\end{spelltargetinginfo}
\end{spellcontent}

\begin{monsterfooter}
\pari \textbf{Awareness} \plus6
\pari \textbf{Speed} 20 ft.;
\textbf{Space} 2.5 ft.;
\textbf{Reach} 2.5 ft.
\pari \textbf{Attributes}:
Str \minus10,
Dex 4,
Con \minus4,
Int \minus7,
Per 1,
Wil \minus2
\end{monsterfooter}
\end{monsection}

\begin{monsection}{Ichor Bear}[Black]{2}[2]
\vspace{-1em}\spelltwocol{}{Medium animal}\vspace{-1em}
\begin{spellcontent}
\begin{spelltargetinginfo}
\spelltwocol{
\textbf{HP} 36;
\textbf{Bloodied} 18;
}
{\textbf{AP} 2/0}

\pari \textbf{Armor} 7;
\textbf{Fort} 16;
\textbf{Ref} 8;
\textbf{Ment} 9
\pari \textbf{Strike} Bite \plus3 (1d10) or claw \plus3 (1d8)
\pari \textbf{Immune} staggered


\pari \textbf{Behavior} Attack highest threat
\end{spelltargetinginfo}
\end{spellcontent}

\begin{monsterfooter}
\pari \textbf{Awareness} \plus6
\pari \textbf{Speed} 30 ft.;
\textbf{Space} 5 ft.;
\textbf{Reach} 5 ft.
\pari \textbf{Attributes}:
Str 5,
Dex 2,
Con 5,
Int \minus7,
Per 2,
Wil 0
\end{monsterfooter}
\end{monsection}


\subsubsection{Ichor Bear Abilities}

\begin{freeability}{Rend}
The bear makes a claw strike against two targets within reach.
\end{freeability}

\parhead{Ichor Healing}
The ichor bear heals 2 hit points at the end of each round.

\begin{monsection}{Pony}{2}[1]
\vspace{-1em}\spelltwocol{}{Medium animal}\vspace{-1em}
\begin{spellcontent}
\begin{spelltargetinginfo}
\spelltwocol{
\textbf{HP} 16;
\textbf{Bloodied} 8;
}
{\textbf{AP} 1/0}

\pari \textbf{Armor} 6;
\textbf{Fort} 13;
\textbf{Ref} 7;
\textbf{Ment} 4
\pari \textbf{Strike} Bite \plus2 (1d8)



\pari \textbf{Behavior} Attack highest threat
\end{spelltargetinginfo}
\end{spellcontent}

\begin{monsterfooter}
\pari \textbf{Awareness} \plus6
\pari \textbf{Speed} 30 ft.;
\textbf{Space} 5 ft.;
\textbf{Reach} 5 ft.
\pari \textbf{Attributes}:
Str 2,
Dex 2,
Con 4,
Int \minus7,
Per 2,
Wil 0
\end{monsterfooter}
\end{monsection}

\begin{monsection}{Raven}{1}[1]
\vspace{-1em}\spelltwocol{}{Tiny animal}\vspace{-1em}
\begin{spellcontent}
\begin{spelltargetinginfo}
\spelltwocol{
\textbf{HP} 1;
\textbf{Bloodied} 0;
}
{\textbf{AP} 1/0}

\pari \textbf{Armor} 5;
\textbf{Fort} 3;
\textbf{Ref} 14;
\textbf{Ment} 3
\pari \textbf{Strike} Talon \plus3 (1d4)



\pari \textbf{Behavior} Attack highest threat
\end{spelltargetinginfo}
\end{spellcontent}

\begin{monsterfooter}
\pari \textbf{Awareness} \plus6
\pari \textbf{Speed} 20 ft.;
\textbf{Space} 2.5 ft.;
\textbf{Reach} 2.5 ft.
\pari \textbf{Attributes}:
Str \minus13,
Dex 3,
Con \minus4,
Int \minus6,
Per 2,
Wil 0
\end{monsterfooter}
\end{monsection}

\begin{monsection}{Roc}{9}[4]
\vspace{-1em}\spelltwocol{}{Gargantuan animal}\vspace{-1em}
\begin{spellcontent}
\begin{spelltargetinginfo}
\spelltwocol{
\textbf{HP} 252;
\textbf{Bloodied} 126;
}
{\textbf{AP} 5/0}

\pari \textbf{Armor} 19;
\textbf{Fort} 21;
\textbf{Ref} 13;
\textbf{Ment} 14
\pari \textbf{Strike} Talon \plus14 (5d10)


\pari \textbf{Actions} One in action phase, one in delayed action phase
\pari \textbf{Behavior} Attack highest threat
\end{spelltargetinginfo}
\end{spellcontent}

\begin{monsterfooter}
\pari \textbf{Awareness} \plus6
\pari \textbf{Speed} 80 ft. fly ft.;
\textbf{Space} 20 ft.;
\textbf{Reach} 20 ft.
\pari \textbf{Attributes}:
Str 20,
Dex 10,
Con 10,
Int \minus7,
Per 11,
Wil 0
\end{monsterfooter}
\end{monsection}


\subsubsection{Roc Abilities}

\begin{freeability}{Flyby Attack}
The roc flies up to its flying movement speed.
It can make a talon strike or use the \textit{grapple} ability at any point during this movement.
\end{freeability}

\begin{monsection}{Spider}[Colossal]{12}[3]
\vspace{-1em}\spelltwocol{}{Colossal animal}\vspace{-1em}
\begin{spellcontent}
\begin{spelltargetinginfo}
\spelltwocol{
\textbf{HP} 216;
\textbf{Bloodied} 108;
}
{\textbf{AP} 4/0}

\pari \textbf{Armor} 23;
\textbf{Fort} 21;
\textbf{Ref} 17;
\textbf{Ment} 16
\pari \textbf{Strike} Bite \plus16 (7d10)


\pari \textbf{Actions} One in action phase, one in delayed action phase
\pari \textbf{Behavior} Attack highest threat
\end{spelltargetinginfo}
\end{spellcontent}

\begin{monsterfooter}
\pari \textbf{Awareness} \plus6
\pari \textbf{Speed} 70 ft.;
\textbf{Space} 30 ft.;
\textbf{Reach} 30 ft.
\pari \textbf{Attributes}:
Str 23,
Dex 15,
Con 7,
Int \minus9,
Per 14,
Wil 0
\end{monsterfooter}
\end{monsection}


\subsubsection{Spider Abilities}

\begin{freeability}{Web Spit}
The spider makes a \plus16 vs. Reflex attack against one creature within \rnglong range.
\hit The target is \glossterm{immobilized} as a \glossterm{condition}.
\end{freeability}

\begin{monsection}{Spider}[Gargantuan]{9}[3]
\vspace{-1em}\spelltwocol{}{Gargantuan animal}\vspace{-1em}
\begin{spellcontent}
\begin{spelltargetinginfo}
\spelltwocol{
\textbf{HP} 162;
\textbf{Bloodied} 81;
}
{\textbf{AP} 4/0}

\pari \textbf{Armor} 20;
\textbf{Fort} 18;
\textbf{Ref} 16;
\textbf{Ment} 13
\pari \textbf{Strike} Bite \plus13 (4d10)


\pari \textbf{Actions} One in action phase, one in delayed action phase
\pari \textbf{Behavior} Attack highest threat
\end{spelltargetinginfo}
\end{spellcontent}

\begin{monsterfooter}
\pari \textbf{Awareness} \plus6
\pari \textbf{Speed} 60 ft.;
\textbf{Space} 20 ft.;
\textbf{Reach} 20 ft.
\pari \textbf{Attributes}:
Str 17,
Dex 12,
Con 5,
Int \minus9,
Per 11,
Wil 0
\end{monsterfooter}
\end{monsection}


\subsubsection{Spider Abilities}

\begin{freeability}{Web Spit}
The spider makes a \plus13 vs. Reflex attack against one creature within \rnglong range.
\hit The target is \glossterm{immobilized} as a \glossterm{condition}.
\end{freeability}

\begin{monsection}{Wasp}[Giant]{6}[1]
\vspace{-1em}\spelltwocol{}{Large animal}\vspace{-1em}
\begin{spellcontent}
\begin{spelltargetinginfo}
\spelltwocol{
\textbf{HP} 30;
\textbf{Bloodied} 15;
}
{\textbf{AP} 1/\minus1}

\pari \textbf{Armor} 14;
\textbf{Fort} 12;
\textbf{Ref} 13;
\textbf{Ment} 7
\pari \textbf{Strike} Bite \plus7 (2d6)



\pari \textbf{Behavior} Attack highest threat
\end{spelltargetinginfo}
\end{spellcontent}

\begin{monsterfooter}
\pari \textbf{Awareness} \plus6
\pari \textbf{Speed} 50 ft. fly (good) ft.;
\textbf{Space} 10 ft.;
\textbf{Reach} 10 ft.
\pari \textbf{Attributes}:
Str 6,
Dex 8,
Con 0,
Int \minus8,
Per 7,
Wil \minus1
\end{monsterfooter}
\end{monsection}

\begin{monsection}{Wolf}{1}[1]
\vspace{-1em}\spelltwocol{}{Medium animal}\vspace{-1em}
\begin{spellcontent}
\begin{spelltargetinginfo}
\spelltwocol{
\textbf{HP} 6;
\textbf{Bloodied} 3;
}
{\textbf{AP} 1/0}

\pari \textbf{Armor} 6;
\textbf{Fort} 8;
\textbf{Ref} 10;
\textbf{Ment} 3
\pari \textbf{Strike} Bite \plus2 (1d6)



\pari \textbf{Behavior} Attack highest threat
\end{spelltargetinginfo}
\end{spellcontent}

\begin{monsterfooter}
\pari \textbf{Awareness} \plus6
\pari \textbf{Speed} 30 ft.;
\textbf{Space} 5 ft.;
\textbf{Reach} 5 ft.
\pari \textbf{Attributes}:
Str 1,
Dex 3,
Con 1,
Int \minus6,
Per 2,
Wil 0
\end{monsterfooter}
\end{monsection}

\section{Humanoids}
\begin{monsection}{Cultist}{1}[1]
\vspace{-1em}\spelltwocol{}{Medium humanoid}\vspace{-1em}
\begin{spellcontent}
\begin{spelltargetinginfo}
\spelltwocol{
\textbf{HP} 5;
\textbf{Bloodied} 2;
}
{\textbf{AP} 1/3}

\pari \textbf{Armor} 4;
\textbf{Fort} 5;
\textbf{Ref} 5;
\textbf{Ment} 10
\pari \textbf{Strike} Club \plus1 (1d4)



\pari \textbf{Behavior} Attack highest threat
\end{spelltargetinginfo}
\end{spellcontent}

\begin{monsterfooter}
\pari \textbf{Awareness} \plus6
\pari \textbf{Speed} 30 ft.;
\textbf{Space} 5 ft.;
\textbf{Reach} 5 ft.
\pari \textbf{Attributes}:
Str 0,
Dex 0,
Con 0,
Int \minus1,
Per 1,
Wil 3
\end{monsterfooter}
\end{monsection}


\subsubsection{Cultist Abilities}

\begin{freeability}{Hex}
The cultist makes a \plus1 vs. Fortitude attack against one creature in Medium range.
\hit The target takes 1d8 life damage and is \glossterm{sickened} as a \glossterm{condition}.",
\end{freeability}

\begin{monsection}{Goblin Shouter}{2}[2]
\vspace{-1em}\spelltwocol{}{Small humanoid}\vspace{-1em}
\begin{spellcontent}
\begin{spelltargetinginfo}
\spelltwocol{
\textbf{HP} 24;
\textbf{Bloodied} 12;
}
{\textbf{AP} 2/3}

\pari \textbf{Armor} 6;
\textbf{Fort} 8;
\textbf{Ref} 12;
\textbf{Ment} 12
\pari \textbf{Strike} Club \plus4 (1d6) or sling \plus4 (1d6)



\pari \textbf{Behavior} Attack lowest threat
\end{spelltargetinginfo}
\end{spellcontent}

\begin{monsterfooter}
\pari \textbf{Awareness} \plus6
\pari \textbf{Speed} 25 ft.;
\textbf{Space} 5 ft.;
\textbf{Reach} 5 ft.
\pari \textbf{Attributes}:
Str 0,
Dex 3,
Con 2,
Int \minus2,
Per 3,
Wil 4
\end{monsterfooter}
\end{monsection}


\subsubsection{Goblin Shouter Abilities}

\begin{freeability}{Shout of Stabbing}[\glossterm{Sustain} (standard)]
The shouter's \glossterm{allies} that can hear it gain a \plus2 bonus to \glossterm{power} with \glossterm{strikes}.
\end{freeability}

\begin{monsection}{Goblin Stabber}{1}[1]
\vspace{-1em}\spelltwocol{}{Small humanoid}\vspace{-1em}
\begin{spellcontent}
\begin{spelltargetinginfo}
\spelltwocol{
\textbf{HP} 4;
\textbf{Bloodied} 2;
}
{\textbf{AP} 1/0}

\pari \textbf{Armor} 5;
\textbf{Fort} 4;
\textbf{Ref} 12;
\textbf{Ment} 5
\pari \textbf{Strike} Shortsword \plus3 (1d4) or sling \plus2 (1d4)



\pari \textbf{Behavior} Attack lowest threat
\end{spelltargetinginfo}
\end{spellcontent}

\begin{monsterfooter}
\pari \textbf{Awareness} \plus6
\pari \textbf{Speed} 25 ft.;
\textbf{Space} 5 ft.;
\textbf{Reach} 5 ft.
\pari \textbf{Attributes}:
Str \minus2,
Dex 3,
Con \minus1,
Int \minus2,
Per 2,
Wil 0
\end{monsterfooter}
\end{monsection}


\subsubsection{Goblin Stabber Abilities}

\begin{freeability}{Sneeky Stab}
The stabber makes a shortsword strike.
If the target is defenseless, overwhelmed, or unaware, the damage becomes 1d8.
\end{freeability}

\begin{monsection}{Orc Chieftain}{5}[3]
\vspace{-1em}\spelltwocol{}{Medium humanoid}\vspace{-1em}
\begin{spellcontent}
\begin{spelltargetinginfo}
\spelltwocol{
\textbf{HP} 120;
\textbf{Bloodied} 60;
}
{\textbf{AP} 3/3}

\pari \textbf{Armor} 11;
\textbf{Fort} 16;
\textbf{Ref} 14;
\textbf{Ment} 16
\pari \textbf{Strike} Greataxe \plus8 (4d6)


\pari \textbf{Actions} One in action phase, one in delayed action phase
\pari \textbf{Behavior} Attack highest threat
\end{spelltargetinginfo}
\end{spellcontent}

\begin{monsterfooter}
\pari \textbf{Awareness} \plus6
\pari \textbf{Speed} 30 ft.;
\textbf{Space} 5 ft.;
\textbf{Reach} 5 ft.
\pari \textbf{Attributes}:
Str 10,
Dex 6,
Con 7,
Int 0,
Per 6,
Wil 7
\end{monsterfooter}
\end{monsection}


\subsubsection{Orc Chieftain Abilities}

\begin{freeability}{Hit Everyone Else}[\glossterm{Sustain} (standard)]
The chieftain's \glossterm{allies} that can hear it gain a \plus2 bonus to \glossterm{accuracy} with \glossterm{strikes}.
\end{freeability}

\vspace{0.5em}
\begin{freeability}{Hit Hardest}
The chieftain makes a greataxe strike.
The strike deals 4d10 damage.
\end{freeability}

\vspace{0.5em}
\begin{freeability}{Hit Fast}
The chieftain makes a greataxe strike.
Its accuracy is increased to 10.
\end{freeability}

\begin{monsection}{Orc Grunt}{2}[1]
\vspace{-1em}\spelltwocol{}{Medium humanoid}\vspace{-1em}
\begin{spellcontent}
\begin{spelltargetinginfo}
\spelltwocol{
\textbf{HP} 12;
\textbf{Bloodied} 6;
}
{\textbf{AP} 1/0}

\pari \textbf{Armor} 5;
\textbf{Fort} 7;
\textbf{Ref} 6;
\textbf{Ment} 6
\pari \textbf{Strike} Greataxe \plus2 (2d8)



\pari \textbf{Behavior} Attack highest threat
\end{spelltargetinginfo}
\end{spellcontent}

\begin{monsterfooter}
\pari \textbf{Awareness} \plus6
\pari \textbf{Speed} 30 ft.;
\textbf{Space} 5 ft.;
\textbf{Reach} 5 ft.
\pari \textbf{Attributes}:
Str 6,
Dex 0,
Con 2,
Int \minus1,
Per 0,
Wil 0
\end{monsterfooter}
\end{monsection}


\subsubsection{Orc Grunt Abilities}

\begin{freeability}{Hit Harder}
The grunt makes a greataxe strike.
Its accuracy is reduced to 0, but the strike deals 4d6 damage.
\end{freeability}

\begin{monsection}{Orc Loudmouth}{3}[2]
\vspace{-1em}\spelltwocol{}{Medium humanoid}\vspace{-1em}
\begin{spellcontent}
\begin{spelltargetinginfo}
\spelltwocol{
\textbf{HP} 36;
\textbf{Bloodied} 18;
}
{\textbf{AP} 2/2}

\pari \textbf{Armor} 8;
\textbf{Fort} 9;
\textbf{Ref} 11;
\textbf{Ment} 11
\pari \textbf{Strike} Greataxe \plus5 (2d8)



\pari \textbf{Behavior} Attack highest threat
\end{spelltargetinginfo}
\end{spellcontent}

\begin{monsterfooter}
\pari \textbf{Awareness} \plus6
\pari \textbf{Speed} 30 ft.;
\textbf{Space} 5 ft.;
\textbf{Reach} 5 ft.
\pari \textbf{Attributes}:
Str 6,
Dex 4,
Con 2,
Int \minus1,
Per 4,
Wil 4
\end{monsterfooter}
\end{monsection}


\subsubsection{Orc Loudmouth Abilities}

\begin{freeability}{Hit Harder}
The loudmouth makes a greataxe strike.
Its accuracy is reduced to 3, but the strike deals 4d6 damage.
\end{freeability}

\vspace{0.5em}
\begin{freeability}{Hit That One Over There}[\glossterm{Sustain} (standard)]
The loudmouth chooses an enemy within Long range.
The loudmouth's \glossterm{allies} that can hear it gain a \plus2 bonus to \glossterm{accuracy} with \glossterm{strikes} against that target.
\end{freeability}

\begin{monsection}{Orc Shaman}{3}[2]
\vspace{-1em}\spelltwocol{}{Medium humanoid}\vspace{-1em}
\begin{spellcontent}
\begin{spelltargetinginfo}
\spelltwocol{
\textbf{HP} 42;
\textbf{Bloodied} 21;
}
{\textbf{AP} 2/3}

\pari \textbf{Armor} 8;
\textbf{Fort} 11;
\textbf{Ref} 11;
\textbf{Ment} 13
\pari \textbf{Strike} Greatstaff \plus4 (2d6)



\pari \textbf{Behavior} Attack highest threat
\end{spelltargetinginfo}
\end{spellcontent}

\begin{monsterfooter}
\pari \textbf{Awareness} \plus6
\pari \textbf{Speed} 30 ft.;
\textbf{Space} 5 ft.;
\textbf{Reach} 5 ft.
\pari \textbf{Attributes}:
Str 6,
Dex 4,
Con 4,
Int \minus1,
Per 0,
Wil 5
\end{monsterfooter}
\end{monsection}


\subsubsection{Orc Shaman Abilities}

\begin{freeability}{Hit Worse}
The shaman makes a \plus4 vs. Mental attack against one creature in Close range.
\hit The target takes a \minus3 penalty to accuracy with strikes as a \glossterm{condition}.
\crit As above, except that the penalty is increased to -6.
\end{freeability}

\vspace{0.5em}
\begin{freeability}{Hurt Less}
One ally in Close range heals 4d6 hit points.
\end{freeability}

\begin{monsection}{Orc Savage}{4}[1]
\vspace{-1em}\spelltwocol{}{Medium humanoid}\vspace{-1em}
\begin{spellcontent}
\begin{spelltargetinginfo}
\spelltwocol{
\textbf{HP} 24;
\textbf{Bloodied} 12;
}
{\textbf{AP} 1/0}

\pari \textbf{Armor} 8;
\textbf{Fort} 9;
\textbf{Ref} 11;
\textbf{Ment} 8
\pari \textbf{Strike} Greataxe \plus4 (2d10)



\pari \textbf{Behavior} Attack highest threat
\end{spelltargetinginfo}
\end{spellcontent}

\begin{monsterfooter}
\pari \textbf{Awareness} \plus6
\pari \textbf{Speed} 30 ft.;
\textbf{Space} 5 ft.;
\textbf{Reach} 5 ft.
\pari \textbf{Attributes}:
Str 9,
Dex 5,
Con 3,
Int \minus1,
Per 0,
Wil 0
\end{monsterfooter}
\end{monsection}


\subsubsection{Orc Savage Abilities}

\begin{freeability}{Hit Fast}
The savage makes a greataxe strike.
Its accuracy is 6.
\end{freeability}

\section{Magical Beasts}
\begin{monsection}{Dragon}[Large Red]{6}[4]
\vspace{-1em}\spelltwocol{}{Large magical beast}\vspace{-1em}
\begin{spellcontent}
\begin{spelltargetinginfo}
\spelltwocol{
\textbf{HP} 192;
\textbf{Bloodied} 96;
}
{\textbf{AP} 4/3}

\pari \textbf{Armor} 16;
\textbf{Fort} 18;
\textbf{Ref} 11;
\textbf{Ment} 18
\pari \textbf{Strike} Bite \plus11 (2d8)


\pari \textbf{Actions} One in action phase, one in delayed action phase
\pari \textbf{Behavior} Attack highest threat
\end{spelltargetinginfo}
\end{spellcontent}

\begin{monsterfooter}
\pari \textbf{Awareness} \plus6
\pari \textbf{Speed} 40 ft.;
\textbf{Space} 10 ft.;
\textbf{Reach} 10 ft.
\pari \textbf{Attributes}:
Str 9,
Dex 0,
Con 8,
Int 8,
Per 8,
Wil 8
\end{monsterfooter}
\end{monsection}

\begin{monsection}{Wyvern}{5}[3]
\vspace{-1em}\spelltwocol{}{Large magical beast}\vspace{-1em}
\begin{spellcontent}
\begin{spelltargetinginfo}
\spelltwocol{
\textbf{HP} 120;
\textbf{Bloodied} 60;
}
{\textbf{AP} 3/0}

\pari \textbf{Armor} 13;
\textbf{Fort} 16;
\textbf{Ref} 10;
\textbf{Ment} 11
\pari \textbf{Strike} Sting \plus8 (4d6) or bite \plus8 (2d10)


\pari \textbf{Actions} One in action phase, one in delayed action phase
\pari \textbf{Behavior} Attack highest threat
\end{spelltargetinginfo}
\end{spellcontent}

\begin{monsterfooter}
\pari \textbf{Awareness} \plus6
\pari \textbf{Speed} 40 ft.;
\textbf{Space} 10 ft.;
\textbf{Reach} 10 ft.
\pari \textbf{Attributes}:
Str 10,
Dex 3,
Con 7,
Int \minus7,
Per 6,
Wil 0
\end{monsterfooter}
\end{monsection}

\begin{monsection}{Ankheg}{7}[2]
\vspace{-1em}\spelltwocol{}{Large magical beast}\vspace{-1em}
\begin{spellcontent}
\begin{spelltargetinginfo}
\spelltwocol{
\textbf{HP} 112;
\textbf{Bloodied} 56;
}
{\textbf{AP} 3/0}

\pari \textbf{Armor} 14;
\textbf{Fort} 17;
\textbf{Ref} 11;
\textbf{Ment} 12
\pari \textbf{Strike} Bite \plus9 (4d6)



\pari \textbf{Behavior} Attack highest threat
\end{spelltargetinginfo}
\end{spellcontent}

\begin{monsterfooter}
\pari \textbf{Awareness} \plus6
\pari \textbf{Speed} 40 ft.;
\textbf{Space} 10 ft.;
\textbf{Reach} 10 ft.
\pari \textbf{Attributes}:
Str 12,
Dex 4,
Con 9,
Int \minus7,
Per 8,
Wil 0
\end{monsterfooter}
\end{monsection}


\subsubsection{Ankheg Abilities}

\begin{freeability}{Drag Prey}
This ability functions like the \textit{shove} ability (see \pcref{Shove}), except that the ankheg's accuracy is \plus14.
In addition, the ankheg can move with the target up to a maximum distance equal to its \glossterm{base speed}.
\end{freeability}

\vspace{0.5em}
\begin{freeability}{Spit Acid}
The ankheg makes a \plus9 vs. Armor attack against everything in a 5 ft. wide Medium line.
\hit Each target takes 4d8 acid damage, and creatures are \glossterm{sickened} as a \glossterm{condition}.
\end{freeability}

\begin{monsection}{Aranea}{5}[2]
\vspace{-1em}\spelltwocol{}{Medium magical beast}\vspace{-1em}
\begin{spellcontent}
\begin{spelltargetinginfo}
\spelltwocol{
\textbf{HP} 60;
\textbf{Bloodied} 30;
}
{\textbf{AP} 2/3}

\pari \textbf{Armor} 11;
\textbf{Fort} 11;
\textbf{Ref} 13;
\textbf{Ment} 15
\pari \textbf{Strike} Bite \plus8 (1d10)



\pari \textbf{Behavior} Attack highest threat
\end{spelltargetinginfo}
\end{spellcontent}

\begin{monsterfooter}
\pari \textbf{Awareness} \plus6
\pari \textbf{Speed} 30 ft.;
\textbf{Space} 5 ft.;
\textbf{Reach} 5 ft.
\pari \textbf{Attributes}:
Str 3,
Dex 6,
Con 3,
Int 6,
Per 7,
Wil 7
\end{monsterfooter}
\end{monsection}


\subsubsection{Aranea Abilities}

\begin{freeability}{Shapeshift}
The aranea makes a Disguise check to change its appearance.
It ignores all penalties for differences between its natural appearance and its intended appearance.
\end{freeability}

\begin{monsection}{Basilisk}{5}[2]
\vspace{-1em}\spelltwocol{}{Medium magical beast}\vspace{-1em}
\begin{spellcontent}
\begin{spelltargetinginfo}
\spelltwocol{
\textbf{HP} 80;
\textbf{Bloodied} 40;
}
{\textbf{AP} 2/0}

\pari \textbf{Armor} 12;
\textbf{Fort} 15;
\textbf{Ref} 9;
\textbf{Ment} 10
\pari \textbf{Strike} Bite \plus8 (2d6)



\pari \textbf{Behavior} Attack highest threat
\end{spelltargetinginfo}
\end{spellcontent}

\begin{monsterfooter}
\pari \textbf{Awareness} \plus6
\pari \textbf{Speed} 30 ft.;
\textbf{Space} 5 ft.;
\textbf{Reach} 5 ft.
\pari \textbf{Attributes}:
Str 6,
Dex \minus1,
Con 7,
Int \minus6,
Per 7,
Wil 0
\end{monsterfooter}
\end{monsection}


\subsubsection{Basilisk Abilities}

\begin{freeability}{Petrifying Gaze}
The basilisk makes a \plus8 vs. Fortitude attack against one creature in Medium range.
\hit The target is \glossterm{nauseated} as a \glossterm{condition}.
\crit As above, and as an additional condition, the target takes 2d6 physical damage at the end of each action phase.
If it takes vital damage in this way, it is petrified permanently.
\end{freeability}

\begin{monsection}{Behir}{8}[3]
\vspace{-1em}\spelltwocol{}{Huge magical beast}\vspace{-1em}
\begin{spellcontent}
\begin{spelltargetinginfo}
\spelltwocol{
\textbf{HP} 216;
\textbf{Bloodied} 108;
}
{\textbf{AP} 4/0}

\pari \textbf{Armor} 16;
\textbf{Fort} 21;
\textbf{Ref} 11;
\textbf{Ment} 14
\pari \textbf{Strike} Bite \plus10 (4d10) or claw \plus10 (4d8)


\pari \textbf{Actions} One in action phase, one in delayed action phase
\pari \textbf{Behavior} Attack highest threat
\end{spelltargetinginfo}
\end{spellcontent}

\begin{monsterfooter}
\pari \textbf{Awareness} \plus6
\pari \textbf{Speed} 50 ft.;
\textbf{Space} 15 ft.;
\textbf{Reach} 15 ft.
\pari \textbf{Attributes}:
Str 16,
Dex 5,
Con 11,
Int \minus3,
Per 5,
Wil 0
\end{monsterfooter}
\end{monsection}


\subsubsection{Behir Abilities}

\begin{freeability}{Electric Breath}
The behir makes a \plus10 vs. Armor attack against everything in a \areamed cone.
\hit Each target takes 7d10 electricity damage, and is \glossterm{dazed} as a \glossterm{condition}.
\end{freeability}

\vspace{0.5em}
\begin{freeability}{Natural Grab}
The behir makes a bite \glossterm{strike}.
In addition to the effects of the strike, it also makes a \plus14 vs. Fortitude and Reflex attack against the same target.
\hit The target is \glossterm{grappled} by the behir.
\end{freeability}

\vspace{0.5em}
\begin{apability}{Rake}
The behir makes four claw \glossterm{strikes} against a target that is \glossterm{grappled} by it.
\end{apability}

\begin{monsection}{Blink Dog}{3}[1]
\vspace{-1em}\spelltwocol{}{Medium magical beast}\vspace{-1em}
\begin{spellcontent}
\begin{spelltargetinginfo}
\spelltwocol{
\textbf{HP} 15;
\textbf{Bloodied} 7;
}
{\textbf{AP} 1/0}

\pari \textbf{Armor} 9;
\textbf{Fort} 7;
\textbf{Ref} 12;
\textbf{Ment} 7
\pari \textbf{Strike} Bite \plus4 (1d10)



\pari \textbf{Behavior} Attack highest threat
\end{spelltargetinginfo}
\end{spellcontent}

\begin{monsterfooter}
\pari \textbf{Awareness} \plus6
\pari \textbf{Speed} 30 ft.;
\textbf{Space} 5 ft.;
\textbf{Reach} 5 ft.
\pari \textbf{Attributes}:
Str 4,
Dex 5,
Con 0,
Int 0,
Per 4,
Wil 0
\end{monsterfooter}
\end{monsection}


\subsubsection{Blink Dog Abilities}

\begin{freeability}{Blink}
As a \glossterm{move action}, the blink dog can use this ability.
If it does, it teleports to an unoccupied location within Medium range.
\end{freeability}

\begin{monsection}{Centaur}{3}[2]
\vspace{-1em}\spelltwocol{}{Large magical beast}\vspace{-1em}
\begin{spellcontent}
\begin{spelltargetinginfo}
\spelltwocol{
\textbf{HP} 42;
\textbf{Bloodied} 21;
}
{\textbf{AP} 2/1}

\pari \textbf{Armor} 9;
\textbf{Fort} 11;
\textbf{Ref} 9;
\textbf{Ment} 9
\pari \textbf{Strike} Longsword \plus5 (1d10) or longbow \plus5 (1d10) or hoof \plus5 (1d8)



\pari \textbf{Behavior} Attack highest threat
\end{spelltargetinginfo}
\end{spellcontent}

\begin{monsterfooter}
\pari \textbf{Awareness} \plus6
\pari \textbf{Speed} 40 ft.;
\textbf{Space} 10 ft.;
\textbf{Reach} 10 ft.
\pari \textbf{Attributes}:
Str 4,
Dex 4,
Con 4,
Int 0,
Per 4,
Wil 2
\end{monsterfooter}
\end{monsection}

\begin{monsection}{Cockatrice}{3}[1]
\vspace{-1em}\spelltwocol{}{Small magical beast}\vspace{-1em}
\begin{spellcontent}
\begin{spelltargetinginfo}
\spelltwocol{
\textbf{HP} 15;
\textbf{Bloodied} 7;
}
{\textbf{AP} 1/2}

\pari \textbf{Armor} 9;
\textbf{Fort} 7;
\textbf{Ref} 14;
\textbf{Ment} 10
\pari \textbf{Strike} Bite \plus4 (1d8)



\pari \textbf{Behavior} Attack highest threat
\end{spelltargetinginfo}
\end{spellcontent}

\begin{monsterfooter}
\pari \textbf{Awareness} \plus6
\pari \textbf{Speed} 25 ft.;
\textbf{Space} 5 ft.;
\textbf{Reach} 5 ft.
\pari \textbf{Attributes}:
Str \minus4,
Dex 5,
Con 0,
Int \minus8,
Per 4,
Wil 4
\end{monsterfooter}
\end{monsection}


\subsubsection{Cockatrice Abilities}

\begin{freeability}{Petrifying Bite}
The cockatrice makes a bite \glossterm{strike}.
In addition to the strike's normal effects, the cockatrice also makes a \plus4 vs. Fortitude attack against the target.
\hit If the strike also hit, the target is \glossterm{nauseated} as a \glossterm{condition}.
\crit As above, and as an additional condition, the target takes 1d6 physical damage at the end of each action phase.
If it takes vital damage in this way, it is petrified permanently.
\end{freeability}

\begin{monsection}{Darkmantle}{1}[1]
\vspace{-1em}\spelltwocol{}{Small magical beast}\vspace{-1em}
\begin{spellcontent}
\begin{spelltargetinginfo}
\spelltwocol{
\textbf{HP} 6;
\textbf{Bloodied} 3;
}
{\textbf{AP} 1/0}

\pari \textbf{Armor} 5;
\textbf{Fort} 6;
\textbf{Ref} 7;
\textbf{Ment} 5
\pari \textbf{Strike} Slam \plus1 (1d6)



\pari \textbf{Behavior} Attack highest threat
\end{spelltargetinginfo}
\end{spellcontent}

\begin{monsterfooter}
\pari \textbf{Awareness} \plus6
\pari \textbf{Speed} 25 ft.;
\textbf{Space} 5 ft.;
\textbf{Reach} 5 ft.
\pari \textbf{Attributes}:
Str 1,
Dex 0,
Con 1,
Int \minus8,
Per 0,
Wil 0
\end{monsterfooter}
\end{monsection}


\subsubsection{Darkmantle Abilities}

\begin{freeability}{Natural Grab}
The darkmantle makes a slam \glossterm{strike}.
In addition to the effects of the strike, it also makes a +\minus1 vs. Fortitude and Reflex attack against the same target.
\hit The target is \glossterm{grappled} by the darkmantle.
\end{freeability}

\begin{monsection}{Frost Worm}{12}[3]
\vspace{-1em}\spelltwocol{}{Gargantuan magical beast}\vspace{-1em}
\begin{spellcontent}
\begin{spelltargetinginfo}
\spelltwocol{
\textbf{HP} 360;
\textbf{Bloodied} 180;
}
{\textbf{AP} 4/0}

\pari \textbf{Armor} 20;
\textbf{Fort} 27;
\textbf{Ref} 12;
\textbf{Ment} 18
\pari \textbf{Strike} Bite \plus15 (7d10) or slam \plus15 (7d10)
\pari \textbf{Immune} cold

\pari \textbf{Actions} One in action phase, one in delayed action phase
\pari \textbf{Behavior} Attack highest threat
\end{spelltargetinginfo}
\end{spellcontent}

\begin{monsterfooter}
\pari \textbf{Awareness} \plus6
\pari \textbf{Speed} 60 ft.;
\textbf{Space} 20 ft.;
\textbf{Reach} 20 ft.
\pari \textbf{Attributes}:
Str 22,
Dex 0,
Con 16,
Int \minus8,
Per 13,
Wil 0
\end{monsterfooter}
\end{monsection}


\subsubsection{Frost Worm Abilities}

\begin{freeability}{Frost Breath}[\glossterm{Cold}]
The frost worm makes a \plus15 vs. Fortitude attack against everything in a \arealarge cone from it.
\hit Each target takes 11d10 cold damage.
\end{freeability}

\vspace{0.5em}
\begin{apability}{Trill}[\glossterm{Compulsion}]
The frost worm emits a piercing noise that compels prey to stay still.
It makes a \plus15 vs. Mental attack against creatures in a \areahuge radius from it.
This area can pass through solid objects, including the ground, but every 5 feet of solid obstacle counts as 20 feet of distance.
\hit Each target is \glossterm{dazed} and \glossterm{immobilized} as two separate \glossterm{conditions}.
\crit Each target is \glossterm{stunned} and \glossterm{immobilized} as two separate \glossterm{conditions}.
\end{apability}

\parhead{Bitter Cold}
The frost worm's bite and slam strikes deal cold damage in addition to their other damage types.

\vspace{0.5em}\parhead{Death Throes}
When a frost worm is killed, its corpse turns to ice and shatters in a violent explosion.
It makes a \plus15 vs. Fortitude attack against everything in a \areahuge radius from it.
\hit Each target takes 13d10 cold and piercing damage.

\begin{monsection}{Girallon}{5}[2]
\vspace{-1em}\spelltwocol{}{Large magical beast}\vspace{-1em}
\begin{spellcontent}
\begin{spelltargetinginfo}
\spelltwocol{
\textbf{HP} 60;
\textbf{Bloodied} 30;
}
{\textbf{AP} 2/\minus1}

\pari \textbf{Armor} 13;
\textbf{Fort} 11;
\textbf{Ref} 15;
\textbf{Ment} 9
\pari \textbf{Strike} Claw \plus9 (2d6)



\pari \textbf{Behavior} Attack highest threat
\end{spelltargetinginfo}
\end{spellcontent}

\begin{monsterfooter}
\pari \textbf{Awareness} \plus6
\pari \textbf{Speed} 40 ft.;
\textbf{Space} 10 ft.;
\textbf{Reach} 10 ft.
\pari \textbf{Attributes}:
Str 9,
Dex 8,
Con 3,
Int \minus8,
Per 6,
Wil \minus1
\end{monsterfooter}
\end{monsection}

\begin{monsection}{Griffon}{4}[2]
\vspace{-1em}\spelltwocol{}{Large magical beast}\vspace{-1em}
\begin{spellcontent}
\begin{spelltargetinginfo}
\spelltwocol{
\textbf{HP} 48;
\textbf{Bloodied} 24;
}
{\textbf{AP} 2/0}

\pari \textbf{Armor} 12;
\textbf{Fort} 10;
\textbf{Ref} 14;
\textbf{Ment} 9
\pari \textbf{Strike} Talon \plus8 (1d10)



\pari \textbf{Behavior} Attack highest threat
\end{spelltargetinginfo}
\end{spellcontent}

\begin{monsterfooter}
\pari \textbf{Awareness} \plus6
\pari \textbf{Speed} 80 ft. fly ft.;
\textbf{Space} 10 ft.;
\textbf{Reach} 10 ft.
\pari \textbf{Attributes}:
Str 7,
Dex 7,
Con 3,
Int \minus4,
Per 5,
Wil 0
\end{monsterfooter}
\end{monsection}


\subsubsection{Griffon Abilities}

\begin{freeability}{Flyby Attack}
The griffin flies up to its flying movement speed.
It can make a talon strike at any point during this movement.
\end{freeability}

\begin{monsection}{Hydra, 5 Headed}{5}[4]
\vspace{-1em}\spelltwocol{}{Huge magical beast}\vspace{-1em}
\begin{spellcontent}
\begin{spelltargetinginfo}
\spelltwocol{
\textbf{HP} 180;
\textbf{Bloodied} 90;
}
{\textbf{AP} 4/0}

\pari \textbf{Armor} 12;
\textbf{Fort} 19;
\textbf{Ref} 8;
\textbf{Ment} 12
\pari \textbf{Strike} Bite \plus9 (4d6)


\pari \textbf{Actions} Five in action phase
\pari \textbf{Behavior} Attack highest threat
\end{spelltargetinginfo}
\end{spellcontent}

\begin{monsterfooter}
\pari \textbf{Awareness} \plus6
\pari \textbf{Speed} 50 ft.;
\textbf{Space} 15 ft.;
\textbf{Reach} 15 ft.
\pari \textbf{Attributes}:
Str 12,
Dex 0,
Con 8,
Int \minus8,
Per 6,
Wil 0
\end{monsterfooter}
\end{monsection}


\subsubsection{Hydra, 5 Headed Abilities}

\parhead{Multi-Headed}
A hydra can take a number of actions in each \glossterm{action phase} equal to the number of heads it has active.
At the end of each action phase, if the hydra took at least 35 damage during that phase, it loses one of its heads.
Severed heads leave behind a stump that can quickly grow new heads.

At the end of each delayed action phase, if the hydra has a severed stump, the stump is either sealed or it grows two new heads.
If the hydra took 15 acid, cold, or fire damage during that phase, the stump is sealed, and will stop growing new heads.
Otherwise, the hydra grows two new heads from the stump.
This grants it additional actions during the action phase as normal.

A hydra cannot sustain too many excess heads for a prolonged period of time.
At the end of each round, if the hydra has more heads than twice its normal head count, it loses an \glossterm{action point}.
If it has no action points remaining, the hydra collapses unconscious for 8 hours.
During that time, the excess heads shrivel and die, and any sealed stumps heal, restoring the hydra to its normal head count.

\begin{monsection}{Hydra, 6 Headed}{6}[4]
\vspace{-1em}\spelltwocol{}{Huge magical beast}\vspace{-1em}
\begin{spellcontent}
\begin{spelltargetinginfo}
\spelltwocol{
\textbf{HP} 216;
\textbf{Bloodied} 108;
}
{\textbf{AP} 4/0}

\pari \textbf{Armor} 13;
\textbf{Fort} 20;
\textbf{Ref} 9;
\textbf{Ment} 13
\pari \textbf{Strike} Bite \plus10 (4d6)


\pari \textbf{Actions} Six in action phase
\pari \textbf{Behavior} Attack highest threat
\end{spelltargetinginfo}
\end{spellcontent}

\begin{monsterfooter}
\pari \textbf{Awareness} \plus6
\pari \textbf{Speed} 50 ft.;
\textbf{Space} 15 ft.;
\textbf{Reach} 15 ft.
\pari \textbf{Attributes}:
Str 13,
Dex 0,
Con 9,
Int \minus8,
Per 7,
Wil 0
\end{monsterfooter}
\end{monsection}


\subsubsection{Hydra, 6 Headed Abilities}

\parhead{Multi-Headed}
A hydra can take a number of actions in each \glossterm{action phase} equal to the number of heads it has active.
At the end of each action phase, if the hydra took at least 40 damage during that phase, it loses one of its heads.
Severed heads leave behind a stump that can quickly grow new heads.

At the end of each round, if the hydra has a severed stump, the stump is either sealed or it grows two new heads.
If the hydra took 20 acid, cold, or fire damage during that phase, the stump is sealed, and will stop growing new heads.
Otherwise, the hydra grows two new heads from the stump.
This grants it additional actions during the action phase as normal.

A hydra cannot sustain too many excess heads for a prolonged period of time.
At the end of each round, if the hydra has more heads than twice its normal head count, it loses an \glossterm{action point}.
If it has no action points remaining, the hydra collapses unconscious for 8 hours.
During that time, the excess heads shrivel and die, and any sealed stumps heal, restoring the hydra to its normal head count.

\begin{monsection}{Minotaur}{4}[1]
\vspace{-1em}\spelltwocol{}{Large magical beast}\vspace{-1em}
\begin{spellcontent}
\begin{spelltargetinginfo}
\spelltwocol{
\textbf{HP} 24;
\textbf{Bloodied} 12;
}
{\textbf{AP} 1/0}

\pari \textbf{Armor} 8;
\textbf{Fort} 9;
\textbf{Ref} 7;
\textbf{Ment} 8
\pari \textbf{Strike} Greataxe \plus5 (2d10) or gore \plus5 (2d8)



\pari \textbf{Behavior} Attack highest threat
\end{spelltargetinginfo}
\end{spellcontent}

\begin{monsterfooter}
\pari \textbf{Awareness} \plus6
\pari \textbf{Speed} 40 ft.;
\textbf{Space} 10 ft.;
\textbf{Reach} 10 ft.
\pari \textbf{Attributes}:
Str 8,
Dex 3,
Con 3,
Int \minus2,
Per 5,
Wil 0
\end{monsterfooter}
\end{monsection}


\subsubsection{Minotaur Abilities}

\begin{freeability}{Impaling Charge}
The minotaur moves up to its speed in a single straight line and makes a gore \glossterm{strike} from its new location.
\end{freeability}

\parhead{Labyrinth Dweller}
The minotaur never gets lost or loses track of its current location.

\begin{monsection}{Thaumavore}{3}[1]
\vspace{-1em}\spelltwocol{}{Small magical beast}\vspace{-1em}
\begin{spellcontent}
\begin{spelltargetinginfo}
\spelltwocol{
\textbf{HP} 15;
\textbf{Bloodied} 7;
}
{\textbf{AP} 1/0}

\pari \textbf{Armor} 11;
\textbf{Fort} 7;
\textbf{Ref} 14;
\textbf{Ment} 7
\pari \textbf{Strike} Bite \plus3 (1d8)



\pari \textbf{Behavior} Attack highest threat that has a source of magic; if no souces of magic exist, attack highest threat
\end{spelltargetinginfo}
\end{spellcontent}

\begin{monsterfooter}
\pari \textbf{Awareness} \plus6
\pari \textbf{Speed} 25 ft.;
\textbf{Space} 5 ft.;
\textbf{Reach} 5 ft.
\pari \textbf{Attributes}:
Str 2,
Dex 5,
Con 0,
Int \minus7,
Per 0,
Wil 0
\end{monsterfooter}
\end{monsection}


\subsubsection{Thaumavore Abilities}

\parhead{Consume Magic}
The thaumavore gains a \plus4 bonus to \glossterm{defenses} against \glossterm{magical} abilities.
Whenever it resists a \glossterm{magical} attack, it heals hit points equal to twice the \glossterm{power} of the effect.

\vspace{0.5em}\parhead{Sense Magic}
The thaumavore can sense the location of all sources of magic within 100 feet of it.
This includes magic items, attuned magical abilities, and so on.

\begin{monsection}{Banehound}{5}[4]
\vspace{-1em}\spelltwocol{}{Huge magical beast}\vspace{-1em}
\begin{spellcontent}
\begin{spelltargetinginfo}
\spelltwocol{
\textbf{HP} 100;
\textbf{Bloodied} 50;
}
{\textbf{AP} 4/0}

\pari \textbf{Armor} 17;
\textbf{Fort} 12;
\textbf{Ref} 15;
\textbf{Ment} 12
\pari \textbf{Strike} Bite \plus10 (2d10)


\pari \textbf{Actions} One in action phase, one in delayed action phase
\pari \textbf{Behavior} Attack highest threat
\end{spelltargetinginfo}
\end{spellcontent}

\begin{monsterfooter}
\pari \textbf{Awareness} \plus6
\pari \textbf{Speed} 50 ft.;
\textbf{Space} 15 ft.;
\textbf{Reach} 15 ft.
\pari \textbf{Attributes}:
Str 10,
Dex 8,
Con 0,
Int 3,
Per 7,
Wil 0
\end{monsterfooter}
\end{monsection}

\begin{monsection}{Fleshfeeder}{4}[1]
\vspace{-1em}\spelltwocol{}{Medium magical beast}\vspace{-1em}
\begin{spellcontent}
\begin{spelltargetinginfo}
\spelltwocol{
\textbf{HP} 28;
\textbf{Bloodied} 14;
}
{\textbf{AP} 1/0}

\pari \textbf{Armor} 10;
\textbf{Fort} 11;
\textbf{Ref} 13;
\textbf{Ment} 8
\pari \textbf{Strike} Bite \plus5 (1d10)



\pari \textbf{Behavior} Attack highest threat
\end{spelltargetinginfo}
\end{spellcontent}

\begin{monsterfooter}
\pari \textbf{Awareness} \plus6
\pari \textbf{Speed} 30 ft.;
\textbf{Space} 5 ft.;
\textbf{Reach} 5 ft.
\pari \textbf{Attributes}:
Str 5,
Dex 6,
Con 5,
Int 0,
Per 5,
Wil 0
\end{monsterfooter}
\end{monsection}

\section{Monstrous Humanoids}
\begin{monsection}{Banshee}{3}[2]
\vspace{-1em}\spelltwocol{}{Medium monstrous humanoid}\vspace{-1em}
\begin{spellcontent}
\begin{spelltargetinginfo}
\spelltwocol{
\textbf{HP} 30;
\textbf{Bloodied} 15;
}
{\textbf{AP} 2/4}

\pari \textbf{Armor} 9;
\textbf{Fort} 8;
\textbf{Ref} 11;
\textbf{Ment} 15
\pari \textbf{Strike} Claw \plus5 (1d8)



\pari \textbf{Behavior} Attack highest threat
\end{spelltargetinginfo}
\end{spellcontent}

\begin{monsterfooter}
\pari \textbf{Awareness} \plus6
\pari \textbf{Speed} 30 ft.;
\textbf{Space} 5 ft.;
\textbf{Reach} 5 ft.
\pari \textbf{Attributes}:
Str 4,
Dex 4,
Con 0,
Int 0,
Per 4,
Wil 6
\end{monsterfooter}
\end{monsection}


\subsubsection{Banshee Abilities}

\begin{freeability}{Wail}
The banshee makes a \plus5 vs. Fortitude attack against everything in a Large radius.
\hit Each target takes 2d6 sonic damage, and creatures are sickened as a condition.
\end{freeability}

\begin{monsection}{Giant}[Hill]{6}[1]
\vspace{-1em}\spelltwocol{}{Large monstrous humanoid}\vspace{-1em}
\begin{spellcontent}
\begin{spelltargetinginfo}
\spelltwocol{
\textbf{HP} 42;
\textbf{Bloodied} 21;
}
{\textbf{AP} 1/0}

\pari \textbf{Armor} 11;
\textbf{Fort} 13;
\textbf{Ref} 7;
\textbf{Ment} 10
\pari \textbf{Strike} Greatclub \plus6 (4d6) or boulder \plus6 (2d10)



\pari \textbf{Behavior} Attack highest threat
\end{spelltargetinginfo}
\end{spellcontent}

\begin{monsterfooter}
\pari \textbf{Awareness} \plus6
\pari \textbf{Speed} 40 ft.;
\textbf{Space} 10 ft.;
\textbf{Reach} 10 ft.
\pari \textbf{Attributes}:
Str 11,
Dex \minus1,
Con 7,
Int \minus2,
Per 0,
Wil 0
\end{monsterfooter}
\end{monsection}


\subsubsection{Giant Abilities}

\begin{freeability}{Boulder Toss}
The giant makes a ranged boulder strike, treating it as a thrown weapon with a 100 ft.\ range increment.
\end{freeability}

\begin{monsection}{Giant}[Stone]{9}[1]
\vspace{-1em}\spelltwocol{}{Huge monstrous humanoid}\vspace{-1em}
\begin{spellcontent}
\begin{spelltargetinginfo}
\spelltwocol{
\textbf{HP} 63;
\textbf{Bloodied} 31;
}
{\textbf{AP} 2/0}

\pari \textbf{Armor} 16;
\textbf{Fort} 16;
\textbf{Ref} 8;
\textbf{Ment} 13
\pari \textbf{Strike} Greatclub \plus9 (5d10) or boulder \plus9 (4d10)



\pari \textbf{Behavior} Attack highest threat
\end{spelltargetinginfo}
\end{spellcontent}

\begin{monsterfooter}
\pari \textbf{Awareness} \plus6
\pari \textbf{Speed} 50 ft.;
\textbf{Space} 15 ft.;
\textbf{Reach} 15 ft.
\pari \textbf{Attributes}:
Str 16,
Dex \minus1,
Con 10,
Int \minus1,
Per 5,
Wil 0
\end{monsterfooter}
\end{monsection}


\subsubsection{Giant Abilities}

\begin{freeability}{Boulder Toss}
The giant makes a ranged boulder strike, treating it as a thrown weapon with a 100 ft.\ range increment.
\end{freeability}

\begin{monsection}{Giant}[Storm]{15}[1]
\vspace{-1em}\spelltwocol{}{Gargantuan monstrous humanoid}\vspace{-1em}
\begin{spellcontent}
\begin{spelltargetinginfo}
\spelltwocol{
\textbf{HP} 105;
\textbf{Bloodied} 52;
}
{\textbf{AP} 3/2}

\pari \textbf{Armor} 22;
\textbf{Fort} 22;
\textbf{Ref} 13;
\textbf{Ment} 22
\pari \textbf{Strike} Greatsword \plus16 (9d10)
\pari \textbf{Immune} deafened


\pari \textbf{Behavior} Attack highest threat
\end{spelltargetinginfo}
\end{spellcontent}

\begin{monsterfooter}
\pari \textbf{Awareness} \plus6
\pari \textbf{Speed} 60 ft.;
\textbf{Space} 20 ft.;
\textbf{Reach} 20 ft.
\pari \textbf{Attributes}:
Str 25,
Dex 0,
Con 16,
Int 8,
Per 16,
Wil 16
\end{monsterfooter}
\end{monsection}


\subsubsection{Giant Abilities}

\begin{freeability}{Lightning Javelin}
The storm giant makes a \plus16 vs. Fortitude attack against everything in a 10 ft. wide Large line.
\hit Each target takes 10d10 electricity damage.
\end{freeability}

\vspace{0.5em}
\begin{freeability}{Thunderstrike}
The storm giant makes a greatsword strike against a target.
If its attack result beats the target's Fortitude defense,
the target also takes 8d10 sonic damage
and is deafened as a condition.
\end{freeability}

\begin{monsection}{Hag}[Green]{5}[2]
\vspace{-1em}\spelltwocol{}{Medium monstrous humanoid}\vspace{-1em}
\begin{spellcontent}
\begin{spelltargetinginfo}
\spelltwocol{
\textbf{HP} 70;
\textbf{Bloodied} 35;
}
{\textbf{AP} 2/2}

\pari \textbf{Armor} 11;
\textbf{Fort} 13;
\textbf{Ref} 13;
\textbf{Ment} 13
\pari \textbf{Strike} Claw \plus8 (1d8)



\pari \textbf{Behavior} Attack highest threat
\end{spelltargetinginfo}
\end{spellcontent}

\begin{monsterfooter}
\pari \textbf{Awareness} \plus6
\pari \textbf{Speed} 30 ft.;
\textbf{Space} 5 ft.;
\textbf{Reach} 5 ft.
\pari \textbf{Attributes}:
Str 0,
Dex 6,
Con 6,
Int 6,
Per 7,
Wil 6
\end{monsterfooter}
\end{monsection}


\subsubsection{Hag Abilities}

\begin{freeability}{Vital Surge}
The hag makes a \plus8 vs. Fortitude attack against one creature within Medium range.
\hit The target takes 2d10 life damage.
\end{freeability}

\vspace{0.5em}
\begin{freeability}{Green Hag's Curse}
The hag makes a \plus8 vs. Mental atack aginst one creature within Medium range.
\hit As a condition, the target is either dazed, fatigued, or sickened, as the hag chooses.
\crit As three separate conditions, the target is dazed, fatigued, and sickened.
\end{freeability}

\parhead{Coven Rituals}
When three or more hags work together, they form a coven.
All members of the coven gain the ability to perform nature rituals as long as they work together.
Hags of any type can form a coven together.

\begin{monsection}{Medusa}{7}[2]
\vspace{-1em}\spelltwocol{}{Medium monstrous humanoid}\vspace{-1em}
\begin{spellcontent}
\begin{spelltargetinginfo}
\spelltwocol{
\textbf{HP} 70;
\textbf{Bloodied} 35;
}
{\textbf{AP} 3/2}

\pari \textbf{Armor} 12;
\textbf{Fort} 12;
\textbf{Ref} 13;
\textbf{Ment} 15
\pari \textbf{Strike} Longbow \plus10 (2d6) or snakes \plus10 (1d10)



\pari \textbf{Behavior} Attack highest threat
\end{spelltargetinginfo}
\end{spellcontent}

\begin{monsterfooter}
\pari \textbf{Awareness} \plus6
\pari \textbf{Speed} 30 ft.;
\textbf{Space} 5 ft.;
\textbf{Reach} 5 ft.
\pari \textbf{Attributes}:
Str 0,
Dex 4,
Con 0,
Int 4,
Per 9,
Wil 8
\end{monsterfooter}
\end{monsection}


\subsubsection{Medusa Abilities}

\begin{freeability}{Petrifying Gaze}
The medusa makes a \plus10 vs. Fortitude attack against one creature in Medium range.
\hit The target is \glossterm{nauseated} as a \glossterm{condition}.
\crit As above, and as an additional condition, the target takes 2d6 physical damage at the end of each action phase.
If it takes vital damage in this way, it is petrified permanently.
\end{freeability}

\begin{monsection}{Harpy}[Harpy Archer]{12}[1]
\vspace{-1em}\spelltwocol{}{Medium monstrous humanoid}\vspace{-1em}
\begin{spellcontent}
\begin{spelltargetinginfo}
\spelltwocol{
\textbf{HP} 72;
\textbf{Bloodied} 36;
}
{\textbf{AP} 2/3}

\pari \textbf{Armor} 21;
\textbf{Fort} 17;
\textbf{Ref} 23;
\textbf{Ment} 21
\pari \textbf{Strike} Longbow \plus15 (4d6)



\pari \textbf{Behavior} Attack highest threat
\end{spelltargetinginfo}
\end{spellcontent}

\begin{monsterfooter}
\pari \textbf{Awareness} \plus6
\pari \textbf{Speed} 30 ft.;
\textbf{Space} 5 ft.;
\textbf{Reach} 5 ft.
\pari \textbf{Attributes}:
Str 13,
Dex 15,
Con 7,
Int 7,
Per 15,
Wil 14
\end{monsterfooter}
\end{monsection}

\section{Outsiders}
\begin{monsection}{Angel}[Astral Deva]{14}[2]
\vspace{-1em}\spelltwocol{}{Medium outsider}\vspace{-1em}
\begin{spellcontent}
\begin{spelltargetinginfo}
\spelltwocol{
\textbf{HP} 224;
\textbf{Bloodied} 112;
}
{\textbf{AP} 4/3}

\pari \textbf{Armor} 25;
\textbf{Fort} 22;
\textbf{Ref} 24;
\textbf{Ment} 26
\pari \textbf{Strike} Mace \plus17 (4d10)



\pari \textbf{Behavior} Attack highest threat
\end{spelltargetinginfo}
\end{spellcontent}

\begin{monsterfooter}
\pari \textbf{Awareness} \plus6
\pari \textbf{Speed} 30 ft.;
\textbf{Space} 5 ft.;
\textbf{Reach} 5 ft.
\pari \textbf{Attributes}:
Str 16,
Dex 16,
Con 16,
Int 16,
Per 16,
Wil 16
\end{monsterfooter}
\end{monsection}


\subsubsection{Angel Abilities}

\begin{freeability}{Smite}
The angel makes a melee \glossterm{strike}.
If its target is evil, it gains a \plus2 bonus to accuracy and a \plus2d bonus to damage on the strike.
\end{freeability}

\vspace{0.5em}
\begin{freeability}{Angel's Grace}
One \glossterm{ally} within reach heals 8d10 hit points.
\end{freeability}

\begin{monsection}{Arrowhawk}{3}[1]
\vspace{-1em}\spelltwocol{}{Medium outsider}\vspace{-1em}
\begin{spellcontent}
\begin{spelltargetinginfo}
\spelltwocol{
\textbf{HP} 12;
\textbf{Bloodied} 6;
}
{\textbf{AP} 1/0}

\pari \textbf{Armor} 10;
\textbf{Fort} 4;
\textbf{Ref} 14;
\textbf{Ment} 9
\pari \textbf{Strike} Bite \plus4 (1d8)



\pari \textbf{Behavior} Attack lowest threat
\end{spelltargetinginfo}
\end{spellcontent}

\begin{monsterfooter}
\pari \textbf{Awareness} \plus6
\pari \textbf{Speed} 60 ft. fly (good) ft.;
\textbf{Space} 5 ft.;
\textbf{Reach} 5 ft.
\pari \textbf{Attributes}:
Str 2,
Dex 6,
Con \minus1,
Int 0,
Per 4,
Wil 0
\end{monsterfooter}
\end{monsection}


\subsubsection{Arrowhawk Abilities}

\begin{freeability}{Electrobolt}
The arrowhawk makes a \plus4 vs. Fortitude attack against one creature or object in Medium range.
\hit The target takes 2d6 electricity damage.
\end{freeability}

\begin{monsection}{Demon}[Bebelith]{11}[3]
\vspace{-1em}\spelltwocol{}{Huge outsider}\vspace{-1em}
\begin{spellcontent}
\begin{spelltargetinginfo}
\spelltwocol{
\textbf{HP} 231;
\textbf{Bloodied} 115;
}
{\textbf{AP} 4/0}

\pari \textbf{Armor} 21;
\textbf{Fort} 18;
\textbf{Ref} 18;
\textbf{Ment} 19
\pari \textbf{Strike} Bite \plus14 (4d10)


\pari \textbf{Actions} One in action phase, one in delayed action phase
\pari \textbf{Behavior} Attack highest threat
\end{spelltargetinginfo}
\end{spellcontent}

\begin{monsterfooter}
\pari \textbf{Awareness} \plus6
\pari \textbf{Speed} 50 ft.;
\textbf{Space} 15 ft.;
\textbf{Reach} 15 ft.
\pari \textbf{Attributes}:
Str 16,
Dex 13,
Con 12,
Int 0,
Per 12,
Wil 0
\end{monsterfooter}
\end{monsection}


\subsubsection{Demon Abilities}

\begin{freeability}{Venomous Bite}
The bebelith makes a bite strike.
If it hits, and the attack result beats the target's Fortitude defense, the target is also poisoned as a condition.
If the target is poisoned, it takes 6d10 poison damage at the end of each action phase after the first round.
\end{freeability}

\begin{monsection}{Hell Hound}{4}[1]
\vspace{-1em}\spelltwocol{}{Medium outsider}\vspace{-1em}
\begin{spellcontent}
\begin{spelltargetinginfo}
\spelltwocol{
\textbf{HP} 20;
\textbf{Bloodied} 10;
}
{\textbf{AP} 1/0}

\pari \textbf{Armor} 10;
\textbf{Fort} 6;
\textbf{Ref} 13;
\textbf{Ment} 10
\pari \textbf{Strike} Bite \plus5 (1d10)
\pari \textbf{Immune} fire damage


\pari \textbf{Behavior} Attack highest threat
\end{spelltargetinginfo}
\end{spellcontent}

\begin{monsterfooter}
\pari \textbf{Awareness} \plus6
\pari \textbf{Speed} 30 ft.;
\textbf{Space} 5 ft.;
\textbf{Reach} 5 ft.
\pari \textbf{Attributes}:
Str 3,
Dex 6,
Con 0,
Int \minus3,
Per 5,
Wil 0
\end{monsterfooter}
\end{monsection}


\subsubsection{Hell Hound Abilities}

\begin{freeability}{Fire Breath}
The hell hound makes a \plus5 vs. Armor attack against everything in a Medium cone.
\hit Each target takes 2d6 fire damage.
\end{freeability}

\begin{monsection}{Salamander}[Flamebrother]{4}[1]
\vspace{-1em}\spelltwocol{}{Medium outsider}\vspace{-1em}
\begin{spellcontent}
\begin{spelltargetinginfo}
\spelltwocol{
\textbf{HP} 20;
\textbf{Bloodied} 10;
}
{\textbf{AP} 1/0}

\pari \textbf{Armor} 11;
\textbf{Fort} 6;
\textbf{Ref} 11;
\textbf{Ment} 10
\pari \textbf{Strike} Spear \plus4 (2d6) or tail slam \plus4 (2d6)
\pari \textbf{Immune} fire damage


\pari \textbf{Behavior} Attack highest threat
\end{spelltargetinginfo}
\end{spellcontent}

\begin{monsterfooter}
\pari \textbf{Awareness} \plus6
\pari \textbf{Speed} 30 ft.;
\textbf{Space} 5 ft.;
\textbf{Reach} 5 ft.
\pari \textbf{Attributes}:
Str 7,
Dex 5,
Con 0,
Int 3,
Per 3,
Wil 0
\end{monsterfooter}
\end{monsection}


\subsubsection{Salamander Abilities}

\begin{apability}{Flame Aura}[\glossterm{Sustain} (standard)]
The salamander intensifies its natural body heat, creating a burning aura around it.
At the end of each action phase, the salamander makes a \plus4 vs. Armor
attack against everything within a Medium radius emanation of it.
\hit Each target takes 2d6 fire damage.
\end{apability}

\vspace{0.5em}
\begin{freeability}{Natural Grab}
The salamander makes a tail slam \glossterm{strike}.
In addition to the effects of the strike, it also makes a \plus4 vs. Fortitude and Reflex attack against the same target.
\hit The target is \glossterm{grappled} by the salamander.
\end{freeability}

\begin{monsection}{Janni}{7}[1]
\vspace{-1em}\spelltwocol{}{Medium outsider}\vspace{-1em}
\begin{spellcontent}
\begin{spelltargetinginfo}
\spelltwocol{
\textbf{HP} 35;
\textbf{Bloodied} 17;
}
{\textbf{AP} 2/1}

\pari \textbf{Armor} 17;
\textbf{Fort} 9;
\textbf{Ref} 16;
\textbf{Ment} 14
\pari \textbf{Strike} Shortsword \plus9 (2d6)



\pari \textbf{Behavior} Attack highest threat
\end{spelltargetinginfo}
\end{spellcontent}

\begin{monsterfooter}
\pari \textbf{Awareness} \plus6
\pari \textbf{Speed} 30 ft.;
\textbf{Space} 5 ft.;
\textbf{Reach} 5 ft.
\pari \textbf{Attributes}:
Str 8,
Dex 9,
Con 0,
Int 4,
Per 8,
Wil 4
\end{monsterfooter}
\end{monsection}

\begin{monsection}{Salamander}[Battlemaster]{5}[3]
\vspace{-1em}\spelltwocol{}{Medium outsider}\vspace{-1em}
\begin{spellcontent}
\begin{spelltargetinginfo}
\spelltwocol{
\textbf{HP} 75;
\textbf{Bloodied} 37;
}
{\textbf{AP} 3/1}

\pari \textbf{Armor} 14;
\textbf{Fort} 9;
\textbf{Ref} 14;
\textbf{Ment} 14
\pari \textbf{Strike} Spear \plus8 (2d8) or tail slam \plus8 (2d8)


\pari \textbf{Actions} One in action phase, one in delayed action phase
\pari \textbf{Behavior} Attack highest threat
\end{spelltargetinginfo}
\end{spellcontent}

\begin{monsterfooter}
\pari \textbf{Awareness} \plus6
\pari \textbf{Speed} 30 ft.;
\textbf{Space} 5 ft.;
\textbf{Reach} 5 ft.
\pari \textbf{Attributes}:
Str 8,
Dex 6,
Con 0,
Int 3,
Per 6,
Wil 3
\end{monsterfooter}
\end{monsection}


\subsubsection{Salamander Abilities}

\begin{apability}{Flame Aura}[\glossterm{Sustain} (standard)]
The salamander intensifies its natural body heat, creating a burning aura around it.
At the end of each action phase, the salamander makes a \plus4 vs. Armor
attack against everything within a Medium radius emanation of it.
\hit Each target takes 2d6 fire damage.
\end{apability}

\vspace{0.5em}
\begin{freeability}{Natural Grab}
The salamander makes a tail slam \glossterm{strike}.
In addition to the effects of the strike, it also makes a \plus8 vs. Fortitude and Reflex attack against the same target.
\hit The target is \glossterm{grappled} by the salamander.
\end{freeability}

\section{Undead}
\begin{monsection}{Allip}{4}[1]
\vspace{-1em}\spelltwocol{}{Medium undead}\vspace{-1em}
\begin{spellcontent}
\begin{spelltargetinginfo}
\spelltwocol{
\textbf{HP} 20;
\textbf{Bloodied} 10;
}
{\textbf{AP} 1/3}

\pari \textbf{Armor} 10;
\textbf{Fort} 6;
\textbf{Ref} 13;
\textbf{Ment} 15
\pari \textbf{Strike} Draining touch \plus6 vs. Reflex (2d6)



\pari \textbf{Behavior} Attack highest threat
\end{spelltargetinginfo}
\end{spellcontent}

\begin{monsterfooter}
\pari \textbf{Awareness} \plus6
\pari \textbf{Speed} 30 ft.;
\textbf{Space} 5 ft.;
\textbf{Reach} 5 ft.
\pari \textbf{Attributes}:
Str 0,
Dex 6,
Con 0,
Int 0,
Per 0,
Wil 6
\end{monsterfooter}
\end{monsection}

\begin{monsection}{Spectre}{7}[2]
\vspace{-1em}\spelltwocol{}{Medium undead}\vspace{-1em}
\begin{spellcontent}
\begin{spelltargetinginfo}
\spelltwocol{
\textbf{HP} 70;
\textbf{Bloodied} 35;
}
{\textbf{AP} 3/4}

\pari \textbf{Armor} 15;
\textbf{Fort} 10;
\textbf{Ref} 19;
\textbf{Ment} 21
\pari \textbf{Strike} Draining touch \plus11 vs. Reflex (2d10)



\pari \textbf{Behavior} Attack highest threat
\end{spelltargetinginfo}
\end{spellcontent}

\begin{monsterfooter}
\pari \textbf{Awareness} \plus6
\pari \textbf{Speed} 30 ft.;
\textbf{Space} 5 ft.;
\textbf{Reach} 5 ft.
\pari \textbf{Attributes}:
Str 0,
Dex 10,
Con 0,
Int 0,
Per 8,
Wil 10
\end{monsterfooter}
\end{monsection}

\begin{monsection}{Dirgewalker}{4}[3]
\vspace{-1em}\spelltwocol{}{Medium undead}\vspace{-1em}
\begin{spellcontent}
\begin{spelltargetinginfo}
\spelltwocol{
\textbf{HP} 60;
\textbf{Bloodied} 30;
}
{\textbf{AP} 3/3}

\pari \textbf{Armor} 15;
\textbf{Fort} 8;
\textbf{Ref} 17;
\textbf{Ment} 17
\pari \textbf{Strike} Claw \plus9 (1d8)


\pari \textbf{Actions} One in action phase, one in delayed action phase
\pari \textbf{Behavior} Attack highest threat
\end{spelltargetinginfo}
\end{spellcontent}

\begin{monsterfooter}
\pari \textbf{Awareness} \plus6
\pari \textbf{Speed} 30 ft.;
\textbf{Space} 5 ft.;
\textbf{Reach} 5 ft.
\pari \textbf{Attributes}:
Str 0,
Dex 7,
Con 0,
Int 3,
Per 6,
Wil 6
\end{monsterfooter}
\end{monsection}


\subsubsection{Dirgewalker Abilities}

\begin{attuneability}{Animating Caper}[\glossterm{Attune} (self)]
One corpse within Close range is animated as a skeleton under the dirgewalker's control.
\end{attuneability}

\vspace{0.5em}
\begin{freeability}{Mournful Dirge}
The dirgewalker makes a \plus8 vs. Mental attack against all creatures in a Medium radius.
\hit Each target is dazed as a condition.
\crit Each target is stunned as a condition.
\end{freeability}

\begin{monsection}{Skeleton}{1}[1]
\vspace{-1em}\spelltwocol{}{Medium undead}\vspace{-1em}
\begin{spellcontent}
\begin{spelltargetinginfo}
\spelltwocol{
\textbf{HP} 5;
\textbf{Bloodied} 2;
}
{\textbf{AP} 1/0}

\pari \textbf{Armor} 7;
\textbf{Fort} 3;
\textbf{Ref} 8;
\textbf{Ment} 7
\pari \textbf{Strike} Claw \plus2 (1d6)



\pari \textbf{Behavior} Attack highest threat
\end{spelltargetinginfo}
\end{spellcontent}

\begin{monsterfooter}
\pari \textbf{Awareness} \plus6
\pari \textbf{Speed} 30 ft.;
\textbf{Space} 5 ft.;
\textbf{Reach} 5 ft.
\pari \textbf{Attributes}:
Str 2,
Dex 2,
Con 0,
Int 0,
Per 0,
Wil 0
\end{monsterfooter}
\end{monsection}


\subsubsection{Skeleton Abilities}

\parhead{Hard Bones}
The skeleton has \glossterm{damage reduction} 1 against piercing and slashing damage.

\begin{monsection}{Skeleton}[Warrior]{4}[1]
\vspace{-1em}\spelltwocol{}{Medium undead}\vspace{-1em}
\begin{spellcontent}
\begin{spelltargetinginfo}
\spelltwocol{
\textbf{HP} 20;
\textbf{Bloodied} 10;
}
{\textbf{AP} 1/0}

\pari \textbf{Armor} 11;
\textbf{Fort} 6;
\textbf{Ref} 13;
\textbf{Ment} 10
\pari \textbf{Strike} Longsword \plus4 (1d10)



\pari \textbf{Behavior} Attack highest threat
\end{spelltargetinginfo}
\end{spellcontent}

\begin{monsterfooter}
\pari \textbf{Awareness} \plus6
\pari \textbf{Speed} 30 ft.;
\textbf{Space} 5 ft.;
\textbf{Reach} 5 ft.
\pari \textbf{Attributes}:
Str 5,
Dex 6,
Con 0,
Int 0,
Per 0,
Wil 0
\end{monsterfooter}
\end{monsection}


\subsubsection{Skeleton Abilities}

\parhead{Hard Bones}
The skeleton has \glossterm{damage reduction} 1 against piercing and slashing damage.

\begin{monsection}{Skeleton}[Mage]{4}[1]
\vspace{-1em}\spelltwocol{}{Medium undead}\vspace{-1em}
\begin{spellcontent}
\begin{spelltargetinginfo}
\spelltwocol{
\textbf{HP} 20;
\textbf{Bloodied} 10;
}
{\textbf{AP} 1/3}

\pari \textbf{Armor} 11;
\textbf{Fort} 6;
\textbf{Ref} 13;
\textbf{Ment} 15
\pari \textbf{Strike} Claw \plus6 (1d8)



\pari \textbf{Behavior} Attack highest threat
\end{spelltargetinginfo}
\end{spellcontent}

\begin{monsterfooter}
\pari \textbf{Awareness} \plus6
\pari \textbf{Speed} 30 ft.;
\textbf{Space} 5 ft.;
\textbf{Reach} 5 ft.
\pari \textbf{Attributes}:
Str 0,
Dex 6,
Con 0,
Int 0,
Per 0,
Wil 6
\end{monsterfooter}
\end{monsection}


\subsubsection{Skeleton Abilities}

\begin{freeability}{Drain Life}[\glossterm{Life}]
The skeleton mage makes a \plus4 vs. Fortitude attack against a creature in \rngmed range.
\hit The target takes 2d6 life damage.
In addition, the skeleton mage heals 6 hit points.
\end{freeability}

\vspace{0.5em}
\begin{freeability}{Terror}[\glossterm{Life}]
The skeleton mage makes a \plus4 vs. Mental attack against a creature in \rngmed range.
\hit The target is \frightened by you as a \glossterm{condition}.
\crit The target is \panicked by you as a \glossterm{condition}.
\end{freeability}

\parhead{Hard Bones}
The skeleton has \glossterm{damage reduction} 1 against piercing and slashing damage.

\begin{monsection}{Skeleton}[Warrior]{4}[3]
\vspace{-1em}\spelltwocol{}{Large undead}\vspace{-1em}
\begin{spellcontent}
\begin{spelltargetinginfo}
\spelltwocol{
\textbf{HP} 72;
\textbf{Bloodied} 36;
}
{\textbf{AP} 3/0}

\pari \textbf{Armor} 14;
\textbf{Fort} 9;
\textbf{Ref} 15;
\textbf{Ment} 12
\pari \textbf{Strike} Greatsword \plus6 (2d10)


\pari \textbf{Actions} One in action phase, one in delayed action phase
\pari \textbf{Behavior} Attack highest threat
\end{spelltargetinginfo}
\end{spellcontent}

\begin{monsterfooter}
\pari \textbf{Awareness} \plus6
\pari \textbf{Speed} 40 ft.;
\textbf{Space} 10 ft.;
\textbf{Reach} 10 ft.
\pari \textbf{Attributes}:
Str 8,
Dex 7,
Con 3,
Int 0,
Per 0,
Wil 0
\end{monsterfooter}
\end{monsection}


\subsubsection{Skeleton Abilities}

\parhead{Hard Bones}
The skeleton has \glossterm{damage reduction} 1 against piercing and slashing damage.

\begin{monsection}{Zombie}{1}[1]
\vspace{-1em}\spelltwocol{}{Medium undead}\vspace{-1em}
\begin{spellcontent}
\begin{spelltargetinginfo}
\spelltwocol{
\textbf{HP} 8;
\textbf{Bloodied} 4;
}
{\textbf{AP} 1/0}

\pari \textbf{Armor} 4;
\textbf{Fort} 8;
\textbf{Ref} 5;
\textbf{Ment} 7
\pari \textbf{Strike} Slam \plus1 (1d6)



\pari \textbf{Behavior} Attack highest threat
\end{spelltargetinginfo}
\end{spellcontent}

\begin{monsterfooter}
\pari \textbf{Awareness} \plus6
\pari \textbf{Speed} 30 ft.;
\textbf{Space} 5 ft.;
\textbf{Reach} 5 ft.
\pari \textbf{Attributes}:
Str 1,
Dex 0,
Con 3,
Int 0,
Per 0,
Wil 0
\end{monsterfooter}
\end{monsection}


\subsubsection{Zombie Abilities}

\parhead{Slow}
The zombie does not act during the \glossterm{action phase}.
Instead, it acts during the \glossterm{delayed action phase}.

\vspace{0.5em}\parhead{Soft Flesh}
The zombie has \glossterm{damage reduction} 3 against piercing and bludgeoning damage.

\begin{monsection}{Zombie}[Warrior]{2}[2]
\vspace{-1em}\spelltwocol{}{Medium undead}\vspace{-1em}
\begin{spellcontent}
\begin{spelltargetinginfo}
\spelltwocol{
\textbf{HP} 36;
\textbf{Bloodied} 18;
}
{\textbf{AP} 2/0}

\pari \textbf{Armor} 6;
\textbf{Fort} 12;
\textbf{Ref} 7;
\textbf{Ment} 9
\pari \textbf{Strike} Slam \plus3 (1d10)



\pari \textbf{Behavior} Attack highest threat
\end{spelltargetinginfo}
\end{spellcontent}

\begin{monsterfooter}
\pari \textbf{Awareness} \plus6
\pari \textbf{Speed} 30 ft.;
\textbf{Space} 5 ft.;
\textbf{Reach} 5 ft.
\pari \textbf{Attributes}:
Str 4,
Dex 0,
Con 5,
Int 0,
Per 0,
Wil 0
\end{monsterfooter}
\end{monsection}


\subsubsection{Zombie Abilities}

\parhead{Slow}
The zombie does not act during the \glossterm{action phase}.
Instead, it acts during the \glossterm{delayed action phase}.

\vspace{0.5em}\parhead{Soft Flesh}
The zombie has \glossterm{damage reduction} 5 against piercing and bludgeoning damage.

\begin{monsection}{Zombie}[Hulking]{3}[2]
\vspace{-1em}\spelltwocol{}{Large undead}\vspace{-1em}
\begin{spellcontent}
\begin{spelltargetinginfo}
\spelltwocol{
\textbf{HP} 54;
\textbf{Bloodied} 27;
}
{\textbf{AP} 2/0}

\pari \textbf{Armor} 8;
\textbf{Fort} 13;
\textbf{Ref} 6;
\textbf{Ment} 10
\pari \textbf{Strike} Slam \plus4 (2d6)



\pari \textbf{Behavior} Attack highest threat
\end{spelltargetinginfo}
\end{spellcontent}

\begin{monsterfooter}
\pari \textbf{Awareness} \plus6
\pari \textbf{Speed} 40 ft.;
\textbf{Space} 10 ft.;
\textbf{Reach} 10 ft.
\pari \textbf{Attributes}:
Str 6,
Dex 0,
Con 6,
Int 0,
Per 0,
Wil 0
\end{monsterfooter}
\end{monsection}


\subsubsection{Zombie Abilities}

\parhead{Slow}
The zombie does not act during the \glossterm{action phase}.
Instead, it acts during the \glossterm{delayed action phase}.

\vspace{0.5em}\parhead{Soft Flesh}
The zombie has \glossterm{damage reduction} 6 against piercing and bludgeoning damage.

\begin{monsection}{Zombie}[Captain]{3}[3]
\vspace{-1em}\spelltwocol{}{Medium undead}\vspace{-1em}
\begin{spellcontent}
\begin{spelltargetinginfo}
\spelltwocol{
\textbf{HP} 81;
\textbf{Bloodied} 40;
}
{\textbf{AP} 3/0}

\pari \textbf{Armor} 8;
\textbf{Fort} 14;
\textbf{Ref} 9;
\textbf{Ment} 11
\pari \textbf{Strike} Slam \plus5 (2d6)


\pari \textbf{Actions} One in action phase, one in delayed action phase
\pari \textbf{Behavior} Attack highest threat
\end{spelltargetinginfo}
\end{spellcontent}

\begin{monsterfooter}
\pari \textbf{Awareness} \plus6
\pari \textbf{Speed} 30 ft.;
\textbf{Space} 5 ft.;
\textbf{Reach} 5 ft.
\pari \textbf{Attributes}:
Str 6,
Dex 0,
Con 6,
Int 0,
Per 0,
Wil 0
\end{monsterfooter}
\end{monsection}


\subsubsection{Zombie Abilities}

\parhead{Slow}
The zombie does not act during the \glossterm{action phase}.
Instead, it acts during the \glossterm{delayed action phase}.

\vspace{0.5em}\parhead{Soft Flesh}
The zombie has \glossterm{damage reduction} 6 against piercing and bludgeoning damage.

\begin{monsection}{Zombie}[Elite]{4}[1]
\vspace{-1em}\spelltwocol{}{Medium undead}\vspace{-1em}
\begin{spellcontent}
\begin{spelltargetinginfo}
\spelltwocol{
\textbf{HP} 40;
\textbf{Bloodied} 20;
}
{\textbf{AP} 1/0}

\pari \textbf{Armor} 8;
\textbf{Fort} 15;
\textbf{Ref} 8;
\textbf{Ment} 10
\pari \textbf{Strike} Slam \plus4 (2d6)



\pari \textbf{Behavior} Attack highest threat
\end{spelltargetinginfo}
\end{spellcontent}

\begin{monsterfooter}
\pari \textbf{Awareness} \plus6
\pari \textbf{Speed} 30 ft.;
\textbf{Space} 5 ft.;
\textbf{Reach} 5 ft.
\pari \textbf{Attributes}:
Str 6,
Dex 0,
Con 8,
Int 0,
Per 0,
Wil 0
\end{monsterfooter}
\end{monsection}


\subsubsection{Zombie Abilities}

\parhead{Slow}
The zombie does not act during the \glossterm{action phase}.
Instead, it acts during the \glossterm{delayed action phase}.

\vspace{0.5em}\parhead{Soft Flesh}
The zombie has \glossterm{damage reduction} 8 against piercing and bludgeoning damage.

\begin{monsection}{Unliving Mother}{2}[2]
\vspace{-1em}\spelltwocol{}{Medium undead}\vspace{-1em}
\begin{spellcontent}
\begin{spelltargetinginfo}
\spelltwocol{
\textbf{HP} 32;
\textbf{Bloodied} 16;
}
{\textbf{AP} 2/2}

\pari \textbf{Armor} 6;
\textbf{Fort} 10;
\textbf{Ref} 7;
\textbf{Ment} 12
\pari \textbf{Strike} Bite \plus3 (1d10)



\pari \textbf{Behavior} Attack highest threat
\end{spelltargetinginfo}
\end{spellcontent}

\begin{monsterfooter}
\pari \textbf{Awareness} \plus6
\pari \textbf{Speed} 30 ft.;
\textbf{Space} 5 ft.;
\textbf{Reach} 5 ft.
\pari \textbf{Attributes}:
Str 4,
Dex 0,
Con 4,
Int 0,
Per 0,
Wil 3
\end{monsterfooter}
\end{monsection}

\begin{monsection}{Unliving Mother}{2}[3]
\vspace{-1em}\spelltwocol{}{Medium undead}\vspace{-1em}
\begin{spellcontent}
\begin{spelltargetinginfo}
\spelltwocol{
\textbf{HP} 48;
\textbf{Bloodied} 24;
}
{\textbf{AP} 3/0}

\pari \textbf{Armor} 9;
\textbf{Fort} 11;
\textbf{Ref} 11;
\textbf{Ment} 10
\pari \textbf{Strike} Bite \plus6 (1d10)


\pari \textbf{Actions} One in action phase, one in delayed action phase
\pari \textbf{Behavior} Attack highest threat
\end{spelltargetinginfo}
\end{spellcontent}

\begin{monsterfooter}
\pari \textbf{Awareness} \plus6
\pari \textbf{Speed} 30 ft.;
\textbf{Space} 5 ft.;
\textbf{Reach} 5 ft.
\pari \textbf{Attributes}:
Str 4,
Dex 3,
Con 4,
Int 0,
Per 4,
Wil 0
\end{monsterfooter}
\end{monsection}

\begin{monsection}{Corrupted}[Mage]{4}[3]
\vspace{-1em}\spelltwocol{}{Medium undead}\vspace{-1em}
\begin{spellcontent}
\begin{spelltargetinginfo}
\spelltwocol{
\textbf{HP} 96;
\textbf{Bloodied} 48;
}
{\textbf{AP} 3/4}

\pari \textbf{Armor} 10;
\textbf{Fort} 13;
\textbf{Ref} 10;
\textbf{Ment} 19
\pari \textbf{Strike} Slam \plus8 (1d10)


\pari \textbf{Actions} One in action phase, one in delayed action phase
\pari \textbf{Behavior} Attack highest threat
\end{spelltargetinginfo}
\end{spellcontent}

\begin{monsterfooter}
\pari \textbf{Awareness} \plus6
\pari \textbf{Speed} 30 ft.;
\textbf{Space} 5 ft.;
\textbf{Reach} 5 ft.
\pari \textbf{Attributes}:
Str 5,
Dex 0,
Con 6,
Int 0,
Per 6,
Wil 7
\end{monsterfooter}
\end{monsection}


\subsubsection{Corrupted Abilities}

\begin{freeability}{Cone of Cold}
The mage makes a 8 vs. Fortitude attack against everything in a \areamed cone from it.
\hit Each target takes 2d10 cold damage.
\end{freeability}

\vspace{0.5em}
\begin{freeability}{Frost Bombs}
The mage creates three orbs of cold energy at locations of its choice within \rngmed range.
At the end of the next round's \glossterm{action phase},
the mage makes a 8 vs. Fortitude attack against everything in a \areasmall radius burst around each orb.
If a target is in the area of multiple orbs, it is only affected once.
\hit Each target takes 4d6 cold damage.
\end{freeability}

\vspace{0.5em}
\begin{freeability}{Telekinetic Crush}
The mage makes a 8 vs. Mental attack against one creature or object within \rngmed range of it.
\hit The target takes 4d8 bludgeoning damage.
\end{freeability}

