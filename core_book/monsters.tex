\chapter{Monsters}

Monsters are all of the various non-humanoid creatures that exist in the world of Rise.
Many of them are dangerous, and adventurers may need to fight them.
This chapter describes the rules for monsters, and the combat statistics for a variety of monsters.

\section{Monster Statistics}
    Like player characters, monsters have levels, attributes, and abilities.
    However, they do not have classes, legacy items, or many other elements of characters.
    This section defines how monsters function.

    \subsection{Level Scaling}
        Each monster has a level that indicates its approximate strength.
        Monsters have some specific level scaling rules that differ from the rules for player characters.
        In general, their automatic level scaling is designed to keep them on par with the power that player characters gain at higher levels, but without all of the complexity of high-level PCs.
        High level monsters should have slightly higher base values for their attributes, though this is not quantified, since monster attributes can be highly varied.
        They have the following specific changes.
        \begin{itemize}
            \item Monsters add only half their Constitution to their level for the purpose of determining their hit points and damage resistance
            \item Monsters have a \plus5 bonus to all defenses that increases to \plus6 at 3rd level and \plus7 at 15th level
            \item Monsters gain a \plus1d damage bonus with strikes at 4th level and every 3 levels thereafter
            \item Monsters gain a \plus1 bonus to power with all abilities that gradually increases at 4th level and every 3 levels thereafter, as described in \tref{Monster Advancement}
            \item High CR monsters automatically remove conditions at the end of each round, and become more resistant to conditions at 12th level and 21st level (see \pcref{Conditions on Monsters})
        \end{itemize}

        Monster level scaling is summarized in \tref{Monster Advancement}.
        The values in that table are accurate for CR 1 monsters, since they are individually the most similar to player characters and have the fewest multipliers and modifiers.
        The statistics for monsters with a different CR can be extrapolated from those base values.

        Monsters gain the same statistical benefits from their attributes that player characters do.
        The values listed in the Monster Advancement table do not include bonuses from attributes.
        When creating a new monster, make sure to decide its attributes appropriately, since they can have a large effect on the monster's overall power level and combat style.
        In particular, attributes make monster defenses more varied.

        The Monster Advancement table includes a Bite Damage column for convenient reference.
        The monster bonus damage with strikes applies to all natural weapons, not just strikes.
        However, bites are one of the most common monster natural weapons, and it's much more convenient to see a die value rather than a large +d value at high levels.
        The damage value for other natural weapons, such as claws, can be derived from the bite damage listed.

    \begin{dtable}
        \lcaption{Monster Advancement}
        \begin{dtabularx}{\columnwidth}{l l >{\lcol}X >{\lcol}X >{\lcol}X}
            \tb{Level}\fn{1} & \tb{Max Rank} & \tb{Accuracy} & \tb{Defenses} & \tb{Bite Damage} \tableheaderrule
            1st              & 1             & \plus0        & 5             & 1d8\plus2  \\
            2nd              & 1             & \plus1        & 6             & 1d8\plus2  \\
            3rd              & 1             & \plus1        & 7             & 1d8\plus2  \\
            4th              & 2             & \plus2        & 8             & 1d10\plus3 \\
            5th              & 2             & \plus2        & 8             & 1d10\plus3 \\
            6th              & 2             & \plus3        & 9             & 1d10\plus3 \\
            7th              & 3             & \plus3        & 9             & 2d6\plus4  \\
            8th              & 3             & \plus4        & 10            & 2d6\plus4  \\
            9th              & 3             & \plus4        & 10            & 2d6\plus4  \\
            10th             & 4             & \plus5        & 11            & 2d8\plus6  \\
            11th             & 4             & \plus5        & 11            & 2d8\plus6  \\
            12th             & 4             & \plus6        & 13            & 2d8\plus6  \\
            13th             & 5             & \plus6        & 13            & 2d10\plus8 \\
            14th             & 5             & \plus7        & 14            & 2d10\plus8 \\
            15th             & 5             & \plus7        & 14            & 2d10\plus8 \\
            16th             & 6             & \plus8        & 15            & 4d6\plus12 \\
            17th             & 6             & \plus8        & 15            & 4d6\plus12 \\
            18th             & 6             & \plus9        & 16            & 4d6\plus12 \\
            19th             & 7             & \plus9       & 16            & 4d8\plus16 \\
            20th             & 7             & \plus10       & 17            & 4d8\plus16 \\
            21st             & 7             & \plus10       & 18            & 4d8\plus16 \\
        \end{dtabularx}
        1. The statistics in this table are accurate for CR 1 monsters. \\
        2. Assuming the monster has no free hands (see \pcref{Monster Natural Weapon Bonuses}). \\
        % 3. For each level beyond 21, a monster increases its maximum hit points by 25 and its maximum damage resistance by 12. \\
    \end{dtable}

    \subsection{Challenge Rating}\label{Challenge Rating}
        Each monster has a challenge rating that indicates its approximate strength within its level, ranging from 1/2 to 6.
        A monster's challenge rating is a guideline to how many of that monster should typically be present in an encounter (see \pcref{Encounter Balancing}).
        This has several effects on the monster's statistics, as described in \trefnp{Challenge Rating Effects}.

        \begin{dtable*}
            \lcaption{Challenge Rating Effects}
            \begin{dtabularx}{\textwidth}{l l l l l X X l}
                \tb{CR} & \tb{HP} & \tb{DR} & \tb{Accuracy} & \tb{Defenses} & \tb{Power Scaling}\fn{1} & \tb{Max Attribute} & \tb{Max Rank}\fn{2} \tableheaderrule
                1/2     & \mult1  & \mult0  & \plus0        & \minus1       & \mult1/2                 & 3                  & \minus1 \\
                1       & \mult1  & \mult2  & \plus0        & \plus0        & \mult1                   & 4                  & \tdash  \\
                2       & \mult3  & \mult4  & \plus1        & \plus0        & \mult2                   & 5                  & \tdash  \\
                4       & \mult4  & \mult8  & \plus1        & \plus1        & \mult2                   & 6                  & \tdash  \\
                6       & \mult6  & \mult16 & \plus1        & \plus2        & \mult3                   & 7                  & \plus1  \\
            \end{dtabularx}
            1. This modifier applies to the special power bonus that monsters get based on their level, not to any other sources of power such as Strength or Willpower. \\
            2. See \pcref{Rank-Based Ability Access}. \\
        \end{dtable*}

        \subsubsection{Monster Vital Wounds}
            Monsters do not normally make vital rolls like player characters do.
            Unless otherwise specified on the monster's description, vital wounds have no negative effects on monsters.
            Instead, once a monster gains a vital wound, it is defeated.
            Defeated monsters may be unconscious, dead, or merely flee the fight at the discretion of the Game Master.

        \subsubsection{Conditions on Monsters}\label{Conditions on Monsters}
            Monsters cannot normally use the \textit{recover} ability.
            However, monsters with a high challenge rating can remove conditions automatically.
            At the end of each round, CR 4 monsters automatically remove one \glossterm{condition} that was not applied during that round.
            CR 6 monsters automatically remove two conditions instead of only one.
            Monsters without a sufficient understanding of the conditions affecting them generally choose randomly.

            \parhead{Condition Avoidance} At 12th level, high-CR monsters become more resistant to conditions.
            CR 4 and CR 6 monsters are immune to the first condition they gain between \glossterm{short rests}.
            They cannot choose to suffer the effects of a condition in order to retain this one-time immunity for a later, more dangerous condition.
            In addition, this does not provide any protection against detrimental \glossterm{brief} effects.
            At 21st level, CR 4 and CR 6 monsters also gain immunity to the second condition they would gain between short rests.

        \subsubsection{Monster Resources}
            Unless otherwise noted in their description, monsters have no resources, and their fatigue tolerance is treated as 0.
            They are unable to use abilities that would cause them to increase their fatigue level, such as the \textit{desperate exertion} ability.
            At the GM's discretion, individual monsters may technically be attuned to their abilities or be attuned to magic items they find, but this is not an explicit part of monster definitions.

        \subsubsection{Rank-Based Ability Access}\label{Rank-Based Ability Access}
            Although monsters do not gain archetypes, their special abilities should still be balanced appropriately for their maximum accessible rank.
            Like a player character, a monster's maximum accessible rank is based on its level, as shown in \tref{Monster Advancement}.
            CR 1/2 monsters take a \minus1 penalty to their maximum rank, and generally do not have particularly powerful or complex special abilities.
            CR 6 monsters gain a \plus1 bonus to their maximum rank, allowing them to use powerful abilities before player characters can.

            It's not always meaningful to give every monster multiple maneuvers, especially low CR monsters.
            However, maneuvers provide significant power at high levels, so monsters shouldn't just use basic strikes.
            All monsters have access to the \textit{monstrous strike} maneuver, which they use whenever they aren't using a specific maneuver from a defined combat style.
            This keeps their simple strikes on par with any special abilities they might have.
            It also means that high level monsters are appropriately weaker when they make special strikes from other sources, such as the \ability{charge} ability, just like high level player characters.
            \begin{freeability}{Monstrous Strike}
                The monster makes a strike.

                \rankline
                \rank{3} The monster gains a \plus1 accuracy bonus with the strike.
                \rank{5} The accuracy bonus increases to \plus2.
                \rank{7} The accuracy bonus increases to \plus3.
            \end{freeability}

    \subsection{Attributes}
        Each of a monster's base attributes can range from \minus9 to whatever the maximum base attribute is for a monster of its CR.
        A monster's total attributes scale with level in the same way as player character attributes.
        In general, a monster with higher base attributes will be stronger, but not all monsters need to start with the same base attribute total.

    \subsection{Monster Natural Weapon Bonuses}\label{Monster Natural Weapon Bonuses}
        Monsters do not use the \textit{offhand strike} ability in combat.
        Rolling offhand strikes for monsters in combat requires too much time and effort, and monsters aren't balanced around the ability to make offhand strikes.

        In addition, monsters that do not use hands for any purpose gain a \plus1d damage bonus with all strikes.
        The base statistics for natural weapons that do not require free hands, like bite and gore, are balanced for player usage.
        Being able to attack while both of your hands are occupied is powerful.
        However, monsters that do not take advantage of that opportunity need increased damage to keep pace with other attack options.

    \subsection{Monster Natural Armor}\label{Monster Natural Armor}
        Monsters use slightly different armor mechanics than player characters.
        They always add half their Dexterity to their Armor defense instead of their full Dexterity, as if they were wearing medium armor.
        In addition, they gain a bonus equal to half their Constitution to their Armor defense.
        This represents hardened skin, tough scales, or similar natural armor that monsters can develop.

        More generally, Armor defense is intended to be a reliable and relatively consistent defense.
        It is the most commonly attacked defense, and some character archetypes are only able to target Armor defense.
        As a result, it should rarely be a monster's highest or lowest defense, and this calculation makes such extremes less likely.

\section{Monster Combat Mechanics}

    \subsection{Monster Actions}\label{Monster Actions}
        All monsters are able to take free actions, move actions, and standard actions in the same way as player characters.
        CR 2 or higher monsters can take minor actions, though most monsters do not have any relevant minor actions to take.
        All CR 4 monsters can take an additional standard action each round, and CR 6 monsters can take two additional standard actions each round.
        However, they cannot use the same ability or weapon twice in the same round, unless they have two different versions of the same weapon (such as two claws).
        These special abilities are listed in their descriptions.

        In general, all monsters of CR 4 or 6 should be designed to attack multiple different PCs in every round.
        If the full damage output of a high-CR monster is brought to bear on a single PC, it will likely kill that PC quickly, taking a player out of the fight too early.
        Alternately, if that PC is able to avoid or resist the damage, it will trivialize the fight and remove the sense of danger.
        Instead, damage should be spread throughout the party to ensure that multiple people feel threatened and have to adjust their actions based on their personal danger tolerance.

        Although splitting attacks between multiple defenders is usually a suboptimal battle strategy, it can be narratively justified in a variety of ways.
        It may be physically impractical for a monster to use all of its natural weapons on the same creature.
        Monsters may want to inflict debuffs on multiple attackers to reduce the damage they take, which may require dealing enough damage to inflict HP-based debuffs.

    \subsection{Encounter Balancing}\label{Encounter Balancing}
        In general, a group of PCs of a given level will have an appropriate challenge from fighting monsters of the same level with a combined challenge rating equal to the number of PCs.
        Fighting monsters of a lower level, or monsters whose combined challenge rating is less than the number of PCs, will yield an easier encounter.
        Fighting monsters of a higher level, or fighting monsters whose combined challenge rating is greater than the number of PCs, will yield a more difficult encounter.

        It is generally not a good idea for PCs to fight monsters more than four levels higher or lower than their own level.
        They may find that their attacks always miss, or never miss, and other aspects of the encounter may be similarly imbalanced in ways that are not easily remedied by simply changing the number of enemies.

        However, intentionally using lower-level monsters with a high CR can change the tone of an encounter in ways that may be beneficial.
        A fight against four CR 1 monsters of the party's level has a different pace and tone than a fight against four CR 2 monsters that are two or three levels lower than the party, but both encounters can be similarly challenging.

        \subsubsection{Balancing By Player Capabilities}
            You can use your knowledge of the capabilities of your players to guess the difficulty of an encounter.
            If they have the perfect spell for a situation, they might be able to trivialize an encounter that looks numerically difficult.

            CR 1/2 monsters are generally vulnerable to large area attacks due to their numbers.
            Most CR 1/2 monsters have only melee attacks, so they are also weaker in tight corridors or against Barrier abilities, whch prevent them from making effective use of their numerical superiority.
            Since they have no damage resistance, they are exceptionally vulnerable to area attacks that care about damage resistance, such as the \textit{prismatic spray} spell.
            Debuffing attacks and high-damage single-target attacks are an inefficient way to deal with them.

            CR 1 monsters are the most well-rounded type of monster.
            Debuffs, area attacks, and single-target attacks are all reasonably effective without being devastating.
            They to be most vulnerable to attacks that target a small number of enemies without sacrificing too much damage, like Sweeping weapons.

            CR 2 monsters are most vulnerable to debilitating conditions.
            They are individually strong enough to be worth debuffing, and they lack the intrinsic condition resilience that higher CR monsters have.
            Area attacks are generally not worth the damage loss compared to single-target attacks, but Sweeping weapons are very useful if they group up.

            CR 4 and CR 6 monsters are most vulnerable to single-target attacks, especially single-target attacks that also inflict brief debuffs.
            Debuffs are powerful, but conditions are unlikely to last long.
            It is possible to apply more conditions to a CR 4 monster than it can remove with effort, but that strategy tends to be ineffective against CR 6 monsters.

    \subsection{Monster Knowledge}
        You can remember relevant knowledge about monsters by making Knowledge skill checks.
        Each monster has a set of associated information that you can learn with a knowledge check of the listed difficulty value.
        Most monsters have multiple tiers of information that you can learn depending on how good your Knowledge check is.
        All information recorded in these descriptions is accurate, though lower check results may not provide full context for a monster's behavior or nature.

\section{Monster Descriptions}


\section{Aberrations}
\begin{monsection}{Aboleth}{12}[4]
\vspace{-1em}\spelltwocol{}{Huge aberration}\vspace{-1em}
\begin{spellcontent}
\begin{spelltargetinginfo}
\spelltwocol{
\textbf{HP} 384;
\textbf{Bloodied} 192;
}
{\textbf{AP} 5/5}

\pari \textbf{Armor} 21;
\textbf{Fort} 22;
\textbf{Ref} 15;
\textbf{Ment} 30
\pari \textbf{Strike} Tentacle \plus17 (4d10)


\pari \textbf{Actions} One in action phase, one in delayed action phase
\pari \textbf{Behavior} Attack highest threat
\end{spelltargetinginfo}
\end{spellcontent}

\begin{monsterfooter}
\pari \textbf{Awareness} \plus6
\pari \textbf{Speed} 50 ft. swim ft.;
\textbf{Space} 15 ft.;
\textbf{Reach} 15 ft.
\pari \textbf{Attributes}:
Str 17,
Dex 0,
Con 14,
Int 15,
Per 14,
Wil 16
\end{monsterfooter}
\end{monsection}


\subsubsection{Aboleth Abilities}

\begin{freeability}{Mind Crush}[\glossterm{Compulsion}]
The aboleth makes a \plus17 vs. Mental attack against a creature in \rnglong range.
\hit The target takes 10d10 psionic damage and is \glossterm{stunned} as a \glossterm{condition}.
\crit The aboleth can spend an \glossterm{action point} to \glossterm{attune} to this ability.
If it does, the target is \glossterm{dominated} by the aboleth as long as the ability lasts.
Otherwise, the target takes double the damage of a non-critical hit.
\end{freeability}

\vspace{0.5em}
\begin{freeability}{Psionic Blast}[\glossterm{Compulsion}]
The aboleth makes a \plus17 vs. Mental attack against enemies in a Large cone.
\hit Each target takes 8d10 psionic damage and is \glossterm{stunned} as a \glossterm{condition}.
\end{freeability}

\parhead{Rituals}
The aboleth can learn and perform arcane rituals of up to 6th level.

\section{Animates}
\begin{monsection}{Elemental}[Air]{10}[1]
\vspace{-1em}\spelltwocol{}{Large animate}\vspace{-1em}
\begin{spellcontent}
\begin{spelltargetinginfo}
\spelltwocol{
\textbf{HP} 70;
\textbf{Bloodied} 35;
}
{\textbf{AP} 2/0}

\pari \textbf{Armor} 19;
\textbf{Fort} 19;
\textbf{Ref} 23;
\textbf{Ment} 12
\pari \textbf{Strike} Slam \plus12 (4d6)



\pari \textbf{Behavior} Attack highest threat
\end{spelltargetinginfo}
\end{spellcontent}

\begin{monsterfooter}
\pari \textbf{Awareness} \plus6
\pari \textbf{Speed} 40 ft.;
\textbf{Space} 10 ft.;
\textbf{Reach} 10 ft.
\pari \textbf{Attributes}:
Str 13,
Dex 15,
Con 11,
Int 0,
Per 12,
Wil 0
\end{monsterfooter}
\end{monsection}

\begin{monsection}{Animus}[Ram]{6}[4]
\vspace{-1em}\spelltwocol{}{Huge animate}\vspace{-1em}
\begin{spellcontent}
\begin{spelltargetinginfo}
\spelltwocol{
\textbf{HP} 216;
\textbf{Bloodied} 108;
}
{\textbf{AP} 4/0}

\pari \textbf{Armor} 16;
\textbf{Fort} 22;
\textbf{Ref} 12;
\textbf{Ment} 11
\pari \textbf{Strike} Slam \plus10 (4d6) or hoof \plus10 (2d10)


\pari \textbf{Actions} One in action phase, one in delayed action phase
\pari \textbf{Behavior} Attack highest threat
\end{spelltargetinginfo}
\end{spellcontent}

\begin{monsterfooter}
\pari \textbf{Awareness} \plus6
\pari \textbf{Speed} 50 ft.;
\textbf{Space} 15 ft.;
\textbf{Reach} 15 ft.
\pari \textbf{Attributes}:
Str 13,
Dex 7,
Con 9,
Int 0,
Per 7,
Wil 0
\end{monsterfooter}
\end{monsection}


\subsubsection{Animus Abilities}

\begin{freeability}{Forceful Smash}
The ram makes a slam strike.
In addition to the strike's normal effects, compare the attack result against the target's Fortitude defense.
\hit The target moves up to 10 feet in a direction of the ram's choice, as the \textit{shove} ability (see \pcref{Shove}).
The ram does not have to move with the target to push it back.
\end{freeability}

\begin{monsection}{Earth Elemental}[Elder]{12}[3]
\vspace{-1em}\spelltwocol{}{Huge animate}\vspace{-1em}
\begin{spellcontent}
\begin{spelltargetinginfo}
\spelltwocol{
\textbf{HP} 360;
\textbf{Bloodied} 180;
}
{\textbf{AP} 4/3}

\pari \textbf{Armor} 20;
\textbf{Fort} 29;
\textbf{Ref} 14;
\textbf{Ment} 21
\pari \textbf{Strike} Slam \plus16 (5d10)


\pari \textbf{Actions} One in action phase, one in delayed action phase
\pari \textbf{Behavior} Attack highest threat
\end{spelltargetinginfo}
\end{spellcontent}

\begin{monsterfooter}
\pari \textbf{Awareness} \plus6
\pari \textbf{Speed} 50 ft.;
\textbf{Space} 15 ft.;
\textbf{Reach} 15 ft.
\pari \textbf{Attributes}:
Str 19,
Dex 0,
Con 16,
Int 7,
Per 14,
Wil 14
\end{monsterfooter}
\end{monsection}

\section{Animals}
\begin{monsection}{Bear}[Black]{2}[2]
\vspace{-1em}\spelltwocol{}{Medium animal}\vspace{-1em}
\begin{spellcontent}
\begin{spelltargetinginfo}
\spelltwocol{
\textbf{HP} 36;
\textbf{Bloodied} 18;
}
{\textbf{AP} 2/0}

\pari \textbf{Armor} 7;
\textbf{Fort} 16;
\textbf{Ref} 8;
\textbf{Ment} 5
\pari \textbf{Strike} Bite \plus3 (1d10) or claw \plus3 (1d8)
\pari \textbf{Immune} staggered


\pari \textbf{Behavior} Attack highest threat
\end{spelltargetinginfo}
\end{spellcontent}

\begin{monsterfooter}
\pari \textbf{Awareness} \plus6
\pari \textbf{Speed} 30 ft.;
\textbf{Space} 5 ft.;
\textbf{Reach} 5 ft.
\pari \textbf{Attributes}:
Str 5,
Dex 2,
Con 5,
Int \minus7,
Per 2,
Wil 0
\end{monsterfooter}
\end{monsection}


\subsubsection{Bear Abilities}

\begin{freeability}{Rend}
The bear makes a claw strike against two targets within reach.
\end{freeability}

\begin{monsection}{Bear}[Brown]{4}[2]
\vspace{-1em}\spelltwocol{}{Large animal}\vspace{-1em}
\begin{spellcontent}
\begin{spelltargetinginfo}
\spelltwocol{
\textbf{HP} 72;
\textbf{Bloodied} 36;
}
{\textbf{AP} 2/0}

\pari \textbf{Armor} 9;
\textbf{Fort} 18;
\textbf{Ref} 8;
\textbf{Ment} 7
\pari \textbf{Strike} Bite \plus5 (2d8) or claw \plus5 (2d6)
\pari \textbf{Immune} staggered


\pari \textbf{Behavior} Attack highest threat
\end{spelltargetinginfo}
\end{spellcontent}

\begin{monsterfooter}
\pari \textbf{Awareness} \plus6
\pari \textbf{Speed} 40 ft.;
\textbf{Space} 10 ft.;
\textbf{Reach} 10 ft.
\pari \textbf{Attributes}:
Str 9,
Dex 3,
Con 7,
Int \minus7,
Per 3,
Wil 0
\end{monsterfooter}
\end{monsection}


\subsubsection{Bear Abilities}

\begin{freeability}{Rend}
The bear makes a claw strike against two targets within reach.
\end{freeability}

\begin{monsection}{Beetle}[Dire]{7}[2]
\vspace{-1em}\spelltwocol{}{Large animal}\vspace{-1em}
\begin{spellcontent}
\begin{spelltargetinginfo}
\spelltwocol{
\textbf{HP} 126;
\textbf{Bloodied} 63;
}
{\textbf{AP} 3/0}

\pari \textbf{Armor} 14;
\textbf{Fort} 21;
\textbf{Ref} 10;
\textbf{Ment} 10
\pari \textbf{Strike} Bite \plus9 (4d6)



\pari \textbf{Behavior} Attack highest threat
\end{spelltargetinginfo}
\end{spellcontent}

\begin{monsterfooter}
\pari \textbf{Awareness} \plus6
\pari \textbf{Speed} 40 ft.;
\textbf{Space} 10 ft.;
\textbf{Reach} 10 ft.
\pari \textbf{Attributes}:
Str 12,
Dex 0,
Con 10,
Int \minus9,
Per 8,
Wil 0
\end{monsterfooter}
\end{monsection}

\begin{monsection}{Centipede}[Huge]{8}[3]
\vspace{-1em}\spelltwocol{}{Huge animal}\vspace{-1em}
\begin{spellcontent}
\begin{spelltargetinginfo}
\spelltwocol{
\textbf{HP} 240;
\textbf{Bloodied} 120;
}
{\textbf{AP} 4/0}

\pari \textbf{Armor} 16;
\textbf{Fort} 25;
\textbf{Ref} 11;
\textbf{Ment} 12
\pari \textbf{Strike} Bite \plus11 (4d8)


\pari \textbf{Actions} One in action phase, one in delayed action phase
\pari \textbf{Behavior} Attack highest threat
\end{spelltargetinginfo}
\end{spellcontent}

\begin{monsterfooter}
\pari \textbf{Awareness} \plus6
\pari \textbf{Speed} 50 ft.;
\textbf{Space} 15 ft.;
\textbf{Reach} 15 ft.
\pari \textbf{Attributes}:
Str 15,
Dex 5,
Con 12,
Int \minus9,
Per 9,
Wil 0
\end{monsterfooter}
\end{monsection}

\begin{monsection}{Dire Wolf}{5}[2]
\vspace{-1em}\spelltwocol{}{Large animal}\vspace{-1em}
\begin{spellcontent}
\begin{spelltargetinginfo}
\spelltwocol{
\textbf{HP} 70;
\textbf{Bloodied} 35;
}
{\textbf{AP} 2/0}

\pari \textbf{Armor} 12;
\textbf{Fort} 15;
\textbf{Ref} 13;
\textbf{Ment} 8
\pari \textbf{Strike} Bite \plus8 (2d8)



\pari \textbf{Behavior} Attack highest threat
\end{spelltargetinginfo}
\end{spellcontent}

\begin{monsterfooter}
\pari \textbf{Awareness} \plus6
\pari \textbf{Speed} 40 ft.;
\textbf{Space} 10 ft.;
\textbf{Reach} 10 ft.
\pari \textbf{Attributes}:
Str 9,
Dex 7,
Con 6,
Int \minus6,
Per 7,
Wil 0
\end{monsterfooter}
\end{monsection}


\subsubsection{Dire Wolf Abilities}

\begin{freeability}{Pounce}
The dire wolf moves up to its speed in a single straight line and makes a bite \glossterm{strike} from its new location.
\end{freeability}

\begin{monsection}{Eel}{6}[2]
\vspace{-1em}\spelltwocol{}{Large animal}\vspace{-1em}
\begin{spellcontent}
\begin{spelltargetinginfo}
\spelltwocol{
\textbf{HP} 84;
\textbf{Bloodied} 42;
}
{\textbf{AP} 2/0}

\pari \textbf{Armor} 15;
\textbf{Fort} 16;
\textbf{Ref} 14;
\textbf{Ment} 9
\pari \textbf{Strike} Bite \plus9 (2d8)



\pari \textbf{Behavior} Attack highest threat
\end{spelltargetinginfo}
\end{spellcontent}

\begin{monsterfooter}
\pari \textbf{Awareness} \plus6
\pari \textbf{Speed} 40 ft.;
\textbf{Space} 10 ft.;
\textbf{Reach} 10 ft.
\pari \textbf{Attributes}:
Str 9,
Dex 8,
Con 7,
Int \minus8,
Per 8,
Wil 0
\end{monsterfooter}
\end{monsection}

\begin{monsection}{Ferret}{1}[1]
\vspace{-1em}\spelltwocol{}{Tiny animal}\vspace{-1em}
\begin{spellcontent}
\begin{spelltargetinginfo}
\spelltwocol{
\textbf{HP} 1;
\textbf{Bloodied} 0;
}
{\textbf{AP} 1/\minus2}

\pari \textbf{Armor} 6;
\textbf{Fort} 3;
\textbf{Ref} 16;
\textbf{Ment} 1
\pari \textbf{Strike} Bite \plus1 (1d6)



\pari \textbf{Behavior} Attack highest threat
\end{spelltargetinginfo}
\end{spellcontent}

\begin{monsterfooter}
\pari \textbf{Awareness} \plus6
\pari \textbf{Speed} 20 ft.;
\textbf{Space} 2.5 ft.;
\textbf{Reach} 2.5 ft.
\pari \textbf{Attributes}:
Str \minus10,
Dex 4,
Con \minus4,
Int \minus7,
Per 1,
Wil \minus2
\end{monsterfooter}
\end{monsection}

\begin{monsection}{Ichor Bear}[Black]{2}[2]
\vspace{-1em}\spelltwocol{}{Medium animal}\vspace{-1em}
\begin{spellcontent}
\begin{spelltargetinginfo}
\spelltwocol{
\textbf{HP} 36;
\textbf{Bloodied} 18;
}
{\textbf{AP} 2/0}

\pari \textbf{Armor} 7;
\textbf{Fort} 16;
\textbf{Ref} 8;
\textbf{Ment} 9
\pari \textbf{Strike} Bite \plus3 (1d10) or claw \plus3 (1d8)
\pari \textbf{Immune} staggered


\pari \textbf{Behavior} Attack highest threat
\end{spelltargetinginfo}
\end{spellcontent}

\begin{monsterfooter}
\pari \textbf{Awareness} \plus6
\pari \textbf{Speed} 30 ft.;
\textbf{Space} 5 ft.;
\textbf{Reach} 5 ft.
\pari \textbf{Attributes}:
Str 5,
Dex 2,
Con 5,
Int \minus7,
Per 2,
Wil 0
\end{monsterfooter}
\end{monsection}


\subsubsection{Ichor Bear Abilities}

\begin{freeability}{Rend}
The bear makes a claw strike against two targets within reach.
\end{freeability}

\parhead{Ichor Healing}
The ichor bear heals 2 hit points at the end of each round.

\begin{monsection}{Pony}{2}[1]
\vspace{-1em}\spelltwocol{}{Medium animal}\vspace{-1em}
\begin{spellcontent}
\begin{spelltargetinginfo}
\spelltwocol{
\textbf{HP} 16;
\textbf{Bloodied} 8;
}
{\textbf{AP} 1/0}

\pari \textbf{Armor} 6;
\textbf{Fort} 13;
\textbf{Ref} 7;
\textbf{Ment} 4
\pari \textbf{Strike} Bite \plus2 (1d8)



\pari \textbf{Behavior} Attack highest threat
\end{spelltargetinginfo}
\end{spellcontent}

\begin{monsterfooter}
\pari \textbf{Awareness} \plus6
\pari \textbf{Speed} 30 ft.;
\textbf{Space} 5 ft.;
\textbf{Reach} 5 ft.
\pari \textbf{Attributes}:
Str 2,
Dex 2,
Con 4,
Int \minus7,
Per 2,
Wil 0
\end{monsterfooter}
\end{monsection}

\begin{monsection}{Raven}{1}[1]
\vspace{-1em}\spelltwocol{}{Tiny animal}\vspace{-1em}
\begin{spellcontent}
\begin{spelltargetinginfo}
\spelltwocol{
\textbf{HP} 1;
\textbf{Bloodied} 0;
}
{\textbf{AP} 1/0}

\pari \textbf{Armor} 5;
\textbf{Fort} 3;
\textbf{Ref} 14;
\textbf{Ment} 3
\pari \textbf{Strike} Talon \plus3 (1d4)



\pari \textbf{Behavior} Attack highest threat
\end{spelltargetinginfo}
\end{spellcontent}

\begin{monsterfooter}
\pari \textbf{Awareness} \plus6
\pari \textbf{Speed} 20 ft.;
\textbf{Space} 2.5 ft.;
\textbf{Reach} 2.5 ft.
\pari \textbf{Attributes}:
Str \minus13,
Dex 3,
Con \minus4,
Int \minus6,
Per 2,
Wil 0
\end{monsterfooter}
\end{monsection}

\begin{monsection}{Roc}{9}[4]
\vspace{-1em}\spelltwocol{}{Gargantuan animal}\vspace{-1em}
\begin{spellcontent}
\begin{spelltargetinginfo}
\spelltwocol{
\textbf{HP} 252;
\textbf{Bloodied} 126;
}
{\textbf{AP} 5/0}

\pari \textbf{Armor} 19;
\textbf{Fort} 21;
\textbf{Ref} 13;
\textbf{Ment} 14
\pari \textbf{Strike} Talon \plus14 (5d10)


\pari \textbf{Actions} One in action phase, one in delayed action phase
\pari \textbf{Behavior} Attack highest threat
\end{spelltargetinginfo}
\end{spellcontent}

\begin{monsterfooter}
\pari \textbf{Awareness} \plus6
\pari \textbf{Speed} 80 ft. fly ft.;
\textbf{Space} 20 ft.;
\textbf{Reach} 20 ft.
\pari \textbf{Attributes}:
Str 20,
Dex 10,
Con 10,
Int \minus7,
Per 11,
Wil 0
\end{monsterfooter}
\end{monsection}


\subsubsection{Roc Abilities}

\begin{freeability}{Flyby Attack}
The roc flies up to its flying movement speed.
It can make a talon strike or use the \textit{grapple} ability at any point during this movement.
\end{freeability}

\begin{monsection}{Spider}[Colossal]{12}[3]
\vspace{-1em}\spelltwocol{}{Colossal animal}\vspace{-1em}
\begin{spellcontent}
\begin{spelltargetinginfo}
\spelltwocol{
\textbf{HP} 216;
\textbf{Bloodied} 108;
}
{\textbf{AP} 4/0}

\pari \textbf{Armor} 23;
\textbf{Fort} 21;
\textbf{Ref} 17;
\textbf{Ment} 16
\pari \textbf{Strike} Bite \plus16 (7d10)


\pari \textbf{Actions} One in action phase, one in delayed action phase
\pari \textbf{Behavior} Attack highest threat
\end{spelltargetinginfo}
\end{spellcontent}

\begin{monsterfooter}
\pari \textbf{Awareness} \plus6
\pari \textbf{Speed} 70 ft.;
\textbf{Space} 30 ft.;
\textbf{Reach} 30 ft.
\pari \textbf{Attributes}:
Str 23,
Dex 15,
Con 7,
Int \minus9,
Per 14,
Wil 0
\end{monsterfooter}
\end{monsection}


\subsubsection{Spider Abilities}

\begin{freeability}{Web Spit}
The spider makes a \plus16 vs. Reflex attack against one creature within \rnglong range.
\hit The target is \glossterm{immobilized} as a \glossterm{condition}.
\end{freeability}

\begin{monsection}{Spider}[Gargantuan]{9}[3]
\vspace{-1em}\spelltwocol{}{Gargantuan animal}\vspace{-1em}
\begin{spellcontent}
\begin{spelltargetinginfo}
\spelltwocol{
\textbf{HP} 162;
\textbf{Bloodied} 81;
}
{\textbf{AP} 4/0}

\pari \textbf{Armor} 20;
\textbf{Fort} 18;
\textbf{Ref} 16;
\textbf{Ment} 13
\pari \textbf{Strike} Bite \plus13 (4d10)


\pari \textbf{Actions} One in action phase, one in delayed action phase
\pari \textbf{Behavior} Attack highest threat
\end{spelltargetinginfo}
\end{spellcontent}

\begin{monsterfooter}
\pari \textbf{Awareness} \plus6
\pari \textbf{Speed} 60 ft.;
\textbf{Space} 20 ft.;
\textbf{Reach} 20 ft.
\pari \textbf{Attributes}:
Str 17,
Dex 12,
Con 5,
Int \minus9,
Per 11,
Wil 0
\end{monsterfooter}
\end{monsection}


\subsubsection{Spider Abilities}

\begin{freeability}{Web Spit}
The spider makes a \plus13 vs. Reflex attack against one creature within \rnglong range.
\hit The target is \glossterm{immobilized} as a \glossterm{condition}.
\end{freeability}

\begin{monsection}{Wasp}[Giant]{6}[1]
\vspace{-1em}\spelltwocol{}{Large animal}\vspace{-1em}
\begin{spellcontent}
\begin{spelltargetinginfo}
\spelltwocol{
\textbf{HP} 30;
\textbf{Bloodied} 15;
}
{\textbf{AP} 1/\minus1}

\pari \textbf{Armor} 14;
\textbf{Fort} 12;
\textbf{Ref} 13;
\textbf{Ment} 7
\pari \textbf{Strike} Bite \plus7 (2d6)



\pari \textbf{Behavior} Attack highest threat
\end{spelltargetinginfo}
\end{spellcontent}

\begin{monsterfooter}
\pari \textbf{Awareness} \plus6
\pari \textbf{Speed} 50 ft. fly (good) ft.;
\textbf{Space} 10 ft.;
\textbf{Reach} 10 ft.
\pari \textbf{Attributes}:
Str 6,
Dex 8,
Con 0,
Int \minus8,
Per 7,
Wil \minus1
\end{monsterfooter}
\end{monsection}

\begin{monsection}{Wolf}{1}[1]
\vspace{-1em}\spelltwocol{}{Medium animal}\vspace{-1em}
\begin{spellcontent}
\begin{spelltargetinginfo}
\spelltwocol{
\textbf{HP} 6;
\textbf{Bloodied} 3;
}
{\textbf{AP} 1/0}

\pari \textbf{Armor} 6;
\textbf{Fort} 8;
\textbf{Ref} 10;
\textbf{Ment} 3
\pari \textbf{Strike} Bite \plus2 (1d6)



\pari \textbf{Behavior} Attack highest threat
\end{spelltargetinginfo}
\end{spellcontent}

\begin{monsterfooter}
\pari \textbf{Awareness} \plus6
\pari \textbf{Speed} 30 ft.;
\textbf{Space} 5 ft.;
\textbf{Reach} 5 ft.
\pari \textbf{Attributes}:
Str 1,
Dex 3,
Con 1,
Int \minus6,
Per 2,
Wil 0
\end{monsterfooter}
\end{monsection}

\section{Humanoids}
\begin{monsection}{Cultist}{1}[1]
\vspace{-1em}\spelltwocol{}{Medium humanoid}\vspace{-1em}
\begin{spellcontent}
\begin{spelltargetinginfo}
\spelltwocol{
\textbf{HP} 5;
\textbf{Bloodied} 2;
}
{\textbf{AP} 1/3}

\pari \textbf{Armor} 4;
\textbf{Fort} 5;
\textbf{Ref} 5;
\textbf{Ment} 10
\pari \textbf{Strike} Club \plus1 (1d4)



\pari \textbf{Behavior} Attack highest threat
\end{spelltargetinginfo}
\end{spellcontent}

\begin{monsterfooter}
\pari \textbf{Awareness} \plus6
\pari \textbf{Speed} 30 ft.;
\textbf{Space} 5 ft.;
\textbf{Reach} 5 ft.
\pari \textbf{Attributes}:
Str 0,
Dex 0,
Con 0,
Int \minus1,
Per 1,
Wil 3
\end{monsterfooter}
\end{monsection}


\subsubsection{Cultist Abilities}

\begin{freeability}{Hex}
The cultist makes a \plus1 vs. Fortitude attack against one creature in Medium range.
\hit The target takes 1d8 life damage and is \glossterm{sickened} as a \glossterm{condition}.",
\end{freeability}

\begin{monsection}{Goblin Shouter}{2}[2]
\vspace{-1em}\spelltwocol{}{Small humanoid}\vspace{-1em}
\begin{spellcontent}
\begin{spelltargetinginfo}
\spelltwocol{
\textbf{HP} 24;
\textbf{Bloodied} 12;
}
{\textbf{AP} 2/3}

\pari \textbf{Armor} 6;
\textbf{Fort} 8;
\textbf{Ref} 12;
\textbf{Ment} 12
\pari \textbf{Strike} Club \plus4 (1d6) or sling \plus4 (1d6)



\pari \textbf{Behavior} Attack lowest threat
\end{spelltargetinginfo}
\end{spellcontent}

\begin{monsterfooter}
\pari \textbf{Awareness} \plus6
\pari \textbf{Speed} 25 ft.;
\textbf{Space} 5 ft.;
\textbf{Reach} 5 ft.
\pari \textbf{Attributes}:
Str 0,
Dex 3,
Con 2,
Int \minus2,
Per 3,
Wil 4
\end{monsterfooter}
\end{monsection}


\subsubsection{Goblin Shouter Abilities}

\begin{freeability}{Shout of Stabbing}[\glossterm{Sustain} (standard)]
The shouter's \glossterm{allies} that can hear it gain a \plus2 bonus to \glossterm{power} with \glossterm{strikes}.
\end{freeability}

\begin{monsection}{Goblin Stabber}{1}[1]
\vspace{-1em}\spelltwocol{}{Small humanoid}\vspace{-1em}
\begin{spellcontent}
\begin{spelltargetinginfo}
\spelltwocol{
\textbf{HP} 4;
\textbf{Bloodied} 2;
}
{\textbf{AP} 1/0}

\pari \textbf{Armor} 5;
\textbf{Fort} 4;
\textbf{Ref} 12;
\textbf{Ment} 5
\pari \textbf{Strike} Shortsword \plus3 (1d4) or sling \plus2 (1d4)



\pari \textbf{Behavior} Attack lowest threat
\end{spelltargetinginfo}
\end{spellcontent}

\begin{monsterfooter}
\pari \textbf{Awareness} \plus6
\pari \textbf{Speed} 25 ft.;
\textbf{Space} 5 ft.;
\textbf{Reach} 5 ft.
\pari \textbf{Attributes}:
Str \minus2,
Dex 3,
Con \minus1,
Int \minus2,
Per 2,
Wil 0
\end{monsterfooter}
\end{monsection}


\subsubsection{Goblin Stabber Abilities}

\begin{freeability}{Sneeky Stab}
The stabber makes a shortsword strike.
If the target is defenseless, overwhelmed, or unaware, the damage becomes 1d8.
\end{freeability}

\begin{monsection}{Orc Chieftain}{5}[3]
\vspace{-1em}\spelltwocol{}{Medium humanoid}\vspace{-1em}
\begin{spellcontent}
\begin{spelltargetinginfo}
\spelltwocol{
\textbf{HP} 120;
\textbf{Bloodied} 60;
}
{\textbf{AP} 3/3}

\pari \textbf{Armor} 11;
\textbf{Fort} 16;
\textbf{Ref} 14;
\textbf{Ment} 16
\pari \textbf{Strike} Greataxe \plus8 (4d6)


\pari \textbf{Actions} One in action phase, one in delayed action phase
\pari \textbf{Behavior} Attack highest threat
\end{spelltargetinginfo}
\end{spellcontent}

\begin{monsterfooter}
\pari \textbf{Awareness} \plus6
\pari \textbf{Speed} 30 ft.;
\textbf{Space} 5 ft.;
\textbf{Reach} 5 ft.
\pari \textbf{Attributes}:
Str 10,
Dex 6,
Con 7,
Int 0,
Per 6,
Wil 7
\end{monsterfooter}
\end{monsection}


\subsubsection{Orc Chieftain Abilities}

\begin{freeability}{Hit Everyone Else}[\glossterm{Sustain} (standard)]
The chieftain's \glossterm{allies} that can hear it gain a \plus2 bonus to \glossterm{accuracy} with \glossterm{strikes}.
\end{freeability}

\vspace{0.5em}
\begin{freeability}{Hit Hardest}
The chieftain makes a greataxe strike.
The strike deals 4d10 damage.
\end{freeability}

\vspace{0.5em}
\begin{freeability}{Hit Fast}
The chieftain makes a greataxe strike.
Its accuracy is increased to 10.
\end{freeability}

\begin{monsection}{Orc Grunt}{2}[1]
\vspace{-1em}\spelltwocol{}{Medium humanoid}\vspace{-1em}
\begin{spellcontent}
\begin{spelltargetinginfo}
\spelltwocol{
\textbf{HP} 12;
\textbf{Bloodied} 6;
}
{\textbf{AP} 1/0}

\pari \textbf{Armor} 5;
\textbf{Fort} 7;
\textbf{Ref} 6;
\textbf{Ment} 6
\pari \textbf{Strike} Greataxe \plus2 (2d8)



\pari \textbf{Behavior} Attack highest threat
\end{spelltargetinginfo}
\end{spellcontent}

\begin{monsterfooter}
\pari \textbf{Awareness} \plus6
\pari \textbf{Speed} 30 ft.;
\textbf{Space} 5 ft.;
\textbf{Reach} 5 ft.
\pari \textbf{Attributes}:
Str 6,
Dex 0,
Con 2,
Int \minus1,
Per 0,
Wil 0
\end{monsterfooter}
\end{monsection}


\subsubsection{Orc Grunt Abilities}

\begin{freeability}{Hit Harder}
The grunt makes a greataxe strike.
Its accuracy is reduced to 0, but the strike deals 4d6 damage.
\end{freeability}

\begin{monsection}{Orc Loudmouth}{3}[2]
\vspace{-1em}\spelltwocol{}{Medium humanoid}\vspace{-1em}
\begin{spellcontent}
\begin{spelltargetinginfo}
\spelltwocol{
\textbf{HP} 36;
\textbf{Bloodied} 18;
}
{\textbf{AP} 2/2}

\pari \textbf{Armor} 8;
\textbf{Fort} 9;
\textbf{Ref} 11;
\textbf{Ment} 11
\pari \textbf{Strike} Greataxe \plus5 (2d8)



\pari \textbf{Behavior} Attack highest threat
\end{spelltargetinginfo}
\end{spellcontent}

\begin{monsterfooter}
\pari \textbf{Awareness} \plus6
\pari \textbf{Speed} 30 ft.;
\textbf{Space} 5 ft.;
\textbf{Reach} 5 ft.
\pari \textbf{Attributes}:
Str 6,
Dex 4,
Con 2,
Int \minus1,
Per 4,
Wil 4
\end{monsterfooter}
\end{monsection}


\subsubsection{Orc Loudmouth Abilities}

\begin{freeability}{Hit Harder}
The loudmouth makes a greataxe strike.
Its accuracy is reduced to 3, but the strike deals 4d6 damage.
\end{freeability}

\vspace{0.5em}
\begin{freeability}{Hit That One Over There}[\glossterm{Sustain} (standard)]
The loudmouth chooses an enemy within Long range.
The loudmouth's \glossterm{allies} that can hear it gain a \plus2 bonus to \glossterm{accuracy} with \glossterm{strikes} against that target.
\end{freeability}

\begin{monsection}{Orc Shaman}{3}[2]
\vspace{-1em}\spelltwocol{}{Medium humanoid}\vspace{-1em}
\begin{spellcontent}
\begin{spelltargetinginfo}
\spelltwocol{
\textbf{HP} 42;
\textbf{Bloodied} 21;
}
{\textbf{AP} 2/3}

\pari \textbf{Armor} 8;
\textbf{Fort} 11;
\textbf{Ref} 11;
\textbf{Ment} 13
\pari \textbf{Strike} Greatstaff \plus4 (2d6)



\pari \textbf{Behavior} Attack highest threat
\end{spelltargetinginfo}
\end{spellcontent}

\begin{monsterfooter}
\pari \textbf{Awareness} \plus6
\pari \textbf{Speed} 30 ft.;
\textbf{Space} 5 ft.;
\textbf{Reach} 5 ft.
\pari \textbf{Attributes}:
Str 6,
Dex 4,
Con 4,
Int \minus1,
Per 0,
Wil 5
\end{monsterfooter}
\end{monsection}


\subsubsection{Orc Shaman Abilities}

\begin{freeability}{Hit Worse}
The shaman makes a \plus4 vs. Mental attack against one creature in Close range.
\hit The target takes a \minus3 penalty to accuracy with strikes as a \glossterm{condition}.
\crit As above, except that the penalty is increased to -6.
\end{freeability}

\vspace{0.5em}
\begin{freeability}{Hurt Less}
One ally in Close range heals 4d6 hit points.
\end{freeability}

\begin{monsection}{Orc Savage}{4}[1]
\vspace{-1em}\spelltwocol{}{Medium humanoid}\vspace{-1em}
\begin{spellcontent}
\begin{spelltargetinginfo}
\spelltwocol{
\textbf{HP} 24;
\textbf{Bloodied} 12;
}
{\textbf{AP} 1/0}

\pari \textbf{Armor} 8;
\textbf{Fort} 9;
\textbf{Ref} 11;
\textbf{Ment} 8
\pari \textbf{Strike} Greataxe \plus4 (2d10)



\pari \textbf{Behavior} Attack highest threat
\end{spelltargetinginfo}
\end{spellcontent}

\begin{monsterfooter}
\pari \textbf{Awareness} \plus6
\pari \textbf{Speed} 30 ft.;
\textbf{Space} 5 ft.;
\textbf{Reach} 5 ft.
\pari \textbf{Attributes}:
Str 9,
Dex 5,
Con 3,
Int \minus1,
Per 0,
Wil 0
\end{monsterfooter}
\end{monsection}


\subsubsection{Orc Savage Abilities}

\begin{freeability}{Hit Fast}
The savage makes a greataxe strike.
Its accuracy is 6.
\end{freeability}

\section{Magical Beasts}
\begin{monsection}{Dragon}[Large Red]{6}[4]
\vspace{-1em}\spelltwocol{}{Large magical beast}\vspace{-1em}
\begin{spellcontent}
\begin{spelltargetinginfo}
\spelltwocol{
\textbf{HP} 192;
\textbf{Bloodied} 96;
}
{\textbf{AP} 4/3}

\pari \textbf{Armor} 16;
\textbf{Fort} 18;
\textbf{Ref} 11;
\textbf{Ment} 18
\pari \textbf{Strike} Bite \plus11 (2d8)


\pari \textbf{Actions} One in action phase, one in delayed action phase
\pari \textbf{Behavior} Attack highest threat
\end{spelltargetinginfo}
\end{spellcontent}

\begin{monsterfooter}
\pari \textbf{Awareness} \plus6
\pari \textbf{Speed} 40 ft.;
\textbf{Space} 10 ft.;
\textbf{Reach} 10 ft.
\pari \textbf{Attributes}:
Str 9,
Dex 0,
Con 8,
Int 8,
Per 8,
Wil 8
\end{monsterfooter}
\end{monsection}

\begin{monsection}{Wyvern}{5}[3]
\vspace{-1em}\spelltwocol{}{Large magical beast}\vspace{-1em}
\begin{spellcontent}
\begin{spelltargetinginfo}
\spelltwocol{
\textbf{HP} 120;
\textbf{Bloodied} 60;
}
{\textbf{AP} 3/0}

\pari \textbf{Armor} 13;
\textbf{Fort} 16;
\textbf{Ref} 10;
\textbf{Ment} 11
\pari \textbf{Strike} Sting \plus8 (4d6) or bite \plus8 (2d10)


\pari \textbf{Actions} One in action phase, one in delayed action phase
\pari \textbf{Behavior} Attack highest threat
\end{spelltargetinginfo}
\end{spellcontent}

\begin{monsterfooter}
\pari \textbf{Awareness} \plus6
\pari \textbf{Speed} 40 ft.;
\textbf{Space} 10 ft.;
\textbf{Reach} 10 ft.
\pari \textbf{Attributes}:
Str 10,
Dex 3,
Con 7,
Int \minus7,
Per 6,
Wil 0
\end{monsterfooter}
\end{monsection}

\begin{monsection}{Ankheg}{7}[2]
\vspace{-1em}\spelltwocol{}{Large magical beast}\vspace{-1em}
\begin{spellcontent}
\begin{spelltargetinginfo}
\spelltwocol{
\textbf{HP} 112;
\textbf{Bloodied} 56;
}
{\textbf{AP} 3/0}

\pari \textbf{Armor} 14;
\textbf{Fort} 17;
\textbf{Ref} 11;
\textbf{Ment} 12
\pari \textbf{Strike} Bite \plus9 (4d6)



\pari \textbf{Behavior} Attack highest threat
\end{spelltargetinginfo}
\end{spellcontent}

\begin{monsterfooter}
\pari \textbf{Awareness} \plus6
\pari \textbf{Speed} 40 ft.;
\textbf{Space} 10 ft.;
\textbf{Reach} 10 ft.
\pari \textbf{Attributes}:
Str 12,
Dex 4,
Con 9,
Int \minus7,
Per 8,
Wil 0
\end{monsterfooter}
\end{monsection}


\subsubsection{Ankheg Abilities}

\begin{freeability}{Drag Prey}
This ability functions like the \textit{shove} ability (see \pcref{Shove}), except that the ankheg's accuracy is \plus14.
In addition, the ankheg can move with the target up to a maximum distance equal to its \glossterm{base speed}.
\end{freeability}

\vspace{0.5em}
\begin{freeability}{Spit Acid}
The ankheg makes a \plus9 vs. Armor attack against everything in a 5 ft. wide Medium line.
\hit Each target takes 4d8 acid damage, and creatures are \glossterm{sickened} as a \glossterm{condition}.
\end{freeability}

\begin{monsection}{Aranea}{5}[2]
\vspace{-1em}\spelltwocol{}{Medium magical beast}\vspace{-1em}
\begin{spellcontent}
\begin{spelltargetinginfo}
\spelltwocol{
\textbf{HP} 60;
\textbf{Bloodied} 30;
}
{\textbf{AP} 2/3}

\pari \textbf{Armor} 11;
\textbf{Fort} 11;
\textbf{Ref} 13;
\textbf{Ment} 15
\pari \textbf{Strike} Bite \plus8 (1d10)



\pari \textbf{Behavior} Attack highest threat
\end{spelltargetinginfo}
\end{spellcontent}

\begin{monsterfooter}
\pari \textbf{Awareness} \plus6
\pari \textbf{Speed} 30 ft.;
\textbf{Space} 5 ft.;
\textbf{Reach} 5 ft.
\pari \textbf{Attributes}:
Str 3,
Dex 6,
Con 3,
Int 6,
Per 7,
Wil 7
\end{monsterfooter}
\end{monsection}


\subsubsection{Aranea Abilities}

\begin{freeability}{Shapeshift}
The aranea makes a Disguise check to change its appearance.
It ignores all penalties for differences between its natural appearance and its intended appearance.
\end{freeability}

\begin{monsection}{Basilisk}{5}[2]
\vspace{-1em}\spelltwocol{}{Medium magical beast}\vspace{-1em}
\begin{spellcontent}
\begin{spelltargetinginfo}
\spelltwocol{
\textbf{HP} 80;
\textbf{Bloodied} 40;
}
{\textbf{AP} 2/0}

\pari \textbf{Armor} 12;
\textbf{Fort} 15;
\textbf{Ref} 9;
\textbf{Ment} 10
\pari \textbf{Strike} Bite \plus8 (2d6)



\pari \textbf{Behavior} Attack highest threat
\end{spelltargetinginfo}
\end{spellcontent}

\begin{monsterfooter}
\pari \textbf{Awareness} \plus6
\pari \textbf{Speed} 30 ft.;
\textbf{Space} 5 ft.;
\textbf{Reach} 5 ft.
\pari \textbf{Attributes}:
Str 6,
Dex \minus1,
Con 7,
Int \minus6,
Per 7,
Wil 0
\end{monsterfooter}
\end{monsection}


\subsubsection{Basilisk Abilities}

\begin{freeability}{Petrifying Gaze}
The basilisk makes a \plus8 vs. Fortitude attack against one creature in Medium range.
\hit The target is \glossterm{nauseated} as a \glossterm{condition}.
\crit As above, and as an additional condition, the target takes 2d6 physical damage at the end of each action phase.
If it takes vital damage in this way, it is petrified permanently.
\end{freeability}

\begin{monsection}{Behir}{8}[3]
\vspace{-1em}\spelltwocol{}{Huge magical beast}\vspace{-1em}
\begin{spellcontent}
\begin{spelltargetinginfo}
\spelltwocol{
\textbf{HP} 216;
\textbf{Bloodied} 108;
}
{\textbf{AP} 4/0}

\pari \textbf{Armor} 16;
\textbf{Fort} 21;
\textbf{Ref} 11;
\textbf{Ment} 14
\pari \textbf{Strike} Bite \plus10 (4d10) or claw \plus10 (4d8)


\pari \textbf{Actions} One in action phase, one in delayed action phase
\pari \textbf{Behavior} Attack highest threat
\end{spelltargetinginfo}
\end{spellcontent}

\begin{monsterfooter}
\pari \textbf{Awareness} \plus6
\pari \textbf{Speed} 50 ft.;
\textbf{Space} 15 ft.;
\textbf{Reach} 15 ft.
\pari \textbf{Attributes}:
Str 16,
Dex 5,
Con 11,
Int \minus3,
Per 5,
Wil 0
\end{monsterfooter}
\end{monsection}


\subsubsection{Behir Abilities}

\begin{freeability}{Electric Breath}
The behir makes a \plus10 vs. Armor attack against everything in a \areamed cone.
\hit Each target takes 7d10 electricity damage, and is \glossterm{dazed} as a \glossterm{condition}.
\end{freeability}

\vspace{0.5em}
\begin{freeability}{Natural Grab}
The behir makes a bite \glossterm{strike}.
In addition to the effects of the strike, it also makes a \plus14 vs. Fortitude and Reflex attack against the same target.
\hit The target is \glossterm{grappled} by the behir.
\end{freeability}

\vspace{0.5em}
\begin{apability}{Rake}
The behir makes four claw \glossterm{strikes} against a target that is \glossterm{grappled} by it.
\end{apability}

\begin{monsection}{Blink Dog}{3}[1]
\vspace{-1em}\spelltwocol{}{Medium magical beast}\vspace{-1em}
\begin{spellcontent}
\begin{spelltargetinginfo}
\spelltwocol{
\textbf{HP} 15;
\textbf{Bloodied} 7;
}
{\textbf{AP} 1/0}

\pari \textbf{Armor} 9;
\textbf{Fort} 7;
\textbf{Ref} 12;
\textbf{Ment} 7
\pari \textbf{Strike} Bite \plus4 (1d10)



\pari \textbf{Behavior} Attack highest threat
\end{spelltargetinginfo}
\end{spellcontent}

\begin{monsterfooter}
\pari \textbf{Awareness} \plus6
\pari \textbf{Speed} 30 ft.;
\textbf{Space} 5 ft.;
\textbf{Reach} 5 ft.
\pari \textbf{Attributes}:
Str 4,
Dex 5,
Con 0,
Int 0,
Per 4,
Wil 0
\end{monsterfooter}
\end{monsection}


\subsubsection{Blink Dog Abilities}

\begin{freeability}{Blink}
As a \glossterm{move action}, the blink dog can use this ability.
If it does, it teleports to an unoccupied location within Medium range.
\end{freeability}

\begin{monsection}{Centaur}{3}[2]
\vspace{-1em}\spelltwocol{}{Large magical beast}\vspace{-1em}
\begin{spellcontent}
\begin{spelltargetinginfo}
\spelltwocol{
\textbf{HP} 42;
\textbf{Bloodied} 21;
}
{\textbf{AP} 2/1}

\pari \textbf{Armor} 9;
\textbf{Fort} 11;
\textbf{Ref} 9;
\textbf{Ment} 9
\pari \textbf{Strike} Longsword \plus5 (1d10) or longbow \plus5 (1d10) or hoof \plus5 (1d8)



\pari \textbf{Behavior} Attack highest threat
\end{spelltargetinginfo}
\end{spellcontent}

\begin{monsterfooter}
\pari \textbf{Awareness} \plus6
\pari \textbf{Speed} 40 ft.;
\textbf{Space} 10 ft.;
\textbf{Reach} 10 ft.
\pari \textbf{Attributes}:
Str 4,
Dex 4,
Con 4,
Int 0,
Per 4,
Wil 2
\end{monsterfooter}
\end{monsection}

\begin{monsection}{Cockatrice}{3}[1]
\vspace{-1em}\spelltwocol{}{Small magical beast}\vspace{-1em}
\begin{spellcontent}
\begin{spelltargetinginfo}
\spelltwocol{
\textbf{HP} 15;
\textbf{Bloodied} 7;
}
{\textbf{AP} 1/2}

\pari \textbf{Armor} 9;
\textbf{Fort} 7;
\textbf{Ref} 14;
\textbf{Ment} 10
\pari \textbf{Strike} Bite \plus4 (1d8)



\pari \textbf{Behavior} Attack highest threat
\end{spelltargetinginfo}
\end{spellcontent}

\begin{monsterfooter}
\pari \textbf{Awareness} \plus6
\pari \textbf{Speed} 25 ft.;
\textbf{Space} 5 ft.;
\textbf{Reach} 5 ft.
\pari \textbf{Attributes}:
Str \minus4,
Dex 5,
Con 0,
Int \minus8,
Per 4,
Wil 4
\end{monsterfooter}
\end{monsection}


\subsubsection{Cockatrice Abilities}

\begin{freeability}{Petrifying Bite}
The cockatrice makes a bite \glossterm{strike}.
In addition to the strike's normal effects, the cockatrice also makes a \plus4 vs. Fortitude attack against the target.
\hit If the strike also hit, the target is \glossterm{nauseated} as a \glossterm{condition}.
\crit As above, and as an additional condition, the target takes 1d6 physical damage at the end of each action phase.
If it takes vital damage in this way, it is petrified permanently.
\end{freeability}

\begin{monsection}{Darkmantle}{1}[1]
\vspace{-1em}\spelltwocol{}{Small magical beast}\vspace{-1em}
\begin{spellcontent}
\begin{spelltargetinginfo}
\spelltwocol{
\textbf{HP} 6;
\textbf{Bloodied} 3;
}
{\textbf{AP} 1/0}

\pari \textbf{Armor} 5;
\textbf{Fort} 6;
\textbf{Ref} 7;
\textbf{Ment} 5
\pari \textbf{Strike} Slam \plus1 (1d6)



\pari \textbf{Behavior} Attack highest threat
\end{spelltargetinginfo}
\end{spellcontent}

\begin{monsterfooter}
\pari \textbf{Awareness} \plus6
\pari \textbf{Speed} 25 ft.;
\textbf{Space} 5 ft.;
\textbf{Reach} 5 ft.
\pari \textbf{Attributes}:
Str 1,
Dex 0,
Con 1,
Int \minus8,
Per 0,
Wil 0
\end{monsterfooter}
\end{monsection}


\subsubsection{Darkmantle Abilities}

\begin{freeability}{Natural Grab}
The darkmantle makes a slam \glossterm{strike}.
In addition to the effects of the strike, it also makes a +\minus1 vs. Fortitude and Reflex attack against the same target.
\hit The target is \glossterm{grappled} by the darkmantle.
\end{freeability}

\begin{monsection}{Frost Worm}{12}[3]
\vspace{-1em}\spelltwocol{}{Gargantuan magical beast}\vspace{-1em}
\begin{spellcontent}
\begin{spelltargetinginfo}
\spelltwocol{
\textbf{HP} 360;
\textbf{Bloodied} 180;
}
{\textbf{AP} 4/0}

\pari \textbf{Armor} 20;
\textbf{Fort} 27;
\textbf{Ref} 12;
\textbf{Ment} 18
\pari \textbf{Strike} Bite \plus15 (7d10) or slam \plus15 (7d10)
\pari \textbf{Immune} cold

\pari \textbf{Actions} One in action phase, one in delayed action phase
\pari \textbf{Behavior} Attack highest threat
\end{spelltargetinginfo}
\end{spellcontent}

\begin{monsterfooter}
\pari \textbf{Awareness} \plus6
\pari \textbf{Speed} 60 ft.;
\textbf{Space} 20 ft.;
\textbf{Reach} 20 ft.
\pari \textbf{Attributes}:
Str 22,
Dex 0,
Con 16,
Int \minus8,
Per 13,
Wil 0
\end{monsterfooter}
\end{monsection}


\subsubsection{Frost Worm Abilities}

\begin{freeability}{Frost Breath}[\glossterm{Cold}]
The frost worm makes a \plus15 vs. Fortitude attack against everything in a \arealarge cone from it.
\hit Each target takes 11d10 cold damage.
\end{freeability}

\vspace{0.5em}
\begin{apability}{Trill}[\glossterm{Compulsion}]
The frost worm emits a piercing noise that compels prey to stay still.
It makes a \plus15 vs. Mental attack against creatures in a \areahuge radius from it.
This area can pass through solid objects, including the ground, but every 5 feet of solid obstacle counts as 20 feet of distance.
\hit Each target is \glossterm{dazed} and \glossterm{immobilized} as two separate \glossterm{conditions}.
\crit Each target is \glossterm{stunned} and \glossterm{immobilized} as two separate \glossterm{conditions}.
\end{apability}

\parhead{Bitter Cold}
The frost worm's bite and slam strikes deal cold damage in addition to their other damage types.

\vspace{0.5em}\parhead{Death Throes}
When a frost worm is killed, its corpse turns to ice and shatters in a violent explosion.
It makes a \plus15 vs. Fortitude attack against everything in a \areahuge radius from it.
\hit Each target takes 13d10 cold and piercing damage.

\begin{monsection}{Girallon}{5}[2]
\vspace{-1em}\spelltwocol{}{Large magical beast}\vspace{-1em}
\begin{spellcontent}
\begin{spelltargetinginfo}
\spelltwocol{
\textbf{HP} 60;
\textbf{Bloodied} 30;
}
{\textbf{AP} 2/\minus1}

\pari \textbf{Armor} 13;
\textbf{Fort} 11;
\textbf{Ref} 15;
\textbf{Ment} 9
\pari \textbf{Strike} Claw \plus9 (2d6)



\pari \textbf{Behavior} Attack highest threat
\end{spelltargetinginfo}
\end{spellcontent}

\begin{monsterfooter}
\pari \textbf{Awareness} \plus6
\pari \textbf{Speed} 40 ft.;
\textbf{Space} 10 ft.;
\textbf{Reach} 10 ft.
\pari \textbf{Attributes}:
Str 9,
Dex 8,
Con 3,
Int \minus8,
Per 6,
Wil \minus1
\end{monsterfooter}
\end{monsection}

\begin{monsection}{Griffon}{4}[2]
\vspace{-1em}\spelltwocol{}{Large magical beast}\vspace{-1em}
\begin{spellcontent}
\begin{spelltargetinginfo}
\spelltwocol{
\textbf{HP} 48;
\textbf{Bloodied} 24;
}
{\textbf{AP} 2/0}

\pari \textbf{Armor} 12;
\textbf{Fort} 10;
\textbf{Ref} 14;
\textbf{Ment} 9
\pari \textbf{Strike} Talon \plus8 (1d10)



\pari \textbf{Behavior} Attack highest threat
\end{spelltargetinginfo}
\end{spellcontent}

\begin{monsterfooter}
\pari \textbf{Awareness} \plus6
\pari \textbf{Speed} 80 ft. fly ft.;
\textbf{Space} 10 ft.;
\textbf{Reach} 10 ft.
\pari \textbf{Attributes}:
Str 7,
Dex 7,
Con 3,
Int \minus4,
Per 5,
Wil 0
\end{monsterfooter}
\end{monsection}


\subsubsection{Griffon Abilities}

\begin{freeability}{Flyby Attack}
The griffin flies up to its flying movement speed.
It can make a talon strike at any point during this movement.
\end{freeability}

\begin{monsection}{Hydra, 5 Headed}{5}[4]
\vspace{-1em}\spelltwocol{}{Huge magical beast}\vspace{-1em}
\begin{spellcontent}
\begin{spelltargetinginfo}
\spelltwocol{
\textbf{HP} 180;
\textbf{Bloodied} 90;
}
{\textbf{AP} 4/0}

\pari \textbf{Armor} 12;
\textbf{Fort} 19;
\textbf{Ref} 8;
\textbf{Ment} 12
\pari \textbf{Strike} Bite \plus9 (4d6)


\pari \textbf{Actions} Five in action phase
\pari \textbf{Behavior} Attack highest threat
\end{spelltargetinginfo}
\end{spellcontent}

\begin{monsterfooter}
\pari \textbf{Awareness} \plus6
\pari \textbf{Speed} 50 ft.;
\textbf{Space} 15 ft.;
\textbf{Reach} 15 ft.
\pari \textbf{Attributes}:
Str 12,
Dex 0,
Con 8,
Int \minus8,
Per 6,
Wil 0
\end{monsterfooter}
\end{monsection}


\subsubsection{Hydra, 5 Headed Abilities}

\parhead{Multi-Headed}
A hydra can take a number of actions in each \glossterm{action phase} equal to the number of heads it has active.
At the end of each action phase, if the hydra took at least 35 damage during that phase, it loses one of its heads.
Severed heads leave behind a stump that can quickly grow new heads.

At the end of each delayed action phase, if the hydra has a severed stump, the stump is either sealed or it grows two new heads.
If the hydra took 15 acid, cold, or fire damage during that phase, the stump is sealed, and will stop growing new heads.
Otherwise, the hydra grows two new heads from the stump.
This grants it additional actions during the action phase as normal.

A hydra cannot sustain too many excess heads for a prolonged period of time.
At the end of each round, if the hydra has more heads than twice its normal head count, it loses an \glossterm{action point}.
If it has no action points remaining, the hydra collapses unconscious for 8 hours.
During that time, the excess heads shrivel and die, and any sealed stumps heal, restoring the hydra to its normal head count.

\begin{monsection}{Hydra, 6 Headed}{6}[4]
\vspace{-1em}\spelltwocol{}{Huge magical beast}\vspace{-1em}
\begin{spellcontent}
\begin{spelltargetinginfo}
\spelltwocol{
\textbf{HP} 216;
\textbf{Bloodied} 108;
}
{\textbf{AP} 4/0}

\pari \textbf{Armor} 13;
\textbf{Fort} 20;
\textbf{Ref} 9;
\textbf{Ment} 13
\pari \textbf{Strike} Bite \plus10 (4d6)


\pari \textbf{Actions} Six in action phase
\pari \textbf{Behavior} Attack highest threat
\end{spelltargetinginfo}
\end{spellcontent}

\begin{monsterfooter}
\pari \textbf{Awareness} \plus6
\pari \textbf{Speed} 50 ft.;
\textbf{Space} 15 ft.;
\textbf{Reach} 15 ft.
\pari \textbf{Attributes}:
Str 13,
Dex 0,
Con 9,
Int \minus8,
Per 7,
Wil 0
\end{monsterfooter}
\end{monsection}


\subsubsection{Hydra, 6 Headed Abilities}

\parhead{Multi-Headed}
A hydra can take a number of actions in each \glossterm{action phase} equal to the number of heads it has active.
At the end of each action phase, if the hydra took at least 40 damage during that phase, it loses one of its heads.
Severed heads leave behind a stump that can quickly grow new heads.

At the end of each round, if the hydra has a severed stump, the stump is either sealed or it grows two new heads.
If the hydra took 20 acid, cold, or fire damage during that phase, the stump is sealed, and will stop growing new heads.
Otherwise, the hydra grows two new heads from the stump.
This grants it additional actions during the action phase as normal.

A hydra cannot sustain too many excess heads for a prolonged period of time.
At the end of each round, if the hydra has more heads than twice its normal head count, it loses an \glossterm{action point}.
If it has no action points remaining, the hydra collapses unconscious for 8 hours.
During that time, the excess heads shrivel and die, and any sealed stumps heal, restoring the hydra to its normal head count.

\begin{monsection}{Minotaur}{4}[1]
\vspace{-1em}\spelltwocol{}{Large magical beast}\vspace{-1em}
\begin{spellcontent}
\begin{spelltargetinginfo}
\spelltwocol{
\textbf{HP} 24;
\textbf{Bloodied} 12;
}
{\textbf{AP} 1/0}

\pari \textbf{Armor} 8;
\textbf{Fort} 9;
\textbf{Ref} 7;
\textbf{Ment} 8
\pari \textbf{Strike} Greataxe \plus5 (2d10) or gore \plus5 (2d8)



\pari \textbf{Behavior} Attack highest threat
\end{spelltargetinginfo}
\end{spellcontent}

\begin{monsterfooter}
\pari \textbf{Awareness} \plus6
\pari \textbf{Speed} 40 ft.;
\textbf{Space} 10 ft.;
\textbf{Reach} 10 ft.
\pari \textbf{Attributes}:
Str 8,
Dex 3,
Con 3,
Int \minus2,
Per 5,
Wil 0
\end{monsterfooter}
\end{monsection}


\subsubsection{Minotaur Abilities}

\begin{freeability}{Impaling Charge}
The minotaur moves up to its speed in a single straight line and makes a gore \glossterm{strike} from its new location.
\end{freeability}

\parhead{Labyrinth Dweller}
The minotaur never gets lost or loses track of its current location.

\begin{monsection}{Thaumavore}{3}[1]
\vspace{-1em}\spelltwocol{}{Small magical beast}\vspace{-1em}
\begin{spellcontent}
\begin{spelltargetinginfo}
\spelltwocol{
\textbf{HP} 15;
\textbf{Bloodied} 7;
}
{\textbf{AP} 1/0}

\pari \textbf{Armor} 11;
\textbf{Fort} 7;
\textbf{Ref} 14;
\textbf{Ment} 7
\pari \textbf{Strike} Bite \plus3 (1d8)



\pari \textbf{Behavior} Attack highest threat that has a source of magic; if no souces of magic exist, attack highest threat
\end{spelltargetinginfo}
\end{spellcontent}

\begin{monsterfooter}
\pari \textbf{Awareness} \plus6
\pari \textbf{Speed} 25 ft.;
\textbf{Space} 5 ft.;
\textbf{Reach} 5 ft.
\pari \textbf{Attributes}:
Str 2,
Dex 5,
Con 0,
Int \minus7,
Per 0,
Wil 0
\end{monsterfooter}
\end{monsection}


\subsubsection{Thaumavore Abilities}

\parhead{Consume Magic}
The thaumavore gains a \plus4 bonus to \glossterm{defenses} against \glossterm{magical} abilities.
Whenever it resists a \glossterm{magical} attack, it heals hit points equal to twice the \glossterm{power} of the effect.

\vspace{0.5em}\parhead{Sense Magic}
The thaumavore can sense the location of all sources of magic within 100 feet of it.
This includes magic items, attuned magical abilities, and so on.

\begin{monsection}{Banehound}{5}[4]
\vspace{-1em}\spelltwocol{}{Huge magical beast}\vspace{-1em}
\begin{spellcontent}
\begin{spelltargetinginfo}
\spelltwocol{
\textbf{HP} 100;
\textbf{Bloodied} 50;
}
{\textbf{AP} 4/0}

\pari \textbf{Armor} 17;
\textbf{Fort} 12;
\textbf{Ref} 15;
\textbf{Ment} 12
\pari \textbf{Strike} Bite \plus10 (2d10)


\pari \textbf{Actions} One in action phase, one in delayed action phase
\pari \textbf{Behavior} Attack highest threat
\end{spelltargetinginfo}
\end{spellcontent}

\begin{monsterfooter}
\pari \textbf{Awareness} \plus6
\pari \textbf{Speed} 50 ft.;
\textbf{Space} 15 ft.;
\textbf{Reach} 15 ft.
\pari \textbf{Attributes}:
Str 10,
Dex 8,
Con 0,
Int 3,
Per 7,
Wil 0
\end{monsterfooter}
\end{monsection}

\begin{monsection}{Fleshfeeder}{4}[1]
\vspace{-1em}\spelltwocol{}{Medium magical beast}\vspace{-1em}
\begin{spellcontent}
\begin{spelltargetinginfo}
\spelltwocol{
\textbf{HP} 28;
\textbf{Bloodied} 14;
}
{\textbf{AP} 1/0}

\pari \textbf{Armor} 10;
\textbf{Fort} 11;
\textbf{Ref} 13;
\textbf{Ment} 8
\pari \textbf{Strike} Bite \plus5 (1d10)



\pari \textbf{Behavior} Attack highest threat
\end{spelltargetinginfo}
\end{spellcontent}

\begin{monsterfooter}
\pari \textbf{Awareness} \plus6
\pari \textbf{Speed} 30 ft.;
\textbf{Space} 5 ft.;
\textbf{Reach} 5 ft.
\pari \textbf{Attributes}:
Str 5,
Dex 6,
Con 5,
Int 0,
Per 5,
Wil 0
\end{monsterfooter}
\end{monsection}

\section{Monstrous Humanoids}
\begin{monsection}{Banshee}{3}[2]
\vspace{-1em}\spelltwocol{}{Medium monstrous humanoid}\vspace{-1em}
\begin{spellcontent}
\begin{spelltargetinginfo}
\spelltwocol{
\textbf{HP} 30;
\textbf{Bloodied} 15;
}
{\textbf{AP} 2/4}

\pari \textbf{Armor} 9;
\textbf{Fort} 8;
\textbf{Ref} 11;
\textbf{Ment} 15
\pari \textbf{Strike} Claw \plus5 (1d8)



\pari \textbf{Behavior} Attack highest threat
\end{spelltargetinginfo}
\end{spellcontent}

\begin{monsterfooter}
\pari \textbf{Awareness} \plus6
\pari \textbf{Speed} 30 ft.;
\textbf{Space} 5 ft.;
\textbf{Reach} 5 ft.
\pari \textbf{Attributes}:
Str 4,
Dex 4,
Con 0,
Int 0,
Per 4,
Wil 6
\end{monsterfooter}
\end{monsection}


\subsubsection{Banshee Abilities}

\begin{freeability}{Wail}
The banshee makes a \plus5 vs. Fortitude attack against everything in a Large radius.
\hit Each target takes 2d6 sonic damage, and creatures are sickened as a condition.
\end{freeability}

\begin{monsection}{Giant}[Hill]{6}[1]
\vspace{-1em}\spelltwocol{}{Large monstrous humanoid}\vspace{-1em}
\begin{spellcontent}
\begin{spelltargetinginfo}
\spelltwocol{
\textbf{HP} 42;
\textbf{Bloodied} 21;
}
{\textbf{AP} 1/0}

\pari \textbf{Armor} 11;
\textbf{Fort} 13;
\textbf{Ref} 7;
\textbf{Ment} 10
\pari \textbf{Strike} Greatclub \plus6 (4d6) or boulder \plus6 (2d10)



\pari \textbf{Behavior} Attack highest threat
\end{spelltargetinginfo}
\end{spellcontent}

\begin{monsterfooter}
\pari \textbf{Awareness} \plus6
\pari \textbf{Speed} 40 ft.;
\textbf{Space} 10 ft.;
\textbf{Reach} 10 ft.
\pari \textbf{Attributes}:
Str 11,
Dex \minus1,
Con 7,
Int \minus2,
Per 0,
Wil 0
\end{monsterfooter}
\end{monsection}


\subsubsection{Giant Abilities}

\begin{freeability}{Boulder Toss}
The giant makes a ranged boulder strike, treating it as a thrown weapon with a 100 ft.\ range increment.
\end{freeability}

\begin{monsection}{Giant}[Stone]{9}[1]
\vspace{-1em}\spelltwocol{}{Huge monstrous humanoid}\vspace{-1em}
\begin{spellcontent}
\begin{spelltargetinginfo}
\spelltwocol{
\textbf{HP} 63;
\textbf{Bloodied} 31;
}
{\textbf{AP} 2/0}

\pari \textbf{Armor} 16;
\textbf{Fort} 16;
\textbf{Ref} 8;
\textbf{Ment} 13
\pari \textbf{Strike} Greatclub \plus9 (5d10) or boulder \plus9 (4d10)



\pari \textbf{Behavior} Attack highest threat
\end{spelltargetinginfo}
\end{spellcontent}

\begin{monsterfooter}
\pari \textbf{Awareness} \plus6
\pari \textbf{Speed} 50 ft.;
\textbf{Space} 15 ft.;
\textbf{Reach} 15 ft.
\pari \textbf{Attributes}:
Str 16,
Dex \minus1,
Con 10,
Int \minus1,
Per 5,
Wil 0
\end{monsterfooter}
\end{monsection}


\subsubsection{Giant Abilities}

\begin{freeability}{Boulder Toss}
The giant makes a ranged boulder strike, treating it as a thrown weapon with a 100 ft.\ range increment.
\end{freeability}

\begin{monsection}{Giant}[Storm]{15}[1]
\vspace{-1em}\spelltwocol{}{Gargantuan monstrous humanoid}\vspace{-1em}
\begin{spellcontent}
\begin{spelltargetinginfo}
\spelltwocol{
\textbf{HP} 105;
\textbf{Bloodied} 52;
}
{\textbf{AP} 3/2}

\pari \textbf{Armor} 22;
\textbf{Fort} 22;
\textbf{Ref} 13;
\textbf{Ment} 22
\pari \textbf{Strike} Greatsword \plus16 (9d10)
\pari \textbf{Immune} deafened


\pari \textbf{Behavior} Attack highest threat
\end{spelltargetinginfo}
\end{spellcontent}

\begin{monsterfooter}
\pari \textbf{Awareness} \plus6
\pari \textbf{Speed} 60 ft.;
\textbf{Space} 20 ft.;
\textbf{Reach} 20 ft.
\pari \textbf{Attributes}:
Str 25,
Dex 0,
Con 16,
Int 8,
Per 16,
Wil 16
\end{monsterfooter}
\end{monsection}


\subsubsection{Giant Abilities}

\begin{freeability}{Lightning Javelin}
The storm giant makes a \plus16 vs. Fortitude attack against everything in a 10 ft. wide Large line.
\hit Each target takes 10d10 electricity damage.
\end{freeability}

\vspace{0.5em}
\begin{freeability}{Thunderstrike}
The storm giant makes a greatsword strike against a target.
If its attack result beats the target's Fortitude defense,
the target also takes 8d10 sonic damage
and is deafened as a condition.
\end{freeability}

\begin{monsection}{Hag}[Green]{5}[2]
\vspace{-1em}\spelltwocol{}{Medium monstrous humanoid}\vspace{-1em}
\begin{spellcontent}
\begin{spelltargetinginfo}
\spelltwocol{
\textbf{HP} 70;
\textbf{Bloodied} 35;
}
{\textbf{AP} 2/2}

\pari \textbf{Armor} 11;
\textbf{Fort} 13;
\textbf{Ref} 13;
\textbf{Ment} 13
\pari \textbf{Strike} Claw \plus8 (1d8)



\pari \textbf{Behavior} Attack highest threat
\end{spelltargetinginfo}
\end{spellcontent}

\begin{monsterfooter}
\pari \textbf{Awareness} \plus6
\pari \textbf{Speed} 30 ft.;
\textbf{Space} 5 ft.;
\textbf{Reach} 5 ft.
\pari \textbf{Attributes}:
Str 0,
Dex 6,
Con 6,
Int 6,
Per 7,
Wil 6
\end{monsterfooter}
\end{monsection}


\subsubsection{Hag Abilities}

\begin{freeability}{Vital Surge}
The hag makes a \plus8 vs. Fortitude attack against one creature within Medium range.
\hit The target takes 2d10 life damage.
\end{freeability}

\vspace{0.5em}
\begin{freeability}{Green Hag's Curse}
The hag makes a \plus8 vs. Mental atack aginst one creature within Medium range.
\hit As a condition, the target is either dazed, fatigued, or sickened, as the hag chooses.
\crit As three separate conditions, the target is dazed, fatigued, and sickened.
\end{freeability}

\parhead{Coven Rituals}
When three or more hags work together, they form a coven.
All members of the coven gain the ability to perform nature rituals as long as they work together.
Hags of any type can form a coven together.

\begin{monsection}{Medusa}{7}[2]
\vspace{-1em}\spelltwocol{}{Medium monstrous humanoid}\vspace{-1em}
\begin{spellcontent}
\begin{spelltargetinginfo}
\spelltwocol{
\textbf{HP} 70;
\textbf{Bloodied} 35;
}
{\textbf{AP} 3/2}

\pari \textbf{Armor} 12;
\textbf{Fort} 12;
\textbf{Ref} 13;
\textbf{Ment} 15
\pari \textbf{Strike} Longbow \plus10 (2d6) or snakes \plus10 (1d10)



\pari \textbf{Behavior} Attack highest threat
\end{spelltargetinginfo}
\end{spellcontent}

\begin{monsterfooter}
\pari \textbf{Awareness} \plus6
\pari \textbf{Speed} 30 ft.;
\textbf{Space} 5 ft.;
\textbf{Reach} 5 ft.
\pari \textbf{Attributes}:
Str 0,
Dex 4,
Con 0,
Int 4,
Per 9,
Wil 8
\end{monsterfooter}
\end{monsection}


\subsubsection{Medusa Abilities}

\begin{freeability}{Petrifying Gaze}
The medusa makes a \plus10 vs. Fortitude attack against one creature in Medium range.
\hit The target is \glossterm{nauseated} as a \glossterm{condition}.
\crit As above, and as an additional condition, the target takes 2d6 physical damage at the end of each action phase.
If it takes vital damage in this way, it is petrified permanently.
\end{freeability}

\begin{monsection}{Harpy}[Harpy Archer]{12}[1]
\vspace{-1em}\spelltwocol{}{Medium monstrous humanoid}\vspace{-1em}
\begin{spellcontent}
\begin{spelltargetinginfo}
\spelltwocol{
\textbf{HP} 72;
\textbf{Bloodied} 36;
}
{\textbf{AP} 2/3}

\pari \textbf{Armor} 21;
\textbf{Fort} 17;
\textbf{Ref} 23;
\textbf{Ment} 21
\pari \textbf{Strike} Longbow \plus15 (4d6)



\pari \textbf{Behavior} Attack highest threat
\end{spelltargetinginfo}
\end{spellcontent}

\begin{monsterfooter}
\pari \textbf{Awareness} \plus6
\pari \textbf{Speed} 30 ft.;
\textbf{Space} 5 ft.;
\textbf{Reach} 5 ft.
\pari \textbf{Attributes}:
Str 13,
Dex 15,
Con 7,
Int 7,
Per 15,
Wil 14
\end{monsterfooter}
\end{monsection}

\section{Outsiders}
\begin{monsection}{Angel}[Astral Deva]{14}[2]
\vspace{-1em}\spelltwocol{}{Medium outsider}\vspace{-1em}
\begin{spellcontent}
\begin{spelltargetinginfo}
\spelltwocol{
\textbf{HP} 224;
\textbf{Bloodied} 112;
}
{\textbf{AP} 4/3}

\pari \textbf{Armor} 25;
\textbf{Fort} 22;
\textbf{Ref} 24;
\textbf{Ment} 26
\pari \textbf{Strike} Mace \plus17 (4d10)



\pari \textbf{Behavior} Attack highest threat
\end{spelltargetinginfo}
\end{spellcontent}

\begin{monsterfooter}
\pari \textbf{Awareness} \plus6
\pari \textbf{Speed} 30 ft.;
\textbf{Space} 5 ft.;
\textbf{Reach} 5 ft.
\pari \textbf{Attributes}:
Str 16,
Dex 16,
Con 16,
Int 16,
Per 16,
Wil 16
\end{monsterfooter}
\end{monsection}


\subsubsection{Angel Abilities}

\begin{freeability}{Smite}
The angel makes a melee \glossterm{strike}.
If its target is evil, it gains a \plus2 bonus to accuracy and a \plus2d bonus to damage on the strike.
\end{freeability}

\vspace{0.5em}
\begin{freeability}{Angel's Grace}
One \glossterm{ally} within reach heals 8d10 hit points.
\end{freeability}

\begin{monsection}{Arrowhawk}{3}[1]
\vspace{-1em}\spelltwocol{}{Medium outsider}\vspace{-1em}
\begin{spellcontent}
\begin{spelltargetinginfo}
\spelltwocol{
\textbf{HP} 12;
\textbf{Bloodied} 6;
}
{\textbf{AP} 1/0}

\pari \textbf{Armor} 10;
\textbf{Fort} 4;
\textbf{Ref} 14;
\textbf{Ment} 9
\pari \textbf{Strike} Bite \plus4 (1d8)



\pari \textbf{Behavior} Attack lowest threat
\end{spelltargetinginfo}
\end{spellcontent}

\begin{monsterfooter}
\pari \textbf{Awareness} \plus6
\pari \textbf{Speed} 60 ft. fly (good) ft.;
\textbf{Space} 5 ft.;
\textbf{Reach} 5 ft.
\pari \textbf{Attributes}:
Str 2,
Dex 6,
Con \minus1,
Int 0,
Per 4,
Wil 0
\end{monsterfooter}
\end{monsection}


\subsubsection{Arrowhawk Abilities}

\begin{freeability}{Electrobolt}
The arrowhawk makes a \plus4 vs. Fortitude attack against one creature or object in Medium range.
\hit The target takes 2d6 electricity damage.
\end{freeability}

\begin{monsection}{Demon}[Bebelith]{11}[3]
\vspace{-1em}\spelltwocol{}{Huge outsider}\vspace{-1em}
\begin{spellcontent}
\begin{spelltargetinginfo}
\spelltwocol{
\textbf{HP} 231;
\textbf{Bloodied} 115;
}
{\textbf{AP} 4/0}

\pari \textbf{Armor} 21;
\textbf{Fort} 18;
\textbf{Ref} 18;
\textbf{Ment} 19
\pari \textbf{Strike} Bite \plus14 (4d10)


\pari \textbf{Actions} One in action phase, one in delayed action phase
\pari \textbf{Behavior} Attack highest threat
\end{spelltargetinginfo}
\end{spellcontent}

\begin{monsterfooter}
\pari \textbf{Awareness} \plus6
\pari \textbf{Speed} 50 ft.;
\textbf{Space} 15 ft.;
\textbf{Reach} 15 ft.
\pari \textbf{Attributes}:
Str 16,
Dex 13,
Con 12,
Int 0,
Per 12,
Wil 0
\end{monsterfooter}
\end{monsection}


\subsubsection{Demon Abilities}

\begin{freeability}{Venomous Bite}
The bebelith makes a bite strike.
If it hits, and the attack result beats the target's Fortitude defense, the target is also poisoned as a condition.
If the target is poisoned, it takes 6d10 poison damage at the end of each action phase after the first round.
\end{freeability}

\begin{monsection}{Hell Hound}{4}[1]
\vspace{-1em}\spelltwocol{}{Medium outsider}\vspace{-1em}
\begin{spellcontent}
\begin{spelltargetinginfo}
\spelltwocol{
\textbf{HP} 20;
\textbf{Bloodied} 10;
}
{\textbf{AP} 1/0}

\pari \textbf{Armor} 10;
\textbf{Fort} 6;
\textbf{Ref} 13;
\textbf{Ment} 10
\pari \textbf{Strike} Bite \plus5 (1d10)
\pari \textbf{Immune} fire damage


\pari \textbf{Behavior} Attack highest threat
\end{spelltargetinginfo}
\end{spellcontent}

\begin{monsterfooter}
\pari \textbf{Awareness} \plus6
\pari \textbf{Speed} 30 ft.;
\textbf{Space} 5 ft.;
\textbf{Reach} 5 ft.
\pari \textbf{Attributes}:
Str 3,
Dex 6,
Con 0,
Int \minus3,
Per 5,
Wil 0
\end{monsterfooter}
\end{monsection}


\subsubsection{Hell Hound Abilities}

\begin{freeability}{Fire Breath}
The hell hound makes a \plus5 vs. Armor attack against everything in a Medium cone.
\hit Each target takes 2d6 fire damage.
\end{freeability}

\begin{monsection}{Salamander}[Flamebrother]{4}[1]
\vspace{-1em}\spelltwocol{}{Medium outsider}\vspace{-1em}
\begin{spellcontent}
\begin{spelltargetinginfo}
\spelltwocol{
\textbf{HP} 20;
\textbf{Bloodied} 10;
}
{\textbf{AP} 1/0}

\pari \textbf{Armor} 11;
\textbf{Fort} 6;
\textbf{Ref} 11;
\textbf{Ment} 10
\pari \textbf{Strike} Spear \plus4 (2d6) or tail slam \plus4 (2d6)
\pari \textbf{Immune} fire damage


\pari \textbf{Behavior} Attack highest threat
\end{spelltargetinginfo}
\end{spellcontent}

\begin{monsterfooter}
\pari \textbf{Awareness} \plus6
\pari \textbf{Speed} 30 ft.;
\textbf{Space} 5 ft.;
\textbf{Reach} 5 ft.
\pari \textbf{Attributes}:
Str 7,
Dex 5,
Con 0,
Int 3,
Per 3,
Wil 0
\end{monsterfooter}
\end{monsection}


\subsubsection{Salamander Abilities}

\begin{apability}{Flame Aura}[\glossterm{Sustain} (standard)]
The salamander intensifies its natural body heat, creating a burning aura around it.
At the end of each action phase, the salamander makes a \plus4 vs. Armor
attack against everything within a Medium radius emanation of it.
\hit Each target takes 2d6 fire damage.
\end{apability}

\vspace{0.5em}
\begin{freeability}{Natural Grab}
The salamander makes a tail slam \glossterm{strike}.
In addition to the effects of the strike, it also makes a \plus4 vs. Fortitude and Reflex attack against the same target.
\hit The target is \glossterm{grappled} by the salamander.
\end{freeability}

\begin{monsection}{Janni}{7}[1]
\vspace{-1em}\spelltwocol{}{Medium outsider}\vspace{-1em}
\begin{spellcontent}
\begin{spelltargetinginfo}
\spelltwocol{
\textbf{HP} 35;
\textbf{Bloodied} 17;
}
{\textbf{AP} 2/1}

\pari \textbf{Armor} 17;
\textbf{Fort} 9;
\textbf{Ref} 16;
\textbf{Ment} 14
\pari \textbf{Strike} Shortsword \plus9 (2d6)



\pari \textbf{Behavior} Attack highest threat
\end{spelltargetinginfo}
\end{spellcontent}

\begin{monsterfooter}
\pari \textbf{Awareness} \plus6
\pari \textbf{Speed} 30 ft.;
\textbf{Space} 5 ft.;
\textbf{Reach} 5 ft.
\pari \textbf{Attributes}:
Str 8,
Dex 9,
Con 0,
Int 4,
Per 8,
Wil 4
\end{monsterfooter}
\end{monsection}

\begin{monsection}{Salamander}[Battlemaster]{5}[3]
\vspace{-1em}\spelltwocol{}{Medium outsider}\vspace{-1em}
\begin{spellcontent}
\begin{spelltargetinginfo}
\spelltwocol{
\textbf{HP} 75;
\textbf{Bloodied} 37;
}
{\textbf{AP} 3/1}

\pari \textbf{Armor} 14;
\textbf{Fort} 9;
\textbf{Ref} 14;
\textbf{Ment} 14
\pari \textbf{Strike} Spear \plus8 (2d8) or tail slam \plus8 (2d8)


\pari \textbf{Actions} One in action phase, one in delayed action phase
\pari \textbf{Behavior} Attack highest threat
\end{spelltargetinginfo}
\end{spellcontent}

\begin{monsterfooter}
\pari \textbf{Awareness} \plus6
\pari \textbf{Speed} 30 ft.;
\textbf{Space} 5 ft.;
\textbf{Reach} 5 ft.
\pari \textbf{Attributes}:
Str 8,
Dex 6,
Con 0,
Int 3,
Per 6,
Wil 3
\end{monsterfooter}
\end{monsection}


\subsubsection{Salamander Abilities}

\begin{apability}{Flame Aura}[\glossterm{Sustain} (standard)]
The salamander intensifies its natural body heat, creating a burning aura around it.
At the end of each action phase, the salamander makes a \plus4 vs. Armor
attack against everything within a Medium radius emanation of it.
\hit Each target takes 2d6 fire damage.
\end{apability}

\vspace{0.5em}
\begin{freeability}{Natural Grab}
The salamander makes a tail slam \glossterm{strike}.
In addition to the effects of the strike, it also makes a \plus8 vs. Fortitude and Reflex attack against the same target.
\hit The target is \glossterm{grappled} by the salamander.
\end{freeability}

\section{Undead}
\begin{monsection}{Allip}{4}[1]
\vspace{-1em}\spelltwocol{}{Medium undead}\vspace{-1em}
\begin{spellcontent}
\begin{spelltargetinginfo}
\spelltwocol{
\textbf{HP} 20;
\textbf{Bloodied} 10;
}
{\textbf{AP} 1/3}

\pari \textbf{Armor} 10;
\textbf{Fort} 6;
\textbf{Ref} 13;
\textbf{Ment} 15
\pari \textbf{Strike} Draining touch \plus6 vs. Reflex (2d6)



\pari \textbf{Behavior} Attack highest threat
\end{spelltargetinginfo}
\end{spellcontent}

\begin{monsterfooter}
\pari \textbf{Awareness} \plus6
\pari \textbf{Speed} 30 ft.;
\textbf{Space} 5 ft.;
\textbf{Reach} 5 ft.
\pari \textbf{Attributes}:
Str 0,
Dex 6,
Con 0,
Int 0,
Per 0,
Wil 6
\end{monsterfooter}
\end{monsection}

\begin{monsection}{Spectre}{7}[2]
\vspace{-1em}\spelltwocol{}{Medium undead}\vspace{-1em}
\begin{spellcontent}
\begin{spelltargetinginfo}
\spelltwocol{
\textbf{HP} 70;
\textbf{Bloodied} 35;
}
{\textbf{AP} 3/4}

\pari \textbf{Armor} 15;
\textbf{Fort} 10;
\textbf{Ref} 19;
\textbf{Ment} 21
\pari \textbf{Strike} Draining touch \plus11 vs. Reflex (2d10)



\pari \textbf{Behavior} Attack highest threat
\end{spelltargetinginfo}
\end{spellcontent}

\begin{monsterfooter}
\pari \textbf{Awareness} \plus6
\pari \textbf{Speed} 30 ft.;
\textbf{Space} 5 ft.;
\textbf{Reach} 5 ft.
\pari \textbf{Attributes}:
Str 0,
Dex 10,
Con 0,
Int 0,
Per 8,
Wil 10
\end{monsterfooter}
\end{monsection}

\begin{monsection}{Dirgewalker}{4}[3]
\vspace{-1em}\spelltwocol{}{Medium undead}\vspace{-1em}
\begin{spellcontent}
\begin{spelltargetinginfo}
\spelltwocol{
\textbf{HP} 60;
\textbf{Bloodied} 30;
}
{\textbf{AP} 3/3}

\pari \textbf{Armor} 15;
\textbf{Fort} 8;
\textbf{Ref} 17;
\textbf{Ment} 17
\pari \textbf{Strike} Claw \plus9 (1d8)


\pari \textbf{Actions} One in action phase, one in delayed action phase
\pari \textbf{Behavior} Attack highest threat
\end{spelltargetinginfo}
\end{spellcontent}

\begin{monsterfooter}
\pari \textbf{Awareness} \plus6
\pari \textbf{Speed} 30 ft.;
\textbf{Space} 5 ft.;
\textbf{Reach} 5 ft.
\pari \textbf{Attributes}:
Str 0,
Dex 7,
Con 0,
Int 3,
Per 6,
Wil 6
\end{monsterfooter}
\end{monsection}


\subsubsection{Dirgewalker Abilities}

\begin{attuneability}{Animating Caper}[\glossterm{Attune} (self)]
One corpse within Close range is animated as a skeleton under the dirgewalker's control.
\end{attuneability}

\vspace{0.5em}
\begin{freeability}{Mournful Dirge}
The dirgewalker makes a \plus8 vs. Mental attack against all creatures in a Medium radius.
\hit Each target is dazed as a condition.
\crit Each target is stunned as a condition.
\end{freeability}

\begin{monsection}{Skeleton}{1}[1]
\vspace{-1em}\spelltwocol{}{Medium undead}\vspace{-1em}
\begin{spellcontent}
\begin{spelltargetinginfo}
\spelltwocol{
\textbf{HP} 5;
\textbf{Bloodied} 2;
}
{\textbf{AP} 1/0}

\pari \textbf{Armor} 7;
\textbf{Fort} 3;
\textbf{Ref} 8;
\textbf{Ment} 7
\pari \textbf{Strike} Claw \plus2 (1d6)



\pari \textbf{Behavior} Attack highest threat
\end{spelltargetinginfo}
\end{spellcontent}

\begin{monsterfooter}
\pari \textbf{Awareness} \plus6
\pari \textbf{Speed} 30 ft.;
\textbf{Space} 5 ft.;
\textbf{Reach} 5 ft.
\pari \textbf{Attributes}:
Str 2,
Dex 2,
Con 0,
Int 0,
Per 0,
Wil 0
\end{monsterfooter}
\end{monsection}


\subsubsection{Skeleton Abilities}

\parhead{Hard Bones}
The skeleton has \glossterm{damage reduction} 1 against piercing and slashing damage.

\begin{monsection}{Skeleton}[Warrior]{4}[1]
\vspace{-1em}\spelltwocol{}{Medium undead}\vspace{-1em}
\begin{spellcontent}
\begin{spelltargetinginfo}
\spelltwocol{
\textbf{HP} 20;
\textbf{Bloodied} 10;
}
{\textbf{AP} 1/0}

\pari \textbf{Armor} 11;
\textbf{Fort} 6;
\textbf{Ref} 13;
\textbf{Ment} 10
\pari \textbf{Strike} Longsword \plus4 (1d10)



\pari \textbf{Behavior} Attack highest threat
\end{spelltargetinginfo}
\end{spellcontent}

\begin{monsterfooter}
\pari \textbf{Awareness} \plus6
\pari \textbf{Speed} 30 ft.;
\textbf{Space} 5 ft.;
\textbf{Reach} 5 ft.
\pari \textbf{Attributes}:
Str 5,
Dex 6,
Con 0,
Int 0,
Per 0,
Wil 0
\end{monsterfooter}
\end{monsection}


\subsubsection{Skeleton Abilities}

\parhead{Hard Bones}
The skeleton has \glossterm{damage reduction} 1 against piercing and slashing damage.

\begin{monsection}{Skeleton}[Mage]{4}[1]
\vspace{-1em}\spelltwocol{}{Medium undead}\vspace{-1em}
\begin{spellcontent}
\begin{spelltargetinginfo}
\spelltwocol{
\textbf{HP} 20;
\textbf{Bloodied} 10;
}
{\textbf{AP} 1/3}

\pari \textbf{Armor} 11;
\textbf{Fort} 6;
\textbf{Ref} 13;
\textbf{Ment} 15
\pari \textbf{Strike} Claw \plus6 (1d8)



\pari \textbf{Behavior} Attack highest threat
\end{spelltargetinginfo}
\end{spellcontent}

\begin{monsterfooter}
\pari \textbf{Awareness} \plus6
\pari \textbf{Speed} 30 ft.;
\textbf{Space} 5 ft.;
\textbf{Reach} 5 ft.
\pari \textbf{Attributes}:
Str 0,
Dex 6,
Con 0,
Int 0,
Per 0,
Wil 6
\end{monsterfooter}
\end{monsection}


\subsubsection{Skeleton Abilities}

\begin{freeability}{Drain Life}[\glossterm{Life}]
The skeleton mage makes a \plus4 vs. Fortitude attack against a creature in \rngmed range.
\hit The target takes 2d6 life damage.
In addition, the skeleton mage heals 6 hit points.
\end{freeability}

\vspace{0.5em}
\begin{freeability}{Terror}[\glossterm{Life}]
The skeleton mage makes a \plus4 vs. Mental attack against a creature in \rngmed range.
\hit The target is \frightened by you as a \glossterm{condition}.
\crit The target is \panicked by you as a \glossterm{condition}.
\end{freeability}

\parhead{Hard Bones}
The skeleton has \glossterm{damage reduction} 1 against piercing and slashing damage.

\begin{monsection}{Skeleton}[Warrior]{4}[3]
\vspace{-1em}\spelltwocol{}{Large undead}\vspace{-1em}
\begin{spellcontent}
\begin{spelltargetinginfo}
\spelltwocol{
\textbf{HP} 72;
\textbf{Bloodied} 36;
}
{\textbf{AP} 3/0}

\pari \textbf{Armor} 14;
\textbf{Fort} 9;
\textbf{Ref} 15;
\textbf{Ment} 12
\pari \textbf{Strike} Greatsword \plus6 (2d10)


\pari \textbf{Actions} One in action phase, one in delayed action phase
\pari \textbf{Behavior} Attack highest threat
\end{spelltargetinginfo}
\end{spellcontent}

\begin{monsterfooter}
\pari \textbf{Awareness} \plus6
\pari \textbf{Speed} 40 ft.;
\textbf{Space} 10 ft.;
\textbf{Reach} 10 ft.
\pari \textbf{Attributes}:
Str 8,
Dex 7,
Con 3,
Int 0,
Per 0,
Wil 0
\end{monsterfooter}
\end{monsection}


\subsubsection{Skeleton Abilities}

\parhead{Hard Bones}
The skeleton has \glossterm{damage reduction} 1 against piercing and slashing damage.

\begin{monsection}{Zombie}{1}[1]
\vspace{-1em}\spelltwocol{}{Medium undead}\vspace{-1em}
\begin{spellcontent}
\begin{spelltargetinginfo}
\spelltwocol{
\textbf{HP} 8;
\textbf{Bloodied} 4;
}
{\textbf{AP} 1/0}

\pari \textbf{Armor} 4;
\textbf{Fort} 8;
\textbf{Ref} 5;
\textbf{Ment} 7
\pari \textbf{Strike} Slam \plus1 (1d6)



\pari \textbf{Behavior} Attack highest threat
\end{spelltargetinginfo}
\end{spellcontent}

\begin{monsterfooter}
\pari \textbf{Awareness} \plus6
\pari \textbf{Speed} 30 ft.;
\textbf{Space} 5 ft.;
\textbf{Reach} 5 ft.
\pari \textbf{Attributes}:
Str 1,
Dex 0,
Con 3,
Int 0,
Per 0,
Wil 0
\end{monsterfooter}
\end{monsection}


\subsubsection{Zombie Abilities}

\parhead{Slow}
The zombie does not act during the \glossterm{action phase}.
Instead, it acts during the \glossterm{delayed action phase}.

\vspace{0.5em}\parhead{Soft Flesh}
The zombie has \glossterm{damage reduction} 3 against piercing and bludgeoning damage.

\begin{monsection}{Zombie}[Warrior]{2}[2]
\vspace{-1em}\spelltwocol{}{Medium undead}\vspace{-1em}
\begin{spellcontent}
\begin{spelltargetinginfo}
\spelltwocol{
\textbf{HP} 36;
\textbf{Bloodied} 18;
}
{\textbf{AP} 2/0}

\pari \textbf{Armor} 6;
\textbf{Fort} 12;
\textbf{Ref} 7;
\textbf{Ment} 9
\pari \textbf{Strike} Slam \plus3 (1d10)



\pari \textbf{Behavior} Attack highest threat
\end{spelltargetinginfo}
\end{spellcontent}

\begin{monsterfooter}
\pari \textbf{Awareness} \plus6
\pari \textbf{Speed} 30 ft.;
\textbf{Space} 5 ft.;
\textbf{Reach} 5 ft.
\pari \textbf{Attributes}:
Str 4,
Dex 0,
Con 5,
Int 0,
Per 0,
Wil 0
\end{monsterfooter}
\end{monsection}


\subsubsection{Zombie Abilities}

\parhead{Slow}
The zombie does not act during the \glossterm{action phase}.
Instead, it acts during the \glossterm{delayed action phase}.

\vspace{0.5em}\parhead{Soft Flesh}
The zombie has \glossterm{damage reduction} 5 against piercing and bludgeoning damage.

\begin{monsection}{Zombie}[Hulking]{3}[2]
\vspace{-1em}\spelltwocol{}{Large undead}\vspace{-1em}
\begin{spellcontent}
\begin{spelltargetinginfo}
\spelltwocol{
\textbf{HP} 54;
\textbf{Bloodied} 27;
}
{\textbf{AP} 2/0}

\pari \textbf{Armor} 8;
\textbf{Fort} 13;
\textbf{Ref} 6;
\textbf{Ment} 10
\pari \textbf{Strike} Slam \plus4 (2d6)



\pari \textbf{Behavior} Attack highest threat
\end{spelltargetinginfo}
\end{spellcontent}

\begin{monsterfooter}
\pari \textbf{Awareness} \plus6
\pari \textbf{Speed} 40 ft.;
\textbf{Space} 10 ft.;
\textbf{Reach} 10 ft.
\pari \textbf{Attributes}:
Str 6,
Dex 0,
Con 6,
Int 0,
Per 0,
Wil 0
\end{monsterfooter}
\end{monsection}


\subsubsection{Zombie Abilities}

\parhead{Slow}
The zombie does not act during the \glossterm{action phase}.
Instead, it acts during the \glossterm{delayed action phase}.

\vspace{0.5em}\parhead{Soft Flesh}
The zombie has \glossterm{damage reduction} 6 against piercing and bludgeoning damage.

\begin{monsection}{Zombie}[Captain]{3}[3]
\vspace{-1em}\spelltwocol{}{Medium undead}\vspace{-1em}
\begin{spellcontent}
\begin{spelltargetinginfo}
\spelltwocol{
\textbf{HP} 81;
\textbf{Bloodied} 40;
}
{\textbf{AP} 3/0}

\pari \textbf{Armor} 8;
\textbf{Fort} 14;
\textbf{Ref} 9;
\textbf{Ment} 11
\pari \textbf{Strike} Slam \plus5 (2d6)


\pari \textbf{Actions} One in action phase, one in delayed action phase
\pari \textbf{Behavior} Attack highest threat
\end{spelltargetinginfo}
\end{spellcontent}

\begin{monsterfooter}
\pari \textbf{Awareness} \plus6
\pari \textbf{Speed} 30 ft.;
\textbf{Space} 5 ft.;
\textbf{Reach} 5 ft.
\pari \textbf{Attributes}:
Str 6,
Dex 0,
Con 6,
Int 0,
Per 0,
Wil 0
\end{monsterfooter}
\end{monsection}


\subsubsection{Zombie Abilities}

\parhead{Slow}
The zombie does not act during the \glossterm{action phase}.
Instead, it acts during the \glossterm{delayed action phase}.

\vspace{0.5em}\parhead{Soft Flesh}
The zombie has \glossterm{damage reduction} 6 against piercing and bludgeoning damage.

\begin{monsection}{Zombie}[Elite]{4}[1]
\vspace{-1em}\spelltwocol{}{Medium undead}\vspace{-1em}
\begin{spellcontent}
\begin{spelltargetinginfo}
\spelltwocol{
\textbf{HP} 40;
\textbf{Bloodied} 20;
}
{\textbf{AP} 1/0}

\pari \textbf{Armor} 8;
\textbf{Fort} 15;
\textbf{Ref} 8;
\textbf{Ment} 10
\pari \textbf{Strike} Slam \plus4 (2d6)



\pari \textbf{Behavior} Attack highest threat
\end{spelltargetinginfo}
\end{spellcontent}

\begin{monsterfooter}
\pari \textbf{Awareness} \plus6
\pari \textbf{Speed} 30 ft.;
\textbf{Space} 5 ft.;
\textbf{Reach} 5 ft.
\pari \textbf{Attributes}:
Str 6,
Dex 0,
Con 8,
Int 0,
Per 0,
Wil 0
\end{monsterfooter}
\end{monsection}


\subsubsection{Zombie Abilities}

\parhead{Slow}
The zombie does not act during the \glossterm{action phase}.
Instead, it acts during the \glossterm{delayed action phase}.

\vspace{0.5em}\parhead{Soft Flesh}
The zombie has \glossterm{damage reduction} 8 against piercing and bludgeoning damage.

\begin{monsection}{Unliving Mother}{2}[2]
\vspace{-1em}\spelltwocol{}{Medium undead}\vspace{-1em}
\begin{spellcontent}
\begin{spelltargetinginfo}
\spelltwocol{
\textbf{HP} 32;
\textbf{Bloodied} 16;
}
{\textbf{AP} 2/2}

\pari \textbf{Armor} 6;
\textbf{Fort} 10;
\textbf{Ref} 7;
\textbf{Ment} 12
\pari \textbf{Strike} Bite \plus3 (1d10)



\pari \textbf{Behavior} Attack highest threat
\end{spelltargetinginfo}
\end{spellcontent}

\begin{monsterfooter}
\pari \textbf{Awareness} \plus6
\pari \textbf{Speed} 30 ft.;
\textbf{Space} 5 ft.;
\textbf{Reach} 5 ft.
\pari \textbf{Attributes}:
Str 4,
Dex 0,
Con 4,
Int 0,
Per 0,
Wil 3
\end{monsterfooter}
\end{monsection}

\begin{monsection}{Unliving Mother}{2}[3]
\vspace{-1em}\spelltwocol{}{Medium undead}\vspace{-1em}
\begin{spellcontent}
\begin{spelltargetinginfo}
\spelltwocol{
\textbf{HP} 48;
\textbf{Bloodied} 24;
}
{\textbf{AP} 3/0}

\pari \textbf{Armor} 9;
\textbf{Fort} 11;
\textbf{Ref} 11;
\textbf{Ment} 10
\pari \textbf{Strike} Bite \plus6 (1d10)


\pari \textbf{Actions} One in action phase, one in delayed action phase
\pari \textbf{Behavior} Attack highest threat
\end{spelltargetinginfo}
\end{spellcontent}

\begin{monsterfooter}
\pari \textbf{Awareness} \plus6
\pari \textbf{Speed} 30 ft.;
\textbf{Space} 5 ft.;
\textbf{Reach} 5 ft.
\pari \textbf{Attributes}:
Str 4,
Dex 3,
Con 4,
Int 0,
Per 4,
Wil 0
\end{monsterfooter}
\end{monsection}

\begin{monsection}{Corrupted}[Mage]{4}[3]
\vspace{-1em}\spelltwocol{}{Medium undead}\vspace{-1em}
\begin{spellcontent}
\begin{spelltargetinginfo}
\spelltwocol{
\textbf{HP} 96;
\textbf{Bloodied} 48;
}
{\textbf{AP} 3/4}

\pari \textbf{Armor} 10;
\textbf{Fort} 13;
\textbf{Ref} 10;
\textbf{Ment} 19
\pari \textbf{Strike} Slam \plus8 (1d10)


\pari \textbf{Actions} One in action phase, one in delayed action phase
\pari \textbf{Behavior} Attack highest threat
\end{spelltargetinginfo}
\end{spellcontent}

\begin{monsterfooter}
\pari \textbf{Awareness} \plus6
\pari \textbf{Speed} 30 ft.;
\textbf{Space} 5 ft.;
\textbf{Reach} 5 ft.
\pari \textbf{Attributes}:
Str 5,
Dex 0,
Con 6,
Int 0,
Per 6,
Wil 7
\end{monsterfooter}
\end{monsection}


\subsubsection{Corrupted Abilities}

\begin{freeability}{Cone of Cold}
The mage makes a 8 vs. Fortitude attack against everything in a \areamed cone from it.
\hit Each target takes 2d10 cold damage.
\end{freeability}

\vspace{0.5em}
\begin{freeability}{Frost Bombs}
The mage creates three orbs of cold energy at locations of its choice within \rngmed range.
At the end of the next round's \glossterm{action phase},
the mage makes a 8 vs. Fortitude attack against everything in a \areasmall radius burst around each orb.
If a target is in the area of multiple orbs, it is only affected once.
\hit Each target takes 4d6 cold damage.
\end{freeability}

\vspace{0.5em}
\begin{freeability}{Telekinetic Crush}
The mage makes a 8 vs. Mental attack against one creature or object within \rngmed range of it.
\hit The target takes 4d8 bludgeoning damage.
\end{freeability}

