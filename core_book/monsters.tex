\chapter{Monsters}

Monsters are all of the various non-humanoid creatures that exist in the world of Rise.
Many of them are dangerous, and adventurers may need to fight them.
This chapter describes the rules for monsters, and the combat statistics for a variety of monsters.

\section{Monster Mechanics}
    Like player characters, monsters have levels, attributes, and abilities.
    However, they do not have classes, legacy items, or many other elements of characters.
    This section defines how monsters function.

    \subsection{Level}
        Each monster has a level that indicates its approximate strength.
        This has all of the same effects as the level for a player character, except that monsters do not gain any of the benefits described in \tref{Character Advancement}.

    \subsection{Challenge Rating}\label{Challenge Rating}
        Each monster has a \glossterm{challenge rating} that indicates its approximate strength within its level.
        A monster's challenge rating ranges from 1/2 to 4.
        This has several effects on the monster's statistics, and is a guideline to how many of that monster should typically be present in an encounter (see \pcref{Encounter Balancing}).

        \parhead{Vital Wounds} Monsters do not normally make \glossterm{wound rolls} like player characters do.
        Instead, once a monster takes at least as many \glossterm{vital wounds} as its \glossterm{challenge rating}, it is defeated.
        Defeated monsters may be unconscious, dead, or merely flee the fight at the discretion of the Game Master.

        For monsters with a CR greater than 1, each \glossterm{vital wound} imposes the normal penalties to Fortitude and Mental defenses, but the vital wounds have no additional effects.
        Some rare monsters may use \glossterm{wound roll} mechanics like player characters do.

        \parhead{Action Points} A monster has a number of \glossterm{action points} equal to its challenge rating \add 1 (rounded down).

        \parhead{Cleanse} When a monster uses the \textit{cleanse} ability, it removes a number of conditions equal to its \glossterm{challenge rating} (minimum 1).

        \parhead{Bonuses} For each challenge rating a monster has above 1, it gains a \plus1 bonus to \glossterm{accuracy} and \glossterm{defenses}.

    \subsection{Attributes}
        Each of a monster's starting attributes can range from \minus9 to 3, as appropriate for the type of monster.
        A monster's attributes scale with level in the same way as character attributes.
        A monster can also have up to two attributes starting at 4 or 5.
        In general, a monster with higher starting attributes will be slightly stronger, but not all monsters need to start with the same starting attribute total.

    \subsection{Encounter Balancing}\label{Encounter Balancing}
        In general, a group of PCs of a given level will have an appropriate challenge from fighting monsters of the same level with a combined challenge rating equal to the number of PCs.
        Fighting monsters of a lower level, or monsters whose combined challenge rating is less than the number of PCs, will yield an easier encounter.
        Fighting monsters of a higher level, or fighting monsters whose combined challenge rating is greater than the number of PCs, will yield an easier encounter.

        It is generally not a good idea for PCs to fight monsters more than three levels higher or lower than their own.
        They may find that their attacks never miss, or always hit, and other aspects of the encounter may be similarly imbalanced in ways that are not easily remedied by simply changing the number of enemies.


        \section{Aberrations}

        All aberrations have the following properties unless noted otherwise in their description:
        
    
    \parhead{Defenses}
    \plus2 Armor,
    \plus3 Fortitude,
    \plus2 Reflex,
    \plus4 Mental
  
  
      
  \begin{monsection}{Aboleth}{12}[4]
    \vspace{-1em}\spelltwocol{}{Huge aberration}\vspace{-1em}
    \vspace{0em}

    
    The aboleth is a revolting fishlike amphibian found primarily in subterranean lakes and rivers.
    It has a pink belly.
    Four pulsating dark blue orifices line the bottom of its body and secrete gray slime that smells like rancid grease.
    It uses its tail for propulsion in the water and drags itself along with its tentacles on land.
  
    

    \begin{spellcontent}
      \begin{spelltargetinginfo}
        \pari \textbf{HP} 52 \monsep
          \textbf{AD} 20 \monsep
          \textbf{Fort} 22 \monsep
          \textbf{Ref} 16 \monsep
          \textbf{Ment} 23
        \pari \textbf{WR} Physical 49, Energy 46 \monsep
        \textbf{VR} Physical 102, Energy 99
        \pari \textbf{Strike:}
            Tentacle \plus15 (11d10 bludgeoning)
      \end{spelltargetinginfo}
    \end{spellcontent}
    \begin{monsterfooter}
      \pari \textbf{Speeds} Land 50 ft., Swim 50 ft. \monsep
        \textbf{Space} 15 ft. \monsep
        \textbf{Reach} 15 ft.
      \pari \textbf{Awareness} \plus3
      \pari \textbf{Attributes}
        Str 15, Dex -1,
        Con 15, Int 14,
        Per 7, Wil 15
      \pari \textbf{Accuracy} 15 \monsep
        \textbf{Power} 28
      \pari \textbf{Alignment} Usually lawful evil
    \end{monsterfooter}
  \end{monsection}
  \begin{freeability}{Mind Control}
      \target{One creature within \rnglong range}
      An aboleth can use this ability as a \glossterm{minor action}. The aboleth makes a \plus15 attack
        vs. Mental against the target.
    
    \hit The target is \glossterm{confused} as a \glossterm{condition}.
    \crit 
        The aboleth can spend an action point to attune to this ability.
        If it does, the target is dominated by the aboleth as long as the ability lasts.
        Otherwise, the target takes double the damage of a non-critical hit.
    \end{freeability}
  

    \begin{freeability}{Mind Crush}
      \target{One creature within \rnglong range}
      An aboleth can use this ability as a \glossterm{minor action}. The aboleth makes a \plus15 attack
        vs. Mental against the target.
    
    \hit The target takes 12d10 energy damage.
    \end{freeability}
  

    \begin{freeability}{Psionic Blast}
      \targets{Each enemy in a \arealarge cone from the aboleth}
      An aboleth can use this ability as a \glossterm{minor action}. The aboleth makes a \plus15 attack
        vs. Mental against each target.
    
    \hit Each target takes 10d10 energy damage. Any target \glossterm{wounded} by this damage is \glossterm{dazed} as a \glossterm{condition}.
    \end{freeability}
  
      \parhead{Psionic Barrier} 
        The aboleth is \glossterm{attuned} to this ability.
        It gains a \plus14 \glossterm{magic bonus} to \glossterm{resistances}.
      
    \parhead{Slime} 
        Whenever a creature is \glossterm{wounded} by the aboleth's tentacle,
          the damaged creature becomes \glossterm{poisoned} by aboleth slime.
        The creature is \glossterm{sickened} as long as it remains poisoned.
        On a third hit, the creature gains a \glossterm{vital wound}.

        Instead of making a \glossterm{vital roll} for the \glossterm{vital wound},
          the target's skin is transformed into a clear, slimy membrane.
        Every 5 minutes, an afflicted creature must be moistened with cool, fresh water
          or it will lose a \glossterm{hit point}.
        This effect lasts until the \glossterm{vital wound} is removed.
      
    \par \textbf{Telepathy 1 000 Ft}
  
  \begin{monsection}{Umber Hulk}{12}[2]
    \vspace{-1em}\spelltwocol{}{Gargantuan aberration}\vspace{-1em}
    \vspace{0em}

    
    An umber hulk is a massive insectoid creature with bulging compound eyes that it uses to
    confuse nearby creatures. Umber hulks are simple-minded and easily manipulated.
  
    

    \begin{spellcontent}
      \begin{spelltargetinginfo}
        \pari \textbf{HP} 12 \monsep
          \textbf{AD} 19 \monsep
          \textbf{Fort} 19 \monsep
          \textbf{Ref} 15 \monsep
          \textbf{Ment} 14
        \pari \textbf{WR} Physical 32, Energy 29 \monsep
        \textbf{VR} Physical 81, Energy 78
        
      \end{spelltargetinginfo}
    \end{spellcontent}
    \begin{monsterfooter}
      \pari \textbf{Speed} 60 ft. \monsep
        \textbf{Space} 20 ft. \monsep
        \textbf{Reach} 20 ft.
      \pari \textbf{Awareness} \plus0
      \pari \textbf{Attributes}
        Str 15, Dex 0,
        Con 14, Int -3,
        Per 0, Wil -3
      \pari \textbf{Accuracy} 13 \monsep
        \textbf{Mundane Power} 20 \monsep
      \textbf{Magical Power} 13
      \pari \textbf{Alignment} Usually chaotic evil
    \end{monsterfooter}
  \end{monsection}
  \begin{freeability}{Confusing Gaze}
      \targets{Each creature within a \areamed cone}
       The umber hulk makes a \plus13 attack
        vs. Mental against each target.
    
    \hit Each target is \glossterm{dazed} as a \glossterm{condition}.
    \crit Each target is \glossterm{confused} as a \glossterm{condition}.
    \end{freeability}
  
        \section{Animals}

        All animals have the following properties unless noted otherwise in their description:
        
    
    \parhead{Defenses}
    \plus2 Armor,
    \plus4 Fortitude,
    \plus3 Reflex,
    \plus2 Mental
  
  
      
  \begin{monsection}{Black Bear}{3}[2]
    \vspace{-1em}\spelltwocol{}{Medium animal}\vspace{-1em}
    \vspace{0em}

    
      The black bear is a forest-dwelling omnivore that usually is not dangerous unless an interloper threatens its cubs or food supply.

      Black bears can be pure black, blond, or cinnamon in color and are rarely more than 5 feet long.
    
    

    \begin{spellcontent}
      \begin{spelltargetinginfo}
        \pari \textbf{HP} 12 \monsep
          \textbf{AD} 9 \monsep
          \textbf{Fort} 11 \monsep
          \textbf{Ref} 7 \monsep
          \textbf{Ment} 5
        \pari \textbf{WR} 9 \monsep
        \textbf{VR} 26
        \pari \textbf{Strike:}
            Bite \plus4 (2d10 bludgeoning and piercing)
      \end{spelltargetinginfo}
    \end{spellcontent}
    \begin{monsterfooter}
      \pari \textbf{Speed} 30 ft. \monsep
        \textbf{Space} 5 ft. \monsep
        \textbf{Reach} 5 ft.
      \pari \textbf{Awareness} \plus0
      \pari \textbf{Attributes}
        Str 5, Dex 0,
        Con 5, Int -8,
        Per 0, Wil -1
      \pari \textbf{Accuracy} 4 \monsep
        \textbf{Mundane Power} 6 \monsep
      \textbf{Magical Power} 2
      \pari \textbf{Alignment} Always true neutral
    \end{monsterfooter}
  \end{monsection}
  
  
  \begin{monsection}{Brown Bear}{5}[2]
    \vspace{-1em}\spelltwocol{}{Large animal}\vspace{-1em}
    \vspace{0em}

    
      These massive omnivores are bad-tempered and territorial.
      The brown bear's statistics can be used for almost any big bear, including the grizzly.
    
    

    \begin{spellcontent}
      \begin{spelltargetinginfo}
        \pari \textbf{HP} 13 \monsep
          \textbf{AD} 11 \monsep
          \textbf{Fort} 14 \monsep
          \textbf{Ref} 9 \monsep
          \textbf{Ment} 7
        \pari \textbf{WR} 13 \monsep
        \textbf{VR} 40
        \pari \textbf{Strike:}
            Bite \plus6 (4d8 bludgeoning and piercing)
      \end{spelltargetinginfo}
    \end{spellcontent}
    \begin{monsterfooter}
      \pari \textbf{Speed} 40 ft. \monsep
        \textbf{Space} 10 ft. \monsep
        \textbf{Reach} 10 ft.
      \pari \textbf{Awareness} \plus0
      \pari \textbf{Attributes}
        Str 8, Dex 0,
        Con 8, Int -8,
        Per 0, Wil -1
      \pari \textbf{Accuracy} 6 \monsep
        \textbf{Mundane Power} 11 \monsep
      \textbf{Magical Power} 6
      \pari \textbf{Alignment} Always true neutral
    \end{monsterfooter}
  \end{monsection}
  
  
  \begin{monsection}{Colossal Centipede}{13}[4]
    \vspace{-1em}\spelltwocol{}{Colossal animal}\vspace{-1em}
    \vspace{0em}

    
    Monstrous centipedes tend to attack anything that resembles food, biting with their jaws and injecting their poison.
  
    

    \begin{spellcontent}
      \begin{spelltargetinginfo}
        \pari \textbf{HP} 56 \monsep
          \textbf{AD} 21 \monsep
          \textbf{Fort} 25 \monsep
          \textbf{Ref} 18 \monsep
          \textbf{Ment} 18
        \pari \textbf{WR} Physical 42, Energy 39 \monsep
        \textbf{VR} Physical 106, Energy 103
        \pari \textbf{Strike:}
            Bite \plus16 (13d10 bludgeoning and piercing)
      \end{spelltargetinginfo}
    \end{spellcontent}
    \begin{monsterfooter}
      \pari \textbf{Speed} 70 ft. \monsep
        \textbf{Space} 30 ft. \monsep
        \textbf{Reach} 30 ft.
      \pari \textbf{Awareness} \plus0
      \pari \textbf{Attributes}
        Str 17, Dex -1,
        Con 17, Int -9,
        Per 0, Wil 0
      \pari \textbf{Accuracy} 16 \monsep
        \textbf{Mundane Power} 30 \monsep
      \textbf{Magical Power} 25
      \pari \textbf{Alignment} Always true neutral
    \end{monsterfooter}
  \end{monsection}
  \parhead{Poison Sting} 
    Whenever a creature is \glossterm{wounded} by the colossal centipede's bite,
      the damaged creature becomes \glossterm{poisoned}.
    Its initial effect makes the target lose a \glossterm{hit point}.
    On the poison's third hit, the target is lose two \glossterm{hit points}.
  
  \begin{monsection}{Dire Wolf}{5}[2]
    \vspace{-1em}\spelltwocol{}{Large animal}\vspace{-1em}
    \vspace{0em}

    
      Dire wolves are efficient pack hunters that will kill anything they can catch.
      Their fur is usually mottled gray or black.
    
    

    \begin{spellcontent}
      \begin{spelltargetinginfo}
        \pari \textbf{HP} 11 \monsep
          \textbf{AD} 14 \monsep
          \textbf{Fort} 12 \monsep
          \textbf{Ref} 12 \monsep
          \textbf{Ment} 8
        \pari \textbf{WR} 10 \monsep
        \textbf{VR} 30
        \pari \textbf{Strike:}
            Bite \plus7 (4d8 bludgeoning and piercing)
      \end{spelltargetinginfo}
    \end{spellcontent}
    \begin{monsterfooter}
      \pari \textbf{Speed} 40 ft. \monsep
        \textbf{Space} 10 ft. \monsep
        \textbf{Reach} 10 ft.
      \pari \textbf{Awareness} \plus3
      \pari \textbf{Attributes}
        Str 8, Dex 7,
        Con 6, Int -7,
        Per 6, Wil 0
      \pari \textbf{Accuracy} 7 \monsep
        \textbf{Mundane Power} 11 \monsep
      \textbf{Magical Power} 7
      \pari \textbf{Alignment} Always true neutral
    \end{monsterfooter}
  \end{monsection}
  
  
  \begin{monsection}{Gargantuan Centipede}{10}[4]
    \vspace{-1em}\spelltwocol{}{Gargantuan animal}\vspace{-1em}
    \vspace{0em}

    
    Monstrous centipedes tend to attack anything that resembles food, biting with their jaws and injecting their poison.
  
    

    \begin{spellcontent}
      \begin{spelltargetinginfo}
        \pari \textbf{HP} 52 \monsep
          \textbf{AD} 18 \monsep
          \textbf{Fort} 21 \monsep
          \textbf{Ref} 15 \monsep
          \textbf{Ment} 15
        \pari \textbf{WR} Physical 29, Energy 26 \monsep
        \textbf{VR} Physical 74, Energy 71
        \pari \textbf{Strike:}
            Bite \plus13 (8d10 bludgeoning and piercing)
      \end{spelltargetinginfo}
    \end{spellcontent}
    \begin{monsterfooter}
      \pari \textbf{Speed} 60 ft. \monsep
        \textbf{Space} 20 ft. \monsep
        \textbf{Reach} 20 ft.
      \pari \textbf{Awareness} \plus0
      \pari \textbf{Attributes}
        Str 13, Dex -1,
        Con 13, Int -9,
        Per 0, Wil 0
      \pari \textbf{Accuracy} 13 \monsep
        \textbf{Mundane Power} 20 \monsep
      \textbf{Magical Power} 16
      \pari \textbf{Alignment} Always true neutral
    \end{monsterfooter}
  \end{monsection}
  \parhead{Poison Sting} 
    Whenever a creature is \glossterm{wounded} by the gargantuan centipede's bite,
      the damaged creature becomes \glossterm{poisoned}.
    Its initial effect makes the target lose a \glossterm{hit point}.
    On the poison's third hit, the target is lose two \glossterm{hit points}.
  
  \begin{monsection}{Giant Bombardier Beetle}{7}[2]
    \vspace{-1em}\spelltwocol{}{Large animal}\vspace{-1em}
    \vspace{0em}

    
      These creatures feed primarily on carrion and offal, gathering heaps of the stuff in which to build nests and lay eggs.
      A giant bombardier beetle is about 6 feet long. Giant bombardier beetles normally attack only to defend themselves, their nests, or their eggs.
    
    

    \begin{spellcontent}
      \begin{spelltargetinginfo}
        \pari \textbf{HP} 13 \monsep
          \textbf{AD} 13 \monsep
          \textbf{Fort} 16 \monsep
          \textbf{Ref} 10 \monsep
          \textbf{Ment} 10
        \pari \textbf{WR} Physical 20, Energy 17 \monsep
        \textbf{VR} Physical 55, Energy 52
        \pari \textbf{Strike:}
            Bite \plus8 (4d10 bludgeoning and piercing)
      \end{spelltargetinginfo}
    \end{spellcontent}
    \begin{monsterfooter}
      \pari \textbf{Speed} 40 ft. \monsep
        \textbf{Space} 10 ft. \monsep
        \textbf{Reach} 10 ft.
      \pari \textbf{Awareness} \plus0
      \pari \textbf{Attributes}
        Str 10, Dex -1,
        Con 10, Int -9,
        Per 0, Wil 0
      \pari \textbf{Accuracy} 8 \monsep
        \textbf{Mundane Power} 13 \monsep
      \textbf{Magical Power} 9
      \pari \textbf{Alignment} Always true neutral
    \end{monsterfooter}
  \end{monsection}
  \begin{freeability}{Acid Breath}
      \targets{Everything in a \areamed cone}
       The giant bombardier beetle makes a \plus8 attack
        vs. Armor against each target.
    
    \hit Each target takes 4d8 acid damage.
    \end{freeability}
  
  \begin{monsection}{Giant Wasp}{6}[2]
    \vspace{-1em}\spelltwocol{}{Large animal}\vspace{-1em}
    \vspace{0em}

    
      Giant wasps attack when hungry or threatened, stinging their prey to death.
      They take dead or incapacitated opponents back to their lairs as food for their unhatched young.
    
    

    \begin{spellcontent}
      \begin{spelltargetinginfo}
        \pari \textbf{HP} 10 \monsep
          \textbf{AD} 17 \monsep
          \textbf{Fort} 12 \monsep
          \textbf{Ref} 14 \monsep
          \textbf{Ment} 7
        \pari \textbf{WR} Physical 13, Energy 10 \monsep
        \textbf{VR} Physical 33, Energy 30
        \pari \textbf{Strike:}
            Stinger \plus8 (4d6 piercing)
      \end{spelltargetinginfo}
    \end{spellcontent}
    \begin{monsterfooter}
      \pari \textbf{Speed} 40 ft. \monsep
        \textbf{Space} 10 ft. \monsep
        \textbf{Reach} 10 ft.
      \pari \textbf{Awareness} \plus3
      \pari \textbf{Attributes}
        Str 4, Dex 9,
        Con 4, Int -8,
        Per 7, Wil -2
      \pari \textbf{Accuracy} 8 \monsep
        \textbf{Mundane Power} 9 \monsep
      \textbf{Magical Power} 6
      \pari \textbf{Alignment} Always true neutral
    \end{monsterfooter}
  \end{monsection}
  \parhead{Poison Sting} 
    Whenever a creature is \glossterm{wounded} by the giant wasp's stinger,
      the damaged creature becomes \glossterm{poisoned}.
    Its initial effect makes the target \glossterm{sickened}.
    On the poison's third hit, the target is \glossterm{paralyzed}.
  
  \begin{monsection}{Huge Centipede}{7}[4]
    \vspace{-1em}\spelltwocol{}{Huge animal}\vspace{-1em}
    \vspace{0em}

    
    Monstrous centipedes tend to attack anything that resembles food, biting with their jaws and injecting their poison.
  
    

    \begin{spellcontent}
      \begin{spelltargetinginfo}
        \pari \textbf{HP} 48 \monsep
          \textbf{AD} 15 \monsep
          \textbf{Fort} 17 \monsep
          \textbf{Ref} 12 \monsep
          \textbf{Ment} 12
        \pari \textbf{WR} Physical 18, Energy 15 \monsep
        \textbf{VR} Physical 49, Energy 46
        \pari \textbf{Strike:}
            Bite \plus10 (6d10 bludgeoning and piercing)
      \end{spelltargetinginfo}
    \end{spellcontent}
    \begin{monsterfooter}
      \pari \textbf{Speed} 50 ft. \monsep
        \textbf{Space} 15 ft. \monsep
        \textbf{Reach} 15 ft.
      \pari \textbf{Awareness} \plus0
      \pari \textbf{Attributes}
        Str 9, Dex -1,
        Con 9, Int -9,
        Per 0, Wil 0
      \pari \textbf{Accuracy} 10 \monsep
        \textbf{Mundane Power} 16 \monsep
      \textbf{Magical Power} 13
      \pari \textbf{Alignment} Always true neutral
    \end{monsterfooter}
  \end{monsection}
  \parhead{Poison Sting} 
    Whenever a creature is \glossterm{wounded} by the huge centipede's bite,
      the damaged creature becomes \glossterm{poisoned}.
    Its initial effect makes the target lose a \glossterm{hit point}.
    On the poison's third hit, the target is lose two \glossterm{hit points}.
  
  \begin{monsection}{Large Centipede}{4}[4]
    \vspace{-1em}\spelltwocol{}{Large animal}\vspace{-1em}
    \vspace{0em}

    
    Monstrous centipedes tend to attack anything that resembles food, biting with their jaws and injecting their poison.
  
    

    \begin{spellcontent}
      \begin{spelltargetinginfo}
        \pari \textbf{HP} 44 \monsep
          \textbf{AD} 12 \monsep
          \textbf{Fort} 13 \monsep
          \textbf{Ref} 9 \monsep
          \textbf{Ment} 9
        \pari \textbf{WR} Physical 12, Energy 9 \monsep
        \textbf{VR} Physical 29, Energy 26
        \pari \textbf{Strike:}
            Bite \plus7 (2d10 bludgeoning and piercing)
      \end{spelltargetinginfo}
    \end{spellcontent}
    \begin{monsterfooter}
      \pari \textbf{Speed} 40 ft. \monsep
        \textbf{Space} 10 ft. \monsep
        \textbf{Reach} 10 ft.
      \pari \textbf{Awareness} \plus0
      \pari \textbf{Attributes}
        Str 5, Dex -1,
        Con 5, Int -9,
        Per 0, Wil 0
      \pari \textbf{Accuracy} 7 \monsep
        \textbf{Mundane Power} 6 \monsep
      \textbf{Magical Power} 4
      \pari \textbf{Alignment} Always true neutral
    \end{monsterfooter}
  \end{monsection}
  \parhead{Poison Sting} 
    Whenever a creature is \glossterm{wounded} by the large centipede's bite,
      the damaged creature becomes \glossterm{poisoned}.
    Its initial effect makes the target lose a \glossterm{hit point}.
    On the poison's third hit, the target is lose two \glossterm{hit points}.
  
  \begin{monsection}{Pony}{2}[1]
    \vspace{-1em}\spelltwocol{}{Medium animal}\vspace{-1em}
    \vspace{0em}

    
      The statistics presented here describe a small horse, under 5 feet tall at the shoulder.
      Ponies are similar to light horses and cannot fight while carrying a rider.
      A pony's maximum load is 120 pounds, and it can drag up to 800 pounds.
    
    

    \begin{spellcontent}
      \begin{spelltargetinginfo}
        \pari \textbf{HP} 6 \monsep
          \textbf{AD} 7 \monsep
          \textbf{Fort} 9 \monsep
          \textbf{Ref} 5 \monsep
          \textbf{Ment} 3
        \pari \textbf{WR} 7 \monsep
        \textbf{VR} 21
        \pari \textbf{Strike:}
            Bite \plus2 (2d6 bludgeoning and piercing)
      \end{spelltargetinginfo}
    \end{spellcontent}
    \begin{monsterfooter}
      \pari \textbf{Speed} 30 ft. \monsep
        \textbf{Space} 5 ft. \monsep
        \textbf{Reach} 5 ft.
      \pari \textbf{Awareness} \plus0
      \pari \textbf{Attributes}
        Str 2, Dex 0,
        Con 4, Int -7,
        Per 0, Wil -1
      \pari \textbf{Accuracy} 2 \monsep
        \textbf{Mundane Power} 3 \monsep
      \textbf{Magical Power} 1
      \pari \textbf{Alignment} Always true neutral
    \end{monsterfooter}
  \end{monsection}
  
  
  \begin{monsection}{Roc}{9}[4]
    \vspace{-1em}\spelltwocol{}{Gargantuan animal}\vspace{-1em}
    \vspace{0em}

    
      Rocs are massive and incredibly strong birds with the ability to carry off horses.
      It is typically 30 feet long from the beak to the base of the tail, with a wingspan as wide as 80 feet.
      Its plumage is either dark brown or golden from head to tail.

      A roc attacks from the air, swooping earthward to snatch prey in its powerful talons and carry it off for itself and its young to devour.
      A solitary roc is typically hunting and will attack any Medium or larger creature that appears edible.
      A mated pair of rocs attack in concert, fighting to the death to defend their nests or hatchlings.
    
    

    \begin{spellcontent}
      \begin{spelltargetinginfo}
        \pari \textbf{HP} 48 \monsep
          \textbf{AD} 18 \monsep
          \textbf{Fort} 19 \monsep
          \textbf{Ref} 17 \monsep
          \textbf{Ment} 13
        \pari \textbf{WR} 18 \monsep
        \textbf{VR} 56
        \pari \textbf{Strike:}
            Bite \plus12 (7d10 bludgeoning and piercing)
      \end{spelltargetinginfo}
    \end{spellcontent}
    \begin{monsterfooter}
      \pari \textbf{Speed} 60 ft. \monsep
        \textbf{Space} 20 ft. \monsep
        \textbf{Reach} 20 ft.
      \pari \textbf{Awareness} \plus0
      \pari \textbf{Attributes}
        Str 12, Dex 10,
        Con 11, Int -7,
        Per 0, Wil -1
      \pari \textbf{Accuracy} 12 \monsep
        \textbf{Mundane Power} 19 \monsep
      \textbf{Magical Power} 14
      \pari \textbf{Alignment} Always true neutral
    \end{monsterfooter}
  \end{monsection}
  
  
  \begin{monsection}{Spider}[Colossal]{12}[4]
    \vspace{-1em}\spelltwocol{}{Colossal animal}\vspace{-1em}
    \vspace{0em}

    
    All monstrous spiders are aggressive predators that use their poisonous bites to subdue or kill prey.
  
    

    \begin{spellcontent}
      \begin{spelltargetinginfo}
        \pari \textbf{HP} 36 \monsep
          \textbf{AD} 25 \monsep
          \textbf{Fort} 19 \monsep
          \textbf{Ref} 22 \monsep
          \textbf{Ment} 17
        \pari \textbf{WR} Physical 26, Energy 23 \monsep
        \textbf{VR} Physical 67, Energy 64
        \pari \textbf{Strike:}
            Bite \plus18 (12d10 bludgeoning and piercing)
      \end{spelltargetinginfo}
    \end{spellcontent}
    \begin{monsterfooter}
      \pari \textbf{Speed} 70 ft. \monsep
        \textbf{Space} 30 ft. \monsep
        \textbf{Reach} 30 ft.
      \pari \textbf{Awareness} \plus7
      \pari \textbf{Attributes}
        Str 15, Dex 15,
        Con 0, Int -9,
        Per 15, Wil 0
      \pari \textbf{Accuracy} 18 \monsep
        \textbf{Mundane Power} 28 \monsep
      \textbf{Magical Power} 24
      \pari \textbf{Alignment} Always true neutral
    \end{monsterfooter}
  \end{monsection}
  \parhead{Poison Sting} 
    Whenever a creature is \glossterm{wounded} by the colossal spider's bite,
      the damaged creature becomes \glossterm{poisoned}.
    Its initial effect makes the target \glossterm{sickened}.
    On the poison's third hit, the target is \glossterm{paralyzed}.
  
  \begin{monsection}{Spider}[Gargantuan]{9}[4]
    \vspace{-1em}\spelltwocol{}{Gargantuan animal}\vspace{-1em}
    \vspace{0em}

    
    All monstrous spiders are aggressive predators that use their poisonous bites to subdue or kill prey.
  
    

    \begin{spellcontent}
      \begin{spelltargetinginfo}
        \pari \textbf{HP} 36 \monsep
          \textbf{AD} 22 \monsep
          \textbf{Fort} 16 \monsep
          \textbf{Ref} 19 \monsep
          \textbf{Ment} 14
        \pari \textbf{WR} Physical 18, Energy 15 \monsep
        \textbf{VR} Physical 49, Energy 46
        \pari \textbf{Strike:}
            Bite \plus14 (7d10 bludgeoning and piercing)
      \end{spelltargetinginfo}
    \end{spellcontent}
    \begin{monsterfooter}
      \pari \textbf{Speed} 60 ft. \monsep
        \textbf{Space} 20 ft. \monsep
        \textbf{Reach} 20 ft.
      \pari \textbf{Awareness} \plus5
      \pari \textbf{Attributes}
        Str 11, Dex 12,
        Con 0, Int -9,
        Per 11, Wil 0
      \pari \textbf{Accuracy} 14 \monsep
        \textbf{Mundane Power} 18 \monsep
      \textbf{Magical Power} 15
      \pari \textbf{Alignment} Always true neutral
    \end{monsterfooter}
  \end{monsection}
  \parhead{Poison Sting} 
    Whenever a creature is \glossterm{wounded} by the gargantuan spider's bite,
      the damaged creature becomes \glossterm{poisoned}.
    Its initial effect makes the target \glossterm{sickened}.
    On the poison's third hit, the target is \glossterm{paralyzed}.
  
  \begin{monsection}{Spider}[Huge]{6}[4]
    \vspace{-1em}\spelltwocol{}{Huge animal}\vspace{-1em}
    \vspace{0em}

    
    All monstrous spiders are aggressive predators that use their poisonous bites to subdue or kill prey.
  
    

    \begin{spellcontent}
      \begin{spelltargetinginfo}
        \pari \textbf{HP} 36 \monsep
          \textbf{AD} 18 \monsep
          \textbf{Fort} 13 \monsep
          \textbf{Ref} 15 \monsep
          \textbf{Ment} 11
        \pari \textbf{WR} Physical 13, Energy 10 \monsep
        \textbf{VR} Physical 33, Energy 30
        \pari \textbf{Strike:}
            Bite \plus11 (5d10 bludgeoning and piercing)
      \end{spelltargetinginfo}
    \end{spellcontent}
    \begin{monsterfooter}
      \pari \textbf{Speed} 50 ft. \monsep
        \textbf{Space} 15 ft. \monsep
        \textbf{Reach} 15 ft.
      \pari \textbf{Awareness} \plus4
      \pari \textbf{Attributes}
        Str 7, Dex 8,
        Con 0, Int -9,
        Per 8, Wil 0
      \pari \textbf{Accuracy} 11 \monsep
        \textbf{Mundane Power} 14 \monsep
      \textbf{Magical Power} 12
      \pari \textbf{Alignment} Always true neutral
    \end{monsterfooter}
  \end{monsection}
  \parhead{Poison Sting} 
    Whenever a creature is \glossterm{wounded} by the huge spider's bite,
      the damaged creature becomes \glossterm{poisoned}.
    Its initial effect makes the target \glossterm{sickened}.
    On the poison's third hit, the target is \glossterm{paralyzed}.
  
  \begin{monsection}{Spider}[Large]{3}[4]
    \vspace{-1em}\spelltwocol{}{Large animal}\vspace{-1em}
    \vspace{0em}

    
    All monstrous spiders are aggressive predators that use their poisonous bites to subdue or kill prey.
  
    

    \begin{spellcontent}
      \begin{spelltargetinginfo}
        \pari \textbf{HP} 36 \monsep
          \textbf{AD} 15 \monsep
          \textbf{Fort} 10 \monsep
          \textbf{Ref} 12 \monsep
          \textbf{Ment} 8
        \pari \textbf{WR} Physical 10, Energy 7 \monsep
        \textbf{VR} Physical 23, Energy 20
        \pari \textbf{Strike:}
            Bite \plus7 (2d8 bludgeoning and piercing)
      \end{spelltargetinginfo}
    \end{spellcontent}
    \begin{monsterfooter}
      \pari \textbf{Speed} 40 ft. \monsep
        \textbf{Space} 10 ft. \monsep
        \textbf{Reach} 10 ft.
      \pari \textbf{Awareness} \plus2
      \pari \textbf{Attributes}
        Str 2, Dex 5,
        Con 0, Int -9,
        Per 4, Wil 0
      \pari \textbf{Accuracy} 7 \monsep
        \textbf{Mundane Power} 4 \monsep
      \textbf{Magical Power} 3
      \pari \textbf{Alignment} Always true neutral
    \end{monsterfooter}
  \end{monsection}
  \parhead{Poison Sting} 
    Whenever a creature is \glossterm{wounded} by the large spider's bite,
      the damaged creature becomes \glossterm{poisoned}.
    Its initial effect makes the target \glossterm{sickened}.
    On the poison's third hit, the target is \glossterm{paralyzed}.
  
  \begin{monsection}{Vampire Eel}{6}[2]
    \vspace{-1em}\spelltwocol{}{Large animal}\vspace{-1em}
    \vspace{0em}

    
      Vampire eels are large, slimy snakelike carnivores.
      They swim through murky water, looking for edible creatures.
    
    

    \begin{spellcontent}
      \begin{spelltargetinginfo}
        \pari \textbf{HP} 11 \monsep
          \textbf{AD} 15 \monsep
          \textbf{Fort} 13 \monsep
          \textbf{Ref} 13 \monsep
          \textbf{Ment} 8
        \pari \textbf{WR} Physical 12, Energy 10 \monsep
        \textbf{VR} Physical 36, Energy 34
        
      \end{spelltargetinginfo}
    \end{spellcontent}
    \begin{monsterfooter}
      \pari \textbf{Speed} 40 ft. \monsep
        \textbf{Space} 10 ft. \monsep
        \textbf{Reach} 10 ft.
      \pari \textbf{Awareness} \plus0
      \pari \textbf{Attributes}
        Str 8, Dex 8,
        Con 7, Int -8,
        Per 0, Wil -1
      \pari \textbf{Accuracy} 7 \monsep
        \textbf{Mundane Power} 11 \monsep
      \textbf{Magical Power} 7
      \pari \textbf{Alignment} Always true neutral
    \end{monsterfooter}
  \end{monsection}
  
  
  \begin{monsection}{Wolf}{2}[1]
    \vspace{-1em}\spelltwocol{}{Medium animal}\vspace{-1em}
    \vspace{0em}

    
      Wolves are pack hunters known for their persistence and cunning.
    
    

    \begin{spellcontent}
      \begin{spelltargetinginfo}
        \pari \textbf{HP} 5 \monsep
          \textbf{AD} 9 \monsep
          \textbf{Fort} 7 \monsep
          \textbf{Ref} 7 \monsep
          \textbf{Ment} 3
        \pari \textbf{WR} 6 \monsep
        \textbf{VR} 18
        \pari \textbf{Strike:}
            Bite \plus2 (2d6 bludgeoning and piercing)
      \end{spelltargetinginfo}
    \end{spellcontent}
    \begin{monsterfooter}
      \pari \textbf{Speed} 30 ft. \monsep
        \textbf{Space} 5 ft. \monsep
        \textbf{Reach} 5 ft.
      \pari \textbf{Awareness} \plus0
      \pari \textbf{Attributes}
        Str 2, Dex 3,
        Con 2, Int -7,
        Per 0, Wil -1
      \pari \textbf{Accuracy} 2 \monsep
        \textbf{Mundane Power} 3 \monsep
      \textbf{Magical Power} 1
      \pari \textbf{Alignment} Always true neutral
    \end{monsterfooter}
  \end{monsection}
  
  
  \begin{monsection}{Raven}{1}[1]
    \vspace{-1em}\spelltwocol{}{Tiny animal}\vspace{-1em}
    \vspace{0em}

    
      These glossy black birds are about 2 feet long and have wingspans of about 4 feet.
      The statistics presented here can describe most nonpredatory birds of similar size.
    
    

    \begin{spellcontent}
      \begin{spelltargetinginfo}
        \pari \textbf{HP} 2 \monsep
          \textbf{AD} 8 \monsep
          \textbf{Fort} 1 \monsep
          \textbf{Ref} 7 \monsep
          \textbf{Ment} 3
        \pari \textbf{WR} 3 \monsep
        \textbf{VR} 14
        \pari \textbf{Strike:}
            Talon \plus4 (1d8 piercing and slashing)
      \end{spelltargetinginfo}
    \end{spellcontent}
    \begin{monsterfooter}
      \pari \textbf{Speed} 20 ft. \monsep
        \textbf{Space} 2-1/2 ft. \monsep
        \textbf{Reach} 0 ft.
      \pari \textbf{Awareness} \plus1
      \pari \textbf{Attributes}
        Str -8, Dex 3,
        Con -4, Int -6,
        Per 2, Wil 0
      \pari \textbf{Accuracy} 2 \monsep
        \textbf{Mundane Power} -7 \monsep
      \textbf{Magical Power} 1
      \pari \textbf{Alignment} Always true neutral
    \end{monsterfooter}
  \end{monsection}
  
  
        \section{Animates}

        All animates have the following properties unless noted otherwise in their description:
        
    
    \parhead{Defenses}
    \plus2 Armor,
    \plus3 Fortitude,
    \plus3 Reflex,
    \plus3 Mental
  
    \parhead{Animated Life} Animates are living creatures that are fundamentally composed of inanimate matter.
    They are considered to be both objects and creatures, and are affected equally by abilities that affect both.
    \parhead{Nonsentient} Animates may have an intelligence of a sort, depending on the nature of their animation, but they are fundamentally not sentient creatures.
    All animates are immune to \glossterm{Compulsion} and \glossterm{Delusion} abilities.
  
      
    \subsection{Air Elementals}
      
    Air elementals are the embodiment of the natural element of air.
    Their rapid flying speed makes them useful on vast battlefields or in extended aerial combat.
  

      

      
  \begin{monsubsection}{Air Elemental}[Large]{5}[2]
    \vspace{-1em}\spelltwocol{}{Large animate}\vspace{-1em}
    \vspace{0em}

    
    

    \begin{spellcontent}
      \begin{spelltargetinginfo}
        \pari \textbf{HP} 10 \monsep
          \textbf{AD} 15 \monsep
          \textbf{Fort} 10 \monsep
          \textbf{Ref} 13 \monsep
          \textbf{Ment} 9
        \pari \textbf{WR} 8 \monsep
        \textbf{VR} 25
        \pari \textbf{Strike:}
            Slam \plus7 (4d8 bludgeoning)
      \end{spelltargetinginfo}
    \end{spellcontent}
    \begin{monsterfooter}
      \pari \textbf{Speed} 40 ft. \monsep
        \textbf{Space} 10 ft. \monsep
        \textbf{Reach} 10 ft.
      \pari \textbf{Awareness} \plus3
      \pari \textbf{Attributes}
        Str 3, Dex 8,
        Con 3, Int -2,
        Per 6, Wil 0
      \pari \textbf{Accuracy} 7 \monsep
        \textbf{Mundane Power} 8 \monsep
      \textbf{Magical Power} 7
      \pari \textbf{Alignment} Usually true neutral
    \end{monsterfooter}
  \end{monsubsection}
  \begin{freeability}{Whirlwind}
      \targets{Each \glossterm{enemy} within reach}
       The air elemental makes a \plus7
        \glossterm{strike} vs. Armor
        with its slam against each target.
    
    \hit Each target takes 4d8 bludgeoning damage.
    \end{freeability}
  ,
  \begin{monsubsection}{Air Elemental}[Huge]{8}[2]
    \vspace{-1em}\spelltwocol{}{Huge animate}\vspace{-1em}
    \vspace{0em}

    
    

    \begin{spellcontent}
      \begin{spelltargetinginfo}
        \pari \textbf{HP} 10 \monsep
          \textbf{AD} 18 \monsep
          \textbf{Fort} 13 \monsep
          \textbf{Ref} 16 \monsep
          \textbf{Ment} 12
        \pari \textbf{WR} 12 \monsep
        \textbf{VR} 39
        \pari \textbf{Strike:}
            Slam \plus10 (4d10 bludgeoning)
      \end{spelltargetinginfo}
    \end{spellcontent}
    \begin{monsterfooter}
      \pari \textbf{Speed} 50 ft. \monsep
        \textbf{Space} 15 ft. \monsep
        \textbf{Reach} 15 ft.
      \pari \textbf{Awareness} \plus4
      \pari \textbf{Attributes}
        Str 5, Dex 11,
        Con 5, Int -2,
        Per 9, Wil 0
      \pari \textbf{Accuracy} 10 \monsep
        \textbf{Mundane Power} 11 \monsep
      \textbf{Magical Power} 10
      \pari \textbf{Alignment} Usually true neutral
    \end{monsterfooter}
  \end{monsubsection}
  \begin{freeability}{Whirlwind}
      \targets{Each \glossterm{enemy} within reach}
       The air elemental makes a \plus10
        \glossterm{strike} vs. Armor
        with its slam against each target.
    
    \hit Each target takes 4d10 bludgeoning damage.
    \end{freeability}
  ,
  \begin{monsubsection}{Air Elemental}[Elder]{11}[4]
    \vspace{-1em}\spelltwocol{}{Huge animate}\vspace{-1em}
    \vspace{0em}

    
    

    \begin{spellcontent}
      \begin{spelltargetinginfo}
        \pari \textbf{HP} 40 \monsep
          \textbf{AD} 24 \monsep
          \textbf{Fort} 18 \monsep
          \textbf{Ref} 22 \monsep
          \textbf{Ment} 17
        \pari \textbf{WR} 19 \monsep
        \textbf{VR} 57
        \pari \textbf{Strike:}
            Slam \plus16 (11d10 bludgeoning)
      \end{spelltargetinginfo}
    \end{spellcontent}
    \begin{monsterfooter}
      \pari \textbf{Speed} 50 ft. \monsep
        \textbf{Space} 15 ft. \monsep
        \textbf{Reach} 15 ft.
      \pari \textbf{Awareness} \plus6
      \pari \textbf{Attributes}
        Str 6, Dex 15,
        Con 6, Int 0,
        Per 13, Wil 0
      \pari \textbf{Accuracy} 16 \monsep
        \textbf{Mundane Power} 24 \monsep
      \textbf{Magical Power} 23
      \pari \textbf{Alignment} Usually true neutral
    \end{monsterfooter}
  \end{monsubsection}
  \begin{freeability}{Whirlwind}
      \targets{Each \glossterm{enemy} within reach}
       The air elemental makes a \plus16
        \glossterm{strike} vs. Armor
        with its slam against each target.
    
    \hit Each target takes 11d10 bludgeoning damage.
    \end{freeability}
  
  
    \subsection{Animated Objects}
      
    Animated objects come in all sizes, shapes, and colors. They owe their existence as creatures to magical effects.

    Animated objects fight only as directed by the animator. They follow orders without question and to the best of their abilities. Since they do not need to breathe and never tire, they can be extremely capable minions.
  

      

      
  \begin{monsubsection}{Animated Object}[Tiny]{1}[0.5]
    \vspace{-1em}\spelltwocol{}{Tiny animate}\vspace{-1em}
    \vspace{0em}

    
    

    \begin{spellcontent}
      \begin{spelltargetinginfo}
        \pari \textbf{HP} 0 \monsep
          \textbf{AD} 8 \monsep
          \textbf{Fort} 0 \monsep
          \textbf{Ref} 8 \monsep
          \textbf{Ment} -4
        \pari \textbf{WR} 2.5 \monsep
        \textbf{VR} 8
        \pari \textbf{Strike:}
            Slam \plus1 (1d8 bludgeoning)
      \end{spelltargetinginfo}
    \end{spellcontent}
    \begin{monsterfooter}
      \pari \textbf{Speed} 20 ft. \monsep
        \textbf{Space} 2-1/2 ft. \monsep
        \textbf{Reach} 0 ft.
      \pari \textbf{Awareness} \plus0
      \pari \textbf{Attributes}
        Str -4, Dex 4,
        Con -4, Int N/A,
        Per 0, Wil -8
      \pari \textbf{Accuracy} 1 \monsep
        \textbf{Mundane Power} -3 \monsep
      \textbf{Magical Power} -7
      \pari \textbf{Alignment} Always true neutral
    \end{monsterfooter}
  \end{monsubsection}
  
  ,
  \begin{monsubsection}{Animated Object}[Small]{1}[1]
    \vspace{-1em}\spelltwocol{}{Small animate}\vspace{-1em}
    \vspace{0em}

    
    

    \begin{spellcontent}
      \begin{spelltargetinginfo}
        \pari \textbf{HP} 3 \monsep
          \textbf{AD} 6 \monsep
          \textbf{Fort} 2 \monsep
          \textbf{Ref} 6 \monsep
          \textbf{Ment} -4
        \pari \textbf{WR} 5 \monsep
        \textbf{VR} 16
        \pari \textbf{Strike:}
            Slam \plus1 (1d10 bludgeoning)
      \end{spelltargetinginfo}
    \end{spellcontent}
    \begin{monsterfooter}
      \pari \textbf{Speed} 25 ft. \monsep
        \textbf{Space} 5 ft. \monsep
        \textbf{Reach} 5 ft.
      \pari \textbf{Awareness} \plus0
      \pari \textbf{Attributes}
        Str -2, Dex 2,
        Con -2, Int N/A,
        Per 0, Wil -8
      \pari \textbf{Accuracy} 1 \monsep
        \textbf{Mundane Power} -1 \monsep
      \textbf{Magical Power} -7
      \pari \textbf{Alignment} Always true neutral
    \end{monsterfooter}
  \end{monsubsection}
  
  ,
  \begin{monsubsection}{Animated Object}[Medium]{2}[2]
    \vspace{-1em}\spelltwocol{}{Medium animate}\vspace{-1em}
    \vspace{0em}

    
    

    \begin{spellcontent}
      \begin{spelltargetinginfo}
        \pari \textbf{HP} 9 \monsep
          \textbf{AD} 6 \monsep
          \textbf{Fort} 6 \monsep
          \textbf{Ref} 6 \monsep
          \textbf{Ment} -2
        \pari \textbf{WR} 6 \monsep
        \textbf{VR} 18
        \pari \textbf{Strike:}
            Slam \plus3 (2d8 bludgeoning)
      \end{spelltargetinginfo}
    \end{spellcontent}
    \begin{monsterfooter}
      \pari \textbf{Speed} 30 ft. \monsep
        \textbf{Space} 5 ft. \monsep
        \textbf{Reach} 5 ft.
      \pari \textbf{Awareness} \plus0
      \pari \textbf{Attributes}
        Str 0, Dex 0,
        Con 0, Int N/A,
        Per 0, Wil -8
      \pari \textbf{Accuracy} 3 \monsep
        \textbf{Mundane Power} 2 \monsep
      \textbf{Magical Power} -6
      \pari \textbf{Alignment} Always true neutral
    \end{monsterfooter}
  \end{monsubsection}
  
  ,
  \begin{monsubsection}{Animated Object}[Large]{4}[2]
    \vspace{-1em}\spelltwocol{}{Large animate}\vspace{-1em}
    \vspace{0em}

    
    

    \begin{spellcontent}
      \begin{spelltargetinginfo}
        \pari \textbf{HP} 11 \monsep
          \textbf{AD} 8 \monsep
          \textbf{Fort} 10 \monsep
          \textbf{Ref} 8 \monsep
          \textbf{Ment} 0
        \pari \textbf{WR} 9 \monsep
        \textbf{VR} 26
        \pari \textbf{Strike:}
            Slam \plus5 (4d6 bludgeoning)
      \end{spelltargetinginfo}
    \end{spellcontent}
    \begin{monsterfooter}
      \pari \textbf{Speed} 40 ft. \monsep
        \textbf{Space} 10 ft. \monsep
        \textbf{Reach} 10 ft.
      \pari \textbf{Awareness} \plus0
      \pari \textbf{Attributes}
        Str 5, Dex 0,
        Con 5, Int N/A,
        Per 0, Wil -8
      \pari \textbf{Accuracy} 5 \monsep
        \textbf{Mundane Power} 6 \monsep
      \textbf{Magical Power} -4
      \pari \textbf{Alignment} Always true neutral
    \end{monsterfooter}
  \end{monsubsection}
  
  ,
  \begin{monsubsection}{Animated Object}[Huge]{7}[2]
    \vspace{-1em}\spelltwocol{}{Huge animate}\vspace{-1em}
    \vspace{0em}

    
    

    \begin{spellcontent}
      \begin{spelltargetinginfo}
        \pari \textbf{HP} 12 \monsep
          \textbf{AD} 10 \monsep
          \textbf{Fort} 14 \monsep
          \textbf{Ref} 10 \monsep
          \textbf{Ment} 3
        \pari \textbf{WR} 15 \monsep
        \textbf{VR} 46
        \pari \textbf{Strike:}
            Slam \plus8 (5d10 bludgeoning)
      \end{spelltargetinginfo}
    \end{spellcontent}
    \begin{monsterfooter}
      \pari \textbf{Speed} 50 ft. \monsep
        \textbf{Space} 15 ft. \monsep
        \textbf{Reach} 15 ft.
      \pari \textbf{Awareness} \plus0
      \pari \textbf{Attributes}
        Str 9, Dex -1,
        Con 9, Int N/A,
        Per 0, Wil -8
      \pari \textbf{Accuracy} 8 \monsep
        \textbf{Mundane Power} 12 \monsep
      \textbf{Magical Power} 1
      \pari \textbf{Alignment} Always true neutral
    \end{monsterfooter}
  \end{monsubsection}
  
  ,
  \begin{monsubsection}{Animated Object}[Gargantuan]{9}[3]
    \vspace{-1em}\spelltwocol{}{Gargantuan animate}\vspace{-1em}
    \vspace{0em}

    
    

    \begin{spellcontent}
      \begin{spelltargetinginfo}
        \pari \textbf{HP} 26 \monsep
          \textbf{AD} 12 \monsep
          \textbf{Fort} 18 \monsep
          \textbf{Ref} 12 \monsep
          \textbf{Ment} 6
        \pari \textbf{WR} 23 \monsep
        \textbf{VR} 64
        \pari \textbf{Strike:}
            Slam \plus11 (7d10 bludgeoning)
      \end{spelltargetinginfo}
    \end{spellcontent}
    \begin{monsterfooter}
      \pari \textbf{Speed} 60 ft. \monsep
        \textbf{Space} 20 ft. \monsep
        \textbf{Reach} 20 ft.
      \pari \textbf{Awareness} \plus0
      \pari \textbf{Attributes}
        Str 12, Dex -2,
        Con 12, Int N/A,
        Per 0, Wil -8
      \pari \textbf{Accuracy} 11 \monsep
        \textbf{Mundane Power} 17 \monsep
      \textbf{Magical Power} 5
      \pari \textbf{Alignment} Always true neutral
    \end{monsterfooter}
  \end{monsubsection}
  
  ,
  \begin{monsubsection}{Animated Object}[Colossal]{11}[3]
    \vspace{-1em}\spelltwocol{}{Gargantuan animate}\vspace{-1em}
    \vspace{0em}

    
    

    \begin{spellcontent}
      \begin{spelltargetinginfo}
        \pari \textbf{HP} 28 \monsep
          \textbf{AD} 13 \monsep
          \textbf{Fort} 21 \monsep
          \textbf{Ref} 13 \monsep
          \textbf{Ment} 8
        \pari \textbf{WR} 32 \monsep
        \textbf{VR} 85
        \pari \textbf{Strike:}
            Slam \plus13 (11d10 bludgeoning)
      \end{spelltargetinginfo}
    \end{spellcontent}
    \begin{monsterfooter}
      \pari \textbf{Speed} 60 ft. \monsep
        \textbf{Space} 20 ft. \monsep
        \textbf{Reach} 20 ft.
      \pari \textbf{Awareness} \plus0
      \pari \textbf{Attributes}
        Str 15, Dex -3,
        Con 15, Int N/A,
        Per 0, Wil -8
      \pari \textbf{Accuracy} 13 \monsep
        \textbf{Mundane Power} 24 \monsep
      \textbf{Magical Power} 11
      \pari \textbf{Alignment} Always true neutral
    \end{monsterfooter}
  \end{monsubsection}
  
  
  
        \section{Humanoids}

        All humanoids have the following properties unless noted otherwise in their description:
        
    
    \parhead{Defenses}
    \plus0 Armor,
    \plus3 Fortitude,
    \plus3 Reflex,
    \plus3 Mental
  
  
      
  \begin{monsection}{Cultist}{2}[1]
    \vspace{-1em}\spelltwocol{}{Medium humanoid}\vspace{-1em}
    \vspace{0em}

    
      Cultists may serve many masters.
      They are united in their generally malign intentions and their magical abilities.
    
    

    \begin{spellcontent}
      \begin{spelltargetinginfo}
        \pari \textbf{HP} 4 \monsep
          \textbf{AD} 2 \monsep
          \textbf{Fort} 5 \monsep
          \textbf{Ref} 5 \monsep
          \textbf{Ment} 7
        \pari \textbf{WR} 3 \monsep
        \textbf{VR} 15
        \pari \textbf{Strike:}
            Club \plus2 (1d8 bludgeoning)
      \end{spelltargetinginfo}
    \end{spellcontent}
    \begin{monsterfooter}
      \pari \textbf{Speed} 30 ft. \monsep
        \textbf{Space} 5 ft. \monsep
        \textbf{Reach} 5 ft.
      \pari \textbf{Awareness} \plus0
      \pari \textbf{Attributes}
        Str -1, Dex 0,
        Con 0, Int -1,
        Per 0, Wil 3
      \pari \textbf{Accuracy} 2 \monsep
        \textbf{Mundane Power} 1 \monsep
      \textbf{Magical Power} 4
      \pari \textbf{Alignment} Usually lawful evil
    \end{monsterfooter}
  \end{monsection}
  \begin{freeability}{Drain Life}
      \target{One creature within \rngmed range}
       The cultist makes a \plus2 attack
        vs. Fortitude against the target.
    
    \hit The target loses a \glossterm{hit point}
    \end{freeability}
  
  \begin{monsection}{Lizardfolk}[Elite]{10}[2]
    \vspace{-1em}\spelltwocol{}{Medium humanoid}\vspace{-1em}
    \vspace{0em}

    
    Lizardfolk are bipedal creatures covered in reptilian scales.
    Their tail resembles that of a crocodile, and is typically 3 to 4 feet long.
    They use it for balance on land and to accelerate their swimming while in water.
    Their scales are typically green, gray, or brown.

    Lizardfolk fight as unorganized individuals.
    They prefer frontal assaults and massed rushes, sometimes trying to force foes into the water, where the lizardfolk have an advantage.
    If outnumbered or if their territory is being invaded, they set snares, plan ambushes, and make raids to hinder enemy supplies.
    Advanced tribes use more sophisticated tactics and have better traps and ambushes.
  
    

    \begin{spellcontent}
      \begin{spelltargetinginfo}
        \pari \textbf{HP} 11 \monsep
          \textbf{AD} 19 \monsep
          \textbf{Fort} 16 \monsep
          \textbf{Ref} 14 \monsep
          \textbf{Ment} 14
        \pari \textbf{WR} Physical 27, Energy 21 \monsep
        \textbf{VR} Physical 65, Energy 59
        \pari \textbf{Strike:}
            Spear \plus11 (4d10 piercing)
      \end{spelltargetinginfo}
    \end{spellcontent}
    \begin{monsterfooter}
      \pari \textbf{Speed} 30 ft. \monsep
        \textbf{Space} 5 ft. \monsep
        \textbf{Reach} 5 ft.
      \pari \textbf{Awareness} \plus0
      \pari \textbf{Attributes}
        Str 12, Dex 0,
        Con 11, Int 0,
        Per 0, Wil 0
      \pari \textbf{Accuracy} 11 \monsep
        \textbf{Mundane Power} 15 \monsep
      \textbf{Magical Power} 12
      \pari \textbf{Alignment} Usually true neutral
    \end{monsterfooter}
  \end{monsection}
  \parhead{Hold Breath} A lizardfolk can hold its breath for ten times the normal length of time.
  
  \begin{monsection}{Lizardfolk}[Grunt]{10}[1]
    \vspace{-1em}\spelltwocol{}{Medium humanoid}\vspace{-1em}
    \vspace{0em}

    
    Lizardfolk are bipedal creatures covered in reptilian scales.
    Their tail resembles that of a crocodile, and is typically 3 to 4 feet long.
    They use it for balance on land and to accelerate their swimming while in water.
    Their scales are typically green, gray, or brown.

    Lizardfolk fight as unorganized individuals.
    They prefer frontal assaults and massed rushes, sometimes trying to force foes into the water, where the lizardfolk have an advantage.
    If outnumbered or if their territory is being invaded, they set snares, plan ambushes, and make raids to hinder enemy supplies.
    Advanced tribes use more sophisticated tactics and have better traps and ambushes.
  
    

    \begin{spellcontent}
      \begin{spelltargetinginfo}
        \pari \textbf{HP} 5 \monsep
          \textbf{AD} 18 \monsep
          \textbf{Fort} 15 \monsep
          \textbf{Ref} 13 \monsep
          \textbf{Ment} 13
        \pari \textbf{WR} Physical 27, Energy 21 \monsep
        \textbf{VR} Physical 65, Energy 59
        \pari \textbf{Strike:}
            Spear \plus10 (4d8 piercing)
      \end{spelltargetinginfo}
    \end{spellcontent}
    \begin{monsterfooter}
      \pari \textbf{Speed} 30 ft. \monsep
        \textbf{Space} 5 ft. \monsep
        \textbf{Reach} 5 ft.
      \pari \textbf{Awareness} \plus0
      \pari \textbf{Attributes}
        Str 11, Dex 0,
        Con 11, Int 0,
        Per 0, Wil 0
      \pari \textbf{Accuracy} 10 \monsep
        \textbf{Mundane Power} 12 \monsep
      \textbf{Magical Power} 10
      \pari \textbf{Alignment} Usually true neutral
    \end{monsterfooter}
  \end{monsection}
  \parhead{Hold Breath} A lizardfolk can hold its breath for ten times the normal length of time.
  
  \begin{monsection}{Orc}[Elite]{8}[1]
    \vspace{-1em}\spelltwocol{}{Medium humanoid}\vspace{-1em}
    \vspace{0em}

    
      Elite orcs are battle-hardened war veterans who are deadly at any range.
      They tend to prioritize raw strength over subtlety.
    
    

    \begin{spellcontent}
      \begin{spelltargetinginfo}
        \pari \textbf{HP} 5 \monsep
          \textbf{AD} 11 \monsep
          \textbf{Fort} 13 \monsep
          \textbf{Ref} 11 \monsep
          \textbf{Ment} 10
        \pari \textbf{WR} Physical 19, Energy 14 \monsep
        \textbf{VR} Physical 50, Energy 45
        \pari \textbf{Strikes:}
            Greataxe \plus8 (5d10 slashing; Sweeping 1);
\par Light Crossbow \plus8 (4d8 piercing)
      \end{spelltargetinginfo}
    \end{spellcontent}
    \begin{monsterfooter}
      \pari \textbf{Speed} 30 ft. \monsep
        \textbf{Space} 5 ft. \monsep
        \textbf{Reach} 5 ft.
      \pari \textbf{Awareness} \plus0
      \pari \textbf{Attributes}
        Str 11, Dex 0,
        Con 9, Int -2,
        Per 0, Wil -1
      \pari \textbf{Accuracy} 8 \monsep
        \textbf{Mundane Power} 12 \monsep
      \textbf{Magical Power} 7
      \pari \textbf{Alignment} Usually lawful evil
    \end{monsterfooter}
  \end{monsection}
  \begin{freeability}{Power Smash}
      \target{One creature or object within \glossterm{reach}}
       The orc makes a \plus6
        \glossterm{strike} vs. Armor
        with its greataxe against the target.
    
    \hit The target takes 7d10 slashing damage.
    \end{freeability}
  
  \begin{monsection}{Pyromancer}{5}[1]
    \vspace{-1em}\spelltwocol{}{Medium humanoid}\vspace{-1em}
    \vspace{0em}

    
      Pyromancers wield powerful fire magic to attack their foes.
    
    

    \begin{spellcontent}
      \begin{spelltargetinginfo}
        \pari \textbf{HP} 5 \monsep
          \textbf{AD} 5 \monsep
          \textbf{Fort} 9 \monsep
          \textbf{Ref} 8 \monsep
          \textbf{Ment} 10
        \pari \textbf{WR} 6 \monsep
        \textbf{VR} 23
        \pari \textbf{Strike:}
            Club \plus5 (2d6 bludgeoning)
      \end{spelltargetinginfo}
    \end{spellcontent}
    \begin{monsterfooter}
      \pari \textbf{Speed} 30 ft. \monsep
        \textbf{Space} 5 ft. \monsep
        \textbf{Reach} 5 ft.
      \pari \textbf{Awareness} \plus0
      \pari \textbf{Attributes}
        Str -1, Dex 0,
        Con 3, Int -1,
        Per 0, Wil 6
      \pari \textbf{Accuracy} 5 \monsep
        \textbf{Mundane Power} 4 \monsep
      \textbf{Magical Power} 7
      \pari \textbf{Alignment} Any
    \end{monsterfooter}
  \end{monsection}
  \begin{freeability}{Combustion}
      \target{One creature or object within \rngmed range}
       The pyromancer makes a \plus5 attack
        vs. Reflex against the target.
    
    \hit The target takes 4d6 fire damage.
    \end{freeability}
  

    \begin{freeability}{Fireball}
      \targets{Everything in a \areasmall radius within \rngmed range}
       The pyromancer makes a \plus5 attack
        vs. Armor against each target.
    
    \hit Each target takes 2d8 fire damage.
    \end{freeability}
  
        \section{Magical Beasts}

        All magical beasts have the following properties unless noted otherwise in their description:
        
    
    \parhead{Defenses}
    \plus2 Armor,
    \plus3 Fortitude,
    \plus3 Reflex,
    \plus3 Mental
  
  
      
  \begin{monsection}{Ankheg}{5}[2]
    \vspace{-1em}\spelltwocol{}{Large magical beast}\vspace{-1em}
    \vspace{0em}

    
      An ankheg is a burrowing monster with a taste for fresh meat. It has six legs, and some specimens are yellow rather than brown. It is about 10 feet long and weighs about 800 pounds.

      An ankheg burrows with legs and mandibles. A burrowing ankheg usually does not make a usable tunnel, but can construct a tunnel; it burrows at half speed when it does so. It often digs a winding tunnel up to 40 feet below the surface in the rich soil of forests or farmlands. The tunnel is 5 feet tall and wide, and usually 60 to 150 feet long.
    
    

    \begin{spellcontent}
      \begin{spelltargetinginfo}
        \pari \textbf{HP} 11 \monsep
          \textbf{AD} 11 \monsep
          \textbf{Fort} 11 \monsep
          \textbf{Ref} 8 \monsep
          \textbf{Ment} 7
        \pari \textbf{WR} Physical 13, Energy 10 \monsep
        \textbf{VR} Physical 33, Energy 30
        \pari \textbf{Strike:}
            Bite \plus6 (4d8 acid, bludgeoning, and piercing)
      \end{spelltargetinginfo}
    \end{spellcontent}
    \begin{monsterfooter}
      \pari \textbf{Speeds} Burrow 20 ft., Land 40 ft. \monsep
        \textbf{Space} 10 ft. \monsep
        \textbf{Reach} 10 ft.
      \pari \textbf{Awareness} \plus3
      \pari \textbf{Attributes}
        Str 8, Dex -1,
        Con 6, Int -8,
        Per 0, Wil -2
      \pari \textbf{Accuracy} 6 \monsep
        \textbf{Mundane Power} 11 \monsep
      \textbf{Magical Power} 5
      \pari \textbf{Alignment} Always true neutral
    \end{monsterfooter}
  \end{monsection}
  \begin{freeability}{Spit Acid}
      \targets{Everything in a \areamed line}
       The ankheg makes a \plus6 attack
        vs. Reflex against each target.
    
    \hit Each target takes 2d10 acid damage.
    \end{freeability}
  

    \begin{freeability}{Drag Prey}
      \target{One Medium or smaller creature or object within \glossterm{reach}}
       The ankheg makes a \plus12 attack
        vs. Fortitude against the target.
    
    \hit The ankheg \glossterm{pushes} the target up to 30 feet.
          It can move the same distance that it pushes the target.
    \end{freeability}
  
      \par \textbf{Darkvision 50 Ft}
    \par \textbf{Tremorsense 50 Ft}
  
  \begin{monsection}{Nightcrawler}{7}[2]
    \vspace{-1em}\spelltwocol{}{Large magical beast}\vspace{-1em}
    \vspace{0em}

    
      A nightcrawler is a large worm imbued with umbramantic power.
      Its body is colored only in shades of gray.
      It is about 9 feet long and weighs about 700 pounds.
      Nightcrawlers move slowly, but they are surprisingly agile in combat.
      They can easily contort their body to avoid attacks or wrap around the defenses of foes.
    
    

    \begin{spellcontent}
      \begin{spelltargetinginfo}
        \pari \textbf{HP} 9 \monsep
          \textbf{AD} 14 \monsep
          \textbf{Fort} 11 \monsep
          \textbf{Ref} 14 \monsep
          \textbf{Ment} 13
        \pari \textbf{WR} 11 \monsep
        \textbf{VR} 35
        
      \end{spelltargetinginfo}
    \end{spellcontent}
    \begin{monsterfooter}
      \pari \textbf{Speeds} Climb 20 ft., Land 40 ft. \monsep
        \textbf{Space} 10 ft. \monsep
        \textbf{Reach} 10 ft.
      \pari \textbf{Awareness} \plus2
      \pari \textbf{Attributes}
        Str 4, Dex 9,
        Con 0, Int -8,
        Per 4, Wil 8
      \pari \textbf{Accuracy} 8 \monsep
        \textbf{Mundane Power} 10 \monsep
      \textbf{Magical Power} 11
      \pari \textbf{Alignment} Always true neutral
    \end{monsterfooter}
  \end{monsection}
  \begin{freeability}{Crawling Darkness}
      \targets{Enemies in a \areamed radius}
       The nightcrawler makes a \plus8 attack
        vs. Fortitude against each target.
    \end{freeability}
  

    \begin{freeability}{Dark Embrace}
      \target{One enemy within \reach}
       The nightcrawler makes a \plus8 attack
        vs. Reflex against the target.
    \end{freeability}
  
        \section{Monstrous Humanoids}

        All monstrous humanoids have the following properties unless noted otherwise in their description:
        
    
    \parhead{Defenses}
    \plus2 Armor,
    \plus3 Fortitude,
    \plus3 Reflex,
    \plus3 Mental
  
  
      
  \begin{monsection}{Giant}[Hill]{7}[3]
    \vspace{-1em}\spelltwocol{}{Huge monstrous humanoid}\vspace{-1em}
    \vspace{0em}

    
      Skin color among hill giants ranges from light tan to deep ruddy brown. Their hair is brown or black, with eyes the same color. Hill giants wear layers of crudely prepared hides with the fur left on. They seldom wash or repair their garments, preferring to simply add more hides as their old ones wear out.

      Adults are about 15 feet tall. Hill giants can live to be 70 years old.
    
    \parhead{Tactics} 
      Hill giants prefer to fight from high, rocky outcroppings, where they can pelt opponents with rocks and boulders while limiting the risk to themselves.
      They lack the intelligence or desire to retreat if their enemies survive to approach them, and prefer to draw their massive clubs and enter melee.
      If possible, they smash their foes off of cliffs.
    

    \begin{spellcontent}
      \begin{spelltargetinginfo}
        \pari \textbf{HP} 24 \monsep
          \textbf{AD} 13 \monsep
          \textbf{Fort} 15 \monsep
          \textbf{Ref} 10 \monsep
          \textbf{Ment} 10
        \pari \textbf{WR} Physical 19, Energy 17 \monsep
        \textbf{VR} Physical 50, Energy 48
        \pari \textbf{Strikes:}
            Greatclub \plus7 (7d10 bludgeoning; Forceful);
\par Boulder \plus7 (5d10 bludgeoning; ; 100 ft. range)
      \end{spelltargetinginfo}
    \end{spellcontent}
    \begin{monsterfooter}
      \pari \textbf{Speed} 50 ft. \monsep
        \textbf{Space} 15 ft. \monsep
        \textbf{Reach} 15 ft.
      \pari \textbf{Awareness} \minus1
      \pari \textbf{Attributes}
        Str 11, Dex -2,
        Con 9, Int -2,
        Per -2, Wil -2
      \pari \textbf{Accuracy} 7 \monsep
        \textbf{Mundane Power} 16 \monsep
      \textbf{Magical Power} 9
      \pari \textbf{Alignment} Usually chaotic evil
    \end{monsterfooter}
  \end{monsection}
  \parhead{Massive Sweep} A hill giant's greatclub has the Sweeping (2) tag.
  
  \begin{monsection}{Giant}[Stone]{10}[3]
    \vspace{-1em}\spelltwocol{}{Gargantuan monstrous humanoid}\vspace{-1em}
    \vspace{0em}

    
      Stone giants prefer thick leather garments, dyed in shades of brown and gray to match the stone around them. Adults stand about 20 feet tall. Stone giants can live to be 300 years old.

      Young stone giants can be capricious, hunting tiny creatures like goats and humanoids on a whim.
      Elder stone giants tend to be wiser and more cautious, and avoid unnecessary conflict.
    
    \parhead{Tactics} 
      Stone giants fight from a great distance whenever possible, using their ability to hurl stones up to 1,000 feet.
    

    \begin{spellcontent}
      \begin{spelltargetinginfo}
        \pari \textbf{HP} 24 \monsep
          \textbf{AD} 17 \monsep
          \textbf{Fort} 18 \monsep
          \textbf{Ref} 14 \monsep
          \textbf{Ment} 13
        \pari \textbf{WR} 25 \monsep
        \textbf{VR} 66
        \pari \textbf{Strikes:}
            Greatclub \plus12 (9d10 bludgeoning; Forceful);
\par Boulder \plus12 (7d10 bludgeoning; ; 200 ft. range)
      \end{spelltargetinginfo}
    \end{spellcontent}
    \begin{monsterfooter}
      \pari \textbf{Speed} 60 ft. \monsep
        \textbf{Space} 20 ft. \monsep
        \textbf{Reach} 20 ft.
      \pari \textbf{Awareness} \plus0
      \pari \textbf{Attributes}
        Str 16, Dex -1,
        Con 12, Int 0,
        Per 0, Wil -2
      \pari \textbf{Accuracy} 12 \monsep
        \textbf{Mundane Power} 21 \monsep
      \textbf{Magical Power} 12
      \pari \textbf{Alignment} Usually true neutral
    \end{monsterfooter}
  \end{monsection}
  \parhead{Massive Sweep} A stone giant's greatclub has the Sweeping (2) tag.
  
  \begin{monsection}{Minotaur}{6}[2]
    \vspace{-1em}\spelltwocol{}{Large monstrous humanoid}\vspace{-1em}
    \vspace{0em}

    
      Minotaurs are bull-headed creatures known for their poor sense of direction.
      They can be cunning in battle, but have a tendency to become trapped in dungeons of even moderate complexity.
    
    

    \begin{spellcontent}
      \begin{spelltargetinginfo}
        \pari \textbf{HP} 11 \monsep
          \textbf{AD} 11 \monsep
          \textbf{Fort} 12 \monsep
          \textbf{Ref} 9 \monsep
          \textbf{Ment} 11
        \pari \textbf{WR} 11 \monsep
        \textbf{VR} 35
        \pari \textbf{Strikes:}
            Gore \plus7 (4d10 piercing; Impact);
\par Ram \plus7 (4d10 bludgeoning; Forceful)
      \end{spelltargetinginfo}
    \end{spellcontent}
    \begin{monsterfooter}
      \pari \textbf{Speed} 40 ft. \monsep
        \textbf{Space} 10 ft. \monsep
        \textbf{Reach} 10 ft.
      \pari \textbf{Awareness} \plus0
      \pari \textbf{Attributes}
        Str 9, Dex -1,
        Con 7, Int 0,
        Per 0, Wil 4
      \pari \textbf{Accuracy} 7 \monsep
        \textbf{Mundane Power} 12 \monsep
      \textbf{Magical Power} 9
      \pari \textbf{Alignment} Usually chaotic evil
    \end{monsterfooter}
  \end{monsection}
  \begin{freeability}{Forceful Shove}
      \targets{As a ram strike}
      For each size category larger or smaller than the target that the minotaur is, it gains a +4 bonus or penalty to \glossterm{accuracy}. The minotaur makes a \plus9
        \glossterm{strike} vs. Fortitude
        with its ram against each target.
    
    \hit The target is knocked back 10 feet and takes 2d10 bludgeoning damage.
    \end{freeability}
  
        \section{Outsiders}

        All outsiders have the following properties unless noted otherwise in their description:
        
    
    \parhead{Defenses}
    \plus3 Armor,
    \plus3 Fortitude,
    \plus3 Reflex,
    \plus3 Mental
  
    \parhead{Planar Essence} An outsider is fundamentally composed of the essence of its home plane.
    When an outsider dies, its essence returns to its plane.
    Weak outsiders lose their independent identity and become part of the core composition of the plane once more.
    Strong outsiders can retain their identity and reform from that raw material given time, making them difficult or impossible to kill completely.
    In either case, outsiders cannot be resurrected by mortal magic such as the \spell{resurrection} spell.
  
      
    \subsection{Angels}
      
      Angels are divine beings native to the good-aligned Outer Planes.
      All angels have a striking and highly memorable appearance that evokes strong emotions in most viewers.
      Most angels evoke an overpowering sense of awe and beauty, but individual angels may have highly varied appearances.
    

      

      
  \begin{monsubsection}{Astral Deva}{12}[4]
    \vspace{-1em}\spelltwocol{}{Medium outsider}\vspace{-1em}
    \vspace{0em}

    
         An astral deva is about 7 1/2 feet tall and weighs about 250 pounds.
        
    

    \begin{spellcontent}
      \begin{spelltargetinginfo}
        \pari \textbf{HP} 44 \monsep
          \textbf{AD} 22 \monsep
          \textbf{Fort} 21 \monsep
          \textbf{Ref} 21 \monsep
          \textbf{Ment} 22
        \pari \textbf{WR} Physical 40, Energy 46 \monsep
        \textbf{VR} Physical 85, Energy 91
        \pari \textbf{Strike:}
            Greatmace \plus17 (13d10 bludgeoning; Impact)
      \end{spelltargetinginfo}
    \end{spellcontent}
    \begin{monsterfooter}
      \pari \textbf{Speeds} Fly 40 ft., Land 30 ft. \monsep
        \textbf{Space} 5 ft. \monsep
        \textbf{Reach} 5 ft.
      \pari \textbf{Awareness} \plus6
      \pari \textbf{Attributes}
        Str 14, Dex 13,
        Con 13, Int 13,
        Per 13, Wil 14
      \pari \textbf{Accuracy} 17 \monsep
        \textbf{Power} 29
      \pari \textbf{Alignment} Always good
    \end{monsterfooter}
  \end{monsubsection}
  \begin{freeability}{Proclamation}
      \targets{\glossterm{Enemies} within a \arealarge radius from the astral deva}
       The astral deva makes a \plus17 attack
        vs. Mental against each target.
    
    \hit Each target takes 10d10 energy damage.
    \end{freeability}
  

    \begin{freeability}{Glimpse Of Divinity}
      \target{One creature within \rngmed range of the astral deva}
       The astral deva makes a \plus19 attack
        vs. Mental against the target.
    
    \hit The target is \glossterm{dazzled} as a \glossterm{condition}.
    \crit The target is \glossterm{blinded} as a condition.
    \end{freeability}
  

    \begin{freeability}{Stunning Smash}
      \target{One creature or object within \glossterm{reach}}
       The astral deva makes a \plus17
        \glossterm{strike} vs. Armor
        with its greatmace against the target.
    
    \hit The target takes 13d10 bludgeoning damage. If the target is \glossterm{wounded} by the attack, it is \glossterm{stunned} as a \glossterm{condition}.
    \end{freeability}
  
      \parhead{Agent Of The Divine} 
              The astral deva is \glossterm{attuned} to this ability.
              It gains a \plus1 \glossterm{magic bonus} to accuracy and all defenses, and a \plus2 magic bonus to power.
            
    \parhead{Divine Alacrity} 
              The astral deva can take two standard actions each round.
              It cannot use the same action twice in the same round.
            
    \parhead{Divine Intervention} 
              An astral deva can learn and perform divine spells and rituals as a rank 5 caster.
              It has access to two spheres and knows five spells from among those spheres.
              They do not use verbal or somatic components to cast spells, and do not suffer any \glossterm{focus penalty}.
              A typical astral deva has access to the Bless and Channel Divinity spheres, and knows the \spell{agent of the divine}, \spell{boon of cleansing}, \spell{glimpse of divinity}, \spell{mantle of faith} and \spell{proclamation} spells.
            
    \parhead{Mantle Of Faith} 
              The astral deva is \glossterm{attuned} to this ability.
              It gains a \plus14 \glossterm{magic bonus} to \glossterm{resistances}.
  ,
  \begin{monsubsection}{Planetar}{16}[4]
    \vspace{-1em}\spelltwocol{}{Medium outsider}\vspace{-1em}
    \vspace{0em}

    
          A planetar is nearly 9 feet tall and weighs about 500 pounds.
        
    

    \begin{spellcontent}
      \begin{spelltargetinginfo}
        \pari \textbf{HP} 44 \monsep
          \textbf{AD} 25 \monsep
          \textbf{Fort} 24 \monsep
          \textbf{Ref} 24 \monsep
          \textbf{Ment} 25
        \pari \textbf{WR} Physical 54, Energy 62 \monsep
        \textbf{VR} Physical 118, Energy 126
        \pari \textbf{Strike:}
            Greatmace \plus20 (14d10 bludgeoning; Impact)
      \end{spelltargetinginfo}
    \end{spellcontent}
    \begin{monsterfooter}
      \pari \textbf{Speeds} Fly 50 ft., Land 30 ft. \monsep
        \textbf{Space} 5 ft. \monsep
        \textbf{Reach} 5 ft.
      \pari \textbf{Awareness} \plus8
      \pari \textbf{Attributes}
        Str 18, Dex 17,
        Con 17, Int 17,
        Per 17, Wil 18
      \pari \textbf{Accuracy} 20 \monsep
        \textbf{Power} 31
      \pari \textbf{Alignment} Always good
    \end{monsterfooter}
  \end{monsubsection}
  
  ,
  \begin{monsubsection}{Solar}{20}[4]
    \vspace{-1em}\spelltwocol{}{Medium outsider}\vspace{-1em}
    \vspace{0em}

    
          A solar has a deep and commanding voice, and stands about 9 feet tall. It weighs about 500 pounds.
        
    

    \begin{spellcontent}
      \begin{spelltargetinginfo}
        \pari \textbf{HP} 44 \monsep
          \textbf{AD} 29 \monsep
          \textbf{Fort} 28 \monsep
          \textbf{Ref} 28 \monsep
          \textbf{Ment} 29
        \pari \textbf{WR} Physical 78, Energy 88 \monsep
        \textbf{VR} Physical 163, Energy 173
        \pari \textbf{Strike:}
            Greatmace \plus24 (19d10 bludgeoning; Impact)
      \end{spelltargetinginfo}
    \end{spellcontent}
    \begin{monsterfooter}
      \pari \textbf{Speeds} Fly 60 ft., Land 30 ft. \monsep
        \textbf{Space} 5 ft. \monsep
        \textbf{Reach} 5 ft.
      \pari \textbf{Awareness} \plus10
      \pari \textbf{Attributes}
        Str 22, Dex 21,
        Con 21, Int 21,
        Per 21, Wil 22
      \pari \textbf{Accuracy} 24 \monsep
        \textbf{Power} 41
      \pari \textbf{Alignment} Always good
    \end{monsterfooter}
  \end{monsubsection}
  
  
  
        \section{Undead}

        All undead have the following properties unless noted otherwise in their description:
        
    
    \parhead{Defenses}
    \plus2 Armor,
    \plus2 Fortitude,
    \plus3 Reflex,
    \plus4 Mental
  
    \parhead{Unliving} Undead are not living creatures, and are immune to most abilities that only affect living creatures.
    However, they are affected in a specific way by effects from the \sphere{vivimancy} mystic sphere.
    Undead are always considered an \glossterm{ally} for \sphere{vivimancy} spells.
    Any effect from the \sphere{vivimancy} sphere that would directly cause an undead creature to lose hit points without dealing damage causes that creature to regain that many lost hit points instead.
    In addition, any effect from the \sphere{vivimancy} sphere that would cause an undead creature to regain lost hit points instead causes it to lose that many hit points instead.
    \parhead{Unnatural Mind} Undead are controlled by fragments of the souls of deceased creatures.
    Many undead are \glossterm{mindless}, and even intelligent undead are immune to \glossterm{Compulsion} and \glossterm{Delusion} abilities.
  
      
  \begin{monsection}{Allip}{3}[4]
    \vspace{-1em}\spelltwocol{}{Medium undead}\vspace{-1em}
    \vspace{0em}

    
      An allip is the spectral remains of someone driven to suicide by a madness that afflicted it in life.
      It craves only revenge and unrelentingly pursues those who tormented it in life and pushed it over the brink.

      An allip cannot speak intelligibly.
    
    

    \begin{spellcontent}
      \begin{spelltargetinginfo}
        \pari \textbf{HP} 36 \monsep
          \textbf{AD} 11 \monsep
          \textbf{Fort} 8 \monsep
          \textbf{Ref} 12 \monsep
          \textbf{Ment} 12
        \pari \textbf{WR} 4 \monsep
        \textbf{VR} 17
        
      \end{spelltargetinginfo}
    \end{spellcontent}
    \begin{monsterfooter}
      \pari \textbf{Speed} Fly 30 ft. \monsep
        \textbf{Space} 5 ft. \monsep
        \textbf{Reach} 5 ft.
      \pari \textbf{Awareness} \plus7
      \pari \textbf{Attributes}
        Str N/A, Dex 5,
        Con N/A, Int 2,
        Per 2, Wil 4
      \pari \textbf{Accuracy} 6 \monsep
        \textbf{Mundane Power} 3 \monsep
      \textbf{Magical Power} 5
      \pari \textbf{Alignment} Always neutral evil
    \end{monsterfooter}
  \end{monsection}
  \begin{freeability}{Draining Touch}
      \target{One living creature within \glossterm{reach}}
       The allip makes a \plus6 attack
        vs. Reflex against the target.
    
    \hit The target loses a \glossterm{hit point}.
    \crit The target loses two \glossterm{hit points}.
    \end{freeability}
  
      \parhead{Babble} 
          During each \glossterm{action phase}, the allip makes an attack vs. Mental against each creature
          within an \arealarge radius \glossterm{emanation} from it.
          It cannot make this attack more than once against any individual target between \glossterm{short rests}.
          \hit Each target is \glossterm{dazed} as a \glossterm{condition}.
        
    \parhead{Incorporeal} 
      The allip has no physical body.
      It makes no sound while moving, and may be unaffected by other abilities that only affect corporeal creatures, such as \glossterm{tremorsense}.
      It is immune to \glossterm{physical} damage and all \glossterm{mundane} abilities that do not deal damage.
      Whenever it would take damage, it has a 50\% chance to take no damage instead.

      The allip can enter or pass through solid objects, but it must remain adjacent to the object's exterior at all times.
      While completely inside a solid object, the object provides \glossterm{total cover}, so it must emerge from the object to attack or be attacked.
  
