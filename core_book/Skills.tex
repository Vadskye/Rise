\chapter{Skills}\label{Skills}

Skills represent the myriad of talents that people can have, such as cooking or swimming.
This chapter describes each skill, including common uses for those skills.

As a reminder, the GM should feel especially free to improvise or expand on the rules governing skills.
Unlike maneuvers or spells, skills have a broad and ambiguous purview.
If it feels relevant to the skill, it's generally fine to allow a check to make a player's action succeed, especially for fun-sounding improvisations.

\section{Skill Overview}
    This section desribes how you use skills.
    For deatails about how you become trained with particular skills, see \pcref{Trained Skills}.

        \begin{dtable!*}
            \lcaption{Class Skills}
            \begin{dtabularx}{\textwidth}{l *{11}{>{\ccol}X} >{\ccol}p{4em}}
                \tb{Skill}        & \tb{Bbn} & \tb{Clr} & \tb{Drd} & \tb{Ftr} & \tb{Mnk} & \tb{Pal} & \tb{Rgr} & \tb{Rog} & \tb{Sor} & \tb{War} & \tb{Wiz} & \tb{Key Ability} \tableheaderrule
                Climb             & C        & \tdash   & C        & C        & C        & \tdash   & C        & C        & \tdash   & \tdash   & \tdash   & Str          \\
                Jump              & C        & \tdash   & C        & C        & C        & \tdash   & C        & C        & \tdash   & \tdash   & \tdash   & Str          \\
                Swim              & C        & \tdash   & C        & C        & C        & \tdash   & C        & C        & \tdash   & \tdash   & \tdash   & Str          \\
                Balance           & C        & \tdash   & C        & C        & C        & \tdash   & C        & C        & \tdash   & \tdash   & \tdash   & Dex          \\
                Flexibility       & C        & \tdash   & \tdash   & C        & C        & \tdash   & \tdash   & C        & \tdash   & \tdash   & \tdash   & Dex          \\
                Perform           & \tdash   & \tdash   & \tdash   & \tdash   & C        & \tdash   & \tdash   & C        & \tdash   & \tdash   & \tdash   & Dex          \\
                Ride              & C        & \tdash   & \tdash   & C        & \tdash   & C        & \tdash   & \tdash   & \tdash   & C        & \tdash   & Dex          \\
                Sleight of Hand   & \tdash   & \tdash   & \tdash   & \tdash   & \tdash   & \tdash   & \tdash   & C        & \tdash   & \tdash   & \tdash   & Dex          \\
                Stealth           & \tdash   & \tdash   & \tdash   & \tdash   & C        & \tdash   & C        & C        & \tdash   & \tdash   & \tdash   & Dex          \\
                Endurance         & C        & \tdash   & C        & C        & C        & C        & C        & \tdash   & \tdash   & \tdash   & \tdash   & Con          \\
                Craft             & C        & C        & C        & C        & C        & C        & C        & C        & C        & C        & C        & Int          \\
                Deduction         & \tdash   & C        & C        & \tdash   & C        & C        & C        & C        & C        & C        & C        & Int          \\
                Devices           & \tdash   & \tdash   & \tdash   & \tdash   & \tdash   & \tdash   & \tdash   & C        & \tdash   & \tdash   & \tdash   & Int          \\
                Disguise          & \tdash   & \tdash   & \tdash   & \tdash   & \tdash   & \tdash   & \tdash   & C        & \tdash   & \tdash   & \tdash   & Int          \\
                Knowledge         & \tdash   & C        & \tdash   & \tdash   & C        & \tdash   & \tdash   & \tdash   & C        & C        & C        & Int          \\
                Linguistics       & \tdash   & C        & \tdash   & \tdash   & \tdash   & \tdash   & \tdash   & C        & C        & C        & C        & Int          \\
                Medicine          & C        & C        & C        & \tdash   & C        & C        & C        & \tdash   & \tdash   & \tdash   & \tdash   & Int          \\
                Awareness         & C        & C        & C        & C        & C        & C        & C        & C        & C        & C        & C        & Per          \\
                Creature Handling & C        & \tdash   & C        & \tdash   & C        & C        & C        & \tdash   & \tdash   & \tdash   & \tdash   & Per          \\
                Deception         & C        & C        & C        & C        & C        & C        & C        & C        & C        & C        & C        & Per          \\
                Persuasion        & C        & C        & C        & C        & C        & C        & C        & C        & C        & C        & C        & Per          \\
                Social Insight    & \tdash   & C        & \tdash   & \tdash   & C        & C        & \tdash   & C        & \tdash   & C        & \tdash   & Per          \\
                Survival          & C        & \tdash   & C        & \tdash   & C        & \tdash   & C        & \tdash   & \tdash   & \tdash   & \tdash   & Per          \\
                Intimidate        & C        & C        & C        & C        & C        & C        & C        & C        & C        & C        & C        & Varies\fn{1} \\
                Profession        & C        & C        & C        & C        & C        & C        & C        & C        & C        & C        & C        & Varies\fn{1} \\
            \end{dtabularx}
            C: class skill. \\
            1. Any attribute could apply depending on how the skill is used. \\
        \end{dtable!*}

    \subsection{Trained Skills}\label{Trained Skills}
        You are \glossterm{trained} in certain skills.
        Your \glossterm{class} gives you a certain number of trained skills from among the \glossterm{class skills} for that class.
        The class skills for each class are summarized in \tref{Class Skills}.

        There are other ways to become trained in skills that are not part of your class.
        If your Intelligence is positive, you gain additional trained skills equal to your Intelligence.
        Some abilities can grant additional trained skills.

    \subsection{Skill Modifiers}
        If you are \glossterm{untrained} in a skill, your bonus with that skill is equal to its associated attribute (if any).
        If you are \glossterm{trained} in a skill, your bonus with that skill is equal to 3 \add half your level \add its associated attribute (if any).
        Many abilities can increase or decrease your bonus with particular skills.

    \subsection{Tasks}\label{Tasks}
        Each skill contains a brief description of how the skill is usually used.
        This description is followed by a series of \glossterm{tasks}, which are particular ways to use skills.
        These tasks are simply examples, and do not list everything the skill can be used for.
        You should be creative with your skills, rather than only using the tasks explicitly listed.

\newpage
\skill{Awareness}{Per}
    The Awareness skill represents your ability to observe things which you might otherwise fail to notice.
    It can be used to observe hidden creatures and traps, as well as to identify fleeting or subtle sensations. 

    The Awareness skill governs the result regardless of the specific sense or senses used.
    It is most commonly used with sight and hearing, though other senses can be used, such as smell or touch.
    Whenever you make an Awareness check, you roll only once, and most creatures have the same Awareness modifier with all of their senses.
    However, the \glossterm{difficulty value} of the check, and the information granted by success, can be very different between senses.
    For example, it is impossible to see through walls or without light, but that does not make hearing impossible.

    \subsection{Common Awareness Tasks}

        \parhead{Identify Disguise} If you succeed at an opposed Awareness vs. Disguise check, you can determine whether a creature or object is disguised.
        \parhead{Identify Forgery} If you succeed at an opposed Awareness vs. Craft check, you can determine whether an object is a forgery.
        \parhead{Notice Hidden Creature} If you succeed at an opposed Awareness vs. Stealth check, you can notice a hidden creature (see \pcref{Hide}).
        Success with a sight-based Awareness check means you can see the creature perfectly.
        Success with any other sense just means you know its exact location, and are still \partiallyunaware of it.
        \parhead{Notice Hidden Object} If you succeed at an opposed Awareness vs. Craft or Devices check, you can notice hidden objects such as traps and secret doors.
        \parhead{Notice Magic Trap} If you succeed at an Awareness check, you can notice hidden magical effects such as traps.
        The \glossterm{difficulty value} to is equal to 15 \add twice the \glossterm{rank} of the ability.
        \parhead{Notice Sleight of Hand} If you succeed at an opposed Awareness vs. Sleight of Hand check, you can notice a creature's attempt to use the Sleight of Hand skill.
        \parhead{Notice Subtle Effect} If you succeed at an Awareness check, you can notice the general effects of a \abilitytag{Subtle} ability affecting you.
        The \glossterm{difficulty value} to notice the effect when it is first applied to you is 15 \add twice the \glossterm{rank} of the ability.
        In addition, you can make another check when the ability ends to notice that you feel normal again.

    \subsection{Common Awareness Modifiers}
        While sleeping, you take a \minus10 penalty to the Awareness skill.
        You gain a \plus20 bonus to notice the presence of creatures and events that directly touch or damage you, such as a creature shoving you or making a \glossterm{strike} against you.

        There are three common circumstances that can make Awareness checks more difficult: obstructions, distance, and similar background sensations.
        Minor obstructions, short distances, and slightly similar backgrounds increase the \glossterm{difficulty value} by 2.
        Significant obstructions, long distances, and very similar backgrounds increase the DV by 5 or more.
        If a sensation is difficult to detect for multiple reasons, the difficulty modifiers stack.

        The \glossterm{difficulty value} of non-opposed checks changes depending on the size of the sensation.
        The difficulty value increases by 5 for each size category larger than Medium, and decreases by 5 for each size category smaller than Medium.
        Multiple sensations of the same type can also be treated as a single larger sensation, which makes them easier to detect.
        Non-visual sensations may not have a literal size category to rely on, so the GM can decide how this modifier applies.

\newpage
\skill{Balance}{Dex}
    The Balance skill represents your ability to maintain your balance and poise on unsteady surfaces.
    The base \glossterm{difficulty value} to balance on a normal surface is 0.
    Generally, creatures only have to roll Balance checks if the surface is unsteady for some reason.

    \subsection{Common Balance Tasks}
        \parhead{Agile Charge}\label{Agile Charge} You can make a \glossterm{difficulty value} 15 Balance check while using the \ability{charge} ability to change directions while charging (see \pcref{Charge}).
        Success means you can make a single turn of up to 90 degrees during the movement.
        \parhead{Maintain Balance} Whenever you take damage while on an unsteady surface, you must make a Balance check based on the surface.
        Failure means that you fall prone.
        \parhead{Rapid Stand}\label{Rapid Stand} You make a \glossterm{difficulty value} 15 Balance check as a \glossterm{minor action} to stand up from a prone position.
        Success means you stand up.
        \parhead{Walk While Balancing} When you move using your \glossterm{land speed} on an unsteady surface, you must make an Balance check based on the surface.
        If you choose to move at half speed, you gain a \plus5 bonus to the check.
        Success means you move along the surface.
        Critical failure means you fall prone.

    \subsection{Common Balance Modifiers}
        There are four common circumstances that make Balance checks more difficult: slippery, mobile, narrow, and uneven surfaces.
        Slightly impaired surfaces increase the \glossterm{difficulty value} by 2.
        Significantly impaired surfaces increase the DV by 5 or more.
        If a surface is impaired for multiple reasons, the difficulty modifiers stack.

\newpage
\skill{Climb}{Str}
    The Climb skill represents your ability to climb obstacles.
    A creature that is climbing without a \glossterm{climb speed} takes a \minus4 penalty to its \glossterm{accuracy} and Armor and Reflex defenses.

    \subsection{Common Climb Tasks}
        \parhead{Creature Climb} As a standard action, you can make a Climb vs. Reflex attack against a creature adjacent to you.
        This requires one \glossterm{free hand}, and the target must be two or more size categories larger than you.
        On a hit, you latch onto the target and can climb on it as if it was a surface with a \glossterm{difficulty value} equal to its Reflex defense.
        This can allow you to use \abilitytag{Size-Based} abilities against the creature as if you were one size category larger than normal.
        The creature takes a \minus4 penalty to \glossterm{accuracy} with \glossterm{strikes} against you.
        \parhead{Grab Surface} You can make a Climb check as part of movement to grab a surface that you are passing by.
        The \glossterm{difficulty value} is 10 higher than normal for the surface if you were moving for reasons out of your control (such as if you are falling).
        Success means you grab onto the wall and interrupt your movement.
        This does not prevent you from taking \glossterm{falling damage} appropriate for the distance you fell.
        \parhead{Maintain Hold} Whenever you take damage while climbing on a surface, suddenly acquire significantly more weight (such as by catching a falling character), or otherwise are significantly distracted, you must make a Climb check based on the surface.
        Failure means you fall off of the surface, and are \prone when you land.
        \parhead{Move} You can make a Climb check as a \glossterm{move action} while you are touching a solid surface.
        This requires two \glossterm{free hands}, or one free hand if you take a \minus5 penalty.
        The \glossterm{difficulty value} is based on the surface.
        Success means that you move along the surface, up to a maximum distance equal to the vertical size of your space (see \pcref{Size Categories}).
        Critical success means the maximum distance you can move is doubled.
        Critical failure means you fall.
        \parhead{Wallrun} You can make a Climb check as part of your movement while you are touching a solid surface.
        The \glossterm{difficulty value} is 10 higher than normal for the surface.
        Success means you can move using your \glossterm{land speed} along the wall during the current phase.
        You move at half speed while going up.
        Failure means you fall.
        For every phase in which you use this ability on the same wall without reaching a stable stopping point, the DV increases by 5.

    \subsection{Climb Speed}\label{Climb Speed}
        Some creatures have a listed climb speed.
        A creature with a passive climb speed must still make a Climb check to climb on surfaces.
        However, the distance it can move if it succeeds on the Climb check is equal to its listed climb speed, regardless of its size or whether it gets a critical success.

\newpage
\skill{Craft}{Int}
    The Craft skills represent your ability to construct objects from raw materials.
    Like Knowledge, Perform, and Profession, Craft is actually a number of separate skills.
    You could have several Craft skills, each with a separate degree of training.
    Common Craft skills are listed below, with additional description for some skills.

    % TODO: note which skills often have magic items?
    \begin{itemize}
        \item Alchemy (Alchemist's fire, tanglefoot bags, potions)
        \item Bone
        \item Ceramics (Glass, pottery)
        \item Jewelry (gemcutting, amulets, rings)
        \item Leather
        \item Manuscripts (Books, official documents, scrolls)
        \item Metal
        \item Poison
        \item Stone
        \item Textiles (Cloth, fabric)
        \item Traps
        \item Wood
    \end{itemize}

    A Craft skill is specifically focused on physical objects. If nothing can be created by an endeavor, it probably falls under the heading of a Profession skill. Complex structures, such as buildings or siege engines, may require Knowledge (engineering) in addition to an appropriate Craft skill.

    \subsection{Common Craft Tasks}
        \parhead{Create Item} You can make a Craft check to create an item.
        For details, see \pcref{Crafting Items}.
        \parhead{Create Forgery} You can make a Craft check to create a false or defective version of an item.
        This functions like creating the item normally, except that you treat the item's \glossterm{rank} as being one lower than it actually is (to a minimum of 0).
        Forgeries which have a function, such as a weapon, are always defective in some way which makes them unsuitable for sustained usage.
        However, a forgery may function once or twice to pass cursory inspection.
        \parhead{Identify Forgery} If you succeed at an opposed Craft vs. Craft check, you can determine whether an object is a forgery.
        \parhead{Identify Item} You can make a Craft check to identify any unusual properties or functions of a magic item or esoteric mundane item.
        The \glossterm{difficulty value} is equal to 5 \add twice the item's \glossterm{rank}.
        Items that are particularly common in a particular setting may be easier to identify, which can reduce the \glossterm{difficulty value} by 2 or more.
        Success means that you know the item's general purpose, and how to activate its functions, including any magical effects.
        You also know the item's rank, which lets you estimate its value.
        \parhead{Repair Item} You can make a Craft check to repair a broken item. This functions like creating the item normally, except that you treat the item's \glossterm{rank} as being two lower than it actually is (to a minimum of 0).

    \subsection{Crafting Items}
        You can use the Craft skill to create an item by expending time and material components. Creating an item often requires multiple consecutive Craft checks. Success on a check means you make progress on completing the item. Failure means you failed to make progress, but can try again without penalty. Critical failure means you botched the item, negating all progress and ruining the material components.

        Each item takes a certain amount of working time to craft, as shown on \tref{Craft Requirements}.
        In addition, each item requires the expenditure of raw materials with a value equal to an item one rank lower than the item you are trying to craft (minimum rank 0).
        Note that raw materials for some items, particularly alchemical items, may be hard to come by in less civilized areas.

        You can attempt to craft items from inferior or ad-hoc materials.
        The materials do not have to be well-suited to the item's construction, but they must be physically capable of performing any necessary actions.
        For example, you could construct a simple arrow-throwing trap from bent sticks or creatively strung rope, but not from sand.
        Generally, using ill-suited materials increases the \glossterm{difficulty value} of the Craft check by at least 5, and it may negatively impact the item's function or longevity.

        In order to craft an item, you must make a Craft check against the item's Craft \glossterm{difficulty value}, as shown on \trefnp{Craft Requirements}.
        If you succeed, you make progress on the item based on how long you spend crafting.
        For every 5 points by which you beat the Craft check, you accomplish twice as much work in the same amount of time.
        Once your total effective working time exceeds the time required to craft the item, you have finished the item.

        All item creation requires artisan's tools to give the best chance of success. If improvised tools are used, the check may be made with a penalty of \minus5 or greater, or it may be impossible, depending on the tools available and the item to be crafted. For example, crafting a bow with improvised woodworking tools would impose a \minus5 penalty, but cutting a diamond without specialized tools is impossible.

        The time and difficulty involved in crafting an item depends on the item's rank, as defined in \trefnp{Craft Requirements}.

        \begin{dtable}
            \lcaption{Craft Requirements}
            \begin{dtabularx}{\columnwidth}{>{\lcol}X l l l}
                \tb{Item}                   & \tb{Subskill} & \tb{DV}            & \tb{Crafting Time}\fn{1} \tableheaderrule
                Alchemical item             & Alchemy       & 5 \add twice rank  & Eight hours        \\
                Body armor                  & Varies        & 10 \add twice rank & One month per rank\fn{2} \\
                Exotic weapon               & Varies        & 10 \add twice rank & 24 hours per rank  \\
                Shield or non-exotic weapon & Varies        & 5 \add twice rank  & 24 hours per rank  \\
                Poison                      & Poison        & 5 \add twice rank  & Eight hours        \\
                Other item                  & Varies        & 5 \add twice rank  & 24 hours per rank  \\
            \end{dtabularx}
            1. For the purpose of crafting times, treat rank 0 items as having a rank of 1/2.
            2. Assuming eight-hour working days for each day of the month.
        \end{dtable}

\newpage
\skill{Creature Handling}{Per}
    % TODO: is non-sapient the right word?
    The Creature Handling skill represents your ability to influence non-sapient creatures.
    With it, you can convince them to do what you want or train them to follow commands.
    This skill has no effect on creatures with an Intelligence of \minus5 or higher.

    \subsection{Common Creature Handling Tasks}

        \begin{freeability}{Command}[\abilitytag{Auditory}, \abilitytag{Compulsion}, \abilitytag{Sustain} (standard)]
            \label{Command}
            Make a Creature Handling vs. Mental attack against a creature within \rngmed range.
            In addition, choose and state an action that the creature could take.
            \hit The target is unable to take any actions except to use the \textit{total defense} ability (see \pcref{Universal Abilities}).
            \crit The target performs the chosen action if it is physically capable of performing it.
            This can include convincing creatures to perform forced marches and similar activities (see \pcref{Overland Exertion}).
            
            You take a \minus10 accuracy penalty against an actively hostile target.
            You take a \minus5 penalty to accuracy with this attack if the target is not an animal, as normal for Creature Handling attacks and checks.
            If the target is damaged or feels that it is in danger, this effect is automatically ended.
        \end{freeability}
        \parhead{Perform Trained Action} You can make a \glossterm{difficulty value} 5 Creature Handling check to convince an \glossterm{ally} to perform an action it is already trained to perform.
        \parhead{Teach Trick} You can make a Creature handling check to teach an \glossterm{ally} a trick.
        A trick is a specific behavior that generally requires a single-word command, like ``fetch'' or ``stay''.
        A creature can learn a maximum of two tricks per point of Intelligence it has above \minus10.
        Teaching a trick generally takes at least a week of intermittent training.
        Simple tricks have a \glossterm{difficulty value} of 5, while complex tricks have a \glossterm{difficulty value} of 10 or more.

    \subsection{Common Creature Handling Modifiers}
        Animals are easier to handle than other kinds of creatures.
        You take a \minus5 penalty to your Creature Handling skill when using it to affect non-animals.

\newpage
\skill{Deception}{Per}
        The Deception skill represents your ability to lie or otherwise mislead people without being caught.
        Using a Deception check is part of conversation or other actions, so it requires no special action to perform.

    \subsection{Common Deception Tasks}
        \parhead{Blend In} If you succeed at an opposed Deception vs. Awareness check, you can avoid notice among a crowd of similar creatures.
        If you act or look significantly different from the creatures around you, observers gain a bonus to their Awareness check to notice you.
        \parhead{Convey Hidden Message} If you succeed at an opposed Deception vs. Social Insight check, you can convey a hidden message in the guise of an ordinary conversation.
        Failure means that an observer recognizes that a hidden message is being conveyed, while critical failure means that an observer recognizes exactly what the hidden message is.
        In general, you must already have a pre-established code or understanding with the intended recipients of your hidden message so they can grasp its true meaning more easily than outside observers.
        \parhead{Distract} If you succeed at an opposed Deception vs. Social Insight check, you can distract a creature you are talking with.
        This generally makes the distracted creature \glossterm{briefly} \partiallyunaware of you, which can allow you to hide or backstab them.
        Normally, the creature realizes that you deceived them once the distraction ends, which prevents them from being distracted again and may influence their behavior.
        \parhead{Fascinate} If you succeed at an opposed Deception vs. Social Insight check, you can keep attention focused on you during a conversation.
        This generally gives distracted creatures a \minus5 penalty to the Awareness and Social Insight skills to notice anything other than you.
        You repeat this check once per minute, with a cumulative \minus5 penalty for each minute that the distraction has lasted.
        \parhead{Impersonate} If you succeed at an opposed Deception vs. Social Insight check, you can impersonate another creature's mannerisms and speech patterns.
        If you are unable to replicate important aspects of the impersonation, such as the beautiful singing voice of a famous bard, you may suffer a penalty to the Deception check.
        This does not allow you to mimic a creature's appearance, which requires the Disguise skill.
        \parhead{Lie} If you succeed at an opposed Deception vs. Social Insight check, you can lie without giving any indication that you are lying.
        Failure means that the observer recognizes that you are intentionally lying.
        Even if you succeed at this check, you still need the Persuasion skill to believe or take actions based on your lies.
        This check only prevents a creature from recognizing the lie based on your body language and behavior.

        When your overall intention is to mislead or conceal information in a conversation, you may need to make this check even if everything you are saying is technically true.
        Generally, using half-truths and similar trickery instead of bald-faced lies gives you a bonus to your Deception check, but a skilled observer can still see through your ruse.


\newpage
\skill{Deduction}{Int}
    The Deduction skill represents your ability to make logical deductions based on evidence.
    It includes both determining which facts and observations are relevant to use as evidence, and reaching conclusions based on that evidence.
    However, this skill cannot protect you from coming to inaccurate conclusions if you rely on inaccurate or incomplete facts and observations.

    \subsection{Common Deduction Tasks}
        \parhead{Identify Surroundings} You can make a Deduction check as a standard action to understand what aspects of your environment are important and why.
        This may require a successful Awareness check to locate hidden objects or subtle clues.
        \parhead{Reach Conclusion} You can combine information that you know to reach a specific conclusion.
        This may require other checks, such as Knowledge or Awareness checks, to ensure that you have enough information to work with.
        The time required to reach a conclusion can vary dramatically depending on how much evidence you have to work with and how easy the conclusion is to reach.
        You can reach simple conclusions immediately after learning all of the relevant information, but complicated scenarios might require days of study and analysis to eliminate all possibilities.
        In general, sifting through a mixture of helpful and misleading evidence increases the difficulty of the Deduction check and the time required to complete it.

\newpage
\skill{Devices}{Int}
    The Devices skill represents your ability to to manipulate mechanical devices such as locks, traps, and other contraptions.
    With enough skill, you can even manipulate magical devices.
    Each device has a base \glossterm{difficulty value} based on its complexity.
    Some tasks are much easier than others, and modify the difficulty value accordingly.

    Many Devices checks require the use of thieves' tools, which contains items like lockpicks and precision cutting implements.
    If you do not have a proper set of tools, you may be able to improvise from your surroundings.
    Generally, this imposes a penalty of at least \minus5 to the Devices check.

    \subsection{Common Devices Tasks}
        \parhead{Activate Device} You can make a Devices check using thieves' tools to make a device perform a function it was designed to do, even if you lack the normal requirements.
        For example, you could tie or untie a knot, lock or unlock a lock without its key, or activate a trap without using its normal triggering mechanism (hopefully without being in its line of fire).
        \parhead{Analyze Device} You can make a Devices check to study a device and understand how it functions.
        The \glossterm{difficulty value} to analyze a device is 5 lower than the device's base difficulty value.
        \parhead{Break Device} You can make a Devices check using thieves' tools to break a device.
        The device ceases to function in its intended way, and the sabotage is obvious to an observer.
        For example, you could jam a lock so it becomes unlocked and can never be locked again.
        Breaking a trap generally triggers the trap in an unpredictable way, which may be dangerous.
        The \glossterm{difficulty value} to break a device is 2 lower than the device's base difficulty value.
        \parhead{Create Bindings} You can make a Devices check with a \plus5 bonus to create bindings from rope or similar materials.
        Binding a helpless foe in this way generally requires a minute of work, though typing up very large creatures may take longer.
        The Flexibility \glossterm{difficulty value} to escape the binding is equal to your check result.
        \parhead{Improvise} You can make a Devices check to construct ad-hoc devices.
        This functions like creating the item with a Craft check (see \pcref{Crafting Items}), with two exceptions.
        First, the item is flimsy, and it breaks after being used once or twice.
        Second, the time requirement is dramatically reduced.
        It takes five minutes to make a device of up to Tiny size.
        You can make a Small device in the time required to make four Tiny devices, a Medium device in the time required to make four Small devices, and so on.
        \parhead{Remove Device} You can make a Devices check using thieves' tools to fully disable a device and remove it if possible.
        This can allow you to bypass traps without ever triggering them, and even take them with you if they are small and portable.
        Magical traps and large-scale physical traps, such as pit traps, are generally not portable.
        The \glossterm{difficulty value} to remove a device is 5 higher than the device's base difficulty value.

\newpage
\skill{Disguise}{Int}
    The Disguise skill represents your ability to create disguises to conceal the appearance of creatures or objects.
    This skill does not help you act appropriately while disguised; see Perform (acting) and Deception.

    Many Disguise checks require the use of a disguise kit, which contains items like makeup and false beards.
    If you do not have a proper kit, you may be able to improvise from your surroundings.
    Generally, this imposes a penalty of at least \minus5 to the Disguise check.

    \subsection{Common Disguise Tasks}
        \parhead{Camouflage} You can make an opposed Disguise vs. Awareness check to blend into your surroundings and avoid being noticed.
        This generally takes at least a minute of work to prepare your disguise to match your exact surroundings.
        It only protects you from visual observation, so you would generally need the Stealth skill to avoid being heard while moving (see \pcref{Stealth}).
        \parhead{Change Appearance}\label{Change Appearance} 
        You can make an opposed Disguise vs. Awareness check using a disguise kit to change a creature's appearance.
        Generally, applying a disguise takes at least a minute, though complex makeup applications or clothing changes can take much longer.
        You take a penalty to the Disguise check based on how radical your changes are, especially to the creature's basic proportions.
        \parhead{Emulate Appearance} You can make a creature look like a different specific creature.
         This functions like the \textit{change appearance} task, except that the result of your Disguise check can't exceed the result of an Awareness check you or someone helping you made to observe the creature you are trying to emulate.

\newpage
\skill{Endurance}{Con}
    The Endurance skill represents your ability to persevere through physical trials.

    \subsection{Common Endurance Tasks}
        \parhead{Delay Vital Wound}\label{Delay Vital Wound}
        Whenever you gain a \glossterm{vital wound}, you can make \glossterm{difficulty value} 15 Endurance check to \glossterm{briefly} ignore the special effect of the vital wound on you.
        The vital wound still penalizes your future \glossterm{vital rolls}.
        At the end of the next round, you must make this check again.
        The difficulty value increases by 10 for each consecutive round that you ignore the same vital wound.

        You can only delay one of your vital wounds in this way.
        If you gain a new vital wound, you can choose to either delay the new vital wound or continue delaying the old vital wound.
        You can make this choice after learning the \glossterm{vital roll} for the new vital wound.
        The difficulty value to delay a new vital wound starts at 15, regardless of whether you were previously delaying a different vital wound.
        \parhead{Hold Breath}\label{Hold Breath}
        You can make an Endurance check to hold your breath.
        While holding your breath, you must make an Endurance check at the end of every 5 rounds that you spend without taking any actions, or at the end of any round in which you take an action.
        The \glossterm{difficulty value} starts at 0, and increases by 1 for each subsequent check until you breathe in air.
        Failure means that you try to breathe in air, and you gain a \glossterm{vital wound} if there is no air available to breathe.

        Essentially, you can fight while holding your breath for a number of rounds equal to your Endurance modifier with no risk of failure.
        If you stay still, you can hold your breath for a number of minutes equal to half your Endurance modifier with no risk of failure.
        \parhead{Stay Awake} You can make an Endurance check to stay awake beyond healthy limits.
        A typical creature needs a minimum of 6 hours of sleep for every 18 hours spent awake, and a minimum of 50 hours of sleep every week.
        The \glossterm{difficulty value} starts at 5, and increases by 5 for each subsequent check until you catch up on your missed sleep.
        Failure means you gain a \glossterm{vital wound} from \glossterm{subdual damage}.
        You must make another check every 8 hours as long as you are still beyond your normal sleep limits.

    % This is thematically appropriate, but it breaks the rule about not introducing ``shadow'' resources
    % \subsection{Ignore Fatigue}\label{Ignore Fatigue}
    %     You can temporarily ignore your fatigue to try to accomplish your objectives.
    %     Ignoring your fatigue requires a \glossterm{minor action}.
    %     The \glossterm{difficulty value} starts at 10.
    %     If you succeed, you reduce your \glossterm{fatigue penalty} by 1 until the end of the round.
    %     For every 10 points by which you succeed, you reduce your fatigue penalty by an additional 1.
    %     Ignoring your fatigue is a \abilitytag{Swift} ability.
    %     You can only use this ability once per \glossterm{short rest}.

\newpage
\skill{Flexibility}{Dex}
    The Flexibility skill represents your ability to escape bindings and move through small areas by contorting your body.

    \subsection{Common Flexibility Tasks}

        \parhead{Escape Bindings} As a standard action, you can make an Flexibility check to escape physical bindings.
        For simple restraints like nets and manacles, the \glossterm{difficulty value} generally depends on the quality of the restraint.
        For complex restraints like carefully tied rope bindings, make an opposed Flexibility vs. Devices check against the creature that created the restraint.
        \parhead{Escape Grapple} As a standard action, you can make a Flexibility check to escape a grapple.
        For details, see \pcref{Grapple Actions}.
        \parhead{Tight Squeeze} You can make a Flexibility check to squeeze into spaces too small to normally fit you (see \pcref{Squeezing}).
        A \glossterm{difficulty value} 15 check allows you to move in a space that can fit your head and shoulders, but which is too tight to allow normal crawling.
        For a typical human, this means squeezing through a space about a foot and a half in diameter.
        A \glossterm{difficulty value} 25 check allows you to move in a space that can fit your head, but not your shoulders.

\newpage
\skill{Intimidate}{Varies}
    The Intimidate skill represents your ability to intimidate and coerce people into doing what you want.

        \parhead{Choosing an Attribute}
        No attribute is a key attribute for Intimidate.
        However, depending on how you are trying to intimidate creatures, you can add any attribute's base value to your Intimidate check.
        For example, if you intimidate a creature by smashing a table and threatening to smash its head in, you can add your Strength to the Intimidate check.
        On the other hand, if you intimidate a creature by staring it down in a cold fury, you can add your Willpower to the Intimidate check.

    \subsection{Common Intimidate Tasks}
        \parhead{Coerce} You can make an Intimidate check to convince a creature to do what you want. This functions like a Persuasion check to form an agreement with a group, except that you do not apply a relationship modifier (see \pcref{Persuasion}).
        Generally, people dislike being coerced.
        \begin{freeability}{Demoralize}
            Make an Intimidate vs. Mental attack against a creature within \rngmed range.
            \hit The target is \glossterm{briefly} \shaken by you.
        \end{freeability}

\newpage
\skill{Jump}{Str}
    The Jump skill represents your ability to jump.
    All Jump checks are made as part of movement, so they take no special action to perform.

    \subsection{Jumping and Movement}
    Distance moved with Jump checks is considered to be part of your land speed.
    The maximum distance that you can cover with a Jump check is equal to your remaining land speed during the current phase.
    This distance is measured only for the farthest extent that you travel from your starting location, not for a round trip or for the entire distance travelled along the arc of your jump.
    For example, if your land speed is thirty feet and you get a ten-foot running start, you can jump no more than twenty feet forward, or fifteen feet and five feet vertically, and so on.
    If your Jump skill is extremely high and your land speed is low, you may need to use the \ability{sprint} ability to make use of your full jumping potential (see \pcref{Sprint}).

    \subsection{Common Jump Tasks}
        \parhead{Leap} As part of movement, you can make a Jump check to leap into the air.
        Your maximum height is equal to half of your Jump check result, and your maximum forward distance is equal to your Jump check result.
        If you have a running start of at least ten feet before jumping, you can add a quarter of your land sped to your maximum horizontal jump distance.
        \parhead{Mitigate Fall} As you hit the ground after a fall, you can make an Jump check to reduce falling damage.
        A \glossterm{difficulty value} 5 check allows you to treat a fall as if it were 10 feet shorter.
        For every 5 points by which you beat that \glossterm{difficulty value}, you can reduce the falling damage by 10 additional feet.
        \parhead{Share Impact} As a standard action, you can make a precise leap to land on an enemy, forcing them to share your falling damage.
        This functions like the \textit{leap} task, except that you also make a Jump vs. Reflex attack against one creature in a space at the end of your motion, including both your initial jump and your fall afterwards (if any).
        On a hit, the target takes half of the \glossterm{falling damage} that you would normally take based on the height of the jump, ignoring any of your abilities that reduce that damage.
        This does not reduce the falling damage you take, and you cannot combine this ability with the \textit{mitigate fall} ability to reduce your falling damage.

    \subsection{Common Jump Modifiers}
        If you jump off of an unstable surface that you cannot stand on normally, such as a wall, you take a \minus5 penalty to your Jump check.

\newpage
\skill{Knowledge}{Int}
    The Knowledge skills represent your understanding of particular aspects of the world.
    Like the Craft, Profession, and Perform skills, Knowledge actually encompasses a number of separate skills.
    Knowledge represents a study of some body of lore, such an academic or even scientific discipline.
    Typical fields of study are listed below, but the GM may create additional fields or decide that some fields are irrelevant in a particular setting.
        \begin{itemize}
            \item Arcana (arcane spells, dragons, magical beasts)
            \item Dungeoneering (aberrations, caverns, oozes, spelunking, subterranean monsters)
            \item Engineering (architecture, buildings, bridges, fortifications, siege weapons)
            \item Items (magic items, artifacts, constructs)
            \item Local (myths and legends, laws and customs, history, nobility and royalty, nearby monsters)
            \item Nature (nature spells, animals, fae, monstrous humanoids, plants, terrain and climate)
            \item Planes (pact spells, the Primal Planes, the Aligned Planes, the Astral Plane,
                planeforged, magic related to the planes, extraplanar monsters)
            \item Religion (divine spells, undead, deities, mythic history, religious traditions, holy symbols)
        \end{itemize}

    \subsection{Common Knowledge Tasks}
        \parhead{Identify Monster} You can make a Knowledge check to identify a monster and recall its special powers or vulnerabilities.
        Each monster notes in its description the specific information that you learn from a successful Knowledge check.
        In general, the \glossterm{difficulty value} for basic information is equal to 5 \add the monster's level.
        Legendary monsters such as dragons can be much easier to recognize.
        \parhead{Identify Item} You can make a Knowledge check to identify any unusual properties or functions of a magic item or esoteric mundane item.
        The \glossterm{difficulty value} is equal to 5 \add twice the item's \glossterm{rank}.
        Items that are particularly common in a particular setting may be easier to identify, which can reduce the \glossterm{difficulty value} by 2 or more.
        Success means that you know the item's general purpose, and how to activate its functions, including any magical effects.
        You also know the item's rank, which lets you estimate its value.
        \parhead{Recall Information} You can make a Knowledge check to remember information related to your field of study.
        The \glossterm{difficulty value} varies depending on the difficulty of the question (see \pcref{Standard Difficulty Values}).
        \parhead{Identify Magical effect} You can make a Knowledge check to identify the general nature of a magical effect that you observe.
        The \glossterm{difficulty value} is generally equal to 10 \add twice the effect's \glossterm{rank}.
        Unusually obscure or obvious magical effects can have higher or lower difficulty values.

        You must use a Knowledge skill relevant to the magical effect.
        Arcane effects require Knowledge (arcana), divine effects require Knowledge (nature), nature effects require Knowledge (nature), and pact effects require Knowledge (planes).
        In some circumstances, other Knowledge skills could be used if they are directly relevant to the magical effect.
        For example, Knowledge (dungeoneering) could be used to identify many spells from the \sphere{terramancy} \glossterm{mystic sphere}.

\newpage
\skill{Linguistics}{Int}
        The Linguistics skill represents your mastery of spoken and written languages (see \pcref{Languages}).
        Normally, you don't make Linguistics checks to speak or understand languages.
        You either know a language or you don't.
        However, training in Linguistics causes you to learn additional languages, and you can use Linguistics to attempt to decipher unfamiliar languages.

        \parhead{Automatic Languages}
        If you are trained in Linguistics, you learn either two \glossterm{common languages} or one \glossterm{rare language} (see \pcref{Communication and Languages}).

    \subsection{Decipher Script}
        You can make a Linguistics check to decipher writing in an unfamiliar language or a message written in an incomplete or archaic form. The base \glossterm{difficulty value} is 10 for the simplest messages, 15 for standard texts, and 20 or higher for intricate, exotic, or very old writing. In addition, the \glossterm{difficulty value} increases by 5 if you do not know any languages that use the same alphabet as the writing being deciphered. Deciphering the equivalent of a single page of script takes 1 minute.

        Success means you understand the general content of a piece of writing about one page long (or the equivalent). Failure means you fail to understand the writing. Critical failure means you to draw a false conclusion about the text. The check to decipher the writing is made secretly, so that you can't tell whether the conclusion you draw is true or false.

    \subsection{Identify Language}
        You make a \glossterm{difficulty value} 10 Linguistics check to identify the language used in speech or writing, even if you can't understand the language.
        For details about languages, see \pcref{Languages}.

\newpage
\skill{Medicine}{Int}
    The Medicine skill represents your practical understanding of how to tend to the wounds of living creatures.
    In order to use this skill to aid a creature, you must be able to see and touch it.

    \subsection{Accelerate Recovery}\label{Accelerate Recovery}
        You can make a \glossterm{difficulty value} 15 Medicine check to accelerate the recovery of up to four creatures from among yourself and your \glossterm{allies} during a \glossterm{long rest}.
        Success means that each creature removes an additional vital wound (see \pcref{Removing Vital Wounds}).
        For every 10 points by which you succeed, each creature removes an additional vital wound.

        To accelerate a creature's recovery, you need a few items and supplies (bandages, salves, and so on) that are easy to come by in civilized areas.
        You can accelerate the recovery of additional creatures during the same rest by taking a cumulative \minus2 penalty per additional creature.

    \subsection{First Aid}\label{First Aid}
        As a standard action, you can make a Medicine check to prevent yourself or an \glossterm{ally} from dying from a \glossterm{vital wound} with a negative \glossterm{vital roll}.
        The \glossterm{difficulty value} is equal to 0 \add 10 for each point by which the vital roll is below 1.
        Success means that the target treats the \glossterm{vital roll} as a 1 instead of its original value.
        This changes the effect of the vital wound, generally preventing the target from dying.
        For details, see \pcref{Vital Wounds}.

        You can use this ability to treat multiple creatures within your \glossterm{reach}.
        You take a \minus5 penalty to the check for each additional target beyond the first.

    \subsection{Identify Poison or Disease}\label{Identify Poison or Disease}
        You can make a Medicine check to identify a poison or disease currently affecting a creature.
        The \glossterm{difficulty value} is equal to 5 \add twice the \glossterm{rank} of the poison or disease.
        Identifying a poison takes a standard action.
        Identifying a disease takes five minutes of work.

    \subsection{Treat Condition}\label{Treat Condition}
        As a standard action, you can make a Medicine check to treat some specific conditions.
        Success usually means the condition is gone, as indicated by the effect's description.
        A condition cannot be removed by this ability unless says this ability can remove it.

    \subsection{Treat Poison or Disease}
        You can make a Medicine check to treat a creature that is currently poisoned or diseased.
        The next time it is attacked by its current poison or disease, it can use your Medicine check or its Fortitude defense, whichever is higher.
        Treating a poison takes a standard action.
        Treating a disease takes five minutes of work.

\newpage
\skill{Perform}{Dex}
    The Perform skills represent your ability to create particular forms of entertainment.
        Like Craft, Knowledge, and Profession, Perform is actually a number of separate skills.
        You could have several Perform skills, each with a separate degree of training.
        Each of the nine categories of the Perform skill includes a variety of methods, instruments, or techniques, a small list of which is provided for each category below.

        \begin{itemize}
            \item Acting (drama, impersonation, mime)
            \item Comedy (buffoonery, limericks, joke-telling)
            \item Dance (ballet, waltz, jig)
            \item Keyboard instruments (harpsichord, piano, pipe organ)
            \item Oratory (epic, ode, storytelling)
            \item Percussion instruments (bells, chimes, drums, gong)
            \item Singing (ballad, chant, melody)
            \item String instruments (fiddle, harp, lute, mandolin)
            \item Wind instruments (flute, pan pipes, recorder, shawm, trumpet)
        \end{itemize}

        \parhead{Performance Types}
        In general, there are three types of performances: instrumental performances, visual performances, and vocal performances.
        \subparhead{Instrumental Performances} These performances require an instrument of some sort to create sound, which always requires at least one hand to manipulate the instrument.
        They affect all creatures who can hear the performance.
        \subparhead{Visual Performances} These performances require some sort of motion, such as juggling.
        They affect all creatures who can see the performance.
        \subparhead{Vocal Performances} These performances use your voice to create sound.
        These are simpler to create than instrumental performances, since they don't require an instrument.
        However, vocal performances are always in a particular language.
        They affect all creatures who can hear the performance and understand the language the performance is in.

        \parhead{Limitations while Performing}
        While you are performing, your actions are slightly limited.
        It takes a \glossterm{minor action} to initiate and sustain a performance.

        While you are performing, you cannot cast spells.
        In addition, you take a \minus5 penalty to the Perform skill for any other performances.
        This penalty stacks, and applies separately for each simultaneous performance.
        For example, if you were playing a lyre, singing, and juggling balls with your feet, you would take a \minus20 penalty to all three performances.
        These limitations are in addition to any restrictions imposed by your method of performing, such as your hands being occupied playing an instrument.

        You can otherwise act normally while performing, including attacking in combat, if doing so is physically possible.

        \parhead{Performance Time}
        In general, you can maintain a performance for up to an hour.
        After that time, you must take a \glossterm{short rest} before performing again.

    \subsection{Entertain}
        You can make a Perform check to provide entertainment or to show off your skills.

    \subsection{Earn Income}
        You can make a Perform check to practice your trade and make a decent living, earning about half your Perform check result in gold pieces per week of dedicated performance.

\newpage
\skill{Persuasion}{Per}
        The Persuasion skill represents your ability to convince people to think what you want them to.
        Depending on how it is used, it represents a combination of verbal acuity, tact, argumentative ability, grace, etiquette, and personal magnetism.
        Using a Persuasion check usually takes at least a minute of sustained conversation.

        Persuasion checks are usually made against a group. For the purposes of a Persuasion check, a ``group'' consists of creatures who consider themselves to be \glossterm{allies} and who share similar information or backgrounds. For example, in a king's court, you cannot simply influence the king alone; his trusted advisors are also part of the same group. It is possible to influence one group without influencing another. For example, the king and his advisors may be unpersuaded, but the prince may find your arguments compelling. Groups can also consist of a single creature. The game master decides what the groups are.

        The base \glossterm{difficulty value} for a Persuasion check against a group is equal to 5 \add the highest Social Insight of any character in the group or half the highest level of any character in the group, whichever is higher.

        Not all social interactions require Persuasion checks. Much of the time, being extraordinarily persuasive is unnecessary, and creatures can be convinced with normal, inartful conversation and good reasoning. Persuasion checks should only be used when your personal persuasiveness matters.

    \subsection{Compel Belief}\label{Compel Belief}
        As part of conversation, you can make a Persuasion check to cause creatures to believe something you say. If you are lying, you must also make a Deception check to lie. Success means the group believes what you are saying is true. What they choose to do with that information is up to them, however. Failure means they do not believe you, but they do not react poorly; perhaps they simply want more verification. You may be able to try again, depending on their patience. Critical failure means the group reacts poorly, and you may have permanently damaged your credibility with them.

        Your check is modified by how believable your argument is, as well as whether the group has strong feelings about the truth of your story.

        \begin{dtable}
            \lcaption{Believability Modifiers}
            \begin{dtabularx}{\columnwidth}{X l}
                \tb{Description}                                                             & \tb{Difficulty Modifier}  \tableheaderrule
                Expected to be true (``Nothing interesting happened while I was on patrol'') & \minus5         \\
                Plausible (``The mayor is too busy to see you now.'')                        & \plus0          \\
                Unlikely (``The north gate is under attack!'')                               & \plus5          \\
                Extremely unlikely (``The mayor is secretly a werewolf.'')                   & \plus10         \\
                Virtually impossible (``Your husband is secretly a werewolf.'')              & \plus15 or more \\
                Demonstratably untrue (``You are secretly a werewolf.'')                     & \tdash\fn{1}    \\
            \end{dtabularx}
            1 You cannot convince someone of something that is proven to be false.  \\
        \end{dtable}

        \begin{dtable}
            \lcaption{Motivation Modifiers}
            \begin{dtabularx}{\columnwidth}{X l}
                \tb{Description}                                                           & \tb{Difficulty Modifier} \tableheaderrule
                Target wants to believe (``That dress looks lovely on you.'')              & \minus5 \\
                Target does not have strong feelings (``I'm busy.'')                       & \plus0  \\
                Target doesn't want the story to be true (``Your brother is a murderer.'') & \plus5  \\
            \end{dtabularx}
        \end{dtable}

    \subsection{Form Agreement}
        You can make a Persuasion check to convince a group to accept a deal you offer. This can be used to persuade the chamberlain to let you see the king, to negotiate peace between feuding barbarian tribes, or to convince the ogre mages that have captured you that they should ransom you back to your friends instead of twisting your limbs off one by one.

        Success means the group will fulfill their end of the deal. Of course, if you don't fulfill your part, they are likely to react poorly. Failure means they did not accept the deal, but they may propose an alternate arrangement, or you can propose another deal without penalty. Critical failure means that there is virtually no chance to reach an agreement, and the group may become hostile or take other steps to end the conversation. They may have been insulted by how unfair the deal was, or you may have made a critical verbal misstep.

        Your check is modified by your relationship with the group and how favorable the deal is.

        \begin{dtable}
            \lcaption{Relationship Modifiers}
            \begin{dtabularx}{\columnwidth}{>{\lcol}X r}
                \tb{Relationship}                                                                                                                                                                 & \tb{Difficulty Modifier} \tableheaderrule
                Intimate: Someone who with whom you have an implicit trust.
                Example: A lover or spouse.                                                                                                                                                       & \minus15 \\
                Friend: Someone with whom you have a regularly positive personal relationship.
                Example: A long-time buddy or a sibling.                                                                                                                                          & \minus10 \\
                Ally: Someone on the same team, but with whom you have no personal relationship.
                Example: A cleric of the same religion or a knight serving the same king.                                                                                                         & \minus5  \\
                Acquaintance (Positive): Someone you have met several times with no particularly negative experiences. Example: The blacksmith that buys your looted equipment regularly.         & \minus2  \\
                Just Met: No relationship whatsoever.
                Example: A guard at a castle or a traveler on a road.                                                                                                                             & \plus0   \\
                Acquaintance (Negative): Someone you have met several times with no particularly positive experiences. Example: A town guard that has arrested you for drunkenness once or twice. & \plus2   \\
                Opposition: Someone who is part of a group that consistently works against your interests, with whom you have no personal relationship.
                Example: An outlaw (to a law-abiding person), a paladin of law (to an outlaw), or a soldier who fights for a country at war with your country.                                                                                      & \plus5   \\
                Enemy: Someone with whom you have a specifically antagonistic relationship.
                Example: An evil warlord whom you are attempting to thwart, a bounty hunter who is tracking you down for your crimes, or a bandit currently robbing you.                                                          & \plus10  \\
                Nemesis: Someone who has sworn to do you, personally, harm, or vice versa. Example: The brother of a man you murdered in cold blood, or the person who murdered your brother in cold blood.                                                             & \plus15  \\
            \end{dtabularx}
        \end{dtable}
        \begin{dtable*}
            \begin{dtabularx}{\textwidth}{>{\lcol}X r}
                \tb{Risk vs. Reward Judgement (Persuasion)}                                                                                                                                                                                                                                                                                          & \tb{Difficulty Modifier} \tableheaderrule
                Fantastic: The reward for accepting the deal is very worthwhile; the risk is either acceptable or extremely unlikely. The best-case scenario is a virtual guarantee. Example: An offer to pay 10gp for directions to the well-known local tavern.                                                  & \minus15 or more                                                   \\
                Good: The reward is good and the risk is minimal. The target is very likely to profit from the deal. Example: An offer to pay someone twice their normal daily wage to spend their evening in a seedy tavern and later report on everyone they saw there.                  & \minus10                                                           \\
                Favorable: The reward is appealing, but there's risk involved. If all goes according to plan, though, the deal will end up benefiting the target. Example: A request for a mercenary to aid the party in battle against a weak goblin tribe in return for a cut of the money and first pick of the magic items. & \minus5                                                            \\
                Even: The reward and risk more of less even out; or the deal involves neither reward nor risk. Example: A request for directions to a place that isn't a secret.                                                                                                                                                                     & \plus0 \\
                Unfavorable: The reward is not enough compared to the risk involved. Even if all goes according to plan, chances are it will end badly for the target. Example: A request to free a prisoner the target is guarding for a small amount of money.                                                                 & \plus5                                                             \\
                Bad: The reward is poor and the risk is high. The target is very likely to get the raw end of the deal. Example: A request for a mercenary to aid the party in battle against an fearsome dragon for a small cut of any non-magical treasure.                                                                    & \plus10                                                            \\
                Horrible: There is no conceivable way that the proposed plan could end up with the target ahead or the worst-case scenario is guaranteed to occur. Example: An offer to trade a broken sword hilt for a shiny new longsword.                                                                                      & \plus15 or more                                                    \\
            \end{dtabularx}
        \end{dtable*}

    \subsection{Gather Information}
        An evening's time, a few gold pieces for buying drinks and making friends, and a \glossterm{difficulty value} 5 Persuasion check get you a general idea of a city's major news items, assuming there are no obvious reasons why the information would be withheld. The higher your check result, the better the information.

        If you want to find out about a specific rumor, or a specific item, or obtain a map, or do something else along those lines, the \glossterm{difficulty value} for the check is generally at least 5.
        The difficulty depends on how widely known and shared the information you seek is.

\newpage
\skill{Profession}{Varies}
    The Profession skills represent your practical understanding of a particular profession.
        Like Craft, Knowledge, and Perform, Profession is actually a number of separate skills.
        You could have several Profession skills, each with a separate degree of training.
        While a Craft skill represents ability in creating or making an item, a Profession skill represents an aptitude in a vocation requiring a broader range of less specific knowledge.
        Most commoners have some training in a Profession skill.

        \parhead{Choosing an Attribute}
        No attribute is a key attribute for Profession.
        However, depending on how you are using your Profession, you can add any attribute's base value to your Profession check.
        For example, if you use your experience as a farmer to harrow a field, you can add your Strength to the Profession check.
        On the other hand, if you use your experience as a sailor to determine the right angle for sails in the current wind, you can add your Perception to the Profession check.

    \subsection{Identify Item}
        You can make a Profession check to identify any unusual properties or functions of an item.
        The Profession skill used must match the item's purpose, ignoring any magical properties it might have.
        This is most commonly used to identify magic items, but it can also be used to understand esoteric nonmagical items.

        Identifying an item takes a minute of careful evaluation.
        The \glossterm{difficulty value} depends on how obscure the item's nature is.
        As a rough guideline, identifying a magic item usually has a base \glossterm{difficulty value} of 5 \add twice the item's rank.
        Items that are particularly common in a particular setting may be easier to identify, which can reduce the \glossterm{difficulty value} by 5 or more.
        For example, a \mitem{potion of healing} is usually a fairly common magic item, so it may only require a DV 2 check to recognize.
        The GM is responsible for determining exactly which magic items, if any, are common enough to easily identify.

        Success means that you know the item's general purpose, and how to activate its functions, including any magical effects.
        You also know the item's rank, which lets you estimate its value.
        Failure means you learn nothing about the item.

        If an item is intentionally crafted to be misleading or hard to identify, the \glossterm{difficulty value} to identify it increases by 5 or more.
        Critical failure on a check to identify it may cause an observer to misidentify it as a different item.
        For example, a poison carefully contructed to resemble a \mitem{potion of healing} may require a DV 17 check to correctly identify.
        A result of 11 or less would cause an observer to confidently conclude that the poison was in fact a potion of healing.

    \subsection{Earn Income}
        You can make a Profession check to practice your trade and make a decent living, earning about half your Profession check result in gold pieces per week of dedicated work. You know how to use the tools of your trade, how to perform the profession's daily tasks, how to supervise helpers, and how to handle common problems.

    \subsection{Perform Task}
        You can make a Profession check to perform some tasks related to your profession. This allows you to use Profession in place of other skills when it is appropriate. For example, a sailor could use Profession to tie common knots in place of Devices or Survival, or a farmer could use Profession to identify common animals and plants in place of Knowledge (nature). The \glossterm{difficulty value} when using Profession may be higher than it would be to use the normal skill for the task.

\newpage
\skill{Ride}{Dex}
    The Ride skill represents your ability to ride and control horses and other mounts.
    Typical riding actions don't require checks. You can saddle, mount, ride, and dismount from a mount without a problem. However, some special actions require Ride checks. Some modifiers apply to all Ride checks, as described at \pcref{Ride Modifiers}.

    Unless an ability says otherwise, you can only use this skill to ride creatures exactly one size category larger than you.

    \parhead{Ride Modifiers}\label{Ride Modifiers}
    If a mount is not trained as a mount, the \glossterm{difficulty value} to ride it increases by 5.
    If it lacks a saddle and other riding gear, the \glossterm{difficulty value} to ride it increases by 5.
    If it takes a standard action other than movement, such as attacking, the \glossterm{difficulty value} to ride it that round increases by 5.

    \subsection{Control Mount}
        When riding a non-sentient \glossterm{ally} in combat that is not trained for battle, you must a \glossterm{difficulty value} 10 Ride check as a \glossterm{move action} to control it. Success means it obeys your commands that round. Failure means it remains still and refuses to obey your commands until you make a successful check to control it. Critical failure means the mount acts of its own volition.

        Magically augmented animals such as a druid's \textit{natural servant} and a ranger's \textit{animal companion} are considered trained for battle, making this ability unnecessary.

    \subsection{Fall}
        If you fall off your mount, or if your mount is downed in battle, you take damage from the fall.
        Falling off of a typical Large creature, such as a horse, deals 1d6 bludgeoning damage, while falling from larger creatures deals damage appropriate to the distance fallen.

    \subsection{Guide Mount}
        While riding an \glossterm{ally}, you must make a \glossterm{difficulty value} 0 Ride check at the start of your turn to guide it with your knees. Success means it obeys your commands, and you can have both hands free to take other actions. Failure means you must use a hand to control the mount that round. Critical failure means the mount acts of its own volition. If you cannot use a hand to control the mount, you fall off the mount if it moves during your turn.

    \subsection{Leap}
        You can make a \glossterm{difficulty value} 5 Ride check to stay on your mount as it jumps. This check is made as part of your mount's movement. Success means that you stay on the mount. Failure means you fall off your mount as it starts to jump.

    \subsection{Spur Mount}
        You can make a \glossterm{difficulty value} 5 Ride check as a move action to get your mount to move faster. Success means it takes the \textit{sprint} action to move faster (see \pcref{Sprint}).
        Failure means your action was wasted.

    \subsection{Stay in Saddle}
        If you take damage or your mount rears or bolts unexpectedly, you must make a \glossterm{difficulty value} 5 Ride check to stay in your saddle. This does not take an action. Success means you stay in your saddle. Failure means you fall off your mount.

    \subsection{Take Cover}
        You can make a \glossterm{difficulty value} 10 Ride check as a move action to drop low and take \glossterm{cover} behind your mount. This requires the use of both your hands. Failure means you can't get low enough and gain no benefit from the action. Critical failure means you fall off your mount. You can leave this cover and resume your normal position as a move action that does not require a Ride check.

        If you take a \minus5 penalty to this check, you can attempt it with only one free hand.
        If you take a \minus10 penalty to this check, you can attempt it without any free hands.

\newpage
\skill{Sleight of Hand}{Dex}
        The Sleight of Hand skill represents your ability to pick pockets, palm objects, and perform other feats of legerdemain.
        All Sleight of Hand checks apply a special modifier based on the size of the action taken or object affected, as shown on \trefnp{Sleight of Hand Difficulty Modifiers}.

        \begin{dtable}
            \lcaption{Sleight of Hand Difficulty Modifiers}
            \begin{dtabularx}{\columnwidth}{X l}
                \tb{Size}   & {Difficulty Modifier} \tableheaderrule
                Fine        & \minus10   \\
                Diminuitive & \minus5   \\
                Tiny        & \plus0   \\
                Small       & \plus5  \\
                Medium      & \plus10  \\
                Large       & \plus15 \\
                Huge        & \plus20 \\
                Gargantuan  & \plus25 \\
                Colossal    & \plus30 \\
            \end{dtabularx}
        \end{dtable}

        \parhead{Conceal Object} You can make an opposed Disguise vs. Awareness check to conceal an \glossterm{ally} or \glossterm{unattended} object on your person.
        The target must be at least one size category smaller than you are.
        If it is only one size category smaller than you, you take a \minus10 penalty to the check.

    \subsection{Conceal Action}
        You can make a Sleight of Hand check to conceal an action you take, preventing observers from noticing that you took the action. You can conceal any physical action that only requires the use of your hands and arms. Your check is opposed by the Awareness check of any observers. Success means your action is unnoticed. Failure means they are able to notice your action normally.

        The space required to perform the action is the size of the action, and applies a bonus or penalty as noted above. For example, winking only requires moving your eye, so you gain a \plus10 bonus to conceal the action. Firing a longbow requires manipulating an object as large as you are, so you take a \minus10 penalty to conceal the action.

        Observers that you touch as part of the action gain a \plus10 bonus to their Awareness check. If you strike them hard, as with an attack, the bonus increases to a \plus20 bonus.

        If you successfully conceal an attack, the attack is treated as if it were an attack from an invisible creature. The target may be \unaware or \partiallyunaware of the attack. If the target is hit, it can tell the direction the attack came from, but not that you made the attack.

        Nonhumanoid creatures without hands and arms may conceal actions using different parts of their body. Large movements can never be concealed with this ability; see the Stealth skill.

    \subsection{Conceal Object}
        As a standard action, you can make a Sleight of Hand check to conceal a creature or object on your person.
        The target must be at least one size category smaller than you are.
        If it is only one size category smaller than you, you take a \minus10 penalty to the check.
        This penalty is separate from the normal modifier based on the object's size.
        A creature must be an \glossterm{ally}, and an object must be unattended.

        Your Sleight of Hand check is opposed by the Awareness check of anyone observing you.
        Success means they fail to notice the item. A creature directly interacting you to search for the object, such as by frisking you, gains a \plus5 bonus to its Awareness check.

    \subsection{Pickpocket}
        You can make a Sleight of Hand check as a standard action to steal an object from another creature. The object must be loose and accessible, such as in a pocket. The \glossterm{difficulty value} depends on whether the creature notices your attempt using Awareness. If the creature's Awareness check exceeds your Sleight of Hand check, the creature notices your attempt and the \glossterm{difficulty value} is equal to the creature's Reflex defense. Otherwise, the creature does not notice your attempt, and the \glossterm{difficulty value} is 10. Success means you successfully steal the object. Failure means you do not steal the object.

\newpage
\skill{Social Insight}{Per}
        The Social Insight skill represents your ability to read body language and emotion.
        Most Social Insight tasks are \glossterm{hidden tasks}.

    \subsection{Discern Enchantment [Hidden]}\label{Discern Enchantment}
        When you interact with a creature, you can try to notice whether it is affected by mind-affecting abilities with a Social Insight check.
        If the creature is not affected by any such abilities, the check automatically fails.
        If the creature is affected by Compulsion or Emotion effects that are not currently altering its behavior, the check also automatically fails.
        If the creature's behavior is currently being altered by a \abilitytag{Compulsion} effect, the \glossterm{difficulty value} is 10, and success means you identify the presence of a Compulsion effect.
        If the creature's behavior is currently being altered by an \abilitytag{Emotion} effect, the \glossterm{difficulty value} is 20, and success means you identify the presence of an Emotion effect.
        Failure means you do not notice any such effects on the creature.

        You can also make this check to identify \abilitytag{Subtle} effects on yourself, using the same \glossterm{difficulty value}.

    \subsection{Discern Lies [Hidden]}
        When you observe a creature speak, you can make a Social Insight check.
        The \glossterm{difficulty value} is equal to the speaking creature's Deception check result.
        Success means you identify whether the creature was lying.
        Failure means you do not notice any indication that the creature is lying.

    \subsection{Discern Secret Message}
        When you observe a hidden message being conveyed, you can make a Social Insight check.
        The \glossterm{difficulty value} is equal to the \glossterm{difficulty value} of the secret message (see \pcref{Deception}).
        Success means you recognize that a hidden message is present, but not its contents.
        Critical success means you can understand the message.
        Failure means you don't notice the hidden message.

    \subsection{Social Assessment}\label{Social Assessment}
        You can make a \glossterm{difficulty value} 5 Social Insight check to get a general assessment of a social situation.
        This does not take an action by itself, but you must observe the group for at least a minute.
        Success means you learn a piece of useful information about the situation, such as a general understanding of expected behaviors or a rough understanding of the social hierarchy.
        For every 5 points by which you beat the \glossterm{difficulty value}, you gain an additional insight into the situation.

        You can make a social assessment after only a single round of observation, but you take a \minus10 penalty on the check.
        If you don't understand the language the group is using, you take a \minus10 penalty on the check.
        The information gained at a given \glossterm{difficulty value} may vary in usefulness depending on how obvious or subtle the situation is.

\newpage
\skill{Stealth}{Dex}
        The Stealth skill represents your ability to escape detection while moving or taking large-scale actions.
        All Stealth checks are made as part of movement or other actions, so they require no special action to perform. If you have been noticed by a creature, you automatically fail all Stealth checks against that creature until you can escape its notice, such as by disappearing out of sight.

        \parhead{Size and Stealth}\label{Size and Stealth} A creature smaller than Medium size gains a \plus5 bonus to the Stealth skill for each size category by which it is smaller than Medium.
        Similarly, a creature larger than Medium size takes a \minus5 bonus to the Stealth skill for each size category by which it is smaller than Medium.
        These effects are summarized below.
            \begin{itemize}
                \item Fine: \plus20
                \item Diminutive: \plus15
                \item Tiny: \plus10
                \item Small: \plus5
                \item Medium: \plus0
                \item Large: \minus5
                \item Huge: \minus10
                \item Gargantuan: \minus15
                \item Colossal: \minus20
            \end{itemize}

    \subsection{Avoid Notice}
        As part of any movement or \glossterm{standard action}, or as a \glossterm{move action} if you hide in place, you can make a Stealth check to prevent creatures from becoming aware of you as a result of that action.
        Your Stealth check is opposed by the Awareness checks of any creatures who are capable of observing you and who are not already aware of you.
        Success means that the observer's awareness of you does not change.
        If it was originally \unaware of you, it stays unaware of you, and if it was originally \partiallyunaware of you, it stays partially unaware of you.
        This effect lasts until you take an action or your circumstances otherwise meaningfully change in a way that would make you easier to observe.
        Failure means that the observer can observe you using any senses they detected you with.
        Generally, success with sight-based senses causes creatures to become aware of you, while success with other senses causes creatures to be \partiallyunaware of you.

        If you do not have \glossterm{cover} or \glossterm{concealment} from a creature (see \pcref{Cover} and \pcref{Concealment}), that creature gains a \plus20 bonus to Awareness checks to observe you.
        For this purpose, do not consider any cover that would be hidden as a result of a successful check, such as an object you hold in front of you.
        While you are invisible, you are treated as having \glossterm{concealment} for this purpose.
        If a creature can see you regardless of cover or concealment with an ability like \trait{lifesight} or \trait{tremorsight}, you are not considered to have cover or concealment from it, so it gains that \plus20 bonus.

        If a creature would automatically know your location with an ability like \trait{lifesense} or \trait{blindsense}, it gains a \plus10 bonus to Awareness checks to observe you.
        This does not stack with the Awareness bonus for not having cover or concealment.

        Larger, more obvious actions are more difficult to hide.
        If you use a movement speed to move, you take a penalty to your Stealth check to conceal that movement.
        This is a \minus5 penalty if you move at no more than half your speed.
        If you use the \textit{sprint} ability or move faster than half your speed, this penalty increases to \minus10.

        Making a \glossterm{strike}, using \glossterm{somatic components}, and taking other similar large-scale actions imposes a \minus10 penalty to the Stealth check.
        If you make a strike with a medium or heavy weapon, this penalty increases to \minus20.
        This is separate from and stacks with the \plus20 bonus that a creature gets to notice you if you hit it with a \glossterm{strike} (see \pcref{Notice Creatures and Events}).

    \subsection{Blend In}
        You can make a Stealth check to blend in with a crowd. Your Stealth check is opposed by the Awareness checks of anyone looking for you specifically. Success means you remain indistiguishable from any other random member of the crowd. Failure means the person looking for you identifed you.

        If you act differently from the crowd you are trying to blend in with, anyone observing you gets a \plus5 bonus to find you. You may need to make Awareness or Social Insight checks to figure out how to act, depending on the situation. If you are extremely dissimilar from the crowd, such as a naked person in church or a human among halflings, anyone observing you may gain a \plus10 or greater bonus to find you.

    \subsection{Hide [Hidden]}\label{Hide}
        As part of any movement or teleportation, you can make a Stealth check to make creatures that are aware of you lose track of your position.
        Your Stealth check is opposed by the Awareness checks of any creatures who were aware of you before your movement.
        In order to use this ability, you must move in a way that makes observers lose sight of you for at least ten feet of your motion.
        This can be achieved by moving through total darkness, moving out of \glossterm{line of sight}, teleporting at least ten feet, or similar activities.
        Note that most teleportation effects make noise, which means that creatures can detect you by hearing that noise instead of beating your Stealth check.
        In addition, you must have \glossterm{cover} or \glossterm{concealment} for the entire duration of your movement.
        Success means that the observer becomes \partiallyunaware of you instead of fully aware of you.

        If you do not have \glossterm{cover} or \glossterm{concealment} from a creature (see \pcref{Cover} and \pcref{Concealment}), that creature gains gain a \plus20 bonus to sight-based Awareness checks to observe you.
        For this purpose, do not consider any cover that would be hidden as a result of a successful check, such as an object you hold in front of you.

\newpage
\skill{Survival}{Per}
        The Survival skill represents your ability to take care of yourself and others in the wilderness, including the ability to follow tracks.

    \subsection{Navigate Wilderness}
        You can make a Survival check while moving overland to avoid natural hazards and getting lost. The \glossterm{difficulty value} depends on the terrain, as shown on \trefnp{Terrain Difficulty Values}. Success means that you and any group you lead can move without any trouble. Failure means you may encounter a natural hazard, depending on the terrain. Critical failure means you become lost.

        You can notice that you are lost the next time that you succeed on a check to navigate in the wilderness. Rediscovering your location requires backtracking for 1d6 hours and another Survival check against the same \glossterm{difficulty value}.

        This check is made once every 8 hours you spend travelling overland. If you move at half speed, you gain a \plus5 bonus on the check.

    \subsection{Sustenance}
        You can make a Survival check to move overland at half speed while hunting and foraging. The \glossterm{difficulty value} depends on the terrain, as shown on \trefnp{Terrain Difficulty Values}. Success means that you can feed yourself, and require no additional food or water. For every 2 points by which you succeed, you can provide food and water to an additional person.

        This check is made once every 8 hours you spend travelling overland. If you move at one-quarter speed, you gain a \plus5 bonus on the check.

        \begin{dtable}
            \lcaption{Terrain Difficulty Values}
            \begin{dtabularx}{\columnwidth}{l X X}
                \tb{Terrain} & \tb{Navigation Difficulty Value} & \tb{Sustenance Difficulty Value} \tableheaderrule
                Desert       & 10                                & 20 \\
                Forest       & 10                                & 15 \\
                Jungle       & 15                                & 10 \\
                Mountains    & 10                                & 15 \\
                Hills        & 5                                 & 10 \\
                Plains       & 5                                 & 10 \\
                Swamp        & 15                                & 15 \\
            \end{dtabularx}
        \end{dtable}

    \subsection{Predict Weather}
        You can make a \glossterm{difficulty value} 10 Survival check to predict the weather. This requires a minute of observation. Success means you know what the weather will be like up to 24 hours in advance. For every 5 points by which you succeed, you can predict the weather one additional day in advance.

    \subsection{Track}\label{Track}
        As a standard action, you can make a Survival check to follow tracks.
        The \glossterm{difficulty value} of the check depends on how easy the tracks are to notice, as shown on \trefnp{Track Difficulty Values} and \trefnp{Track Difficulty Modifiers}.
        You must use this ability each round to continue following the trail, though you do not have to make an additional Survival check each round.
        You must make another Survival check if you change your movement speed, if you follow the trail for 1 mile, or if it becomes especially difficult to follow for any reason.

        If you move at up to half your normal speed as the same round that you use this ability, you take no penalty on the check.
        If you move at your full speed, you take a \minus5 penalty to the check.

        The \glossterm{difficulty value} depends on the surface and the prevailing conditions, as given on the table below:
        The base \glossterm{difficulty value} to follow tracks is 5 if you use scent to track, regardless of the condition of the ground.

        A creature can use the Awareness skill to notice signs of passage, but the Survival skill is necessary to follow tracks for any distance.

        \begin{dtable}
            \lcaption{Track Difficulty Values}
            \begin{dtabularx}{\columnwidth}{l >{\lcol}X l}
                \tb{Surface}     & \tb{Description}                                                                                                                                                     & \tb{Difficulty Value} \tableheaderrule
                Very soft ground & Any surface (fresh snow, thick dust, wet mud) that holds deep, clear impressions of footprints.                                                                      & 0  \\
                Soft ground      & Any surface soft enough to yield to pressure, but firmer than wet mud or fresh snow, in which a creature leaves frequent but shallow footprints                      & 5  \\
                Firm ground      & Most normal outdoor surfaces (such as lawns, fields, woods, and the like) or exceptionally soft or dirty indoor surfaces (thick rugs and very dirty or dusty floors) & 10 \\
                Hard ground      & Any surface that doesn't hold footprints at all, such as bare rock or a streambed                                                                                    & 15 \\
            \end{dtabularx}
        \end{dtable}

        \begin{dtable}
            \lcaption{Track Difficulty Modifiers}
            \begin{dtabularx}{\columnwidth}{>{\lcol}X l}
                \tb{Condition}                                      & \tb{Difficulty Modifier} \tableheaderrule
                Every three creatures in the group being tracked    & \minus1      \\
                Size of creature or creatures being tracked:\fn{1}  &              \\
                Fine                                                & \plus20      \\
                Diminutive                                          & \plus15      \\
                Tiny                                                & \plus10       \\
                Small                                               & \plus5       \\
                Medium                                              & \plus0       \\
                Large                                               & \minus5      \\
                Huge                                                & \minus10      \\
                Gargantuan                                          & \minus15     \\
                Colossal                                            & \minus20     \\
                Every 24 hours since the trail was made             & \plus1\fn{3} \\
                Every hour of rain since the trail was made         & \plus1       \\
                Fresh snow cover since the trail was made           & \plus10      \\
                \tb{Poor visibility:\fn{2}}                         &              \\
                Overcast or moonless night                          & \plus6       \\
                Moonlight                                           & \plus3       \\
                Fog or precipitation                                & \plus3       \\
                Tracked party hides trail (and moves at half speed) & \plus5
            \end{dtabularx}
            1 For a group of mixed sizes, apply only the modifier for the largest size category. \\
            2 Apply only the largest modifier from this category. \\
            3 With scent-based tracking, apply this modifier per hour since the trail was made. \\
        \end{dtable}

        If you fail a Survival check to track, you can retry after 5 minutes of searching.

    \subsection{Use Rope}
        You can make a Survival check as a standard action to tie knots. Success means you tie the knot successfully. Failure means your action is wasted, but you can try again. Tying a typical knot is \glossterm{difficulty value} 5. Tying a special knot, such as one that slips, slides slowly, or loosens with a tug is \glossterm{difficulty value} 10. Knots can also be tied with the Devices skill.

\newpage
\skill{Swim}{Str}
        The Swim skill represents your ability to swim.
        A creature that is swimming without a \glossterm{swim speed} takes a \minus4 penalty to its \glossterm{accuracy} and Armor and Reflex defenses.
        In addition, ranged weapons have difficulty working underwater.
        All ranged weapons have \glossterm{range limits} of 5/15 when used by a creature that is underwater, or when used against a target that is underwater, regardless of the attack's normal range limits or any other modifiers.

    \subsection{Swimming}
        You can make a Swim check as a \glossterm{move action} while you are in water or some other thick liquid that allows swimming.
        The \glossterm{difficulty value} depends on the turbulence of the water, as shown on \trefnp{Swim Difficulty Values}.
        Since moving in this way is a \glossterm{move action}, you cannot also use other move actions like the \ability{block} or \ability{follow} abilities.

        Success means you can move up to one quarter of the \glossterm{base speed} for your size.
        Critical success means that you can instead move up to half the base speed for your size.
        Failure means your action is wasted and you do not move.
        Critical failure means you sink five feet deeper into the liquid, which can cause you to start drowning.

        If you succeed or critically succeed on a Swim check to move, you can keep using the same result in future phases as long as you continue swimming in the same liquid.
        However, if you are attacked or otherwise significantly distracted, you must make a new roll for any subsequent movement.

        \begin{dtable}
            \lcaption{Swim Difficulty Values}
            \begin{dtabularx}{\columnwidth}{>{\lcol}X >{\lcol}X}
                \tb{Water}   & \tb{Difficulty Value} \tableheaderrule
                Calm water   & 5  \\
                Rough water  & 10 \\
                Stormy water & 15 \\
            \end{dtabularx}
        \end{dtable}

    \subsection{Swimming Underwater}
        You can swim underwater just like you can above water.
        If you are underwater, either because you failed a Swim check or because you are swimming underwater intentionally, you must hold your breath (see \pcref{Hold Breath}).

    \subsection{Swim Speed}\label{Swim Speed}
        Some creatures have a listed swim speed.
        A creature with a passive swim speed must still make a Swim check to swim through liquid.
        However, the distance it can move if it succeeds on the Swim check is equal to its listed swim speed, regardless of its size or whether it gets a critical success on the check.
