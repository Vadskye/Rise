\chapter{Skills}\label{Skills}

Skills represent the myriad of talents that people can have, such as cooking or swimming.
This chapter describes each skill, including common uses for those skills.

\section{Skill Overview}

    This section desribes how you acquire and use skills.

    \subsection{Skill Points}\label{Skill Points}

        At 1st level, you gain eight skill points.
        Skill points can be spent to improve your abilities with particular skills (see Skill Training, below).
        Unless otherwise noted, skill points can be spent to improve any skill.

        You gain additional skill points equal to twice your starting Intelligence.
        If your Intelligence is negative, you similarly lose skill points equal to twice your starting Intelligence.
        Some other abilities, such as the rogue \textit{skill lore} ability, can grant additional skill points (see Skill Lore, page \pref{Rog:Skill Lore}).

    \subsection{Skill Training}\label{Skill Training}

        You can spend skill points to become \glossterm{trained} or \glossterm{mastered} in skills.
        Your level of training determines your base modifier for attacks and checks using that skill, as described below.
        If you are already \glossterm{trained} in a skill, you only pay the difference in skill point costs to become \glossterm{mastered} in that skill.

        \begin{itemize}
            \itemhead{Untrained} Becoming untrained in a skill costs no skill points.
                You are untrained in all skills by default.
                Your modifier with an untrained skill is equal to half the skill's \glossterm{key attribute}.
                If the skill does not have a key attribute, your modifier is \plus0.
            \itemhead{Trained} Becoming trained in a skill costs one skill point.
                Your modifier with a trained skill is equal to either the skill's \glossterm{key attribute} (if any) or 1 \add half your level, whichever is higher.
            \itemhead{Mastered} Mastering a skill costs three skill points, or two skill points if the skill is a \glossterm{class skill} for you.
                Your modifier with a mastered skill is equal to 4 \add either the skill's key attribute (if any) or your level, whichever is higher.
        \end{itemize}

    \subsection{Skill Modifier}\label{Skill Modifier}

        Your bonus with a skill is calculated as follows:

        \begin{figure}[h]
            \centering Training modifier (see \pcref{Skill Training}) \add other bonuses and penalties
        \end{figure}

        \subparhead{Bonuses and Penalties} Species abilities, class abilities, penalties from \glossterm{encumbrance}, and other effects can increase or decrease your bonus with a skill.

    \subsection{Class Skills}

        The class skills for each class are summarized on \trefnp{Class Skills}.
        \begin{dtable!*}
            \lcaption{Class Skills}
            \begin{dtabularx}{\textwidth}{>{\lcol}p{12em} *{11}{>{\ccol}X} >{\ccol}p{4em}}
                \tb{Skill}        & \tb{Bbn} & \tb{Clr} & \tb{Drd} & \tb{Ftr} & \tb{Mge} & \tb{Mnk} & \tb{Pal} & \tb{Rgr} & \tb{Rog} & \tb{War} & \tb{Key Ability} \tableheaderrule
                Climb             & C        & \tdash   & C        & C        & \tdash   & C        & \tdash   & C        & C        & \tdash   & Str              \\
                Jump              & C        & \tdash   & C        & C        & \tdash   & C        & \tdash   & C        & C        & \tdash   & Str              \\
                Swim              & C        & \tdash   & C        & C        & \tdash   & C        & \tdash   & C        & C        & \tdash   & Str              \\
                Acrobatics        & C        & \tdash   & C        & C        & \tdash   & C        & \tdash   & C        & C        & \tdash   & Dex              \\
                Escape Artist     & \tdash   & \tdash   & \tdash   & C        & \tdash   & C        & \tdash   & \tdash   & C        & \tdash   & Dex              \\
                Ride              & \tdash   & \tdash   & \tdash   & C        & \tdash   & \tdash   & C        & \tdash   & \tdash   & C        & Dex              \\
                Sleight of Hand   & \tdash   & \tdash   & \tdash   & \tdash   & \tdash   & \tdash   & \tdash   & \tdash   & C        & \tdash   & Dex              \\
                Stealth           & \tdash   & \tdash   & \tdash   & \tdash   & \tdash   & C        & \tdash   & C        & C        & \tdash   & Dex              \\
                Craft             & \tdash   & C        & C        & C        & C        & C        & C        & C        & C        & C        & Int              \\
                Deduction         & \tdash   & C        & C        & \tdash   & C        & C        & C        & C        & C        & C        & Int              \\
                Devices           & \tdash   & \tdash   & \tdash   & \tdash   & \tdash   & \tdash   & \tdash   & \tdash   & C        & \tdash   & Int              \\
                Disguise          & \tdash   & \tdash   & \tdash   & \tdash   & \tdash   & \tdash   & \tdash   & \tdash   & C        & \tdash   & Int              \\
                Heal              & \tdash   & C        & C        & \tdash   & \tdash   & C        & C        & C        & \tdash   & \tdash   & Int              \\
                Knowledge         & \tdash   & C        & \tdash   & \tdash   & C        & C        & \tdash   & \tdash   & \tdash   & C        & Int              \\
                Linguistics       & \tdash   & C        & \tdash   & \tdash   & C        & \tdash   & \tdash   & \tdash   & C        & C        & Int              \\
                Awareness         & C        & C        & C        & C        & C        & C        & C        & C        & C        & C        & Per              \\
                Creature Handling & C        & \tdash   & C        & \tdash   & \tdash   & \tdash   & C        & C        & \tdash   & \tdash   & Per              \\
                Sense Motive      & \tdash   & C        & \tdash   & \tdash   & \tdash   & \tdash   & C        & \tdash   & C        & C        & Per              \\
                Spellcraft        & \tdash   & C        & C        & \tdash   & C        & C        & \tdash   & \tdash   & \tdash   & C        & Per              \\
                Survival          & C        & \tdash   & C        & \tdash   & \tdash   & C        & \tdash   & C        & \tdash   & \tdash   & Per              \\
                Intimidate        & C        & C        & C        & C        & C        & C        & C        & C        & C        & C        & Varies\fn{1}     \\
                Perform           & \tdash   & \tdash   & \tdash   & \tdash   & \tdash   & C        & \tdash   & \tdash   & C        & \tdash   & Varies\fn{1}     \\
                Profession        & C        & C        & C        & C        & C        & C        & C        & C        & C        & C        & Varies\fn{1}     \\
                Bluff             & C        & C        & C        & C        & C        & C        & C        & C        & C        & C        & \tdash\fn{2}     \\
                Persuasion        & C        & C        & C        & C        & C        & C        & C        & C        & C        & C        & \tdash\fn{2}     \\
            \end{dtabularx}
            C\@: class skill \\
            1. Attribute varies depending on skill usage \\
            2. No attribute applies \\
        \end{dtable!*}

    \subsection{Trying Again}
        In general, you can try a skill check again if you fail, and you can keep trying indefinitely. Some skills, however, have consequences of failure that must be taken into account. A few skills are virtually useless once a check has failed on an attempt to accomplish a particular task. For most skills, when a character has succeeded once at a given task, additional successes are meaningless.

        Skills that require a free action to use can never be used more than once for the same purpose within a round. For example, if you fail to notice a creature sneaking up on you, you can't keep making Awareness checks as a free action until you notice. You could try again in the next round, however.

    \subsection{Special Skill Checks}

        \subsubsection{Opposed Checks}
            An opposed check is a check whose success or failure is determined by comparing the check result to another character's check result. In an opposed check, the higher result succeeds, while the lower result fails. In case of a tie, the higher check modifier wins. If these scores are the same, roll again to break the tie. Typical opposed checks are described in \trefnp{Example Opposed Checks}

            \begin{dtable}
                \lcaption{Example Opposed Checks}
                \begin{dtabularx}{\columnwidth}{*{3}{>{\lcol}X}}
                    \tb{Task} & \tb{Skill (Key Ability)} & \tb{Opposing Skill (Key Ability)} \tableheaderrule
                    Lie                                 & Bluff (\tdash)        & Sense Motive (Per)                   \\
                    Create a forged artwork             & Craft (Int)           & Craft (Int) or Awareness (Per)       \\
                    Create a false map                  & Craft (Int)           & Craft (Int) or Knowledge (geography) \\
                    Make a bully back down              & Intimidate (varies)   & Special\fn{1}                        \\
                    Make someone look like someone else & Disguise (Int)        & Awareness (Per)                      \\
                    Sneak up on someone                 & Stealth (Dex)         & Awareness (Per)                      \\
                    Steal a coin pouch                  & Sleight of Hand (Dex) & Awareness (Per)                      \\
                    Tie a prisoner securely             & Devices (Int)\fn{2}   & Escape Artist (Dex)                  \\
                \end{dtabularx}
                1 An Intimidate check can be opposed by the target's Mental defense, not a skill check. See the Intimidate skill description for more information. \\
                2 You can also tie up a creature while grappling with them (see \pcref{Grapple Actions}).
            \end{dtable}

        \subsubsection{Group Skill Checks}
            When multiple characters are trying to use the same skill simultaneously, they may be able to work together. Most skills can be done as a group skill check, where the more skilled characters assist the less skilled characters and work together to find the best result. There are two kinds of group skill checks.

            \parhead{Collaborative Checks} When making a collaborative skill check, each member of the group gets their own result. Collaborative checks might include a group of scouts sneaking up on an enemy camp, a group of adventurers climbing a cliff, or a group of spies lying about their true allegiances.

            When making a collaborative check, the group must choose a leader before making the check.
            Each member of the group can make the check using the higher of their skill modifier and half the leader's skill modifier. Other modifiers apply normally.

            The group leader must remain in a position to help the other members of the group for the group members to gain this benefit. For example, if a group is making a collaborative check to leap across a chasm, the leader must jump last. If the group is making a collaborative Stealth check, the leader must remain adjacent to the other members of the group to prevent their mistakes quietly. The exact circumstances depend on the situation.

            \parhead{Collective Checks} The group works together to get a single result. Collective checks might include a group of diplomats persuading a noble to go to war, a group of medics tending to a dying warrior, or a group of scouts keeping watch for enemy attacks.

            When making a collective check, the group simply uses the highest result from any character making the check.

            \parhead{Making Group Skill Checks} A group skill check is not always possible. If the whole group is leaping over a chasm at once to escape a dragon, there is not enough time to designate a leader and have them aid the others to make a collaborative check. In general, a group skill check is only possible if the group has at least five rounds to work together, though this time may vary widely depending on the situation and the skill being used.

    \subsection{Tasks}\label{Tasks}

        Each skill contains a brief description of how the skill is usually used.
        This description is followed by a series of \glossterm{tasks}, which are particular ways to use skills.
        These tasks are simply examples, and do not list everything the skill can be used for.
        You should be creative with your skills, rather than only using the tasks explicitly listed.

    \subsubsection{Hidden Tasks}\label{Hidden Tasks}
        Some \glossterm{tasks} are called \glossterm{hidden tasks}, and are marked with a [Hidden] tag in the task name.
        These tasks rely on hidden information that you should not have access to.
        For example, you can make a Sense Motive check to identify whether a creature is lying.
        If you are told to make a Sense Motive check when a creature talks, you can deduce that it is probably lying regardless of the success or failure of the check.
        To solve this issue, any checks for hidden tasks should be made secretly by the GM.\@
        Usually, you should not even know that you made a check unless you learn a result from it.

        If you are suspicious of a situation, you can ask the GM to make a relevant check for you.
        This usually should not grant a bonus to the check, but it can ensure that the GM did not forget to make the check!

\newpage
\skill{Acrobatics}{Dex}
        The Acrobatics skill represents your ability to balance, tumble, and perform similar feats of agility and poise.

    \subsection{Agile Charge}
        You can make a DR 10 Acrobatics check while \glossterm{charging} to change directions while charging.
        Success means you can make a single turn of up to 90 degrees during the movement.
        Failure means you can't change direction, though you can continue your movement or stop.
        Critical failure means you stop where you tried to change direction and fall \prone.

    \subsection{Balance}\label{Balance}

        When you are on a slippery or narrow surface, you must make an Acrobatics check to move.
        Success means you move along the surface at half speed.
        Critical success means you move along the surface at full speed.
        Failure means your action is wasted, and you do not move.
        Critical failure means you fall prone.
        If you do not have enough room to fall prone, you may fall off of the edge you are balancing on.

        The DR of Acrobatics checks to balance varies with the surface, as described in \trefnp{Balancing DRs}.
        In addition, if you are forcibly moved while on a slippery or narrow surface, you must make an Acrobatics check against the same DR.\@
        Success means you stay standing.
        Failure means you fall prone.

        \begin{dtable}
            \lcaption{Balancing DRs}
            \begin{dtabularx}{\columnwidth}{l X}
                \tb{Narrow Surface}           & \tb{DR} \tableheaderrule
                At least one foot wide        & 0       \\
                At least six inches wide      & 5      \\
                At least two inches wide      & 10      \\
                At least one inch wide        & 15      \\
                Less than than one inch wide  & 20      \\
                \tb{Surface Condition}       & \tb{DR Modifier} \\
                Water covered                 & \plus2      \\
                Slightly mobile (rope bridge) & \plus2      \\
                Ice or oil covered            & \plus5      \\
                Very mobile (slack rope)      & \plus5      \\
            \end{dtabularx}
        \end{dtable}

    \subsection{Mitigate Fall}
        As you hit the ground after a fall, you can make an Acrobatics check to reduce falling damage.
        A DR 5 check allows you to treat a fall as if it were 10 feet shorter.
        For every 10 points by which you beat that DR, you can reduce the falling damage by 10 additional feet.

    \subsection{Rapid Stand}\label{Rapid Stand}
        You can use the \textit{rapid stand} ability as a \glossterm{minor action}.
        \begin{freeability}{Rapid Stand}[Swift]
            You make a DR 15 Acrobatics check to stand up from a prone position quickly.
            Success means you stand up.
            Since this is a \glossterm{Swift} ability, standing up in this way means you do not suffer the penalties for being prone during the current phase.
            Failure means you fail to stand up.
        \end{freeability}

\newpage
\skill{Awareness}{Per}
    The Awareness skill represents your ability to observe things which you might otherwise fail to notice. It can be used to spot concealed things, to correctly identify sounds, or to smell the distinctive rotting flesh smell that accompanies undead.

    Each creature has a variety of senses. The Awareness skill governs the use of every sense. It is most commonly used with sight, hearing, and smell, though other uses are possible. Often, bonuses or penalties will apply only to certain senses. For example, it is extremely difficult to see when there is no light, but that has no effect on how difficult it is for you to hear sounds. In such cases, you only roll one Awareness check, and you apply the modifiers separately for each sense.

    While sleeping, you take a \minus10 penalty to the Awareness skill.

    \subsection{Discern Illusion [Hidden]}
        When you observe the effect of a \glossterm{Sensation} ability, you can make an Awarenss check to notice its unreal nature.
        The DR is specified in the description of the ability creating the illusion, but is usually equal to a check result made when using the ability.
        Success means you recognize the effect as an illusion, and can see through it as if it was almost entirely transparent.
        Failure means you don't notice anything amiss.

        If the illusion is completely missing a sense that should logically be present, such as a \spell{silent image} of people marching in heavy armor, the DR to interact with the illusion with that sense is lowered by 5.

        You cannot retry this check until you gain meaningful new information that would help you identify the illusion.

    \subsection{Identify Disguise [Hidden]}\label{Identify Disguise}
        When you observe a disguised creature or object, you can make an Awareness check to identify the disguise.
        The DR is equal to the Disguise check result used to create the disguise (see \pcref{Disguise}).
        Success means you know that the creature or object is disguised.
        Critical success means you can also discern its true appearance beneath the disguise.
        Failure means you don't notice anything unusual.

        You cannot retry this check until you gain meaningful new information that would help you identify the disguise.

    \subsection{Identify Forgery}
        As a standard action, you can make an Awareness check to identify forgeries.
        The DR to identify a forgery is equal to the Craft check result used to make the item (see \pcref{Craft}).
        Success means you correctly identify whether the item is a forgery or not.
        Failure means you don't notice anything indicating the item is a forgery.

        You cannot retry this check until you gain meaningful new information that would help you identify the forgery.

    \subsection{Notice Creatures and Events}
        As a free action, you can notice creatures and events around you.
        The DR depends on the sense used and the obviousness of the event, as described on tables below.
        Success means you notice something, but you don't know any details -- only its general direction.
        For every 5 points by which you beat the DR, you learn an additional piece of information, such as what caused the event or the location of the creature.
        Failure means you don't notice anything.

        This can be used to determine the precise location of a creature or object, even if you can't see it. The DR to identify the location is equal to the DR to notice the creature or object using the appropriate sense \add 5 (since you need to learn an additional piece of information).

    \subsection{Read Lips}
        When you see a creature speaking, you can make an sight-based Awareness check to read its lips.
        The DR is 10 for ordinary conversation, or up to 20 if the speaking creature makes an effort to avoid moving its lips.
        You must be able to understand the language spoken.
        Success means you can understand the general content of the message, but not the exact words.
        Critical success means you understand the exact words.
        Failure means you don't understand the message.

    \subsection{Search}\label{Search}
        As a standard action, you can use the \textit{search} ability to closely investigate a small area.
        \begin{freeability}{Search}
            Make an Awareness check to notice things in a single 5-ft.\ square within 10 feet of you.
            You gain a \plus5 bonus to this check.
        \end{freeability}

    \subsection{Senses}\label{Senses}

        \parhead{Sight} The DR to see something depends on the obviousness of the sight, as shown on \trefnp{Sight-based DRs}, and other modifiers given at \trefnp{Awareness DR Modifiers}.

        \begin{dtable}
            \lcaption{Sight-based DRs}
            \begin{dtabularx}{\columnwidth}{X l}
                \tb{Situation} & \tb{Base DR\fn{1}} \tableheaderrule
                Creature or object & 0 \\
                Creature trying to hide & Stealth check result\fn{2} \\
                Hidden trap, secret door, or mechanism & Craft or Devices check result \\
                Magic trap & 15 \add double level of spell used to create trap \\
            \end{dtabularx}
            1 Always add any appropriate modifiers from \tref{Awareness DR Modifiers} \\
            2 Don't add size-based DR modifiers to the Awareness check.
        \end{dtable}

        \parhead{Sound} The DR to hear a sound depends on the intensity of the sound, as shown on \trefnp{Sound-based DRs}, and other modifiers given at \trefnp{Awareness DR Modifiers}.

        \begin{dtable}
            \lcaption{Sound-based DRs}
            \begin{dtabularx}{\columnwidth}{X l}
                \tb{Situation} & \tb{Base DR\fn{1}} \tableheaderrule
                Creature shouting & \minus5\fn{1} \\
                Creature talking normally or fighting & 0 \\
                Creature whispering & 5 \\
                Creature standing still & 10 \\
                Creature trying to be quiet & Stealth check result\fn{2} \\
            \end{dtabularx}
            1 Always add any appropriate modifiers from \tref{Awareness DR Modifiers} \\
            2 Don't add size-based DR modifiers.
        \end{dtable}

        \parhead{Scent} The DR to smell something depends on the intensity of the scent, as shown on \trefnp{Scent-based DRs}, and other modifiers given at \trefnp{Awareness DR Modifiers}.

        The DRs given are for a creature with an ordinary sense of smell, like a human.

        Smell intensity can vary widely depending on the circumstances. In general, an unusually strong smell, such as a creature wearing perfume, has a DR which is 5 lower. An unusually weak smell, such as a creature who has just taken an unscented bath, has a DR which is 5 higher.

        Smells dissipate quickly with distance. Double the normal distance modifiers for scent-based perception checks.

        \subparhead{Scent Ability}\label{Scent} Some creatures have an unusually good sense of smell. Creatures with the scent ability gain a \plus5 bonus to scent-based Awareness checks.

        \begin{dtable}
            \lcaption{Scent-based DRs}
            \begin{dtabularx}{\columnwidth}{X l}
                \tb{Situation} & \tb{Base DR\fn{1}} \tableheaderrule
                Strong scent (rotting flesh, pungent spices) & 0 \\
                Moderate scent (fresh food) & 5 \\
                Living creature & 10 \\
            \end{dtabularx}
            1 Always add any appropriate modifiers from \tref{Awareness DR Modifiers} \\
        \end{dtable}

        \parhead{Other Senses} Other senses can exist, and creatures can make Awareness checks to use those other senses appropriately.

    \subsection{Modifiers}
        All Awareness checks share the same set of modifiers, in addition to certain modifiers which depend on the sense. These are noted on \trefnp{Awareness DR Modifiers}.

        This table should not generally be consulted during a game. It is provided to make it easier to estimate a task's difficulty.

        \begin{dtable}
            \lcaption{Awareness DR Modifiers}
            \begin{dtabularx}{\columnwidth}{X l}
                \tb{Distance} & \tb{DR Modifier\fn{1}} \tableheaderrule
                Up to 20 feet away             & \plus0           \\
                21--100 feet away              & \plus2           \\
                101--500 feet away             & \plus5           \\
                501--2500 feet away            & \plus10          \\
                2500--10000 feet away          & \plus15          \\
                \tb{Number}                    & \tb{DR Modifier} \\
                1--4 creatures or objects      & \plus0           \\
                5--20 creatures or objects     & \minus2          \\
                21--100 creatures or objects   & \minus5          \\
                101--500 creatures or objects  & \minus10         \\
                501--2500 creatures or objects & \minus15         \\
                \tb{Background}                & \tb{DR Modifier} \\
                Nothing similar                & 0                \\
                Many similar sensations        & \plus5           \\
                Many more intense sensations   & \plus10           \\
            \end{dtabularx}
            1 Doubled for scent-based Awareness checks.
        \end{dtable}

\newpage
\skill{Bluff}{---}
        The Bluff skill represents your ability to mislead people with your words. It is usually used when you are lying. Using a Bluff check is part of conversation or other actions, so it requires no special action to perform.

    \subsection{Blend In}
        You can make a Bluff check to blend in with a crowd. Your Bluff check is opposed by the Awareness checks of anyone looking for you. Success means you remain unobserved. Failure means the person looking for you found you.

        If you act differently from the crowd you are trying to blend in with, anyone observing you gets a \plus5 bonus to find you. You may need to make Awareness or Sense Motive checks to figure out how to act, depending on the situation. If you are extremely dissimilar from the crowd, such as a naked person in a temple or a human among halflings, anyone observing you may gain a \plus10 or greater bonus to find you.

    \subsection{Conceal Threat}\label{Conceal Threat}
        As a \glossterm{minor action}, you can use the \textit{conceal threat} ability to look less threatening.
        \begin{freeability}{Conceal Threat}[\glossterm{Sustain} (minor)]
            Choose a \glossterm{threat} value you wish to portray and make a Bluff check.
            The intended threat cannot exceed your actual threat.
            Your Bluff check is opposed by the Sense Motive checks of anyone observing you.
            \hit Your threat against each target is equal to your chosen threat.
        \end{freeability}

    \subsection{Distract}\label{Distract}
        As a standard action, you can make a Bluff check to distract a creature you are interacting with.
        Your Bluff check is opposed by your target's Sense Motive check.
        Success means they take a \minus5 penalty to the Awareness and Sense Motive skills against targets other than you until the end of the next round.
        Failure means they take no penalty, and realize you were trying to distract them.
        You can continue distracting the target by using this ability against them each round.
        The DR increases by 2 for each consecutive round that you have distracted the same creature.

        Normally, distracting a creature requires both visible motion and sound.
        If you take a \minus5 penalty to the Bluff check, you can distract a creature without moving, or without making sound, but not without both.
        In addition, you can take a \minus5 penalty to your Bluff check to distract everyone who can see or hear you.

        If you successfully distract a creature, you can attempt to hide from that creature as if you were not being observed (see \pcref{Stealth}, for details).

    \subsection{Impersonate}\label{Impersonate}
        When you are pretending to be another creature, you can assume the mannerisms and speech patterns of the creature you are impersonating.
        To do so, you must make a Bluff check.
        Anyone observing you can oppose your check with a Sense Motive check to identify the impersonation (see the \textit{identify disguise} ability, page \pref{Identify Disguise}).
        If you succeed, the observer thinks your impersonation is accurate.
        If you critically succeed, they also take a \minus5 penalty to any other check to see through your impersonation, such as to notice a flawed disguise.
        If they succeed, they notice inconsistencies or mistakes in your impersonation, and may realize you are not what you seem.

        % Technically these should be bonuses to the checks observers make, but that's confusing
        If you do not know how you are supposed to act, or are physically unable to perform necessary actions, impersonation is more difficult.
        You take a \minus2 penalty if you cannot replicate minor details of an impersonation, such as a deep voice beyond your vocal range.
        You take a \minus5 penalty if you cannot replicate significant details of an impersonation, such as the singing voice of a famous bard or the noble manners of a crown prince.
        You take a \minus10 or greater penalty if you cannot replicate fundamental aspects of the impersonation, such as the actions required to lead a complex ritual as an archmage.
        Observers who do not know your impersonation is inaccurate can take similar penalties; see the \textit{identify disguise} ability for details.

        A creature may not believe your impersonation even if you make a successful Bluff check.
        For example, a halfling can impersonate an orc's voice perfectly with a Bluff check, but without a disguise anyone who sees the halfling will immediately realize it is not an orc (see \pcref{Disguise}).

    \subsection{Lie}
        As a free action, when you say something which you know is untrue, you can make a Bluff check to avoid revealing your deception.
        Anyone observing you lie can oppose your check with a Sense Motive check.
        If you succeed, the observer does not notice any indication that you are lying.
        If they succeed, they realize that you are lying.

        A creature that fails its Sense Motive check may choose not to believe you for other reasons.
        This check only prevents a creature from recognizing the lie based on your body language and behavior.
        To convince creatures to believe or take actions based on lies, you need the Persuasion skill (see \pcref{Compel Belief}).

    \subsection{Secret Message}
        As part of normal speech, you can make a Bluff check to attempt to convey a hidden message to another character without others understanding it using codes, metaphors, and similar misdirection tools. The DR is 10 for simple messages and 15 for complex messages. If the message contains completely new information, the DR increases by 5. You can freely increase the DR to make the message more difficult to intercept, but doing so also makes it more difficult for the intended recipient to interpret.

        Anyone witnessing the exchange may make a Sense Motive check against the same DR to identify the hidden message.
        Creatures who know your system for conveying hidden messages -- normally, the intended recipient -- receive a \plus10 bonus.
        Creatures who know in advance that a message will be conveyed also receive a \plus5 bonus on this check.

\newpage
\skill{Climb}{Str}
    The Climb skill represents your ability to climb obstacles.

    \subsection{Climb}
        You can make a Climb check as a move action to move up, down, or across a slope, a wall, or some other steep incline (or even a ceiling with handholds). Success means you move a number of feet equal to the size of your space, as described on \trefnp{Climb Speeds}. Critical success means you move at twice that speed. Failure means your action is wasted and you do not move. Critical failure means you fall.

        You need both hands free to climb, but you may cling to a wall with one hand while you cast a spell or take some other action that requires only one hand.

        \begin{dtable}
            \lcaption{Climb Speeds}
            \begin{dtabularx}{\columnwidth}{l X l X}
                \tb{Size} & \tb{Speed} & \tb{Size} & \tb{Speed} \tableheaderrule
                Medium      & 5 ft.     &            &        \\
                Small       & 5 ft.     & Large      & 10 ft. \\
                Tiny        & 2-1/2 ft. & Huge       & 15 ft. \\
                Diminuitive & 1 ft.     & Gargantuan & 20 ft. \\
                Fine        & 1/2 ft.   & Colossal   & 40 ft. \\
            \end{dtabularx}
        \end{dtable}

        The DR of the check depends on the difficulty of the task and the conditions of the climbing surface, as shown on \trefnp{Climb DRs} and \trefnp{Climb DR Modifiers}.

        \begin{dtable}
            \lcaption{Climb DRs}
            \begin{dtabularx}{\columnwidth}{l X X}
                \tb{DR} & \tb{Surface or Activity} & \tb{Example} \tableheaderrule
                0 & Steep slope & A hill too steep to walk up \\
                5 & Surface with large hand and foot holds & Knotted rope, Very rough rocks, ship's rigging \\
                10 & Surface with some hand and foot holds & Surface with pitons or carved holes, rough wall \\
                10 & Surface with only large hand holds & Pulling yourself up by your hands while dangling \\
                15 & Uneven surface with narrow hand and foot holds & Unknotted rope, unweathered natural rock, typical ruin wall, brick wall \\
                15 & Overhang or ceiling with large handholds & Tree limbs, butcher's ceiling with meat hooks \\
                20 & Rough surface with no holds & Weathered natural rock, well-made stone wall \\
                30 & Bracing between two opposite smooth surfaces & Parallel glass windows \\
                35 & Smooth surface & Glass window \\
            \end{dtabularx}
        \end{dtable}

        \begin{dtable}
            \lcaption{Climb DR Modifiers}
            \begin{dtabularx}{\columnwidth}{l X}
                \tb{DR Modifier\fn{1}} & \tb{Description} \tableheaderrule
                \minus5 & Climbing a chimney (artificial or natural) or other location where you can brace against two opposite walls \\
                \minus2 & Climbing a corner where you can brace against perpendicular walls \\
                \minus2 & Inclined surface (between 45 and 60 degrees) \\
                \minus2 & Climbing a free-hanging object, such as a rope, where you can brace against a nearby wall \\
                \plus2 & Surface is slightly slippery, such as damp stone \\
                \plus5 & Surface is very slippery, such as ice or oil-covered stone \\
            \end{dtabularx}
            1 These modifiers are cumulative with modifiers of different kinds. Use any that apply (but not if one modifier is just a more extreme version of another).
        \end{dtable}

        \parhead{Climbing Distractions}
        If you take damage while climbing, suddenly acquire significantly more weight (such as by catching a falling character), or otherwise are significantly distracted, you must make another Climb check against the wall's DR to avoid falling.

        \parhead{Climb Speed}\label{Climb Speed}
            A creature with a climb speed can move a distance equal to its climb speed with a successful Climb check.
            However, it does not move double its speed if it gets a critical success on a Climb check.

    \subsubsection{Grab Edge}\label{Grab Edge}
        If you are next to the edge of a wall or cliff, you can grab it.
        Grabbing an edge is done as part of other movement, and does not take an action in itself.
        The DR of the check depends on the nature of the edge, but a typical stone or similarly solid edge has a DR of 5.
        You can pull yourself up from a grabbed edge as a move action that requires a Climb check against the edge's DR\@.

        Your ability to grab an edge depends on your reach. The maximum vertical reach (height the creature can reach without jumping) for an average creature of a given size is shown on the table below. (As a Medium creature, a typical human can reach 8 feet without jumping.) The maximum vertical reach for an individual creature is usually given by the 1 and 1/2 times the creature's height. Quadrupedal creatures don't have the same vertical reach as a bipedal creature; treat them as being one size category smaller.

        \begin{dtable}
            \begin{dtabularx}{\columnwidth}{>{\lcol}X >{\lcol}X}
                \tb{Creature Size}  & \tb{Vertical Reach} \tableheaderrule
                Colossal   & 128 ft. \\
                Gargantuan & 64 ft.  \\
                Huge       & 32 ft.  \\
                Large      & 16 ft.  \\
                Medium     & 8 ft.   \\
                Small      & 4 ft.   \\
                Tiny       & 2 ft.   \\
                Diminutive & 1 ft.   \\
                Fine       & 1/2 ft.
            \end{dtabularx}
        \end{dtable}

        If you can't reach an edge, you can jump to grab it (see \pcref{Leap}).

    \subsection{Stop Fall}\label{Stop Fall}
        It is possible, but difficult, to make a Climb check to stop yourself from falling when near a wall. To catch yourself while falling, make a Climb check against a DR equal to the wall's DR \add 10.

    \subsection{Wallrun}\label{Wallrun}
        As part of movement, you can make a Climb check to run along a wall rather than climbing it. The DR is 5 higher than normal for the wall, but this does not require free hands. Success means you move up to half your land speed horizontally, or up to a quarter of your land speed vertically. Failure means you fall. Critical failure means you are prone when you land. For every round you spend running on a wall, the DR increases by 5.

        Wallrunning on a ceiling is impossible.

    \subsection{Creature Climb}
        As a standard action, you can make an attack vs. Reflex against a creature adjacent to you.
        Your \glossterm{accuracy} is equal to your Climb skill.
        The creature must be three or more size categories larger than you.
        Success means you can climb the creature as if it were a solid object with a Climb DR equal to its Reflex defense.
        The creature takes a \minus4 penalty to \glossterm{accuracy} on physical attacks against you, but is not otherwise encumbered by your presence unless it cannot support your weight.
        It can attempt to remove you by dealing damage to you or with an appropriate ability, such as the \textit{shove} ability (see \pcref{Shove}).

\newpage
\skill{Craft}{Int}
    Like Knowledge, Perform, and Profession, Craft is actually a number of separate skills.
    You could have several Craft skills, each with a separate degree of training.
    Common Craft skills are listed below, with additional description for some skills.

    \begin{itemize}
        \item Alchemy (Alchemist's fire, tanglefoot bags, potions)
        \item Bone
        \item Ceramics (Glass, pottery)
        \item Jewelry (gemcutting, amulets, rings)
        \item Leather
        \item Manuscripts (Books, official documents, scrolls)
        \item Metal
        \item Poison
        \item Stone
        \item Textiles (Cloth, fabric)
        \item Traps
        \item Wood
    \end{itemize}

    % TODO: note which skills often have magic items?

    A Craft skill is specifically focused on physical objects. If nothing can be created by an endeavor, it probably falls under the heading of a Profession skill. Complex structures, such as buildings or siege engines, may require Knowledge (engineering) in addition to an appropriate Craft skill.

    \subsection{Create Item}
        You can use the Craft skill to create an item by expending time and material components. Creating an item often requires multiple consecutive Craft checks. Success on a check means you make progress on completing the item. If you make enough progress, you complete the item. Failure means you failed to make progress, but can try again without penalty. Critical failure means you botched the item, and negating all progress and ruining half of the material components. You can start again from scratch with your remaining components.

        Each item takes a certain amount of working time to craft, as shown on \tref{Crafting Time}, and the expenditure of one quarter of the item's price in raw materials. In order to craft an item, you must make a Craft check against the item's Craft DR, as shown on \trefnp{Craft DRs}. If you succeed, you make progress on the item based on how long you spend crafting. For every 5 points by which you beat the Craft check, you accomplish twice as much work in the same amount of time. Once your total effective working time exceeds the time required to craft the item, you have finished the item.

        All crafts require artisan's tools to give the best chance of success. If improvised tools are used, the check may be made with a penalty, or may be impossible, depending on the tools available and the item to be crafted. For example, crafting a bow with improvised woodworking tools would impose a \minus5 penalty, but cutting a diamond without specialized tools is impossible. Note that raw materials for some items, particularly alchemical items, may be hard to come by in some areas. A typical day's work consists of 8 hours of work.

        To determine the time required to craft an item, consult the table below.
        \begin{dtable}
            \lcaption{Crafting Time}
            \begin{dtabularx}{\columnwidth}{l X}
                \tb{Item Price} & \tb{Crafting Time} \tableheaderrule
                1gp or less    & One hour         \\
                10gp or less   & Eight hours      \\
                100gp or less  & One week\fn{1}   \\
                500gp or less  & One month\fn{1}  \\
                1000gp or less & Two months\fn{1} \\
            \end{dtabularx}
            1 Assuming 8 hours of work each day.
        \end{dtable}

        \begin{dtable}
            \lcaption{Craft DRs}
            \begin{dtabularx}{\columnwidth}{>{\lcol}X l l}
                \tb{Item} & \tb{Craft Skill} & \tb{Craft DR} \tableheaderrule
                Acid & Alchemy & 5 \\
                Alchemist's fire, smokestick, or tindertwig & Alchemy & 10 \\
                Antitoxin, sunrod, tanglefoot bag, or thunderstone & Alchemy & 15 \\
                Armor or shield & Metal or wood & 5 \add AD bonus \\
                Longbow or shortbow & Wood & 10 \\
                Crossbow & Wood & 10 \\
                Simple melee or thrown weapon & Metal or wood & 5 \\
                Martial or exotic melee or thrown weapon & Metal or wood & 10 \\
                Mechanical trap & Traps & Varies\fn{1} \\
                Very simple item (wooden spoon) & Varies & 2 \\
                Typical item (iron pot) & Varies & 5 \\
                High-quality item (bell, average lock) & Varies & 10 \\
                Complex or superior item (fine china, document with official seal)  & Varies & 15\add \\
            \end{dtabularx}
            1 Traps have their own rules for construction.
        \end{dtable}

    \subsection{Appraise Item}
        You can make a Craft check to estimate the value of an item with 1 minute of careful study. The Craft skill used must be related to the item.

        The DR depends on the rarity of the item. Common items, such as nonmagical equipment and utensils, are DR 5. Rare items, such as valuable gems and magic items worth less than 100,000 gp, are DR 10. Extraordinarily rare items, such as unique gems and magic items worth at least 100,000 gp, are DR 20.

        Success means you know the value of the item. Failure means you think the item is worth (d10 \add 5)/10 \x the item's actual value. (The d10 roll is made secretly, so you don't know how close you are.) Critical failure means you can't estimate the item's value at all.

        After you appraise an item, you can't appraise it again with any skill until you gain a level.

    \subsection{Create Forgery}
        You can use the Craft skill to create false or defective versions of objects. For example, you may wish to forge official-looking documents with Craft (manuscripts), or you may need to make items in a rush to satisfy an employer. This functions like crafting the item from scratch, except that the Craft DR is 5 lower than normal, and you use one-half the item's price to determine the price of raw materials and the crafting time.

        Forgeries which have a function, such as a weapon, are always defective in some way which makes them unsuitable for use. A forgery can be detected with the Identify Forgery uses of the Craft and Awareness skills.

    \subsection{Identify Forgery}
        As a standard action, you can make a Craft check to evaluate whether an item is a forgery. The DR to identify a forgery is equal to the Craft check used to make the item. Success means you correctly identify whether the item is a forgery or not. Failure means you are unsure. Critical failure means you randomly identify the item as genuine or forged. The check is made secretly, so you can't be sure how good the result is.

    \subsection{Repair Item}
        You can make a Craft check to repair a broken item. This functions like crafting the item from scratch, except that you use one-tenth of the item's price to determine the price of raw materials and the crafting time.

    \subsection{Other Tasks}
        You can use the Craft skill for various tasks related to the object of your craft. For example, you could use Craft (wood) to sabotage a wagon so it will break at a later time, or Craft (alchemy) to purify spoiled food or ingredients. The DRs of such tasks, and what can be accomplished, can vary widely. In general, the more difficult the task, and the more loosely it is related to the skill, the higher the DR\@.

\newpage
\skill{Creature Handling}{Per}
    The Creature Handling skill represents your ability to handle creatures without being able to speak with them. With it, you can convince them to do what you want or train them to follow commands. This skill can only be used with creatures with an Intelligence of \minus6 or lower.

        Animals are easier to handle than other kinds of creatures.
        The DRs listed are for animals; the DRs to handle other kinds of creatures are 5 higher.

    \subsection{Handling Creatures}
        You can use Creature Handling to control a creature's actions using these abilities.
        Critical failure with these abilities may make the target hostile, depending on the circumstances.

        As a standard action, you can use the \textit{command} ability to control the actions of a creature.

        \begin{freeability}{\labeltext{Command}}[\glossterm{Auditory}, \glossterm{Compulsion}, \glossterm{Sustain (standard)}]
            Make an attack vs. Mental against a creature within \rngmed range.
            Your \glossterm{accuracy} is equal to your Creature Handling skill.
            In addition, choose and state an action that the creature could take.
            \hit The target is unable to take any actions except to use the \textit{cleanse} ability (see \pcref{Cleanse}).
            \crit The target performs the chosen action if it is physically capable of performing it.
            This can include convincing creatures to perform forced marches and similar activities (see \pcref{Forced March}).
            
            The target's defense is increased if it is not an animal, as normal for Creature Handling attacks and checks.
            You take a \minus10 penalty to accuracy against an actively hostile target.
            If the target is damaged or feels that it is in danger, this effect is automatically ended.
        \end{freeability}

        As a \glossterm{free action}, you can use the \textit{perform trained action} ability to convince a creature to perform an action it knows.

        \begin{freeability}{Perform Trained Action}
            Make a DR 5 Creature Handling check on an \glossterm{ally} within \rnglong range and choose an action that creature could take.
            If you succeed, the target performs the chosen action if it is trained to perform it.
            Generally, wild animals are not trained in any actions, so this is not effective on them.
        \end{freeability}

    \subsection{Training Creatures}\label{Training Creatures}
        You can use Creature Handling to train a creature. Success means the creature learns a trick or becomes domesticated. Failure means your time is wasted, and you must try again. Exceptional failure may indicate that the animal becomes unable to learn the trick, becomes hostile, or other consequences depending on the situation. Training a creature takes a week or more; this requires spending at least four hours each day with the creature. It is not generally possible to accelerate the process by spending more time each day; the creature must take time to learn the new behavior. If the training is interrupted for a week or more, the attempt automatically fails.

        \parhead{Teach a Trick} You can teach a creature a specific trick with one week of work and a successful Creature Handling check against the indicated DR\@. A creature can learn two tricks per point of Intelligence it has above \minus10. Thus, a creature with an Intelligence of \minus9 can learn two tricks, while a creature with an Intelligence of \minus5 can learn ten tricks. Possible tricks (and their associated DRs) include, but are not necessarily limited to, the following.

        \subsk{Attack (DR 10)} The creature attacks apparent enemies. You may point to a particular creature that you wish the creature to attack, and it will comply if able. Normally, a creature will attack only humanoids, monstrous humanoids, giants, or other creatures. Teaching a creature to attack all creatures (including such unnatural creatures as undead and aberrations) counts as two tricks.
        \subsk{Come (DR 5)} The creature comes to you, even if it normally would not do so.
        \subsk{Defend (DR 10)} The creature defends you (or is ready to defend you if no threat is present), even without any command being given. Alternatively, you can command the creature to defend a specific other character.
        \subsk{Down (DR 5)} The creature breaks off from combat or otherwise backs down. A creature that doesn't know this trick continues to fight until it must flee (due to injury, a fear effect, or the like) or its opponent is defeated.
        \subsk{Fetch (DR 5)} The creature goes and gets something. If you do not point out a specific item, the creature fetches some random object.
        \subsk{Guard (DR 10)} The creature stays in place and prevents others from approaching.
        \subsk{Heel (DR 5)} The creature follows you closely, even to places where it normally wouldn't go.
        \subsk{Messenger (DR 15)} The creature carries a small item to a destination.
        Once it arrives, it waits for up to 24 hours for someone to take the item from it.
        The destination must be known to the creature.
        \par When you instruct the creature to deliver the item, you must communicate the destination to the creature.
        This normally requires a DR 20 Creature Handling check as a standard action.
        The DR of this check is lowered to 15 for locations the creature is extremely familiar with, such as its home.
        If you have other means of communicating the destination to the creature, such as the \textit{wild speech} druid ability (see Wild Speech, page \pref{Drd:Wild Speech}), that check is unnecessary.
        \subsk{Perform (DR 10)} The creature performs a variety of simple tricks, such as sitting up, rolling over, roaring or barking, and so on.
        \subsk{Seek (DR 5)} The creature moves into an area and looks around for anything that is obviously alive or animate.
        \subsk{Stay (DR 5)} The creature stays in place, waiting for you to return. It does not challenge other creatures that come by, though it still defends itself if it needs to.
        \subsk{Track (DR 10)} The creature tracks the scent presented to it. (This requires the creature to have the scent ability)
        \subsk{Work (DR 5)} The creature pulls or pushes a medium or heavy load.

        \parhead{Rear a Wild Creature} To rear a creature means to raise a wild creature from infancy so that it becomes domesticated. The time required depends on the creature in question. The DR for this check is equal to 5 \add the number of levels the creature has. You can rear as many as three creatures of the same kind at once without penalty. You can rear additional creatures, but you take a cumulative \minus2 penalty to checks you make to rear all of the animals. A successfully domesticated creature can be taught tricks at the same time it's being raised, or it can be taught as a domesticated creature later.

        \parhead{Bonus Tricks}\label{Bonus Tricks} Some trainers can teach creatures bonus tricks in addition to their normal maximum number of tricks known.
        Once a creature has learned a bonus trick, that trick may not be retrained into a different trick by any trainer without the same ability to grant bonus tricks.
        However, any trainer may untrain the trick.

\newpage
\skill{Deduction}{Int}
    You can use the Deduction skill to make logical deductions based on evidence.
    It includes both determining which facts and observations are relevant to use as evidence, and reaching conclusions based on that evidence.
    However, this skill cannot protect you from coming to inaccurate conclusions if you rely on inaccurate or incomplete facts and observations.

    \subsection{Analyze Evidence}
        As a standard action, you can make use the \textit{analyze evidence} ability.
        \begin{freeability}{Analyze Evidence}
            Make a Deduction check to analyze evidence available to you and try to reach an accurate conclusion.
            This includes both determining which evidence is relevant and deciding what that evidence proves.
            Most deductions have two components: observations you make, and knowledge you have.
            When you use this ability, you can decide to trust your own observations, your own knowledge, or both.

            If you trust your own observations, and the deduction requires making observations, your Deduction modifier on the check is limited to be no greater than your modifier with the skill used to make observations.
            This skill is typically Awareness or Sense Motive.
            If you trust your own knowledge, and the deduction requires knowledge, your Deduction modifier on the check is limited to be no greater than twice your Knowledge modifier with any relevant knowledge.
            If you trust both your observations and your knowledge, both limits apply.

            Alternately, you can explicitly specify either the observations or knowledge your deduction is relying on.
            For example, you could make a deduction based on information given to you by another creature.
            If you do, your Deduction check is not limited, but your conclusions may be inaccurate if your assumptions are inaccurate.

            The base DR for this check is 10.
            This DR is modified depending on the difficulty of the deduction and the quality of the evidence available to you, as shown on \trefnp{Deduction DR Modifiers}.
        \end{freeability}

    \begin{dtable*}
        \lcaption{Deduction DR Modifiers}
        \begin{dtabularx}{\textwidth}{X X l}
            \tb{Evidence Quality} & \tb{Example} & \tb{Modifier} \tableheaderrule
            No irrelevant or misleading evidence & Determining a historical truth by reading the relevant passage in a history book & \plus5 \\
            Some evidence is irrelevant or misleading & Determining a historical truth by reading contemporary accounts & \plus0 \\
            About half of the evidence is irrelevant or misleading & Determining a historical truth by reading eyewitness accounts & \minus5 \\
            Almost all evidence is irrelevant or misleading & Determining a historical truth by reading military propaganda & \minus10 or more\fn{1} \\
            All evidence is irrelevant or misleading & Determining a historical truth by reading a cookbook & \tdash\fn{2} \\

            \tb{Complexity} & \tb{Example} & \tb{Modifier} \tableheaderrule
            Exceptionally simple logic using no more than one piece of evidence & The sun is out; therefore, it is daytime & \plus5 \\
            Simple logic using one or two pieces of evidence & It is raining, and the cleric's clothes and boots are dry; therefore, he was not out in the rain & \plus0 \\
            Moderately complex logic using at least three pieces of evidence & It is raining, the mage's clothes are wet but her boots are dry, she was observed stepping into the bar, and there are no tracks leading up to the door; therefore, she can probably fly & \minus5 \\
            Exceptionally complex logic & A difficult logic puzzle & \minus10 or more \\
        \end{dtabularx}
        1. If there is an exceptionally large amount of irrelevant or misleading evidence relative to the amount of useful evidence, this penalty may be even larger.
        2. It is impossible to make a correct deduction if there is no relevant and accurate evidence.
    \end{dtable*}

\newpage
\skill{Devices}{Int}
        You can use the Devices skill to manipulate mechanical devices such as locks, traps, and other contraptions. With enough skill, you can even manipulate magical devices.

        The DR of a Devices check depends on the device being manipulated. In addition, some actions are easier than others, and modify the DR accordingly. DRs are listed on \trefnp{Devices DRs}.

        \begin{dtable}
            \lcaption{Devices DRs}
            \begin{dtabularx}{\columnwidth}{>{\lcol}X c}
                \tb{Device Type} & \tb{Base DR} \tableheaderrule
                Simple device (wagon wheel, typical knot) & 5 \\
                Average device (door hinge, complex knot) & 10 \\
                Challenging device (typical lock or trap) & 15 \\
                Difficult device (good lock, complex trap) & 20 \\
                Magic trap & 15 \add double spell level \\
                Extraordinary device (extraordinary lock) & 25 \\
            \end{dtabularx}
        \end{dtable}

    \parhead{Special Circumstances}

        You can attempt to leave no trace of your tampering while making a Devices check to activate or subvert a device. This increases the Devices DR by 5, but increases the Awareness DR to notice the tampering by 10.

    \subsection{Activate Device}
        As a standard action, you can make a Devices check to make a device perform a function it was designed to do, even if you lack the normal requirements. For example, you could tie or untie a knot, apply a poison, lock or unlock a lock without its key, or activate a trap without using its normal triggering mechanism (hopefully without being in its line of fire).

    \subsection{Analyze Device}
        As a standard action, you can evaluate a device to understand how it functions. The DR is 10 lower than normal. Success grants you an insight into the device's mechanisms. For every 5 points by which you succeed, you glean additional information about the device, which can also help you estimate the device's true difficulty. Failure means you learn nothing about the device.

    \subsection{Break Device}
        As a standard action, you can make a Devices check to break a device. The DR is 5 lower than normal. This is generally not subtle, such as jamming a lock or breaking a trap while triggering it in the process. It cannot change the state of a device, but you can prevent it from being used normally; a locked door will remain locked, but you can prevent it from being unlocked with its key. Failure means the device continues to function. Critical failure may cause you to think that you successfully broke the device, while in fact it functions normally.

    \subsection{Create Bindings}
        As a standard action, you can make a Devices check to tie a knot or create a binding, including binding a helpless foe. The DR to escape the binding is equal to your check result.

    \subsection{Improvise}\label{Improvise}
        You can construct ad-hoc devices from available materials.
        It takes five minutes to make a device of up to Tiny size.
        You can make a Small device in the time required to make two Tiny devices, a Medium device in the time required to make two Small devices, and so on.
        You make a Devices check against the DR required to craft the item normally.
        Success means you create a device that lasts long enough for a single use before breaking.
        For every 5 points by which you succeed, the device lasts for an additional use.

        Normally, you must have materials at hand which are designed for the construction of the device.
        You can jury-rig devices together from inappropriate materials by increasing the DR by 10.
        The materials do not have to be well-suited to the device's construction, but they must be physically capable of performing any necessary actions.
        For example, you could construct a simple arrow-throwing trap from bent sticks or creatively strung rope, but not from sand.
        Especially appropriate or inappropriate materials may decrease or further increase the DR.\

    \subsection{Subvert Device}
        As a standard action, you can make a Devices check to subvert the normal functionality of a device. The DR is 5 higher than normal. This could allow you to sabotage a hinge or lock so it breaks once it is used, disable a trap without triggering it, or disable a lock so it is perpetually closed or open. Failure means you were unsuccessful, and your action was wasted. Critical failure means you break the device immediately, think you sabotaged the device when you did not, or activate the trap, depending on the circumstances.

        Once you have disabled a device, you can attempt to recover it for later use if you can carry it. This requires a separate Devices check to subvert the device. The DR is 5 higher than normal, as usual for a check to subvert a device.

\newpage
\skill{Disguise}{Int}
        The Disguise skill represents your ability to create disguises to conceal the appearance of creatures or objects. This skill does not help you act appropriately while disguised; see Perform (acting) and Bluff.

    \subsection{Conceal Object}
        As a standard action, you can make a Disguise check to conceal a creature or object on your person.
        The target must be at least two size categories smaller than you are.
        A creature must be an \glossterm{ally}, and an object must be unattended.

        Your Disguise check is opposed by the Awareness check of anyone observing you.
        Success means they fail to notice the item. A creature directly interacting you to search for the object, such as by frisking you, gains a \plus5 bonus to its Awareness check.

    \subsection{Disguise Creature}\label{Disguise Creature}
        You can make a Disguise check to change the appearance of a creature using makeup, costumes, and so forth. An observer can perceive the presence of a disguise with a Disguise or Awareness check to identify a disguise that is higher than your Disguise check. The effectiveness of a disguise depends in part on how much you're attempting to change the creature's appearance, as shown on the table below. The modifiers are cumulative; use any that apply.

        The Disguise check is made secretly, so that you can't be sure how good the result is. However, you can attempt to judge its effectiveness with an identify disguise check using either Disguise or Awareness.

        \parhead{Creation Time} Creating a disguise takes 30 minutes. You can take a \minus5 penalty to reduce the time to 5 minutes, a \minus10 penalty to reduce the time to 5 rounds, or a \minus15 penalty to reduce the time to a \glossterm{standard action}.

        \begin{dtable}
            \begin{dtabularx}{\columnwidth}{l >{\ccol}X}
                \tb{Characteristic} & \tb{Disguise Check Modifier} \tableheaderrule
                Different gender          & \minus2       \\
                Different species or subtype & \minus2       \\
                Different age category    & \minus2\fn{1} \\
                Different creature type   & \minus5       \\
                Additional limb           & \minus5\fn{2} \\
                Different size category   & \tdash\fn{3}  \\
            \end{dtabularx}
            1 Per step of difference between your actual age category and your
            disguised age category. The steps are: young, adulthood, middle age, old, and venerable. \\
            2 Per limb. You must have suitable disguise materials available. \\
            3 You cannot disguise yourself as a different size category.
        \end{dtable}

    \subsection{Emulate Creature}
        You can make a Disguise check to make a creature look like another creature you have previously observed. This functions like the \textit{disguise creature} ability, but the result of your Disguise check can't exceed the result of an Awareness check you have made to observe the creature. People viewing the disguise who know what that person looks like get a \plus5 bonus on their checks to identify the disguise.

    \subsection{Identify Disguise [Hidden]}
        When you observe a disguised creature or object, you can make an Disguise check to identify the disguise.
        The DR is equal to the Disguise check result used to create the disguise.
        Success means you know that the creature or object is disguised.
        Critical success means you can also discern its true appearance beneath the disguise.
        Failure means you don't notice anything unusual.

        You cannot retry this check until you gain meaningful new information that would help you identify the disguise.

\newpage
\skill{Escape Artist}{Dex}
        The Escape Artist skill represents your ability to escape bindings and move through small areas by contorting your body.

    \subsection{Escape Bindings}
        As a standard action, you can make an Escape Artist check to escape bindings and restraints. The DRs of various restraints are given on the table below.

        \begin{dtable}
            \begin{dtabularx}{\columnwidth}{>{\lcol}X l}
                \tb{Restraint}  & \tb{Escape Artist DR} \tableheaderrule
                Ropes                               & Binder's grapple or Devices check \\
                Net                                 & 10                                \\
                Manacles                            & 20                                \\
                Masterwork manacles                 & 30                                \\
                Grappler                            & Grappler's attack result  \\
                \spell{Entangle} and similar spells & Spellcaster's attack result       \\
            \end{dtabularx}
        \end{dtable}

    \subsection{Tight Squeeze}
        As a standard action, you can use the \textit{tight squeeze} ability to squeeze into spaces too small to normally fit you.
        \begin{freeability}{Tight Squeeze}
            Make an Escape Artist check to move one foot forward into a tight space.
            A DR 15 check allows you to move in a space that can fit your head and shoulders, but which is too tight to allow normal crawling.
            A DR 20 check allows you to move in a space that can fit your head, but not your shoulders.
            Success means you make progress through the space, while failure means your action is wasted.

            This functions like \glossterm{squeezing}, except that the penalties are increased to \minus4.
            If you are squeezing in a space that cannot fit your shoulders, you are also treated as \glossterm{helpless}.
        \end{freeability}

\newpage
\skill{Heal}{Int}
        The Heal skill allows you to tend the wounds of others. In order to heal a creature, you must be able to see and touch it.

    \subsection{Accelerate Recovery}\label{Accelerate Recovery}
        You can make a DR 10 Heal check to accelerate the recovery of up to four willing creatures during a \glossterm{long rest}.
        Success means that each creature removes an additional \glossterm{vital wound} during the rest.
        For every 10 points by which you succeed, each creature can removes an additional \glossterm{vital wound}.

        To accelerate a creature's recovery, you need a few items and supplies (bandages, salves, and so on) that are easy to come by in civilized areas.
        You can accelerate the recovery of additional creatures during the same rest by taking a cumulative \minus2 penalty per additional creature.

    \subsection{First Aid}
        As a standard action, you can make a Heal check to prevent a willing creature from dying from a \glossterm{vital wound} with a negative \glossterm{wound roll}.
        The DR is equal to 5 \add 4 for each \glossterm{vital wound} the creature has.
        Success means that the negative \glossterm{wound roll} for that creature's \glossterm{vital wound} changes to 0, leaving the target unconscious but alive.
        For details, see \pcref{Vital Wounds}.

    \subsection{Treat Poison or Disease}
        You can make a Heal check to treat poison or disease in a character.
        To resist the next attack by the poison or disease, it can use your Heal check or its Fortitude defense, whichever is higher.
        Treating a poison takes a standard action. Treating a disease takes five minutes of work.

    \subsection{Treat Wound}
        As a standard action, you can make a Heal check to treat some specific wounds, such as from a caltrop. Success usually means the wound is gone, as indicated by the effect's description.

\newpage
\skill{Intimidate}{Varies}
        The Intimidate skill represents your ability to intimidate and coerce people into doing what you want.

        \parhead{Check Modifiers}
        You gain a bonus of up to \plus10 on Intimidate checks if the target thinks you or your group is stronger than it is, or that it is otherwise in some real danger from you. Likewise, you take a penalty of up to \minus10 if the target thinks you or your group is weaker than it is, or that there is otherwise no chance that you could cause it harm.

        \parhead{Choosing an Attribute}
        Depending on how you are trying to intimidate creatures, you can use any attribute as a key attribute for Intimidate.
        For example, if you intimidate a creature by smashing a table and threatening to smash its head in, you can use Strength to make the Intimidate check.
        On the other hand, if you intimidate a creature by staring it down in a cold fury, you can use Willpower to make the Intimidate check.

    \subsection{Augment Threat}\label{Augment Threat}
        As a \glossterm{minor action}, you can use the \textit{augment threat} ability to look more threatening.
        \begin{freeability}{Augment Threat}[\glossterm{Sustain} (minor)]
            Choose any number of creatures within \rnglong range.
            Make an Intimidate check against the Sense Motive checks of each target.
            \hit You gain a \plus2 bonus to \glossterm{threat} against each target.
        \end{freeability}

    \subsection{Coerce}
        You can make an Intimidate check to convince a creature to do what you want. This functions like a Persuasion check to form an agreement with a group, except that you do not apply a relationship modifier. In addition, the DR is up to 10 lower if the group thinks your group is significantly stronger than them, or up to 10 higher if the group thinks your group is significantly weaker.

    \subsection{Demoralize}\label{Demoralize}

        As a standard action, you can use the \textit{demoralize} ability to intimidate creatures in combat.

        \begin{freeability}{Demoralize}
            Make an attack vs. Mental against a creature within \rngmed range.
            Your \glossterm{accuracy} is equal to your Intimidate skill.
            \hit The target is \shaken by you until the end of the next round.
        \end{freeability}

\newpage
\skill{Jump}{Str}
        The Jump skill represents your ability to jump. All Jump checks are made as part of movement, so they take no special action to perform. Distance moved with Jump checks is counted against your normal maximum movement in a phase.

        Several modifiers apply to all Jump checks, which are described below.

        \parhead{Running Start}\label{Running Start} If you move at least twenty feet before jumping, you have a running start.
        Jumping without a running start is difficult.
        If you make a Jump attack or check without a running start, you roll twice and take the lower result.

    \subsection{Hop Up}
        You can make a DR 5 Jump check while moving to jump up onto an object as tall as your waist, such as a table or small boulder. Doing so counts as 5 feet of movement. Success means you land on top of the object and can continue your movement. Failure indicates you stop your movement where you are. You do not need to get a running start to hop up.

    \subsection{Jump Down}
        You can make a DR 5 Jump check while moving to intentionally jump down from a height to reduce your falling damage. Success means you take damage as if you had dropped 10 fewer feet than you actually did. Failure means you take the normal falling damage. You do not need to get a running start to hop up.

    \subsection{Leap}\label{Leap}
        As part of movement, you can make a Jump check to jump.
        You move forward any number of feet, up to a maximum equal to one quarter of your land speed \add your Jump check result.
        Your maximum height must be no greater than half of your Jump check result, and at least equal to a quarter of your forward distance travelled.
        For example, if you have a land speed of 30 feet and you get a Jump check result of 20, you can move forward a maximum of 25 feet.
        If you instead jump forward twenty feet, your maximum height must be between 5 and 10 feet.

        You always reach your maximum height at the midpoint of the jump.
        However, you can interrupt your leap before travelling the full horizontal distance.
        For example, if you need to travel five feet forward and five feet vertically to reach a rope, you can start a leap which would take you ten feet forward and reach a maximum height of five feet.
        Making such a leap would require a Jump check result of 10.
        When you reach the rope, you can stop your movement there, ignoring the forward motion which would make you travel the full ten feet.

        When leaping, your movement may not stopped by hitting the ground after travelling the normal distance, such as if you jump off of a ledge.
        In that case, you move one quarter of your jump distance farther forward as you fall before your fall becomes entirely downward.
        If an insufficiently long jump would cause you to fall into a gap, you can attempt to stop your fall (see \pcref{Stop Fall}) if you can reach the wall.

    \subsection{Rebounding Leap}\label{Rebounding Leap}
        While in midair, if you make contact with a solid object that can support your weight, you can jump off of that object, as the \textit{leap} ability.
        You are not considered to have a running start.
        In addition, you take a \minus5 penalty to the check (in addition to the penalty for not having a running start), because rebounding off of an object in midair is difficult.
        You must travel at least 10 feet in the air between each rebounding jump.

\newpage
\skill{Knowledge}{Int}
        Like the Craft, Profession, and Perform skills, Knowledge actually encompasses a number of separate skills. Knowledge represents a study of some body of lore, possibly an academic or even scientific discipline. Below are listed typical fields of study.
        \begin{itemize}
            \item Arcana (ancient mysteries, magic traditions, arcane symbols,
                cryptic phrases, constructs, dragons, magical beasts)
            \item Engineering (architecture, buildings, bridges, fortifications, siege weapons)
            \item Dungeoneering (aberrations, caverns, oozes, spelunking, subterranean monsters)
            \item Geography (lands, terrain, climate, people, outdoor monsters)
            \item Local (humanoids, legends, inhabitants, laws, customs, history, nobility, royalty)
            \item Nature (animals, fey, giants, monstrous humanoids, plants, seasons and cycles, weather, vermin)
            \item Planes (the Primal Planes, the Aligned Planes, the Astral Plane,
                outsiders, elementals, magic related to the planes, extraplanar monsters)
            \item Religion (gods and goddesses, mythic history, ecclesiastic tradition, holy symbols, undead)
        \end{itemize}

        You cannot retry Knowledge checks until you are presented with significant new information about the subject that could jog your memory.

    \subsection{Identify Monster}
        You can make a Knowledge check reactively to identify a monster and recall its special powers or vulnerabilities. In general, the DR is equal to 5 \add the monster's level. Success allows you to remember the monster's name and its most well-known features. For every 5 points by which you beat the DR, you remember an additional piece of useful information. Failure indicates you don't remember anything important about the monster. Critical failure means you remember incorrect information.

    \subsection{Answer Question/Recall Information}
        You can make a Knowledge check reactively to remember information related to your field of study. The DR varies depending on the difficulty of the question. Answering a simple or trivial question that most students would be familiar with is DR 5. Answering a challenging question which would be beyond the reach of most initiates is DR 15. Answering a deeply complex or obscure question which might require information known only by the most learned masters in the field would be DR 20 or higher.

    \subsection{Appraise Item}
        You can make a Knowledge check to estimate the value of an item with 1 minute of careful study. The Knowledge skill used must be related to the item.

        The DR depends on the rarity of the item. Common items, such as nonmagical equipment and utensils, are DR 5. Rare items, such as valuable gems and magic items worth less than 100,000 gp, are DR 10. Extraordinarily rare items, such as unique gems and magic items worth at least 100,000 gp, are DR 20.

        Success means you know the value of the item. Failure means you think the item is worth (d10 \add 5)/10 \x the item's actual value. (The d10 roll is made secretly, so you don't know how close you are.) Critical failure means you can't estimate the item's value at all.

        After you appraise an item, you can't appraise it again with any skill until you gain a level.

\newpage
\skill{Linguistics}{Int}
        The Linguistics skill represents your mastery of spoken and written languages (see \pcref{Languages}).
        Normally, you don't make Linguistics checks to speak or understand languages.
        You either know a language or you don't.
        However, training in Linguistics causes you to learn additional languages, and you can use Linguistics to attempt to decipher unfamiliar languages.

        \parhead{Learning Languages}\label{Learning Languages}
        If you are trained in Linguistics, you learn additional \glossterm{common languages} equal to one plus one quarter of your level.
        If you have mastered Linguistics, you instead learn additional common languages equal to two plus half your level.
        In place of two common languages, you may instead learn a \glossterm{rare language}.
        Rare languages are more difficult to learn, and are usually only spoken by unusual creatures.

        Druidic is a special language.
        Druids learn Druidic as part of their initation as druids, and are forbidden from teaching it to non-druids.
        As a result, it cannot be learned through ordinary means.
        Druidic uses its own alphabet.

    \subsection{Decipher Script}
        You can make a Linguistics check to decipher writing in an unfamiliar language or a message written in an incomplete or archaic form. The base DR is 10 for the simplest messages, 15 for standard texts, and 20 or higher for intricate, exotic, or very old writing. In addition, the DR increases by 5 if you do not know any languages that use the same alphabet as the writing being deciphered. Deciphering the equivalent of a single page of script takes 1 minute.

        Success means you understand the general content of a piece of writing about one page long (or the equivalent). Failure means you fail to understand the writing. Critical failure means you to draw a false conclusion about the text. The check to decipher the writing is made secretly, so that you can't tell whether the conclusion you draw is true or false.

    \subsection{Identify Language}
        You make a DR 10 Linguistics check to identify the language used in speech or writing, even if you can't understand the language.
        For details about languages, see \pcref{Languages}.

\newpage
\skill{Perform}{Varies}
        \par Like Craft, Knowledge, and Profession, Perform is actually a number of separate skills.
        You could have several Perform skills, each with a separate degree of training.
        Each of the nine categories of the Perform skill includes a variety of methods, instruments, or techniques, a small list of which is provided for each category below.

        \begin{itemize}
            \item Acting (drama, impersonation, mime)
            \item Comedy (buffoonery, limericks, joke-telling)
            \item Dance (ballet, waltz, jig)
            \item Keyboard instruments (harpsichord, piano, pipe organ)
            \item Oratory (epic, ode, storytelling)
            \item Percussion instruments (bells, chimes, drums, gong)
            \item String instruments (fiddle, harp, lute, mandolin)
            \item Wind instruments (flute, pan pipes, recorder, shawm, trumpet)
            \item Sing (ballad, chant, melody)
        \end{itemize}

        \parhead{Choosing an Attribute}
        Depending on how you are trying to perform, you can use any attribute as a key attribute for Perform.

        \parhead{Performance Types}
        In general, there are three types of performances: instrumental performances, visual performances, and vocal performances.
        \subparhead{Instrumental Performances} These performances require an instrument of some sort to create sound, which always requires at least one hand to manipulate the instrument.
        They affect all creatures who can hear the performance.
        \subparhead{Visual Perforamances} These performances require some sort of motion, such as juggling.
        They affect all creatures who can see the performance.
        \subparhead{Vocal Performances} These performances use your voice to create sound.
        These are simpler to create than instrumental performances, since they don't require an instrument.
        However, vocal performances are always in a particular language.
        They affect all creatures who can hear the performance and understand the language the performance is in.

        \parhead{Limitations while Performing}
        While you are performing, your actions are slightly limited.
        You cannot cast spells or take other actions requiring similar levels of focus and concentration.
        In addition, you take a \minus10 penalty to the Perform skill for any other performances.
        This penalty stacks.
        For example, if you were playing a lyre, singing, and juggling balls with your feet, you would take a \minus10 penalty to your singing and a \minus20 penalty to your juggling.
        These limitations are in addition to any restrictions imposed by your method of performing, such as your hands being occupied playing an instrument.

        You can otherwise act normally while performing, including attacking in combat, if doing so is physically possible.

        \parhead{Performance Time}
        In general, you can maintain a performance for a number of minutes equal to 5 \add your Constitution.
        After that time, you must rest for 5 minutes before performing again.

    \subsection{Entertain}
        You can make a Perform check to provide entertainment or to show off your skills.

    \subsection{Earn Income}
        You can make a Perform check to practice your trade and make a decent living, earning about half your Perform check result in gold pieces per week of dedicated performance.

\newpage
\skill{Persuasion}{---}
        You can use the Persuasion skill to convince people to do or think what you want. Depending on how it is used, it represents a combination of verbal acuity, tact, argumentative ability, grace, etiquette, and personal magnetism. Using a Persuasion check usually takes at least a minute of sustained conversation.

        Persuasion checks are usually made against a group. For the purposes of a Persuasion check, a ``group'' consists of creatures who consider themselves to be \glossterm{allies} and who share similar information or backgrounds. For example, in a king's court, you cannot simply influence the king alone; his trusted advisors are also part of the same group. It is possible to influence one group without influencing another. For example, the king and his advisors may be unpersuaded, but the prince may find your arguments compelling. Groups can also consist of a single creature. The game master decides what the groups are.

        The base DR for a Persuasion check against a group is equal to 5 \add the highest level of any character in the group or the highest Sense Motive of any character in the group, whichever is higher.

        Not all social interactions require Persuasion checks. Much of the time, being extraordinarily persuasive is unnecessary, and creatures can be convinced with normal, inartful conversation and good reasoning. Persuasion checks should only be used when your personal persuasiveness matters.

    \subsection{Compel Belief}\label{Compel Belief}
        As part of conversation, you can make a Persuasion check to cause creatures to believe something you say. If you are lying, you must also make a Bluff check to lie. Success means the group believes what you are saying is true. What they choose to do with that information is up to them, however. Failure means they do not believe you, but they do not react poorly; perhaps they simply want more verification. You may be able to try again, depending on their patience. Critical failure means the group reacts poorly, and you may have permanently damaged your credibility with them.

        Your check is modified by how believable your argument is, as well as whether the group has strong feelings about the truth of your story.

        \begin{dtable}
            \lcaption{Believability Modifiers}
            \begin{dtabularx}{\columnwidth}{X l}
                \tb{Description} & \tb{DR Modifier}  \tableheaderrule
                Expected to be true (``Nothing interesting happened while I was on patrol'') & \minus5         \\
                Plausible (``The mayor is too busy to see you now.'')                        & \plus0          \\
                Unlikely (``The north gate is under attack!'')                               & \plus5          \\
                Extremely unlikely (``The mayor is secretly a werewolf.'')                   & \plus10         \\
                Virtually impossible (``Your husband is secretly a werewolf.'')              & \plus15 or more \\
                Demonstratably untrue (``You are secretly a werewolf.'')                     & \tdash\fn{1}    \\
            \end{dtabularx}
            1 You cannot convince someone of something that is proven to be false.  \\
        \end{dtable}

        \begin{dtable}
            \lcaption{Motivation Modifiers}
            \begin{dtabularx}{\columnwidth}{X l}
                \tb{Description}                                                           & \tb{DR Modifier} \tableheaderrule
                Target wants to believe (``That dress looks lovely on you.'')              & \minus5          \\
                Target does not have strong feelings (``I'm busy.'')                       & \plus0           \\
                Target doesn't want the story to be true (``Your brother is a murderer.'') & \plus5           \\
            \end{dtabularx}
        \end{dtable}

    \subsection{Form Agreement}
        You can make a Persuasion check to convince a group to accept a deal you offer. This can be used to persuade the chamberlain to let you see the king, to negotiate peace between feuding barbarian tribes, or to convince the ogre mages that have captured you that they should ransom you back to your friends instead of twisting your limbs off one by one.

        Success means the group will fulfill their end of the deal. Of course, if you don't fulfill your part, they are likely to react poorly. Failure means they did not accept the deal, but they may propose an alternate arrangement, or you can propose another deal without penalty. Critical failure means that there is virtually no chance to reach an agreement, and the group may become hostile or take other steps to end the conversation. They may have been insulted by how unfair the deal was, or you may have made a critical verbal misstep.

        Your check is modified by your relationship with the group and how favorable the deal is.

        \begin{dtable}
            \begin{dtabularx}{\columnwidth}{>{\lcol}X r}
                \tb{Relationship} & \tb{Modifier} \tableheaderrule
                Intimate: Someone who with whom you have an implicit trust.
                Example: A lover or spouse. & \minus15 \\
                Friend: Someone with whom you have a regularly positive personal relationship.
                Example: A long-time buddy or a sibling. & \minus10 \\
                Ally: Someone on the same team, but with whom you have no personal relationship.
                Example: A cleric of the same religion or a knight serving the same king. & \minus5 \\
                Acquaintance (Positive): Someone you have met several times with no particularly negative experiences. Example: The blacksmith that buys your looted equipment regularly. & \minus2 \\
                Just Met: No relationship whatsoever.
                Example: A guard at a castle or a traveler on a road. & \plus0 \\
                Acquaintance (Negative): Someone you have met several times with no particularly positive experiences. Example: A town guard that has arrested you for drunkenness once or twice. & \plus2 \\
                Enemy: Someone on an opposed team, with whom you have no personal relationship.
                Example: A cleric of a philosophically-opposed religion or an orc bandit who is robbing you. & \plus5 \\
                Personal Foe: Someone with whom you have a regularly antagonistic personal relationship.
                Example: An evil warlord whom you are attempting to thwart, or a bounty hunter who is tracking you down for your crimes. & \plus10 \\
                Nemesis: Someone who has sworn to do you, personally, harm. Example: The brother of a man you murdered in cold blood. & \plus15 \\
            \end{dtabularx}
        \end{dtable}
        \begin{dtable*}
            \begin{dtabularx}{\textwidth}{>{\lcol}X r}
                \tb{Risk vs. Reward Judgement (Persuasion)} & \tb{Modifier} \tableheaderrule
                Fantastic: The reward for accepting the deal is very worthwhile; the risk is either acceptable or extremely unlikely. The best-case scenario is a virtual guarantee. Example: An offer to pay a lot of gold for information that isn't important to the character. & \minus15 or more \\
                Good: The reward is good and the risk is minimal. The subject is very likely to profit from the deal. Example: An offer to pay someone twice their normal daily wage to spend their evening in a seedy tavern with a reputation for vicious brawls and later report on everyone they saw there. & \minus10 \\
                Favorable: The reward is appealing, but there's risk involved. If all goes according to plan, though, the deal will end up benefiting the subject. Example: A request for a mercenary to aid the party in battle against a weak goblin tribe in return for a cut of the money and first pick of the magic items. & \minus5\\
                Even: The reward and risk more of less even out; or the deal involves neither reward nor risk. Example: A request for directions to a place that isn't a secret. & \plus0 \\
                Unfavorable: The reward is not enough compared to the risk involved. Even if all goes according to plan, chances are it will end badly for the subject. Example: A request to free a prisoner the target is guarding for a small amount of money. & \plus5\\
                Bad: The reward is poor and the risk is high. The subject is very likely to get the raw end of the deal. Example: A request for a mercenary to aid the party in battle against an ancient red dragon for a small cut of any non-magical treasure. & \plus10 \\
                Horrible: There is no conceivable way that the proposed plan could end up with the subject ahead or the worst-case scenario is guaranteed to occur. Example: An offer to trade a rusty kitchen knife for a shiny new longsword. & \plus15 or more \\
            \end{dtabularx}
        \end{dtable*}

    \subsection{Gather Information}
        An evening's time, a few gold pieces for buying drinks and making friends, and a DR 5 Persuasion check get you a general idea of a city's major news items, assuming there are no obvious reasons why the information would be withheld. The higher your check result, the better the information.

        If you want to find out about a specific rumor, or a specific item, or obtain a map, or do something else along those lines, the DR for the check is generally at least 5.
        The difficulty depends on how widely known and shared the information you seek is.

\newpage
\skill{Profession}{Varies}
        Like Craft, Knowledge, and Perform, Profession is actually a number of separate skills.
        You could have several Profession skills, each with a separate degree of training.
        While a Craft skill represents ability in creating or making an item, a Profession skill represents an aptitude in a vocation requiring a broader range of less specific knowledge.
        Most commoners have some training in a Profession skill.

        \parhead{Choosing an Attribute}
        Depending on your profession, you can use any attribute as a key attribute for Profession.

    \subsection{Appraise Item}
        You can make a Profession check to estimate the value of an item with 1 minute of careful study. The Profession skill used must be related to the item.
        This check is always Intelligence-based, regardless of your profession.

        The DR depends on the rarity of the item. Common items, such as nonmagical equipment and utensils, are DR 5. Rare items, such as valuable gems and magic items worth less than 100,000 gp, are DR 10. Extraordinarily rare items, such as unique gems and magic items worth at least 100,000 gp, are DR 20.

        Success means you know the value of the item. Failure means you think the item is worth (d10 \add 5)/10 \x the item's actual value. (The d10 roll is made secretly, so you don't know how close you are.) Critical failure means you can't estimate the item's value at all.

        After you appraise an item, you can't appraise it again with any skill until you gain a level.

    \subsection{Earn Income}
        You can make a Profession check to practice your trade and make a decent living, earning about half your Profession check result in gold pieces per week of dedicated work. You know how to use the tools of your trade, how to perform the profession's daily tasks, how to supervise helpers, and how to handle common problems.

    \subsection{Perform Task}
        You can make a Profession check to perform some tasks related to your profession. This allows you to use Profession in place of other skills when it is appropriate. For example, a sailor could use Profession to tie common knots in place of Devices or Survival, or a farmer could use Profession to identify common animals and plants in place of Knowledge (nature). The DR when using Profession may be higher than it would be to use the normal skill for the task.

\newpage
\skill{Ride}{Dex}
        The Ride skill allows you to ride horses and other mounts. Typical riding actions don't require checks. You can saddle, mount, ride, and dismount from a mount without a problem. However, some special actions require Ride checks. Some modifiers apply to all Ride checks, as described at \pcref{Ride Modifiers}.

    \subsection{Control Mount}
        When riding an \glossterm{ally} in combat that is not trained for battle, you must a DR 10 Ride check as a move action to control it. Success means it obeys your commands that round. Failure means it remains still and refuses to obey your commands until you make a successful check to control it. Critical failure means the mount acts of its own volition.

    \subsection{Fall}
        If you fall off your mount, or if your mount is downed in battle, you take damage from the fall.
        Falling off of a typical Large creature, such as a horse, deals 1d6 bludgeoning damage, while falling from larger creatures deals damage appropriate to the distance fallen.

    \subsection{Guide Mount}
        While riding an \glossterm{ally}, you must make a DR 0 Ride check at the start of your turn to guide it with your knees. Success means it obeys your commands, and you can have both hands free to take other actions. Failure means you must use a hand to control the mount that round. Critical failure means the mount acts of its own volition. If you cannot use a hand to control the mount, you fall off the mount if it moves during your turn.

    \subsection{Leap}
        You can make a DR 5 Ride check to stay on your mount as it jumps. This check is made as part of your mount's movement. Success means that you stay on the mount. Failure means you fall off your mount as it starts to jump.

    \subsection{Spur Mount}
        You can make a DR 5 Ride check as a move action to get your mount to move faster. Success means it takes the \textit{sprint} action to move faster (see \pcref{Sprint}).
        Failure means your action was wasted.

    \subsection{Stay in Saddle}
        If you take damage or your mount rears or bolts unexpectedly, you must make a DR 5 Ride check to stay in your saddle. This does not take an action. Success means you stay in your saddle. Failure means you fall off your mount.

    \subsection{Take Cover}
        You can make a DR 10 Ride check as a move action to drop low and take cover behind your mount. This requires the use of both your hands. Success means you gain the benefits of active cover. Failure means you can't get low enough and gain no benefit from the action. Critical failure means you fall off your mount. You can leave this cover and resume your normal position as a move action that does not require a Ride check.

    \subsection{Ride Modifiers}\label{Ride Modifiers}
        If a mount is not trained as a mount, the DR to ride it increases by 10.
        If it lacks a saddle and other riding gear, the DR to ride it increases by 5. If it takes a standard action other than movement, such as attacking, the DR to ride it that round increases by 5.

\newpage
\skill{Sense Motive}{Per}
        The Sense Motive skill represents your ability to read body language and emotion.
        Most Sense Motive tasks are \glossterm{hidden tasks}.

    \subsection{Discern Enchantment [Hidden]}
        When you interact with a creature, you can try to notice whether it is affected by mind-affecting abilities with a Sense Motive check.
        If the creature is not affected by any such abilities, the check automatically fails.
        If the creature is affected by Compulsion or Emotion effects that are not currently altering its behavior, the check also automatically fails.
        If the creature's behavior is currently being altered by a \glossterm{Compulsion} effect, the DR is 10, and success means you identify the presence of a Compulsion effect.
        If the creature's behavior is currently being altered by an \glossterm{Emotion} effect, the DR is 20, and success means you identify the presence of an Emotion effect.
        Failure means you do not notice any such effects on the creature.

        You can also make this check to identify \glossterm{Subtle} effects on yourself, using the same DRs.

    \subsection{Discern Lies [Hidden]}
        When you observe a creature speak, you can make a Sense Motive check.
        The DR is equal to the speaking creature's Bluff check result.
        Success means you identify whether the creature was lying.
        Failure means you do not notice any indication that the creature is lying.

    \subsection{Discern Secret Message}
        When you observe a hidden message being conveyed, you can make a Sense Motive check.
        The DR is equal to the DR of the secret message (see \pcref{Bluff}).
        Success means you recognize that a hidden message is present, but not its contents.
        Critical success means you can understand the message.
        Failure means you don't notice the hidden message.

    \subsection{Social Assessment}\label{Social Assessment}
        You can make a DR 5 Sense Motive check to get a general assessment of a social situation.
        This does not take an action by itself, but you must observe the group for at least a minute.
        Success means you learn a piece of useful information about the situation, such as a general understanding of expected behaviors or a rough understanding of the social hierarchy.
        For every 5 points by which you beat the DR, you gain an additional insight into the situation.

        You can make a social assessment after only a single round of observation, but you take a \minus10 penalty on the check.
        If you don't understand the language the group is using, you take a \minus10 penalty on the check.
        The information gained at a given DR may vary in usefulness depending on how obvious or subtle the situation is.

\newpage
\skill{Sleight of Hand}{Dex}

        The Sleight of Hand skill represents your ability to pick pockets, palm objects, and perform other feats of legerdemain.

        All Sleight of Hand checks apply a special modifier based on the size of the action taken or object affected, as shown on \trefnp{Sleight of Hand Modifiers}.

        \begin{dtable}
            \lcaption{Sleight of Hand Modifiers}
            \begin{dtabularx}{\columnwidth}{X l}
                \tb{Size} & {Check Modifier} \tableheaderrule
                Fine        & \plus8   \\
                Diminuitive & \plus4   \\
                Tiny        & \plus0   \\
                Small       & \minus4  \\
                Medium      & \minus8  \\
                Large       & \minus12 \\
                Huge        & \minus16 \\
                Gargantuan  & \minus20 \\
                Colossal    & \minus24 \\
            \end{dtabularx}
        \end{dtable}

    \subsection{Conceal Action}
        You can make a Sleight of Hand check to conceal an action you take, preventing observers from noticing that you took the action. You can conceal any physical action that only requires the use of your hands and arms. Your check is opposed by the Awareness check of any observers. Success means your action is unnoticed. Failure means they are able to notice your action normally.

        The space required to perform the action is the size of the action, and applies a bonus or penalty as noted above. For example, winking only requires moving your eye, so you gain a \plus8 bonus to conceal the action. Firing a longbow requires manipulating an object as large as you are, so you take a \minus8 penalty to conceal the action.

        Observers that you touch as part of the action gain a \plus10 bonus to their Awareness check. If you strike them hard, as with an attack, the bonus increases to a \plus20 bonus.

        If you successfully conceal a ranged attack, the attack is treated as if it were an attack from an invisible creature. The target may be \unaware of the attack. If the target is hit, it can tell the direction the attack came from, but not that you made the attack.

        Nonhumanoid creatures without hands and arms may conceal actions using different parts of their body. Large movements can never be concealed with this ability; see the Stealth skill.

    \subsection{Conceal Object}
        You can make a Sleight of Hand check as a standard action to conceal an item on your person. The object must be at least two size categories smaller than you are. Your Sleight of Hand check is opposed by the Awareness check of anyone observing you. Success means they fail to notice the item. A creature directly interacting you to search for the object, such as by frisking you, gains a \plus5 bonus to its Awareness check.

    \subsection{Pickpocket}
        You can make a Sleight of Hand check as a standard action to steal an object from another creature. The object must be loose and accessible, such as in a pocket. The DR depends on whether the creature notices your attempt using Awareness. If the creature's Awareness check exceeds your Sleight of Hand check, the creature notices your attempt and the DR is equal to the creature's Reflex defense. Otherwise, the creature does not notice your attempt, and the DR is 10. Success means you successfully steal the object. Failure means you do not steal the object.

\newpage
\skill{Spellcraft}{Per}
        The Spellcraft skill represents your ability to notice and understand spells and magical effects.

        While sleeping, you take a \minus10 penalty to the Spellcraft skill.

        % TODO: add ability to identify that a creature is attuned to active effects.

    \subsection{Identify Magical Effect}
        When you observe a magical effect, you can make a Spellcraft check to identify its nature.
        This grants you no special ability to notice hidden magical effects, but can allow you to understand magical effects you have already noticed.
        The DR is equal to 5 \add the \glossterm{power} of the effect.
        Success means you know in general terms what the effect does.
        Critical success means you know exactly what the effect does, and if it is a common effect, what ability caused it.
        Failure means you do not recognize the effect.

        If the effect has obvious visual or other cues to its true nature, such as a wall of fire, the DR is lowered by 5.
        If the effect has obvious cues that are misleading, such as a wall of fire that heals creatures that pass through it, the DR is increased by 5.

        You cannot retry this check until you gain meaningful new information that would help you identify the effect.

    \subsection{Identify Spellcasting}
        You can identify spells being cast within 100 foot \glossterm{range} of you.
        The DR is equal to 5 \add the spell level of the spell.
        Success means you know what spell is being cast.
        Failure means you do not.

    \subsection{Identify Potion}
        You can make a DR Spellcraft check to identify a potion.
        This takes a minute of careful evaluation.
        For most potions, the DR is 15, and success means you identify what spell the potion contains.
        Failure means you do not learn anything about the potion's nature.

        Potions can be crafted to conceal their true nature.
        The DR to identify such potions is usually 25.
        Success means you know what spell the potion contains.
        Failure means you identify the potion as whatever spell the potion is intended to resemble.
        Critical failure means you do not learn anything about the potion's nature.

    % TODO: this should give partial information on failure or only give complete information on crit
    \subsection{Identify Magical Writing}\label{Identify Magical Writing}
        You can make a Spellcraft check as a standard action to identify a ritual or similar piece of magical writing.
        The \glossterm{difficulty rating} depends on the complexity of the writing.
        If the writing describes a spell or ritual, the DR equal to 10 \add three times the level of the spell or ritual.
        Success means you understand the magical writing.
        Once you decipher a particular magical writing, you do not need to decipher it again.

    \subsection{Teleport Trace}
        As a standard action, you can make a Spellcraft check to learn information about a teleportation within \rngmed range of you.
        The DR is equal to 10 \add 1 per round since the teleportation occurred.
        Success means you identify the direction of the teleportation.
        Critical success means you also identify the distance.
        Failure means you learn no information about the teleportation.

\newpage
\skill{Stealth}{Dex}
        The Stealth skill represents your ability to escape detection while moving or hiding. All Stealth checks are made as part of movement or other actions, so they require no special action to perform. If you have been noticed by a creature, you automatically fail all Stealth checks against that creature until you can escape its notice, such as by disappearing out of sight.

        \parhead{Size and Stealth}\label{Size and Stealth} A creature larger or smaller than Medium gains an bonus or penalty to the Stealth skill equal to \plus4 per size larger than Medium, or \minus4 for per size smaller than Medium: Fine \plus16, Diminutive \plus12, Tiny \plus8, Small \plus4, Large \minus4, Huge \minus8, Gargantuan \minus12, Colossal \minus16.

    \subsection{Blend In}
        You can make a Stealth check to blend in with a crowd. Your Stealth check is opposed by the Awareness checks of anyone looking for you. Success means you remain unobserved. Failure means the person looking for you found you.

        If you act differently from the crowd you are trying to blend in with, anyone observing you gets a \plus5 bonus to find you. You may need to make Awareness or Sense Motive checks to figure out how to act, depending on the situation. If you are extremely dissimilar from the crowd, such as a naked person in church or a human among halflings, anyone observing you may gain a \plus10 or greater bonus to find you.

    \subsection{Hide [Hidden]}
        As a move action, or as part of movement, you can make a Stealth check to hide so people can't see you.
        Your Stealth check is opposed by the Awareness checks of any observers.
        Success means that you can't be seen, heard, or detected in any way.
        Failure means that the observer can observe you using any senses they detected you with.

        If you do not have passive cover or concealment from a creature (see \pcref{Cover} and \pcref{Concealment}), your Stealth check is automatically treated as a 0 against sight-based Awareness checks that creature makes, regardless of your modifiers.
        For this purpose, do not consider any cover that would be hidden as a result of a successful check, such as a shield you hold in front of you.

        If you move at up to half your speed during your turn, you take a \minus5 penalty to the Stealth skill.
        If you move at up to your full speed during your turn, you take a \minus10 penalty to the Stealth skill.
        It's practically impossible (\minus20 penalty) to remain unobserved while attacking, sprinting, or charging.

\newpage
\skill{Survival}{Per}
        The Survival skill represents your ability to take care of yourself and others in the wilderness, as well as your mastery of various survival-related tasks.

    \subsection{Navigate Wilderness}
        You can make a Survival check while moving overland to avoid natural hazards and getting lost. The DR depends on the terrain, as shown on \trefnp{Terrain DRs}. Success means that you and any group you lead can move without any trouble. Failure means you may encounter a natural hazard, depending on the terrain. Critical failure means you become lost.

        You can notice that you are lost the next time that you succeed on a check to navigate in the wilderness. Rediscovering your location requires backtracking for 1d6 hours and another Survival check against the same DR\@.

        This check is made once every 8 hours you spend travelling overland. If you move at half speed, you gain a \plus5 bonus on the check.

    \subsection{Sustenance}
        You can make a Survival check to move overland at half speed while hunting and foraging. The DR depends on the terrain, as shown on \trefnp{Terrain DRs}. Success means that you can feed yourself, and require no additional food or water. For every 2 points by which you succeed, you can provide food and water to an additional person.

        This check is made once every 8 hours you spend travelling overland. If you move at one-quarter speed, you gain a \plus5 bonus on the check.

        \begin{dtable}
            \lcaption{Terrain DRs}
            \begin{dtabularx}{\columnwidth}{X l l}
                \tb{Terrain} & \tb{Navigation DR} & \tb{Sustenance DR} \tableheaderrule
                Desert    & 15 & 20 \\
                Forest    & 10 & 15 \\
                Jungle    & 10 & 10 \\
                Mountains & 10 & 15 \\
                Hills     & 5  & 10 \\
                Plains    & 5  & 10 \\
                Swamp     & 15 & 15 \\
            \end{dtabularx}
        \end{dtable}

    \subsection{Predict Weather}
        You can make a DR 10 Survival check to predict the weather. This requires a minute of observation. Success means you know what the weather will be like up to 24 hours in advance. For every 5 points by which you succeed, you can predict the weather one additional day in advance.

    \subsection{Track}\label{Track}
        As a standard action, you can make a Survival check to follow tracks.
        The DR of the check depends on how easy the tracks are to notice, as shown on \trefnp{Track DRs} and \trefnp{Track Modifiers}.
        You must use this ability each round to continue following the trail, though you do not have to make an additional Survival check each round.
        You must make another Survival check if you change your movement speed, if you follow the trail for 1 mile, or if it becomes especially difficult to follow for any reason.

        If you move at up to half your normal speed as the same round that you use this ability, you take no penalty on the check.
        If you move at your full speed, you take a \minus5 penalty to the check.

        The DR depends on the surface and the prevailing conditions, as given on the table below:
        The base DR to follow tracks is 5 if you use scent to track, regardless of the condition of the ground.

        A creature can use the Awareness skill to notice signs of passage, but the Survival skill is necessary to follow tracks for any distance.

        \begin{dtable}
            \lcaption{Track DRs}
            \begin{dtabularx}{\columnwidth}{l >{\lcol}X l}
                \tb{Surface} & \tb{Description} & \tb{Survival DR} \tableheaderrule
                Very soft ground & Any surface (fresh snow, thick dust, wet mud) that holds deep, clear impressions of footprints. & 0 \\
                Soft ground & Any surface soft enough to yield to pressure, but firmer than wet mud or fresh snow, in which a creature leaves frequent but shallow footprints & 5 \\
                Firm ground & Most normal outdoor surfaces (such as lawns, fields, woods, and the like) or exceptionally soft or dirty indoor surfaces (thick rugs and very dirty or dusty floors) & 10 \\
                Hard ground & Any surface that doesn't hold footprints at all, such as bare rock or a streambed & 15 \\
            \end{dtabularx}
        \end{dtable}

        \begin{dtable}
            \lcaption{Track Modifiers}
            \begin{dtabularx}{\columnwidth}{>{\lcol}X l}
                \tb{Condition} & \tb{DR Modifier} \tableheaderrule
                Every three creatures in the group being tracked    & \minus1      \\
                Size of creature or creatures being tracked:\fn{1}  &              \\
                Fine                                                & \plus16      \\
                Diminutive                                          & \plus12      \\
                Tiny                                                & \plus8       \\
                Small                                               & \plus4       \\
                Medium                                              & \plus0       \\
                Large                                               & \minus4      \\
                Huge                                                & \minus8      \\
                Gargantuan                                          & \minus12     \\
                Colossal                                            & \minus16     \\
                Every 24 hours since the trail was made             & \plus1\fn{3} \\
                Every hour of rain since the trail was made         & \plus1       \\
                Fresh snow cover since the trail was made           & \plus10      \\
                \tb{Poor visibility:\fn{2}}                         &              \\
                Overcast or moonless night                          & \plus6       \\
                Moonlight                                           & \plus3       \\
                Fog or precipitation                                & \plus3       \\
                Tracked party hides trail (and moves at half speed) & \plus5
            \end{dtabularx}
            1 For a group of mixed sizes, apply only the modifier for the largest size category. \\
            2 Apply only the largest modifier from this category. \\
            3 With scent-based tracking, apply this modifier per hour since the trail was made. \\
        \end{dtable}

        If you fail a Survival check to track, you can retry after 5 minutes of searching.

    \subsection{Use Rope}
        You can make a Survival check as a standard action to tie knots. Success means you tie the knot successfully. Failure means your action is wasted, but you can try again. Tying a typical knot is DR 10. Tying a special knot, such as one that slips, slides slowly, or loosens with a tug is DR 15. Knots can also be tied with the Devices skill.

        You can also make a Survival check in place of a ranged attack to set a grappling hook. You take a \minus2 penalty per 10 feet.

\newpage
\skill{Swim}{Str}
        The Swim skill represents your ability to swim.

    \subsection{Swimming}
        You can make a Swim check to move through water as a \glossterm{move action}. The DR depends on the turbulence of the water, as shown on \trefnp{Swim DRs}. Success means you move forward by up to one-quarter your speed. Critical success means you move twice as fast. Failure means you make no progress through the water.

        \begin{dtable}
            \lcaption{Swim DRs}
            \begin{dtabularx}{\columnwidth}{>{\lcol}X >{\lcol}X}
                \tb{Water} & \tb{Swim DR} \tableheaderrule
                Calm water   & 5 \\
                Rough water  & 10 \\
                Stormy water & 15 \\
            \end{dtabularx}
        \end{dtable}

    \subsection{Swimming Underwater}
        You can swim underwater just like you can above water. If you are underwater, either because you failed a Swim check or because you are swimming underwater intentionally, you must hold your breath. You can hold your breath for a number of rounds equal to 5 \add your Constitution. After that period of time, you must make a DR 10 Constitution check every round to continue holding your breath. Each round, the DR for the check increases by 5. If you fail, you begin to drown.

    \subsection{Swim Speed}\label{Swim Speed}
        A creature with a swim speed can move a distance equal to its swim speed with a successful Swim check.
        In addition, it gains a \plus10 bonus to any Swim checks it makes.
