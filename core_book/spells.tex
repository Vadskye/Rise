\chapter{Spells and Rituals}\label{Spells}

\section{Spell and Ritual Mechanics}\label{Spell and Ritual Mechanics}

    Spells and rituals share many common properties, defined here.

    \subsection{Categories of Magic}
        There are three \glossterm{magic sources} that spells can come from: arcane (cast by mages and warlocks), divine (cast by clerics and paladins), and nature (cast by druids).
        In addition, there are nine schools of magic.
        Each school of magic has a set of thematically related effects.
        Every spell and ritual belongs to at least one school of magic.
        The schools of magic are described below.

        \subsubsection{Abjuration}
            Abjuration spells and rituals reduce or negate damage, magic, and other effects.
            They can be used to protect allies and remove harmful magic.

        \subsubsection{Channeling}
            Channeling spells and rituals call upon the power of deities or other supernatural entities.
            They can be used to do anything those entities could do.
            Arcane spellcasters do not have access to Channeling spells.

        \subsubsection{Conjuration}
            Conjuration spells and rituals create and transport objects and creatures.
            They can be used to summon allies, transport creatures, and create objects from thin air.

        \subsubsection{Divination}
            Divination spells and rituals grant knowledge.
            They can be used to reveal hidden truths, predict the future, or communicate at great distances.

        \subsubsection{Enchantment}
            Enchantment spells and rituals alter the minds of creatures.
            They can be used to influence, control, or debilitate creatures.
            Almost all enchantment spells and rituals have the \glossterm{Mind} tag, and many have the \glossterm{Subtle} as well (see \pcref{Ability Tags}).

        \subsubsection{Evocation}
            Evocation spells and rituals create and manipulate energy.
            They can be used to inflict damage with energy blasts or manipulate the environment.

        \subsubsection{Illusion}
            Illusion spells and rituals create or manipulate sensory impressions.
            They can be used to create or remove light, conceal things that exist, or cause creatures to perceive things that do not exist.

        \subsubsection{Transmutation}
            Transmutation spells and rituals change the properties of creatures and objects.
            They can be used to grant new abilities, enhance existing abilities, change a target's form, or even alter the flow of time itself.

        \subsubsection{Vivimancy}
            Vivimancy spells and rituals manipulate the power of life and death, as well as souls.
            They can be used to heal or inflict wounds, resurrect the dead, create undead monsters, and cripple the bodies of creatures.

    \subsection{Casting Components}\label{Casting Components}
        Unless otherwise noted, all spells and rituals require both verbal and somatic components to cast or perform.
        You cannot start casting a spell or performing a ritual without all required components.
        If you lose those components before the ability resolves, it is \glossterm{miscast}.

        To provide the verbal component for a spell or ritual, you must speak in a strong voice with a volume at least as loud as ordinary conversation.
        To provide the somatic component for a spell or ritual, you must make a measured and precise movement of at least one free hand.

    \subsection{Augments}\label{Augments}
        There are a number of \glossterm{augments} that can be applied to spells and rituals to increase their power.
        Each augment has a name, a level, and an effect.
        Whenever you cast a spell or perform a ritual, can choose to apply any number of augments you know to the spell or ritual.
        For each augment you apply, you increase the spell or ritual's level by an amount equal to the augment's level.
        In exchange, the ability gains the effects of that augment.
        If an augment would increase the spell or ritual's level beyond the maximum level you can cast, you cannot apply the augment to that ability.

        \parhead{Augments and Cantrips}
        You can apply augments to \glossterm{cantrips}.
        Cantrips are considered to be 0th level spells for this purpose.

        \parhead{Augments and Subspells}
        If a spell or ritual changes its properties with a subspell or subritual, it may become eligible for different augments.
        % TODO: fix example
        For example, if you apply the Fireball subspell to the \spell{pyromancy} spell, it changes to affect an area.
        You would then be able to apply the Widened augment to increase its area.

        \subsubsection{Augment Descriptions}\label{Augment Descriptions}

            \augment{1}{Silent} You do not need to use \glossterm{verbal components} to cast the spell.
            \par This augment can be applied to any spell.

            \augment{1}{Stilled} You do not need to use \glossterm{somatic components} to cast the spell.
            \par This augment can be applied to any spell.

            \augment{2}{Cryptic} The spell's visual effects and magical aura changes to mimic a different spell of your choice.
            You may choose any combination of spell or \glossterm{subspell} you know, along with any other augments, that result in a spell of the same level or lower as the spell you are casting.
            This affects inspection of the spell itself by any means, such as with the Spellcraft skill (see \pcref{Spellcraft}).
            However, it does not alter the mechanical effects of the spell in any way.
            If the spell's effects depend on visual components, the spell may fail to work if you alter the spell's visuals too much. 

            \augment{2}{Extended} The ability's range increases by one step, to a maximum of \rngext.
            The steps are, in order: \rngtouch, \rngclose, \rngmed, \rnglong, and \rngext.
            This augment can be applied multiple times.
            Each time, the ability's range increases by an additional step.
            \par This augment can be applied to any spell or ritual with a range that is one of the above ranges.

            \augment{2}{Giant} The ability can affect a target one size category larger.
            This augment can be applied multiple times.
            Its effects stack.
            \par This augment can be applied to any spell or ritual that that has a maximum size category of targets that it can affect.

            \augment{2}{Quickened} You can cast the spell as a \glossterm{minor action}.
            In exchange, you cannot take any actions during the \glossterm{action phase} or \glossterm{delayed action phase} of the next round.
            \par This augment can be applied to any spell.

            \augment{2}{Selective} You may freely exclude any areas from the spell's effect.
            However, all squares in the final area of the spell must be contiguous.
            You cannot create split a spell's area into multiple completely separate areas.
            \par This augment can be applied to any spell or ritual that affects an area.

            \augment{2}{Widened} The ability's area increases by one step, to a maximum of \areahuge.
            The steps are, in order: \areasmall, \areamed, \arealarge, and \areahuge.
            Normally, a Small or Medium line is 5 ft.\ wide, while a Large or Huge line is 10 ft.\ wide.
            A line used to define a wall does not have a width.
            This augment can be applied multiple times.
            Each time, the ability's area increases by an additional step.
            \par This augment can be applied to any spell or ritual with an area that is one of the above areas.

            \augment{3}{Intensified} The damage inflicted and healing granted by the ability increases by \plus1d.
            This augment can be applied multiple times.
            Its effects stack.
            \par This augment can be applied to any spell or ritual that deals damage or grants healing based on a dice pool.

            \augment{3}{Phasing} When determining whether you have \glossterm{line of sight} and \glossterm{line of effect} to a particular location with the spell, you can ignore a single solid obstacle up to five feet thick.
            This can allow you to cast spells through solid walls, though it does not grant you the ability to see through the wall.
            \par This augment can be applied to any spell with a range.

            \augment{3}{Precise} You gain a \plus1 bonus to accuracy with the ability.
            This augment can be applied multiple times.
            Its effects stack.
            \par This augment can be applied to any spell or ritual that has an attack roll.

            \augment{4}{Accelerated} The ritual takes half the normal amount of time to perform.
            \par This augment can be applied to any ritual.

            % \augment{5}{Dual} The spell targets an additional creature within range.
            % This augment can be applied multiple times.
            % Its effects stack.
            % \par This augment can be applied to any spell that has a range and affects a single target of the caster's choice.
            % It cannot be applied to spells that affect a single specific target, such as the caster.
            % In addition, it cannot be applied to \glossterm{Attune} spells.

            \augment{6}{Echoing} During the \glossterm{delayed action phase} of the next round, the spell's effect occurs again.
            All choices you made for the original casting of the spell are made identically for the repeat casting.
            It affects the same area, targets, and so on.

            If the spell is now invalid, such as if all of its targets are out of range, the additional casting has no effect.
            This augment does not allow you to \glossterm{attune} to the same spell more than once.

    \subsection{Dismissal}
        As a \glossterm{minor action}, you can dismiss any spells or rituals you used that have lasting effects.
        This requires the same casting components (verbal and somatic) as casting the spell or performing the ritual normally.
        Spells and rituals can also be dismissed in other ways, such as after their effects have finished.
        When a spell or ability is dismissed, all of its lingering effects immediately end.
        % Should a spell be dismissed for all targets simultaneously, or per target?

    \subsection{Spellpower}

        The \glossterm{power} of your spells and rituals is determined by your \glossterm{spellpower}.
        Normally, your spellpower is equal to your level or an \glossterm{attribute}, whichever is higher.

        \parhead{Multiple Spellcasting Abilities} If you have the ability to cast spells from more than one separate ability, use the spellpower appropriate to the ability that you are casting the spell with.

        \parhead{Reducing Spellpower} You can voluntarily reduce the \glossterm{power} of the spells you cast by using a lower spellpower.
        However, you cannot use a spellpower lower than the minimum spellpower required to cast the spell, which is equal to twice the spell's level.

    \subsection{Resurrecting the Dead}\label{Resurrecting the Dead}
        Several rituals have the power to restore dead characters to life.

        When a living creature dies, its soul departs its body, travels through the Astral Plane, and goes to abide on the plane where the creature's deity resides.
        If the creature did not worship a deity, its soul departs to the plane corresponding to its alignment.
        Bringing a creature back from the dead means retrieving their soul and returning it to their body.

        \subparhead{Death and Old Age} While a creature is dead, it still tracks that time towards its maximum age.
        A creature's maximum age is largely determined by the strength of its soul, not the condition of its body.
        No magic can return a creature to life when it has passed its maximum age.

        \subparhead{Preventing Revivification} Enemies can take steps to make it more difficult for a character to be returned from the dead.
        Except for \spell{true resurrection}, every ritual to raise the dead requires a body, so keeping or destroying the body is an effective deterrent.
        The \spell{soul bind} ritual prevents any sort of revivification unless the soul is first released.

        \subparhead{Revivification against One's Will} A soul cannot be returned to life if it does not wish to be.
        A soul infallibly knows the name, alignment, and patron deity (if any) of the character attempting to revive it and may refuse to return on that basis.

\section{Spell Mechanics}\label{Spell Mechanics}

    \subsection{Cantrips}\label{Cantrips}
        Every spell can be cast as a \glossterm{cantrip}.
        You do not have to spend an action point to cast a spell as a cantrip.
        Cantrips are considered to be 0th level spells for the purpose of abilities that care about spell level, such as \glossterm{augments}.

    \subsection{Miscasting}\label{Miscasting}

        If you start casting a spell and fail to cast the spell successfully, such as if your concentration is broken (see \pcref{Concentration}), you miscast the spell.
        A miscast spell resolves at the same time that the spell normally would, but it does not have its normal effect.
        Instead, a wave of magical energy causes a \glossterm{miscast backlash}.
        When a mystic backlash occurs, make an attack against the Mental defense of yourself and all creatures in a 5 foot radius from you.
        Your \glossterm{power} with this ability is equal to your \glossterm{spellpower} with the spell you tried to cast.
        % damage type?
        On a hit, each target takes \glossterm{standard damage} \minus1d.

        \parhead{Impossible Spells} Whenever you try to cast a spell in an impossible way, the spell is \glossterm{miscast} instead.
        This most commonly happens if you attempt to declare an invalid target for the spell.
        For example, if you try to cast a spell that only affects living creatures on a creature that is undead, the spell would be miscast.

        \parhead{Voluntary Miscasting} At the start of each phase while you are casting a spell, you can choose to stop casting the spell, causing you to \glossterm{miscast} it instead.

    \subsection{Delayed Resolution}\label{Delayed Resolution}
        Every spell has the \glossterm{Delayed} tag, causing it to resolve later than other actions (see \pcref{Delayed Actions}).

    \subsection{Concentration}\label{Concentration}
        Every spell requires concentration to cast.
        If you are distracted while casting a spell, such as if you are riding a galloping horse, you need to make a \glossterm{concentration} check to avoid losing concentration on the spell.
        In addition, if the spell has the \glossterm{Delayed} tag, as all spells normally do, you must maintain this concentration for the entire \glossterm{action phase} before the spell resolves.
        If you are damaged during that phase, you also need to make a \glossterm{concentration} check to avoid losing the spell.
        If your concentration is broken while casting a spell, you \glossterm{miscast} that spell (see \pcref{Miscasting}).

        Your bonus on concentration \glossterm{checks} is normally equal to your level or your Willpower, whichever is higher.
        You apply any \glossterm{overwhelm penalties} you suffer to your concentration checks.
        If you take damage while also being distracted by some other effect, you take a \minus2 penalty to the concentration check.
        The \glossterm{difficulty rating} of the check is equal to 5 \add twice the level of the spell you are casting.

        The amount of damage you take during the action phase does not directly affect your concentration check.
        % TODO: timing of being staggered
        However, if you become \glossterm{staggered}, you take a \minus4 penalty to checks, which affects your concentration check.

    \subsection{Subspells}\label{Subspells}
        % TODO: some of this wording is weird
        All spells have a number of \glossterm{subspells}.
        Each subspell has a name, a level, and an effect.
        Whenever you cast a spell, you can choose to apply a single subspell you know from that spell.
        If you do, the spell's level becomes equal to the subspell's level.
        In exchange, the spell gains the effects of the subspell.
        You cannot learn or cast subspells whose spell level exceeds your maximum spell level.

        Some subspells simply add additional properties to a spell's normal effect.
        Others change the targets or effects of the spell significantly.
        After choosing whether to cast a subspell, you can apply any number of \glossterm{augments}, described below.

\section{Ritual Mechanics}
    Rituals are ceremonies that create magical effects.
    Some spellcasting characters can learn and perform rituals.
    You don't memorize a ritual as you would a normal spell or subspell.
    Rituals are too complex for all but the most knowledgeable sages to commit to memory.
    To perform a ritual, you need to read from a book or a scroll containing it.
    Rituals are similar to spells, but abilities that affect spells do not affect rituals unless they say they do in their descriptions.

    \subsection{Ritual Descriptions}
        % TODO: proper chapter references
        Rituals are described in Chapter 13.
        The description of each ritual follows the same format as the description of spells in Chapter 12, except that every ritual has a level, like \glossterm{subspells} do.

        \subparhead{Ritual Sources}
        A ritual always matches the magic source of the person performing the ritual.
        For example, \spell{scrying} is an arcane ritual when performed by a wizard, but a divine ritual when performed by a cleric.

    \subsection{Ritual Requirements}
        In order to learn and perform a ritual, you must be able to cast at least one spell of the same level as the ritual.

    \subsection{Ritual Books}
        A ritual book contains one or more rituals that you can use as frequently as you want, as long as you can spend the time and \glossterm{action point} to perform the ritual.
        Scribing a ritual in a ritual book costs an amount of precious inks.

    \subsection{Ritual Costs}\label{Ritual Costs}
        The costs to scribe rituals are described on \trefnp{Ritual Costs}.
        \begin{dtable}
            \lcaption{Ritual Costs}
            \begin{dtabularx}{\columnwidth}{X l l}
                \tb{Ritual Level} & \tb{Cost to Scribe} & \tb{Item Level} \\
                \bottomrule
                1st-Level & 50 gp & 1st \\
                2nd-Level & 200 gp & 3rd \\
                3rd-Level & 500 gp & 4th \\
                4th-Level & 1,250 gp & 7th \\
                5th-Level & 3,000 gp & 9th \\
                6th-Level & 7,500 gp & 11th \\
                7th-Level & 15,000 gp & 12th \\
                8th-Level & 35,000 gp & 14th \\
                9th-Level & 75,000 gp & 16th \\
            \end{dtabularx}
        \end{dtable}

    \subsection{Subrituals}\label{Subrituals}
        Many rituals have \glossterm{subrituals}, just like many spells have \glossterm{subspells} (see \pcref{Subspells}).
        Subrituals work in the same way as subspells, except that they are applied to rituals instead of spells.

    \subsection{Performing Rituals}
        To perform a ritual, you must have a ritual book containing the ritual and the material components required for the ritual.
        Unless otherwise specified, performing a ritual requires spending a single \glossterm{action point}.
        Some rituals require multiple action points to complete.
        Other creatures can supply action points to help you perform rituals; see Ritual Participants, below.

        If you are distracted during the ritual, you must make a Concentration check, just as if you were casting a spell of the ritual's level.
        If you fail, the ritual is ruined and you must start from the beginning.
        % TODO: can we remove the ``casting a ritual'' wording?
        \par Performing a ritual and casting a ritual mean the same thing.

        \subsubsection{Ritual Participants}
            Creatures can assist in the performance of rituals even if they are unable to perform rituals themselves.
            A creature that helps perform a ritual is called a ritual participant, and the creature performing the ritual is called the ritual leader.
            A ritual participant may spend an action point in place of or in addition to the action point spent by the creature performing the ritual.
            It may also \glossterm{attune} to the effect of the ritual in place of the creature performing the ritual.
            Only one creature may attune to the ritual's effect in this way.
            If multiple creatures are willing to spend action points or attune to effects, the ritual leader decides which creatures spend action points or attune to the ritual's effects.

            The steps required to participate in rituals can be complex.
            Ritual participants must be given specific instructions for the actions they must perform during a ritual by a creature who knows how to perform the ritual.
            This instruction generally takes half the time required to perform the ritual.
            A creature cannot participate in rituals unless it has an Intelligence of at least 0, can speak at least one language, and has the fine motor control required to perform the somatic components of spells.

            Normally, a ritual participant can only contribute one action point.
            If the participant can cast spells from the same source as the ritual, they can contribute any number of action points.

            \parhead{Changing Ritual Participation}
            Rituals are deeply complex magic, and they cannot be abandoned or paused partway through.
            If the number of ritual participants in a ritual decreases below its initial value, the ritual fails at the end of the next round if the number of participants is not restored.
            However, ritual participants can transfer their participation to other creatures without disrupting the ritual.

            In order to transfer ritual participation, the new creature must be able to participate in the ritual, and must immediately spend the same number of action points as the creature that it is taking over from.
            Similarly, the ritual leader can transfer their leadership to another creature.
            In addition to the requirements for transferring ritual participation, the new leader must know the ritual and be able to perform it themselves.

            Changing ritual participation and leadership is usually done when performing extraordinarily long or demanding rituals.

    \subsection{Magical Writings}
        To record a spell in written form, a character uses complex notation that describes the magical forces involved in the spell.
        The notation constitutes a universal language that spellcasters have discovered, not invented.
        Each writer uses this universal system regardless of their native language or culture.
        However, each character uses the system in their own way.
        Another person's magical writing remains incomprehensible to even the most powerful spellcaster until they take the time to study and decipher it.

        To decipher an magical writing (such as a single spell in written form on a scroll), you must make a Spellcraft check (DR 10 \add the spell's level). If the skill check fails, you cannot attempt to read that particular spell again until the next day.
        A \ritual{read magic} ritual automatically deciphers a magical writing without a skill check.
        If the person who created the magical writing is on hand to help the reader, success is also automatic.

        Once a character deciphers a particular magical writing, they do not need to decipher it again.
        Deciphering a magical writing allows the reader to identify the spell and gives some idea of its effects (as explained in the spell or ritual description).
        

\section{Spell Lists}


\small
\subsection{Arcane Magic}\label{Arcane Magic}
\subsubsection{Arcane Spells}\label{Arcane Spells}
\begin{spelllist}
\spellhead{Astromancy} Transport creatures and objects instantly through space.
\spellhead{Barrier} Shield allies from hostile forces.
\spellhead{Chronomancy} Manipulate the passage of time to inhibit foes and aid allies.
\spellhead{Compel} Bend creatures to your will by controlling their actions.
\spellhead{Corruption} Weaken the life force of foes, reducing their combat prowess.
\spellhead{Cryomancy} Drain heat to injure and freeze foes.
\spellhead{Delusion} Instill false emotions to influence creatures.
\spellhead{Electromancy} Create electricity to injure and stun foes.
\spellhead{Fabrication} Create objects to damage and impair foes.
\spellhead{Glamer} Change how creatures and objects are perceived.
\spellhead{Photomancy} Create bright light to blind foes and illuminate your surroundings.
\spellhead{Polymorph} Change the physical forms of objects and creatures.
\spellhead{Pyromancy} Create fire to incinerate foes.
\spellhead{Revelation} Share visions of the present and future, granting insight or combat prowess.
\spellhead{Scry} See and hear at great distances.
\spellhead{Summon} Summon creatures to fight with you.
\spellhead{Telekinesis} Manipulate creatures and objects at a distance.
\spellhead{Thaumaturgy} Suppress and manipulate magical effects.
\spellhead{Weaponcraft} Create and manipulate weapons to attack foes.
\end{spelllist}

\subsubsection{Arcane Rituals}\label{Arcane Rituals}
\begin{spelllist}
\spellhead[1]{Create Water} TODO
\spellhead[1]{Endure Elements} TODO
\spellhead[1]{Fortify} TODO
\spellhead[1]{Light} TODO
\spellhead[1]{Magic Mouth} TODO
\spellhead[1]{Purify Sustenance} TODO
\spellhead[1]{Read Magic} TODO
\spellhead[2]{Animate Dead} TODO
\spellhead[2]{Create Object} TODO
\spellhead[2]{Create Sustenance} TODO
\spellhead[2]{Gentle Repose} TODO
\spellhead[2]{Mount} TODO
\spellhead[2]{Mystic Lock} TODO
\spellhead[2]{Purge Curse} TODO
\spellhead[2]{Scryward} TODO
\spellhead[2]{Seek Legacy} TODO
\spellhead[2]{Water Breathing} TODO
\spellhead[3]{Explosive Runes} TODO
\spellhead[3]{Plane Shift} TODO
\spellhead[3]{Retrieve Legacy} TODO
\spellhead[3]{Sending} TODO
\spellhead[3]{Telepathic Bond} TODO
\spellhead[4]{Discern Location} TODO
\spellhead[4]{Overland Teleportation} TODO
\spellhead[4]{Private Sanctum} TODO
\spellhead[4]{Scry Creature} TODO
\spellhead[5]{Soul Bind} TODO
\spellhead[7]{Gate} TODO
\end{spelllist}



\small
\subsection{Divine Magic}\label{Divine Magic}
\subsubsection{Divine Spells}\label{Divine Spells}
\begin{spelllist}
\spellhead{Bless} Grant divine blessings to aid allies and improve combat prowess.
\spellhead{Compel} Bend creatures to your will by controlling their actions.
\spellhead{Corruption} Weaken the life force of foes, reducing their combat prowess.
\spellhead{Delusion} Instill false emotions to influence creatures.
\spellhead{Divine Judgment} Smite foes with divine power.
\spellhead{Photomancy} Create bright light to blind foes and illuminate your surroundings.
\spellhead{Revelation} Share visions of the present and future, granting insight or combat prowess.
\spellhead{Scry} See and hear at great distances.
\spellhead{Summon} Summon creatures to fight with you.
\spellhead{Thaumaturgy} Suppress and manipulate magical effects.
\spellhead{Vital Surge} Alter life energy to cure or inflict wounds.
\spellhead{Weaponcraft} Create and manipulate weapons to attack foes.
\end{spelllist}

\subsubsection{Divine Rituals}\label{Divine Rituals}
\begin{spelllist}
\spellhead[1]{Bless Water} TODO
\spellhead[1]{Create Water} TODO
\spellhead[1]{Curse Water} TODO
\spellhead[1]{Endure Elements} TODO
\spellhead[1]{Fortify} TODO
\spellhead[1]{Light} TODO
\spellhead[1]{Purify Sustenance} TODO
\spellhead[1]{Read Magic} TODO
\spellhead[2]{Animate Dead} TODO
\spellhead[2]{Create Object} TODO
\spellhead[2]{Create Sustenance} TODO
\spellhead[2]{Gentle Repose} TODO
\spellhead[2]{Mystic Lock} TODO
\spellhead[2]{Purge Curse} TODO
\spellhead[2]{Scryward} TODO
\spellhead[2]{Seek Legacy} TODO
\spellhead[2]{Water Breathing} TODO
\spellhead[3]{Plane Shift} TODO
\spellhead[3]{Restoration} TODO
\spellhead[3]{Retrieve Legacy} TODO
\spellhead[3]{Sending} TODO
\spellhead[4]{Blessed Transit} TODO
\spellhead[4]{Discern Location} TODO
\spellhead[4]{Scry Creature} TODO
\spellhead[5]{Soul Bind} TODO
\spellhead[7]{Gate} TODO
\end{spelllist}



\small
\subsection{Nature Magic}\label{Nature Magic}
\subsubsection{Nature Spells}\label{Nature Spells}
\begin{spelllist}
\spellhead{Aeromancy} Command air to protect allies and blast foes.
\spellhead{Aquamancy} Command water to crush and drown foes.
\spellhead{Corruption} Weaken the life force of foes, reducing their combat prowess.
\spellhead{Cryomancy} Drain heat to injure and freeze foes.
\spellhead{Electromancy} Create electricity to injure and stun foes.
\spellhead{Photomancy} Create bright light to blind foes and illuminate your surroundings.
\spellhead{Polymorph} Change the physical forms of objects and creatures.
\spellhead{Pyromancy} Create fire to incinerate foes.
\spellhead{Revelation} Share visions of the present and future, granting insight or combat prowess.
\spellhead{Scry} See and hear at great distances.
\spellhead{Summon} Summon creatures to fight with you.
\spellhead{Thaumaturgy} Suppress and manipulate magical effects.
\spellhead{Vital Surge} Alter life energy to cure or inflict wounds.
\end{spelllist}

\subsubsection{Nature Rituals}\label{Nature Rituals}
\begin{spelllist}
\spellhead[1]{Create Water} TODO
\spellhead[1]{Endure Elements} TODO
\spellhead[1]{Fortify} TODO
\spellhead[1]{Light} TODO
\spellhead[1]{Purify Sustenance} TODO
\spellhead[1]{Read Magic} TODO
\spellhead[2]{Create Object} TODO
\spellhead[2]{Create Sustenance} TODO
\spellhead[2]{Fertility} TODO
\spellhead[2]{Gentle Repose} TODO
\spellhead[2]{Infertility} TODO
\spellhead[2]{Mystic Lock} TODO
\spellhead[2]{Purge Curse} TODO
\spellhead[2]{Scryward} TODO
\spellhead[2]{Seek Legacy} TODO
\spellhead[2]{Water Breathing} TODO
\spellhead[3]{Ironwood} TODO
\spellhead[3]{Plane Shift} TODO
\spellhead[3]{Restoration} TODO
\spellhead[3]{Resurrection} TODO
\spellhead[3]{Retrieve Legacy} TODO
\spellhead[3]{Sending} TODO
\spellhead[4]{Discern Location} TODO
\spellhead[4]{Lifeweb Transit} TODO
\spellhead[4]{Reincarnation} TODO
\spellhead[4]{Scry Creature} TODO
\spellhead[5]{Awaken} TODO
\spellhead[7]{Gate} TODO
\end{spelllist}



\small
\subsection{Pact Magic}\label{Pact Magic}
\subsubsection{Pact Spells}\label{Pact Spells}
\begin{spelllist}
\spellhead{Astromancy} Transport creatures and objects instantly through space.
\spellhead{Chronomancy} Manipulate the passage of time to inhibit foes and aid allies.
\spellhead{Compel} Bend creatures to your will by controlling their actions.
\spellhead{Corruption} Weaken the life force of foes, reducing their combat prowess.
\spellhead{Cryomancy} Drain heat to injure and freeze foes.
\spellhead{Delusion} Instill false emotions to influence creatures.
\spellhead{Electromancy} Create electricity to injure and stun foes.
\spellhead{Fabrication} Create objects to damage and impair foes.
\spellhead{Photomancy} Create bright light to blind foes and illuminate your surroundings.
\spellhead{Polymorph} Change the physical forms of objects and creatures.
\spellhead{Pyromancy} Create fire to incinerate foes.
\spellhead{Telekinesis} Manipulate creatures and objects at a distance.
\spellhead{Weaponcraft} Create and manipulate weapons to attack foes.
\end{spelllist}



\section{Spell Descriptions}

\section{Spell Descriptions}
\begin{spellsection}{Agony}
\begin{spellheader}
\spelldesc{You inflict debilitating pain on your foe}
\end{spellheader}
\begin{spellcontent}
\begin{spelltargetinginfo}
\spelltwocol{\spelltgt{One creature}}{\spellrng{\rngclose}}
\end{spelltargetinginfo}
\begin{spelleffects}
\begin{spellattack}{Spellpower vs. Mental}
\spellsuccess Physical damage dealt to the target is increased by \plus2d.
\spellcritical Physical damage dealt to the target is increased by \plus4d.
\end{spellattack}
\spelldur Condition
\spelltags{\glossterm{Delusion}, \glossterm{Mind}}
\end{spelleffects}
\end{spellcontent}
\begin{spellfooter}
\spellinfo{Enchantment}{Arcane, Divine}
\parhead*{Augments} Extended, Mass, Quickened, Silent, Stilled
\spellnotes This damage increase applies before other effects that modify the total damage dealt, such as \glossterm{damage reduction}.
\end{spellfooter}
\begin{spellsubcontent}
\begin{spellcantrip}
The spell's duration becomes Sustain (swift).
Its effect is still a \glossterm{condition}, and can be removed by abilites that remove conditions.
\end{spellcantrip}
\end{spellsubcontent}
\end{spellsection}
\subsubsection{Subspells}
\augment{3}{Complete}
The damage increase applies to all damage, not just physical damage.
\begin{spellsection}{Antimagic}
\begin{spellcontent}
\begin{spelltargetinginfo}
\spelltwocol{\spelltgt{One creature, object, or active magical effect}}{\spellrng{\rngmed}}
\end{spelltargetinginfo}
\begin{spelleffects}
\begin{spellattack}{Spellpower vs. Special}
\spellspecial
The attack result is applied to every \glossterm{magical} effect on the target.
The DR for each effect is equal to 10 + the \glossterm{power} of that effect.
\spellsuccess
Success against a magical effect causes that effect to be \glossterm{suppressed}.
\end{spellattack}
\spelldur Sustain (swift)
\spelltags{\glossterm{Thaumaturgy}}
\end{spelleffects}
\end{spellcontent}
\begin{spellfooter}
\spellinfo{Abjuration}{Arcane, Divine, Magic, Nature}
\parhead*{Augments} Extended, Mass, Quickened, Silent, Stilled
\end{spellfooter}
\begin{spellsubcontent}
\begin{spellcantrip}
The spell's duration becomes Sustain (standard).
\end{spellcantrip}
\end{spellsubcontent}
\end{spellsection}
\subsubsection{Subspells}
\augment{2}{Alter Magic Aura}
Replace the spell's targets and effects with the following:
\begin{spellcontent}
\begin{augmenttargetinginfo}
\spelltwocol{\spelltgt{One magical object (Large or smaller)}}{\spellrng{\rngmed}}
\end{augmenttargetinginfo}
\begin{augmenteffects}
\begin{spellattack}{Spellpower vs. Mental}
\spellsuccess
One of the target's magic auras is altered (see \pcref{Spellcraft}).
You can change the school and descriptors of the aura.
In addition, you can decrease the spellpower of the aura by up to half your spellpower, or increase the spellpower of the aura up to a maximum of your spellpower.
\end{spellattack}
\spelldur Attunement
\spelltags{\glossterm{Thaumaturgy}}
\end{augmenteffects}
\end{spellcontent}
\augment{2}{Dimensional Anchor}
Replace the spell's effects with the following:
\begin{spellcontent}
\begin{augmenteffects}
\begin{spellattack}{Spellpower vs. Mental}
\spellsuccess
The target cannot travel extradimensionally.
This prevents all \glossterm{Manifestation}, \glossterm{Planar}, and \glossterm{Translocation} effects.
\end{spellattack}
\spelldur Condition
\spelltags{\glossterm{T}, \glossterm{a}, \glossterm{a}, \glossterm{g}, \glossterm{h}, \glossterm{m}, \glossterm{r}, \glossterm{t}, \glossterm{u}, \glossterm{u}, \glossterm{y}}
\end{augmenteffects}
\end{spellcontent}
\augment{2}{Suppress Item}
Replace the spell's targets and effects with the following:
\begin{spellcontent}
\begin{augmenttargetinginfo}
\spelltwocol{\spelltgt{One object}}{\spellrng{\rngmed}}
\end{augmenttargetinginfo}
\begin{augmenteffects}
\begin{spellattack}{Spellpower vs. Special}
\spellspecial
The DR is equal to 10 + the target's spellpower.
\spellsuccess
The target object is \glossterm{suppressed}.
\end{spellattack}
\spelldur Sustain (swift)
\spelltags{\glossterm{Thaumaturgy}}
\end{augmenteffects}
\end{spellcontent}
\augment{3}{Banishing}
Replace the spell's effects with the following:
\begin{spellcontent}
\begin{augmenteffects}
\begin{spellattack}{Spellpower vs. Special}
\spellspecial
If the target is an effect of an ongoing \glossterm{magical} ability, such as a summoned monster, the DR is equal to 10 + the target's spellpower.
Otherwise, this ability has no effect.
\spellsuccess
The target is treated as if the spell that created it was \glossterm{dispelled}.
This usually causes the target to disappear.
\end{spellattack}
\spelltags{\glossterm{Thaumaturgy}}
\end{augmenteffects}
\end{spellcontent}
\augment{5}{Dimensional Lock}
Replace the spell's targets and effects with the following:
\begin{spellcontent}
\begin{augmenttargetinginfo}
\spelltwocol{\spellburst{\arealarge radius}}{\spellrng{\rngmed}}
\spelltgts{Everything in the area}
\end{augmenttargetinginfo}
\begin{augmenteffects}
\spelleffect
Extradimensional travel into or out of the spell's area is impossible.
This prevents all \glossterm{Manifestation}, \glossterm{Planar}, and \glossterm{Translocation} effects.
\spelldur Attunement
\spelltags{\glossterm{T}, \glossterm{a}, \glossterm{a}, \glossterm{g}, \glossterm{h}, \glossterm{m}, \glossterm{r}, \glossterm{t}, \glossterm{u}, \glossterm{u}, \glossterm{y}}
\end{augmenteffects}
\end{spellcontent}
\augment{7}{Antimagic Field}
Replace the spell's targets and effects with the following:
\begin{spellcontent}
\begin{augmenttargetinginfo}
\spellspecial This emanation always includes you in its area
\spellemanation{\areasmall radius centered on you}
\end{augmenttargetinginfo}
\begin{augmenteffects}
\spelleffect
All magical abilities and objects are \glossterm{suppressed} in the area.
In addition, magical abilities and objects cannot be activated within the area.
\par Creatures within the area cannot concentrate on or dismiss spells. However, you can concentrate on and dismiss your own \spell{antimagic field}.
\spelldur Sustain (swift)
\spelltags{\glossterm{Thaumaturgy}}
\end{augmenteffects}
\end{spellcontent}
\begin{spellsection}{Barkskin}
\begin{spellheader}
\spelldesc{You toughen a creature's skin, giving it the appearance of tree bark.}
\end{spellheader}
\begin{spellcontent}
\begin{spelltargetinginfo}
\spelltwocol{\spelltgt{One living creature}}{\spellrng{\rngclose}}
\end{spelltargetinginfo}
\begin{spelleffects}
\spelleffect
The target gains \glossterm{damage reduction} against physical damage equal to your spellpower.
In addition, it is \glossterm{vulnerable} to fire damage.
\spelldur Attunement
\spelltags{\glossterm{Enhancement}}
\end{spelleffects}
\end{spellcontent}
\begin{spellfooter}
\spellinfo{Transmutation}{Nature}
\parhead*{Augments} Extended, Quickened, Silent, Stilled
\end{spellfooter}
\begin{spellsubcontent}
\begin{spellcantrip}
The spell's duration becomes Sustain (swift).
\end{spellcantrip}
\end{spellsubcontent}
\end{spellsection}
\subsubsection{Subspells}
\augment{3}{Stoneskin}
The spell does not make the target vulnerable to fire damage.
Instead, it makes the target \glossterm{vulnerable} to damage from adamantine weapons.
\augment{5}{Empowered}
The damage reduction granted by this spell increases by an amount equal to your spellpower.
\begin{spellsection}{Barrier}
\begin{spellcontent}
\begin{spelltargetinginfo}
\spellzone{\areamed radius centered on you}
\end{spelltargetinginfo}
\begin{spelleffects}
\spelleffect
Whenever a creature makes physical contact with the spell's area for the first time, you make a Spellpower vs. Mental attack against it.
Success means the creature is unable to enter the spell's area with any part of its body.
The rest of its movement in the current phase is cancelled.
Failure means the creature can enter the area unimpeded.
Creatures in the area at the time that the spell is cast are unaffected by the spell.
\spelldur Sustain (swift)
\end{spelleffects}
\end{spellcontent}
\begin{spellfooter}
\spellinfo{Abjuration}{Divine, Nature}
\parhead*{Augments} Quickened, Silent, Stilled, Widened
\end{spellfooter}
\begin{spellsubcontent}
\begin{spellcantrip}
The spell's duration becomes Sustain (standard)
\end{spellcantrip}
\end{spellsubcontent}
\end{spellsection}
\subsubsection{Subspells}
\augment{4}{Selective}
Whenever a creature attempts to pass through the barrier for the first time, you can allow it to pass through unimpeded.
You must be aware of a creature attempting to pass through the barrier to allow it through.
\augment{7}{Antilife Shell}
The spell only affects living creatures.
However, it affects them automatically, without requiring an attack.
\par
This is a \glossterm{Life} effect from the \glossterm{Vivimancy} school.
\begin{spellsection}{Bless}
\begin{spellheader}
\spelldesc{You invoke a divine blessing to aid your ally.}
\end{spellheader}
\begin{spellcontent}
\begin{spelltargetinginfo}
\spelltwocol{\spelltgt{One creature}}{\spellrng{\rngclose}}
\end{spelltargetinginfo}
\begin{spelleffects}
\spelleffect The target gains a \plus2d bonus to damage with all attacks.
\spelldur Attunement
\end{spelleffects}
\end{spellcontent}
\begin{spellfooter}
\spellinfo{Channeling}{Divine}
\parhead*{Augments} Extended, Quickened, Silent, Stilled
\end{spellfooter}
\begin{spellsubcontent}
\begin{spellcantrip}
The spell's duration becomes Sustain (swift).
\end{spellcantrip}
\end{spellsubcontent}
\end{spellsection}
\subsubsection{Subspells}
\augment{6}{Protection}
The target gains \glossterm{damage reduction} against all damage equal to your spellpower.
\begin{spellsection}{Boon of Mastery}
\begin{spellheader}
\spelldesc{You grant your ally great mastery over a particular domain.}
\end{spellheader}
\begin{spellcontent}
\begin{spelltargetinginfo}
\spelltwocol{\spelltgt{One willing creature}}{\spellrng{\rngclose}}
\end{spelltargetinginfo}
\begin{spelleffects}
\spellspecial
When you cast this spell, choose a skill.
You must have mastered the chosen skill.
\spelleffect
The target gains a \plus5 bonus to the chosen skill.
\spelldur Attunement
\spelltags{\glossterm{Enhancement}}
\end{spelleffects}
\end{spellcontent}
\begin{spellfooter}
\spellinfo{Transmutation}{Arcane, Divine, Nature}
\parhead*{Augments} Extended, Quickened, Silent, Stilled
\end{spellfooter}
\begin{spellsubcontent}
\begin{spellcantrip}
The spell's duration becomes Sustain (swift).
\end{spellcantrip}
\end{spellsubcontent}
\end{spellsection}
\subsubsection{Subspells}
\augment{4}{Myriad}
You may choose an additional skill that you have mastered as you cast the spell.
The target gains the same bonus to all chosen skills.
\begin{spellsection}{Charm Person}
\begin{spellheader}
\spelldesc{You manipulate a person's mind so they think of you as a trusted friend and ally.}
\end{spellheader}
\begin{spellcontent}
\begin{spelltargetinginfo}
\spelltwocol{\spelltgt{One humanoid creature}}{\spellrng{\rngclose}}
\end{spelltargetinginfo}
\begin{spelleffects}
\begin{spellattack}{Spellpower vs. Mental}
\spellspecial If the target thinks that you or your allies are threatening it, you take a \minus5 penalty to accuracy on the attack.
\spellsuccess
The target is \charmed by you.
Any act by you or your apparent allies that threatens or damages the \spell{charmed} person breaks the effect.
\spellcritical As above, but the effect's duration becomes permanent.
\end{spellattack}
\spelldur Sustain (swift)
\spelltags{\glossterm{Delusion}, \glossterm{Mind}, \glossterm{Subtle}}
\end{spelleffects}
\end{spellcontent}
\begin{spellfooter}
\spellinfo{Enchantment}{Arcane}
\parhead*{Augments} Extended, Mass, Quickened, Silent, Stilled
\end{spellfooter}
\begin{spellsubcontent}
\begin{spellcantrip}
The spell has no additional effects on a critical hit.
In addition, its duration becomes Sustain (standard).
\end{spellcantrip}
\end{spellsubcontent}
\end{spellsection}
\subsubsection{Subspells}
\augment{2}{Silent}
The spell does not require verbal components to cast.
\augment{3}{Monstrous}
The spell can target creatures of any creature type.
\augment{4}{Attuned}
The spell's duration becomes Attunement.
A critical sucess still makes the effect permanent.
\augment{5}{Amnesia}
When the spell ends, the target forgets all events that transpired during the spell's duration.
It becomes aware of its surroundings as if waking up from a daydream.
It is not directly aware of any magical influence on its mind, though unusually paranoid or perceptive creatures may deduce that their minds were affected.
\augment{5}{Dominating}
Replace the spell's effects with the following:
\begin{spellcontent}
\begin{augmenteffects}
\begin{spellattack}{Spellpower vs. Mental}
\spellsuccess The target is \confused for 2 rounds.
\spellcritical
The target is \dominated for 2 rounds.
If the target was already dominated by you, this effect lasts for 24 hours instead.
\end{spellattack}
\spelltags{\glossterm{Compulsion, Mind}}
\end{augmenteffects}
\end{spellcontent}
\begin{spellsection}{Cone of Cold}
\begin{spellheader}
\spelldesc{You drain the heat from an area, creating a field of extreme cold.}
\end{spellheader}
\begin{spellcontent}
\begin{spelltargetinginfo}
\spellburst{\areamed cone}
\spelltgts{Everything in the area}
\end{spelltargetinginfo}
\begin{spelleffects}
\begin{spellattack}{Spellpower vs. Fortitude}
\spellsuccess
\spelldamage{cold}[1d4].
In addition, the target is \fatigued as a condition.
\spellcritical As above, but double damage.
\end{spellattack}
\spelltags{\glossterm{Cold}}
\end{spelleffects}
\end{spellcontent}
\begin{spellfooter}
\spellinfo{Evocation}{Arcane, Nature}
\parhead*{Augments} Intensified, Quickened, Silent, Stilled, Widened
\end{spellfooter}
\begin{spellsubcontent}
\begin{spellcantrip}
The spell deals no damage.
\end{spellcantrip}
\end{spellsubcontent}
\end{spellsection}
\begin{spellsection}{Control Air}
\begin{spellheader}
\spelldesc{You shield your ally with a barrier of wind, protecting them from harm.}
\end{spellheader}
\begin{spellcontent}
\begin{spelltargetinginfo}
\spelltwocol{\spelltgt{One willing creature (Medium or smaller)}}{\spellrng{\rngclose}}
\end{spelltargetinginfo}
\begin{spelleffects}
\spelleffect
The target gains a \plus2 bonus to \glossterm{physical defenses}.
This bonus is increased to \plus5 against ranged \glossterm{strikes} from weapons or projectiles that are Small or smaller.
Any effect which increases the size of creature this spell can affect also increases the size of ranged weapon it defends against by the same amount.
\spelldur Attunement
\spelltags{\glossterm{Air}, \glossterm{Imbuement}}
\end{spelleffects}
\end{spellcontent}
\begin{spellfooter}
\spellinfo{Transmutation}{Air, Nature}
\parhead*{Augments} Extended, Quickened, Silent, Stilled
\end{spellfooter}
\begin{spellsubcontent}
\begin{spellcantrip}
The spell's duration becomes Sustain (swift).
\end{spellcantrip}
\end{spellsubcontent}
\end{spellsection}
\subsubsection{Subspells}
\augment{2}{Gentle Descent}
The target gains a 30 foot glide speed.
A creature with a glide speed can glide through the air at the indicated speed (see \pcref{Gliding}).
\augment{2}{Windstrike}
Replace the spell's targets and effects with the following:
\begin{spellcontent}
\begin{augmenttargetinginfo}
\spelltwocol{\spelltgt{One creature or object}}{\spellrng{\rngmed}}
\end{augmenttargetinginfo}
\begin{augmenteffects}
\begin{spellattack}{Spellpower vs. Fortitude}
\spellsuccess \spelldamage{bludgeoning}.
\spellcritical As above, but double damage.
\end{spellattack}
\spelltags{\glossterm{Air}}
\end{augmenteffects}
\end{spellcontent}
\augment{3}{Accelerated}
The glide speed granted by this spell increases to 60 feet.
\augment{3}{Gust of Wind}
Replace the spell's targets and effects with the following:
\begin{spellcontent}
\begin{augmenttargetinginfo}
\spellburst{\arealarge line, 10 ft. wide}
\spelltgts{Everything in the area}
\end{augmenttargetinginfo}
\begin{augmenteffects}
\begin{spellattack}{Spellpower vs. Fortitude}
\spellsuccess \spelldamage{bludgeoning}[1d4].
\spellcritical As above, but double damage.
\end{spellattack}
\spelltags{\glossterm{Air}}
\end{augmenteffects}
\end{spellcontent}
\augment{4}{Air Walk}
The target can walk on air as if it were solid ground.
The magic only affects the target's legs and feet.
By choosing when to treat the air as solid, it can traverse the air with ease.
\augment{4}{Wind Screen}
The miss chance for ranged strikes against the target increases to 50\%.
\augment{5}{Stormlord}
Whenever a creature within \rngclose range of the target attacks it, wind strikes the attacking creature.
The wind deals 1d4 bludgeoning damage \add 1d per two spellpower.
Any individual creature can only be dealt damage in this way once per round.
\par Any effect which increases this spell's range increases the range of this effect by the same amount.
\par
This is a \glossterm{Shielding} effect from the \glossterm{Evocation} school.
\augment{7}{Control Weather}
Replace the spell's targets and effects with the following:
\begin{spellcontent}
\begin{augmenttargetinginfo}
\spellzone{2 mile radius cylinder from your location}
\end{augmenttargetinginfo}
\begin{augmenteffects}
\spelleffect
When you cast this spell, you choose a new weather pattern.
You can only choose weather which would be possible in the climate and season of the area you are in.
For example, you can normally create a thunderstorm, but not if you are in a desert.
The weather begins to take effect in the area when you complete the spell.
After five minutes, your chosen weather pattern fully takes effect.
You can control the general tendencies of the weather, such as the direction and intensity of the wind.
You cannot control specific applications of the weather -- where lightning strikes, for example, or the exact path of a tornado.
Contradictory weather conditions are not possible simultaneously.
After the spell's duration ends, the weather continues on its natural course, which may cause your chosen weather pattern to end.
% TODO: This should be redundant with generic spell mechanics
If another ability would magically manipulate the weather in the same area, the most recently used ability takes precedence.
\spelldur Attunement
\spelltags{\glossterm{Air}}
\end{augmenteffects}
\end{spellcontent}
\begin{spellsection}{Corruption}
\begin{spellheader}
\spelldesc{You corrupt your foe's life force, weakening them.}
\end{spellheader}
\begin{spellcontent}
\begin{spelltargetinginfo}
\spelltwocol{\spelltgt{One living creature}}{\spellrng{\rngclose}}
\end{spelltargetinginfo}
\begin{spelleffects}
\begin{spellattack}{Spellpower vs. Fortitude}
\spellsuccess
The target takes a \minus2 penalty to \glossterm{accuracy}, \glossterm{checks}, and \glossterm{defenses}.
\spellcritical
As above, but the penalty is increased by 2.
\end{spellattack}
\spelldur Condition
\spelltags{\glossterm{Life}}
\end{spelleffects}
\end{spellcontent}
\begin{spellfooter}
\spellinfo{Vivimancy}{Arcane, Divine, Nature}
\parhead*{Augments} Extended, Mass, Quickened, Silent, Stilled
\end{spellfooter}
\begin{spellsubcontent}
\begin{spellcantrip}
The spell's duration becomes Sustain (swift).
Its effect is still a condition, and can be removed by abilites that remove conditions.
\end{spellcantrip}
\end{spellsubcontent}
\end{spellsection}
\subsubsection{Subspells}
\augment{3}{Eyebite}
If the spell's attack succeeds, the target is also \partiallyblinded. If it critically hits, the target is \blinded instead of partially blinded.
\augment{3}{Finger of Death}
If the spell's attack critically hits, the target immediately dies.
\par
This is a \glossterm{Death} effect.
\augment{4}{Empowered}
The penalty increases by 1.
\augment{5}{Corruption of Blood and Bone}
If the spell's attack succeeds, at the end of each round, the target takes life damage equal to your spellpower.
The target's maximum hit points are reduced by the amount of damage it takes in this way.
When the spell ends, the target's maximum hit points are restored.
\augment{6}{Corrupting Curse}
The spell's attack is made against Mental defense instead of Fortitude defense.
In addition, if it critically hits, the spell's effect becomes a permanent curse.
It is no longer a condition, and cannot be removed by abilities that remove conditions.
This is a \glossterm{Curse} effect.
\begin{spellsection}{Create Acid}
\begin{spellheader}
\spelldesc{You create a magical orb of acid in your hand that speeds to its target.}
\end{spellheader}
\begin{spellcontent}
\begin{spelltargetinginfo}
\spelltwocol{\spelltgt{One creature or object}}{\spellrng{\rngmed}}
\end{spelltargetinginfo}
\begin{spelleffects}
\begin{spellattack}{Spellpower vs. Reflex}
\spellsuccess \spelldamage{acid}.
\spellcritical As above, but double damage.
\end{spellattack}
\spelltags{\glossterm{Acid}, \glossterm{Manifestation}}
\end{spelleffects}
\end{spellcontent}
\begin{spellfooter}
\spellinfo{Conjuration}{Arcane}
\parhead*{Augments} Extended, Intensified, Mass, Quickened, Silent, Stilled
\end{spellfooter}
\begin{spellsubcontent}
\begin{spellcantrip}
The spell's range becomes \rngclose, and it deals \minus1d damage.
\end{spellcantrip}
\end{spellsubcontent}
\end{spellsection}
\subsubsection{Subspells}
\augment{3}{Corrosive}
The spell deals double damage to objects.
\augment{4}{Lingering}
The acid deals half damage on initial impact.
However, it deals damage to the target again at the end of each round for 2 rounds, including the initial round.
\begin{spellsection}{Cure Wounds}
\begin{spellcontent}
\begin{spelltargetinginfo}
\spelltwocol{\spelltgt{One creature}}{\spellrng{\rngmed}}
\end{spelltargetinginfo}
\begin{spelleffects}
\begin{spellattack}{Spellpower vs. Fortitude}
\spellsuccess The target is healed for \spelldamage{}.
\end{spellattack}
\spelltags{\glossterm{Life}}
\end{spelleffects}
\end{spellcontent}
\begin{spellfooter}
\spellinfo{Vivimancy}{Divine, Life, Nature}
\parhead*{Augments} Extended, Intensified, Mass, Quickened, Silent, Stilled
\end{spellfooter}
\begin{spellsubcontent}
\begin{spellcantrip}
Instead of healing, the spell grants \glossterm{temporary hit points} equal to twice your spellpower.
The duration of the temporary hit points is Sustain (swift).
\end{spellcantrip}
\end{spellsubcontent}
\end{spellsection}
\subsubsection{Subspells}
\augment{2}{Moderate Wounds}
For every 5 points of healing, this spell can instead cure 1 vital damage.
\augment{2}{Restore Senses}
Replace the spell's effects with the following:
\begin{spellcontent}
\begin{augmenteffects}
\spelleffect
One of the target's physical senses, such as sight or hearing, is restored to full capacity.
This can heal both magical and mundane conditions, but it cannot completely replace missing body parts required for a sense to function (such as missing eyes).
\spelltags{\glossterm{Flesh}}
\end{augmenteffects}
\end{spellcontent}
\augment{2}{Undead Bane}
If the target is undead, the spell gains a \plus2 bonus to accuracy and deals double damage on a critical hit.
\augment{3}{Remove Disease}
Replace the spell's effects with the following:
\begin{spellcontent}
\begin{augmenteffects}
\spelleffect
All diseases affecting the target are removed.
\spelltags{\glossterm{Flesh}}
\end{augmenteffects}
\end{spellcontent}
\augment{3}{Serious Wounds}
For every 2 points of healing, this spell can instead cure 1 vital damage.
\augment{4}{Critical Wounds}
For every point of healing, this spell can instead cure 1 vital damage.
\begin{spellsection}{Distort Image}
\begin{spellcontent}
\begin{spelltargetinginfo}
\spelltwocol{\spelltgt{One willing creature}}{\spellrng{\rngmed}}
\end{spelltargetinginfo}
\begin{spelleffects}
\spelleffect
The target's physical outline is distorted so it appears blurred, shifting, and wavering.
Targeted physical attacks against the target have a 20\% miss chance.
Spells and other non-physical attacks suffer no miss chance.
\spelldur Attunement
\spelltags{\glossterm{Glamer}, \glossterm{Visual}}
\end{spelleffects}
\end{spellcontent}
\begin{spellfooter}
\spellinfo{Illusion}{Arcane}
\parhead*{Augments} Extended, Quickened, Silent, Stilled
\end{spellfooter}
\begin{spellsubcontent}
\begin{spellcantrip}
The spell's duration becomes Sustain (swift).
\end{spellcantrip}
\end{spellsubcontent}
\end{spellsection}
\subsubsection{Subspells}
\augment{2}{Distort Light}
Replace the spell's targets and effects with the following:
\begin{spellcontent}
\begin{augmenttargetinginfo}
\spelltwocol{\spellemanation{\areamed radius from the target}}{\spellrng{\rngclose}}
\spelltgt{One object (Small or smaller)}
\end{augmenttargetinginfo}
\begin{augmenteffects}
\spelleffect
Light within or passing through the area is dimmed to be no brighter than shadowy illumination.
Any effect or object which blocks light also blocks this spell's emanation.
\spelldur Attunement (multiple)
\spelltags{\glossterm{Glamer}, \glossterm{Light}}
\end{augmenteffects}
\end{spellcontent}
\augment{3}{Disguise Image}
Replace the spell's effects with the following:
\begin{spellcontent}
\begin{augmenteffects}
\spelleffect
You make a Disguise check to alter the target's appearance (see \pcref{Disguise Creature}).
You gain a \plus5 bonus on the check, and you can freely alter the appearance of the target's clothes and equipment, regardless of their original form.
However, this effect is unable to alter the sound, smell, texture, or temperature of the target or its clothes and equipment.
\spelldur Attunement
\spelltags{\glossterm{Glamer}, \glossterm{Visual}}
\end{augmenteffects}
\end{spellcontent}
\augment{3}{Mirror Image}
Replace the spell's effects with the following:
\begin{spellcontent}
\begin{augmenteffects}
\spelleffect
Four illusory duplicates appear around the target that mirror its every move.
The duplicates shift chaotically in its space, making it difficult to identify the real creature.
All targeted attacks against the target have a 50% miss chance.
Whenever an attack misses in this way, it affects an image, destroying it.
\spelldur Sustain (swift)
\spelltags{\glossterm{Figment}, \glossterm{Visual}}
\end{augmenteffects}
\end{spellcontent}
\augment{4}{Shadow Mantle}
The spell's deceptive nature extends beyond merely altering light to affect the nature of reality itself.
The spell's miss chance changes to a failure chance, and applies to non-physical attacks as well as physical attacks.
In addition, it loses the \glossterm{Visual} tag, allowing it to affect creatures who do not rely on sight to affect the target.
\augment{5}{Displacement}
The target's image is futher distorted, and appears to be two to three feet from its real location.
The spell's miss chance increases to 50\%.
\begin{spellsection}{Elemental Blade}
\begin{spellheader}
\spelldesc{You transform the active part of a weapon into air, increasing its reach.}
\end{spellheader}
\begin{spellcontent}
\begin{spelltargetinginfo}
\spelltwocol{\spelltgt{One unattended weapon}}{\spellrng{\rngclose}}
\end{spelltargetinginfo}
\begin{spelleffects}
\spelleffect
The target weapon gains an additional five feet of reach, extending the wielder's threatened area.
This has no effect on ranged attacks with the weapon.
\spelldur Attunement
\spelltags{\glossterm{Air}, \glossterm{Shaping}}
\end{spelleffects}
\end{spellcontent}
\begin{spellfooter}
\spellinfo{Transmutation}{Arcane, Nature, War, Water}
\parhead*{Augments} Extended, Quickened, Silent, Stilled
\end{spellfooter}
\begin{spellsubcontent}
\begin{spellcantrip}
The spell's duration becomes Sustain (swift).
\end{spellcantrip}
\end{spellsubcontent}
\end{spellsection}
\subsubsection{Subspells}
\augment{2}{Aqueous Blade}
Replace the spell's effects with the following:
\begin{spellcontent}
\begin{augmenteffects}
\spelleffect
\glossterm{Strikes} with the affected weapon are made against Reflex defense instead of Armor defense.
However, damage with the weapon is halved, including any bonuses to damage.
\spelldur Attunement
\spelltags{\glossterm{Shaping}, \glossterm{Water}}
\end{augmenteffects}
\end{spellcontent}
\augment{4}{Zephyr Blade}
The weapon's reach is increased by ten feet instead of five feet.
\augment{7}{Greater Aqueous Blade}
Replace the spell's effects with the following:
\begin{spellcontent}
\begin{augmenteffects}
\spelleffect
\glossterm{Strikes} with the affected weapon are made against Reflex defense instead of Armor defense.
\spelldur Attunement
\spelltags{\glossterm{Shaping}, \glossterm{Water}}
\end{augmenteffects}
\end{spellcontent}
\begin{spellsection}{Fear}
\begin{spellheader}
\spelldesc{You terrify your foe.}
\end{spellheader}
\begin{spellcontent}
\begin{spelltargetinginfo}
\spelltwocol{\spelltgt{One creature}}{\spellrng{\rngmed}}
\end{spelltargetinginfo}
\begin{spelleffects}
\begin{spellattack}{Spellpower vs. Mental}
\spellsuccess The target is \frightened by you.
\spellcritical The target is \panicked by you.
\spellfailure The target is \shaken by you.
\end{spellattack}
\spelldur Condition
\spelltags{\glossterm{Delusion}, \glossterm{Mind}}
\end{spelleffects}
\end{spellcontent}
\begin{spellfooter}
\spellinfo{Enchantment}{Arcane}
\parhead*{Augments} Extended, Mass, Quickened, Silent, Stilled
\end{spellfooter}
\begin{spellsubcontent}
\begin{spellcantrip}
The spell's duration becomes Sustain (swift).
\end{spellcantrip}
\end{spellsubcontent}
\end{spellsection}
\subsubsection{Subspells}
\augment{2}{Redirected}
The target is afraid of a willing ally within the spell's range instead of being afraid of you.
\begin{spellsection}{Fireball}
\begin{spellheader}
\spelldesc{You create a small burst of flame.}
\end{spellheader}
\begin{spellcontent}
\begin{spelltargetinginfo}
\spelltwocol{\spellburst{\areasmall radius}}{\spellrng{\rngclose}}
\spelltgts{Everything in the area}
\end{spelltargetinginfo}
\begin{spelleffects}
\begin{spellattack}{Spellpower vs. Reflex}
\spellsuccess \spelldamage{fire}[1d4].
\spellcritical As above, but double damage.
\end{spellattack}
\spelltags{\glossterm{Fire}}
\end{spelleffects}
\end{spellcontent}
\begin{spellfooter}
\spellinfo{Evocation}{Arcane, Fire, Nature}
\parhead*{Augments} Extended, Intensified, Quickened, Silent, Stilled, Widened
\end{spellfooter}
\begin{spellsubcontent}
\begin{spellcantrip}
The spell affects a 5 foot radius, and it deals \minus1d damage.
\end{spellcantrip}
\end{spellsubcontent}
\end{spellsection}
\subsubsection{Subspells}
\augment{2}{Burning Hands}
Replace the spell's targets with the following:
\begin{spellcontent}
\begin{augmenttargetinginfo}
\spellburst{\arealarge cone}
\spelltgts{Everything in the area}
\end{augmenttargetinginfo}
\end{spellcontent}
\augment{2}{Flame Blade}
Replace the spell's targets and effects with the following:
\begin{spellcontent}
\begin{augmenttargetinginfo}
\spelltwocol{\spelltgt{One unattended weapon}}{\spellrng{\rngclose}}
\end{augmenttargetinginfo}
\begin{augmenteffects}
\spelleffect
The target weapon deals \plus2d damage with \glossterm{strikes}.
In addition, all damage dealt with the weapon with strikes becomes fire damage in addition to its normal damage types.
\spelldur Attunement
\spelltags{\glossterm{Fire}}
\end{augmenteffects}
\end{spellcontent}
\augment{3}{Fire Trap}
Replace the spell's targets and effects with the following:
\begin{spellcontent}
\begin{augmenttargetinginfo}
\spelltwocol{\spelltgt{One openable object (Large or smaller)}}{\spellrng{\rngclose}}
\end{augmenttargetinginfo}
\begin{augmenteffects}
\spelleffect
If a creature opens the target object, it explodes.
You make an attack against everything within an \areamed radius burst centered on the target.
After the object explodes in this way, the spell ends.
\begin{spellattack}{Spellpower vs. Reflex}
\spellsuccess \spelldamage{fire}[1d4].
\spellcritical As above, but double damage.
\end{spellattack}
\spelldur Attunement
\spelltags{\glossterm{Fire}, \glossterm{Trap}}
\end{augmenteffects}
\end{spellcontent}
\begin{spellsection}{Flare}
\begin{spellcontent}
\begin{spelltargetinginfo}
\spelltwocol{\spellburst{5 foot radius}}{\spellrng{\rngmed}}
\spelltgts{All creatures in the area}
\end{spelltargetinginfo}
\begin{spelleffects}
\spelleffect
A brilliant light appears in the area until the end of the round.
It illuminates a 100 foot radius around the area with bright light.
\begin{spellattack}{Spellpower vs. Reflex}
\spellsuccess
The target is \partiallyblinded.
\spellcritical
The target is \blinded instead.
\end{spellattack}
\spelldur Condition
\spelltags{\glossterm{Figment}, \glossterm{Light}, \glossterm{Visual}}
\end{spelleffects}
\end{spellcontent}
\begin{spellfooter}
\spellinfo{Illusion}{Arcane, Divine, Nature}
\parhead*{Augments} Extended, Quickened, Silent, Stilled, Widened
\end{spellfooter}
\begin{spellsubcontent}
\begin{spellcantrip}
The spell affects a single creature, rather than an area.
\end{spellcantrip}
\end{spellsubcontent}
\end{spellsection}
\subsubsection{Subspells}
\augment{2}{Dancing Lights}
Replace the spell's effects with the following:
\begin{spellcontent}
\begin{augmenteffects}
\spelleffect
Up to four glowing lights appear in the area.
The lights resemble lanterns or torches, and shed bright light in the same 20 foot radius.
However, you can freely choose the color of the lights when you cast the spell.
During each movement phase, you can move the lights up to 100 feet in any direction.
If one of the lights ever goes out of range from you, it immediately winks out.
\spelldur Sustain (swift)
\spelltags{\glossterm{Figment}, \glossterm{Light}, \glossterm{Visual}}
\end{augmenteffects}
\end{spellcontent}
\augment{2}{Expanded}
The spell's area increases to \areasmall.
This allows the standard Widened augment to be used to expand the spell's area further.
\augment{3}{Faerie Fire}
Each target is surrounded with a pale glow made of hundreds of ephemeral points of lights, causing it to bright light in a 5 foot radius as a candle.
The lights impose a \minus10 penalty to Stealth checks.
In addition, they reveal the outline of the creatures if they become \glossterm{invisible}.
This allows observers to see their location, though not to see them perfectly.
\augment{3}{Illuminating}
The brilliant light persists as long as you spend a \glossterm{swift action} each round to sustain it.
The light has no additional effects on creatures in the area.
\augment{4}{Flashbang}
An intense sound accompanies the flash of light caused by the spell.
If the spell's attack is successful, the target is also \deafened as a condition.
This is an \glossterm{Auditory}, \glossterm{Figment} effect.
\augment{5}{Blinding}
The spell's critical effect makes the target \blinded as a condition, rather than just for one round.
In addition, the blindness replaces the spell's normal success effect, rather than being applied in addition to it.
\augment{5}{Universal}
The light radiates from every point in the area simultaneously, making it impossible to avoid.
The spell's attack is made against Fortitude instead of Reflex.
\begin{spellsection}{Foresight}
\begin{spellheader}
\spelldesc{You grant a creature the ability to see fractions of a second into the future.}
\end{spellheader}
\begin{spellcontent}
\begin{spelltargetinginfo}
\spelltwocol{\spelltgt{One willing creature}}{\spellrng{\rngclose}}
\end{spelltargetinginfo}
\begin{spelleffects}
\spelleffect
The target gains a \plus2 bonus to \glossterm{accuracy} with physical attacks.
\spelldur Attunement
\spelltags{\glossterm{Enhancement}}
\end{spelleffects}
\end{spellcontent}
\begin{spellfooter}
\spellinfo{Divination}{Arcane, Divine, Nature}
\parhead*{Augments} Extended, Quickened, Silent, Stilled
\end{spellfooter}
\begin{spellsubcontent}
\begin{spellcantrip}
The spell's duration becomes Sustain (swift).
\end{spellcantrip}
\end{spellsubcontent}
\end{spellsection}
\subsubsection{Subspells}
\augment{2}{Augury}
Replace the spell's effects with the following:
\begin{spellcontent}
\begin{augmenteffects}
\spelleffect
Choose an action that the target could take.
You learn whether the stated action is likely to bring good or bad results for it within the next hour.
This spell provides one of four results:
\begin{itemize}
\item Weal (if the action will probably bring good results).
\item Woe (for bad results).
\item Weal and woe (for both).
\item No response (for actions that don't have especially good or bad results).
\end{itemize}
This spell does not describe the future with certainty.
It describes which result is most probable.
The more unambiguous the action's effects, the more likely the spell is to be correct.
% TODO: inconsistent wording; this spell vs. this subspell
After using this subspell, you cannot cast it again until the hour affected by the previous casting is over, regardless of whether the action was taken.
\end{augmenteffects}
\end{spellcontent}
\augment{7}{Foresee Actions}
The target can learn what actions all creatures it can observe intend to take during each phase before it decides its actions for that phase.
It learns this information in the instant before it acts, and normally does not have time to communicate it to other creatures.
\begin{spellsection}{Inertial Shield}
\begin{spellheader}
\spelldesc{You create a barrier around your ally that resists physical intrusion.}
\end{spellheader}
\begin{spellcontent}
\begin{spelltargetinginfo}
\spelltwocol{\spelltgt{One creature}}{\spellrng{\rngclose}}
\end{spelltargetinginfo}
\begin{spelleffects}
\spelleffect
The target gains \glossterm{damage reduction} against \glossterm{physical damage} equal to your spellpower.
In addition, it is \glossterm{vulnerable} to arcane damage.
\spelldur Attunement
\spelltags{\glossterm{Shielding}}
\end{spelleffects}
\end{spellcontent}
\begin{spellfooter}
\spellinfo{Abjuration}{Arcane}
\parhead*{Augments} Extended, Quickened, Silent, Stilled
\end{spellfooter}
\begin{spellsubcontent}
\begin{spellcantrip}
The spell's duration becomes Sustain (swift).
\end{spellcantrip}
\end{spellsubcontent}
\end{spellsection}
\subsubsection{Subspells}
\augment{3}{Complete}
The damage reduction applies against all damage, not just physical damage.
\augment{4}{Immunity}
Replace the spell's effects with the following:
\begin{spellcontent}
\begin{augmenteffects}
\spelleffect
Choose a type of damage.
The target becomes immune to damage of the chosen type.
Attacks that deal damage of multiple types still inflict damage normally unless the target is immune to all types of damage dealt.
\end{augmenteffects}
\end{spellcontent}
\augment{4}{Retributive}
Damage resisted by this spell is reflected back to the attacker as life damage.
If the attacker is beyond \rngclose range of the target, this reflection fails.
\par Any effect which increases this spell's range increases the range of this effect by the same amount.
\par
This is a \glossterm{Life} effect from the \glossterm{Vivimancy} school.
\augment{5}{Empowered}
The damage reduction increases by an amount equal to your spellpower.
\begin{spellsection}{Inflict Wounds}
\begin{spellcontent}
\begin{spelltargetinginfo}
\spelltwocol{\spelltgt{One creature}}{\spellrng{\rngmed}}
\end{spelltargetinginfo}
\begin{spelleffects}
\begin{spellattack}{Spellpower vs. Fortitude}
\spellsuccess \spelldamage{life}.
\spellcritical As above, but double damage.
\end{spellattack}
\spelltags{\glossterm{Life}}
\end{spelleffects}
\end{spellcontent}
\begin{spellfooter}
\spellinfo{Vivimancy}{Arcane, Divine, Nature}
\parhead*{Augments} Extended, Intensified, Mass, Quickened, Silent, Stilled
\end{spellfooter}
\begin{spellsubcontent}
\begin{spellcantrip}
The spell's range becomes \rngclose, and it deals \minus1d damage.
\end{spellcantrip}
\end{spellsubcontent}
\end{spellsection}
\subsubsection{Subspells}
\augment{3}{Drain Life}
You gain temporary hit points equal to half the damage you deal with this spell.
\augment{4}{Death Knell}
If the spell's attack succeeds, the target suffers a death knell.
At the end of each round, if the target has 0 hit points, it immediately dies.
This effect lasts until the target removes this condition.
\par
This is a \glossterm{Death} effect.
\begin{spellsection}{Lightning Bolt}
\begin{spellheader}
\spelldesc{You create a bolt of electricity that fries your foes.}
\end{spellheader}
\begin{spellcontent}
\begin{spelltargetinginfo}
\spellburst{\arealarge line, 10 ft\. wide}
\spelltgts{Everything in the area}
\end{spelltargetinginfo}
\begin{spelleffects}
\begin{spellattack}{Spellpower vs. Reflex}
\spellspecial You gain a \plus2 bonus to accuracy against creatures wearing metal armor or otherwise carrying a significant amount of metal.
\spellsuccess
\spelldamage{electricity}[1d4].
\spellcritical As above, but double damage.
\end{spellattack}
\spelltags{\glossterm{Electricity}}
\end{spelleffects}
\end{spellcontent}
\begin{spellfooter}
\spellinfo{Evocation}{Arcane, Nature}
\parhead*{Augments} Intensified, Quickened, Silent, Stilled, Widened
\end{spellfooter}
\begin{spellsubcontent}
\begin{spellcantrip}
The spell's area becomes a 5 ft\. wide \areamed line.
\end{spellcantrip}
\end{spellsubcontent}
\end{spellsection}
\subsubsection{Subspells}
\augment{4}{Instantaneous}
The lightning bolt created by the spell is faster, but less penetrating.
The spell's attack is made against Fortitude defense instead of Reflex defense.
\begin{spellsection}{Planar Disruption}
\begin{spellheader}
\spelldesc{You disrupt a creature's body by partially thrusting it into another plane.}
\end{spellheader}
\begin{spellcontent}
\begin{spelltargetinginfo}
\spelltwocol{\spelltgt{One creature}}{\spellrng{\rngmed}}
\end{spelltargetinginfo}
\begin{spelleffects}
\begin{spellattack}{Spellpower vs. Mental}
\spellsuccess \spelldamage{physical}.
\spellcritical
As above, but double damage.
In addition, if the creature is an \glossterm{outsider} native to another plane, it is sent back to its home plane.
\end{spellattack}
\spelltags{\glossterm{Planar}, \glossterm{Teleportation}}
\end{spelleffects}
\end{spellcontent}
\begin{spellfooter}
\spellinfo{Conjuration}{Arcane, Divine}
\parhead*{Augments} Extended, Intensified, Mass, Quickened, Silent, Stilled
\end{spellfooter}
\begin{spellsubcontent}
\begin{spellcantrip}
The spell's range becomes \rngclose, and it deals \minus1d damage.
\end{spellcantrip}
\end{spellsubcontent}
\end{spellsection}
\begin{spellsection}{Poison}
\begin{spellheader}
\spelldesc{You weaken your foe with a potent poison.}
\end{spellheader}
\begin{spellcontent}
\begin{spelltargetinginfo}
\spelltwocol{\spelltgt{One living creature}}{\spellrng{\rngclose}}
\end{spelltargetinginfo}
\begin{spelleffects}
\spelleffect
At the end of each round, you make a Spellpower vs. Fortitude attack against the target.
Success means the target takes poison damage equal to your spellpower.
If this is the second successful attack, the target takes a \minus2 penalty to \glossterm{accuracy}, \glossterm{checks}, and \glossterm{defenses}.
If this is the third successful attack, the penalty increases to -5.
\spelldur Condition
\spelltags{\glossterm{Poison}}
\end{spelleffects}
\end{spellcontent}
\begin{spellfooter}
\spellinfo{Transmutation}{Destruction, Divine, Nature}
\parhead*{Augments} Extended, Mass, Quickened, Silent, Stilled
\end{spellfooter}
\begin{spellsubcontent}
\begin{spellcantrip}
The spell does not have additional effects other than damage.
\end{spellcantrip}
\end{spellsubcontent}
\end{spellsection}
\begin{spellsection}{Polymorph}
\begin{spellheader}
\spelldesc{You change the target's physical form.}
\end{spellheader}
\begin{spellcontent}
\begin{spelltargetinginfo}
\spelltwocol{\spelltgt{One willing creature}}{\spellrng{\rngmed}}
\end{spelltargetinginfo}
\begin{spelleffects}
\spelleffect
You increase or decrease the target's size by one size category.
\spelldur Attunement
\spelltags{\glossterm{Shaping}, \glossterm{Sizing}}
\end{spelleffects}
\end{spellcontent}
\begin{spellfooter}
\spellinfo{Transmutation}{Arcane, Nature}
\parhead*{Augments} Extended, Quickened, Silent, Stilled
\end{spellfooter}
\begin{spellsubcontent}
\begin{spellcantrip}
The spell's duration becomes Sustain (swift).
\end{spellcantrip}
\end{spellsubcontent}
\end{spellsection}
\subsubsection{Subspells}
\augment{3}{Alter Appearance}
You can also make a Disguise check to alter the target's appearance (see \pcref{Disguise Creature}).
You gain a \plus5 bonus on the check, and you ignore penalties for changing the target's gender, race, subtype, or age.
However, this effect is unable to alter the target's clothes or equipment in any way.
\augment{4}{Fabricate}
Replace the spell's targets and effects with the following:
\begin{spellcontent}
\begin{augmenttargetinginfo}
\spelltwocol{\spelltgts{One or more unattended, nonmagical objects (Large or smaller); see text}}{\spellrng{\rngclose}}
\end{augmenttargetinginfo}
\begin{augmenteffects}
\spelleffect
You make a Craft check to transform the targets into a new item (or items) made of the same materials.
You require none of the tools or time expenditure that would normally be necessary.
The total size of all targets combined must be Large size or smaller.
\spelltags{\glossterm{Shaping}}
\end{augmenteffects}
\end{spellcontent}
\begin{spellsection}{Protection from Alignment}
\begin{spellcontent}
\begin{spelltargetinginfo}
\spelltwocol{\spelltgt{One creature}}{\spellrng{\rngclose}}
\end{spelltargetinginfo}
\begin{spelleffects}
\spellspecial
Choose an alignment other than neutral (chaotic, good, evil, lawful).
This spell gains the tag for that alignment's \glossterm{opposed alignment}.
\spelleffect
The target gains damage reduction equal to your spellpower against physical effects that have the chosen alignment, and physical attacks made by creatures with the chosen alignment.
\spelltags{\glossterm{Shielding}}
\end{spelleffects}
\end{spellcontent}
\begin{spellfooter}
\spellinfo{Abjuration}{Arcane, Chaos, Divine, Evil, Good, Law}
\parhead*{Augments} Extended, Quickened, Silent, Stilled
\end{spellfooter}
\begin{spellsubcontent}
\begin{spellcantrip}
The spell's duration becomes Sustain (swift).
\end{spellcantrip}
\end{spellsubcontent}
\end{spellsection}
\subsubsection{Subspells}
\augment{3}{Complete}
The damage reduction also applies against non-physical effects.
\augment{4}{Retributive}
Whenever a creature with the chosen alignment makes a physical melee attack against the target, you make a Spellpower vs. Mental attack against the attacking creature.
Success means the attacker takes \spelldamage{divine}[d4].
\begin{spellsection}{Scry}
\begin{spellheader}
\spelldesc{You create a scrying sensor that allows you to see at a distance.}
\end{spellheader}
\begin{spellcontent}
\begin{spelltargetinginfo}
\spelltwocol{\spelltgt{One square}}{\spellrng{\rngmed}}
\end{spelltargetinginfo}
\begin{spelleffects}
\spelleffect
A Fine object appears floating in the air in the target space.
It resembles a human eye in size and shape, though it is \glossterm{invisible}.
At the start of each round, you choose whether you see from this sensor or from your body.
The sensor's visual acuity is the same as your own, except that it does not share the benefits of any \glossterm{magical} effects that improve your vision.
You otherwise act normally, though you may have difficulty moving or taking actions if the sensor cannot see your body or your intended targets, effectively making you \blinded.
If undisturbed, the sensor floats in the air in its position.
As a standard action, you can concentrate to move the sensor up to 30 feet in any direction, even vertically.
You can only have one casting of this spell active at once.
If you cast it again, any previous castings of the spell are dismissed.
\spelldur Attunement
\spelltags{\glossterm{Scrying}}
\end{spelleffects}
\end{spellcontent}
\begin{spellfooter}
\spellinfo{Divination}{Arcane, Divine, Nature}
\parhead*{Augments} Extended, Mass, Quickened, Silent, Stilled
\end{spellfooter}
\begin{spellsubcontent}
\begin{spellcantrip}
The sensor cannot be moved after it is originally created, and the spell's duration becomes Sustain (swift).
\end{spellcantrip}
\end{spellsubcontent}
\end{spellsection}
\subsubsection{Subspells}
\augment{2}{Alarm}
The sensor continues to observe its surroundings while you are not sensing through it.
If it sees a creature or object of Tiny size or larger moving within 50 feet of it, it will trigger a mental "ping" that only you can notice.
You must be within 1 mile of the sensor to receive this mental alarm.
This mental sensation is strong enough to wake you from normal sleep, but does not otherwise disturb concentration.
\augment{2}{Auditory}
At the start of each round, you can choose whether you hear from the sensor or from your body.
This choice is made independently from your sight.
The sensor's auditory acuity is the same as your own, except that it does not share the benefits of any \glossterm{magical} effects that improve your hearing.
\augment{3}{Accelerated}
When you move the sensor, you can move it up to 100 feet, instead of up to 30 feet.
\augment{3}{Dual}
You create an additional sensor in the same location.
You must move and see through each sensor individually.
\augment{3}{Penetrating}
The spell's range becomes \rngunrestricted, allowing you to cast it into areas where you do not have \glossterm{line of sight} or \glossterm{line of effect}.
\augment{4}{Reverse Scrying}
% TODO: wording
The sensor created by this spell appears at the location of the source of the ability that created the target sensor.
Replace the spell's targets with the following:
\begin{spellcontent}
\begin{augmenttargetinginfo}
\spelltwocol{\spelltgt{One magical sensor}}{\spellrng{\rngmed}}
\end{augmenttargetinginfo}
\end{spellcontent}
\augment{4}{Semi-Autonomous}
You can move the sensor as a \glossterm{swift action} rather than as a standard action.
\augment{5}{Scry Creature}
You must make a Spellpower vs. Mental attack against the target.
Success means the sensor appears in the target's space.
Failure means the sensor does not appear at all.
Replace the spell's targets with the following:
\begin{spellcontent}
\begin{augmenttargetinginfo}
\spellspecial
You must specify your target with a precise mental image of its appearance.
The image does not have to be perfect, but it must unambiguously identify the target.
If you specify its appearance incorrectly, or if the target has changed its appearance, you may accidentally target a different creature, or the spell may simply fail.
\spelltwocol{\spelltgt{One creature}}{\spellrng{Same plane (Unrestricted)}}
\end{augmenttargetinginfo}
\end{spellcontent}
\augment{6}{Split Senses}
You do not have to choose whether to sense from the perspective of the sensor or from the perspective of your own body.
You constantly receive sensory input from both your body and the sensor.
\begin{spellsection}{Smite}
\begin{spellheader}
\spelldesc{You smite a foe with holy (or unholy) power.}
\end{spellheader}
\begin{spellcontent}
\begin{spelltargetinginfo}
\spelltwocol{\spelltgt{One creature}}{\spellrng{\rngmed}}
\end{spelltargetinginfo}
\begin{spelleffects}
\begin{spellattack}{Spellpower vs. Mental}
\spellsuccess \spelldamage{divine}.
\spellcritical As above, but double damage.
\end{spellattack}
\end{spelleffects}
\end{spellcontent}
\begin{spellfooter}
\spellinfo{Channeling}{Divine}
\parhead*{Augments} Extended, Intensified, Mass, Quickened, Silent, Stilled
\end{spellfooter}
\begin{spellsubcontent}
\begin{spellcantrip}
The spell's range becomes \rngclose, and it deals \minus1d damage.
\end{spellcantrip}
\end{spellsubcontent}
\end{spellsection}
\begin{spellsection}{Summon Monster}
\begin{spellheader}
\spelldesc{You summon a creature to fight by your side.}
\end{spellheader}
\begin{spellcontent}
\begin{spelltargetinginfo}
\spelltwocol{\spelltgt{One unoccupied square}}{\spellrng{\rngmed}}
\end{spelltargetinginfo}
\begin{spelleffects}
\spelleffect
A creature appears in the target location.
It visually appears to be a common Small or Medium animal of your choice, though in reality it is a manifestation of magical energy.
Regardless of the appearance and size chosen, the creature has hit points equal to twice your spellpower.
All of its defenses are equal to your 5 \add your spellpower, and its land speed is equal to 30 feet.
Each round, you choose the creature's actions.
There are only two actions it can take.
As a move action, it can move as you direct.
As a standard action, it can make a melee \glossterm{strike} against a creature it threatens.
Its accuracy is equal to your spellpower.
If it hits, it deals 1d3 damage \plus1d per two spellpower.
The type of damage dealt by this attack depends on the creature's appearance.
Most animals bite or claw their foes, which deals bludgeoning and slashing damage.
\spelldur Sustain (swift)
\spelltags{\glossterm{Manifestation}}
\end{spelleffects}
\end{spellcontent}
\begin{spellfooter}
\spellinfo{Conjuration}{Arcane, Divine, Nature}
\parhead*{Augments} Extended, Quickened, Silent, Stilled
\end{spellfooter}
\begin{spellsubcontent}
\begin{spellcantrip}
The spell's duration becomes Sustain (standard).
\end{spellcantrip}
\end{spellsubcontent}
\end{spellsection}
\subsubsection{Subspells}
\augment{2}{Summon Bear}
The creature appears to be a Medium bear.
As a standard action, it can make a \glossterm{grapple} attack against a creature it threatens.
Its accuracy is the same as its accuracy with strikes.
While grappling, the manifested creature can either make a strike or attempt to escape the grapple.
This augment replaces the effects of any other augments that change the appearance of the creature.
\begin{spellsection}{Telekinesis}
\begin{spellcontent}
\begin{spelltargetinginfo}
\spelltwocol{\spelltgt{One Medium or smaller creature or object}}{\spellrng{\rngclose}}
\end{spelltargetinginfo}
\begin{spelleffects}
\begin{spellattack}{Spellpower vs. Mental}
\spellsuccess
You move the target up five feet per spellpower. Moving the target upwards costs twice the normal movement cost.
\spellcritical
As above, but you move the target ten feet per spellpower instead of five feet per spellpower.
\end{spellattack}
\spelltags{\glossterm{Telekinesis}}
\end{spelleffects}
\end{spellcontent}
\begin{spellfooter}
\spellinfo{Evocation}{Arcane}
\parhead*{Augments} Extended, Mass, Quickened, Silent, Stilled
\end{spellfooter}
\begin{spellsubcontent}
\begin{spellcantrip}
If your attack succeeds, you move the target one foot per spellpower. In addition, the spell has no additional effects on a critical hit.
\end{spellcantrip}
\end{spellsubcontent}
\end{spellsection}
\subsubsection{Subspells}
\augment{2}{Mending}
Replace the spell's targets and effects with the following:
\begin{spellcontent}
\begin{augmenttargetinginfo}
\spelltwocol{\spelltgt{One unattended object}}{\spellrng{\rngclose}}
\end{augmenttargetinginfo}
\begin{augmenteffects}
\spelleffect
The target is healed for \spelldamage{}.
\end{augmenteffects}
\end{spellcontent}
\augment{2}{Precise}
Replace the spell's effects with the following:
\begin{spellcontent}
\begin{augmenteffects}
\begin{spellattack}{Spellpower vs. Mental}
\spellsuccess
You move the target up to five feet in any direction.
In addition, you can make a check to manipulate the target as if you were using your hands.
The check's result has a maximum equal to your attack result.
\end{spellattack}
\spelltags{\glossterm{Telekinesis}}
\end{augmenteffects}
\end{spellcontent}
\augment{3}{Binding}
If your attack roll beat both the target's Fortitude and Mental defenses, it is \immobilized after the forced movement is finished.
This is a \glossterm{condition}, and lasts until removed.
\augment{3}{Levitate}
Replace the spell's targets and effects with the following:
\begin{spellcontent}
\begin{augmenttargetinginfo}
\spelltwocol{\spelltgt{One unattended object or willing creature (Medium or smaller)}}{\spellrng{\rngclose}}
\end{augmenttargetinginfo}
\begin{augmenteffects}
\spelleffect
The target floats in midair, unaffected by gravity.
During the movement phase, you can move the target up to ten feet in any direction.
\spelldur Sustain (swift)
\end{augmenteffects}
\end{spellcontent}
\begin{spellsection}{Water Mastery}
\begin{spellheader}
\spelldesc{You create a wave of water to crush your foes.}
\end{spellheader}
\begin{spellcontent}
\begin{spelltargetinginfo}
\spellburst{\arealarge line, 10 ft\. wide}
\spelltgts{Everything in the area}
\end{spelltargetinginfo}
\begin{spelleffects}
\begin{spellattack}{Spellpower vs. Fortitude}
\spellsuccess \spelldamage{bludgeoning}[1d4].
\spellcritical As above, but double damage.
\end{spellattack}
\spelltags{\glossterm{Manifestation}, \glossterm{Water}}
\end{spelleffects}
\end{spellcontent}
\begin{spellfooter}
\spellinfo{Conjuration}{Nature, Water}
\parhead*{Augments} Intensified, Quickened, Silent, Stilled, Widened
\end{spellfooter}
\begin{spellsubcontent}
\begin{spellcantrip}
The spell's area becomes a 5 ft.\ wide, \areamed line.
\end{spellcantrip}
\end{spellsubcontent}
\end{spellsection}
\subsubsection{Subspells}
\augment{2}{Aqueuous Sphere}
Replace the spell's targets with the following:
\begin{spellcontent}
\begin{augmenttargetinginfo}
\spelltwocol{\spellburst{\areasmall radius}}{\spellrng{\rngclose}}
\spelltgts{Everything in the area}
\end{augmenttargetinginfo}
\end{spellcontent}
\augment{2}{Create Water}
Replace the spell's targets and effects with the following:
\begin{spellcontent}
\begin{augmenttargetinginfo}
\spellrng{\rngclose}
\end{augmenttargetinginfo}
\begin{augmenteffects}
\spelleffect
You create up to one gallon of wholesome, drinkable water.
The water can be created at multiple locations within the ritual's range, allowing you to fill multiple small water containers.
\spelltags{\glossterm{Creation}, \glossterm{Water}}
\end{augmenteffects}
\end{spellcontent}
\augment{4}{Sustained}
The area affected by this spell becomes completely filled with water.
You can sustain the water as a \glossterm{swift action}.
Creatures in this \glossterm{zone} suffer penalties appropriate for fighting underwater, and may be unable to breathe.
\begin{spellsection}{Web}
\begin{spellheader}
\spelldesc{
You create a many-layered mass of strong, stricky strands that trap creatures caught within them.
The strands are similar to spider webs, but larger and tougher.
}
\end{spellheader}
\begin{spellcontent}
\begin{spelltargetinginfo}
\spelltwocol{\spellzone{\areasmall radius}}{\spellrng{\rngclose}}
\spelltgts{Everything in the area}
\end{spelltargetinginfo}
\begin{spelleffects}
\spelleffect
The area becomes filled with webs, making it \glossterm{difficult terrain}.
Each 5-ft.\ square of webbing has hit points equal to your spellpower, and is \glossterm{vulnerable} to fire.
\begin{spellattack}{Spellpower vs. Reflex}
\spellsuccess The target is \immobilized as long as it has webbing from this spell in its space.
\end{spellattack}
\spelldur Sustain (swift)
\spelltags{\glossterm{Manifestation}}
\end{spelleffects}
\end{spellcontent}
\begin{spellfooter}
\spellinfo{Conjuration}{Arcane, Nature}
\parhead*{Augments} Extended, Quickened, Silent, Stilled, Widened
\end{spellfooter}
\begin{spellsubcontent}
\begin{spellcantrip}
The spell's duration becomes Sustain (standard).
\end{spellcantrip}
\end{spellsubcontent}
\end{spellsection}
\subsubsection{Subspells}
\augment{3}{Reinforced}
Each 5-ft.\ square of webbing gains additional hit points equal to your spellpower.
In addition, the webs are no longer vulnerable to fire.
\begin{spellsection}{Word of Faith}
\begin{spellheader}
\spelldesc{You speak an utterance that rebukes those who do not share your faith.}
\end{spellheader}
\begin{spellcontent}
\begin{spelltargetinginfo}
\spellburst{\areamed radius from you}
\spelltgts{Creatures in the area that do not worship your deity}
\end{spelltargetinginfo}
\begin{spelleffects}
\begin{spellattack}{Spellpower vs. Mental}
\spellsuccess \spelldamage{divine}.
\spellcritical As above, but double damage.
\end{spellattack}
\end{spelleffects}
\end{spellcontent}
\begin{spellfooter}
\spellinfo{Channeling}{Divine}
\parhead*{Augments} Intensified, Quickened, Silent, Stilled, Widened
\end{spellfooter}
\begin{spellsubcontent}
\begin{spellcantrip}
The spell's area becomes an \areasmall radius.
\end{spellcantrip}
\end{spellsubcontent}
\end{spellsection}
\subsubsection{Subspells}
\augment{4}{Bolstering}
Creatures in the spell's area that worship your deity heal 1d4 damage \plus1d per two spellpower.

\section{Ritual Descriptions}


\begin{spellsection}{Animate Dead}[3]

\begin{spellheader}
\spelldesc{You bind a fragment of a dead creature's soul to its corpse, reanimating it as an undead skeleton or zombie.}
\end{spellheader}


\begin{ability}{Animate Dead}[\glossterm{Attune} (multiple)]

Choose any number of corpses within \rngclose range.
The combined levels of all targets cannot exceed your spellpower.
The target becomes an undead creature that obeys your spoken commands.
You choose whether to create a skeleton or a zombie.
Creating a zombie require a mostly intact corpse, including most of the flesh.
Creating a skeleton only requires a mostly intact skeleton.
If a skeleton is made from an intact corpse, the flesh quickly falls off the animated bones.

This ritual takes one hour to perform.

\end{ability}




\parhead{Schools} Vivimancy

\parhead{Spell Lists} Arcane, Divine
\end{spellsection}


\begin{spellsection}{Awaken}[7]


\begin{ability}{Awaken}

Choose a Large or smaller willing animal within \rngclose range.
The target becomes sentient.
Its Intelligence becomes 1d6 \sub 5.
Its type changes from animal to magical beast.
It gains the ability to speak and understand one language that you know of your choice.
This effect is permanent.

This ritual takes 24 hours to perform, and requires 49 action points from its participants.

\end{ability}




\parhead{Schools} Transmutation

\parhead{Spell Lists} Nature
\end{spellsection}


\begin{spellsection}{Binding}[3]


\begin{ability}{Binding}[\glossterm{Attune}]

This ritual creates a \areasmall radius zone.
The outlines of the zone are denoted by a magic circle physically inscribed on the ground during the ritual.
The circle is obvious, but a DR 16 Perception or Spellcraft check is required to verify that the circle belongs to a \ritual{binding} ritual.
If the circle's perimeter is broken, the ritual's effects end immediately.
Whenever a creature enters the area, you make a Spellpower vs. Mental attack against it.
\hit The target is unable to escape the ritual's area physically or alter the circle in any way.
It treats the edge of the area as an impassable barrier, preventing the effects of any of its abilities from extending outside that area.

This ritual takes one hour to perform.

\end{ability}




\parhead{Schools} Abjuration

\parhead{Spell Lists} Arcane, Divine
\end{spellsection}


\subsubsection{Subrituals}


\begin{ability}[\nth{5}]{Dimension Lock}
This subritual functions like the \ritual{binding} ritual, except that a struck creature also cannot travel extradimensionally.
This prevents all \glossterm{Manifestation}, \glossterm{Planar}, and \glossterm{Teleportation} effects.
\end{ability}
\vspace{0.25em}


\begin{spellsection}{Bless Water}[1]


\begin{ability}{Bless Water}[\glossterm{Attune} (multiple)]

Choose one pint of unattended, nonmagical water within \rngclose range.
The target becomes holy water.
Holy water can be can be thrown as a splash weapon, dealing 1d8 points of damage to a struck undead creature or an evil outsider.

This ritual takes one minute to perform.

\end{ability}




\parhead{Schools} Channeling

\parhead{Spell Lists} Divine
\end{spellsection}


\begin{spellsection}{Blessed Transit}[5]


\begin{ability}{Blessed Transit}[\glossterm{Teleportation}]

This ritual functions like the \ritual{overland teleporation} ritual, except that the destination must be a temple or equivalent holy site to your deity.

\end{ability}




\parhead{Schools} Conjuration

\parhead{Spell Lists} divine
\end{spellsection}


\begin{spellsection}{Create Object}[3]


\begin{ability}{Create Object}[\glossterm{Attune} (multiple), \glossterm{Manifestation}]

Make a Craft check to create an object of no greater than Small size.
The object appears out of thin air in an unoccupied square within \rngclose range.
% TODO: add ability to create objects of other sizes/materials
It must be made of nonliving, nonreactive vegetable matter, such as wood or cloth.

This ritual takes one hour to perform.

\end{ability}




\parhead{Schools} Conjuration

\parhead{Spell Lists} Arcane, Divine, Nature
\end{spellsection}


\begin{spellsection}{Create Sustenance}[3]


\begin{ability}{Create Sustenance}[\glossterm{Creation}]

Choose an unoccupied square within \rngclose range.
This ritual creates food and drink in that square that is sufficient to sustain two Medium creatures per spellpower for 24 hours.
The food that this ritual creates is simple fare of your choice -- highly nourishing, if rather bland.

This ritual takes one hour to perform.

\end{ability}




\parhead{Schools} Conjuration

\parhead{Spell Lists} Arcane, Divine, Nature
\end{spellsection}


\begin{spellsection}{Curse Water}[1]


\begin{ability}{Curse Water}[\glossterm{Attune} (multiple)]

Choose one pint of unattended, nonmagical water within \rngclose range.
The target becomes unholy water.
Unholy water can be can be thrown as a splash weapon, dealing 1d8 points of damage to a struck good outsider.

This ritual takes one minute to perform.

\end{ability}




\parhead{Schools} Channeling

\parhead{Spell Lists} Divine
\end{spellsection}


\begin{spellsection}{Discern Location}[5]


\begin{ability}{Discern Location}

Choose a creature or object on the same plane as you.
You do not need \glossterm{line of sight} or \glossterm{line of effect} to the target.
However, you must specify your target with a precise mental image of its appearance.
The image does not have to be perfect, but it must unambiguously identify the target.
You learn the location (place, name, business name, or the like), community, country, and continent where the target lies.

This ritual takes 24 hours to perform, and it requires 25 action points from its participants.

\end{ability}




\parhead{Schools} Divination

\parhead{Spell Lists} Arcane, Divine, Nature
\end{spellsection}


\subsubsection{Subrituals}


\begin{ability}[\nth{7}]{Interplanar}
This subritual functions like the \ritual{discern location} ritual, except that the target does not have to be on the same plane as you.
It gains the \glossterm{Planar} tag in addition to the tags from the \ritual{discern location} ritual.

This ritual takes 24 hours to perform, and it requires 49 action points from its participants.
\end{ability}
\vspace{0.25em}


\begin{spellsection}{Endure Elements}[1]


\begin{ability}{Endure Elements}[\glossterm{Attune} (multiple)]

Choose a willing creature or unattended object within \rngclose range.
The target suffers no harm from being in a hot or cold environment.
It can exist comfortably in conditions between \minus50 and 140 degrees Fahrenheit.
Its equipment, if any, is also protected.
This does not protect the target from fire or cold damage.

This ritual takes one minute to perform.

\end{ability}




\parhead{Schools} Abjuration

\parhead{Spell Lists} Arcane, Divine, Nature
\end{spellsection}


\begin{spellsection}{Explosive Runes}[3]


\begin{ability}{Explosive Runes}[\glossterm{Attune} (multiple), \glossterm{Trap}]

Choose a Small or smaller unattended object with writing on it within \rngclose range.
In addition, choose a type of \glossterm{energy damage} (cold, electricity, fire, or sonic).
This ritual gains the tag appropriate to the chosen energy type.
If a creature reads the target, the target explodes.
You make a Spellpower vs. Reflex attack against everything within an \areamed radius from the target.
\hit Each target takes \glossterm{standard damage} \minus1d of the damage type chosen.

After the target explodes in this way, the ritual is \glossterm{dismissed}.
If the target object is destroyed or rendered illegible, the ritual is dismissed without exploding.
This ritual takes one hour to perform.

\end{ability}




\parhead{Schools} Evocation

\parhead{Spell Lists} Arcane
\end{spellsection}


\begin{spellsection}{Fertility}[3]


\begin{ability}{Fertility}

This ritual creates an area of bountiful growth in a one mile radius zone from your location.
Normal plants within the area become twice as productive as normal for the next year.
This ritual does not stack with itself.
If the \ritual{infertility} ritual is also applied to the same area, the most recently performed ritual takes precedence.

This ritual takes 24 hours to perform, and requires 9 action points from its participants.

\end{ability}




\parhead{Schools} Transmutation

\parhead{Spell Lists} Nature
\end{spellsection}


\begin{spellsection}{Gate}[9]


\begin{ability}{Gate}[\glossterm{Planar}, \glossterm{Sustain} (standard), \glossterm{Teleportation}]

Choose a plane that connects to your current plane, and a location within that plane.
This ritual creates an interdimensional connection between your current plane and the location you choose, allowing travel between those two planes in either direction.
The gate takes the form of an \areasmall radius circular disk, oriented a direction you choose (typically vertical).
It is a two-dimensional window looking into the plane you specified when casting the spell, and anyone or anything that moves through it is shunted instantly to the other location.
The gate cannot be \glossterm{sustained} for more than 5 rounds, and is automatically dismissed at the end of that time.

You must specify the gate's destination with a precise mental image of its appearance.
The image does not have to be perfect, but it must unambiguously identify the location.
Incomplete or incorrect mental images may result in the ritual leading to an unintended destination within the same plane, or simply failing entirely.

% TODO: Is this planar cosmology correct?
The Astral Plane connects to every plane, but transit from other planes is usually more limited.
From the Material Plane, you can only reach the Astral Plane.

This ritual takes one week to perform, and requires 81 action points from its participants.

\end{ability}




\parhead{Schools} Conjuration

\parhead{Spell Lists} Arcane, Divine, Nature
\end{spellsection}


\begin{spellsection}{Gentle Repose}[2]


\begin{ability}{Gentle Repose}[\glossterm{Attune} (multiple), \glossterm{Temporal}]

Choose an unattended, nonmagical object within \rngclose range.
Time does not pass for the target, preventing it from decaying or spoiling.
This can extend the time a poison or similar item lasts before becoming inert.
If used on a corpse, this effectively extends the time limit on raising that creature from the dead (see \ritual{resurrection}) and similar effects that require a fresh body.
Additionally, this can make transporting a fallen comrade more pleasant.

This ritual takes one minute to perform.

\end{ability}




\parhead{Schools} Transmutation

\parhead{Spell Lists} Arcane, Divine, Nature
\end{spellsection}


\begin{spellsection}{Infertility}[3]


\begin{ability}{Fertility}

This ritual creates an area of death and decay in a one mile radius zone from your location.
Normal plants within the area become half as productive as normal for the next year.
This ritual does not stack with itself.
If the \ritual{fertility} ritual is also applied to the same area, the most recently performed ritual takes precedence.

This ritual takes 24 hours to perform, and requires 9 action points from its participants.

\end{ability}




\parhead{Schools} Transmutation

\parhead{Spell Lists} Nature
\end{spellsection}


\begin{spellsection}{Ironwood}[3]


\begin{ability}{Ironwood}[\glossterm{Shaping}]

Choose a Small or smaller unattended, nonmagical wooden object within \rngclose range.
The target is transformed into ironwood.
While remaining natural wood in almost every way, ironwood is as strong, heavy, and resistant to fire as iron.
Metallic armor and weapons, such as full plate, can be crafted from ironwood.

% Should this have an action point cost? May be too rare...
This ritual takes 24 hours to perform.

\end{ability}




\parhead{Schools} Transmutation

\parhead{Spell Lists} Nature
\end{spellsection}


\begin{spellsection}{Lifeweb Transit}[5]


\begin{ability}{Lifeweb Transit}[\glossterm{Teleportation}]

This ritual functions like the \ritual{overland teleporation} ritual, except that both the starting and ending points must be living plants.
Both plants must be larger than the largest creature being teleported in this way.

\end{ability}




\parhead{Schools} Conjuration

\parhead{Spell Lists} Nature
\end{spellsection}


\begin{spellsection}{Light}[1]


\begin{ability}{Light}[\glossterm{Attune} (multiple), \glossterm{Figment}, \glossterm{Light}]

Choose a Medium or smaller willing creature or unattended object within \rngclose range.
The target glows like a torch, shedding bright light in an \areamed radius (and dim light for an additional 20 feet).

This ritual takes one minute to perform.

\end{ability}




\parhead{Schools} Illusion

\parhead{Spell Lists} Arcane, Divine, Nature
\end{spellsection}


\begin{spellsection}{Magic Mouth}[1]


\begin{ability}{Magic Mouth}[\glossterm{Attune} (multiple), \glossterm{Figment}]

Choose a Large or smaller willing creature or unattended object within \rngclose range.
In addition, choose a triggering condition and a message of twenty-five words or less.
The condition must be something that a typical human in the target's place could detect.

When the triggering condition occurs, the target appears to grow a magically animated mouth.
The mouth speaks the chosen message aloud.
After the message is spoken, this effect is \glossterm{dismissed}.

This ritual takes 24 hours to perform.

\end{ability}




\parhead{Schools} Illusion

\parhead{Spell Lists} Arcane
\end{spellsection}


\begin{spellsection}{Mount}[3]


\begin{ability}{Mount}[\glossterm{Attune} (multiple), \glossterm{Manifestation}]

This ritual creates your choice of a light horse or a pony to serve as a mount.
The creature appears in an unoccupied location within \rngclose range.
It comes with a bit and bridle and a riding saddle, and will readily accept any creature as a rider.

\end{ability}




\parhead{Schools} Conjuration

\parhead{Spell Lists} Arcane
\end{spellsection}


\begin{spellsection}{Mystic Lock}[2]


\begin{ability}{Mystic Lock}[\glossterm{Attune} (multiple)]

Choose a Large or smaller closable, nonmagical object within \rngclose range, such as a door or box.
The target object becomes magically locked.
It can be unlocked with a Devices check against a DR equal to 20 \add your spellpower.
The DR to break it open forcibly increases by 10.

You can freely pass your own \ritual{arcane lock} as if the object were not locked.
This effect lasts as long as you \glossterm{attune} to it.
If you use this ability multiple times, you can attune to it each time.

This ritual takes one minute to perform.

\end{ability}




\parhead{Schools} Transmutation

\parhead{Spell Lists} Arcane, Divine, Nature
\end{spellsection}


\subsubsection{Subrituals}


\begin{ability}[\nth{5}]{Resilient}
This subritual functions like the \ritual{mystic lock} ritual, except that the DR to unlock the target with a Devices check is instead equal to 30 + your spellpower.
In addition, the DR to break it open increases by 20 instead of by 10.
\end{ability}
\vspace{0.25em}


\begin{spellsection}{Overland Teleportation}[5]


\begin{ability}{Overland Teleportation}[\glossterm{Teleportation}]

Choose up to five willing, Medium or smaller ritual participants.
Choose a destination up to 100 miles away from you on your current plane.
Each target is teleproted to the chosen destination.

You must specify the destination with a precise mental image of its appearance.
The image does not have to be perfect, but it must unambiguously identify the destination.
If you specify its appearance incorrectly, or if the area has changed its appearance, the destination may be a different area than you intended.
The new destination will be one that more closely resembles your mental image.
If no such area exists, the ritual simply fails.
% TODO: does this need more clarity about what teleportation works?

This ritual takes 24 hours to perform and requires 25 action points from its ritual participants.

\end{ability}




\parhead{Schools} Conjuration

\parhead{Spell Lists} Arcane
\end{spellsection}


\begin{spellsection}{Plane Shift}[5]


\begin{ability}{Plane Shift}[\glossterm{Planar}, \glossterm{Teleportation}]

Choose up to five Medium or smaller willing ritual participants.
In addition, choose a plane that connects to your current plane and a location within that plane.
The targets teleport to a random location on that plane 1d100 miles away from the intended destination.

You must specify the destination with a precise mental image of its appearance.
The image does not have to be perfect, but it must unambiguously identify the location.
Incomplete or incorrect mental images may result in the ritual leading to an unintended destination within the same plane, or simply failing entirely.

% TODO: Is this planar cosmology correct?
The Astral Plane connects to every plane, but transit from other planes is usually more limited.
From the Material Plane, you can only reach the Astral Plane.

This ritual takes 24 hours to perform, and requires 25 action points from its participants.

\end{ability}




\parhead{Schools} Conjuration

\parhead{Spell Lists} Arcane, Divine, Nature
\end{spellsection}


\subsubsection{Subrituals}


\begin{ability}[\nth{8}]{Precise}
This subritual functions like the \ritual{plane shift} ritual, except that the actual destination is the same as the intended destination, rather than being a random distance away.
This ritual takes 24 hours to perform, and requires 64 action points from its participants.
\end{ability}
\vspace{0.25em}


\begin{spellsection}{Private Sanctum}[5]


\begin{ability}{Private Sanctum}[\glossterm{Mystic}]

This ritual creates a ward against any external perception in an \arealarge radius zone centered on your location.
This effect is permanent.
Everything in the area is completely imperceptible from outside the area.
Anyone observing the area from outside sees only a dark, silent void, regardless of darkvision and similar abilities.
In addition, all \glossterm{Scrying} effects fail to function in the area.
Creatures inside the area can see within the area and outside of it without any difficulty.

This ritual takes 24 hours to perform, and requires 25 action points from its participants.

\end{ability}




\parhead{Schools} Abjuration

\parhead{Spell Lists} Arcane
\end{spellsection}


\begin{spellsection}{Purge Curse}[3]


\begin{ability}{Purge Curse}[\glossterm{Mystic}]

Choose a willing creature within \rngclose range.
All curses affecting the target are removed.
This ritual cannot remove a curse that is part of the effect of an item the target has equipped.
However, it can allow the target to remove any cursed items it has equipped.

This ritual takes 24 hours to perform, and requires 9 action points from its participants.

\end{ability}




\parhead{Schools} Abjuration

\parhead{Spell Lists} Arcane, Divine, Nature
\end{spellsection}


\begin{spellsection}{Purify Sustenance}[1]


\begin{ability}{Purify Sustenance}[\glossterm{Shaping}]

All food and water in a single square within \rngclose range is purified.
Spoiled, rotten, poisonous, or otherwise contaminated food and water becomes pure and suitable for eating and drinking.
This does not prevent subsequent natural decay or spoiling.

This ritual takes one hour to perform.

\end{ability}




\parhead{Schools} Transmutation

\parhead{Spell Lists} Arcane, Divine, Nature
\end{spellsection}


\begin{spellsection}{Read Magic}[1]


\begin{ability}{Read Magic}[\glossterm{Attune}]

You gain the ability to decipher magical inscriptions that would otherwise be unintelligible.
This can allow you to read ritual books and similar objects created by other creatures.
After you have read an inscription in this way, you are able to read that particular writing without the use of this ritual.

This ritual takes one minute to perform.

\end{ability}




\parhead{Schools} Divination

\parhead{Spell Lists} Arcane, Divine, Nature
\end{spellsection}


\begin{spellsection}{Regeneration}[4]


\begin{ability}{Regeneration}[\glossterm{Flesh}]

Choose a willing creature within \rngclose range.
All of the target's hit points, \glossterm{subdual damage}, and \glossterm{vital damage} are healed.
In addition, any of the target's severed body parts or missing organs grow back by the end of the next round.

This ritual takes 24 hours to perform, and requires 16 action points from its participants.

\end{ability}




\parhead{Schools} Vivimancy

\parhead{Spell Lists} Divine, Nature
\end{spellsection}


\begin{spellsection}{Reincarnation}[5]


\begin{ability}{Reincarnation}[\glossterm{Creation}, \glossterm{Flesh}, \glossterm{Life}]

Choose one Diminuitive or larger piece of a humanoid corpse.
The target must have been part of the original creature's body at the time of death.
The creature the target corpse belongs to returns to life in a new body.
It must not have died due to old age.

This ritual creates an entirely new body for the creature's soul to inhabit from the natural elements at hand.
During the ritual, the body ages to match the age of the original creature at the time it died.
The creature has 0 hit points when it returns to life.

A reincarnated creature is identical to the original creature in all respects, except for its race.
The creature's race is replaced with a random race from \tref{Humanoid Reincarnations}.
Its appearance changes as necessary to match its new race, though it retains the general shape and distinguishing features of its original appearance.
The creature loses all attribute modifiers and abilities from its old race, and gains those of its new race.
However, its racial bonus feat and languages are unchanged.

A creature's soul naturally rejects being placed into a different body than its original home.
Until the creature is restored to its initial race, its maximum action points are reduced by 1.
This penalty does not stack if the creature is reincarnated multiple times.

Coming back from the dead is an ordeal.
All of the creature's action points and other daily abilities are expended when it returns to life.
In addition, its maximum action points are reduced by 1.
This penalty lasts for thirty days, or until the creature gains a level.
If this would reduce a creature's maximum action points below 0, the creature cannot be resurrected.

This ritual takes 24 hours to perform, and requires 25 action points from its participants.

\end{ability}




\parhead{Schools} Conjuration, Vivimancy

\parhead{Spell Lists} Nature
\end{spellsection}


\subsubsection{Subrituals}


\begin{ability}[\nth{7}]{Fated}
This subritual functions like the \ritual{reincarnation} ritual, except that the target is reincarnated as its original race instead of as a random race.
This ritual takes 24 hours to perform, and requires 49 action points from its participants.
\end{ability}
\vspace{0.25em}


\begin{dtable}
\lcaption{Humanoid Reincarnations}
\begin{dtabularx}{\columnwidth}{l X}
d\% & Incarnation \\
\bottomrule
01 & Bugbear \\
02-\minus13 & Dwarf \\
14-\minus25 & Elf \\
26 & Gnoll \\
27-\minus38 & Gnome \\
39-\minus42 & Goblin \\
43-\minus52 & Half-elf \\
53-\minus62 & Half-orc \\
63-\minus74 & Halfling \\
75-\minus89 & Human \\
90-\minus93 & Kobold \\
94 & Lizardfolk \\
95-\minus99 & Orc \\
100 & Other
\end{dtabularx}
\end{dtable}


\begin{spellsection}{Resurrection}[4]


\begin{ability}{Resurrection}[\glossterm{Flesh}, \glossterm{Life}]

Choose one intact humanoid corpse within \rngclose range.
The target returns to life.
It must not have died due to old age.

The creature has 0 hit points when it returns to life.
It is cured of all \glossterm{vital damage} and other negative effects, but the body's shape is unchanged.
Any missing or irreparably damaged limbs or organs remain missing or damaged.
The creature may therefore die shortly after being resurrected if its body is excessively damaged.

Coming back from the dead is an ordeal.
All of the creature's action points and other daily abilities are expended when it returns to life.
In addition, its maximum action points are reduced by 1.
This penalty lasts for thirty days, or until the creature gains a level.
If this would reduce a creature's maximum action points below 0, the creature cannot be resurrected.

This ritual takes 24 hours to perform, and requires 16 action points from its participants.

\end{ability}




\parhead{Schools} Conjuration, Vivimancy

\parhead{Spell Lists} Nature
\end{spellsection}


\subsubsection{Subrituals}


\begin{ability}[\nth{7}]{Complete}
This subritual functions like the \ritual{resurrection} ritual, except that it does not have to target a fully intact corpse.
Instead, it targets a Diminuitive or larger piece of a humanoid corpse.
The target must have been part of the original creature's body at the time of death.
The resurrected creature's body is fully restored to its healthy state before dying, including regenerating all missing or damaged body parts.

This ritual takes 24 hours to perform, and requires 49 action points from its participants.
\end{ability}
\vspace{0.25em}


\begin{spellsection}{Retrieve Legacy}[4]


\begin{ability}{Retrieve Legacy}[\glossterm{Teleportation}]

Choose a willing creature within \rngclose range.
If the target's \glossterm{legacy item} is on the same plane and \glossterm{unattended}, it is teleported into the target's hand.

This ritual takes 24 hours to perform, and requires 16 action points from its ritual participants.

\end{ability}




\parhead{Schools} Conjuration, Divination

\parhead{Spell Lists} Arcane, Divine, Nature
\end{spellsection}


\begin{spellsection}{Scryward}[3]


\begin{ability}{Scryward}[\glossterm{Mystic}]

This ritual creates a ward against scrying in an \arealarge radius zone centered on your location.
All \glossterm{Scrying} effects fail to function in the area.
This effect is permanent.

This ritual takes 24 hour to perform, and requires 9 action points from its participants.

\end{ability}




\parhead{Schools} Abjuration

\parhead{Spell Lists} Arcane, Divine, Nature
\end{spellsection}


\begin{spellsection}{Seek Legacy}[2]


\begin{ability}{Seek Legacy}

Choose a willing creature within \rngclose range.
The target learns the precise distance and direction to their \glossterm{legacy item}, if it is on the same plane.

This ritual takes 24 hours to perform.

\end{ability}




\parhead{Schools} Divination

\parhead{Spell Lists} Arcane, Divine, Nature
\end{spellsection}


\begin{spellsection}{Sending}[4]


\begin{ability}{Sending}[\glossterm{Sustain} (standard)]

Choose a creature on the same plane as you.
You do not need \glossterm{line of sight} or \glossterm{line of effect} to the target.
However,  must specify your target with a precise mental image of its appearance.
The image does not have to be perfect, but it must unambiguously identify the target.
If you specify its appearance incorrectly, or if the target has changed its appearance, you may accidentally target a different creature, or the spell may simply fail.

You send the target a short verbal message.
The message must be twenty-five words or less, and speaking the message must not take longer than five rounds.

After the the target receives the message, it may reply with a message of the same length as long as the ritual's effect continues.
Once it speaks twenty-five words, or you stop sustaining the effect, the ritual is \glossterm{dismissed}.

This ritual takes one hour to perform.

\end{ability}




\parhead{Schools} Divination

\parhead{Spell Lists} Arcane, Divine, Nature
\end{spellsection}


\subsubsection{Subrituals}


\begin{ability}[\nth{6}]{Interplanar}
This subritual functions like the \ritual{sending} ritual, except that the target does not have to be on the same plane as you.
It gains the \glossterm{Planar} tag in addition to the tags from the \ritual{sending} ritual.
\end{ability}
\vspace{0.25em}


\begin{spellsection}{Soul Bind}[8]


\begin{ability}{Soul Bind}[\glossterm{Life}]

Choose one intact corpse within \rngclose range.
In addition, choose a gem you hold that is worth at least 1,000 gp.
The soul of the creature that the target corpse belongs to is imprisoned in the chosen gem.
This prevents the creature from being resurrected, and prevents the corpse from being used to create undead creatures, as long as the gem is intact.
A creature holding the gem may still resurrect or reanimate the creature.

This ritual takes one hour to perform.

\end{ability}




\parhead{Schools} Vivimancy

\parhead{Spell Lists} Arcane, Divine
\end{spellsection}


\begin{spellsection}{Telepathic Bond}[4]


\begin{ability}{Telepathic Bond}[\glossterm{Attune} (shared)]

Choose up to five willing ritual participants.
Each target can communicate mentally through telepathy with each other target.
This communication is instantaneous across any distance, but cannot reach across planes.

% Is this grammatically correct?
This effect lasts as long as you and each target \glossterm{attune} to it.
Each target must attune to this ritual independently.
If a target breaks its attunement, it stops being able to send and receive mental messages with other targets.
However, the effect continues as long as you attune to it.
If you stop attuning to it, the ritual is \glossterm{dismissed} as usual.

This ritual takes 24 hours to perform.

\end{ability}




\parhead{Schools} Divination

\parhead{Spell Lists} Arcane
\end{spellsection}


\subsubsection{Subrituals}


\begin{ability}[\nth{8}]{Interplanar}
This subritual functions like the \ritual{telepathic bond} ritual, except that each target can communicate telepathically even across different planes.
It gains the \glossterm{Planar} tag in addition to the tags from the \ritual{telepathic bond} ritual.
\end{ability}
\vspace{0.25em}


\begin{spellsection}{Water Breathing}[2]


\begin{ability}{Water Breathing}[\glossterm{Attune} (multiple)]

Choose a Medium or smaller willing creature within \rngclose range.
The target can breathe water as easily as a human breathes air, preventing it from drowning or suffocating underwater.
This effect lasts as long as you \glossterm{attune} to it.
If you use this ability multiple times, you can attune to it each time.

This ritual takes one minute to perform.

\end{ability}




\parhead{Schools} Transmutation

\parhead{Spell Lists} Arcane, Divine, Nature
\end{spellsection}

