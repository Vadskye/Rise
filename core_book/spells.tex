\chapter{Spells and Rituals}\label{Spells}

\section{Spell and Ritual Mechanics}\label{Spell and Ritual Mechanics}

    Spells and rituals share many common properties, defined here.

    \subsection{Categories of Magic}
        There are three \glossterm{magic sources} that spells can come from: arcane (cast by mages and warlocks), divine (cast by clerics and paladins), and nature (cast by druids).
        In addition, there are nine schools of magic.
        Each school of magic has a set of thematically related effects.
        Every spell and ritual belongs to at least one school of magic.
        The schools of magic are described below.

        \subsubsection{Abjuration}
            Abjuration spells and rituals reduce or negate damage, magic, and other effects.
            They can be used to protect allies and remove harmful magic.

        \subsubsection{Channeling}
            Channeling spells and rituals call upon the power of deities or other supernatural entities.
            They can be used to do anything those entities could do.
            Arcane spellcasters do not have access to Channeling spells.

        \subsubsection{Conjuration}
            Conjuration spells and rituals create and transport objects and creatures.
            They can be used to summon allies, transport creatures, and create objects from thin air.

        \subsubsection{Divination}
            Divination spells and rituals grant knowledge.
            They can be used to reveal hidden truths, predict the future, or communicate at great distances.

        \subsubsection{Enchantment}
            Enchantment spells and rituals alter the minds of creatures.
            They can be used to influence, control, or debilitate creatures.
            Almost all enchantment spells and rituals have the \glossterm{Mind} tag, and many have the \glossterm{Subtle} as well (see \pcref{Ability Tags}).

        \subsubsection{Evocation}
            Evocation spells and rituals create and manipulate energy.
            They can be used to inflict damage with energy blasts or manipulate the environment.

        \subsubsection{Illusion}
            Illusion spells and rituals create or manipulate sensory impressions.
            They can be used to create or remove light, conceal things that exist, or cause creatures to perceive things that do not exist.

        \subsubsection{Transmutation}
            Transmutation spells and rituals change the properties of creatures and objects.
            They can be used to grant new abilities, enhance existing abilities, change a target's form, or even alter the flow of time itself.

        \subsubsection{Vivimancy}
            Vivimancy spells and rituals manipulate the power of life and death, as well as souls.
            They can be used to heal or inflict wounds, resurrect the dead, create undead monsters, and cripple the bodies of creatures.

    \subsection{Casting Components}\label{Casting Components}
        Unless otherwise noted, all spells and rituals require both verbal and somatic components to cast or perform.
        You cannot start casting a spell or performing a ritual without all required components.
        If you lose those components before the ability resolves, it is \glossterm{miscast}.

        To provide the verbal component for a spell or ritual, you must speak in a strong voice with a volume at least as loud as ordinary conversation.
        To provide the somatic component for a spell or ritual, you must make a measured and precise movement of at least one free hand.

    \subsection{Augments}\label{Augments}
        There are a number of \glossterm{augments} that can be applied to spells and rituals to increase their power.
        Each augment has a name, a level, and an effect.
        Whenever you cast a spell or perform a ritual, can choose to apply any number of augments you know to the spell or ritual.
        For each augment you apply, you increase the spell or ritual's level by an amount equal to the augment's level.
        In exchange, the ability gains the effects of that augment.
        If an augment would increase the spell or ritual's level beyond the maximum level you can cast, you cannot apply the augment to that ability.

        \parhead{Augments and Cantrips}
        You can apply augments to \glossterm{cantrips}.
        Cantrips are considered to be 0th level spells for this purpose.

        \parhead{Augments and Subspells}
        If a spell or ritual changes its properties with a subspell or subritual, it may become eligible for different augments.
        % TODO: fix example
        For example, if you apply the Fireball subspell to the \spell{pyromancy} spell, it changes to affect an area.
        You would then be able to apply the Widened augment to increase its area.

        \subsubsection{Augment Descriptions}\label{Augment Descriptions}

            \augment{1}{Silent} You do not need to use \glossterm{verbal components} to cast the spell.
            \par This augment can be applied to any spell.

            \augment{1}{Stilled} You do not need to use \glossterm{somatic components} to cast the spell.
            \par This augment can be applied to any spell.

            \augment{2}{Cryptic} The spell's visual effects and magical aura changes to mimic a different spell of your choice.
            You may choose any combination of spell or \glossterm{subspell} you know, along with any other augments, that result in a spell of the same level or lower as the spell you are casting.
            This affects inspection of the spell itself by any means, such as with the Spellcraft skill (see \pcref{Spellcraft}).
            However, it does not alter the mechanical effects of the spell in any way.
            If the spell's effects depend on visual components, the spell may fail to work if you alter the spell's visuals too much. 

            \augment{2}{Extended} The ability's range increases by one step, to a maximum of \rngext.
            The steps are, in order: \rngtouch, \rngclose, \rngmed, \rnglong, and \rngext.
            This augment can be applied multiple times.
            Each time, the ability's range increases by an additional step.
            \par This augment can be applied to any spell or ritual with a range that is one of the above ranges.

            \augment{2}{Giant} The ability can affect a target one size category larger.
            This augment can be applied multiple times.
            Its effects stack.
            \par This augment can be applied to any spell or ritual that that has a maximum size category of targets that it can affect.

            \augment{2}{Quickened} You can cast the spell as a \glossterm{minor action}.
            In exchange, you cannot take any actions during the \glossterm{action phase} or \glossterm{delayed action phase} of the next round.
            \par This augment can be applied to any spell.

            \augment{2}{Selective} You may freely exclude any areas from the spell's effect.
            However, all squares in the final area of the spell must be contiguous.
            You cannot create split a spell's area into multiple completely separate areas.
            \par This augment can be applied to any spell or ritual that affects an area.

            \augment{2}{Widened} The ability's area increases by one step, to a maximum of \areahuge.
            The steps are, in order: \areasmall, \areamed, \arealarge, and \areahuge.
            Normally, a Small or Medium line is 5 ft.\ wide, while a Large or Huge line is 10 ft.\ wide.
            A line used to define a wall does not have a width.
            This augment can be applied multiple times.
            Each time, the ability's area increases by an additional step.
            \par This augment can be applied to any spell or ritual with an area that is one of the above areas.

            \augment{3}{Dual} The spell targets an additional creature within range.
            This augment can be applied multiple times.
            Its effects stack.
            \par This augment can be applied to any spell that has a range and affects a single target of the caster's choice.
            It cannot be applied to spells that affect a single specific target, such as the caster.
            In addition, it cannot be applied to \glossterm{Attune} spells.

            \augment{3}{Intensified} The damage inflicted and healing granted by the ability increases by \plus1d.
            This augment can be applied multiple times.
            Its effects stack.
            \par This augment can be applied to any spell or ritual that deals damage or grants healing based on a dice pool.

            \augment{3}{Phasing} When determining whether you have \glossterm{line of sight} and \glossterm{line of effect} to a particular location with the spell, you can ignore a single solid obstacle up to five feet thick.
            This can allow you to cast spells through solid walls, though it does not grant you the ability to see through the wall.
            \par This augment can be applied to any spell with a range.

            \augment{3}{Precise} You gain a \plus1 bonus to accuracy with the ability.
            This augment can be applied multiple times.
            Its effects stack.
            \par This augment can be applied to any spell or ritual that has an attack roll.

            \augment{4}{Accelerated} The ritual takes half the normal amount of time to perform.
            \par This augment can be applied to any ritual.

            \augment{6}{Echoing} During the \glossterm{delayed action phase} of the next round, the spell's effect occurs again.
            All choices you made for the original casting of the spell are made identically for the repeat casting.
            It affects the same area, targets, and so on.

            If the spell is now invalid, such as if all of its targets are out of range, the additional casting has no effect.
            This augment does not allow you to \glossterm{attune} to the same spell more than once.

    \subsection{Dismissal}
        As a \glossterm{minor action}, you can dismiss any spells or rituals you used that have lasting effects.
        This requires the same casting components (verbal and somatic) as casting the spell or performing the ritual normally.
        Spells and rituals can also be dismissed in other ways, such as after their effects have finished.
        When a spell or ability is dismissed, all of its lingering effects immediately end.
        % Should a spell be dismissed for all targets simultaneously, or per target?

    \subsection{Spellpower}

        The \glossterm{power} of your spells and rituals is determined by your \glossterm{spellpower}.
        Normally, your spellpower is equal to your level or an \glossterm{attribute}, whichever is higher.

        \parhead{Multiple Spellcasting Abilities} If you have the ability to cast spells from more than one separate ability, use the spellpower appropriate to the ability that you are casting the spell with.

        \parhead{Reducing Spellpower} You can voluntarily reduce the \glossterm{power} of the spells you cast by using a lower spellpower.
        However, you cannot use a spellpower lower than the minimum spellpower required to cast the spell, which is equal to twice the spell's level.

    \subsection{Resurrecting the Dead}\label{Resurrecting the Dead}
        Several rituals have the power to restore dead characters to life.

        When a living creature dies, its soul departs its body, travels through the Astral Plane, and goes to abide on the plane where the creature's deity resides.
        If the creature did not worship a deity, its soul departs to the plane corresponding to its alignment.
        Bringing a creature back from the dead means retrieving their soul and returning it to their body.

        \subparhead{Preventing Revivification} Enemies can take steps to make it more difficult for a character to be returned from the dead.
        Except for \spell{true resurrection}, every ritual to raise the dead requires a body, so keeping or destroying the body is an effective deterrent.
        The \spell{soul bind} ritual prevents any sort of revivification unless the soul is first released.

        \subparhead{Revivification against One's Will} A soul cannot be returned to life if it does not wish to be.
        A soul infallibly knows the name, alignment, and patron deity (if any) of the character attempting to revive it and may refuse to return on that basis.

\section{Spell Mechanics}\label{Spell Mechanics}

    \subsection{Cantrips}\label{Cantrips}
        Every spell can be cast as a \glossterm{cantrip}.
        You do not have to spend an action point to cast a spell as a cantrip.
        Cantrips are considered to be 0th level spells for the purpose of abilities that care about spell level, such as \glossterm{augments}.

    \subsection{Miscasting}\label{Miscasting}

        If you start casting a spell and fail to cast the spell successfully, such as if your concentration is broken (see \pcref{Concentration}), you miscast the spell.
        A miscast spell resolves at the same time that the spell normally would, but it does not have its normal effect.
        Instead, a wave of magical energy causes a \glossterm{miscast backlash}.
        When a mystic backlash occurs, make an attack against the Mental defense of yourself and all creatures in a 5 foot radius from you.
        Your \glossterm{power} with this ability is equal to your \glossterm{spellpower} with the spell you tried to cast.
        % damage type?
        On a hit, each target takes \glossterm{standard damage} \minus1d.

        \parhead{Impossible Spells} Whenever you try to cast a spell in an impossible way, the spell is \glossterm{miscast} instead.
        This most commonly happens if you attempt to declare an invalid target for the spell.
        For example, if you try to cast a spell that only affects living creatures on a creature that is undead, the spell would be miscast.

        \parhead{Voluntary Miscasting} At the start of each phase while you are casting a spell, you can choose to stop casting the spell, causing you to \glossterm{miscast} it instead.

    \subsection{Delayed Resolution}\label{Delayed Resolution}
        Every spell has the \glossterm{Delayed} tag, causing it to resolve later than other actions (see \pcref{Delayed Actions}).

    \subsection{Concentration}\label{Concentration}
        Every spell requires concentration to cast.
        If you are distracted while casting a spell, such as if you are riding a galloping horse, you need to make a \glossterm{concentration} check to avoid losing concentration on the spell.
        In addition, if the spell has the \glossterm{Delayed} tag, as all spells normally do, you must maintain this concentration for the entire \glossterm{action phase} before the spell resolves.
        If you are damaged during that phase, you also need to make a \glossterm{concentration} check to avoid losing the spell.
        If your concentration is broken while casting a spell, you \glossterm{miscast} that spell (see \pcref{Miscasting}).

        Your bonus on concentration \glossterm{checks} is normally equal to your level or your Willpower, whichever is higher.
        You apply any \glossterm{overwhelm penalties} you suffer to your concentration checks.
        If you take damage while also being distracted by some other effect, you take a \minus2 penalty to the concentration check.
        The \glossterm{difficulty rating} of the check is equal to 5 \add twice the level of the spell you are casting.

        The amount of damage you take during the action phase does not directly affect your concentration check.
        % TODO: timing of being staggered
        However, if you become \glossterm{staggered}, you take a \minus4 penalty to checks, which affects your concentration check.

    \subsection{Subspells}\label{Subspells}
        % TODO: some of this wording is weird
        All spells have a number of \glossterm{subspells}.
        Each subspell has a name, a level, and an effect.
        Whenever you cast a spell, you can choose to apply a single subspell you know from that spell.
        If you do, the spell's level becomes equal to the subspell's level.
        In exchange, the spell gains the effects of the subspell.
        You cannot learn or cast subspells whose spell level exceeds your maximum spell level.

        Some subspells simply add additional properties to a spell's normal effect.
        Others change the targets or effects of the spell significantly.
        After choosing whether to cast a subspell, you can apply any number of \glossterm{augments}, described below.

\section{Ritual Mechanics}
    Rituals are ceremonies that create magical effects.
    Some spellcasting characters can learn and perform rituals.
    You don't memorize a ritual as you would a normal spell or subspell.
    Rituals are too complex for all but the most knowledgeable sages to commit to memory.
    To perform a ritual, you need to read from a book or a scroll containing it.
    Rituals are similar to spells, but abilities that affect spells do not affect rituals unless they say they do in their descriptions.

    \subsection{Ritual Descriptions}
        % TODO: proper chapter references
        Rituals are described in Chapter 13.
        The description of each ritual follows the same format as the description of spells in Chapter 12, except that every ritual has a level, like \glossterm{subspells} do.

        \subparhead{Ritual Sources}
        A ritual always matches the magic source of the person performing the ritual.
        For example, \spell{scrying} is an arcane ritual when performed by a wizard, but a divine ritual when performed by a cleric.

    \subsection{Ritual Requirements}
        In order to learn and perform a ritual, you must be able to cast at least one spell of the same level as the ritual.

    \subsection{Ritual Books}
        A ritual book contains one or more rituals that you can use as frequently as you want, as long as you can spend the time and \glossterm{action point} to perform the ritual.
        Scribing a ritual in a ritual book costs an amount of precious inks.

    \subsection{Ritual Costs}\label{Ritual Costs}
        The costs to scribe rituals are described on \trefnp{Ritual Costs}.
        \begin{dtable}
            \lcaption{Ritual Costs}
            \begin{dtabularx}{\columnwidth}{X l l}
                \tb{Ritual Level} & \tb{Cost to Scribe} & \tb{Item Level} \\
                \bottomrule
                1st-Level & 50 gp & 1st \\
                2nd-Level & 200 gp & 3rd \\
                3rd-Level & 500 gp & 4th \\
                4th-Level & 1,250 gp & 7th \\
                5th-Level & 3,000 gp & 9th \\
                6th-Level & 7,500 gp & 11th \\
                7th-Level & 15,000 gp & 12th \\
                8th-Level & 35,000 gp & 14th \\
                9th-Level & 75,000 gp & 16th \\
            \end{dtabularx}
        \end{dtable}

    \subsection{Subrituals}\label{Subrituals}
        Many rituals have \glossterm{subrituals}, just like many spells have \glossterm{subspells} (see \pcref{Subspells}).
        Subrituals work in the same way as subspells, except that they are applied to rituals instead of spells.

    \subsection{Performing Rituals}
        To perform a ritual, you must have a ritual book containing the ritual and the material components required for the ritual.
        Unless otherwise specified, performing a ritual requires spending a single \glossterm{action point}.
        Some rituals require multiple action points to complete.
        Other creatures can supply action points to help you perform rituals; see Ritual Participants, below.

        If you are distracted during the ritual, you must make a Concentration check, just as if you were casting a spell of the ritual's level.
        If you fail, the ritual is ruined and you must start from the beginning.
        % TODO: can we remove the ``casting a ritual'' wording?
        \par Performing a ritual and casting a ritual mean the same thing.

        \subsubsection{Ritual Participants}
            Creatures can assist in the performance of rituals even if they are unable to perform rituals themselves.
            A creature that helps perform a ritual is called a ritual participant, and the creature performing the ritual is called the ritual leader.
            A ritual participant may spend an action point in place of or in addition to the action point spent by the creature performing the ritual.
            It may also \glossterm{attune} to the effect of the ritual in place of the creature performing the ritual.
            Only one creature may attune to the ritual's effect in this way.
            If multiple creatures are willing to spend action points or attune to effects, the ritual leader decides which creatures spend action points or attune to the ritual's effects.

            The steps required to participate in rituals can be complex.
            Ritual participants must be given specific instructions for the actions they must perform during a ritual by a creature who knows how to perform the ritual.
            This instruction generally takes half the time required to perform the ritual.
            A creature cannot participate in rituals unless it has an Intelligence of at least 0, can speak at least one language, and has the fine motor control required to perform the somatic components of spells.

            Normally, a ritual participant can only contribute one action point.
            If the participant can cast spells from the same source as the ritual, they can contribute any number of action points.

            \parhead{Changing Ritual Participation}
            Rituals are deeply complex magic, and they cannot be abandoned or paused partway through.
            If the number of ritual participants in a ritual decreases below its initial value, the ritual fails at the end of the next round if the number of participants is not restored.
            However, ritual participants can transfer their participation to other creatures without disrupting the ritual.

            In order to transfer ritual participation, the new creature must be able to participate in the ritual, and must immediately spend the same number of action points as the creature that it is taking over from.
            Similarly, the ritual leader can transfer their leadership to another creature.
            In addition to the requirements for transferring ritual participation, the new leader must know the ritual and be able to perform it themselves.

            Changing ritual participation and leadership is usually done when performing extraordinarily long or demanding rituals.

    \subsection{Magical Writings}
        To record a spell in written form, a character uses complex notation that describes the magical forces involved in the spell.
        The notation constitutes a universal language that spellcasters have discovered, not invented.
        Each writer uses this universal system regardless of their native language or culture.
        However, each character uses the system in their own way.
        Another person's magical writing remains incomprehensible to even the most powerful spellcaster until they take the time to study and decipher it.

        To decipher an magical writing (such as a single spell in written form on a scroll), you must make a Spellcraft check (DR 10 \add the spell's level). If the skill check fails, you cannot attempt to read that particular spell again until the next day.
        A \ritual{read magic} ritual automatically deciphers a magical writing without a skill check.
        If the person who created the magical writing is on hand to help the reader, success is also automatic.

        Once a character deciphers a particular magical writing, they do not need to decipher it again.
        Deciphering a magical writing allows the reader to identify the spell and gives some idea of its effects (as explained in the spell or ritual description).
        

\section{Spell Lists}

\small
\section{Arcane Magic}\label{Arcane Magic}
\subsection{Arcane Spells}\label{Arcane Spells}
\begin{spelllist}
\spellhead{Agony} TODO
\spellhead{Antimagic} TODO
\spellhead{Boon of Mastery} TODO
\spellhead{Charm Person} TODO
\spellhead{Cone of Cold} TODO
\spellhead{Corruption} TODO
\spellhead{Create Acid} TODO
\spellhead{Distort Image} TODO
\spellhead{Elemental Blade} TODO
\spellhead{Fear} TODO
\spellhead{Fireball} TODO
\spellhead{Flare} TODO
\spellhead{Foresight} TODO
\spellhead{Inertial Shield} TODO
\spellhead{Inflict Wounds} TODO
\spellhead{Lightning Bolt} TODO
\spellhead{Planar Disruption} TODO
\spellhead{Polymorph} TODO
\spellhead{Protection from Alignment} TODO
\spellhead{Scry} TODO
\spellhead{Summon Monster} TODO
\spellhead{Telekinesis} TODO
\spellhead{Web} TODO
\end{spelllist}
\subsection{Arcane Rituals}\label{Arcane Rituals}
\begin{spelllist}
\spellhead[1]{Endure Elements} TODO
\spellhead[1]{Light} TODO
\spellhead[1]{Magic Mouth} TODO
\spellhead[1]{Purify Sustenance} TODO
\spellhead[1]{Read Magic} TODO
\spellhead[2]{Gentle Repose} TODO
\spellhead[2]{Mystic Lock} TODO
\spellhead[2]{Water Breathing} TODO
\spellhead[3]{Animate Dead} TODO
\spellhead[3]{Binding} TODO
\spellhead[3]{Create Object} TODO
\spellhead[3]{Create Sustenance} TODO
\spellhead[3]{Explosive Runes} TODO
\spellhead[3]{Mount} TODO
\spellhead[3]{Purge Curse} TODO
\spellhead[3]{Scryward} TODO
\spellhead[4]{Sending} TODO
\spellhead[4]{Telepathic Bond} TODO
\spellhead[5]{Discern Location} TODO
\spellhead[5]{Plane Shift} TODO
\spellhead[5]{Private Sanctum} TODO
\spellhead[6]{Overland Teleportation} TODO
\spellhead[8]{Soul Bind} TODO
\spellhead[9]{Gate} TODO
\end{spelllist}
\small
\section{Divine Magic}\label{Divine Magic}
\subsection{Divine Spells}\label{Divine Spells}
\begin{spelllist}
\spellhead{Agony} TODO
\spellhead{Antimagic} TODO
\spellhead{Barrier} TODO
\spellhead{Bless} TODO
\spellhead{Boon of Mastery} TODO
\spellhead{Corruption} TODO
\spellhead{Cure Wounds} TODO
\spellhead{Flare} TODO
\spellhead{Foresight} TODO
\spellhead{Inflict Wounds} TODO
\spellhead{Planar Disruption} TODO
\spellhead{Poison} TODO
\spellhead{Protection from Alignment} TODO
\spellhead{Scry} TODO
\spellhead{Smite} TODO
\spellhead{Summon Monster} TODO
\spellhead{Word of Faith} TODO
\end{spelllist}
\subsection{Divine Rituals}\label{Divine Rituals}
\begin{spelllist}
\spellhead[1]{Bless Water} TODO
\spellhead[1]{Curse Water} TODO
\spellhead[1]{Endure Elements} TODO
\spellhead[1]{Light} TODO
\spellhead[1]{Purify Sustenance} TODO
\spellhead[1]{Read Magic} TODO
\spellhead[2]{Gentle Repose} TODO
\spellhead[2]{Mystic Lock} TODO
\spellhead[2]{Water Breathing} TODO
\spellhead[3]{Animate Dead} TODO
\spellhead[3]{Binding} TODO
\spellhead[3]{Create Object} TODO
\spellhead[3]{Create Sustenance} TODO
\spellhead[3]{Purge Curse} TODO
\spellhead[3]{Scryward} TODO
\spellhead[4]{Regeneration} TODO
\spellhead[4]{Sending} TODO
\spellhead[5]{Discern Location} TODO
\spellhead[5]{Plane Shift} TODO
\spellhead[8]{Soul Bind} TODO
\spellhead[9]{Gate} TODO
\end{spelllist}
\small
\section{Nature Magic}\label{Nature Magic}
\subsection{Nature Spells}\label{Nature Spells}
\begin{spelllist}
\spellhead{Antimagic} TODO
\spellhead{Barkskin} TODO
\spellhead{Barrier} TODO
\spellhead{Boon of Mastery} TODO
\spellhead{Cone of Cold} TODO
\spellhead{Control Air} TODO
\spellhead{Corruption} TODO
\spellhead{Cure Wounds} TODO
\spellhead{Elemental Blade} TODO
\spellhead{Fireball} TODO
\spellhead{Flare} TODO
\spellhead{Foresight} TODO
\spellhead{Inflict Wounds} TODO
\spellhead{Lightning Bolt} TODO
\spellhead{Poison} TODO
\spellhead{Polymorph} TODO
\spellhead{Scry} TODO
\spellhead{Summon Monster} TODO
\spellhead{Water Mastery} TODO
\spellhead{Web} TODO
\end{spelllist}
\subsection{Nature Rituals}\label{Nature Rituals}
\begin{spelllist}
\spellhead[1]{Endure Elements} TODO
\spellhead[1]{Light} TODO
\spellhead[1]{Purify Sustenance} TODO
\spellhead[1]{Read Magic} TODO
\spellhead[2]{Gentle Repose} TODO
\spellhead[2]{Mystic Lock} TODO
\spellhead[2]{Water Breathing} TODO
\spellhead[3]{Create Object} TODO
\spellhead[3]{Create Sustenance} TODO
\spellhead[3]{Fertility} TODO
\spellhead[3]{Infertility} TODO
\spellhead[3]{Ironwood} TODO
\spellhead[3]{Purge Curse} TODO
\spellhead[3]{Scryward} TODO
\spellhead[4]{Regeneration} TODO
\spellhead[4]{Resurrection} TODO
\spellhead[4]{Sending} TODO
\spellhead[5]{Discern Location} TODO
\spellhead[5]{Plane Shift} TODO
\spellhead[5]{Reincarnation} TODO
\spellhead[6]{Lifeweb Transit} TODO
\spellhead[7]{Awaken} TODO
\spellhead[9]{Gate} TODO
\end{spelllist}


\section{Spell Descriptions}


\begin{spellsection}{Aeromancy}

\begin{spellheader}
\spelldesc{You blast a foe with wind.}
\end{spellheader}


\begin{ability}{Cantrip}
Make an attack vs. Armor against a creature or object within \rngmed range.
\hit The target takes bludgeoning \glossterm{standard damage} \minus1d.
\end{ability}




\parhead{Schools} Transmutation

\parhead{Spell Lists} Nature
\end{spellsection}


\subsubsection{Subspells}


\begin{ability}[\nth{1}]{Propulsion}[\glossterm{Air}, \glossterm{Swift}]
Choose a willing creature in \rngclose range.
You move the target up to 50 feet in any direction.
You cannot change direction partway through the movement.
Moving the target upwards cost twice the normal movement cost.
\end{ability}
\vspace{0.25em}


\begin{ability}[\nth{1}]{Wind Screen}[\glossterm{Air}, \glossterm{Attune} (target), \glossterm{Shielding}]
Choose a willing creature in \rngclose range.
The target gains a \plus1 \glossterm{magic bonus} to Armor defense.
This bonus is increased to \plus5 against ranged \glossterm{physical attacks} from weapons or projectiles that are Small or smaller.

Any effect which increases the size of creature this subspell can affect also increases the size of ranged weapon it defends against by the same amount.
You can cast this subspell as a \glossterm{minor action}.
\end{ability}
\vspace{0.25em}


\begin{ability}[\nth{1}]{Windblade}[\glossterm{Air}, \glossterm{Attune} (target), \glossterm{Shaping}]
Choose a willing creature within \rngclose range.
Melee weapons wielded by the target gain an additional five feet of \glossterm{reach}.
This has no effect on ranged attacks the target makes.

You can cast this subspell as a \glossterm{minor action}.
\end{ability}
\vspace{0.25em}


\begin{ability}[\nth{1}]{Windstrike}[\glossterm{Air}]
Make an attack vs. Armor against a creature or object within \rngmed range.
\hit The target takes bludgeoning \glossterm{standard damage} \plus1d.
\end{ability}
\vspace{0.25em}


\begin{ability}[\nth{2}]{Gentle Descent}
Choose a willing, Large or smaller creature in \rngclose range.
The target gains a 30 foot \glossterm{glide speed} (see \pcref{Gliding}).
\end{ability}
\vspace{0.25em}


\begin{ability}[\nth{2}]{Greater Propulsion}
This subspell functions like the \spell{propulsion} subspell, except that the distance you can move the target is increased to 100 feet.
In addition, the target gains a \plus1d bonus to damage with melee \glossterm{strikes} during the same phase.
\end{ability}
\vspace{0.25em}


\begin{ability}[\nth{2}]{Gust of Wind}[\glossterm{Air}]
Make an attack vs. Armor against everything in a \arealarge, 10 ft. wide line from you.
\hit Each target takes bludgeoning \glossterm{standard damage} \minus1d.
\end{ability}
\vspace{0.25em}


\begin{ability}[\nth{2}]{Stripping Windstrike}
This subspell functions like the \spell{windstrike} subspell, except that the attack result is also compared to the target's Reflex defense.
% Clarify: this can hit even if the damaging effect misses
\hit The target drops all items it is holding that are not well secured (such as a ring) or held in two hands.
\end{ability}
\vspace{0.25em}


\begin{ability}[\nth{3}]{Greater Wind Screen}
This subspell functions like the \spell{wind screen} subspell, except that the defense bonus against ranged attacks increases to \plus10.
\end{ability}
\vspace{0.25em}


\begin{ability}[\nth{3}]{Greater Windblade}[\glossterm{Air}, \glossterm{Attune} (target), \glossterm{Shaping}]
Choose a willing creature within \rngclose range.
Melee weapons wielded by the target gain an additional fifteen feet of \glossterm{reach}.
This has no effect on ranged attacks the target makes.

You can cast this subspell as a \glossterm{minor action}.
\end{ability}
\vspace{0.25em}


\begin{ability}[\nth{3}]{Stormlord}
This subspell functions like the \spell{wind screen} subspell, except that the air also retaliates against creatures that attack the target.
Whenever a creature within \rngclose range of the target attacks it, make an attack vs. Armor against the attacking creature.
A hit deals bludgeoning \glossterm{standard damage} \minus2d.
Any individual creature can only be dealt damage in this way once per round.

Any effect which increases this spell's range increases the range of this retaliation by the same amount.
\end{ability}
\vspace{0.25em}


\begin{ability}[\nth{3}]{Stripping Gust of Wind}
This subspell functions like the \spell{gust of wind} subspell, except that the attack result is also compared to each target's Reflex defense.
\hit Each target drops all items it is holding that are not well secured (such as a ring) or held in two hands.
\end{ability}
\vspace{0.25em}


\begin{ability}[\nth{4}]{Air Walk}
Choose a willing creature in \rngclose range.
The target can walk on air as if it were solid ground.
The magic only affects the target's legs and feet.
By choosing when to treat the air as solid, it can traverse the air with ease.
\end{ability}
\vspace{0.25em}


\begin{ability}[\nth{4}]{Control Weather}[\glossterm{Air}, \glossterm{Attune} (self)]
When you cast this subspell, you choose a new weather pattern.
You can only choose weather which would be possible in the climate and season of the area you are in.
For example, you can normally create a thunderstorm, but not if you are in a desert.

When you complete the spell, the weather begins to take effect in a two mile radius cylinder-shaped zone centered on from your location.
After five minutes, your chosen weather pattern fully takes effect.

You can control the general tendencies of the weather, such as the direction and intensity of the wind.
You cannot control specific applications of the weather -- where lightning strikes, for example, or the exact path of a tornado.
Contradictory weather conditions are not possible simultaneously.

After the spell's effect ends, the weather continues on its natural course, which may cause your chosen weather pattern to end.
% TODO: This should be redundant with generic spell mechanics
If another ability would magically manipulate the weather in the same area, the most recently used ability takes precedence.
\end{ability}
\vspace{0.25em}


\begin{ability}[\nth{4}]{Cyclone}[\glossterm{Air}]
Make an attack vs. Armor against everything in a \areamed radius within \rngmed range.
\hit Each target takes bludgeoning \glossterm{standard damage} \minus1d.
\end{ability}
\vspace{0.25em}


\begin{ability}[\nth{4}]{Supreme Propulsion}
This subspell functions like the \spell{propulsion} subspell, except that the distance you can move the target is increased to 300 feet.
In addition, the target gains a \plus2d bonus to damage with melee \glossterm{strikes} during the same phase.
\end{ability}
\vspace{0.25em}


\begin{ability}[\nth{6}]{Stripping Cyclone}
This subspell functions like the \spell{cyclone} subspell, except that the attack result is also compared to each target's Reflex defense.
\hit Each target drops all items it is holding that are not well secured (such as a ring) or held in two hands.
\end{ability}
\vspace{0.25em}

\newpage
\begin{spellsection}{Aquamancy}

\begin{spellheader}
\spelldesc{You create a wave of water to crush your foes.}
\end{spellheader}


\begin{ability}{Cantrip}[\glossterm{Manifestation}, \glossterm{Water}]
Make an attack vs. Fortitude against everything in a \areamed, 5 ft.\ wide line from you.
\hit Each target takes bludgeoning \glossterm{standard damage} \minus2d.
\end{ability}




\parhead{Schools} Conjuration

\parhead{Spell Lists} Nature
\end{spellsection}


\subsubsection{Subspells}


\begin{ability}[\nth{1}]{Crushing Wave}[\glossterm{Manifestation}, \glossterm{Water}]
Make an attack vs. Fortitude against everything in a \arealarge, 10 ft.\ wide line from you.
\hit Each target takes bludgeoning \glossterm{standard damage} \minus1d.
\end{ability}
\vspace{0.25em}


\begin{ability}[\nth{1}]{Dampen}[\glossterm{Attune} (target)]
Choose up to five willing creatures within \rngclose range.
Each target gains damage reduction against fire damage equal to your \glossterm{power}.

You can cast this subspell as a \glossterm{minor action}.
\end{ability}
\vspace{0.25em}


\begin{ability}[\nth{1}]{Underwater Freedom}[\glossterm{Attune} (target)]
Choose a willing creature within \rngclose range.
The target suffers no penalties for acting underwater, except for those relating to using ranged weapons.

You can cast this subspell as a \glossterm{minor action}.
\end{ability}
\vspace{0.25em}


\begin{ability}[\nth{2}]{Aqueous Blade}[\glossterm{Attune} (target), \glossterm{Shaping}, \glossterm{Water}]
Choose a willing creature within \rngclose range.
Whenever the target makes a \glossterm{strike} with a melee weapon, the attack is made against Reflex defense instead of Armor defense.
However, the target takes a \minus2d penalty to damage with \glossterm{strikes}.

You can cast this subspell as a \glossterm{minor action}.
This subspell is from the Transmutation school instead of the Conjuration school.
\end{ability}
\vspace{0.25em}


\begin{ability}[\nth{2}]{Aqueous Sphere}
This subspell functions like the \spell{crushing wave} subspell, except that it targets everything in a \areasmall radius within \rngclose range.
\end{ability}
\vspace{0.25em}


\begin{ability}[\nth{2}]{Geyser}[\glossterm{Manifestation}, \glossterm{Water}]
Make an attack vs. Fortitude against everything in a \arealarge, 5 ft.\ wide vertical line within \rngmed range.
If this subspell has its area increased, such as with the Widened \glossterm{augment}, only the length of the line increases.
\hit Each target takes takes bludgeoning \glossterm{standard damage} \plus1d.
\end{ability}
\vspace{0.25em}


\begin{ability}[\nth{3}]{Greater Underwater Freedom}
This subspell functions like the \spell{underwater freedom} subspell, except that the target can also breathe water as if it was air.
\end{ability}
\vspace{0.25em}


\begin{ability}[\nth{3}]{Overpowering Wave}
This subspell functions like the \spell{crushing wave} subspell, except that it attacks Reflex defense instead of Fortitude defense.
\end{ability}
\vspace{0.25em}


\begin{ability}[\nth{3}]{Raging River}
This subspell functions like the \spell{crushing wave} subspell, except that it gains the \glossterm{Sustain} (standard) tag.
The area affected by the spell becomes a \glossterm{zone} that is continuously filled with rushing water.
Creatures in area suffer penalties appropriate for fighting underwater, and may be unable to breathe.
In addition, at the end of each \glossterm{action phase} in subsequent rounds, the attack is repeated in that area.
\end{ability}
\vspace{0.25em}


\begin{ability}[\nth{5}]{Greater Aqueous Blade}
This subspell functions like the \spell{aqueous blade} subspell, except that the penalty to damage is reduced to \minus1d.
\end{ability}
\vspace{0.25em}


\begin{ability}[\nth{6}]{Greater Raging River}
This subspell functions like the \textit{raging river} subspell, except that the spell gains the \glossterm{Sustain} (minor) tag instead of the \glossterm{Sustain} (standard) tag.
\end{ability}
\vspace{0.25em}

\newpage
\begin{spellsection}{Astromancy}

\begin{spellheader}
\spelldesc{You disrupt a creature's body by partially thrusting it into another plane.}
\end{spellheader}


\begin{ability}{Cantrip}[\glossterm{Planar}, \glossterm{Teleportation}]
Make an attack vs. Mental against a creature within \rngmed range.
\hit The target takes physical \glossterm{standard damage} \minus1d.
\end{ability}




\parhead{Schools} Conjuration

\parhead{Spell Lists} Arcane, Pact
\end{spellsection}


\subsubsection{Subspells}


\begin{ability}[\nth{1}]{Dimensional Jaunt}[\glossterm{Planar}, \glossterm{Teleportation}]
Make an attack vs. Mental against a creature within \rngmed range.
\hit The target takes physical \glossterm{standard damage} \plus1d.
\end{ability}
\vspace{0.25em}


\begin{ability}[\nth{1}]{Teleport}[\glossterm{Teleportation}]
Choose a Medium or smaller willing creature or unattended object within \rngclose range.
The target teleports into an unoccupied destination within range.
If the destination is invalid, this subspell is \glossterm{miscast}.
\end{ability}
\vspace{0.25em}


\begin{ability}[\nth{2}]{Banishment}
This subspell functions like the \spell{dimensional jaunt} subspell, except that it gains a \plus2 bonus to \glossterm{accuracy} against \glossterm{outsiders} not on their home planes and creatures created by \glossterm{Manifestation} abilities.
\crit The target takes double damage.
In addition, if it is an outsider not on its home plane, it is teleported to a random location on its home plane.
If it is a creature created by a \glossterm{Manifestation} ability, it immediately disappears.
\end{ability}
\vspace{0.25em}


\begin{ability}[\nth{2}]{Dimensional Jaunt -- Plane of Earth}
This subspell functions like the \spell{dimensional jaunt} subspell, except that the target is partially teleported into the Plane of Earth.
The damage becomes bludgeoning damage, and a struck target is \glossterm{slowed} as a \glossterm{condition}.
\end{ability}
\vspace{0.25em}


\begin{ability}[\nth{2}]{Dimensional Shuffle}[\glossterm{Teleportation}]
Choose up to five willing creatures within \rngmed range.
Each target teleports into the location of a different target.
\end{ability}
\vspace{0.25em}


\begin{ability}[\nth{3}]{Dimension Door}[\glossterm{Teleportation}]
You teleport to a location within \rngext range of you.
You must clearly visualize the destination's appearance, but you do not need \glossterm{line of sight} or \glossterm{line of effect} to your destination.
\end{ability}
\vspace{0.25em}


\begin{ability}[\nth{3}]{Dimensional Jaunt -- Plane of Fire}
This subspell functions like the \spell{dimensional jaunt} subspell, except that the target is partially teleported into the Plane of Fire.
The damage becomes fire damage and increases by \plus1d.
In addition, a struck target is \glossterm{ignited} until it puts out the fire.
\end{ability}
\vspace{0.25em}


\begin{ability}[\nth{3}]{Greater Teleport}
This subspell functions like the \textit{teleport} subspell, except that the range is increased to \rngext.
\end{ability}
\vspace{0.25em}


\begin{ability}[\nth{4}]{Blink}[\glossterm{Attune} (target), \glossterm{Planar}, \glossterm{Teleportation}]
Choose a willing creature within \rngclose range.
The target randomly blinks between its current plane and the Astral Plane.
This blinking stops if the target takes actions on its current plane.
In any phase where it does not take any actions, the target has a 50\% chance to completely ignore any effect that targets it directly.
It is still affected normally by abilities that affect an area.
\end{ability}
\vspace{0.25em}


\begin{ability}[\nth{5}]{Dimensional Jaunt -- Deep Astral Plane}
This subspell functions like the \spell{dimensional jaunt} subspell, except that the target is partially teleported into the deep Astral Plane.
The damage increases by \plus1d.
In addition, a struck target is \glossterm{stunned} as a \glossterm{condition}.
\end{ability}
\vspace{0.25em}


\begin{ability}[\nth{7}]{Dimensional Jaunt -- Myriad}
This subspell functions like the \spell{dimensional jaunt} subspell, except that the target is partially teleported through a dizzying array of planes.
The damage increases by \plus3d and becomes damage of all types.
\end{ability}
\vspace{0.25em}


\begin{ability}[\nth{7}]{Greater Blink}
This subspell functions like the \subspell{blink} subspell, except that the target also has a 20\% chance to completely ignore any effect that targets it directly during phases where it takes an action.
\end{ability}
\vspace{0.25em}

\newpage
\begin{spellsection}{Barrier}

\begin{spellheader}
\spelldesc{You create a barrier around your ally that resists physical intrusion.}
\end{spellheader}


\begin{ability}{Cantrip}[\glossterm{Sustain} (standard)]
Choose a willing creature in \rngclose range.
The target gains \glossterm{damage reduction} equal to your \glossterm{power} against \glossterm{physical} damage.
\end{ability}




\parhead{Schools} Abjuration

\parhead{Spell Lists} Arcane
\end{spellsection}


\subsubsection{Subspells}


\begin{ability}[\nth{1}]{Ablative Shield}[\glossterm{Attune} (target), \glossterm{Shielding}]
Choose a willing creature in \rngclose range.
The target gains \glossterm{damage reduction} equal to your \glossterm{power} against all damage except for \glossterm{energy damage}.

You can cast this spell as a \glossterm{minor action}.
\end{ability}
\vspace{0.25em}


\begin{ability}[\nth{1}]{Resist Energy}
Choose a willing creature in \rngclose range.
The target gains \glossterm{damage reduction} equal to your \glossterm{power} against \glossterm{energy damage}.

You can cast this spell as a \glossterm{minor action}.
\end{ability}
\vspace{0.25em}


\begin{ability}[\nth{2}]{Complete Shield}
This subspell functions like the \spell{ablative shield} spell, except that the damage reduction applies against all damage.
\end{ability}
\vspace{0.25em}


\begin{ability}[\nth{2}]{Repulsion Field}[\glossterm{Sustain} (minor)]
This subspell creates a repulsive field in a \areamed radius zone from your location.
Whenever an enemy makes physical contact with the spell's area for the first time, you make an attack vs. Mental against it.
\hit The target is unable to enter the spell's area with any part of its body.
The rest of its movement in the current phase is cancelled.

Creatures in the area at the time that the spell is cast are unaffected by the spell.
\end{ability}
\vspace{0.25em}


\begin{ability}[\nth{3}]{Deflective Shield}
This subspell functions like the \spell{ablative shield} spell, except that the target also gains a \plus1 \glossterm{magic bonus} to Armor defense.
\end{ability}
\vspace{0.25em}


\begin{ability}[\nth{3}]{Immunity}
Choose a willing creature in \rngclose range, and a type of damage other than physical damage (see \pcref{Damage Types}).
The target becomes immune to damage of the chosen type.
Attacks that deal damage of multiple types still inflict damage normally unless the target is immune to all types of damage dealt.
\end{ability}
\vspace{0.25em}


\begin{ability}[\nth{3}]{Retributive Shield}
This subspell functions like the \spell{ablative shield} spell, except that damage resisted by this subspell is dealt back to the attacker as life damage.
If the attacker is beyond \rngclose range of the target, this reflection fails.

Any effect which increases this subspell's range increases the range of this effect by the same amount.
This subspell is from both the Abjuration and Vivimancy schools and gains the \glossterm{Life} tag in addition to the tags from the \spell{ablative shield} spell.
\end{ability}
\vspace{0.25em}


\begin{ability}[\nth{4}]{Empowered Shield}
This subspell functions like the \spell{ablative shield} spell, except that the damage reduction increases by an amount equal to your \glossterm{power}.
\end{ability}
\vspace{0.25em}


\begin{ability}[\nth{5}]{Antilife Shell}
This effect functions like the \spell{repulsion} subspell, except that you gain a \plus10 bonus to accuracy with the attack against living creatures.
\end{ability}
\vspace{0.25em}

\newpage
\begin{spellsection}{Bless}

\begin{spellheader}
\spelldesc{You ask your deity for a blessing of resilience to aid an ally.}
\end{spellheader}


\begin{ability}{Cantrip}
Choose a willing creature within \rngclose range.
The target removes one \glossterm{condition} affecting it.
\end{ability}




\parhead{Schools} Channeling

\parhead{Spell Lists} Divine
\end{spellsection}


\subsubsection{Subspells}


\begin{ability}[\nth{1}]{Battle Blessing}[\glossterm{Attune} (target)]
The target gains a \plus1 \glossterm{magic bonus} to \glossterm{accuracy} with all attacks.

You can cast this spell as a \glossterm{minor action}.
\end{ability}
\vspace{0.25em}


\begin{ability}[\nth{1}]{Blessing of Resilience}[\glossterm{Attune} (target)]
Choose a willing creature within \rngclose range.
The target ignores the next two \glossterm{conditions} it would receive.
After resisting two conditions in this way, this spell ends.

You can cast this subspell as a \glossterm{minor action}.
\end{ability}
\vspace{0.25em}


\begin{ability}[\nth{1}]{Cleansing Blessing}
All allies within \arealarge radius from you can remove one \glossterm{condition} affecting them.
\end{ability}
\vspace{0.25em}


\begin{ability}[\nth{2}]{Blessed Blade}[\glossterm{Attune} (target), \glossterm{Fire}]
Choose a willing creature within \rngclose range.
\glossterm{Strikes} made with melee weapons wielded by the target are made against Mental defense instead of Armor defense.
However, the target takes a \minus2d penalty to damage with \glossterm{strikes}.

You can cast this subspell as a \glossterm{minor action}.
\end{ability}
\vspace{0.25em}


\begin{ability}[\nth{2}]{Blessing of Might}[\glossterm{Attune} (target)]
The target gains a \plus1d \glossterm{magic bonus} to damage with all abilities.
You can cast this subspell as a \glossterm{minor action}.
\end{ability}
\vspace{0.25em}


\begin{ability}[\nth{3}]{Divine Might}[\glossterm{Attune} (self), \glossterm{Shaping}, \glossterm{Sizing}]
You increase your size by one size category.
This increases your \glossterm{overwhelm value}, \glossterm{overwhelm resistance}, and usually increases your \glossterm{reach} (see \pcref{Size in Combat}).
However, your muscles are not increased fully to match its new size, and your Strength is unchanged.

You can cast this subspell as a \glossterm{minor action}.
\end{ability}
\vspace{0.25em}


\begin{ability}[\nth{3}]{Greater Blessing of Resilience}
This subspell functions like the \textit{blessing of resilience} subspell, except that the spell does not end until it resists four \glossterm{conditions}.
\end{ability}
\vspace{0.25em}


\begin{ability}[\nth{4}]{Divine Shield}[\glossterm{Attune} (target)]
Choose a willing creature within \rngclose range.
The target gains \glossterm{damage reduction} equal to your \glossterm{power} against all damage.

You can cast this subspell as a \glossterm{minor action}.
\end{ability}
\vspace{0.25em}


\begin{ability}[\nth{4}]{Greater Cleansing Blessing}
This subspell functions like the \spell{cleansing blessing} subspell, except that it can remove two conditions instead of one.
\end{ability}
\vspace{0.25em}


\begin{ability}[\nth{5}]{Divine Might, Greater}
This subspell functions like the \textit{divine might} subspell, except that you gain a \plus2 bonus to Strength
\end{ability}
\vspace{0.25em}


\begin{ability}[\nth{5}]{Greater Battle Blessing}
This subspell functions like the \spell{battle blessing} subspell, except that the bonus increases to \plus2.
\end{ability}
\vspace{0.25em}


\begin{ability}[\nth{5}]{Greater Blessed Blade}
This subspell functions like the \spell{blessed blade} subspell, except that the penalty to damage is reduced to \minus1d.
\end{ability}
\vspace{0.25em}


\begin{ability}[\nth{5}]{Greater Blessing of Might}
The target gains a \plus2d \glossterm{magic bonus} to damage with all abilities.
\end{ability}
\vspace{0.25em}


\begin{ability}[\nth{5}]{Supreme Blessing of Resilience}
This subspell functions like the \textit{blessing of resilience} subspell, except that the spell can resist any number of conditions without ending.
\end{ability}
\vspace{0.25em}


\begin{ability}[\nth{7}]{Divine Might, Supreme}
This subspell functions like the \spell{divine might} subspell, except that your size is increased by two size categories.
Your Strength is increased by 2 to partially match your new size.
\end{ability}
\vspace{0.25em}

\newpage
\begin{spellsection}{Chronomancy}

\begin{spellheader}
\spelldesc{You slow a foe's passage through time, inhibiting its actions.}
\end{spellheader}


\begin{ability}{Cantrip}[\glossterm{Temporal}]
Make an attack vs. Mental against a creature within \rngmed range.
\hit The target is \glossterm{slowed} as a \glossterm{condition}.
\end{ability}




\parhead{Schools} Transmutation

\parhead{Spell Lists} Arcane, Pact
\end{spellsection}


\subsubsection{Subspells}


\begin{ability}[\nth{1}]{Haste}[\glossterm{Attune} (target), \glossterm{Temporal}]
Choose a willing creature within \rngmed range.
The target gains a \plus10 foot \glossterm{magic bonus} to its \glossterm{base speed}.
You can cast this subspell as a \glossterm{minor action}.
\end{ability}
\vspace{0.25em}


\begin{ability}[\nth{1}]{Slow}[\glossterm{Temporal}]
Make an attack vs. Mental against a creature within \rngmed range.
\hit The target is \glossterm{slowed} and \glossterm{dazed} as a single \glossterm{condition}.
\crit the target is \glossterm{immobilized} and \glossterm{dazed} as a single \glossterm{condition}.
\end{ability}
\vspace{0.25em}


\begin{ability}[\nth{2}]{Time Hop}[\glossterm{Temporal}]
Choose a Medium or smalller willing creature or unattended object within \rngmed range.
You send the target into the future, causing it to temporarily cease to exist.
When you cast this subspell, you choose how many rounds the target ceases to exist for, up to a maximum of five rounds.
At the end of the last round, it reappears in the same location where it disappeared.

The area the target occupied can be physically crossed, but it is treated as an invalid destination for teleportation and other similar magic.
When the target reappears, all of its surroundings are adjusted as if the object had retroactively always existed in its space.
For example, if the location is occupied by a creature that walked into the area, the creature is relocated to the closest unoccupied space along the path it took to reach the target.

You can cast this subspell as a \glossterm{minor action}.
\end{ability}
\vspace{0.25em}


\begin{ability}[\nth{3}]{Delay Damage}[\glossterm{Sustain} (minor), \glossterm{Temporal}]
Whenever you take damage, half of the damage (rounded down) is not dealt to you immediately.
This damage is tracked separately.
When the ends, you take all of the delayed damage at once.
This damage has no type, and ignores all effects that reduce or negate damage.
Damage dealt in this way in excess of your hit points is dealt as \glossterm{vital damage}.

You can cast this subspell as a \glossterm{minor action}.
\end{ability}
\vspace{0.25em}


\begin{ability}[\nth{3}]{Greater Haste}[\glossterm{Attune} (target), \glossterm{Temporal}]
Choose a willing creature within \rngmed range.
The target gains a \plus30 foot \glossterm{magic bonus} to its \glossterm{base speed}, up to a maximum of double its \glossterm{base speed}.
In addition, it gains a \plus2 \glossterm{magic bonus} to Reflex defense.

You can cast this subspell as a \glossterm{minor action}.
\end{ability}
\vspace{0.25em}


\begin{ability}[\nth{3}]{Temporal Stasis}[\glossterm{Attune} (self), \glossterm{Temporal}]
Choose a Medium or smaller willing creature within \rngmed range.
The target is placed into stasis, rendering it unconscious.
While in stasis, it cannot take any actions and cannot be targeted, moved, damaged, or otherwise affected by outside forces in any way.

% TODO: wording
This effect normally lasts as long as you \glossterm{attune} to it, and until the end of the round when you release the attunement.
If you use this ability on yourself, it instead lasts for a number of rounds you choose when you cast the spell, up to a maximum of five rounds.

You can cast this subspell as a \glossterm{minor action}.
\end{ability}
\vspace{0.25em}


\begin{ability}[\nth{4}]{Temporal Duplicate}[\glossterm{Temporal}]
Choose a willing creature within \rngmed range.
You reach into a possible future and create a duplicate of the target.
The duplicate is identical in all ways to the target when the spell resolves, except that it has no \glossterm{legend points}.
The target and its duplicate can act during the next round.
At the end of that round, the target and its duplicate cease to exist.
At the end of the following round, the target reappears in the place where it ceased to exist.
If that space is occupied, it appears in the closest unoccupied space.

When the target reappears, its condition is unchanged from when it left, except that it loses all action points, spell points, and all similar resources equal to the amount used by its duplicate.
Its hit points, conditions, and all other statistics are unaffected, regardless of any damage or other negative effects suffered by the duplicate.
If this would reduce any of the target's resources below 0, it takes physical \glossterm{standard damage} \plus3d from the paradox and becomes \glossterm{stunned} as a \glossterm{condition}.

You can cast this subspell as a \glossterm{minor action}.
\end{ability}
\vspace{0.25em}


\begin{ability}[\nth{4}]{Time Lock}[\glossterm{Sustain} (minor), \glossterm{Temporal}]
Choose a willing creature within \rngmed range.
You lock the state of the target's body in time.
Note the target's hit points, vital damage, and active conditions.
If the target dies, this effect ends immediately.

As a \glossterm{standard action}, you can reach through time to restore the target's state.
If you do, the target's hit points, vital damage, and active conditions become identical to what they were when you cast this subspell.
This does not affect any other properties of the target, such as any resources expended.
After you restore the target's state in this way, the spell ends.

You can cast this subspell as a \glossterm{minor action}.
\end{ability}
\vspace{0.25em}


\begin{ability}[\nth{7}]{Greater Temporal Duplicate}
This subspell functions like the \spell{temporal duplicate} subspell, except that you can reach up to five minutes into the future to summon the duplicate.
When you cast the spell, you choose the length of time before the target disappears.
The duplicate still only exists for a single round.
\end{ability}
\vspace{0.25em}


\begin{ability}[\nth{7}]{Greater Time Lock}[\glossterm{Attune} (self), \glossterm{Temporal}]
This subspell functions like the \textit{time lock} subspell, except that the effect is not ended if the target dies, and restoring the target's state can also restore it to life.
If the target is restored to life in this way, all of its properties not locked by this subspell, such as any resources expended, are identical to what they were when the target died.
In addition, this spell has the \glossterm{Attune} (self) tag instead of the \glossterm{Sustain} (minor) tag.
\end{ability}
\vspace{0.25em}


\begin{ability}[\nth{7}]{Time Stop}[\glossterm{Temporal}]
You can take two full rounds of actions immediately.
During this time, all other creatures and objects are fixed in time, and cannot be targeted, moved, damaged, or otherwise affected by outside forces in any way.
You can still affect yourself and create areas or new effects.

You are still vulnerable to danger, such as from heat or dangerous gases.
However, you cannot be detected by any means while you travel.

After casting this spell, you cannot cast it again until you take a \glossterm{short rest}.
\end{ability}
\vspace{0.25em}

\newpage
\begin{spellsection}{Compel}

\begin{spellheader}
\spelldesc{You compel a foe to fall down.}
\end{spellheader}


\begin{ability}{Cantrip}[\glossterm{Compulsion}, \glossterm{Mind}]
Make an attack vs. Mental against a creature within \rngmed range.
\hit The target falls \glossterm{prone}.
\end{ability}




\parhead{Schools} Enchantment

\parhead{Spell Lists} Arcane, Divine, Pact
\end{spellsection}


\subsubsection{Subspells}


\begin{ability}[\nth{1}]{Dance}[\glossterm{Compulsion}, \glossterm{Mind}]
Make an attack vs. Mental against a creature within \rngmed range.
\hit As a \glossterm{condition}, the target is compelled to dance.
It can spend a \glossterm{move action} to dance, if it is physically capable of dancing.
At the end of each round, if the target did not dance during that round, it takes a \minus2 penalty to \glossterm{accuracy}, \glossterm{checks}, and \glossterm{defenses} as the compulsion intensifies.
This penalty stacks each round until the target dances, which resets the penalties to 0.
\crit As above, except that the target must dance as a \glossterm{standard action} to reset the penalties, instead of as a move action.
\end{ability}
\vspace{0.25em}


\begin{ability}[\nth{1}]{Fall}
Make an attack vs. Mental against all enemies in a \areamed radius from you.
\hit Each target falls \glossterm{prone}.
\crit As above, and as a \glossterm{condition}, each target is unable to stand up.
If a target is somehow brought into a standing position, it will immediately fall and become prone again.
\end{ability}
\vspace{0.25em}


\begin{ability}[\nth{2}]{Stay}[\glossterm{Compulsion}, \glossterm{Mind}]
Make an attack vs. Mental against a creature within \rngmed range.
\hit The target falls \glossterm{prone} and is \glossterm{slowed} as a \glossterm{condition}.
\crit The target falls prone and is \glossterm{immobilized} as a \glossterm{condition}.
\end{ability}
\vspace{0.25em}


\begin{ability}[\nth{3}]{Confusion}[\glossterm{Compulsion}, \glossterm{Mind}]
Make an attack vs. Mental against a creature within \rngmed range.
\hit The target is \disoriented as a \glossterm{condition}.
\crit The target is \confused as a \glossterm{condition}.
\end{ability}
\vspace{0.25em}


\begin{ability}[\nth{3}]{Sleep}[\glossterm{Compulsion}, \glossterm{Mind}]
Make an attack vs. Mental against a creature within \rngclose range.
\hit The target is \blinded as a \glossterm{condition}.
\crit The target falls asleep.
It cannot be awakened by any means while the spell lasts.
After that time, it can wake up normally, though it continues to sleep until it would wake up naturally.
% Awkward to sustain without the Sustain tag
This effect lasts as long as you \glossterm{sustain} it as a \glossterm{minor action}.
However, it is a \glossterm{condition}, and can be removed by effects which remove conditions.
\end{ability}
\vspace{0.25em}


\begin{ability}[\nth{4}]{Discordant Song}[\glossterm{Compulsion}, \glossterm{Mind}]
Make an attack vs. Mental against all enemies in a \areamed radius from you.
\hit Each target is \disoriented as a \glossterm{condition}.
\crit Each target is \confused as a \glossterm{condition}.
\end{ability}
\vspace{0.25em}


\begin{ability}[\nth{4}]{Dominate}[\glossterm{Compulsion}, \glossterm{Mind}]
Make an attack vs. Mental against a creature within \rngmed range.
\hit The target is \glossterm{confused} as a \glossterm{condition}.
\crit The target is \glossterm{stunned} as a \glossterm{condition}.
As a standard action, you can make an additional attack vs. Mental against the target as long as it remains stunned in this way and is within \rngmed range of you.
On a hit, the target becomes stunned in the same way as an additional condition, continuing the effect even if the target removed the original condition in the same phase.
On a critical hit, the target becomes \glossterm{dominated} by you as long as you \glossterm{attune} to this ability.
\end{ability}
\vspace{0.25em}


\begin{ability}[\nth{6}]{Irresistible Dance}
This subspell functions like the \textit{dance} subspell, except that you gain a \plus4 bonus to accuracy on the attack.
\end{ability}
\vspace{0.25em}

\newpage
\begin{spellsection}{Corruption}

\begin{spellheader}
\spelldesc{You corrupt your foe's life force, weakening it.}
\end{spellheader}


\begin{ability}{Corruption}[\glossterm{Life}]
Make an attack vs. Fortitude against a living creature within \rngclose range.
\hit The target is \glossterm{sickened} as a \glossterm{condition}.
\end{ability}




\parhead{Schools} Vivimancy

\parhead{Spell Lists} Arcane, Divine, Nature, Pact
\end{spellsection}


\subsubsection{Subspells}


\begin{ability}[\nth{1}]{Miasma}
Make an attack vs. Fortitude against all living enemies within an \areamed radius from you.
\hit Each target is \glossterm{sickened} as a \glossterm{condition}.
\end{ability}
\vspace{0.25em}


\begin{ability}[\nth{1}]{Sickening Decay}[\glossterm{Life}]
Make an attack vs. Fortitude against a living creature within \rngclose range.
\hit The target is \glossterm{sickened} as a \glossterm{condition}.
% TODO: clarify when exactly this damage is taken (should be at the end of the phase)
In addition, it takes life \glossterm{standard damage} \minus3d whenever it takes a \glossterm{standard action}.
It can only take damage in this way once per round.
\crit The target is \glossterm{nauseated} as a \glossterm{condition}.
In addition, it takes life \glossterm{standard damage} \minus1d whenever it takes a \glossterm{standard action}.
It can only take damage in this way once per round.
\end{ability}
\vspace{0.25em}


\begin{ability}[\nth{3}]{Bleed}
This subspell functions like the \spell{sickening decay} subspell, except that a struck target also begins bleeding as an additional \glossterm{condition}.
At the end of every subsequent \glossterm{action phase}, the target takes slashing \glossterm{standard damage} \minus2d.
\end{ability}
\vspace{0.25em}


\begin{ability}[\nth{3}]{Corruption of Blood and Bone}
This subspell functions like the \spell{sickening decay} spell, except that it gains a \plus1d bonus to damage.
In addition, damage from the spell reduces the target's maximum hit points by the same amount.
This hit point reduction is part of the same \glossterm{condition} as the spell's other effects.
When the condition is removed, the target's maximum hit points are restored.
\end{ability}
\vspace{0.25em}


\begin{ability}[\nth{3}]{Eyebite}[\glossterm{Life}]
Make an attack vs. Fortitude against a living creature within \rngclose range.
\hit The target is \glossterm{blinded} as a \glossterm{condition}.
\crit The target is \glossterm{blinded} twice by two separate \glossterm{conditions}.
Both conditions must be removed before the target can see again.
\end{ability}
\vspace{0.25em}


\begin{ability}[\nth{3}]{Greater Miasma}
This subspell functions like the \spell{miasma} subspell, except that a struck target is \glossterm{nauseated} instead of sickened.
\end{ability}
\vspace{0.25em}


\begin{ability}[\nth{4}]{Crippling Decay}
This subspell functions like the \spell{sickening decay} spell, except that a struck target is also \glossterm{immobilized} as an additional \glossterm{condition}.
\end{ability}
\vspace{0.25em}


\begin{ability}[\nth{4}]{Curse of Decay}[\glossterm{Curse}]
This subspell functions like the \spell{sickening decay} subspell, except that the attack is made against Mental defense instead of Fortitude defense.
In addition, if the attack critically hits, the spell's effect becomes a permanent curse.
It is no longer a condition, and cannot be removed by abilities that remove conditions.
\end{ability}
\vspace{0.25em}


\begin{ability}[\nth{5}]{Finger of Death}[\glossterm{Death}, \glossterm{Life}]
Make an attack vs. Fortitude against a living creature within \rngclose range.
\hit The target takes life \glossterm{standard damage} \plus3d.
\crit The target immediately dies.
\end{ability}
\vspace{0.25em}

\newpage
\begin{spellsection}{Cryomancy}

\begin{spellheader}
\spelldesc{You drain the heat from a foe.}
\end{spellheader}


\begin{ability}{Cantrip}[\glossterm{Cold}]
Make an attack vs. Fortitude against one creature or object within \rngmed range.
\hit The target takes cold \glossterm{standard damage} \minus1d.
\end{ability}




\parhead{Schools} Evocation

\parhead{Spell Lists} Arcane, Nature, Pact
\end{spellsection}


\subsubsection{Subspells}


\begin{ability}[\nth{1}]{Cone of Cold}[\glossterm{Cold}]
Make an attack vs. Fortitude against everything in a \areamed cone from you.
\hit Each target takes cold \glossterm{standard damage} \minus1d, and is \glossterm{fatigued} as a \glossterm{condition}.
\end{ability}
\vspace{0.25em}


\begin{ability}[\nth{1}]{Frostbite}[\glossterm{Cold}]
Make an attack vs. Fortitude against one creature or object within \rngmed range.
\hit The target takes cold \glossterm{standard damage} \plus1d.
\end{ability}
\vspace{0.25em}


\begin{ability}[\nth{1}]{Icecraft}[\glossterm{Attune} (self), \glossterm{Cold}]
Choose a pool of unattended, nonmagical water within \rngclose range.
This subspell creates an icy weapon or a suit of icy armor from the target pool of water.
You can create any weapon, shield, or body armor that you are proficient with, and which would normally be made entirely from metal, except for heavy body armor.
The pool of water targeted must be at least as large as the item you create.

The item functions like a normal item of its type, except that it is more fragile.
It has hit points equal to twice your \glossterm{power}, does not have any \glossterm{hardness}, and is \glossterm{vulnerable} to fire damage.
If the item would take cold damage, it instead heals that many hit points.

Whenever a creature wearing armor created in this way takes physical damage, cold damage, or fire damage, that damage is also dealt to the armor.
Likewise, whenever a creature wielding a weapon created in this way deals damage with the weapon, that damage is also dealt to the weapon.
If the item loses all of its hit points, this effect is \glossterm{dismissed}.
\end{ability}
\vspace{0.25em}


\begin{ability}[\nth{2}]{Blizzard}
This subspell functions like the \spell{cone of cold} subspell, except that the area becomes a \areamed radius from you.
\end{ability}
\vspace{0.25em}


\begin{ability}[\nth{2}]{Cold Snap}
This subspell functions like the \spell{cone of cold} subspell, except that it gains the \glossterm{Sustain} (standard) tag.
The area affected by the spell becomes a \glossterm{zone} that is supernaturally chilled.
At the end of each \glossterm{action phase} in subsequent rounds, the attack is repeated in that area.
\end{ability}
\vspace{0.25em}


\begin{ability}[\nth{2}]{Sturdy Icecraft}
This subspell functions like the \subspell{icecraft} subspell, except that the item created has hit points equal to four times your \glossterm{power}.
In addition, you can create heavy body armor.
\end{ability}
\vspace{0.25em}


\begin{ability}[\nth{3}]{Deep Freeze}
This subspell functions like the \spell{cone of cold} subspell, except that it attacks Reflex defense instead of Fortitude defense.
\end{ability}
\vspace{0.25em}


\begin{ability}[\nth{3}]{Greater Frostbite}
This subspell functions like the \subspell{frostbite} subspell, except that a struck target is also \glossterm{exhausted} as a \glossterm{condition}.
\end{ability}
\vspace{0.25em}


\begin{ability}[\nth{4}]{Enhanced Icecraft}
This subspell functions like the \subspell{sturdy icecraft} subspell, except that the item created is magically enhanced.
A weapon gains a \plus1d \glossterm{magic bonus} to damage with \glossterm{strikes}, and armor grants a \plus1 \glossterm{magic bonus} to the defenses it improves.
\end{ability}
\vspace{0.25em}


\begin{ability}[\nth{4}]{Freezing}
This subspell functions like the \spell{cone of cold} subspell, except that each struck target is also \glossterm{immobilized} as an additional \glossterm{condition}.
\end{ability}
\vspace{0.25em}


\begin{ability}[\nth{4}]{Greater Cold Snap}
This subspell functions like the \textit{cold snap} subspell, except that the spell gains the \glossterm{Sustain} (minor) tag instead of the \glossterm{Sustain} (standard) tag.
\end{ability}
\vspace{0.25em}

\newpage
\begin{spellsection}{Delusion}

\begin{spellheader}
\spelldesc{You terrify your foe.}
\end{spellheader}


\begin{ability}{Cantrip}[\glossterm{Emotion}, \glossterm{Mind}]
Make an attack vs. Mental against a creature within \rngmed range.
\hit The target is \glossterm{shaken} by you as a \glossterm{condition}.
\crit The target is \glossterm{frightened} by you as a \glossterm{condition}.
\end{ability}




\parhead{Schools} Enchantment

\parhead{Spell Lists} Arcane, Divine, Pact
\end{spellsection}


\subsubsection{Subspells}


\begin{ability}[\nth{1}]{Agony}[\glossterm{Emotion}, \glossterm{Mind}]
Make an attack vs. Mental against a creature within \rngmed range.
\hit The target is inflicted with agonizing pain as a \glossterm{condition}.
It suffers a \minus2 penalty to Mental defense.
% Does this need to clarify that it takes effect in the round the spell was cast?
In addition, at the end of each \glossterm{delayed action phase}, if the target took damage that round, it takes \glossterm{standard damage} \minus2d.
This damage is of all damage types that the target was damaged by during that round.
\end{ability}
\vspace{0.25em}


\begin{ability}[\nth{1}]{Cause Fear}[\glossterm{Emotion}, \glossterm{Mind}]
Make an attack vs. Mental against a creature within \rngmed range.
\hit The target is \frightened by you as a \glossterm{condition}.
\crit The target is \panicked by you as a \glossterm{condition}.
\end{ability}
\vspace{0.25em}


\begin{ability}[\nth{2}]{Calm Emotions}[\glossterm{Emotion}, \glossterm{Mind}, \glossterm{Sustain} (standard)]
Make an attack vs. Mental against all creatures within a \areamed radius from you.
\hit Each target has its emotions calmed.
The effects of all other \glossterm{Emotion} abilities on that target are \glossterm{suppressed}.
It cannot take violent actions (although it can defend itself) or do anything destructive.
If the target takes damage or feels that it is in danger, this effect is \glossterm{dismissed}.
\end{ability}
\vspace{0.25em}


\begin{ability}[\nth{2}]{Charm}[\glossterm{Attune} (self), \glossterm{Emotion}, \glossterm{Mind}, \glossterm{Subtle}]
Make an attack vs. Mental against a creature within \rngmed range.
If the target thinks that you or your allies are threatening it, you take a \minus5 penalty to accuracy on the attack.
\hit The target is \charmed by you.
Any act by you or your apparent allies that threatens or damages the \spell{charmed} person breaks the effect.
This effect is automatically \glossterm{dismissed} after one hour.
\crit As above, except that the effect is not automatically dismissed.
\end{ability}
\vspace{0.25em}


\begin{ability}[\nth{2}]{Enrage}[\glossterm{Emotion}, \glossterm{Mind}]
Make an attack vs. Mental against up to five creatures within \rngmed range.
\hit As a \glossterm{condition}, each target is unable to take any \glossterm{standard actions} that do not cause it to make an attack.
For example, it could make a \glossterm{strike} or cast an offensive spell, but it could not heal itself or summon an ally.
This cannot prevent it from taking the \textit{recover} or \textit{desperate recovery} actions.
\end{ability}
\vspace{0.25em}


\begin{ability}[\nth{2}]{Redirected Fear}
This subspell functions like the \spell{cause fear} subspell, except that you also choose a willing ally within the spell's range.
The target is afraid of the chosen ally instead of being afraid of you.
\end{ability}
\vspace{0.25em}


\begin{ability}[\nth{5}]{Amnesiac Charm}
This subspell functions like the \spell{charm} subspell, except that when the spell ends, an affected target forgets all events that transpired during the spell's duration.
It becomes aware of its surroundings as if waking up from a daydream.
The target is not directly aware of any magical influence on its mind, though unusually paranoid or perceptive creatures may deduce that their minds were affected.
\end{ability}
\vspace{0.25em}

\newpage
\begin{spellsection}{Divine Judgment}

\begin{spellheader}
\spelldesc{You smite a foe with holy (or unholy) power.}
\end{spellheader}


\begin{ability}{Cantrip}
Make an attack vs. Mental against a creature within \rngmed range.
\hit The target takes divine \glossterm{standard damage} \minus1d.
\end{ability}




\parhead{Schools} Channeling

\parhead{Spell Lists} Divine
\end{spellsection}


\subsubsection{Subspells}


\begin{ability}[\nth{1}]{Divine Judgment}
Make an attack vs. Mental against a creature within \rngmed range.
\hit The target takes divine \glossterm{standard damage} \plus1d.
\end{ability}
\vspace{0.25em}


\begin{ability}[\nth{2}]{Word of Faith}
Make an attack vs. Mental against all enemies in a \areamed radius from you.
\hit Each target takes divine \glossterm{standard damage} \minus1d.
\end{ability}
\vspace{0.25em}

\newpage
\begin{spellsection}{Electromancy}

\begin{spellheader}
\spelldesc{You create a bolt of electricity that fries your foes.}
\end{spellheader}


\begin{ability}{Cantrip}[\glossterm{Electricity}]
Make an attack vs. Fortitude against everything in a \areamed, 5 ft.\ wide line from you.
\hit Each target takes electricity \glossterm{standard damage} \minus2d.
\end{ability}




\parhead{Schools} Evocation

\parhead{Spell Lists} Arcane, Nature, Pact
\end{spellsection}


\subsubsection{Subspells}


\begin{ability}[\nth{1}]{Lightning Bolt}[\glossterm{Electricity}]
Make an attack vs. Fortitude against everything in a \arealarge, 10 ft.\ wide line from you.
\hit Each target takes electricity \glossterm{standard damage} \minus1d.
\end{ability}
\vspace{0.25em}


\begin{ability}[\nth{1}]{Shocking Grasp}[\glossterm{Electricity}]
Make an attack vs. Fortitude against one creature or object you \glossterm{threaten}.
You gain a \plus5 bonus to \glossterm{concentration} checks to cast this subspell.
\hit The target takes electricity \glossterm{standard damage} \plus1d.
\end{ability}
\vspace{0.25em}


\begin{ability}[\nth{2}]{Call Lightning}[\glossterm{Electricity}]
Make an attack vs. Fortitude against everything in a \arealarge, 5 ft.\ wide vertical line within \rngmed range.
If you are outdoors in cloudy or stormy weather, you gain a \plus2 bonus to \glossterm{accuracy} with the attack.
If this subspell has its area increased, such as with the Widened \glossterm{augment}, only the length of the line increases.
\hit Each target takes takes electricity \glossterm{standard damage} \plus1d.
\end{ability}
\vspace{0.25em}


\begin{ability}[\nth{2}]{Dynamo}
This subspell functions like the \spell{lightning bolt} subspell, except that it gains the \glossterm{Sustain} (standard) tag.
The area affected by the spell becomes a \glossterm{zone} that is continuously filled with electrical pulses.
At the end of each \glossterm{action phase} in subsequent rounds, the attack is repeated in that area.
\end{ability}
\vspace{0.25em}


\begin{ability}[\nth{2}]{Forked Lightning}
This subspell functions like the \spell{lightning bolt} subspell, except that you create two separate line-shaped areas instead of one.
The two areas can overlap, but targets in the overlapping area are only affected once.
\end{ability}
\vspace{0.25em}


\begin{ability}[\nth{2}]{Magnetic}
This subspell functions like the \spell{lightning bolt} subspell, except that you gain a \plus2 bonus to accuracy against targets wearing metal armor or otherwise carrying or composed of a significant amount of metal.
\end{ability}
\vspace{0.25em}


\begin{ability}[\nth{2}]{Uncontrolled Discharge}
Make an attack vs. Fortitude against everything in a \areamed radius from you.
\hit Each target takes electricity \glossterm{standard damage} \minus1d.
\end{ability}
\vspace{0.25em}


\begin{ability}[\nth{3}]{Magnetic Blade}[\glossterm{Attune} (self), \glossterm{Electricity}]
% TODO: unattended or attended by a willing ally
Choose an unattended metal weapon within \rngclose range.
Attacks with the weapon gain a \plus2 bonus to \glossterm{accuracy} against targets wearing metal armor or otherwise carrying or composed of a significant amount of metal.

You can cast this subspell as a \glossterm{minor action}.
\end{ability}
\vspace{0.25em}


\begin{ability}[\nth{3}]{Shocking}
This subspell functions like the \spell{lightning bolt} subspell, except that each struck target is also \glossterm{dazed} as a \glossterm{condition}.
Each critically struck target is \glossterm{stunned} instead of dazed.
\end{ability}
\vspace{0.25em}


\begin{ability}[\nth{4}]{Chain Lightning}[\glossterm{Electricity}]
Make an attack vs. Fortitude against one creature or object within \rngmed range.
\hit The target takes electricity \glossterm{standard damage} \plus2d.
In addition, make an additional attack vs. Fortitude against any number of creatures in a \areamed radius from the struck target.
\hit Each secondary target takes electricity \glossterm{standard damage}.
\end{ability}
\vspace{0.25em}


\begin{ability}[\nth{4}]{Greater Dynamo}
This subspell functions like the \textit{dynamo} subspell, except that the spell gains the \glossterm{Sustain} (minor) tag instead of the \glossterm{Sustain} (standard) tag.
\end{ability}
\vspace{0.25em}

\newpage
\begin{spellsection}{Fabrication}

\begin{spellheader}
\spelldesc{You conjure acid from thin air to land on a foe.}
\end{spellheader}


\begin{ability}{Cantrip}[\glossterm{Acid}, \glossterm{Manifestation}]
Make an attack vs. Armor against one creature or object within \rngmed range.
\hit The target takes acid \glossterm{standard damage} \minus1d.
\end{ability}




\parhead{Schools} Conjuration

\parhead{Spell Lists} Arcane, Pact
\end{spellsection}


\subsubsection{Subspells}


\begin{ability}[\nth{1}]{Acid Splash}[\glossterm{Acid}, \glossterm{Manifestation}]
Make an attack vs. Armor against one creature or object within \rngmed range.
\hit The target takes acid \glossterm{standard damage} \plus1d.
\end{ability}
\vspace{0.25em}


\begin{ability}[\nth{1}]{Forge}[\glossterm{Attune} (self)]
Choose a type of weapon or shield that you are proficient with.
You create a normal item of that type anywhere within \rngclose range.

The item cannot be constructed of any magical or extraordinary material.
% This should allow the Giant augment; is this worded to allow that?
It is sized appropriately for you, up to a maximum of a Medium size item.
\end{ability}
\vspace{0.25em}


\begin{ability}[\nth{2}]{Corrosive Splash}
This subspell functions like the \spell{acid splash} spell, except that it deals double damage to objects.
\end{ability}
\vspace{0.25em}


\begin{ability}[\nth{2}]{Greater Forge}
This subspell functions like the \spell{forge} subspell, except that you can also create any type of body armor you are proficient with.
If you create body armor, you can create it already equipped to a willing creature within range.
\end{ability}
\vspace{0.25em}


\begin{ability}[\nth{2}]{Poison}[\glossterm{Manifestation}, \glossterm{Poison}]
Make an attack vs. Fortitude against a creature within \rngmed range.

\hit The target takes poison \glossterm{standard damage} and is poisoned as a \glossterm{condition}.
If the target is poisoned, repeat this attack at the end of each \glossterm{action phase} after the first round.
On the second hit, the target takes poison \glossterm{standard damage} \minus1d and becomes \glossterm{sickened}.
On the third hit, the target takes poison \glossterm{standard damage} \plus1d and becomes \glossterm{nauseated} instead of sickened.
After the third hit, no further attacks are made, but the target remains nauseated until the condition is removed.
\end{ability}
\vspace{0.25em}


\begin{ability}[\nth{2}]{Web}[\glossterm{Manifestation}, \glossterm{Sustain} (minor)]
You fill a \areasmall radius zone in \rngclose range with webs.
The webs make the area \glossterm{difficult terrain}.
Each 5-ft.\ square of webbing has hit points equal to your \glossterm{power}, and is \glossterm{vulnerable} to fire.

In addition, you make an attack vs. Reflex against all creatures in the area when the spell is cast.
\hit Each target is \immobilized as long as it has webbing from this ability in its space.
\end{ability}
\vspace{0.25em}


\begin{ability}[\nth{3}]{Lingering Acid Splash}
This subspell functions like the \spell{acid splash} spell, except that the acid lingers on a struck target.
At the end of each \glossterm{action phase} in subsequent rounds, the target takes acid \glossterm{standard damage} \minus1d.
This is a \glossterm{condition}, and lasts until removed.
\end{ability}
\vspace{0.25em}


\begin{ability}[\nth{3}]{Meteor}[\glossterm{Manifestation}]
You create a meteor in midair that falls to the ground, crushing foes in its path.
The meteor takes up a \areamed radius, and must be created in unoccupied space.
After being summoned, it falls up to 100 feet before disappearing.
Make an attack vs. Armor against everything in its path.
\hit Each target takes bludgeoning and fire \glossterm{standard damage}.
\end{ability}
\vspace{0.25em}


\begin{ability}[\nth{3}]{Reinforced Webbing}
This subspell functions like the \textit{web} subspell, except that each 5-ft.\ square of webbing gains additional hit points equal to your \glossterm{power}.
In addition, the webs are no longer \glossterm{vulnerable} to fire damage.
\end{ability}
\vspace{0.25em}


\begin{ability}[\nth{5}]{Meteor Storm}
This subspell functions like the \textit{meteor} subspell, except that you can create up to five different meteors.
The areas affected by two different meteors cannot overlap.
If one of the meteors is created in an invalid area, that meteor is not created, but the others are created and dealt their damage normally.
\end{ability}
\vspace{0.25em}

\newpage
\begin{spellsection}{Glamer}

\begin{spellheader}
\spelldesc{}
\end{spellheader}


\begin{ability}{Cantrip}[\glossterm{Sensation}, \glossterm{Visual}]
You can only cast this spell during the \glossterm{action phase}.
Choose a willing creature within \rngclose range.
The target's weapons become briefly invisible, and its hands are blurred.
On the first melee \glossterm{strike} the target makes during the \glossterm{delayed action phase} of this round, the target of the attack is treated as \glossterm{unaware} of the attack.

This effect provides no offensive benefit against creatures immune to \glossterm{Visual} abilities.
\end{ability}




\parhead{Schools} Illusion

\parhead{Spell Lists} Arcane
\end{spellsection}


\subsubsection{Subspells}


\begin{ability}[\nth{1}]{Blur}[\glossterm{Attune} (target), \glossterm{Sensation}, \glossterm{Visual}]
Choose a willing creature within \rngmed range.
The target's physical outline is distorted so it appears blurred, shifting, and wavering.
It gains a \plus1 \glossterm{magic bonus} to Armor defense and Stealth (see \pcref{Stealth}).
This effect provides no defensive benefit against creatures immune to \glossterm{Visual} abilities.

You can cast this subspell as a \glossterm{minor action}.
\end{ability}
\vspace{0.25em}


\begin{ability}[\nth{1}]{Hidden Blade}[\glossterm{Attune} (target), \glossterm{Sensation}, \glossterm{Visual}]
Choose a willing creature within \rngclose range.
The target's weapons become invisible, and its hands are blurred.
On the first melee \glossterm{strike} the target makes, the target of the attack is treated as \glossterm{unaware} of the attack.
After the target makes an attack in this way, this effect ends.

This effect provides no offensive benefit against creatures immune to \glossterm{Visual} abilities.
\end{ability}
\vspace{0.25em}


\begin{ability}[\nth{1}]{Suppress Light}[\glossterm{Attune} (self), \glossterm{Light}, \glossterm{Sensation}]
Choose a Small or smaller unattended object within \rngclose range.
This subspell suppresses light in a \areamed radius emanation from the target.
Light within or passing through the area is dimmed to be no brighter than shadowy illumination.
Any object or effect which blocks light also blocks this spell's emanation.
\end{ability}
\vspace{0.25em}


\begin{ability}[\nth{2}]{Disguise Image}[\glossterm{Attune} (target), \glossterm{Sensation}, \glossterm{Visual}]
Choose a willing creature within \rngclose range.
You make a Disguise check to alter the target's appearance (see \pcref{Disguise Creature}).
You gain a \plus5 bonus on the check, and you can freely alter the appearance of the target's clothes and equipment, regardless of their original form.
However, this effect is unable to alter the sound, smell, texture, or temperature of the target or its clothes and equipment.
\end{ability}
\vspace{0.25em}


\begin{ability}[\nth{2}]{Mirror Image}[\glossterm{Attune} (target), \glossterm{Sensation}, \glossterm{Visual}]
Choose a willing creature within \rngclose range.
Four illusory duplicates appear around the target that mirror its every move.
The duplicates shift chaotically in its space, making it difficult to identify the real creature.

All \glossterm{targeted} \glossterm{physical attacks} against the target have a 50\% miss chance.
Whenever an attack misses in this way, it affects an image, destroying it.
This ability provides no defensive benefit against creatures immune to \glossterm{Visual} abilities.
\end{ability}
\vspace{0.25em}


\begin{ability}[\nth{3}]{Shadow Mantle}
This subspell functions like the \spell{blur} subspell, except that the spell's deceptive nature extends beyond altering light to affect the nature of reality itself.
The defense bonus it provides applies to all defenses.
In addition, the spell loses the \glossterm{Visual} tag, and can protect against attacks from creatures immune to Visual abilities.
\end{ability}
\vspace{0.25em}


\begin{ability}[\nth{4}]{Greater Mirror Image}
This subspell functions like the \textit{mirror image} subspell, except that destroyed images can reappear.
At the end of each \glossterm{action phase}, one destroyed image reappears, to a maximum of four images.
\end{ability}
\vspace{0.25em}


\begin{ability}[\nth{6}]{Displacement}[\glossterm{Attune} (target), \glossterm{Sensation}, \glossterm{Visual}]
Choose a willing creature within \rngmed range.
The target's image appears to be two to three feet from its real location.
\glossterm{Targeted} \glossterm{physical attacks} against the target suffer a 50\% miss chance.
This ability provides no defensive benefit against creatures immune to \glossterm{Visual} abilities.
\end{ability}
\vspace{0.25em}

\newpage
\begin{spellsection}{Photomancy}


\begin{ability}{Cantrip}[\glossterm{Light}, \glossterm{Sensation}, \glossterm{Visual}]
Make an attack vs. Fortitude against one creature, object, or location within \rngmed range.
Bright light illuminates a 100 foot radius around the target until the end of the round.
\hit The target is \dazzled as a \glossterm{condition}.
\crit As above, and target is also \dazed as an additional \glossterm{condition}.
\end{ability}




\parhead{Schools} Illusion

\parhead{Spell Lists} Arcane, Divine, Nature, Pact
\end{spellsection}


\subsubsection{Subspells}


\begin{ability}[\nth{1}]{Flare}[\glossterm{Light}, \glossterm{Sensation}, \glossterm{Visual}]
A burst of light light fills a \areasmall radius within \rngmed range of you.
Bright light illuminates a 100 foot radius around the area until the end of the round.
Make an attack vs. Fortitude against all creatures in the source area.
\hit Each target is \dazzled as a \glossterm{condition}.
\crit As above, and target is also \dazed as an additional \glossterm{condition}.
\end{ability}
\vspace{0.25em}


\begin{ability}[\nth{2}]{Faerie Fire}
This subspell functions like the \spell{flare} subspell, except that each struck target is surrounded with a pale glow made of hundreds of ephemeral points of light.
This causes the struck target to radiate bright light in a 5 foot radius, as a candle.
The lights impose a \minus10 penalty to Stealth checks.
In addition, they reveal the outline of the creatures if they become \glossterm{invisible}.
This allows observers to see their location, though not to see them perfectly.
\end{ability}
\vspace{0.25em}


\begin{ability}[\nth{2}]{Illuminating}
This subspell functions like the \spell{flare} subspell, except that it gains the \glossterm{Sustain} (minor) tag.
The area affected by the spell becomes an illuminated \glossterm{zone}.
At the end of each \glossterm{action phase} in subsequent rounds, the attack is repeated in that area.
\end{ability}
\vspace{0.25em}


\begin{ability}[\nth{3}]{Flashbang}
This subspell functions like the \spell{flare} subspell, except that an intense sound accompanies the flash of light caused by the spell.
Each struck target is also \glossterm{deafened} as an additional \glossterm{condition}.
This subspell gains the \glossterm{Auditory} tag in addition to the tags from the \spell{flare} subspell.
\end{ability}
\vspace{0.25em}


\begin{ability}[\nth{3}]{Kaleidoscopic}[\glossterm{Light}, \glossterm{Mind}, \glossterm{Sensation}, \glossterm{Visual}]
This subspell creates a brilliant, rapidly shifting rainbow of lights in a \areasmall radius within \rngmed range of you.
They illuminate a 100 foot radius around the area with bright light until the end of the round.
Make an attack vs. Mental against all creatures in the source area.
\hit Each target is \disoriented as a \glossterm{condition}.
\crit Each target is \confused as a \glossterm{condition}.
\end{ability}
\vspace{0.25em}


\begin{ability}[\nth{3}]{Pillars of Light}
This subspell functions like the \spell{flare} subspell, except that it affects up to five different \areasmall radius, 50 ft. tall cylinders within range.
The areas can overlap, but targets in the overlapping area are only affected once.
\end{ability}
\vspace{0.25em}


\begin{ability}[\nth{4}]{Blinding}
This subspell functions like the \spell{flare} subspell, except that each struck target is \glossterm{blinded} instead of \glossterm{dazzled}.
\end{ability}
\vspace{0.25em}


\begin{ability}[\nth{4}]{Solar Flare}
This subspell functions like the \spell{flare} subspell, except that you gain a \plus2 bonus to accuracy with the attack.
In addition, the light is treated as being natural sunlight for the purpose of abilities.
This can allow it to destroy vampires and have similar effects.
\end{ability}
\vspace{0.25em}


\begin{ability}[\nth{7}]{Greater Solar Flare}
This subspell functions like the \spell{solar flare} subspell, except that the accuracy bonus is increased to \plus4.
\end{ability}
\vspace{0.25em}

\newpage
\begin{spellsection}{Polymorph}

\begin{spellheader}
\spelldesc{You transform a foe's body into a more broken state.}
\end{spellheader}


\begin{ability}{Polymorph}[\glossterm{Shaping}]
Make an attack vs. Fortitude against a creature within \rngmed range.
\hit The target takes physical \glossterm{standard damage} \minus1d.
\end{ability}




\parhead{Schools} Transmutation

\parhead{Spell Lists} Arcane, Nature, Pact
\end{spellsection}


\subsubsection{Subspells}


\begin{ability}[\nth{1}]{Baleful Polymorph}[\glossterm{Shaping}]
Make an attack vs. Fortitude against a creature within \rngmed range.
\hit The target takes physical \glossterm{standard damage} \plus1d.
\end{ability}
\vspace{0.25em}


\begin{ability}[\nth{1}]{Mending}
% TODO: unattended or attended by a willing creature
Choose an unattended object within \rngclose range.
The target is healed for hit points equal to \glossterm{standard damage} \plus1d.
\end{ability}
\vspace{0.25em}


\begin{ability}[\nth{1}]{Shrink}[\glossterm{Attune} (target), \glossterm{Shaping}, \glossterm{Sizing}]
Choose a willing creature within \rngclose range.
The target's size decreases by one size category, to a minimum of Tiny.
This decreases its Strength by 2 and usually decreases its \glossterm{reach} (see \pcref{Size in Combat}).

You can cast this subspell as a \glossterm{minor action}.
\end{ability}
\vspace{0.25em}


\begin{ability}[\nth{2}]{Alter Appearance}[\glossterm{Attune} (target), \glossterm{Shaping}]
Choose a Large or smaller willing creature within \rngclose range.
You make a Disguise check to alter the target's appearance (see \pcref{Disguise Creature}).
You gain a \plus5 bonus on the check, and you ignore penalties for changing the target's gender, race, subtype, or age.
However, this effect is unable to alter the target's clothes or equipment in any way.

You can cast this subspell as a \glossterm{minor action}.
\end{ability}
\vspace{0.25em}


\begin{ability}[\nth{2}]{Barkskin}[\glossterm{Attune} (target)]
Choose a willing creature within \rngclose range.
The target gains \glossterm{damage reduction} equal to your \glossterm{power} against damage dealt by \glossterm{physical attacks}.
In addition, it is \glossterm{vulnerable} to fire damage.

You can cast this subspell as a \glossterm{minor action}.
\end{ability}
\vspace{0.25em}


\begin{ability}[\nth{3}]{Craft Object}[\glossterm{Shaping}]
Choose any number of unattended, nonmagical objects within \rngclose range.
You make a Craft check to transform the targets into a new item (or items) made of the same materials.
You require none of the tools or time expenditure that would normally be necessary.
The total size of all targets combined must be Large size or smaller.
\end{ability}
\vspace{0.25em}


\begin{ability}[\nth{3}]{Enlarge}[\glossterm{Attune} (target), \glossterm{Shaping}, \glossterm{Sizing}]
Choose a Large or smaller willing creature within \rngclose range.
The target's size increases by one size category.
This increases its \glossterm{overwhelm value}, \glossterm{overwhelm resistance}, and usually increases its \glossterm{reach} (see \pcref{Size in Combat}).
However, the target's muscles are not increased fully to match its new size, and its Strength is unchanged.

You can cast this subspell as a \glossterm{minor action}.
\end{ability}
\vspace{0.25em}


\begin{ability}[\nth{3}]{Stoneskin}[\glossterm{Attune} (target)]
Choose a willing creature within \rngclose range.
The target gains \glossterm{damage reduction} equal to your \glossterm{power} against damage dealt by \glossterm{physical attacks}, except for damage from adamantine weapons.
You can cast this subspell as a \glossterm{minor action}.
\end{ability}
\vspace{0.25em}


\begin{ability}[\nth{4}]{Greater Shrink}
This subspell functions like the \spell{shrink} subspell, except that the target's size decreases by two size categories, to a minimum of Diminuitive.
\end{ability}
\vspace{0.25em}


\begin{ability}[\nth{4}]{Regeneration}[\glossterm{Attune} (target)]
Choose a willing creature within \rngclose range.
A the end of each round, the target heals hit points equal to your \glossterm{power}.
\end{ability}
\vspace{0.25em}


\begin{ability}[\nth{5}]{Disintegrate}[\glossterm{Shaping}]
Make an attack vs. Fortitude against a creature within \rngmed range.
\hit The target takes physical \glossterm{standard damage} \plus3d.
In addition, if the target has no hit points remaining, it dies.
Its body is completely disintegrated, leaving behind only a pinch of fine dust.
Its equipment is unaffected.
\end{ability}
\vspace{0.25em}


\begin{ability}[\nth{5}]{Enlarge, Greater}
This subspell functions like the \textit{enlarge} subspell, except that the target's Strength is increased by 2 to match its new size.
\end{ability}
\vspace{0.25em}


\begin{ability}[\nth{6}]{Ironskin}
This subspell functions like the \textit{stoneskin} subspell, except that the damage reduction is equal to twice your \glossterm{power}.
\end{ability}
\vspace{0.25em}


\begin{ability}[\nth{7}]{Enlarge, Supreme}
This subspell functions like the \spell{enlarge} subspell, except that the target's size is increased by two size categories.
Its Strength is increased by 2 to partially match its new size.
\end{ability}
\vspace{0.25em}


\begin{ability}[\nth{7}]{Greater Regeneration}
This subspell functions like the \textit{regeneration} subspell, except that the healing is equal to twice your \glossterm{power}.
\end{ability}
\vspace{0.25em}

\newpage
\begin{spellsection}{Pyromancy}

\begin{spellheader}
\spelldesc{You create a small burst of flame.}
\end{spellheader}


\begin{ability}{Cantrip}[\glossterm{Fire}]
Make an attack vs. Armor against one creature or object within \rngmed range.
\hit The target takes fire \glossterm{standard damage} \minus1d.
\end{ability}




\parhead{Schools} Evocation

\parhead{Spell Lists} Arcane, Fire, Nature, Pact
\end{spellsection}


\subsubsection{Subspells}


\begin{ability}[\nth{1}]{Firebolt}[\glossterm{Fire}]
Make an attack vs. Armor against one creature within \rngmed range.
\hit The target takes fire \glossterm{standard damage} \plus1d.
\end{ability}
\vspace{0.25em}


\begin{ability}[\nth{1}]{Fireburst}[\glossterm{Fire}]
Make an attack vs. Armor against everything in a \areasmall radius within \rngclose range.
\hit Each target takes fire \glossterm{standard damage} \minus1d.
\end{ability}
\vspace{0.25em}


\begin{ability}[\nth{2}]{Blast Furnace}
This subspell functions like the \spell{fireburst} subspell, except that it gains the \glossterm{Sustain} (standard) tag.
The area affected by the spell becomes a \glossterm{zone} that is continuously engulfed in flames.
At the end of each \glossterm{action phase} in subsequent rounds, the attack is repeated in that area.
\end{ability}
\vspace{0.25em}


\begin{ability}[\nth{2}]{Burning Hands}[\glossterm{Fire}]
Make an attack vs. Armor against everything in a \arealarge cone.
\hit Each target takes fire \glossterm{standard damage} \minus1d.
\end{ability}
\vspace{0.25em}


\begin{ability}[\nth{2}]{Flame Blade}[\glossterm{Attune} (target), \glossterm{Fire}]
Choose a willing creature within \rngclose range.
% Is this clear enough at not stacking with magic bonuses intrinsic to the creature?
Weapons wielded by the target gain a \plus1d \glossterm{magic bonus} to damage with \glossterm{strikes}.
In addition, all damage dealt with strikes using its weapons becomes fire damage in addition to the attack's normal damage types.

You can cast this subspell as a \glossterm{minor action}.
\end{ability}
\vspace{0.25em}


\begin{ability}[\nth{2}]{Ignition}
This subspell functions like the \spell{fireburst} subspell, except that each struck target is also \glossterm{ignited} as a \glossterm{condition}.
\end{ability}
\vspace{0.25em}


\begin{ability}[\nth{3}]{Fearsome Flame}
This subspell functions like the \spell{fireburst} subspell, except that the attack result is also compared to each target's Mental defense.
\hit Each target is \glossterm{shaken} as a \glossterm{condition}.
\end{ability}
\vspace{0.25em}


\begin{ability}[\nth{3}]{Fireball}
Make an attack vs. Armor against everything in a \areamed radius within \rngmed range.
\hit Each target takes fire \glossterm{standard damage} \minus1d.
\end{ability}
\vspace{0.25em}


\begin{ability}[\nth{3}]{Flame Serpent}[\glossterm{Fire}]
Make an attack vs. Armor against everything in a \arealarge, 5 ft.\ wide shapeable line within \rngmed range.
\hit Each target takes fire \glossterm{standard damage} \minus1d.
\end{ability}
\vspace{0.25em}


\begin{ability}[\nth{3}]{Inferno}[\glossterm{Fire}]
Make an attack vs. Armor against everything in a \arealarge radius from you.
\hit Each target takes fire \glossterm{standard damage} \minus1d.
\end{ability}
\vspace{0.25em}


\begin{ability}[\nth{3}]{Superheated Fireburst}
This subspell functions like the \spell{fireburst} subspell, except that it attacks Reflex defense instead of Armor defense.
\end{ability}
\vspace{0.25em}


\begin{ability}[\nth{4}]{Flame Aura}[\glossterm{Attune} (target), \glossterm{Fire}]
Choose a willing creature within \rngclose range.
Heat constantly radiates in a \areamed radius emanation from the target.
At the end of each \glossterm{action phase}, make an attack vs. Armor against everything in the area.
\hit Each target takes fire \glossterm{standard damage} \minus2d.

You can cast this subspell as a \glossterm{minor action}.
In addition, you can apply the Widened \glossterm{augment} to this subspell.
If you do, it increases the area of the emanation.
\end{ability}
\vspace{0.25em}


\begin{ability}[\nth{4}]{Greater Blast Furnace}
This subspell functions like the \textit{blast furnace} subspell, except that the spell gains the \glossterm{Sustain} (minor) tag instead of the \glossterm{Sustain} (standard) tag.
\end{ability}
\vspace{0.25em}


\begin{ability}[\nth{4}]{Greater Ignition}
This subspell functions like the \spell{fireburst} subspell, except that each target hit is also \glossterm{ignited} as a \glossterm{condition}.
Unlike the normal ignited effect, this condition cannot be removed by putting out the fire.
In addition, the ignited effect deals fire \glossterm{standard damage} \minus3d instead of the normal 1d6 fire damage each round.
\end{ability}
\vspace{0.25em}


\begin{ability}[\nth{5}]{Greater Inferno}
This subspell functions like the \textit{inferno} subspell, except that it affects everything in a 200 ft.\ radius from you.
\end{ability}
\vspace{0.25em}


\begin{ability}[\nth{6}]{Supreme Ignition}
This subspell functions like the \textit{greater ignition} subspell, except that the condition must be removed twice before the effect ends.
\end{ability}
\vspace{0.25em}

\newpage
\begin{spellsection}{Revelation}

\begin{spellheader}
\spelldesc{You grant a creature the ability to see fractions of a second into the future.}
\end{spellheader}


\begin{ability}{Cantrip}
You can only cast this spell during the \glossterm{action phase}.
Choose a willing creature within \rngclose range.
On the first \glossterm{strike} the target makes during the \glossterm{delayed action phase} of this round, it rolls twice and takes the higher result.
\end{ability}




\parhead{Schools} Divination

\parhead{Spell Lists} Arcane, Divine, Nature
\end{spellsection}


\subsubsection{Subspells}


\begin{ability}[\nth{1}]{Precognitive Defense}
Choose a willing creature within \rngclose range.
The target gains a \plus1 \glossterm{magic bonus} to Armor defense and Reflex defense.
\end{ability}
\vspace{0.25em}


\begin{ability}[\nth{1}]{Precognitive Offense}[\glossterm{Attune} (target)]
Choose a willing creature within \rngclose range.
The target gains a \plus1 \glossterm{magic bonus} to \glossterm{accuracy} with all attacks.

You can cast this spell as a \glossterm{minor action}.
\end{ability}
\vspace{0.25em}


\begin{ability}[\nth{1}]{True Strike}
You can only cast this spell during the \glossterm{action phase}.
Choose a willing creature within \rngclose range.
On the first \glossterm{strike} the target makes during the \glossterm{delayed action phase} of this round, it gains a \plus4 bonus to \glossterm{accuracy} and rolls twice and takes the higher result.
\end{ability}
\vspace{0.25em}


\begin{ability}[\nth{2}]{Boon of Many Eyes}
Choose a willing creature within \rngclose range.
The target gains a \plus1 \glossterm{magic bonus} to \glossterm{overwhelm resistance}.
\end{ability}
\vspace{0.25em}


\begin{ability}[\nth{2}]{Boon of Mastery}
Choose a willing creature within \rngclose range.
The target gains a \plus2 \glossterm{magic bonus} to all skills.
\end{ability}
\vspace{0.25em}


\begin{ability}[\nth{2}]{Discern Lies}[\glossterm{Attune} (self), \glossterm{Detection}]
Make an attack vs. Mental against a creature within \rngmed range.
\hit You know when the target deliberately and knowingly speaks a lie.
This ability does not reveal the truth, uncover unintentional inaccuracies, or necessarily reveal evasions.
\end{ability}
\vspace{0.25em}


\begin{ability}[\nth{3}]{Boon of Knowledge}
The target gains a \plus4 \glossterm{magic bonus} to all Knowledge skills (see \pcref{Knowledge}).
\end{ability}
\vspace{0.25em}


\begin{ability}[\nth{3}]{Greater True Strike}
This subspell functions like the \textit{true strike} subspell, except that the bonus is increased to \plus6.
\end{ability}
\vspace{0.25em}


\begin{ability}[\nth{3}]{Third Eye}
Choose a willing creature within \rngclose range.
The target gains \glossterm{blindsight} with a 50 foot range.
This can allow it to see perfectly without any light, regardless of concealment or invisibility.
\end{ability}
\vspace{0.25em}


\begin{ability}[\nth{4}]{Greater Precognitive Defense}
This subspell functions like the \spell{precognitive defense} subspell, except that the bonus is increased to \plus2.
\end{ability}
\vspace{0.25em}


\begin{ability}[\nth{4}]{Greater Precognitive Offense}
This subspell functions like the \spell{precognitive offense} subspell, except that the bonus is increased to \plus2.
\end{ability}
\vspace{0.25em}


\begin{ability}[\nth{5}]{Greater Boon of Mastery}
This subspell functions like the \spell{boon of mastery} spell, except that the bonus is increased to \plus4.
\end{ability}
\vspace{0.25em}


\begin{ability}[\nth{5}]{Supreme True Strike}
This subspell functions like the \textit{true strike} subspell, except that the bonus is increased to \plus8.
\end{ability}
\vspace{0.25em}

\newpage
\begin{spellsection}{Scry}

\begin{spellheader}
\spelldesc{You create a scrying sensor that allows you to see at a distance.}
\end{spellheader}


\begin{ability}{Cantrip}[\glossterm{Scrying}, \glossterm{Sustain} (minor)]
This cantrip functions like the \textit{arcane eye} subspell, except that it gains the \glossterm{Sustain} (minor) tag in place of the \glossterm{Attune} (self) tag.",
In addition, the sensor cannot be moved after it is originally created.
\end{ability}




\parhead{Schools} Divination

\parhead{Spell Lists} Arcane, Divine, Nature
\end{spellsection}


\subsubsection{Subspells}


\begin{ability}[\nth{1}]{Alarm}
A \glossterm{scrying sensor} appears floating in the air in an unoccupied square within \rngmed range.
The sensor passively observes its surroundings.
If it sees a creature or object of Tiny size or larger moving within 50 feet of it, it will trigger a mental "ping" that only you can notice.
You must be within 1 mile of the sensor to receive this mental alarm.
This mental sensation is strong enough to wake you from normal sleep, but does not otherwise disturb concentration.
\end{ability}
\vspace{0.25em}


\begin{ability}[\nth{1}]{Arcane Eye}[\glossterm{Attune} (self), \glossterm{Scrying}]
A \glossterm{scrying sensor} appears floating in the air in an unoccupied square within \rngmed range.
At the start of each round, you choose whether you see from this sensor or from your body.

While viewing through the sensor, your visual acuity is the same as your normal body, except that it does not share the benefits of any \glossterm{magical} effects that improve your vision.
You otherwise act normally, though you may have difficulty moving or taking actions if the sensor cannot see your body or your intended targets, effectively making you \blinded.

If undisturbed, the sensor floats in the air in its position.
As a \glossterm{minor action}, you can concentrate to move the sensor up to 30 feet in any direction, even vertically.
At the end of each round, if the sensor is does not have \glossterm{line of effect} from you, it is destroyed.
\end{ability}
\vspace{0.25em}


\begin{ability}[\nth{2}]{Accelerated Eye}
This subspell functions like the \spell{arcane eye} subspell, except that the sensor moves up to 100 feet when moved instead of up to 30 feet.
\end{ability}
\vspace{0.25em}


\begin{ability}[\nth{2}]{Auditory Eye}
This subspell functions like the \spell{arcane eye} subspell, except that you can you can also hear through the sensor.
At the start of each round, you can choose whether you hear from the sensor or from your body.
This choice is made independently from your sight.
The sensor's auditory acuity is the same as your own, except that it does not share the benefits of any \glossterm{magical} effects that improve your hearing.
\end{ability}
\vspace{0.25em}


\begin{ability}[\nth{2}]{Greater Alarm}
This subspell functions like the \textit{alarm} subspell, except that the sensor gains 100 ft.\ \glossterm{darkvision} and its Awareness bonus is equal to your \glossterm{power}.
\end{ability}
\vspace{0.25em}


\begin{ability}[\nth{2}]{Reverse Scrying}
Choose a magical sensor within \rngmed range.
A new scrying sensor appears at the location of the source of the the ability that created the target sensor.
This sensor functions like the sensor created by the \spell{autonomous eye} subspell, except that the sensor cannot move.
\end{ability}
\vspace{0.25em}


\begin{ability}[\nth{3}]{Autonomous Eye}
This subspell functions like the \spell{arcane eye} subspell, except that the sensor is not destroyed when it loses \glossterm{line of effect} to you.
\end{ability}
\vspace{0.25em}


\begin{ability}[\nth{3}]{Twin Eye}
This subspell functions like the \spell{arcane eye} subspell, except that you constantly receive sensory input from both your body and the sensor.
This allows you to see simultaneously from your body and from the sensor.
\end{ability}
\vspace{0.25em}


\begin{ability}[\nth{4}]{Penetrating Eye}
This subspell functions like the \spell{autonomous eye} subspell, except that you do not need \glossterm{line of sight} or \glossterm{line of effect} to target a location.
You must specify a distance and direction to target a location you cannot see.
This can allow you to cast the spell beyond walls and similar obstacles.
As normal, if the intended location is occupied or otherwise impossible, the spell is \glossterm{miscast}.
\end{ability}
\vspace{0.25em}

\newpage
\begin{spellsection}{Summon}

\begin{spellheader}
\spelldesc{You summon a creature to fight by your side.}
\end{spellheader}


\begin{ability}{Summon Monster}[\glossterm{Manifestation}, \glossterm{Sustain} (standard)]
This cantrip functions like the \spell{summon monster} subspell, except that it has the \glossterm{Sustain} (standard) tag instead of the \glossterm{Sustain} (minor) tag.
\end{ability}




\parhead{Schools} Conjuration

\parhead{Spell Lists} Arcane, Divine, Nature
\end{spellsection}


\subsubsection{Subspells}


\begin{ability}[\nth{1}]{Summon Monster}[\glossterm{Manifestation}, \glossterm{Sustain} (minor)]
You summon a creature in an unoccupied square within \rngmed range.
It visually appears to be a common Small or Medium animal of your choice, though in reality it is a manifestation of magical energy.
Regardless of the appearance and size chosen, the creature has hit points equal to twice your \glossterm{power}.
All of its defenses are equal to your 4 \add your \glossterm{power}, and its \glossterm{land speed} is equal to 30 feet.

Each round, you choose the creature's actions.
There are only two actions it can take.
As a move action, it can move as you direct.
As a standard action, it can make a melee \glossterm{strike} against a creature it threatens.
Its accuracy is equal to your \glossterm{accuracy}.
If it hits, it deals \glossterm{standard damage} \minus2d.
The type of damage dealt by this attack depends on the creature's appearance.
Most animals bite or claw their foes, which deals bludgeoning and slashing damage.
\end{ability}
\vspace{0.25em}


\begin{ability}[\nth{2}]{Summon Bear}
This subspell functions like the \spell{summon monster} spell, except that the creature appears to be a Medium bear.
As a standard action, it can make a \glossterm{grapple} attack against a creature it threatens.
Its accuracy is the same as its accuracy with \glossterm{strikes}.
While grappling, the manifested creature can either make a strike or attempt to escape the grapple.

This augment replaces the effects of any other augments that change the appearance of the creature.
\end{ability}
\vspace{0.25em}

\newpage
\begin{spellsection}{Telekinesis}

\begin{spellheader}
\spelldesc{You crush your foe's body with telekinetic force.}
\end{spellheader}


\begin{ability}{Cantrip}
Make an attack vs. Mental against one creature or object within \rngmed range.
\hit The target takes bludgeoning \glossterm{standard damage} \minus1d.
\end{ability}




\parhead{Schools} Evocation

\parhead{Spell Lists} Arcane, Pact
\end{spellsection}


\subsubsection{Subspells}


\begin{ability}[\nth{1}]{Telekinetic Crush}
Make an attack vs. Mental against one creature or object within \rngmed range.
\hit The target takes bludgeoning \glossterm{standard damage} \plus1d.
\end{ability}
\vspace{0.25em}


\begin{ability}[\nth{1}]{Telekinetic Lift}
Choose a Medium or smaller willing creature or unattended object within \rngclose range.
The target is reduced to one quarter of its normal weight.
This gives it a \plus4 bonus to Jump, if applicable, and makes it easier to lift and move.
\end{ability}
\vspace{0.25em}


\begin{ability}[\nth{1}]{Telekinetic Throw}
Make an attack vs. Mental against a Medium or smaller creature or object within \rngmed range.
\hit You move the target up to 50 feet in any direction.
You can change direction partway through the movement.
Moving the target upwards costs twice the normal movement cost.

% Wording?
If the target is willing, you can move it up to 100 feet.
\end{ability}
\vspace{0.25em}


\begin{ability}[\nth{2}]{Binding Crush}
This subspell functions like the \spell{telekinetic crush} subspell, except that the struck creature is also \glossterm{slowed} as a \glossterm{condition} if it is Large or smaller.
\end{ability}
\vspace{0.25em}


\begin{ability}[\nth{3}]{Greater Telekinetic Lift}
This subspell functions like the \spell{telekinetic lift} subspell, except that the target is reduced to one tenth of its normal weight.
This increases the Jump bonus to \plus8.
\end{ability}
\vspace{0.25em}


\begin{ability}[\nth{3}]{Greater Telekinetic Throw}
This subspell functions like the \textit{telekinetic throw} spell, except that you can move the target up to 100 feet.
If the target is willing, you can move it up to 200 feet.
\end{ability}
\vspace{0.25em}


\begin{ability}[\nth{4}]{Levitate}[\glossterm{Attune} (self)]
Choose a Medium or smaller willing creature or unattended object within \rngclose range.
% TODO: Wording
As long as the target remains within 50 feet above a surface that could support its weight, it floats in midair, unaffected by gravity.
During the movement phase, you can move the target up to ten feet in any direction as a \glossterm{free action}.
\end{ability}
\vspace{0.25em}


\begin{ability}[\nth{5}]{Greater Binding Crush}
This subspell functions like the \spell{telekinetic crush} subspell, except that the struck creature is also \glossterm{immobilized} as a \glossterm{condition} if it is Large or smaller.
\end{ability}
\vspace{0.25em}

\newpage
\begin{spellsection}{Thaumaturgy}


\begin{ability}{Cantrip}[\glossterm{Mystic}, \glossterm{Sustain} (standard)]
Make an attack against one creature within \rngmed range.
The attack result is applied to every \glossterm{magical} effect on the target.
The DR for each effect is equal to 5 + the \glossterm{power} of that effect.
\hit Each effect is \glossterm{suppressed}.
\end{ability}




\parhead{Schools} Abjuration

\parhead{Spell Lists} Arcane, Divine, Nature
\end{spellsection}


\subsubsection{Subspells}


\begin{ability}[\nth{1}]{Alter Magic Aura}[\glossterm{Attune} (self), \glossterm{Mystic}]
Make an attack vs. Mental against one Large or smaller magical object in \rngmed range.
\hit One of the target's magic auras is altered (see \pcref{Spellcraft}).
You can change the school and descriptors of the aura.
In addition, you can decrease the \glossterm{power} of the aura by up to half your power, or increase the power of the aura up to a maximum of your power.
\end{ability}
\vspace{0.25em}


\begin{ability}[\nth{1}]{Suppress Item}[\glossterm{Mystic}, \glossterm{Sustain} (minor)]
Make an attack vs. Mental against one Large or smaller magical object in \rngmed range.
\hit All magical properties the target has are \glossterm{suppressed}.
\end{ability}
\vspace{0.25em}


\begin{ability}[\nth{1}]{Suppress Magic}[\glossterm{Mystic}, \glossterm{Sustain} (standard)]
Make an attack against one creature, object, or magical effect within \rngmed range.
If you target a creature or object, the attack result is applied to every \glossterm{magical} effect on the target.
% Is this clear enough?
This does not affect the passive effects of any magic items the target has equipped.
If you target a magical effect directly, the attack result is applied against the effect itself.
The DR for each effect is equal to 5 + the \glossterm{power} of that effect.
\hit Each effect is \glossterm{suppressed}.
\end{ability}
\vspace{0.25em}


\begin{ability}[\nth{2}]{Dimensional Anchor}[\glossterm{Mystic}, \glossterm{Sustain} (minor)]
Make an attack vs. Mental against a creature or object within \rngmed range.
\hit The target is unable to travel extradimensionally.
This prevents all \glossterm{Manifestation}, \glossterm{Planar}, and \glossterm{Teleportation} effects.
\end{ability}
\vspace{0.25em}


\begin{ability}[\nth{2}]{Dismissal}[\glossterm{Mystic}]
Make an attack against one creature or object within \rngmed range.
If the target is an effect of an ongoing \glossterm{magical} ability, such as a summoned monster or created object, the DR is equal to the \glossterm{power} of the ability.
Otherwise, this subspell has no effect.
\hit The target is treated as if the ability that created it was \glossterm{dismissed}.
This usually causes the target to disappear.
\end{ability}
\vspace{0.25em}


\begin{ability}[\nth{2}]{Dispel Magic}
This subspell functions like the \spell{suppress magic} spell, except that a hit against an effect causes it to be \glossterm{dismissed} instead of suppressed.
\end{ability}
\vspace{0.25em}


\begin{ability}[\nth{2}]{Malign Transferance}[\glossterm{Mystic}]
Choose a willing ally within \rngmed range.
The ally must be currently affected by a \glossterm{magical} \glossterm{condition}.
In addition, make an attack vs. Mental against a creature within \rngmed range.
\hit One magical condition of your choice is removed from the chosen ally and applied to the struck creature.
\crit As above, except that you can transfer any number of magical conditions in this way.
\end{ability}
\vspace{0.25em}


\begin{ability}[\nth{3}]{Enhance Magic}[\glossterm{Attune} (target), \glossterm{Mystic}]
Choose a willing creature within \rngmed range.
The target gains a \plus1 enhancement bonus to \glossterm{power} with spells.
\end{ability}
\vspace{0.25em}


\begin{ability}[\nth{4}]{Dimensional Lock}[\glossterm{Attune} (self), \glossterm{Mystic}]
This subspell creates a dimensional lock in a \arealarge radius zone from your location.
Extraplanar travel into or out of the area is impossible.
This prevents all \glossterm{Manifestation}, \glossterm{Planar}, and \glossterm{Teleportation} effects.
\end{ability}
\vspace{0.25em}


\begin{ability}[\nth{5}]{Greater Malign Transferance}
Choose any number of willing allies within \rngmed range.
Each ally must be currently affected by a \glossterm{magical} \glossterm{condition}.
In addition, make an attack vs. Mental against a creature within \rngmed range.
\hit Up to two magical conditions of your choice are removed from the chosen allies and applied to the struck creature.
\crit As above, except that you can transfer any number of magical conditions in this way.
\end{ability}
\vspace{0.25em}


\begin{ability}[\nth{5}]{Resist Magic}
Choose a willing creature within \rngmed range.
The target gains \glossterm{magic resistance} equal to 5 + your \glossterm{power}.
\end{ability}
\vspace{0.25em}


\begin{ability}[\nth{6}]{Greater Enhance Magic}[\glossterm{Attune} (target), \glossterm{Mystic}]
This subspell functions like the \textit{enhance magic} subspell, except that the bonus is increased to \plus2.
\end{ability}
\vspace{0.25em}


\begin{ability}[\nth{7}]{Antimagic Field}[\glossterm{Mystic}, \glossterm{Sustain} (minor)]
All other magical abilities and objects are \glossterm{suppressed} within a \areamed radius emanation from you.
% How much of this is redundant with suppression?
Creatures within the area cannot activate, sustain, or dismiss magical abilities.
% TODO: wording
This does not affect aspects of creatures that cannot be suppressed, such as the knowledge of abilities.
You cannot exclude yourself from this emanation.
However, this subspell does not prevent you from sustaining or dismissing this subspell.
\end{ability}
\vspace{0.25em}

\newpage
\begin{spellsection}{Vital Surge}


\begin{ability}{Vital Surge}[\glossterm{Life}]
Make an attack vs. Fortitude against a creature within \rngmed range.
\hit The target takes life damage equal to \glossterm{standard damage} \minus1d.
\end{ability}




\parhead{Schools} Vivimancy

\parhead{Spell Lists} Divine, Nature
\end{spellsection}


\subsubsection{Subspells}


\begin{ability}[\nth{1}]{Cure Wounds}[\glossterm{Life}]
Choose a willing creature within \rngmed range.
The target heals hit points equal to \glossterm{standard damage} \plus1d.
\end{ability}
\vspace{0.25em}


\begin{ability}[\nth{1}]{Inflict Wounds}[\glossterm{Life}]
Make an attack vs. Fortitude against a creature within \rngmed range.
\hit The target takes life damage equal to \glossterm{standard damage} \plus1d.
\end{ability}
\vspace{0.25em}


\begin{ability}[\nth{2}]{Drain Life}
This subspell functions like the \spell{inflict wounds} subspell, except that you heal hit points equal to your \glossterm{power} if you hit, up to a maximum equal to the damage dealt.
\end{ability}
\vspace{0.25em}


\begin{ability}[\nth{2}]{Greater Cure Wounds}
This subspell functions like the \spell{cure wounds} subspell, except that for every 5 points of healing you provide, you can instead heal one point of \glossterm{vital damage}.
\end{ability}
\vspace{0.25em}


\begin{ability}[\nth{2}]{Remove Disease}[\glossterm{Flesh}]
This subspell functions like the \spell{cure wounds} subspell, except that all diseases affecting the target are also removed.
\end{ability}
\vspace{0.25em}


\begin{ability}[\nth{2}]{Restore Senses}[\glossterm{Flesh}]
This subspell functions like the \spell{cure wounds} subspell, except that one of the target's physical senses, such as sight or hearing, is also restored to full capacity.
This can heal both magical and mundane effects, but it cannot completely replace missing body parts required for a sense to function (such as missing eyes).
\end{ability}
\vspace{0.25em}


\begin{ability}[\nth{3}]{Circle of Death}[\glossterm{Life}, \glossterm{Sustain} (minor)]
You are surrounded by an aura of death in a \areamed radius emanation from you.
When this spell resolves, and the end of each \glossterm{action phase} in subsequent rounds, make an attack vs. Fortitude against all enemies in the area.
\hit Each target takes life \glossterm{standard damage} \minus3d.
\end{ability}
\vspace{0.25em}


\begin{ability}[\nth{3}]{Circle of Healing}[\glossterm{Life}, \glossterm{Sustain} (minor)]
You are surrounded by an aura of healing in a \areamed radius emanation from you.
When this spell resolves, and the end of each \glossterm{action phase} in subsequent rounds, all allies in the area heal hit points equal to half your \glossterm{power}.
\end{ability}
\vspace{0.25em}


\begin{ability}[\nth{3}]{Death Knell}[\glossterm{Death}]
This subspell functions like the \spell{inflict wounds} subspell, except that a struck target suffers a death knell as a \glossterm{condition}.
At the end of each round, if the target has 0 hit points, it immediately dies.
% TODO: wording
If the target dies while the condition is active, you heal hit points equal to twice your \glossterm{power}.
\end{ability}
\vspace{0.25em}


\begin{ability}[\nth{4}]{Heal}
This subspell functions like the \spell{cure wounds} subspell, except that it gains a \plus1d bonus to healing.
In addition, it heals \glossterm{vital damage} as easily as it heals hit points.
\end{ability}
\vspace{0.25em}

\newpage
\begin{spellsection}{Weaponcraft}

\begin{spellheader}
\spelldesc{You create a dancing blade that attacks nearby foes}
\end{spellheader}


\begin{ability}{Cantrip}[\glossterm{Manifestation}]
Make an attack vs. Armor against one creature or object within \rngmed range.
\hit The target takes piercing \glossterm{standard damage} \minus1d.
\end{ability}




\parhead{Schools} Conjuration, Transmutation

\parhead{Spell Lists} Arcane, Divine, Pact
\end{spellsection}


\subsubsection{Subspells}


\begin{ability}[\nth{1}]{Blade Barrier}[\glossterm{Sustain} (minor)]
A wall of whirling blades appears within \rngmed range.
The wall either takes the form of a 20 ft.\ high, \arealarge line.
The wall provides \glossterm{active cover} (20\% miss chance) against attacks made through it.
Attacks that miss in this way harmlessly strike the wall.
Whenever a creature or object passes through the wall, make an attack vs. Armor against it.
\hit The target takes slashing \glossterm{standard damage} \minus1d.
\end{ability}
\vspace{0.25em}


\begin{ability}[\nth{1}]{Mystic Bow}[\glossterm{Manifestation}]
Make an attack vs. Armor against one creature or object within \rngmed range.
\hit The target takes piercing \glossterm{standard damage} \plus1d.
\end{ability}
\vspace{0.25em}


\begin{ability}[\nth{1}]{Summon Weapon}[\glossterm{Manifestation}, \glossterm{Sustain} (minor)]
A melee weapon that you are proficient with appears in an unoccupied square within \rngmed range.
The weapon floats about three feet off the ground, and is sized appropriately for a creature of your size.
The specific weapon you choose affects the type of damage it deals.
Regardless of the weapon chosen, it has hit points equal to twice your \glossterm{power}.
All of its defenses are equal to 3 \add your \glossterm{power}, and it has a 30 foot fly speed with good maneuverability, though it cannot travel farther than five feet above the ground.

Each round, the weapon automatically moves towards the creature closest to it during the \glossterm{movement phase}.
During the \glossterm{action phase}, it makes a melee \glossterm{strike} against a random creature adjacent to it.
Its accuracy is equal to your \glossterm{accuracy}.
If it hits, it deals \glossterm{standard damage} \minus1d.
\end{ability}
\vspace{0.25em}


\begin{ability}[\nth{2}]{Aerial}
This subspell functions like the \spell{weaponcraft} spell, except that the weapon's height above the ground is not limited.
This allows the weapon to fly up to fight airborne foes.
\end{ability}
\vspace{0.25em}


\begin{ability}[\nth{2}]{Blade Perimeter}
This subspell functions like the \spell{blade barrier} subspell, except that the wall is an 20 ft.\ high, \areamed radius circle.
\end{ability}
\vspace{0.25em}


\begin{ability}[\nth{2}]{Create Ballista}
This subspell functions like the \spell{weaponcraft} spell, except that it creates a fully functional Large ballista instead of a weapon of your choice.
The ballista functions like any other weapon, with the following exceptions.

It cannot move, and makes ranged \glossterm{strikes} instead of melee strikes.
Its attacks have a maximum range of 100 feet.
Its attacks deal piercing damage, and its hit points are equal to three times your \glossterm{power}.
In addition, the ballista attacks the creature farthest from it, instead of the creature closest to it.
\end{ability}
\vspace{0.25em}


\begin{ability}[\nth{3}]{Blade Barrier, Dual}
This subspell functions like the \spell{blade barrier} subspell, except that the area must be a line.
In addition, the spell creates two parallel walls of the same length, five feet apart.
\end{ability}
\vspace{0.25em}


\begin{ability}[\nth{3}]{Contracting Blade Perimeter}
This subspell functions like the \spell{blade perimeter} subspell, except that the wall's radius shrinks by 5 feet at the end of every \glossterm{action phase}, dealing damage to everything it moves through.
% Clarify interaction with solid obstacles that block contraction?
\end{ability}
\vspace{0.25em}


\begin{ability}[\nth{3}]{Giant Blade}
This subspell functions like the \spell{weaponcraft} spell, except that the weapon takes the form of a Large greatsword.
The weapon's attacks hit everything in a \areasmall cone from it.
It aims the cone to hit as many creatures as possible.
\end{ability}
\vspace{0.25em}


\begin{ability}[\nth{4}]{Create Ballista, Dual Track}
This subspell functions like the \spell{create ballista} subspell, except that the ballista is created with two separate bolt tracks.
This allows it to fire at two different targets in the same round whenever you command it to fire.
It cannot fire at the same target twice.
Each round, it attacks the two creatures farthest from it.
\end{ability}
\vspace{0.25em}


\begin{ability}[\nth{6}]{Titan Blade}
This subspell functions like the \spell{weaponcraft} spell, except that the weapon takes the form of a Gargantuan greatsword.
The weapon's attacks hit everything in a \areamed cone from it.
It aims the cone to hit as many creatures as possible.
\end{ability}
\vspace{0.25em}


\begin{ability}[\nth{7}]{Paired Weapons}
This subspell functions like the \spell{summon weapon} subspell, except that you summon two weapons instead of one.
Each weapon attacks independently.
\end{ability}
\vspace{0.25em}


\section{Ritual Descriptions}


\begin{spellsection}{Animate Dead}[3]

\begin{spellheader}
\spelldesc{You bind a fragment of a dead creature's soul to its corpse, reanimating it as an undead skeleton or zombie.}
\end{spellheader}


\begin{ability}{Animate Dead}[\glossterm{Attune} (multiple)]

Choose any number of corpses within \rngclose range.
The combined levels of all targets cannot exceed your spellpower.
The target becomes an undead creature that obeys your spoken commands.
You choose whether to create a skeleton or a zombie.
Creating a zombie require a mostly intact corpse, including most of the flesh.
Creating a skeleton only requires a mostly intact skeleton.
If a skeleton is made from an intact corpse, the flesh quickly falls off the animated bones.

This ritual takes one hour to perform.

\end{ability}




\parhead{Schools} Vivimancy

\parhead{Spell Lists} Arcane, Divine
\end{spellsection}


\begin{spellsection}{Awaken}[7]


\begin{ability}{Awaken}

Choose a Large or smaller willing animal within \rngclose range.
The target becomes sentient.
Its Intelligence becomes 1d6 \sub 5.
Its type changes from animal to magical beast.
It gains the ability to speak and understand one language that you know of your choice.
This effect is permanent.

This ritual takes 24 hours to perform, and requires 49 action points from its participants.

\end{ability}




\parhead{Schools} Transmutation

\parhead{Spell Lists} Nature
\end{spellsection}


\begin{spellsection}{Binding}[3]


\begin{ability}{Binding}[\glossterm{Attune}]

This ritual creates a \areasmall radius zone.
The outlines of the zone are denoted by a magic circle physically inscribed on the ground during the ritual.
The circle is obvious, but a DR 16 Perception or Spellcraft check is required to verify that the circle belongs to a \ritual{binding} ritual.
If the circle's perimeter is broken, the ritual's effects end immediately.
Whenever a creature enters the area, you make a Spellpower vs. Mental attack against it.
\hit The target is unable to escape the ritual's area physically or alter the circle in any way.
It treats the edge of the area as an impassable barrier, preventing the effects of any of its abilities from extending outside that area.

This ritual takes one hour to perform.

\end{ability}




\parhead{Schools} Abjuration

\parhead{Spell Lists} Arcane, Divine
\end{spellsection}


\subsubsection{Subrituals}


\begin{ability}[\nth{5}]{Dimension Lock}
This subritual functions like the \ritual{binding} ritual, except that a struck creature also cannot travel extradimensionally.
This prevents all \glossterm{Manifestation}, \glossterm{Planar}, and \glossterm{Teleportation} effects.
\end{ability}
\vspace{0.25em}


\begin{spellsection}{Bless Water}[1]


\begin{ability}{Bless Water}[\glossterm{Attune} (multiple)]

Choose one pint of unattended, nonmagical water within \rngclose range.
The target becomes holy water.
Holy water can be can be thrown as a splash weapon, dealing 1d8 points of damage to a struck undead creature or an evil outsider.

This ritual takes one minute to perform.

\end{ability}




\parhead{Schools} Channeling

\parhead{Spell Lists} Divine
\end{spellsection}


\begin{spellsection}{Blessed Transit}[5]


\begin{ability}{Blessed Transit}[\glossterm{Teleportation}]

This ritual functions like the \ritual{overland teleporation} ritual, except that the destination must be a temple or equivalent holy site to your deity.

\end{ability}




\parhead{Schools} Conjuration

\parhead{Spell Lists} divine
\end{spellsection}


\begin{spellsection}{Create Object}[3]


\begin{ability}{Create Object}[\glossterm{Attune} (multiple), \glossterm{Manifestation}]

Make a Craft check to create an object of no greater than Small size.
The object appears out of thin air in an unoccupied square within \rngclose range.
% TODO: add ability to create objects of other sizes/materials
It must be made of nonliving, nonreactive vegetable matter, such as wood or cloth.

This ritual takes one hour to perform.

\end{ability}




\parhead{Schools} Conjuration

\parhead{Spell Lists} Arcane, Divine, Nature
\end{spellsection}


\begin{spellsection}{Create Sustenance}[3]


\begin{ability}{Create Sustenance}[\glossterm{Creation}]

Choose an unoccupied square within \rngclose range.
This ritual creates food and drink in that square that is sufficient to sustain two Medium creatures per spellpower for 24 hours.
The food that this ritual creates is simple fare of your choice -- highly nourishing, if rather bland.

This ritual takes one hour to perform.

\end{ability}




\parhead{Schools} Conjuration

\parhead{Spell Lists} Arcane, Divine, Nature
\end{spellsection}


\begin{spellsection}{Curse Water}[1]


\begin{ability}{Curse Water}[\glossterm{Attune} (multiple)]

Choose one pint of unattended, nonmagical water within \rngclose range.
The target becomes unholy water.
Unholy water can be can be thrown as a splash weapon, dealing 1d8 points of damage to a struck good outsider.

This ritual takes one minute to perform.

\end{ability}




\parhead{Schools} Channeling

\parhead{Spell Lists} Divine
\end{spellsection}


\begin{spellsection}{Discern Location}[5]


\begin{ability}{Discern Location}

Choose a creature or object on the same plane as you.
You do not need \glossterm{line of sight} or \glossterm{line of effect} to the target.
However, you must specify your target with a precise mental image of its appearance.
The image does not have to be perfect, but it must unambiguously identify the target.
You learn the location (place, name, business name, or the like), community, country, and continent where the target lies.

This ritual takes 24 hours to perform, and it requires 25 action points from its participants.

\end{ability}




\parhead{Schools} Divination

\parhead{Spell Lists} Arcane, Divine, Nature
\end{spellsection}


\subsubsection{Subrituals}


\begin{ability}[\nth{7}]{Interplanar}
This subritual functions like the \ritual{discern location} ritual, except that the target does not have to be on the same plane as you.
It gains the \glossterm{Planar} tag in addition to the tags from the \ritual{discern location} ritual.

This ritual takes 24 hours to perform, and it requires 49 action points from its participants.
\end{ability}
\vspace{0.25em}


\begin{spellsection}{Endure Elements}[1]


\begin{ability}{Endure Elements}[\glossterm{Attune} (multiple)]

Choose a willing creature or unattended object within \rngclose range.
The target suffers no harm from being in a hot or cold environment.
It can exist comfortably in conditions between \minus50 and 140 degrees Fahrenheit.
Its equipment, if any, is also protected.
This does not protect the target from fire or cold damage.

This ritual takes one minute to perform.

\end{ability}




\parhead{Schools} Abjuration

\parhead{Spell Lists} Arcane, Divine, Nature
\end{spellsection}


\begin{spellsection}{Explosive Runes}[3]


\begin{ability}{Explosive Runes}[\glossterm{Attune} (multiple), \glossterm{Trap}]

Choose a Small or smaller unattended object with writing on it within \rngclose range.
In addition, choose a type of \glossterm{energy damage} (cold, electricity, fire, or sonic).
This ritual gains the tag appropriate to the chosen energy type.
If a creature reads the target, the target explodes.
You make a Spellpower vs. Reflex attack against everything within an \areamed radius from the target.
\hit Each target takes \glossterm{standard damage} \minus1d of the damage type chosen.

After the target explodes in this way, the ritual is \glossterm{dismissed}.
If the target object is destroyed or rendered illegible, the ritual is dismissed without exploding.
This ritual takes one hour to perform.

\end{ability}




\parhead{Schools} Evocation

\parhead{Spell Lists} Arcane
\end{spellsection}


\begin{spellsection}{Fertility}[3]


\begin{ability}{Fertility}

This ritual creates an area of bountiful growth in a one mile radius zone from your location.
Normal plants within the area become twice as productive as normal for the next year.
This ritual does not stack with itself.
If the \ritual{infertility} ritual is also applied to the same area, the most recently performed ritual takes precedence.

This ritual takes 24 hours to perform, and requires 9 action points from its participants.

\end{ability}




\parhead{Schools} Transmutation

\parhead{Spell Lists} Nature
\end{spellsection}


\begin{spellsection}{Gate}[9]


\begin{ability}{Gate}[\glossterm{Planar}, \glossterm{Sustain} (standard), \glossterm{Teleportation}]

Choose a plane that connects to your current plane, and a location within that plane.
This ritual creates an interdimensional connection between your current plane and the location you choose, allowing travel between those two planes in either direction.
The gate takes the form of an \areasmall radius circular disk, oriented a direction you choose (typically vertical).
It is a two-dimensional window looking into the plane you specified when casting the spell, and anyone or anything that moves through it is shunted instantly to the other location.
The gate cannot be \glossterm{sustained} for more than 5 rounds, and is automatically dismissed at the end of that time.

You must specify the gate's destination with a precise mental image of its appearance.
The image does not have to be perfect, but it must unambiguously identify the location.
Incomplete or incorrect mental images may result in the ritual leading to an unintended destination within the same plane, or simply failing entirely.

% TODO: Is this planar cosmology correct?
The Astral Plane connects to every plane, but transit from other planes is usually more limited.
From the Material Plane, you can only reach the Astral Plane.

This ritual takes one week to perform, and requires 81 action points from its participants.

\end{ability}




\parhead{Schools} Conjuration

\parhead{Spell Lists} Arcane, Divine, Nature
\end{spellsection}


\begin{spellsection}{Gentle Repose}[2]


\begin{ability}{Gentle Repose}[\glossterm{Attune} (multiple), \glossterm{Temporal}]

Choose an unattended, nonmagical object within \rngclose range.
Time does not pass for the target, preventing it from decaying or spoiling.
This can extend the time a poison or similar item lasts before becoming inert.
If used on a corpse, this effectively extends the time limit on raising that creature from the dead (see \ritual{resurrection}) and similar effects that require a fresh body.
Additionally, this can make transporting a fallen comrade more pleasant.

This ritual takes one minute to perform.

\end{ability}




\parhead{Schools} Transmutation

\parhead{Spell Lists} Arcane, Divine, Nature
\end{spellsection}


\begin{spellsection}{Infertility}[3]


\begin{ability}{Fertility}

This ritual creates an area of death and decay in a one mile radius zone from your location.
Normal plants within the area become half as productive as normal for the next year.
This ritual does not stack with itself.
If the \ritual{fertility} ritual is also applied to the same area, the most recently performed ritual takes precedence.

This ritual takes 24 hours to perform, and requires 9 action points from its participants.

\end{ability}




\parhead{Schools} Transmutation

\parhead{Spell Lists} Nature
\end{spellsection}


\begin{spellsection}{Ironwood}[3]


\begin{ability}{Ironwood}[\glossterm{Shaping}]

Choose a Small or smaller unattended, nonmagical wooden object within \rngclose range.
The target is transformed into ironwood.
While remaining natural wood in almost every way, ironwood is as strong, heavy, and resistant to fire as iron.
Metallic armor and weapons, such as full plate, can be crafted from ironwood.

% Should this have an action point cost? May be too rare...
This ritual takes 24 hours to perform.

\end{ability}




\parhead{Schools} Transmutation

\parhead{Spell Lists} Nature
\end{spellsection}


\begin{spellsection}{Lifeweb Transit}[5]


\begin{ability}{Lifeweb Transit}[\glossterm{Teleportation}]

This ritual functions like the \ritual{overland teleporation} ritual, except that both the starting and ending points must be living plants.
Both plants must be larger than the largest creature being teleported in this way.

\end{ability}




\parhead{Schools} Conjuration

\parhead{Spell Lists} Nature
\end{spellsection}


\begin{spellsection}{Light}[1]


\begin{ability}{Light}[\glossterm{Attune} (multiple), \glossterm{Figment}, \glossterm{Light}]

Choose a Medium or smaller willing creature or unattended object within \rngclose range.
The target glows like a torch, shedding bright light in an \areamed radius (and dim light for an additional 20 feet).

This ritual takes one minute to perform.

\end{ability}




\parhead{Schools} Illusion

\parhead{Spell Lists} Arcane, Divine, Nature
\end{spellsection}


\begin{spellsection}{Magic Mouth}[1]


\begin{ability}{Magic Mouth}[\glossterm{Attune} (multiple), \glossterm{Figment}]

Choose a Large or smaller willing creature or unattended object within \rngclose range.
In addition, choose a triggering condition and a message of twenty-five words or less.
The condition must be something that a typical human in the target's place could detect.

When the triggering condition occurs, the target appears to grow a magically animated mouth.
The mouth speaks the chosen message aloud.
After the message is spoken, this effect is \glossterm{dismissed}.

This ritual takes 24 hours to perform.

\end{ability}




\parhead{Schools} Illusion

\parhead{Spell Lists} Arcane
\end{spellsection}


\begin{spellsection}{Mount}[3]


\begin{ability}{Mount}[\glossterm{Attune} (multiple), \glossterm{Manifestation}]

This ritual creates your choice of a light horse or a pony to serve as a mount.
The creature appears in an unoccupied location within \rngclose range.
It comes with a bit and bridle and a riding saddle, and will readily accept any creature as a rider.

\end{ability}




\parhead{Schools} Conjuration

\parhead{Spell Lists} Arcane
\end{spellsection}


\begin{spellsection}{Mystic Lock}[2]


\begin{ability}{Mystic Lock}[\glossterm{Attune} (multiple)]

Choose a Large or smaller closable, nonmagical object within \rngclose range, such as a door or box.
The target object becomes magically locked.
It can be unlocked with a Devices check against a DR equal to 20 \add your spellpower.
The DR to break it open forcibly increases by 10.

You can freely pass your own \ritual{arcane lock} as if the object were not locked.
This effect lasts as long as you \glossterm{attune} to it.
If you use this ability multiple times, you can attune to it each time.

This ritual takes one minute to perform.

\end{ability}




\parhead{Schools} Transmutation

\parhead{Spell Lists} Arcane, Divine, Nature
\end{spellsection}


\subsubsection{Subrituals}


\begin{ability}[\nth{5}]{Resilient}
This subritual functions like the \ritual{mystic lock} ritual, except that the DR to unlock the target with a Devices check is instead equal to 30 + your spellpower.
In addition, the DR to break it open increases by 20 instead of by 10.
\end{ability}
\vspace{0.25em}


\begin{spellsection}{Overland Teleportation}[5]


\begin{ability}{Overland Teleportation}[\glossterm{Teleportation}]

Choose up to five willing, Medium or smaller ritual participants.
Choose a destination up to 100 miles away from you on your current plane.
Each target is teleproted to the chosen destination.

You must specify the destination with a precise mental image of its appearance.
The image does not have to be perfect, but it must unambiguously identify the destination.
If you specify its appearance incorrectly, or if the area has changed its appearance, the destination may be a different area than you intended.
The new destination will be one that more closely resembles your mental image.
If no such area exists, the ritual simply fails.
% TODO: does this need more clarity about what teleportation works?

This ritual takes 24 hours to perform and requires 25 action points from its ritual participants.

\end{ability}




\parhead{Schools} Conjuration

\parhead{Spell Lists} Arcane
\end{spellsection}


\begin{spellsection}{Plane Shift}[5]


\begin{ability}{Plane Shift}[\glossterm{Planar}, \glossterm{Teleportation}]

Choose up to five Medium or smaller willing ritual participants.
In addition, choose a plane that connects to your current plane and a location within that plane.
The targets teleport to a random location on that plane 1d100 miles away from the intended destination.

You must specify the destination with a precise mental image of its appearance.
The image does not have to be perfect, but it must unambiguously identify the location.
Incomplete or incorrect mental images may result in the ritual leading to an unintended destination within the same plane, or simply failing entirely.

% TODO: Is this planar cosmology correct?
The Astral Plane connects to every plane, but transit from other planes is usually more limited.
From the Material Plane, you can only reach the Astral Plane.

This ritual takes 24 hours to perform, and requires 25 action points from its participants.

\end{ability}




\parhead{Schools} Conjuration

\parhead{Spell Lists} Arcane, Divine, Nature
\end{spellsection}


\subsubsection{Subrituals}


\begin{ability}[\nth{8}]{Precise}
This subritual functions like the \ritual{plane shift} ritual, except that the actual destination is the same as the intended destination, rather than being a random distance away.
This ritual takes 24 hours to perform, and requires 64 action points from its participants.
\end{ability}
\vspace{0.25em}


\begin{spellsection}{Private Sanctum}[5]


\begin{ability}{Private Sanctum}[\glossterm{Mystic}]

This ritual creates a ward against any external perception in an \arealarge radius zone centered on your location.
This effect is permanent.
Everything in the area is completely imperceptible from outside the area.
Anyone observing the area from outside sees only a dark, silent void, regardless of darkvision and similar abilities.
In addition, all \glossterm{Scrying} effects fail to function in the area.
Creatures inside the area can see within the area and outside of it without any difficulty.

This ritual takes 24 hours to perform, and requires 25 action points from its participants.

\end{ability}




\parhead{Schools} Abjuration

\parhead{Spell Lists} Arcane
\end{spellsection}


\begin{spellsection}{Purge Curse}[3]


\begin{ability}{Purge Curse}[\glossterm{Mystic}]

Choose a willing creature within \rngclose range.
All curses affecting the target are removed.
This ritual cannot remove a curse that is part of the effect of an item the target has equipped.
However, it can allow the target to remove any cursed items it has equipped.

This ritual takes 24 hours to perform, and requires 9 action points from its participants.

\end{ability}




\parhead{Schools} Abjuration

\parhead{Spell Lists} Arcane, Divine, Nature
\end{spellsection}


\begin{spellsection}{Purify Sustenance}[1]


\begin{ability}{Purify Sustenance}[\glossterm{Shaping}]

All food and water in a single square within \rngclose range is purified.
Spoiled, rotten, poisonous, or otherwise contaminated food and water becomes pure and suitable for eating and drinking.
This does not prevent subsequent natural decay or spoiling.

This ritual takes one hour to perform.

\end{ability}




\parhead{Schools} Transmutation

\parhead{Spell Lists} Arcane, Divine, Nature
\end{spellsection}


\begin{spellsection}{Read Magic}[1]


\begin{ability}{Read Magic}[\glossterm{Attune}]

You gain the ability to decipher magical inscriptions that would otherwise be unintelligible.
This can allow you to read ritual books and similar objects created by other creatures.
After you have read an inscription in this way, you are able to read that particular writing without the use of this ritual.

This ritual takes one minute to perform.

\end{ability}




\parhead{Schools} Divination

\parhead{Spell Lists} Arcane, Divine, Nature
\end{spellsection}


\begin{spellsection}{Regeneration}[4]


\begin{ability}{Regeneration}[\glossterm{Flesh}]

Choose a willing creature within \rngclose range.
All of the target's hit points, \glossterm{subdual damage}, and \glossterm{vital damage} are healed.
In addition, any of the target's severed body parts or missing organs grow back by the end of the next round.

This ritual takes 24 hours to perform, and requires 16 action points from its participants.

\end{ability}




\parhead{Schools} Vivimancy

\parhead{Spell Lists} Divine, Nature
\end{spellsection}


\begin{spellsection}{Reincarnation}[5]


\begin{ability}{Reincarnation}[\glossterm{Creation}, \glossterm{Flesh}, \glossterm{Life}]

Choose one Diminuitive or larger piece of a humanoid corpse.
The target must have been part of the original creature's body at the time of death.
The creature the target corpse belongs to returns to life in a new body.
It must not have died due to old age.

This ritual creates an entirely new body for the creature's soul to inhabit from the natural elements at hand.
During the ritual, the body ages to match the age of the original creature at the time it died.
The creature has 0 hit points when it returns to life.

A reincarnated creature is identical to the original creature in all respects, except for its race.
The creature's race is replaced with a random race from \tref{Humanoid Reincarnations}.
Its appearance changes as necessary to match its new race, though it retains the general shape and distinguishing features of its original appearance.
The creature loses all attribute modifiers and abilities from its old race, and gains those of its new race.
However, its racial bonus feat and languages are unchanged.

A creature's soul naturally rejects being placed into a different body than its original home.
Until the creature is restored to its initial race, its maximum action points are reduced by 1.
This penalty does not stack if the creature is reincarnated multiple times.

Coming back from the dead is an ordeal.
All of the creature's action points and other daily abilities are expended when it returns to life.
In addition, its maximum action points are reduced by 1.
This penalty lasts for thirty days, or until the creature gains a level.
If this would reduce a creature's maximum action points below 0, the creature cannot be resurrected.

This ritual takes 24 hours to perform, and requires 25 action points from its participants.

\end{ability}




\parhead{Schools} Conjuration, Vivimancy

\parhead{Spell Lists} Nature
\end{spellsection}


\subsubsection{Subrituals}


\begin{ability}[\nth{7}]{Fated}
This subritual functions like the \ritual{reincarnation} ritual, except that the target is reincarnated as its original race instead of as a random race.
This ritual takes 24 hours to perform, and requires 49 action points from its participants.
\end{ability}
\vspace{0.25em}


\begin{dtable}
\lcaption{Humanoid Reincarnations}
\begin{dtabularx}{\columnwidth}{l X}
d\% & Incarnation \\
\bottomrule
01 & Bugbear \\
02-\minus13 & Dwarf \\
14-\minus25 & Elf \\
26 & Gnoll \\
27-\minus38 & Gnome \\
39-\minus42 & Goblin \\
43-\minus52 & Half-elf \\
53-\minus62 & Half-orc \\
63-\minus74 & Halfling \\
75-\minus89 & Human \\
90-\minus93 & Kobold \\
94 & Lizardfolk \\
95-\minus99 & Orc \\
100 & Other
\end{dtabularx}
\end{dtable}


\begin{spellsection}{Resurrection}[4]


\begin{ability}{Resurrection}[\glossterm{Flesh}, \glossterm{Life}]

Choose one intact humanoid corpse within \rngclose range.
The target returns to life.
It must not have died due to old age.

The creature has 0 hit points when it returns to life.
It is cured of all \glossterm{vital damage} and other negative effects, but the body's shape is unchanged.
Any missing or irreparably damaged limbs or organs remain missing or damaged.
The creature may therefore die shortly after being resurrected if its body is excessively damaged.

Coming back from the dead is an ordeal.
All of the creature's action points and other daily abilities are expended when it returns to life.
In addition, its maximum action points are reduced by 1.
This penalty lasts for thirty days, or until the creature gains a level.
If this would reduce a creature's maximum action points below 0, the creature cannot be resurrected.

This ritual takes 24 hours to perform, and requires 16 action points from its participants.

\end{ability}




\parhead{Schools} Conjuration, Vivimancy

\parhead{Spell Lists} Nature
\end{spellsection}


\subsubsection{Subrituals}


\begin{ability}[\nth{7}]{Complete}
This subritual functions like the \ritual{resurrection} ritual, except that it does not have to target a fully intact corpse.
Instead, it targets a Diminuitive or larger piece of a humanoid corpse.
The target must have been part of the original creature's body at the time of death.
The resurrected creature's body is fully restored to its healthy state before dying, including regenerating all missing or damaged body parts.

This ritual takes 24 hours to perform, and requires 49 action points from its participants.
\end{ability}
\vspace{0.25em}


\begin{spellsection}{Retrieve Legacy}[4]


\begin{ability}{Retrieve Legacy}[\glossterm{Teleportation}]

Choose a willing creature within \rngclose range.
If the target's \glossterm{legacy item} is on the same plane and \glossterm{unattended}, it is teleported into the target's hand.

This ritual takes 24 hours to perform, and requires 16 action points from its ritual participants.

\end{ability}




\parhead{Schools} Conjuration, Divination

\parhead{Spell Lists} Arcane, Divine, Nature
\end{spellsection}


\begin{spellsection}{Scryward}[3]


\begin{ability}{Scryward}[\glossterm{Mystic}]

This ritual creates a ward against scrying in an \arealarge radius zone centered on your location.
All \glossterm{Scrying} effects fail to function in the area.
This effect is permanent.

This ritual takes 24 hour to perform, and requires 9 action points from its participants.

\end{ability}




\parhead{Schools} Abjuration

\parhead{Spell Lists} Arcane, Divine, Nature
\end{spellsection}


\begin{spellsection}{Seek Legacy}[2]


\begin{ability}{Seek Legacy}

Choose a willing creature within \rngclose range.
The target learns the precise distance and direction to their \glossterm{legacy item}, if it is on the same plane.

This ritual takes 24 hours to perform.

\end{ability}




\parhead{Schools} Divination

\parhead{Spell Lists} Arcane, Divine, Nature
\end{spellsection}


\begin{spellsection}{Sending}[4]


\begin{ability}{Sending}[\glossterm{Sustain} (standard)]

Choose a creature on the same plane as you.
You do not need \glossterm{line of sight} or \glossterm{line of effect} to the target.
However,  must specify your target with a precise mental image of its appearance.
The image does not have to be perfect, but it must unambiguously identify the target.
If you specify its appearance incorrectly, or if the target has changed its appearance, you may accidentally target a different creature, or the spell may simply fail.

You send the target a short verbal message.
The message must be twenty-five words or less, and speaking the message must not take longer than five rounds.

After the the target receives the message, it may reply with a message of the same length as long as the ritual's effect continues.
Once it speaks twenty-five words, or you stop sustaining the effect, the ritual is \glossterm{dismissed}.

This ritual takes one hour to perform.

\end{ability}




\parhead{Schools} Divination

\parhead{Spell Lists} Arcane, Divine, Nature
\end{spellsection}


\subsubsection{Subrituals}


\begin{ability}[\nth{6}]{Interplanar}
This subritual functions like the \ritual{sending} ritual, except that the target does not have to be on the same plane as you.
It gains the \glossterm{Planar} tag in addition to the tags from the \ritual{sending} ritual.
\end{ability}
\vspace{0.25em}


\begin{spellsection}{Soul Bind}[8]


\begin{ability}{Soul Bind}[\glossterm{Life}]

Choose one intact corpse within \rngclose range.
In addition, choose a gem you hold that is worth at least 1,000 gp.
The soul of the creature that the target corpse belongs to is imprisoned in the chosen gem.
This prevents the creature from being resurrected, and prevents the corpse from being used to create undead creatures, as long as the gem is intact.
A creature holding the gem may still resurrect or reanimate the creature.

This ritual takes one hour to perform.

\end{ability}




\parhead{Schools} Vivimancy

\parhead{Spell Lists} Arcane, Divine
\end{spellsection}


\begin{spellsection}{Telepathic Bond}[4]


\begin{ability}{Telepathic Bond}[\glossterm{Attune} (shared)]

Choose up to five willing ritual participants.
Each target can communicate mentally through telepathy with each other target.
This communication is instantaneous across any distance, but cannot reach across planes.

% Is this grammatically correct?
This effect lasts as long as you and each target \glossterm{attune} to it.
Each target must attune to this ritual independently.
If a target breaks its attunement, it stops being able to send and receive mental messages with other targets.
However, the effect continues as long as you attune to it.
If you stop attuning to it, the ritual is \glossterm{dismissed} as usual.

This ritual takes 24 hours to perform.

\end{ability}




\parhead{Schools} Divination

\parhead{Spell Lists} Arcane
\end{spellsection}


\subsubsection{Subrituals}


\begin{ability}[\nth{8}]{Interplanar}
This subritual functions like the \ritual{telepathic bond} ritual, except that each target can communicate telepathically even across different planes.
It gains the \glossterm{Planar} tag in addition to the tags from the \ritual{telepathic bond} ritual.
\end{ability}
\vspace{0.25em}


\begin{spellsection}{Water Breathing}[2]


\begin{ability}{Water Breathing}[\glossterm{Attune} (multiple)]

Choose a Medium or smaller willing creature within \rngclose range.
The target can breathe water as easily as a human breathes air, preventing it from drowning or suffocating underwater.
This effect lasts as long as you \glossterm{attune} to it.
If you use this ability multiple times, you can attune to it each time.

This ritual takes one minute to perform.

\end{ability}




\parhead{Schools} Transmutation

\parhead{Spell Lists} Arcane, Divine, Nature
\end{spellsection}

