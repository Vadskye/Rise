\chapter{Equipment}

\section{Weapons}\label{Weapons}

    Weapons are grouped into several interlocking sets of categories. These categories pertain to what weapon group the weapon belongs to (axes, bows, and so on), the weapon's usefulness either in close combat (melee) or at a distance (ranged, which includes both thrown and projectile weapons), and its relative \glossterm{usage class} (light, medium, or heavy).

    \subsection{Weapon Groups}\label{Weapon Groups}

        Weapons are organized into thematically related categories called weapon groups. They are described in \trefnp{Weapon Groups}. For example, all axes belong to the ``axes'' weapon group. Some weapons can be found in multiple weapon groups. For example, a dagger is a simple weapon, a blade, and a thrown weapon.

        \parhead{Exotic Weapons}\label{Exotic Weapons} Some weapons are rare and have unusual fighting styles.
        These weapons are called exotic weapons.
        Proficiency with a weapon group does not grant you with exotic weapons from that group.
        Some class abilities grant proficiency with exotic weapons.

        \begin{dtable!*}
            \lcaption{Weapon Groups}
            \begin{dtabularx}{\textwidth}{l >{\lcol}X >{\lcol}X}
                \tb{Group}        & \tb{Weapons}                                                         & \tb{Exotic Weapons} \tableheaderrule
                Armor weapons     & Standard shield (and spiked), spiked armor                           &                                 \\
                Axes              & Battleaxe, greataxe, handaxe, throwing axe                           & Double axe, dwarven waraxe      \\
                Blades            & Greatsword, longsword, rapier, short sword, dagger                   & Katana, kukri, two-bladed sword \\
                Bows              & Longbow, shortbow                                                    &                                 \\
                Club-like weapons & Club, greatclub, mace, morningstar, sap                              &                                 \\
                Crossbows         & Hand crossbow, heavy crossbow, light crossbow                        & Repeating crossbows             \\
                Flexible Weapons  & Flail, heavy flail, whip                                             &                                 \\
                Headed weapons    & Greathammer, Heavy pick, light hammer, light pick, sickle, warhammer &                                 \\
                Monk weapons      & Kama, nunchaku, quarterstaff, sai, shuriken, siangham                &                                 \\
                Polearms          & Glaive, guisarme, halberd, ranseur, scythe                           &                                 \\
                Simple weapons    & Club, dagger, light crossbow, quarterstaff, unarmed strike           &                                 \\
                Spears            & Javelin, lance, longspear, shortspear, spear                         & Greatspear, pike                \\
                Thrown weapons    & Dagger, dart, handaxe, javelin, light hammer, shuriken, sling        & Bolas, net                      \\
            \end{dtabularx}
        \end{dtable!*}

        \subsubsection{Weapon Proficiency}\label{Weapon Proficiency}
            Each character is proficient with different weapon groups. These indicate the weapons that you can use effectively. You take a \minus2 penalty to accuracy with weapons you are not proficient with, and you cannot use them to defend yourself, which can cause you to be \defenseless.

    \subsection{Weapon Usage Classes}\label{Weapon Usage Classes}
        A weapon's \glossterm{usage class} is a measure of how much effort it takes to wield a weapon in combat.
        It indicates whether a melee weapon, when wielded by a creature the weapon is sized for, is considered a light weapon, a medium weapon, or a heavy weapon.

        \parhead{Light Weapons}\label{Light Weapons} Light weapons are easier to use while making attacking with two weapons at once (see \pcref{Two-Weapon Strikes}) or while grappling.
        They cannot be held in two hands.
        Light weapons tend to have higher \glossterm{accuracy} than heavier weapons.

        \parhead{Medium Weapon} A medium weapon can be used in one hand. You can also hold a medium weapon in two hands. Changing grips to hold it in one hand or two hands can be done as a move action.

        \parhead{Heavy Weapon} Two hands are required to wield a heavy weapon.
        You can hold it in one hand, but while doing so you cannot attack or defend yourself with it.
        This usually causes you to be \defenseless.
        Changing grips to hold it in one hand or two hands is a move action.
        Heavy weapons tend to have higher damage than lighter weapons.

        % TODO: explain how double weapons work

    \subsection{Melee Weapons}
        Melee weapons are used for making melee attacks against foes within your reach (see \pcref{Reach}). Most weapons are melee weapons.

    \subsection{Ranged Weapons}
        Ranged weapons are used to make attacks against distant foes within the range of your weapon. There are two kinds of ranged weapons: projectile weapons and thrown weapons.

        \subsubsection{Range Increments}\label{Range Increment} All ranged weapons have a ``range increment'', which indicates the distance at which you can effectively attack with the weapon. Any attack against a target within your range increment takes no penalty. For each range increment by which the target is distant from you, you take a cumulative \minus1 penalty to your accuracy. A thrown weapon has a maximum range of five range increments. A projectile weapon can shoot out to ten range increments.

        \subsubsection{Projectile Weapons} Projectile weapons fire ammunition at a target to deal damage. The ammunition generally breaks when used.

        \subsubsection{Thrown Weapons}\label{Thrown Weapons} Thrown weapons are thrown at a target to deal damage. They generally do not break when thrown.

            \parhead{Throwing Other Weapons} It is possible to throw a weapon that is not designed to be thrown. The range increment is 10 feet. You are treated as being nonproficient with the weapon. If the attack hits, the weapon deals its normal damage.

            \parhead{Heavy Weapons} Heavy thrown weapons require a standard action to throw, rather than an attack action like normal. They are not always physically thrown with two hands, but they require the use of your entire body to propel the weapon, preventing you from using your hands for anything else. This can cause you to be \defenseless.

        \subsubsection{Ranged Weapons in Melee}

            You take a \minus4 penalty to accuracy with medium and large ranged weapons against creatures adjacent to you. In addition, you are usually \defenseless while using ranged weapons.

            \parhead{Thrown Weapons in Melee}\label{Thrown Weapons in Melee} You cannot defend yourself with a weapon you are throwing, which can cause you to be \defenseless. To avoid this, you can attempt to use the weapon as a melee weapon. Some thrown weapon are not designed for use in melee, such as shurikens. When using such a weapon as a melee weapon, you are always treated as nonproficient with it, which means you cannot defend yourself with it.

            \parhead{Physical Size} Like all objects and creatures, weapons have a size category that represents how physically large they are. In general, a light weapon is an object two size categories smaller than the wielder, a medium weapon is an object one size category smaller than the wielder, and a heavy weapon is an object of the same size category as the wielder.

            \parhead{Inappropriately Sized Weapons}\label{Inappropriately Sized Weapons} A weapon's \glossterm{usage class} is altered by one step for each size category of difference between the wielder's size and the size of the creature for which the weapon was designed.
            For example, a light weapon sized for a Large creature can be used by a Medium creature as if it had a medium usage class.
            The wielder also gains a \plus1d bonus to damage with the weapon per size category if the weapon is unusually large, or a \minus1d penalty to damage with the weapon if the weapon is unusually small.
            In addition, the wielder takes a \minus2 penalty to accuracy with the weapon per size category of difference.
            If a weapon's usage class would be changed to something other than light, medium, or heavy by this alteration, the creature can't wield the weapon at all.

    \subsection{Improvised Weapons}\label{Improvised Weapons} Sometimes objects not crafted to be weapons nonetheless see use in combat. Because such objects are not designed for this use, any creature that uses one in combat is considered to be nonproficient with it.

        To determine the size category and appropriate damage for an improvised weapon, compare its relative size and damage potential to the weapon list to find a reasonable match. An improvised weapon will generally deal damage as if it were one size category smaller than a similar manufactured weapon. An improvised thrown weapon has a range increment of 10 feet.

    \subsection{Drawing and Sheathing Weapons}\label{Drawing and Sheathing Weapons}
        Sheathing any weapon is a move action.
        The time it takes to draw a weapon depends on how encumbering the weapon is.
        You can draw a \glossterm{light weapon} as a \glossterm{free action} once per round.
        You can draw any non-hidden weapon as a \glossterm{move action}.
        You can draw a hidden weapon as a \glossterm{standard action}.

    \subsection{Natural Weapons}\label{Natural Weapons}
        Every creature can attack with its body using an unarmed strike. Many monsters can attack more effectively with specific parts of their body, such as teeth and claws. These are called natural weapons. You are automatically proficient with any natural weapons you possess. Most humanoids possess no natural weapons. Natural weapons are described on \tref{Natural Weapons}.

        A creature with multiple natural weapons of the same type can designate one as a primary attack and one as a secondary attack, allowing it to fight with both at once (see \pcref{Two-Weapon Strikes}). You are only considered to have one unarmed strike, so you cannot dual wield with only your unarmed strike (but see the unarmed warrior monk ability, \pref{Mnk:Unarmed Warrior}).

    \subsection{Weapon Qualities}
        Here is the format for weapon entries (given as column headings on \trefnp{Weapons}, below).

        \parhead{Usage Class} Describes whether the weapon is a light, medium, heavy, double, or ranged weapon.

        \par This cost is the same for a Small or Medium version of the weapon. A Large version costs twice the listed price.
        \parhead{Damage} The Damage column gives the damage dealt by the weapon on a successful hit.
        These values are accurate for creatures using weapons sized appropriately for them.
        For details about using weapons of other sizes, see \pcref{Inappropriately Sized Weapons}.

        If two damage ranges are given, the weapon is a double weapon. Use the second damage figure given for the double weapon's other side.

        \parhead{Range Increment} The range increment of the weapon.

        \parhead{Type} Weapons are classified according to the type of damage they deal: bludgeoning, piercing, or slashing. Some monsters may be resistant or immune to attacks from certain types of weapons.

        Some weapons deal damage of multiple types. If a weapon is of two types, the damage it deals is not half one type and half another; all of it is both types. Therefore, a creature would have to be immune to both types of damage to ignore any of the damage from such a weapon.

        In other cases, a weapon can deal either of two types of damage. In a situation when the damage type is significant, the wielder can choose which type of damage to deal with such a weapon.
        \parhead{Cost} This value is the weapon's cost in gold pieces (gp) or silver pieces (sp). The cost includes miscellaneous gear that goes with the weapon.
        \parhead{Weight} This column gives the weight of a Medium version of
        the weapon. Halve this number for Small weapons, and double it for Large weapons.

        \parhead{Special} Some weapons have special properties. See the weapon
        descriptions for details.

        \begin{longtabuwrapper}
            \begin{longtabu}{p{12em} c c c >{\ccol}p{10em} c c >{\ccol}X}
                \lcaption{Weapons}\\
                \tb{Name}                              & \tb{Usage Class} & \tb{Accuracy} & \tb{Damage\fn{1}} & \tb{Damage Type\fn{2}}   & \tb{Cost} & \tb{Weight\fn{3}} & \tb{Special}                 \\
                Armor weapons\label{Armor Weapons}     &                 &         &                   &                          &         &         &                              \\
                \tind Shield, standard\fn{4}           & Medium          & \plus0  & \minus2d          & Bludgeoning              & special & special & Forceful                     \\
                \tind Spiked armor\fn{4}               & Medium          & \plus0  & \minus1d          & Piercing                 & special & special & Grappling                    \\
                \tind Spiked shield, standard\fn{4}    & Medium          & \plus0  & \minus1d          & Piercing                 & special & special & Forceful                     \\

                Axes                                   &                 &         &                   &                          &         &         &                              \\
                \tind Battleaxe                        & Medium          & \plus0  & \plus0d           & Slashing                 & 10 gp   & 6 lb.   & \tdash                       \\
                \tind Greataxe                         & Heavy           & \plus0  & \plus1d           & Slashing                 & 20 gp   & 12 lb.  & \tdash                       \\
                \tind Handaxe                          & Light           & \plus1  & \minus1d          & Slashing                 & 8 gp    & 2 lb.   & Throwing (10 ft.)            \\

                Blades                                 &                 &         &                   &                          &         &         &                              \\
                \tind Greatsword                       & Heavy           & \plus0  & \plus1d           & Slashing                 & 25 gp   & 8 lb.   & \tdash                       \\
                \tind Longsword                        & Medium          & \plus0  & \plus0d           & Slashing                 & 15 gp   & 4 lb.   & \tdash                       \\
                \tind Dagger                           & Light           & \plus2  & \minus2d          & Piercing or slashing     & 2 gp    & 1 lb.   & Compact, Throwing (10 ft.)   \\
                \tind Rapier                           & Medium          & \plus1  & \minus1d          & Piercing                 & 20 gp   & 2 lb.   & Disarming                    \\
                \tind Short sword                      & Light           & \plus1  & \minus1d          & Piercing or slashing     & 10 gp   & 2 lb.   & \tdash                       \\

                Bows                                   &                 &         &                   &                          &         &         &                              \\
                \tind Longbow\fn{4}                    & Heavy (Ranged)  & \plus0  & \plus0            & \tdash                   & 40 gp   & 3 lb.   & Projectile (100 ft.)         \\
                \tind Shortbow\fn{4}                   & Medium (Ranged) & \plus0  & \plus0            & \tdash                   & 30 gp   & 2 lb.   & Projectile (50 ft.)          \\
                \tind Arrows (20)                      & \tdash          & \plus0  & \plus0            & Piercing                 & 1 gp    & 3 lb.   & Ammunition                   \\
                \tind Blunted arrows (20)              & \tdash          & \plus0  & \minus1d          & Bludgeoning              & 5 gp    & 3 lb.   & Ammunition, Subdual          \\
                \tind Fire arrows (20)\fn{4}           & \tdash          & \minus1 & 0                 & Piercing and fire        & 100 gp  & 3 lb.   & Ammunition                   \\

                Club-like weapons                      &                 &         &                   &                          &         &         &                              \\
                \tind Club                             & Medium          & \minus1 & \plus0d           & Bludgeoning              & \tdash  & 3 lb.   & \tdash                       \\
                \tind Greatclub                        & Heavy           & \minus1 & \plus1d           & Bludgeoning              & 5 gp    & 8 lb.   & \tdash                       \\
                \tind Mace                             & Light           & \plus1  & \minus1d          & Bludgeoning              & 12 gp   & 4 lb.   & \tdash                       \\
                \tind Morningstar                      & Medium          & \plus0  & \plus0d           & Bludgeoning and piercing & 8 gp    & 6 lb.   & \tdash                       \\
                \tind Sap                              & Light           & \plus1  & \minus1d          & Bludgeoning              & 1 gp    & 2 lb.   & Subdual                      \\

                Crossbows                              &                 &         &                   &                          &         &         &                              \\
                \tind Hand crossbow\fn{4}             & Light (Ranged)  & \plus0  & \minus1d          & Piercing                 & 100 gp  & 2 lb.   & Projectile (50 ft.)          \\
                \tind Heavy crossbow\fn{4}            & Heavy (Ranged)  & \plus0  & \plus1d           & Piercing                 & 50 gp   & 8 lb.   & Projectile (100 ft.)         \\
                \tind Light crossbow\fn{4}            & Medium (Ranged) & \plus0  & \plus0d           & Piercing                 & 40 gp   & 4 lb.   & Projectile (50 ft.)          \\
                \tind Crossbow bolts (10)             & \tdash          & \plus0  & \plus0            & \tdash                   & 1 gp    & 1 lb.   & Ammunition                   \\
                \tind Hand crossbow bolts (10)        & \tdash          & \plus0  & \plus0            & \tdash                   & 2 gp    & 1/2 lb. & Ammunition                   \\

                Flexible weapons                       &                 &         &                   &                          &         &         &                              \\
                \tind Flail                            & Medium          & \minus1 & \plus0d           & Bludgeoning              & 8 gp    & 5 lb.   & Tripping                     \\
                \tind Heavy flail                      & Heavy           & \minus1 & \plus1d           & Bludgeoning              & 15 gp   & 10 lb.  & Tripping                     \\
                \tind Whip\fn{4}                       & Light           & \plus1  & \minus3d          & Slashing                 & 1 gp    & 2 lb.   & Disarming, Subdual, Tripping \\

                Headed weapons                         &                 &         &                   &                          &         &         &                              \\
                \tind Greathammer                      & Heavy           & \plus0  & \plus1d           & Bludgeoning              & 12 gp   & 5 lb.   & \tdash                       \\
                \tind Light hammer                     & Light           & \plus1  & \minus2d          & Bludgeoning              & 1 gp    & 2 lb.   & Throwing (20 ft.)            \\
                \tind Pickaxe                          & Heavy           & \minus2 & \plus2d           & Piercing                 & 8 gp    & 6 lb.   & \tdash                       \\
                \tind Sickle                           & Light           & \plus1  & \minus1d          & Slashing                 & 6 gp    & 2 lb.   & Tripping                     \\
                \tind Warhammer                        & Medium          & \plus0  & \plus0d           & Bludgeoning              & 12 gp   & 5 lb.   & \tdash                       \\

                Monk weapons                           &                 &         &                   &                          &         &         &                              \\
                \tind Kama                             & Light           & \plus1  & \minus1d          & Slashing                 & 2 gp    & 2 lb.   & Tripping                     \\
                \tind Nunchaku                         & Light           & \plus1  & \minus1d          & Bludgeoning              & 2 gp    & 2 lb.   & Disarming                    \\
                \tind Quarterstaff                     & Heavy           & \plus0  & \minus1d/\minus1d & Bludgeoning              & \tdash  & 4 lb.   & Double                       \\
                \tind Sai                              & Light           & \plus1  & \minus2d          & Piercing or bludgeoning  & 1 gp    & 1 lb.   & Disarming                    \\
                \tind Shuriken (5)                     & Light (Ranged)  & \plus1  & \minus2d          & Piercing and slashing    & 1 gp    & 1/2 lb. & Ammunition, Thrown (10 ft.)  \\
                \tind Siangham                         & Light           & \plus1  & \minus1d          & Piercing                 & 3 gp    & 1 lb.   & \tdash                       \\

                Polearms                               &                 &         &                   &                          &         &         &                              \\
                \tind Glaive                           & Heavy           & \plus0  & \plus1d           & Slashing                 & 8 gp    & 10 lb.  & Reach                        \\
                \tind Guisarme                         & Heavy           & \plus0  & \plus1d           & Slashing                 & 9 gp    & 12 lb.  & Reach, Tripping              \\
                \tind Halberd                          & Heavy           & \plus0  & \plus1d           & Piercing or slashing     & 10 gp   & 12 lb.  & Reach                        \\
                \tind Quarterstaff                     & Heavy           & \plus0  & \minus1d/\minus1d & Bludgeoning              & \tdash  & 4 lb.   & Double                       \\
                \tind Ranseur                          & Heavy           & \plus0  & \plus1d           & Piercing                 & 10 gp   & 12 lb.  & Disarming, Reach             \\
                \tind Scythe                           & Heavy           & \plus0  & \plus1d           & Slashing                 & 18 gp   & 10 lb.  & \tdash                       \\

                Simple weapons                         &                 &         &                   &                          &         &         &                              \\
                \tind Club                             & Medium          & \plus0  & \minus1d          & Bludgeoning              & \tdash  & 3 lb.   & \tdash                       \\
                \tind Dagger                           & Light           & \plus2  & \minus2d          & Piercing or slashing     & 2 gp    & 1 lb.   & Compact, Thrown (10 ft.)     \\
                \tind Light crossbow\fn{4}            & Medium (Ranged) & \plus0  & \plus0d           & Piercing                 & 40 gp   & 4 lb.   & Projectile (50 ft.)          \\
                \tind Quarterstaff                     & Heavy           & \plus0  & \minus1d/\minus1d & Bludgeoning              & \tdash  & 4 lb.   & Double                       \\
                \tind Unarmed strike                   & Light           & \plus0  & \minus3d          & Bludgeoning              & \tdash  & \tdash  & Subdual, Unarmed             \\

                Spears                                 &                 &         &                   &                          &         &         &                              \\
                \tind Javelin                          & Medium (Ranged) & \plus0  & \minus1d          & Piercing                 & 1 gp    & 2 lb.   & Thrown (30 ft.)              \\
                \tind Lance                            & Heavy           & \plus0  & \plus0d           & Piercing                 & 10 gp   & 10 lb.  & Mounted, Reach               \\
                \tind Longspear                        & Heavy           & \plus0  & \plus0d           & Piercing                 & 5 gp    & 9 lb.   & Bracing, Reach               \\
                \tind Spear                            & Medium          & \plus0  & \minus1d          & Piercing                 & 2 gp    & 6 lb.   & Bracing, Thrown (10 ft.)     \\

                Thrown weapons                         &                 &         &                   &                          &         &         &                              \\
                \tind Dagger                           & Light           & \plus2  & \minus2d          & Piercing or slashing     & 2 gp    & 1 lb.   & Compact, Thrown (10 ft.)     \\
                \tind Dart (5)                         & Light (Ranged)  & \plus1  & \minus2d          & Piercing                 & 1 gp    & 1/2 lb. & Ammunition, Thrown (20 ft.)  \\
                \tind Handaxe                          & Light           & \plus1  & \minus1d          & Slashing                 & 8 gp    & 2 lb.   & Thrown (10 ft.)              \\
                \tind Light hammer                     & Light           & \plus1  & \minus2d          & Bludgeoning              & 1 gp    & 2 lb.   & Thrown (20 ft.)              \\
                \tind Javelin                          & Medium (Ranged) & \plus0  & \minus1d          & Piercing                 & 1 gp    & 2 lb.   & Thrown (30 ft.)              \\
                \tind Shuriken\fn{4} (5)               & Light (Ranged)  & \plus2  & \minus2d          & Piercing and slashing    & 1 gp    & 1/2 lb. & Ammunition, Thrown (10 ft.)  \\
                \tind Sling\fn{4}                      & Light (Ranged)  & \plus0  & \minus1d          & Bludgeoning              & 2 gp    & 0 lb.   & Projectile (50 ft.)          \\
                \tind Bullets, sling (20)              & \tdash          & \tdash  & \tdash            & \tdash                   & 1 gp    & 5 lb.   & Ammunition                   \\

                Unarmed weapons\label{Unarmed Weapons} &                 &         &                   &                          &         &         &                              \\
                \tind Claw sheath\fn{4}                & \tdash          & \tdash  & \tdash            & \tdash                   & 50 gp   & 3 lb.   & Special                      \\
                \tind Gauntlet                         & Light           & \plus2  & \minus3d          & Bludgeoning              & 2 gp    & 1 lb.   & Unarmed                      \\
                \tind Gauntlet, spiked                 & Light           & \plus2  & \minus3d          & Piercing                 & 5 gp    & 1 lb.   & Unarmed                      \\
                \tind Unarmed strike                   & Light           & \plus2  & \minus3d          & Bludgeoning              & \tdash  & \tdash  & Subdual, Unarmed             \\
            \end{longtabu}
            1 Applies as a modifier to your damage with \glossterm{strikes} using the weapon. \\
            2 When two types are given, the weapon is both types if the entry specifies ``and,'' or either type (attacker's choice) if the entry specifies ``or.'' \\
            3 Weight figures are for Medium weapons. A Small weapon weighs half as much, and a Large weapon weighs twice as much. \\
            4 This weapon has special rules. \\
        \end{longtabuwrapper}

        \begin{dtable!*}
            \begin{dtabularx}{\textwidth}{p{12em} c c c >{\ccol}p{10em} c c >{\ccol}X}
                \tb{Exotic Weapons}                  & \tb{Usage Class} & \tb{Accuracy} & \tb{Damage\fn{1}} & \tb{Damage Type\fn{2}} & \tb{Cost} & \tb{Weight\fn{3}} & \tb{Special} \tableheaderrule
                Armor                                &                  &               &                   &                        &           &                   &                                \\
                Axes                                 &                  &               &                   &                        &           &                   &                                \\
                \tind Dwarven waraxe                 & Medium           & \plus0        & \plus0d           & Slashing               & 75 gp     & 8 lb.             & Thrown (20 ft.)                \\
                \tind Orc double axe                 & Heavy            & \plus0        & \plus0d/\plus0d   & Slashing               & 60 gp     & 15 lb.            & Double                         \\
                Blades                               &                  &               &                   &                        &           &                   &                                \\
                \tind Katana                         & Medium           & \plus1        & \plus0d           & Slashing               & 75 gp     & 6 lb.             & \\
                \tind Kukri                          & Light            & \plus2        & \minus1d          & Slashing               & 8 gp      & 2 lb.             &                                \\
                \tind Two-bladed sword               & Heavy            & \plus0        & \plus0d/\plus0d   & Slashing               & 100 gp    & 10 lb.            & Double                         \\
                Bows                                 &                  &               &                   &                        &           &                   &                                \\
                \tind Flatbow                        & Heavy (Ranged)   & \plus1        & \plus0d           & \tdash                 & 30 gp     & 3 lb.             & Projectile (100 ft.)           \\
                \tind Takedown bow\fn{4}             & Special (Ranged) & \plus0        & \plus0d           & \tdash                 & 200 gp    & 4 lb.             & Projectile (100 ft. or 50 ft.) \\
                Club-like weapons                    &                  &               &                   &                        &           &                   &                                \\
                Crossbows                            &                  &               &                   &                        &           &                   &                                \\
                \tind Repeating heavy crossbow\fn{4} & Heavy (Ranged)   & \plus0        & \plus1d           & Piercing               & 400 gp    & 12 lb.            & Projectile (100 ft.)           \\
                \tind Repeating light crossbow\fn{4} & Medium (Ranged)  & \plus0        & \plus0d           & Piercing               & 250 gp    & 6 lb.             & Projectile (50 ft.)            \\
                \tind Repeating bolts (5)            & \tdash           & \plus0        & \plus0d           & \tdash                 & 1 gp      & 1 lb.             & Ammunition                     \\
                Flexible weapons                     &                  &               &                   &                        &           &                   &                                \\
                Headed weapons                       &                  &               &                   &                        &           &                   &                                \\
                Monk weapons                         &                  &               &                   &                        &           &                   &                                \\
                Polearms                             &                  &               &                   &                        &           &                   &                                \\
                Simple weapons                       &                  &               &                   &                        &           &                   &                                \\
                Spear                                &                  &               &                   &                        &           &                   &                                \\
                \tind Greatspear                     & Heavy            & \plus0        & \plus1d           & Piercing               & 15 gp     & 12 lb.            & Bracing                        \\
                \tind Pike\fn{4}                     & Heavy            & \plus0        & \plus0d           & Piercing               & 15 gp.    & 10 lb.            & Reach                          \\
                Thrown weapons                       &                  &               &                   &                        &           &                   &                                \\
                \tind Bolas                          & Light (Ranged)   & \plus2        & \minus2d\fn{4}    & Bludgeoning            & 5 gp      & 2 lb.             & Thrown (10 ft.), Tripping      \\
                \tind Net\fn{4}                      & Medium (Ranged)  & \plus0        & \tdash            & \tdash                 & 20 gp     & 6 lb.             & Thrown (10 ft.)                \\
                Unarmed weapons                      &                  &               &                   &                        &           &                   &                                \\
            \end{dtabularx}
            1 Applies as a modifier to your damage with \glossterm{strikes} using the weapon.
            2 When two types are given, the weapon is both types if the entry specifies ``and,'' or either type (attacker's choice) if the entry specifies ``or.'' \\
            3 Weight figures are for Medium weapons. A Small weapon weighs half as much, and a Large weapon weighs twice as much. \\
            4 This weapon has special rules. \\
        \end{dtable!*}

        \begin{dtable!*}
            \lcaption{Natural Weapons}
            \begin{dtabularx}{\textwidth}{p{12em} c c c >{\ccol}p{15em} >{\ccol}X}
                \tb{Natural Weapons} & \tb{Usage Class} & \tb{Accuracy} & \tb{Damage} & \tb{Damage Type\fn{2}} & \tb{Special} \tableheaderrule
                Bite            & Medium & \plus0 & \plus0d  & Piercing and bludgeoning & \tdash    \\
                Claw            & Light  & \plus1 & \plus0d  & Slashing and piercing    & \tdash    \\
                Constrict\fn{2} & Heavy  & \plus0 & \plus1d  & Bludgeoning              & Grappling \\
                Ram             & Heavy  & \plus0 & \plus0d  & Bludgeoning              & Forceful  \\
                Slam            & Medium & \plus0 & \plus0d  & Bludgeoning              & \tdash    \\
                Talon           & Light  & \plus1 & \minus1d & Piercing                 & \tdash    \\
                Unarmed Strike  & Light  & \plus2 & \minus3d & Bludgeoning              & \tdash    \\
            \end{dtabularx}
            1 When two types are given, the weapon is both types if the entry specifies ``and,'' or either type (attacker's choice) if the entry specifies ``or''. \\
            2 This attack can only be used against a foe you are grappling with. \\
        \end{dtable!*}

    \subsection{Weapon Tags}\label{Weapon Tags}
        Some weapons found on \trefnp{Weapons} have tags that indicate that they have special abilities. The list of abilities that weapons can have is given below.
        \parhead{Ammunition} This weapon is designed to thrown or fired by a projectile weapon in large quantities. It is cheaper to buy and craft, but ammunition is usually \glossterm{broken} after being fired.
        % Timing is weird
        \parhead{Bracing} As a \glossterm{move action}, you can brace this weapon against a charge until the end of the round.
        While you are bracing your weapon, you gain a \plus2d bonus to damage on attacks with that weapon against creatures that use the \textit{charge} ability to move up to you that round.
        \parhead{Compact} This weapon is unusually small. It is one size category smaller than normal for a light weapon (that is, three size categories smaller than the creature it is intended for). This makes it easier to conceal (see \pcref{Sleight of Hand}).
        \parhead{Disarming} You gain a \plus2 bonus to accuracy when you use the \textit{disarm} ability to attack with this weapon.
        \parhead{Double} This weapon has more than one striking surface. You can fight with both ends simultaneously, just like wielding two weapons at once (see \pcref{Two-Weapon Strikes}). Alternately, you can attack with one end at a time. If you have the ability to use a double weapon in one hand, you can only fight with one end at a time, not both.
        \parhead{Forceful} You use the \textit{shove} ability with this weapon instead of with a free hand (see \pcref{Shove}).
        That does not cause the shove attack to deal damage, but it means you do not need a free hand to use the ability.
        \parhead{Grappling} You gain a \plus2 bonus to \glossterm{accuracy} on \glossterm{melee attacks} with this weapon against creatures who are \glossterm{grappled} by you.
        In addition, it is treated as a light weapon for the purpose of determining whether you take accuracy penalties when attacking with it in a grapple.
        \parhead{Mounted}\label{Mounted Weapon} If you are mounted, and mount moves in the same phase that you make a \glossterm{strike} with a Mounted weapon, you gain a \plus1d bonus to damage with the strike.
        You take a \minus1d penalty to damage with Mounted weapons while not mounted.
        \parhead{Projectile} This weapon fires projectiles at range. Projectile weapons have a \glossterm{range increment} listed in their description, which indicates the distance they can be easily fired. Projectile weapons must be reloaded. The time required to reload a projectile weapon is given in the weapon description.
        Unless otherwise noted, projectile weapons cannot be used while \prone.
        \parhead{Reach}\label{Reach Weapon} This weapon can be used to attack at double your natural reach (so 10 feet for a typical Small or Medium creature). However, it cannot attack a creature within your natural reach.

        Reach weapons can held using a different grip to strike nearby foes. This is called ``short hafting''. While short hafting a Reach weapon, you ignore the weapon's Reach property, but you take a \minus2 penalty to accuracy with it.
        \parhead{Subdual} This weapon deals \glossterm{subdual damage} (see \pcref{Subdual Damage}).
        \parhead{Throwing} This weapon is designed to be thrown. Throwing weapons have a range increment listed in their description, which indicates the distance they can be easily thrown. See \pcref{Thrown Weapons}.
        Unless otherwise noted in a weapon's description, a throwing weapon can be used to attack in melee without penalty.
        \parhead{Tripping} When you use the \textit{trip} ability, you can attack with this weapon instead of with a free hand (see \pcref{Trip}).
        If you do, you gain a \plus2 bonus to accuracy on the attack.

    \subsection{Weapon Descriptions}
        Some weapons in \trefnp{Weapons} have additional abilities which are described below.
        \parhead{Claw Sheath} A claw sheath is not a weapon in itself, but a covering over a single claw that a creature can use to make claw attacks. Claw sheaths do not grant a creature without claws a claw attack. Normally, a sheath does not improve the claw attack, but magical claw sheaths can be made which grant bonuses to attacks made with the claw. It may be possible to find or craft unusual sheaths for natural weapons other than claws.
        \parhead{Crossbow, Hand} You can draw a hand crossbow back by hand. Loading a hand crossbow is a move action that requires one hand (but not the hand wielding the crossbow).
        \par You can fire a crossbow while \prone without penalty.
        \parhead{Crossbow, Heavy} You draw a heavy crossbow back by turning a small winch. Loading a heavy crossbow is a standard action that requires both hands.
        \par You can fire a crossbow while \prone without penalty.
        \parhead{Crossbow, Light} You draw a light crossbow back by pulling a lever. Loading a light crossbow is a move action that requires both hands.
        \par You can fire a crossbow while \prone without penalty.
        \parhead{Crossbow, Repeating} The repeating crossbow (whether heavy or light) holds 5 crossbow bolts. As long as it holds bolts, you can reload it as a \glossterm{free action} by pulling the reloading lever. Loading a new case of 5 bolts is a \glossterm{standard action} that requires both hands.
        \par You can fire a crossbow while \prone without penalty.
        \parhead{Flatbow} You need both hands to fire a bow, regardless of its size. One hand is free when not firing the bow. Loading a bow is a free action that requires both hands. A flatbow is too unwieldy to use while you are mounted.
        Unlike a longbow, a flatbow is flat when not under tension and has approximately rectangular limbs.
        This spreads stress more evenly over the bow's structure, allowing more precise shots, though the firing technique is different and less commonly known.
        \parhead{Fire Arrows} These arrows are treated with alchemist's fire so they can be ignited before being shot.
        The process requires thickening the arrow shaft, reducing the precision of the arrow.
        It takes a \glossterm{move action} to ignite a fire arrow assuming you have access to an active flame the size of a torch or larger.
        In addition, fire arrows lose their flame quickly.
        If the target is farther than two \glossterm{range increments} away, the fire goes out, and the damage dealt by the hit is not fire damage.
        \parhead{Hammer, Gnome Hooked} This weapon has a hammer head which deals \plus0d damage, and a hook which deals \minus1d damage. The hook is a tripping weapon.
        \parhead{Longbow} You need both hands to fire a bow, regardless of its size. One hand is free when not firing the bow. Loading a bow is a free action that requires both hands. A longbow is too unwieldy to use while you are mounted.
        \par When attacking with a bow, you take a \minus4 penalty to accuracy against creatures that threaten you.
        \parhead{Net} A net is used to entangle enemies. When you throw a net, you make an attack vs. Reflex against your target. If you hit, the target is \slowed. If you control the trailing rope by succeeding on an opposed Strength check while holding it, the netted creature can move only within the limits that the rope allows.
        \par A netted creature can escape with a \glossterm{difficulty rating} 10 Escape Artist check (normally a standard action). The net has (2/10) and can be burst with a \glossterm{difficulty rating} 10 Strength check as a standard action.
        \par A net is useful only against creatures within one size category of you.
        \par A net must be folded to be thrown effectively. The first time you throw your net in a fight, you make a normal ranged touch attack roll. After the net is unfolded, you take a \minus4 penalty to accuracy with it. It takes 2 standard actions for a proficient user to fold a net and twice that many for a nonproficient one to do so.
        \parhead{Pike} A pike can be used to atttack at up to triple your natural reach (so 15 feet for a typical Small or Medium creature).
        However, it cannot be short hafted (see \pcref{Reach Weapon}).
        \parhead{Shield, Standard} You can bash with a shield in addition to defending with it. See Armor for details.
        \parhead{Shortbow} You need both hands to fire a bow, regardless of its size. One hand is free when not firing the bow. Loading a bow is a free action that requires both hands.
        \par When attacking with a bow, you take a \minus4 penalty to accuracy against creatures that threaten you.
        \parhead{Sling} You can fire, but not load, a sling with one hand. Loading a sling is a free action that requires both hands.
        \par You can hurl ordinary stones with a sling, but stones are not as dense or as round as bullets. You take a \minus1d penalty to damage with ordinary stones.
        \parhead{Spiked Armor} Any \glossterm{body armor} can be spiked.
        You cannot normally attack with spiked armor.
        However, if your armor is spiked and you are proficient with it, you deal damage with it when you make a successful \glossterm{grapple} or \glossterm{shove} attack.
        Magical abilities on a suit of armor does not improve the spikes' effectiveness, but the spikes can be made into magic weapons in their own right.
        \parhead{Spiked Shield, Standard} You can bash with a spiked shield in addition to defending with it. See Armor for details.
        \parhead{Takedown Bow} A takedown bow is a bow assembled from multiple independent components that can be reconfigured into two different combinations.
        In its longbow configuration, it has a heavy (ranged) usage class and a 100 ft.\ range increment.
        In its shortbow configuration, it has a medium (ranged) usage class and a 50 ft.\ range increment.
        In addition, when it is fully disassembled, it takes up space equivalent to a Light usage class weapon, making it easier to transport and conceal.

        \parhead{Unarmed Strike} Anyone can attack unarmed, but it is more dangerous than attacking with a weapon. See \pcref{Unarmed Combat}, for details.

        \parhead{Whip} A whip is a light melee weapon with 15 foot reach.
        You can use a whip against foes anywhere within your reach, including adjacent foes.
        However, you can't defend yourself with a whip, which can make you \defenseless.
        % This shouldn't need to specify threatening mechanics because there are no more effects that trigger off of threatened area that a whip shouldn't be able to do.
        % and you don't \glossterm{threaten} the area into which you can make an attack.

\section{Armor}\label{Armor}

    Most characters use armor to protect themselves. There are two kinds of armor: \glossterm{body armor}, such as full plate armor, and \glossterm{shields}.

    \subsection{Armor Usage Classes}\label{Armor Usage Classes}
        An armor's \glossterm{usage class} is a measure of how the armor is used, and how much effort is required to use it.
        It indicates whether armor, when used by a creature the armor is sized for, is considered light armor, medium armor, heavy armor, or a shield.
        Light armor, medium armor, and heavy armor are all types of \glossterm{body armor}.

        \parhead{Light Armor} Light armor has a low \glossterm{encumbrance} and imposes no special penalties on a creature wearing it.

        \parhead{Medium Armor} A creature wearing medium armor has its \glossterm{base speed} reduced by five feet (to a minimum of five feet).

        \parhead{Heavy Armor} A creature wearing heavy armor has its \glossterm{base speed} reduced by ten feet (to a minimum of five feet).
        In addition, the creature's current Dexterity is reduced to half its normal value.
        % TODO: wording
        The creature's base Dexterity is also reduced to half its normal value for the purpose of calculations that rely on base Dexterity directly, such as Armor and Reflex defenses.
        If the creature's Dexterity or base Dexterity are negative, they are not halved.

        \parhead{Shield} A shield requires a free hand instead of being worn on the body.
        Most shields impose no special penalties on a creature wielding them.
        However, a creature wielding a tower shield increases \glossterm{encumbrance} by 2 and halves the creature's Dexterity, just like heavy armor.

        \subsubsection{Armor Proficiency}\label{Armor Proficiency}
            Unlike weapons, proficiency with armor is defined by the armor's usage class.
            If you wear or use armor you are not proficient with, it provides half its normal defense bonus.
            In addition, you apply that armor's \glossterm{encumbrance} as a penalty to your \glossterm{accuracy} with \glossterm{physical attacks}.
            Since standard shields have no \glossterm{encumbrance}, you can use them without penalizing your attacks.

    \subsection{Armor Qualities}
        \par Here is the format for armor entries (given as column headings on \trefnp{Armor and Shields}, below).

        \parhead{Cost} The cost of the armor for Small or Medium humanoid creatures.
        See \trefnp{Armor for Unusual Creatures}, below, for armor prices for other creatures.
        \parhead{Armor/Shield Bonus} Both body armor and shields improve your Armor defense.
        Wearing multiple suits of armor or wielding multiple shields does not improve your defenses any further.

        \parhead{Encumbrance} This value indicates how much the armor increases your \glossterm{encumbrance}.
        % TODO: awkward that this doesn't affect, say, Climb-based attacks.
        You apply your encumbrance as a penalty to all Strength and Dexterity-based checks you make.
        For details, see \pcref{Encumbrance}.

        \parhead{Weight} This column gives the weight of the armor sized for a Medium wearer. Armor fitted for Small characters weighs half as much, and armor for Large characters weighs twice as much.

    \subsection{Getting Into And Out Of Armor}
        The time required to don armor depends on its type; see \trefnp{Donning Armor}. Donning armor of any kind takes both hands.
        \parhead{Don} This column tells how long it takes a character to put the armor on. (One minute is 10 rounds.) Readying (strapping on) a shield is only a move action.
        \parhead{Don Hastily} This column tells how long it takes to put the armor on in a hurry. The \glossterm{encumbrance} and defense bonus for hastily donned armor are each 1 point worse than normal.
        \parhead{Remove} This column tells how long it takes to get the armor off. Loosing a shield (removing it from the arm and dropping it) is only a move action.

        \begin{dtable}
            \lcaption{Donning Armor}
            \begin{dtabularx}{\columnwidth}{>{\lcol}X c c c}
                \tb{Armor Type} & \tb{Don} & \tb{Don Hastily} & \tb{Remove} \tableheaderrule
                Shield (any)      & 1 move action   & n/a             & 1 move action           \\
                Light body armor  & 1 minute        & 5 rounds        & 1 minute\fn{1}          \\
                Medium body armor & 4 minutes\fn{1} & 1 minute        & 1 minute\fn{1}          \\
                Heavy body armor & 4 minutes\fn{2} & 4 minutes\fn{1} & 1d4\plus1 minutes\fn{1} \\
            \end{dtabularx}
            1 If the character has some help, cut this time in half. A single character doing nothing else can help one or two adjacent characters. Two characters can't help each other don armor at the same time. \\
            2 The wearer must have help to don this armor. Without help, it can be donned only hastily.
        \end{dtable}

        \begin{dtable!*}
            \lcaption{Armor and Shields}
            \begin{dtabularx}{\textwidth}{>{\lcol}X c c c c c c}
                \tb{Armor} & \tb{Armor Defense} & \tb{Resistances}\fn{1} & \tb{Encumbrance} & \tb{Material} & \tb{Cost} & \tb{Weight}\fn{2} \tableheaderrule
                Light armor            &              &         &              &                   &            &             \\
                \tind Leather          & \plus2       & \plus0  & \plus1       & Leather           & 10 gp      & 15 lb.      \\
                \tind Studded leather  & \plus2       & \plus1  & \plus1       & Leather and metal & 25 gp      & 20 lb.      \\
                \tind Chain shirt      & \plus2       & \plus2  & \plus2       & Metal             & 40 gp      & 25 lb.      \\
                Medium armor\fn{3}     &              &         &              &                   &            &             \\
                \tind Hide             & \plus3       & \plus1  & \plus3       & Leather           & 15 gp      & 25 lb.      \\
                \tind Scale mail       & \plus3       & \plus2  & \plus5       & Metal             & 50 gp      & 30 lb.      \\
                \tind Breastplate      & \plus3       & \plus3  & \plus4       & Metal             & 150 gp     & 30 lb.      \\
                Heavy armor\fn{4}      &              &         &              &                   &            &             \\
                \tind Plated mail      & \plus4       & \plus4  & \plus5       & Metal             & 250 gp     & 35 lb.      \\
                \tind Half-plate       & \plus4       & \plus5  & \plus7       & Metal             & 500 gp     & 50 lb.      \\
                \tind Full plate       & \plus4       & \plus6  & \plus6       & Metal             & 1000 gp    & 50 lb.      \\
                Shields                &              &         &              &                   &            &             \\
                \tind Buckler          & \plus1       & \tdash  & \tdash       & Metal or wood     & 15 gp      & 5 lb.       \\
                \tind Shield, standard & \plus2       & \tdash  & \tdash\fn{5} & Metal or wood     & 15 gp      & 10 lb.      \\
                \tind Shield, tower    & \plus3\fn{6} & \tdash  & \plus2\fn{5} & Metal or wood     & 30 gp      & 45 lb.      \\
                Extras                 &              &         &              &                   &            &             \\
                \tind Armor spikes     & \tdash       & \minus1 & \plus1       & Metal             & \plus50 gp & \plus10 lb. \\
                \tind Gauntlet, locked & \tdash       & \tdash  & Special      & Metal             & 8 gp       & \plus5 lb.  \\
                \tind Shield spikes    & \minus1      & \tdash  & \tdash       & Metal             & \plus10 gp & \plus5 lb.  \\
            \end{dtabularx}
            1 This increases your \glossterm{resistances} against \glossterm{physical damage} (see \pcref{Resistances}). \\
            2 Weight figures are for armor sized to fit Medium characters. Armor fitted for Small characters weighs half as much, and armor fitted for Large characters weighs twice as much. \\
            3 Speed is reduced by 5 feet (see \pcref{Armor Usage Classes}). \\
            4 Speed is reduced by 10 feet and Dexterity is halved (see \pcref{Armor Usage Classes}). \\
            5 The hand holding the shield is not free, which may limit your actions. \\
            6 Tower shields can grant you cover. See the description. \\
        \end{dtable!*}

    \subsection{Armor Descriptions}
        Any special benefits or accessories to the types of armor found on \trefnp{Armor and Shields} are described below.
        \parhead{Armor Spikes} You can add armor spikes to any \glossterm{body armor}.
        Spiked armor is a \glossterm{weapon} that you can deal damage with (see \pcref{Armor Weapons}).
        Magical abilities on a suit of armor does not improve the spikes' effectiveness, but the spikes can be made into magic weapons in their own right.
        \parhead{Breastplate} It comes with a helmet and greaves.
        \parhead{Buckler} This small metal shield is worn strapped to your forearm.
        You can hold items in a hand holding a buckler.
        However, if you wield weapons or otherwise take actions using the arm bearing the buckler, you do not gain the buckler's defensive bonus during that time.
        \par You can't bash someone with a buckler.
        \parhead{Chain Shirt} A chain shirt comes with a steel cap.
        \parhead{Full Plate} The suit includes gauntlets, heavy leather boots, a visored helmet, and a thick layer of padding that is worn underneath the armor. Each suit of full plate must be individually fitted to its owner by a master armorsmith, although a captured suit can be resized to fit a new owner with a day of work and a \glossterm{difficulty rating} 10 Craft (metalworking) check. The new owner must still be of the same size category as the size category that the suit was designed for.
        \parhead{Gauntlet, Locked} This armored gauntlet has small chains and braces that allow the wearer to attach a weapon to the gauntlet so that it cannot be dropped easily.
        As a \glossterm{standard action}, you can lock or unlock the gauntlet with a different free hand.
        While the gauntlet is locked, any item held in that hand is extraordinarily well secured.
        This can prevent you from dropping the item if you are affected by the \textit{disarm} ability or similar effects (see \pcref{Disarm}).
        However, you are unable to use that hand for any purpose other than holding the item until you unlock the gauntlet.
        \par The price given is for a single locked gauntlet.
        If you are wearing armor that normally has gauntlets, you can replace one or both of those gauntlets with a locked gauntlet with no significant weight increase.
        Like a normal gauntlet, a locked gauntlet lets you deal normal damage rather than \glossterm{subdual damage} with unarmed attacks (see \pcref{Unarmed Combat}).
        \parhead{Half-Plate} The suit includes gauntlets.
        \parhead{Scale Mail} The suit includes gauntlets.
        \parhead{Shield, Standard, Wooden or Steel} You strap a shield to your forearm and grip it with your hand. A standard shield is so cumbersome that you can't use your shield hand for anything else.
        \subparhead{Shield Bash Attacks} You can bash an opponent with a standard shield, using it as a medium bludgeoning weapon. See \trefnp{Weapons} for the damage dealt by a shield bash.
        Magical abilities on a shield do not affect shield bash attacks made with it, but the shield can be made into a magic weapon in its own right.
        \parhead{Shield, Tower} This massive shield is nearly as tall as an average human.
        When you take the \textit{total defense} action with a tower shield, you can treat the shield as a wall along one edge of your square, providing you with \glossterm{total cover} against attacks.
        You cannot attack with a tower shield, and you cannot use your shield hand for anything else.

        While wielding a tower shield, you take a \minus2 penalty to \glossterm{accuracy} with \glossterm{strikes} because of the shield's unwieldy nature.
        \parhead{Shield Spikes} When added to your shield, these spikes turn it into a piercing weapon that increases the damage dealt by a shield bash as if the shield were designed for a creature one size category larger than you. You can't put spikes on a buckler or a tower shield. Otherwise, attacking with a spiked shield is like making a shield bash attack (see above).
        \parhead{Studded Leather} Only the studs on studded leather are made of metal.
        Studded leather armor made with studs from special materials does not grant the wearer the properties of the special material.

    \subsection{Armor for Unusual Creatures}\label{Armor for Unusual Creatures}
        Armor and shields for unusually big creatures, unusually little creatures, and nonhumanoid creatures have different costs and weights from those given on \trefnp{Armor and Shields}. Refer to the appropriate line on the table below and apply the multipliers to cost and weight for the armor type in question.
        \begin{dtable}
            \lcaption{Armor for Unusual Creatures}
            \begin{dtabularx}{\columnwidth}{>{\lcol}X c c c c}
                & \multicolumn{2}{c}{\tb{Humanoid}} & \multicolumn{2}{c}{\tb{Nonhumanoid}} \tableheaderrule
                \tb{Size} & \tb{Cost} & \tb{Weight} & \tb{Cost} & \tb{Weight} \\
                Tiny or smaller\fn{1} & \mult1/2 & \mult1/10 & \mult1 & \mult1/10 \\
                Small & \mult1 & \mult1/2 & \mult2 & \mult1/2 \\
                Medium & \mult1 & \mult1 & \mult2 & \mult1 \\
                Large & \mult2 & \mult2 & \mult4 & \mult2 \\
                Huge & \mult4 & \mult4 & \mult8 & \mult4 \\
                Gargantuan & \mult8 & \mult8 & \mult16 & \mult8 \\
                Colossal & \mult16 & \mult12 & \mult32 & \mult12 \\
            \end{dtabularx}
            1 Divide defense bonuses from armor by 2.
        \end{dtable}

    \subsection{Special Materials}
        Armor and shields can be made from special materials, which can alter the properties of the item. The most common special material is ironwood, which is made from wood magically treated using the \spell{ironwood} ritual to be as strong as steel. Ironwood armor functions exactly like  armor of the same type, except that it is made of wood. This causes it to weigh half the normal weight, and allows druids to wear it without penalty. However, it costs twice as much as armor of the same type would normally cost, to a minimum of an additional 200 gp.

\section{Goods And Services}

    \subsubsection{Standard Adventuring Kit}
        % Technically 14.7 gp and 49.5 pounds
        A standard adventuring kit costs 15 gp, weighs 50 pounds, and contains the following items:
        \begin{itemize}
            \item Backpack
            \item Bedroll
            \item Flint and steel
            \item Rations, trail (7 days)
            \item Rope, hempen (50 ft.)
            \item Sack (empty)
            \item Tent
            \item Torch
            \item Waterskin
        \end{itemize}

    \begin{dtable!*}
        \lcaption{Goods and Services}
        \begin{dtabularx}{\textwidth}{>{\lcol}X c c >{\lcol}X c c >{\lcol}X c c}
            \tb{Item} & \tb{Cost} & \tb{Weight} & \tb{Item} & \tb{Cost} & \tb{Weight} & \tb{Item} & \tb{Cost} & \tb{Weight} \tableheaderrule
            \multicolumn{3}{>{\columncolor{tbrown}}c}{\tb{Adventuring Gear}} & \tind Average & 40 gp & 1 lb. & \multicolumn{3}{>{\columncolor{tbrown}}c}{\tb{Special Substances and Items}} \\
            Backpack (empty) & 2 gp & 2 lb.\fn{1} & \tind Good & 80 gp & 1 lb. & Acid (flask) & 5 gp & 1 lb. \\
            Barrel (empty) & 2 gp & 30 lb. & \tind Amazing & 150 gp & 1 lb. & Alchemist's fire (flask) & 5 gp & 1 lb. \\
            Basket (empty) & 4 sp & 1 lb. & Manacles & 15 gp & 2 lb. & Antitoxin (vial) & 10 gp & \tdash \\
            Bedroll & 1 sp & 5 lb.\fn{1} & Manacles, masterwork & 50 gp & 2 lb. & Everburning torch & 100 gp & 1 lb. \\
            Bell & 1 gp & \tdash & Mirror, small steel & 10 gp & 1/2 lb. & Holy water (flask) & 10 gp & 1 lb. \\
            Blanket, winter & 5 sp & 3 lb.\fn{1} & Mug/Tankard, clay & 2 cp & 1 lb. & Smokestick & 5 gp & 1/2 lb. \\
            Block and tackle & 5 gp & 5 lb. & Oil (1-pint flask) & 1 sp & 1 lb. & Sunrod & 2 gp & 1 lb. \\
            Bottle, wine, glass & 2 gp & \tdash & Paper (sheet) & 4 sp & \tdash & Tanglefoot bag & 15 gp & 4 lb. \\
            Bucket (empty) & 5 sp & 2 lb. & Parchment (sheet) & 2 sp & \tdash & Thunderstone & 10 gp & 1 lb. \\
            Caltrops & 1 gp & 2 lb. & Pick, miner's & 3 gp & 10 lb. & Tindertwig & 5 sp & \tdash \\
            Candle & 1 cp & \tdash & Pitcher, clay & 2 cp & 5 lb. & \multicolumn{3}{c}{\tb{Tools and Skill Kits}}  \\
            Canvas (sq.\ yd.) & 1 sp & 1 lb. & Piton & 1 sp & 1/2 lb. & Item & Cost & Weight \\
            Case, map or scroll & 1 gp & 1/2 lb. & Pole, 10 foot & 5 cp & 8 lb. & Alchemist's lab & 500 gp & 40 lb. \\
            Chain (10 ft.) & 30 gp & 2 lb. & Pot, iron & 5 sp & 10 lb. & Artisan's tools & 5 gp & 5 lb. \\
            Chalk, 1 piece & 1 cp & \tdash & Pouch, belt (empty) & 1 gp & 1/2 lb.\fn{1} & Artisan's tools, masterwork & 55 gp & 5 lb. \\
            Chest (empty) & 2 gp & 25 lb. & Ram, portable & 10 gp & 20 lb. & Climber's kit & 80 gp & 5 lb.\fn{1} \\
            Crowbar & 2 gp & 5 lb. & Rations, trail (per day) & 5 sp & 1 lb.\fn{1} & Disguise kit & 50 gp & 8 lb.\fn{1} \\
            Firewood (per day) & 1 cp & 20 lb. & Rope, hempen (50 ft.) & 1 gp & 10 lb. & Healer's kit & 50 gp & 1 lb. \\
            Fishhook & 1 sp & \tdash & Rope, silk (50 ft.) & 10 gp & 5 lb. & Holly and mistletoe & \tdash & \tdash \\
            Fishing net, 25 sq.\ ft. & 4 gp & 5 lb. & Sack (empty) & 1 sp & 1/2 lb.\fn{1} & Holy symbol, wooden & 1 gp & \tdash \\
            Flask (empty) & 3 cp & 1-1/2 lb. & Sealing wax & 1 gp & 1 lb. & Holy symbol, silver & 25 gp & 1 lb. \\
            Flint and steel & 1 gp & \tdash & Sewing needle & 5 sp & \tdash & Hourglass & 25 gp & 1 lb. \\
            Grappling hook & 1 gp & 4 lb. & Signal whistle & 8 sp & \tdash & Magnifying glass & 100 gp & \tdash \\
            Hammer & 5 sp & 2 lb. & Signet ring & 5 gp & \tdash & Musical instrument, common & 5 gp & 3 lb.\fn{1} \\
            Ink (1 oz.\ vial) & 8 gp & \tdash & Sledge & 1 gp & 10 lb. & Musical instrument, masterwork & 100 gp & 3 lb.\fn{1} \\
            Inkpen & 1 sp & \tdash & Soap (per lb.) & 5 sp & 1 lb. & Scale, merchant's & 2 gp & 1 lb. \\
            Jug, clay & 3 cp & 9 lb. & Spade or shovel & 2 gp & 8 lb. & Spell component pouch & 5 gp & 2 lb. \\
            Ladder, 10 foot & 2 sp & 20 lb. & Spyglass & 1,000 gp & 1 lb. & Spellbook, wizard's (blank) & 15 gp & 3 lb. \\
            Lamp, common & 1 sp & 1 lb. & Tent & 5 gp & 20 lb.\fn{1} & Thieves' tools & 30 gp & 1 lb. \\
            Lantern, bullseye & 12 gp & 3 lb. & Torch & 1 cp & 1 lb. & Thieves' tools, masterwork & 100 gp & 2 lb. \\
            Lantern, hooded & 7 gp & 2 lb. & Vial, ink or potion & 1 gp & 1/10 lb. & Tool, masterwork & 50 gp & 1 lb. \\
            Lock &   & 1 lb. & Waterskin & 1 gp & 4 lb.\fn{1} & Water clock & 1,000 gp & 200 lb. \\
            \tind Very simple & 20 gp & 1 lb. & Whetstone & 2 cp & 1 lb. &  &  &  \\
        \end{dtabularx}
        1 These items weigh one-quarter this amount when made for Small characters. Containers for Small characters also carry one-quarter the normal amount. \\
        \tdash No weight, or no weight worth noting.	
    \end{dtable!*}

    \begin{dtable!*}
        \begin{dtabularx}{\textwidth}{>{\lcol}X c c >{\lcol}X c c >{\lcol}X c c}
            \tb{Clothing} &  &  & \tb{Mounts and Related Gear} &  &  & \tb{Transport} &  & \tableheaderrule
            \tb{Item} & \tb{Cost} & \tb{Weight} & \tb{Item} & \tb{Cost} & \tb{Weight} & \tb{Item} & \tb{Cost} & \tb{Weight} \\
            Artisan's outfit              & 1 gp   & 4 lb.     & Barding               &              &              & Carriage                       & 100 gp                             & 600 lb. \\
            Cleric's vestments            & 5 gp   & 6 lb.     & \tind Medium creature & \mult2\fn{2} & \mult1\fn{2} & Cart                           & 15 gp                              & 200 lb. \\
            Cold weather outfit           & 8 gp   & 7 lb.     & \tind Large creature  & \mult4\fn{2} & \mult2\fn{2} & Galley                         & 30,000 gp                          & \tdash  \\
            Courtier's outfit             & 30 gp  & 6 lb.     & Bit and bridle        & 2 gp         & 1 lb.        & Keelboat                       & 3,000 gp                           & \tdash  \\
            Entertainer's outfit          & 3 gp   & 4 lb.     & Dog, guard            & 25 gp        & \tdash       & Longship                       & 10,000 gp                          & \tdash  \\
            Explorer's outfit             & 10 gp  & 8 lb.     & Dog, riding           & 150 gp       & \tdash       & Rowboat                        & 50 gp                              & 100 lb. \\
            Monk's outfit                 & 5 gp   & 2 lb.     & Donkey or mule        & 8 gp         & \tdash       & Oar                            & 2 gp                               & 10 lb.  \\
            Noble's outfit                & 75 gp  & 10 lb.    & Feed (per day)        & 5 cp         & 10 lb.       & Sailing ship                   & 10,000 gp                          & \tdash  \\
            Peasant's outfit              & 1 sp   & 2 lb.     & Horse                 &              &              & Sled                           & 20 gp                              & 300 lb. \\
            Royal outfit                  & 200 gp & 15 lb.    & \tind Horse, heavy    & 200 gp       & \tdash       & Wagon                          & 35 gp                              & 400 lb. \\
            Scholar's outfit              & 5 gp   & 6 lb.     & \tind Horse, light    & 75 gp        & \tdash       & Warship                        & 25,000 gp                          & \tdash  \\
            Traveler's outfit             & 1 gp   & 5 lb.     & \tind Pony            & 30 gp        & \tdash       & \tb{Spellcasting and Services} &                                    &         \\
            \tb{Food, Drink, and Lodging} &        &           & \tind Warhorse, heavy & 400 gp       & \tdash       & Service                        & \multicolumn{2}{l}{Cost}          \\
            Item                          & Cost   & Weight    & \tind Warhorse, light & 150 gp       & \tdash       & Coach cab                      & \multicolumn{2}{l}{3 cp per mile} \\
            Ale                           &        &           & \tind Warpony         & 100 gp       & \tdash       & Hireling, trained              & \multicolumn{2}{l}{3 sp per day}  \\
            \tind Gallon                  & 2 sp   & 8 lb.     & Saddle                &              &              & Hireling, untrained            & \multicolumn{2}{l}{1 sp per day}  \\
            \tind Mug                     & 4 cp   & 1 lb.     & \tind Military        & 20 gp        & 30 lb.       & Messenger                      & \multicolumn{2}{l}{2 cp per mile} \\
            Banquet (per person)          & 10 gp  & \tdash    & \tind Pack            & 5 gp         & 15 lb.       & Road or gate toll              & \multicolumn{2}{l}{1 cp}          \\
            Bread, per loaf               & 2 cp   & 1/2 lb.   & \tind Riding          & 10 gp        & 25 lb.       & Ship's passage                 & \multicolumn{2}{l}{1 sp per mile} \\
            Cheese, hunk of               & 1 sp   & 1/2 lb.   & \tb{Saddle, Exotic}   &              &              &                                &                                    &         \\
            Inn stay (per day)            &        &           & \tind Military        & 60 gp        & 40 lb.       &                                &                                    &         \\
            \tind Good                    & 2 gp   & \tdash    & \tind Pack            & 15 gp        & 20 lb.       &                                &                                    &         \\
            \tind Common                  & 5 sp   & \tdash    & \tind Riding          & 30 gp        & 30 lb.       &                                &                                    &         \\
            \tind Poor                    & 2 sp   & \tdash    & Saddlebags            & 4 gp         & 8 lb.        &                                &                                    &         \\
            Meals (per day)               &        &           & Stabling (per day)    & 5 sp         & \tdash       &                                &                                    &         \\
            \tind Good                    & 5 sp   & \tdash    &                       &              &              &                                &                                    &         \\
            \tind Common                  & 3 sp   & \tdash    &                       &              &              &                                &                                    &         \\
            \tind Poor                    & 1 sp   & \tdash    &                       &              &              &                                &                                    &         \\
            Meat, chunk of                & 3 sp   & 1/2 lb.   &                       &              &              &                                &                                    &         \\
            Wine                          &        &           &                       &              &              &                                &                                    &         \\
            \tind Common (pitcher)        & 2 sp   & 6 lb.     &                       &              &              &                                &                                    &         \\
            \tind Fine (bottle)           & 10 gp  & 1-1/2 lb. &                       &              &              &                                &                                    &         \\
        \end{dtabularx}
        1 See spell description for additional costs. If the additional costs put the spell's total cost above 2,000 gp, that spell is not generally available. \\
        2 Relative to normal armor of the same type
    \end{dtable!*}

    \subsection{Adventuring Gear}
        A few of the pieces of adventuring gear found on \trefnp{Goods and Services} are described below, along with any special benefits they confer on the user (``you'').
        \parhead{Caltrops} A caltrop is a four-pronged iron spike crafted so that one prong faces up no matter how the caltrop comes to rest. You scatter caltrops on the ground in the hope that your enemies step on them or are at least forced to slow down to avoid them. One 2- pound bag of caltrops covers an area 5 feet square.
        \par Each time a creature moves into an area covered by caltrops (or spends a round fighting while standing in such an area), it might step on one. The caltrops make an attack roll with an accuracy of \plus0 against the creature's Armor defense. If the caltrops succeed on the attack, the creature has stepped on one. The caltrop deals 1d6 damage. If the creature takes damage this way, it moves at half speed as a \glossterm{condition}. This condition can be removed with a \glossterm{difficulty rating} 10 Heal check. Any creature moving at half speed or slower can pick its way through a bed of caltrops without stepping on any.
        \par Caltrops may not be effective against unusual opponents.
        \parhead{Candle} A candle dimly illuminates a 5 foot radius and burns for 1 hour.
        \parhead{Chain} Chain has \glossterm{damage resistance} of 10 and a \glossterm{wound resistance} of 20. It can be burst with a \glossterm{difficulty rating} 18 Strength check.
        \parhead{Crowbar} A crowbar it grants a \plus2 bonus on Strength checks made for such purposes. If used in combat, treat a crowbar as an improvised weapon equivalent to a club.
        \parhead{Flint and Steel} Lighting a torch with flint and steel is a standard action.
        Lighting any other fire may take additional standard actions, depending on the size of the fire.
        \parhead{Grappling Hook} Throwing a grappling hook successfully requires a ranged attack (\glossterm{difficulty rating} 10, \plus2 per 10 feet of distance thrown).
        \parhead{Hammer} If a hammer is used in combat, treat it as a medium improvised weapon that deals bludgeoning damage equal to that of a spiked gauntlet of its size.
        \parhead{Ink} This is black ink. You can buy ink in other colors, but it costs twice as much.
        \parhead{Jug, Clay} This basic ceramic jug is fitted with a stopper and holds 1 gallon of liquid.
        \parhead{Lamp, Common} A lamp clearly illuminates a 20 foot radius, provides shadowy illumination out to a 40 foot radius, and burns for 6 hours on a pint of oil. You can carry a lamp in one hand.
        \parhead{Lantern, Bullseye} A bullseye lantern provides bright illumination in a 50 foot cone and shadowy illumination in a 100 foot cone. It burns for 6 hours on a pint of oil. You can carry a bullseye lantern in one hand.
        \parhead{Lantern, Hooded} A hooded lantern clearly illuminates a 20 foot radius and provides shadowy illumination in a 40 foot radius. It burns for 6 hours on a pint of oil. You can carry a hooded lantern in one hand.
        \parhead{Lock} The \glossterm{difficulty rating} to open a lock with the Open Lock skill depends on the lock's quality: simple (\glossterm{difficulty rating} 20), average (\glossterm{difficulty rating} 25), good (\glossterm{difficulty rating} 30), or superior (\glossterm{difficulty rating} 40).
        \parhead{Manacles and Manacles, Masterwork} Manacles can bind a Medium creature. A manacled creature can use the Escape Artist skill to slip free (\glossterm{difficulty rating} 30, or \glossterm{difficulty rating} 35 for masterwork manacles). Breaking the manacles requires a Strength check (\glossterm{difficulty rating} 26, or \glossterm{difficulty rating} 28 for masterwork manacles). Manacles have a \glossterm{damage resistance} 10 and a \glossterm{wound resistance} of 20.
        \par Most manacles have locks; add the cost of the lock you want to the cost of the manacles.
        \par For the same cost, you can buy manacles for a Small creature.
        \par For a Large creature, manacles cost ten times the indicated amount, and for a Huge creature, one hundred times this amount. Gargantuan, Colossal, Tiny, Diminutive, and Fine creatures can be held only by specially made manacles.
        \parhead{Oil} A pint of oil burns for 6 hours in a lantern. You can use a flask of oil as a splash weapon. Use the rules for alchemist's fire, except that it takes a standard action to prepare a flask with a fuse. Once it is thrown, there is a 50\% chance of the flask igniting successfully.
        \par You can pour a pint of oil on the ground to cover an area 5 feet square, provided that the surface is smooth. If lit, the oil burns for 2 rounds and deals 1d3 points of fire damage to each creature in the area.
        \parhead{Ram, Portable} This iron-shod wooden beam gives you a \plus2 bonus on Strength checks made to break open a door and it allows a second person to help you without having to roll, increasing your bonus by 2.
        \parhead{Rope, Hempen} This rope has a \glossterm{wound resistance} of 4 and can be burst with a \glossterm{difficulty rating} 23 Strength check.
        \parhead{Rope, Silk} This rope has a \glossterm{wound resistance} of 5 and can be burst with a \glossterm{difficulty rating} 24 Strength check.
        \parhead{Spyglass} Objects viewed through a spyglass are magnified to twice their size.
        \parhead{Torch} A torch burns for 1 hour, clearly illuminating a 20 foot radius and providing shadowy illumination out to a 40 foot radius. If a torch is used in combat, treat it as a light improvised weapon that deals bludgeoning damage equal to that of a gauntlet of its size, plus 1 point of fire damage. Wielding a torch gives a \plus2 bonus to dirty trick attempts made to dazzle an opponent.
        \parhead{Vial} A vial holds 1 ounce of liquid. The stoppered container usually is no more than 1 inch wide and 3 inches high.

    \subsection{Special Substances And Items}
        Any of these substances except for the everburning torch and holy water can be made by a character with the Craft (alchemy) skill.
        \parhead{Acid} You can throw a flask of acid as a weapon.
        When you do, make an attack against the target's Armor defense.
        A typical flask has a \glossterm{range increment} of 10 feet.
        On a hit, the target takes 1d6 acid damage.
        \parhead{Alchemist's Fire} You can throw a vial of alchemist's fire as a weapon.
        When you do, make an attack against the target's Armor defense.
        A typical vial has a \glossterm{range increment} of 10 feet.
        On a hit, the target takes 1d6 fire damage becomes \glossterm{ignited} as a \glossterm{condition}.
        This condition can be removed if the target makes a \glossterm{difficulty rating} 10 Dexterity check as a \glossterm{move action} to put out the flames.
        Dropping \glossterm{prone} as part of this action gives a \plus5 bonus to this check.
        \parhead{Antitoxin} If you drink antitoxin, you gain a \plus5 bonus to Fortitude defense against \glossterm{Poison} abilities for 1 hour.
        \parhead{Everburning Torch} This otherwise normal torch has a continual flame spell cast upon it. An everburning torch clearly illuminates a 20 foot radius and provides shadowy illumination out to a 40 foot radius.
        \parhead{Holy Water} You can throw a flask of holy water as a weapon.
        When you do, make an attack against the target's Armor defense.
        A typical flask has a \glossterm{range increment} of 10 feet.
        On a hit, if the target is an undead creature or evil outsider, it takes 1d8 energy damage.
        % TODO: define fog clouds
        % \parhead{Smokestick} This alchemically treated wooden stick instantly creates thick, opaque smoke when ignited. The smoke fills a 10 foot cube (treat the effect as a fog cloud spell, except that a moderate or stronger wind dissipates the smoke in 1 round). The stick is consumed after 1 round, and the smoke dissipates naturally.
        \parhead{Sunrod} This 1 foot long, gold-tipped, iron rod glows brightly when struck. It clearly illuminates a 20 foot radius and provides shadowy illumination in a 40 foot radius. It glows for 6 hours, after which the gold tip is burned out and worthless.
        \parhead{Tanglefoot Bag} You can throw a tanglefoot bag as a weapon.
        When you do, make an attack against the target's Armor defense.
        The bag has a \glossterm{range increment} of 10 feet.
        On a hit, the target is \slowed as a \glossterm{condition}.
        \par In addition to being removed as a condition, this effect can be broken by dealing 5 points of damage to the goo on the target.
        Unless the attacker uses a slashing weapon, damage dealt to the goo is also dealt to the creature the goo is on.
        The goo's Armor defense is 0. % that doesn't make sense
        In addition, when the target moves, it can make a Strength check as part of the movement.
        If it beats a \glossterm{difficulty rating} of 5, the condition is removed after the movement is complete.
        \parhead{Thunderstone} You can throw a thunderstone as a weapon.
        % TODO: define hitting a location with a thrown item generically
        When you do, make an attack to hit the correct location.
        The thunderstone's \glossterm{range increment} is 20 feet.
        If the thunderstone strikes a hard surface (or is struck hard), it creates a deafening bang.
        When you detonate a thunderstone, make an attack vs Fortitude with a \plus2 total \glossterm{accuracy} against all creatures within a \areasmall radius burst from the stone.
        On a hit, each target is \glossterm{deafened} as a \glossterm{condition}.
        \parhead{Tindertwig} The alchemical substance on the end of this small, wooden stick ignites when struck against a rough surface. Creating a flame with a tindertwig is much faster than creating a flame with flint and steel (or a magnifying glass) and tinder. Lighting a torch with a tindertwig is a \glossterm{minor action}, and lighting any other fire with one is at least a standard action.

    \subsection{Tools and Skill Kits}
        \parhead{Alchemist's Lab} An alchemist's lab always has the perfect tool for making alchemical items, so it provides a \plus2 bonus on Craft (alchemy) checks. It has no bearing on the costs related to the Craft (alchemy) skill. Without this lab, a character with the Craft (alchemy) skill is assumed to have enough tools to use the skill but not enough to get the \plus2 bonus that the lab provides.
        \parhead{Artisan's Tools} These special tools include the items needed to pursue any craft. Without them, you have to use improvised tools (\minus2 penalty on Craft checks), if you can do the job at all (see \pcref{Craft}).
        \parhead{Artisan's Tools, Masterwork} These tools serve the same purpose as artisan's tools (above), but masterwork artisan's tools are the perfect tools for the job, so you get a \plus2 bonus on Craft checks made with them.
        \parhead{Climber's Kit} This is the perfect tool for climbing and gives you a \plus2 bonus on Climb checks.
        \parhead{Disguise Kit} The kit is the perfect tool for disguise and provides a \plus2 bonus on Disguise checks. A disguise kit is expended after ten uses.
        \parhead{Healer's Kit} It is the perfect tool for healing and provides a \plus2 bonus on Heal checks. A healer's kit is expended after ten uses.
        \parhead{Holy Symbol, Silver or Wooden} A holy symbol focuses good energy. A cleric or paladin uses it as the focus for his spells and as a tool for turning undead. Each religion has its own holy symbol.
        \parhead{Unholy Symbols} An unholy symbol is like a holy symbol except that it focuses evil energy and is used by evil clerics (or by neutral clerics who want to cast evil spells or command undead).
        \parhead{Magnifying Glass} This simple lens allows a closer look at small objects. It is also useful as a substitute for flint and steel when starting fires in sunlight. Lighting a fire with a magnifying glass requires light as bright as sunlight to focus, tinder to ignite, and at least a standard action.
        A magnifying glass grants a \plus2 bonus on Appraise checks involving any item that is small or highly detailed.
        \parhead{Musical Instrument, Common or Masterwork} A masterwork instrument grants a \plus2 bonus on Perform checks involving its use.
        \parhead{Pitons} You can make your own handholds and footholds by pounding pitons into a wall. Doing so takes 1 minute per piton, and one piton is needed per 3 feet of distance. As with any surface that offers handholds and footholds, a wall with pitons in it has a \glossterm{difficulty rating} of 15.
        \parhead{Spell Component Pouch} A spellcaster with a spell component pouch is assumed to have all the material components and focuses needed for spellcasting, except for those components that have a specific cost, divine focuses, and focuses that wouldn't fit in a pouch.
        \parhead{Spellbook, Wizard's (Blank)} A spellbook has 100 pages of parchment, and each spell takes up one page per spell level (one page each for 0-level spells).
        \parhead{Thieves' Tools} This kit contains the tools you need to use the Disable Device and Open Lock skills. Without these tools, you must improvise tools, and you take a \minus2 circumstance penalty on Disable Device and Open Locks checks.
        \parhead{Thieves' Tools, Masterwork} This kit contains extra tools and tools of better make, which grant a \plus2 bonus on Disable Device and Open Lock checks.
        \parhead{Tool, Masterwork} This well-made item is the perfect tool for the job. It grants a \plus2 bonus on a related skill check (if any). Bonuses provided by multiple masterwork items used toward the same skill check do not stack.
        \parhead{Water Clock} This large, bulky contrivance gives the time accurate to within half an hour per day since it was last set. It requires a source of water, and it must be kept still because it marks time by the regulated flow of droplets of water.

    \subsection{Clothing}
        \parhead{Artisan's Outfit} This outfit includes a shirt with buttons, a skirt or pants with a drawstring, shoes, and perhaps a cap or hat. It may also include a belt or a leather or cloth apron for carrying tools.
        \parhead{Cleric's Vestments} These ecclesiastical clothes are for performing priestly functions, not for adventuring.
        % What does this actually affect? How does cold weather work?
        \parhead{Cold Weather Outfit} A cold weather outfit includes a wool coat, linen shirt, wool cap, heavy cloak, thick pants or skirt, and
        boots. This outfit grants a \plus2 bonus to Fortitude defense against exposure to cold weather.
        \parhead{Courtier's Outfit} This outfit includes fancy, tailored clothes in whatever fashion happens to be the current style in the courts of the nobles. If you wear this outfit without jewelry (costing an additional 50 gp), you look like an out-of-place commoner.
        \parhead{Entertainer's Outfit} This set of flashy, perhaps even gaudy, clothes is for entertaining. While the outfit looks whimsical, its practical design lets you tumble, dance, walk a tightrope, or just run (if the audience turns ugly).
        \parhead{Explorer's Outfit} This is a full set of clothes for someone who never knows what to expect. It includes sturdy boots, leather breeches or a skirt, a belt, a shirt (perhaps with a vest or jacket), gloves, and a cloak. Rather than a leather skirt, a leather overtunic may be worn over a cloth skirt. The clothes have plenty of pockets (especially the cloak). The outfit also includes any extra items you might need, such as a scarf or a wide-brimmed hat.
        \parhead{Monk's Outfit} This simple outfit includes sandals, loose breeches, and a loose shirt, and is all bound together with sashes. The outfit is designed to give you maximum mobility, and it's made of high-quality fabric. You can hide small weapons in pockets hidden in the folds, and the sashes are strong enough to serve as short ropes.
        \parhead{Noble's Outfit} This set of clothes is designed specifically to be expensive and to show it. Precious metals and gems are worked into the clothing. To fit into the noble crowd, every would-be noble also needs a signet ring (see Adventuring Gear, above) and jewelry (worth at least 100 gp).
        \parhead{Peasant's Outfit} This set of clothes consists of a loose shirt and baggy breeches, or a loose shirt and skirt or overdress. Cloth wrappings are used for shoes.
        \parhead{Royal Outfit} This is just the clothing, not the royal scepter, crown, ring, and other accoutrements. Royal clothes are ostentatious, with gems, gold, silk, and fur in abundance.
        \parhead{Scholar's Outfit} Perfect for a scholar, this outfit includes a robe, a belt, a cap, soft shoes, and possibly a cloak.
        \parhead{Traveler's Outfit} This set of clothes consists of boots, a wool skirt or breeches, a sturdy belt, a shirt (perhaps with a vest or jacket), and an ample cloak with a hood.

    \subsection{Food, Drink, and Lodging}
        \parhead{Inn} Poor accommodations at an inn amount to a place on the floor near the hearth. Common accommodations consist of a place on a raised, heated floor, the use of a blanket and a pillow. Good accommodations consist of a small, private room with one bed, some amenities, and a covered chamber pot in the corner.
        \parhead{Meals} Poor meals might be composed of bread, baked turnips, onions, and water. Common meals might consist of bread, chicken stew, carrots, and watered-down ale or wine. Good meals might be composed of bread and pastries, beef, peas, and ale or wine.

    \subsection{Mounts and Related Gear}
        \parhead{Barding, Medium Creature and Large Creature} Barding is a type of armor that covers the head, neck, chest, body, and possibly legs of a horse or other mount. Barding made of medium or heavy armor provides better protection than light barding, but at the expense of speed. Barding can be made of any of the armor types found on \tref{Armor and Shields}.
        \par Armor for a horse (a Large nonhumanoid creature) costs four times as much as armor for a human (a Medium humanoid creature) and also weighs twice as much as the armor found on \tref{Armor and Shields}. If the barding is for a pony or other Medium mount, the cost is only double, and the weight is the same as for Medium armor worn by a humanoid. Medium or heavy barding slows a mount that wears it, as shown on the table below.

        \par Flying mounts can't fly in medium or heavy barding.
        \par Removing and fitting barding takes five times as long as the figures given on \tref{Donning Armor}. A barded animal cannot be used to carry any load other than the rider and normal saddlebags.
        \parhead{Dog, Riding} This Medium dog is specially trained to carry a Small humanoid rider. It is brave in combat like a warhorse. You take no damage when you fall from a riding dog.
        \parhead{Donkey or Mule} Donkeys and mules are stolid in the face of danger, hardy, surefooted, and capable of carrying heavy loads over vast distances. Unlike a horse, a donkey or a mule is willing (though not eager) to enter dungeons and other strange or threatening places.
        \parhead{Feed} Horses, donkeys, mules, and ponies can graze to sustain themselves, but providing feed for them is much better. If you have a riding dog, you have to feed it at least some meat.
        \parhead{Horse} A horse (other than a pony) is suitable as a mount for a human, dwarf, elf, half-elf, or half-orc. A pony is smaller than a horse and is a suitable mount for a gnome or halfling.
        \par Warhorses and warponies can be ridden easily into combat. Light horses, ponies, and heavy horses are hard to control in combat.
        \parhead{Saddle, Exotic} An exotic saddle is like a normal saddle of the same sort except that it is designed for an unusual mount. Exotic saddles come in military, pack, and riding styles.
        \parhead{Saddle, Military} A military saddle braces the rider, providing a \plus2 bonus on Ride checks related to staying in the saddle. If you're knocked unconscious while in a military saddle, you stay in the saddle instead of falling to the ground.
        \parhead{Saddle, Pack} A pack saddle holds gear and supplies, but not a rider. It holds as much gear as the mount can carry.
        \parhead{Saddle, Riding} The standard riding saddle supports a rider.

    \subsection{Transport}
        \parhead{Carriage} This four-wheeled vehicle can transport as many as four people within an enclosed cab, plus two drivers. In general, two horses (or other beasts of burden) draw it. A carriage comes with the harness needed to pull it.
        \parhead{Cart} This two-wheeled vehicle can be drawn by a single horse (or other beast of burden). It comes with a harness.
        \parhead{Galley} This three-masted ship has seventy oars on either side and requires a total crew of 200. A galley is 130 feet long and 20 feet wide, and it can carry 150 tons of cargo or 250 soldiers. For 8,000 gp more, it can be fitted with a ram and castles with firing platforms fore, aft, and amidships. This ship cannot make sea voyages and sticks to the coast. It moves about 4 miles per hour when being rowed or under sail.
        \parhead{Keelboat} This 50- to 75 foot long ship is 15 to 20 feet wide and has a few oars to supplement its single mast with a square sail. It has a crew of eight to fifteen and can carry 40 to 50 tons of cargo or 100 soldiers. It can make sea voyages, as well as sail down rivers (thanks to its flat bottom). It moves about 1 mile per hour.
        \parhead{Longship} This 75 foot long ship with forty oars requires a total crew of 50. It has a single mast and a square sail, and it can carry 50 tons of cargo or 120 soldiers. A longship can make sea voyages. It moves about 3 miles per hour when being rowed or under sail.
        \parhead{Rowboat} This 8- to 12 foot long boat holds two or three Medium passengers. It moves about one and a half miles per hour.
        \parhead{Sailing Ship} This larger, seaworthy ship is 75 to 90 feet long and 20 feet wide and has a crew of 20. It can carry 150 tons of cargo. It has square sails on its two masts and can make sea voyages. It moves about 2 miles per hour.
        \parhead{Sled} This is a wagon on runners for moving through snow and over ice. In general, two horses (or other beasts of burden) draw it. A sled comes with the harness needed to pull it.
        \parhead{Wagon} This is a four-wheeled, open vehicle for transporting heavy loads. In general, two horses (or other beasts of burden) draw it. A wagon comes with the harness needed to pull it.
        \parhead{Warship} This 100 foot long ship has a single mast, although oars can also propel it. It has a crew of 60 to 80 rowers. This ship can carry 160 soldiers, but not for long distances, since there isn't room for supplies to support that many people. The warship cannot make sea voyages and sticks to the coast. It is not used for cargo. It moves about 2-1/2 miles per hour when being rowed or under sail.

    \subsection{Spellcasting And Services}
        Sometimes the best solution for a problem is to hire someone else to take care of it.
        \parhead{Coach Cab} The price given is for a ride in a coach that transports people (and light cargo) between towns. For a ride in a cab that transports passengers within a city, 1 copper piece usually takes you anywhere you need to go.
        \parhead{Hireling, Trained} The amount given is the typical daily wage for mercenary warriors, masons, craftsmen, scribes, teamsters, and other trained hirelings. This value represents a minimum wage; many such hirelings require significantly higher pay.
        \parhead{Hireling, Untrained} The amount shown is the typical daily wage for laborers, porters, cooks, maids, and other menial workers.
        \parhead{Messenger} This entry includes horse-riding messengers and runners. Those willing to carry a message to a place they were going anyway may ask for only half the indicated amount.
        \parhead{Road or Gate Toll} A toll is sometimes charged to cross a well-trodden, well-kept, and well-guarded road to pay for patrols on it and for its upkeep. Occasionally, a large walled city charges a toll to enter or exit (or sometimes just to enter).
        \parhead{Ship's Passage} Most ships do not specialize in passengers, but many have the capability to take a few along when transporting cargo. Double the given cost for creatures larger than Medium or creatures that are otherwise difficult to bring aboard a ship.
        \parhead{Spell} The indicated amount is how much it costs to get a spellcaster to cast a spell for you. This cost assumes that you can go to the spellcaster and have the spell cast at his or her convenience. If you want to bring the spellcaster somewhere to cast a spell you need to negotiate with him or her, and the default answer is no.
        \par The cost given is for a spell with no cost for a material component or focus component. If the spell includes a material component, add the cost of that component to the cost of the spell.
        \par If the spell has a focus component (other than a divine focus), add 1/10 the cost of that focus to the cost of the spell.
        \par Furthermore, if a spell has dangerous consequences, the spellcaster will certainly require proof that you can and will pay for dealing with any such consequences (that is, assuming that the spellcaster even agrees to cast such a spell, which isn't certain). In the case of spells that transport the caster and characters over a distance, you will likely have to pay for two castings of the spell, even if you aren't returning with the caster.
        \par In addition, not every town or village has a spellcaster of sufficient level to cast any spell. In general, you must travel to a small town (or larger settlement) to be reasonably assured of finding a spellcaster capable of casting 1st-level spells, a large town for 2nd-level spells, a small city for 3rd- or 4th-level spells, a large city for 5th- or 6th-level spells, and a metropolis for 7th- or 8th-level spells. Even a metropolis isn't guaranteed to have a local spellcaster able to cast 9th-level spells. Additionally, while you may find spellcasters able to cast some spells, there is no guarantee that they are able to cast the spell you desire. In general, the more common the spell, the more likely you are to find a spellcaster able to cast it.

\section{Consumable Items}\label{Consumable Items}

    Many substances exist that can aid adventurers.

    \subsection{Poisons}\label{Poisons}

        \begin{dtable*}
            \lcaption{Typical Poisons}\pdfbookmark[3]{Typical Poisons}{}
            \begin{dtabularx}{\textwidth}{l l l l l X l}
                \tb{Poison}         & \tb{Transmission} & \tb{Form} & \tb{Potency} & \tb{Primary Effect}                                 & \tb{Secondary Effect}                       & \tb{Type}  \\
                Nitharit            & Contact           & Powder    & 3            & Sickened                                            & Nauseated                                   & Plant      \\
                Arsenic             & Ingestion         & Powder    & 5            & Lose a hit point                                    & Gain a \glossterm{vital wound}, poison ends & Plant      \\
                Sassone leaf        & Contact           & Powder    & 6            & Lose a hit point                                    & Lose two hit points, poison ends            & Plant      \\
                Dragon bile         & Contact           & Liquid    & 10           & Sickened, lose a hit point                          & Nauseated, lose two hit points              & Venom      \\
                Insanity mist       & Ingestion         & Gas       & 10           & Disoriented                                         & Confused                                    & Alchemical \\
                Black lotus extract & Contact           & Liquid    & 15           & Lose a hit point from each successful poison attack & Gain a \glossterm{vital wound}              & Plant      \\
            \end{dtabularx}
        \end{dtable*}

        Poisons can deal damage, weaken creatures, or even kill them.
        Some effects which are not literally poisonous, such as animal venom or fungal spores, are considered poisons.

        \subsubsection{Poison Transmission}\label{Poison Transmission}\label{Transmission}

            There are three ways that poisons can be contracted.

            \parhead{Contact} A contact poison affects any creature that touches it with bare skin.
            \parhead{Ingestion} An ingestion poison affects any creature that eats, drinks, or breathes it, depending on the type of poison.
            Ingestion poisons have no effect when used to coat weapons.
            \parhead{Injury} An injury poison affects any creature injured by something carrying the poison.
            Almost all injury poisons take liquid form, and are typically used to coat weapons.
            An attack that deals no damage cannot transmit injury poisons.

        \subsubsection{Poison Forms}\label{Poison Forms}

            There are four forms of poison.

            \parhead{Gas} Gaseous poisons are difficult to store, but easy to affect foes with.
            \parhead{Liquid} Liquid poisons are the most common type of poison.
            Liquid poisons can be used to coat weapons, slipped into food, or simply thrown at foes.
            A dose of a liquid poison is usually about one ounce of the poison.
            \parhead{Pellet} Some rare poisons come in small, solid pellets or cubes.
            Typically, these pellets contain a powerful liquid poison that becomes inert quickly after being exposed.
            Pellet poisons cannot be used to coat weapons or thrown at foes, but can be slipped into food.
            \parhead{Powder} Poison in powder form cannot be used to coat weapons, but can be slipped into food or thrown at foes.

        \subsubsection{Poison Effects}\label{Poison Effects}

            Poisons can have a wide variety of effects, as determined by the type of poison used.
            However, most poison share certain common properties.

            \parhead{Becoming Poisoned}
            All poisons have a potency.
            Unless otherwise noted, a poison's accuracy is equal to its potency.
            When a creature first comes into contact with a poison, the poison makes an attack roll against the Fortitude defense of the poisoned creature.
            On a hit, the target becomes \glossterm{poisoned} and suffers the primary effect of the poison. 
            On a miss, the target is not \glossterm{poisoned}.

            Some attacks make the target poisoned if they hit the target.
            In that case, the ability's accuracy defines the poison's potency.

            \parhead{Poison Attacks}\label{Potency}\label{Poison Potency}
            At the end of each round, the poison makes an attack roll against the Fortitude defense of the poisoned creature.
            On a hit, the creature gets closer to the poison's secondary effect (see Secondary Effects, below).
            For every 10 points by which the attack hits, it counts as an additional successful attack for the purpose of reaching the poison's secondary effect.
            On a miss, the creature does not suffect the effect of the poison that round, and gets closer to resisting the poison (see Resisting Poisons, below).
            For every 10 points by which the attack misses, it counts as an additional failed attack for the purpose of resisting the poison.

            \parhead{Resisting Poisons}
            If a poisoned creature resists a poison three times, the creature stops being poisoned by that poison.
            Unless otherwise noted, this removes any lingering effects from the poison.

            \parhead{Primary Effects}
            Most poisons have primary effects.
            If the poison successfully attacks a poisoned creature, the creature suffers the poison's primary effect as long as the creature remains poisoned.
            Repeated primary effects, such as losing a hit point with each successful poison attack, occur whenever the trigger is met.
            % TODO: wording
            The first attack that inflicts the poison counts as a poison attack for this purpose.

            \parhead{Secondary Effects}
            Most poisons have a secondary effect based on the type of the poison.
            If the poison successfully attacks a poisoned creature three times,
                including the initial attack that inflicted the poison,
                the creature suffers the poison's secondary effect in addition to its primary effect.
            Some poisons end after the secondary effect is applied.
            Unless otherwise noted, a creature continues to be poisoned until it resists the poison three times or until the poison is removed by other means.

            \parhead{Multiple Doses}
            A creature can be affected by multiple doses of the same poison.
            This does not cause the same effect to occur multiple times, but each extra dose increases the potency of the poison by 1.

            A poison is considered the same if it has the same name and comes from the same source.
            For example, a creature bitten multiple times by the same giant spider suffers multiple doses of the same poison.
            A creature bitten multiple times by different giant spiders considers each spider's poison separately.

            \parhead{Poison Quality} Some poisons are unusually high or low quality.

        \subsubsection{Creating Poisons}\label{Creating Poisons}

            You can use the Craft (poison) skill to create poisons.
            To create a poison, you must make a Craft (poison) check against a \glossterm{difficulty rating} equal to 10 \add the poison's potency.
            For every 2 points by which you beat this \glossterm{difficulty rating}, the created poison's potency increases by 1.

            Creating a poison requires special materials.
            The type of materials required, and how those materials can be acquired, depend on the type of poison.

            \begin{itemize}
                \itemhead{Plant}: Plant-based poisons can typically be harvested by making a Survival check to search in appropriate terrain.
                    The \glossterm{difficulty rating} of this check is usually equal to 10 \add the potency of the poison.
                \itemhead{Venom}: Venom requires an appropriate body part from a creature -- often, poison it naturally produces.
                \itemhead{Alchemical}: Alchemical poisons require alchemical materials.
                    These cannot normally be found in nature.
                    In unusual circumstances, these components can be synthesized from natural chemicals or magical materials with a Craft (alchemy) check equal to 10 \add the potency of the poison.
            \end{itemize}
