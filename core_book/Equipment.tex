\chapter{Equipment}

\section{Weapons}

Weapons are grouped into several interlocking sets of categories. These categories pertain to what weapon group the weapon belongs to (axes, bows, and so on), the weapon's usefulness either in close combat (melee) or at a distance (ranged, which includes both thrown and projectile weapons), and its relative encumbrance (light, medium, or heavy).

\subsection{Weapon Groups}

Weapons are organized into thematically related categories called weapon groups. They are described in \trefnp{Weapon Groups}. For example, all axes belong to the ``axes'' weapon group. Some weapons can be found in multiple weapon groups. For example, a dagger is a simple weapon, a light blade, and a thrown weapon.

\begin{dtable!*}
    \lcaption{Weapon Groups}
    \begin{dtabularx}{\textwidth}{l >{\lcol}X >{\lcol}X}
        \tb{Group} & \tb{Weapons} & \tb{Exotic Weapons} \\
        \hline
        Armor weapons    & Heavy shield (and spiked), light shield (and spiked), spiked armor &                                              \\
        Axes             & Battleaxe, greataxe, handaxe, throwing axe                         & Double axe, dwarven urgrosh,  dwarven waraxe \\
        Blunt weapons    & Club, greatclub, mace, morningstar, quarterstaff, sap              & Gnome hooked hammer, maul                    \\
        Blades, heavy    & Falchion, greatsword, longsword, scimitar                          & Bastard sword, two-bladed sword              \\
        Blades, light    & Dagger, punching dagger, rapier, short sword                       & Kukri                                        \\
        Bows             & Longbow, shortbow                                                  &                                              \\
        Crossbows        & Hand crossbow, heavy crossbow, light crossbow, & repeating crossbows           \\
        Flexible Weapons & Flail, heavy flail, nunchaku                                       & Whip                                         \\
        Headed weapons   & Heavy pick, light hammer, light pick, sickle, warhammer            &                                              \\
        Monk weapons     & Kama, nunchaku, quarterstaff, sai, shuriken, siangham              &                                              \\
        Polearms         & Glaive, guisarme, halberd, ranseur, scythe                         &                                              \\
        Simple weapons   & Club, dagger, light crossbow, quarterstaff, unarmed strike         &                                              \\
        Spears           & Javelin, lance, longspear, shortspear, spear                       &                                              \\
        Thrown weapons   & Dagger, dart, handaxe, javelin, light hammer, shuriken, sling      & Bolas, net                                   \\
    \end{dtabularx}
\end{dtable!*}

\subsubsection{Weapon Proficiency}\label{Weapon Proficiency}
Each character is proficient with different weapon groups. These indicate the weapons that you can use effectively. You gain a \plus4 bonus to accuracy with a weapon you are proficient with. You can wield weapons you are not proficient with, but you cannot use them to defend yourself, which can cause you to be \defenseless.

\subsection{Weapon Encumbrance}
This is a measure of how much effort it takes to wield a weapon in combat. A weapon's encumbrance indicates whether a melee weapon, when wielded by a character of the weapon's size category, is considered a light weapon, a medium weapon, or a heavy weapon.

\parhead{Light} A light weapon can be used with more finesse than a medium weapon. The wielder can use Dexterity to attack with the weapon (in place of Strength and combat prowess). In addition, light weapons are easier to use while dual attacking (see \pcref{Dual Attacking}) or while grappling. Light weapons cannot be held in two hands.

\parhead{Medium} A medium weapon can be used in one hand. You can also hold a medium weapon in two hands. Changing grips to hold it in one hand or two hands can be done as a move action.

\parhead{Heavy} Two hands are required to wield a heavy weapon. You can hold it in one hand, but while doing so you cannot attack or defend yourself with it. This usually causes you to be \defenseless. Changing grips to hold it in one hand or two hands is a move action.

\subsection{Using Weapons in Two Hands}
Whenever you use a melee weapon in two hands, you gain a \plus1 bonus to damage.
This is included in the description of heavy weapons in \trefnp{Weapons}.

\subsection{Melee Weapons}
Melee weapons are used for making melee attacks against foes within your reach (see \pcref{Reach}). Most weapons are melee weapons.

\subsection{Ranged Weapons}
Ranged weapons are used to make attacks against distant foes within the range of your weapon. There are two kinds of ranged weapons: projectile weapons and thrown weapons.

\subsubsection{Range Increments}\label{Range Increment} All ranged weapons have a ``range increment'', which indicates the distance at which you can effectively attack with the weapon. Any attack against a target within your range increment takes no penalty. For each range increment by which the target is distant from you, you take a cumulative \minus2 penalty to your accuracy. A thrown weapon has a maximum range of five range increments. A projectile weapon can shoot out to ten range increments.

\subsubsection{Projectile Weapons} Projectile weapons fire ammunition at a target to deal damage. The ammunition generally breaks when used.

\subsubsection{Thrown Weapons}\label{Thrown Weapons} Thrown weapons are thrown at a target to deal damage. They generally do not break when thrown.

\parhead{Throwing Other Weapons} It is possible to throw a weapon that is not designed to be thrown. The range increment is 10 feet. You are treated as being nonproficient with the weapon. If the attack hits, the weapon deals its normal damage.

\parhead{Heavy Weapons} Heavy thrown weapons require a standard action to throw, rather than an attack action like normal. They are not always physically thrown with two hands, but they require the use of your entire body to propel the weapon, preventing you from using your hands for anything else. This can cause you to be \defenseless.

\subsubsection{Ranged Weapons in Melee}

You take a \minus4 penalty to accuracy with medium and large ranged weapons against creatures adjacent to you. In addition, you are usually \defenseless while using ranged weapons.

\parhead{Thrown Weapons in Melee}\label{Thrown Weapons in Melee} You cannot defend yourself with a weapon you are throwing, which can cause you to be \defenseless. To avoid this, you can attempt to use the weapon as a melee weapon. Some thrown weapon are not designed for use in melee, such as shurikens. When using such a weapon as a melee weapon, you are always treated as nonproficient with it, which means you cannot defend yourself with it and do not gain the \plus4 bonus for being proficient with your weapon.

\subsection{Weapon Size}\label{Weapon Size} Every weapon has a size category. This designation indicates the size of the creature for which the weapon was designed. Weapons for unusually large creatures deal more damage, while weapons for unusually small creatures deal less damage.

    Damage dice increase using the following pattern:
    \begin{itemize}
        \item 1d2
        \item 1d3
        \item 1d4
        \item 1d6
        \item 1d8
        \item 1d10
        \item 2d6
        \item 2d8
        \item 2d10
        \item 4d6
    \end{itemize}
    For each incremental increase, move one space down the list.
    Likewise, for each incremental decrease, move one space up the list.
    If you would increase a pool of damage dice beyond 4d6, simply add one more d6 for each additional increase.

    \parhead{Creature Size Increases}
    Larger creatures deal more damage with their weapons. Every creature size above Medium increases your damage dice by two increments, while every size below Medium decreases your damage dice by one increment.  These differences are summarized on \trefnp{Weapon Damage and Creature Size}.

\begin{dtable}
    \lcaption{Weapon Damage and Creature Size}
    \begin{dtabularx}{\columnwidth}{*{6}{l} >{\lcol}X}
        \tb{Medium} & \tb{Tiny} & \tb{Small} & \tb{Large} & \tb{Huge} & \tb{Gargantuan} & \tb{Colossal}\\
        \hline
        1d2  & \tdash  & 1   & 1d4  & 1d8  & 2d6  & 2d10  \\
        1d3  & 1   & 1d2 & 1d6  & 1d10 & 2d8  & 4d6  \\
        1d4  & 1d2 & 1d3 & 1d8  & 2d6  & 2d10 & 5d6 \\
        1d6  & 1d3 & 1d4 & 1d10 & 2d8  & 4d6  & 6d6  \\
        1d8  & 1d4 & 1d6 & 2d6  & 2d10 & 5d6  & 7d6  \\
        1d10 & 1d6 & 1d8 & 2d8  & 4d6  & 6d6  & 8d6 \\
    \end{dtabularx}
\end{dtable}

\parhead{Physical Size} Like all objects and creatures, weapons have a size category that represents how physically large they are. In general, a light weapon is an object two size categories smaller than the wielder, a medium weapon is an object one size category smaller than the wielder, and a heavy weapon is an object of the same size category as the wielder.

\parhead{Inappropriately Sized Weapons} A weapon's encumbrance is altered by one step for each size category of difference between the wielder's size and the size of the creature for which the weapon was designed. For example, a light weapon sized for a Large creature can be used by a Medium creature as if it had medium encumbrance. In addition, the wielder takes a \minus2 penalty to accuracy on physical attacks per size difference. If a weapon's encumbrance would be changed to something other than light, medium, or heavy by this alteration, the creature can't wield the weapon at all.

\subsection{Improvised Weapons} Sometimes objects not crafted to be weapons nonetheless see use in combat. Because such objects are not designed for this use, any creature that uses one in combat is considered to be nonproficient with it.

To determine the size category and appropriate damage for an improvised weapon, compare its relative size and damage potential to the weapon list to find a reasonable match. An improvised weapon will generally deal damage as if it were one size category smaller than a similar manufactured weapon. An improvised thrown weapon has a range increment of 10 feet.

\subsection{Drawing and Sheathing Weapons}\label{Drawing and Sheathing Weapons}
Sheathing any weapon is a move action. The time it takes to draw a weapon depends on how encumbering the weapon is. Drawing a light weapon is a swift action, while drawing a medium or heavy weapon is a move action. Drawing a hidden weapon of any type is a standard action.

\subsection{Natural Weapons}\label{Natural Weapons}
Every creature can attack with its body using an unarmed strike. Many monsters can attack more effectively with specific parts of their body, such as teeth and claws. These are called natural weapons. You are automatically proficient with any natural weapons you possess. Most humanoids possess no natural weapons. Natural weapons are described on \tref{Natural Weapons}.

A creature with multiple natural weapons of the same type can designate one as a primary attack and one as a secondary attack, allowing it to fight with both at once (see \pcref{Dual Attacking}). You are only considered to have one unarmed strike, so you cannot dual wield with only your unarmed strike (but see the unarmed warrior monk ability, \pref{Mnk:Unarmed Warrior}).

\subsection{Weapon Qualities}
Here is the format for weapon entries (given as column headings on \trefnp{Weapons}, below).

\parhead{Encumbrance} Describes whether the weapon is a light, medium, heavy, double, or ranged weapon.

\par This cost is the same for a Small or Medium version of the weapon. A Large version costs twice the listed price.
\parhead{Damage} The Damage column gives the damage dealt by the weapon on a successful hit. The damage given is for weapons used by Medium creatures. Weapons used by Small creatures have a damage die that is one size smaller, as shown on \trefnp{Weapon Damage and Creature Size}.

If two damage ranges are given, the weapon is a double weapon. Use the second damage figure given for the double weapon's other side.

\parhead{Range Increment} The range increment of the weapon.

\parhead{Type} Weapons are classified according to the type of damage they deal: bludgeoning, piercing, or slashing. Some monsters may be resistant or immune to attacks from certain types of weapons.

Some weapons deal damage of multiple types. If a weapon is of two types, the damage it deals is not half one type and half another; all of it is both types. Therefore, a creature would have to be immune to both types of damage to ignore any of the damage from such a weapon.

In other cases, a weapon can deal either of two types of damage. In a situation when the damage type is significant, the wielder can choose which type of damage to deal with such a weapon.
\parhead{Cost} This value is the weapon's cost in gold pieces (gp) or silver pieces (sp). The cost includes miscellaneous gear that goes with the weapon.
\parhead{Weight} This column gives the weight of a Medium version of
the weapon. Halve this number for Small weapons, and double it for
Large weapons.

\parhead{Special} Some weapons have special properties. See the weapon
descriptions for details.

\onecolumn
\begin{longtabuwrapper}
    \begin{longtabu}{p{12em} c c >{\ccol}p{10em} c c >{\ccol}X}
        \lcaption{Weapons}\\
        \tb{Name} & \tb{Encumbrance} & \tb{Damage} & \tb{Damage Type\fn{1}} & \tb{Cost} & \tb{Weight\fn{2}} & \tb{Special} \\
        Armor weapons &&&&&& \\
        \tind Shield, heavy\fn{3} & Medium & 1d4 & Bludgeoning & special & special & Forceful \\
        \tind Shield, light\fn{3} & Light & 1d3 & Bludgeoning & special & special & Forceful \\
        \tind Spiked armor\fn{3} & Light & 1d6 & Piercing & special & special & Grappling \\
        \tind Spiked shield, heavy\fn{3} & Medium & 1d6 & Piercing & special & special & Forceful \\
        \tind Spiked shield, light\fn{3} & Light & 1d4 & Piercing & special & special & Forceful \\

        Axes &&&&&& \\
        \tind Axe, throwing & Light & 1d6 & Slashing & 8 gp & 2 lb. & Throwing (10 ft.) \\
        \tind Battleaxe & Medium & 1d8 & Slashing & 10 gp & 6 lb. & \tdash \\
        \tind Greataxe & Heavy & 1d10\plus1 & Slashing & 20 gp & 12 lb. & \tdash \\
        \tind Handaxe & Light & 1d6 & Slashing & 6 gp & 3 lb. & \tdash \\
        \tind Waraxe, dwarven & Heavy & 1d10\plus1 & Slashing & 75 gp & 8 lb. & Exotic Grip \\

        Blades, heavy &&&&&& \\
        \tind Falchion & Heavy & 1d10\plus1 & Slashing & 50 gp & 8 lb. & \tdash \\
        \tind Greatsword & Heavy & 1d10\plus1 & Slashing & 25 gp & 8 lb. & \tdash \\
        \tind Katana\fn{3} & Medium & 1d10 & Slashing & 75 gp & 6 lb. & \tdash \\
        \tind Longsword & Medium & 1d8 & Slashing & 15 gp & 4 lb. & \tdash \\
        \tind Scimitar & Medium & 1d8 & Slashing & 15 gp & 4 lb. & \tdash \\

        Blades, light &&&&&& \\
        \tind Dagger & Light & 1d4 & Piercing or slashing & 2 gp & 1 lb. & Small, Throwing (10 ft.) \\
        \tind Dagger, punching & Light & 1d4 & Piercing & 2 gp & 1 lb. & Small \\
        \tind Rapier & Medium & 1d6 & Piercing & 20 gp & 2 lb. & Disarming \\
        \tind Sword, short & Light & 1d6 & Piercing or slashing & 10 gp & 2 lb. & \tdash \\

        Blunt weapons &&&&&& \\
        \tind Club & Medium & 1d6 & Bludgeoning & \tdash & 3 lb. & \tdash \\
        \tind Greatclub & Heavy & 1d10\plus1 & Bludgeoning & 5 gp & 8 lb. & \tdash \\
        \tind Mace & Light & 1d6 & Bludgeoning & 12 gp & 8 lb. & \tdash \\
        \tind Morningstar & Medium & 1d8 & Bludgeoning and piercing & 8 gp & 6 lb. & \tdash \\
        \tind Quarterstaff & Heavy & 1d6/1d6 & Bludgeoning & \tdash & 4 lb. & Double \\
        \tind Sap & Light & 1d6 & Bludgeoning & 1 gp & 2 lb. & Nonlethal \\

        Bows &&&&&& \\
        \tind Longbow\fn{3} & Heavy (Ranged) & 1d8 & Piercing & 40 gp & 3 lb. & Projectile (100 ft.) \\
        \tind Shortbow\fn{3} & Medium (Ranged) & 1d6 & Piercing & 30 gp & 2 lb. & Projectile (50 ft.) \\
        \tind Arrows (20) & \tdash & \tdash & \tdash & 1 gp & 3 lb. & Ammunition \\

        Crossbows &&&&&& \\
        \tind Crossbow, hand\fn{3} & Light (Ranged) & 1d6 & Piercing & 100 gp & 2 lb. & Projectile (50 ft.) \\
        \tind Crossbow, heavy\fn{3} & Heavy (Ranged) & 1d10 & Piercing & 50 gp & 8 lb. & Projectile (100 ft.) \\
        \tind Crossbow, light\fn{3} & Medium (Ranged) & 1d8 & Piercing & 40 gp & 4 lb. & Projectile (50 ft.) \\
        \tind Bolts, crossbow (10) & \tdash & \tdash & \tdash & 1 gp & 1 lb. & Ammunition \\
        \tind Bolts, hand (10) & \tdash & \tdash & \tdash & 1 gp & 1/2 lb. & Ammunition \\

        Flexible weapons &&&&&& \\
        \tind Flail  & Medium & 1d8 & Bludgeoning & 8 gp & 5 lb. & Disarming, Tripping \\
        \tind Flail, heavy & Heavy & 1d10\plus1 & Bludgeoning & 15 gp & 10 lb. & Disarming, Tripping \\

        Headed weapons &&&&&& \\
        \tind Hammer, light & Light & 1d4 & Bludgeoning & 1 gp & 2 lb. & Throwing (20 ft.) \\
        \tind Pick, heavy & Medium & 1d8 & Piercing & 8 gp & 6 lb. & Unbalanced \\
        \tind Pick, light & Light & 1d6 & Piercing & 4 gp & 3 lb. & Unbalanced \\
        \tind Sickle & Light & 1d6 & Slashing & 6 gp & 2 lb. & Tripping \\
        \tind Warhammer & Medium & 1d8 & Bludgeoning & 12 gp & 5 lb. & \tdash \\

        Monk weapons &&&&&& \\
        \tind Kama & Light & 1d6 & Slashing & 2 gp & 2 lb. & Tripping \\
        \tind Nunchaku & Light & 1d6 & Bludgeoning & 2 gp & 2 lb. & Disarming, Unbalanced \\
        \tind Quarterstaff & Heavy & 1d6/1d6 & Bludgeoning & \tdash & 4 lb. & Double \\
        \tind Sai & Light & 1d4 & Piercing or bludgeoning & 1 gp & 1 lb. & Disarming \\
        \tind Shuriken (5) & Light (Ranged) & 1d4 & Piercing and slashing & 1 gp & 1/2 lb. & Ammunition, Thrown (10 ft.) \\
        \tind Siangham & Light & 1d6 & Piercing & 3 gp & 1 lb. & \tdash \\

        Polearms &&&&&& \\
        \tind Glaive & Heavy & 1d10\plus1 & Slashing & 8 gp & 10 lb. & Reach \\
        \tind Guisarme & Heavy & 1d10\plus1 & Slashing & 9 gp & 12 lb. & Reach, Tripping \\
        \tind Halberd & Heavy & 1d10\plus1 & Piercing or slashing & 10 gp & 12 lb. & Reach \\
        \tind Quarterstaff & Heavy & 1d6/1d6 & Bludgeoning & \tdash & 4 lb. & Double \\
        \tind Ranseur & Heavy & 1d10\plus1 & Piercing & 10 gp & 12 lb. & Disarming, Reach \\
        \tind Scythe & Heavy & 1d10\plus1 & Slashing & 18 gp & 10 lb. & \tdash \\

        Simple weapons &&&&&& \\
        \tind Club & Medium & 1d6 & Bludgeoning & \tdash & 3 lb. & \tdash \\
        \tind Crossbow, light\fn{3} & Medium (Ranged) & 1d8 & Piercing & 35 gp & 4 lb. & Projectile (100 ft.) \\
        \tind Dagger & Light & 1d4 & Piercing or slashing & 2 gp & 1 lb. & Small, Thrown (10 ft.) \\
        \tind Quarterstaff & Heavy & 1d6/1d6 & Bludgeoning & \tdash & 4 lb. & Double \\
        \tind Unarmed strike & Light & 1d3 & Bludgeoning & \tdash & \tdash & Nonlethal, Unarmed \\

        Spears &&&&&& \\
        \tind Javelin & Medium (Ranged) & 1d6 & Piercing & 1 gp & 2 lb. & Thrown (30 ft.) \\
        \tind Lance & Heavy & 1d8\plus1 & Piercing & 10 gp & 10 lb. & Charging, Reach \\
        \tind Longspear & Heavy & 1d8\plus1 & Piercing & 5 gp & 9 lb. & Bracing, Reach \\
        \tind Shortspear & Light & 1d6 & Piercing & 1 gp & 3 lb. & Bracing, Thrown (20 ft.) \\
        \tind Spear & Medium & 1d8 & Piercing & 2 gp & 6 lb. & Bracing, Thrown (20 ft.) \\

        Thrown weapons &&&&&& \\
        \tind Axe, throwing & Light & 1d6 & Slashing & 8 gp & 2 lb. & Thrown (10 ft.) \\
        \tind Dagger & Light & 1d4 & Piercing or slashing & 2 gp & 1 lb. & Small, Thrown (10 ft.) \\
        \tind Dart (5) & Light (Ranged) & 1d4 & Piercing & 1 gp & 1/2 lb. & Ammunition, Thrown (20 ft.) \\
        \tind Hammer, light & Light & 1d4 & Bludgeoning & 1 gp & 2 lb. & Thrown (20 ft.) \\
        \tind Javelin & Medium (Ranged) & 1d6 & Piercing & 1 gp & 2 lb. & Thrown (30 ft.) \\
        \tind Shuriken (5) & Light (Ranged) & 1d4 & Piercing and slashing & 1 gp & 1/2 lb. & Ammunition, Thrown (10 ft.) \\
        \tind Sling\fn{3} & Light (Ranged) & 1d4 & Bludgeoning & 2 gp & 0 lb. & Projectile (50 ft.) \\
        \tind Bullets, sling (20) & \tdash & \tdash & \tdash & 1 gp & 5 lb. & Ammunition \\

        Unarmed weapons\label{Unarmed Weapons} &&&&&&\\
        \tind Claw Sheath\fn{3} & \tdash & \tdash & \tdash & 50 gp & 3 lb. & Special \\
        \tind Gauntlet & Light & 1d3 & Bludgeoning & 2 gp & 1 lb. & Unarmed \\
        \tind Gauntlet, spiked & Light & 1d4 & Piercing & 5 gp & 1 lb. & Unarmed \\
        \tind Unarmed strike & Light & 1d3 & Bludgeoning & \tdash & \tdash & Nonlethal, Unarmed \\
    \end{longtabu}
\end{longtabuwrapper}
\twocolumn

\begin{dtable!*}
    \begin{dtabularx}{\textwidth}{p{12em} c c >{\ccol}p{10em} c c >{\ccol}X}
        \tb{Exotic Weapons} & \tb{Encumbrance} & \tb{Damage} & \tb{Damage Type\fn{1}} & \tb{Cost} & \tb{Weight\fn{2}} & \tb{Special} \\
        \hline
        Armor &&&&&& \\
        Axes &&&&&& \\
        \tind Axe, orc double & Heavy & 1d8/1d8 & Slashing & 60 gp & 15 lb. & Double \\
        \tind Urgrosh, dwarven\fn{3} & Heavy & 1d8/1d6 & Slashing or piercing & 50 gp & 12 lb. & Bracing, Double \\
        Blunt weapons &&&&&& \\
        Blades, heavy &&&&&& \\
        \tind Sword, two-bladed & Heavy & 1d8/1d8 & Slashing & 100 gp & 10 lb. & Double \\
        Blades, light &&&&&& \\
        \tind Kukri & Light & 1d8 & Slashing & 8 gp & 2 lb. & \\
        Bows &&&&&& \\
        Crossbows &&&&&& \\
        \tind Crossbow, repeating heavy\fn{3} & Heavy (Ranged) & 1d10 & Piercing & 400 gp & 12 lb. & Projectile (100 ft.) \\
        \tind Crossbow, repeating light\fn{3} & Medium (Ranged) & 1d8 & Piercing & 250 gp & 6 lb. & Projectile (50 ft.) \\
        \tind Bolts, repeating (5) & \tdash & \tdash & \tdash & 1 gp & 1 lb. & Ammunition \\
        Flexible weapons &&&&&& \\
        \tind Flail, dire & Heavy & 1d8/1d8 & Bludgeoning & 90 gp & 10 lb. & Disarming, Double, Tripping \\
        \tind Whip\fn{3} & Light & 1d3 & Slashing & 1 gp & 2 lb. & Disarming, Nonlethal, Tripping \\
        Headed weapons &&&&&& \\
        \tind Hammer, gnome hooked\fn{3} & Heavy & 1d8/1d6 & Bludgeoning or piercing & 20 gp & 6 lb. & Double, Impact, Tripping \\
        Monk weapons &&&&&& \\
        Polearms &&&&&& \\
        Simple weapons &&&&&& \\
        Spear &&&&&& \\
        \tind Urgrosh, dwarven\fn{3} & Heavy & 1d8/1d6 & Slashing or piercing & 50 gp & 12 lb. & Bracing, Double, Impact \\
        Thrown weapons &&&&&& \\
        \tind Bolas & Light (Ranged) & 1d4\fn{3} & Bludgeoning & 5 gp & 2 lb. & Thrown (10 ft.), Tripping \\
        \tind Net\fn{3} & Medium (Ranged) & \tdash & \tdash & 20 gp & 6 lb. & Thrown (10 ft.) \\
        Unarmed weapons &&&&&&\\
    \end{dtabularx}
    1 When two types are given, the weapon is both types if the entry specifies ``and,'' or either type (attacker's choice) if the entry specifies ``or.'' \\
    2 Weight figures are for Medium weapons. A Small weapon weighs half as much, and a Large weapon weighs twice as much. \\
    3 This weapon has special rules. \\
\end{dtable!*}

\begin{dtable!*}
    \lcaption{Natural Weapons}
    \begin{dtabularx}{\textwidth}{p{12em} c c >{\ccol}p{15em} >{\ccol}X}
        \tb{Natural Weapons} & \tb{Encumbrance} & \tb{Damage} & \tb{Damage Type\fn{2}} & \tb{Special} \\
        \hline
        Bite            & Medium & 1d8       & Piercing and bludgeoning & \tdash   \\
        Claw            & Light  & 1d6       & Slashing and piercing    & \tdash     \\
        Constrict\fn{2} & Heavy  & 1d10      & Bludgeoning              & \tdash       \\
        Gore            & Heavy  & 1d8       & Piercing                 & Forceful \\
        Slam            & Medium & 1d8       & Bludgeoning              & \tdash       \\
        Talon           & Light  & 1d6       & Piercing                 & \tdash     \\
        Unarmed Strike  & Light  & 1d3\fn{3} & Bludgeoning              & Unarmed  \\
    \end{dtabularx}
    1 When two types are given, the weapon is both types if the entry specifies ``and,'' or either type (attacker's choice) if the entry specifies ``or''. \\
    2 This attack can only be used against a foe you are grappling with. \\
\end{dtable!*}

\subsection{Weapon Tags}
Some weapons found on \trefnp{Weapons} have tags that indicate that they have special abilities. The list of abilities that weapons can have is given below.
\parhead{Ammunition} This weapon is designed to thrown or fired by a projectile weapon in large quantities. It is cheaper to buy and craft, but ammunition is usually \glossterm{broken} after being fired.
\parhead{Bracing} As a move action, you can brace this weapon against a charge for 1 round. While you are bracing your weapon, you gain a \plus5 bonus to damage on attacks with that weapon against creatures that charge you that round.
\parhead{Charging} This weapon deals double damage when used from the back of a charging mount.
\parhead{Disarming} You gain a \plus2 bonus to accuracy on disarm attacks using this weapon.
\parhead{Double} This weapon has more than one striking surface. You can fight with both ends simultaneously, just like wielding two weapons at once (see \pcref{Dual Attacking}). Alternately, you can attack with one end at a time. If you have the ability to use a double weapon in one hand, you can only fight with one end at a time, not both.
\parhead{Exotic Grip} If you have proficiency with exotic weapons, you can use this in one hand. While wielding it in one hand, you do not gain the \plus1 bonus to damage from wielding the weapon in two hands.
\parhead{Finesse} You can apply your Dexterity instead of your Strength when determining your accuracy with physical attacks using the weapon, even if it isn't a light weapon for you.
This property has no effect if the weapon is not sized appropriately for you.
\parhead{Forceful} You can use this weapon to make shove attacks to push people away from you. This allows you to add the weapon's proficiency bonus to accuracy with the shove attack.
\parhead{Grappling} You gain a \plus2 bonus to accuracy on physical attacks with this weapon in a grapple.
\parhead{Nonlethal} This weapon deals nonlethal damage rather than lethal damage. See \pcref{Nonlethal Damage}.
\parhead{Projectile} This weapon fires projectiles at range. Projectile weapons have a \glossterm{range increment} listed in their description, which indicates the distance they can be easily fired. Projectile weapons must be reloaded. The time required to reload a projectile weapon is given in the weapon description.
Unless otherwise noted, projectile weapons cannot be used while \prone.
\parhead{Reach} This weapon strikes at double your natural reach (so 10 feet for a typical Small or Medium creature). However, it cannot attack a creature within your natural reach.

Reach weapons can held using a different grip to strike nearby foes. This is called ``short hafting''. While short hafting a reach weapon, you ignore the weapon's reach property, but you take a \minus4 penalty to accuracy with it.
\parhead{Small} This weapon is unusually small. It is one size category smaller than normal for a light weapon (that is, three size categories smaller than the creature it is intended for). This makes it easier to conceal (see \pcref{Sleight of Hand}).
\parhead{Throwing} This weapon is designed to be thrown. Throwing weapons have a range increment listed in their description, which indicates the distance they can be easily thrown. See \pcref{Thrown Weapons}.
\parhead{Tripping} You can use this weapon to make trip attacks, allowing you to use your accuracy with your weapon (including the \plus4 proficiency bonus) in place of your unarmed accuracy.
\parhead{Unarmed} This weapon is used as part of an unarmed strike. It cannot be disarmed. Unless you have the Improved Unarmed Strike feat (see \featref{Improved Unarmed Strike}), you can't defend yourself with this weapon, which usually makes you \defenseless.
\parhead{Unbalanced} If you roll either a 1 or 2 when attacking this weapon, you critically fail the attack (see \pcref{Critical Success and Failure}).

\subsection{Weapon Descriptions}
Some weapons in \trefnp{Weapons} have additional abilities which are described below.
\parhead{Claw Sheath} A claw sheath is not a weapon in itself, but a covering over a single claw that a creature can use to make claw attacks. Claw sheaths do not grant a creature without claws a claw attack. Normally, a sheath does not improve the claw attack, but magical claw sheaths can be made which grant bonuses to attacks made with the claw. It may be possible to find or craft unusual sheaths for natural weapons other than claws.
\parhead{Crossbow, Hand} You can draw a hand crossbow back by hand. Loading a hand crossbow is a move action that requires one hand (but not the hand wielding the crossbow).
\par You can fire a crossbow while \prone without penalty.
\parhead{Crossbow, Heavy} You draw a heavy crossbow back by turning a small winch. Loading a heavy crossbow is a full-round action that requires both hands.
\par You can fire a crossbow while \prone without penalty.
\parhead{Crossbow, Light} You draw a light crossbow back by pulling a lever. Loading a light crossbow is a move action that requires both hands.
\par You can fire a crossbow while \prone without penalty.
\parhead{Crossbow, Repeating} The repeating crossbow (whether heavy or light) holds 10 crossbow bolts. As long as it holds bolts, you can reload it as a \glossterm{free action} by pulling the reloading lever. Loading a new case of 10 bolts is a \glossterm{standard action} that requires both hands.
\par You can fire a crossbow while \prone without penalty.
\parhead{Hammer, Gnome Hooked} This weapon has a hammer head which deals 1d8 points of damage, and a hook which deals 1d6 points of damage. The hook is a tripping weapon.
\parhead{Katana} This weapon must be held in two hands unless you have proficiency with exotic weapons, granting the normal \plus1 bonus to damage for wielding a weapon in two hands.
\parhead{Longbow} You need both hands to fire a bow, regardless of its size. One hand is free when not firing the bow. Loading a bow is a free action that requires both hands. A longbow is too unwieldy to use while you are mounted.
\par When attacking with a bow, you take a \minus4 penalty to accuracy against creatures that threaten you.
\parhead{Net} A net is used to entangle enemies. When you throw a net, you make a ranged touch attack against your target. If you hit, the target is \entangled. If you control the trailing rope by succeeding on an opposed Strength check while holding it, the entangled creature can move only within the limits that the rope allows.
\par An entangled creature can escape with a DR 20 Escape Artist check (a full-round action). The net has 5 hit points and can be burst with a DR 15 Strength check (also a full-round action).
\par A net is useful only against creatures within one size category of you.
\par A net must be folded to be thrown effectively. The first time you throw your net in a fight, you make a normal ranged touch attack roll. After the net is unfolded, you take a \minus4 penalty to accuracy with it. It takes 2 full-round actions for a proficient user to fold a net and twice that long for a nonproficient one to do so.
\parhead{Rapier} A rapier is treated as a medium weapon if it is used as a secondary weapon when dual attacking (see \pcref{Dual Attacking}).
\parhead{Shield, Heavy or Light} You can bash with a shield instead of using it for defense. See Armor for details.
\parhead{Shortbow} You need both hands to fire a bow, regardless of its size. One hand is free when not firing the bow. Loading a bow is a free action that requires both hands.
\par When attacking with a bow, you take a \minus4 penalty to accuracy against creatures that threaten you.
\parhead{Sling} You can fire, but not load, a sling with one hand. Loading a sling is a free action that requires both hands.
\par You can hurl ordinary stones with a sling, but stones are not as dense or as round as bullets. You take a \minus1 penalty to accuracy and damage with ordinary stones.
\parhead{Spiked Armor} You can outfit your armor with spikes, which can deal damage in a grapple. See Armor for details.
\parhead{Spiked Shield, Heavy or Light} You can bash with a spiked shield instead of using it for defense. See Armor for details.

\parhead{Unarmed Strike} Anyone can attack unarmed, but it is more dangerous than attacking with a weapon. See \pcref{Unarmed Combat}, for details.

\parhead{Urgrosh, Dwarven} This weapon has an axe head which deals 1d8 points of damage, and a spear which deals 1d6 points of damage. The spear is a bracing weapon. You must be proficient with axes to use the axe head, and proficient with spears to use the spear head.
\parhead{Whip} A whip is treated as a light melee weapon with 15 foot reach. However, you can't defend yourself with a whip, which can make you \defenseless, and you don't threaten the area into which you can make an attack. In addition, unlike most other weapons with reach, you can use it against foes anywhere within your reach (including adjacent foes).

\section{Armor}

Most characters use armor to protect themselves. There are two kinds of armor: body armor, such as full plate armor, and shields.

\subsection{Armor Qualities}
\par Here is the format for armor entries (given as column headings on \trefnp{Armor and Shields}, below).

\parhead{Encumbrance} All armor restricts a character's movement to some degree. The heavier the armor, the more restrictive it is. Shields are divided into light shields (including bucklers), heavy shields, or tower shields, while body armor can be light, medium, or heavy. Many classes are not proficient with heavier kinds of armor.

\parhead{Cost} The cost of the armor for Small or Medium humanoid
creatures. See \trefnp{Armor for Unusual Creatures}, below, for armor prices for other creatures.
\parhead{Armor/Shield Bonus} Body armor improves your Armor defense, while shields improve all your physical defenses. Wearing multiple suits of armor or wielding multiple shields does not improve your defenses any further.

\parhead{Dexterity} Heavy body armor limits mobility and agility, halving the character's Dexterity. This halving is applied in addition to the armor's \glossterm{encumbrance penalty} (if any). A Dexterity penalty is not halved.

Your character's encumbrance (the amount of gear he or she carries) may also affect your character's Dexterity.
\subparhead{Shields} Most shields do not affect a character's Dexterity. However, a tower shield halves a character's Dexterity. This is not cumulative with the halving from wearing medium or heavy armor.
\parhead{Encumbrance Penalty} All armor has an associated \glossterm{encumbrance penalty}.
A character's encumbrance penalty applies to all Strength and Dexterity-based skill checks the character makes.
A character's encumbrance (the amount of gear carried, including armor) may also apply an encumbrance penalty.
\subparhead{Shields} If a character is wearing armor and using a shield, both armor check penalties apply.
\subparhead{Nonproficiency} A character who uses armor she is not proficient with also applies the armor's encumbrance penalty to her accuracy with physical attacks.

\subparhead{Sleeping in Armor} A character who sleeps overnight in medium or heavy body armor is automatically \fatigued the next day. Sleeping in light armor does not cause fatigue.
\parhead{Arcane Spell Failure} Armor interferes with the gestures that a spellcaster must make to cast an arcane spell that has a somatic component. Arcane spellcasters face the possibility of arcane spell failure if they're wearing armor.

\subparhead{Casting an Arcane Spell in Armor} A character who casts an arcane spell while wearing armor must usually make an arcane spell failure roll. The number in the Arcane Spell Failure Chance column on Table: Armor and Shields is the chance that the spell fails and is ruined. If the spell lacks a somatic component, however, it can be cast with no chance of arcane spell failure.

\subparhead{Shields} If a character is wearing armor and using a shield, add the two numbers together to get a single arcane spell failure chance.

\parhead{Weight} This column gives the weight of the armor sized for a Medium wearer. Armor fitted for Small characters weighs half as much, and armor for Large characters weighs twice as much.

\subsubsection{Moving in Armor}
    The heavier the armor, the more it slows your movement. Light armor has no effect on your movement speed, though it does impose \glossterm{encumbrance penalties}. Medium and heavy armor reduce your movement speed by five feet (to a minimum of 5 feet).

\subsection{Getting Into And Out Of Armor}
The time required to don armor depends on its type; see \trefnp{Donning Armor}. Donning armor of any kind takes both hands.
\parhead{Don} This column tells how long it takes a character to put the armor on. (One minute is 10 rounds.) Readying (strapping on) a shield is only a move action.
\parhead{Don Hastily} This column tells how long it takes to put the armor on in a hurry. The encumbrance penalty and defense bonus for hastily donned armor are each 1 point worse than normal.
\parhead{Remove} This column tells how long it takes to get the armor off. Loosing a shield (removing it from the arm and dropping it) is only a move action.

\begin{dtable}
    \lcaption{Donning Armor}
    \begin{dtabularx}{\columnwidth}{>{\lcol}X c c c}
        \tb{Armor Type} & \tb{Don} & \tb{Don Hastily} & \tb{Remove} \\
        \hline
        Shield (any)      & 1 move action   & n/a             & 1 move action           \\
        Light body armor  & 1 minute        & 5 rounds        & 1 minute\fn{1}          \\
        Medium body armor & 4 minutes\fn{1} & 1 minute        & 1 minute\fn{1}          \\
        Heavy body armmor & 4 minutes\fn{2} & 4 minutes\fn{1} & 1d4\plus1 minutes\fn{1} \\
    \end{dtabularx}
    1 If the character has some help, cut this time in half. A single character doing nothing else can help one or two adjacent characters. Two characters can't help each other don armor at the same time. \\
    2 The wearer must have help to don this armor. Without help, it can be donned only hastily.
\end{dtable}

\begin{dtable!*}
    \lcaption{Armor and Shields}
    \begin{dtabularx}{\textwidth}{>{\lcol}X >{\ccol}p{6em} >{\ccol}p{5em} >{\ccol}p{6em} >{\ccol}p{7em} c c c}
        \hline
        \tb{Armor} & \tb{Armor/Shield Bonus} & \tb{Dex Modifier} & \tb{Encumbrance Penalty} & \tb{Arcane Spell Failure Chance} & \tb{Material} & \tb{Cost} & \tb{Weight\fn{1}} \\
        Light armor &  &  &  &  &  &  &  \\
        \tind Leather          & \plus1        & 1\x    & \minus1  & 10\%         & Leather       & 10 gp      & 15 lb.      \\
        \tind Studded leather  & \plus2        & 1\x    & \minus2  & 15\%         & Leather       & 25 gp      & 20 lb.      \\
        \tind Chain shirt      & \plus2        & 1\x    & \minus2  & 20\%         & Metal         & 40 gp      & 25 lb.      \\
        Medium armor           &               &        &          &              &               &            &             \\
        \tind Hide             & \plus3        & 1\x    & \minus3  & 20\%         & Leather       & 15 gp      & 25 lb.      \\
        \tind Scale mail       & \plus3        & 1\x    & \minus4  & 25\%         & Metal         & 50 gp      & 30 lb.      \\
        \tind Breastplate      & \plus4        & 1\x    & \minus4  & 25\%         & Metal         & 150 gp     & 30 lb.      \\
        Heavy armor            &               &        &          &              &               &            &             \\
        \tind Half-plate       & \plus5        & 1/2\x  & \minus6  & 40\%         & Metal         & 200 gp     & 50 lb.      \\
        \tind Full plate       & \plus6        & 1/2\x  & \minus6  & 35\%         & Metal         & 500 gp     & 50 lb.      \\
        Shields                &               &        &          &              &               &            &             \\
        \tind Buckler          & \plus1        & \tdash & \minus1  & 5\%          & Metal         & 15 gp      & 5 lb.       \\
        \tind Shield, light    & \plus2        & \tdash & \minus2  & 5\%\fn{2}    & Metal or wood & 5 gp       & 5 lb.       \\
        \tind Shield, heavy    & \plus3        & \tdash & \minus3  & 15\%\fn{2}   & Metal or wood & 15 gp      & 10 lb.      \\
        \tind Shield, tower    & \plus4\fn{3}  & 1/2\x  & \minus10 & 50\%\fn{2}   & Metal         & 30 gp      & 45 lb.      \\
        Extras                 &               &        &          &              &               &            &             \\
        \tind Armor spikes     & \minus1\fn{4} & \tdash & \minus2  & \tdash       & Metal         & \plus50 gp & \plus10 lb. \\
        \tind Gauntlet, locked & \tdash        & \tdash & Special  & \tdash\fn{2} & Metal         & 8 gp       & \plus5 lb.  \\
        \tind Shield spikes    & \tdash        & \tdash & \minus1  & \tdash       & Metal         & \plus10 gp & \plus5 lb.  \\
    \end{dtabularx}
    1 Weight figures are for armor sized to fit Medium characters. Armor fitted for Small characters weighs half as much, and armor fitted for Large characters weighs twice as much. \\
    2 Hand not free to cast spells. \\
    3 Tower shields can grant you cover. See the description. \\
    4 Armor spikes reduce the defense bonus granted by the armor they are put on by 1. \\
\end{dtable!*}

\subsection{Armor Descriptions}
Any special benefits or accessories to the types of armor found on \trefnp{Armor and Shields} are described below.
\parhead{Armor Spikes} You can have spikes added to your armor, which allow you to deal extra piercing damage (see \tref{Weapons}) on a successful grapple or overrun attack. If you are not proficient with the spikes, you only deal the damage if you succeed at the attack by 10 or more.
\par Magical abilities on a suit of armor does not improve the spikes' effectiveness, but the spikes can be made into magic weapons in their own right.
\parhead{Breastplate} It comes with a helmet and greaves.
\parhead{Buckler} This small metal shield is worn strapped to your forearm.
You can hold items in a hand holding a buckler.
However, if you wield weapons or otherwise take actions using the arm bearing the buckler, you do not gain the buckler's defensive bonus during that time.
\par You can't bash someone with a buckler.
\parhead{Chain Shirt} A chain shirt comes with a steel cap.
\parhead{Full Plate} The suit includes gauntlets, heavy leather boots, a visored helmet, and a thick layer of padding that is worn underneath the armor. Each suit of full plate must be individually fitted to its owner by a master armorsmith, although a captured suit can be resized to fit a new owner with a day of work and a DR 25 Craft (metalworking) check. The new owner must still be of the same size category as the size category that the suit was designed for.
\parhead{Gauntlet, Locked} This armored gauntlet has small chains and braces that allow the wearer to attach a weapon to the gauntlet so that it cannot be dropped easily. It provides a \plus10 bonus on any roll made to keep from being disarmed in combat. Removing a weapon from a locked gauntlet or attaching a weapon to a locked gauntlet is a full-round action.
\par The price given is for a single locked gauntlet. The weight given applies only if you're wearing a breastplate, light armor, or no armor. Otherwise, the locked gauntlet replaces a gauntlet you already have as part of the armor.
\par While the gauntlet is locked, you can't use the hand wearing it for casting spells or employing skills. (You can still cast spells with somatic components, provided that your other hand is free.)
Like a normal gauntlet, a locked gauntlet lets you deal lethal damage rather than nonlethal damage with an unarmed strike.
\parhead{Half-Plate} The suit includes gauntlets.
\parhead{Scale Mail} The suit includes gauntlets.
\parhead{Shield, Heavy, Wooden or Steel} You strap a shield to your forearm and grip it with your hand. A heavy shield is so heavy that you can't use your shield hand for anything else.
\subparhead{Wooden or Steel} Wooden and steel shields offer the same basic protection, though they respond differently to special attacks.
\subparhead{Shield Bash Attacks} You can bash an opponent with a heavy shield, using it as a medium bludgeoning weapon. See \trefnp{Weapons} for the damage dealt by a shield bash.  If you use your shield as a weapon, you lose its defense bonus until your next action (usually until the next round). Magical abilities on a shield do not affect shield bash attacks made with it, but the shield can be made into a magic weapon in its own right.
\parhead{Shield, Light, Wooden or Steel} You strap a shield to your forearm and grip it with your hand. A light shield's weight lets you carry other items in that hand, although you cannot use weapons or cast spells with it.
\subparhead{Wooden or Steel} Wooden and steel shields offer the same basic protection, though they respond differently to special attacks.
\parhead{Shield, Tower} This massive wooden shield is nearly as tall as an average human. In most situations, a tower shield provides the indicated bonus to your physical defenses. However, when you take the total defense action with a tower shield, you can treat the shield as a wall along one edge of your square, providing you with total cover. The shield does not, however, provide cover against targeted spells; a spellcaster can cast a spell on you by targeting the shield you are holding. You cannot bash with a tower shield, nor can you use your shield hand for anything else.

When employing a tower shield in combat, you take a \minus2 penalty to accuracy with physical attacks because of the shield's encumbrance.
\parhead{Shield Spikes} When added to your shield, these spikes turn it into a piercing weapon that increases the damage dealt by a shield bash as if the shield were designed for a creature one size category larger than you. You can't put spikes on a buckler or a tower shield. Otherwise, attacking with a spiked shield is like making a shield bash attack (see above).
\parhead{Studded Leather} Only the studs on studded leather are made of metal. If the studs are made from a special material, only the studs are affected, not the whole armor. For example, ironwood studded leather weighs 17.5 pounds, since the weight of the studs is halved.

\subsection{Armor for Unusual Creatures}\label{Armor for Unusual Creatures}
Armor and shields for unusually big creatures, unusually little creatures, and nonhumanoid creatures have different costs and weights from those given on \trefnp{Armor and Shields}. Refer to the appropriate line on the table below and apply the multipliers to cost and weight for the armor type in question.
\begin{dtable}
    \lcaption{Armor for Unusual Creatures}
\begin{dtabularx}{\columnwidth}{>{\lcol}X c c c c}
  & \multicolumn{2}{c}{\tb{Humanoid}} & \multicolumn{2}{c}{\tb{Nonhumanoid}} \\
\hline
\tb{Size} & \tb{Cost} & \tb{Weight} & \tb{Cost} & \tb{Weight} \\
Tiny or smaller\fn{1} & \mult1/2 & \mult1/10 & \mult1 & \mult1/10 \\
Small & \mult1 & \mult1/2 & \mult2 & \mult1/2 \\
Medium & \mult1 & \mult1 & \mult2 & \mult1 \\
Large & \mult2 & \mult2 & \mult4 & \mult2 \\
Huge & \mult4 & \mult4 & \mult8 & \mult4 \\
Gargantuan & \mult8 & \mult8 & \mult16 & \mult8 \\
Colossal & \mult16 & \mult12 & \mult32 & \mult12 \\
\end{dtabularx}
1 Divide armor bonus by 2.
\end{dtable}

\subsection{Special Materials}
Armor and shields can be made from special materials, which can alter the properties of the item. The most common special material is ironwood, which is made from wood magically treated using the \spell{ironwood} ritual to be as strong as steel. Ironwood armor functions exactly like  armor of the same type, except that it is made of wood. This causes it to weigh half the normal weight, and allows druids to wear it without penalty. However, it costs twice as much as armor of the same type would normally cost, to a minimum of an additional 200 gp.

\section{Consumable Items}\label{Consumable Items}

Many substances exist that can aid adventurers.

\subsection{Poisons}\label{Poisons}

\begin{dtable*}
    \lcaption{Typical Poisons}\pdfbookmark[3]{Typical Poisons}{}
    \begin{dtabularx}{\textwidth}{l l l l l X l}
        \tb{Poison}         & \tb{Transmission} & \tb{Form} & \tb{Potency} & \tb{Primary Effect}            & \tb{Terminal Effect}                              & \tb{Type} \\
        Nitharit            & Contact           & Powder    & 3            & Sickened                       & Nauseated for 1 round                             & Plant           \\
        Sassone leaf        & Contact           & Powder    & 6            & 1d6 damage per round           & 1d6 damage per two potency                        & Plant           \\
        Dragon bile         & Contact           & Liquid    & 10           & Sickened, 1d8 damage per round & 1d8 damage per two potency, nauseated for 1 round & Venom           \\
        Black lotus extract & Contact           & Liquid    & 15           & 1d10 damage per round          & 1d10 damage per two potency                       & Plant           \\
        Arsenic             & Ingestion         & Powder    & 5            & 1d6 damage per round           & 1d6 damage per two potency                        & Plant           \\
        Insanity mist       & Ingestion         & Gas       & 10           & Disoriented                    & Confused for 1 round                              & Alchemical           \\
    \end{dtabularx}
\end{dtable*}

Poisons can deal damage, weaken creatures, or even kill them.
Some effects which are not literally poisonous, such as animal venom or fungal spores, are considered poisons.

\subsubsection{Poison Transmission}\label{Poison Transmission}\label{Transmission}

There are three ways that poisons can be contracted.

\parhead{Contact} A contact poison affects any creature that touches it with bare skin.
\parhead{Ingestion} An ingestion poison affects any creature that eats, drinks, or breathes it, depending on the type of poison.
Ingestion poisons have no effect when used to coat weapons.
\parhead{Injury} An injury poison affects any creature injured by something carrying the poison.
Almost all injury poisons take liquid form, and are typically used to coat weapons.
An attack that deals no damage cannot transmit injury poisons.

\subsubsection{Poison Forms}\label{Poison Forms}

There are four forms of poison.

\parhead{Gas} Gaseous poisons are difficult to store, but easy to affect foes with.
\parhead{Liquid} Liquid poisons are the most common type of poison.
Liquid poisons can be used to coat weapons, slipped into food, or simply thrown at foes.
A dose of a liquid poison is usually about one ounce of the poison.
\parhead{Pellet} Some rare poisons come in small, solid pellets or cubes.
Typically, these pellets contain a powerful liquid poison that becomes inert quickly after being exposed.
Pellet poisons cannot be used to coat weapons or thrown at foes, but can be slipped into food.
\parhead{Powder} Poison in powder form cannot be used to coat weapons, but can be slipped into food or thrown at foes.

\subsubsection{Poison Effects}\label{Poison Effects}

Poisons can have a wide variety of effects, as determined by the type of poison used.
However, most poison share certain common properties.

\parhead{Poison Attacks}\label{Potency}\label{Poison Potency}
All poisons have a potency.
Unless otherwise noted, a poison's accuracy is equal to its potency.
At the end of each round, the poison makes an attack roll against the Fortitude defense of the poisoned creature.
Success means the creature suffers the effect of the poison, and gets closer to the poison's terminal effect (see Terminal Effects, below).
For every 10 points by which the attack succeeds, it counts as an additional successful attack for the purpose of reaching the poison's terminal effect.
If the attack fails, the creature does not suffect the effect of the poison that round, and gets closer to resisting the poison (see Resisting Poisons, below).
For every 10 points by which the attack fails, it counts as an additional failed attack for the purpose of resisting the poison.

\parhead{Resisting Poisons}
If a poisoned creature resists a poison three times, the creature stops being poisoned by that poison.
Unless otherwise noted, this removes any lingering effects from the poison.

\parhead{Primary Effects}
Most poisons have primary effects.
If the poison successfully attacks a poisoned creature, the creature suffers the poison's primary effect as long as the creature remains poisoned.
Repeated primary effects, such as damage per round, occur at the end of each round.
This includes the round in which the creature is initially affected, but not the round it stops being poisoned.

\parhead{Terminal Effects}
Most poisons have a terminal effect based on the type of the poison.
If the poison successfully attacks a poisoned creature three times, the creature suffers the poison's terminal effect.
Unless otherwise noted, the terminal effect occurs in addition to the poison's normal effect that round.
Once a creature suffers a poison's terminal effect, it stops being poisoned.
This does not remove any lingering effects from the poison, but prevents the creature's condition from worsening.

\parhead{Multiple Doses}
A creature can be affected by multiple doses of the same poison.
This does not cause the same effect to occur multiple times, but each extra dose increases the potency of the poison by 1.

A poison is considered the same if it has the same name and comes from the same source.
For example, a creature bitten multiple times by the same giant spider suffers multiple doses of the same poison.
A creature bitten multiple times by different giant spiders considers each spider's poison separately.

\parhead{Poison Quality} Some poisons are unusually high or low quality.

\subsubsection{Creating Poisons}\label{Creating Poisons}

You can use the Craft (poison) skill to create poisons.
To create a poison, you must make a Craft (poison) check against a DR equal to 10 \add the poison's potency.
For every 2 points by which you beat this DR, the created poison's potency increases by 1.

Creating a poison requires special materials.
The type of materials required, and how those materials can be acquired, depend on the type of poison.

\begin{itemize}
    \itemhead{Plant}: Plant-based poisons can typically be harvested by making a Survival check to search in appropriate terrain.
        The DR of this check is usually equal to 10 \add the potency of the poison.
    \itemhead{Venom}: Venom requires an appropriate body part from a creature -- often, poison it naturally produces.
    \itemhead{Alchemical}: Alchemical poisons require alchemical materials.
        These cannot normally be found in nature.
        In unusual circumstances, these components can be synthesized from natural chemicals or magical materials with a Craft (alchemy) check equal to 10 \add the potency of the poison.
\end{itemize}
