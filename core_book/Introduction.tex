\chapter{Introduction}

Rise is a tabletop role-playing game.
This chapter explains what that means, and how Rise is different from other existing games.

\section{What Is A Tabletop Role-Playing Game?}
    In tabletop role-playing games like Rise, you play a specific character of your own design.
    Your character can try to do anything you can imagine in a world that the game master, or GM, creates.
    Of course, you won't always succeed.
    Your character's abilities are defined in the pages ahead; when you're done creating a character, it will have a personality of its own, along with strengths, weaknesses, and special abilities.
    Usually, your character will go on adventures with other characters, each of which is played by other players.
    Together, you will create and experience a story with the Game Master, or GM, who defines the universe that the player characters inhabit.

    \subsection{Describing Actions}
        Most of the time, when you're playing a game of Rise, you simply describe what you want your character to do.
        For example, you can say that your character steps out of their room in the inn and walks over to knock on a friend's door.
        Although Rise has rules that could govern some aspects of that scenario, such as an Awareness check to see if your friend notices you knocking, you wouldn't usually reference those rules explicitly.
        Even in the unlikely scenario that your friend doesn't notice you knock the first time, you can just knock again, so there's no point in worrying about the details.
        If something seems reasonable, it probably is, and you don't need to worry about the fiddly bits.

        Sometimes, when you describe what your character tries to do, the action has a narratively relevant chance of failure.
        Instead of knocking on the door to say hi, you might only have time to bang on it once to warn your sleeping friend about an attack from assassins.
        In that case, there's some chance that your friend is sleeping too deeply to notice the noise the first time you knock.
        You could try knocking again, just like in the first scenario, but in this scenario that failure would cost you valuable time to survive the attack.
        In that scenario, you would roll a die to determine whether you succeed in your action - or in this case, whether your friend would succeed in their attempt to notice you.

        Your character doesn't have to take a specific ``knock on door'' action that the rules define.
        However, Rise has broadly written mechanics to describe many aspects of the universe, including how likely your friend is to notice things while asleep.
        Intuitively, you should be more likely to succeed if your friend is very aware of their surroundings, or has an unusual sleeping practice, such as if they are an elf.
        The rules of Rise attempt to make the results of gameplay match your intuition.

    \subsection{Using Specific Abilities}
        Instead of describing broadly what you want to have happen, you might choose one of a list of clearly defined abilities that your character can use.
        Every character has specific abilities unique to them, such as a wizard's spells known.
        There are also a number of simple abilities that anyone can use, such as the \ability{dirty trick} or \ability{trip} abilities.
        These universal abilities attempt to adequately describe a wide variety of reasonable improvised actions that you might try to use in combat.

        Explicitly defined abilities have rules for determining what happens when you use them.
        Some abilities, such as attacks in combat, require rolling dice to determine how effective they are.
        Of course, you can use your character's abilities at any time, not just in combat.
        Abilities such as the \spell{create water} or \spell{distant hand} spells can be used to solve other kinds of problems entirely.

    \subsection{Rolling Dice}
        Eventually, you'll have to determine whether something succeeds or fails.
        This can happen as part of using a specific ability that tells you exactly what to roll, or because you tried to narrate your character taking an action that has a dramatically relevant chance of failure.
        In either case, you'll roll a single ten-sided die, or a d10.
        You'll add some modifier that represents how skilled your character is at the particular thing that they are trying to do.
        At the GM's discretion, they may also give the roll an extra bonus or penalty based on the circumstances that your character is in.
        If your die roll is high enough, your character succeeds at whatever they were trying to do.
        Otherwise, your character fails, which may sometimes have additional consequences.

        In Rise, it's entirely possible for characters to be so skilled that they succeed at what they are trying to do even if you roll a 1.
        Likewise, there are tasks that are so obviously impossible for your character that they cannot possibly succeed.
        In those cases, there's no reason to roll!
        Of course, the GM is the final arbiter of whether rolling is necessary.
        They may have information that the players do not.

    \subsection{Why Use So Many Rules?}

        Tabletop role-playing games attempt to create rules to define how their universe works.
        Some games are intentionally vague or minimalist about their rules, which can be fun!
        Simple games are easy to start playing, and they try to avoid getting in the way of good role-playing.
        However, Rise takes a different approach.
        It spends a lot of effort - and words - attempting to define an internally consistent universe, and creating a large number of specific abilities that can be used in that universe.
        There are a few important advantages to taking this approach: establishing expectations, supporting multiple play styles, and assisting the GM.

        \subsubsection{Establishing Expectations}
            Different people can have very different ideas about what is realistic - or narratively appropriate - in a made-up fantasy universe.
            To some people, kicking in the tavern door and starting a brawl is just some good clean fun, and you'll take a few good punches and then laugh about it later that evening over drinks.
            But to other people, that might sound like a good way to find yourself imprisoned for the foreseeable future with all of your possessions confiscated by the town guard.
            Another interpretation of that scenario might see the brawler seriously injured with a broken bottle in the eye, leaving them partially blinded for weeks - or indefinitely.

            All of those ideas are valid, and they each match the narrative of a particular type of story.
            However, it's important that everyone setting at a table and playing a game agrees about what to expect.
            Players can get confused or frustrated when their actions have consequences that feel arbitrary or unfair.
            Generally, games are more fun if everyone in the game shares a common set of expectations and conventions.
            Otherwise, games can devolve into disagreements about what is or isn't reasonable.

            One way to establish these expectations is to use a rules system like Rise that defines some expectations explictly.
            If the scenario above happened in Rise, the last outcome of an incapacitating bottle to the eye shouldn't normally be possible, since the rules explicitly define how injury works.
            Knowing what is and isn't possible can help give players and GMs a useful set of guardrails for what they try to do in the universe.
            It's relatively easy to get everyone to agree about simple things that regular human people have experience with, like how difficult it is to climb a tree.
            However, Rise is full superhuman people and monsters, and eventually you'll need to figure out how far a barbarian as strong as Hercules can throw a bear.
            Having a single authoritative resource to consult can cut off long disagreements about details that are difficult or impossible to determine objectively.

            Of course, different games played with a flexible rules system like Rise can have very different tones and themes.
            Either of the first two scenarios in the tavern are still plausible in different games, and a GM can use house rules to make vital wounds have more long-term consequences if they want.
            Using a rules system like Rise can help, but it is not the full answer by itself.
            The GM and players always share responsibility for establishing expectations about what genre a game will be, and conforming to those expectations to the extent that it makes the game more fun.

        \subsubsection{Supporting Multiple Play Styles}
            Some people deeply enjoy the process of role-playing itself.
            They enjoy the process of getting into a character and speaking in their voice, exploring their needs and desires, and building a narrative for them over time.
            These people often do not need the confines of a robust rules system, and can play equally well in games with minimal rules or none at all.

            Other people do not enjoy role-playing as an end in itself, or even at all.
            However, they may still enjoy the \textit{game} aspect of a role-playing game.
            Instead of playing a character for their personality and backstory, they may play a character for their unique mechanics and tactical advantages.

            Still other people may be interested in role-playing as a concept, but find it daunting.
            The blank page in front of you when you start painting a picture or writing an essay can be daunting, and that first step is often the hardest to take.
            Giving people a clearly defined set of abilities and specific tools for interacting with the world can enhance creativity by providing a safe space for interaction and experimentation.
            Even if you don't enjoy or feel confident in speaking in your character's voice, you can still engage with the narrative aspects of the adventure by cast a relevant spell or making a relevant skill check.
            People in this middle ground can sometimes enjoy deeper role-playing games while being be lost in role-playing games with minimal or nonexistent rules.

            One of the joys - and challenges - of Rise is drawing together people with very different desires and play styles to share a single experience.
            Rules-free role-playing games and tactical wargames can both have a narrower appeal than rules-heavy role-playing games like Rise, which try to provide something for everyone.
            You can run games with deep role-players alongside tactical gamers, and it can be a lot of fun.
            It does place a greater burden on the GM to provide the right ratio of content to keep everyone happy, and it does require the players to be patient when their preferred playstyle is put in the background to support the needs of other players.
            A well-blended game can also draw people out of their comfort zones slowly and safely over time as they observe and start to enjoy the playstyles of the other players in the game.

        \subsubsection{Assisting the GM}
            The Game Master carries an extra weight of responsibility to shape the flow of the game.
            Creating narratively consistent universes, appropriate challenges, and engaging storylines out of thin air is deeply challenging.
            If this job is too difficult, no one will want to do it, and then no one will play the game!
            Making the GM's job easier is a critical component of any role-playing game.

            There are several ways that Rise can make the GM's job easier.
            It provides information about the mechanics and tropes of the universe that the game takes place in, which helps establish expectations and resolve disputes that might come up during the game.
            It will provide a clear narrative foundation for the world and the characters in that world, which minimizes the up-front work required to run a game, once that section of the book is more complete.
            It will provide a wealth of pre-packaged challenges appropriate for players of any power level or play style, and advice for how to use those challenges appropriately, once that section of the book is more complete.
            The GM-focused sections are currently the most unfinished part of Rise, and this will be a more useful guide before Rise is done.

\section{What Makes Rise Different?}
    If you haven't played other tabletop role-playing games, feel free to skip this section.
    If you have, you may wonder what makes Rise unique in a crowded sea of games.
    Rise has five fundamental principles that differentiate it from other TTRPGs: minimal resource management, simultaneous combat, optional complexity, unbounded scaling, and a bounded action economy.

    \subsection{Minimal Resource Management}
        Many games make use of resources like mana, spell slots, or timed cooldowns to limit how often characters can use their abilities.
        These systems have fundamental problems that undercut the fun and flow of a TTRPG, and Rise essentially does not use resources to limit character ability usage.
        In Rise, characters can cast spells or use special attacks any number of times in a row without consuming resources.

        Some systems have resources that are designed to ebb and flow in the course of a typical combat.
        You might expend mana to use a powerful spell, and then regain mana over time by using weaker spells or fulfilling certain conditions.
        Alternately, you might use a spell and then wait some number of in-game turns before you can use that same spell again.
        This can be fiddly to track and hard to recover from if you forget what happened to your resource pool, which is why this approach is more common in video games than in TTRPGs.
        More importantly, this system has no clear way to handle ability usage outside of combat.
        It effectively gives unlimited ability usage when time is no obstacle, but only in an awkward and convoluted way.
        This category of system is unsuitable for Rise because it is too fiddly in combat and doesn't make sense out of combat.

        Some systems have finite-use resources that are tied to the expenditure of in-game time, such as taking long rests, or session breaks.
        You might spend a spell slot to use a powerful spell, and then be unable to cast that spell again until your character rests for some period of time.
        This can be manageable from a complexity perspective if the number of unique resources is small.
        However, it can get dangerously convoluted if characters have a large number of separate or partially interchangeable resource pools, such as using separate pools for individual spell levels.

        The real problem is that this limitation requires you to make your decisions based on not just the current situation, but also on your prediction of all future situations you will encounter before you have the opportunity to rest.
        This contributes significantly to the tactical complexity of deciding each individual action in combat, which slows down the pace of the game.
        It is also punishing to newer players who have less experience with the metagaming required to deduce how many resources an individual fight is worth.
        This strategic complexity is compounded if hit points are treated as an additional resource, since you now have to trade off the potential impact of one limited resource against another limited resource.

        Optimization of resource usage can be unintuitive and out of character, but failure to correctly manage your resources can leave you with no useful abilities remaining.
        This concern can be exacerbated if some characters are extremely resource-intensive while others have no meaningful resources to track.
        No one likes being forced to hide from a difficult fight or take only insignificant actions while your more resource-savvy or resource-independent allies continue using dramatic and powerful abilities.
        It can also add stress to the party dynamics when one character frequently asks for long rests after fights because they expended resources and no one else needs to rest.
        This category of system is unsuitable for Rise because it creates complexity in ways that detract from the fun and narrative of a game instead of adding to it.

        Rise does not use resources to limit normal actions in combat.
        The vast majority of spells, special martial attacks, and other abilities that affect enemies or your environment can be used any number of times.
        There are a small number of abilities with one-round cooldowns, and a universal ability that can only be used once per short rest.
        However, there is no time tracking in the system longer than ``next round''.
        Small cooldowns are a fine-grained balancing tool that allow characters to have powerful abilities which would have detrimental effects for the game if they could be used every turn.

        Rise does use a single universal resource, called ``fatigue'', that recovers based on long rests.
        This allows some opportunity to invest extra effort into difficult fights, and to eventually tire over the course of a long day, but it is simple to use and track.
        Damage taken during a fight is easily recovered after a ten minute rest, with the exception of rarer vital wounds.
        This gives a GM time to force a party to undergo multiple sequential fights if it is appropriate.
        If multiple fights in a row are possible, that is often obvious from the narrative context, so metagaming about future fights is almost always pointless.

        Overall, Rise uses resource limitations very sparingly.
        This allows it to gain some of their benefits while avoiding the detrimental effects that come from making resource limitations a fundamental part of the system.

    \subsection{Simultaneous Combat}
        In most TTRPGs, combat takes place in a series of turns.
        When your turn comes up, you take all of your actions, and then you wait through everyone else's turn until your turn comes again.
        This system has one foundational disadvantage: it is very, very slow.
        Rise uses a simultaneous combat system that dramatically increases the pace of combat.

        Imagine a typical 4-5 player game with 1-2 enemy groups using a traditional turn-based initiative system.
        In this scenario, you have to wait through about 5 turns before it comes back to your turn.
        This number can increase significantly in large-scale fights.
        Each of those 5 or so turns can meaningfully change the battlefield situation on its own by moving, weakening, or defeating various enemies and allies.
        The state of the battlefield at the end of your turn is often drastically different than the state of that battlefield at the start of your next turn, so there's no point in immediately planning for your next turn.
        Player coordination can be challenging, since they must coordinate in the specific order assigned by the initiative system, and enemy turns can intervene to ruin coordinated plans.

        In theory, every player would accurately track the unfolding battlefield state through each of the intervening turns, so they already know what they will do when their turn comes up.
        In practice, many players find that difficult or impossible, so the first question when someone's turn comes up is: what is the current situation?
        Not everyone asks this explicitly, but it must always be analyzed anew.
        Once this information has been processed, players can choose their actions from among a typically large pool of potential actions.
        Everyone else must wait and do nothing while the active player decides their actions - and there are often multiple actions involved, since this includes both movement and any number of sequential attacks.
        Once the active player has decided their actions, those actions must be fully rolled and resolved before combat can proceed, so there is little point in trying to make plans until the results are revealed.
        All of this combines to make even short combats take an hour or more, and six-person adventuring groups can feel dangerously bloated.

        Rise works differently.
        Combat in Rise is broken up into three phases: the movement phase, the action phase, and the delayed action phase.
        During the movement phase, all creatures move simultaneously, and no attacks are possible.
        Characters can declare certain simple reactive movements like ``stay adjacent to this enemy'' to ensure that they end up in a reasonable position regardless of enemy actions.
        If the movements of characters conflict in impossible ways, initiative checks can temporarily force a linear order of resolution.
        Each player declares their own actions in an arbitrary order as soon as they decide them, so people are not forced to wait and do nothing while slower players contemplate their choices.
        Player coordination is easy, since all actions are happening together.

        During the action phase, all creatures take combat actions simultaneously, unless they delay.
        The delayed action phase works in the same way, with all creatures who delayed their actions acting together.
        The effects of all actions are applied simultaneously, so attacks during this phase cannot ``interrupt'' each other.
        Attackers are always responsible for rolling instead of using ``saving throws'' or similar mechanics that force defenders to roll.
        All of this means that players can fully resolve their own actions without waiting for slower players to decide what they will do.

        The start of each phase still requires a general assessment from all acting players about the current state of the battlefield, which takes just as much time as the assessment in a classic initiative system.
        However, the time required for this tactical analysis only increases marginally as the number of players and enemies in the game increases.
        This makes Rise scale much better to large player counts or large enemy hordes.
        Combat in Rise flows by quickly, making it much easier to balance time between combat and non-combat encounters within the same game session - or to run through multiple separate, individually challenging combats without sacrificing the pace and energy of the game.

    \subsection{Optional Complexity}
        Many games operate at a consistent level of complexity.
        Many rules-light games are always simple, and many rules-dense games are always complex.
        This is a perfectly reasonable design philosophy.
        Among other benefits, it makes it easy to know what to expect from the game, which helps give the game a well-defined niche.

        Rise is designed to allow players to choose their own level of complexity.
        This broadens its potential audience by allowing people with very different play styles or tolerances for complexity to enjoy the same game together.
        This goal is manifested in several key ways in Rise's design.
        First, simplicity is the default.
        Second, complexity is centralized around the character creation and leveling process, rather than gameplay.
        Third, complexity is only loosely connected to narrative and storytelling roles.
        Fourth, gameplay complexity is unconnected to character power.

        In order for complexity to be truly optional, simplicity must be the default.
        The fundamental infrastructure of the game must be easy to pick up and understand.
        Complexity must be isolated to specific areas where people can choose to interact with it if they enjoy that process.

        As a necessary consequence of making simplicity the default, the gameplay in a session for a typical character must not be overly complex.
        You should be able to pick up a typical character and figure out how to play as them without extensive study.
        If gameplay is simple by default, and complexity must live somewhere, then the complexity must be located in the process of character creation.
        Building a perfectly optimized Rise character is designed to be an difficult and rewarding challenge for people who enjoy it.
        The core systems - attributes, classes, attunements, and feats - interact with each other in nuanced ways.
        However, building a Rise character driven solely by narrative concerns is only moderately difficult, since you can make each decision independently.

        Even for simple characters, the process of character creation is still one of the most complicated aspects of Rise.
        That is why Rise provides (or will provide, once that section is done) an extensive selection of premade characters for a wide variety of narrative archetypes.
        Each premade character includes advice for how to play that character and level them up.
        The premade characters make the system more accessible to people who don't want to interact with the complexity of character creation, or who want a starting point that they can customize to match their own character's narrative more closely.

        Not all complexity is contained within the character creation process.
        Characters are allowed to have significant gameplay complexity.
        What's important is that players intentionally opt in to playing a more complex character, and don't feel forced to.

        If players must never feel compelled to take on more complexity than they enjoy, then complexity should be mostly separate from narrative concerns.
        For example, it would be a bad idea to define a system where martial characters are simple and spellcasters are complicated.
        Both of those are rich and evocative narrative constructs.
        Many people who don't enjoy complexity will want to play spellcasters, and many people who enjoy complexity will want to play martial characters.
        Gameplay complexity must be more finely tuned and localized than those sweeping strokes.

        In Rise, gameplay complexity is generally generated by acquiring a large number of increasingly situational abilities.
        Every class has some archetypes that grant additional abilities known and some archetypes that grant additional passive abilities.
        If you like having a lot of unique abilities, you can be a human with a high Intelligence to maximize your insight points, and focus on learning spells and maneuvers that attack your enemies or have situational effects.
        If you like minimizing complexity, you can instead choose archetypes or learn spells that simply grant you passive benefits, and focus on one or two standard attacks that you specialize in.
        Some feats give you new abilities and new circumstances to pay attention to that make you more effective, while others simply increase your passive statistics and defenses.

        Of course, all of this customization of complexity would be mostly pointless if complexity was strongly correlated with character power.
        If exceptionally complicated or hyper-specialized characters were obviously and consistently more effective than other characters, it would push everyone to use those characters.
        Rise makes the tradeoffs between gaining raw power and gaining additional options balanced enough that neither is always superior.
        There will always be specific contexts where one character's mechanics are superior to another's.
        For example, a specialized defensive melee character may excel in a campaign that consists of entirely of series of duels in confined spaces, while being exceptionally weak in a campaign that consists entirely of large-scale battles against cavalry archers on open fields.
        However, each character's strengths and weaknesses are well defined, and their raw power is close to each other, so a GM should not have great difficulty designing a campaign where everyone feels relevant and useful.

    \subsection{Unbounded Scaling}
        Some systems uses bounded bonuses for accuracy or other game statistics.
        This means that every character of the same power level - or in some systems, of any power level - has a similar chance of success with any given skill check or attack roll.
        This can frequently cause narratively inappropriate and even comical events, and Rise explicitly rejects this philosophy.

        Imagine a typical party of four players, with one character being exceptionally skilled at a particular task.
        Perhaps the rogue is exceptionally skilled at lying, or a barbarian is exceptionally skilled at climbing.
        If ``exceptionally skilled'' only means that they have a \plus5 bonus on a d20 compared to \plus0 from the rest of the party, the exceptionally skilled character will only get the best result in the party half the time.
        The other half of the time, some other character with no relevant skills will meet or exceed the skilled character's result - sometimes by a dramatic margin.
        When failure compared to rank amateurs happens this often, it becomes hard to take seriously the idea that any character can be exceptionally skilled at anything.

        Rise characters can have dramatic statistical differences between each other, even at low levels.
        It uses a d10 as the fundamental die, which makes every bonus more significant.
        In addition, a 1st-level character can easily reach a \plus6 bonus with a skill check that is particularly relevant to their character.
        This means that a skilled character can beat a party of rank amateurs 80\% of the time, and at higher levels their success becomes completely guaranteed.
        Likewise, the Mental defense of a powerful sorcerer and a cowardly rogue can allow mind-affecting attacks to almost always hit a rogue while almost never hitting the sorcerer.
        These statistical differences do not always grow with level, but they remain significant at every level.

        One advantage of systems with bounded scaling is that it is easier to guarantee that every character is relevant in any situation.
        Even if your character has no useful abilities of any kind, you might sometimes succeed on important actions through sheer luck.
        However, this design philosophy often breaks the symmetry between magical and non-magical characters.
        Magical characters can often use extremely specific and powerful abilities that are impossible for nonmagical characters to duplicate.
        If magical characters also have similar odds of success with all generic mechanics of the game, they will almost certainly have far more influence over the narrative of the game than any nonmagical character can hope to match.

        The philosophy of Rise is that it's okay for some characters to be irrelevant in specific contexts.
        It's good to give people time in the spotlight where their character's abilities help solve the specific problem that the group is facing when no other character could.
        Rise encourages that, and makes it impossible for one character to be relevant in \textit{all} contexts.
        Each character has their own strengths and weaknesses, and if you try to be good at everything, you'll fall behind people who specialize in a particular area.
        This will naturally rotate the spotlight between different characters, allowing each player to feel relevant and important in turn.

        This dramatic scaling is also used to govern the power of characters over time, in addition to the power of characters relative to each other.
        Rise attempts to model a massive power range for player characters.
        They are expected to start their journeys at level 1 as little more than commoners, and by level 21 they are effectively demigods who can alter the fate of entire worlds.
        This is a critical part of the narrative fabric of Rise, and it is reflected in the statistics and abilities of characters.
        If a level 1 kobold posed even a tiny threat to a level 21 character, the mechanics of the game would sabotage the purported narrative of power and growth.
        In Rise, overall character power doubles approximately every two to three levels.
        The system takes some care to avoid bloating numbers to unwieldy levels on this journey, and the use of the d10 as the standard die helps immensely.

    \subsection{Bounded Action Economy}
        It is dangerous to to give characters too many actions each turn.
        Each additional action a character can take increases how difficult it is for a player to decide what to do on their turn.
        In addition, each additional action increases the complexity of the change between the start of the turn and the end of the turn.
        This is especially risky with Rise's simultaneous initiative system, which combines the actions taken by all characters into a single resolution process.

        Rise places significant limitations on how many relevant actions each character can take on their turn.
        Generally, characters can only move during the movement phase and then take one significant action each turn.
        Some characters can use a minor action to accomplish something useful.
        However, that essentially marks the end of action economy scaling, even up to the maximum level.

        Detrimental effects that could deny actions are also heavily limited.
        Total action denial effects are only usable by high level characters, and even then they only work against weak enemies or enemies that have already been significantly damaged.
        Taking actions is fun, and sitting quietly while everyone else does things can be very frustrating.
        Similarly, completely removing an enemy's ability to act can easily remove the tension from a fight before it's actually over.

\section{The Narrative Universe of Rise}
    Rise does not attempt to define a single geography with specific countries and locations that is shared between all games.
    It is common for GMs to define their own setting when running a game, and that freedom is important.
    However, the universe of Rise does differ in a number of important ways from the real world.
    The fundamental assumptions that Rise makes about the world are listed below.
    These fundamental elements are ambiguous about some details, and GMs are encouraged to fill in those details as they see fit.
    Of course, a GM has absolute power, and can create a world that changes any number of these assumptions.
    However, doing so can significantly change the tone of the game and create logical inconsistencies, so it should be done carefully.

    \subsection{Magic is Common}
        The world of Rise is a magical place.
        Many people care capable of using magic to perform feats that would be impossible in the real world.
        Not everyone is capable of magic, of course.
        As an overly broad generalization, it's reasonable to assume that about a quarter of the civilized people in the world have some ability to use magic.
        In some societies, such as a feudal human-dominated society with a large number of commoners and serfs, the percentage of people with magic can be much lower.
        However, this is balanced by the existence of other societies that tend to be much more magical, such as societies ruled by gnomes and elves.
        Even in low-magic societies, everyone knows that magic exists, and almost everyone has observed or been personally affected by magic at some point in their lives.

        People can have magical abilities for a wide variety of reasons.
        There are three main categories to explain why people can access magic: intrinsic magic, learned magic, and gifted magic.
        Each class with magical abilities belongs to one of these groups.
        Characters with magical feats are free to choose any of those three explanations for their feats.
        The explanation does not have to be the same as for any other magical abilities they possess.
        For example, a cleric may be gifted their magical cleric abilities because they worship a particular deity, but they may also be naturally telepathic.

        Some people are simply intrinsically magical.
        They may require training and experience to improve their natural magical talents, but they had magical capabilities before doing any training.
        This intrinsic magic can come from magical ancestry, unusual birth circumstances, magical experimentation, exposure to powerful magic, simple random chance, or any number of other sources.
        This is the standard explanation for sorcerers.
        In addition, this is the most common explanation for the magical abilities of monsters.

        Some people gain access to magic through personal training or research.
        These people find ways to tap into some pre-existing magical property of the universe and manipulate it at their command.
        This is the standard explanation for monks, rangers with the Beastmaster archetype, rogues with the Bardic Music archetype, and wizards.

        Some people are gifted magic by their association with powerful magical entities or forces.
        They offer worship, allegiance, or their souls, and are granted magical power in exchange.
        This is the standard explanation for clerics, druids, paladins, and warlocks.

    \subsection{Personal Power Comes From Great Deeds}
        The average person in the world of Rise is not particularly more or less competent than the average person in the real world.
        Training can help people improve their skills, but as in the real world, anyone who tries to improve themselves through training and practice eventually reaches an upper limit to their potential.
        However, unlike in the real world, people in Rise can reach beyond their ordinary limitations.
        By defeating powerful foes and performing great deeds that influence the world around them, people can gain levels, which allows them to reach new heights of power.
        At high levels, people can perform clearly superhuman feats that would be impossible for ordinary humans, even without the influence of magic.

        People in Rise wouldn't usually talk about "levels" as a discrete concept ranging from 1 to 21.
        They would perceive the world as a spectrum, and the specific divisions would be more subtle.
        However, they would be aware that some people are fundamentally stronger and more skilled than others.
        Individual scholars or scholastic groups may create their own concepts in-universe to categorize and explain the phenomenon of levels, since the growth of personal power over time is observable and studiable. 
        However, those in-universe concepts would never exactly replicate the metagame concept of a level.

        It is common for people in positions of political power to also wield unusually large amounts of personal power.
        High level individuals can be savvier, wiser, and more persuasive than any ordinary human.
        They are more likely than low-level individuals to be able to gain political power through whatever means they see fit, and more likely to maintain their hold on that power.
        In addition, political power can grant further opportunities for performing great deeds, which helps those in power to gain levels and stay ahead of any competition.

        The fastest path to acquiring personal power does not come from pursuing political power.
        It comes from adventuring.
        Adventurers can defeat powerful monsters, help towns in need, and otherwise have a significant personal influence on the world.
        In the process of these adventures, they can amass personal power much more rapidly than ordinary people.
        Of course, adventuring also has an unusually high risk of death.
        Even worse, people who die while adventuring often leave their corpse in the middle of nowhere - in a monster's stomach - which prevents them from being resurrected without incredibly rare magic.
        Adventurers must constantly seek out new challenges to test their limits, or else they will stagnate and stop acquiring personal power, so it is never a sustainable long-term activity.
        There are many people in the world who were adventurers at some point in their past, and everyone is familiar with the concept, but active adventurers are still unusual.

    \subsection{Deities and Afterlifes}\label{Deities and Afterlifes}
        When a humanoid creature dies in Rise, they know beyond a shadow of a doubt that they will go to an afterlife.
        Most likely, they know exactly which afterlife they will go to, either as a result of their alignment or their worship of a particular deity.
        In that afterlife, they will live again for as long as they want, though they cannot leave without being magically resurrected.
        People are confident that this is true because deities have told them so, and deities are provably real.
        Also, rare and powerful magic can be used to communicate with people in the afterlife.

        It is an undisputed fact that Rise is filled with a wide variety of deities of varying power and influence.
        They divinely empower their clerics to act on their behalf.
        Many people know, though some chain of connections, someone who chose to become a cleric and was quickly rewarded with divine magic far beyond anything they could previously do on their own.
        Everyone has heard legends of deities intervening more directly in the world even without a cleric, though these stories are rare and few have experienced them firsthand.

        There are nine distinct afterlife planes, with one plane for each alignment combination.
        Each of those planes is divided into layers.
        Some of those layers are reserved for deities, with major deities claiming layers that are entirely their own and multiple minor deities sharing territory within a single layer.
        The remaining layers have no specific associated deity.
        People can travel between the layers, though the specific mechanisms for traversing layers are different for each afterlife plane.
        Most people do not know this level of detail about afterlife planes, and a commoner would simply be confident that they will go where they belong.

        It is well known that the afterlife planes for evildoers are much harsher than the other afterlife planes.
        The three evil afterlife planes are collectively referred to the Abyss.
        Demons stalk those planes, tormenting evildoers for their own sadistic reasons.
        One of the reasons that some people worship evil deities is to gain a promise of safety, since evil deities protect their worshippers from demonic torment in the afterlife.
        It is also said that demons only torment the weak-willed, and that those who escape demonic torments are free to live in hedonistic luxury.
        There is truth in this, though there are far more people who are confident that they would rule proudly in the Abyss than people who succeed.

    \subsection{Secrets of the Universe: Power Ultimately Derives From Souls}
        At a surface level, Rise seems to have a deep and fundamental divide between magical and mundane power.
        The physical abilities of a mighty barbarian and the divine magic wielded by a cleric are generally believed to come from completely different sources.
        In truth, these are all just reflections of the ultimate source of power for everything in Rise: the soul.

        When living creatures are born, they enter existence with a new soul.
        This is the fundamental miracle of life, and no one knows where these souls come from.
        Not all souls are equal in power.
        Even the combined power of the souls inhabiting a vast colony of ants is dwarfed by the soul of a dog or cat, and that too pales in comparison to the soul of a humanoid creature like a human or elf.
        Humanoid creatures have unusually potent souls, though some rare monsters have souls of similar intrinsic strength.

        \subsubsection{Transferring Souls}
            The intrinsic power of souls can be transferred.
            The simplest method of transfer is through death.
            When a predator kills its prey, the prey's soul is shattered and vulnerable in the moments after death.
            If the killer's soul is strong enough, it can ingest a fraction of the dead creature's and make its energy a part of its own soul.
            Weak-souled creatures are unable to feed on soul energy in this way.
            No matter how many rabbits a typical wolf kills, it will never gain a level.
            It is simply a wolf, and lacks the capacity to be more than that.

            Strong-souled monsters can gain a great deal of power by feeding on the souls of dead creatures.
            By repeatedly killing creatures with souls and feeding on those soul shards, they grow their own power.
            Likewise, an adventurer that kills a monster claims a piece of that monster's soul - including the combined power of all soul shards the monster absorbed in its life.
            With appropriate magical rituals, it is possible to allow deities or distant creatures to feed on the soul of a dying creature.
            Demons and minor deities sometimes use this principle to feed on souls offered to them in ritual sacrifices by their cultists.

            Transferring a soul's power through death is deeply inefficient.
            Under normal circumstances, only a fraction of a soul's power can be absorbed in this way.
            Some of the soul's power splashes into the surrounding world at the location of a creature's death, where it creates or fuels natural magical phenomena in the area.
            Creatures with strong souls, like humanoid creatures, retain their sense of self and are reborn in an appropriate afterlife with the vast majority of their soul intact (see \pcref{Deities and Afterlifes}).

            A soul's power can be transferred without the inefficiency of death.
            Commonly, it is simply freely given through love and emotional connection.
            Creatures who love each other naturally share small portions of their souls with each other.
            Over time, deeply connected creatures, such as old married couples, can mix their souls so fully that they become virtually indistinguishable.

            Voluntary soul sharing does not have to be perfectly symmmetric, of course.
            Worship is another method of transferring soul shards, and many deities fundamentally derive power from the combined soul shards willingly given by their legions of worshippers.
            In exchange, deities can use their power to protect their worshippers, either through divinely empowered clerics or more rarely through direct intervention.
            More mundanely, adventurers who save a town from a dire threat may earn soul shards freely granted from the gratitude of its inhabitants.

        \subsubsection{Souls and Intrinsic Power}
            As creatures gain soul shards, they may increase their personal power, which is represented in Rise as increasing their level.
            This does not mean that a creature's level or overall combat power is directly correlated to the strength of its soul.
            A well-trained soldier will easily defeat a commoner in battle, but this does not mean that the soldier's soul is stronger.
            Bears are physically much stronger than humans, and a typical bear is higher level than a commoner, but they have much weaker souls.

            Essentially, a creature's intrinsic strength and magical abilities determine the baseline power for a standard adult of that species.
            For monsters, this baseline power can be far beyond an ordinary human.
            Training and experience alone can increase that power slightly, but up to a clear limit, which is generally up to three levels beyond the baseline.
            To develop beyond that point, a creature must draw power from soul shards.

            The strength of a creature's soul determines how much power it can draw from soul shards.
            Creatures with a weak soul cannot master the raw energy contained within soul shards they are exposed to, and cannot gain levels in this way by any means.
            A strong soul allows a creature to gain levels from soul shards, and the strength of the soul determines the upper limit.
            For example, a dire wolf has an unusually strong soul for an animal, but it still eventually reaches a maximum level that it cannot surpass.
            Typically, only about 10\% of the humanoid population has a strong enough soul to exceed 10th level, though of course few even reach that point.
            All player characters are assumed to have have exceptionally strong souls even relative to normal humanoid creatures, and are able to reach 21st level.
            Legendary monsters of epic proportions may have still stronger souls, and be able to surpass that limit.

        \subsubsection{Mysteries of the Soul}
            The mysteries of differing soul strength have no clear and consistent explanation.
            In broad terms, the strength of a creature's soul usually correlates to its emotional and intellectual potential, as well as its force of will.
            Humanoid creatures and dragons are unusually mentally capable - not just in raw intelligence, but also in empathy, determination, and capacity for belief - and correspondingly have unusually strong souls.
            There are individual exceptions that suggest that this is not the entire dimension of what causes strong and weak souls.
            It is not uncommon for animals to have unusually strong souls for no known reason, causing them to develop over time into their ``dire'' variants.
            Dire animals, who have gained levels by feeding on soul shards, do not seem obviously more emotionally or intellectually capable than ordinary animals.
            Perhaps there is simply an element of randomness in the creation of each new soul.

            The fundamental mysteries of souls and their sharing is not widely known in the universe of Rise.
            Individual elements of this truth are widely known, such as the observation that people can become stronger by slaying monsters, but the reverse is not nearly as true.
            Strange phenomena can occur where death occurred, and old battlegrounds are often haunted by naturally occuring undead.
            Rare and learned scholars may understand that souls are important, since demons seem to want them, and may have their own theories that approach fragments of the broader truth.

            Some powerful and unusual entities, such as deities and demons, know particular elements of how soul energy can be transferred.
            Demons are generally aware that they can feed on souls in evil afterlife planes as they lose their cohesion over time.
            They attempt to torment weaker souls to accelerate this breakdown, and avoid souls that are too strong to break.
            However, they are unaware of the subtler aspects of soul sharing, such as willing soul transfer between loved ones.
            Powerful deities know more about souls than any other entities as a result of being worshipped and maintaining the existence of their personal afterlife planes.
            In exceptionally rare occasions they may see fit to share that knowledge if it serves their purposes.

            The secret of how souls power the universe of Rise is presented here because it explains much of the strangeness of the world, not because it is widely known, or because characters would usually understand or discuss it.
            Souls are not easy to investigate or analyze, and there is no good way for an ordinary inhabitant of the world to deduce the details of soul energy and soul shard transfer through observation.

    \subsection{Secrets of the Univere: Soul-Fuelled Phenomena}

        The peculiar nature of soul energy causes a wide variety of strange and unique effect in the Rise universe.

        \subsubsection{Deities}
            Deities are among the most obvious phenomena that are fundamentally created by the energy of souls.
            When hordes of living creatures pay homage to the same entity, that entity can feed on that outpouring of worship and become incredibly powerful if it has a strong enough soul.
            The background of Rise is full of minor deities and demigods who either lack a sufficient base of worshippers to become a true deity or who lack a strong enough soul to effectively use the worship they receive.
            Not every powerful entity powered by souls is a deity.
            Deities are sentient creatures that fundamentally owe their power to voluntary worship.

            A deity that gains a sufficient base of worshippers can claim territory within the afterlife plane associated with its alignment.
            Deities have extraordinary power within their claimed territory, and can reshape it as they see fit.
            Any souls that worship a deity will be reborn within that deity's territory in the appropriate afterlife plane, even if that plane does not match their personal alignment.
            Deities must expend soul energy to maintain their territory, so they are always hungry to gain additional followers, and only successful deities expend the effort to claim any territory at all.

        \subsubsection{Nature}
            Nature itself has an immensely vast soul, but although people can worship Nature, it is not a deity because does not depend on mortal worship for its power.
            Nature claims the greatest tithe of every unclaimed death - every predator hunting a prey, every swatted fly.
            These souls are individually tiny.
            However, the combined soul energy released by billions of deaths over millenia dwarfs the power of any other individual entity in the Rise universe.

            Nature lacks a coherent anthropomorphic representation, and its will is almost never brought to bear in any organized way.
            Druids are granted power by Nature, but they need not agree to any particular ideology, and their usage of that power is virtually never policed or revoked by Nature itself in the way that a misbehaving cleric might be punished by their deity.
            Nature welcomes a diversity of viewpoints, for it is itself almost infinitely diverse.
            It has a wealth of power, and it does not expend soul energy maintaining territory in an afterlife plane, so it does not need to jealously hoard its gifts like deities must.
            The only druids who have had their powers revoked were a rare few who turned their powers to the explicit and intentional destruction of Nature itself.

            People who worship nature do not have any special territory in an afterlife reserved for them, since Nature claims no part of any afterlife.
            The afterlife planes are where Nature's power is weakest, and it can claim no tithe of any deaths there, since the planes themselves absorb the soul energy.
            Instead, devoted worshippers of Nature may have their souls reincarnated instead of going to a normal afterlife.
            This gift is not granted to all worshippers, and indeed many would prefer to go to a normal afterlife.

            Every plane that is not the Astral Plane an afterlife plane is a manifestation of Nature's power in some sense, and it claims deaths that occur on any of those planes.
            The four Elemental Planes - Air, Fire, Earth, and Water - are the grandest manifestations of Nature's power.

        \subsubsection{Pact Magic}
            Demons and other entities of great power can make pacts with mortals.
            In these pacts, the mortals offer their soul to the entity for a period of time after death, and the entity who becomes their soulkeeper.
            In exchange, the soulkeeper grants the mortal soul energy from its own supply.
            The soulkeeper's goal is to have the mortal gain a great wealth of its own soul energy in its life, and then to break the will of the soul while it is in the soulkeeper's clutches.
            If the soulkeeper succeeds, it gains the rare and powerful ability to feed on the mortal's entire soul.
            This is a vast wealth of soul energy compared to the normal shards extracted from death and worship, and it annihilates the mortal's soul, preventing it from travelling it to its normal afterlife.

            Successful soulkeepers can therefore amass great power.
            However, it is a risky business, much like adventuring is for mortals.
            If the mortal resists the soulkeeper's torments during its time in the afterlife, it may take its entire soul intact to its normal afterlife.
            When this happens, the soulkeeper loses the bounty of the soul, all of the soul energy it originally invested in the mortal, and time it wasted trying to break the mortal's spirit.
            This is particularly likely if the mortal dies soon after making the pact, so soulkeepers must choose their mortal partners wisely.

            Failing to break a mortal's spirit is not the worst thing that can happen to an overly successful soulkeeper.
            It may may attract attention from more powerful entities within its own plane.
            When a soulkeeper is killed, ownership of the soul is transferred to whatever killed it.
            This means that soulkeepers with active contracts - especially active contracts with mortals who are nearing death after a long life - are extremely attractive targets for anyone who wants to steal the reward of the soul.

        \subsubsection{Ambient Magic and Magical Creatures}
            The world of Rise is full of strange creatures that have superhuman strength or magical abilities, like minotaurs and manticores.
            It is common knowledge that such creatures are typically found only in distant wilderness or in deep dungeons.
            In general, the farther you get from civilization, the more powerful the monsters in the area become, and the more likely you are to encounter strange magical phenomena.
            Small towns seem to cause a subtle warding effect, and powerful monsters in the area will typically avoid them.
            Even monsters that lack the intellectual capacity to understand complex causation chains like ``if I attack the town, they may send powerful warriors to hunt me down'' will typically avoid interacting with civilization unless necessary.

            All of this can be explained by the behavior of souls.
            The constant cycle of life and death in nature produces a great wealth of soul energy.
            Most of it is claimed by Nature itself, but some spills out at the location of each death.
            This soul energy lingers and can build up over time in the form of ambient magic.
            Many monsters can instinctively feed on this ambient magic.
            This naturally allows them to build their power to near the limit of their soul's potential by the time they are adults.

            Civilization disrupts the natural cycles of life and death, reducing the soul energy present in an area.
            Although humanoid creatures have powerful souls, they die less frequently, and the vast majority of the soul energy of their death moves with them to their afterlife.
            From the perspective of creatures that feed on ambient magic, civilized areas stand out as a dead zone.
