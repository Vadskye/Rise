\chapter{Introduction}

\section{What Is Rise?}
    Rise is a role-playing game.
    In Rise, you play a character of your own design.
    Your character can try to do anything you can imagine in a world that the game master, or GM, creates.
    Of course, you won't always succeed.
    Your character's abilities are defined in the pages ahead; when you're done creating a character, it will have a personality of its own, along with strengths, weaknesses, and special abilities.
    Usually, your character will go on adventures with other characters, each of which is played by other players.
    Together, you will create and experience a story with the Game Master, or GM, who defines the universe that the player characters inhabit.

    \subsection{Defining the Undefined}
        This book does not attempt to include specific rules for every aspect of a realistic world.
        Unless defined otherwise - or if it's not worth the effort to look up Rise's exact rules in the flow of a game - you should assume that the universe works more or less like the real world does, and as long as everyone agrees that something is reasonable, it's not worth worrying about in more detail.
        For example, Rise does not have specific rules for how long it takes to eat a meal, the arc that a thrown ball takes through the air, or how much extra weight a well-made chandelier can hold without breaking.
        It's possible to imagine situations where each of those might be important to a game, however, so you'll have to guess what would be reasonable as obscure situations arise.
        The Game Master's has the final word when defining ambiguities like this.

\section{How To Do Things}

    \subsection{Mechanical and Narrative Actions}
        There are two broad categories of actions that a character can take in Rise: mechanical actions and narrative actions.
        Sometimes, you choose one of a list of clearly defined actions that your character can take.
        For example, if you are a wizard who knows the \spell{fireball} spell, you can cast that spell, and there are specific rules for determining its area and effects.
        You can think of that as a mechanical action: there is a specific set of rules and mechanics associated with your chosen action.
        Mechanical actions aren't limited to combat.
        Abilities such as the \spell{create water} spell or the \text{leap} ability can be relevant outside of any combat or even outside of a dramatically interesting scene.

        At other times, you simply describe what you want your character to do.
        For example, you can say that your character steps out of their room in the inn and walks over to knock on a friend's door.
        In those cases, your character isn't taking a specific "knock on door" action that the rules define.
        You can think of that as a narrative action: you're narrating what you want to do.
        Although Rise has rules concerning some aspects of that scenario, such as an Awareness check to see if your friend notices you knocking, you don't have to use the rules all the time.
        If something seems reasonable, it probably is, and you don't need to worry about the fiddly bits.

        Sometimes, it's useful to use rules and dice rolls to determine the effects of narrative actions.
        There is no specific "do a backflip off of a wall and dive into a small pool of water" action, but only particularly skilled or lucky characters should be able to do that successfully.
        You can use one of the broad category abilities that Rise provides to resolving narrative actions.
        That usually involves a \glossterm{check} with a skill or attribute.

        Each character has a set of specific mechanical actions they can take.
        Some are unique to that particular character, such as a wizard's spells known.
        Others are common to all characters, such as the \textit{sprint} action.
        In addition, each character has a set of numerical statistics that describe how good they are at various things.
        These statistics can be referenced by mechanical actions, but they are also used to resolve narrative actions that need dice rolls to resolve.

    \subsection{Making Dice Rolls}
        In general, you should use rules and dice rolls to resolve narrative actions if they meet two conditions.
        First, the action should be dramatically interesting.
        No one wants to waste time determining the success or failure of a series of completely irrelevant actions.
        Second, the action should have a meaningful chance of both success and failure.
        Sufficiently skilled characters may not need to roll to accomplish simple tasks, and there's no point in rolling for things that are obviously impossible.

        If both conditions are met, you can roll a ten-sided die, or d10.
        You add a number based on how competent your character is with the action to the number you rolled on the die.
        This result is compared to a number representing how difficult the action is.
        If your result is at least as high as the number you were trying to get, your character's action succeeded.
        Otherwise, the action fails.
        Depending on the context, failure may be simply equivalent to having never attempted the action, or it may have serious consequences.

    \subsection{Attacks and Checks}
        Almost all rolls you will need to make can be described as an \glossterm{attack roll} or a \glossterm{check}.
        Aything that affects another creature in a potentially harmful way is considered an attack.
        When you make an attack roll, you add a bonus based on your \glossterm{accuracy} with the attack.
        If the result of your roll is at least as high as your target's \glossterm{defense}, your attack hits.
        Otherwise, it misses.
        Attacks often deal damage, but they an also impose penalties or have other effects.

        Dice rolls that don't directly affect other creatures are generally checks.
        When you make a check, you add a bonus based on your competence with the activity, which is often a \glossterm{skill}.
        If the result of your roll is at least as high as the \glossterm{difficulty rating} of the activity, your check succeeds.
        Otherwise, it fails.

        \subsubsection{Opposed Actions}
            Sometimes, you're competing with another creature, such as when you are trying to hold a door closed and a ferocious ogre is trying to shove it open.
            In that case, you both make a \glossterm{check}, and the creature with the higher result wins.
            This is called an opposed check.
            If you both get the same result, roll again to break the tie.

\section{Narrative Pacing}
    A typical game of Rise is broken up into two distinct concepts.
    Sometimes, you will interact with the world in broad, sweeping descriptions, as you describe in general terms what your character does.
    For example, you and your group may decide to travel overland for seven days to reach a distant city.
    Just like you don't need to mention every time your character breathes or takes a step, you don't need to track every second of time spent on that journey.
    Until something of dramatic significance happens, the GM should narrate the events of the journey in general terms to avoid spending time on irrelevant details.

    If something important happens, the pace of the game should change to focus on the current events.
    For example, if your group is ambushed by roaming bandits, the exact actions you take become important.
    This is called an encounter.
    During an encounter, the GM should ask each player to describe their character's actions more precisely.
    If the encounter is time-sensitive, such as combat, the GM should track the flow of time in \glossterm{rounds}.
    A round represents six seconds of time, so there are ten rounds per minute.

    Not all encounters require tracking time precisely.
    For example, if you and your party encounter a broken-down caravan carrying an angry prince, your exact actions can be important.
    Whether you show appropriate deference and fix his caravan, ignore him, or kill him to steal his jewelry can have important repurcussions in the game world.
    However, the exact time it takes to make that decision and execute on it is not usually important.

    Not every challenge needs to be handled with the narrative pacing of an encounter, especially if the outcome of the challenge is uninteresting or predetermined.
    Even combat can be described in broad terms, such as if you are significantly more powerful than your foes and your victory is assured.
    In all cases, the GM decides how to handle the flow of time and the narrative pacing.
    They may have information you don't about what is important and what is irrelevant.

\section{Principles of Rise}

    \begin{itemize}
        \item \textbf{The Game Master controls the world.} Everything about the game world is up to the GM\@: the people your character meets, the challenges they will overcome, and the very ground under their feet.
            Even the ``rules'' of the game are completely subject to the GM's whim.
        \item \textbf{You control your character.} A GM should never tell you how your character feels or what they try to do --- unless, of course, your character is being controlled by hostile magic or some other power.
        \item \textbf{Respect and trust are critical.} The GM has a great deal of power in Rise, and the players have to trust that the GM knows what they are doing.
            Likewise, the GM needs to let the players do what they want --- even if it doesn't suit their idea of what ``should'' happen.
            Some of the most memorable events happen when players do things that are totally unexpected.
        \item \textbf{Everything is flexible.} Rise contains hundreds of pages of rules.
            We think they're pretty good rules.
            But sometimes, you don't need them all --- or you think you've come up with something better.
            Do whatever works for you and your group.
        \item \textbf{Do what makes sense.} This book doesn't contain rules for how to drink water, sleep, or blink.
            If something isn't described explicitly here, assume that it works the same way it does in reality.
        \item \textbf{It's just a game, so have fun.}
    \end{itemize}

    \subsection{Rule of Drama}

        The Rule of Drama is simple: \textbf{Only dramatic actions matter}.
        In general, an action has dramatic significance if there is a consequence for failure, the exact time required to perform the task is important, or the game master says so.
        This has several effects.

        \subsubsection{Checks without Rolls}
            A check represents an attempt to accomplish a dramatically significant goal, usually while under some sort of time pressure or distraction.
            When there is no dramatic significance to the task, a check should not be rolled.

            \parhead{Taking 5}\label{Taking 5}
            Normally, if a check would be required but there is no dramatic significance to the check, assume the character rolled a 5 when determining whether the check is successful.
            This is called ``taking 5''.

            \parhead{Taking 10}\label{Taking 10}
            If a character would not succeed when taking 5, the character can try to ``take 10'' instead.
            Taking 10 requires spending ten times the amount of time normally required to accomplish the task.
            If a task takes a variable amount of time, assume it took the average amount of time required to make a check.
            In exchange, the character calculates their check result if they had rolled a 10.

            Essentially, taking 10 means the character repeatedly attempts the task until they succeed.
            It is possible to take 10 on a task that has consequences for failure, but taking 10 guarantees that those consequences occur.

        \subsubsection{Narrative Time}
            In most cases, the exact time of day and exactly how long an action takes is not very important.
            No one really cares whether the orc horde broke through the town gates at the crack of dawn, or at 6:55 AM\@.
            Likewise, travelling from a tavern in a city to the blacksmith probably takes a few minutes, depending on the size of the city, but it's not usually important.

            Rise contains rules which you can use to decide exactly how much time things take, but when it's not important, it's generally better to only worry about time in broad strokes.
            It makes everyone's life a bit easier --- especially for the GM\@.

\section{Character Creation}

    The first thing you will probably want to do in Rise is create a character.
    This involves a mixture of thematic and mechanical decisions that will work together to create a fun character that is rewarding to play.
    There are five thematic considerations when creating a character: concept, personality, motivation, background, and appearance.
    There are five mechanical considerations: attributes, species, class, skills, and feats.
    These decisions are described below in a recommended order.
    However, you can make these decisions in any order, and you may find it easier to create a character in a different way.  


    \subsection{Narrative Choices}

        \subsubsection{Character Concept}

            Fundamentally, who is your character?
            You should think of a short phrase that describes the core concept behind the character you will create.
            It's best to think in broad strokes when creating a character concept.
            Your concept should be more than just a factual description of your species or what you do.
            It should be something that makes you memorable.
            Some sample character concepts are given below for inspiration.

            \begin{itemize}
                \item Pragmatic wanderer
                \item Artistic pixie
                \item Mushroom-obsessed hermit
                \item Bumbling do-gooder
                \item Dim-witted bodyguard
                \item Cowardly storyteller
                \item Bear-barian
                \item Parsimonious law enforcer
                \item Peaceful naturalist
                \item Trigger-happy pyromaniac
                \item Heroic, simple-minded warrior
                \item Friendly necromancer
                \item Chaotic speed demon
                \item Pompous ex-noble
                \item Sarcastic mercenary
                \item Battle-scarred priest
                \item Ambitious arcane prodigy
                \item Charismatic musician
                \item Aloof scholar
                \item Blunt-spoken warrior
                \item Crazed prophet
                \item Polite warrior
                \item World-weary pirate
                \item Devout cultist
                \item Con artist with a heart of gold
            \end{itemize}

        \subsubsection{Personality}

            How does your character behave?
            You should decide, in broad terms, what your character's personality is.
            This will change over time, especially as you start playing the character in the game, so you don't need to define everything perfectly.
            However, having a general sense of how your character behaves is helpful.

            For most games, it's important to have a personality that can tolerate working with others in a group.
            A character that is excessively aloof, moody, or obnoxious can make the game more difficult to enjoy for everyone.
            Likewise, a character who tries to speak for everyone or who repeatedly steals the spotlight from others can be frustrating to work with.
            You should figure out the right balance with your fellow players and your GM.\@

        \subsubsection{Motivation and Goal}
            Why does your character put in all of the effort that adventuring requires?
            They probably have a goal that they are trying to achieve, or an ideal that they are trying to embody.
            Writing down a specific goal or ideal can be helpful as an anchor point when defining the character.

        \subsubsection{Alignment}
            Your character's alignment reflects their moral character: are they more inclined to good or evil, and to chaos or order?
            Alignments are described in more detail at \pcref{Alignment}.

        \subsubsection{Background}
            What happened in their character's past to make them the way that they are?
            What were their parents like, and where are they now?
            You don't have to have all of the answers when you first create a character, but it's good to have some idea.
            The richer your backstory, the more the GM can weave that into the narrative of the current story.
            Sometimes, it's fun to take a break from saving the world to go visit someone's grandma.

        \subsubsection{Appearance}
            What does your character look like?
            What would someone's first impression of them be?
            This can be helpful for understanding how other characters in the game world - or even monsters - would react to you.

    \subsection{Mechanical Choices}

        \subsubsection{Species}
            Is your character a surly dwarf, an agile elf, or a dim-witted half-orc?
            There are six core species in Rise: Humans, dwarves, elves, gnomes, halflings, and half-orcs.
            Your character's species has only a mild effect on your character's abilities, but it can be important when thinking about your personality and background.
            Species are described in more detail at \pcref{Species}.

        \subsubsection{Attributes}
            Is your character strong and stupid, frail but agile, or just generally talented?
            There are six core attributes in Rise: Strength, Dexterity, Constitution, Intelligence, Perception, and Willpower.
            These attributes determine a great deal about your character's strengths and weaknesses.
            Attributes are described in more detail at \pcref{Attributes}.

        \subsubsection{Class and Archetypes}
            Your character's class is what they have chosen to focus on, and their source of power --- the fundamental element that makes them rise above a mere commoner.
            It is the most mechanically significant choice you make.
            Each class has five \glossterm{archetypes}.
            You choose three of those five archetypes for your character.
            Classes are described in more detail at \pcref{Classes}.

            You can choose to be a multiclass character instead of only having a single class.
            This allows you to choose archetypes from any of the classes that you have.
            For details, see \pcref{Multiclass Characters}.

        \subsubsection{Spend Insight Points}
            You can spend \glossterm{insight points} to gain new special abilities or to learn new proficiencies.
            For details, see \pcref{Insight Points}.

        \subsubsection{Spend Skill Points}
            You can spend \glossterm{skill points} to gain training in skills.
            For details, see \pcref{Skill Points}.

        \subsubsection{Choose Proficiencies}
            You are automatically proficient in specific languages and \glossterm{weapon groups} based on your species and class.
            For details, see \pcref{Communication and Languages}, and \pcref{Weapon Groups}.

        \subsubsection{Equipment}
            Items have item levels that indicate the approximate level that characters can reasonably get access to them.
            Typically, you can start with a single level 1 or lower item, up to three level 1/2 items, and a standard adventuring kit.
            Individual campaigns or character backstories may change what starting equipment is available, so check with your GM.
