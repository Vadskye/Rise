\chapter{Running The Game}

This book provides helpful guidance on how to play the Rise role-playing game.
It includes some context for the narrative universe of Rise, mechanics for monsters and other antagonists, and clarifying rules for special circumstances.

\sectiongraphic{Why Use So Many Rules?}{width=\columnwidth}{introduction/so many rules}

  Tabletop role-playing games attempt to create rules to define how their universe works.
  Some games are intentionally vague or minimalist about their rules, which can be fun!
  Simple games are easy to start playing, and they try to avoid getting in the way of good role-playing.
  However, Rise takes a different approach.
  It spends a lot of effort - and words - attempting to define an internally consistent universe, and creating a large number of specific abilities that can be used in that universe.
  There are a few important advantages to taking this approach: establishing expectations, supporting multiple play styles, and assisting the GM.

  \subsection{Establishing Expectations}
    Different people can have very different ideas about what is realistic - or narratively appropriate - in a made-up fantasy universe.
    To some people, kicking in the tavern door and starting a brawl is just some good clean fun, and you'll take a few good punches and then laugh about it later that evening over drinks.
    But to other people, that might sound like a good way to find yourself imprisoned for the foreseeable future with all of your possessions confiscated by the town guard.
    Another interpretation of that scenario might see the brawler seriously injured with a broken bottle in the eye, leaving them partially blinded for weeks - or indefinitely.

    All of those ideas are valid, and they each match the narrative of a particular type of story.
    However, it's important that everyone sitting at a table and playing a game agrees about what to expect.
    Players can get confused or frustrated when their actions have consequences that feel arbitrary or unfair.
    Generally, games are more fun if everyone in the game shares a common set of expectations and conventions.
    Otherwise, games can devolve into disagreements about what is or isn't reasonable.

    One way to establish these expectations is to use a rules system like Rise that defines some expectations explicitly.
    If the scenario above happened in Rise, the last outcome of an incapacitating bottle to the eye shouldn't normally be possible, since the rules explicitly define how injury works.
    Knowing what is and isn't possible can help give players and GMs a useful set of guardrails for what they try to do in the universe.
    It's relatively easy to get everyone to agree about simple things that regular human people have experience with, like how difficult it is to climb a tree.
    However, Rise is full of superhuman people and monsters, and eventually you'll need to figure out how far a barbarian as strong as Hercules can throw a bear.
    Having a single authoritative resource to consult can cut off long disagreements about details that are difficult or impossible to determine objectively.

    Of course, different games played with a flexible rules system like Rise can have very different tones and themes.
    Either of the first two scenarios in the tavern are still plausible in different games, and a GM can use house rules to make vital wounds have more long-term consequences if they want.
    Using a rules system like Rise can help, but it is not the full answer by itself.
    The GM and players always share responsibility for establishing expectations about what genre a game will be, and conforming to those expectations to the extent that it makes the game more fun.

  \subsectiongraphic{Supporting Multiple Play Styles}{width=\columnwidth}{introduction/multiple play styles}
    Some people deeply enjoy the process of role-playing itself.
    They enjoy the process of getting into a character and speaking in their voice, exploring their needs and desires, and building a narrative for them over time.
    These people often do not need the confines of a robust rules system, and can play equally well in games with minimal rules or none at all.

    Other people do not enjoy role-playing as an end in itself, or even at all.
    However, they may still enjoy the \textit{game} aspect of a role-playing game.
    Instead of playing a character for their personality and backstory, they may play a character for their unique mechanics and tactical advantages.

    Still other people may be interested in role-playing as a concept, but find it daunting.
    The blank page in front of you when you start painting a picture or writing an essay can be daunting, and that first step is often the hardest to take.
    Giving people a clearly defined set of abilities and specific tools for interacting with the world can enhance creativity by providing a safe space for interaction and experimentation.
    Even if you don't enjoy or feel confident in speaking in your character's voice, you can still engage with the narrative aspects of the adventure by casting a relevant spell or making a relevant skill check.
    People in this middle ground can sometimes enjoy deeper role-playing games while being feeling lost in role-playing games with minimal or nonexistent rules.

    One of the joys - and challenges - of Rise is drawing together people with very different desires and play styles to share a single experience.
    Rules-free role-playing games and tactical wargames can both have a narrower appeal than rules-heavy role-playing games like Rise, which try to provide something for everyone.
    You can run games with deep role-players alongside tactical gamers, and it can be a lot of fun.
    It does place a greater burden on the GM to provide the right ratio of content to keep everyone happy, and it does require the players to be patient when their preferred playstyle is put in the background to support the needs of other players.
    A well-blended game can also draw people out of their comfort zones slowly and safely over time as they observe and start to enjoy the playstyles of the other players in the game.

  \subsection{Assisting the GM}
    The Game Master carries an extra weight of responsibility to shape the flow of the game.
    Creating narratively consistent universes, appropriate challenges, and engaging storylines out of thin air is deeply challenging.
    If this job is too difficult, no one will want to do it, and then no one will play the game!
    Making the GM's job easier is a critical component of any role-playing game.

    There are several ways that Rise can make the GM's job easier.
    It provides information about the mechanics and tropes of the universe that the game takes place in, which helps establish expectations and resolve disputes that might come up during the game.
    It will provide a clear narrative foundation for the world and the characters in that world, which minimizes the up-front work required to run a game, once that section of the book is more complete.
    It will provide a wealth of pre-packaged challenges appropriate for players of any power level or play style, and advice for how to use those challenges appropriately, once that section of the book is more complete.
    The GM-focused sections are currently the most unfinished part of Rise, and this will be a more useful guide before Rise is done.

\section{Ad-Hoc Circumstantial Modifiers}
  Circumstances frequently modify a creature's odds of success when making attacks and checks, or when defending itself from attacks.
  Rise defines a number of specific circumstances with explicit effects, but as the GM, you should feel free to decide that any circumstances should modify the odds of success.

  There are two kinds of circumstantial modifiers.
  Circumstances that make a creature better or worse at its task give it a bonus or penalty to its attack or check.
  Circumstances that make the task easier or harder increase or decrease the difficulty value of the task, or the defense of the attacked creature.

  Most circumstances grant a \plus2 bonus or impose a \minus2 penalty.
  Of course, you can get more specific than that, especially in unusually significant circumstances.

\section{Narrative Pacing}
  A typical game of Rise is broken up into two distinct concepts.
  Sometimes, you will interact with the world in broad, sweeping descriptions, as you describe in general terms what your character does.
  For example, you and your group may decide to travel overland for seven days to reach a distant city.
  Just like you don't need to mention every time your character breathes or takes a step, you don't need to track every second of time spent on that journey.
  Until something of dramatic significance happens, the GM should narrate the events of the journey in general terms to avoid spending time on irrelevant details.

  If something important happens, the pace of the game should change to focus on the current events.
  For example, if your group is ambushed by roaming bandits, the exact actions you take become important.
  This is called an encounter.
  During an encounter, the GM should ask each player to describe their character's actions more precisely.
  If the encounter is time-sensitive, such as combat, the GM should track the flow of time in \glossterm{rounds}.
  A round represents six seconds of time, so there are ten rounds per minute.

  Not all encounters require tracking time precisely.
  For example, if you and your party encounter a broken-down caravan carrying an angry prince, your exact actions can be important.
  Whether you show appropriate deference and fix his caravan, ignore him, or kill him to steal his jewelry can have important repurcussions in the game world.
  However, the exact time it takes to make that decision and execute on it is not usually important.

  Not every challenge needs to be handled with the narrative pacing of an encounter, especially if the outcome of the challenge is uninteresting or predetermined.
  Even combat can be described in broad terms, such as if you are significantly more powerful than your foes and your victory is assured.
  In all cases, the GM decides how to handle the flow of time and the narrative pacing.
  They may have information you don't about what is important and what is irrelevant.

\section{Principles of Rise}

  \begin{itemize}
    \item \textbf{The Game Master controls the world.} Everything about the game world is up to the GM\@: the people your character meets, the challenges they will overcome, and the very ground under their feet.
      Even the ``rules'' of the game are completely subject to the GM's whim.
    \item \textbf{You control your character.} A GM should never tell you how your character feels or what they try to do --- unless, of course, your character is being controlled by hostile magic or some other power.
    \item \textbf{Respect and trust are critical.} The GM has a great deal of power in Rise, and the players have to trust that the GM knows what they are doing.
      Likewise, the GM needs to let the players do what they want --- even if it doesn't suit their idea of what ``should'' happen.
      Some of the most memorable events happen when players do things that are totally unexpected.
    \item \textbf{Everything is flexible.} Rise contains hundreds of pages of rules.
      We think they're pretty good rules.
      But sometimes, you don't need them all --- or you think you've come up with something better.
      Do whatever works for you and your group.
    \item \textbf{Do what makes sense.} This book doesn't contain rules for how to drink water, sleep, or blink.
      If something isn't described explicitly here, assume that it works the same way it does in reality.
    \item \textbf{It's just a game, so have fun.}
  \end{itemize}

  \subsection{Rule of Drama}

    The Rule of Drama is simple: \textbf{Only dramatic actions matter}.
    In general, an action has dramatic significance if there is a consequence for failure, the exact time required to perform the task is important, or the game master says so.
    This has several effects.

    \subsubsection{Checks without Rolls}
      A check represents an attempt to accomplish a dramatically significant goal, usually while under some sort of time pressure or distraction.
      When there is no dramatic significance to the task, a check should not be rolled.

      \parhead{Taking 5}\nonsectionlabel{Taking 5}
      Normally, if a check would be required but there is no dramatic significance to the check, assume the character rolled a 5 when determining whether the check is successful.
      This is called ``taking 5''.

      \parhead{Taking 10}\nonsectionlabel{Taking 10}
      If a character would not succeed when taking 5, the character can try to ``take 10'' instead.
      Taking 10 requires spending ten times the amount of time normally required to accomplish the task.
      If a task takes a variable amount of time, assume it took the average amount of time required to make a check.
      In exchange, the character calculates their check result if they had rolled a 10.

      Essentially, taking 10 means the character repeatedly attempts the task until they succeed.
      It is possible to take 10 on a task that has consequences for failure, but taking 10 guarantees that those consequences occur.

\section{Uncommon Combat Circumstances}

  Even if the attack had no obvious physical or visual effects, a creature that resists an attack still feels a hostile force or a tingle, but cannot usually deduce the exact nature of the attack.

  Creatures can voluntarily lower their defenses against attacks that they are aware of.
  When they do, their defense is treated as 0 against the attack.

  \parhead{Dealing Damage, Taking Damage, and Losing Hit Points}
  Some abilities trigger when a creature deals damage or is dealt damage.
  Other abilities trigger when a creature loses hit points or causes another creature to lose hit points.
  An attack deals damage even if all damage dealt by the attack is applied to \glossterm{damage resistance} instead of \glossterm{hit points}.

  \subsection{Improvised Combat Abilities}
    Sometimes, players will try to improvise attacks that don't match any of their written abilities.
    In general, clever improvisation using the environment can be a fun way to add excitement to a combat.
    The safest way to reward improvisation is to give a bonus to a character's existing abilities, like an accuracy or damage bonus.
    Be careful about letting players effectively create new abilities for their character out of thin air.

    Constant improvisation can slow the game down and make your life harder.
    It's hard to invent reasonable new abilities on the fly.
    If you make the effect of their improvised action too weak, they can feel like they wasted their time or were punished for their creativity.
    If it's too strong, it can change the game balance in significant ways and make other players wonder why they bother using their abilities as written.
    That can make players who don't enjoy or who aren't good at thinking on the fly like that feel like their characters can't contribute as much.

    For example, players might reasonably try to improvise an attack that would disarm an opponent of their weapon.
    Disarming is a classic narrative trope, and it sounds plausible at a glance.
    However, losing your weapon is a debilitating effect numerically, especially if you allow the player to steal the dropped weapon.
    If warriors in Rise could be disarmed with a simple attack roll, it would massively reduce the power of melee martial characters, including player characters.
    Melee-focused combatants already have more complex positioning requirements and are in danger from enemy attacks, so giving them an additional penalty would discourage anyone from going into melee.
    It's better to treat disarming as being a loss condition akin to being defeated or running out of hit points.

  \subsection{Jump Arcs}
    Generally, it's way too complicated to deal with the exact path that a jumping creature takes during its jump.
    According to the Rise rules, some of the jumping arcs look very strange.
    For example, if a creature jumps forward at a 45 degree angle into thin air, it will fall to the ground in a straight line at the end of its jump, which creates an unrealistic triangle-shaped trajectory.
    Try not to get bogged down in the details of exactly what space creatures occupy in midair, or the exact arc.

    If you really want to be more detailed, you can say that a creature's maximum height during a horizontal jump must happen in the middle of its jump.
    That maximum height jump must be no less than a quarter of the forward distance travelled.
    Those rules generate more realistic outcomes if a creature tries to make a thirty-foot long jump in a room with a five-foot ceiling, since the creature should hit its head on the ceiling and be unable to complete its jump.
    However, although this creates more realistic results, it is much more convoluted to resolve, which can take time.
    In addition, it weakens the mobility of mundane characters, which makes magical forms of mobility like flight even more powerful than they already are.
    As always, use the rules and conventions that keep you and your player group happiest.

  \subsection{Object Statistics}
    An object's size primarily influences the number of \glossterm{hit points} it has.
    The primary material it is constructed from determines its \glossterm{damage resistance}, and can modify the number of hit points it has.
    Details are given in \tref{Object Statistics By Size} and \tref{Object Statistics By Material}.

    These rules are more detailed than you should really need.
    During a typical game session, it's often best to just guess whether a character could plausibly sunder or smash an object rather than consulting these tables.

    \begin{dtable}
      \lcaption{Object Statistics By Size}
      \begin{dtabularx}{\textwidth}{l X X}
        \tb{Size}  & \tb{Hit Points} & \tb{Sunder DV Modifier} \tableheaderrule
        Fine       & 1               & 0\fn{1} \\
        Diminutive & 2               & 0       \\
        Tiny       & 5               & 5       \\
        Small      & 10              & 10      \\
        Medium     & 20              & 15      \\
        Large      & 50              & 20      \\
        Huge       & 100             & 25      \\
        Gargantuan & 200             & 30      \\
        Colossal   & 500             & 35      \\
      \end{dtabularx}
      1. Extremely small objects may be difficult to grip effectively, which can significantly increase the difficulty to sunder them.
    \end{dtable}

    \begin{dtable}
      \lcaption{Object Statistics By Material}
      \begin{dtabularx}{\textwidth}{l X X X}
        \tb{Material}   & \tb{DR}\fn{1} & \tb{HP Multiplier}\fn{2} \tableheaderrule
        Adamantine      & 30            & \mult3   \\
        Glass           & 5             & \mult1/2 \\
        Ice             & 0             & \mult1/2 \\
        Iron or steel   & 15            & \mult2   \\
        Leather or hide & 5             & \tdash   \\
        Mithral         & 20            & \mult2   \\
        Paper or cloth  & 0             & \mult1/2 \\
        Rope            & 5             & \tdash   \\
        Stone           & 10            & \mult2   \\
        Wood            & 5             & \tdash   \\
      \end{dtabularx}
      1. An object's \glossterm{damage resistance} also increases the difficulty value of checks to sunder it with raw Strength. \\
      2. Any value here modifies the number of hit points the object would normally have based on its size.
    \end{dtable}

\section{Player vs Player Combat}
  Most of the rules of Rise function in the same way for monsters and players.
  Monsters calculate their statistics in a simpler way, but they still have the same fundamental set of actions.
  However, there is a small quirk in the timing of combat declaration that doesn't work if there are players on both sides of an encounter.

  Normally, players are allowed to declare \abilitytag{Swift} actions during their normal turn.
  This can create a situation where a player decides whether or not to take a Swift action based on seeing the results of an earlier non-Swift action within their allied group.
  For example, a melee fighter might choose to use the \ability{total defense} ability because her allies killed all of the adjacent enemies.
  However, what if one of those enemies was making a decision about whether to use \ability{total defense} in the same way?
  Suppose the melee fighter will also drop unconscious or die from incoming damage from that monster's allies.
  There's a paradox - if that monster defends itself knowing that the figher will die, it might turn a hit into a miss, which means it would be alive for the melee fighter to attack.
  However, if the fighter defends herself, the monster will attack instead of defending itself because the fighter won't be dead.

  Monsters don't make complicated contingent action decisions.
  This timing oddity shouldn't come up during normal gameplay.
  However, it can cause problems when players are on both sides of a combat.

  Fortunately, this issue is fairly easy to solve.
  Players must be required to pre-declare any \abilitytag{Swift} actions they plan on taking before anyone's turn actually begins.
  Each player still takes turns in the normal order, which matters for triggered effects that happen during a player's action.
  This prevents contigent Swift actions from creating weird paradoxes.

  You can choose to run this way all the time, if you want.
  Forcing players to pre-declare Swift actions makes slightly more sense, and you might find it useful in specific combat scenarios against intelligent monsters.
  However, it's a bit of a hassle.
  Forcing each player to decide whether they want to take a Swift action before anyone resolves any actions can slow down combat significantly.

\section{All Items}

  Sometimes, it can be useful to know the entire list of items that exist.

  \begin{longcolumn}
    \input{generated/everything_table.tex}
  \end{longcolumn}
