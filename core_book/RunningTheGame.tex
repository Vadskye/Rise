\chapter{Running The Game}

This book provides helpful guidance on how to play the Rise role-playing game.
It includes some context for the narrative universe of Rise, mechanics for monsters and other antagonists, and clarifying rules for special circumstances.

\section{Ad-Hoc Circumstantial Modifiers}
    Circumstances frequently modify a creature's odds of success when making attacks and checks, or when defending itself from attacks.
    Rise defines a number of specific circumstances with explicit effects, but as the GM, you should feel free to decide that any circumstances should modify the odds of success.

    There are two kinds of circumstantial modifiers.
    Circumstances that make a creature better or worse at its task give it a bonus or penalty to its attack or check.
    Circumstances that make the task easier or harder increase or decrease the difficulty value of the task, or the defense of the attacked creature.

    Most circumstances grant a \plus2 bonus or impose a \minus2 penalty.
    Of course, you can get more specific than that, especially in unusually significant circumstances.

\section{Narrative Pacing}
    A typical game of Rise is broken up into two distinct concepts.
    Sometimes, you will interact with the world in broad, sweeping descriptions, as you describe in general terms what your character does.
    For example, you and your group may decide to travel overland for seven days to reach a distant city.
    Just like you don't need to mention every time your character breathes or takes a step, you don't need to track every second of time spent on that journey.
    Until something of dramatic significance happens, the GM should narrate the events of the journey in general terms to avoid spending time on irrelevant details.

    If something important happens, the pace of the game should change to focus on the current events.
    For example, if your group is ambushed by roaming bandits, the exact actions you take become important.
    This is called an encounter.
    During an encounter, the GM should ask each player to describe their character's actions more precisely.
    If the encounter is time-sensitive, such as combat, the GM should track the flow of time in \glossterm{rounds}.
    A round represents six seconds of time, so there are ten rounds per minute.

    Not all encounters require tracking time precisely.
    For example, if you and your party encounter a broken-down caravan carrying an angry prince, your exact actions can be important.
    Whether you show appropriate deference and fix his caravan, ignore him, or kill him to steal his jewelry can have important repurcussions in the game world.
    However, the exact time it takes to make that decision and execute on it is not usually important.

    Not every challenge needs to be handled with the narrative pacing of an encounter, especially if the outcome of the challenge is uninteresting or predetermined.
    Even combat can be described in broad terms, such as if you are significantly more powerful than your foes and your victory is assured.
    In all cases, the GM decides how to handle the flow of time and the narrative pacing.
    They may have information you don't about what is important and what is irrelevant.

\section{Principles of Rise}

    \begin{itemize}
        \item \textbf{The Game Master controls the world.} Everything about the game world is up to the GM\@: the people your character meets, the challenges they will overcome, and the very ground under their feet.
            Even the ``rules'' of the game are completely subject to the GM's whim.
        \item \textbf{You control your character.} A GM should never tell you how your character feels or what they try to do --- unless, of course, your character is being controlled by hostile magic or some other power.
        \item \textbf{Respect and trust are critical.} The GM has a great deal of power in Rise, and the players have to trust that the GM knows what they are doing.
            Likewise, the GM needs to let the players do what they want --- even if it doesn't suit their idea of what ``should'' happen.
            Some of the most memorable events happen when players do things that are totally unexpected.
        \item \textbf{Everything is flexible.} Rise contains hundreds of pages of rules.
            We think they're pretty good rules.
            But sometimes, you don't need them all --- or you think you've come up with something better.
            Do whatever works for you and your group.
        \item \textbf{Do what makes sense.} This book doesn't contain rules for how to drink water, sleep, or blink.
            If something isn't described explicitly here, assume that it works the same way it does in reality.
        \item \textbf{It's just a game, so have fun.}
    \end{itemize}

    \subsection{Rule of Drama}

        The Rule of Drama is simple: \textbf{Only dramatic actions matter}.
        In general, an action has dramatic significance if there is a consequence for failure, the exact time required to perform the task is important, or the game master says so.
        This has several effects.

        \subsubsection{Checks without Rolls}
            A check represents an attempt to accomplish a dramatically significant goal, usually while under some sort of time pressure or distraction.
            When there is no dramatic significance to the task, a check should not be rolled.

            \parhead{Taking 5}\label{Taking 5}
            Normally, if a check would be required but there is no dramatic significance to the check, assume the character rolled a 5 when determining whether the check is successful.
            This is called ``taking 5''.

            \parhead{Taking 10}\label{Taking 10}
            If a character would not succeed when taking 5, the character can try to ``take 10'' instead.
            Taking 10 requires spending ten times the amount of time normally required to accomplish the task.
            If a task takes a variable amount of time, assume it took the average amount of time required to make a check.
            In exchange, the character calculates their check result if they had rolled a 10.

            Essentially, taking 10 means the character repeatedly attempts the task until they succeed.
            It is possible to take 10 on a task that has consequences for failure, but taking 10 guarantees that those consequences occur.

\section{Overland Movement}\label{Overland Movement}

    This section provides rules governing overland movement speeds.
    Not every game should think about overland movement travel speed in a detailed way.
    It's fine to just say that characters spend ``a few days'' walking around between various important locations.

    However, sometimes you do care about the details.
    You might be running a low fantasy campaign where characters track rations and struggle against their environment more often than fantastic monsters.
    Alternately, you might present players with a specific deadline in-game, like ``the Ritual of Corruption will be finished in two weeks'', and the players might be interested in figuring out exactly how much travelling they can do before the deadline is up.

    \subsection{Standard Travel Days}
        Characters covering long distances cross-country use overland movement.
        Overland movement is measured in miles per hour or miles per day.
        A day normally represents 10 hours of actual travel time.
        However, sailing ships and other methods of travel that keep moving without requiring a rest are listed with a full 24 hours of travel time.

        Creatures can make an Endurance check to push beyond a standard 10-hour travel day.
        In addition, they can make an Endurance check to travel faster within a normal travel day.
        For details, see \pcref{Overland Exertion}.

        Standard travel distances on foot are listed in \tref{Travel Distance By Movement Speed}.
        When using mounts or ships, \tref{Mounts and Vehicles} will be more convenient.

    \begin{dtable}
        \lcaption{Travel Distance By Movement Speed}
        \begin{dtabularx}{\columnwidth}{>{\lcol}X c c c c}
            & \multicolumn{4}{c}{\tdash\tdash\tdash Speed \tdash\tdash\tdash} \tableheaderrule
                                 & 15 feet     & 20 feet  & 30 feet     & 40 feet  \\
            One Hour (Overland)  &             &          &             &          \\
            Walk                 & 3/4 mile    & 1 mile   & 1-1/2 miles & 2 miles  \\
            Hustle               & 1-1/2 miles & 2 miles  & 3 miles     & 4 miles  \\
            One Day (Overland)   &             &          &             &          \\
            Walk                 & 7-1/2 miles & 10 miles & 15 miles    & 20 miles \\
            Hustle               & \tdash      & \tdash   & \tdash      & \tdash   \\
        \end{dtabularx}
    \end{dtable}

    \begin{dtable}
        \lcaption{Mounts and Vehicles}
        \begin{dtabularx}{\columnwidth}{>{\lcol}X l l}
            \tb{Mount/Vehicle}                         & \tb{Per Hour} & \tb{Per Day} \tableheaderrule
            Mount (carrying load)                      &               &          \\
            \tind Light horse or light warhorse        & 6 miles       & 60 miles \\
            \tind Light horse                          & 4 miles       & 40 miles \\
            \tind Light warhorse                       & 4 miles       & 40 miles \\
            \tind Heavy horse or heavy warhorse        & 5 miles       & 50 miles \\
            \tind Heavy horse                          & 3-1/2 miles   & 35 miles \\
            \tind Heavy warhorse                       & 3-1/2 miles   & 35 miles \\
            \tind Pony or warpony                      & 4 miles       & 40 miles \\
            \tind Pony                                 & 3 miles       & 30 miles \\
            \tind Warpony                              & 3 miles       & 30 miles \\
            \tind Donkey or mule                       & 3 miles       & 30 miles \\
            \tind Donkey                               & 2 miles       & 20 miles \\
            \tind Mule                                 & 2 miles       & 20 miles \\
            \tind Dog, riding                          & 4 miles       & 40 miles \\
            \tind Dog, riding                          & 3 miles       & 30 miles \\
            \tind Cart or wagon                        & 2 miles       & 20 miles \\
            \tb{Ship}                                  &               &          \\
            \tind Raft or barge (poled or towed)\fn{1} & 1/2 mile      & 5 miles  \\
            \tind Keelboat (rowed)\fn{1}               & 1 mile        & 10 miles \\
            \tind Rowboat (rowed)\fn{1}                & 1-1/2 miles   & 15 miles \\
            \tind Sailing ship (sailed)                & 2 miles       & 48 miles \\
            \tind Warship (sailed and rowed)           & 2-1/2 miles   & 60 miles \\
            \tind Longship (sailed and rowed)          & 3 miles       & 72 miles \\
            \tind Galley (rowed and sailed)            & 4 miles       & 96 miles \\
        \end{dtabularx}
        1 Rafts, barges, keelboats, and rowboats are used on lakes and rivers.
        If going downstream, add the speed of the current (typically 3 miles per hour) to the speed of the vehicle. In addition to 10 hours of being rowed, the vehicle can also float an additional 14 hours, if someone can guide it, so add an additional 42 miles to the daily distance traveled. These vehicles can't be rowed against any significant current, but they can be pulled upstream by draft animals on the shores.
    \end{dtable}

    \subsection{Overland Terrain}
        Travelling over a flat, paved highway is much faster than trailblazing through a jungle.
        You can use \tref{Terrain and Overland Movement} as a reference for common terrain.

        A highway is a straight, major, paved road.
        A road is typically a dirt track.
        A trail is like a road, except that it allows only single-file travel and does not benefit a party traveling with vehicles.
        Trackless terrain is a wild area with no significant paths.

    \begin{dtable}
        \lcaption{Terrain and Overland Movement}
        \begin{dtabularx}{\columnwidth}{>{\lcol}X c c c}
            \tb{Terrain}   & \tb{Highway} & \tb{Road or Trail} & \tb{Trackless} \tableheaderrule
            Desert, sandy  & \mult1       & \mult1/2           & \mult1/2 \\
            Forest         & \mult1       & \mult1             & \mult1/2 \\
            Hills          & \mult1       & \mult3/4           & \mult1/2 \\
            Jungle         & \mult1       & \mult3/4           & \mult1/4 \\
            Moor           & \mult1       & \mult1             & \mult3/4 \\
            Mountains      & \mult3/4     & \mult3/4           & \mult1/2 \\
            Plains         & \mult1-1/2   & \mult1             & \mult3/4 \\
            Swamp          & \mult1       & \mult3/4           & \mult1/2 \\
            Tundra, frozen & \mult1       & \mult3/4           & \mult3/4
        \end{dtabularx}
    \end{dtable}

\section{Uncommon Combat Circumstances}

    Even if the attack had no obvious physical or visual effects, a creature that resists an attack still feels a hostile force or a tingle, but cannot usually deduce the exact nature of the attack.

    Creatures can voluntarily lower their defenses against attacks that they are aware of.
    When they do, their defense is treated as 0 against the attack.

    \parhead{Dealing Damage, Taking Damage, and Losing Hit Points}
    Some abilities trigger when a creature deals damage or is dealt damage.
    Other abilities trigger when a creature loses hit points or causes another creature to lose hit points.
    An attack deals damage even if all damage dealt by the attack is applied to \glossterm{damage resistance} instead of \glossterm{hit points}.

    \subsection{Object Statistics}
        An object's size primarily influences the number of \glossterm{hit points} it has.
        The primary material it is constructed from determines its \glossterm{damage resistance}, and can modify the number of hit points it has.
        Details are given in \tref{Object Statistics By Size} and \tref{Object Statistics By Material}.

        These rules are more detailed than you should really need.
        During a typical game session, it's often best to just guess whether a character could plausibly sunder or smash an object rather than consulting these tables.

        \begin{dtable}
            \lcaption{Object Statistics By Size}
            \begin{dtabularx}{\textwidth}{l X X}
                \tb{Size}  & \tb{Hit Points} & \tb{Sunder Difficulty Value} \tableheaderrule
                Fine       & 1               & 1\fn{1} \\
                Diminutive & 2               & 2       \\
                Tiny       & 5               & 5       \\
                Small      & 10              & 10      \\
                Medium     & 20              & 15      \\
                Large      & 50              & 20      \\
                Huge       & 100             & 25      \\
                Gargantuan & 200             & 30      \\
                Colossal   & 500             & 35      \\
            \end{dtabularx}
            1. Extremely small objects may be difficult to grip effectively, which can significantly increase the difficulty to sunder them.
        \end{dtable}

        \begin{dtable}
            \lcaption{Object Statistics By Material}
            \begin{dtabularx}{\textwidth}{l X X X}
                \tb{Material}   & \tb{DR}\fn{1} & \tb{Hit Points Multiplier}\fn{2} & \tb{Sunder Difficulty Value Modifier}  \tableheaderrule
                Adamantine      & 30                    & \mult3                & \plus20              \\
                Glass           & 5                     & \mult1/2              & \tdash               \\
                Ice             & 1                     & \mult1/2              & \minus5              \\
                Iron or steel   & 12                    & \mult2                & \plus10              \\
                Leather or hide & 3                     & \tdash                & \tdash               \\
                Mithral         & 15                    & \mult2                & \plus10              \\
                Paper or cloth  & 1                     & \mult1/2              & \minus5              \\
                Rope            & 2                     & \tdash                & \tdash               \\
                Stone           & 8                     & \mult2                & \plus5               \\
                Wood            & 5                     & \tdash                & \tdash               \\
            \end{dtabularx}
            1. See \pcref{Damage Resistance}. \\
            2. Any value here modifies the number of hit points the object would normally have based on its size.
        \end{dtable}
