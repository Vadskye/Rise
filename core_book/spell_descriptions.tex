\section{Spell Descriptions}
\begin{spellsection}{Aeromancy}
\begin{spellheader}
\spelldesc{You shield your ally with a barrier of wind, protecting them from harm.}
\end{spellheader}
\begin{spellcontent}
\begin{spelltargetinginfo}
\spelltwocol{\spelltgt{One willing creature (Medium or smaller)}}{\spellrng{\rngclose}}
\spelltime{minor action}
\end{spelltargetinginfo}
\begin{spelleffects}
\spelleffect
The target gains a \plus1 bonus to \glossterm{physical defenses}.
This bonus is increased to \plus5 against ranged \glossterm{physical attacks} from weapons or projectiles that are Small or smaller.
Any effect which increases the size of creature this spell can affect also increases the size of ranged weapon it defends against by the same amount.
\spelldur Attunement (shared)
\spelltags{\glossterm{Air}, \glossterm{Imbuement}}
\end{spelleffects}
\end{spellcontent}
\begin{spellfooter}
\spellinfo{Transmutation}{Air, Nature}
\end{spellfooter}
\begin{spellsubcontent}
\begin{spellcantrip}
The spell's casting time becomes a standard action, and its duration becomes Sustain (minor, shared).
\end{spellcantrip}
\end{spellsubcontent}
\end{spellsection}
\subsubsection{Subspells}
\augment{2}{Gentle Descent}
The target gains a 30 foot glide speed.
A creature with a glide speed can glide through the air at the indicated speed (see \pcref{Gliding}).
\augment{2}{Windstrike}
Replace the spell's targets and effects with the following:
\begin{spellcontent}
\begin{augmenttargetinginfo}
\spelltwocol{\spelltgt{One creature or object}}{\spellrng{\rngmed}}
\spelltime{standard action}
\end{augmenttargetinginfo}
\begin{augmenteffects}
\begin{spellattack}{Spellpower vs. Fortitude}
\spellsuccess Bludgeoning \glossterm{standard damage} \plus1d.
\spellcritical As above, but double damage.
\end{spellattack}
\spelltags{\glossterm{Air}}
\end{augmenteffects}
\end{spellcontent}
\augment{3}{Gust of Wind}
Replace the spell's targets and effects with the following:
\begin{spellcontent}
\begin{augmenttargetinginfo}
\spellburst{\arealarge line, 10 ft. wide}
\spelltgts{Everything in the area}
\end{augmenttargetinginfo}
\begin{augmenteffects}
\begin{spellattack}{Spellpower vs. Fortitude}
\spellsuccess Bludgeoning \glossterm{standard damage} \minus1d.
\spellcritical As above, but double damage.
\end{spellattack}
\spelltags{\glossterm{Air}}
\end{augmenteffects}
\end{spellcontent}
\augment{3}{Windblade}
Replace the spell's targets and effects with the following:
\begin{spellcontent}
\begin{augmenttargetinginfo}
\spelltwocol{\spelltgt{One unattended weapon}}{\spellrng{\rngclose}}
\spelltime{standard action}
\end{augmenttargetinginfo}
\begin{augmenteffects}
\spelleffect
The target weapon gains an additional ten feet of reach, extending the wielder's threatened area.
This has no effect on ranged attacks with the weapon.
\spelldur Sustain (minor)
\spelltags{\glossterm{Air}, \glossterm{Shaping}}
\end{augmenteffects}
\end{spellcontent}
\augment{4}{Air Walk}
The target can walk on air as if it were solid ground.
The magic only affects the target's legs and feet.
By choosing when to treat the air as solid, it can traverse the air with ease.
\augment{4}{Wind Screen}
The defense bonus against ranged attacks increases to \plus10.
\augment{5}{Stormlord}
Whenever a creature within \rngclose range of the target attacks it, wind strikes the attacking creature.
The wind deals \\glossterm{standard damage} \minus1d.
Any individual creature can only be dealt damage in this way once per round.
\par Any effect which increases this spell's range increases the range of this effect by the same amount.
\par
This is a \glossterm{Shielding} effect from the Evocation school.
\augment{6}{Innate}
You can cast the spell without spending an action point.
\augment{7}{Control Weather}
Replace the spell's targets and effects with the following:
\begin{spellcontent}
\begin{augmenttargetinginfo}
\spellzone{2 mile radius cylinder from your location}
\end{augmenttargetinginfo}
\begin{augmenteffects}
\spelleffect
When you cast this spell, you choose a new weather pattern.
You can only choose weather which would be possible in the climate and season of the area you are in.
For example, you can normally create a thunderstorm, but not if you are in a desert.
The weather begins to take effect in the area when you complete the spell.
After five minutes, your chosen weather pattern fully takes effect.
You can control the general tendencies of the weather, such as the direction and intensity of the wind.
You cannot control specific applications of the weather -- where lightning strikes, for example, or the exact path of a tornado.
Contradictory weather conditions are not possible simultaneously.
After the spell's duration ends, the weather continues on its natural course, which may cause your chosen weather pattern to end.
% TODO: This should be redundant with generic spell mechanics
If another ability would magically manipulate the weather in the same area, the most recently used ability takes precedence.
\spelldur Attunement
\spelltags{\glossterm{Air}}
\end{augmenteffects}
\end{spellcontent}
\begin{spellsection}{Antimagic}
\begin{spellcontent}
\begin{spelltargetinginfo}
\spelltwocol{\spelltgt{One creature, object, or active magical effect}}{\spellrng{\rngmed}}
\end{spelltargetinginfo}
\begin{spelleffects}
\begin{spellattack}{Spellpower vs. Special}
\spellspecial
The attack result is applied to every \glossterm{magical} effect on the target.
The DR for each effect is equal to 10 + the \glossterm{power} of that effect.
\spellsuccess
Success against a magical effect causes that effect to be \glossterm{suppressed}.
\end{spellattack}
\spelldur Sustain (minor)
\spelltags{\glossterm{Thaumaturgy}}
\end{spelleffects}
\end{spellcontent}
\begin{spellfooter}
\spellinfo{Abjuration}{Arcane, Divine, Magic, Nature}
\end{spellfooter}
\begin{spellsubcontent}
\begin{spellcantrip}
The spell's duration becomes Sustain (standard).
\end{spellcantrip}
\end{spellsubcontent}
\end{spellsection}
\subsubsection{Subspells}
\augment{2}{Alter Magic Aura}
Replace the spell's targets and effects with the following:
\begin{spellcontent}
\begin{augmenttargetinginfo}
\spelltwocol{\spelltgt{One magical object (Large or smaller)}}{\spellrng{\rngmed}}
\end{augmenttargetinginfo}
\begin{augmenteffects}
\begin{spellattack}{Spellpower vs. Mental}
\spellsuccess
One of the target's magic auras is altered (see \pcref{Spellcraft}).
You can change the school and descriptors of the aura.
In addition, you can decrease the spellpower of the aura by up to half your spellpower, or increase the spellpower of the aura up to a maximum of your spellpower.
\end{spellattack}
\spelldur Attunement
\spelltags{\glossterm{Thaumaturgy}}
\end{augmenteffects}
\end{spellcontent}
\augment{2}{Dimensional Anchor}
Replace the spell's effects with the following:
\begin{spellcontent}
\begin{augmenteffects}
\begin{spellattack}{Spellpower vs. Mental}
\spellsuccess
The target cannot travel extradimensionally.
This prevents all \glossterm{Manifestation}, \glossterm{Planar}, and \glossterm{Teleportation} effects.
\end{spellattack}
\spelldur Condition
\spelltags{\glossterm{Thaumaturgy}}
\end{augmenteffects}
\end{spellcontent}
\augment{2}{Suppress Item}
Replace the spell's targets and effects with the following:
\begin{spellcontent}
\begin{augmenttargetinginfo}
\spelltwocol{\spelltgt{One object}}{\spellrng{\rngmed}}
\end{augmenttargetinginfo}
\begin{augmenteffects}
\begin{spellattack}{Spellpower vs. Special}
\spellspecial
The DR is equal to 10 + the target's spellpower.
\spellsuccess
The target object is \glossterm{suppressed}.
\end{spellattack}
\spelldur Sustain (minor)
\spelltags{\glossterm{Thaumaturgy}}
\end{augmenteffects}
\end{spellcontent}
\augment{3}{Banishing}
Replace the spell's effects with the following:
\begin{spellcontent}
\begin{augmenteffects}
\begin{spellattack}{Spellpower vs. Special}
\spellspecial
If the target is an effect of an ongoing \glossterm{magical} ability, such as a summoned monster, the DR is equal to 10 + the target's spellpower.
Otherwise, this ability has no effect.
\spellsuccess
The target is treated as if the spell that created it was \glossterm{dismissed}.
This usually causes the target to disappear.
\end{spellattack}
\spelltags{\glossterm{Thaumaturgy}}
\end{augmenteffects}
\end{spellcontent}
\augment{5}{Dimensional Lock}
Replace the spell's targets and effects with the following:
\begin{spellcontent}
\begin{augmenttargetinginfo}
\spelltwocol{\spellburst{\arealarge radius}}{\spellrng{\rngmed}}
\spelltgts{Everything in the area}
\end{augmenttargetinginfo}
\begin{augmenteffects}
\spelleffect
Extradimensional travel into or out of the spell's area is impossible.
This prevents all \glossterm{Manifestation}, \glossterm{Planar}, and \glossterm{Teleportation} effects.
\spelldur Attunement
\spelltags{\glossterm{Thaumaturgy}}
\end{augmenteffects}
\end{spellcontent}
\augment{6}{Innate}
You can cast the spell without spending an action point.
\augment{7}{Antimagic Field}
Replace the spell's targets and effects with the following:
\begin{spellcontent}
\begin{augmenttargetinginfo}
\spellspecial This emanation always includes you in its area
\spellemanation{\areasmall radius centered on you}
\end{augmenttargetinginfo}
\begin{augmenteffects}
\spelleffect
All magical abilities and objects are \glossterm{suppressed} in the area.
In addition, magical abilities and objects cannot be activated within the area.
\par Creatures within the area cannot concentrate on or dismiss spells. However, you can concentrate on and dismiss your own \spell{antimagic field}.
\spelldur Sustain (minor)
\spelltags{\glossterm{Thaumaturgy}}
\end{augmenteffects}
\end{spellcontent}
\begin{spellsection}{Aquamancy}
\begin{spellheader}
\spelldesc{You create a wave of water to crush your foes.}
\end{spellheader}
\begin{spellcontent}
\begin{spelltargetinginfo}
\spellburst{\arealarge line, 10 ft\. wide}
\spelltgts{Everything in the area}
\end{spelltargetinginfo}
\begin{spelleffects}
\begin{spellattack}{Spellpower vs. Fortitude}
\spellsuccess Bludgeoning \glossterm{standard damage} \minus1d.
\spellcritical As above, but double damage.
\end{spellattack}
\spelltags{\glossterm{Manifestation}, \glossterm{Water}}
\end{spelleffects}
\end{spellcontent}
\begin{spellfooter}
\spellinfo{Conjuration}{Nature, Water}
\end{spellfooter}
\begin{spellsubcontent}
\begin{spellcantrip}
The spell's area becomes a 5 ft.\ wide, \areamed line.
\end{spellcantrip}
\end{spellsubcontent}
\end{spellsection}
\subsubsection{Subspells}
\augment{2}{Aqueuous Sphere}
Replace the spell's targets with the following:
\begin{spellcontent}
\begin{augmenttargetinginfo}
\spelltwocol{\spellburst{\areasmall radius}}{\spellrng{\rngclose}}
\spelltgts{Everything in the area}
\end{augmenttargetinginfo}
\end{spellcontent}
\augment{2}{Create Water}
Replace the spell's targets and effects with the following:
\begin{spellcontent}
\begin{augmenttargetinginfo}
\spellrng{\rngclose}
\end{augmenttargetinginfo}
\begin{augmenteffects}
\spelleffect
You create up to one gallon of wholesome, drinkable water.
The water can be created at multiple locations within the ritual's range, allowing you to fill multiple small water containers.
\spelltags{\glossterm{Manifestation}, \glossterm{Water}}
\end{augmenteffects}
\end{spellcontent}
\augment{3}{Aqueous Blade}
Replace the spell's targets and effects with the following:
\begin{spellcontent}
\begin{augmenttargetinginfo}
\spelltwocol{\spelltgt{One unattended weapon}}{\spellrng{\rngclose}}
\end{augmenttargetinginfo}
\begin{augmenteffects}
\spelleffect
\glossterm{Strikes} with the affected weapon are made against Reflex defense instead of Armor defense.
However, attacks with the weapon take a \minus2d penalty to \glossterm{strike damage}.
\spelldur Sustain (minor)
\spelltags{\glossterm{Shaping}, \glossterm{Water}}
\end{augmenteffects}
\end{spellcontent}
\augment{4}{Sustained}
The area affected by this spell becomes completely filled with water.
You can sustain the water as a \glossterm{minor action}.
Creatures in this \glossterm{zone} suffer penalties appropriate for fighting underwater, and may be unable to breathe.
\augment{6}{Innate}
You can cast the spell without spending an action point.
\augment{8}{Greater Aqueous Blade}
Replace the spell's targets and effects with the following:
\begin{spellcontent}
\begin{augmenttargetinginfo}
\spelltwocol{\spelltgt{One unattended weapon}}{\spellrng{\rngclose}}
\end{augmenttargetinginfo}
\begin{augmenteffects}
\spelleffect
\glossterm{Strikes} with the affected weapon are made against Reflex defense instead of Armor defense.
\spelldur Sustain (minor)
\spelltags{\glossterm{Shaping}, \glossterm{Water}}
\end{augmenteffects}
\end{spellcontent}
\begin{spellsection}{Astromancy}
\begin{spellheader}
\spelldesc{You disrupt a creature's body by partially thrusting it into another plane.}
\end{spellheader}
\begin{spellcontent}
\begin{spelltargetinginfo}
\spelltwocol{\spelltgt{One creature}}{\spellrng{\rngmed}}
\end{spelltargetinginfo}
\begin{spelleffects}
\begin{spellattack}{Spellpower vs. Mental}
\spellsuccess Physical \glossterm{standard damage} \plus1d.
\spellcritical
As above, but double damage.
In addition, if the creature is an \glossterm{outsider} native to another plane, it is sent back to its home plane.
\end{spellattack}
\spelltags{\glossterm{Planar}, \glossterm{Teleportation}}
\end{spelleffects}
\end{spellcontent}
\begin{spellfooter}
\spellinfo{Conjuration}{Arcane}
\end{spellfooter}
\begin{spellsubcontent}
\begin{spellcantrip}
The spell deals \minus2d damage.
\end{spellcantrip}
\end{spellsubcontent}
\end{spellsection}
\subsubsection{Subspells}
\augment{2}{Dimensional Jaunt -- Plane of Fire}
The spell's damage becomes fire damage.
In addition, if the attack hits, the target is \glossterm{ignited} as a \glossterm{condition}.
\augment{2}{Teleport}
Replace the spell's targets and effects with the following:
\begin{spellcontent}
\begin{augmenttargetinginfo}
\spelltwocol{\spelltgt{One willing creature (Medium or smaller)}}{\spellrng{\rngclose}}
\end{augmenttargetinginfo}
\begin{augmenteffects}
\spelleffect
The target teleports into an unoccupied destination within \rngmed range of its original location.
If the destination is invalid, the spell fails.
\spelltags{\glossterm{Teleportation}}
\end{augmenteffects}
\end{spellcontent}
\augment{4}{Dimension Door}
You teleport to a location within \rngext range of you.
You must clearly visualize the destination's appearance, but you do not need \glossterm{line of sight} or \glossterm{line of effect} to your destination.
Replace the spell's targets with the following:
\begin{spellcontent}
\begin{augmenttargetinginfo}
\spelltgt{You}
\end{augmenttargetinginfo}
\end{spellcontent}
\augment{4}{Dimensional Jaunt -- Plane of Earth}
The spell's damage becomes bludgeoning damage.
In addition, if the attack hits, the target is \glossterm{immobilized} as a \glossterm{condition}.
\augment{6}{Innate}
You can cast the spell without spending an action point.
\augment{8}{Dimensional Jaunt -- Deep Astral Plane}
If the attack hits, the target is \glossterm{stunned} as a \glossterm{condition}.
\begin{spellsection}{Barrier}
\begin{spellheader}
\spelldesc{You create a barrier around your ally that resists physical intrusion.}
\end{spellheader}
\begin{spellcontent}
\begin{spelltargetinginfo}
\spelltwocol{\spelltgt{One creature}}{\spellrng{\rngclose}}
\spelltime{minor action}
\end{spelltargetinginfo}
\begin{spelleffects}
\spelleffect
The target gains \glossterm{damage reduction} equal to your spellpower.
In addition, it is \glossterm{vulnerable} to arcane damage.
\spelldur Attunement (shared)
\spelltags{\glossterm{Shielding}}
\end{spelleffects}
\end{spellcontent}
\begin{spellfooter}
\spellinfo{Abjuration}{Arcane}
\end{spellfooter}
\begin{spellsubcontent}
\begin{spellcantrip}
The spell's casting time becomes a standard action, and its duration becomes Sustain (minor, shared).
\end{spellcantrip}
\end{spellsubcontent}
\end{spellsection}
\subsubsection{Subspells}
\augment{2}{Repulsion}
Replace the spell's targets and effects with the following:
\begin{spellcontent}
\begin{augmenttargetinginfo}
\spellzone{\areamed radius centered on your location}
\end{augmenttargetinginfo}
\begin{augmenteffects}
\spelleffect
Whenever a creature makes physical contact with the spell's area for the first time, you make a Spellpower vs. Mental attack against it.
Success means the creature is unable to enter the spell's area with any part of its body.
The rest of its movement in the current phase is cancelled.
Failure means the creature can enter the area unimpeded.
Creatures in the area at the time that the spell is cast are unaffected by the spell.
\spelldur Sustain (minor)
\end{augmenteffects}
\end{spellcontent}
\augment{4}{Immunity}
Replace the spell's effects with the following:
\begin{spellcontent}
\begin{augmenteffects}
\spelleffect
Choose a type of damage.
The target becomes immune to damage of the chosen type.
Attacks that deal damage of multiple types still inflict damage normally unless the target is immune to all types of damage dealt.
\end{augmenteffects}
\end{spellcontent}
\augment{4}{Retributive}
Damage resisted by this spell is reflected back to the attacker as life damage.
If the attacker is beyond \rngclose range of the target, this reflection fails.
\par Any effect which increases this spell's range increases the range of this effect by the same amount.
\par
This is a \glossterm{Life} effect from the Vivimancy school.
\augment{5}{Empowered}
The damage reduction increases by an amount equal to your spellpower.
\augment{6}{Innate}
You can cast the spell without spending an action point.
\augment{7}{Antilife Shell}
Replace the spell's targets and effects with the following:
\begin{spellcontent}
\begin{augmenttargetinginfo}
\spellzone{\areamed radius centered on your location}
\end{augmenttargetinginfo}
\begin{augmenteffects}
\spelleffect
Living creatures are unable to enter the spell's area.
Creatures in the area at the time that the spell is cast are unaffected by the spell.
\spelldur Sustain (minor)
\end{augmenteffects}
\end{spellcontent}
\begin{spellsection}{Bless}
\begin{spellheader}
\spelldesc{You invoke a divine blessing to aid your ally.}
\end{spellheader}
\begin{spellcontent}
\begin{spelltargetinginfo}
\spelltwocol{\spelltgt{One creature}}{\spellrng{\rngclose}}
\spelltime{minor action}
\end{spelltargetinginfo}
\begin{spelleffects}
\spelleffect The target gains \glossterm{temporary hit points} equal to twice your spellpower.
\spelldur Attunement (shared)
\end{spelleffects}
\end{spellcontent}
\begin{spellfooter}
\spellinfo{Channeling}{Divine}
\end{spellfooter}
\begin{spellsubcontent}
\begin{spellcantrip}
The spell's casting time becomes a standard action, and its duration becomes Sustain (minor, shared).
\end{spellcantrip}
\end{spellsubcontent}
\end{spellsection}
\subsubsection{Subspells}
\augment{4}{Battle Blessing}
The target gains a \plus1d bonus to damage with all attacks that deal damage measured in dice.
\augment{6}{Innate}
You can cast the spell without spending an action point.
\augment{6}{Protection}
The target gains \glossterm{damage reduction} equal to your spellpower against all damage.
\begin{spellsection}{Chronomancy}
\begin{spellheader}
\spelldesc{You slow a foe's passage through time, inhibiting its actions.}
\end{spellheader}
\begin{spellcontent}
\begin{spelltargetinginfo}
\spelltwocol{\spelltgt{One creature}}{\spellrng{\rngmed}}
\end{spelltargetinginfo}
\begin{spelleffects}
\begin{spellattack}{Spellpower vs. Mental}
\spellsuccess
The target is \glossterm{slowed} and \glossterm{dazed}.
\spellcritical
the target is \glossterm{immobilized} and \glossterm{dazed}.
\end{spellattack}
\spelldur Condition
\spelltags{\glossterm{Temporal}}
\end{spelleffects}
\end{spellcontent}
\begin{spellfooter}
\spellinfo{Transmutation}{Arcane}
\end{spellfooter}
\begin{spellsubcontent}
\begin{spellcantrip}
You take a \minus2 penalty to accuracy with the spell.
\end{spellcantrip}
\end{spellsubcontent}
\end{spellsection}
\subsubsection{Subspells}
\augment{2}{Haste}
Replace the spell's targets and effects with the following:
\begin{spellcontent}
\begin{augmenttargetinginfo}
\spelltwocol{\spelltgt{One willing creature}}{\spellrng{\rngmed}}
\spelltime{minor action}
\end{augmenttargetinginfo}
\begin{augmenteffects}
\spelleffect
The target gains a \plus30 foot bonus to its speed in all its movement modes, up to a maximum of double its original speed.
\spelldur Attunement
\spelltags{\glossterm{Temporal}}
\end{augmenteffects}
\end{spellcontent}
\augment{4}{Delay Damage}
Replace the spell's targets and effects with the following:
\begin{spellcontent}
\begin{augmenttargetinginfo}
\spelltgt{You}
\end{augmenttargetinginfo}
\begin{augmenteffects}
\spelleffect
Whenever you take damage, half of the damage (rounded down) is not dealt to you immediately.
This damage is tracked separately.
When the ends, you take all of the delayed damage at once.
When this happens, any damage in excess of your hit points is dealt as \glossterm{vital damage}.
\spelldur Sustain (minor)
\spelltags{\glossterm{Temporal}}
\end{augmenteffects}
\end{spellcontent}
\augment{6}{Innate}
You can cast the spell without spending an action point.
\augment{9}{Time Stop}
Replace the spell's targets and effects with the following:
\begin{spellcontent}
\begin{augmenttargetinginfo}
\spelltgt{You}
\end{augmenttargetinginfo}
\begin{augmenteffects}
\spelleffect
You can take two full rounds of actions immediately.
During this time, all other creatures and objects are fixed in time, and cannot be moved or altered by any effect.
You can still affect yourself and create areas or new effects.
You are still vulnerable to danger, such as from heat or dangerous gases. However, you cannot be detected by any means while you travel.
\spelltags{\glossterm{Temporal}}
\end{augmenteffects}
\end{spellcontent}
\begin{spellsection}{Compel}
\begin{spellheader}
\spelldesc{Compel a foe to freeze in place.}
\end{spellheader}
\begin{spellcontent}
\begin{spelltargetinginfo}
\spelltwocol{\spelltgt{One creature}}{\spellrng{\rngmed}}
\end{spelltargetinginfo}
\begin{spelleffects}
\begin{spellattack}{Spellpower vs. Mental}
\spellsuccess The target is \immobilized.
\spellcritical
The target is \immobilized twice by two separate conditions.
Each condition must be removed independently.
\end{spellattack}
\spelldur Condition
\spelltags{\glossterm{Compulsion}, \glossterm{Mind}}
\end{spelleffects}
\end{spellcontent}
\begin{spellfooter}
\spellinfo{Enchantment}{Arcane, Divine}
\end{spellfooter}
\begin{spellsubcontent}
\begin{spellcantrip}
You take a \minus2 penalty to accuracy with the spell.
\end{spellcantrip}
\end{spellsubcontent}
\end{spellsection}
\subsubsection{Subspells}
\augment{2}{Confusion}
Replace the spell's effects with the following:
\begin{spellcontent}
\begin{augmenteffects}
\begin{spellattack}{Spellpower vs. Mental}
\spellsuccess The target is \disoriented.
\spellcritical The target is \confused.
\end{spellattack}
\spelldur Condition
\spelltags{\glossterm{Compulsion}, \glossterm{Mind}}
\end{augmenteffects}
\end{spellcontent}
\augment{3}{Dance}
Replace the spell's effects with the following:
\begin{spellcontent}
\begin{augmenteffects}
\begin{spellattack}{Spellpower vs. Mental}
\spellsuccess
The target is compelled to dance.
It can spend a \glossterm{minor action} or standard action to dance, if it is physically capable of dancing.
Whenever it takes a minor action or standard action in a round where it has not danced, it takes mental \glossterm{standard damage} \plus1d.
Regardless of whether the target dances, it takes a \minus2 penalty to \glossterm{physical defenses} due to its limited control over its limbs.
\spellcritical
As above, except that dancing as a minor action does not prevent the target from taking damage.
Only dancing as a standard action can prevent the target from taking damage.
\end{spellattack}
\spelldur Condition
\spelltags{\glossterm{Compulsion}, \glossterm{Mind}}
\end{augmenteffects}
\end{spellcontent}
\augment{4}{Sleep}
Replace the spell's effects with the following:
\begin{spellcontent}
\begin{augmenteffects}
\begin{spellattack}{Spellpower vs. Mental}
\spellspecial This ability's effect can be removed by effects which remove \glossterm{conditions}.
\spellsuccess The target is \blinded.
\spellcritical
The target falls asleep.
It cannot be awakened by any means while the spell lasts.
After that time, it can wake up normally, though it continues to sleep until it would wake up naturally.
\end{spellattack}
\spelldur Sustain (minor)
\spelltags{\glossterm{Compulsion}, \glossterm{Mind}}
\end{augmenteffects}
\end{spellcontent}
\augment{5}{Dominate}
Replace the spell's effects with the following:
\begin{spellcontent}
\begin{augmenteffects}
\begin{spellattack}{Spellpower vs. Mental}
\spellsuccess The target is \glossterm{confused}.
\spellcritical
The target is \glossterm{dominated} by you.
If the target was already dominated by you, this ability's duration becomes Attunement instead of Sustain (minor).
\end{spellattack}
\spelldur Sustain (minor)
\spelltags{\glossterm{Compulsion}, \glossterm{Mind}}
\end{augmenteffects}
\end{spellcontent}
\augment{6}{Innate}
You can cast the spell without spending an action point.
\augment{9}{Irresistible Dance}
This subspell functions like the \textit{dance} subspell, except that you gain a \plus4 bonus to accuracy on the attack.
\begin{spellsection}{Corruption}
\begin{spellheader}
\spelldesc{You corrupt your foe's life force, weakening it.}
\end{spellheader}
\begin{spellcontent}
\begin{spelltargetinginfo}
\spelltwocol{\spelltgt{One living creature}}{\spellrng{\rngclose}}
\end{spelltargetinginfo}
\begin{spelleffects}
\begin{spellattack}{Spellpower vs. Fortitude}
\spellsuccess
The target is \glossterm{sickened}.
In addition, it takes life \glossterm{standard damage} \minus3d whenever it takes a \glossterm{standard action}.
\spellcritical
The target is \glossterm{nauseated}.
In addition, it takes life \glossterm{standard damage} whenever it takes a \glossterm{standard action}.
\end{spellattack}
\spelldur Condition
\spelltags{\glossterm{Life}}
\end{spelleffects}
\end{spellcontent}
\begin{spellfooter}
\spellinfo{Vivimancy}{Arcane, Divine, Nature}
\end{spellfooter}
\begin{spellsubcontent}
\begin{spellcantrip}
You take a \minus2 penalty to accuracy with the spell.
\end{spellcantrip}
\end{spellsubcontent}
\end{spellsection}
\subsubsection{Subspells}
\augment{3}{Corruption of Blood and Bone}
Whenever the target takes damage from this spell, its maximum hit points are reduced by the same amount.
When the spell ends, the target's maximum hit points are restored.
\augment{3}{Eyebite}
If the spell's attack hits, the target is also \dazzled. If it critically hits, the target is \blinded instead of dazzled.
\augment{4}{Finger of Death}
If the spell's attack critically hits, the target immediately dies.
\par
This is a \glossterm{Death} effect.
\augment{5}{Cripple}
If the attack hits, the target is also \glossterm{immobilized} as part of the same condition.
On a \glossterm{critical hit}, the target is \glossterm{paralyzed} instead of immobilized.
\augment{6}{Corrupting Curse}
The spell's attack is made against Mental defense instead of Fortitude defense.
In addition, if it critically hits, the spell's effect becomes a permanent curse.
It is no longer a condition, and cannot be removed by abilities that remove conditions.
This is a \glossterm{Curse} effect.
\augment{6}{Innate}
You can cast the spell without spending an action point.
\begin{spellsection}{Cryomancy}
\begin{spellheader}
\spelldesc{You drain the heat from an area, creating a field of extreme cold.}
\end{spellheader}
\begin{spellcontent}
\begin{spelltargetinginfo}
\spellburst{\areamed cone}
\spelltgts{Everything in the area}
\end{spelltargetinginfo}
\begin{spelleffects}
\begin{spellattack}{Spellpower vs. Fortitude}
\spellsuccess
Cold \glossterm{standard damage} \minus1d.
In addition, the target is \fatigued as a condition.
\spellcritical As above, but double damage.
\end{spellattack}
\spelltags{\glossterm{Cold}}
\end{spelleffects}
\end{spellcontent}
\begin{spellfooter}
\spellinfo{Evocation}{Arcane, Nature}
\end{spellfooter}
\begin{spellsubcontent}
\begin{spellcantrip}
The spell deals no damage.
\end{spellcantrip}
\end{spellsubcontent}
\end{spellsection}
\subsubsection{Subspells}
\augment{2}{Slick}
The spell's area is covered with a film of slick ice.
Creatures moving across the area must make Acrobatics checks to balance (see \pcref{Balance}).
This ice lasts as long as you \glossterm{sustain} it as a \glossterm{minor action}.
\augment{4}{Freezing}
If the attack hits against a target, it is also \glossterm{immobilized} as a \glossterm{condition}.
\augment{6}{Innate}
You can cast the spell without spending an action point.
\begin{spellsection}{Distort Image}
\begin{spellcontent}
\begin{spelltargetinginfo}
\spelltwocol{\spelltgt{One willing creature}}{\spellrng{\rngmed}}
\spelltime{minor action}
\end{spelltargetinginfo}
\begin{spelleffects}
\spelleffect
The target's physical outline is distorted so it appears blurred, shifting, and wavering.
It gains a \plus1 bonus to \glossterm{physical defenses} and Stealth (see \pcref{Stealth}).
This bonus is increases to \plus2 while in \glossterm{shadowy illumination}.
This effect provides no defensive benefit against creatures immune to \glossterm{Visual} abilities.
\spelldur Attunement (shared)
\spelltags{\glossterm{Glamer}, \glossterm{Visual}}
\end{spelleffects}
\end{spellcontent}
\begin{spellfooter}
\spellinfo{Illusion}{Arcane}
\end{spellfooter}
\begin{spellsubcontent}
\begin{spellcantrip}
The spell's casting time becomes a standard action, and its duration becomes Sustain (minor, shared).
\end{spellcantrip}
\end{spellsubcontent}
\end{spellsection}
\subsubsection{Subspells}
\augment{2}{Distort Light}
Replace the spell's targets and effects with the following:
\begin{spellcontent}
\begin{augmenttargetinginfo}
\spelltwocol{\spellemanation{\areamed radius from the target}}{\spellrng{\rngclose}}
\spelltgt{One object (Small or smaller)}
\spelltime{standard action}
\end{augmenttargetinginfo}
\begin{augmenteffects}
\spelleffect
Light within or passing through the area is dimmed to be no brighter than shadowy illumination.
Any effect or object which blocks light also blocks this spell's emanation.
\spelldur Attunement (multiple)
\spelltags{\glossterm{Glamer}, \glossterm{Light}}
\end{augmenteffects}
\end{spellcontent}
\augment{2}{Mirror Image}
Replace the spell's effects with the following:
\begin{spellcontent}
\begin{augmenteffects}
\spelleffect
Four illusory duplicates appear around the target that mirror its every move.
The duplicates shift chaotically in its space, making it difficult to identify the real creature.
All targeted attacks against the target have a 50\% miss chance.
Whenever an attack misses in this way, it affects an image, destroying it.
This ability provides no defensive benefit against creatures immune to \glossterm{Visual} abilities.
\spelldur Attunement (shared)
\spelltags{\glossterm{Figment}, \glossterm{Visual}}
\end{augmenteffects}
\end{spellcontent}
\augment{3}{Disguise Image}
Replace the spell's effects with the following:
\begin{spellcontent}
\begin{augmenteffects}
\spelleffect
You make a Disguise check to alter the target's appearance (see \pcref{Disguise Creature}).
You gain a \plus5 bonus on the check, and you can freely alter the appearance of the target's clothes and equipment, regardless of their original form.
However, this effect is unable to alter the sound, smell, texture, or temperature of the target or its clothes and equipment.
\spelldur Attunement (shared)
\spelltags{\glossterm{Glamer}, \glossterm{Visual}}
\end{augmenteffects}
\end{spellcontent}
\augment{3}{Shadow Mantle}
The spell's deceptive nature extends beyond merely altering light to affect the nature of reality itself.
The defense bonus applies to all defenses, not just physical defenses.
In addition, the spell loses the \glossterm{Visual} tag, and can protect against attacks from creatures immune to Visual abilities.
\augment{5}{Greater Mirror Image}
This subspell functions like the \textit{mirror image} subspell, except that you regain a destroyed mirror image at the end of every round, to a maximum of four images.
\augment{6}{Innate}
You can cast the spell without spending an action point.
\augment{7}{Displacement}
The target's image is futher distorted, and appears to be two to three feet from its real location.
Targeted \glossterm{physical attacks} against the target suffer a 50\% miss chance.
\begin{spellsection}{Electromancy}
\begin{spellheader}
\spelldesc{You create a bolt of electricity that fries your foes.}
\end{spellheader}
\begin{spellcontent}
\begin{spelltargetinginfo}
\spellburst{\arealarge line, 10 ft\. wide}
\spelltgts{Everything in the area}
\end{spelltargetinginfo}
\begin{spelleffects}
\begin{spellattack}{Spellpower vs. Reflex}
\spellspecial You gain a \plus2 bonus to accuracy against creatures wearing metal armor or otherwise carrying a significant amount of metal.
\spellsuccess
Electricity \glossterm{standard damage} \minus1d.
\spellcritical As above, but double damage.
\end{spellattack}
\spelltags{\glossterm{Electricity}}
\end{spelleffects}
\end{spellcontent}
\begin{spellfooter}
\spellinfo{Evocation}{Arcane, Nature}
\end{spellfooter}
\begin{spellsubcontent}
\begin{spellcantrip}
The spell's area becomes a 5 ft\. wide \areamed line.
\end{spellcantrip}
\end{spellsubcontent}
\end{spellsection}
\subsubsection{Subspells}
\augment{3}{Forked Lightning}
You create two separate areas instead of one.
The two areas can overlap, but targets in the overlapping area are only affected once.
\augment{4}{Instantaneous}
The lightning bolt created by the spell is faster, but less penetrating.
The spell's attack is made against Fortitude defense instead of Reflex defense.
\augment{5}{Shocking}
If the attack hits, the target is \glossterm{dazed} as a \glossterm{condition}.
On a \glossterm{critical hit}, the target is \glossterm{stunned} instead of dazed.
\augment{6}{Innate}
You can cast the spell without spending an action point.
\begin{spellsection}{Emotion}
\begin{spellheader}
\spelldesc{You terrify your foe.}
\end{spellheader}
\begin{spellcontent}
\begin{spelltargetinginfo}
\spelltwocol{\spelltgt{One creature}}{\spellrng{\rngmed}}
\end{spelltargetinginfo}
\begin{spelleffects}
\begin{spellattack}{Spellpower vs. Mental}
\spellsuccess The target is \frightened by you.
\spellcritical The target is \panicked by you.
\spellfailure The target is \shaken by you.
\end{spellattack}
\spelldur Condition
\spelltags{\glossterm{Delusion}, \glossterm{Mind}}
\end{spelleffects}
\end{spellcontent}
\begin{spellfooter}
\spellinfo{Enchantment}{Arcane}
\end{spellfooter}
\begin{spellsubcontent}
\begin{spellcantrip}
You take a \minus2 penalty to accuracy with the spell.
\end{spellcantrip}
\end{spellsubcontent}
\end{spellsection}
\subsubsection{Subspells}
\augment{2}{Agony}
Replace the spell's effects with the following:
\begin{spellcontent}
\begin{augmenteffects}
\begin{spellattack}{Spellpower vs. Mental}
\spellsuccess At the end of each \glossterm{delayed action phase}, if the target took damage that round, it takes mental \glossterm{standard damage} \minus2d.
\spellcritical As above, but double damage.
\end{spellattack}
\spelldur Condition
\spelltags{\glossterm{Delusion}, \glossterm{Mind}}
\end{augmenteffects}
\end{spellcontent}
\augment{2}{Redirected}
The target is afraid of a willing ally within the spell's range instead of being afraid of you.
\augment{3}{Charm}
Replace the spell's effects with the following:
\begin{spellcontent}
\begin{augmenteffects}
\begin{spellattack}{Spellpower vs. Mental}
\spellspecial If the target thinks that you or your allies are threatening it, you take a \minus5 penalty to accuracy on the attack.
\spellsuccess
The target is \charmed by you.
Any act by you or your apparent allies that threatens or damages the \spell{charmed} person breaks the effect.
\spellcritical As above, but this ability's duration becomes Attunement instead of Sustain (swift).
\end{spellattack}
\spelldur Attunement
\spelltags{\glossterm{Delusion}, \glossterm{Mind}, \glossterm{Subtle}}
\end{augmenteffects}
\end{spellcontent}
\augment{6}{Innate}
You can cast the spell without spending an action point.
\augment{7}{Amnesiac Charm}
This subspell functions like the \textit{charm} subspell, except that when the spell ends, the target forgets all events that transpired during the spell's duration.
It becomes aware of its surroundings as if waking up from a daydream.
The target is not directly aware of any magical influence on its mind, though unusually paranoid or perceptive creatures may deduce that their minds were affected.
\begin{spellsection}{Fabrication}
\begin{spellheader}
\spelldesc{You conjure acid from thin air onto a foe's flesh}
\end{spellheader}
\begin{spellcontent}
\begin{spelltargetinginfo}
\spelltwocol{\spelltgt{One creature or object}}{\spellrng{\rngmed}}
\end{spelltargetinginfo}
\begin{spelleffects}
\begin{spellattack}{Spellpower vs. Fortitude}
\spellsuccess
Acid \glossterm{standard damage} \plus1d.
\spellcritical
As above, but double damage.
In addition, the target is \glossterm{sickened} as a \glossterm{condition}.
\end{spellattack}
\spelltags{\glossterm{Acid}, \glossterm{Manifestation}}
\end{spelleffects}
\end{spellcontent}
\begin{spellfooter}
\spellinfo{Conjuration}{Arcane}
\end{spellfooter}
\begin{spellsubcontent}
\begin{spellcantrip}
The spell deals \minus2d damage.
\end{spellcantrip}
\end{spellsubcontent}
\end{spellsection}
\subsubsection{Subspells}
\augment{2}{Poison}
Replace the spell's effects with the following:
\begin{spellcontent}
\begin{augmenteffects}
\spelleffect
When this spell resolves, and at the end of each \glossterm{action phase}, you make a Spellpower vs. Fortitude attack against the target.
A hit means the target takes poison \glossterm{standard damage} \minus3d.
If this is the second hit, the target also becomes \glossterm{sickened}.
If this is the third hit, the target becomes \glossterm{nauseated} instead of sickened.
\spelldur Condition
\spelltags{\glossterm{Manifestation}, \glossterm{Poison}}
\end{augmenteffects}
\end{spellcontent}
\augment{2}{Web}
Replace the spell's targets and effects with the following:
\begin{spellcontent}
\begin{augmenttargetinginfo}
\spelltwocol{\spellzone{\areasmall radius}}{\spellrng{\rngclose}}
\spelltgts{Everything in the area}
\end{augmenttargetinginfo}
\begin{augmenteffects}
\spelleffect
The area becomes filled with webs, making it \glossterm{difficult terrain}.
Each 5-ft.\ square of webbing has hit points equal to your spellpower, and is \glossterm{vulnerable} to fire.
\begin{spellattack}{Spellpower vs. Reflex}
\spellsuccess The target is \immobilized as long as it has webbing from this spell in its space.
\end{spellattack}
\spelldur Sustain (minor)
\spelltags{\glossterm{Manifestation}}
\end{augmenteffects}
\end{spellcontent}
\augment{3}{Corrosive}
The spell deals double damage to objects.
\augment{4}{Reinforced Webbing}
This subspell functions like the \textit{web} subspell, except that each 5-ft.\ square of webbing gains additional hit points equal to your spellpower.
In addition, the webs are no longer \glossterm{vulnerable} to fire damage.
\augment{5}{Lingering}
The spell deals \minus3d damage.
However, the damage deals its damage again at the end of every round after the first.
This is a \glossterm{condition}, and lasts until removed.
\augment{5}{Meteor}
Replace the spell's targets and effects with the following:
\begin{spellcontent}
\begin{augmenttargetinginfo}
\spelltwocol{\spellburst{\areamed radius cylinder, 100 ft\. high}}{\spellrng{\rnglong}}
\spelltgts{Everything in the area}
\end{augmenttargetinginfo}
\begin{augmenteffects}
\spelleffect
A meteor appears in midair at the top of the area and slams through the area, striking everything in it before disappearing.
The meteor is a sphere with the same radius as the area.
The distance that the meteor falls does not affect the damage dealt, but it must not share space with any creature or object when it is created.
If it does, this ability fails without effect.
\begin{spellattack}{Spellpower vs. Reflex}
\spellsuccess Bludgeoning and fire \glossterm{standard damage} \minus1d.
\spellcritical As above, but double damage.
\end{spellattack}
\spelltags{\glossterm{Manifestation}}
\end{augmenteffects}
\end{spellcontent}
\augment{6}{Innate}
You can cast the spell without spending an action point.
\augment{8}{Meteor Swarm}
This subspell functions like the \textit{meteor} subspell, except that you can target up to five different areas within range with separate meteors.
The areas affected by two different meteors cannot overlap.
If one of the meteors is created in an invalid area, that meteor is not created, but the others are created and dealt their damage normally.
\begin{spellsection}{Flare}
\begin{spellcontent}
\begin{spelltargetinginfo}
\spelltwocol{\spellburst{\areasmall radius}}{\spellrng{\rngmed}}
\spelltgts{All creatures in the area}
\end{spelltargetinginfo}
\begin{spelleffects}
\spelleffect
A brilliant light appears in the area until the end of the round.
It illuminates a 100 foot radius around the area with bright light.
\begin{spellattack}{Spellpower vs. Reflex}
\spellsuccess
The target is \dazzled.
\spellcritical
The target is \blinded.
\end{spellattack}
\spelldur Condition
\spelltags{\glossterm{Figment}, \glossterm{Light}, \glossterm{Visual}}
\end{spelleffects}
\end{spellcontent}
\begin{spellfooter}
\spellinfo{Illusion}{Arcane, Divine, Nature}
\end{spellfooter}
\begin{spellsubcontent}
\begin{spellcantrip}
The spell affects a single target within range instead of creating a burst.
\end{spellcantrip}
\end{spellsubcontent}
\end{spellsection}
\subsubsection{Subspells}
\augment{2}{Dancing Lights}
Replace the spell's effects with the following:
\begin{spellcontent}
\begin{augmenteffects}
\spelleffect
Up to four glowing lights appear in the area.
The lights resemble lanterns or torches, and shed bright light in the same 20 foot radius.
However, you can freely choose the color of the lights when you cast the spell.
During each movement phase, you can move the lights up to 100 feet in any direction.
If one of the lights ever goes out of range from you, it immediately winks out.
\spelldur Sustain (minor)
\spelltags{\glossterm{Figment}, \glossterm{Light}, \glossterm{Visual}}
\end{augmenteffects}
\end{spellcontent}
\augment{3}{Faerie Fire}
Each target is surrounded with a pale glow made of hundreds of ephemeral points of lights, causing it to bright light in a 5 foot radius as a candle.
The lights impose a \minus10 penalty to Stealth checks.
In addition, they reveal the outline of the creatures if they become \glossterm{invisible}.
This allows observers to see their location, though not to see them perfectly.
\augment{3}{Illuminating}
The brilliant light persists as long as you spend a \glossterm{minor action} each round to sustain it.
The light has no additional effects on creatures in the area.
\augment{3}{Kaleidoscopic}
Replace the spell's effects with the following:
\begin{spellcontent}
\begin{augmenteffects}
\spelleffect
A brilliant, rapidly shifting rainbow of lights appear in the area until the end of the round.
They illuminate a 100 foot radius around the area with bright light.
\begin{spellattack}{Spellpower vs. Mental}
\spellsuccess
The target is \disoriented.
\spellcritical
The target is \confused.
\end{spellattack}
\spelldur Condition
\spelltags{\glossterm{Figment}, \glossterm{Light}, \glossterm{Mind}, \glossterm{Visual}}
\end{augmenteffects}
\end{spellcontent}
\augment{4}{Flashbang}
An intense sound accompanies the flash of light caused by the spell.
If the spell's attack is successful, the target is also \deafened as a condition.
This is an \glossterm{Auditory}, \glossterm{Figment} effect.
\augment{6}{Innate}
You can cast the spell without spending an action point.
\begin{spellsection}{Polymorph}
\begin{spellheader}
\spelldesc{You shape the ground into an animated spike that drives into a foe.}
\end{spellheader}
\begin{spellcontent}
\begin{spelltargetinginfo}
\spelltwocol{\spelltgt{One creature or object}}{\spellrng{\rngmed}}
\end{spelltargetinginfo}
\begin{spelleffects}
\begin{spellattack}{Spellpower vs. Reflex}
\spellsuccess Piercing \glossterm{standard damage} \plus1d.
\spellcritical As above, but double damage.
\end{spellattack}
\spelltags{\glossterm{Physical}, \glossterm{Shaping}}
\end{spelleffects}
\end{spellcontent}
\begin{spellfooter}
\spellinfo{Transmutation}{Arcane, Nature}
\end{spellfooter}
\begin{spellsubcontent}
\begin{spellcantrip}
The spell deals \minus2d damage.
\end{spellcantrip}
\end{spellsubcontent}
\end{spellsection}
\subsubsection{Subspells}
\augment{2}{Barkskin}
Replace the spell's targets and effects with the following:
\begin{spellcontent}
\begin{augmenttargetinginfo}
\spelltwocol{\spelltgt{One willing creature}}{\spellrng{\rngmed}}
\spelltime{minor action}
\end{augmenttargetinginfo}
\begin{augmenteffects}
\spelleffect
The target gains \glossterm{damage reduction} equal to your spellpower against damage dealt by \glossterm{physical attacks}.
In addition, it is \glossterm{vulnerable} to fire damage.
\spelldur Attunement (shared)
\end{augmenteffects}
\end{spellcontent}
\augment{2}{Shrink}
Replace the spell's targets and effects with the following:
\begin{spellcontent}
\begin{augmenttargetinginfo}
\spelltwocol{\spelltgt{One willing creature (Small or larger)}}{\spellrng{\rngmed}}
\spelltime{minor action}
\end{augmenttargetinginfo}
\begin{augmenteffects}
\spelleffect
You decrease the target's size by one size category.
This decreases its \glossterm{strike damage} and usually decreases its \glossterm{reach} (see \pcref{Size in Combat}).
\spelldur Attunement (shared)
\spelltags{\glossterm{Shaping}, \glossterm{Sizing}}
\end{augmenteffects}
\end{spellcontent}
\augment{3}{Alter Appearance}
Replace the spell's targets and effects with the following:
\begin{spellcontent}
\begin{augmenttargetinginfo}
\spelltwocol{\spelltgt{One willing creature (Large or smaller)}}{\spellrng{\rngmed}}
\spelltime{minor action}
\end{augmenttargetinginfo}
\begin{augmenteffects}
\spelleffect
You make a Disguise check to alter the target's appearance (see \pcref{Disguise Creature}).
You gain a \plus5 bonus on the check, and you ignore penalties for changing the target's gender, race, subtype, or age.
However, this effect is unable to alter the target's clothes or equipment in any way.
\spelldur Attunement (shared)
\spelltags{\glossterm{Shaping}}
\end{augmenteffects}
\end{spellcontent}
\augment{3}{Enlarge}
Replace the spell's targets and effects with the following:
\begin{spellcontent}
\begin{augmenttargetinginfo}
\spelltwocol{\spelltgt{One willing creature (Large or smaller)}}{\spellrng{\rngmed}}
\spelltime{minor action}
\end{augmenttargetinginfo}
\begin{augmenteffects}
\spelleffect
You increase the target's size by one size category.
This increases its \glossterm{strike damage} and usually increases its \glossterm{reach} (see \pcref{Size in Combat}).
However, the target takes a \minus1d penalty to \glossterm{strike damage}, as its muscles are not increased fully to match its new size.
\spelldur Attunement (shared)
\spelltags{\glossterm{Shaping}, \glossterm{Sizing}}
\end{augmenteffects}
\end{spellcontent}
\augment{3}{Stoneskin}
This subspell functions like the \textit{barkskin} subspell, except that the target is \glossterm{vulnerable} to damage from adamantine weapons instead of fire damage.
\augment{4}{Craft Object}
Replace the spell's targets and effects with the following:
\begin{spellcontent}
\begin{augmenttargetinginfo}
\spelltwocol{\spelltgts{One or more unattended, nonmagical objects (Large or smaller); see text}}{\spellrng{\rngclose}}
\spelltime{standard action}
\end{augmenttargetinginfo}
\begin{augmenteffects}
\spelleffect
You make a Craft check to transform the targets into a new item (or items) made of the same materials.
You require none of the tools or time expenditure that would normally be necessary.
The total size of all targets combined must be Large size or smaller.
\spelltags{\glossterm{Shaping}}
\end{augmenteffects}
\end{spellcontent}
\augment{4}{Impaling}
If the attack hits, the target is \glossterm{immobilized} as a \glossterm{condition}.
\augment{6}{Disintegrate}
Replace the spell's effects with the following:
\begin{spellcontent}
\begin{augmenteffects}
\begin{spellattack}{Spellpower vs. Fortitude}
\spellsuccess
Physical \glossterm{standard damage} \plus1d.
In addition, if the target has no hit points remaining, it dies.
Its body is completely disintegrated, leaving behind only a pinch of fine dust.
Its equipment is unaffected.
\spellcritical
As above, but double damage.
\end{spellattack}
\spelltags{\glossterm{Shaping}}
\end{augmenteffects}
\end{spellcontent}
\augment{6}{Innate}
You can cast the spell without spending an action point.
\augment{7}{Greater Enlarge}
This subspell functions like the \textit{enlarge} subspell, except that the target does not take a penalty to strike damage.
\augment{7}{Ironskin}
This subspell functions like the \textit{stoneskin} subspell, except that the damage reduction is equal to twice your spellpower.
\begin{spellsection}{Protection from Alignment}
\begin{spellcontent}
\begin{spelltargetinginfo}
\spelltwocol{\spelltgt{One creature}}{\spellrng{\rngclose}}
\spelltime{minor action}
\end{spelltargetinginfo}
\begin{spelleffects}
\spellspecial
Choose an alignment other than neutral (chaotic, good, evil, or lawful).
This spell gains the tag for that alignment's \glossterm{opposed alignment}.
\spelleffect
The target gains damage reduction equal to your spellpower against physical effects that have the chosen alignment, and physical attacks made by creatures with the chosen alignment.
\spelldur Attunement (shared)
\spelltags{\glossterm{Shielding}}
\end{spelleffects}
\end{spellcontent}
\begin{spellfooter}
\spellinfo{Abjuration}{Arcane, Chaos, Divine, Evil, Good, Law}
\end{spellfooter}
\begin{spellsubcontent}
\begin{spellcantrip}
The spell's casting time becomes a standard action, and its duration becomes Sustain (minor, shared).
\end{spellcantrip}
\end{spellsubcontent}
\end{spellsection}
\subsubsection{Subspells}
\augment{3}{Complete}
The damage reduction also applies against non-physical effects.
\augment{4}{Retributive}
Whenever a creature with the chosen alignment makes a physical melee attack against the target, you make a Spellpower vs. Mental attack against the attacking creature.
Success means the attacker takes divine \glossterm{standard damage} \minus1d.
\augment{6}{Innate}
You can cast the spell without spending an action point.
\begin{spellsection}{Pyromancy}
\begin{spellheader}
\spelldesc{You create a small burst of flame.}
\end{spellheader}
\begin{spellcontent}
\begin{spelltargetinginfo}
\spelltwocol{\spellburst{\areasmall radius}}{\spellrng{\rngclose}}
\spelltgts{Everything in the area}
\end{spelltargetinginfo}
\begin{spelleffects}
\begin{spellattack}{Spellpower vs. Reflex}
\spellsuccess Fire \glossterm{standard damage} \minus1d.
\spellcritical As above, but double damage.
\end{spellattack}
\spelltags{\glossterm{Fire}}
\end{spelleffects}
\end{spellcontent}
\begin{spellfooter}
\spellinfo{Evocation}{Arcane, Fire, Nature}
\end{spellfooter}
\begin{spellsubcontent}
\begin{spellcantrip}
The spell affects a single target within range instead of creating a burst.
\end{spellcantrip}
\end{spellsubcontent}
\end{spellsection}
\subsubsection{Subspells}
\augment{2}{Burning Hands}
Replace the spell's targets with the following:
\begin{spellcontent}
\begin{augmenttargetinginfo}
\spellburst{\arealarge cone}
\spelltgts{Everything in the area}
\end{augmenttargetinginfo}
\end{spellcontent}
\augment{2}{Flame Blade}
Replace the spell's targets and effects with the following:
\begin{spellcontent}
\begin{augmenttargetinginfo}
\spelltwocol{\spelltgt{One unattended weapon}}{\spellrng{\rngclose}}
\end{augmenttargetinginfo}
\begin{augmenteffects}
\spelleffect
The target weapon gains a \plus1d bonus to \glossterm{strike damage}.
In addition, all damage dealt with the weapon with strikes becomes fire damage in addition to its normal damage types.
\spelldur Sustain (minor)
\spelltags{\glossterm{Fire}}
\end{augmenteffects}
\end{spellcontent}
\augment{3}{Fire Trap}
Replace the spell's targets and effects with the following:
\begin{spellcontent}
\begin{augmenttargetinginfo}
\spelltwocol{\spelltgt{One openable object (Large or smaller)}}{\spellrng{\rngclose}}
\end{augmenttargetinginfo}
\begin{augmenteffects}
\spelleffect
If a creature opens the target object, it explodes.
You make an attack against everything within an \areamed radius burst centered on the target.
After the object explodes in this way, the spell ends.
\begin{spellattack}{Spellpower vs. Reflex}
\spellsuccess Fire \glossterm{standard damage} \minus1d.
\spellcritical As above, but double damage.
\end{spellattack}
\spelldur Attunement
\spelltags{\glossterm{Fire}, \glossterm{Trap}}
\end{augmenteffects}
\end{spellcontent}
\augment{6}{Innate}
You can cast the spell without spending an action point.
\begin{spellsection}{Revelation}
\begin{spellheader}
\spelldesc{You grant a creature the ability to see fractions of a second into the future.}
\end{spellheader}
\begin{spellcontent}
\begin{spelltargetinginfo}
\spelltwocol{\spelltgt{One willing creature}}{\spellrng{\rngclose}}
\spelltime{minor action}
\end{spelltargetinginfo}
\begin{spelleffects}
\spelleffect
The target gains a \plus1 bonus to \glossterm{accuracy} with all attacks.
\spelldur Attunement (shared)
\spelltags{\glossterm{Enhancement}}
\end{spelleffects}
\end{spellcontent}
\begin{spellfooter}
\spellinfo{Divination}{Arcane, Divine, Nature}
\end{spellfooter}
\begin{spellsubcontent}
\begin{spellcantrip}
The spell's casting time becomes a standard action, and its duration becomes Sustain (minor, shared).
\end{spellcantrip}
\end{spellsubcontent}
\end{spellsection}
\subsubsection{Subspells}
\augment{2}{Discern Lies}
Replace the spell's targets and effects with the following:
\begin{spellcontent}
\begin{augmenttargetinginfo}
\spellemanation{\areamed radius from you}
\spelltgts{All creatures in the area}
\end{augmenttargetinginfo}
\begin{augmenteffects}
\spelleffect
You know when the target deliberately and knowingly speaks a lie.
This ability does not reveal the truth, uncover unintentional inaccuracies, or necessarily reveal evasions.
\spelldur Attunement
\spelltags{\glossterm{D}, \glossterm{c}, \glossterm{e}, \glossterm{e}, \glossterm{i}, \glossterm{n}, \glossterm{o}, \glossterm{t}, \glossterm{t}}
\end{augmenteffects}
\end{spellcontent}
\augment{3}{Augury}
Replace the spell's targets and effects with the following:
\begin{spellcontent}
\begin{augmenttargetinginfo}
\spelltwocol{\spelltgt{One willing creature}}{\spellrng{\rngclose}}
\spelltime{standard action}
\end{augmenttargetinginfo}
\begin{augmenteffects}
\spelleffect
Choose an action that the target could take.
You learn whether the stated action is likely to bring good or bad results for it within the next hour.
This spell provides one of four results:
\begin{itemize}
\item Weal (if the action will probably bring good results).
\item Woe (for bad results).
\item Weal and woe (for both).
\item No response (for actions that don't have especially good or bad results).
\end{itemize}
This spell does not describe the future with certainty.
It describes which result is most probable.
The more unambiguous the action's effects, the more likely the spell is to be correct.
% TODO: inconsistent wording; this spell vs. this subspell
After using this subspell, you cannot cast it again until the hour affected by the previous casting is over, regardless of whether the action was taken.
\end{augmenteffects}
\end{spellcontent}
\augment{3}{Boon of Mastery}
The target also gains a \plus2 bonus to all skills.
\augment{4}{Boon of Knowledge}
The target also gains a \plus5 bonus to all Knowledge skills (see \pcref{Knowledge}).
\augment{5}{Third Eye}
The target also gains \glossterm{blindsight} out to a 100 foot range, allowing it to see perfectly without any light, regardless of concealment or invisibility.
\augment{6}{Innate}
You can cast the spell without spending an action point.
\augment{7}{Foresee Actions}
The target can learn what actions all creatures it can observe intend to take during each phase before it decides its actions for that phase.
It learns this information in the instant before it acts, and normally does not have time to communicate it to other creatures.
\augment{7}{Greater Boon of Mastery}
This subspell functions like the \textit{boon of mastery} subspell, except that the skill bonus is increased to \plus5.
\begin{spellsection}{Scry}
\begin{spellheader}
\spelldesc{You create a scrying sensor that allows you to see at a distance.}
\end{spellheader}
\begin{spellcontent}
\begin{spelltargetinginfo}
\spelltwocol{\spelltgt{One square}}{\spellrng{\rngmed}}
\end{spelltargetinginfo}
\begin{spelleffects}
\spelleffect
A Fine object appears floating in the air in the target space.
It resembles a human eye in size and shape, though it is \glossterm{invisible}.
At the start of each round, you choose whether you see from this sensor or from your body.
The sensor's visual acuity is the same as your own, except that it does not share the benefits of any \glossterm{magical} effects that improve your vision.
You otherwise act normally, though you may have difficulty moving or taking actions if the sensor cannot see your body or your intended targets, effectively making you \blinded.
If undisturbed, the sensor floats in the air in its position.
As a standard action, you can concentrate to move the sensor up to 30 feet in any direction, even vertically.
You can only have one casting of this spell active at once.
If you cast it again, any previous castings of the spell are dismissed.
\spelldur Attunement
\spelltags{\glossterm{Scrying}}
\end{spelleffects}
\end{spellcontent}
\begin{spellfooter}
\spellinfo{Divination}{Arcane, Divine, Nature}
\end{spellfooter}
\begin{spellsubcontent}
\begin{spellcantrip}
The sensor cannot be moved after it is originally created, and the spell's duration becomes Sustain (minor).
\end{spellcantrip}
\end{spellsubcontent}
\end{spellsection}
\subsubsection{Subspells}
\augment{2}{Alarm}
The sensor continues to observe its surroundings while you are not sensing through it.
If it sees a creature or object of Tiny size or larger moving within 50 feet of it, it will trigger a mental "ping" that only you can notice.
You must be within 1 mile of the sensor to receive this mental alarm.
This mental sensation is strong enough to wake you from normal sleep, but does not otherwise disturb concentration.
\augment{2}{Auditory}
At the start of each round, you can choose whether you hear from the sensor or from your body.
This choice is made independently from your sight.
The sensor's auditory acuity is the same as your own, except that it does not share the benefits of any \glossterm{magical} effects that improve your hearing.
\augment{3}{Accelerated}
When you move the sensor, you can move it up to 100 feet, instead of up to 30 feet.
\augment{3}{Dual}
You create an additional sensor in the same location.
You must move and see through each sensor individually.
\augment{3}{Penetrating}
The spell's range becomes \rngunrestricted, allowing you to cast it into areas where you do not have \glossterm{line of sight} or \glossterm{line of effect}.
\augment{4}{Reverse Scrying}
% TODO: wording
The sensor created by this spell appears at the location of the source of the ability that created the target sensor.
Replace the spell's targets with the following:
\begin{spellcontent}
\begin{augmenttargetinginfo}
\spelltwocol{\spelltgt{One magical sensor}}{\spellrng{\rngmed}}
\end{augmenttargetinginfo}
\end{spellcontent}
\augment{4}{Semi-Autonomous}
You can move the sensor as a \glossterm{minor action} rather than as a standard action.
\augment{5}{Scry Creature}
You must make a Spellpower vs. Mental attack against the target.
Success means the sensor appears in the target's space.
Failure means the sensor does not appear at all.
Replace the spell's targets with the following:
\begin{spellcontent}
\begin{augmenttargetinginfo}
\spellspecial
You must specify your target with a precise mental image of its appearance.
The image does not have to be perfect, but it must unambiguously identify the target.
If you specify its appearance incorrectly, or if the target has changed its appearance, you may accidentally target a different creature, or the spell may simply fail.
\spelltwocol{\spelltgt{One creature}}{\spellrng{Same plane (Unrestricted)}}
\end{augmenttargetinginfo}
\end{spellcontent}
\augment{6}{Innate}
You can cast the spell without spending an action point.
\augment{6}{Split Senses}
You do not have to choose whether to sense from the perspective of the sensor or from the perspective of your own body.
You constantly receive sensory input from both your body and the sensor.
\begin{spellsection}{Smite}
\begin{spellheader}
\spelldesc{You smite a foe with holy (or unholy) power.}
\end{spellheader}
\begin{spellcontent}
\begin{spelltargetinginfo}
\spelltwocol{\spelltgt{One creature}}{\spellrng{\rngmed}}
\end{spelltargetinginfo}
\begin{spelleffects}
\begin{spellattack}{Spellpower vs. Mental}
\spellsuccess Divine \glossterm{standard damage} \plus1d.
\spellcritical As above, but double damage.
\end{spellattack}
\end{spelleffects}
\end{spellcontent}
\begin{spellfooter}
\spellinfo{Channeling}{Divine}
\end{spellfooter}
\begin{spellsubcontent}
\begin{spellcantrip}
The spell deals \minus2d damage.
\end{spellcantrip}
\end{spellsubcontent}
\end{spellsection}
\subsubsection{Subspells}
\augment{2}{Word of Faith}
Replace the spell's targets and effects with the following:
\begin{spellcontent}
\begin{augmenttargetinginfo}
\spellburst{\areamed radius from your location}
\spelltgts{Creatures in the area that do not worship your deity}
\end{augmenttargetinginfo}
\begin{augmenteffects}
\begin{spellattack}{Spellpower vs. Mental}
\spellsuccess Divine \glossterm{standard damage} \minus1d.
\spellcritical As above, but double damage.
\end{spellattack}
\end{augmenteffects}
\end{spellcontent}
\augment{6}{Innate}
You can cast the spell without spending an action point.
\begin{spellsection}{Summon Monster}
\begin{spellheader}
\spelldesc{You summon a creature to fight by your side.}
\end{spellheader}
\begin{spellcontent}
\begin{spelltargetinginfo}
\spelltwocol{\spelltgt{One unoccupied square}}{\spellrng{\rngmed}}
\end{spelltargetinginfo}
\begin{spelleffects}
\spelleffect
A creature appears in the target location.
It visually appears to be a common Small or Medium animal of your choice, though in reality it is a manifestation of magical energy.
Regardless of the appearance and size chosen, the creature has hit points equal to twice your spellpower.
All of its defenses are equal to your 5 \add your spellpower, and its land speed is equal to 30 feet.
Each round, you choose the creature's actions.
There are only two actions it can take.
As a move action, it can move as you direct.
As a standard action, it can make a melee \glossterm{strike} against a creature it threatens.
Its accuracy is equal to your spellpower.
If it hits, it deals \glossterm{standard damage} \minus2d.
The type of damage dealt by this attack depends on the creature's appearance.
Most animals bite or claw their foes, which deals bludgeoning and slashing damage.
\spelldur Sustain (minor)
\spelltags{\glossterm{Manifestation}}
\end{spelleffects}
\end{spellcontent}
\begin{spellfooter}
\spellinfo{Conjuration}{Arcane, Divine, Nature}
\end{spellfooter}
\begin{spellsubcontent}
\begin{spellcantrip}
The spell's duration becomes Sustain (standard).
\end{spellcantrip}
\end{spellsubcontent}
\end{spellsection}
\subsubsection{Subspells}
\augment{2}{Summon Bear}
The creature appears to be a Medium bear.
As a standard action, it can make a \glossterm{grapple} attack against a creature it threatens.
Its accuracy is the same as its accuracy with \glossterm{physical attacks}.
While grappling, the manifested creature can either make a strike or attempt to escape the grapple.
This augment replaces the effects of any other augments that change the appearance of the creature.
\augment{6}{Innate}
You can cast the spell without spending an action point.
\begin{spellsection}{Telekinesis}
\begin{spellcontent}
\begin{spelltargetinginfo}
\spelltwocol{\spelltgt{One Medium or smaller creature or object}}{\spellrng{\rngclose}}
\end{spelltargetinginfo}
\begin{spelleffects}
\begin{spellattack}{Spellpower vs. Mental}
\spellsuccess
You move the target up five feet per spellpower. Moving the target upwards costs twice the normal movement cost.
\spellcritical
As above, but you move the target ten feet per spellpower instead of five feet per spellpower.
\end{spellattack}
\spelltags{\glossterm{Telekinesis}}
\end{spelleffects}
\end{spellcontent}
\begin{spellfooter}
\spellinfo{Evocation}{Arcane}
\end{spellfooter}
\begin{spellsubcontent}
\begin{spellcantrip}
You take a \minus2 penalty to accuracy with the spell.
\end{spellcantrip}
\end{spellsubcontent}
\end{spellsection}
\subsubsection{Subspells}
\augment{2}{Mending}
Replace the spell's targets and effects with the following:
\begin{spellcontent}
\begin{augmenttargetinginfo}
\spelltwocol{\spelltgt{One unattended object}}{\spellrng{\rngclose}}
\end{augmenttargetinginfo}
\begin{augmenteffects}
\spelleffect
The target is healed for hit points equal to \glossterm{standard damage} \plus1d.
\end{augmenteffects}
\end{spellcontent}
\augment{2}{Precise}
Replace the spell's effects with the following:
\begin{spellcontent}
\begin{augmenteffects}
\begin{spellattack}{Spellpower vs. Mental}
\spellsuccess
You move the target up to five feet in any direction.
In addition, you can make a check to manipulate the target as if you were using your hands.
The check's result has a maximum equal to your attack result.
\end{spellattack}
\spelltags{\glossterm{Telekinesis}}
\end{augmenteffects}
\end{spellcontent}
\augment{3}{Binding}
If your attack roll beat both the target's Fortitude and Mental defenses, it is \immobilized after the forced movement is finished.
This is a \glossterm{condition}, and lasts until removed.
\augment{3}{Levitate}
Replace the spell's targets and effects with the following:
\begin{spellcontent}
\begin{augmenttargetinginfo}
\spelltwocol{\spelltgt{One unattended object or willing creature (Medium or smaller)}}{\spellrng{\rngclose}}
\end{augmenttargetinginfo}
\begin{augmenteffects}
\spelleffect
The target floats in midair, unaffected by gravity.
During the movement phase, you can move the target up to ten feet in any direction.
\spelldur Sustain (minor)
\end{augmenteffects}
\end{spellcontent}
\augment{6}{Innate}
You can cast the spell without spending an action point.
\begin{spellsection}{Vital Surge}
\begin{spellcontent}
\begin{spelltargetinginfo}
\spelltwocol{\spelltgt{One creature}}{\spellrng{\rngmed}}
\end{spelltargetinginfo}
\begin{spelleffects}
\spellspecial
When you cast this spell, you choose whether the target is healed or takes damage.
\begin{spellattack}{Spellpower vs. Fortitude}
\spellsuccess The target heals hit points or takes life damage equal to \glossterm{standard damage} \plus1d.
\spellcritical As above, except that if you chose damage, the spell deals double damage.
\end{spellattack}
\spelltags{\glossterm{Life}}
\end{spelleffects}
\end{spellcontent}
\begin{spellfooter}
\spellinfo{Vivimancy}{Arcane, Divine, Nature}
\end{spellfooter}
\begin{spellsubcontent}
\begin{spellcantrip}
You cannot choose for the spell to heal the target, and the spell deals \minus2d damage.
\end{spellcantrip}
\end{spellsubcontent}
\end{spellsection}
\subsubsection{Subspells}
\augment{2}{Cure Wounds}
You cannot choose for the spell to deal damage.
In addition, for every 10 points of healing you provide, you can also heal one point of \glossterm{vital damage}.
\augment{2}{Restore Senses}
Replace the spell's effects with the following:
\begin{spellcontent}
\begin{augmenteffects}
\spelleffect
One of the target's physical senses, such as sight or hearing, is restored to full capacity.
This can heal both magical and mundane conditions, but it cannot completely replace missing body parts required for a sense to function (such as missing eyes).
\spelltags{\glossterm{Flesh}}
\end{augmenteffects}
\end{spellcontent}
\augment{2}{Undead Bane}
If the target is undead, the spell gains a \plus2 bonus to accuracy.
\augment{3}{Circle of Death}
Replace the spell's targets and effects with the following:
\begin{spellcontent}
\begin{augmenttargetinginfo}
\spellemanation{\areamed radius from you}
\spelltgts{Enemies in the area}
\end{augmenttargetinginfo}
\begin{augmenteffects}
\spelleffect
When this spell resolves, and the end of each \glossterm{action phase}, you make an attack against all targets to deal damage.
\begin{spellattack}{Spellpower vs. Mental}
\spellsuccess
The target takes life \glossterm{standard damage} \minus3d.
\end{spellattack}
\spelldur Sustain (minor)
\spelltags{\glossterm{Life}}
\end{augmenteffects}
\end{spellcontent}
\augment{3}{Cure Serious Wounds}
This subspell functions like the \textit{cure wounds} subspell, except that it heals one point of vital damage for every 5 points of healing instead of for every 10.
\augment{3}{Drain Life}
You gain temporary hit points equal to half the damage you deal with this spell.
\augment{3}{Remove Disease}
Replace the spell's effects with the following:
\begin{spellcontent}
\begin{augmenteffects}
\spelleffect
All diseases affecting the target are removed.
\spelltags{\glossterm{Flesh}}
\end{augmenteffects}
\end{spellcontent}
\augment{4}{Circle of Healing}
Replace the spell's targets and effects with the following:
\begin{spellcontent}
\begin{augmenttargetinginfo}
\spellemanation{\areamed radius from you}
\spelltgts{Allies in the area}
\end{augmenttargetinginfo}
\begin{augmenteffects}
\spelleffect
When this spell resolves, and the end of each \glossterm{action phase}, you make an attack against all targets to heal them.
\begin{spellattack}{Spellpower vs. Mental}
\spellsuccess
The target heals hit points equal to \glossterm{standard damage} \minus3d.
\end{spellattack}
\spelldur Sustain (minor)
\spelltags{\glossterm{Life}}
\end{augmenteffects}
\end{spellcontent}
\augment{4}{Cure Critical Wounds}
This subspell functions like the \textit{cure wounds} subspell, except that it heals one point of vital damage for every 2 points of healing instead of for every 10.
\augment{4}{Death Knell}
If the attack hits, the target suffers a death knell.
At the end of each round, if the target has 0 hit points, it immediately dies.
This effect lasts until the target removes this condition.
\par
This is a \glossterm{Death} effect.
\augment{6}{Innate}
You can cast the spell without spending an action point.