\chapter{Magic}\label{Magic}

Magic comes in many forms, but it is most commonly wielded with spells.
A spell is a one-time magical effect.
There are three types of spells: arcane (cast by mages), divine (cast by clerics), and nature (cast by druids). Cutting across these categories are the nine schools of magic.
Each of the nine schools represents a different type of mastery over the world, based on fundamentally distinct principles.

\section{Casting Spells}\label{Casting Spells}
    Whether a spell is arcane, divine, or natural, casting a spell works the same way.

    \subsection{Casting Process}

        \begin{itemize}
            \itemhead{Choose spell}: You must choose which spell to cast from among the spells you know.
                If a spell has \glossterm{subspells}, you must choose which subspell to use when you cast it.
            \itemhead{Choose augments}: If you know any \glossterm{augments}, you can apply any number of augments to the spell.
                If you apply an augment, you increase the spell's level by an amount equal to that augment's level.
                For details, see \pcref{Augments}.
            \itemhead{Pay action point}: If necessary, you must expend an action point to cast the spell.
                If you do not have an action point to spend, your attempt to cast the spell fails.
                Effects that replace action point costs also happen at this time.
            \itemhead{Perform spell components}: All spells have verbal and somatic components unless their description indicates otherwise (see \pcref{Components}).
            \itemhead{Choose effects}: You make choices about the spell's effects as you finish casting the spell.
                This includes deciding which creatures to target, where the spell takes effect, and so on.
        \end{itemize}

        All of the above steps take place at the start of the action phase, at the same time that other actions are decided.
        However, spells take time to cast, and their effects do not resolve during the \glossterm{action phase}.
        Instead, the targeting and effects of spells are resolved simultaneously with other actions during the \glossterm{delayed action phase}.
        If you take damage or are otherwise distracted during a phase in which you attempt to cast a spell, you may \glossterm{miscast} the spell (see \pcref{Concentration}).

    \subsection{Focusing and Concentration}\label{Concentration}\label{Focus}\label{Focusing and Concentration}

        Some actions require focusing, such as casting spells.
        If you are damaged or distracted while taking an action that requires focus, your concentration may be broken.

        \parhead{Concentration Checks}\label{Concentration Checks}

        To make a concentration check, roll d10 \add your level or Willpower \sub \glossterm{overwhelm penalties}.
        The DR is equal to 5 \add (twice the level of the spell you are casting).
        If the total damage you took in the current round exceeds your Mental defense, you take a \minus5 penalty to this check.
        If the damage exceeds the defense by 10, you take a \minus10 penalty instead.

        Success means you cast the spell successfully.
        Failure means you miscast the spell (see \pcref{Miscasting}).

        \parhead{Casting a Spell} You must concentrate to cast spells.
        When you finish casting a spell, if you took any damage while casting it, make a Concentration check (see \pcref{Concentration Checks}). Failure means you miscast the spell (see \pcref{Miscasting}), but you still lose the spell slot used to cast it.

        \parhead{Focusing on Existing Spells} Many spells allow you to spend a standard action focusing to extend their effects.
        At the end of every round you focus, if you took any damage, make a Concentration check.
        Failure means your focus ends, but the spell may continue to have effects, as indicated in the spell description.
        Most spells do not allow you to resume focusing on them after your concentration is broken.

        \parhead{Performing Rituals} You must focus to perform rituals.
        At the end of every round, if you took any damage, make a Concentration check.
        Failure means the ritual fails and has no effect.

        \parhead{Distracting Circumstances} In some circumstances, you need to Concentration make a concentration check to cast spells or take other actions even if you haven't taken damage.
        Examples include being on a galloping horse, in a storm-tossed ship, or in an earthquake.

        \parhead{Focus Limits} Focusing on a spell is mentally tiring.
        You can focus on a spell for up to 5 minutes without penalty.
        After 5 minutes, and every minute thereafter, you must make a Concentration check even if you haven't taken damage.
        If you fail, you lose your focus on the spell and become fatigued.
        The difficulty of the test increases by 2 for every additional minute of focus.

    \subsection{Miscasting}\label{Miscasting}

        If you start casting a spell and fail to complete it, such as if your concentration is broken or your armor interferes with your spellcasting, you miscast the spell.
        When you miscast a spell, the spell does not have its normal effect.
        Instead, a wave of magical energy causes a \glossterm{miscast backlash}, as described below.
        This is a \glossterm{magical} effect.
        Your \glossterm{power} with this effect is equal to your \glossterm{spellpower} with the spell you tried to cast.
        \begin{ability}
            \begin{spelltargetinginfo}
                \spellarea{5 foot radius burst centered on you}
                \spelltgts{Everything in the area}
            \end{spelltargetinginfo}
            \begin{spelleffects}
                \spelleffect The target takes \glossterm{standard damage} \minus1d.
            \end{spelleffects}
        \end{ability}

        At the start of each phase while you are casting a spell, you can choose to stop casting the spell, causing you to \glossterm{miscast} it instead.

    \subsection{Subspells}\label{Subspells}
        % TODO: some of this wording is weird
        All spells have a number of \glossterm{subspells}.
        Each subspell has a name, a level, and an effect.
        Whenever you cast a spell, you can choose to apply a single subspell you know from that spell.
        If you do, the spell's level becomes equal to the subspell's level.
        In exchange, the spell gains the effects of the subspell.
        You cannot learn or cast subspells whose spell level exceeds your maximum spell level.

        Some subspells simply add additional properties to a spell's normal effect.
        Others change the targets or effects of the spell significantly.
        After choosing whether to cast a subspell, you can apply any number of \glossterm{augments}, described below.

    \subsection{Augments}\label{Augments}
        There are a number of \glossterm{augments} that can be applied to spells and rituals to increase their power.
        Each augment has a name, a level, and an effect.
        Whenever you cast a spell or perform a ritual, you can choose to apply any number of augments you know to the spell or ritual.
        For each augment you apply, you increase the spell or ritual's level by an amount equal to the augment's level.
        In exchange, the ability gains the effects of that augment.
        If an augment would increase the spell or ritual's level beyond the maximum level you can cast, you cannot apply the augment to that ability.

        \parhead{Augments and Subspells}
        If a spell or ritual changes its properties with a subspell or subritual, it may become eligible for different augments.
        % TODO: fix example
        For example, if you apply the Fireball subspell to the \spell{pyromancy} spell, it changes to affect an area.
        You would then be able to apply the Widened augment to increase its area.

        \subsubsection{Augment Descriptions}\label{Augment Descriptions}

            \augment{1}{Extended} The ability's range increases by one step, to a maximum of \rngext.
            The steps are, in order: \rngtouch, \rngclose, \rngmed, \rnglong, and \rngext.
            This augment can be applied multiple times.
            Each time, the ability's range increases by an additional step.
            \par This augment can be applied to any spell or ritual with a range that is one of the above ranges.

            \augment{1}{Giant} The ability can affect a target one size category larger.
            This augment can be applied multiple times.
            Its effects stack.
            \par This augment can be applied to any spell or ritual that that has a maximum size category of targets that it can affect.

            \augment{2}{Cryptic} The spell's visual effects and magical aura changes to mimic a different spell of your choice.
            You may choose any combination of spell or \glossterm{subspell} you know, along with any other augments, that result in a spell of the same level or lower as the spell you are casting.
            This affects inspection of the spell itself by any means, such as with the Spellcraft skill (see \pcref{Spellcraft}).
            However, it does not alter the mechanical effects of the spell in any way.
            If the spell's effects depend on visual components, the spell may fail to work if you alter the spell's visuals too much. 

            \augment{2}{Quickened} You can cast the spell as a \glossterm{minor action}.
            In exchange, you cannot take any actions during the \glossterm{action phase} or \glossterm{delayed action phase} of the next round.
            \par This augment can be applied to any spell.

            \augment{2}{Selective} You may freely exclude any areas from the spell's effect.
            However, all squares in the final area of the spell must be contiguous.
            You cannot create split a spell's area into multiple completely separate areas.

            \augment{2}{Silent} You do not need to use \glossterm{verbal components} to cast the spell.
            \par This augment can be applied to any spell.

            \augment{2}{Stilled} You do not need to use \glossterm{somatic components} to cast the spell.
            \par This augment can be applied to any spell.

            \augment{2}{Widened} The ability's area increases by one step, to a maximum of \areahuge.
            The steps are, in order: \areasmall, \areamed, \arealarge, and \areahuge.
            This augment cannot affect abilities with other areas.
            A Small or Medium line is 5 ft.\ wide, while a Large or Huge line is 10 ft.\ wide.
            This augment can be applied multiple times.
            Each time, the ability's area increases by an additional step.
            \par This augment can be applied to any spell or ritual with an area that is one of the above areas.

            \augment{3}{Intensified} The ability deals \plus1d damage.
            This augment can be applied multiple times.
            Its effects stack.
            \par This augment can be applied to any spell or ritual that deals a dice pool of damage.

            \augment{3}{Mass} The spell affects up to five targets.
            If it deals damage, that damage is reduced by \minus2d.
            \par This augment can be applied to any spell that affects a single target of the caster's choice.
            It cannot be applied to spells that affect a single specific target, such as the caster.

            \augment{3}{Phasing} When determining whether you have \glossterm{line of sight} and \glossterm{line of effect} to a particular location with the spell, you can ignore a single solid obstacle up to five feet thick.
            This can allow you to cast spells through solid walls, though it does not grant you the ability to see through the wall.
            \par This augment can be applied to any spell with a range.

            \augment{3}{Precise} You gain a \plus1 bonus to accuracy with the ability.
            This augment can be applied multiple times.
            Its effects stack.
            \par This augment can be applied to any spell or ritual that has an attack roll.

            \augment{4}{Accelerated} The ritual takes half the normal amount of time to perform.
            \par This augment can be applied to any ritual.

            \augment{6}{Echoing} During the \glossterm{delayed action phase} of the next round, the spell's effect occurs again.
            All choices you made for the original casting of the spell are made identically for the repeat casting.
            It affects the same area, targets, and so on.

            If the spell is now invalid, such as if all of its targets are out of range, the additional casting has no effect.
            This augment not allow you to \glossterm{attune} to the same spell more than once.

    \subsection{Dismissing Spells}

        As a \glossterm{minor action}, you can dismiss any spells you cast that have lasting effects.
        This requires the same casting components (verbal and somatic) as casting the spell normally.
        The effects of a dismissed spell immediately end.

    \subsection{Impossible Spell Effects}
        If you try to cast a spell in circumstances that make the spell's effect impossible, the spell fails and has no effect.
        You still lose the spell slot used to cast it.

\section{Determining Spell Effects}

    \subsection{Spellpower}

        Both the accuracy and power of your spells is determined by your spellpower.
        Normally, your spellpower is equal to your level or an \glossterm{attribute}, whichever is higher.
        Effects that increase spellpower never increase spells per day or spells known.
        Only your class levels affect those values.

        \parhead{Multiple Spell Sources} If you the ability to cast spells from more than one separate ability, use the spellpower appropriate to the ability that you are casting the spell with.

        \parhead{Reducing Spellpower} You can voluntarily reduce the power of the spells you cast by using a lower spellpower.
        However, you cannot use a spellpower lower than the minimum spellpower required to cast the spell, which is equal to twice the spell's level.

        \parhead{Magic Resistance} Some creatures have magic resistance, which is an ability which allows them to resist \glossterm{magical} effects such as spells.
        You can overcome magic resistance by making an attack with an accuracy equal to your spellpower.
        See \pcref{Magic Resistance}, for details.

    \subsection{Magical Attacks}

        To affect an unwilling creature with a spell, you must make a magical attack.
        Your accuracy is normally equal to your spellpower.

        \subsubsection{Resisting a Spell} A creature that successfully resists a spell that has no obvious physical effects feels a hostile force or a tingle, but cannot deduce the exact nature of the attack without the use of the Spellcraft skill (see \pcref{Spellcraft}).

        \subsubsection{Not Resisting a Spell} A creature can voluntarily forego its defenses and willingly accept a spell's result.
            However, a character with a special immunity to specific magical effects cannot suppress that quality.

    \subsection{Line of Effect}\label{Line of Effect}

        Almost all abilities must have a \glossterm{line of effect} to function.
        Unless otherwise noted in an ability's description, you cannot target a creature you do not have line of effect to.
        In addition, spells that affect an area do not affect targets that the spell does not have line of effect to.

        A line of effect is a straight, unblocked path that indicates what a spell can affect.
        A line of effect is canceled by a solid barrier.
        It's like line of sight for ranged weapons, except that it's not blocked by fog, darkness, and other factors that limit normal sight.

        You must have a clear line of effect to any target that you cast a spell on or to any space in which you wish to create an effect.
        You must have a clear line of effect to the point of origin of any spell you cast.

        A burst, cone, cylinder, or emanation spell affects only an area, creatures, or objects to which it has line of effect from its origin (a spherical burst's center point, a cone-shaped burst's starting point, a cylinder's circle, or an emanation's point of origin).

        An otherwise solid barrier with a hole of at least 1 square foot through it does not block a spell's line of effect.
        Such an opening means that the 5-foot length of wall containing the hole is no longer considered a barrier for purposes of a spell's line of effect.

        \subsubsection{Destroying Barriers}\label{Destroying Barriers}
            Some abilities, such as the \spell{fireball} spell, deal damage to both creatures and objects.
            If a physical barrier is destroyed by an ability, that barrier does not affect the ability's line of effect.
            For example, a thin curtain of silk normally blocks line of effect.
            However, a spell that destroyed the curtain would have its full effect on everything behind the curtain.

    \subsection{Targeting Spells}

        \parhead{Midair Locations} A creature or object brought into being or transported to your location by a spell cannot appear floating in an empty space.
        It must arrive in an open location on a surface capable of supporting it.

        \parhead{Targeting Inside Creatures} Creatures block line of effect to the inside of their own bodies.
        As a result, you cannot cast a spell that takes effect inside a creature unless you are also inside the creature.
        This restriction applies even if there is no physical barrier to the inside of the creature; you cannot detonate a \spell{fireball} inside a creature's mouth, even if it has its mouth open at the time.

    \subsection{Special Spell Effects}

        \subsubsection{Attacks}
            Some spell descriptions refer to attacking.
            All abilities that affect any unwilling creatures, even if they don't deal damage, are considered attacks.
            If all creatures affected by a spell are \glossterm{willing}, the spell is not considered an attack.
            Spells that damage objects or summon allies are not attacks because the spells themselves don't harm anyone.

        \subsubsection{Resurrecting the Dead}\label{Resurrecting the Dead}

            Several spells have the power to restore slain characters to life.

            When a living creature dies, its soul departs its body, leaves the Material Plane, travels through the Astral Plane, and goes to abide on the plane where the creature's deity resides.
            If the creature did not worship a deity, its soul departs to the plane corresponding to its alignment.
            Bringing someone back from the dead means retrieving his or her soul and returning it to his or her body.

            \subparhead{Preventing Revivification} Enemies can take steps to make it more difficult for a character to be returned from the dead.
            Except for \spell{true resurrection}, every ritual to raise the dead requires a body, so keeping or destroying the body is an effective deterrent.
            The \spell{soul bind} ritual prevents any sort of revivification unless the soul is first released.

            \subparhead{Revivification against One's Will} A soul cannot be returned to life if it does not wish to be.
            A soul infallibly knows the name, alignment, and patron deity (if any) of the character attempting to revive it and may refuse to return on that basis.

    \subsection{Combining Effects}
        Spells or magical effects usually work as described, no matter how many other spells or magical effects happen to be operating in the same area or on the same recipient.
        Except in special cases, a spell does not affect the way another spell operates.
        Whenever a spell has a specific effect on other spells, the spell description explains that effect.

        However, spells, feats, and other abilities that have very similar effects may not both help their target.
        A character can only be increased so far beyond his or her normal limits; even layered with powerful magical effects, a commoner is no serious threat to a giant.
        The limitations on these effects are provided by the stacking rules described below.

        \subsubsection{Stacking Effects}
            Spells that provide bonuses or penalties usually do not stack with themselves.
            More generally, two enhancement bonuses don't stack even if they come from different spells; see \pcref{Stacking Rules}, for more details.

            \parhead{Same Effect More than Once in Different Strengths} In cases when two or more identical spells are operating in the same area or on the same target, but at different strengths, only the best one applies.
            This is called overlapping.

            \parhead{Same Effect with Differing Results} The same spell can sometimes produce varying effects if applied to the same recipient more than once.
            Usually, the last spell in the series trumps the others.
            None of the previous spells are actually removed or \glossterm{dismissed}, but their effects become irrelevant while the final spell in the series lasts.

            \parhead{One Effect Makes Another Irrelevant} Sometimes, one spell can render a later spell irrelevant.
            Both spells are still active, but one has rendered the other useless in some fashion.

            \parhead{Multiple Mind Control Effects} Sometimes magical effects that affect a creature's mind render each other irrelevant, such as a spell that removes the target's ability to act.
            Mental controls that don't remove the recipient's ability to act usually do not interfere with each other.
            If a creature is under the mental control of two or more creatures, it tends to obey each to the best of its ability, and to the extent of the control each effect allows.
            If the controlled creature receives conflicting orders simultaneously, the competing controllers must make opposed spellpower checks to determine which one the creature obeys.

            \parhead{Spells with Opposite Effects} Spells with opposite effects apply normally, with all bonuses, penalties, or changes accruing in the order that they apply.

            \parhead{Instantaneous Effects} Two or more spells with instantaneous durations work cumulatively when they affect the same target.

        % TODO: pull these out into generic descriptions of suppressing/dismissing spells and abilities
        \subsubsection{Suppressing Spells}\label{Suppressing Spells}
            Spells can be \glossterm{suppressed} by effects such as the \spell{antimagic} spell.
            While a spell is suppressed, it has no effect.
            However, if it stops being suppressed, its effects continue as if they had not been interrupted.

        \subsubsection{Dismissing Spells}\label{Dismissing Spells}
            When a spell is \glossterm{dismissed}, all of its effects with a duration end.
            Unless otherwise specified, any spell with a duration can be dismissed.

            If a spell affects multiple targets, it must be dismissed individually on each target.
            Dismissing the effect on one target does not affect the other targets of the spell.
            When you voluntarily dismiss a spell, you can choose to dismiss it for any number of targets with no more effort than dismissing it for a single target.

% TODO: make this whole section refer to abilities instead of spells
\section{Spell Descriptions}
    The description of each spell is presented in a standard format.
    Each category of information is explained and defined below.

    \subsection{Name}
        The first line of every spell description gives the name by which the spell is generally known.

    \subsection{Description}
        Beneath the spell name is a brief description of the spell's effect.
        This description has no mechanical significance, and simply describes how the spell usually appears or is used.

    \subsection{Schools of Magic}\label{Schools of Magic}
        The next line describes the schools of magic that the spell belongs to.
        Almost every spell belongs to at least one of nine schools of magic.
        A school of magic is a group of related spells that work in similar ways.
        They are described below.

        Some spells belong to more than one school of magic.
        Treat these spells for all purposes as if they were a member of both schools simultaneously.
        If you are prohibited from casting spells from a certain school, you cannot cast a spell which belongs to that school, even if it also belongs to another school.
        Likewise, any benefits which apply to casting spells from a specific school apply normally.
        If you have abilities which apply when casting spells from both schools that make up a spell, the abilities do not stack.

        A small number of spells (\spell{limited wish}, \spell{permanency}, \spell{prestidigitation}, and \spell{wish}) are universal, belonging to no school.

        \subsubsection{Abjuration}
            Abjuration spells reduce or negate damage, magic, and other effects.
            They can be used to protect allies and remove harmful magic.

        \subsubsection{Channeling}
            Channeling spells call upon the power of deities or other supernatural entities.
            They can be used to do anything those entities could do.
            Arcane spellcasters do not have access to Channeling spells.

        \subsubsection{Conjuration}
            Conjuration spells create and transport objects and creatures.
            They can be used to summon allies, transport creatures, and create objects from thin air.

        \subsubsection{Divination}
            Divination spells grant knowledge.
            They can be used to reveal hidden truths, predict the future, or communicate at great distances.

        \subsubsection{Enchantment}
            Enchantment spells alter the minds of creatures.
            They can be used to influence, control, or debilitate creatures.
            Almost all enchantment spells are \glossterm{Mind} spells, and many are \glossterm{Subtle} as well.

        \subsubsection{Evocation}
            Evocation spells create and manipulate energy.
            They can be used to inflict damage with energy blasts or manipulate the environment.

        \subsubsection{Illusion}
            Illusion spells create or manipulate sensory impressions.
            They can be used to create or remove light, conceal things that exist, or cause creatures to perceive things that do not exist.

        \subsubsection{Transmutation}
            Transmutation spells change the properties of creatures and objects.
            They can be used to grant new abilities, enhance existing abilities, change a target's form, or even alter the flow of time itself.

        \subsubsection{Vivimancy}
            Vivimancy spells manipulate the power of life and death, as well as souls.
            They can be used to heal or inflict wounds, resurrect the dead, create undead monsters, and cripple the bodies of creatures.

    \subsection{[Tags]}
        Appearing on the same line as the school, when applicable, are tags which further categorizes the spell in some way.
        Some spells have more than one tag.
        Ability tags are described at \pcref{Ability Tags}.

    \subsection{Level}
        The next line of a spell description gives the spell's level, a number between 1 and 9 that defines the spell's relative power.
        This number is preceded by an abbreviation for the class whose members can cast the spell.
        The Level entry also indicates whether a spell is a domain spell and, if so, what its domain and its level as a domain spell are.

        Names of spellcasting classes are abbreviated as follows: cleric Clr; druid Drd; mage Mge.

        The domains a spell can be associated with include Air, Chaos, Death, Destruction, Earth, Evil, Fire, Good, Knowledge, Law, Leadership, Magic, Nature, Protection, Strength, Travel, Trickery, Vitality, War, and Water.

    \subsection{Components}\label{Components}
        A spell's components are what you must do or possess to cast it.
        All spells have verbal and somatic components unless the spell description says otherwise.
        The Components entry in a spell description includes abbreviations that tell you what type of components it has.
        Specifics for material components and focuses are given at the end of the descriptive text.

        \parhead{Verbal (V)} A verbal component is a spoken incantation.
        To provide a verbal component, you speak in a strong voice with a volume at least as loud as ordinary conversation.

        A gag spoils the incantation (and thus the spell). A spellcaster who has been deafened has a 20\% chance to spoil any spell with a verbal component that he or she tries to cast.
        Likewise, a \spell{silence} spell imposes a 20\% chance of failure.

        \parhead{Somatic (S)} A somatic component is a measured and precise movement of at least one hand.
        While casting a spell with somatic components, one hand is used to cast the spell, and cannot be used to defend yourself or take other actions.
        % Touch range spells often include the act of touching the spell recipient as part of the somatic component.

        \parhead{Material (M)} A material component is one or more physical substances or objects that are annihilated by the spell energies in the casting process.

    \subsection{Casting Time}
        All spells have a casting time of one standard action unless otherwise specified in the spell description.
        Some spells and subspells require only a \glossterm{minor action} to cast.
        If a spell can be cast as a minor action, any of its subspells that require a standard action to cast state that explicitly.

        You make all pertinent decisions about a spell (range, target, area, effect, version, and so forth) when you finish casting the spell, not when you start casting.

    \subsection{Range}
        A spell's range indicates how far from you it can reach, as defined in the Range entry of the spell description.
        It indicates the maximum distance at which you can designate the spell's point of origin.
        The effect of a spell can extend beyond that range if it affects an area.
        A spell without a range simply affects the area specified in the spell's description; if it becomes relevant, you are considered to be its point of origin.
        Standard ranges include the following.
        \parhead{Touch} You must touch a creature or object to affect it.
        Touching a creature requires a successful attack against its Reflex defense.
        A touch spell that deals damage can score a critical hit just as a weapon can.
        A touch spell threatens a critical hit on a natural roll of 20 and deals double damage on a successful critical hit.
        Some touch spells allow you to touch multiple targets.
        You can touch as many willing targets as you can reach as part of the casting, but all targets of the spell must be touched in the same round that you finish casting the spell.
        Normally, you can touch no more than six targets per round of casting.

        If you have the ability to make multiple touch attacks, such as from the \spell{chill touch} spell, and you can make multiple attacks in a round, you can make a touch attack on each of those attacks.

        \parhead{Close} The spell reaches as far as 30 feet.
        \parhead{Medium} The spell reaches as far as 100 feet.
        \parhead{Far} The spell reaches as far as 300 feet.
        \parhead{Unlimited} The spell reaches anywhere on the same plane of existence.
        \parhead{Range Expressed in Feet} Some spells have no standard range category, just a range expressed in feet.

        \subsubsection{Unrestricted Ranges}

            Some spells have an unrestricted range, as denoted by \rngunrestricted.
            A spell with an unrestricted range does not require line of sight or line of effect.

    \subsection{Area}\label{Spell Area}

        Some spells affect an area.
        Sometimes a spell description specifies a specially defined area, but usually an area falls into one of the categories defined below.

        When casting an area spell, you select the point where the spell originates.
        The point of origin of a spell is always a grid intersection.
        When determining whether a given creature is within the area of a spell, count out the distance from the point of origin in squares just as you do when moving a character or when determining the range for a ranged attack.
        The only difference is that instead of counting from the center of one square to the center of the next, you count from intersection to intersection.

        You can freely decrease a spell's area, provided that you decrease it uniformly across all of the spell's dimensions.
        For example, you can cast a \spell{fireball} that affects a 5 foot radius if you choose to do so, but you can't cast a \spell{fireball} with any shape other than a sphere.

        You can count diagonally across a square, but remember that every second diagonal counts as 2 squares of distance.
        If the far edge of a square is within the spell's area, anything within that square is within the spell's area.
        If the spell's area only touches the near edge of a square, however, anything within that square is unaffected by the spell.

        \subsubsection{Area Types}\label{Area Types}

            \parhead{Burst} A burst spell has an immediate effect on all valid targets within an area.

            \parhead{Emanation} An emanation spell has effects within an area for the duration of the spell.
            It emanates from a specific creature or object, rather than a location.
            If that creature or object moves, the emanation moves with it.

            \parhead{Zone} A zone spell has effects within an area for the duration of the spell.
            Unless otherwise noted, it does not move after being created.

        \subsubsection{Area Shapes}

            \parhead{Cone} A cone extends from the point of origin in a quarter-sphere, up to the given length.

            \parhead{Cylinder} A cylinder extends out from the point of origin in a circle, up to the given radius.
            Cylinders also have a specific height.
            Unless otherwise specified, a cylinder's height is the same as its radius.
            Cylinders ignore obstacles that partially block line of effect, as long as there is a path around the obstacle that lies entirely within the spell's area.

            \parhead{Line} A line extends from the point of origin in a straight line, up to the given length.
            Lines also have a specific width and height.
            Unless otherwise specified, a line-shaped spell affects an area 5 feet wide and 5 feet high.
            The affected squares are chosen such that they stay close to the chosen line as possible.
            All squares affected by a line must be contiguous, so every square is adjacent to another affected square, disregarding diagonals.

            \parhead{Sphere} A sphere extends from the point of origin in all directions.
            Any spell which only specifies a radius for its area is sphere-shaped.

            \parhead{Wall} A wall is like a line, except that it has no width.
            Instead, it affects the boundary between squares.
            Walls can also be shapeable.

            Walls can normally be created within occupied squares, but not within solid objects.
            Some walls are called solid walls, and cannot be created within occupied squares.

            \parhead{Specific Shapes} Some spells specify a series of volumes that make up the area of the spell.
            Most commonly, the volumes are cubes.
            You may arrange the volumes as you want, with the restriction that each volume in the spell's area must be adjacent to one other volume in the spell's area.

        \subsubsection{Area Sizes}

            The area affected by many spells falls into one of three sizes.
            Each size defines the extent to which the spell extends out from its origin, whether as a radius or as a length.
            Some spells have specific sizes, as given in the spell description.

            \parhead{Small} Small spells extend 10 feet from their point of origin.
            \parhead{Medium} Medium spells extend 20 feet from their point of origin.
            \parhead{Large} Large spells extend 50 feet from their point of origin.

    \subsection{Targets}
        Some spells have a target or targets.
        You cast these spells on creatures or objects, as defined by the spell itself.
        You must be able to see or touch the target, and you must specifically choose that target.
        You do not have to select your target until you finish casting the spell.

        \subparhead{Multiple Targets} Most spells which have multiple targets also specify an area that the targets must reside in.
        If the spell says ``all creatures'', you do not have the ability to choose which creatures it affects; otherwise, you may pick and choose creatures within the area.

        \subparhead{Redirecting a Spell} Some spells allow you to redirect the effect to new targets or areas after you cast the spell.
        Redirecting a spell is a minor action.

        \subparhead{Targeting Restrictions} Many spells affect ``living creatures'', which means all creatures other than constructs and undead.
        Creatures in the spell's area that are not of the appropriate type do not count against the creatures affected.

        \subparhead{Willing Targets} Some spells restrict you to willing targets only.
        You can choose to be a willing target at any time.
        Unconscious creatures and objects are automatically considered willing, but a character who is conscious but immobile or helpless (such as one who is bound or paralyzed) is not automatically willing.

        \subparhead{Invalid Targets} You can always attempt to cast a spell on an invalid target.
        If the target is still invalid when the spell resolves, the spell is automatically \glossterm{miscast}.
        For example, you could attempt to cast the \spell{finger of death} spell, which only targets living creatures, on a creature that is secretly undead.
        The spell would automatically be miscast, which may reveal the target's true nature.

    \subsection{Duration}

        An ability's duration entry tells you how long the magical energy of the spell lasts.

        % TODO: pull this out into a better section
        \parhead{Sustain}\label{Sustain} The ability lasts as long as you take an action to sustain it.
        A Sustain duration always specifies an action type, such as Sustain (standard).
        At the end of each round, the ability is dismissed unless you used the ability that round or took the action to sustain the ability that round.
        Sustaining spells does not take concentration, and cannot be disrupted in the same way that casting spells can.

        Taking an action to sustain an ability only allows you to sustain a single use of that ability.
        However, you can sustain multiple abilities at once if you have available actions.

        You can only sustain an ability for up to 5 minutes.
        After that time, the ability's effect is dismissed.

        \subparhead{Sustain (shared)} The ability lasts as long as you and all targets sustain the ability each round.
        If you use a an ability with a Sustain (shared) ability on yourself, you only need to sustain the ability once per round, not twice.

        \parhead{Attunement} The ability lasts as long as you \glossterm{attune} to it (see \pcref{Attunement}).

        \subparhead{Attunement (multiple)} The ability lasts as long as you \glossterm{attune} to it.
        In addition, you can attune to multiple activations of the same ability at once (see \pcref{Multiple Attunement}).

        \subparhead{Attunement (shared)} The ability lasts as long as you and all targets \glossterm{attune} to it (see \pcref{Shared Attunement}).

        \parhead{Condition} The ability lasts until its target removes it, such as by taking the Recover action (see \pcref{Recover}).
        Only abilities that affect creatures can have the Condition duration.

        \parhead{Permanent} The ability lasts until it is somehow cancelled or removed, such as with the \spell{antimagic} spell.

        \parhead{Instantaneous} Abilities without a listed duration are instantaneous.

        \parhead{Targets, Effects, and Areas} If an ability affects creatures directly, the effects travel with the targets for the ability's duration.
        If an ability creates or summons objects or creatures, they last for the duration, and are capable of moving outside the ability's initial range.
        Such effects can sometimes be destroyed prior to when their duration ends.
        % Seems redundant with the definition of emanations elsewhere
        % If an ability creates an emanation, then the spell stays with that area for its duration.
        Creatures become subject to the spell when they enter the emanation and are no longer subject to it when they leave.

        % \parhead{Discharge} Occasionally a spells lasts for a set duration or until triggered or discharged.

    \subsection{Magic Resistance}\label{Magic Resistance}
        Magic resistance is an additional defense against \glossterm{magical} abilities such as spells.
        To affect a magic resistant creature with a magical ability, you must make an additional magical attack against the creature's magic resistance value.
        Your accuracy is equal to your \glossterm{power} with the ability you using, such as your spellpower with spells.
        If your attack result beats the creature's magic resistance, the ability works normally.
        Otherwise, the ability has no effect on the creature.

        Magic resistance does not prevent a magical ability from having its normal effect on other creatures or objects.
        Magical abilities which do not directly affect targets, such as the \spell{summon monster} or \spell{create image} spells, do not allow magic resistance.
        In addition, Thaumaturgy and Physical abilities do not allow magic resistance (see \pcref{Ability Tags}).

        Normally, creatures with magic resistance can choose to allow spells through their resistance.
        Some creatures cannot control their magic resistance, so an attack is always necessary to affect them.
        This is specified in the description of the creature's magic resistance.

    \subsection{Effect}
        This portion of a spell description details what the spell does and how it works.
        If one of the previous entries in the description included ``see text'', this is where the explanation is found.
        There are several key parts of a spell which are also contained here.

        \subsubsection{Damage}
            This is the amount of damage the spell deals.
            Typically, the effect will specify who takes the damage.
            If no effect is specified, the spell damages all of its targets, or all creatures (but not objects) in the area.
            A spell with this entry is considered a damaging spell.
            A spell without this entry is not, even if it could be used to deal damage.

            Spells can inflict many kinds of damage.
            Common damage types include acid, arcane, bludgeoning, cold, divine, electricity, fire, life, physical, piercing, slashing, solar, and sonic.

            \parhead{Damaging Items} Unless the descriptive text for the spell specifies otherwise, all items carried or worn by a creature are assumed to survive a magical attack.
            If an item is not carried or worn and is not magical, it does not get any defenses.
            It simply is dealt the appropriate damage.

        \subsubsection{Healing}
            This is the amount of damage the spell heals.
            Typically, the effect will specify who receives the healing.
            If no effect is specified, the spell heals all of its targets, or all creatures (but not objects) in the area.

\section{Ability Tags}\label{Ability Tags}

    Many spells and other abilities have tags that describe the ability's nature.
    Many of these tags have no game effect by themselves, but they govern how the ability interacts with spells, other abilities, unusual creatures, and so on.
    They are described below.

    \parhead{Acid} Acid abilities use corrosive acid.
    They do not function underwater.

    \parhead{Air} Air abilities control the surrounding air.
    They do not function in environments without air.

    \parhead{Animation} Animation abilities grant a semblance of life to objects.

    \parhead{Auditory} Auditory abilities use sound to cause their effects.
    Creatures and objects that cannot hear the effect are immune to it.

    \parhead{Chaotic} Chaotic abilities channel the essence of chaos.

    \parhead{Cold} Cold abilities use cold \glossterm{energy}. It is possible to freeze liquids and perform similar feats with cold abilities.

    \parhead{Compulsion} Compulsion abilities forcibly alter a creature's actions, but do not necessarily affect its opinions or personality.
    All Compulsion abilities are also \glossterm{Mind} abilities.

    \parhead{Creation} Creation abilities create permanent physical objects.
    Objects created with Creation abilities are identical to objects created through more mundane means.
    Unless otherwise specified, magical Creation abilities do not allow \glossterm{magic resistance}. 

    \parhead{Curse} Curse abilities lay supernatural curses on their targets.
    They cannot be dismissed, but can be removed with the \spell{break enchantment} or \spell{remove curse} spells.

    \parhead{Death} Death abilities only affect living creatures.
    A creature killed by a death effect cannot be returned to life by \spell{resurrection} or similar abilities that depend on an intact corpse.

    \parhead{Delusion} Delusion abilities alter a creature's opinons or personality, but do not necessarily affect their actions.
    All Delusion abilities are also Mind abilities.

    \parhead{Detection} Detection abilities reveal magical auras or information within an area.
    They can penetrate up to 1 foot of stone, 1 inch of common metal, a thin sheet of lead, or 3 feet of wood or dirt.
    For its ability to penetrate other materials, use the most similar substance from the list above.

    \parhead{Earth} Earth abilities manipulate the ground or other forms of dirt.
    They do not function if no earth is accessible.

    \parhead{Electricity} Electricity abilities use electrical \glossterm{energy}.

    \parhead{Enhancement} Enhancement abilities enhance the existing abilities of their targets.

    \parhead{Evil} Evil abilities channel the essence of evil.

    \parhead{Figment} Figment abilities create light, sound, or other sensations.
    Figments cannot remove real sensations present in their area, but they can add additional sensations.
    You can only create figments of sensations you understand; for example, you cannot create a figment which speaks in a language you do not understand.
    \par A figment's physical defenses are equal to 10 \add its size modifier.

    \parhead{Fire} Fire abilities use fire \glossterm{energy}. They do not function underwater.
    \par Fire abilities provide light equivalent to a torch for their duration.
    Abilities without a duration create a brief burst of torchlight.

    \parhead{Flesh} Flesh abilities manipulate the physical flesh of creatures.
    They have no effect on creatures without flesh, such as ghosts or oozes.

    \parhead{Glamer} Glamer abilities alter sensations present in an area or on a target.
    They can be used to change how something real appears, or to remove it from perception entirely.

    \parhead{Good} Good abilities channel the essence of good.

    \parhead{Imbuement} Imbuement abilities imbue their targets with magic, granting them new abilities.

    \parhead{Lawful} Good abilities channel the essence of law.

    \parhead{Life} Life abilities attack, restore, or manipulate the life force of creatures.
    They have no effect on objects and creatures that are not alive.
    \par Undead creatures are affected in a special way by Life abilities.
    In addition to any differences given in the effect's description, life damage instead heals undead creatures, and healing instead deals life damage.

    \parhead{Light} Light abilities create visible light.
    Their area is blocked by barriers that prevent sight, even if the barriers would not otherwise block effect areas.
    Similarly, their area of effect is not blocked by barriers which do not prevent sight, even if the barriers would normally block effect areas.

    \parhead{Manifestation} Manifestation abilities create temporary constructs formed from raw magical energy.
    Objects and creatures created with manifestation abilities seem real on the surface, but they have no internal structure.
    When an object or creature created by a Manifestation ability is destroyed or killed, or when the duration of the ability that created it expires, it disappears without a trace.
    Unlike \glossterm{Creation} abilities, magical Manifestation abilities allow \glossterm{magic resistance}.

    \parhead{Mind} Mind abilities manipulate the minds of creatures.
    They have no effect on objects or creatures without minds.

    \parhead{Physical} Physical abilities manipulate physical objects rather than having a direct magical effect on their targets.
    They do not allow magic resistance.
    Some abilities are not themselves Physical, but have Physical effects, such as \spell{mighty throw}.

    \parhead{Planar} Planar abilities transport matter or information between planes.

    \parhead{Poison} Poison abilities use substances to weaken the foe's body.

    \parhead{Scrying} Scrying abilities create one or more invisible magical sensors that send you information.
    Unless otherwise noted, the sensor created has the same powers of sensory acuity that you possess.
    This includes the effect of any abilities which target you personally, such as spells to increase your visual acuity, but not abilities which affect an area around you.
    However, the sensor is treated as a separate, independent sensory organ, and it functions normally even if you have been blinded, deafened, or otherwise suffered sensory impairment.
    \par Any creature trained in Spellcraft can notice the sensor by making a DR 20 Spellcraft check.
    The sensor can be dismissed as if it were an active spell.
    You cannot create a sensor in a location with lead sheeting between you and the location, and you sense that the effect is blocked in this way.
    \\

    \parhead{Shaping} Shaping abilities change the shape or structure of their targets.

    \parhead{Shielding} Shielding abilities improve the defenses of their targets.

    \parhead{Sizing} Sizing abilities alter the size of their targets.
    Unless otherwise stated, multiple effects which increase or decrease size do not stack.
    Opposing size modifications cancel each other out on a one for one basis, and any remaining effects occur normally.

    \parhead{Sonic} Sonic abilities use sonic \glossterm{energy}.

    \parhead{Speech} Speech abilities use words to achieve their ends.
    You must specify a language when using a Speech effect, and the language must be one you know (or have memorized the correct words to say). They have no effect on objects or creatures that do not understand the chosen language.

    \parhead{Subtle} Subtle abilities have no visual or otherwise perceivable manifestation.
    Creatures affected by Subtle abilities do not generally know that they are being magically influenced.
    Subtle spells can still be identified with the Spellcraft skill (see \pcref{Spellcraft}), but the DR is 10 higher than normal.

    \parhead{Telekinesis} Telekinesis abilities use telekinesis, the power of the mind.
    Many telekinesis abilities create fields of solid telekinetic force.

    \parhead{Teleportation} Teleportation abilities move creature or objects through the Astral Plane to a distant destination.
    A teleported creature can bring along equipment and held objects as long as their weight does not exceed the creature's maximum load (see \tref{Weight Limits}). Any excess items are left behind, in order of their distance from the creature's body.

    \parhead{Temporal} Temporal abilities alter the flow of time.

    \parhead{Thaumaturgy} Thaumaturgy abilities alter or destroy magic itself.
    They do not allow \glossterm{magic resistance}.

    \parhead{Trap} Trap abilities do not have their full effect immediately.
    All Trap abilities specify a condition or circumstance, such as opening a door, which triggers the full effect of the ability.
    \par Unless otherwise noted, active Trap effects can be detected with the Awareness skill and disabled with the Devices skill before their effect triggers (see \pcref{Awareness}, and \pcref{Devices}).
    The DR to detect and disable the effect is equal to 20 \add the \glossterm{power} of the effect.
    \par No more than one Trap ability can be placed on the same object or in the same area.
    Only the first trap placed has any effect.
    It must be dismissed before any new traps can be placed.

    \parhead{Visual} Visual abilities use visible objects or forces to cause their effects.
    Creatures and objects that cannot see the effect are immune to it.

    \parhead{Water} Water abilities use water to cause their effects.

\section{Cantrips}\label{Cantrips}
    Cantrips are special spells that arcane casters can use at will.
    Like other spells, they have verbal and somatic components.
    All cantrips take a standard action to cast unless specified otherwise in the description.
    Cantrips are considered to be 0th level for the purpose of spells and abilities which reference spell level.
    They are described at the end of Chapter 12.

\section{Rituals}\label{Rituals}
    Rituals are ceremonies that create magical effects.
    Spellcasting characters can learn and perform rituals.
    You don't memorize a ritual as you would a normal spell; rituals are too complex for all but the most knowledgeable sages to commit to memory.
    To perform a ritual, you need to read from a book or a scroll containing it.
    Rituals are similar to spells, but they are not considered spells.
    \subsection{Ritual Descriptions}
        \par Like a spell, each ritual has a school, a level, and a magical effect.
        % TODO: proper chapter references
        Rituals are described in Chapter 13. The description of each ritual follows the same format as the description of spells in Chapter 12, except that every ritual has a level, like \glossterm{subspells} do.

        \subparhead{Ritual Sources}
        A ritual always matches the magic source of the person performing the ritual.
        For example, \spell{scrying} is an arcane ritual when performed by a wizard, but a divine ritual when performed by a cleric.
    \subsection{Ritual Requirements}
        In order to learn and perform a ritual, you must be able to cast at least one spell of the same level as the ritual.
    \subsection{Ritual Books}
        A ritual book contains one or more rituals that you can use as frequently as you want, as long as you can spend the time and \glossterm{action point} to perform the ritual.
        Scribing a ritual in a ritual book costs an amount of precious inks.
    \subsection{Ritual Costs}\label{Ritual Costs}
        The costs to scribe rituals are described on \trefnp{Ritual Costs}.
        \begin{dtable}
            \lcaption{Ritual Costs}
            \begin{dtabularx}{\columnwidth}{X l l}
                \tb{Ritual Level} & \tb{Cost to Scribe} & \tb{Item Level} \\
                \bottomrule
                1st-Level & 50 gp & 1st \\
                2nd-Level & 200 gp & 3rd \\
                3rd-Level & 500 gp & 4th \\
                4th-Level & 1,250 gp & 7th \\
                5th-Level & 3,000 gp & 9th \\
                6th-Level & 7,500 gp & 11th \\
                7th-Level & 15,000 gp & 12th \\
                8th-Level & 35,000 gp & 14th \\
                9th-Level & 75,000 gp & 16th \\
            \end{dtabularx}
        \end{dtable}

    \subsection{Subrituals}\label{Subrituals}
        Many rituals have \glossterm{subrituals}, just like many spells have \glossterm{subspells} (see \pcref{Subspells}).
        Subrituals work in the same way as subspells, except that they are applied to rituals instead of spells.

    \subsection{Performing Rituals}
        To perform a ritual, you must have a ritual book containing the ritual and the material components required for the ritual.
        Unless otherwise specified, performing a ritual requires spending a single \glossterm{action point}.
        Some rituals require multiple action points to complete.
        Other creatures can supply action points to help you perform rituals; see Ritual Participants, below.

        If you are distracted during the ritual, you must make a Concentration check, just as if you were casting a spell of the ritual's level.
        If you fail, the ritual is ruined and you must start from the beginning.
        % TODO: can we remove the ``casting a ritual'' wording?
        \par Performing a ritual and casting a ritual mean the same thing.

        \subsubsection{Ritual Participants}
            Creatures can assist in the performance of rituals even if they are unable to perform rituals themselves.
            A creature that helps perform a ritual is called a ritual participant, and the creature performing the ritual is called the ritual leader.
            A ritual participant may spend an action point in place of or in addition to the action point spent by the creature performing the ritual.
            It may also \glossterm{attune} to the effect of the ritual in place of the creature performing the ritual.
            Only one creature may attune to the ritual's effect in this way.
            If multiple creatures are willing to spend action points or attune to effects, the ritual leader decides which creatures spend action points or attune to the ritual's effects.

            The steps required to participate in rituals can be complex.
            Ritual participants must be given specific instructions for the actions they must perform during a ritual by a creature who knows how to perform the ritual.
            This instruction generally takes half the time required to perform the ritual.
            A creature cannot participate in rituals unless it has an Intelligence of at least 0, can speak at least one language, and has the fine motor control required to perform the somatic components of spells.

            Normally, a ritual participant can only contribute one action point.
            If the participant can cast spells from the same source as the ritual, they can contribute any number of action points.

            \parhead{Changing Ritual Participation}
            Rituals are deeply complex magic, and they cannot be abandoned or paused partway through.
            If the number of ritual participants in a ritual decreases below its initial value, the ritual fails at the end of the next round if the number of participants is not restored.
            However, ritual participants can transfer their participation to other creatures without disrupting the ritual.

            In order to transfer ritual participation, the new creature must be able to participate in the ritual, and must immediately spend the same number of action points as the creature that it is taking over from.
            Similarly, the ritual leader can transfer their leadership to another creature.
            In addition to the requirements for transferring ritual participation, the new leader must know the ritual and be able to perform it themselves.

            Changing ritual participation and leadership is usually done when performing extraordinarily long or demanding rituals.

    \subsection{Magical Writings}
        To record a spell in written form, a character uses complex notation that describes the magical forces involved in the spell.
        The notation constitutes a universal language that spellcasters have discovered, not invented.
        Each writer uses this universal system regardless of their native language or culture.
        However, each character uses the system in their own way.
        Another person's magical writing remains incomprehensible to even the most powerful spellcaster until they take the time to study and decipher it.

        To decipher an magical writing (such as a single spell in written form on a scroll), you must make a Spellcraft check (DR 10 \add the spell's level). If the skill check fails, you cannot attempt to read that particular spell again until the next day.
        A \ritual{read magic} ritual automatically deciphers a magical writing without a skill check.
        If the person who created the magical writing is on hand to help the reader, success is also automatic.

        Once a character deciphers a particular magical writing, they do not need to decipher it again.
        Deciphering a magical writing allows the reader to identify the spell and gives some idea of its effects (as explained in the spell or ritual description).

\section{Types of Abilities}

    There are two types of abilities: magical abilities and physical abilities.

    \parhead{Magical Abilities}\label{Magical Abilities} A magical ability is an ability that has no physical explanation.
    Examples include spells, a medusa's petrifying gaze, and a cleric's domain invocations.
    Magical attacks often target Fortitude and Mental defenses, and can be resisted by \glossterm{magic resistance}.
    Abilities that are magical in nature are indicated with a [Mag] tag.
    Abilities that are not magical are \glossterm{mundane}.

    Many abilities which fundamentally concern magical effects are not themselves magical in nature.
    This is most commmon with abilities that represent choices the character makes or knowledge the character has.
    For example, although all spells are magical abilities, the ability to cast spells is not itself a magical ability.
    It is simply knowledge that the creature possesses.
    Of course, that knowledge would be useless if the creature had no access to magic.

    \parhead{Physical Abilities}\label{Physical Abilities} A physical ability has a tangible component and some form of natural explanation.
    Examples include weapon attacks, a dragon's breath weapon, and a barbarian's rage.
    Physical attacks often target Armor and Reflex defenses.
    Unless otherwise indicated, all abilities are physical in nature.
    Abilities that are not physical are \glossterm{magical}.
