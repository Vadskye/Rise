\chapter{Description}

\section{Alignment}\label{Alignment}
    A creature's general moral and personal attitudes are represented by its alignment: lawful good, neutral good, chaotic good, lawful neutral, neutral, chaotic neutral, lawful evil, neutral evil, or chaotic evil.

    Alignment is a tool for developing your identity.
    It is not a straitjacket for restricting your actions.
    Each alignment represents a broad range of personality types or personal philosophies, so two characters of the same alignment can still be quite different from each other.
    In addition, few people are completely consistent.

    \subsection{Good vs. Evil}
        The ancient battle between good and evil takes many forms, and distinguishing good from evil is a deeply complex task.
        For the purposes of Rise, good and evil are strictly defined according to selfishness vs\. altruism.
        The actions of good characters may at times be morally reprehensible, and the actions of evil characters may seem to be virtuous.
        However, this narrow definition of good and evil avoids the complexities of defining a more robust moral system while preserving the fundamental conflict between good and evil.

        \parhead{Good} Good characters are altruistic.
        They take other creatures into account when making decisions, and actively try to help or improve others around them.
        Good characters may have significant disagreements about what actions are best, but they consistently prioritize the good of others or the ``greater good'' over their own desires.
        Different good characters may also have different perspectives on who they should take into account when making decisions.
        For example, some good characters actively work to protect animals and plants, while others only care about sentient beings.

        \parhead{Evil} Evil characters are selfish.
        They consistently prioritize their own desires and needs over the desires of others, even their allies or friends.
        Evil characters may perform good deeds, but their ultimate motivation is to help themselves or make themselves feel better, not to help others.

        \parhead{Neutral} Characters that are neutral between good and evil are neither consistently altruistic nor consistently selfish.
        Most neutral characters behave altruistically in some ways and selfishly in other ways -- either at different times, or about different aspects of life.
        They often have strong bonds to particular individuals who they care about selflessly, but are not altruistic in a general sense.
        Non-sentient beings such as animals are neutral rather than good or evil.

    \subsection{Law vs. Chaos}
        \parhead{Law} Lawful characters value consistency.
        They obey rules that guide their actions.
        Some lawful characters draw their rules from external forces, such as serving a particular master or following the legal laws of the land.
        Other lawful characters follow rules they make for themselves.

        \parhead{Chaos} Chaotic characters value flexibility and freedom.
        They make decisions based on what they think or feel at the time, even if it is inconsistent with their previous statements or action.

        \parhead{Neutral} Characters that are neutral between law and chaos are neither exceptionally consistent nor exceptionally inconsistent.
        They tend to be generally consistent but may change their minds under the right circumstances.
        Non-sentient beings such as animals are neutral rather than lawful or chaotic.

\section{Vital Statistics}

    \subsection{Age}
        You can choose or randomly generate your age.
        If you choose it, it must be at least the minimum age for your species and class (see \trefnp{Random Starting Ages}). Your minimum starting age is the adulthood age of your species plus the number of dice indicated in the entry corresponding to the character's species and class on \trefnp{Random Starting Ages}.

        Alternatively, refer to \trefnp{Random Starting Ages} and roll dice to determine how old you are.

        \begin{dtable}
            \lcaption{Random Starting Ages}
            \begin{dtabularx}{\columnwidth}{l c *{3}{>{\ccol}X}}
                % TODO: add spellwarped back once they are implemented
                \tb{Species} & \tb{Adulthood} & \tb{Barbarian Rogue} & \tb{Fighter Mage Paladin Ranger} & \tb{Cleric Druid Monk} \tableheaderrule
                Human    & 15 years  & \plus1d4 & \plus1d6 & \plus2d6  \\
                Dwarf    & 40 years  & \plus3d6 & \plus5d6 & \plus7d6  \\
                Elf      & 110 years & \plus4d6 & \plus6d6 & \plus10d6 \\
                Gnome    & 40 years  & \plus4d6 & \plus6d6 & \plus9d6  \\
                Half-elf & 20 years  & \plus1d6 & \plus2d6 & \plus3d6  \\
                Half-orc & 14 years  & \plus1d4 & \plus1d6 & \plus2d6  \\
                Halfling & 20 years  & \plus2d4 & \plus3d6 & \plus4d6  \\
            \end{dtabularx}
        \end{dtable}

        With age, your \glossterm{checks} based on physical attributes decrease and your checks based on mental attributes increase (see \trefnp{Aging Effects}).

        When you reach venerable age, the GM secretly rolls your maximum age, which is the number from the Venerable column on \trefnp{Aging Effects} plus the result of the dice roll indicated on the Maximum Age column on that table.
        They record the result.
        If you reach your maximum age, you die of old age at some time during the following year.

        The maximum ages are for player characters. Most people in the world at large die from pestilence, accidents, infections, or violence before getting to venerable age.

        \begin{dtable}
            \lcaption{Aging Effects}
            \begin{dtabularx}{\columnwidth}{l *{4}{>{\ccol}X}}
                \tb{Species}  & \tb{Middle Age\fn{1}} & \tb{Old\fn{2}} & \tb{Venerable\fn{3}} & \tb{Maximum Age} \tableheaderrule
                Human    & 35 years  & 53 years  & 70 years  & \plus4d10 years \\
                Dwarf    & 125 years & 188 years & 250 years & \plus2d\% years \\
                Elf      & 175 years & 263 years & 350 years & \plus4d\% years \\
                Gnome    & 100 years & 150 years & 200 years & \plus3d\% years \\
                Half-elf & 62 years  & 93 years  & 125 years & \plus6d10 years \\
                Half-orc & 30 years  & 45 years  & 60 years  & \plus2d10 years \\
                Halfling & 50 years  & 75 years  & 100 years & \plus1d\% years \\
            \end{dtabularx}
            1 At middle age, \minus1 to \glossterm{checks} based on Str, Dex, and Con; \plus1 to \glossterm{checks} based on Int, Per, and Wil. \\
            2 At old age, the aging modifiers change to \minus2 and \plus2.
            2 At venerable age, the aging modifiers change to \minus3 and \plus3.
        \end{dtable}

    \subsection{Height and Weight}
        The dice roll given in the Height Modifier column determines the character's extra height beyond the base height. That same number multiplied by the dice roll or quantity given in the Weight Modifier column determines the character's extra weight beyond the base weight.

        \begin{dtable}
            \lcaption{Random Height and Weight}
            \begin{dtabularx}{\columnwidth}{l *{3}{>{\lcol}X} >{\lcol}p{5em}}
                \tb{Species} & \tb{Base Height} & \tb{Height Modifier} & \tb{Base Weight} & \tb{Weight Modifier} \tableheaderrule
                Human, male      & 4' 10'' & \plus2d10 & 120 lb. & \x (2d4) lb. \\
                Human, female    & 4' 5''  & \plus2d10 & 85 lb.  & \x (2d4) lb. \\
                Dwarf, male      & 3' 9''  & \plus2d4  & 130 lb. & \x (2d6) lb. \\
                Dwarf, female    & 3' 7''  & \plus2d4  & 100 lb. & \x (2d6) lb. \\
                Elf, male        & 4' 5''  & \plus2d6  & 85 lb.  & \x (1d6) lb. \\
                Elf, female      & 4' 5''  & \plus2d6  & 80 lb.  & \x (1d6) lb. \\
                Gnome, male      & 3' 0''  & \plus2d4  & 40 lb.  & \x 1 lb.     \\
                Gnome, female    & 2' 10'' & \plus2d4  & 35 lb.  & \x 1 lb.     \\
                Half-elf, male   & 4' 7''  & \plus2d8  & 100 lb. & \x (2d4) lb. \\
                Half-elf, female & 4' 5''  & \plus2d8  & 80 lb.  & \x (2d4) lb. \\
                Half-orc, male   & 5' 0''  & \plus2d10 & 150 lb. & \x (2d6) lb. \\
                Half-orc, female & 4' 8''  & \plus2d10 & 110 lb. & \x (2d6) lb. \\
                Halfling, male   & 2' 8''  & \plus2d4  & 30 lb.  & \x 1 lb.     \\
                Halfling, female & 2' 6''  & \plus2d4  & 25 lb.  & \x 1 lb.
            \end{dtabularx}
        \end{dtable}

% Is this the right location?
\section{Planes}\label{Planes}
    The universe of Rise is divided into \glossterm{planes}.
    A plane is a distinct realm of existence.
    Except for the connections between planes through \glossterm{planar rifts}, each plane is effectively an isolated universe, and different planes can obey different fundamental laws.
    For example, the Material Plane has gravity that exerts a consistent acceleration in a single absolute direction.
    However, the Astral Plane has subjective gravity, where each creature on the plane chooses the direction that gravity pulls it in, if any.

    \subsection{General Cosmology}
        The planes of Rise are divided up into groups.

        \parhead{Primal Planes} The primal planes are manifestations of the basic building blocks of the universe.
        Each plane in this group is predominantly composed of a single element or type of energy.
        There are four primal planes: Air, Earth, Fire, and Water.

        \parhead{Aligned Planes} The aligned planes are manifestations of the nine alignments that define the morality of the universe.
        Each plane in this group is strongly associated with a particular alignment.
        The souls of creatures with the corresponding alignment often spend their afterlife in the Aligned Planes.
        There are nine aligned planes, one for each alignment combination (see \pcref{Alignment}).

        \parhead{Nexus Planes} The nexus planes are composite planes with a number of distinct environments and filled with creatures of myriad alignments.
        Nexus planes comprise the majority of civilization across all planes.
        They do not have their own unique planar essence, and no planar creatures are native to nexus planes.
        There are two nexus planes: the Material Plane and the Astral Plane.

        \parhead{Demiplanes} These planes are small, fragmentary realms that are greatly limited in their scope.
        There is no specific list of demiplanes, and they share few common properties.
        Most demiplanes were created for particular purposes by beings of great power, though some simply came into existence through unknown means.

    \subsection{Planar Rifts}\label{Planar Rifts}
        Normally, there are boundaries between different planes that prevent direct passage between them.
        However, \glossterm{planar rifts} are places where these boundaries have weakened, making interplanar travel easier.
        A planar rift joins a specific location on one plane to a specific location on a different plane.
        Most planar rifts lead to and from the Astral Plane, which is the space between the other planes (see \pcref{The Astral Plane}).

        Most planar rifts still require the use of magic, such as the \spell{plane shift} ritual, to actually cross between planes.
        Some especially large rifts enable physical travel between planes without the use of any magic.

    \subsection{Planar Traits}
        \subsubsection{Gravity Direction}
            The direction of gravity on a plane can take one of the following forms:
            \begin{itemize}
                \item Fixed Gravity: Gravity points in a fixed direction and with a fixed strength at all locations on the plane.
                    Almost all planes with a fixed gravity have a perfectly flat surface.
                \item Absolute Directional Gravity: Gravity points in a consistent direction according to a rule that applies equally to everything on the plane, but which is not in a fixed direction.
                    For example, a plane filled with floating spheres where gravity always points towards the closest sphere has absolute directional gravity.
                \item Subjective Gravity: Each creature on the plane chooses the direction of gravity for that creature.
                    The plane has no gravity for unattended objects and nonsentient creatures.
                    A creature on the plane can make use the \textit{control gravity} ability as a \glossterm{minor action}.
                    \begin{freeability}{Control Gravity}
                        Make a Willpower check with a \glossterm{difficulty rating} of 10.
                        Success means that you choose the direction of gravity that applies to you on the current plane.
                        Alternately, you can choose for gravity to not apply to you.

                        Failure means you gain a \plus2 bonus to the next \textit{control gravity} ability you use on this plane.
                        This bonus stacks with itself and lasts until you succeed at a \textit{control gravity} ability on this plane.
                    \end{freeability}
            \end{itemize}

        \subsubsection{Gravity Strength} The strength of gravity on a plane can take one of the following forms:
            \begin{itemize}
                \item Normal Gravity: Gravity is about the strength of Earth.
                \item No Gravity: There is no gravity on the plane.
                    The \glossterm{range limits} of ranged weapons are quadrupled.
                    % TODO: what additional effects are there?
                \item Light Gravity: Gravity is about half the strength of Earth.
                    % Should there be check penalties?
                    The weight of all items is halved.
                    The \glossterm{range limits} of ranged weapons are doubled.
                \item Heavy Gravity: Gravity is about twice the strength of Earth.
                    Creatures take a \minus2 penalty to Strength and Dexterity-based checks.
                    The weight of all items is doubled.
                    The \glossterm{range limits} of ranged weapons are halved, to a minimum of 5 feet.
                    % TODO: falling damage
                \item Extreme Gravity: Gravity is about four times the strength of Earth.
                    Creatures take a \minus4 penalty to Strength and Dexterity-based checks.
                    The weight of all items is quadrupled.
                    The \glossterm{range limits} of ranged weapons are reduced to one quarter of the normal value, to a minimum of 5 feet.
                    % TODO: falling damage
            \end{itemize}

        \subsubsection{Light} Various planes are illuminated in different ways.
            \begin{itemize}
                \item Fixed Source: There is a single constant source of light on the plane.
                \item Mobile Source: There is a single source of light on the plane that moves around it, illuminating different parts of the plane at different times.
                \item None: There is no natural source of light on the plane.
                    Other sources of light, such as torches, function normal.
            \end{itemize}

        \subsubsection{Limits} The behavior of a plane at its limits can vary widely.
            Some planes have different behaviors at different limits depending on their shape.
            \begin{itemize}
                \item Astral Gate: If you reach the limits of the plane, you find a planar gate to the Astral Plane.
                \item Barrier: If you reach the limits of the plane, you find an impassable barrier.
                    The barrier takes the form of a substance relevant to the plane's nature.
                    It may be possible to dig tunnels into the barrier to some depth, but there is nothing behind the barrier.
                    As you progress past the limit, the barrier becomes increasingly difficult to break through, and eventually it becomes completely impenetrable.
                \item Looped: If you go beyond the limits of the plane, you wrap around to the opposite side of the plane.
                    There is no obvious transition point or perception of transportation when this occurs - the shape of the plane simply connects to itself.
                    On very small planes, this can allow you to see your own back, though looped planes of that size are rare.
                \item Infinite: The plane has no limits. This is extremely rare.
            \end{itemize}

        \subsubsection{Planar Connectivity}
            Different planes have different degrees of connection to other planes.
            \begin{itemize}
                \item Isolated: The plane is difficult to reach or leave.
                    It has no permanent \glossterm{planar rifts}, and temporary rifts are rare or nonexistent.
                \item Stable Connected: The plane has multiple permanent \glossterm{planar rifts}.
                    However, temporary rifts are rare.
                \item Unstable Connected: The plane has no permanent \glossterm{planar rifts}, but temporary rifts are common.
                \item Conduit: The plane has a large number of permanent \glossterm{planar rifts}, and temporary rifts are common.
            \end{itemize}

        \subsubsection{Shape} The shape of a plane defines the shape of its core surface, and what happens if you travel beyond that surface.

            \begin{itemize}
                \item Flat Surface: The plane consists of a flat surface generally made of earth or similar material.
                    Most activity and civilization on the plane happens on this surface.
                    It is usually possible to construct tunnels into a flat surface plane to some depth, depending on the size of the plane.
                \item Hollow Sphere: The plane consists of a hollow sphere with an outer boundary generally made of earth or similar material.
                    Most activity and civilization on the plane happens on the inner surface of the sphere.
                    Some hollow sphere planes have an outer surface that can also be accessed, but most have a limit before any outer surface can be reached.
                \item Solid Sphere: The plane consists of a solid sphere generally made of earth or similar material.
                    Most activity and civilization on the plane happens on the surface of the sphere.
                    It is possible to construct tunnels into a solid sphere plane, but it may become increasingly difficult to traverse the plane as you approach the center of the sphere.
                    In general, the limit of a solid sphere plane is located at ten times the radius of the plane's primary sphere.
                \item Uniform: The plane has no well-defined surface or ground layer.
                    Some uniform planes have no ground or solid obstacles, while others are composed almost entirely of ground and firmament.
                    Uniform planes almost always still have limits of some kind.
            \end{itemize}

            % TODO: morphicness, alignment, magic

    \subsection{Plane Descriptions}

        \subsubsection{Primal Planes}

            \parhead{The Plane of Air}
            The Plane of Air is a a soaring landscape unencumbered by gravity or ground.
            The vast expanses of empty air are littered with clouds and unpredictable winds.
            Any inhabitants of the plane must adapt to a highly mobile lifestyle.
            A number of towns and structures have been built in the plane using raw materials brought from other planes.
            They sail through the air at the whims of the wind, and are occasionally battered by intersections with other wind streams.

            The Plane of Air has the following planar traits:
            \begin{itemize}
                \item Gravity strength: No Gravity
                \item Light: Fixed Source, from a sun outside the limits of the plane
                \item Limits: Looped
                \item Planar connectivity: Unstable Connected
                \item Shape: Uniform, in a sphere with a radius of about 2,000 miles.
            \end{itemize}

            \parhead{The Plane of Earth}
            The Plane of Earth is a titanically large body of earth and stone.
            A labyrinthine series of mostly airless tunnels weave their way through the plane, connecting the few cities.
            The plane is a major source of valuable gems, diamonds, and rare metals like mithral, but the dense rock and airless environment make successful mining difficult.
            In addition, earthquakes periodically reshape the environment by collapsing old tunnel systems and constructing new ones.
            Some cities have been carved out in vast underground rooms reinforced to survive the earthquakes.

            The Plane of Earth has the following planar traits:
            \begin{itemize}
                \item Gravity direction: Fixed
                \item Gravity strength: Normal
                \item Light: None
                \item Limits: Barrier, formed from increasingly dense rock that eventually becomes so hard that no known material or magic can damage it.
                \item Planar connectivity: Stable Connected
                \item Shape: Uniform, in a sphere with a radius of about 500 miles.
            \end{itemize}

            \parhead{The Plane of Fire}
            The Plane of Fire is an endless searing inferno.
            The plane's essence is highly combustible, allowing fires to burn indefinitely without any obvious fuel.
            However, the intensity of flames on the plane are highly uneven, as the plane generates fuel in various locations that shift over time.
            Some pockets on the surface are devoid of natural fuel, allowing the allow the construction of trading hubs where the few inhabitants of the plane who are not naturally immune to fire can survive.
            A variety of large tunnels and magma flows run through the sphere, and the intensity of the heat generally increases as you approach the center.

            The Plane of Fire has the following planar traits:
            \begin{itemize}
                \item Gravity direction: Absolute Directional, pointing to the center of the sphere
                \item Gravity strength: Normal
                \item Light: None, though the constant fires provide sufficient illumination in most locations on the plane
                \item Limits: Looped
                \item Planar connectivity: Unstable Connected
                \item Shape: Solid Sphere, with a radius of about 1,000 miles.
            \end{itemize}

            \parhead{The Plane of Water}
            The Plane of Water is an impossibly vast ocean.
            Powerful currents sweep through the ocean, but much of it is calm, and many forms of aquatic life abound in the water.
            The Plane of Water is the most densely populated Primal Plane, both by sentient creatures and monsters.
            Magnificant underwater cities are carved from huge rocks that float peacefully suspended in the water.
            Though there is no sun, simple creatures akin to plankton form the base of the food chain by feeding directly on the plane's essence.

            The Plane of Earth has the following planar traits:
            \begin{itemize}
                \item Gravity strength: No Gravity
                \item Light: None, though bioluminescent creatures like plankton are extremely common, making many parts of the plane well-lit
                \item Limits: Looped
                \item Planar connectivity: Stable Connected
                \item Shape: Uniform, in a sphere with a radius of about 1,000 miles.
            \end{itemize}

        \subsubsection{Nexus Planes}

            \parhead{The Material Plane}\label{The Material Plane}
            The Material Plane is the plane that most Rise adventures begin on.
            The surface of the plane is a massive sphere with a radius of about 4,000 miles.
            It is the most familiar to most humanoid creatures.

            The Material Plane has the following planar traits:
            \begin{itemize}
                \item Gravity direction: Absolute Directional, pointing to the center of the sphere
                \item Gravity strength: Normal
                \item Light: Mobile Source, from a sun and moon outside of the plane's limits
                \item Limits: Looped
                \item Planar connectivity: Isolated
                \item Shape: Solid Sphere, with a radius of about 4,000 miles.
            \end{itemize}

        \parhead{The Astral Plane}\label{The Astral Plane}
            The Astral Plane is the space between the other planes.
            It is a necessary intermediate destination for virtually all planar journeys, as all \glossterm{planar rifts} lead to and from the Astral Plane.
            Most activity on the Astral Plane occurs in a space called the Inner Astral Plane, a massive but finite region where all planar rifts on the Astral Plane appear.
            % Are there any other infinite planes?
            However, unlike all other planes, the Astral Plane has no known limits to its extent, and may in fact be infinite.
            The area outside the Inner Astral Plane is known as the Deep Astral Plane, and few venture into those sparsely populated realms.
            % This should be more clearly defined
            The Deep Astral Plane has magical turbulence that interferes with long-range communication and transportation magic, making exploration difficult.

            The Astral Plane has the following planar traits:
            \begin{itemize}
                \item Directional gravity: Subjective
                \item Gravity strength: Normal
                \item Light: Fixed Source, from the infinite reaches of the Deep Astral Plane
                \item Limits: Infinite
                \item Planar connectivity: Conduit
                \item Shape: Uniform
            \end{itemize}


\section{Creatures and Objects}
    In the world of Rise, creatures and objects are meaningfully different.
    Many abilities affect only creatures or only objects.
    The difference between a creature and an object is defined as being agency.
    Creatures have agency, and objects do not.
    This is not the same as sentience or life, which either creatures objects may have.

    For example, zombies are nonsentient, nonliving creatures.
    Conversely, trees are a nonsentient, living objects.
    Some rare magic items can be made intelligent by magic, making them sentient, nonliving objects.
    Some unintelligent animals and magical beasts like ants and giant spiders are nonsentient, living creatures.

    \subsection{Animates}
        One type of entity in the world is both an object and a creature.
        Animates are a type of creature that are made of nonsentient matter given a semblance of life and sentience by some form of magic.
        Fire elementals, clay golems, and plant creatures like treants are all animates.
        Animates are considered to be both creatures and objects, and are affected fully by abilities that affect both.
