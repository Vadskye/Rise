\chapter{Optional Rules}

\section{Attributes}

    \subsection{Other Methods of Attribute Generation}
        Point buy offers the fairest and most customizable system for determining attribute scores, ensuring that players can be almost any character they want to be. However, some groups may wish to determine attribute scores differently. Other options are provided below.

        \subsubsection{Simple Random Point Buy}
            With this method, you have only a small degree of control over your attribute scores, but all characters generated in this way are equally powerful.
            As with the point buy method, all your attribute scores start at 0, and you get 15 points to distribute among your attribute scores.
            However, you do not have full control over how to distribute those points.

            For each attribute, starting with the attributes you care about most, roll 1d8.
            You spend that many points on that attribute, ignoring any extra points that can't be spent
            For example, if you roll a 4, you spend 3 points on the attribute, causing you to start with a 2.
            If you do not have enough points remaining to spend the amount indicated by the die roll, spend as many as you can and move on to the next attribute.

            If you have points remaining after rolling all of your attribute scores, you may distribute the points freely among your abilities, using the normal point buy rules.
            You cannot increase the starting value of any individual attribute by more than 1 during this stage.
            If any of your attributes start as a 0, you may choose to lower them to gain the normal benefits from having low attributes (see \pcref{Attribute Penalties}).

            To further limit your character creation options, you may choose to randomize the order in which you roll your attributes instead of rolling them in an order of your choice.

        \subsubsection{Smoothed Random Point Buy}
            This method functions like the Simple Random Point Buy method, except that the resulting attribute values have a smoother distribution, and you can randomly end up with attribute penalties.

            For each attribute, starting with the attributes you care about most, roll 4d6.
            Then, remove any one of the rolls after seeing the results.
            Sum the results of the remaining three dice and spend the appropriate number of attribute points as indicated in \trefnp{Smoothed Random Point Buy Results}.
            If you do not have enough points remaining to spend the amount indicated by the die roll, spend as many as you can and move on to the next ability.

            If you have points remaining after rolling all of your attribute scores, you may distribute the points freely among your abilities, using the normal point buy rules.
            You cannot increase the starting value of any individual attribute by more than 1 during this stage.

            To further limit your character creation options, you may choose to randomize the order in which you roll your attributes instead of rolling them in an order of your choice.

            \begin{dtable}
                \lcaption{Smoothed Random Point Buy Results}
                \begin{dtabularx}{\columnwidth}{X X X}
                    \tb{Roll} & \tb{Base Attribute} & \tb{Point Cost} \tableheaderrule
                    3-4       & \minus2             & 0\fn{1} \\
                    5-6       & \minus1             & 0\fn{2} \\
                    7-8       & 0                   & 0       \\
                    9-10      & 1                   & 1       \\
                    11-12     & 2                   & 3       \\
                    13-15     & 3                   & 5       \\
                    16-18     & 4                   & 8       \\
                \end{dtabularx}
                1 You gain one \glossterm{insight point}. \\
                2 You gain an additional \glossterm{trained skill}. \\
            \end{dtable}

        \subsubsection{Classic Hardcore}

            This method is completely random and can generate very overpowered or underpowered characters.
            It represents the unfairness of the world, where some people are just better or worse than others.
            For each attribute, roll 2d6, take the average (rounded down), and subtract 2.
            If you roll a 1 on both dice, treat the average as a 0.
            The result is your base value for that attribute.

\section{Epic Fates}
    After 21st level, characters no longer gain levels normally.
    However, they can still increase their personal power as they make progress towards their ultimate fate.

    When you reach 21st level, you may choose an epic fate that you qualify for, or you may delay choosing until you meet the prerequisites for your desired fate.
    You do not start with any ranks in you chosen epic fate.
    Each epic fate specifies ways that you can make progress towards that epic fate.
    Whenever you make dramatic progress towards your epic fate, your rank in that epic fate may increase, at the discretion of the Game Master.

    None of the epic fate abilities have a tag to indicate that they are \magical abilities.
    Many of them are not fundamentally \glossterm{mundane} in nature, but they are beyond normal magic, and effects like an \spell{antimagic field} cannot interact with or suppress them.

    \subsection{Artificial Immortality}
        You have sought out strange magical power in search of a way to artificially prolong your life.
        As your power grows, you become increasingly able to resist death and return from it.
        Eventually, you will transcend death entirely.

        \parhead{Prerequisites} You must perform a series of rituals to prepare yourself for immortality, at least one of which must be rank 7 or higher. There are many kinds of immortality that you can pursue with this epic fate, and the exact nature of the rituals will change depending on the type of immortality you pursue.

        \parhead{Progression} You must discover powerful new magic rituals that support your particular form of immortality. This generally requires exploring sites of ancient magic, gaining favor with powerful creatures who have relevant knowledge or abilities, and independent experimentation based on your findings.

        \subsubsection{Artifical Immortality Ranks}

            \parhead{Rank 1 -- Life After Death} If you die from any cause other than old age, you resurrect according to nature of your chosen path to immortality.
            For example, you can have a phylactery regenerate a new body for you like a lich, or you can create clones of yourself or golems that you inhabit if your first body dies.
            You must always return in a new body of some sort.

            Your specific form of immortality determines where you return, such as at the site of your death or at your personal sanctum.
            However, it cannot cannot be based on the location or state of your old corpse, since that corpse is no longer ``you''.
            The timing of your resurrection may also differ based on your immortality, but you cannot complete your resurrection sooner than one day after the time of your death. After you resurrect in this way, this ability does not function for one week, allowing you to be killed normally.

            This immortality may change your base species, such as if you become a lich or move your body into a flesh golem. If it does, you retain all benefits and modifiers from your original species other than size, and you gain the effects of the new species in addition.

            \parhead{Rank 2 -- Death Familiarity} You become so familiar with the trauma of injury and death that your mind and body adapt to it.
            You gain a \plus10 bonus to vital rolls.
            In addition, the time of your vulnerability to true death after resurrection is reduced to 48 hours.

            \parhead{Rank 3 -- Artificial Life} Whenever you resurrect with your \textit{life after death} ability, your new body gains a \plus2 bonus to two random attributes. The attributes are randomized differently for each new body. In addition, that resurrection functions even if the cause of your death was old age, and you can control the physical age of your new body.

            \parhead{Rank 4 -- Deathcaller} You are deeply familiar with death, and know how to most effectively inflict it on others.
            Whenever you cause a living creature to lose at least half its hit points in a single round, you may kill that creature outright.
            In addition, your \textit{artificial life} ability grants a bonus to three random attributes instead of two random attributes.

            \parhead{Rank 5 -- True Immortality} You become fully immortal. There is no time limit after the resurrection from your \textit{life after death} ability where you become vulnerable to a true death. In addition, the resurrection can complete as quickly as one minute after your death. If a physical component limits your immortality, such as a phylactery, it can no longer be damaged or destroyed without the direct intervention of a rank 5 Slayer.

    \subsection{Ascendant}
        You have begun to see through the weave of the world and glimpse the higher truths beyond.
        As your insight into the true nature of reality grows, you begin to transcend the physical realm.
        Eventually, you become a being of pure energy.

        \parhead{Prerequisites} You must have spent at least a week living in each of the following planes: Air, Astral, Earth, Fire, Material, and Water.
        In addition, you must have an Intelligence or Perception of at least 2.

        \parhead{Progression} You must discover and spend time in exotic environments with unusual properties, especially with energy-related phenomena, to discern the underlying structure of the universe revealed in extremes.
        This involves a mix of meditation, observation, and potentially dangerous personal experience.
        Discovering potentially valuable locations may require extensive research.
        In order to reach the highest ranks, you must journey into forbidden realms of powerful magic, like the inner sanctums of major deities or the horrific depths of the Far Realm.

        \subsubsection{Ascendant Ranks}

            \parhead{Rank 1 -- Energetic Soul} Whenever you deal damage, you can treat at as energy damage in addition to its other types.
            In addition, whenever you take or deal energy damage while not \trait{incorporeal}, you and your equipment \glossterm{briefly} become incorporeal.
            While you are incorporeal in this way, you gain a \glossterm{fly speed} equal to your base speed with a \glossterm{height limit} of 60 feet (see \pcref{Flight}).
            Your \glossterm{maneuverability} with this fly speed is perfect (see \pcref{Flying Maneuverability}).

            \parhead{Rank 2 -- See Through the Weave} You can see everything within 120 feet of you perfectly, regardless of obstacles of any kind or light levels.
            This is similar to \trait{blindsight}, except that it also ignores solid obstacles of any kind, allowing you to have \glossterm{line of sight} through walls.
            You can perceive the presence of obstacles just as well as you can see what lies behind them.

            \parhead{Rank 3 -- Reach Through the Weave} When you use any of your abilities, you can treat yourself as being up to 60 feet away from your true location.
            You do not need \glossterm{line of effect} to your chosen location.
            For example, this allows you to make melee attacks against creatures up to 60 feet away.
            This changes your \glossterm{line of effect}, but does not change your \glossterm{line of sight}.

            % Does this need to have an explicit Recover reminder? Seems like that just works as intended
            \parhead{Rank 4 -- Become Energy} You become permanently \trait{incorporeal}, along with any equipment you carry.
            The fly speed from your \textit{energetic soul} ability also becomes permanent.
            In addition, you no longer age and no longer have hit points.
            Instead, you gain a bonus to your \glossterm{damage resistance} equal to the number of hit points you would normally have from your level, base class, and Constitution.
            Other effects that would increase or decrease your maximum hit points have no effect on you.
            You gain vital wounds based on taking damage in excess of your damage resistance rather than in excess of your hit points, including the extra vital wounds for taking massive damage.

            \parhead{Rank 5 -- Ascension} You cannot be killed, only dissipated.
            When you die, you automatically reform at a random location within a mile of your death after 10 minutes.
            Reforming in this way returns you to full damage resistance and removes all conditions and vital wounds, but your \glossterm{fatigue} and other effects remain the same.

    \subsection{Deity}
        People have begun to worship you, putting you on the path to become a deity.
        As your followers grow, you become capable of ever greater miraculous acts, and you can grant your followers some of your power.
        Eventually, you ascend into the pantheon of gods.

        \parhead{Prerequisites} You must have at least a hundred worshippers with souls to choose this epic fate.
        In addition, you must not have any cleric archetypes.

        \parhead{Progression} To progress towards this epic fate, you must gain a significant number of additional worshippers.
        In general, you must at least double your worshippers to progress towards each new rank of this fate, though this can vary widely.
        Having worshippers among many different places is more valuable than converting an isolated group to worship you, though both are helpful.

        \subsubsection{Deity Ranks}
            \parhead{Rank 1 -- Domain Influence} Choose a cleric domain.
            You gain all abilities from that domain except for its mastery ability.
            In addition, your worshippers become eligible to gain cleric archetypes, though they cannot exceed a maximum rank in those archetypes of twice your rank in this epic fate (to a maximum of 8).
            This does not grant additional archetypes to worshippers who have already chosen their three archetypes, and is usually only relevant to NPC worshippers.

            \parhead{Rank 2 -- Prayers} You hear all prayers directed to you.
            Once per week, you can teleport yourself and up to ten \glossterm{allies} any distance within the same plane as a \glossterm{standard action}.
            Your destination must either be a worshipper actively praying to you or a holy place dedicated to you.
            In addition, choose a second cleric domain.
            You gain all abilities from that domain except for its mastery ability.

            \parhead{Rank 3 -- Domain Mastery} Choose a third cleric domain.
            You gain all abilities from that domain.
            In addition, you gain the mastery ability from the domains you chose with your \textit{domain influence} and \textit{prayers} abilities.

            \parhead{Rank 4 -- Demigod} You become a demigod.
            You no longer age normally, and you cannot die from old age.
            You become a planeforged native to an Aligned Plane matching your alignment.
            For details about the aligned planes, see the Tome of Guidance.
            While you are on that plane, you can teleport to any plane with your \textit{prayers} ability from this epic fate.
            In addition, you can use that teleportation ability once per hour instead of once per week.

            \parhead{Rank 5 -- Deification} You become a deity.
            You are transported to an Aligned Plane matching your alignment, and you gain divine dominion over an amount of territory in that plane.
            While you are in your territory, you can can freely reshape your territory with a thought to match your desires, and you are immune to all damage and \glossterm{conditions}.

            Regardless of which plane you are on, you can teleport to anywhere within your home plane as a \glossterm{standard action}.
            In addition, there is no limit on the number of times you can teleport with your \textit{prayers} ability from this epic fate.

    \subsection{Hero of Legend}
        You are widely known as a hero, rescuing those in need.
        As your deeds of heroism spread, you gain abilities to help you protect others.
        Although you will eventually die, your legend will live on, inspiring others to save people as you did.

        \parhead{Prerequisites} You must be publicly known to be involved with saving at least one major country or similarly large group of people from some sort of disaster to choose this epic fate.

        \parhead{Progression} To progress towards this epic fate, you must publicly contribute to saving large numbers of people from death or other major disasters in a way that builds your reputation.
        Reaching the higher ranks typically requires saving a significant fraction of a major plane from some sort of catastrophe.

        \subsubsection{Hero of Legend Ranks}
            \parhead{Rank 1 -- Worthy Hero} You and all \glossterm{allies} who can see or hear you are immune to being frightened and panicked.
            In addition, you gain a \plus100 bonus to your \glossterm{hit points}.

            \parhead{Rank 2 -- Heroic Intervention} As a \glossterm{free action}, you may choose any number of \glossterm{allies} within \shortrange of you.
            Whenever a chosen creature would be attacked that round, that attack is made against you instead.
            If the attack would have targeted both you and that ally, the attack only targets you once, not twice.
            This ability has the \abilitytag{Swift} tag.

            \parhead{Rank 3 -- Invincible Hero} You gain a \plus4 bonus to all defenses.
            In addition, you cannot be \vulnerable for any reason.

            \parhead{Rank 4 -- Answer the Call} You gain an intuitive sense for when people need your aid.
            Whenever someone on the same plane as you is in danger, you are aware of the existence of that danger.
            You can sense the general category of danger (fire, combat, drowning, etc.) and a very approximate direction and distance.
            This generally allows you to sense if a large number of people are in danger from the same thing.
            As a \glossterm{standard action}, you can teleport any distance within that plane to reach a person in danger.

            \parhead{Rank 5 -- Heroic Legacy} If you die, your legend lives on.
            You may choose a worthy successor, either before your death or after your death from your afterlife plane.
            When you die, or as soon as you choose a successor while dead, your successor immediately gains the Rank 1 benefit from this archetype.
            As long as your successor lives and remains worthy, you cannot choose to be resurrected from your afterlife.
            As a player, you may choose to play as your successor instead of your original character if the GM allows it.

    \subsection{Mutant}
        Your body has been altered by battle scars and strange experiments.
        As your mutations grow ever more extreme, you become more powerful - and more monstrous.
        Eventually, you can regenerate from death itself, though some scars never fade.

        \parhead{Prerequisites} Dangerous experiments to mutate your body in extreme ways must have been performed on you.
        You can choose the nature of these experiments, such as alchemical or magical.
        Some mutants do this to themselves, while others find willing collaborators.
        In addition, you must have a Constitution of at least 2.

        \parhead{Progression} To progress towards this epic fate, you must continue ever more radical forms of experimentation.
        As you become inured to ordinary alterations to your body, you must travel and research to find unique substances of immense power to fuel the experiments.
        To reach the higher ranks, you must undergo experiments that kill you far more often than they succeed, so you will need to be resurrected multiple times to continue down this path.

        \subsubsection{Mutant Ranks}
            \parhead{Rank 1 -- Unnatural Arsenal} You grow an extra functioning arm and hand.
            In addition, you gain a wide variety of natural weapons (see \tref{Natural Weapons}).
            You gain a bite, horn, ram, stinger, and tentacle.
            In addition, two of your hands become claws.
            You gain these weapons in addition to any natural weapons you already have, no matter how biologically implausible that may be.
            % TODO: mutants will often have Massive, so Sweeping is oddly redundant.
            In addition, each of your natural weapons has the \weapontag{Sweeping} (3) weapon tag in addition to its other weapon tags.

            \parhead{Rank 2 -- Regeneration} At the end of each round, you regain hit points equal to a quarter of your maximum hit points.
            In addition, you may increase the result of one of your \glossterm{vital wounds} by 1, to a maximum of 10.
            If you are unconscious, this automatically applies to your most severe vital wounds first.

            \parhead{Rank 3 -- Monstrous Form} Your size increases by one size category.
            In addition, you gain a wide variety of movement modes.

            You gain a \glossterm{climb speed} and \glossterm{swim speed} equal to your base speed, and a 10 foot \glossterm{burrow speed}.
            For each of those movement modes, if you already have it, you gain a \plus10 foot bonus to your speed with that movement mode instead.
            Wings also grow from your back, granting you a \glossterm{fly speed} equal to your base speed with a 60 foot \glossterm{height limit} (see \pcref{Flight}).

            \parhead{Rank 4 -- Two Heads Are Better Than One} You grow a second head.
            Whenever you gain a \glossterm{condition}, you choose which head gains the condition, with the restriction that the chosen head must not already have more conditions than the other head.
            At the start of each round, you can choose which heads are active during that round.
            You are only subject to the effects of the conditions affecting active heads.
            If you choose for both heads to be active, you can use an extra \glossterm{minor action} during the \glossterm{action phase}.

            \parhead{Rank 5 -- Regenerative Immortality} You can regenerate from any wounds, even lethal ones.
            When you die, your \textit{unnatural regeneration} ability continues functioning.
            Once that ability improves your vital wounds so they are all above 0, you return to life.
            However, each time you die, you gain a new scar that your regeneration always recreates.

            If your corpse is mutilated, burned, immersed in acid, or fully destroyed, this process can take much longer to complete, but it cannot be fully stopped.
            Some drop of blood, flake of skin, or other remnant of your corpse will always persist and regenerate eventually.
            If your body is separated into pieces while you are dead, each piece will attempt to regenerate individually.
            Your soul will automatically return to the first piece that regenerates completely, at which point the remaining fragments will wither and die.

    \subsection{Paradox}
        You exist partly outside of the ordinary flow of time.
        Your very existence wreaks havoc on prophecies and the orderly sequence of events.
        As your alterations to the natural timeline of the universe grow in scope, your ability to bend time to your whims grows in turn.
        Eventually, you become a fixed point across all of time.

        \parhead{Prerequisites} You must have been directly involved with an action that resulted in a significant change to at least one major country or similarly significant entity.
        Any type of change is acceptable, as long as it would be historically important would not have happened without your intervention.

        \parhead{Progression} To progress towards this epic fate, you must be alter the course of other major events that will be remembered to history.
        Reaching the higher ranks typically requires changing the fate of major planes, or creatures of similar importance.

        \subsubsection{Paradox Ranks}
            \parhead{Rank 1 -- Temporal Aberration} Your actions, and events involving you, cannot be observed in any effect that sees or predicts the future.
            This applies against both magical abilities and abilities that rely on direct observation.
            Prophecies are only able to describe how events would happen without your intervention, and are blind to any changes you might cause.

            In addition, whenever you would make a \glossterm{movement}, you can make two different movements and then decide which one was the movement you actually made.
            You can make this decision after observing how other creatures react to your movement, but before taking any other actions.

            The other movement never happened, and had no effect.
            Only the movement itself is reverted in this way.
            Any other abilities you used during the resolution of that movement, such as the \ability{sprint} ability, still happened, so you would still gain fatigue and resolve other effects.

            \parhead{Rank 2 -- A Fork In Time's Road} Whenever you would take a standard action, you can take two different standard actions and then decide which one was the action you actually took.
            You can make this decision after seeing all die results and observing all effects of both actions, but before taking any other actions.

            The other action never happened, and has no effect.
            Only the standard action itself is reverted in this way.
            Any other abilities you used during the resolution of that action, such as the \ability{desperate exertion} ability, still happened, so you would still gain fatigue and resolve other effects.

            \parhead{Rank 3 -- Choose Fate} Once per \glossterm{short rest}, whenever you or any other creature you are aware of rolls an attack or check, you can choose the result of that die.
            You can choose to use this ability after learning the result of the action using that die roll, including whether it succeeded or failed and the result of any damage dice based on the attack.
            However, you must use it before any other actions resolve.
            If you use this to make an attack \glossterm{explode}, subsequent dice after the die you modify in this way are rolled normally.
            Using this to affect an enemy's action may change the actions taken by other enemies in that enemy's \glossterm{allied group}, which the GM should resolve.

            \parhead{Rank 4 -- Paradoxical Defense} Whenever a creature attacks you, it must roll twice and take the lower result.
            This does not protect any other targets of the attack.

            \parhead{Rank 5 -- Fixed Point} You become an immutable fact across all of time.
            You no longer age.
            Whenever you die, history is rewritten so you retroactively never died instead.
            The changes are as subtle and believable as possible, but even extraordinary coincidences can occur to save you from death.
            You typically still end up unconscious from vital wounds, and are always removed from combat or otherwise unable to usefully act for at least ten minutes, but you survive.

    \subsection{Slayer}
        You are a killer of legendary skill.
        As your body count increases, you gain abilities to help you track down and kill increasingly powerful foes.
        Eventually, your powers threaten the gods themselves, allowing you a unique ability to transcend death.

        \parhead{Prerequisites} You must be directly involved with slaying at least one \glossterm{elite} creature with a level of at least 21.

        \parhead{Progression} To progress towards this epic fate, you must publicly contribute to slaying increasingly dangerous and fearsome foes.
        To reach the higher ranks, you must kill creatures of singular power whose influence is felt across multiple planes.
        This might include demon princes, supreme dragons beyond even the power of wyrms, or the nightmarish aberration progenitors in the Far Realm.

        \subsubsection{Slayer Ranks}
            \parhead{Rank 1 -- Lethality} You gain a \plus4 bonus to your accuracy for the purpose of determining whether you get a \glossterm{critical hit}.
            This bonus stacks with other abilities with the same effect, such as \weapontag{Keen} weapons.

            \parhead{Rank 2 -- Precision Killer} You gain a \plus4 bonus to your \glossterm{accuracy}.
            In addition, you can inflict \glossterm{critical hits} on any creature, regardless of its size, body structure, or other abilities.

            \parhead{Rank 3 -- Mark of the Slayer} As a \glossterm{minor action}, you can choose to mark any creature you can unambiguously identify.
            This includes any creature you can see, as well as any creature you know the name of and can differentiate from other similar creatures.
            You can only mark one creature at a time, and applying a new mark replaces any previous mark.
            You cannot use this ability to replace a mark that is less than a week old if the recipient of the previous mark still lives.

            This mark is visible on the creature's body with a design that is recognizably yours.
            It appears on top of any clothing or other attempt to conceal it, even if the creature is invisible.
            Anyone can recognize the significance of the mark with a \glossterm{difficulty value} 15 Knowledge (arcana or local) check, and creatures that understand the significance of the mark may refuse to give your target aid of any kind to avoid risking your wrath.

            You know the exact distance and direction to any creature you have marked with this ability that is on the same plane as you.
            As a \glossterm{standard action}, you can create a \glossterm{scrying sensor} adjacent to them that you can see and hear through.
            The sensor lasts as long as you \glossterm{sustain} it as a \glossterm{free action}.
            It moves to stay adjacent to the target, regardless of its speed.

            \parhead{Rank 4 -- Slayer's Journey} As a \glossterm{standard action}, you can \glossterm{teleport} yourself and up to ten \glossterm{allies} any distance within the same plane to the location of a creature affected by your \textit{mark of the slayer} ability from this epic fate.
            You cannot precisely choose the destination of this ability, and it does not leave you immediately adjacent to the marked creature.
            Generally, it leaves you just outside any sort of fortress or defenses the marked creature has constructed.
            After you use this ability, you cannot use it to travel to the same creature for a day.
            This does not limit your ability to travel to a different creature if you mark a different creature.

            \parhead{Rank 5 -- Godslayer}
            Damage from your attacks ignores all forms of invulnerability and immunity.
            You can deal fire damage to fire elementals, physical damage to ghosts, and so on.
            In addition, you can destroy artifacts and even inflict damage on deities in their divine dominion.
            As a result, even deities fear to interfere with you directly.
            If you ever die, you can generally threaten or fight your way past any planar guardians to leave your afterlife whenever you want.
            After you do this once, you become a planeforged native to your afterlife plane, since your new body is formed from the raw material of that plane.

\section{Uncommon Species}\label{Uncommon Species}

    \subsection{Animal Hybrid}
        Animal hybrids are humanoid creatures that are a combination of humans and animals.
        The abilities of an animal hybrid depend on the type of animal it is based on.

        \parhead{Size} Medium.
        \parhead{Attributes} The attributes of an awakened animal depend on its size.
        \parhead{Special Abilities} As the original animal.
        \parhead{Automatic Languages} Common and any one \glossterm{common language} (see \tref{Common Languages}).

        \subsubsection{Sample Animal Hybrids}

            \parhead{Hybrid Bee}

            \subparhead{Special Abilities}
            \parhead{Attribute} \plus1 Dexterity, \minus1 Constitution.
            \begin{itemize}
                \itemhead{Low-light Vision} A hybrid bee has \trait{low-light vision}, allowing it to see clearly in \glossterm{shadowy illumination} (see \pcref{Low-light Vision}).
                \itemhead{Stinger} A hybrid bee has a stinger natural weapon (see \pcref{Natural Weapons}).
                    Whenever it causes a creature to lose \glossterm{hit points} with that natural weapon, the struck creature is poisoned by giant wasp venom (see \pcref{Poison}).
                    Its stage 1 effect makes the target \slowed while the poison lasts.
                    Its stage 3 effect makes the target \immobilized while the poison lasts.
                \itemhead{Winged Agility} A hybrid bee has wings that are not strong enough to help it fly.
                    However, the wings still help it stabilize its movements.
                    It gains a \plus3 bonus to the Balance and Jump skills, and it gains a \plus5 foot bonus to its maximum horizontal jump distance (see \pcref{Jumping}).
                    This increases its maximum vertical jump distance normally.
            \end{itemize}

            \parhead{Hybrid Shark}

            \subparhead{Special Abilities}
            \begin{itemize}
                \itemhead{Bloodscent} A hybrid shark has the scent ability (see \pcref{Scent}).
                    In addition, it gains a \plus10 bonus to Awareness checks to detect blood.
                \itemhead{Bite} A hybrid shark's mouth is elongated, which it can use as a bite attack (see \pcref{Natural Weapons}).
                    A hybrid shark's bite deals 1d6 damage.
                \itemhead{Gills} You can breathe water as easily as a human breathes air, preventing you from drowning or suffocating underwater.
                \itemhead{Swim Speed} A hybrid shark has a swim speed equal to the base speed for its size.
            \end{itemize}

            \parhead{Hybrid Wolf}

            \subparhead{Special Abilities}
            \begin{itemize}
                \itemhead{Scent} A hybrid wolf has the scent ability (see \pcref{Scent}).
                \itemhead{Bite} A hybrid wolf's mouth is elongated, which it can use as a bite attack (see \pcref{Natural Weapons}).
                    A hybrid wolf's bite deals 1d6 damage.
                \itemhead{Low-light Vision} A hybrid wolf has \trait{low-light vision}, allowing it to see clearly in \glossterm{shadowy illumination} (see \pcref{Low-light Vision}).
            \end{itemize}

    \subsection{Automaton}
        An automaton appears to be a humanoid construct, like a golem.
        Its body is made from some combination of stone, wood, and metal.
        However, its artificial body is inhabited by a true soul, making it an indwelt (see \pcref{Indwelt}).

        \parhead{Size} Medium.
        \parhead{Attributes} \plus1 Constitution or Intelligence, \minus1 Dexterity.
        \parhead{Special Abilities}
        \begin{itemize}
            \itemhead{Artificial Life} Automatons are not alive. They ignore abilities which only affect living creatures, including poisons and most healing abilities. In addition, they do not need to eat, drink, or sleep.
            \itemhead{Automaton Archetype} Automatons only gain two class archetypes instead of three.
                Instead, they treat the Automaton archetype as one of their archetypes, and they gain ranks in it just like they gain ranks in class archetypes.
            \itemhead{Manual Repair} A Craft skill relevant to the automaton's body can be used to achieve the same effects that the Medicine skill would have on a living creature.
            \itemhead{Mechanical Body} Automatons are considered both objects and creatures, and are affected by abilities which affect either.
                They are always considered to be \glossterm{attended} by themselves, so they are never affected by abilities that only affect unattended objects, even while unconscious.
            \itemhead{Mechanical Intuition} Automatons gain a \plus2 bonus to the Devices skill and one Craft skill of their choice.
        \end{itemize}

        \subsubsection{Automaton Archetype}

            \cf{Aut}[1]{Modular Carapace} You can adjust the density and layering of your hardened exterior to augment your defenses.
            Changing your configuration in this way requires 10 minutes of work, and spare armor parts that you generally keep with you.
            You can choose to treat your carapace as being light, medium, or heavy armor.
            The benefits from this ability are considered to come from body armor, and do not stack with actual body armor.

            \begin{itemize}
                \item Light armor: You gain a \plus3 bonus to your Armor defense, and a damage resistance bonus equal to three times your rank in this archetype.
                    You can wear body armor on top of this carapace.
                    Although the benefits of that armor do not stack with the carapace, you can use the higher Armor defense value and damage resistance bonus from either armor.
                \item Medium armor: You gain a \plus4 bonus to your Armor defense, and a damage resistance bonus equal to four times your rank in this archetype.
                    However, your Dexterity bonus to your Armor defense is halved, and you cannot wear body armor.
                \item Heavy armor: You gain a \plus5 bonus to your Armor defense, and a damage resistance bonus equal to five times your rank in this archetype.
                    However, your Dexterity bonus to your Armor defense is removed, you take a \minus10 foot penalty to your speed with all movement modes, and you cannot wear body armor.
            \end{itemize}

            If you lose your original armor parts, you can create or buy new parts that are suited to your body.
            These parts are considered a Rank 1 (40 gp) item.

            \cf{Aut}[2]{Modular Weaponry} You can customize your arms to augment your combat prowess.
            Changing your configuration in this way requires 10 minutes of work, and spare arm and weapon parts that you generally keep with you.
            \begin{itemize}
                % Two improvements over baseline
                \item Blade: You convert one of your arms into a blade natural weapon.
                    It has a \plus1 accuracy bonus, deals 1d8 slashing damage, and has the \weapontag{Keen} and \weapontag{Sweeping} (1) weapon tags (see \pcref{Weapon Tags}).
                    However, that arm no longer has a \glossterm{free hand}.
                % Two improvements over baseline
                \item Bulk: You convert one of your arms into a bulky, overburdened slam natural weapon.
                    It deals 1d10 bludgeoning damage and has the \weapontag{Impact} and \weapontag{Resonating} weapon tags.
                    However, that arm no longer has a \glossterm{free hand}.
                \item Fortified: You add additional protective plating to your arms.
                    You gain a \plus1 bonus to your Armor defense.
                    This does not require a \glossterm{free hand}, but it is still considered to come from a shield, and it does not stack with the benefit from using a shield.
                \item Slim: You trim away the excess protection from your arms to make their movements more precise.
                    You gain a \plus1 accuracy bonus with any attack that uses your hands, including \glossterm{strikes} and abilities like \ability{grapple}.
                    This does not apply to spells that simply have \glossterm{somatic components}, though it does apply to touch spells like \spell{freezing grasp}.
                    However, you take a \minus1 penalty to your Armor defense.
            \end{itemize}

            If you lose your original arm and weapon parts, you can create or buy new parts that are suited to your body.
            These parts are considered a Rank 1 (40 gp) item.

            \magicalcf{Aut}[3]{Imbued Carapace} When you gain this ability, choose one magic body armor property for each of your three \ability{modular carapace} types (see \tref{Magic Armor}).
            The property must be rank 3 or lower.
            Whenever you use a carapace type, you can gain the benefit of its magic armor property, though you must still \glossterm{attune} to the property.
            Each carapace can have a unique property, or you can choose the same property for multiple carapace types.
            If you are using a light carapace, you can gain the benefit of its magic armor property in addition to any magic armor you wear on top of the carapace.

            Whenever you gain a rank in this archetype, you can choose a new property for each of your carapace types.
            Each property's rank must be no higher than your rank in this archetype.
            Magic armor of rank 3 or higher does not multiply the damage resistance provided by your carapace, as magic armor normally does (see \pcref{Magic Armor Damage Resistance}).

            \magicalcf{Aut}[4]{Embedded Armory} Whenever you use your \ability{modular weaponry} ability to change your configuration, you can also embed one magic weapon, implement, glove, gauntlet, or bracer into one of your arms.
            You can attune to that item without having it physically present on your arm.
            If you attune to a weapon embedded this way, you apply its magical properties to any nonmagical weapons you use with that arm, including natural weapons and weapons made from special materials.
            You cannot embed a \glossterm{legacy item} in this way.

            \cf{Aut}[5]{Modular Carapace+} The damage resistance from your \textit{modular carapace} armor increases.
            Light carapace increases to four times your rank in this archetype, medium carapace increases to five times your rank, and heavy carapace increases to seven times your rank.

            \cf{Aut}[5]{Reassembly} You can recover from vital wounds more easily by simply replacing broken parts.
            You can remove a vital wound with ten minutes of work.
            This increases your \glossterm{fatigue level} by three, and it requires replacement parts that you generally keep with you.
            The parts are considered a consumable Rank 3 (200 gp) item.

            This can even save you from death, though that is more difficult and requires more advanced parts.
            A creature can spend eight hours replacing broken parts of your corpse to \glossterm{resurrect} you (see \pcref{Resurrection}).
            % Baseline trained bonus at rank 5 is +9.
            This requires a \glossterm{difficulty value} 15 Craft check appropriate to the composition of your body.
            The parts required to perform this feat are considered a consumable Rank 5 (5,000 gp) item.

            \cf{Aut}[6]{Artificial Mind} You become immune to \abilitytag{Compulsion} and \abilitytag{Emotion} attacks.

            \cf{Aut}[7]{Hypercognition} You can take an additional \glossterm{minor action} each round, as long as that minor action is purely mental.
            As normal, you cannot use the same ability twice in the same round.

            Alternately, you can take an additional purely mental standard action instead of a minor action.
            When you do, you increase your \glossterm{fatigue level} by one and you \glossterm{briefly} cannot take any additional actions with this ability.

        \subsubsection{Base Class Abilities}
            If you choose automaton as your base class, you gain the following abilities.

            \cf{Aut}{Defenses}
            You gain the following bonuses to your \glossterm{defenses}: \plus4 Fortitude, \plus2 Reflex, \plus3 Mental.

            \cf{Aut}{Hit Points}
                You have 8 hit points \add twice your Constitution, plus 2 hit points per level beyond 1.
                This increases as your level increases, as indicated below.
                \begin{itemize}
                    \itemhead{Level 7} 20 hit points \add three times your Constitution, plus 3 hit points per level beyond 7.
                    \itemhead{Level 13} 40 hit points \add six times your Constitution, plus 6 hit points per level beyond 13.
                    \itemhead{Level 19} 80 hit points \add twelve times your Constitution, plus 12 hit points per level beyond 19.
                \end{itemize}

            \cf{Aut}{Resources} You have the following \glossterm{resources}:
            \begin{itemize}
                \item Three \glossterm{attunement points}, which you can use to attune to items and abilities that affect you (see \pcref{Attunement Points}).
                \item A \glossterm{fatigue tolerance} equal to 4 \add your Constitution.
                    Your fatigue tolerance makes it easier for you to use powerful abilities that fatigue you (see \pcref{Fatigue}).
                \item A number of \glossterm{insight points} equal to 1 \add your Intelligence.
                    You can spend insight points to gain additional abilities (see \pcref{Insight Points}).
                \item Three \glossterm{trained skills} from among your \glossterm{class skills}, plus additional trained skills equal to your Intelligence (see \pcref{Skills}).
            \end{itemize}

            \cf{Aut}{Weapon Proficiencies} 
            You are proficient with simple weapons and one weapon group of your choice.

            \cf{Aut}{Armor Proficiencies} 
            You are not proficient with armor.

            \cf{Aut}{Skills}
            You have the following \glossterm{class skills}:
            \begin{itemize}
                \itemhead{Strength} Climb.
                \itemhead{Dexterity} Balance, Flexibility.
                \itemhead{Constitution} Endurance.
                \itemhead{Intelligence} Craft (any), Deduction, Devices, Disguise.
                \itemhead{Perception} Awareness.
            \end{itemize}

    \subsection{Awakened Animal}

        Awakened animals are animals that have been granted sentience by the \spell{awaken} ritual.
        The abilities of an awakened animal depend on the type of animal it is.

        \parhead{Size} Small or Medium, as original animal.
        \parhead{Attributes} The attributes of an awakened animal depend on its size.
        \subparhead{Medium} No change.
        \subparhead{Small} \minus2 Strength, \plus1 Dexterity.
        \parhead{Special Abilities} As the original animal.
        \parhead{Automatic Languages} Common.

        \subsubsection{Sample Awakened Animals}

            \parhead{Cat}

            \subparhead{Size} Small. This gives a cat a 20 foot \glossterm{base speed} and a \plus5 bonus to the Stealth skill, among other effects (see \pcref{Size Categories}).
            \subparhead{Attributes} \minus2 Strength, \plus1 Dexterity
            \subparhead{Special Abilities}
            \begin{itemize}
                \itemhead{Claws} A cat's paws end in claws, which it can use to attack (see \pcref{Natural Weapons}). A cat's claws have a \plus2 accuracy bonus and deal 1d4 damage.
                \itemhead{Low-light Vision} A cat has \trait{low-light vision}, allowing it to see clearly in \glossterm{shadowy illumination} (see \pcref{Low-light Vision}).
                \itemhead{Multipedal} A cat is \trait{multipedal}, which gives it a \plus10 foot bonus to its \glossterm{land speed} and a \plus5 bonus to Balance.
                \itemhead{Scent} A cat has the scent ability (see \pcref{Scent}).
            \end{itemize}

    \subsection{Changeling}

        \parhead{Size} Medium.
        \parhead{Attributes} No change.
        \parhead{Special Abilities}
        \begin{itemize}
            \itemhead{Alter Shape} A changeling can alter its physical form in minor ways. As a standard action, a changeling can make a Disguise check with a \plus10 bonus to alter its body. This ability does not alter the changeling's equipment, which may give away its identity unless disguised normally.

            This is a \magical ability.
        \end{itemize}
        \parhead{Bonus Languages} Any.
        \parhead{Automatic Languages} Common, any two \glossterm{common languages}.

    \subsection{Dragon}
        Ancient dragons are magical creatures of immense power and wisdom, and are far more powerful than any ordinary character of the same level.
        However, young dragons can be played as characters, though their unique abilities do pose unique challenges.

        \parhead{Creature Type} Unlike most other playable species, dragons are magical beasts instead of humanoids.
        \parhead{Size} Small. This gives a dragon a 20 foot \glossterm{base speed} and a \plus5 bonus to the Stealth skill, among other effects (see \pcref{Size Categories}).
        \parhead{Attributes} \minus2 Strength, \plus1 Dexterity.
        \parhead{Special Abilities}
        \begin{itemize}
            \itemhead{Dragon Archetype} Dragons only gain two class archetypes instead of three.
                Instead, they treat the Dragon archetype as one of their archetypes, and they gain ranks in it just like they gain ranks in class archetypes.
            \itemhead{Draconic Senses} Dragons have \trait{darkvision} with a 60 foot range, allowing them to see in complete darkness (see \pcref{Darkvision}).
                In addition, dragons gain \trait{low-light vision}, allowing them to see clearly in \glossterm{shadowy illumination} (see \pcref{Low-light Vision}).
            \itemhead{Draconic Scales} Dragons gain a \plus2 bonus to their Armor defense.
            \itemhead{Draconic Weapons} Dragons have a bite natural weapon and two claw natural weapons.
                For details, see \pcref{Natural Weapons}.
            \itemhead{Draconic Wings} Dragons have scaly wings that sprout from their backs.
                These wings grant them a glide speed equal to the \glossterm{base speed} for their size (see \pcref{Gliding}).
                The wings themselves are \glossterm{mundane}, but the ability to fly and glide with them is \magical.
            \itemhead{Dragon Type} Each dragon has a single type from among the dragon types on \trefnp{Dragon Types}.
                They are immune to the damage type dealt by their type's breath weapon.
            \itemhead{Limited Equipment} A dragon's claws are not able to effectively wield shields or manufactured weapons.
                They can wear armor, but it is treated as \glossterm{barding} instead of normal armor, reducing its effectiveness (see \pcref{Barding}).
        \end{itemize}
        \parhead{Automatic Languages} Common, Draconic, any one \glossterm{common language}.

        \begin{dtable}
            % Don't use lcaption because there is already a Dragon Types table with a label
            \caption[]{Dragon Types}
            \begin{dtabularx}{\columnwidth}{l >{\lcol}X >{\lcol}X}
                \tb{Dragon} & \tb{Damage Type} & \tb{Breath Weapon} \tableheaderrule
                Black       & Acid             & \areamed, 5 ft. wide line \\
                Blue        & Electricity      & \areamed, 5 ft. wide line \\
                Brass       & Fire             & \areamed, 5 ft. wide line \\
                Bronze      & Electricity      & \areamed, 5 ft. wide line \\
                Copper      & Acid             & \areamed, 5 ft. wide line \\
                Gold        & Fire             & \areasmall cone           \\
                Green       & Acid             & \areasmall cone           \\
                Red         & Fire             & \areasmall cone           \\
                Silver      & Cold             & \areasmall cone           \\
                White       & Cold             & \areasmall cone           \\
            \end{dtabularx}
        \end{dtable}

        \subsubsection{Dragon Archetype}

            \cf{Dgn}[1]{Draconic Breath} You can use the \textit{breath weapon} ability as a \glossterm{standard action}.
            % +1r damage to compensate for cooldown... ish. Kind of awkward scaling right now.
            \begin{activeability}{Breath Weapon}
                \rankline
                Make an attack vs. Reflex against everything in the area defined by your dragon type (see Dragon Types, above).
                After you use this ability, you \glossterm{briefly} cannot use it again.
                \hit \damageranktwo{}.
                The damage type is defined by your dragon type.
                \miss Half damage.

                \rankline
                % T2 area
                \rank{2} The area increases.
                    A line breath weapon becomes a \arealarge, 5 ft.\ wide line.
                    A cone breath weapon becomes a \areamed cone.
                \rank{3} The damage increases to \damagerankthree{}.
                \rank{4} The damage increases to \damagerankfour{}.
                % T4 area
                \rank{5} The area increases.
                    A line breath weapon becomes a \areahuge, 10 ft.\ wide line.
                    A cone breath weapon becomes a \arealarge cone.
                \rank{6} The damage increases to \damageranksix{}.
                \rank{7} The damage increases to \damagerankseven{}.
            \end{activeability}

            \cf{Dgn}[2]{Draconic Body} You gain a \plus1 bonus to your Armor defense.

            \magicalcf{Dgn}[3]{Draconic Flight} Your wings grow larger, granting you a limited ability to fly.
            You gain a \glossterm{fly speed} equal to the \glossterm{base speed} for your size with a maximum height of 15 feet (see \pcref{Flight}).
            As a \glossterm{free action}, you can increase your \glossterm{fatigue level} by one to ignore this height limit until the end of the round.

            \cf{Dgn}[4]{Draconic Bulk} Your size category increases to Medium.
            This increases your \glossterm{base speed} to 30 feet.
            You reduce your Dexterity by 1 and increase your Strength by 2.
            In addition, you gain a \plus1 bonus to your \glossterm{power} with all abilities.

            \cf{Dgn}[5]{Draconic Body+} The Armor bonus from your \textit{draconic body} ability increases to \plus2.

            \magicalcf{Dgn}[6]{Draconic Flight+} The maximum height from your \textit{draconic flight} ability increases to 60 feet.
            In addition, you gain a \plus10 foot bonus to your fly speed with that ability.

            \cf{Dgn}[7]{Draconic Bulk+} Your size category increases to Large.
            This increases your \glossterm{base speed} to 40 feet.
            In addition, the attribute modifiers to Dexterity and Strength increase to \minus2 and \plus3 respectively, and the power bonus increases to \plus2.
            % Natural dragons treat their tail slam as heavy, but that may not work for a PC? They also treat their Bite as heavy, which these dragons don't do.
            You also gain a tail slam \glossterm{natural weapon}.
            It deals 1d10 bludgeoning damage and has the \weapontag{Impact} weapon tag (see \pcref{Weapon Tags}).
            % TODO: too fast?
            In addition, you gain a \plus20 foot bonus to your fly speed with your \textit{draconic flight} ability, but your maneuverability drops to poor maneuverability (see \pcref{Flying Maneuverability}).

        \subsubsection{Base Class Abilities}
            If you choose dragon as your base class, you gain the following abilities.

            \cf{Drg}{Defenses}
            You gain the following bonuses to your \glossterm{defenses}: \plus4 Fortitude, \plus2 Reflex, \plus3 Mental.

            \cf{Drg}{Hit Points}
                You have 8 hit points \add twice your Constitution, plus 2 hit points per level beyond 1.
                This increases as your level increases, as indicated below.
                \begin{itemize}
                    \itemhead{Level 7} 20 hit points \add three times your Constitution, plus 3 hit points per level beyond 7.
                    \itemhead{Level 13} 40 hit points \add six times your Constitution, plus 6 hit points per level beyond 13.
                    \itemhead{Level 19} 80 hit points \add twelve times your Constitution, plus 12 hit points per level beyond 19.
                \end{itemize}

            \cf{Drg}{Resources} You have the following \glossterm{resources}:
            \begin{itemize}
                \item Three \glossterm{attunement points}, which you can use to attune to items and abilities that affect you (see \pcref{Attunement Points}).
                \item A \glossterm{fatigue tolerance} equal to 3 \add your Constitution.
                    Your fatigue tolerance makes it easier for you to use powerful abilities that fatigue you (see \pcref{Fatigue}).
                \item A number of \glossterm{insight points} equal to 2 \add your Intelligence.
                    You can spend insight points to gain additional abilities (see \pcref{Insight Points}).
                \item Three \glossterm{trained skills} from among your \glossterm{class skills}, plus additional trained skills equal to your Intelligence (see \pcref{Skills}).
            \end{itemize}

            \cf{Drg}{Weapon Proficiencies} 
            You are not proficient with any weapon groups, even simple weapons.
            You are still proficient with your natural weapons.

            \cf{Drg}{Armor Proficiencies} 
            You are proficient with light and medium armor.
            Armor shaped appropriately for dragons can be hard to find, and may need to be crafted individually for the dragon.

            \cf{Drg}{Skills}
            You have the following \glossterm{class skills}:
            \begin{itemize}
                \item \subparhead{Strength} Climb, Swim.
                \item \subparhead{Dexterity} Balance, Stealth.
                \item \subparhead{Constitution} Endurance.
                \item \subparhead{Intelligence} Craft, Deduction, Knowledge (arcana), Medicine.
                \item \subparhead{Perception} Awareness, Creature Handling, Social Insight, Survival.
                \item \subparhead{Other} Deception, Intimidate, Persuasion.
            \end{itemize}

    \subsection{Drow}

        Drow are an offshoot group of elves that live deep underground.
        The deep caves are a far harsher environment than the surface world.
        Resources are scarce, and dangerous monsters are far more common.
        In order to survive, drow were forced to adopt a variety of practices condemned by surface civilizations.
        The most notorious are their frequent use of poison, their refusal to take prisoners, their willingness to eat any non-drow creatures they kill, even sentient creatures.
        In addition, drow society tends to reward selfishness and ambition more explicitly than surface civilizations, and the vast majority of drow are evil.

        When drow find opportunities to reach the surface world, they seek to conquer territory for themselves, usually with great violence.
        They have always been defeated and banished back to their caves, but surface civilizations still remember the danger that drow pose.
        Even more so than tieflings or orcs, who are already viewed with suspicion, drow are anathema in almost any civilized society.
        Drow who escape the deep caves are more likely to find a peaceful existence on other planes that do not fear an underground invasion.

        \parhead{Size} Medium.
        \parhead{Attributes} \minus1 Constitution, \plus1 Dexterity
        \parhead{Special Abilities}
        \begin{itemize}
            \itemhead{Darkvision} Drow have \trait{darkvision} with a 120 foot range, allowing them to see in complete darkness (see \pcref{Darkvision}).
            \itemhead{Deep Darkness}[\sparkle] A drow can use the \textit{deep darkness} ability as a \glossterm{standard action}.
                \begin{magicalsustainability}{Deep Darkness}{\abilitytag{Sustain} (minor)}
                    \rankline
                    \target{One \glossterm{zone} within \rngmed range}
                    You can choose this ability's radius, up to a maximum of a \areamed radius.
                    Light within or passing through the area is dimmed to be no brighter than \glossterm{shadowy illumination}
                    Any object or effect which blocks light also blocks this spell's effect.
                \end{magicalsustainability}
            \itemhead{Drow Prejudice} Almost all surface-dwellers have negative associations with drow.
                Drow have an Opposition relationship with most people that they meet, which influences people's behavior and makes Persuasion checks harder (see \pcref{Persuasion}).
                People in some locations, such as deep underground, do not have this attitude.
            \itemhead{Keen Senses} Drow gain a \plus2 bonus to the Awareness skill (see \pcref{Awareness}).
            \itemhead{Poison Tolerance} Drow are \trait{impervious} to poison.
            \itemhead{Sensitive Eyes} Drow take a \minus2 penalty to \glossterm{accuracy} while they are in \glossterm{bright illumination}.
                This penalty is doubled while they are in \glossterm{brilliant illumination}.
            \itemhead{Trance} Drow do not sleep, and are immune to \magical effects that would cause them to sleep.
                Instead of sleeping, drow can trance for 4 hours.
                An elf in trance may make Perception-based checks at a \minus5 penalty.
                Drow must still avoid strenuous activity for 8 hours to heal and gain other benefits of taking a \glossterm{long rest}.
        \end{itemize}
        \parhead{Automatic Languages} Common, Elven, Undercommon

    \subsection{Dryaidi}

        Dryaidi are humanoid creatures with plantlike characteristics.
        They might have leaves instead of hair, a green skin tone, or rough, barky skin.
        They are descended from dryads, and share some fey heritage and an affinity for trees.

        \parhead{Size} Medium.
        \parhead{Attributes} No change.
        \parhead{Special Abilities}
        \begin{itemize}
            \itemhead{Dryad Archetype} Dryaidi only gain two class archetypes instead of three.
                Instead, they treat the Dryad archetype as one of their archetypes, and they gain ranks in it just like they gain ranks in class archetypes.
            \itemhead{Enchanting Appearance} A dryaidi gains a \plus2 bonus to the Creature Handling, Perform, and Persuasion skills.
            \itemhead{Fey Vulnerability} Dryaidi are \vulnerable to cold iron weapons.
            \itemhead{Tree Bond} A dryaidi must be bonded with a specific tree.
            The tree must be at least a hundred years old, healthy, and intact.
            Forming a bond or severing a bond takes one week of meditation and ritual, periodically interrupted by rest.
            Forming a bond also requires asking permission from the tree through the ritual.
            Any individual tree can only be bonded to one dryad or dryaidi in this way.

            As long as the bonded tree remains healthy and intact, the dryaidi gains a \plus1 bonus to Mental defense and a \plus1 bonus to its \glossterm{fatigue tolerance}.
            If the bonded tree becomes unhealthy, is seriously damaged, or is killed, these bonuses are inverted into penalties until the dryaidi forms a bond with a new tree.
            A bonded dryaidi can passively observe the general health and status of the tree it bonded to.
            \itemhead{Verdant Flourishing} Dryaidi can use the \spell{fertile patch} and \spell{rapid growth} \glossterm{cantrips} from the \sphere{verdamancy} sphere.
            If they already have access to that sphere, they can cast each of those cantrips as a \glossterm{minor action}.
        \end{itemize}
        \parhead{Automatic Languages} Common, Sylvan.

        \subsubsection{Dryad Archetype}

            \cf{Dry}[1]{Tree Stride} You can walk into and through living trees.
            Moving through a tree does not impede your movement in any way, and you can end your movement inside a tree.
            When you do, you can choose to be partially melded or fully melded with the tree.
            While partially melded, the tree provides \glossterm{cover} against all attacks against you.
            While fully melded, the tree blocks \glossterm{line of sight} or \glossterm{line of effect} between you and the outside world as long as it remains intact.

            At the end of each round, if you are fully or partially melded with a tree that you are bonded with using your \textit{tree bond} ability, you regain hit points equal to half your maximum hit points.

            \cf{Dry}[2]{Natural Speech} You can speak with plants and animals as if they were capable of ordinary speech.
                This ability does not make them any more friendly or cooperative than normal.
                Wary and cunning animals are likely to be terse and evasive, while stupid ones tend to make inane comments and are unlikely to say or understand anything of use.
                Plants do not have complex thought processes, but can provide information about events that have happened near them.
                In general, plants can remember events that happened within the most recent quarter of their lifespan.

            \cf{Dry}[3]{Tree Stride+} You can \glossterm{teleport} between living trees instead of moving using your \glossterm{land speed}.
            Teleporting a given distance costs movement equal to half that distance.
            If this teleportation fails for any reason, you still expend that movement.

            \cf{Dry}[4]{Fey Charm} You can use the \ability{fey charm} ability as a \glossterm{minor action}.
            \begin{magicalsustainability}{Fey Charm}{\abilitytag{Emotion}, \abilitytag{Subtle}, \abilitytag{Sustain} (minor)}
                \rankline
                \noindent

                Make an attack vs. Mental against a creature within \medrange that is an animal, plant, or humanoid.
                You take a \minus10 penalty to \glossterm{accuracy} with this attack against creatures who have made an attack or been attacked since the start of the last round.
                \hit The target is \charmed by you.
                Any act by you or by creatures that appear to be your allies that threatens or harms the charmed person breaks the effect.
                Harming the target is not limited to dealing it damage, but also includes causing it significant subjective discomfort.
                An observant target may interpret overt threats to its allies as a threat to itself.

                \rankline

                \noindent The attack's \glossterm{accuracy} increases by \plus2 for each rank beyond 4.
            \end{magicalsustainability}

            \cf{Dry}[5]{Tree Bond+} You can bond to a grove of trees instead of a single tree.
            The cumulative age of all trees in the grove must be at least a thousand years, and the grove must fit within a 500 foot radius.
            While bonded to a grove, the bonuses from your \textit{tree bond} and \textit{enchanting appearance} abilities double.

            \cf{Dry}[6]{Tree Union} When you meld with a tree using your \textit{tree stride} ability, you can fully unite with it.
            When you do, you have \glossterm{line of sight} and \glossterm{line of effect} from all areas of the tree simultaneously, as if you were everywhere in the tree's body.
            Attacks against the tree simultaneously affect both you and the tree, but both you and the tree are \impervious to all damage except for fire damage and damage from cold iron weapons.
            You and the tree are instead \vulnerable to fire damage and damage from cold iron weapons.

            \cf{Dry}[7]{Acorns of Life} Whenever you visit a tree you are bonded to with your \textit{tree bond} ability, you can gather acorns of life.
            You can have up to ten acorns of life at once.
            As a \glossterm{minor action}, you can throw an acorn of life onto an unoccupied \glossterm{grounded} space within \medrange of you.
            The space must be made of dirt, earth, or stone.
            When the acorn lands, a tree immediately grows in that space.
            The tree has a five foot diameter trunk and grows vertically until it reaches a hundred feet tall or until it encounters a solid obstacle preventing its growth.

        \subsubsection{Base Class Abilities}
            If you choose dryad as your base class, you gain the following abilities.

            \cf{Dry}{Defenses}
            You gain the following bonuses to your \glossterm{defenses}: \plus2 Fortitude, \plus3 Reflex, \plus4 Mental.

            \cf{Dry}{Hit Points}
                You have 8 hit points \add  your Constitution, plus 1 hit points per level beyond 1.
                This increases as your level increases, as indicated below.
                \begin{itemize}
                    \itemhead{Level 7} 18 hit points \add twice your Constitution, plus 2 hit points per level beyond 7.
                    \itemhead{Level 13} 35 hit points \add five times your Constitution, plus 5 hit points per level beyond 13.
                    \itemhead{Level 19} 70 hit points \add ten times your Constitution, plus 10 hit points per level beyond 19.
                \end{itemize}

            \cf{Dry}{Resources} You have the following \glossterm{resources}:
            \begin{itemize}
                \item Three \glossterm{attunement points}, which you can use to attune to items and abilities that affect you (see \pcref{Attunement Points}).
                \item A \glossterm{fatigue tolerance} equal to 3 \add your Constitution.
                    Your fatigue tolerance makes it easier for you to use powerful abilities that fatigue you (see \pcref{Fatigue}).
                \item A number of \glossterm{insight points} equal to 2 \add your Intelligence.
                    You can spend insight points to gain additional abilities (see \pcref{Insight Points}).
                \item Five \glossterm{trained skills} from among your \glossterm{class skills}, plus additional trained skills equal to your Intelligence (see \pcref{Skills}).
            \end{itemize}

            \cf{Dry}{Weapon Proficiencies} 
            You are proficient with simple weapons and bows.

            \cf{Dry}{Armor Proficiencies} 
            You are proficient with light armor.

            \cf{Dry}{Skills}
            You have the following \glossterm{class skills}:
            \begin{itemize}
                \item \subparhead{Strength} Climb, Jump, Swim.
                \item \subparhead{Dexterity} Balance, Flexibility, Perform, Stealth.
                \item \subparhead{Intelligence} Craft (wood), Knowledge (arcana, nature), Medicine
                \item \subparhead{Perception} Awareness, Creature Handling, Deception, Persuasion, Social Insight, Survival.
                \item \subparhead{Other} Intimidate.
            \end{itemize}

    \subsection{Eladrin}

        \parhead{Size} Medium.
        \parhead{Attributes} \minus1 Constitution, either \plus1 Dexterity or \plus1 Willpower
        \parhead{Special Abilities}
        \begin{raggeditemize}
            \itemhead{Fae Step} As a standard action, you can use the \ability{fae step} ability.
            \begin{magicalactiveability}{Fae Step}
                \rankline
                You \glossterm{teleport} horizontally to a location within \shortrange.

                \rankline
                This ability improves based on your rank in your highest-rank archetype.
                \rank{3} The range increases to \medrange.
                \rank{5} The range increases to \longrange.
                \rank{7} The range increases to \distrange.
            \end{magicalactiveability}
            \itemhead{Fae Season} Eladrin respond strongly to their emotions, and change their abilities based on the season they currently represent.
                An eladrin must choose one of the following seasons when it finishes a \glossterm{short rest}.
                The chosen season lasts until it changes to a different season.
                \subcf{Spring} \plus1 bonus to Mental defense, \minus1 penalty to Fortitude defense.
                Eladrin expressing the spring season are filled with the joy of a new year.
                However, they are also visibly thinner and more frail, as if recovering from a long winter.
                \subcf{Summer} \plus1 bonus to Fortitude defense, \minus1 penalty to Reflex defense.
                Eladrin expressing the summer season are visibly hearty and a little more plump.
                However, they also move with all the alacrity of a long summer day.
                \subcf{Autumn} \plus1 bonus to all checks, \minus1 penalty to \glossterm{accuracy}.
                Eladrin expressing the autumn season embody the spirit of the harvest.
                They are filled with goodwill towards all creatures, and prefer finding peaceful solutions to problems.
                Their bodies tend to be firm and toned, reflecting the hard work required to prepare for the winter.
                \subcf{Winter} \plus1 bonus to \glossterm{vital rolls}, \minus1 penalty to Mental defense.
                Eladrin expressing the winter season are prepared for the worst.
                They tend to be dour and pessimistic, but they press on despite the certainty of doom.
            \itemhead{Low-light Vision} Eladrin have \trait{low-light vision}, allowing them to see clearly in \glossterm{shadowy illumination} (see \pcref{Low-light Vision}).
            \itemhead{Trance} Eladrin do not sleep, and are immune to \magical effects that would cause them to sleep.
                Instead of sleeping, eladrin can trance for 4 hours.
                An eladrin in trance may make Perception-based checks at a \minus5 penalty.
                Eladrin must still avoid strenuous activity for 8 hours to heal and gain other benefits of taking a \glossterm{long rest}.
        \end{raggeditemize}
        \parhead{Species Feat Options} 
        \parhead{Automatic Languages} Common, Sylvan, and any one \glossterm{common language} (see \tref{Common Languages}).

    \subsection{Harpy}
        Harpies are winged creatures with the upper body of a humanoid and the lower body of a bird.
        Most harpies are female, but male harpies do exist.

        \parhead{Creature Type} Unlike most other playable species, harpies are monstrous humanoids instead of humanoids.
        \parhead{Size} Medium.
        \parhead{Attributes} \minus1 Intelligence, \plus1 Dexterity.
        \parhead{Special Abilities}
        \begin{itemize}
            \itemhead{Harpy Archetype} Harpies only gain two class archetypes instead of three.
                Instead, they treat the Harpy archetype as one of their archetypes, and they gain ranks in it just like they gain ranks in class archetypes.
            \itemhead{Limited Equipment} Harpies can wear armor, but it is treated as \glossterm{barding} instead of normal armor, reducing its effectiveness (see \pcref{Barding}).
                Harpy talons are not able to effectively wield shields or manufactured weapons.
            \itemhead{Prehensile Talons} Harpies have a talon natural weapon on each foot (see \pcref{Natural Weapons}).
                In addition, they can use their feet as \glossterm{free hands}.
            \itemhead{Wings} Harpies have a feathered wings that sprout from their shoulders.
                These wings grant them a glide speed equal to the \glossterm{base speed} for their size (see \pcref{Gliding}).
                However, they have no arms or hands.
        \end{itemize}
        \parhead{Automatic Languages} Common.
        
        \subsubsection{Harpy Archetype}

            \cf{Hrp}[1]{Winged Agility} While you are able to use your wings, you gain a \plus2 bonus to Armor defense and a \plus4 bonus to the Balance and Jump skills.
            In addition, you gain a \plus10 foot bonus to your maximum horizontal jump distance (see \pcref{Jumping}).
            This increases your maximum vertical jump distance normally.

            \cf{Hrp}[1]{Winged Combat} As a \glossterm{free action}, you can make a short hop off of solid ground to hover in your space using your wings.
            This allows you to use both of your feet without standing on them during that phase, so you can attack with both talons at once.
            You must land at the end of the phase, and you can only use this ability once per round.
            This brief hovering does not cause you to suffer penalties for flying.

            \magicalcf{Hrp}[2]{Luring Song} You can use the \textit{luring song} ability as a standard action.
            \begin{magicalactiveability}{Luring Song}[\abilitytag{Auditory}, \abilitytag{Compulsion}]
                    \rankline
                    Make an attack vs. Mental against a creature within \longrange.
                    \hit As a \glossterm{condition}, the target must move towards you as best it can during each \glossterm{movement phase}.
                    In addition, it cannot move farther away from you at any time, except as necessary to get closer to you (such as to avoid an intervening obstacle).
                    It can otherwise act freely, and is still able to attack you and your allies.

                    If you attack the target with any ability other than this one, this effect is automatically broken.
                    When this effect ends, the target becomes immune to this effect until it finishes a \glossterm{short rest}.
                    \crit The condition must be removed twice before the effect ends.

                    \rankline
                    \rank{4} You can target an additional creature within range.
                    \rank{6} The maximum number of targets increases to 5.
                \end{magicalactiveability}

            \cf{Hrp}[3]{Flight} You gain a \glossterm{fly speed} equal to the \glossterm{base speed} for your size with a maximum height of 15 feet (see \pcref{Flight}).
            As a \glossterm{free action}, you can increase your \glossterm{fatigue level} by one to ignore this height limit until the end of the round.

            \cf{Hrp}[4]{Greater Winged Agility} You can use your wings to help you maneuver on the ground more effectively, such as by briefly hovering or gliding over obstacles.
            The Balance and Jump bonuses from your \textit{winged agility} ability increase to \plus8.
            In addition, you ignore \glossterm{difficult terrain}.

            \cf{Hrp}[4]{Greater Winged Combat} You can use your \textit{winged combat} any number of times in the same round.
            You must still land at the end of each phase.

            % TODO: define correct rank
            \magicalcf{Hrp}[5]{Siren Song} You can use the \textit{siren song} ability as a standard action.
            \begin{magicalsustainability}{Siren Song}{\abilitytag{Auditory}, \abilitytag{Emotion}, \abilitytag{Sustain} (minor)}
                \rankline
                Make an attack vs. Mental against all \glossterm{enemies} within a \medarea radius from you.
                \hit Each target is both \charmed by you and \stunned as long as it can still hear you sing.
                It remains stunned even if it stops being charmed, such as if you or your allies attack it.
                This ability does not have the \abilitytag{Subtle} tag, so an observant target may notice that it is being influenced.
                \rankline
                \rank{7} The area increases to a \largearea radius.
            \end{magicalsustainability}

            \cf{Hrp}[6]{Agile Flight} Your flight improves to have good maneuverability (see \pcref{Flying Maneuverability}).
            In addition, your maximum height increases to 30 feet.

            \magicalcf{Hrp}[7]{Mythic Siren} You gain a \plus5 \glossterm{accuracy} bonus with your \ability{luring song} and \ability{siren song} abilities.

        \subsubsection{Base Class Abilities}
            If you choose harpy as your base class, you gain the following abilities.

            \cf{Hrp}{Defenses}
            You gain the following bonuses to your \glossterm{defenses}: \plus2 Fortitude, \plus4 Reflex, \plus3 Mental.

            \cf{Hrp}{Hit Points}
                You have 8 hit points \add twice your Constitution, plus 2 hit points per level beyond 1.
                This increases as your level increases, as indicated below.
                \begin{itemize}
                    \itemhead{Level 7} 20 hit points \add three times your Constitution, plus 3 hit points per level beyond 7.
                    \itemhead{Level 13} 40 hit points \add six times your Constitution, plus 6 hit points per level beyond 13.
                    \itemhead{Level 19} 80 hit points \add twelve times your Constitution, plus 12 hit points per level beyond 19.
                \end{itemize}

            \cf{Hrp}{Resources} You have the following \glossterm{resources}:
            \begin{itemize}
                \item Two \glossterm{attunement points}, which you can use to attune to items and abilities that affect you (see \pcref{Attunement Points}).
                \item A \glossterm{fatigue tolerance} equal to 4 \add your Constitution.
                    Your fatigue tolerance makes it easier for you to use powerful abilities that fatigue you (see \pcref{Fatigue}).
                \item A number of \glossterm{insight points} equal to 2 \add your Intelligence.
                    You can spend insight points to gain additional abilities (see \pcref{Insight Points}).
                \item Five \glossterm{trained skills} from among your \glossterm{class skills}, plus additional trained skills equal to your Intelligence (see \pcref{Skills}).
            \end{itemize}

            \cf{Hrp}{Weapon Proficiencies} 
            You are proficient with simple weapons.

            \cf{Hrp}{Armor Proficiencies} 
            You are proficient with light armor.

            \cf{Hrp}{Skills}
            You have the following \glossterm{class skills}:
            \begin{itemize}
                \item \subparhead{Strength} Climb, Jump.
                \item \subparhead{Dexterity} Balance, Flexibility, Perform, Stealth.
                \item \subparhead{Perception} Awareness, Creature Handling, Deception, Persuasion, Survival.
                \item \subparhead{Other} Intimidate.
            \end{itemize}

    \subsection{Kit}

        Kit are humanoid creatures that have noticeable foxlike characteristics.
        They are descended from natural fox spirits.
        All kit have at least one tail, and some have multiple tails.
        Their tails are distinctly fluffy and fox-like, and most kit put effort into concealing their tails to avoid revealing their true nature.

        \parhead{Size} Medium.
        \parhead{Attributes} No change.
        \parhead{Special Abilities}
        \begin{itemize}
            \itemhead{Foxlike Agility} Kit gain a \plus2 bonus to the Balance and Stealth skills.
            \itemhead{Illusory Guise} As a standard action, a kit can magically disguise its physical appearance in minor ways.
                This functions like the \textit{change appearance} ability with a \plus4 bonus, except that a kit cannot change the appearance of its equipment, creature type, or number of limbs, including any tails it may have (see \pcref{Change Appearance}).
                This is a \magical ability.
                It lasts until the kit \glossterm{dismisses} it as a free action or uses this ability again.
            \itemhead{Instictive Trickster} Kit gain a \plus2 bonus to the Deception and Social Insight skills.
            \itemhead{Low-light Vision} Kit have \trait{low-light vision}, allowing them to see clearly in \glossterm{shadowy illumination} (see \pcref{Low-light Vision}).
        \end{itemize}
        \parhead{Automatic Languages} Common, any one \glossterm{common language}.

    \subsection{Naiadi}
        Naiadi are humanoid creatures descended from naiads.
        Most naiadi are unusually physically appealing, but show no other outward signs of their heritage.

        \parhead{Size} Medium.
        \parhead{Attributes} No change.
        \parhead{Special Abilities}
        \begin{itemize}
            \itemhead{Create Water} A naiadi can cast the \spell{create water} cantrip.
                When they do so, they do not require verbal or somatic \glossterm{casting components}, and their spellcasting rank is considered to be equal to their rank in their highest rank archetype.
                If they would already know that cantrip through the Aquamancy sphere, the volume of water created with the cantrip doubles.
            \itemhead{Enchanting Appearance} A naiadi gains a \plus2 bonus to the Creature Handling, Perform, and Persuasion skills.
            \itemhead{Low-light Vision} Naiadi have \trait{low-light vision}, allowing them to see clearly in \glossterm{shadowy illumination} (see \pcref{Low-light Vision}).
            \itemhead{Naiad Archetype} You may choose three class archetypes, as normal.
                However, you may choose the Naiad archetype in place of one of your class archetypes.
                If you do, you gain ranks in it just like you gain ranks in class archetypes.
                You cannot choose naiad as your base class.
            \itemhead{Water Affinity} A naiadi has a \glossterm{swim speed} equal to the \glossterm{base speed} for their size.
                In addition, they can breathe clean water like a human breathes air.
        \end{itemize}
        \parhead{Automatic Languages} Common, Sylvan, any one \glossterm{common language}.

        \subsubsection{Naiad Archetype}

            \magicalcf{Nai}[1]{Water Bond} You can form a bond with a fresh stream, lake, or other Gargantuan or larger body of fresh water (not salt water).
            Forming a bond or severing a bond takes one week of meditation and ritual, periodically interrupted by rest.
            Forming a bond also requires asking permission from the water.
            Any individual body of water can only be bonded to one naiad or naiadi in this way.

            As long as your bonded water remains clean, pure, and large enough to be a valid subject of bonding, you gain a \plus1 bonus to Mental defense and a bonus equal to twice your rank in this archetype to your \glossterm{hit points}.
            If your bonded water becomes contaminated or shrinks below the minimum size, these bonuses are inverted into penalties until you sever the bond.
            You can passively observe the general health and status of water you are bonded to, including knowing when significant pollutants enter the water and when the water grows or shrinks significantly.

            \magicalcf{Nai}[2]{Fluidseeker} You gain a \plus1 bonus to \glossterm{accuracy} against creatures significantly composed of water or watery fluids.
            This is true of almost all living creatures.

            \magicalcf{Nai}[2]{Freshwater Fountain} The volume of water you can create with the \spell{create water} cantrip increases by five times.

            \magicalcf{Nai}[3]{Aqueous Form} You can cast the \spell{aqueous form} spell.
            When you do, you do not require verbal or somatic \glossterm{casting components}, and you use your rank in this archetype as your your spellcasting rank.
            In addition, it has the Attune tag instead of the Attune (deep) tag.

            \magicalcf{Nai}[4]{Greater Water Bond} The bonuses from your \textit{water bond} ability increase to \plus2 Mental defense and three times your rank in this archetype to your hit points.

            \magicalcf{Nai}[5]{Greater Fluidseeker} The accuracy bonus from your \textit{fluidseeker} ability increases to \plus2.

            \magicalcf{Nai}[5]{Greater Freshwater Fountain} The multiplier from your \textit{freshwater fountain} ability increases to twenty times the normal volume of water.

            \magicalcf{Nai}[6]{Greater Aqueous Form} When you cast the \spell{aqueous form} spell, it does not have the \abilitytag{Attune} tag.
            Instead, it lasts until you \glossterm{dismiss} it as a \glossterm{free action}.

            \magicalcf{Nai}[7]{Supreme Water Bond} The bonuses from your \textit{water bond} ability increase to \plus3 Mental defense and three times your rank in this archetype to your hit points.

    \subsection{Oozeborn}
        \includegraphics[width=\columnwidth]{optional rules/oozeborn}
        Oozeborn are ooze creatures that have gained true sentience through a strange quirk of their birth.
        They are very rare to see in civilized lands, as most oozeborn lack the opportunity to discover more than the dark caves in which they were spawned.
        Since they often grow up without mentorship from any civilized creature, oozeborn tend to have odd mannerisms and a poor ability to mask their emotions, even after spending years in civilization.
        Old oozeborn may eventually adapt to societal norms and act perfectly natural, or they may abandon civilized company entirely.

        The body of an oozeborn is amorphous, and they lack any identifiable internal organs.
        Their natural color depends on the nature of the ooze that spawned them, so green and gray are the most common colors.
        Adventuring oozeborn typically assume a bipedal shape for both practical and social convenience, but their natural shape is a loosely spherical blob.
        Unconscious oozeborn revert to their default state automatically, though some learn to maintain a semblance of cohesion while asleep.

        \parhead{Creature Type} Unlike most other playable species, oozeborn are animates instead of humanoids.
        \parhead{Size} Medium.
        \parhead{Attributes} \minus1 Intelligence, \plus1 Constitution.
        \parhead{Special Abilities}
        \begin{itemize}
            \itemhead{Acidic Body} Ooozeborn are \trait{impervious} to acid damage and poisons.
            \itemhead{Amorphous Form} An oozeborn's natural form is a loosely spherical blob.
                They have a \minus10 foot penalty to their \glossterm{land speed}, but they gain a \plus5 bonus to the Flexibility skill (see \pcref{Flexibility}).
                They can use the \ability{mold body} ability as a standard action to adopt a particular shape.
                \begin{sustainability}{Mold Body}{\abilitytag{Sustain} (free)}
                    \rankline
                    You make a Disguise check to alter your appearance (see \pcref{Change Appearance}).
                    This physically changes your body to match the results of your disguise.
                    You gain a \plus4 bonus on the check, and you ignore penalties for changing your gender, species, subtype, age, and number of limbs (up to 4).
                    However, this effect is unable to alter your equipment in any way.

                    You cannot create more than two \glossterm{free hands} with this ability.
                    If you add at least two legs, you gain a \plus10 foot bonus to your land speed.
                    % TODO: awkwardly worded
                    This speed bonus does not stack with the bonus for becoming \trait{multipedal}, so the only benefit you gain from creating three or more legs is a \plus5 bonus to the Balance skill.
                    If you give yourself a standard humanoid shape, you can wear armor designed for humanoids without suffering the normal penalties for \glossterm{barding} (see \pcref{Barding}).

                    You can sustain this ability for any length of time without mental strain, ignoring the normal 5 minute limit.
                \end{sustainability}
            \itemhead{Compressible Body} Oozeborn can compress their head and shoulders down to a minimum of a one inch radius, allowing them to squeeze through very small areas.
                Their clothing or armor is not compressed, so they may limit their ability to move through extremely narrow spaces.
            \itemhead{Darkvision} Oozeborn have \trait{darkvision} with a 60 foot range, allowing them to see in complete darkness (see \pcref{Darkvision}).
            \itemhead{Oozeborn Archetype} Oozeborn only gain two class archetypes instead of three.
                Instead, they treat the Oozeborn archetype as one of their archetypes, and they gain ranks in it just like they gain ranks in class archetypes.
        \end{itemize}
        \parhead{Automatic Languages} Common.
        
        \subsubsection{Oozeborn Archetype}

            \cf{Ooz}[1]{Acidic Pseudopod} One of your arms becomes a pseudopod \glossterm{natural weapon}.
            It deals 1d10 bludgeoning and acid damage and has the \weapontag{Long} weapon tag (see \pcref{Weapon Tags}).
            You do not have a \glossterm{free hand} on that arm while using it as a weapon in this way.

            In addition, all damage you deal with natural weapons is acid damage in addition to its other types.
            This does not affect damage you deal with manufactured weapons.

            \cf{Ooz}[2]{Darkborn Senses} You gain \trait{blindsense} with a 60 foot range, allowing you to sense your surroundings without light (see \pcref{Blindsense}).
            If you already have the blindsense ability, you increase its range by 60 feet.
            In addition, you gain \trait{blindsight} with a 15 foot range, allowing you to see without light (see \pcref{Blindsight}).
            If you already have the blindsight ability, you increase its range by 15 feet.

            \cf{Ooz}[2]{Ingest Object} You can use the \textit{ingest object} ability as a standard action.
            This functions like the \spell{absorb object} spell, except that the maximum size of the object is equal to your size.
            Anything you absorb in this way takes a single point of \glossterm{environmental} acid damage during each of your actions while it remains absorbed.
            This damage is insufficient to hurt most objects made from wood, stone, or metal, but it can destroy more fragile objects like paper or complex mechanical traps.

            \cf{Ooz}[3]{Greater Amorphous Form} You gain a \plus4 bonus to your defenses when determining whether a \glossterm{strike} gets a \glossterm{critical hit} against you instead of a normal hit.
            In addition, your \ability{mold body} ability loses the \abilitytag{Sustain} (free) tag.
            Instead, it lasts until you choose to \glossterm{dismiss} it as a \glossterm{free action}.
            This allows you to maintain your shape while unconscious.

            \cf{Ooz}[3]{Greater Compressible Body} You reduce your penalties for \squeezing by 1.

            \cf{Ooz}[4]{Acidic Body+} You are \trait{immune} to acid damage and poisons.

            \cf{Ooz}[5]{Greater Darkborn Senses} The range of your \trait{blindsense} increases by 60 feet.
            In addition, the range of your \trait{blindsight} increases by 15 feet.

            \cf{Ooz}[5]{Greater Ingest Object} The maximum number of objects you can absorb with your \textit{ingest object} ability increases to 2.
            In addition, you may absorb \glossterm{allies} with that ability in addition to unattended objects.

            \cf{Ooz}[6]{Supreme Amorphous Form} The bonus from your \textit{greater amorphous form} ability increases to \plus8.

            \cf{Ooz}[6]{Supreme Compressible Body} You reduce your penalties for squeezing by 2, which means you take no penalties for squeezing unless you use the \ability{tight squeeze} ability (see \pcref{Flexibility}).

            \cf{Ooz}[7]{Third Arm} When you use your \ability{mold body} ability, you can create three arms instead of two.
            You can use all three hands as free hands.
            For example, this can allow you to use a \weapontag{Heavy} weapon and a shield simultaneously.

            In addition, your arms become stronger and more agile.
            You can use any of your arms as a pseudopod natural weapon, and your pseudopods gain the \weapontag{Light} weapon tag (see \pcref{Weapon Tags}).

        \subsubsection{Base Class Abilities}
            If you choose oozeborn as your base class, you gain the following abilities.

            \cf{Ooz}{Defenses}
            You gain the following bonuses to your \glossterm{defenses}: \plus4 Fortitude, \plus2 Reflex, \plus3 Mental.

            \cf{Ooz}{Hit Points}
                You have 10 hit points \add twice your Constitution, plus 2 hit points per level beyond 1.
                This increases as your level increases, as indicated below.
                \begin{itemize}
                    \itemhead{Level 7} 24 hit points \add four times your Constitution, plus 4 hit points per level beyond 7.
                    \itemhead{Level 13} 50 hit points \add eight times your Constitution, plus 8 hit points per level beyond 13.
                    \itemhead{Level 19} 100 hit points \add fifteen times your Constitution, plus 15 hit points per level beyond 19.
                \end{itemize}

            \cf{Ooz}{Resources} You have the following \glossterm{resources}:
            \begin{itemize}
                \item Two \glossterm{attunement points}, which you can use to attune to items and abilities that affect you (see \pcref{Attunement Points}).
                \item A \glossterm{fatigue tolerance} equal to 4 \add your Constitution.
                    Your fatigue tolerance makes it easier for you to use powerful abilities that fatigue you (see \pcref{Fatigue}).
                \item A number of \glossterm{insight points} equal to 1 \add your Intelligence.
                    You can spend insight points to gain additional abilities (see \pcref{Insight Points}).
                \item Four \glossterm{trained skills} from among your \glossterm{class skills}, plus additional trained skills equal to your Intelligence (see \pcref{Skills}).
            \end{itemize}

            \cf{Ooz}{Weapon Proficiencies} 
            You are proficient with simple weapons.

            \cf{Ooz}{Armor Proficiencies} 
            You are proficient with light and medium armor.
            Depending on whether you are sustaining your \ability{mold body} and the form you choose, you may need \glossterm{barding} instead of regular armor (see \pcref{Barding}).

            \cf{Ooz}{Skills}
            You have the following \glossterm{class skills}:
            \begin{itemize}
                \item \subparhead{Strength} Climb, Swim.
                \item \subparhead{Dexterity} Balance, Flexibility, Sleight of Hand, Stealth.
                \item \subparhead{Constitution} Endurance.
                \item \subparhead{Intelligence} Craft, Knowledge (dungeoneering).
                \item \subparhead{Perception} Awareness, Survival.
                \item \subparhead{Other} Intimidate.
            \end{itemize}

    \subsection{Sapling}

        Saplings are young treants that have left their forest home in search of adventure.
        They tend to be slow to think and act, but resilient once they have made up their mind.

        \parhead{Creature Type} Unlike most other playable species, saplings are considered animates instead of humanoids.
        \parhead{Size} Medium.
        \parhead{Attributes} \plus1 Constitution, \plus1 Willpower, \minus1 Dexterity, \minus1 Intelligence
        \parhead{Special Abilities}
        \begin{itemize}
            \itemhead{Barkskin} A sapling gains a \plus2 bonus to Armor defense.
            \itemhead{Ingrain} A sapling can use the \textit{ingrain} ability as a \glossterm{minor action} while it is \glossterm{grounded}.
                \begin{activeability}{Ingrain}
                    \rankline
                    The sapling's land speed becomes 5 feet, regardless of any modifiers that normally apply.
                    It cannot voluntarily stop being \glossterm{grounded} while this ability lasts.
                    It gains a \plus4 bonus to Fortitude defense and a \plus2 bonus to Armor defense.

                    If the sapling finishes a \glossterm{long rest} with this ability active for the duration of the rest, it acquires nutrients sufficient to replace a day's worth of food and water.
                    This ability lasts until the sapling ends it as a standard action, or until it stops being \glossterm{grounded}.
                \end{activeability}
            \itemhead{Limited Equipment} Saplings can wear armor, but it is treated as \glossterm{barding} instead of normal armor, reducing its effectiveness (see \pcref{Barding}).
            \itemhead{Made of Wood} Saplings are \vulnerable to fire damage. In addition, they are both creatures and plants.
            \itemhead{Treant Archetype} Saplings only gain two class archetypes instead of three.
                Instead, they treat the Treant archetype as one of their archetypes, and they gain ranks in it just like they gain ranks in class archetypes.
            \itemhead{Tree Appearance} When a sapling stays perfectly still, observers must make a DV 15 Awareness check to recognize that it is not an ordinary tree.
            Careful observers may still notice that the ordinary tree has appeared where no tree used to be, so they may be suspicious of a sapling even if they do not pass this check.
            \itemhead{Unhurried and Unfaltering} Saplings have a \minus10 penalty to speed with all \glossterm{movement modes}.
                However, saplings ignore all penalties to their land speed, such as from wearing heavy armor.
                Those penalties still apply to any other movement modes normally.
                This also does not change effects that completely replace the sapling's land speed, such as the \ability{ingrain} ability.
                In addition, saplings ignore \glossterm{difficult terrain} from inanimate natural sources, such as \glossterm{heavy undergrowth}.
        \end{itemize}
        \parhead{Automatic Languages} Common, Sylvan.

        \subsubsection{Treant Archetype}
            % Nourishing Ingrain:
            % * R1, 2 HP - 25% of fighter base HP (8)
            % * R2, 4 HP - 28.5% of fighter base HP (14)
            % * R3, 6 HP - 30% of fighter base HP (20)
            % * R4, 8 HP - 27.5% (29)
            % * R5, 15 HP - 37.5% (40)
            % * R6, 18 HP - 31% (58)
            % * R7, 21 HP - 26% (80)

            % This intentionally can bring you above half max
            \cf{Tre}[1]{Nourishing Ingrain} At the end of each round while you are \ability{ingrained}, you regain hit points equal to twice your rank in this archetype, and you may choose to remove a \glossterm{condition}.
            If you do, you increase your \glossterm{fatigue level} by one.

            \cf{Tre}[2]{Sturdy as the Mighty Oak} You gain a bonus equal to three times your rank in this archetype to your \glossterm{hit points} (see \pcref{Hit Points}).
            In addition, you gain a \plus1 bonus to your \glossterm{vital rolls} (see \pcref{Vital Wounds}).

            \cf{Tre}[3]{Animate Plants} You can use the \textit{animate plants} ability as a standard action.
            \begin{activeability}{Animate Plants}
                \spelltwocol{}{\abilitytag{Manifestation}}
                \rankline
                Make an attack vs. Reflex against one Large or smaller \glossterm{grounded} creature within \medrange.
                You gain a \plus2 accuracy bonus if the target is in \glossterm{undergrowth}.

                \hit The target is \slowed as a \glossterm{condition}.
                In addition, it takes 1d8 bludgeoning damage immediately, and during each of your subsequent actions while this condition lasts.

                This condition can be removed if the target makes a \glossterm{difficulty value} 10 Strength check as a \glossterm{movement} to break the plants.
                If the target makes this check as a standard action, it gains a \plus5 bonus.
                In addition, this condition is removed if the target takes fire damage.
                \crit The condition must be removed an additional time before the effect ends.
                \rankline
                For each rank beyond 3, the attack's \glossterm{accuracy} increases by \plus2 and the \glossterm{difficulty value} to break the plants increases by 2.
                In addition, the damage increases at each rank as described below.
                \rank{4} 1d10 bludgeoning damage.
                \rank{5} 2d6 bludgeoning damage.
                \rank{6} 2d8 bludgeoning damage.
                \rank{7} 2d10 bludgeoning damage.
            \end{activeability}

            \cf{Tre}[4]{Tall as the Noble Pine} Your size category increases to Large.
            This would normally increase your \glossterm{base speed} to 40 feet.
            However, you take a \minus10 foot penalty to your base speed, so it is still 30 feet.
            You also gain a \plus1 bonus to your Strength, and a \minus1 penalty to your Dexterity.

            \cf{Tre}[5]{Nourishing Ingrain+} The healing from your \textit{nourishing ingrain} ability increases to three times your rank in this archetype.
            In addition, removing a condition with that ability no longer increases your fatigue level.

            \cf{Tre}[6]{Sturdy as the Mighty Oak+} The hit point bonus increases to four times your rank in this archetype.

            \cf{Tre}[7]{Tall as the Noble Pine+} Your size category increases to Huge.
            This increases your \glossterm{base speed} to 40 feet.
            The modifiers to Strength and Dexterity increase to \plus2 and \minus2, respectively.

        \subsubsection{Base Class Abilities}
            If you choose treant as your base class, you gain the following abilities.

            \cf{Tre}{Defenses}
            You gain the following bonuses to your \glossterm{defenses}: \plus5 Fortitude, \plus3 Reflex, \plus4 Mental.

            \cf{Tre}{Hit Points}
                You have 10 hit points \add twice your Constitution, plus 2 hit points per level beyond 1.
                This increases as your level increases, as indicated below.
                \begin{itemize}
                    \itemhead{Level 7} 24 hit points \add four times your Constitution, plus 4 hit points per level beyond 7.
                    \itemhead{Level 13} 50 hit points \add eight times your Constitution, plus 8 hit points per level beyond 13.
                    \itemhead{Level 19} 100 hit points \add fifteen times your Constitution, plus 15 hit points per level beyond 19.
                \end{itemize}

            \cf{Tre}{Resources} You have the following \glossterm{resources}:
            \begin{itemize}
                \item Two \glossterm{attunement points}, which you can use to attune to items and abilities that affect you (see \pcref{Attunement Points}).
                \item A \glossterm{fatigue tolerance} equal to 4 \add your Constitution.
                    Your fatigue tolerance makes it easier for you to use powerful abilities that fatigue you (see \pcref{Fatigue}).
                \item A number of \glossterm{insight points} equal to 1 \add your Intelligence.
                    You can spend insight points to gain additional abilities (see \pcref{Insight Points}).
                \item Three \glossterm{trained skills} from among your \glossterm{class skills}, plus additional trained skills equal to your Intelligence (see \pcref{Skills}).
            \end{itemize}

            \cf{Tre}{Weapon Proficiencies} 
            You are proficient with simple weapons and club-like weapons.

            \cf{Tre}{Armor Proficiencies} 
            You are proficient with light, medium, and heavy armor.

            \cf{Tre}{Skills}
            You have the following \glossterm{class skills}:
            \begin{itemize}
                \item \subparhead{Dexterity} Balance.
                \item \subparhead{Constitution} Endurance.
                \item \subparhead{Intelligence} Knowledge (nature).
                \item \subparhead{Perception} Awareness, Creature Handling, Survival.
                \item \subparhead{Other} Intimidate.
            \end{itemize}

    \subsection{Tiefling}
        \includegraphics[width=\columnwidth]{optional rules/tiefling}

        Tieflings are humanoid creatures descended from fiends.
        They inherit a tendency towards evil from their ancestors, and are therefore viewed with great suspicion by most civilized societies.
        Good-aligned tieflings exist, but they may have difficulty using their natural talents for subterfuge and deceit for noble ends, and they often struggle with hidden vices.

        \parhead{Size} Medium.
        \parhead{Attributes} No change.
        \parhead{Special Abilities}
        \begin{itemize}
            \itemhead{Darkvision} Tieflings have \trait{darkvision} with a 60 foot range, allowing them to see in complete darkness (see \pcref{Darkvision}).
            \itemhead{Demonic Prejudice} Most people have negative associations with tieflings thanks to the malign influence that demons have on the world.
                Tieflings have an Opposition relationship with most people that they meet, which influences people's behavior and makes Persuasion checks harder (see \pcref{Persuasion}).
                People in some locations, such as the Abyss, do not have this attitude.
            \itemhead{Hellfire Tolerance} Tieflings are \trait{impervious} to fire damage.
            \itemhead{Infernal Presence} Tieflings gain a \plus2 bonus to the Deception and Intimidate skills.
            \itemhead{Tiefling Archetype} You may choose three class archetypes, as normal.
                However, you may choose the Tiefling archetype in place of one of your class archetypes.
                If you do, you gain ranks in it just like you gain ranks in class archetypes.
                You cannot choose tiefling as your base class.
        \end{itemize}
        \parhead{Automatic Languages} Abyssal, Common, any one \glossterm{common language}.

        \subsubsection{Tiefling Archetype}
            \magicalcf{Tif}[1]{Abyssal Hop} You can use the \ability{abyssal hop} ability as a standard action.
            \begin{magicalactiveability}{Abyssal Hop}
                \rankline
                You teleport horizontally into an unoccupied location within \shortrange on a stable surface that can support your weight.
                If the destination is invalid, this spell fails with no effect.
                In addition, make an attack vs. Reflex against each \glossterm{enemy} adjacent to your location after you arrive.
                \hit \damagerankzero{fire}.
                \miss Half damage.

                \rankline
                % Stay at 1 rank lower than par
                \rank{2} The damage bonus from your power increases to \plus1 per 2 power.
                \rank{3} The base damage increases to 1d8.
                \rank{4} The damage bonus increases to be equal to your power.
                \rank{5} The base damage increases to 1d10.
                \rank{6} The damage bonus increases to 1d8 per 3 power.
                \rank{7} The base damage increases to 2d8.
            \end{magicalactiveability}

            \cf{Tif}[1]{Infernal Resilience} You gain a bonus equal to twice your rank in this archetype to your \glossterm{damage resistance}.

            \magicalcf{Tif}[2]{Infernal Ancestry} You deepen your connection to a particular aspect of your demonic ancestry.
            Choose one of the following infernal ancestries: hellfire conduit, tempting allure, or unholy might.
            You gain a benefit based on your chosen ancestry.
            \begin{itemize}
                \item Infernal Rebuke: You can use the \textit{infernal rebuke} ability as a standard action.
                    \begin{magicalactiveability}{Infernal Rebuke}
                        \rankline
                        Make an attack vs. Fortitude against one creature within \shortrange.
                        You gain a \plus2 bonus to \glossterm{accuracy} with this attack if the target attacked you during the previous round.
                        \hit \damagerankthree{fire}.

                        \rankline
                        % Stay at rank+1 damage
                        \rank{3} The base damage increases to 1d10.
                        \rank{4} The damage bonus from your power increases to 1d8 per 3 power.
                        \rank{5} The base damage increases to 2d8.
                        \rank{6} The damage bonus from your power increases to 1d8 per 2 power.
                        \rank{7} The base damage increases to 4d8.
                    \end{magicalactiveability}
                \item Tempting Allure: You gain a \plus2 bonus to the Deception, Disguise, and Persuasion skills.
                    In addition, you can use the \ability{charming temptation} ability as a standard action.
                    \begin{magicalsustainability}{Charming Temptation}{\abilitytag{Emotion}, \abilitytag{Subtle}, \abilitytag{Sustain} (minor)}
                        \rankline
                        \noindent

                        Make an attack vs. Mental against a humanoid creature within \medrange.
                        You take a \minus10 penalty to \glossterm{accuracy} with this attack against creatures who have made an attack or been attacked since the start of the last round.
                        \vspace{0.25em}
                        \hit The target is \charmed by you.
                        Any act by you or by creatures that appear to be your allies that threatens or harms the charmed person breaks the effect.
                        Harming the target is not limited to dealing it damage, but also includes causing it significant subjective discomfort.
                        An observant target may interpret overt threats to its allies as a threat to itself.

                        \rankline

                        \noindent The attack's \glossterm{accuracy} increases by \plus2 for each rank beyond 2.
                        \vspace{0.1em}
                    \end{magicalsustainability}
                \item Unholy Might: You gain two claw natural weapons and one bite natural weapon (see \pcref{Natural Weapons}).
                    In addition, you gain a \plus1 bonus to your \glossterm{mundane power}.
            \end{itemize}

            \magicalcf{Tif}[3]{Abysswalker} You can use your \ability{abyssal hop} ability to teleport as a \glossterm{movement} instead of as a standard action.
            When you do, you do not deal fire damage at your destination, and you \glossterm{briefly} cannot use that ability as a movement again.

            \magicalcf{Tif}[4]{Greater Infernal Ancestry} The benefits of your \textit{infernal ancestry} ability improve.
            \begin{itemize}
                \item Infernal Conduit: You gain a \plus1 bonus to your \glossterm{magical power}.
                    In addition, the area affected by your \textit{abyssal hop} ability increases to a \smallarea radius from your destination.
                \item Tempting Allure: The skill bonuses from your \textit{infernal ancestry} ability increase to \plus3.
                    In addition, you can use the \ability{dominating temptation} ability as a standard action.
                    \begin{magicalactiveability}{Dominating Temptation}{\abilitytag{Emotion}}
                        \rankline
                        \noindent
                        Make an attack vs. Mental against a humanoid creature within \shortrange.%
                        \vspace{0.25em}
                        \hit The target is \stunned as a \glossterm{condition}.
                        \crit The target is \confused instead of stunned.
                        In addition, if the target is humanoid and was already confused from a previous use of this ability, you may \glossterm{attune} to this ability.
                        When you do, the target becomes \dominated by you for the duration of that attunement.

                        \rankline
                        \noindent The attack's \glossterm{accuracy} increases by \plus1 for each rank beyond 4.
                        \vspace{0.1em}
                    \end{magicalactiveability}
                \item Unholy Might: You can use the \ability{unholy strength} ability as a standard action.
                    \begin{magicalattuneability}{Unholy Strength}{\abilitytag{Attune}}
                        \rankline
                        You gain a \plus1 \glossterm{enhancement bonus} to your Strength.
                    \end{magicalattuneability}
            \end{itemize}

            \cf{Tif}[5]{Greater Hellfire Tolerance} You become \trait{immune} to fire damage.

            \cf{Tif}[5]{Greater Infernal Resilience}  The bonus from your \ability{infernal resilience} ability increases to three times your rank in this archetype.

            \magicalcf{Tif}[6]{Supreme Infernal Ancestry} The benefits of your \textit{infernal ancestry} ability reach their peak.
            \begin{itemize}
                \item Infernal Conduit: The magical power bonus from your \textit{greater infernal ancestry} ability increases to \plus2.
                    In addition, the area affected by your \textit{abyssal hop} ability increases to a \medarea radius from your destination.
                \item Tempting Allure: The skill bonuses from your \textit{infernal ancestry} ability increase to \plus4.
                    In addition, your \textit{tempting domination} ability can dominate non-humanoid creatures.
                \item Unholy Might: The power bonus from your \textit{infernal ancestry} ability increases to \plus2.
                    In addition, your \textit{unholy surge} ability loses the \abilitytag{Attune} tag.
                    Instead, it lasts until you \glossterm{dismiss} it as a \glossterm{free action}.
            \end{itemize}

            \magicalcf{Tif}[7]{Greater Abyssal Hop} When you use your \ability{abyssal hop} ability, you no longer require \glossterm{line of sight} or \glossterm{line of effect} to your destination.
            In addition, when you use it to teleport as a standard action, the range increases to \distrange.

        % \subsubsection{Base Class Abilities}
        %     If you choose tiefling as your base class, you gain the following abilities.

        %     \cf{Tif}{Defenses}
        %     You gain the following bonuses to your \glossterm{defenses}: \plus3 Fortitude, \plus5 Reflex, \plus7 Mental.

        %     \cf{Tif}{Resources} You have the following \glossterm{resources}:
        %     \begin{itemize}
        %         \item Four \glossterm{attunement points}, which you can use to attune to items and abilities that affect you (see \pcref{Attunement Points}).
        %         \item A \plus2 bonus to your \glossterm{fatigue tolerance}, which makes it easier for you to use powerful abilities that fatigue you (see \pcref{Fatigue}).
        %         \item Two \glossterm{insight points}, which you can spend to gain additional abilities or proficiencies (see \pcref{Insight Points}).
        %         \item Five \glossterm{trained skills}, which you can spend to learn skills (see \pcref{Trained Skills}).
        %     \end{itemize}

        %     \cf{Tif}{Weapon Proficiencies} 
        %     You are proficient with simple weapons.

        %     \cf{Tif}{Armor Proficiencies} 
        %     You are proficient with light and medium armor.

        %     \cf{Tif}{Skills}
        %     You have the following \glossterm{class skills}:
        %     \begin{itemize}
        %         \item \subparhead{Strength} Climb.
        %         \item \subparhead{Dexterity} Balance, Sleight of Hand, Stealth.
        %         \item \subparhead{Intelligence} Craft, Disguise, Knowledge (arcana, planes)
        %         \item \subparhead{Perception} Awareness, Social Insight.
        %         \item \subparhead{Other} Deception, Intimidate, Perform, Persuasion.
        %     \end{itemize}

    \subsection{Vampire}
        A vampire is an undead creature that must drink the blood of living creatures to survive.
        Unlike most undead creatures, vampires appear to be alive and human, allowing them to act normally in society.
        Vampires have great power, but also many dangerous weaknesses.

        \parhead{Creature Type} Unlike most other playable species, vampires are undead instead of humanoids.
        \parhead{Size} Medium.
        \parhead{Attributes} \plus1 Strength and Dexterity, \minus1 Constitution.
        \parhead{Special Abilities}
        \begin{itemize}
            \itemhead{Climb Speed} Vampires have a \glossterm{climb speed} 10 feet slower than their \glossterm{base speed}.
            \itemhead{Darkvision} Vampires have \trait{darkvision} with a 90 foot range, allowing them to see in complete darkness (see \pcref{Darkvision}).
            \itemhead{Fangs} Vampires have a bite natural weapon (see \tref{Natural Weapons}).
                These fangs retract when not in use, so vampires cannot be identified as non-human by their fangs unless they choose to expose them.
            \itemhead{Undead} Vampires are \trait{undead} instead of \glossterm{living}, and they take damage from most healing effects (see \pcref{Undead})).
            \itemhead{Unnatural Charm} Vampires gain a \plus3 bonus to the Persuasion skill.
            \itemhead{Vampire Archetype} Vampires only gain two class archetypes instead of three.
                Instead, they treat the Vampire archetype as one of their archetypes, and they gain ranks in it just like they gain ranks in class archetypes.
        \end{itemize}

        \parhead{Special Weaknesses}
            Vampires have a number of specific weaknesses.
        \begin{itemize}
            \itemhead{Blood Dependence} For every 24 hours that a vampire remains awake without ingesting at least one pint of blood from living creatures, its maximum hit points are reduced by 20.
            If its maximum hit points are reduced to 0 in this way, it dies and withers away into a pile of ash.
            This penalty is removed as soon as the vampire drinks a pint of blood.
            A vampire can can enter a torpor to survive indefinitely without blood.
            While in a torpor, it is unconscious until it smells blood nearby.
            \itemhead{Garlic} Whenever a vampire smells or touches garlic, it takes 10 energy damage and becomes \frightened by any creatures bearing garlic as a condition.
            This damage is repeated at the during each subsequent \glossterm{action phase} that the vampire spends exposed to garlic.
            \itemhead{Holy Water} Whenever a vampire takes damage from holy water, it becomes \stunned as a condition.
            \itemhead{Running Water} Whenever a vampire touches or flies over running water, it takes 10 energy damage and \glossterm{briefly} becomes \immobilized.
            This applies as long as the vampire is within 100 feet of the running water, even the water is underground or under a bridge.
            It can use the \ability{struggle} ability to move despite being immobilized, but only towards the closest shore.
            This damage is repeated at the during each subsequent \glossterm{action phase} that the vampire spends touching or flying over running water.
            \itemhead{Silver} Vampires are \vulnerable to strikes using silvered weapons.
            \itemhead{True Sunlight} Whenever a vampire is exposed to true sunlight, it takes 20 energy damage and becomes \blinded as a condition.
            If it suffers a \glossterm{vital wound} from this damage, it immediately dies and dissolves into a pile of ash.
            This damage is repeated at the during each subsequent \glossterm{action phase} that the vampire spends in true sunlight.
            \itemhead{Unmirrored} Vampires have no reflection in mirrors, including their clothes and equipment.
            This can allow careful observers to identify vampires.
            \itemhead{Wooden Stakes} If a vampire loses hit points from a critical strike using a wooden stake, the stake becomes impaled in its heart.
            The vampire becomes \paralyzed until the stake is removed.
            A wooden stake is a light improvised weapon that deals 1d4 piercing damage.
        \end{itemize}

        \subsubsection{Vampire Archetype}

            \cf{Vmp}[1]{Blood Drain} Whenever a living creature loses hit points from a \glossterm{strike} using your bite natural weapon, you can increase your \glossterm{fatigue level} by one.
            When you do, you regain \glossterm{damage resistance} and \glossterm{hit points} at the end of the round.
            The recovery is equal to the hit points the target lost from the attack, ignoring any extra damage from critical hits.

            \cf{Vmp}[2]{Gentle Fangs} Whenever you deal damage using your bite natural weapon, you can choose not to reduce the target's hit points below 0, or you can treat the damage as \glossterm{subdual damage}.
            In addition, damage dealt using your bite natural weapon does not wake sleeping creatures unless you inflict a vital wound.

            \magicalcf{Vmp}[2]{Reviving Coffin} You can designate a coffin as your home by resting in it for 24 consecutive hours.
            When you take a \glossterm{long rest} in your home coffin, you recover two \glossterm{vital wounds} instead of one.
            In addition, you can cross running water without penalty while in your home coffin.

            When you die, if your corpse is placed in your home coffin, you can be resurrected after 24 hours.
            You can only be resurrected in this way if you were killed by a vital wound with a \glossterm{vital roll} of \minus3 or higher.
            If you were killed by a more severe vital wound, or by some other effect, even your home coffin cannot save you.

            \magicalcf{Vmp}[3]{Charming Gaze} You can use the \ability{charming gaze} ability as a standard action.
            \begin{magicalsustainability}{Charming Gaze}{\abilitytag{Emotion}, \abilitytag{Subtle}, \abilitytag{Sustain} (minor), \abilitytag{Visual}}
                \rankline
                Make an attack vs. Mental against all humanoid creatures in a \largearea cone from you.
                You take a \minus10 penalty to \glossterm{accuracy} with this attack against creatures who have made an attack or been attacked since the start of the last round.
                \hit The target is \charmed by you.
                Any act by you or by creatures that appear to be your allies that threatens or harms the charmed person breaks the effect.
                Harming the target is not limited to dealing it damage, but also includes causing it significant subjective discomfort.
                An observant target may interpret overt threats to its allies as a threat to itself.

                \rankline

                \noindent The attack's \glossterm{accuracy} increases by \plus2 for each rank beyond 3.
            \end{magicalsustainability}

            \magicalcf{Vmp}[4]{Creature of the Night} You can use the \ability{creature of the night} ability as a standard action.
            \begin{magicalattuneability}{Creature of the Night}{\abilitytag{Attune}}
                \rankline
                You \glossterm{shapeshift} into the form of a Tiny bat, a Medium cloud of mist, or your normal humanoid form.
                While in your bat form, you gain \trait{blindsense} (120 ft.) and a 40 foot fly speed with a 60 ft. height limit.
                While in your mist form, you become \trait{incorporeal} and gain a 20 foot fly speed with a 30 ft. height limit and perfect maneuverability.

                In either non-humanoid form, you are unable to take any standard action other than \glossterm{movement}.
                You cannot use this ability while \paralyzed.
            \end{magicalattuneability}

            \magicalcf{Vmp}[4]{Reviving Coffin+} You can designate up to three home coffins, rather than only one.
            This can allow you to travel with one coffin while keeping others safe for emergencies.

            \cf{Vmp}[5]{Unholy Perfection} Choose any two attributes other than Constitution.
            You gain a \plus1 bonus to those two attributes.

            \magicalcf{Vmp}[6]{Dominating Gaze} You can use the \ability{dominating gaze} ability as a standard action.
            \begin{magicalactiveability}{Dominating Gaze}{\abilitytag{Emotion}, \abilitytag{Visual}}
                \rankline
                Make an attack vs. Mental against a humanoid creature within \shortrange.
                \hit The target is \stunned as a \glossterm{condition}.
                While it has no remaining \glossterm{damage resistance}, it is \confused instead of stunned.
                \crit If the target is humanoid and was already confused from a previous use of this ability, you may \glossterm{attune} to this ability.
                When you do, the target becomes \dominated by you for the duration of that attunement.

                \rankline
                The attack's \glossterm{accuracy} increases by \plus2 for each rank beyond 6.
            \end{magicalactiveability}

            \magicalcf{Vmp}[7]{Blood Drain+} Using your \ability{blood drain} ability does not increase your fatigue level.

            \magicalcf{Vmp}[7]{Eternal Undeath} Your \ability{reviving coffin} ability can revive you from any cause of death or severity of vital wound.
            As long as some part of your corpse, even just a pinch of ash, is placed inside one of your home coffins, you will resurrect after 24 hours.
            Only the destruction of all of your home coffins or the total annihilation of your corpse can prevent your return.

        \subsubsection{Base Class Abilities}
            If you choose vampire as your base class, you gain the following abilities.

            \cf{Vmp}{Defenses}
            You gain the following bonuses to your \glossterm{defenses}: \plus3 Fortitude, \plus4 Reflex, \plus5 Mental.

            \cf{Vmp}{Hit Points}
                You have 8 hit points \add twice your Constitution, plus 2 hit points per level beyond 1.
                This increases as your level increases, as indicated below.
                \begin{itemize}
                    \itemhead{Level 7} 20 hit points \add three times your Constitution, plus 3 hit points per level beyond 7.
                    \itemhead{Level 13} 40 hit points \add six times your Constitution, plus 6 hit points per level beyond 13.
                    \itemhead{Level 19} 80 hit points \add twelve times your Constitution, plus 12 hit points per level beyond 19.
                \end{itemize}

            \cf{Vmp}{Resources} You have the following \glossterm{resources}:
            \begin{itemize}
                \item Three \glossterm{attunement points}, which you can use to attune to items and abilities that affect you (see \pcref{Attunement Points}).
                \item A \glossterm{fatigue tolerance} equal to 3 \add your Constitution.
                    Your fatigue tolerance makes it easier for you to use powerful abilities that fatigue you (see \pcref{Fatigue}).
                \item A number of \glossterm{insight points} equal to 1 \add your Intelligence.
                    You can spend insight points to gain additional abilities (see \pcref{Insight Points}).
                \item Five \glossterm{trained skills} from among your \glossterm{class skills}, plus additional trained skills equal to your Intelligence (see \pcref{Skills}).
            \end{itemize}

            \cf{Vmp}{Weapon Proficiencies} 
            You are proficient with simple weapons and one weapon group of your choice.

            \cf{Vmp}{Armor Proficiencies} 
            You are proficient with light armor.

            \cf{Vmp}{Skills}
            You have the following \glossterm{class skills}:
            \begin{itemize}
                \itemhead{Strength} Climb, Jump.
                \itemhead{Dexterity} Balance, Stealth.
                \itemhead{Intelligence} Deduction, Disguise, Knowledge (dungeoneering and religion)
                \itemhead{Perception} Awareness, Creature Handling, Deception, Persuasion, Social Insight.
                \itemhead{Other} Intimidate, Persuasion
            \end{itemize}

\section{Classes}
    \subsection{Bard}
        A bard is a rogue with the ability to perform magical feats through music.
        It is unclear whether bards actually draw power from music in the same way that druids draw power from nature, or whether they simply channel their innate magical talent through music.
        The bard class functions like the rogue class, with the following exceptions:
        \begin{itemize}
            \item A bard cannot choose the \textit{assassin} archetype. However, the \textit{arcane magic} sorcerer archetype is considered to be part of their class, and they may choose that archetype without spending insight points to multiclass.
            \item A bard casts spells without \glossterm{somatic components}.
            \item A bard can only cast spells while sustaining a performance with the Perform skill. This performance can be either a mundane performance or a \textit{bardic performance} ability.
        \end{itemize}

    \subsection{Blighter}
        Blighter practice a strange inversion of druidic traditions.
        While druids venerate nature in all its forms, blighters dedicate their lives to the destruction of nature for its own sake.
        They rip power directly from the death of natural beings, using it to fuel their own warped version of nature magic.
        The blighter class functions like the druid class, with the following exceptions:
        \begin{itemize}
            \item Whenever a blighter rests, they automatically destroy nature and kill anything living around them.
                Plants wither and die, insects fall dead in the air, and so on.
                A ten minute rest destroys life in a radius equal to five feet times the blighter's highest rank in the blighter class (minimum 5 feet total).
                In general, Diminuitive or larger creatures and Medium or larger plants suffer no ill effects, though creatures may feel subtle pains.
                An eight hour rest destroys life in ten times that radius, and kills life one size category larger.
                Resting beyond that point does not increase the radius or severity of the effect.
                This destruction spreads out gradually throughout the resting period, and even a partially completed rest destroys some natural life.
            \item A blighter cannot choose the \textit{wildspeaker} archetype.
                However, the \textit{domain influence} cleric archetype is considered to be part of their class, and they may choose that archetype without spending insight points to multiclass.
                A blighter can only choose the Death, Destruction, and Evil domains.
            \item A blighter cannot gain access to the \sphere{verdamancy} mystic sphere by any means.
        \end{itemize}

    \subsection{Faebonder}
        A faebonder is a warlock who made their pact with a fae creature instead of a demon.
        The faebonder class functions like the warlock class, with the following exceptions:
        \begin{itemize}
            \item The magic source for the faebonder class is nature magic instead of pact magic.
                This changes the \glossterm{mystic spheres} a faebonder has access to and all other effects based on their source of magic.
                However, they still require both \glossterm{verbal components} and \glossterm{somatic components} to cast spells from the faebonder class (see \pcref{Casting Components}).
            \item A faebonder cannot choose the \textit{blessings of the abyss} archetype. However, the \textit{elementalist} druid archetype is considered to be part of their class, and they may choose that archetype without spending insight points to multiclass.
            \item Faebonders add Knowledge (nature) to their class skill list and remove Knowledge (planes).
        \end{itemize}

    \subsection{Favored Soul}
        A favored soul is a warlock who made their pact with a deity instead of a demon.
        This is an unusual arrangement, as deities would normally influence their clerics to achieve their aims.
        However, in special circumstances, a deity may want to empower a non-worshipper to influence mortal affairs.
        The favored soul class functions like the warlock class, with the following exceptions:
        \begin{itemize}
            \item The magic source for the favored soul class is divine magic instead of pact magic.
                This changes the \glossterm{mystic spheres} a favored soul has access to and all other effects based on their source of magic.
                However, they still require both \glossterm{verbal components} and \glossterm{somatic components} to cast spells from the favored soul class (see \pcref{Casting Components}).
            \item A favored soul cannot choose the \textit{blessings of the abyss} archetype. However, the \textit{domain influence} cleric archetype is considered to be part of their class, and they may choose that archetype without spending insight points to multiclass.
            \item Favored souls add Knowledge (religion) to their class skill list and remove Knowledge (planes).
        \end{itemize}

    \subsection{Shaman}
        A shaman, like a cleric, is a divine worshipper.
        However, while clerics worship powerful, well-established deities, shamans worship more primitive deities of lesser power.
        As a result, their divine powers are more limited and take different forms.
        Shamans are common among less civilized humanoid societies like bugbears.
        The shaman class functions like the cleric class, with the following exceptions:
        \begin{itemize}
            \item The magic source for the shaman class is nature magic instead of divine magic.
                This changes the \glossterm{mystic spheres} a shaman has access to and all other effects based on their source of magic.
            \item A shaman cannot choose the \textit{divine spell mastery} archetype. However, the \textit{elementalist} druid archetype is considered to be part of their class, and they may choose that archetype without spending insight points to multiclass.
            \item A shaman cannot gain access to more than two \textit{mystic spheres} from the magic source granted by the shaman class by any means.
            \item Shamans add Knowledge (nature) to their class skill list and remove Knowledge (planes).
        \end{itemize}

\section{Alternate Play Styles}

    \subsection{Being Surrounded}\label{Being Surrounded}
        Normally, exact positioning doesn't matter that much in combat.
        This makes it easier to play without a grid, or to just spend less time worrying about the details of everyone's positions on a grid.
        With this optional rule, you can make positioning more important in combat, increasing tactical depth for melee characters.
        This generally has the downside of making movement more complicated, however, as combatants try to surround others and avoid being surrounded themselves.

        If you play with this alternate rule, when you are being attacked by multiple foes at once, you are less able to defend yourself.
        If every space adjacent to you either contains an \glossterm{enemy} or is adjacent to an \glossterm{enemy}, you are surrounded.
        A creature that is surrounded takes a \minus2 penalty to its Armor and Reflex defenses.
        When determining whether you are surrounded, ignore any enemies that are sharing space with you, and ignore any enemies that are at least two size categories smaller than you.

        Any effect that makes a creature immune to being \partiallyunaware, such as the \spell{foresight} spell, also makes that creature immune to being surrounded.

    \subsection{Critical Failure}
        Normally, there is no explicit penalty for catastrophic failure built into the rules.
        Even if you fail at a check by a large amount, it doesn't leave you worse off than when you started.
        Sometimes, it may be narratively appropriate to punish significant failure more severely, at the GM's discretion.
        For example, attempting a difficult Persuasion check and completely botching the execution might leave the target feeling more hostile than if no Persuasion had been attempted at all.
        A good threshold for critical failure would generally be failing a check by 8 or more.
        For specific tasks, it may make more sense to have punishments for failure at lower thresholds as well.

        This is considered an optional rule because it generally makes trying silly ideas or extremely difficult tasks more dangerous, which isn't appropriate for every game.
        It also depends heavily on GM discretion.

    \subsection{Easy Magic Item Reforging}
        The Craft Specialization feat allows characters to transfer magic item properties between different items.
        For example, if the players find a magic meteor hammer that none of them could use, they could reforge that item as a magic battleaxe so they could use its property.
        With this optional rule, skilled item crafters capable of this action are assumed to be common in major cities or towns.
        The typical price to reforge an item in this way is two ranks lower than the item's rank, to a minimum rank of 1.

        The advantage of using this optional rule is that it makes magic items more likely to be useful to the party.
        Without this rule, you may be forced to have the party ``randomly'' only find magic items that they are coincidentally proficient with, or the party may frequently find magic items that they can't use.
        On the other hand, this rule assumes a more magical and highly developed civilization.
        It also may require the party to frequently return to town to reforge useless items into items that are useful for them.
        Either of those requirements may not match the intended tone of your campaign.

    \subsection{Expanded Insight Points}
        Normally, \glossterm{insight points} can only be used to learn new special abilities from your class, or from a small number of feats.
        This alternate rule allows you to spend insight points to gain a wide variety of other proficiencies and benefits.
        This makes character creation more complicated, but it also allows you to personalize your character much more precisely.

        If you play with this alternate rule, you can spend insight points in any of the following ways.
        \begin{raggeditemize}
            \item You can spend an \glossterm{insight point} to gain an additional \glossterm{trained skill}.
            \item You can spend an \glossterm{insight point} to gain proficiency in an additional \glossterm{usage class} of armor (light, medium, or heavy).
                You must be proficient with light armor to become proficient with medium armor, and you must be proficient with medium armor to become proficient with heavy armor.
            \item You can spend an \glossterm{insight point} to gain proficiency in an additional \glossterm{weapon group}.
            \item You can spend two \glossterm{insight points} to gain proficiency with \glossterm{exotic weapons} from a single \glossterm{weapon group} you are already proficient with.
            \item You can spend an \glossterm{insight point} to learn two \glossterm{common languages} or one \glossterm{rare language} (see \pcref{Communication and Languages}).
        \end{raggeditemize}

    \subsection{Longer Rests}\label{Longer Rests}
        Normally, characters can take a short rest in ten minutes and a long rest in eight hours.
        With this optional rule, a short rest instead requires eight hours of rest, and long rest requires a week.

        This dramatically slows down the narrative pacing of the world, and makes the world feel much more brutal and unforgiving.
        Characters will often be forced to start a combat while missing damage resistance or even hit points, and taking vital wounds can be crippling.

    \subsection{Obscure Magic Items}\label{Obscure Magic Items}
        The base rules of Rise make it fairly easy to identify magic items.
        This keeps the pace of the game up when players find magic items frequently.
        However, you may choose to treat magic items as being more rare and mysterious.
        If you do, make the following changes:
        \begin{itemize}
            \item The \ability{identify item} ability from the Craft and Knowledge skills provides no information about how to use a magic item's properties or what they might be.
                It can still be used to identify whether or not an item is magical.
            \item The Knowledge (items) Knowledge skill is removed entirely.
            \item Magic items are more rare, and therefore more valuable.
                Calculate the prices for all magic items as if they were one rank higher than they actually are.
                Rank 7 magic items cannot be bought for any price - they are simply too rare.
            \item All spells with the \abilitytag{Attune} tag require an additional \glossterm{attunement point} to attune to.
                If magic items are hard to find and use, spellcasters gain a powerful benefit, since their personal attunement spells are still reliably available.
                This change ensures that spellcasters still gain a benefit from their personal access to magic, but they are not drastically more powerful than characters who depend on finding useful magic items.
        \end{itemize}

        You may also want to add complex or unintuitive activation conditions to magic items.
        For example, \mitem{boots of speed} may only function while hopping on one foot, or while you are not wearing socks.
        This can encourage players to experiment more with magic items to figure out how to use them.

    \subsection{Rage Accuracy}
        Normally, a barbarian's \ability{rage} ability provides a +2 accuracy bonus.
        With this variant, raging barbarians instead gain no accuracy bonus, but roll 1d12 instead of 1d10 for their attack rolls while raging.
        The attack's \glossterm{explosion target} is reduced by 3.
        A barbarian's overall accuracy and damage output with this rule essentially equivalent to the normal rule, but they are more likely to get critical hits or completely whiff on important attacks.

        This variant can be more fun for people who like big hits and big misses, and for RPG veterans who naturally associate a barbarian with a d12.
        It is considered a variant rule because not everyone owns a 12-sided die, and you shouldn't need to buy one just to play a barbarian.

    \subsection{Restricted Archetype Order}
        Normally, when a character in Rise levels up, they can freely choose which of their class archetypes they want to rank up (as long as they don't exceed their maximum rank).
        However, this means that most levels require making a choice that may be confusing for newer players.
        The process of leveling up can be simplified if each player chooses an order for their archetypes.

        With this variant, each character has a primary archetype, a secondary archetype, and a tertiary archetype.
        This choice is made at character creation.
        Whenever they increase their maximum rank, they increase their rank in their primary archetype.
        In their next level up, they increase their rank in their secondary archetype, and then finally their tertiary archetype.

    \subsection{Sleeping While Encumbered}
        Normally, characters can sleep in their armor without any penalty.
        This is unrealistic, but it can be time-consuming to make everyone track how their sleeping statistics differ from their waking statistics.
        Being ambushed while sleeping is very rare in most games, so it's generally not worth the hassle.
        However, if you want a more realistic game with more punishing night ambushes, you can use this alternate rule.

        If you play with this alternate rule, resting in armor is difficult.
        If you take a \glossterm{long rest} while you have \glossterm{encumbrance}, you finish your rest with a \glossterm{fatigue level} equal to the value of your encumbrance.
        In addition, only half the time you spend sleeping while you have encumbrance counts as sleep for the purpose of determining your fatigue (see \pcref{Sleep and Fatigue}).

    \subsection{Tap Out}
        With this optional rule, whenever you gain a vital wound, you can ``tap out'' to guarantee that you survive while taking yourself out of the fight.
        If you tap out, you treat the result of the vital roll for that vital wound as a 10, regardless of any bonuses or penalties you would normally have to the vital roll.
        However, you fall unconscious immediately, and you cannot regain consciousness by any means until you finish a \glossterm{short rest}.

        This optional rule significantly reduces the likelihood of character death, and makes fights less likely to impose long-term consequences on characters.
        However, it also makes vital wounds more likely to entirely knock characters out of a fight, which can increase the risk that the entire party is defeated.
