\chapter{Optional Rules}

\section{Attributes}

    \subsection{Other Methods of Attribute Generation}
        Point buy offers the fairest and most customizable system for determining attribute scores, ensuring that players can be almost any character they want to be. However, some groups may wish to determine attribute scores differently. Other options are provided below.

        \subsubsection{Semi-Randomized Point Buy}
            With this method, you have only a small degree of control over your character's attribute scores, but all characters generated in this way are equally powerful. As with the point buy method, all your character's attribute scores start at 0, and you get 10 points to distribute among your character's attribute scores. However, you do not have full control over how to distribute those points.

            Roll 4d6 for each attribute score, dropping a die of your choice with each roll. First roll for the attribute scores that you care about most, and save the least important attribute scores for last. After rolling for an attribute score, sum results on the three highest dice and consult \trefnp{Semi-Randomized Point Buy Results} and spend the appropriate number of points to yield an attribute score, as indicated by \trefnp{Attribute Score Point Costs}. If you do not have enough points remaining to spend the amount indicated by the die roll, spend as many as you can and move on to the next ability.

            If you have points remaining after rolling all of your attribute scores, you may distribute the points freely among your abilities, using the normal point buy rules. You cannot increase any attribute above 3 during this stage.

            After all of your points have been spent, you may swap any two of your attribute scores.

            \begin{dtable}
                \lcaption{Semi-Randomized Point Buy Results}
                \begin{dtabularx}{\columnwidth}{X X X}
                    \tb{Roll} & \tb{Attribute Score} & \tb{Point Cost} \\
                    \hline
                    3-7   & \minus2 & \minus2\fn{1} \\
                    8-9   & \minus1 & \minus1\fn{1} \\
                    10-11 & 0  & 0 \\
                    12-13 & 1  & 1 \\
                    13-14 & 2  & 2 \\
                    15-16 & 3  & 3 \\
                    17    & 4  & 5 \\
                    18    & 5  & 8 \\
                \end{dtabularx}
                1 You gain extra points for having low stats. You can gain these points any number of times per character. \\
            \end{dtable}

            For characters with more extreme attribute scores, use the following approach for each attribute score, starting with 10 points as normal:
            \begin{itemize}
                \item Roll 2d8
                \item Take the average, rounding down
                \item Subtract 3
                \item Spend the points as indicated on \trefnp{Attribute Score Point Costs} until you have no points left.
            \end{itemize}

        \subsubsection{Random Point Buy}
            This method gives you no control over the character whatsoever, while still ensuring that all characters generated are equally powerful. It functions as the semi-randomized point buy method, except that you also randomize the order in which the attribute scores are rolled.

\begin{comment}
            \subsubsection{Weighted Semi-Randomized Point Buy}
                This method gives you more control over your attribute scores, while still requiring you to deal with attribute score advantages or disadvantages that you might not have chosen. It functions like the semi-randomized point buy method, except that you do not roll 3d6 for every attribute score. You have a pool of 18 dice. You can distribute those dice as you choose between each attribute score, except the last one, which is automatically assigned your remaining points. If you roll more than three dice for an attribute score, keep only the highest 3 dice. If you assign fewer than three dice to roll for an attribute score, add 1 to the result for every die less than 3 you are rolling. You can roll as many as six dice for a single attribute score, and you cannot roll less than one die for an attribute score.

                For example, player hoping to play a barbarian might assign six dice to Strength, five dice to Constitution, three dice to Dexterity, and two dice to each of Intelligence and Perception, leaving the rest of the points to Willpower.
\end{comment}

        \subsubsection{Classic Hardcore}

            This method is completely random and can generate very overpowered or underpowered characters. It represents the unfairness of the world, where some people are just better or worse than others. Roll 1d8 for each attribute score and subtract 3 from each result. The result is the attribute score.

            \newpage

\section{Races}

    \subsection{Awakened Animal}

        Awakened animals are animals that have been granted sentience by the \spell{awaken} ritual.
        The abilities of an awakened animal depend on the type of animal it is.

        \parhead{Size} Tiny, Small, or Medium, as original animal.
        \parhead{Attributes} The attributes of an awakened animal depend on its size.
        \subparhead{Medium} No change.
        \subparhead{Small} \plus1 Dexterity, \minus1 Strength.
        \subparhead{Tiny} \plus2 Dexterity, \minus2 Strength.
        \parhead{Speed} As the original animal.
        \parhead{Special Abilities} As the original animal.
        \parhead{Racial Bonus Feat} No racial bonus feat.

        \subsubsection{Sample Awakened Animals}

            \parhead{Cat}

            \subparhead{Size} Tiny. As a Tiny character, a cat gains several benefits and penalties, as described at \pcref{Small Characters}.
            \subparhead{Attributes} \plus2 Dexterity, \minus2 Strength.
            \subparhead{Speed} 20 feet.
            \subparhead{Special Abilities}
            \begin{itemize}
                \itemhead{Scent} A cat has the scent ability (see \pcref{Scent}).
                \itemhead{Claws} A cat's paws end in claws, which it can use to attack (see \pcref{Natural Weapons}). A cat's claws do 1d3 damage.
                \itemhead{Low-light Vision} A cat treats sources of light as if they had double their normal illumination range.
            \end{itemize}

        \subsection{Changeling}

            \parhead{Size} Medium.
            \parhead{Attributes} No change.
            \parhead{Speed} 30 feet.
            \parhead{Special Abilities}
            \begin{itemize}
                \itemhead{Alter Shape} A changeling can alter its physical form in minor ways. As a standard action, a changeling can make a Disguise check with a \plus10 bonus to alter its body. This ability does not alter the changeling's equipment, which may give away its identity unless disguised normally.
            \end{itemize}
            \parhead{Racial Bonus Feat} Any one from the following list: Open Minded, Skill Focus (Bluff, Disguise, Intimidate, Persuasion, or Sense Motive).
            \parhead{Automatic Languages} Common and any one language (except Druidic).
            \parhead{Bonus Languages} Any.

    \subsection{Drakkenfel}

        A drakkenfel is created when a dragon's scales are removed while the dragon still lives. The scales retain much of the dragon's power, and without them, the dragon is cursed to continue its life as a diminished, mortal creature. Drakkenfel are extremely rare, as there are few who are powerful enough to subdue a dragon to extract its scales -- and fewer still who would dare to do so, knowing that they would earn the eternal enmity of all dragons.

        Although are drakkenfel are generally considered to be a lesser creature than true dragons, they have hidden power. While separated from her full draconic essence, a drakkenfel can bridge the divide between the different types of true dragons, gaining the affinity for multiple energy types and special powers. It is rumored that a drakkenfel who regains her scales retains these enhanced abilities, becoming even more powerful than ordinary dragons.

        Physically, a drakkenfel resembles a wyrmling dragon, except that it is completely scaleless. Its skin is leathery and rough, and some drakkenfel bear horrific scars from the ritual that removed their scales. Most drakkenfel retain tatters of their wings, but they are always nonfunctional.

        \parhead{Size} Small.
        \parhead{Attributes} No change.
        \parhead{Speed} 20 feet.
        \parhead{Special Abilities}
        \begin{itemize}
            \itemhead{Draconic Essence} Each drakkenfel was once a type of true dragon. This choice must be made at 1st level, and cannot thereafter be changed. A list of dragons and their associated energy type is given on \tref{Dragon Types}. The drakkenfel is treated as if she had the Draconic Heritage feat in this dragon for the purpose of feats and abilities.
            \itemhead{Energy Resistance} A drakkenfel gains damage reduction equal to five times her level against the energy type associated with her draconic essence.
            \itemhead{Sleeping Dragon} If a drakkenfel recovers her stolen scales, she immediately becomes a true dragon again.
        \end{itemize}
        \parhead{Racial Bonus Feat} Draconic Scales.
        \parhead{Special} At least half of a drakkenfel's class levels must be taken in the drakkenfel class.

    \subsection{Dryaidi}

        Dryaidi are humanoid creatures that resemble plants. They are descended from dryads.

        \parhead{Size} Medium.
        \parhead{Attributes} \plus1 Constitution, \minus1 Dexterity.
        \parhead{Speed} 20 feet.
        \parhead{Special Abilities}
        \begin{itemize}
            \itemhead{Ingrain} As a standard action, a dryaidi can plant roots into natural earth. While ingrained, a dryaid's land speed becomes 5 feet, but she gains a \plus4 bonus to her Fortitude defense against attacks that would move her. Resting for 4 hours while ingrained gains the same benefits that a human would gain from 8 hours of rest, including healing. In addition, the dryaidi acquires nutrients sufficient to replace a day's worth of food and water. Withdrawing ingrained roots is a full-round action.
            \itemhead{Photosynthesis} While in sunlight, a dryaidi gains a \plus10 foot bonus to land speed.
        \end{itemize}
        \parhead{Racial Bonus Feat} Dryad Heritage.

    \subsection{Tieflings}

        Tieflings are humanoid creatures descended from fiends.
        \parhead{Size} Medium.
        \parhead{Attributes} No change.
        \parhead{Speed} 30 feet.
        \parhead{Special Abilities}
        \begin{itemize}
            \itemhead{Darkvision} Tieflings can see in the dark clearly up to 50 feet. Beyond that, they can see dimly, treating areas of darkness as shadowy illumination. Darkvision does not function if a tiefling is in a brightly lit area, and does not resume functioning until 1 round after the tiefling leaves the brightly lit area.
            \itemhead{Energy Resistance} A tiefling has damage reduction against cold, electricity, and fire equal to twice its level.
        \end{itemize}
        \parhead{Racial Bonus Feat} Fiendish Heritage.

\section{Racial Templates}

    Racial templates are applied in addition to the effects of a normal race.
    In exchange for special bonuses, a character with a racial template must take a minimum number of levels in the template class.

    \subsection{Half-Dragon}

        Half-dragons are the offspring of dragons.

        \subsubsection{Template Class Requirements}
            A half-dragon must take the half-dragon template class with his first, fifth, ninth, and thirteenth character levels.
            He may not choose half-dragon as his base class, and he may not voluntarily take additional levels in the half-dragon class.

    \begin{dtable}
        \lcaption{Half-Dragon Progression}
        \begin{dtabularx}{\columnwidth}{>{\ccol}p{\levelcol} >{\ccol}p{\babcolgood} *{3}{>{\ccol}p{\savecol}} >{\lcol}X}
            \tb{Level} & \tb{Combat Prowess} & \tb{Fort} & \tb{Ref} & \tb{Ment} & \tb{Special} \\
            \hline
            \halfdragonprogressionrow{1} & Claws, draconic heritage, keen senses, scales       \\
            \halfdragonprogressionrow{2} & Bite, breath weapon, draconic magic, draconic wings \\
            \halfdragonprogressionrow{3} & Energy immunity, flight, mighty breath              \\
            \halfdragonprogressionrow{4} & Draconic apotheosis                                 \\
        \end{dtabularx}
    \end{dtable}

        \subsubsection{Class Abilities}

            \cf{Drg}{Languages}
            Half-dragons automatically know Draconic, in addition to any other languages provided by their base race.

            \cf{Drg}{Claws}
            The half-dragon's hands are fiercely clawed.
            He can use his hands as a claw attack that deals 1d6 damage for a Medium half-dragon.

            \cf{Drg}{Keen Vision}
            The half-dragon gains \glossterm{low-light vision}, allowing him to treat sources of light as if they had double their normal illumination range.
            If he already has low-light vision, he doubles its benefit, allowing him to treat sources of light as if they had four times their normal illumination range.
            In addition, he gains \glossterm{darkvision} with a 50 foot range, allowing it to see in complete darkness.
            If he already has darkvision, he increases its range by 50 feet.

            \cf{Drg}{Draconic Power}
            The strength of a half-dragon's draconic abilities are determined by his draconic power.
            His draconic power is equal to his Constitution or his character level, whichever is higher.

            \cf{Drg}{Draconic Heritage} The half-dragon is descended from a particular type of true dragon, as described in \tref{Dragon Types}. This choice must be made at 1st level, and cannot thereafter be changed.
            His heritage grants him \glossterm{damage reduction} against damage of his dragon's energy type equal to twice his draconic power.

            The half-dragon may not take any Bloodline feats. His direct draconic ancestry cannot be diluted or augmented by such means.

            \cf{Drg}{Scales} The half-dragon gains a bonus to his Armor defense equal to his levels in the half-dragon template class.

            \cf{Drg}[2nd]{Bite} 
            The half-dragon's mouth is elongated and filled with sharp teeth.
            He can use his mouth as a bite attack that deals 1d8 damage for a Medium half-dragon.

            \cf{Drg}[2nd]{Breath weapon} The half-dragon gains a breath weapon based on the type of dragon he is descended from.
            The shape of the breath weapon is given on \trefnp{Dragon Types}: either a \arealarge, 5 ft. wide line or a \areamed cone.
            He makes a Draconic power vs. Reflex attack against everything in the area.
            Success deals 1d8 damage of his dragon's energy type per two draconic power.
            Critical success deals double damage.
            Failure deals half damage.

            After using his breath weapon, a half-dragon must wait 1d4 rounds before he can use it again.

            \cf{Drg}[2nd]{Draconic Magic} If the half-dragon has any levels in spellcasting classes, he may choose one of them.
            His half-dragon levels increase his spellcasting abilities with that class as if he had gained levels in that class.
            This increases his spells per day, spells known, and spellpower appropriately.
            This ability causes half-dragon to be treated as a magical class for the purpose of the normal multiclassing benefits spellcasters gain (see \pcref{Spellcasters and Multiclassing}).

            \cf{Drg}[2nd]{Draconic Wings}
            The half-dragon grows leathery draconic wings from his back.
            They grant him a glide speed equal to his base land speed (see \pcref{Gliding}, for details).

            \cf{Drg}[3rd]{Draconic Flight}
            The half-dragon can use his wings to fly.
            He gains a fly speed equal to his base land speed, with average maneuverability (see \pcref{Flying}, for details).
            He can fly for a number of rounds equal to his draconic power.
            After that limit is reached, he must rest for 5 minutes before flying again.

            \cf{Drg}[3rd]{Energy Immunity} The half-dragon becomes immune to damage of his dragon's energy type.

            \cf{Drg}[3th]{Mighty Breath}
            The half-dragon's breath weapon improves.
            If it is a line, it becomes a \areahuge, 10 ft. wide line.
            If it is a cone, it becomes a \arealarge cone.

            \cf{Drg}[4th]{Draconic Apotheosis}
            The half-dragon reaches the pinnacle of his draconic nature.
            This grants him several benefits.
            \begin{enumerate}
                \item All of the half-dragon's attributes increase by 1.
                \item His creature type becomes dragon, in place of his original creature type.
                \item He may use his draconic power in place of his spellpower when casting spells from any class.
                \item The damage dealt by his draconic natural weapons increases by one size category.
                \item There is no limit on how long he can fly with his draconic wings.
            \end{enumerate}

\section{Feats}

    \subsection{Drakkenfel}

        Only a character with the drakkenfel race can become a drakkenfel. Drakkenfel function like spellwarped, with the following alterations.

        \cf{Drk}{Innate Magic}[Su]
        A drakkenfel treats her draconic nature as her choice of innate magic.
        This replaces the normal choices of innate magic offered to a spellwarped.
        Her good defense is Fortitude, her key attribute is Intelligence, and she treats Awareness, Knowledge (arcana), and Persuasion as class skills.

        \cf{Drk}{Spellwarp Pool}[Su]
        The drakkenfel gains the following minor ability.
        \subcf{Frightful Legacy}
        The drakkenfel can alter her appearance as a swift action to look more draconic for 5 rounds.
        This can grant her a \plus2 bonus to Intimidate checks.

        \cf{Drk}[2nd]{Surge of Power}[Su]
        The drakkenfel gains the following ability based on her innate magic.
        \subcf{Draconic Form}
        The drakkenfel transforms her body to become like a dragon.
        She gains a bite natural attack, and a claw natural attack for each hand (see \pcref{Natural Weapons}).
        For a Medium creature, the bite deals 1d8 damage, and the claws deal 1d6 damage.
        In addition, she gains a \plus2 bonus to her Armor defense from draconic scales.

        \cf{Drk}[2nd]{Spellwarped Body}[Ex]
        The drakkenfel gains the following ability based on her innate magic.
        \subcf{Draconic Superiority}
        The drakkenfel gains a \plus1 bonus to an attribute of her choice.
        At her 10th drakkenfel level, this bonus applies to all attributes.
        At her 20th drakkenfel level, this bonus increases to \plus2.

        \cf{Drk}[3rd]{Magical Senses}
        The drakkenfel gains the following ability based on her innate magic.
        \subcf{Draconic Senses}
        The drakkenfel gains low-light vision, 50 foot darkvision, and 20 foot blindsense for 1 round.
        If she already has low-light vision, she doubles its benefit, allowing him to treat sources of light as if they had four times their normal illumination range.
        If she already has darkvision, she increases its range by 50 feet.
        If she already has blindsense, she increases its range by 20 feet.

        \cf{Drk}[3rd]{Spellwarped Aspect}[Su]
        The drakkenfel has access to the following spellwarped aspects based on her innate magic, in addition to the general aspects.
        \subcf{Keen Senses}
        The drakkenfel gains low-light vision and 50 foot darkvision.
        \subcf{7th -- Blindsense}
        The drakkenfel gains 50 foot blindsense.
        \subcf{7th -- Dragonshape}
        When the drakkenfel uses her surge of power, she can transform completely into a dragon.
        If she does, the following changes occur.
        \begin{itemize}
            \item Her equipment melds into her body. All physical properties of her equipment, such as armor, have no effect. However, she still gains the magical properties of her equipped items.
            \item Her hands transform completely into claws. She cannot use her claws to wield weapons or use items normally, but she can cast spells.
            \item She increases in size by one size category, increasing the damage of her natural weapons.
            \item She gains a \plus2 bonus to spellpower, and a \plus4 bonus to Armor defense.
        \end{itemize}
        \subcf{11th -- Draconic Size}
        When the drakkenfel uses her surge of power, she can increase her size by one size category.
        The size increase lasts as long as her surge of power does.
        This is a sizing effect, and does not stack with most other sizing effects.
        However, it stacks with the size increase from the dragonshape aspect.

        \subsubsection{Drakkenfel Invocations}
            \parhead{1st -- Breath Weapon}
            As a standard action, the drakkenfel makes a special attack vs. Reflex against everything within an Medium area.
            The shape and damage type of the drakkenfel's breath weapon depends on her draconic essence, as described in \tref{Dragon Types}.
            A successful attack deals 1d6 damage per spellpower.
            A failed attack deals half damage.
            \parhead{1st -- Augment Weapons}
            As a standard action, the drakkenfel can give her natural attacks a \plus1 enhancement bonus.
            This gives a \plus1 bonus to damage and grants her an additional offensive legend point for her natural attacks (see \pcref{Weapon Enhancement Bonuses}).
            This bonus increases by \plus1 at spellpower 4, and every 4 spellpower thereafter.
            \parhead{4th -- Water Breathing}
            As a standard action, the drakkenfel gains the ability to breathe water as if it was air for 1 hour.
            \parhead{6th -- Lightning Breath}
            This invocation functions like the \spell{lightning bolt} spell, except that the drakkenfel breathes the effect from her mouth as a breath weapon.
            \parhead{8th -- Tiring Breath}
            This invocation functions like the \spell{waves of fatigue} spell, except that the drakkenfel breathes the effect from her mouth as a breath weapon.

\section{Feats}

    \feat{Body of the Bending Willow}{Bloodline, Fae}
    \featpre Dryad Heritage.
    \featben You gain a \plus2 bonus to Escape Artist and Stealth checks.

    If you have three or more fae bloodline feats, you can also walk between trees. As a move action, you can spend a fae point to step into an adjacent plant of at least Medium size and out of any other plant of at least Medium size within 100 feet.

    \feat{Body of the Mighty Oak}{Bloodline, Fae}
    \featpre Dryad Heritage.
    \featben You gain a \plus1 bonus to Armor defense.

    If you have three or more fae bloodline feats, you can also ingrain in natural earth or stone.

    \feat{Deep Ingrain}{Bloodline, Fae}
    \featpres Dryad Heritage, Con 3.
    \featben When you ingrain, you may spend a fae point to deeply ingrain your roots. While deeply ingrained, your bonus to Maneuver defense increases to \plus5. In addition, you can draw nutrients from the earth to heal hit points equal to your fae power as a swift action. You can only regain hit points in this way 5 times before you deplete the available nutrients in the area.

    \feat{Dryad Heritage}{Bloodline, Fae}
    \featpre Dryaidi.
    \featben This feat functions like the Fae Heritage feat, except that it grants a different special ability.

    As a standard action, you can spend a fae point to gain the ability to speak with trees. This ability functions like the druid's wild speech ability, except that it only allows you to communicate with trees.

    \feat{Fiendish Heritage}{Bloodline, Fiendish}
    \featpre Tiefling or nongood alignment.
    \featben You have the blood of a fiendish creature in your veins, granting you fiendish power.
    Your fiendish power is equal to your Willpower, or your level \add the number of fiendish bloodline feats you possess, whichever is higher.
    You have a pool with a number of fiend points equal to the number of fiendish bloodline feats you possess.

    As a standard action, you can spend a fiend point to surround yourself in \areamed radius emanation of darkness for \durshort duration.
    All light within the area is reduced to be no brighter than shadowy illumination.
    This typically grants you concealment, allowing you to hide.
    \feat{Photosynthesis}{Bloodline, Fae}
    \featpre Dryad Heritage
    \featben For each you spend an hour in sunlight, you regain one spent fae point.
