\chapter{Optional Rules}

\section{Attributes}

    \subsection{Other Methods of Attribute Generation}
        Point buy offers the fairest and most customizable system for determining attribute scores, ensuring that players can be almost any character they want to be. However, some groups may wish to determine attribute scores differently. Other options are provided below.

        \subsubsection{Simple Random Point Buy}
            With this method, you have only a small degree of control over your attribute scores, but all characters generated in this way are equally powerful.
            As with the point buy method, all your attribute scores start at 0, and you get 10 points to distribute among your attribute scores.
            However, you do not have full control over how to distribute those points.

            For each attribute, starting with the attributes you care about most, roll 1d6.
            You spend that many points on that attribute, ignoring any extra points that can't be spent
            For example, if you roll a 5, you spend 4 points on the attribute, causing you to start with a 3.
            If you do not have enough points remaining to spend the amount indicated by the die roll, spend as many as you can and move on to the next attribute.

            If you have points remaining after rolling all of your attribute scores, you may distribute the points freely among your abilities, using the normal point buy rules.
            You cannot increase the starting value of any individual attribute by more than 1 during this stage.
            If any of your attributes start as a 0, you may choose to lower them to gain the normal benefits from having low attributes (see \pcref{Attribute Penalties}).

            To further limit your character creation options, you may choose to randomize the order in which you roll your attributes instead of rolling them in an order of your choice.

        \subsubsection{Smoothed Random Point Buy}
            This method functions like the Simple Random Point Buy method, except that the resulting attribute values have a smoother distribution, and you can randomly end up with attribute penalties.

            For each attribute, starting with the attributes you care about most, roll 4d6.
            Then, remove any one of the rolls after seeing the results.
            Sum the results of the remaining three dice and spend the appropriate number of attribute points as indicated in \trefnp{Smoothed Random Point Buy Results}.
            If you do not have enough points remaining to spend the amount indicated by the die roll, spend as many as you can and move on to the next ability.

            If you have points remaining after rolling all of your attribute scores, you may distribute the points freely among your abilities, using the normal point buy rules.
            You cannot increase the starting value of any individual attribute by more than 1 during this stage.

            To further limit your character creation options, you may choose to randomize the order in which you roll your attributes instead of rolling them in an order of your choice.

            \begin{dtable}
                \lcaption{Smoothed Random Point Buy Results}
                \begin{dtabularx}{\columnwidth}{X X X}
                    \tb{Roll} & \tb{Base Attribute} & \tb{Point Cost} \tableheaderrule
                    3-4       & \minus2             & 0\fn{1} \\
                    5-6       & \minus1             & 0\fn{2} \\
                    7-8       & 0                   & 0       \\
                    9-10      & 1                   & 1       \\
                    11-12     & 2                   & 2       \\
                    13-15     & 3                   & 4       \\
                    16-18     & 4                   & 6       \\
                \end{dtabularx}
                1 You gain one \glossterm{insight point}. \\
                2 You gain one \glossterm{skill point}. \\
            \end{dtable}

        \subsubsection{Classic Hardcore}

            This method is completely random and can generate very overpowered or underpowered characters.
            It represents the unfairness of the world, where some people are just better or worse than others.
            For each attribute, roll 2d6, take the average (rounded down), and subtract 2.
            If you roll a 1 on both dice, treat the average as a 0.
            The result is your base value for that attribute.

\section{Epic Fate}
    After 21st level, characters no longer gain levels normally.
    However, they can still increase their personal power as they make progress towards their ultimate fate.

    When you reach 21st level, you may choose an epic fate that you qualify for, or you may delay choosing until you meet the prerequisites for your desired fate.
    You do not start with any ranks in you chosen epic fate.
    Each epic fate specifies ways that you can make progress towards that epic fate.
    Whenever you make dramatic progress towards your epic fate, your rank in that epic fate may increase, at the discretion of the Game Master.

    None of the epic fate abilities have a tag to indicate that they are \glossterm{magical} abilities.
    Many of them are not fundamentally \glossterm{mundane} in nature, but they are beyond normal magic, and effects like an \spell{antimagic field} cannot interact with or suppress them.

    \subsection{Artificial Immortality}
        You have sought out strange magical power in search of a way to artificially prolong your life.
        As your power grows, you become increasingly able to resist death and return from it.
        Eventually, you will transcend death entirely.

        \parhead{Prerequisites} You must perform a series of rituals to prepare yourself for immortality, at least one of which must be rank 7 or higher. There are many kinds of immortality that you can pursue with this epic fate, and the exact nature of the rituals will change depending on the type of immortality you pursue. For example, you can have a phylactery regenerate a new body for you like a lich, you can create clones of yourself that you inhabit if your first body dies, or you can modify your body to regenerate after death from mortal wounds. This immortality may change your base species, such as if you become a lich or move your body into a flesh golem. If it does, you retain all benefits and modifiers from your original species other than size and gain the effects of the new species in addition.

        \parhead{Progression} You must discover powerful new magic rituals that support your particular form of immortality. This generally requires exploring sites of ancient magic, gaining favor with powerful creatures who have relevant knowledge or abilities, and independent experimentation based on your findings.

        \subsubsection{Artifical Immortality Ranks}

            \parhead{Rank 1 -- Life After Death} If you die from any cause other than old age, you resurrect according to nature of your chosen immortality. Your specific form of immortality determines where you return, such as at the site of your death, at the current location of your corpse, or in an entirely separate location. The timing of your resurrection may also differ based on your immortality, but you cannot complete your resurrection sooner than one day after the time of your death. After you resurrect in this way, this ability does not function for one week, allowing you to be killed normally.

            \parhead{Rank 2 -- Death Familiarity} You become so familiar with the trauma of injury and death that your body adapts to it.
            You gain a \plus2 bonus to vital rolls.
            In addition, you cannot gain more than two vital wounds from a single damage roll, regardless of the amount of damage you take.

            \parhead{Rank 3 -- Artificial Life} Whenever you resurrect with your \textit{life after death} ability, your new body gains a \plus1 bonus to Strength, Dexterity, and Constitution. This bonus does not stack if you resurrect multiple times. In addition, that resurrection functions even if the cause of your death was old age, and you can control the physical age of your new body.

            \parhead{Rank 4 -- Deathcaller} You are deeply familiar with death, and know how to most effectively inflict it on others.
            Whenever you inflict a \glossterm{vital wound} on a living creature, you may kill that creature outright.

            \parhead{Rank 5 -- True Immortality} You become fully immortal. There is no time limit after the resurrection from your \textit{life after death} ability where you become vulnerable to a true death. In addition, the resurrection can complete as quickly as the end of the next round after your death. If a physical component limits your immortality, such as a phylactery, it can no longer be damaged or destroyed without the direct intervention of a rank 5 Slayer.

    \subsection{Deity}
        People have begun to worship you, putting you on the path to become a deity.
        As your followers grow, you become capable of ever greater miraculous acts, and you can grant your followers some of your power.
        Eventually, you ascend into the pantheon of gods.

        \parhead{Prerequisites} You must have at least a hundred sentient worshippers to choose this epic fate.
        In addition, you must not have any cleric archetypes.

        \parhead{Progression} To progress towards this epic fate, you must gain a significant number of additional worshippers.
        In general, you must at least double your worshippers to progress towards each new rank of this fate, though this can vary widely.
        Having worshippers among many different places is more valuable than converting an isolated group to worship you, though both are helpful.

        \subsubsection{Deity Ranks}
            \parhead{Rank 1 -- Domain Influence} Choose a cleric domain.
            You gain all abilities from that domain except for its mastery ability.
            In addition, your worshippers become eligible to gain cleric archetypes, though they cannot exceed a maximum rank in those archetypes of twice your rank in this epic fate (to a maximum of 8).
            This does not grant additional archetypes to worshippers who have already chosen their three archetypes, and is usually only relevant to NPC worshippers.

            \parhead{Rank 2 -- Prayers} You hear all prayers directed to you.
            Once per week, you can teleport yourself and up to ten \glossterm{allies} any distance within the same plane as a \glossterm{standard action}.
            Your destination must either be a worshipper actively praying to you or a holy place dedicated to you.
            In addition, choose a second cleric domain.
            You gain all abilities from that domain except for its mastery ability.

            \parhead{Rank 3 -- Domain Mastery} Choose a third cleric domain.
            You gain all abilities from that domain.
            In addition, you gain the mastery ability from the domains you chose with your \textit{domain influence} and \textit{prayers} abilities.

            \parhead{Rank 4 -- Demigod} You become a demigod.
            You no longer age normally, and you cannot die from old age.
            You become a \glossterm{planeforged} native to an Aligned Plane matching your alignment (see \pcref{Planes}).
            While you are on that plane, you can teleport to any plane with your \textit{prayers} ability from this epic fate.
            In addition, you can use that teleportation ability once per hour instead of once per week.

            \parhead{Rank 5 -- Deification} You become a deity.
            You are transported to an Aligned Plane matching your alignment, and you gain divine dominion over an amount of territory in that plane.
            While you are in your territory, you can can freely reshape your territory with a thought to match your desires, and you are immune to all damage and \glossterm{conditions}.

            Regardless of which plane you are on, you can teleport to anywhere within your home plane as a \glossterm{standard action}.
            In addition, there is no limit on the number of times you can teleport with your \textit{prayers} ability from this epic fate.

    \subsection{Hero of Legend}
        You are widely known as a hero, rescuing those in need.
        As your deeds of heroism spread, you gain abilities to help you protect others.
        Although you will eventually die, your legend will live on, inspiring others to save people as you did.

        \parhead{Prerequisites} You must be publicly known to be involved with saving at least one major country or similarly large group of people from some sort of disaster to choose this epic fate.
        In addition, you must have a base Willpower of at least 1.

        \parhead{Progression} To progress towards this epic fate, you must publicly contribute to saving large numbers of people from death or other major disasters in a way that builds your reputation.

        \subsubsection{Hero of Legend Ranks}
            \parhead{Rank 1 -- Heroic Intervention} At the start of each phase, you may choose an \glossterm{ally} adjacent to you.
            Whenever that creature would be the target of an attack that phase, you are targeted by that attack instead.
            If the attack would have targeted both you and that ally, the attack only targets you once, not twice.

            \parhead{Rank 2 -- Unstoppable Hero} You gain a \plus4 bonus to all defenses, and you gain a \plus50 bonus to your maximum \glossterm{hit points}.
            In addition, you gain a \plus20 foot bonus to your speed with all of your \glossterm{movement modes}.

            \parhead{Rank 3 -- Sheltering Aura} Your \textit{heroic intervention} ability from this epic fate affects any number of \glossterm{allies} within a \areamed radius \glossterm{emanation} from you.
            Whenever an affected ally is attacked, you teleport into an empty space next to that creature, or into its space if no empty space is available.
            If multiple allies are attacked simultaneously, you can choose where you end up at the end of the series of teleportations.

            \parhead{Rank 4 -- Inspiring Hero} The area of your \textit{heroic intervention} ability increases to a \areagarg radius \glossterm{emanation}.
            In addition, each creature with a mind affected by that ability is so inspired by your example that it gains a \plus1 bonus to its base Willpower permanently.
            This bonus does not stack.

            \parhead{Rank 5 -- Answer the Call} You gain an intuitive sense for when people need your aid.
            Whenever someone on the same plane as you is in danger, you are aware of the existence of that danger.
            You can sense the general category of danger (fire, combat, drowning, etc.) and a very approximate direction and distance.
            This generally allows you to sense if a large number of people are in danger from the same thing.
            As a \glossterm{standard action}, you can teleport any distance within that plane to reach a person in danger.

    \subsection{Slayer}
        You are widely known as a killer of legendary skill.
        As your body count increases, you gain abilities to help you track down and kill increasingly powerful foes.
        Eventually, your powers threaten the gods themselves, allowing you a unique ability to transcend death.

        \parhead{Prerequisites} You must be publicly known to be involved with slaying at least one creature with a challenge rating of 4 and a level of at least 21.

        \parhead{Progression} To progress towards this epic fate, you must publicly contribute to slaying increasingly dangerous and fearsome foes in a way that builds your reputation.

        \subsubsection{Slayer Ranks}
            \parhead{Rank 1 -- Lethality} You gain a \plus4 bonus to \glossterm{power}.
            In addition, whenever you would inflict any number of \glossterm{vital wounds} on a creature, you may inflict twice that many vital wounds.

            \parhead{Rank 2 -- Precision Killer} You gain a \plus4 bonus to \glossterm{accuracy}.
            In addition, you can inflict \glossterm{critical hits} on creatures that would otherwise be immune to critical hits from you due to their size or body structure.

            \parhead{Rank 3 -- Mark of the Slayer} As a \glossterm{standard action}, you can choose to mark any creature you can unambiguously identify.
            This includes any creature you can see, as well as any creature you know the name of and can differentiate from other similar creatures.
            You can only mark one creature at a time, and applying a new mark replaces any previous mark.
            You cannot use this ability to replace a mark that is less than a week old if the recipient of the previous mark still lives.

            This mark is visible on the creature's body with a design that is recognizably yours.
            It appears on top of any clothing or other attempt to conceal it, even if the creature is invisible.
            Anyone can recognize the significance of the mark with a \glossterm{difficulty value} 15 Knowledge (arcana or local) check, and creatures that understand the significance of the mark may refuse to give your target aid of any kind to avoid risking your wrath.

            You know the exact distance and direction to any creature you have marked with this ability that is on the same plane as you.
            As a \glossterm{standard action}, you can create a \glossterm{scrying sensor} adjacent to them that you can see and hear through.
            The sensor lasts as long as you \glossterm{sustain} it as a \glossterm{free action}.
            It moves to stay adjacent to the target, regardless of its speed.

            \parhead{Rank 4 -- Slayer's Journey} As a \glossterm{standard action}, you can \glossterm{teleport} yourself and up to ten \glossterm{allies} any distance within the same plane to the location of a creature affected by your \textit{mark of the slayer} ability from this epic fate.
            You cannot precisely choose the destination of this ability, and it does not leave you immediately adjacent to the marked creature.
            Generally, it leaves you just outside any sort of fortress or defenses the marked creature has constructed.
            After you use this ability, you cannot use it to travel to the same creature for a day.
            This does not limit your ability to travel to a different creature if you mark a different creature.

            \parhead{Rank 5 -- Godslayer}
            Your attacks ignore most forms of general immunity.
            This does not help you ignore specific immunities, such as fire elemental's immunity to fire damage.
            However, you can destroy artifacts and even inflict damage and conditions on deities in their divine dominion.
            As a result, even deities fear to interfere with you directly.
            If you ever die, you can generally threaten or fight your way past any planar guardians to leave your afterlife whenever you want.
            After you do this once, you become a \glossterm{planeforged} native to your afterlife plane, since your new body is formed from the raw material of that plane (see \pcref{Planes}).

\section{Uncommon Species}\label{Uncommon Species}

    \subsection{Animal Hybrid}
        Animal hybrids are humanoid creatures that are a combination of humans and animals.
        The abilities of an animal hybrid depend on the type of animal it is based on.

        \parhead{Size} Medium.
        \parhead{Attributes} No change.
        \parhead{Speed} 30 feet.
        \parhead{Special Abilities} As the original animal.
        \parhead{Species Feat Options} Any feat strongly associated with the chosen animal.
        \parhead{Automatic Languages} Common and any one \glossterm{common language} (see \tref{Common Languages}).

        \subsubsection{Sample Animal Hybrids}

            \parhead{Hybrid Shark}

            \subparhead{Special Abilities}
            \begin{itemize}
                \itemhead{Bloodscent} A hybrid shark has the scent ability (see \pcref{Scent}).
                    In addition, it gains a \plus10 bonus to Awareness checks to detect blood.
                \itemhead{Bite} A hybrid shark's mouth is elongated, which it can use as a bite attack (see \pcref{Natural Weapons}).
                    A hybrid shark's bite deals 1d8 damage.
                \itemhead{Swim Speed} A hybrid shark has a swim speed equal to the base speed for its size.
            \end{itemize}
            \subparhead{Species Feat Options} Awareness Specialization, Survival Specialization, Swift, or Swim Specialization.

            \parhead{Hybrid Wolf}

            \subparhead{Special Abilities}
            \begin{itemize}
                \itemhead{Scent} A hybrid wolf has the scent ability (see \pcref{Scent}).
                \itemhead{Bite} A hybrid wolf's mouth is elongated, which it can use as a bite attack (see \pcref{Natural Weapons}).
                    A hybrid wolf's bite deals 1d8 damage.
                \itemhead{Low-light Vision} A hybrid wolf treats sources of light as if they had double their normal illumination range.
            \end{itemize}
            \subparhead{Species Feat Options} Awareness Specialization, Rapid Reaction, Stealth Specialization, Survival Specialization, or Swift.

    \subsection{Awakened Animal}

        Awakened animals are animals that have been granted sentience by the \spell{awaken} ritual.
        The abilities of an awakened animal depend on the type of animal it is.

        \parhead{Size} Small or Medium, as original animal.
        \parhead{Attributes} The attributes of an awakened animal depend on its size.
        \subparhead{Medium} No change.
        \subparhead{Small} \minus2 base Strength, \plus1 base Dexterity.
        \parhead{Speed} As the original animal.
        \parhead{Special Abilities} As the original animal.
        \parhead{Species Feat Options} Any feat strongly associated with the chosen animal.
        \parhead{Automatic Languages} Common.

        \subsubsection{Sample Awakened Animals}

            \parhead{Cat}

            \subparhead{Size} Small. This gives a cat a \plus4 bonus to the Stealth skill, among other effects (see \pcref{Size in Combat}).
            \subparhead{Attributes} Being Small gives cats a \minus2 penalty to base Strength and a \plus1 bonus to base Dexterity.
            \subparhead{Speed} 20 feet.
            \subparhead{Special Abilities}
            \begin{itemize}
                \itemhead{Scent} A cat has the scent ability (see \pcref{Scent}).
                \itemhead{Claws} A cat's paws end in claws, which it can use to attack (see \pcref{Natural Weapons}). A cat's claws do 1d6 damage.
                \itemhead{Low-light Vision} A cat treats sources of light as if they had double their normal illumination range.
            \end{itemize}
        \parhead{Species Feat Options} Awareness Specialization, Climb Specialization, Flexibility Specialization, Rapid Reaction, Stealth Specialization, or Swift.

    \subsection{Changeling}

        \parhead{Size} Medium.
        \parhead{Attributes} No change.
        \parhead{Speed} 30 feet.
        \parhead{Special Abilities}
        \begin{itemize}
            \itemhead{Alter Shape} A changeling can alter its physical form in minor ways. As a standard action, a changeling can make a Disguise check with a \plus10 bonus to alter its body. This ability does not alter the changeling's equipment, which may give away its identity unless disguised normally.

            This is a \glossterm{magical} ability.
        \end{itemize}
        \parhead{Bonus Languages} Any.
        \parhead{Species Feat Options} any Skill feat.
        \parhead{Automatic Languages} Common, any two \glossterm{common languages}.

    \subsection{Dragon}
        Ancient dragons are magical creatures of immense power and wisdom, and are far more powerful than any ordinary character of the same level.
        However, young dragons can be played as characters, though their unique abilities do pose unique challenges.

        \parhead{Size} Small.
        \parhead{Attributes} \minus2 base Strength, \plus1 base Dexterity.
        \parhead{Speed} 20 feet.
        \parhead{Special Abilities}
        \begin{itemize}
            \itemhead{Dragon Archetype} You only gain two class archetypes instead of three.
                Instead, you treat the Dragon Archetype as one of your archetypes, and you gain ranks in that just like you gain ranks in class archetypes.
        \end{itemize}
        \parhead{Automatic Languages} Common, Draconic, any one \glossterm{common language}.

        \subsubsection{Dragon Archetype}
            % TODO: dragons probably do not need a rank 0 armor defense bonus; they certainly can't have that and a +2 AD from base class
            \cf{Dgn}[0]{Draconic Scales} You gain a \plus1 bonus to Armor defense.
            \cf{Dgn}[0]{Draconic Senses} You have  \glossterm{darkvision} with a range of 60 feet and \glossterm{low-light vision}.
            \cf{Dgn}[0]{Draconic Weapons} You have a bite natural weapon and two claw natural weapons.
                For details, see \pcref{Natural Weapons}.
            \cf{Dgn}[0]{Draconic Wings} You gain scaly wings that sprout from your back.
                These wings grant you a glide speed equal to the \glossterm{base speed} for your size (see \pcref{Gliding}).
                The wings themselves are \glossterm{mundane}, but the ability to fly and glide with them is \glossterm{magical}.
            \cf{Dgn}[0]{Dragon Type} Choose a type of dragon from among the dragons on \trefnp{Dragon Types}.
                You are that type of dragon.
                You are immune to the damage type dealt by that dragon's breath weapon.
            \cf{Dgn}[0]{Limited Equipment} A dragon's claws are not able to effectively wield shields or manufactured weapons.
            They can wear armor, but it is treated as barding instead of normal armor, increasing its cost.
            In general, dragon-fitted barding is rare or nonexistent even in large cities, so a dragon's armor must usually be created specifically for them.

            \cf{Dgn}[1]{Draconic Breath} You can use the \textit{breath weapon} ability as a \glossterm{standard action}.
            % +1d to compensate for cooldown
            \begin{freeability}{Breath Weapon}
                Make an attack vs. Reflex against everything in the area defined by your dragon type (see \trefnp{Dragon Types}).
                You may use your Constitution in place of your Strength to determine your \glossterm{power} with this ability.
                After you use this ability, you \glossterm{briefly} cannot use it again.
                \hit Each target takes damage equal to 1d10 plus half your \glossterm{power}.
                The damage type is defined by your dragon type.

                \rankline
                \rank{2} The damage increases to 2d6.
                    In addition, the area affected by your breath weapon increases.
                    A line breath weapon becomes a \arealarge, 5 ft.\ wide line.
                    A cone breath weapon becomes a \areamed cone.
                \rank{3} The damage increases to 2d8.
                \rank{4} The damage increases to 2d10.
                    In addition, the area affected by your breath weapon increases.
                    A line breath weapon becomes a \areahuge, 10 ft.\ wide line.
                    A cone breath weapon becomes a \arealarge cone.
                \rank{5} The damage increases to 4d6.
                \rank{6} The damage increases to 4d8.
                    In addition, the area affected by your breath weapon increases.
                    A line breath weapon becomes a \areagarg, 15 ft.\ wide line.
                    A cone breath weapon becomes a \areahuge cone.
                \rank{7} The damage increases to 4d10.
            \end{freeability}

            \cf{Dgn}[2]{Draconic Flight}[Magical] Your wings grow larger, granting you a limited ability to fly.
            You gain a \glossterm{fly speed} equal to the \glossterm{base speed} for your size with a maximum height of 15 feet (see \pcref{Flying}).
            At the start of each phase, you can increase your \glossterm{fatigue level} by one to ignore this height limit until the end of the round.

            \cf{Dgn}[3]{Draconic Bulk} Your size category increases to Medium.
            This increases the \glossterm{base speed} for your size.
            You reduce your base Dexterity by 1 and increase your base Strength by 2.
            In addition, you gain a \plus2 bonus to your \glossterm{power} with all abilities.

            \cf{Dgn}[4]{Draconic Body} You gain a \plus1 bonus to Armor defense.
            In addition, you gain a \plus1d damage bonus with all \glossterm{natural weapons}.

            \cf{Dgn}[5]{Greater Draconic Flight} The maximum height from your \textit{draconic flight} ability increases to 60 feet.
            In addition, you gain a \plus10 foot bonus to your fly speed with that ability.

            \cf{Dgn}[6]{Greater Draconic Bulk} Your size category increases to Large.
            The speed bonus from your \textit{draconic bulk} ability increases to \plus20 feet, the attribute modifiers to Dexterity and Strength increase to \minus2 and \plus3 respectively, and the power bonus increases to \plus6.
            You gain a slam natural weapon (see \pcref{Natural Weapons}).
            % TODO: too fast?
            In addition, you gain a \plus30 foot bonus to your fly speed with your \textit{draconic flight} ability, but your maneuverability drops to poor maneuverability (see \pcref{Flying Maneuverability}).

            \cf{Dgn}[7]{Greater Draconic Body} The defense bonus from your \textit{draconic body} ability increases to \plus2.
            In addition, the damage bonus increases to \plus2d.

        \subsubsection{Basic Class Abilities}
            If choose dragon as your base class, you gain the following abilities.
            % 4 insight, 5 skill, 8 AP, 6 fatigue, 2 armor, 0 weapons = 25 points

            \cf{Drg}{Defenses}
            You gain the following bonuses to your \glossterm{defenses}: \plus7 Fortitude, \plus3 Reflex, \plus5 Mental.

            \cf{Drg}{Resources} You have the following \glossterm{resources}:
            \begin{itemize}
                \item Two \glossterm{insight points}, which you can spend to gain additional abilities or proficiencies (see \pcref{Insight Points}).
                \item Five \glossterm{trained skills}, which you can spend to learn skills (see \pcref{Trained Skills}).
                \item Three \glossterm{attunement points}, which you can use to attune to items and abilities that affect you (see \pcref{Attunement Points}).
                \item A \plus3 bonus to your \glossterm{fatigue tolerance}, which makes it easier for you to use powerful abilities that fatigue you (see \pcref{Fatigue}).
            \end{itemize}

            \cf{Drg}{Weapon Proficiencies} 
            You are not proficient with any weapon groups, even simple weapons.
            You are still proficient with your natural weapons.

            \cf{Drg}{Armor Proficiencies} 
            You are proficient with light and medium armor.
            Armor shaped appropriately for dragons can be hard to find, and may need to be crafted individually for the dragon.

            \cf{Drg}{Skills}
            You have the following \glossterm{class skills}:
            \begin{itemize}
                \item \subparhead{Strength} Climb, Jump, Swim.
                \item \subparhead{Dexterity} Balance, Flexibility, Stealth.
                \item \subparhead{Constitution} Endurance.
                \item \subparhead{Intelligence} Craft, Deduction, Knowledge (arcana), Medicine.
                \item \subparhead{Perception} Awareness, Creature Handling, Social Insight, Spellsense, Survival.
                \item \subparhead{Other} Deception, Intimidate, Persuasion.
            \end{itemize}

    \subsection{Drakkenfel}

        A drakkenfel is created when a dragon's scales are removed while the dragon still lives. The scales retain much of the dragon's power, and without them, the dragon is cursed to continue its life as a diminished, mortal creature.
        Drakkenfel are extremely rare, as there are few who are powerful enough to subdue a dragon to extract its scales -- and fewer still who would dare to do so, knowing that they would earn the eternal enmity of dragonkind.

        Physically, a drakkenfel resembles a wyrmling dragon, except that it is completely scaleless.
        Its skin is leathery and rough, and some drakkenfel bear scars from the ritual that removed their scales.
        Most drakkenfel retain tatters of their wings, but they are always nonfunctional.

        \parhead{Size} Small.
        \parhead{Attributes} No change. This replaces the normal bonuses and penalties to attributes from being Small.
        \parhead{Speed} 25 feet.
        \parhead{Special Abilities}
        \begin{itemize}
            \itemhead*{Low-light Vision}: Drakkenfel treat sources of light as if they had double their normal illumination range.
            \itemhead{Draconic Essence} Each drakkenfel was once a type of true dragon.
                When creating a drakkenfel, choose which type of dragon it used to be.
                This is inherent to the drakkenfel, and cannot be changed.
                A list of dragons and their associated energy type is given on \tref{Dragon Types}.
            \itemhead{Damage Tolerance} A drakkenfel gains \plus4 bonus to its \glossterm{defenses} against attacks that deal damage of the type associated with its \textit{draconic essence}.
            \itemhead{Bite} A drakkenfel's mouth can be used to bite (see \pcref{Natural Weapons}). A drakkenfel's bite attack deals \plus0d damage.
            \itemhead{Sleeping Dragon} If a drakkenfel recovers its stolen scales, it immediately becomes a true dragon again.
                Its statistics become identical to its statistics before losing its scales, including level.
        \end{itemize}
        \parhead{Species Feat Options} Draconic Heritage. The type of dragon chosen for the drakkenfel's \textit{draconic ancestry} must match its \textit{draconic essence}.
        \parhead{Automatic Languages} Common, Draconic, any one \glossterm{common language}.

    \subsection{Dryaidi}

        Dryaidi are humanoid creatures that resemble plants. They are descended from dryads.

        \parhead{Size} Medium.
        \parhead{Attributes} \plus1 base Constitution, \minus1 base Dexterity.
        \parhead{Speed} 25 feet.
        \parhead{Special Abilities}
        \begin{itemize}
            \itemhead{Ingrain} A dryaidi use the \textit{ingrain} ability as a standard action.
                \begin{freeability}{Ingrain}
                    The dryaidi's land speed becomes 5 feet, regardless of any modifiers that normally apply.
                    It gains a \plus4 bonus to Fortitude defense and a \plus1 bonus to Armor defense.
                    When moving, it may ignore \glossterm{difficult terrain} of any kind.
                    If the dryaidi takes a \glossterm{long rest} while this ability is active, it acquires nutrients sufficient to replace a day's worth of food and water.

                    This ability lasts until the dryaidi ends it as a standard action.
                \end{freeability}
            \itemhead{Photosynthesis} While in sunlight, a dryaidi gains a \plus5 foot bonus to land speed.
            \itemhead{Plant Nature} A dryaidi is considered both a creature and a plant.
        \end{itemize}
        \parhead{Species Feat Options} Herbalist, Mental Magic, Regenerator, Sphere Focus: Verdamancy, or Toughness.
        \parhead{Automatic Languages} Common, Sylvan.

    \subsection{Kit}

        Kit are humanoid creatures that have noticeable foxlike characteristics.
        They are descended from natural fox spirits.
        All kit have at least one tail, and some have multiple tails.
        Their tails are distinctly fluffy and fox-like, and most kit typically put effort into concealing their tails to avoid revealing their true nature.

        \parhead{Size} Medium.
        \parhead{Attributes} No change.
        \parhead{Speed} 30 feet.
        \parhead{Special Abilities}
        \begin{itemize}
            \itemhead{Foxlike Agility} A kit gains a \plus2 bonus to the Balance and Stealth skills.
            \itemhead{Illusory Guise} As a standard action, a kit can magically disguise its physical appearance in minor ways.
                This functions like the \textit{disguise creature} ability with a \plus4 bonus, except that a kit cannot change the appearance of its equipment, creature type, or number of limbs, including any tails it may have (see \pcref{Disguise Creature}).
                This is a \glossterm{magical} ability.
                It lasts until the kit \glossterm{dismisses} it as a free action or uses this ability again.
            \itemhead{Instictive Trickster} A kit gains a \plus2 bonus to the Deception and Social Insight skills.
            \itemhead{Low-Light Vision} A kit treats sources of light as if they had double their normal illumination range.
        \end{itemize}
        \parhead{Species Feat Options} Skill Specialization: Balance, Skill Specialization: Deception, Skill Specialization: Social Insight, Skill Specialization: Stealth, Swift
        \parhead{Automatic Languages} Common, any one \glossterm{common language}.

    \subsection{Naiadi}

        Naiadi are humanoid creatures descended from naiads.
        Most naiadi are unusually physically appealing, but show no other outward signs of their heritage.

        \parhead{Size} Medium.
        \parhead{Attributes} No change.
        \parhead{Speed} 30 feet.
        \parhead{Special Abilities}
        \begin{itemize}
            \itemhead{Enchanting Appearance} A naiadi gains a \plus2 bonus to the Creature Handling, Perform, and Persuasion skills.
            \itemhead{Low-light Vision} A naiadi treats sources of light as if they had double their normal illumination range.
            \itemhead{Water Affinity} A naiadi has a \glossterm{swim speed} equal to the \glossterm{base speed} for their size.
                In addition, they can breathe clean water like a human breathes air.
            \itemhead{Create Water} A naiadi can cast the \spell{create water} cantrip.
                When they do so, they do not require verbal or somatic \glossterm{casting components}, and their spellcasting rank is considered to be equal to their rank in their highest rank archetype.
                If they would already know that cantrip through the Aquamancy sphere, the volume of water created with the cantrip doubles.
        \end{itemize}
        \parhead{Species Feat Options} Boongiver, Leadership, Mental Magic, Perform Specialization, Persuasion Specialization, Sphere Focus: Aquamancy, or Swim Specialization.
        \parhead{Automatic Languages} Common, Sylvan, any one \glossterm{common language}.

    \subsection{Orc}
        Orcs are green-skinned humanoid creatures known for their strength and brutality.

        \parhead{Attributes} \plus1 starting Strength, \minus1 starting Intelligence.
        \parhead{Speed} 30 feet.
        \parhead{Special Abilities}
        \begin{itemize}
            \itemhead*{Darkvision}: Orcs can see in the dark clearly up to 60 feet.
                Darkvision does not function if an orc is in a brightly lit area, and does not resume functioning until the end of the next round after the orc leaves the brightly lit area.
            \itemhead*{Intimidating}: Orcs gain a \plus3 bonus to the Intimidate skill (see \pcref{Intimidate}).
            \itemhead*{Mighty}: You gain a \plus1 bonus to your Strength for the purpose of determining your \glossterm{weight limits}.
            \itemhead*{Powerful}: You gain a bonus equal to a quarter of your level (minimum 1) to your \glossterm{power} with \glossterm{mundane} abilities.
        \end{itemize}
        \parhead{Automatic Languages} Common, Orc.

    \subsection{Tieflings}

        Tieflings are humanoid creatures descended from fiends.
        \parhead{Size} Medium.
        \parhead{Attributes} No change.
        \parhead{Speed} 30 feet.
        \parhead{Special Abilities}
        \begin{itemize}
            \itemhead*{Darkvision}: Tieflings can see in the dark clearly up to 60 feet. Darkvision does not function if a tiefling is in a brightly lit area, and does not resume functioning until the end of the next round after the tiefling leaves the brightly lit area.
            \itemhead*{Fire Tolerance} (Magical): Tieflings are \glossterm{impervious} to fire damage.
            \itemhead*{Infernal Darkness} (Magical): A tiefling can use the \textit{infernal darkness} ability as a \glossterm{standard action}.
                \begin{durationability}{Infernal Darkness}[\abilitytag{Sustain} (minor)]
                    \rankline
                    \target{One \glossterm{zone} within \rngmed range}
                    You can choose this ability's radius, up to a maximum of a \areamed radius.
                    Light within or passing through the area is dimmed to be no brighter than \glossterm{shadowy illumination}
                    Any object or effect which blocks light also blocks this spell's effect.
                \end{durationability}
            \itemhead*{Infernal Presence}: Tieflings gain a \plus2 bonus to the Deception and Intimidate skills.
        \end{itemize}
        \parhead{Species Feat Options} \featref*{Deception Specialization}, \featref*{Executioner}, \featref*{Intimidate Specialization}, \featref*{Spellwarped}, or \featref*{Sphere Focus: Pyromancy}.
        \parhead{Automatic Languages} Abyssal, Common, any one \glossterm{common language}.


\section{Classes}
    \subsection{Bard}
        A bard is a rogue with the ability to perform magical feats through music.
        It is unclear whether bards actually draw power from music in the same way that druids draw power from nature, or whether they simply channel their innate magical talent through music.
        The bard class functions like the rogue class, with the following exceptions:
        \begin{itemize}
            \item A bard cannot choose the \textit{assassin} archetype. However, the \textit{arcane magic} sorcerer archetype is considered to be part of their class, and they may choose that archetype without spending insight points to multiclass.
            \item A bard casts spells without \glossterm{somatic components}.
            \item A bard can only cast spells while sustaining a performance with the Perform skill. In addition, they can cast bard spells with the \abilitytag{Focus} tag while sustaining a performance, despite the normal limitation that Focus abilities cannot be used during a performance. This performance can be either a mundane performance or a \textit{bardic performance} ability.
        \end{itemize}

    \subsection{Faebonder}
        A faebonder is a warlock who made their pact with a fae creature instead of a demon or devil.
        The faebonder class functions like the warlock class, with the following exceptions:
        \begin{itemize}
            \item The magic source for the faebonder class is nature magic instead of pact magic.
                This changes the \glossterm{mystic spheres} a faebonder has access to and all other effects based on their source of magic.
                However, they still require both \glossterm{verbal components} and \glossterm{somatic components} to cast spells from the faebonder class (see \pcref{Casting Components}).
            \item A faebonder cannot choose the \textit{blessings of the abyss} archetype. However, the \textit{elementalist} druid archetype is considered to be part of their class, and they may choose that archetype without spending insight points to multiclass.
            \item Faebonders add Knowledge (nature) to their class skill list and remove Knowledge (planes).
        \end{itemize}

    \subsection{Favored Soul}
        A favored soul is a warlock who made their pact with a deity instead of a demon or devil.
        This is an unusual arrangement, as deities would normally influence their clerics to achieve their aims.
        However, in special circumstances, a deity may want to take a more direct hand in mortal affairs.
        The favored soul class functions like the warlock class, with the following exceptions:
        \begin{itemize}
            \item The magic source for the favored soul class is divine magic instead of pact magic.
                This changes the \glossterm{mystic spheres} a favored soul has access to and all other effects based on their source of magic.
                However, they still require both \glossterm{verbal components} and \glossterm{somatic components} to cast spells from the favored soul class (see \pcref{Casting Components}).
            \item A favored soul cannot choose the \textit{blessings of the abyss} archetype. However, the \textit{domain influence} cleric archetype is considered to be part of their class, and they may choose that archetype without spending insight points to multiclass.
            \item Favored souls add Knowledge (religion) to their class skill list and remove Knowledge (planes).
        \end{itemize}

    \subsection{Shaman}
        A shaman, like a cleric, is a divine worshipper.
        However, while clerics worship powerful, well-established deities, shamans worship more primitive deities of lesser power.
        As a result, their divine powers are more limited and take different forms.
        Shamans are common among less civilized humanoid societies like bugbears.
        The shaman class functions like the cleric class, with the following exceptions:
        \begin{itemize}
            \item The magic source for the shaman class is nature magic instead of divine magic.
                This changes the \glossterm{mystic spheres} a shaman has access to and all other effects based on their source of magic.
            \item A shaman cannot choose the \textit{divine spell mastery} archetype. However, the \textit{elementalist} druid archetype is considered to be part of their class, and they may choose that archetype without spending insight points to multiclass.
            \item A shaman cannot gain access to more than two \textit{mystic spheres} from the magic source granted by the shaman class by any means.
            \item Shamans add Knowledge (nature) to their class skill list and remove Knowledge (planes).
        \end{itemize}


\section{Feats}

% Feat names must follow ``I have'' or ``I am (a)''.
\begin{longtablewrapper}
    \begin{longtable}{>{\lcol}p{11em} >{\lcol}p{12em} l >{\lcol}p{8em} >{\lcol}p{3em}}
        \lcaption{Optional Feats}\\
        \tb{General Feats} & \tb{Prerequisites} & \tb{Benefits}          & \tb{Feat Types} & \tb{Page}                 \\
        \featref{Infernal Heritage}             & Tiefling           & Gain aspects of demons & Bloodline       & \featpref{Infernal Heritage} \\
        \featref{Naiad Heritage}                & Naiadi            & Gain aspects of naiads & Bloodline       & \featpref{Naiad Heritage} \\
    \end{longtable}
\end{longtablewrapper}

    \section{Feat Descriptions}

    \begin{feat}{Infernal Heritage}{Bloodline}
        \featpre Tiefling species.

        \ff[1]{Infernal Rebuke} You can use the \textit{infernal rebuke} ability as a standard action.
        \begin{freeability}{Infernal Rebuke}[Instant]
            \abilitytag{Magical}
            \rankline
            Make an attack vs. Fortitude against one creature within \rngshort range.
            You gain a \plus2 bonus to \glossterm{accuracy} with this attack if the target attacked you during the previous round.
            \hit The target takes 1d8 plus \glossterm{power} fire damage.

            \rankline
            \featlevel{3} The damage increases to 1d10.
            \featlevel{6} The damage increases to 2d8.
            \featlevel{9} The damage increases to 2d10.
            \featlevel{12} The damage increases to 4d8.
            \featlevel{15} The damage increases to 4d10.
            \featlevel{18} The damage increases to 6d10.
            \featlevel{21} The damage increases to 7d10.
        \end{freeability}

        \ff[3]{Infernal Ancestry} You deepen your connection to a particular aspect of your demonic ancestry.
        You gain your choice of one of the following abilities.
        \begin{itemize}
            \item Abysswalker: As a standard action, you can \glossterm{teleport} yourself into an unoccupied location within \medrange on a stable surface that can support your weight.
            If the destination is invalid, this ability fails with no effect.
            \item Tempting Allure: You gain a \plus2 bonus to the Deception, Disguise, and Persuasion skills.
            \item Unholy Might: You gain either two claw natural weapons or one bite natural weapon (see \pcref{Natural Weapons}).
                In addition, you gain a \plus2 bonus to Strength for the purpose of determining your weight limits (see \pcref{Weight Limits}).
        \end{itemize}

        \ff[6]{Greater Infernal Darkness} The maximum radius affected by your \textit{infernal darkness} ability increases to a \arealarge radius.
        In addition, when you use the ability you may choose whether it also blocks \glossterm{darkvision} from seeing into the area.

        \ff[9]{Hellfire Conduit} Whenever you deal fire damage, you may also treat that damage as being pure energy damage.
        This can help you deal damage to enemies that are highly resistant to fire damage.
        In addition, you treat all fire damage you take as \glossterm{environmental damage}.

        \ff[12]{Greater Infernal Ancestry} The benefits of your \textit{infernal ancestry} ability improve.
        \begin{itemize}
            \item Abysswalker: You can use your \textit{infernal ancestry} ability to teleport as a move action instead of as a standard action.
                When you do, you \glossterm{briefly} cannot use it as a move action again.
                In addition, when you use it to teleport as a standard action, the range increases to \distrange.
            \item Tempting Allure: The bonuses from your \textit{infernal ancestry} ability increase to \plus3.
                In addition, you gain the ability to cast the \spell{charm} spell.
                It does not have any \glossterm{somatic components} or \glossterm{verbal components} for you.
            \item Unholy Might: The bonus to Strength from your \textit{infernal ancestry} ability increases to \plus4.
                In addition, you gain a \plus3 bonus to your \glossterm{power}.
        \end{itemize}

        \ff[15]{Supreme Infernal Darkness} You can use your \textit{infernal darkness} ability as a \glossterm{minor action}.
        In addition, your darkvision functions within your own \textit{infernal darkness} even if you block all other sources of darkvision from working.

        \ff[18]{Greater Hellfire Conduit} Whenever you deal fire damage, you may also treat that damage as being physical damage.
        In addition, you are immune to \glossterm{fire damage}.

        \ff[21]{Supreme Infernal Ancestry} The benefits of your \textit{infernal ancestry} ability improve.
        \begin{itemize}
            \item Abysswalker: When you use your \textit{infernal ancestry} ability to teleport as a move action, it no longer briefly prevents you from using it as a move action again.
                In addition, when you use it to teleport as a standard action, the range increases to 1,000 feet.
            \item Tempting Allure: The bonuses from your \textit{infernal ancestry} ability increase to \plus4.
                In addition, you gain the ability to cast the \spell{dominate monster} spell.
                It does not have any \glossterm{somatic components} or \glossterm{verbal components} for you.
            \item Unholy Might: The bonus to Strength from your \textit{infernal ancestry} ability increases to \plus6.
                In addition, the power bonus increases to \plus9.
        \end{itemize}
    \end{feat}

    \begin{feat}{Naiad Heritage}{Bloodline}
        \featpre Naiadi species.

        \ff[1]{Water Bond}[Magical] You can form a bond with a fresh stream, lake, or other Gargantuan or larger body of fresh water (not salt water).
        Forming a bond or severing a bond takes one week of meditation and ritual, periodically interrupted by rest.
        Forming a bond also requires asking permission from the water.
        Any individual body of water can only be bonded to one naiad or naiadi in this way.

        As long as your bonded water remains clean, pure, and large enough to be a valid subject of bonding, you gain a \plus1 bonus to Mental defense and a bonus to your maximum \glossterm{hit points} equal to your level.
        If your bonded water becomes contaminated or shrinks below the minimum size, you take a \minus1 penalty to Mental defense and a penalty to your maximum hit points equal to your level until you sever the bond.
        You can passively observe the general health and status of water you are bonded to, including knowing when significant pollutants enter the water and when the water grows or shrinks significantly.

        \ff[1]{Underwater Freedom} You reduce your penalties for fighting underwater by 2 (see \pcref{Underwater Combat}).

        \ff[3]{Greater Enchanting Appearance} The bonuses from your \textit{enchanting appearance} species ability increase to \plus4.

        \ff[6]{Aqueous Form}[Magical] You can cast the \spell{aqueous form} spell.
        When you do, you do not require verbal or somatic \glossterm{casting components}, and your spellcasting rank is considered to be equal to your rank in your highest rank archetype.

        \ff[6]{Freshwater Fountain}[Magical] The volume of water you can create with the \spell{create water} cantrip increases by five times.

        \ff[9]{Greater Water Bond}[Magical] The bonus to Mental defense from your \textit{water bond} ability increases to \plus2.

        \ff[9]{Greater Underwater Freedom} The penalty reduction from your \textit{underwater freedom} ability increases to 4.

        \ff[12]{Supreme Enchanting Appearance} The bonuses from your \textit{enchanting appearance} species ability increase to \plus6.

        \ff[15]{Fluidseeker}[Magical] You gain a \plus1 bonus to \glossterm{accuracy} against creatures significantly composed of water or watery fluids.
        This is true of almost all living creatures.

        \ff[18]{Supreme Water Bond}[Magical] The number of hit points granted by your \textit{water bond} ability increases to twice your level, and the bonus to Mental defense increases to \plus3.

        \ff[21]{Greater Aqueous Form} When you cast the \spell{aqueous form} spell, it does not have the \abilitytag{Attune} (self) tag.
        Instead, it lasts until you \glossterm{dismiss} it as a \glossterm{free action}.
        In addition, it does not reduce your \glossterm{damage resistance}.
    \end{feat}

\section{Gameplay Options}
    \subsection{Tap Out}
        With this optional rule, whenever you gain a vital wound, you can ``tap out'' to guarantee that you survive while taking your character out of the fight.
        If you tap out, you treat the result of the vital roll for that vital wound as a 10, regardless of any bonuses or penalties you would normally have to the vital roll.
        However, you fall unconscious immediately, and you cannot regain consciousness by any means until you take a \glossterm{short rest}.

        This optional rule significantly reduces the likelihood of character death, and makes fights less likely to impose long-term consequences on characters.
        However, it also makes vital wounds more likely to entirely knock characters out of a fight, which can increase the risk that the entire party is defeated.

