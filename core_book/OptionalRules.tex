\chapter{Optional Rules}

\section{Attributes}

    \subsection{Other Methods of Attribute Generation}
        Point buy offers the fairest and most customizable system for determining attribute scores, ensuring that players can be almost any character they want to be. However, some groups may wish to determine attribute scores differently. Other options are provided below.

        \subsubsection{Semi-Randomized Point Buy}
            With this method, you have only a small degree of control over your attribute scores, but all characters generated in this way are equally powerful.
            As with the point buy method, all your attribute scores start at 0, and you get 10 points to distribute among your attribute scores.
            However, you do not have full control over how to distribute those points.

            Roll 4d6 for each attribute score, dropping a die of your choice with each roll. First roll for the attribute scores that you care about most, and save the least important attribute scores for last. After rolling for an attribute score, sum results on the three highest dice and consult \trefnp{Semi-Randomized Point Buy Results} and spend the appropriate number of points to yield an attribute score, as indicated by \trefnp{Attribute Score Point Costs}. If you do not have enough points remaining to spend the amount indicated by the die roll, spend as many as you can and move on to the next ability.

            If you have points remaining after rolling all of your attribute scores, you may distribute the points freely among your abilities, using the normal point buy rules. You cannot increase any attribute above 2 during this stage.

            After all of your points have been spent, you may swap any two of your attribute scores.

            \begin{dtable}
                \lcaption{Semi-Randomized Point Buy Results}
                \begin{dtabularx}{\columnwidth}{X X X}
                    \tb{Roll} & \tb{Base Attribute} & \tb{Point Cost} \tableheaderrule
                    3-4       & \minus2              & 0\fn{1} \\
                    5-7       & \minus1              & 0\fn{2} \\
                    8-11      & 0                    & 0       \\
                    12-14     & 1                    & 1       \\
                    15-16     & 2                    & 3       \\
                    16-17     & 3                    & 5       \\
                    18        & 4                    & 7       \\
                \end{dtabularx}
                1 You gain one \glossterm{insight point}. \\
                2 You gain one \glossterm{skill point}. \\
            \end{dtable}

            For characters with more extreme attribute scores, use the following approach, starting with 10 points as normal:
            \begin{enumerate}
                \item For each attribute, roll 2d6
                \item Take the average, rounding down, and subtract 2 from it
                \item Spend that many points as indicated on \trefnp{Attribute Score Point Costs} until you have no points left.
            \end{enumerate}

        \subsubsection{Random Point Buy}
            This method gives you no control over your character whatsoever, while still ensuring that all characters generated are equally powerful.
            It functions like the semi-randomized point buy method, except that you also randomize the order in which the attribute scores are rolled.

        \subsubsection{Classic Hardcore}

            This method is completely random and can generate very overpowered or underpowered characters.
            It represents the unfairness of the world, where some people are just better or worse than others.
            For each attribute, roll 2d6, take the average (rounded down), and subtract 2.
            If you roll a 1 on both dice, treat the average as a 0.
            The result is your base value for that attribute.

\newpage

\section{Species}

    \subsection{Animal Hybrid}
        Animal hybrids are humanoid creatures that are a combination of humans and animals.
        The abilities of an animal hybrid depend on the type of animal it is based on.

        \parhead{Size} Medium.
        \parhead{Attributes} No change.
        \parhead{Speed} 30 feet.
        \parhead{Special Abilities} As the original animal.
        \parhead{Species Feat Options} Any feat strongly associated with the chosen animal.

        \subsubsection{Sample Animal Hybrids}

            \parhead{Hybrid Wolf}

            \subparhead{Special Abilities}
            \begin{itemize}
                \itemhead{Scent} A hybrid wolf has the scent ability (see \pcref{Scent}).
                \itemhead{Bite} A hybrid wolf's mouth is elongated, which it can use as a bite attack (see \pcref{Natural Weapons}).
                    A hybrid wolf's bite deals \plus0d damage.
                \itemhead{Low-light Vision} A hybrid wolf treats sources of light as if they had double their normal illumination range.
            \end{itemize}
        \parhead{Species Feat Options} Agility, Awareness Specialization, Stealth Specialization, Survival Specialization, or Swift.

    \subsection{Awakened Animal}

        Awakened animals are animals that have been granted sentience by the \spell{awaken} ritual.
        The abilities of an awakened animal depend on the type of animal it is.

        \parhead{Size} Tiny, Small, or Medium, as original animal.
        \parhead{Attributes} The attributes of an awakened animal depend on its size.
        \subparhead{Medium} No change.
        \subparhead{Small} \minus1 base Strength, \plus1 base Dexterity.
        \subparhead{Tiny} \minus2 base Strength, \plus2 base Dexterity.
        \parhead{Speed} As the original animal.
        \parhead{Special Abilities} As the original animal.
        \parhead{Species Feat Options} Any feat strongly associated with the chosen animal.

        \subsubsection{Sample Awakened Animals}

            \parhead{Cat}

            \subparhead{Size} Tiny. As a Tiny character, a cat gains several benefits and penalties, as described at \pcref{Small Characters}.
            \subparhead{Attributes} Being Tiny gives cats a \minus2 penalty to base Strength and a \plus2 bonus to base Dexterity.
            \subparhead{Speed} 20 feet.
            \subparhead{Special Abilities}
            \begin{itemize}
                \itemhead{Scent} A cat has the scent ability (see \pcref{Scent}).
                \itemhead{Claws} A cat's paws end in claws, which it can use to attack (see \pcref{Natural Weapons}). A cat's claws do \minus1d damage.
                \itemhead{Low-light Vision} A cat treats sources of light as if they had double their normal illumination range.
            \end{itemize}
        \parhead{Species Feat Options} Agility, Awareness Specialization, Climb Specialization, Escape Artist Specialization, Stealth Specialization, or Swift.

    \subsection{Changeling}

        \parhead{Size} Medium.
        \parhead{Attributes} No change.
        \parhead{Speed} 30 feet.
        \parhead{Special Abilities}
        \begin{itemize}
            \itemhead{Alter Shape} A changeling can alter its physical form in minor ways. As a standard action, a changeling can make a Disguise check with a \plus10 bonus to alter its body. This ability does not alter the changeling's equipment, which may give away its identity unless disguised normally.

            This is a \glossterm{magical} ability.
        \end{itemize}
        \parhead{Automatic Languages} Common and any one \glossterm{common language}.
        \parhead{Bonus Languages} Any.
        \parhead{Species Feat Options} any Skill feat.

    \subsection{Drakkenfel}

        A drakkenfel is created when a dragon's scales are removed while the dragon still lives. The scales retain much of the dragon's power, and without them, the dragon is cursed to continue its life as a diminished, mortal creature.
        Drakkenfel are extremely rare, as there are few who are powerful enough to subdue a dragon to extract its scales -- and fewer still who would dare to do so, knowing that they would earn the eternal enmity of dragonkind.

        Physically, a drakkenfel resembles a wyrmling dragon, except that it is completely scaleless.
        Its skin is leathery and rough, and some drakkenfel bear scars from the ritual that removed their scales.
        Most drakkenfel retain tatters of their wings, but they are always nonfunctional.

        \parhead{Size} Small.
        \parhead{Attributes} No change. This replaces the normal bonuses and penalties to attributes from being Small.
        \parhead{Speed} 25 feet.
        \parhead{Special Abilities}
        \begin{itemize}
            \itemhead*{Low-light Vision}: Drakkenfel treat sources of light as if they had double their normal illumination range.
            \itemhead{Draconic Essence} Each drakkenfel was once a type of true dragon.
                When creating a drakkenfel, choose which type of dragon it used to be.
                This is inherent to the drakkenfel, and cannot be changed.
                A list of dragons and their associated energy type is given on \tref{Dragon Types}.
            \itemhead{Energy Resistance} A drakkenfel gains a bonus equal to its level to \glossterm{resistances} against the energy type associated with its \textit{draconic essence}.
            \itemhead{Bite} A drakkenfel's mouth can be used to bite (see \pcref{Natural Weapons}). A drakkenfel's bite attack deals \plus0d damage.
            \itemhead{Sleeping Dragon} If a drakkenfel recovers its stolen scales, it immediately becomes a true dragon again.
                Its statistics become identical to its statistics before losing its scales, including level, except that it keeps all abilities gained from its Scaleless feat while a drakkenfel.
                It does not gain additional abilities from the Scaleless feat as it gains levels.
        \end{itemize}
        \parhead{Species Feat Options} Draconic Heritage. The type of dragon chosen for the drakkenfel's \textit{draconic ancestry} must match its \textit{draconic essence}.

    \subsection{Dryaidi}

        Dryaidi are humanoid creatures that resemble plants. They are descended from dryads.

        \parhead{Size} Medium.
        \parhead{Attributes} \plus1 base Constitution, \minus1 base Dexterity.
        \parhead{Speed} 20 feet.
        \parhead{Special Abilities}
        \begin{itemize}
            \itemhead{Ingrain} A dryaidi use the \textit{ingrain} ability as a standard action.
                \begin{freeability}{Ingrain}
                    The dryaidi's land speed becomes 5 feet, regardless of any bonuses that normally apply.
                    It gains a \plus4 bonus to Fortitude defense and a \plus1 bonus to Armor defense.
                    If the dryaidi takes a \glossterm{long rest} while this ability is active, it acquires nutrients sufficient to replace a day's worth of food and water.

                    This ability lasts until the dryaidi ends it as a standard action.
                \end{freeability}
            \itemhead{Photosynthesis} While in sunlight, a dryaidi gains a \plus10 foot bonus to land speed.
            \itemhead{Plant Nature} A dryaidi is considered both a creature and a plant.
        \end{itemize}
        \parhead{Species Feat Options} Innate Magic, Regenerator, or Toughness.

    \subsection{Tieflings}

        Tieflings are humanoid creatures descended from fiends.
        \parhead{Size} Medium.
        \parhead{Attributes} No change.
        \parhead{Speed} 30 feet.
        \parhead{Special Abilities}
        \begin{itemize}
            \itemhead*{Darkvision}: Tieflings can see in the dark clearly up to 50 feet. Darkvision does not function if a tiefling is in a brightly lit area, and does not resume functioning until the end of the next round after the tiefling leaves the brightly lit area.
            \itemhead*{Fire Resistance} (Magical): Tieflings gain a bonus equal to their level to \glossterm{resistances} against fire damage.
            \itemhead*{Infernal Darkness} (Magical): A tiefling can use the \textit{infernal darkness} ability as a \glossterm{standard action}.
                \begin{attuneability}{Infernal Darkness}[\glossterm{Attune} (self)]
                    Light is suppressed within an \areamed radius \glossterm{zone} within \rngclose range.
                    Light within or passing through the area is dimmed to be no brighter than shadowy illumination.
                \end{attuneability}
            \itemhead*{Infernal Presence}: Tieflings gain a \plus2 bonus to the Bluff and Intimidate skills.
        \end{itemize}
        \parhead{Species Feat Options} \featref*{Executioner}, \featref*{Skill Specialization: Bluff}, \featref*{Skill Specialization: Intimidate}, or \featref*{Spellwarped}.
