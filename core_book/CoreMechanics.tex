\chapter{Core Mechanics}

This chapter describes the core mechanics of Rise.

\section{Defining the Undefined}
    This book does not attempt to include specific rules for every aspect of a realistic world.
    Unless defined otherwise - or if it's not worth the effort to look up Rise's exact rules in the flow of a game - you should assume that the universe works more or less like the real world does, and as long as everyone agrees that something is reasonable, it's not worth worrying about in more detail.
    For example, Rise does not have specific rules for how long it takes to eat a meal, the arc that a thrown ball takes through the air, or how much extra weight a well-made chandelier can hold without breaking.
    It's possible to imagine situations where each of those might be important to a game, however, so you'll have to guess what would be reasonable as obscure situations arise.
    The Game Master has the final word when defining ambiguities like this.

\section{Making Attacks}\label{Attacks}
    Anything that affects another creature in a potentially harmful way, such as striking a creature with a sword, is an attack.
    Many abilities are always considered attacks, even if you use them in a way that you believe is not harmful.
    To make an attack, you must make an \glossterm{attack roll}.

    \subsection{Attack Rolls}\label{Attack Rolls}
        To make an attack roll, roll 1d10 and add your \glossterm{accuracy} with the attack.
        The sum of your die roll and your accuracy is called your \glossterm{attack result}.
        You compare your attack result to a \glossterm{defense} that your \glossterm{target} has (see \pcref{Defenses}).
        All attacks specify which defense they are compared to.
        If your result is equal to or higher than your target's defense, the attack hits.
        This almost always means the target suffers some harmful effect, such as taking \glossterm{damage} (see \pcref{Damage}).
        Otherwise, the attack misses.

        \subsubsection{Exploding Attacks}\label{Exploding Attacks}
            When you make an attack roll, if you roll a 10 on the d10, the die \glossterm{explodes}.
            In addition, some effects can cause your roll to \glossterm{explode} without rolling a 10.

            When an attack roll \glossterm{explodes}, you roll it again and add the second result to the original result before applying your \glossterm{accuracy}.
            If you roll a 10 on the extra roll, you keep rolling until you stop rolling a 10 and add all of the rolls together.

        \subsubsection{Critical Hits}\label{Critical Hits}
            If your attack result is at least 10 higher than your target's defense, your attack is a \glossterm{critical hit}.
            Many attacks have special effects on critical hits.
            Unless its critical hit effects are otherwise noted, any attack that deals damage deals double that damage on a critical hit.

            Objects are not normally subject to critical hits.
            Some creatures are also not subject to critical hits, as noted in their descriptions.

        \subsubsection{Glancing Blows}\label{Glancing Blows}
            When you miss on an attack by 2 or less, it is called a glancing blow.
            Some attacks have effects when you get a glancing blow, as indicated in their descriptions or in other abilities.
            Glancing blow effects are always weaker than the effect of a successful hit, but they are always better than missing entirely.
            A glancing blow is no different from a complete miss unless some abilities have specific effects on glancing blows.

    \subsection{Accuracy}\label{Accuracy}
        Your accuracy with an \glossterm{attack} is the number that you add to the \glossterm{attack roll}.
        Your accuracy with an attack is normally equal to half your level \add half your base Perception.
        In addition to this base number, your accuracy can include any number of bonuses and penalties from other sources.

    \subsection{Dealing Damage}\label{Dealing Damage}
        Many \glossterm{attacks} deal damage to their targets.
        In general, most damaging attacks deal an amount of damage determined by rolling some number of dice and adding some multiplier of your \glossterm{power} with that attack.
        The details are given in each attack's description.

        When a creature is dealt damage, the damage first reduces that creature's \glossterm{damage resistance} (see \pcref{Damage Resistance}).
        Any damage in excess of the creature's remaining damage resistance causes it to lose that many \glossterm{hit points} (see \pcref{Hit Points}).
        If you take damage that would reduce your hit points below 0, you gain one or more \glossterm{vital wounds} (see \pcref{Reaching Zero Hit Points}).
        Monsters typically do not gain vital wounds like player characters do.
        Instead, they simply die or fall unconscious when they reach 0 hit points.

        \parhead{Dealing Damage, Taking Damage, and Losing Hit Points}
        You deal damage whenever you hit with a damaging attack, regardless of whether the target loses hit points or only \glossterm{damage resistance}.
        Likewise, you take damage whenever anything deals damage to you.
        However, you only lose hit points if the damage is not mitigated by your damage resistance.
        Many special abilities require the target to lose hit points from a damaging attack, which does not happen if the damage is resisted.

        \subsubsection{Dice Pools}\label{Dice Pools}
            Almost all attacks deal damage based on the result rolled from a pool of dice.
            Many abilities can increase or decrease your damage.
            Some modifiers add or subtract flat values from the damage you deal.
            Others add or subtract \glossterm{dice increments}.

            A die increment is a single increase or decrease in the value of a dice pool.
            Increasing by one die increment is written as \plus1d, and decreasing by one die increment is written as \minus1d.
            Damage dice change in size according to the following pattern:
            \begin{itemize}
                \item 1 damage (minimum)
                \item 1d2
                \item 1d3
                \item 1d4
                \item 1d6
                \item 1d8
                \item 1d10
                \item 2d6
                \item 2d8
                \item 2d10
                \item 4d6
                \item 4d8
                \item 4d10
                \item 5d10
                \item 6d10
                \item 7d10
                \item 8d10
            \end{itemize}

        For each die increment that increases the damage, move one space down the list.
        Likewise, for each die increment that decreases the damage, move one space up the list.
        After the dice pool reaches 8d10, each additional die increment simply adds an additional 10 flat damage.

        % TODO: different section?
        \subsubsection{Damage Types}\label{Damage Types}
            All damage falls into one of two categories: \glossterm{energy damage} or \glossterm{physical damage}.
            Physical damage is the most common type of damage.
            Energy damage is usually caused by \glossterm{magical} effects.

            \subsubsection{Damage Subtypes}\label{Damage Subtypes}
                Physical damage has three subtypes: bludgeoning damage, piercing damage, and slashing damage.
                Energy damage has five subtypes: acid damage, cold damage, electricity damage, fire damage, and sonic damage.
                Damage of a particular subtype is also considered damage of its primary type.
                For example, if you are \glossterm{impervious} to \glossterm{physical damage}, that applies against bludgeoning damage because bludgeoning damage is a subtype of \glossterm{physical damage}.

                Some damage types have special properties, as described below.
                \parhead{Cold} Abilities that deal cold damage can freeze liquids and have similar effects appropriate to a sudden drop in temperature.
                \parhead{Electricity} Abilities that deal electricity damage can ignite nonmagical fires if they damage combustible objects.
                \parhead{Fire} Abilities that deal fire damage provide light equivalent to a torch for their duration.
                % TODO: what does this actually mean
                Abilities without a duration create a brief burst of torchlight.
                While underwater, they deal half damage and have no nondamaging effects.

            \subsubsection{Multiple Damage Types}\label{Multiple Damage Types}
                Some attacks deal damage that has multiple damage types.
                Defensive abilities such as defense bonuses or damage immunities apply against an attack only if they apply to all damage types dealt by the attack.

        \subsubsection{Special Damage Types}\label{Special Damage Types}

            These special damage types are separate from the standard damage types, like fire damage or energy damage.

            \subsubsection{Subdual Damage}\label{Subdual Damage}
                Some attacks and environmental effects deal subdual damage.
                Subdual damage works in the same way as normal damage, except it cannot inflict \glossterm{vital wounds}.
                If an attack that deals subdual damage would inflict a vital wound, the target increases its \glossterm{fatigue level} by three instead.
                Whenever you make a \glossterm{strike}, you can choose to deal subdual damage instead of normal damage.
                If you do, you deal half damage with the strike.

            \subsubsection{Environmental Damage}\label{Environmental Damage}
                Some abilities and environmental effects deal environmental damage.
                Environmental damage is never dealt as the result of a successful attack roll.
                Environmental damage works in the same way as normal damage, except that environmental damage is reduced by your \glossterm{damage resistance} without subtracting from its remaining value.
                Any environmental damage in excess of a creature's damage resistance is causes the creature to lose hit points just like normal damage.

                It is possible for damage to be both environmental damage and subdual damage.

\section{Being Attacked}\label{Being Attacked}
    When another creature attacks you, you may or may not suffer serious consequences from the attack.
    The attack may miss entirely if your \glossterm{defenses} are high enough.
    If you are hit by a damaging attack, you reduce your \glossterm{damage resistance} first, and then your \glossterm{hit points} second.
    Taking damage in excess of your hit points can cause you to gain a \glossterm{vital wound}, which can kill you.
    This section explains all of those concepts.

    \subsection{Defenses}\label{Defenses}
        Usually, when you are attacked, the attacker has to make an \glossterm{attack roll} against one of your four \glossterm{defenses} (see \pcref{Attack Rolls}).
        If the attack roll is at least as high as that defense, the attack succeeds.
        The four defenses are described below.
        \begin{itemize}
            \item Armor defense (AD): Your Armor defense protects you from normal physical attacks, such as attempts to hit you with a sword.
                It is the most commonly used defense.
            \item Reflex defense: Your Reflex protects you from physical attacks that armor does not help against, such as pit traps or bolts of lightning.
            \item Fortitude defense: Your Fortitude defense protects you from attacks you have to physically endure or resist, such as poisons and life-draining spells.
            \item Mental defense: Your Mental defense protects you from attacks you have to mentally endure or resist, such as terrifying creatures and magical mind manipulation.
        \end{itemize}

        \subsubsection{Defense Values}\label{Defense Values}

            Your defenses are calculated in the following way:
            \begin{itemize}
                \itemhead{Armor}: Half level \add base Dexterity (modified depending on equipped armor) \add half base Constitution \add class defense bonus \add defense bonuses from equipped body armor and shield
                \itemhead{Fortitude}: Half level \add base Constitution \add class defense bonus
                \itemhead{Reflex}: Half level \add base Dexterity \add class defense bonus
                \itemhead{Mental}: Half level \add base Willpower \add class defense bonus
            \end{itemize}
            In addition to the normal calculation, each defense may have additional bonuses or penalties applied by various abilities.

            \parhead{Class Bonuses} Each class provides bonuses to some combination of Armor, Fortitude, Reflex, and Mental defenses.
            \parhead{Natural Armor} Creatures with unusually tough skin or thick hide, including most monsters, gain bonuses to their Armor defense.
            These bonuses stack with any armor such creatures might wear.

        \subsubsection{Resisting Attacks}
            If an attack fails against you, you almost always suffer no effects from the attack.
            Even if the attack had no obvious physical or visual effects, a creature that resists an attack still feels a hostile force or a tingle, but cannot usually deduce the exact nature of the attack.
            The Spellsense skill can be used to learn about failed \glossterm{magical} attacks against you (see \pcref{Spellsense}).

            \parhead{Lowering Defenses} When you are subject to an attack that you are aware of, you can voluntarily lower your defenses against the attack.
            If you do, your defense is treated as 0 against the attack.

    \subsection{Hit Points}\label{Hit Points}
        Your \glossterm{hit points} measure how hard you are to seriously injure or kill.
        Whenever you take damage that is not prevented by your \glossterm{damage resistance}, you lose that many hit points (see \pcref{Damage}).
        When you run out of hit points, you start gaining vital wounds instead (see \pcref{Reaching Zero Hit Points}).

        The number of hit points you have is based on your level, as defined in \tref{Character Advancement}.
        In addition, each creature gains a bonus (or penalty) to its hit points equal to twice its Constitution.
        Some special abilities can give you additional \glossterm{hit points}.

        \parhead{What Hit Points Represent} Hit points represent a combination of durability, luck, divine providence, and sheer determination, depending on the nature of the creature being damaged.
        When lose a single hit point from an orc with a greataxe, the axe did not literally carve into your skin without affecting your ability to fight.
        Instead, you avoided the worst of the blow, but it bruised you through your armor, the effort to dodge the blow fatigued you, or it barely nicked you through sheer luck -- and everyone's luck runs out eventually.

        \subsubsection{Reaching Zero Hit Points}\label{Reaching Zero Hit Points}
        At the end of each phase, if you took damage in excess of your remaining \glossterm{hit points} during that phase, you gain a \glossterm{vital wound} and your hit points are set to 0.
        % TODO: this should be a generic mechanic
        Damage you take at the end of the round is considered to be part of the delayed action phase for this purpose.
        If the combined excess damage you suffered from all sources in a given phase is greater than half your maximum hit points, you gain an additional vital wound.
        You gain an additional vital wound for each increment of half your maximum hit points that you take in excess damage.

        \subsubsection{Regaining Hit Points and Damage Resistance}\label{Regaining Hit Points and Damage Resistance}
            Some abilities cause you to regain lost hit points or damage resistance.
            For example, the \textit{recover} ability allows you to regain all of your lost hit points (see \pcref{Recover}).
            You can't normally exceed your maximum hit points or maximum damage resistance, even with magical abilities.

            \parhead{Timing} If you regain hit points and lose hit points in the same phase, the two values offset before checking other effects or limitations on the healing or damage.
            The same applies to regaining and losing damage resistance.
            For example, if you were at five hit point and both regained six hit points and lost six hit points simultaneously, you would not receive a vital wound for dropping below 0 hit points.
            Instead, you would simply be at one hit point after both the healing and damage were applied.
            Similarly, if you were at your maximum hit points and both regained and lost two hit points simultaneously, you would still be at your maximum hit points.

    \subsection{Damage Resistance}\label{Damage Resistance}
        Most creatures can resist some damage before they start losing hit points.
        Whenever you take damage, you first apply that damage to your \glossterm{damage resistance} before applying it to your hit points.
        You reduce your remaining damage resistance by an amount equal to the damage you take.
        If the damage is less than your remaining damage resistance, you do not lose any hit points.
        If the damage is more than your remaining damage resistance, your damage resistance is reduced to 0 and you lose hit points equal to the excess damage.

        Essentially, your damage resistance function like extra hit points that are always subtracted before you start losing actual hit points.
        Many attacks have specific consequences if you lose hit points from them, and your damage resistance can protect you from those consequences.

        You restore your damage resistance to its full value whenever you take a \glossterm{short rest}.
        Damage resistance cannot be restored by effects that cause creatures to regain hit points.
        Some specific special abilities can restore damage resistance.

        The base value of your damage resistance is based on your level, as defined in \tref{Character Advancement}.
        In addition, all creatures gain a bonus to their damage resistance equal to their Constitution.
        Body armor provides a bonus to damage resistance, and many special abilities also increase your damage resistance.

        \parhead{Noticing Damage Resistance}
        In general, it is impossible to determine whether a creature has damage resistance simply by observing them unless there are obvious visual cues like body armor.
        However, when a creature takes damage from an attack, an observer can determine the result of the attack with an Awareness check with a base \glossterm{difficulty rating} of 10 (before applying the normal modifiers for distance, visibility, and so on).
        The creature dealing the damage gains a \plus10 bonus to this check.
        Success on this check allows an observer to distinguish between the following three possibilities:
        \begin{itemize}
            \item The creature resisted all damage from your attack.
            \item The creature resisted some damage from your attack, but also lost some hit points.
                This means that the target's damage resistance was reduced to 0 during this phase.
            \item The creature did not resist any damage from your attack, and took all damage from the attack from their hit points.
        \end{itemize}

        \parhead{Timing} If you take damage from multiple sources simultaneously, you apply your current damage resistance value against all of them equally for the purpose of determining whether any individual attack caused you to lose hit points.
        For example, if you have 5 damage resistance remaining, and you take 3 damage and 8 damage from different sources in the same round, only the 8 damage attack would be considered to have caused you to lose hit points.
        This is meaningful for determining the effects of attacks that have special effects if they specifically cause you to lose hit points.

    \subsection{Vital Wounds}\label{Vital Wounds}
        A \glossterm{vital wound} represents serious damage to your body.
        Each \glossterm{vital wound} has a specific detrimental effect on you.
        You gain vital wounds by taking damage in excess of your remaining \glossterm{hit points} (see \pcref{Reaching Zero Hit Points}).

        To determine the effect of a \glossterm{vital wound}, make a \glossterm{vital roll} and find the corresponding effect in \trefnp{Vital Wound Effects}.
        The effect of the vital wound lasts until you remove that vital wound.
        The effects of vital wounds stack with each other, even if you roll the same effect twice for different \glossterm{vital wounds}.

        \subsubsection{Vital Rolls}\label{Vital Rolls}
            To make a \glossterm{vital roll}, roll 1d10 \sub the number of \glossterm{vital wounds} you already had, ignoring the vital wound you are rolling for.
            This includes vital wounds that have no specific vital wound effect.
            The result determines the effect of the \glossterm{vital wound}, as listed in \tref{Vital Wound Effects}.
            Vital wound effects from vital rolls below 1 are lethal if untreated, but the Medicine skill can be used to prevent you from dying (see \pcref{Medicine}).

            \parhead{Delaying Death} Vital wounds with a vital roll below 1 can kill you.
            While you are dying in this way, if you receive healing that causes you to regain hit points, you delay your death by one round.
            This benefit applies even if you are already at full hit points.
            You cannot delay your death in this way by more than 5 rounds.

            \begin{dtable}
                \lcaption{Vital Wound Effects}
                \begin{dtabularx}{\textwidth}{l X}
                    \tb{Vital Roll}  & \tb{Effect} \tableheaderrule
                    \minus6 or less  & You immediately die                                                        \\
                    \minus1--\minus5 & You are unconscious, and you die at the end of the next round              \\
                    0                & You are unconscious, and you die after one minute                          \\
                    1                & You are unconscious while you have less than full hit points               \\
                    2                & Your maximum \glossterm{hit points} and \glossterm{damage resistance} is halved \\
                    3                & You take a \minus2 penalty to \glossterm{accuracy}                         \\
                    4                & You take a \minus2 penalty to all \glossterm{defenses}                     \\
                    5                & You take a \minus1 penalty to future \glossterm{vital rolls}               \\
                    6                & You move at half speed while you have less than full hit points            \\
                    7                & Your maximum \glossterm{damage resistance} are halved                            \\
                    8                & You take a \minus1 penalty to \glossterm{accuracy}                         \\
                    9                & You take a \minus1 penalty to all \glossterm{defenses}                     \\
                    10 or more       & No extra vital wound effect                                                \\
                \end{dtabularx}
            \end{dtable}

        \subsubsection{Removing Vital Wounds}\label{Removing Vital Wounds}
            Vital wounds take time to heal.
            Whenever you take a \glossterm{long rest}, you remove one of your vital wounds.
            If you have multiple vital wounds, the vital wounds heal in order from highest vital roll to lowest vital roll.

\section{Making Checks}\label{Checks}\label{Making Checks}
    Checks are required to perform actions that have a chance of failure where the difficulty is not measured by the defense of another creature or object.
    For example, climbing a wall or remembering an obscure piece of trivia may require a check.

    To make a check, roll 1d10 and add your modifier with the check.
    You compare that result to a \glossterm{difficulty rating} that represents the difficulty of the task.
    The more difficult the task, the higher the \glossterm{difficulty rating} will be.
    If your result is equal to or higher than the \glossterm{difficulty rating}, the check succeeds.
    This usually means you accomplish a task successfully.
    Normal Difficulty Ratings are described in \tref{Difficulty Ratings}.

    \begin{dtable}
        \lcaption{Difficulty Ratings}
        \begin{dtabularx}{\columnwidth}{p{8em} X}
            \tb{Difficulty Rating} & \tb{Example (Skill Used)} \tableheaderrule
            Trivial (0)      & Hear a coversation from 10 feet away (Awareness)                          \\
            Average (5)      & Tie or untie a typical knot (Devices)                                     \\
            Tough (10)       & Swim in rough water (Swim)                                                \\
            Challenging (15) & Balance on a one-inch wide wood beam (Balance)                         \\
            Heroic (20)      & Open a high quality lock (Devices)                                        \\
            Legendary (25)   & Leap across a 30-foot chasm with a running start (Jump)                   \\
            Epic (30)        & Convince a wise mayor her husband is secretly a werewolf (Persuasion)     \\
            Godlike (40)     & Track three orcs across firm ground after 24 hours of rainfall (Survival) \\
        \end{dtabularx}
    \end{dtable}

    \subsection{Critical Success}
        If your check result is at least 10 higher than the \glossterm{difficulty rating}, your check is a \glossterm{critical success}.
        Some checks have a special effect on a critical success.
        For example, a critical success while climbing means you move twice as quickly (see \pcref{Climb}).

    \subsection{Critical Failure}
        If your check result is at least 6 lower than the \glossterm{difficulty rating}, your check is a \glossterm{critical failure}.
        Some checks have a special effect on a critical failure, which is usually bad for the character making the check.
        For example, a critical failure while climbing means you fall (see \pcref{Climb}).

    \subsection{Types of Checks}
        There are two types of checks: attribute checks and skill checks.
        Your bonus with an attribute check is normally equal to your total value for that attribute.
        Your bonus with a skill check is based on your training with that skill, as well as your value for any relevant attribute (see \pcref{Skills}).

        Some abilities give you bonuses to checks based on a particular attribute, such as "Strength-based checks".
        Those bonuses apply to both attribute checks and checks with skills based on those attributes.

    \subsection{Opposed Checks}
        Sometimes, you're competing with another creature, such as when you are trying to hold a door closed and a ferocious ogre is trying to shove it open.
        In that case, you both make a \glossterm{check}, and the creature with the higher result wins.
        This is called an opposed check.
        If you both get the same result, roll again to break the tie.

    \subsection{Encumbrance}\label{Encumbrance}
        Your encumbrance is a value that represents how much you are burdened by armor and weight.
        You apply your encumbrance as a penalty to all Strength and Dexterity-based checks you make.
        You can increase your encumbrance by wearing armor.
        In addition, you reduce your total encumbrance from \glossterm{armor} by an amount equal to your base Strength, to a minimum of 0.
        You apply this reduction to the total encumbrance applied by both your body armor and your shield, not individually for each piece of armor.
        Most shields do not cause encumbrance, so this only matters for tower shields.

        Resting in armor is difficult.
        If you take a \glossterm{long rest} while you have any encumbrance, you finish your rest with a \glossterm{fatigue level} equal to the value of your encumbrance.
        In addition, only half the time you spend sleeping while you have encumbrance counts as sleep for the purpose of determining your fatigue (see \pcref{Sleep and Fatigue}).

\section{Resting and Resources}

    \subsection{Attunement Points}\label{Attunement Points}
        You can use \glossterm{attunement points} to \glossterm{attune} to effects such as spells or items (see \pcref{Attunement}).
        Abilities that require attunement have the \abilitytag{Attune} tag (see \pcref{Ability Tags}).

        Your \glossterm{class} gives you a certain number of attunement points.
        At higher levels, you gain additional \glossterm{attunement points} (see \pcref{Character Advancement}).
        A small number of abilities can also grant additional \glossterm{attunement points}.

        When you take a \glossterm{short rest}, you recover all \glossterm{attunement points} that you released from attuned effects.
        For details, see \pcref{Resting}.

    \subsection{Fatigue}\label{Fatigue}
        Thoughout the day, you can become fatigued by your exertions both in and out of combat.
        While \glossterm{hit points} are easy to restore, reducing your \glossterm{fatigue level} generally requires a \glossterm{long rest}.
        Fatigue is still easier to recover from than \glossterm{vital wounds}.

        \subsubsection{Fatigue Level}\label{Fatigue Level}
            Your \glossterm{fatigue level} measures how fatigued you are.
            A number of abilities and attacks can cause you to increase your fatigue level.
            The most common abilities that increase your fatigue level are the \textit{desperate exertion}, \textit{recover}, and \textit{sprint} abilities.
            All of those abilities are described in \pcref{Universal Abilities}.

            \subsubsection{Fatigue Tolerance}\label{Fatigue Tolerance}
                Becoming slightly fatigued is not immediately detrimental.
                Your fatigue level can be as high as your base Strength \add your base Willpower without suffering any consequences (minimum 0).
                This value is called your \glossterm{fatigue tolerance}.
                Your \glossterm{class} gives you a bonus to your fatigue tolerance, and some abilities can also modify it.

            \subsubsection{Fatigue Penalty}\label{Fatigue Penalty}
                You take a penalty to \glossterm{accuracy} and \glossterm{checks} equal your \glossterm{fatigue level} \sub your \glossterm{fatigue tolerance}.
                This penalty is called your \glossterm{fatigue penalty}.

        \subsubsection{Exhaustion}\label{Exhaustion}
            When your \glossterm{fatigue penalty} reaches \minus5, you fall \unconscious until your fatigue penalty is reduced below \minus5.
            Generally, this means that you are unconscious for 8 hours.

        \subsubsection{Recovering From Fatigue}
            When you take a \glossterm{long rest}, your \glossterm{fatigue level} is restored to 0.
            There are no other ways to reduce your fatigue level.

    \subsection{Resting}\label{Resting}
        When you have a moment to relax, you can rest to regain some of your expended resources.
        There are two main types of rests: a \glossterm{short rest} and a \glossterm{long rest}.
        Resting is not actually an ability in the same sense as most other abilities.
        You do not declare that you are using the ``short rest'' ability, and you do not have to differentiate between whether you intend to take a short rest or a long rest.
        The benefits of taking a short rest or long rest happen automatically after you spend enough time avoiding strenuous activity.
        Resting at night is often combined with sleeping, but you can rest at any time without sleeping.

        \subsubsection{Short Rest}\label{Short Rest}
            Resting for ten minutes is considered a \glossterm{short rest}.
            When you take a short rest, you gain the following benefits.
            \begin{itemize}
                \item Your \glossterm{hit points} become equal to your maximum hit points.
                \item Your current \glossterm{damage resistance} becomes equal to your maximum damage resistance.
                \item You regain any \glossterm{attunement points} you released from \glossterm{attuned} abilities (see \pcref{Attunement}).
                \item You remove all \glossterm{conditions} affecting you (unless they cannot be removed normally).
                \item Some other abilities have specific effects that last until you take a short rest.
                    For example, a barbarian cannot use their \textit{rage} ability again after raging until after they take a short rest (see Rage, page \pref{Bbn:Rage}).
            \end{itemize}

        \subsubsection{Long Rest}\label{Long Rest}
            Resting for eight hours is considered a \glossterm{long rest}.
            When you take a long rest, you gain the following benefits.
            \begin{itemize}
                \item You remove one of your vital wounds (see \pcref{Removing Vital Wounds}).
                    The Medicine skill can increase this healing (see \pcref{Accelerate Recovery}).
                \item Your \glossterm{fatigue level} becomes 0.
                \item Some other abilities have specific effects that last until you take a long rest.
            \end{itemize}

            You can take multiple long rests consecutively to recover from extensive vital wounds.
