\chapter{Core Mechanics}

This chapter describes the core mechanics of Rise.
It defines how attributes work and explains how to make attacks in combat.

\section{Attacks and Checks}\label{Attacks and Checks}
    You can take many actions without needing to roll a die at all.
    However, eventually you will need to do something where there is a dramatically significant chance of failure.
    In that case, you will need to roll a die to see if you succeed or fail.
    Almost all rolls you will need to make can be described as an \glossterm{attack roll} or a \glossterm{check}.

    \subsection{Attack Rolls}
        Attack rolls are required to make \glossterm{attacks}.
        Anything that affects another creature in a potentially harmful way, such as striking a creature with a sword, is an attack.
        Many abilities are always considered attacks, even if you use them in a way that you believe is not harmful.

        To make an attack roll, roll 1d10 and add your \glossterm{accuracy} with the attack.
        The sum of your die roll and your accuracy is called your \glossterm{attack result}.
        You compare your attack result to a \glossterm{defense} that your \glossterm{target} has (see \pcref{Defenses}).
        All attacks specify which defense they are compared to.
        If your result is at least equal to your target's defense, the attack hits.
        This almost always means the target suffers some harmful effect, such as taking \glossterm{damage} (see \pcref{Damage}).
        Otherwise, the attack misses.

        \subsubsection{Exploding Attacks}\label{Exploding Attacks}
            When you make an attack roll, if you roll a 10 on the d10, the die \glossterm{explodes}.
            In addition, some effects can cause your roll to \glossterm{explode} without rolling a 10.
            For example, if you attack an \glossterm{unaware} target, your attack roll explodes regadless of the roll.

            When an attack roll \glossterm{explodes}, you roll it again and add the second result to the original result before applying your \glossterm{accuracy}.
            If you roll a 10 on the extra roll, you keep rolling until you stop rolling a 10 and add all of the rolls together.
            % Is this necessary text?
            No effect can cause extra rolls to explode without rolling a 10.

        \subsubsection{Critical Hit}\label{Critical Hit}
            If your attack result is at least 10 higher than your target's defense, your attack is a \glossterm{critical hit}.
            Many attacks have special effects on critical hits.
            Unless its critical hit effects are otherwise noted, any attack that deals damage deals double that damage on a critical hit.

            Objects are not normally subject to critical hits.
            Some creatures are also not subject to critical hits, as noted in their descriptions.

    \subsection{Checks}\label{Checks}
        Checks are required to perform actions that have a chance of failure that are not attacks.
        For example, climbing a wall or remembering an obscure piece of trivia may require a check.

        To make a check, roll 1d10 and add your \glossterm{check modifier} with the check.
        You compare the die result, including your check modifier, to a \glossterm{difficulty rating} (DR) that represents the difficulty of the task.
        The more difficult the task, the higher the DR will be.
        If your result is at least equal to the DR, the check succeeds.
        This usually means you accomplish a task successfully.
        Normal Difficulty Ratings are described in \tref{Difficulty Ratings}.

        \begin{dtable}
            \lcaption{Difficulty Ratings}
            \begin{dtabularx}{\columnwidth}{p{8em} X}
                \tb{Difficulty (DR)} & \tb{Example (Skill Used)} \tableheaderrule
                Trivial (0)      & Hear a coversation from 10 feet away (Awareness)                          \\
                Average (5)      & Tie or untie a typical knot (Devices)                                     \\
                Tough (10)       & Swim in rough water (Swim)                                                \\
                Challenging (15) & Balance on a one-inch wide wood beam (Acrobatics)                         \\
                Heroic (20)      & Open a high quality lock (Devices)                                        \\
                Legendary (25)   & Leap across a 30-foot chasm with a running start (Jump)                   \\
                Epic (30)        & Convince a wise mayor her husband is secretly a werewolf (Persuasion)     \\
                Godlike (40)     & Track three orcs across firm ground after 24 hours of rainfall (Survival) \\
            \end{dtabularx}
        \end{dtable}

        \subsubsection{Critical Success}
            If your check result is at least 10 higher than the DR, your check is a \glossterm{critical success}.
            Some checks have a special effect on a critical success.
            For example, a critical success while climbing means you move twice as quickly (see \pcref{Climb}).

        \subsubsection{Critical Failure}
            If your check result is at least 10 lower than the DR, your check is a \glossterm{critical failure}.
            Some checks have a special effect on a critical failure, which is usually bad for the character making the check.
            For example, a critical failure while climbing means you fall (see \pcref{Climb}).

\section{Combat Time}\label{Combat Time}
    The world of Rise can be a harsh one, and not all disagreements can be resolved peacefully.
    At some point, you will be forced to enter combat.
    This section explains how time passes in combat.

    \subsection{Rounds}\label{Rounds}

        Combat takes place in a series of \glossterm{rounds}, which represent about six seconds of time.
        Each round of a combat is divided into three \glossterm{phases} (see \pcref{Phases}).
        After all phases are complete, the round ends and the next round begins.

    \subsection{Actions}\label{Actions}

        You can take actions in combat to defeat your foes.
        There are four types of actions: \glossterm{standard actions}, \glossterm{minor actions}, \glossterm{move actions}, and \glossterm{free actions}.
        % TODO: define conflicting action limits (drawing a sword and sheathing a shield?)

        \subsubsection{Standard Actions}\label{Standard Actions}
            Most common activities require a \glossterm{standard action}, such as attacking with a weapon, casting a \glossterm{spell}, and using many special abilities.
            Using a standard action generally takes about three seconds of time within the game, and it requires most of your attention during that time.

            You can take one standard action per round.

        \subsubsection{Minor Actions}\label{Minor Actions}
            Some special abilities require a \glossterm{minor action}.
            Using a minor action does not take much time or attention, and it can be done at the same time as any other actions.

            You can normally take one minor action per round.
            However, you can choose to take an additional minor action in place of a \glossterm{standard action}.

        \subsubsection{Move Actions}\label{Move Actions}
            You can move around a battlefield as a \glossterm{move action}.
            Using a move action generally takes about three seconds of time within the game, and it requires most of your attention during that time.

            You can normally take one move action per round.
            However, you can choose to take an additional move action in place of a \glossterm{standard action}.

        \subsubsection{Free Actions}\label{Free Actions}
            Many minor activities require a \glossterm{free action}, such as drawing or sheathing a weapon.
            Using a free action does not take much time or attention, and it can be done at the same time as any other actions.

            You can take any number of free actions per round.

    \subsection{Phases}\label{Phases}

        There are three \glossterm{phases} in each round: a \glossterm{movement phase}, an \glossterm{action phase}, and sometimes a \glossterm{delayed action phase}.
        Each phase specifies the types of actions that can be taken during that phase.
        As a special case, \glossterm{free actions} may be taken during any phase.

        \subsubsection{The Movement Phase}\label{The Movement Phase}
            During the \glossterm{movement phase}, you can take one \glossterm{move action}.
            The most common move action is the \textit{hustle} ability, which allows you to move a distance equal to your \glossterm{speed}.
            For details, see \pcref{Movement and Positioning}.

        \subsubsection{The Action Phase}\label{The Action Phase}
            During the \glossterm{action phase}, you can take one \glossterm{minor action} and one \glossterm{standard action}.
            Alternately, you can take a \glossterm{move action} or additional \glossterm{minor action} in place of your standard action.
            Most of the time, you will simply take a single standard action.

        \subsubsection{The Delayed Action Phase}\label{The Delayed Action Phase}
            During the \glossterm{delayed action phase}, you can take a \glossterm{minor action} or \glossterm{standard action} if you did not use the corresponding action in the \glossterm{action phase}.
            Alternately, you can take a \glossterm{move action} or additional \glossterm{minor action} in place of a standard action.
            In addition, some abilities have effects during the delayed action phase instead of or in addition to their effects in the action phase.
            For example, the \textit{charge} ability allows you to move during the action phase and attack during the delayed action phase (see \pcref{Charge}).

    \subsection{Resolving Actions}\label{Resolving Actions}

        Within each phase, actions of all creatures are simultaneously resolved in the following order.
        Allies with the ability to communicate can freely coordinate their actions with each other, within reasonable limits.

        \begin{enumerate*}
            \item Choose actions.
            \item Determine targets affected by actions.
            \item Apply the results of \glossterm{Swift} abilities.
            \item Check action success.
                Example: Making attack rolls.
            \item Determine action results.
                Example: Making damage rolls.
            \item Apply action results.
                Examples: Adding \glossterm{vital wounds}, moving creature locations, and applying penalties.
        \end{enumerate*}

        In the vast majority of cases, there is no need to go through this order explicitly.
        Combats will run much faster if attack and damage rolls are generally made and announced at the same time as those actions are chosen, even before all characters have explicitly stated their actions.
        The order of resolution matters when creatures take actions that directly conflict with each other.

        \subsubsection{Swift Abilities}\label{Swift Abilities}
            Some abilities resolve before other actions in the same phase.
            These abilities have the \glossterm{Swift} tag.
            They resolve after targets are determined, but before attack rolls are made.
            Swift abilities never require attack rolls, and almost always affect only the creature using the ability.

            For example, the \textit{total defense} ability is a swift ability.
            It increases your defenses against attacks made during the same phase (see \pcref{Total Defense}).

        % This may be too complicated
        \subsubsection{Conflicting Actions}\label{Conflicting Actions}

            Sometimes, actions that occur in the same phase can conflict with each other.
            In this case, each creature involved with conflicting actions in that phase rolls an \glossterm{initiative} check (see \pcref{Initiative}).
            Starting from the highest check result and continuing to the lowest, each creature decides to resolve its chosen action or delay to choose a different action.
            After the lowest initiative check result has made its choice, each creature that delayed can resolve a different action of its choice, starting from the lowest check result and continuing to the highest.
            When deciding, each creature knows the resolved effects of the actions chosen by previous creatures that it can observe.

            When determining whether two actions conflict, it is best to be generous and consider the intention of the action.
            The only downside to treating actions as conflicting is the time required to resolve the initiative checks and consider action changes.

            For example, one of the most common conflicts occurs when a creature tries to move into melee range with a foe that unexpectedly moves away.
            Although the two movements not mutually impossible, the first creature can easily end up out of melee range from all foes at the end of the phase if it doesn't have enough movement to reach its target.
            Treating the movements as conflicting allows the first creature to sprint or change its action if it chooses.

\section{Character Statistics}
    This section explains how character statistics, such as how strong you are or how accurate your attacks are, should be calculated.

    \subsection{Action Points}\label{Action Points}
        You can use a wide variety of special abilities by spending \glossterm{action points}.
        Abilities that require action points have the \glossterm{AP} or \glossterm{Attune} tags (see \pcref{Ability Tags}).

        There are two kinds of action points: \glossterm{recovery action points} and \glossterm{reserve action points}.
        The difference between the two types is how quickly you can get them back after spending them.
        You can get back a spent \glossterm{recovery action point} with a \glossterm{short rest},
            but a spent \glossterm{reserve action point} does not return until after a \glossterm{long rest}.
        For details, see \pcref{Resting}.
        Unless otherwise noted, you can use abilities that require action points by spending either type of action point.

        A typical 1st level character has three \glossterm{recovery action points} and three \glossterm{reserve action points}.
        At higher levels, you gain additional \glossterm{recovery action points} (see \pcref{Character Advancement}).
        You can gain additional \glossterm{reserve action points} with a high Willpower, and some abilities can also grant reserve action points.

        \subsubsection{Regaining Action Points}\label{Regaining Action Points}
            You gain regain spent \glossterm{action points} by resting (see \pcref{Resting}).
            In addition, some abilities allow you to regain action points during combat.
            There are two significant restrictions on regaining action points during combat.

            First, you cannot regain action points and spend action points in the same round.
            If you have spent an action point during a round, including spending an action point to \glossterm{attune} to an ability, you cannot regain action points until the next round.
            Similarly, if you have already regained an action point during a round, you cannot spend action points until the next round.
            If you somehow spend and regain action points simultaneously, you do not regain the action points.

            Second, you cannot regain more than one action point per round, even if multiple different abilities say you regain a spent action point.
            All action points you would regain after the first one in the round are ignored.

    \subsection{Attributes}\label{Attributes}

        Each character has six \glossterm{attributes}: Strength (Str), Dexterity (Dex), Constitution (Con), Intelligence (Int), Perception (Per), and Willpower (Wil).
        Each attribute represents a character's raw talent in that area.
        A 0 in an attribute represents average human capacity.
        That doesn't mean that every commoner has a 0 in every attribute; not everyone is average, after all.

        \subsubsection{Attribute Descriptions}

            \parhead{Strength (Str)}\label{Strength}
                Strength measures muscle and physical power.
                It has the following effects:
                \begin{itemize}
                    \item Strength determines how much a character can carry (see \tref{Carrying Capacity by Strength}).
                    \item Strength affects Strength-based skills: Climb, Jump, and Swim (see \pcref{Skills}).
                    \item You reduce your \textit{encumbrance} for weight or heavy armor by an amount equal to your starting Strength.
                    \item If your Strength is negative, you take a penalty to all Strength-based skills equal to your Strength.
                    % TODO: wording
                    \item If your Strength is negative, you take a penalty to \glossterm{power} with \glossterm{mundane} abilities equal to your Strength.
                \end{itemize}

                If you have a high Strength, you can use it to determine several statistics:
                \begin{itemize}
                    \item Your \glossterm{power} with most \glossterm{mundane} abilities, such as \glossterm{strikes} (see \pcref{Power}).
                    \item Your \glossterm{threat} (see \pcref{Threat}).
                \end{itemize}

            \parhead{Dexterity (Dex)}\label{Dexterity}
                Dexterity measures hand-eye coordination, agility, and reflexes.
                It has the following effects:
                \begin{itemize}
                    \item Dexterity affects Dexterity-based skills: Acrobatics, Escape Artist, Ride, Sleight of Hand, and Stealth (see \pcref{Skills}).
                    \item You gain a bonus (or penalty) to your Reflex defense equal to your starting Dexterity.
                    \item If your Dexterity is negative, you take a penalty to all Dexterity-based skills equal to your Dexterity.
                \end{itemize}

                If you have a high Dexterity, you can use it to determine several statistics:
                \begin{itemize}
                    \item Your \glossterm{accuracy} with melee and thrown \glossterm{strikes} using \glossterm{light weapons} (see \pcref{Accuracy}).
                    \item Your Armor and Reflex defenses (see \pcref{Defenses}).
                \end{itemize}

            \parhead{Constitution (Con)}\label{Constitution}
                Constitution represents your health and stamina.
                It has the following efects:
                \begin{itemize}
                    \item Your \glossterm{wound threshold} is modified based on your starting Constitution (see \pcref{Wound Threshold}).
                    \item You gain a bonus (or penalty) to your Fortitude defense equal to your starting Constitution.
                \end{itemize}

                If you have a high Constitution, you can use it to determine several statistics:
                \begin{itemize}
                    \item Your Fortitude defenses (see \pcref{Defenses}).
                    \item Your \glossterm{fatigue threshold} (see \pcref{Fatigue Threshold}).
                \end{itemize}

            \parhead{Intelligence (Int)}\label{Intelligence}
                Intelligence represents how well you learn and reason.
                It has the following effects:

                \begin{itemize}
                    \item You gain \glossterm{insight points} equal to your starting Intelligence (see \pcref{Insight Points}).
                    \item You gain bonus languages equal to your starting Intelligence (see \pcref{Languages}).
                    \item You gain a bonus (or penalty) to \glossterm{skill points} equal to twice your starting Intelligence (see \pcref{Skill Points}).
                    \item You gain a bonus (or penalty) to \glossterm{insight points} equal to your starting Intelligence (see \pcref{Insight Points}).
                    \item Your Intelligence affects Intelligence-based skills: Craft, Deduction, Disguise, Heal, Knowledge, and Linguistics (see \pcref{Skills}).
                    \item If your Intelligence is negative, you take a penalty to all Intelligence-based skills equal to your Intelligence.
                \end{itemize}

                \par An animal has an Intelligence score of \minus6 or lower.
                A creature of humanlike intelligence has a score of at least a \minus5 Intelligence.

            \parhead{Perception (Per)}\label{Perception}
                Perception describes your ability to observe and be aware of your surroundings.
                It has the following effects:
                \begin{itemize}
                    \item Your Perception affects Perception-based skills: Awareness, Creature Handling, Sense Motive, Spellcraft, and Survival (see \pcref{Skills}).
                    \item If your Perception is negative, you take a penalty to all Perception-based skills equal to your Perception.
                    \item If your Perception is negative, you take a penalty to accuracy with all attacks equal to half your Perception.
                \end{itemize}

                If you have a high Perception, you can use it to determine your \glossterm{accuracy} with all attacks (see \pcref{Accuracy}).

            \parhead{Willpower (Wil)}\label{Willpower}
                Willpower represents your ability to endure mental hardships.
                It has the following effects:
                \begin{itemize}
                    \item You gain a bonus (or penalty) to the number of \glossterm{reserve action points} you have equal to your starting Willpower.
                    \item You gain a bonus (or penalty) to your Mental defense equal to your starting Willpower.
                    \item If your Willpower is negative, you take a penalty to \glossterm{power} with \glossterm{magical} abilities equal to your Willpower.
                \end{itemize}

                If you have a high Willpower, you can use it to determine several statistics:
                \begin{itemize}
                    \item Your \glossterm{power} with most \glossterm{magical} abilities (see \pcref{Power}).
                    \item Your Mental defense (see \pcref{Defenses}).
                \end{itemize}

        \subsubsection{Increasing Attributes}\label{Increasing Attributes}
            As your level increases, your attributes increase as well, as shown on \trefnp{Increasing Attributes with Level}.

            \begin{dtable}
                \lcaption{Increasing Attributes with Level}
                \begin{dtabularx}{\columnwidth}{X X}
                    \tb{Starting Attribute} & \tb{Bonus}   \\
                    0 or lower              & 0            \\
                    1                       & \plus1 per even level \\
                    2                       & \plus1 per level after 1st \\
                    3                       & \plus1 per level after 1st \\
                \end{dtabularx}
            \end{dtable}

        \subsubsection{Determining Attributes}
            There are several options for how to determine attribute scores.

            \parhead{Predefined Attribute Scores} This is the simplest method.
            Simply take the following set of attribute scores and distribute them as you choose among your attributes:

            3, 2, 1, 1, 0, 0

            This set of attribute scores is called the ``elite array''.
            For more extreme characters, you may use the ``savant array'':

            4, 2, 0, 0, 0, 0.

            Finally, for more well-balanced characters, you may use the ``balanced array'':

            2, 2, 2, 1, 0, 0

            Any of these distributions can be altered by taking penalties to any attributes given as 0.
            For each penalty you take, you gain two additional \glossterm{skill points} (see \pcref{Skills}).

            \parhead{Point Buy}
                With this method, you can fully control your attribute scores to match what you want to be able to do.
                All your attribute scores start at 0.
                You get ten points to distribute among your attributes.
                Attributes can be bought according to the costs on \trefnp{Attribute Score Point Costs}.
                The listed cost is the total cost required to gain the listed starting attribute.
                You are 1st level when you start, which adds appropriately to your total attribute score.

                \parhead{Impaired Attributes}\label{Impaired Attributes}
                You can start with up to two attributes below 0.
                If you do, you compensate for your impairment in that area with additional talents in other areas.
                For each point below 0, you gain an additional \glossterm{skill point}.
                Decreasing your attributes below 0 does not grant you additional points to spend on other attributes.

                \begin{dtable}
                    \lcaption{Attribute Score Point Costs}
                    \begin{dtabularx}{\columnwidth}{X X l}
                        \tb{Starting Attribute Score} & \tb{Total Attribute Score} & \tb{Cumulative Point Cost} \tableheaderrule
                        \minus2\fn{1}                 & \minus2                    & 0\fn{2}                    \\
                        \minus1\fn{1}                 & \minus1                    & 0\fn{3}                    \\
                        0                             & 0                          & 0                          \\
                        1                             & 1 \add half level          & 1                          \\
                        2                             & 1 \add level               & 3                          \\
                        3                             & 2 \add level               & 5                          \\
                        4                             & 3 \add level               & 7                          \\
                    \end{dtabularx}
                    1 You cannot reduce more than two attributes below 0 in this way. \\
                    2 You gain a total of four \glossterm{skill points}. \\
                    3 You gain two skill points. \\
                \end{dtable}

        \subsubsection{Extraordinary Attributes}
            Some abilities can increase your starting attributes above 4.
            For each point of starting attribute beyond 4, you increase your current attribute by the same amount.

            For example, a 20th level half-orc cleric with the Strength domain who spent 7 points on her starting Strength would have a total starting Strength of 6.
            Her Strength would be two higher than it would be with a starting attribute of 4, for a total of 25.

    \subsection{Accuracy}\label{Accuracy}
        Your accuracy with an \glossterm{attack} is the number that you add to the \glossterm{attack roll}.
        Your accuracy with an attack is normally equal to the higher of your level and your Perception.
        If you are making a \glossterm{strike} with a \glossterm{light weapon}, you may use your Dexterity in place of your Perception.
        In addition to this base number, your accuracy can include any number of bonuses and penalties from other sources.

        \parhead{Proficiency} Each creature is \glossterm{proficient} with a number of weapons.
        For details about the weapons you can be proficient with, see \pcref{Weapons}.
        Your proficiencies are primarily determined by your class, but some abilities also grant proficiency with additional weapons.
        If you make a \glossterm{physical attack} with a weapon you are not proficient with, you take a \minus2 penalty to accuracy.

    \subsection{Power}\label{Power}
        Many abilities have a \glossterm{power} which represents the overall strength of the ability.
        This determines the ability's \glossterm{standard damage} (see \pcref{Standard Damage}).
        Some abilities also have effects that explicitly depend on the ability's power.
        Unless an ability specifies otherwise, its \glossterm{power} depends on whether it is \glossterm{magical} or \glossterm{mundane}.

        \parhead{Magical Abilities}
        Your power with magical abilities is normally equal to the higher of your level and your Willpower.

        \parhead{Mundane Abilities}
        Your power with mundane abilities is normally equal to the higher of your level and your Strength.

    \subsection{Damage}\label{Damage}
        Many attacks deal damage when they hit.
        \subsubsection{Die Increments}\label{Die Increments}
            Most attacks deal damage equal to the result rolled from a pool of dice.
            Many abilities can increase or decrease your damage with abilities.
            These modifiers always increase or decrease your damage by one \glossterm{die increment}.
            Increasing by one die increment is written as \plus1d, and decreasing by one die increment is written as \minus1d.
            A set of damage dice can increase in size in \glossterm{die increments}.
            Damage dice change in size using the following pattern:
            \begin{itemize}
                \item 1 damage (minimum)
                \item 1d2
                \item 1d3
                \item 1d4
                \item 1d6
                \item 1d8
                \item 1d10
                \item 2d6
                \item 2d8
                \item 2d10
                \item 4d6
                \item 4d8
                \item 4d10
                \item 5d10
                \item 6d10
            \end{itemize}

            For each die increment that increases the damage, move one space down the list.
            Likewise, for each die increment that decreases the damage, move one space up the list.
            After the damage dice reach 4d10, each additional die increment simply adds an extra 1d10 of damage.

        \subsubsection{Standard Damage}\label{Standard Damage}
            Most damaging effects use \glossterm{standard damage} as a baseline to determine the damage they deal.
            Your \glossterm{standard damage} with an ability is based on your \glossterm{power} with that ability (see \pcref{Power}).
            Standard damage starts at 1d8 and increases by \plus1d per two power.
            This is summarized on \trefnp{Standard Damage}.

        \begin{dtable}
            \lcaption{Standard Damage}
            \begin{dtabularx}{\columnwidth}{l X}
                \tb{Power} & \tb{Damage} \\
                0--1         & 1d6  \\
                2--3         & 1d8  \\
                4--5         & 1d10 \\
                6--7         & 2d6  \\
                8--9         & 2d8  \\
                10--11       & 2d10 \\
                12--13       & 4d6  \\
                14--15       & 4d8  \\
                16--17       & 4d10 \\
                18--19       & 5d10 \\
                20--21       & 6d10 \\
                22--23       & 7d10 \\
                24--25\fn{1} & 8d10 \\
            \end{dtabularx}
            1. For values above 25, increase by 1d10 at every even value.
        \end{dtable}

        \subsubsection{Power vs. Die Increments}\label{Power vs. Die Increments}
            Many abilities specify that they deal damage based on some standard calculation, such as your \glossterm{standard damage} or your normal damage with a \glossterm{strike}, plus an additional modifier to the \glossterm{die increment}.
            For example, the \textit{power attack} fighter ability deals \plus2d damage more than a normal \glossterm{strike}.
            Other abilities modifier your \glossterm{power} directly.
            For example, the \textit{wellspring of power} mage ability gives you a \plus1 bonus to \glossterm{power} with arcane spells.

            In general, active abilities do not modify their own \glossterm{power}, and instead increase damage with \glossterm{die increments}
            Abilities that modify the effectivenes of other abilities, such as \textit{wellspring of power}, generally increase \glossterm{power} instead of increasing \glossterm{die increments} directly.

    \subsection{Defenses}\label{Defenses}
        Usually, when you are attacked, the attacker has to make an \glossterm{attack roll} against a specific defense.
        If the attack roll is at least as high as that defense, the attack succeeds.
        \begin{itemize}
            \item Armor defense (AD): Your Armor defense protects you from normal physical attacks, such as attempts to hit you with a sword.
                It is the most commonly used defense.
            \item Reflex defense: Your Reflex protects you from attacks you can only avoid, such as pit traps.
            \item Fortitude defense: Your Fortitude defense protects you from attacks you have to physically endure or resist, such as poisons and deadly spells.
            \item Mental defense: Your Mental defense protects you from attacks you have to mentally endure or resist, such as terrifying creatures and magical manipulation.
        \end{itemize}

        \subsubsection{Defense Values}\label{Defense Values}

            Your defenses are calculated in the following way:
            \begin{itemize}
                \itemhead{Armor}: Level or Dexterity \add defense bonuses from equipped body armor and shield
                \itemhead{Fortitude}: Level or Constitution \add starting Constitution \add class defense bonus
                \itemhead{Reflex}: Level or Dexterity \add starting Dexterity \add class defense bonus
                \itemhead{Mental}: Level or Willpower \add starting Willpower \add class defense bonus
            \end{itemize}
            The attributes and relevant bonuses which apply to each defense are described in \trefnp{Defense Calculations}.
            In addition to the normal calculation, each defense may have additional bonuses or penalties applied by various abilities.

            \begin{dtable!*}
                \lcaption{Defense Calculations}
                \begin{dtabularx}{\textwidth}{l l l l >{\lcol}X}
                    \tb{Defense Name} & \tb{Attribute} & \tb{Starting Attribute Modifier} & \tb{Body Armor Modifier} & \tb{Shield Modifier} \\
                    \midrule
                    Armor defense     & Dex & \tdash & Yes & Yes \\
                    Fortitude defense & Con & Con    & No  & No  \\
                    Reflex defense    & Dex & Dex    & No  & No  \\
                    Mental defense    & Wil & Wil    & No  & No  \\
                \end{dtabularx}
            \end{dtable!*}

            \parhead{Class Bonuses} Each class provides bonuses to some combination of Fortitude, Reflex, and Mental defense.
            \parhead{Natural Armor} Creatures with unusually tough skin or thick hide, including most monsters, gain bonuses to their Armor defense.
            These bonuses stack with any armor such creatures might wear.
            \parhead{Size Modifier} Large creatures have a penalty to Reflex defense.
            For details, see \tref{Size in Combat}.

        \subsubsection{Resisting Attacks}
            If an attack fails against you, you almost always suffer no effects from the attack.
            Even if the attack had no obvious physical or visual effects, a creature that resists an attack still feels a hostile force or a tingle, but cannot usually deduce the exact nature of the attack.
            The Spellcraft skill can be used to learn about failed \glossterm{magical} attacks against you (see \pcref{Spellcraft}).

            \parhead{Lowering Defenses} When you are subject to an attack that you are aware of, you can voluntarily lower your defenses against the attack.
            If you do, your defense is treated as 0 against the attack.

    \subsection{Taking Damage}
        When you take damage, one of three things may happen.

        If the damage does not exceed your \glossterm{fatigue threshold}, you ignore the damage without any negative effects.
        This represents the attack glancing harmlessly off of your armor, scratching you minimally, or otherwise failing to have a significant impact.

        If the damage exceeds your \glossterm{fatigue threshold}, but does not exceed your \glossterm{wound threshold}, you become more vulnerable to future damage but suffer no other negative consequences.
        This represents some combination of endurance, luck, or even divine providence preventing you from suffering more serious injury from a dangerous attack.

        If the damage exceeds your \glossterm{wound threshold}, you suffer a \glossterm{vital wound}, which represents a potentially life-threatening injury (see \pcref{Vital Wounds}).

        \subsubsection{Fatigue Threshold}\label{Fatigue Threshold}
            % Is ``attack'' the wrong word here?
            Your \glossterm{fatigue threshold} indicates the amount of damage an attack has to deal in order to fatigue you.
            It is normally equal to the higher of your level and Constitution.

            When you take at least as much damage as your \glossterm{fatigue threshold} from a single damage roll, you gain a point of \glossterm{fatigue}, which lowers your \glossterm{wound threshold}.

        \subsubsection{Wound Threshold}\label{Wound Threshold}
            Your \glossterm{wound threshold} indicates the amount of damage an attack has to deal to seriously injure you.
            % TODO: do math to figure out correct wound threshold value to balance con vs dex vs str and make combat time work
            It is normally equal to your level \x (5 \add starting Constitution).
            For each point of \glossterm{fatigue} you have, your wound threshold is lowered by an amount equal to your level.

            When you take at least as much damage as your \glossterm{wound threshold} from a single damage roll, you gain a \glossterm{vital wound}.
            In addition, if your \glossterm{wound threshold} is reduced to 0, you gain a \glossterm{vital wound}.
            This can be caused by effects that inflict \glossterm{fatigue} without dealing damage.

        \subsubsection{Vital Wounds}\label{Vital Wounds}
            When you gain a \glossterm{vital wound}, you lose all points of \glossterm{fatigue} except for points of \glossterm{fatigue} that cannot be removed.
            You take a \minus2 penalty to Fortitude and Mental defenses for each \glossterm{vital wound} you have.
            In addition, each \glossterm{vital wound} has a specific detrimental effect on you.

            To determine the effect of a \glossterm{vital wound}, make a \glossterm{wound roll} and find the corresponding effect in \trefnp{Vital Wound Effects}.
            The effect of the wound lasts until you heal that \glossterm{vital wound}.
            The effects of vital wounds stack with each other, even if you roll the same effect twice for different \glossterm{vital wounds}.

            \parhead{Wound Roll}\label{Wound Roll}
            To make a \glossterm{wound roll}, roll 1d10 \sub the number of \glossterm{vital wounds} you have.

            \begin{dtable}
                \lcaption{Vital Wound Effects}
                \begin{dtabularx}{\textwidth}{l X}
                    \tb{Wound Roll} & \tb{Effect} \tableheaderrule
                    \minus1 or less & You are unconscious, and at the end of the next round you die \\
                    0 & You are unconscious \\
                    1 & Your maximum \glossterm{recovery action points} are reduced by 2 \\
                    2 & You gain two points of \glossterm{fatigue} that cannot be removed \\
                    3 & You move at half speed (minimum 5 ft.) \\
                    4 & Your maximum \glossterm{reserve action points} are reduced by 2 \\
                    5 & You take a \minus2 penalty to all defenses \\
                    6 & You take a \minus2 penalty to \glossterm{accuracy} \\
                    7 & Your maximum \glossterm{recovery action points} are reduced by 1 \\
                    8 & You gain a point of \glossterm{fatigue} that cannot be removed \\
                    9 & You take a \minus4 penalty to all skills \\
                    10 or more & No effect \\
                \end{dtabularx}
            \end{dtable}

        \subsubsection{Subdual Damage}\label{Subdual Damage}
            Some attacks and environmental effects deal subdual damage.
            Subdual damage works in the same way as normal damage, except that \glossterm{vital wounds} from subdual damage cannot kill.
            Any \glossterm{wound roll} below 0 from subdual damage is treated as a 0, causing the target to fall unconscious until the \glossterm{vital wound} is healed.

    \subsection{Initiative}\label{Initiative}
        When multiple creatures take mutually impossible actions simultaneously, such as racing to be the first one to a door, they must roll initiative checks.
        For details, see \pcref{Conflicting Actions}.
        Your bonus on \glossterm{initiative} checks is normally equal to your Dexterity or Perception, whichever is higher.

        \parhead{Movement-Based Initiative}\label{Movement-Based Initiative}
        When making \glossterm{initiative} checks to determine the success of movement, having a faster movement speed is helpful (see \pcref{Movement and Positioning}).
        For every 5 feet of movement you would have available after completing your movement, you gain a \plus2 bonus to any initiative checks necessary to determine whether your movement succeeds.
        Regardless of whether your initiative check succeeds or fails, you cannot use that ``excess'' movement to move after making such an initiative check.

    \subsection{Threat}\label{Threat}
        Each creature has a value that represents how threatening it is.
        This value is called a creature's \glossterm{threat}.
        Many monsters choose the targets of their attacks based on the threat of their foes.
        Your base threat is equal to the higher of your level and your Strength.
        In addition, you gain bonuses to your threat based on your equipment and some other effects which can make you appear more or less intimidating.

        \subsubsection{Threat From Equipment}
            If you are visibly wearing \glossterm{body armor}, you gain a bonus to your threat equal to half the defense bonus provided by the armor.
            For this purpose, ignore any abilities that increase the defense bonus you gain from the armor, and only use the base defense bonus from the armor itself.
            If you are visibly wielding a weapon or otherwise obviously capable of inflicting lethal damage, you gain a \plus2 bonus to your threat.

        \subsubsection{Threat From Size}
            You gain a \plus2 bonus to your threat for each size category that you are larger than Medium.
            Likewise, you take a \minus2 penalty to your threat for each size category that you are smaller than Medium.

        \subsubsection{Subjective Threat Modifiers}
            In addition to your base \glossterm{threat}, some actions and abilities modify the threat that particular creatures perceive you have.
            For example, the \textit{Conceal Threat} ability from the Bluff skill allows you to lower your apparent threat to creatures that fail to see through the deception (see \pcref{Conceal Threat}).

        \subsubsection{Consistent Targeting}
            Unless their descriptions state otherwise, monsters prefer to keep attacking the same target rather than changing targets frequently.
            A monster will only change targets based on \glossterm{threat} if the new target has a threat at least 5 points closer to its preference than its current target.

        \subsubsection{Threat While Incapacitated}
            While you are unconscious or otherwise obviously unable to take hostile actions, you take a \minus10 penalty to threat.
            Many creatures stop attacking unconscious creatures, preferring to focus on targets that can still cause harm.
            It is possible to trick creatures into thinking you are not a threat by playing dead or using similar tactics.

    \subsection{Insight Points}\label{Insight Points}
        You can spend \glossterm{insight points} to learn new special abilities.
        You normally have a number of \glossterm{insight points} equal to 1 \add your starting Intelligence.
        Some abilities can also grant insight points.

        You can spend an \glossterm{insight point} to gain two \glossterm{skill points}.
        In addition, every class has at least one way to spend \glossterm{insight points} to learn additional abilities.
        The list of ways to spend \glossterm{insight points} on class abilities is given below.
        \begin{itemize}
            \item Barbarian: Primal maneuvers
            \item Cleric: Augments, divine spells
            \item Druid: Augments, nature spells, wild aspects
            \item Fighter: Battle tactics, martial maneuvers
            \item Mage: Augments, arcane spells
            \item Monk: Estoric maneuvers, ki manifestations
            \item Paladin: Divine spells
            \item Ranger: Hunting styles, wild maneuvers
            \item Rogue: Trick maneuvers
            \item Warlock: Eldritch augments, pact spells
        \end{itemize}

\section{Circumstances, Bonuses, and Penalties}

    Many effects can grant bonuses or penalties to actions you take.

    \subsection{Size in Combat}\label{Size in Combat}
        Your size affects your \glossterm{space} and \glossterm{reach} in combat, your Strength, your Reflex defense, and how easily you overwhelm creatures and are overwhelmed yourself.
        These effects are shown on \trefnp{Size in Combat}.

        \begin{dtable*}
            \lcaption{Size in Combat}
            \begin{dtabularx}{\textwidth}{l l l l p{4em} p{4.5em} p{3em} p{5.5em} X}
                \tb{Size} & \tb{Space}\fn{1} & \tb{Reach}\fn{1} & \tb{Base Speed} & \tb{Strength Modifier}\fn{2} & \tb{Reflex Modifier} & \tb{Overwhelm Value}\fn{3} & \tb{Overwhelm Resistance}\fn{4} & \tb{Example Creature} \tableheaderrule
                Fine              & 1/2 ft.    & 0          & 10 ft. & \minus8 & \plus8  & 1/4 & \tdash & Fly                      \\
                Diminutive        & 1 ft.      & 0          & 15 ft. & \minus6 & \plus6  & 1/2 & \tdash & Toad                     \\
                Tiny              & 2-1/2 ft.  & 0          & 20 ft. & \minus4 & \plus4  & 1/2 & \tdash & Cat                      \\
                Small             & 5 ft.      & 5 ft.      & 25 ft. & \minus2 & \plus2  & 1 & \tdash & Halfling                 \\
                Medium            & 5 ft.      & 5 ft.      & 30 ft. & \tdash & \tdash  & 1 & \tdash & Human                    \\
                Large (tall)      & 10 ft.     & 10 ft.     & 40 ft. & \plus2 & \minus2 & 2 & 1 & Ogre                     \\
                Large (long)      & 10 ft.     & 5 ft.      & 40 ft. & \plus2 & \minus2 & 2 & 1 & Horse                    \\
                Huge (tall)       & 15 ft.     & 15 ft.     & 50 ft. & \plus4 & \minus4 & 3 & 2 & Cloud giant              \\
                Huge (long)       & 15 ft.     & 10 ft.     & 50 ft. & \plus4 & \minus4 & 3 & 2 & Bulette                  \\
                Gargantuan (tall) & 20 ft.     & 20 ft.     & 60 ft. & \plus6 & \minus6 & 4 & 3 & 50-ft.\ animated statue  \\
                Gargantuan (long) & 20 ft.     & 15 ft.     & 60 ft. & \plus6 & \minus6 & 4 & 3 & Kraken                   \\
                Colossal (tall)   & 25\add ft. & 25\add ft. & 70 ft. & \plus8 & \minus8 & 5 & 4 & Colossal animated object \\
                Colossal (long)   & 25\add ft. & 25\add ft. & 70 ft. & \plus8 & \minus8 & 5 & 4 & Great wyrm red dragon    \\
            \end{dtabularx}
            1 Creatures can vary in space and reach.  These are simply typical values.  \\
            2. Applies to total Strength, not starting Strength. \\
            3. See \pcref{Overwhelm Value}. \\
            4. See \pcref{Overwhelm Resistance}. \\
        \end{dtable*}

        \subsubsection{Space}\label{Space}
            A creature's \glossterm{space} is the area its body occupies while fighting.
            All humanoid species take up a 5-ft.\ by 5-ft.\ space in combat, which is a single \glossterm{square}.
            Normally, other creatures can't be in the space you occupy.
            Most creatures have a space significantly larger than the physical space their body occupies because they need room to maneuver in combat.

        \subsubsection{Reach}\label{Reach}
            A creature's \glossterm{reach} is the distance that its \glossterm{melee attacks} can reach.
            Enemies within a creature's reach are considered \glossterm{threatened}.

        \subsubsection{Base Speed}\label{Base Speed}
            A creature's \glossterm{base speed} is the distance that it can usually move.
            In addition to a base speed, most creatures have specific \glossterm{movement modes} that allow them to move in particular ways.
            The most common movement mode is a land speed, which allows creatures to move across the ground.
            Most creatures, including all humanoid species, have a land speed equal to their base speed.
            There are other movement modes that can allow creatures to move in different ways.
            For example, most birds have a \glossterm{fly speed}, which allows them to move through the air.
            For details about other speeds, see \pcref{Movement Modes}.

        \subsubsection{Other Effects}
            A creature's size affects a number of additional skills and abilities.
            For example, larger creatures have a penalty to the Stealth skill (see \pcref{Size and Stealth}).
            The effects of unusual size are described in those skills and abilities.
            Unusually large or small creatures also have other special rules apply to them, as described below.

        \subsubsection{Very Small Creatures}
            \parhead{Space} If a creature takes up less than a single square of space, you can fit multiple creatures in that square. You can fit four Tiny creatures in a square, twenty-five Diminuitive creatures, or 100 Fine creatures.

            \parhead{Reach} Creatures that take up less than 1 square of space typically have a natural reach of 0 feet, meaning they can't reach into adjacent squares. They must enter an opponent's square to attack in melee. You can attack into your own square if you need to, so you can attack such creatures normally. Since they have no natural reach, they do not threaten the squares around them.

            If a creature without a natural reach uses a reach weapon, it gains no benefits or penalties (see \pcref{Reach Weapon}).

            \parhead{Movement} Creatures three size categories smaller than you are not considered obstacles and do not hinder your movement.

        \subsubsection{Very Large Creatures}
            \parhead{Space} Very large creatures take up multiple squares. Anything which affects a single square the creature occupies affects the creature.

            \parhead{Reach} Creatures that take up more than 1 square typically have a natural reach of 10 feet or more, meaning that they can reach targets even if they aren't in adjacent squares. Creatures with a large natural reach can attack anyone within their reach, including adjacent foes.

            Creatures with a large natural reach using reach weapons can strike at up to double their natural reach but can't strike at their natural reach or less, just like Medium sized creatures.

            \parhead{Movement} Creatures three size categories larger than you are not considered obstacles and do not hinder your movement.

            % Should this be two size categories larger, or be more dependent on creature anatomy?
            \parhead{Immunities} Creatures at least three size categories larger than you are difficult to fight. You cannot get a \glossterm{critical hit} with \glossterm{strikes} or contribute to overwhelm penalties against such creatures. If you can reach a vulnerable point on the creature, such as by flying or by using ranged weapons, you can get critical hits and contribute to overwhelm penalties normally.

        \subsubsection{Arbitrary Modifiers}

            Circumstances frequently modify your odds of success when making attacks and checks, or when defending yourself from attacks.
            There are two kinds of circumstantial modifiers.
            Circumstances that make you better or worse at your task give you a bonus or penalty to your attack or check.
            Circumstances that make the task easier or harder increase or decrease the \glossterm{difficulty rating} of the task, or the defense of the attacked creature.

            Most circumstances grant a \plus2 bonus or impose a \minus2 penalty.
            Extraordinary circumstances can potentially have greater modifiers.
            All circumstantial modifiers should be used at the discretion of the GM.\@

        \subsubsection{Overwhelm}\label{Overwhelm}
            When you are being attacked by multiple foes at once, or by a massive foe that can attack from many directions, you are less able to defend yourself.
            You take penalties to your Armor defense equal to half the combined \glossterm{overwhelm value} of all creatures threatening you, rounded down.
            Among equal sized creatures, this usually means you take a penalty equal to half the number of creatures threatening you.
            These penalties are called \glossterm{overwhelm penalties}.
            If you are suffering at least a \minus1 overwhelm penalty, you are \glossterm{overwhelmed}.

            \parhead{Overwhelm Value}\label{Overwhelm Value} Your \glossterm{overwhelm value} affects how much of a penalty you impose on creatures you threaten, as described above.
            For example, Medium creatures have an overwhelm value of 1, and larger creatures have a higher overwhelm value (see \pcref{Size in Combat}).
            Some abilities can affect your overwhelm value.

            Some creatures have fractional overwhelm values.
            For example, a Tiny creature has an overwhelm value of 1/2.
            Fractional overwhelm values are not rounded down until after being added with the overwhelm values from other threatening creatures.
            For example, if five Tiny creatures were threatening a single creature, their combined overwhelm value would be 2 and 1/2.
            That value is rounded down to 2 when determining whether the creature is overwhelmed, and how large that creature's overwhelm penalties are.

            If your \glossterm{overwhelm value} would be reduced below 1, special rules apply.
            The first -1 penalty reduces your overwhelm value to 1/2.
            An additional -1 penalty reduces your overwhelm value to 1/4.
            Your overwhelm value cannot be reduced below 1/4.

            \parhead{Overwhelm Resistance}\label{Overwhelm Resistance} Some abilities grant \glossterm{overwhelm resistance}.
            For example, Large and larger creatures automatically gain overwhelm resistance (see \pcref{Size in Combat}).
            A creature with overwhelm resistance treats creatures threating it as if their \glossterm{overwhelm value} was reduced by amount equal to the creatures overwhelm resistance.
            Some abilities can increase or decrease overwhelm resistance, such as the \spell{boon of many eyes} spell.
            A creature without overwhelm resistance is considered to have an overwhelm resistance of 0.

            For example, Felix the fighter has an \glossterm{overwhelm resistance} of 1.
            If he was \glossterm{threatened} by a giant with an \glossterm{overwhelm value} of 2, he would reduce the giant's overwhelm value to 1.
            As a result, Felix would not be overwhelmed.

            \parhead{Ignoring Attackers} At the start of each phase, you can choose to ignore up to one creature threatening you.
            If you do, you are treated as being \unaware against that creature.
            In exchange, it does not contribute to the number of creatures overwhelming you.

        \subsubsection{Range Increments}\label{Range Increments}
            Most physical ranged attacks are less accurate against distant targets.
            This is represented with a \glossterm{range increment} for the attack, which is always measured in feet.
            You take a \minus1 penalty to accuracy with the ranged attack for each full range increment between you and your target.
            For example, when using a longbow with a range increment of 100 feet against a target 170 feet away, you take a \minus1 penalty to accuracy.
            You cannot make a ranged attack beyond 10 range increments away from you.

        \subsubsection{Cover}\label{Cover}

            Cover represents any obstacle that physically prevents you from striking your target, such as a tree or intervening creature.
            A creature behind cover is more difficult to attack.
            There are three kinds of cover: \glossterm{active cover}, \glossterm{passive cover}, and \glossterm{total cover}.
            All three types of cover are determined by the presence or absence of physical obstacles.

            \parhead{Active Cover}\label{Active Cover} Active cover is provided by mobile obstacles between you and your target, such as creatures or tree branches blowing in the wind.
            Physical attacks against creatures and objects with active cover suffer a 20\% miss chance.
            If an attack misses due to active cover, the attack is made against the intervening obstacle rather than being negated like normal for miss chances.
            The obstacle takes any damage from a successful attack normally.

            \parhead{Passive Cover}\label{Passive Cover} Passive cover is provided by immobile obstacles between you and your target, such as trees and walls.
            Creatures and objects with passive cover from you gain a \plus2 bonus to Armor defense.
            In addition, creatures with passive cover can hide (see \pcref{Stealth}).

            \parhead{Measuring Cover}

            When you make an attack, choose a single square within your \glossterm{space} and a single \glossterm{target square} within your target's space.
            If you are making a ranged attack, choose one corner of your space.
            If you are making a melee attack, choose any two corners of your square.
            These corners are called the \glossterm{points of origin} for your attack.
            For the purpose of determining cover, your attack originates from your chosen \glossterm{points of origin} and travels to the \glossterm{target square}.

            % TODO: pull out ``line of sight'' and ``line of effect'' from this text
            First, check if you can attack the target at all.
            For each \glossterm{point of origin} of your attack, you must be able to draw two lines to any two corners of your attack's \glossterm{target square}.
            These two lines must not overlap each other.
            In addition, each line must not be blocked by solid objects, though they can touch the edges of spaces blocked by solid objects.
            The lines can pass through obstacles that do not take up the entire area within their space (such as most creatures).
            Finally, the line must not be blocked by other squares within the target's space, preventing you from targeting the ``inside'' of large creatures.
            If you cannot draw such a line, the target has \glossterm{total cover} from you.
            This makes all targeted attacks impossible.

            Second, draw a line from the \glossterm{points of origin} of your attack to the center of your attack's target square.
            If any such line touches a square with an obstacle that grants active or passive cover, even at an edge or corner, the target has the appropriate cover from you.
            Otherwise, if the line is uninterrupted, the target does not have cover from you.

            \parhead{Partial Obstacles} Many obstacles, such as trees and low walls, can provide passive cover without normally blocking \glossterm{line of sight} or providing \glossterm{total cover}.
            Unusually small creatures, or creatures who intentionally take cover behind such obstacles, may be able to gain total cover from them.

            \parhead{Improved Cover}
            A creature can benefit from both passive and active cover.
            Cover of the same type generally doesn't stack; a creature behind two trees is not substantially more protected than a creature behind a single tree.
            However, exceptionally well covered creatures, such as a creature behind an arrow slit in a castle, may receive additional benefits.
            In that case, each additional major obstacle increases the miss chance by 10\% or increases the defense bonus by \plus1, as appropriate.

            \parhead{Total Cover}\label{Total Cover}
            If a creature is completely behind an physical object that blocks sight, it has \glossterm{total cover} from attacks.
            A creature with total cover cannot be targeted by any attacks.
            Total cover is the only kind of cover that also affects attacks other than \glossterm{physical attacks}.

        \subsubsection{Concealment}\label{Concealment}
            Concealment represents anything which makes it more difficult to see your target, such as dim lighting.
            A creature or object with concealment from you gains a \plus2 bonus to Armor defense.
            The concealment bonus does not apply if you can't see your opponent (such as if you close your eyes).
            Determining concealment works similarly to determining cover.
            You must use the same \glossterm{points of origin} and \glossterm{target square} when determining concealment that you would use to determine cover.

            \parhead{Determining Concealment} There are two things that can cause a creature to be concealed: poor lighting, and intervening obstacles that block sight.
            Determining concealment from obstacles that block sight works the same way as determining cover.

            Determining concealment from lighting conditions is simpler, since it ignores lighting conditions between you and the target.
            If your \glossterm{target square} square is in lighting that provides concealment, the target has concealment.
            Otherwise, it does not.

        \subsubsection{Helpless Defenders}
            A helpless creature is completely at an opponent's mercy.
            Its Armor and Reflex defenses are calculated as if it had a Dexterity of \minus10.
            Paralyzed, bound, and unconscious creatures are helpless.
            Any \glossterm{physical attack} against a helpless creature automatically \glossterm{explodes} on the first die.

        \subsubsection{Invisibility}\label{Invisibility}
            If it is impossible to see your target, you can't attack him normally. However, you can attack a square that you think he occupies. An attack into a square occupied by an invisible enemy has a 50\% miss chance. If an adjacent invisible creature strikes you, you can automatically identify the square he occupied when he struck you.

        \subsubsection{Surprise Attacks}\label{Surprise Attacks}
            Sometimes, creatures are not aware that combat is taking place when the combat starts. This most commonly happens with ambushes. Any creature that is not aware of the combat continues taking whatever actions it would normally be taking until it becomes aware of the combat. It is \unaware until that point.

    \subsection{Stacking Rules}\label{Stacking Rules}
        Usually, modifiers stack with each other, meaning that you add or subtract all of the modifiers to get the final result.
        However, some modifiers do not stack with each other, as described below.
        When bonuses don't stack with each other, you only apply the largest bonus.
        Likewise, when penalties don't stack with each ather, you only apply the largest penalty.

        \parhead{Special Exceptions}

        \begin{itemize}
            \item Effects from the same source do not stack. Any ability with the same name has the same source.
            \item Magic bonuses do not stack with each other.
            \item If a creature gains the same condition multiple times, the effects do not stack, but each instance of the condition is tracked separately.
                The creature must remove all instances of the condition before the effects are removed.
            \item \glossterm{Sizing} effects do not stack.
                If multiple effects both increase and decrease size, size increases offset size decreases on a one-for-one basis to determine the creature's final size.
            \item If a character has two separate abilities which let them add the same attribute to a given roll or statistic, the attribute is still only added once.
        \end{itemize}

        \subsubsection{Maximum Bonuses}\label{Ability Limits}
            Some bonuses specify that they cannot increase the value beyond a given point.
            These bonuses must always be applied last, and cannot be combined with other bonuses to exceed the maximum value.
            If multiple bonuses specify different maximum values, use the lower maximum value.
            If a bonus with a maximum value is applied to a value that already exceeds the maximum value the bonus can provide, simply ignore the bonus and its maximum value.

    \subsection{Doubling}\label{Doubling}
        If you double any in-game value twice, it becomes three times as large. An additional doubling would make it four times as large, and so on. For example, if you make an attack that deals double damage and you get a critical hit, you deal triple damage.

        \parhead{Real-World Values} Values that are ``real'', such as movement and distance, are an exception.
        Any real value has a unit that it measures, such as feet.
        Abstract values, such as bonuses and penalties to attacks and checks, do not have units.
        If you double a real-world value twice, it becomes four times as large.

    \subsection{Changing Statistics}

        Your modifiers and defenses can change for many reasons.
        In general, all changes take effect immediately, though some side effects of those changes may not happen until you rest or level up.

        \parhead{Numerical Modifiers} Changes to numerical modifiers always take effect immediately.
        For example, if a barbarian enters a rage, their damage and defenses are all adjusted immediately.

        \parhead{Skill Points} Effects that change a character's skill points take effect immediately.
        However, the character cannot spend additional skill points on new skills until they level up.
        If a character's total skill points are decreased below their currently spent skill points, they immediately lose training from skills until their spent skill points are equal to their total skill points.

\section{Movement and Positioning}\label{Movement and Positioning}

    This section describes in more detail how creatures move and position themselves on a battlefield.

    \subsection{Measuring Movement}

        For simplicity, all movement is measured in five-foot increments.
        While it is possible to be more precise than that, it's generally not worth the complexity.

        \parhead{Squares}\label{Squares} Area is commonly measured in 5-ft.\ by 5-ft.\ spaces called \glossterm{squares}.
        A single square represents the area occupied by a single humanoid creature in combat.
        Sometimes, movement and distance are represented by the number of squares travelled.
        A 30-ft.\ movement is the same thing as moving six squares.

        \parhead{Diagonals}\label{Diagonals} When measuring distance, the first diagonal counts as five feet of movement, and the second counds as ten feet of movement.
        The third costs five feet, the fourth costs ten feet, and so on.
        You can move diagonally past corners and enemies.

    \subsection{Movement Abilities}\label{Movement Abilities}

        Almost all creatures can use these abilities to move around a battlefield.
        Many movement abilities are reactive, allowing you to move automatically in response to the movement of other creatures.
        For example, you can try to follow a creature wherever it goes that round.
        In all cases, if you run out of movement speed before accomplishing your intended task, you simply stop where you ran out of movement.

        The most common types of reactive movements are the \textit{block}, \textit{follow}, and \textit{withdraw} abilities, which are described below.
        However, you can can come up with other reactive movements.
        The only requirement is that a reactive movement must have a simple criteria for determining how you move.

        \parhead{Hustle} As a \glossterm{move action}, you can use the \textit{hustle} ability to move.
        This is the most common movement ability.

        \begin{freeability}{Hustle}
            Choose a path that you want to travel.  You travel that path, up to the limit of your movement speed.
        \end{freeability}

        \parhead{Block} As a \glossterm{move action}, you can use the \textit{block} ability to prevent a creature from entering a particular area.

        \begin{freeability}{\labeltext{Block}}
            Choose a creature to block, and the area you want to block it from entering.
            During the current phase, you automatically move to intercept the target as it approaches the blocked area, up to the limit of your movement speed.
            Usually, blocking a target requires an opposed \glossterm{initiative} check against the target.
            Success means you successfully keep ahead of the target as it moves, preventing it from entering the area (unless it can move through you).
            Failure means the target moves around you (if there is room) to enter the area.

            Multiple creatures can coordinate to block a single creature.
            The blocked creature must beat the initiative of all blocking creatures to enter the blocked area.
        \end{freeability}

        \parhead{Follow} As a \glossterm{move action}, you can use the \textit{follow} ability to follow a creature as it moves.

        \begin{freeability}{\labeltext{Follow}}
            Choose a creature to follow, and the maximum distance you want to follow at.
            During the current phase, you automatically move such that your distance to the target is no greater than your desired follow distance, up to the limit of your movement speed.
            If the target uses an ability that makes it impossible to follow with movement, such as teleporting, you stop moving when you become adjacent to the position where it used that ability.
        \end{freeability}

        \parhead{Withdraw} As a \glossterm{move action}, you can use the \textit{withdraw} ability to keep away from creatures as they move.

        \begin{freeability}{\labeltext{Withdraw}}
            This ability functions like the \textit{follow} ability, except that you specify a minimum distance between you and the target instead of a maximum distance.
            In addition, you can specify multiple targets and try to keep away from all of them.
        \end{freeability}

        \parhead{Sprint} As a \glossterm{free action}, you can spend an \glossterm{action point} to use the \textit{sprint} ability, allowing you to briefly move more quickly.
        You cannot use this ability twice in the same round.

        \begin{apability}{\labeltext{Sprint}}[\glossterm{Swift}]
            You double your movement speed until the end of the current phase.
        \end{apability}

    \subsection{Movement Impediments}

        \parhead{Difficult Terrain}\label{Difficult Terrain}
        Some terrain is hard to move through, like thick bushes or a swamp.
        If a square is \glossterm{difficult terrain}, it doubles the movement cost required to move out of the square.
        That generally means it takes ten feet of movement, or fifteen feet if you are moving diagonally.

        If a square is considered difficult terrain for multiple reasons, the cost increases stack.
        For example, a square in a swamp that also has thick bushes blocking your passage would take twenty feet of movement, or thirty feet to move diagonally.

        \parhead{Obstacles}
        An obstacle is anything that gets in your way. Enemies and large solid objects like walls completely block your movement. If you can get past an obstacle, like a low wall, that square is treated as difficult terrain. Some obstacles require a \glossterm{check} to bypass, such as an Acrobatics check (see \pcref{Acrobatics}).

        \parhead{Squeezing}\label{Squeezing}
        In some cases, you may have to squeeze into or through an area that isn't as wide as the space you take up.
        You can squeeze through or into a space that is at least half as wide as your normal space.
        While \glossterm{squeezing}, you move half as fast, and you take a \minus2 penalty to \glossterm{accuracy} with \glossterm{strikes} and Armor and Reflex defenses.
        You can squeeze into tighter spaces with the Escape Artist skill.

        Creatures that take up multiple squares take up half their normal number of squares while squeezing. For example, a Large creature who normally takes up four spaces takes up two spaces while squeezing.

        \parhead{Accidentally Squeezing} Sometimes a character ends its movement while moving through a space where it's not normally allowed to stop. When that happens, the character is squeezing in the space until it can move. If squeezing is impossible, the creature immediately moves to the closest available space. Try not to do this.

        \parhead{Undergrowth}\label{Undergrowth} Vines, roots, bushes, and similar plants that can obstruct movement are common in forested areas.
        These small plants can impede movement in large quantities.
        There are two kinds of undergrowth: \glossterm{light undergrowth} and \glossterm{heavy undergrowth}.

        \subparhead{Light Undergrowth}\label{Light Undergrowth}
        Light undergrowth is \glossterm{difficult terrain} and provides \glossterm{concealment}.

        \subparhead{Heavy Undergrowth}\label{Heavy Undergrowth}
        Heavy undergrowth quadruples the movement cost required to move out of each square and provides \glossterm{concealment}.
        In addition, using the \textit{charge} and \textit{sprint} actions is impossible in heavy undergrowth (see \pcref{Movement Abilities}).

    \subsection{Movement Modes}\label{Movement Modes}
        A movement mode is a method of moving from one location to another.
        % TODO: terminology confusion mode vs. speed
        The most common movement mode is a land speed.
        In addition, some abilities grant creatures the ability to move in unusual ways.
        These forms of movement are described here.

        \parhead{Burrowing}
        A creature with a burrow speed can move through the ground at the indicated speed in any direction, even vertically. Unless otherwise noted, the creature can only burrow through dirt and loose earth, not rock or harder substances. It does not leave behind a usable tunnel for other creatures.

        \parhead{Climbing}
        A creature with a \glossterm{climb speed} can move a distance equal to its climb speed with a successful Climb check (see \pcref{Climb}).
        In addition, it gains a \plus10 bonus to any Climb checks it makes.

        \parhead{Flying}\label{Flying}
        A creature with a \glossterm{fly speed} can fly through the air at the indicated speed.
        It must not be carrying weight in excess of its maximum carrying capacity (see \pcref{Carrying Capacity}).

        Each creature with a fly speed also has a maneuverability: good, average, poor, or special.
        Unless otherwise specified, a creature with a fly speed has average maneuverability.

        Normally, a flying creature must move forward by at least half its fly speed each round. If it does not, it falls.
        Turning by 90 degrees costs 5 feet of movement, and it can't turn in the same place by more than 90 degrees.
        It can move up by only one square vertically per square traveled horizontally, but it can fly directly down if it chooses.
        The creature can fly up at half speed, but can fly down twice as fast.

        \parhead{Maneuverability}\label{Maneuverability} Some creatures have fly speeds with special maneuverability rules.

        \subparhead{Good Maneuverability} If a creature has good maneuverability while flying, it gains three benefits while flying.
        First, it not need to move forward to maintain its flight, allowing it to hover.
        Second, it can turn in place without spending movement.
        Third, it can move up at the same speed as it moves horizontally.

        \parhead{Poor Maneuverability} If a creature has poor maneuverability while flying, it must spend five feet of movement to turn by 45 degrees, and it can't turn in the same place by more than 45 degrees. In addition, it can only descend by up to one square vertically per square traveled horizontally without falling.

        \parhead{Falling} If a flying creature loses control, usually by failing to maintain its minimum forward speed, or loses the ability to fly, it falls just like any other creature would in midair. As long as it still has the ability to fly, it can regain control of its fall as a standard action, causing it to resume flying normally.

        \parhead{Gliding}\label{Gliding}
        A creature with a glide speed can glide through the air at the indicated speed.

        While in the air, a creature with a glide speed can control its fall as a move action. This allows it to move up to its speed horizontally in a direction of its choice while moving only five feet down. If it desires, it can move half as far horizontally and fall down twice as fast. It takes no falling damage if it touches the ground while gliding.

        \parhead{Land}
        A creature with a land speed can move across the ground at the indicated speed.
        Most creatures have a land speed.

\section{Ability Mechanics}\label{Ability Mechanics}

    \subsection{Targets}\label{Targets}
        Almost all abilities affect targets.
        A target of an ability is a creature directly affected by the ability in some way.
        Many abilities affect targets within a specific \glossterm{range}.

        \subsubsection{Targeted Abilities}\label{Targeted Abilities}
            Some abilities allow you to choose specific targets.
            There can be restrictions on the targets of the ability, such as ``a creature or object'' or ``a willing creature''.
            These abilities are called \glossterm{targeted} abilities.

        \subsubsection{Area Abilities}
            Some abilities affect all valid targets within a given area.
            There can be restrictions on the targets of the ability, such as ``all creatures'' or ``all enemies''.
            However, you cannot individually choose to include or exclude specific targets.
            These abilities are not \glossterm{targeted} abilities.

        \subsubsection{Willing Targets}
            Some abilities can only target willing creatures.
            Creatures choose to be considered willing when abilities choose their targets.
            If a creature chose to be willing at the time when the ability targeted it, the ability will affect that creature even if it decides to be unwilling after that step.

        \subsubsection{Invalid Targets}
            % clarify timing
            You can always attempt to use an ability on an invalid target.
            If the target is still invalid when the ability resolves, the ability automatically fails and has no effect on the target.
            A \glossterm{spell} that fails in this way is \glossterm{miscast} (see \pcref{Miscasting}).

    \subsection{Range}\label{Range}
        Many abilities can only affect targets or areas within a given \glossterm{range} of you.
        For abilities that affect specific targets, all targets must be within the range.
        For abilities that affect an area within a range, the area's \glossterm{point of origin} must within the range (see \pcref{Point of Origin}).
        There are four common ranges: \rngclose, \rngmed, \rnglong, and \rngext.
        Unless otherwise noted, all abilities with a range require both \glossterm{line of sight} and \glossterm{line of effect} to the point of origin or to all targets.

    \subsection{Line of Sight}\label{Line of Sight}
        Almost all abilities, including \glossterm{strikes}, must have \glossterm{line of sight} to target creatures or objects.
        Unless otherwise noted in an ability's description, you cannot target a creature, object, or location that you do not have line of effect to.

        A line of sight is a straight, unblocked path between you and a target.
        To check if you have line of sight, find a path from any corner of one \glossterm{square} within your \glossterm{space} to any two corners of one \glossterm{square} within the \glossterm{space} of your target.
        If those lines are not blocked by any obstacles that impede sight, you have line of sight to your target.

    \subsection{Line of Effect}\label{Line of Effect}

        Almost all abilities, including \glossterm{strikes}, must have a \glossterm{line of effect} to function.
        Unless otherwise noted in an ability's description, you cannot target a creature, object, or location that you do not have line of effect to.
        In addition, abilities that affect an area do not affect targets that the ability does not have line of effect to.

        A line of effect is a straight, unblocked path between you and a target.
        It is identified in the same way as \glossterm{line of sight}, except that it is blocked by physical obstacles instead of obstacles that block sight.
        For example, a pane of glass would block line of effect, but not line of sight.

        \subsubsection{Area Line of Effect}\label{Area Line of Effect}
            Abilities that affect areas normally measure line of effect from the area's \glossterm{point of origin}.
            This can allow you to affect targets that you do not have line of effect to as long as the point of origin has line of effect to both you and the target.

            Areas originating from creatures do not have a single point of origin.
            Instead, line of effect is measured from all grid intersections within or touching the creature's space.
            If any such grid intersection has line of effect to a location, the area as a whole is considered to have line of effect to that location.

        \subsubsection{Destroying Barriers}\label{Destroying Barriers}
            Some abilities deal damage to both creatures and objects.
            If a physical barrier is \glossterm{broken} by an ability, that barrier does not affect the ability's line of effect.
            For example, a thin curtain of silk normally blocks line of effect.
            However, an ability that destroyed the curtain would have its full effect on everything behind the curtain.

        \subsubsection{Inside Creatures}
            Creatures block line of effect to the inside of their own bodies.
            As a result, you cannot use an ability that takes effect inside a creature unless you are also inside the creature.
            This restriction applies even if there is no physical barrier to the inside of the creature.
            You cannot place \glossterm{point of origin} for an area inside a creature's mouth, even if the creature has its mouth open at the time.

    \subsection{Area}\label{Area}

        Some abilities affect targets within an area.
        All areas have a \glossterm{point of origin}, an area shape, a measurement of their size in feet, and an area type.

        \subsubsection{Point of Origin}\label{Point of Origin}
            When you use an ability that affects an area within a \glossterm{range}, you choose one grid intersection to serve as a starting point for the area.
            This grid intersection is called the \glossterm{point of origin} for the area.
            Areas that originate from a creature do not have a single point of origin.
            % What ambiguity does having multiple points of origin cause?
            For the purpose of effects that care about the area's point of origin, all grid intersections within or touching the creature's space are used.

        \subsubsection{Area Shape}

            \parhead{Cone} A cone extends from the point of origin in a quarter-sphere, up to the given length.

            \parhead{Cylinder} A cylinder extends out from the point of origin in a circle, up to the given radius.
            Cylinders also have a specific height.
            Unless otherwise specified, a cylinder's height is the same as its radius.
            Cylinders ignore obstacles that partially block line of effect, as long as there is a path around the obstacle that lies entirely within the ability's area.

            \parhead{Line} A line extends from the point of origin in a straight line, up to the given length.
            Lines also have a specific width and height.
            Unless otherwise specified, a line-shaped ability affects an area 5 feet wide and 5 feet high.
            The affected squares are chosen such that they stay close to the chosen line as possible.
            All squares affected by a line must be contiguous, so every square is adjacent to another affected square, disregarding diagonals.

            \parhead{Sphere} A sphere extends from the point of origin in all directions.
            Any ability which only specifies a radius for its area is sphere-shaped.

            \parhead{Wall} A wall is like a line, except that it has no width.
            Instead, it affects the boundary between squares.
            Walls can also be shapeable.

            Walls can normally be created within occupied squares, but not within solid objects.
            Some walls are called solid walls, and cannot be created within occupied squares.

            \parhead{Specific Shapes} Some abilities specify a series of volumes that make up the area of the ability.
            Most commonly, the volumes are cubes.
            You may arrange the volumes as you want, with the restriction that each volume in the ability's area must be adjacent to one other volume in the ability's area.

        \subsubsection{Area Size}

            The area affected by many abilities falls into one of three sizes.
            Each size defines the extent to which the ability extends out from its origin, whether as a radius or as a length.
            Some abilities have specific sizes, as given in the ability description.

            \parhead{Small} Small abilities extend 10 feet from their point of origin.
            \parhead{Medium} Medium abilities extend 20 feet from their point of origin.
            \parhead{Large} Large abilities extend 50 feet from their point of origin.

        \subsubsection{Area Types}\label{Area Types}

            \parhead{Burst} A burst ability has an immediate effect on all valid targets within an area.

            \parhead{Emanation} An emanation ability has effects within an area for the duration of the ability.
            It emanates from a specific creature or object, rather than a location.
            If that creature or object moves, the emanation moves with it.

            \parhead{Zone} A zone ability has effects within an area for the duration of the ability.
            Unless otherwise noted, it does not move after being created.


        When casting an area ability, you select the point where the ability originates.
        The point of origin of a ability is always a grid intersection.
        When determining whether a given creature is within the area of a ability, count out the distance from the point of origin in squares just as you do when moving a character or when determining the range for a ranged attack.
        The only difference is that instead of counting from the center of one square to the center of the next, you count from intersection to intersection.

        You can freely decrease a ability's area, provided that you decrease it uniformly across all of the ability's dimensions.
        For example, you can cast a \spell{fireball} spell that affects a 5 foot radius if you choose to do so, but you can't cast a \spell{fireball} with any shape other than a sphere.

        You can count diagonally across a square, but remember that every second diagonal counts as 2 squares of distance.
        If the far edge of a square is within the ability's area, anything within that square is within the ability's area.
        If the ability's area only touches the near edge of a square, however, anything within that square is unaffected by the ability.

    \subsection{Ability Durations}

        An ability's duration determines how long its effect lasts.
        Abilities can have one of several different kinds of durations.

        \subsection{Conditions}\label{Conditions}
            Many abilities impose \glossterm{conditions} on their targets.
            A condition lasts until it is removed.
            You can remove a condition by taking a \glossterm{short rest} or using the \textit{cleanse} ability (see \pcref{Cleanse}).
            There are several other abilities that can also remove conditions.

        \subsection{Attunement}\label{Attunement}
            Many abilities last as long as a creature \glossterm{attunes} to them.
            % This wording would be simpler if attuning to abilities strictly cost recovery action points. 
            % However, that leads to weird scenarios where it's sometimes better to use reserve AP while you still have recovery AP remaining.
            Attuning to an ability costs an \glossterm{action point}.
            When you attune to an ability, you must choose one of your \glossterm{recovery action points} which has not been used to attune to an ability.
            As long as you remain attuned to that ability, you cannot recover that action point by any means.
            If you cannot select a \glossterm{recovery action point} in this way,
                such as if you have already attuned to as many abilities as you have recovery action points,
                you cannot attune to the ability.
            You must release your attunement before beginning to rest in order to recover the associated \glossterm{recovery action point}.

            You can choose to stop attuning to an ability as a \glossterm{free action}.
            If you do, the ability's effect ends.

            Attuned abilities continue to work across any distance, but not across planar boundaries.
            % TODO: wording
            At the end of each round, your attunement to all abilities created by creatures on a different plane than your current plane ends.
            Planar travel that does not last a full round, such as teleportation within a plane, does not interrupt your attunement.

            % This name is not great
            \subsubsection{Duplicate Attunement Abilities}\label{Duplicate Attunement Abilities}
                When you use an \glossterm{Attune} (self) or \glossterm{Attune} (target) ability, any previous activations of that ability are immediately \glossterm{dismissed}.
                Any creatures that were attuned to the previous activation of the ability have their attunement released.
                \glossterm{Attune} (ritual) abilities do not have this limitation.
                You may have any number of activations of \glossterm{Attune} (ritual) abilities active at once.

                % What does this even mean? Intended to talk about within-ability choices, I think.
                Minor variations of a single ability are considered to be the same ability for this purpose.
                Applying different \glossterm{augments} to a \glossterm{spell} does not allow you to attune to it more than once.

            \subsubsection{Attunement Types}\label{Attunement Types}
                There are three types of attunement abilities: self, target, and ritual.

                \parhead{Attune (self)} A self attunement ability requires the creature using the ability to attune to the effect.

                \parhead{Attune (target)} A target attunement ability requires the target of the ability to attune to the effect.
                If the ability targets multiple creatures, each creature must attune to the ability independently.

                As a special case, if a target attunement ability targets an inanimate object, the creature using the ability must attune to the effect.

                \parhead{Attune (ritual)} Only \glossterm{rituals} have the \glossterm{Attune} (ritual) tag.
                A ritual attunement ability requires any participant in the ritual to attune to the effect.
                In addition, ritual attunement abilities are not subject to the normal restrictions on multiple attunement.
                % TODO: wording
                You can maintain any number of activations of a particular ritual attunement ability at once.

        \subsection{Sustained Abilities}\label{Sustained Abilities}
            Some abilities last as long as you take an action to sustain them each round.
            The type of action required is always specified in the ability.
            At the end of each \glossterm{action phase}, the ability is dismissed unless you used the ability that phase or took the action to sustain the ability that phase.
            Sustaining a spell does not take concentration, and cannot be disrupted in the same way that casting a spell can (see \pcref{Concentration}).

            % TODO: a more robust timing system would make stating this explicitly unnecessary
            If a sustained ability has effects that trigger at the end of the action phase, it ends before having its effects if you fail to sustain the ability.

            Taking an action to sustain an ability only allows you to sustain a single use of that ability.
            However, you can sustain multiple separate abilities at once if you have available actions.

            You can only sustain an ability for up to 5 minutes.
            After that time, the ability's effect is \glossterm{dismissed}.

        \subsection{Permanent}
            Some abilities last permanently.
            Such abilities never expire on their own, but can be \glossterm{dismissed} or removed by other abilities appropriately.

        % Is this necessary?
        % Should this clarify interactions with bursts/zones/emanations?
        \subsection{Targeting and Durations}
            If an ability targets creatures or objects directly, the effects travel with the targets for the ability's duration.
            If an ability creates or summons objects or creatures, they last for the duration of the ability, and are capable of moving outside the ability's initial range.
            Such effects can sometimes be destroyed prior to when their duration ends.

    \subsection{Combining Effects}
        Abilities do not generally affect the way another abilities function.
        However, sometimes multiple effects can be in conflict on a creature.
        If one effect makes another effect irrelevant or impossible, the latter effect is ignored.
        If two effects both conflict with each other, the most recent effect takes precedence, and the other is ignored.
        Unless otherwise noted, two different uses of the same ability are always considered to be conflicting with each other.

        All abilities will still have as much of their effect as possible.
        It is possible for an ability to be partially effective in this way.

    \subsection{Suppressing Abilities}\label{Suppressing Abilities}
        Abilities can be \glossterm{suppressed} by effects such as the \spell{suppress magic} spell.
        While an ability is suppressed, it has no effect.
        However, if it stops being suppressed, its effects continue as if they had not been interrupted.

    \subsection{Ability Tags}\label{Ability Tags}

        Many abilities have tags that describe the nature of the ability.
        Many of these tags have no game effect by themselves, but they govern how the ability interacts with spells, other abilities, unusual creatures, and so on.
        They are described below.

        \parhead{Acid} Acid abilities use corrosive acid.
        They do not function underwater.

        \parhead{Air} Air abilities control the surrounding air.
        They do not function in environments without air.

        \parhead{AP} AP abilities require spending an \glossterm{action point} to use.
        Some AP abilities, such as specific \glossterm{rituals}, require spending multiple action points instead of only one.
        The number of action points required will be given in the ability's description.

        \parhead{Auditory} Auditory abilities use sound to cause their effects.
        Creatures and objects that cannot hear the effect are immune to it.

        \parhead{Cold} Cold abilities use cold \glossterm{energy}. It is possible to freeze liquids and have similar effects with cold abilities.

        \parhead{Compulsion} Compulsion abilities forcibly alter a creature's actions, but do not necessarily affect its opinions or personality.
        They have no effect on objects or creatures without minds.

        \parhead{Creation} Creation abilities create permanent physical objects.
        Objects created with Creation abilities are identical to objects created through more mundane means.

        \parhead{Curse} Curse abilities lay supernatural curses on their targets.
        They cannot be \glossterm{dismissed}, but can be removed with the \spell{remove curse} spell.

        \parhead{Emotion} Emotion abilities alter a creature's opinons or personality, but do not necessarily affect their actions.
        They have no effect on objects or creatures without minds.

        \parhead{Detection} Detection abilities reveal magical auras or information within an area.
        They can penetrate up to 1 foot of stone, 1 inch of common metal, a thin sheet of lead, or 3 feet of wood or dirt.
        For its ability to penetrate other materials, use the most similar substance from the list above.

        \parhead{Earth} Earth abilities manipulate the ground or other forms of dirt.
        They do not function if no earth is accessible.

        \parhead{Electricity} Electricity abilities use electrical \glossterm{energy}.

        \parhead{Sensation} Sensation abilities create or manipulate light, sound, or other sensations.
        You can only create sensations you understand.
        For example, you cannot create an illusory figment which speaks coherently in a language you do not understand.
        \par If a Sensation ability appears to create a physical object or creature, its defenses are equal to 0.

        \parhead{Fire} Fire abilities use fire \glossterm{energy}. They do not function underwater.
        \par Fire abilities provide light equivalent to a torch for their duration.
        Abilities without a duration create a brief burst of torchlight.

        \parhead{Flesh} Flesh abilities manipulate the physical flesh of creatures.
        They have no effect on creatures without flesh, such as ghosts or oozes.

        \parhead{Life} Life abilities attack, restore, or manipulate the life force of creatures.
        They have no effect on objects and creatures that are not alive.
        \par Undead creatures are affected in a special way by Life abilities.
        In addition to any differences given in the effect's description, life damage instead heals undead creatures, and healing instead deals life damage.

        \parhead{Light} Light abilities create visible light.
        Their area is blocked by barriers that prevent sight, even if the barriers would not otherwise block effect areas.
        Similarly, their area of effect is not blocked by barriers which do not prevent sight, even if the barriers would normally block effect areas.

        \parhead{Manifestation} Manifestation abilities create temporary constructs formed from raw magical energy.
        Objects and creatures created with manifestation abilities seem real on the surface, but they have no internal structure.
        When an object or creature created by a Manifestation ability is destroyed or killed, or when the duration of the ability that created it ends, it disappears without a trace.

        \parhead{Mystic} Mystic abilities alter or destroy magic itself.

        \parhead{Physical} Physical abilities manipulate physical objects rather than having a direct magical effect on their targets.
        % TODO: find example? Does this still exist
        Some abilities are not themselves Physical, but have Physical effects.

        \parhead{Planar} Planar abilities transport matter or information between planes.

        \parhead{Poison} Poison abilities use substances to weaken the foe's body.

        \parhead{Scrying} Scrying abilities create one or more invisible magical sensors that send you information.
        Unless otherwise noted, the sensor created has the same powers of sensory acuity that you possess.
        This includes the effect of any abilities which target you personally, such as spells to increase your visual acuity, but not abilities which affect an area around you.
        However, the sensor is treated as a separate, independent sensory organ, and it functions normally even if you have been blinded, deafened, or otherwise suffered sensory impairment.
        \par Any creature trained in Spellcraft can notice the sensor by making a DR 20 Spellcraft check.
        The sensor can be dismissed as if it were an active spell.
        You cannot create a sensor in a location with lead sheeting between you and the location, and you sense that the effect is blocked in this way.
        \\

        \parhead{Shaping} Shaping abilities change the shape or structure of their targets.

        \parhead{Shielding} Shielding abilities improve the defenses of their targets.

        \parhead{Sizing} Sizing abilities alter the size of their targets.
        Unless otherwise stated, multiple effects which increase or decrease size do not stack.
        Opposing size modifications cancel each other out on a one for one basis, and any remaining effects occur normally.

        \parhead{Sonic} Sonic abilities use sonic \glossterm{energy}.

        \parhead{Speech} Speech abilities use words to achieve their ends.
        You must specify a language when using a Speech effect, and the language must be one you know (or have memorized the correct words to say). They have no effect on objects or creatures that do not understand the chosen language.

        \parhead{Subtle} Subtle abilities have no visual or otherwise perceivable manifestation.
        Creatures successfully affected by Subtle abilities do not generally know that they are being magically influenced.
        However, a creature that successfully resists a Subtle ability can generally notice that it resisted a magical effect of some kind, just like a non-Subtle ability.
        Subtle spells can still be identified with the Spellcraft skill (see \pcref{Spellcraft}), but the DR is 10 higher than normal.

        \parhead{Swift} Swift abilities take effect before other abilities used during the same phase.
        For details, see \pcref{Swift Abilities}.

        \parhead{Teleportation} Teleportation abilities move creature or objects through the Astral Plane to a distant destination.
        A teleported creature can bring along equipment and held objects as long as their weight does not exceed the creature's maximum carrying capacity (see \pcref{Carrying Capacity}). Any excess items are left behind, in order of their distance from the creature's body.

        \parhead{Temporal} Temporal abilities alter the flow of time.

        \parhead{Trap} Trap abilities do not have their full effect immediately.
        All Trap abilities specify a condition or circumstance, such as opening a door, which triggers the full effect of the ability.
        \par Unless otherwise noted, active Trap effects can be detected with the Awareness skill and disabled with the Devices skill before their effect triggers (see \pcref{Awareness}, and \pcref{Devices}).
        The DR to detect and disable the effect is equal to 20 \add the \glossterm{power} of the effect.
        \par No more than one Trap ability can be placed on the same object or in the same area.
        Only the first trap placed has any effect.
        It must be dismissed before any new traps can be placed.

        \parhead{Visual} Visual abilities use visible objects or forces to cause their effects.
        Creatures and objects that cannot see the effect are immune to it.

        \parhead{Water} Water abilities use water to cause their effects.

    % TODO: different section?
    \subsection{Damage Types}\label{Damage Types}
        Abilities can deal many kinds of damage.
        The damage types are listed below, along with any special properties that type of damage has.

        \begin{dtable}
            \lcaption{Damage Types}
            \begin{dtabularx}{\columnwidth}{l X}
                \tb{Name} & \tb{Special Effects} \tableheaderrule
                Acid & Effective against many objects \\
                Bludgeoning & A type of physical damage \\
                Cold & At type of energy damage \\
                Divine & \\
                Electricity & A type of energy damage \\
                Fire & A type of energy damage \\
                Life & Heals undead creatures instead of damaging them \\
                Physical & \\
                Poison & \\
                Piercing & A type of physical damage \\
                Slashing & A type of physical damage \\
                Sonic & Effective against many objects, a type of energy damage \\
                Thaumatic & \\
            \end{dtabularx}
        \end{dtable}

    \subsection{Magical and Mundane Abilities}\label{Magical and Mundane Abilities}

        There are two types of abilities: magical abilities and mundane abilities.

        \parhead{Magical Abilities}\label{Magical Abilities} A \glossterm{magical} ability is an ability that has no physical explanation.
        Examples include \glossterm{spells}, a medusa's petrifying gaze, and a cleric's domain invocations.
        Abilities that are magical in nature are indicated with a (Magical) indicator.
        Abilities that are not magical are \glossterm{mundane}.

        \parhead{Mundane Abilities}\label{Mundane Abilities} A \glossterm{mundane} ability has a tangible component and some form of natural explanation.
        Examples include weapon attacks, a dragon's breath weapon, and a barbarian's rage.
        Mundane attacks often target Armor defense.
        Unless otherwise indicated, all abilities are mundane in nature.
        Abilities that are not mundane are \glossterm{magical}.

\section{Universal Abilities}\label{Universal Abilities}
    All creatures can use the following abilities.

    \subsection{Strikes}\label{Strikes}
        A \glossterm{strike} is the most common type of attack.
        There are three kinds of strikes: melee, projectile, and thrown.
        Many abilities allow you to make one or more strikes.
        When you make a strike, you can choose which kind of strike to make.

        All strikes are \glossterm{mundane} abilities.
        Your \glossterm{accuracy} with a strike is equal to your normal accuracy (see \pcref{Accuracy}).
        Your \glossterm{damage} with a strike is determined by your \glossterm{power} and the weapon you hit with (see \pcref{Strike Damage}).

        \parhead{Two-Weapon Strikes}\label{Two-Weapon Strikes}
        All strike abilities require a \glossterm{standard action} and allow you to attack with either one or two weapons.
        If you make a strike with two weapons, you take a \minus1 penalty to \glossterm{accuracy} with both attacks for each non-light weapon you attack with.

        \begin{freeability}{\labeltext{Melee Strike}}
            Choose a creature you \glossterm{threaten} and one or two \glossterm{melee weapons} that you can attack with.
            For each weapon, make a \glossterm{physical attack} with that weapon against the Armor defense of the target.

            On a hit with any weapon, the target takes damage from one weapon you hit with (see \pcref{Strike Damage}).
            On a critical hit with any weapon, the target takes double damage from one weapon you hit with.
        \end{freeability}

        \begin{freeability}{\labeltext{Projectile Strike}}
            Choose a creature you can target and one or two \glossterm{projectile} weapons you can attack with, and a creature you can see.
            The creature must be within ten \glossterm{range increments} of you with all weapons.
            For each weapon, make a \glossterm{physical attack} with that weapon against the Armor defense of the target.
            The attack takes a \minus1 penalty to \glossterm{accuracy} for for each full range increment between you and the target with that weapon.

            On a hit with any weapon, the target takes damage from one weapon you hit with (see \pcref{Strike Damage}).
            On a critical hit with any weapon, the target takes double damage from one weapon you hit with.
        \end{freeability}

        \begin{freeability}{\labeltext{Thrown Strike}}
            Choose a creature you can target and one or two \glossterm{thrown weapons} you can attack with.
            The creature must be within five \glossterm{range increments} of you with all of the weapons.
            For each weapon, make a \glossterm{physical attack} with that weapon against the Armor defense of the target.
            The attack takes a \minus1 penalty to \glossterm{accuracy} for for each full range increment between you and the target with that weapon.

            On a hit with any weapon, the target takes damage from one weapon you hit with (see \pcref{Strike Damage}).
            On a critical hit with any weapon, the target takes double damage from one weapon you hit with.
        \end{freeability}

        \subsubsection{Strike Damage}\label{Strike Damage}
            The damage you deal with a single \glossterm{strike} is equal to \glossterm{standard damage} based on your \glossterm{power} with \glossterm{mundane} abilities (see \pcref{Standard Damage}), plus that weapon's damage modifier, if any (see \tref{Weapons}).
            For example, if you had a \glossterm{power} of 4, your base damage on a strike would be 1d10.
            If you then made a strike with a greatsword, which has a \plus1d damage modifier, your total damage would be 2d6.

    \subsection{Special Combat Abilities}\label{Special Combat Abilities}

        \begin{dtable}
            \lcaption{Special Combat Abilities}
            \begin{dtabularx}{\columnwidth}{>{\lcol}p{6em} l X}
                \tb{Ability}  & \tb{Defense} & \tb{Brief Description} \tableheaderrule
                Charge\fn{1}  & Armor        & Move and attack                         \\
                Cleanse\fn{1} & \tdash       & Quick remove a \glossterm{condition}    \\
                Dirty Trick   & Any          & Impose penalty on a foe                 \\
                Disarm        & Ref          & Attack item, knocking it free           \\
                Feint         & Ref          & Leave foe vulnerable to attacks         \\
                Grapple       & Fort and Ref & Wrestle with a foe                      \\
                Overrun\fn{1} & Fort         & Move through foe's space                \\
                Recover       & \tdash       & Regain a spent \glossterm{action point} \\
                Shove         & Fort         & Move a foe                              \\
                Struggle      & \tdash       & Move 5 feet regardless of penalties     \\
                Total Defense & \tdash       & Gain \plus2 to defenses                 \\
                Trip          & Ref          & Trip a foe                              \\
            \end{dtabularx}
            1. This ability costs an \glossterm{action point} to use.
        \end{dtable}

        \parhead{Charge} You can use the \textit{charge} ability as a standard action during the \glossterm{action phase}.

        \begin{apability}{\labeltext{Charge}}
            Move up to your speed in a single straight line.
            During the \glossterm{delayed action phase}, you can make a melee \glossterm{strike} from your new location.
        \end{apability}

        \parhead{Cleanse} You can use the \textit{cleanse} ability as a \glossterm{minor action}.
        As long as you are conscious, no effect can prevent you from using this ability, even effects that prohibit using any other abilities.
        \begin{apability}{\labeltext{Cleanse}}
            You remove one \glossterm{condition} affecting you.
            This cannot remove a condition applied during the current round.
        \end{apability}

        \parhead{Dirty Trick} As a standard action, you can use the \textit{dirty trick} ability to creatively impair a foe's ability to fight.

        \begin{freeability}{Dirty Trick}\label{Dirty Trick}
            When you use this ability, you must describe the kind of dirty trick you are performing.
            For example, you can pull a creature's pants down, throw sand, or otherwise use your environment to attack.

            Make a melee \glossterm{physical attack} with a free hand against the Fortitude or Reflex defense of a creature you \glossterm{threaten}.
            The target uses whichever defense is appropriate to the nature of the trick you describe.

            On a hit, as a \glossterm{condition}, the target suffers a \minus2 penalty to one of the following statistics:
                \glossterm{accuracy} with \glossterm{physical attacks}, \glossterm{concentration} checks, Armor defense, Fortitude defense, Reflex defense, or Mental defense.
        \end{freeability}

        \parhead{Disarm} As a standard action, you can use the \textit{disarm} ability to knock an item out of a foe's hands.

        % Somewhat oddly? dual wielding is significantly better at disarming than two-handing
        \begin{freeability}{Disarm}\label{Disarm}
            Make a melee \glossterm{strike} against an object.
            Unlike most abilities, this ability can target specific items \glossterm{attended} by creatures.
            This attack must beat the target's Armor defense.
            If the target is attended by a creature, the attack must also beat the attending creature's Reflex defense.

            On a hit, you choose whether the target takes damage from the weapon you hit it with.
            In addition, if the target is \glossterm{attended} and is not held in two hands or well secured (such as a ring), you can choose to knock it loose.
            If you do, it falls to the ground in the square occupied by the attending creature that is closest to you.
        \end{freeability}

        \parhead{Feint} As a standard action during the \glossterm{action phase}, you can use the \textit{feint} ability to distract an opponent.
        This allows you to strike them more accurately, though you sacrifice some power for the feint.

        \begin{freeability}{Feint}\label{Feint}
            Make a melee \glossterm{physical attack} with a weapon you wield against a creature's Reflex defense.
            During the \glossterm{delayed action phase}, you can also make a melee \glossterm{strike} with the same weapon with a \minus1d penalty to damage.

            On a hit, the target takes a \minus2 penalty to defenses against the delayed strike.
            On a miss, you take a \minus2 penalty to accuracy with the strike.
        \end{freeability}

        \parhead{Grapple} As a standard action, you can use the \textit{grapple} ability to physically grab and restrain a creature.

        \begin{freeability}{Grapple}\label{Grapple}
            Make a melee \glossterm{physical attack} with a free hand against a creature's Fortitude and Reflex defenses.

            On a hit against both defenses, you and the target are \grappled by each other.
            For details, see \pcref{Grappling}.
        \end{freeability}

        \parhead{Overrun} At the start of each phase, you can spend an \glossterm{action point} to use the \textit{overrun} ability.

        \begin{freeability}{\labeltext{Overrun}}
            You can try to move directly through creatures in your way during the current phase.
            Each creature in your way can choose to avoid you, allowing you to pass through its square unhindered.
            If a creature does not attempt to avoid you, you make an attack vs. Fortitude against it.
            Your \glossterm{accuracy} is equal to your Strength.

            On a hit, you can move through the creature's space, though you treat it as \glossterm{difficult terrain}.
            On a miss, you end your movement immediately.
        \end{freeability}

        \parhead{Recover} As a standard action, you can use the \textit{recover} ability to regain a spent \glossterm{recovery action point}.

        \begin{freeability}{\labeltext{Recover}}
            At the end of the round, if you did not take damage this round, you may regain one spent \glossterm{recovery action point}.
            For details, see \pcref{Regaining Action Points}.
        \end{freeability}

        \parhead{Shove} As a standard action, you can use the \textit{shove} ability to physically move a creature.

        \begin{freeability}{Shove}\label{Shove}
            Make a melee \glossterm{physical attack} with a free hand against a creature's Fortitude defense.
            You use your Strength in place of your Perception to determine your \glossterm{accuracy} with this attack.
            In addition, for each size category larger or smaller than the target that you are, you gain a \plus4 bonus or penalty to \glossterm{accuracy}.

            On a hit, you move the target up to 10 feet in a direction of your choice.
            Effects that limit movement speed, such as \glossterm{difficult terrain}, similarly limit the distance you can move the target.
            You can move the same distance that you push the target.
            You cannot normally keep moving the target if it stops being adjacent to you.
            If the target encounters a creature or solid object, you must stop moving it.
        \end{freeability}

        \parhead{Struggle} As a standard action, you can use the \textit{struggle} ability to move despite movement impediments.

        \begin{freeability}{\labeltext{Struggle}}
            Until the end of the current phase, your land speed becomes five feet, regardless of all other effects that would modify your land speed.
            In addition, you can move a distance up to your land speed.
            This does not allow you to pass obstacles unrelated to movement speed penalties, such as walls.
        \end{freeability}

        \parhead{Total Defense} As a standard action, you can use the \textit{total defense} ability to focus entirely on defending yourself.

        \begin{freeability}{\labeltext{Total Defense}}[\glossterm{Swift}]
            You gain a \plus2 bonus to your \glossterm{defenses} until the end of the round.
        \end{freeability}

        \parhead{Trip} As a standard action, you can use the \textit{trip} ability to trip a creature.

        \begin{freeability}{\labeltext{Trip}}
            Make a melee \glossterm{physical attack} with a free hand against a creature's Reflex defenses.
            For each size category smaller than the target that you are, you take a \minus4 penalty to \glossterm{accuracy}.

            On a hit, the target becomes \prone.
        \end{freeability}

% TODO: This is a bad name; organize these better
\section{Special Rules}
    \subsection{Encumbrance}\label{Encumbrance}
        Your encumbrance is a value that represents how much you are burdened by armor and weight.
        You apply your encumbrance as a penalty to all Strength and Dexterity-based checks you make.
        You can increase your encumbrance by wearing armor or by carrying an excessive weight (see \pcref{Carrying Capacity}).
        In addition, you reduce your encumbrance by amount equal to your starting Strength.

        Sleeping in armor is difficult.
        % ``have any encumbrance'' is awkward - worth introducing ``encumbered''?
        If you sleep while you have any encumbrance, you become \glossterm{fatigued} when you wake up.

    \subsection{Grappling}\label{Grappling}
        A grappled creature is physically struggling with at least one other creature.
        While grappled, you suffer certain penalties and restrictions, as described below.

        \subsubsection{Being In A Grapple}
            While grappling, you suffer certain penalties and restrictions, as described below. Other than these restrictions, you can act normally. You can also take certain actions in a grapple, as described in \pcref{Grapple Actions}
            \begin{itemize}
                \item You must use a free hand (or equivalent limbs) to grapple, preventing you from taking any actions which would require having two free hands.
                    If you cannot free a hand, you suffer a \minus10 penalty to accuracy on all \glossterm{physical attacks} until you have a free hand.
                \item You take a \minus4 penalty to accuracy with weapons that are not \glossterm{light weapons}, since they are too large and cumbersome to be used effectively in a grapple.
                \item Spellcasting is extremely difficult. You cannot cast \glossterm{spells} or perform \glossterm{rituals} with \glossterm{somatic components}.
                    Casting a spell without somatic components requires a \glossterm{concentration} check with a DR equal to 20 \add three times the spell's level.
                \item You cannot normally move from your location (but see the \textit{move grapple} ability, below).
            \end{itemize}

        \subsubsection{Grapple Actions}\label{Grapple Actions}
            While grappled, you can use four special abilities to try to affect the grapple.
            For all grapple actions, you can use your Strength in place of your Perception to determine your \glossterm{accuracy}.

            \parhead{Bind Foe} As a standard action, you can use the \textit{bind foe} ability to bind a foe you are grappling in restraints.

            \begin{freeability}{Bind Foe}
                You must have physical restraints, such as rope, in hand to use this ability (in addition to the free hand required to grapple).

                Make an attack vs. Fortitude and Reflex against a creature who is grappled by you.
                If you have the time, you can \glossterm{take 10} on this attack to secure bindings on a helpless foe, including a foe rendered helpless through this attack.

                \hit The target is bound, rendering it \glossterm{helpless} and effectively \glossterm{paralyzed}.
                % TODO: clarify how to escape/break bindings?
                The only physical actions a bound creature can take are to escape or break the bindings.
                Escaping the bindings requires a \glossterm{physical attack} or Escape Artist check which beats the attack result made to bind the creature.
                Breaking the bindings requires making a Strength check sufficient to break the item used to bind the creature.
            \end{freeability}

            \parhead{Escape Grapple} As a standard action, you can use the \textit{escape grapple} ability to try to stop being grappled.

            \begin{freeability}{Escape Grapple}
                Make an attack vs. Reflex against every creature that you are grappled by.

                % TODO: awkward wording
                \hit You are not grappled by each target, and each target is not grappled by you.
            \end{freeability}

            \parhead{Move Grapple} As a \glossterm{move action}, you can use the \textit{move grapple} ability to move yourself and all creatures you are grappling with.

            \begin{freeability}{Move Grapple}
                Make an attack vs. Fortitude against every creature grappled by you.
                If a target also uses this ability to affect you during the same phase, you compare your attack result against its attack result instead of against its Fortitude defense.

                % how to word ``hit or beat''?
                If you hit every target or beat every target's attack result, you can move yourself and all other creatures grappled by you a distance up to half your speed.
            \end{freeability}

            \parhead{Pin}\label{Pin} As a standard action, you can use the \textit{pin} ability to further restrict the actions of a creature you are grappling with.

            \begin{freeability}{Pin}
                Make an attack vs. Fortitude and Reflex against a creature who is grappled by you.
                This ability requires two free hands to use.

                \hit The target is completely immobile as long as you use two free hands to hold it still.
                The only physical action the target can take is the \textit{escape grapple} ability (see above), though it can take mental actions.
                At your discretion, you can cover the mouth of a pinned creature to prevent it from speaking.
                You may release the target as a \glossterm{free action}, ending this effect.
            \end{freeability}

        \subsubsection{Asymmetric Grappling}\label{Asymmetric Grappling}
            Normally, when you use the \textit{grapple} ability, both you and the target become grappled by each other.
            Some abilities allow you to grapple other creatures without becoming grappled yourself.
            You can release a creature that you are not grappled by as a \glossterm{free action}.
            If you do, the creatures stops being grappled by you.

    \subsection{Unarmed Combat}\label{Unarmed Combat}
        Every creature can attack with its body using an unarmed attack.
        You are not proficient with your unarmed attack, so you are usually \defenseless while unarmed.
        In addition, an unarmed attack always deals \glossterm{subdual damage}.
        You may use any appropriate part of your body to make an unarmed attack -- fists, feet, elbows, and so on.
        However, you only have one unarmed attack.
        You cannot dual-wield unarmed attacks as if you were fighting with two weapons at once unless you are \glossterm{proficient} with your unarmed attack (see \pcref{Strikes}).

        An unarmed attack is a type of natural weapon.
        Abilities that affect natural weapons can affect your unarmed attack.
        Gauntlets can also be worn to attack with your fists, but attacks with gauntlets are not considered unarmed attacks.

    \subsection{Mounted Combat}\label{Mounted Combat}
        \parhead{Horses in Combat} Warhorses and warponies can serve readily as combat steeds. Light horses, ponies, and heavy horses, however, are frightened by combat.
        At the start of each round, you must make a DR 10 Ride check to control such a horse.
        Success means you can act normally that round, directing the horse's movements as if it was trained for combat.
        Failure means that the horse acts of its own volition that round, usually fleeing in panic.

        \parhead{Space} A horse (not a pony) is a Large creature, and thus takes up a space 10 feet (2 squares) across. While mounted, you share your mount's space completely. Anyone threatening your mount can attack either you or your mount. However, your reach is still that of a creature of your normal size. Thus, a Medium paladin would threaten all squares adjacent to his Large horse with a longsword, and all squares 10 feet away from his mount with a lance.

        In the case of abnormally large mounts (two or more size categories larger than you), you may not completely share space. Such situations should be handled on a case-by-case basis, depending on the nature of the mount.

        \parhead{Combat while Mounted} With a DR 5 Ride check, you can guide your mount with your knees so as to use both hands to attack or defend yourself. This is a free action.

        If your mount is moving in the current phase, you gain a \plus1d bonus to damage with Mounted weapons (see \pcref{Mounted Weapon}).
        You also take a \minus2 penalty to accuracy with ranged strikes.
        If your mount uses the \textit{sprint} ability, this penalty increases to \minus4 (see \pcref{Sprint}).

        \parhead{Casting Spells while Mounted} You can cast a spell normally if your mount only moves during the \glossterm{movement phase}.
        If you have your mount move during the same \glossterm{phase} that you cast a spell, you have to make a \glossterm{concentration} check due to the vigorous motion (DR 5 \add double spell level) to avoid \glossterm{miscasting} the spell.
        If the mount uses the \textit{sprint} ability during that phase, you take a \minus2 penalty to the check due to the violent motion.

        \parhead{If Your Mount Falls in Battle} If your mount falls, you fall to the ground with it.

        \parhead{If You Are Dropped} If you are knocked unconscious, you fall from your mount to the ground, which may cause you to take \glossterm{falling damage}.
        If you have a military saddle, you stay on your mount instead.
        In either case, the mount acts according to its nature.
        Most mounts flee combat without a rider.

    \subsection{Drowning}\label{Drowning}
        You can hold your breath for a number of rounds equal to 5 \add your Constitution.
        After that time, you must roll 1d10.
        This attack gains a \plus5 bonus for each round you hold your breath beyond your limit.
        If the result exceeds your Fortitude defense, you take \glossterm{vital damage} equal to the difference.

    \subsection{Damage Reduction}\label{Damage Reduction}
        Some abilities can give you \glossterm{damage reduction}.
        Each \glossterm{round}, you ignore the first points of damage you would take.
        Damage reduction always specifies an amount of damage it reduces.
        Once it reduces that much damage, it stops functioning until the start of the next round.

    \subsection{Hardness}\label{Hardness}
        Almost all objects have a \glossterm{hardness} that indicates how durable they are.
        In addition, some creatures have a hardness as well.
        When a creature or object with hardness would take damage, if the hardness of the attacking object or creature is lower than the hardness of the defender, the attacking object or creature takes the damage instead.
        For example, if you try to break a stone wall with a wooden club, the club will break instead of the wall.
        % TODO: define hardness for creatures and their natural weapons; natural weapons should generally have higher hardness than creatures to avoid hardness reflection being common

    \subsection{Underwater Combat}\label{Underwater Combat}
        Land-based creatures have considerable difficulty when fighting in water.
        You take a \minus4 penalty to \glossterm{accuracy} with \glossterm{physical attacks}, Strength and Dexterity-based \glossterm{checks}, and Armor and Reflex defenses.
        In addition, all physical ranged attacks are considered to have a \glossterm{range increment} of 5 feet, regardless of the weapon's normal range increment or any other modifiers.

\section{Daily Resources}

    \subsection{Resting}\label{Resting}
        When you have a moment to relax, you can rest to regain some of your expended abilities.
        There are two main types of rests: a \glossterm{short rest} and a \glossterm{long rest}.

        \subsubsection{Short Rest}\label{Short Rest}
            Resting for five minutes is considered a \glossterm{short rest}.
            When you take a short rest, you gain the following benefits.
            \begin{itemize}
                \item You heal all points of \glossterm{fatigue}.
                \item You regain all spent \glossterm{recovery action points}.
                    Being \glossterm{attuned} to effects at the start of your rest can prevent you from regaining spent \glossterm{recovery action points} (see \pcref{Attunement}).
                \item You remove all \glossterm{conditions} affecting you (unless they cannot be removed normally).
                \item Some other abilities have specific effects that last until you take a short rest.
                    For example, a barbarian cannot use their \textit{rage} ability again after raging until after they take a short rest (see Rage, page \pref{Bbn:Rage}).
            \end{itemize}

        \subsubsection{Long Rest}\label{Long Rest}
            Resting for eight hours is considered a \glossterm{long rest}.
            When you take a long rest, you gain the following benefits.
            \begin{itemize}
                \item You heal one \glossterm{vital wound}.
                    The Heal skill can increase this healing (see \pcref{Accelerate Recovery}).
                \item You regain all spent \glossterm{action points}, including \glossterm{reserve action points}.
                    Being \glossterm{attuned} to effects at the start of your rest can prevent you from regaining spent \glossterm{recovery action points} (see \pcref{Attunement}).
                \item You regain all spent \glossterm{legend points}.
                \item You regain all spent \glossterm{item slots}.
                \item Some other abilities have specific effects that last until you take a long rest.
            \end{itemize}

\section{Legend Points}\label{Legend Points}
    At 3rd level, you gain a limited ability to change your fate with \glossterm{legend points}.
    When you unexpectedly fail at a critical moment, or an enemy delivers a particularly devastating blow, you can spend a \glossterm{legend point} to shift the outcome of the situation in your favor.

    \subsubsection{Using Legend Points}
        You can use a legend point to automatically roll a 10 on any \glossterm{attack} or \glossterm{check} you make.
        On attack rolls, this allows you to roll again, just as if you had rolled a 10 normally (see \pcref{Exploding Attacks}).
        Alternately, you can use a legend point to make any \glossterm{attack} or \glossterm{check} against you roll a 1.

        Using a legend point is not an action, and can be done at any time.
        You can decide to use a legend point after you learn whether the original roll succeeded or failed.
        You can even use a legend point after you learn what the effects of a successful attack or check would be, if that is information you could normally learn if it succeeded.
        However, you must use the legend point before the phase is over.

        If you use a legend point defensively against an attack affects multiple targets, your legend point only affects the roll against you, and does not change the attack's effects against the other targets.

    \subsubsection{Gaining Legend Points}
        At the start of each session of gameplay, you gain one legend point if you would otherwise have no legend points.
        In addition, on rare occasions you can gain legend points by performing extraordinary deeds worthy of legends.
        In unusually long gameplay sessions, you may recover a legend point at an appropriate resting point if you have no legend points.

    \subsubsection{Legendary Foes}
        A small number of legendary enemies you encounter may have their own legend points.

\section{Character Advancement}\label{Character Advancement}

    As you accomplish challenges and defeats foes, you gain experience.
    If you have enough experience, you gain a level.
    You gain some abilities at specific levels, as described in \trefnp{Character Advancement}.

    A character that increases in level gains additional benefits.
    \begin{itemize}
        \item At 1st, 3rd, 6th, and 10th level, you gain an \glossterm{item slot}.
        \item At 3rd level, you gain access to \glossterm{legend points}.
        \item At 4th level, and every 4 levels thereafter, you gain a \glossterm{legacy item} upgrade (see \pcref{Legacy Items}).
        \item At 7th level, and every 6 levels thereafter, you gain an additional \glossterm{recovery action point}.
    \end{itemize}

    \begin{dtable}
        \lcaption{Character Advancement}
        \begin{dtabularx}{\columnwidth}{*{2}{>{\lcol}l} *{5}{>{\lcol}X}}
            \tb{Level} & \tb{XP}  & \tb{Max Rank}\fn{1} & \tb{Item Slots} & \tb{Legacy Item}\fn{2} & \tb{Recovery AP} \tableheaderrule
            1st          & 0      & 1      & 1      & \tdash & 3
            \\ 2nd       & 20     & 2      & \tdash & \tdash & \tdash
            \\ 2rd\fn{4} & 50     & \tdash & 2      & \tdash & \tdash
            \\ 4th       & 90     & \tdash & \tdash & 1      & \tdash
            \\ 5th       & 150    & 3      & \tdash & \tdash & \tdash
            \\ 6th       & 230    & \tdash & 3      & \tdash & \tdash
            \\ 7th       & 350    & \tdash & \tdash & \tdash & 4
            \\ 8th       & 510    & 4      & \tdash & 2      & \tdash
            \\ 9th       & 750    & \tdash & \tdash & \tdash & \tdash
            \\ 10th      & 1,050  & \tdash & 4      & \tdash & \tdash
            \\ 11th      & 1,550  & 5      & \tdash & \tdash & \tdash
            \\ 12th      & 2,200  & \tdash & \tdash & 3      & \tdash
            \\ 13th      & 3,150  & \tdash & \tdash & \tdash & 5
            \\ 14th      & 4,450  & 6      & \tdash & \tdash & \tdash
            \\ 15th      & 6,350  & \tdash & \tdash & \tdash & \tdash
            \\ 16th      & 8,900  & \tdash & \tdash & 4      & \tdash
            \\ 17th      & 13,000 & 7      & \tdash & \tdash & \tdash
            \\ 18th      & 18,000 & \tdash & \tdash & \tdash & \tdash
            \\ 19th      & 25,500 & \tdash & \tdash & \tdash & 6
            \\ 20th      & 36,000 & 8      & \tdash & 5      & \tdash
        \end{dtabularx}
        1. This is your maximum rank in your class archetypes (see \pcref{Archetypes}). \\
        2. This is the number of abilities you gain with your \glossterm{legacy item} (see \pcref{Legacy Items}). \\
        3. You gain access to \glossterm{legend points} at this level.
    \end{dtable}

    \subsection{Leveling Up}
        When you gain a level, the following things happen:
        \begin{itemize}
            \item Your \glossterm{attributes} increase (see \pcref{Increasing Attributes})
            \item Your skill modifiers usually increase by 1 (see \pcref{Skill Modifier})
            \item Your \glossterm{mundane power} and \glossterm{magical power} each increase by 1 (see \pcref{Power}).
            \item Your \glossterm{fatigue threshold} and \glossterm{wound threshold} each increase (see \pcref{Taking Damage}).
            \item Your \glossterm{accuracy} increases by 1 (see \pcref{Accuracy})
            \item All of your \glossterm{defenses} increase by 1 (see \pcref{Defenses})
            \item You gain an additional \glossterm{archetype rank} (see \pcref{Archetypes})
        \end{itemize}
