\chapter{Core Mechanics}

This chapter describes the core mechanics of Rise.
It defines how attributes work and explains how to make physical attacks in combat.

\section{Attacks and Checks}\label{Attacks and Checks}
    Your character can take many actions without needing to roll a die at all.
    However, eventually your character will need to do something where there is a dramatically significant chance of failure.
    In that case, you will need to roll a die to see if your character succeeds or fails.
    Almost all rolls you will need to make can be described as an \glossterm{attack roll} or a \glossterm{check}.

    \subsection{Attack Rolls}
        Attack rolls are required to make \glossterm{attacks}.
        Anything that affects another creature in a potentially harmful way, such as striking a creature with a sword, is an attack.
        Some abilities are always considered attacks, even if you use them in a way that you believe cannot be harmful.

        To make an attack roll, roll 1d10 and add your \glossterm{accuracy} with the attack.
        The sum of your die roll and your accuracy is called your \glossterm{attack result}.
        You compare your attack result to a \glossterm{defense} that your \glossterm{target} has.
        All attacks specify which defense they are compared to.
        If your result is at least equal to your target's defense, the attack succeeds.
        This almost always means the target suffers some harmful effect, such as taking \glossterm{damage}.
        Otherwise, the attack fails.

        \subsubsection{Exploding Attacks}\label{Exploding Attacks}
            When you make an attack roll, if you roll a 10 on the d10, the die ``explodes''.
            You roll again and add the second result to the original 10 before applying your \glossterm{accuracy}.
            If you roll a 10 on the extra roll, you keep rolling until you stop rolling a 10 and add all of the rolls together.

        \subsubsection{Critical Hit}\label{Critical Hit}
            If your attack result is at least 10 higher than your target's defense, your attack is a \glossterm{critical hit}.
            Many attacks have a special effect on a critical hit.
            For example, \glossterm{strikes} deal double damage when you get a critical hit.

        \subsubsection{Critical Miss}\label{Critical Miss}
            if your attack result is at least 10 lower than your target's defense, your attack is a \glossterm{critical miss}.
            Some attacks have a special effect on a critical miss, which is usually bad for the attacker.
            % TODO: Is this ever true?
            % Some attacks also have special effects on a non-critical miss, where you fail to beat your target's defense by 9 or less.

    \subsection{Checks}\label{Checks}
        Checks are required to perform actions that have a chance of failure that are not attacks.
        For example, climbing a wall or remembering an obscure piece of trivia may require a check.

        To make a check, roll 1d10 and add your \glossterm{check modifier} with the check.
        You compare the die result, including your check modifier, to a \glossterm{difficulty rating} (DR) that represents the difficulty of the task.
        The more difficult the task, the higher the DR will be.
        If your result is at least equal to the DR, the check succeeds.
        This usually means you accomplish a task successfully.
        Normal Difficulty Ratings are described in \tref{Difficulty Ratings}.

        \begin{dtable}
            \lcaption{Difficulty Ratings}
            \begin{dtabularx}{\columnwidth}{p{8em} X}
                \tb{Difficulty (DR)} & \tb{Example (Skill Used)} \\
                \bottomrule
                Trivial (0)      & Hear a coversation from 10 feet away (Awareness)                          \\
                Average (5)      & Tie or untie a typical knot (Devices)                                     \\
                Tough (10)       & Swim in rough water (Swim)                                                \\
                Challenging (15) & Balance on a one-inch wide wood beam (Acrobatics)                         \\
                Heroic (20)      & Open a high quality lock (Devices)                                        \\
                Legendary (25)   & Leap across a 30-foot chasm with a running start (Jump)                   \\
                Epic (30)        & Convince a wise mayor her husband is secretly a werewolf (Persuasion)     \\
                Godlike (40)     & Track three orcs across firm ground after 24 hours of rainfall (Survival) \\
            \end{dtabularx}
        \end{dtable}

        \subsubsection{Critical Success}
            If your check result is at least 10 higher than the DR, your check is a \glossterm{critical success}.
            Many checks have a special effect on a critical success.
            For example, a critical success while climbing means you move twice as quickly (see \pcref{Climb}).

        \subsubsection{Critical Failure}
            If your check result is at least 10 lower than the DR, your check is a \glossterm{critical failure}.
            Some checks have a special effect on a critical failure, which is usually bad for the character making the check.
            For example, a critical failure while climbing means you fall (see \pcref{Climb}).

\section{Combat Overview}
    The world of Rise can be a harsh one, and not all disagreements can be resolved peacefully.
    At some point, you will be forced to enter combat.
    This section explains how combat works, including how to take actions during a round and how attacks and defenses are calculated.

    \subsection{Combat Time}
        Combat takes place in a series of \glossterm{rounds}, which represent about six seconds of time.
        Each round of a combat is divided into three phases: a \glossterm{movement phase}, an \glossterm{action phase}, and sometimes a \glossterm{delayed action phase}.
        After both phases are complete, the round ends and the next round begins.

        \subsubsection{The Movement Phase}\label{The Movement Phase}
            Almost all creatures have a \glossterm{speed} measured in feet.
            For example, a typical human has a speed of thirty feet.
            During the movement phase, you can move a maximum distance equal to your speed.
            You can sprint to temporarily increase your speed (see \pcref{Sprint}).
            Some creatures have special forms of movement, such as the ability to fly.
            A creature has a listed speed for any special form of movement it has.

            At the start of the movement phase, all creatures designate a location they attempt to move to, or some other type of movement they attempt to make (see \pcref{Movement and Positioning}).
            Once everyone has chosen their actions, those actions are \glossterm{initiated}, and begin resolving.
            Normally, you will simply arrive in the destination you are trying to reach.
            For details about resolving conflicting movements, see \pcref{Resolving Actions}.

            In addition to this movement, you can take other minor actions that require motion, such as drawing a weapon.
            There is no specific limit on the number of minor actions you can perform in this way, but you must be able to perform all of them simultaneously.
            For example, you could move a distance equal to your speed and draw a sword.
            However, you could not put on a cloak and equip a shield in the same movement phase, because you cannot take those actions at the same time.
            When all creatures have resolved their movement, the action phase begins.

        \subsubsection{The Action Phase}\label{The Action Phase}
            During the action phase, each creature can take actions.
            Usually, you will simply take a \glossterm{standard action}.

        \subsubsection{The Delayed Action Phase}\label{The Delayed Action Phase}
            Some abilities cause creatures to take \glossterm{delayed actions}, which are initiated and resolved after other actions in the round.
            For example, spellcasting is a \glossterm{dual action}, which has a component in both the action phase and the delayed action phase.
            Any creature can also take the Delay action (see \pcref{Delay}), which causes them to act during the delayed action phase instead of the action phase.
            If there are no delayed actions, this phase is skipped.

    \subsection{Movement and Positioning}\label{Movement and Positioning}

        \subsubsection{Measuring Movement}

            For simplicity, all movement is measured in five-foot increments.
            While it is possible to be more precise than that, it's generally not worth the complexity.

            \parhead{Diagonals} When measuring distance, the first diagonal counts as five feet of movement, and the second counds as ten feet of movement.
            The third costs five feet, the fourth costs ten feet, and so on.
            You can move diagonally past corners and enemies.

        \subsubsection{Taking up Space}
            A typical human takes up a 5-ft.\ by 5-ft.\ space in combat.
            For convenience, this is often called a \glossterm{square}.
            Differently sized creatures can take up more or less space, as indicated on \tref{Size in Combat}.
            Normally, other creatures can't be in any squares you occupy.

            Sometimes, movement and distance are represented in squares.
            A 30-ft.\ movement is the same thing as moving six squares.

        \subsubsection{Moving Near Foes}\label{Moving Near Foes}
            All squares threatened by any foes cost double the normal movement cost to move out of.

        \subsubsection{Movement Impediments}

            \parhead{Difficult Terrain}\label{Difficult Terrain}
            Some terrain is hard to move through, like thick bushes or a swamp. If a square is difficult terrain, it doubles the movement cost required to move out of the square. That generally means it takes ten feet of movement, or fifteen feet if you are moving diagonally. If a square is considered difficult terrain for multiple reasons, the cost increases stack. For example, a square in a swamp that also has thick bushes blocking your passage would take twenty feet of movement, or thirty feet to move diagonally.

            \parhead{Obstacles}
            An obstacle is anything that gets in your way. Enemies and large solid objects like walls completely block your movement. If you can get past an obstacle, like a low wall, that square is treated as difficult terrain. Some obstacles require a \glossterm{check} to bypass, such as an Acrobatics check (see \pcref{Acrobatics}).

            \parhead{Squeezing}\label{Squeezing}
            In some cases, you may have to squeeze into or through an area that isn't as wide as the space you take up. You can squeeze through or into a space that is at least half as wide as your normal space. While squeezing, you move half as fast, and you take a \minus2 penalty to physical accuracy, physical checks, and physical defenses. You can squeeze into tighter spaces with the Escape Artist skill.

            Creatures that take up multiple squares take up half their normal number of squares while squeezing. For example, a Large creature who normally takes up four spaces takes up two spaces while squeezing.

            \parhead{Accidentally Squeezing} Sometimes a character ends its movement while moving through a space where it's not normally allowed to stop. When that happens, the character is squeezing in the space until it can move. If squeezing is impossible, the creature immediately moves to the closest available space. Try not to do this.

        \subsubsection{Special Movement Modes}\label{Special Movement Modes}
            Some spells and abilities grant creatures the ability to move in unusual ways. These forms of movement are described here.

            \parhead{Burrowing}
            A creature with a burrow speed can move through the ground at the indicated speed in any direction, even vertically. Unless otherwise noted, the creature can only burrow through dirt and loose earth, not rock or harder substances. It does not leave behind a usable tunnel for other creatures.

            \parhead{Climbing}
            A creature with a \glossterm{climb speed} can move a distance equal to its climb speed with a successful Climb check (see \pcref{Climb}).
            In addition, it gains a \plus10 bonus to any Climb checks it makes.

            \parhead{Flying}\label{Flying}
            A creature with a fly speed can fly through the air at the indicated speed. It must be \glossterm{unencumbered} (see \pcref{Encumbrance}). If a creature with a fly speed is encumbered, it is treated as having a glide speed instead (see \pcref{Gliding}), which it can use to glide even though it is encumbered.

            Each creature with a fly speed also has a maneuverability: good, average, poor, or special. Normally, a flying creature must move forward by at least half its fly speed each round. If it does not, it falls. Turning by 90 degrees costs 5 feet of movement, and it can't turn in the same place by more than 90 degrees. It can move up by only one square vertically per square traveled horizontally, but it can fly directly down if it chooses. The creature can fly up at half speed, but can fly down twice as fast.

            \parhead{Maneuverability}\label{Maneuverability} Some creatures have fly speeds with special maneuverability rules.

            \subparhead{Good Maneuverability} If a creature has good maneuverability while flying, it gains three benefits while flying.
            First, it not need to move forward to maintain its flight, allowing it to hover.
            Second, it can turn in place without spending movement.
            Third, it can move up at the same speed as it moves horizontally.

            \parhead{Poor Maneuverability} If a creature has poor maneuverability while flying, it must spend five feet of movement to turn by 45 degrees, and it can't turn in the same place by more than 45 degrees. In addition, it can only descend by up to one square vertically per square traveled horizontally without falling.

            \parhead{Falling} If a flying creature loses control, usually by failing to maintain its minimum forward speed, or loses the ability to fly, it falls just like any other creature would in midair. As long as it still has the ability to fly, it can regain control of its fall as a standard action, causing it to resume flying normally.

            \parhead{Gliding}\label{Gliding}
            A creature with a glide speed can glide through the air at the indicated speed. It must be \glossterm{unencumbered} (see \pcref{Encumbrance}).

            While in the air, a creature with a glide speed can control its fall as a move action. This allows it to move up to its speed horizontally in a direction of its choice while moving only five feet down. If it desires, it can move half as far horizontally and fall down twice as fast. It takes no falling damage if it touches the ground while gliding.

    \subsection{Action Types}

        \parhead{Standard Action} Most common activities require a single standard action, such as attacking with a weapon, casting a spell, and using most special abilities.
        You can only take one standard action per action phase.

        \parhead{Minor and Immediate Actions}\label{Minor and Immediate Actions} Each round, you can take a single \glossterm{minor action} or \glossterm{immediate action}, but not both.
        Minor actions can be taken in either the \glossterm{action phase} or the \glossterm{delayed action phase}.
        They are declared and resolved simultaneously with any other actions you take during that phase.
        This can allow you to perform multiple tasks during the same phase.

        Immediate actions do not need to be declared ahead of time, and can be taken in any phase.
        Instead, abilities that can be used as immediate actions specify triggering conditions that allow the action to be taken.
        Immediate actions are resolved immediately, before the triggering action resolves.
        If multiple minor actions or immediate actions are taken simultaneously, they are resolved using the normal rules for resolving simultaneous actions.

        \subparhead{Minor Actions and Standard Action} Any action that normally requires a minor action to complete can also be done as a \glossterm{standard action}.

        % TODO: should free actions resolve early?
        \parhead{Free Actions}\label{Free Actions} Each round, you can take any number of free actions.
        Free actions can be taken in any phase.
        Like minor actions, free actions are declared and resolved simultaneously with other actions you intend to take during that phase.

        \parhead{Full-Round Actions} A full-round action requires your full attention.
        Unless otherwise specified, you perform any movement required for the action during the movement phase, and the rest of the action during the action phase.

        \parhead{Partial Actions} If you are restricted to only taking a move or standard action, but not both, you can spend a standard action to perform a partial version of a full-round action. For most actions, you spend the first round starting the action, and use a second standard action to complete it. Some full-round actions have specific partial versions described below which you take instead. You can only take these partial actions when you cannot take full-round actions.

    \subsection{Resolving Actions}\label{Resolving Actions}

        Within each phase, actions of all creatures are simultaneously resolved in the following order.
        Italicized steps are less common, and can usually be skipped.
        Allies with the ability to communicate can freely coordinate their actions with each other, within reasonable limits.

        \begin{enumerate*}
            \item Choose actions.
            \item All actions are \glossterm{initiated}.
            \item Determine targets affected by actions.
            \item Check action success.
                Example: Making attack rolls.
            \item Determine action results.
                Example: Making damage rolls.
            \item Apply action results.
                Examples: Reducing hit points, moving creature locations, and applying penalties.
                Effects that trigger when damage is dealt, such as Concentration checks (see \pcref{Concentration}), are resolved now.
        \end{enumerate*}

        In the vast majority of cases, there is no need to go through this order explicitly.
        Combats will run much faster if attack and damage rolls are generally made and announced at the same time as the actions are chosen, even before all characters have explicitly stated their actions.
        The order of resolution matters when creatures take actions that directly conflict with each other.

        \subsubsection{Conflicting Actions}\label{Conflicting Actions}

            Sometimes, actions that occur within the same resolution step can conflict with each other.
            There are two main methods for resolving these conflicts.

            \parhead{Mutually Exclusive Actions} Sometimes, actions that should take place at the same time directly conflict with each other.
            This most commonly happens when two creatures move to the same place.
            In this case, each involved character rolls \glossterm{initiative} (see \pcref{Initiative}).
            The creature with the highest initiative result succeeds.
            All other creatures come as close as possible to completing their intended action.

    \subsection{Special Actions}

\section{Universal Abilities}
    All creatures can use the following abilities.

    \subsection{Strikes}\label{Strikes}
        A \glossterm{strike} is the most common type of attack.
        There are two kinds of strikes: a single strike, with a single weapon, and a dual strike, with two weapons.
        Many abilities allow you to make one or more strikes.
        Whenever you make a strike, you can choose which kind of strike to make.
        All strikes are \glossterm{mundane} abilities.
        Your \glossterm{accuracy} with a strike is equal to your accuracy with \glossterm{physical attacks} (see \pcref{Physical Accuracy}).
        Your \glossterm{damage} with a strike has special rules (see \pcref{Strike Damage}).

        \parhead{Single Strike}
        \begin{ability}
            \begin{spelltargetinginfo}
                \spellspecial Choose a weapon you wield, or your \glossterm{unarmed attack}.
                \spellquicktargeting{One creature or object}{As weapon}
            \end{spelltargetinginfo}
            \begin{spelleffects}
                \begin{spellattack}{Physical vs. Armor}
                    \spellsuccess The target takes \glossterm{strike damage} from the chosen weapon (see \pcref{Strike Damage}).
                    \spellcritical As above, but the target takes double damage.
                \end{spellattack}
            \end{spelleffects}
        \end{ability}

        \parhead{Dual Strike}\label{Dual Strike}
        You can only make a \glossterm{dual strike} if you are wielding two weapons at once.
        \begin{ability}
            \begin{spelltargetinginfo}
                \spellspecial Choose two weapons you wield.
                % clarify - lower range of both weapons?
                \spellquicktargeting{One creature or object}{As weapons}
            \end{spelltargetinginfo}
            \begin{spelleffects}
                \begin{spellattack}{Physical vs. Armor}
                    \spellspecial You roll this attack roll twice, once for each weapon.
                    Roll any miss chances or failure chances independently for each weapon.
                    You take a \minus1 penalty to accuracy on both attack rolls for each non-light weapon you choose when you use this ability.
                    \spellsuccess The target takes \glossterm{strike damage} from one weapon you hit with (see \pcref{Strike Damage}).
                    If you hit with both weapons, you choose which weapon to deal damage from.
                    \spellcritical The target takes \glossterm{strike damage} from both weapons.
                    % is this necessary or just wordy?
                    There is no advantage to getting a \glossterm{critical hit} with both weapons, rather than just one.
                \end{spellattack}
            \end{spelleffects}
        \end{ability}

    \subsection{Combat Maneuvers}\label{Combat Maneuvers}
        A combat maneuver is an attempt to physically hinder your foe with your body, such as by shoving or tripping it.
        Unless otherwise noted, combat maneuvers do not deal damage.

        \begin{dtable}
            \lcaption{Combat Maneuvers}
            \begin{dtabularx}{\columnwidth}{l l X}
                \tb{Maneuver}  & \tb{Defense} & \tb{Brief Description} \\
                \bottomrule
                Dirty Trick & Any                  & Impose penalty on a foe         \\
                Disarm      & Reflex               & Force foe to drop item          \\
                Feint       & Reflex               & Leave foe vulnerable to attacks \\
                Grapple     & Fortitude and Reflex & Wrestle with a foe              \\
                Shove       & Fortitude            & Move a foe                      \\
                Trip        & Reflex               & Trip a foe                      \\
            \end{dtabularx}
        \end{dtable}

        \subsubsection{Maneuver Descriptions}

            \parhead{Dirty Trick}\label{Dirty Trick} You strike your foe in a sensitive spot, pull its pants down, or creatively use your environment to attack.
            You must use a free hand to use this ability.
            \begin{ability}
                \begin{spelltargetinginfo}
                    \spellquicktargeting{One creature}{Adjacent}
                \end{spelltargetinginfo}
                \begin{spelleffects}
                    \begin{spellattack}{Physical vs. Special}
                        \spellspecial Depending on the nature of your dirty trick, this attack can target Fortitude or Reflex defense.
                        \spellsuccess The target suffers a \minus2 penalty to one of the following statistics:
                        \glossterm{accuracy} with \glossterm{physical attacks}, \glossterm{physical checks}, Armor defense, Fortitude defense, Reflex defense, or Mental defense.

                        \spellcritical As above, except that the penalty is increased to \minus5.
                    \end{spellattack}
                    \spelldur Condition
                \end{spelleffects}
            \end{ability}

            \parhead{Disarm}\label{Disarm} You can strike an item your foe is wearing or holding to knock it out of their hands or damage it.
            \begin{ability}
                \begin{spelltargetinginfo}
                    \spellspecial Choose a weapon you wield, or your \glossterm{unarmed attack}.
                    \spellquicktargeting{One object}{As weapon}
                \end{spelltargetinginfo}
                \begin{spelleffects}
                    \begin{spellattack}{Physical vs. Special}
                        \spellspecial If the target is \glossterm{unattended}, this attack is made against its Armor defense.
                        If the target is \glossterm{attended}, this attack is also made against the attending creature's Reflex defense.
                        \spellsuccess You choose whether or not the target takes \glossterm{strike damage} from the chosen weapon.
                        In addition, if the target is \glossterm{attended} and not held in two hands or extraordinarily well secured (such as a ring), it falls to the ground in the closest square to you occupied by the attending creature.
                        \spellcritical As above, but you deal double damage if you choose to deal damage.
                        In addition, the target falls to the ground even if it is held in two hands.
                    \end{spellattack}
                \end{spelleffects}
            \end{ability}

            \parhead{Feint}\label{Feint} You can make a fake attack to leave your foe off-balance.
            \begin{ability}
                \begin{spelltargetinginfo}
                    \spellspecial Choose a weapon you wield, or your \glossterm{unarmed attack}.
                    \spellquicktargeting{One creature}{As weapon}
                \end{spelltargetinginfo}
                \begin{spelleffects}
                    \spelleffect During the \glossterm{delayed action phase}, you make a \glossterm{strike} with the weapon.
                    You can make the strike against any creature within range of your weapon, including the creature this ability targets.
                    \begin{spellattack}{Physical vs. Reflex}
                        \spellsuccess The target takes a \minus2 penalty to Armor defense against the strike.
                        \spellcritical As above, but the penalty is increased to \minus5.
                        \spellfailure You take a \minus2 penalty to accuracy with the strike.
                    \end{spellattack}
                \end{spelleffects}
            \end{ability}

            \parhead{Grapple}\label{Grapple} You physically grab and restrain your foe.
            You must use a free hand to use this ability.
            \begin{ability}
                \begin{spelltargetinginfo}
                    \spellquicktargeting{One creature}{Adjacent}
                \end{spelltargetinginfo}
                \begin{spelleffects}
                    \begin{spellattack}{Strength vs. Fortitude and Reflex}
                        \spellsuccess You and the target are \grappled.
                        For details, see \pcref{Grappling}.
                    \end{spellattack}
                \end{spelleffects}
            \end{ability}

            \parhead{Shove}\label{Shove} You shove your foe where you want it to go.
            You must use a free hand to use this ability.
            \begin{ability}
                \begin{spelltargetinginfo}
                    \spellquicktargeting{One creature}{Adjacent}
                \end{spelltargetinginfo}
                \begin{spelleffects}
                    \begin{spellattack}{Strength vs. Fortitude}
                        \spellsuccess You move the target up to 10 feet in a direction of your choice.
                        Effects that limit movement speed, such as \glossterm{difficult terrain}, similarly limit the distance you can move the target.
                        You can move the same distance that you push the target, up to a maximum distance equal to half your land speed.

                        You cannot normally keep moving the target if it stops being adjacent to you.
                        If the target encounters a creature or solid object, you must stop moving it.
                        \spellcritical As above, except that you can move the target up to five feet per point of Strength you have.
                        This does not change the maximum distance you can move.
                        In addition, you can move the target through creatures, but not solid objects.
                        Each five feet of movement through a creature costs 10 feet of movement.
                        You cannot leave the target inside another creature's space.
                        If that occurs inadvertently, you and the target return to the last open space you both occupied.
                    \end{spellattack}
                \end{spelleffects}
            \end{ability}

            \parhead{Trip}\label{Trip} You try to trip your foe.
            You must use a free hand to use this ability.
            \begin{ability}
                \begin{spelltargetinginfo}
                    \spellquicktargeting{One creature}{Adjacent}
                \end{spelltargetinginfo}
                \begin{spelleffects}
                    \begin{spellattack}{Physical vs. Reflex}
                        \spellsuccess The target becomes \prone.
                        \spellcritical As above. In addition, during the \glossterm{delayed action phase}, you can make a \glossterm{strike} with a weapon you wield.
                        You take a \minus2d penalty to damage with the strike.
                    \end{spellattack}
                \end{spelleffects}
            \end{ability}

    \subsection{Special Combat Actions}

        \parhead{Charge}\label{Charge} As a \glossterm{standard action}, you can move up to your speed in the action phase.
        If you use this ability during the \glossterm{action phase}, during the \glossterm{delayed action phase}, you can make a \glossterm{strike} from your new location.
        If you charge, you take a \minus5 penalty to \glossterm{physical defenses} until the end of the round, including during the phase when you first use this ability.

        \par Your movement while charging has special limitations.
        First, you must move entirely in a single straight line.
        Your path must be clear of all obstacles and movement impediments, including \glossterm{difficult terrain}.
        If your movement becomes impeded while charging, you stop moving immediately, though you still suffer the defense penalty for charging and can still attack from your new location.

        \parhead{Delay}\label{Delay}
        During the action phase, you can delay your action instead of acting immediately.
        If you delay, you do nothing until after the actions of all other creatures have been resolved.
        At that point, you can declare and resolve your actions for the action phase, as described in \pcref{Resolving Actions}.
        You cannot delay during the movement phase.

        Some abilities cause actions to be delayed, such as charging (see \pcref{Charge}).
        If you use an ability that causes actions to be delayed after you have already delayed, any actions which would be delayed are ignored.
        For example, if you charge after delaying, you would not be able to attack after the charge, making it generally pointless.

        \parhead{Desperate Recovery}\label{Desperate Recovery}
        As a standard action, you can use this ability.
        As long as you are conscious, no effect can prevent you from taking this action, even effects that prohibit all other actions.
        \begin{ability}
            \begin{spelleffects}
                \spelleffect At the end of the round, if you did not take damage this round, you may remove one \glossterm{condition} affecting you.
            \end{spelleffects}
        \end{ability}

        \parhead{Recover}\label{Recover}
        As a standard action, you can spend an \glossterm{action point} to use this ability.
        As long as you are conscious, no effect can prevent you from taking this action, even effects that prohibit all other actions.
        \begin{ability}
            \begin{spelleffects}
                \spelleffect You heal hit points equal to your \glossterm{standard damage} \minus1d.
                In addition, you may remove one \glossterm{condition} affecting you.
            \end{spelleffects}
        \end{ability}

        \parhead{Total Defense}\label{Total Defense}
        As a standard action, you can focus entirely on defense.
        If you do, you gain a \plus2 bonus to your defenses until the end of the round, including in the phase that you use this ability.

    \subsection{Special Movements}

        \parhead{Struggle} As a standard action, you can move five feet regardless of movement penalties.
        This does not allow you to pass obstacles unrelated to movement speed penalties, such as walls.
        You can only use this action with a land speed, and not with any other type of speed (see \pcref{Special Movement Modes}).

        \parhead{Overrun}\label{Overrun} As part of movement, you can try to move directly through creatures in your way.
        Each creature in your way can choose to avoid you, allowing you to pass through its square unhindered.
        If a creature does not attempt to avoid you, you make a Strength vs. Fortitude attack against it.
        Success means you can move through the creature's space, though you treat it as \glossterm{difficult terrain}.
        Critical success means the creature is knocked prone, and you do not treat its space as difficult terrain.
        Failure means you end your movement immediately and fall \prone.

    \subsection{Reactive Movements}

        It is possible to declare movement that reacts automatically to the movement of an opponent.
        For example, you can try to follow a creature wherever it goes that round.
        If you declare a reactive movement at the start of the movement phase, you automatically move accordingly.
        In all cases, if you run out of movement speed before accomplishing your intended task, you simply stop where you ran out of movement.
        The most common types of reactive movements are the \textit{block}, \textit{follow}, and \textit{withdraw} abilities, but you can come up with other reactive movements.
        The only requirement is that a reactive movement must have a simple criteria for determining how you move.

        \parhead{Block}\label{Block} You can designate a target creature or object to block, and the area you want to block it from entering.
        When you do, you automatically move to intercept the target as it approaches the blocked area.
        Usually, blocking a target requires an opposed \glossterm{initiative} check against the target.
        Success means you successfully keep ahead of the target as it moves, preventing it from entering the area (unless it can move through you).
        Failure means the target moves around you (if there is room) to enter the area.

        Multiple creatures can coordinate to block a single creature.
        The blocked creature must beat the initiative of all blocking creatures to enter the blocked area.

        \parhead{Follow}\label{Follow} You can designate a target creature or object to follow, and the maximum distance you want to follow at. When you do, you automatically move such that your distance to the target is no greater than your desired follow distance. For the rest of the round, whenever that creature or object moves, you move with it to stay within that follow distance.

        If the target takes an action that makes it impossible to follow with movement, such as teleporting, you cannot follow it for the rest of the round.

        \parhead{Withdraw}\label{Withdraw} Withdrawing functions the same way as following, except that you specify a minimum distance between you and the target instead of a maximum distance.
        In addition, you can specify multiple targets and try to keep away from all of them.

    \section{Attributes}\label{Attributes}

    Each character has six \glossterm[attribute]{attributes}: Strength (Str), Dexterity (Dex), Constitution (Con), Intelligence (Int), Perception (Per), and Willpower (Wil).
    Each attribute represents a character's raw talent in that area.
    A 0 in an attribute represents average human capacity.
    That doesn't mean that every commoner has a 0 in every attribute; not everyone is average, after all.

    \subsection{Attribute Descriptions}

        \subsubsection{Strength (Str)}\label{Strength}
            Strength measures muscle and physical power.
            It has the following effects:
            \begin{itemize}
                \item Strength determines how much a character can carry (see \tref{Weight Limits}).
                \item Strength affects Strength-based skills: Climb, Jump, Sprint, and Swim (see \pcref{Skills}).
                \item If your Strength is negative, you take a penalty to all Strength-based skills equal to your Strength.
                \item If your Strength is negative, you take a penalty to damage with \glossterm{strikes} equal to half your Strength in \glossterm{die increments}.
            \end{itemize}

            If you have a high Strength, you can use it to determine several statistics:
            \begin{itemize}
                \item Your \glossterm{strike damage} (see \pcref{Strike Damage}).
                \item Your Fortitude defense (see \pcref{Defenses}).
            \end{itemize}

        \subsubsection{Dexterity (Dex)}\label{Dexterity}
            Dexterity measures hand-eye coordination, agility, and reflexes.
            It has the following effects:
            \begin{itemize}
                \item Dexterity affects Dexterity-based skills: Acrobatics, Escape Artist, Ride, Sleight of Hand, and Stealth (see \pcref{Skills}).
                \item You gain a bonus (or penalty) to your Reflex defense equal to your starting Dexterity.
                \item If your Dexterity is negative, you take a penalty to all Dexterity-based skills equal to your Dexterity.
            \end{itemize}

            If you have a high Dexterity, you can use it to determine several statistics:
            \begin{itemize}
                \item Your \glossterm{accuracy} with \glossterm{physical attacks} using light melee and thrown weapons (see \pcref{Physical Accuracy}).
                \item Your Armor and Reflex defenses (see \pcref{Defenses}).
            \end{itemize}

        \subsubsection{Constitution (Con)}\label{Constitution}
            Constitution represents your character's health and stamina.
            It has the following efects:
            \begin{itemize}
                \item You gain bonus hit points based on your starting Constitution (see \pcref{Hit Points}).
                \item You heal additional hit points when you take a \glossterm{short rest} based on your starting Constitution (see \pcref{Short Rest}).
                \item You gain a bonus (or penalty) to your Fortitude defense equal to your starting Constitution.
                \item Your ability to perform many feats of physical endurance is limited by your Constitution.
            \end{itemize}

            If you have a high Constitution, you can use it to determine your Armor and Fortitude defenses (see \pcref{Defenses}).

        \subsubsection{Intelligence (Int)}\label{Intelligence}
            Intelligence determines how well your character learns and reasons.
            It has the following effects:

            \begin{itemize}
                \item You gain bonus languages equal to your starting Intelligence (see \pcref{Languages}).
                \item You gain extra skill points equal to twice your starting Intelligence (see \pcref{Skill Points}).
                \item Your Intelligence affects Intelligence-based skills: Craft, Disguise, Heal, Knowledge, and Linguistics (see \pcref{Skills}).
                \item If your starting Intelligence is negative, you lose skill points equal to twice your starting Intelligence.
                \item If your Intelligence is negative, you take a penalty to all Intelligence-based skills equal to your Intelligence.
            \end{itemize}

            If you have a high Intelligence, you can use it to determine your Mental defense (see \pcref{Defenses}).

            \par An animal has an Intelligence score of \minus6 or lower.
            A creature of humanlike intelligence has a score of at least a \minus5 Intelligence.

        \subsubsection{Perception (Per)}\label{Perception}
            Perception describes a character's ability to observe and be aware of one's surroundings.
            It has the following effects:
            \begin{itemize}
                \item Your Perception affects Perception-based skills: Awareness, Creature Handling, Sense Motive, Spellcraft, and Survival (see \pcref{Skills}).
                \item If your Perception is negative, you take a penalty to all Perception-based skills equal to your Perception.
                \item If your Perception is negative, you take a penalty to accuracy with all attacks equal to half your Perception.
            \end{itemize}

            If you have a high Perception, you can use it to determine several statistics:
            \begin{itemize}
                \item Your \glossterm{accuracy} with \glossterm{physical attacks} (see \pcref{Physical Accuracy}).
                \item Your Reflex defense (see \pcref{Defenses}).
            \end{itemize}

        \subsubsection{Willpower (Wil)}\label{Willpower}
            Willpower measures a character's ability to endure mental hardships.
            It has the following effects:
            \begin{itemize}
                \item You gain a bonus (or penalty) to the number of \glossterm{action points} you have equal to your starting Willpower.
                \item You gain a bonus (or penalty) to your Mental defense equal to your starting Willpower.
            \end{itemize}

            If you have a high Willpower, you can use it to determine your Mental defense (see \pcref{Defenses}).

    % TODO: this section is too small
    \subsection{Using Attributes}

        \subsubsection{Choosing Attributes to Use}
            In many cases, multiple attributes can be used for the same thing.
            Whenever more than one attribute could be used, you must choose which one to use (usually, the higher attribute).

    \subsection{Determining Attributes}
        There are several options for how to determine attribute scores.

        \subsubsection{Predefined Attribute Scores}
            This is the simplest method.
            Simply take the following set of attribute scores and distribute them as you choose among your character's abilities:

            3, 2, 1, 1, 0, 0

            This set of attribute scores is called the ``elite array''.
            For more extreme characters, you may use the ``savant array'':

            4, 1, 1, 0, 0, 0.

            Finally, for more well-balanced characters, you may use the ``balanced array'':

            2, 2, 2, 1, 0, 0

            Any of these distributions can be altered by taking penalties to any attributes given as 0.
            For each penalty you take, you gain two additional \glossterm{skill points} (see \pcref{Skills}).

        \subsubsection{Increasing Attributes}
            As your level increases, your attributes increase as well, as shown on \trefnp{Increasing Attributes with Level}.

            \begin{dtable}
                \lcaption{Increasing Attributes with Level}
                \begin{dtabularx}{\columnwidth}{X X}
                    \tb{Starting Attribute} & \tb{Bonus}   \\
                    0 or lower              & 0            \\
                    1                       & \plus1 per level after 1st \\
                    2                       & \plus1 per level after 1st \\
                    3                       & \plus1 per level after 1st \\
                    4                       & \plus1 per level after 1st \\
                \end{dtabularx}
            \end{dtable}

        \subsubsection{Point Buy}
            With this method, you can fully control your character's attribute scores to match what you want your character to be.
            All your character's attribute scores start at 0.
            You get seven points to distribute among your character's attributes.
            Attributes can be bought according to the costs on \trefnp{Attribute Score Point Costs}.
            The listed cost is the total cost required to gain the listed starting attribute.
            A starting character is 1st level, which adds appropriately to the character's total attribute score.

            \parhead{Impaired Attributes}\label{Impaired Attributes}
            You can start with up to two attributes below 0.
            If you do, you compensate for your impairment in that area with additional talents in other areas.
            For each point below 0, you gain an additional \glossterm{skill point}.

            \begin{dtable}
                \lcaption{Attribute Score Point Costs}
                \begin{dtabularx}{\columnwidth}{X X l}
                    \tb{Starting Attribute Score} & \tb{Total Attribute Score} & \tb{Cumulative Point Cost} \\
                    \bottomrule
                    \minus2\fn{1} & \minus2      & 0\fn{2} \\
                    \minus1\fn{1} & \minus1      & 0\fn{3} \\
                    0             & 0            & 0       \\
                    1             & Level        & 1       \\
                    2             & 1 \add level & 2       \\
                    3             & 2 \add level & 3       \\
                    4             & 3 \add level & 5       \\
                \end{dtabularx}
                1 You cannot reduce more than two attributes below 0 in this way. \\
                2 You gain four \glossterm{skill points}. \\
                3 You gain two skill points. \\
            \end{dtable}

\section{Character Statistics}
    This section explains how to calculate commonly used statistics about your character.

    \subsection{Accuracy}\label{Accuracy}
        Your accuracy with an \glossterm{attack} is the number that you add to the \glossterm{attack roll}.
        You will have multiple different attacks you can make.
        The accuracy for an attack depends on the type of attack it is.

        \subsubsection{Physical Accuracy}\label{Physical Accuracy}
            Your accuracy with a \glossterm{physical attack}, such as a \glossterm{strike}, is normally equal to the higher of your level and your Perception.
            If you are using a \glossterm{light weapon}, you may use your Dexterity instead.
            In addition to this base number, your accuracy can include any number of bonuses and penalties from other sources.

            \parhead{Proficiency} Each creature is \glossterm{proficient} with a number of weapons.
            For details about the weapons you can be proficient with, see \pcref{Weapons}.
            Your proficiencies are primarily determined by your class, but some abilities also grant proficiency with additional weapons.
            If you make a strike with a weapon you are not proficient with, you take a \minus2 penalty to accuracy.

    \subsection{Damage}\label{Damage}
        Some attacks deal damage when they hit.
        Damage does not represent serious physical injury to your body.
        Instead, it represents a depletion of some combination of endurance, luck, or even divine providence that prevents you from suffering more serious injury
        When you take damage, you reduce your \glossterm{hit points} by that amount (see \pcref{Hit Points}).
        If you have no hit points remaining, you may take that damage as \glossterm{vital damage} instead, which represents potentially life-threatening injuries (see \pcref{Vital Damage}).

        Most attacks deal damage equal to the result rolled from a pool of dice.

        \subsubsection{Die Increments}\label{Die Increments}
            Many abilities can increase or decrease your damage with abilities.
            These modifiers always increase or decrease your damage by one \glossterm{die increment}.
            Increasing by one die increment is written as \plus1d, and decreasing by one die increment is written as \minus1d.
            A set of damage dice can increase in size in \glossterm{die increments}.
            Damage dice change in size using the following pattern:
            \begin{itemize}
                \item 1 damage (minimum)
                \item 1d2
                \item 1d3
                \item 1d4
                \item 1d6
                \item 1d8
                \item 1d10
                \item 2d6
                \item 2d8
                \item 2d10
                \item 4d6
                \item 4d8
                \item 4d10
                \item 5d10
                \item 6d10
            \end{itemize}

            For each die increment that increases the damage, move one space down the list.
            Likewise, for each die increment that decreases the damage, move one space up the list.
            After the damage dice reach 4d10, each additional die increment simply adds an extra 1d10 of damage.

        \subsubsection{Standard Damage}\label{Standard Damage}
            Most damaging effects use \glossterm{standard damage} as a baseline to determine the damage they deal.
            Your \glossterm{standard damage} with an ability is based on that ability's \glossterm{power}, if it has one (see \pcref{Power}).
            Otherwise, it is based on your level.
            In either case, it starts at 1d8 and increases by \plus1d per two power or per two levels.
            This is summarized on \trefnp{Standard Damage}.

        \begin{dtable}
            \lcaption{Standard Damage}
            \begin{dtabularx}{\columnwidth}{l X}
                \tb{Power} & \tb{Damage} \\
                0--1   & 1d8  \\
                2--3   & 1d10 \\
                4--5   & 2d6  \\
                6--7   & 2d8  \\
                8--9   & 2d10 \\
                10--11 & 4d6  \\
                12--13 & 4d8  \\
                14--15 & 4d10 \\
                16--17 & 5d10 \\
                18--19 & 6d10 \\
                20--21 & 7d10 \\
                22--23 & 8d10 \\
                24--25\fn{1} & 9d10 \\
            \end{dtabularx}
            1. For values above 25, increase by 1d10 at every even value.
        \end{dtable}

        \subsubsection{Strike Damage}\label{Strike Damage}
            The damage you deal with a single \glossterm{strike} from a weapon is called your \glossterm{strike damage}.
            Your base \glossterm{strike damage} is equal to \glossterm{standard damage} based on your level or your Strength, whichever is higher.
            For example, if you have a Strength of 4, your base \glossterm{strike damage} would be equal to 2d6 (1d8 \plus2d).
            When you make a strike with a weapon, you also apply that weapon's damage modifier, if any.
            Some abilities other than \glossterm{strikes} deal damage based on your \glossterm{strike damage}.

            \parhead{Creature Size and Damage}\label{Creature Size and Damage}
            Larger creatures deal more damage with their weapons.
            Each creature size above Medium grants a \plus2d bonus to damage with strikes.
            Likewise, each creature size below Medium imposes a \minus1d penalty to damage with strikes.
            This is described in \tref{Size in Combat}.

    \subsection{Defenses}\label{Defenses}
        Usually, when you are attacked, the attacker has to make an \glossterm{attack roll} against a specific defense.
        If the attack roll is at least as high as that defense, the attack succeeds.
        There are two physical defenses and two non-physical defenses.
        \begin{itemize}
            \item Armor defense (AD): Your Armor defense protects you from normal physical attacks, such as attempts to hit you with a sword.
                It is the most commonly used defense.
                Armor defense is a physical defense.
            \item Reflex defense: Your Reflex protects you from attacks you have to avoid, such as explosions or falling rocks.
                Reflex defense is a physical defense.
            \item Fortitude defense: Your Fortitude defense protects you from attacks you have to physically endure or resist, such as poisons and deadly spells.
                Fortitude defense is not a physical defense.
            \item Mental defense: Your Mental defense protects you from attacks you have to mentally endure or resist, such as terrifying creatures and magical manipulation.
                Mental defense is not a physical defense.
        \end{itemize}

        \subsubsection{Defense Values}\label{Defense Values}

            Each of your defenses is calculated in the following way:

            \begin{figure}[h]
                \centering Level or defense attribute \add racial defense bonus \add class defense bonus \add other bonuses and penalties
            \end{figure}

            The attributes and relevant bonuses which apply to each defense are described in \trefnp{Defense Calculations}.

            \begin{dtable!*}
                \lcaption{Defense Calculations}
                \begin{dtabularx}{\textwidth}{l l l l l >{\lcol}X}
                    \tb{Defense Name} & \tb{Attributes} & \tb{Starting Attribute Modifier} & \tb{Body Armor Modifier} & \tb{Shield Modifier} & \tb{Size Modifier} \\
                    \midrule
                    Armor defense     & Dex or Con & \tdash & Yes & Yes & No  \\
                    Fortitude defense & Con or Str & Con & No  & No  & Yes \\
                    Reflex defense    & Dex or Per & Dex & No  & Yes & Yes \\
                    Mental defense    & Wil or Int & Wil & No  & No  & No  \\
                \end{dtabularx}
            \end{dtable!*}

            \parhead{Class and Racial Bonuses} Each class and race provides bonuses to some combination of Fortitude, Reflex, and Mental defense.
            \parhead{Natural Armor} Creatures with unusually tough skin or thick hide, including most monsters, gain bonuses to their Armor defense.
            These bonuses stack with any armor such creatures might wear.
            \parhead{Size Modifier} Large creatures have a bonus to Fortitude defense and a penalty to Reflex defense and Stealth.
            For details, see \tref{Size in Combat}.

    \subsection{Initiative}\label{Initiative}
        When multiple creatures take mutually impossible actions simultaneously, such as racing to be the first one to a door, they must roll initiative checks.
        The winner completes the action first, and any other creatures complete as much of the action as possible in order from highest to lowest result.

        Your initiative check is calculated as follows:

        \begin{figure}[h]
            \centering Dexterity or Perception \add other bonuses and penalties
        \end{figure}

        For example, if two creatures were racing to reach a door, they would both roll initiative.
        The winner would reach the door and stop in their intended square, and the loser would stop adjacent to their intended square.

        \parhead{Conditionally Impossible Actions} In rare cases, one action may make another action impossible if the first action succeeds.
        However, unlike with mutually exclusive actions, the second action would not make the first action impossible.
        This usually happens if a creature moves during the action phase while being attacked.
        If the attack trips or deals enough damage to kill the moving creature, its movement becomes impossible.
        In this case, the second action is negated, and the creature takes no action during that action phase.

        \subsubsection{Reach}\label{Reach}
            Normally, you can make melee attacks against anyone within five feet of you.
            The range at which you can make melee attacks is called your \glossterm{reach}, and the area that you can attack into is called your \glossterm{threatened area}.
            Reach for larger and smaller creatures is determined by size, as shown on \trefnp{Size in Combat}.

    \subsection{Circumstances, Bonuses, and Penalties}

        Many effects can grant bonuses or penalties to actions you take.

        \subsubsection{Arbitrary Modifiers}

            Circumstances frequently modify your odds of success when making attacks and checks, or when defending yourself from attacks.
            There are two kinds of circumstantial modifiers.
            Circumstances that make you better or worse at your task give you a bonus or penalty to your attack or check.
            Circumstnaces that make the task easier or harder increase or decrease the \glossterm{difficulty rating} of the task, or the defense of the attacked creature.

            Most circumstances grant a \plus2 bonus or impose a \minus2 penalty.
            Extraordinary circumstances can potentially have greater modifiers.
            All circumstantial modifiers should be used at the discretion of the GM.\@

        \subsubsection{Overwhelm}\label{Overwhelm}
            When a creature is being attacked by multiple foes at once, it is less able to defend itself.
            A creature is considered overwhelmed if it is being threatened by more than one creature.
            Multiple creatures occupying the same square count as a single creature when determining overwhelm penalties.
            If a creature is overwhelmed, it takes a penalty to physical defenses equal to the number of creatures threatening it.

            A creature's overwhelm penalties cannot exceed \minus5.

            \parhead{Ignoring Attackers} You can freely ignore a creature attacking you.
            If you do, you are treated as being \unaware against that creature.
            In exchange, it does not contribute to overwhelm penalties against you.

        \subsubsection{Range Increments}\label{Range Increments}
            Physical ranged attacks often have a specific \glossterm{range increment}.
            A range increment is always measured in feet.
            You take a \minus1 penalty to accuracy with the ranged attack for each full range increment between you and your target.
            For example, when using a longbow with a range increment of 100 feet against a target 170 feet away, you take a \minus1 penalty to accuracy.
            You cannot make a ranged attack beyond 10 range increments away from you.

        \subsubsection{Cover}\label{Cover}

            Cover represents any obstacle that physically prevents you from striking your target, such as a tree or intervening creature.
            A creature behind cover is more difficult to attack.
            There are three kinds of cover: \glossterm{active cover}, \glossterm{passive cover}, and \glossterm{total cover}.
            All three types of cover are determined by the presence or absence of physical obstacles.

            \parhead{Active Cover}\label{Active Cover} Active cover is provided by mobile obstacles between you and your target, such as creatures or tree branches blowing in the wind.
            Physical attacks against creatures with active cover suffer a 20\% miss chance.
            If an attack misses due to active cover, the attack is made against the intervening obstacle rather than being negated like normal for miss chances.
            The obstacle takes any damage from a successful attack normally.

            \parhead{Passive Cover}\label{Passive Cover} Passive cover is provided by immobile obstacles between you and your target, such as trees and walls.
            Creatures with passive cover gain a \plus2 bonus to \glossterm{physical defenses}.
            In addition, creatures with passive cover can hide (see \pcref{Stealth}).

            \parhead{Measuring Cover}

            When you make an attack, choose a single square within your \glossterm{space} and a single \glossterm{target square} within your target's space.
            If you are making a ranged attack, choose one corner of your space.
            If you are making a melee attack, choose any two corners of your square.
            These corners are called the \glossterm{points of origin} for your attack.
            For the purpose of determining cover, your attack originates from your chosen \glossterm{points of origin} and travels to the \glossterm{target square}.

            % TODO: pull out ``line of sight'' and ``line of effect'' from this text
            First, check if you can attack the target at all.
            For each \glossterm{point of origin} of your attack, you must be able to draw two lines to any two corners of your attack's \glossterm{target square}.
            These two lines must not overlap each other.
            In addition, each line must not be blocked by solid objects, though they can touch the edges of spaces blocked by solid objects.
            The lines can pass through obstacles that do not take up the entire area within their space (such as most creatures).
            Finally, the line must not be blocked by other squares within the target's space, preventing you from targeting the ``inside'' of large creatures.
            If you cannot draw such a line, the target has \glossterm{total cover} from you.
            This makes all targeted attacks impossible.

            Second, draw a line from the \glossterm{points of origin} of your attack to the center of your attack's target square.
            If any such line touches a square with an obstacle that grants active or passive cover, even at an edge or corner, the target has the appropriate cover from you.
            Otherwise, if the line is uninterrupted, the target does not have cover from you.

            \parhead{Partial Obstacles} Many obstacles, such as trees and low walls, can provide passive cover without normally blocking \glossterm{line of sight} or providing \glossterm{total cover}.
            Unusually small creatures, or creatures who intentionally take cover behind such obstacles, may be able to gain total cover from them.

            \parhead{Improved Cover}
            A creature can benefit from both passive and active cover.
            Cover of the same type generally doesn't stack; a creature behind two trees is not substantially more protected than a creature behind a single tree.
            However, exceptionally well covered creatures, such as a creature behind an arrow slit in a castle, may receive additional benefits.
            In that case, each additional major obstacle increases the miss chance by 10\% or grants an additional \plus1 bonus to physical defenses, as appropriate.

            \parhead{Total Cover}\label{Total Cover}
            If a creature is completely behind an physical object that blocks sight, it has \glossterm{total cover} from attacks.
            A creature with total cover cannot be targeted by any attacks.
            Total cover is the only kind of cover that also affects attacks other than \glossterm{physical attacks}.

        \subsubsection{Concealment}\label{Concealment}
            Concealment represents anything which makes it more difficult to see your target, such as dim lighting. A creature with concealment from you gains a \plus2 bonus to \glossterm{physical defenses}. The concealment bonus does not apply if you can't see your opponent (such as if you close your eyes).
            Determining concealment works similarly to determining cover.
            You must use the same \glossterm{points of origin} and \glossterm{target square} when determining concealment that you would use to determine cover.

            \parhead{Determining Concealment} There are two things that can cause a creature to be concealed: poor lighting, and intervening obstacles that block sight.
            Determining concealment from obstacles that block sight works the same way as determining cover.

            Determining concealment from lighting conditions is simpler, since it ignores lighting conditions between you and the target.
            If your \glossterm{target square} square is in lighting that provides concealment, the target has concealment.
            Otherwise, it does not.

        \subsubsection{Helpless Defenders}
            A helpless creature is completely at an opponent's mercy.
            Its physical defenses are calculated as if it had a Dexterity of \minus10.
            Paralyzed, bound, and unconscious creatures are helpless.

            \parhead{Coup de Grace}\label{Coup de Grace} As a full-round action, you can deliver a coup de grace to a helpless opponent within your \glossterm{reach} that you can see.
            You automatically hit with your weapon and score a \glossterm{critical success}. If the creature takes damage equal to or higher than half its maximum hit points, it immediately dies.

            Delivering a coup de grace requires focused concentration and methodical action. This leaves you \defenseless. If you take damage in excess of your Fortitude defense while delivering a coup de grace, your attempt fails.
            If your target stops being helpless during your coup de grace attempt for any reason, the attempt fails.
            You can't deliver a coup de grace against a creature that is immune to critical hits.

        \subsubsection{Invisibility}\label{Invisibility}
            If it is impossible to see your target, you can't attack him normally. However, you can attack a square that you think he occupies. An attack into a square occupied by an invisible enemy has a 50\% miss chance. If an adjacent invisible creature strikes you, you can automatically identify the square he occupied when he struck you.

        \subsubsection{Surprise Attacks}\label{Surprise Attacks}
            Sometimes, creatures are not aware that combat is taking place when the combat starts. This most commonly happens with ambushes. Any creature that is not aware of the combat continues taking whatever actions it would normally be taking until it becomes aware of the combat. It is \unaware until that point.

    \subsection{Size in Combat}\label{Size in Combat}
        Size affects your space and reach in combat.
        In addition, your Fortitude and Reflex defenses are affected by your size modifier.
        These effects are shown on \trefnp{Size in Combat}.

        \begin{dtable*}
            \lcaption{Size in Combat}
            \begin{dtabularx}{\textwidth}{l l l l l X}
                \tb{Size} & \tb{Space\fn{1}} & \tb{Reach\fn{1}} & \tb{Size Modifier\fn{2}} & \tb{Damage Modifier\fn{3}} & \tb{Example Creature} \\
                \bottomrule
                Fine              & 1/2 ft.    & 0          & \plus8  & \minus4d & Fly                      \\
                Diminutive        & 1 ft.      & 0          & \plus6  & \minus3d & Toad                     \\
                Tiny              & 2-1/2 ft.  & 0          & \plus4  & \minus2d & Cat                      \\
                Small             & 5 ft.      & 5 ft.      & \plus2  & \minus1d & Halfling                 \\
                Medium            & 5 ft.      & 5 ft.      & \tdash  & \tdash & Human                    \\
                Large (tall)      & 10 ft.     & 10 ft.     & \minus2 & \plus2d & Ogre                     \\
                Large (long)      & 10 ft.     & 5 ft.      & \minus2 & \plus2d & Horse                    \\
                Huge (tall)       & 15 ft.     & 15 ft.     & \minus4 & \plus4d & Cloud giant              \\
                Huge (long)       & 15 ft.     & 10 ft.     & \minus4 & \plus4d & Bulette                  \\
                Gargantuan (tall) & 20 ft.     & 20 ft.     & \minus6 & \plus6d & 50-ft.\ animated statue  \\
                Gargantuan (long) & 20 ft.     & 15 ft.     & \minus6 & \plus6d & Kraken                   \\
                Colossal (tall)   & 30\add ft. & 30\add ft. & \minus8 & \plus8d & Colossal animated object \\
                Colossal (long)   & 30\add ft. & 20\add ft. & \minus8 & \plus8d & Great wyrm red dragon    \\
            \end{dtabularx}
            1 Creatures can vary in space and reach.  These are simply typical values.  \\
            2 Applies to Reflex defense.  The same value is applied in reverse to Fortitude defense.
            For example, a Huge creature has a \minus4 penalty to Reflex and a \plus4 bonus to Fortitude defense. \\
            3. This modifier applies to \glossterm{strike damage}.
        \end{dtable*}

        \parhead{Other Effects}
        A creature's size affects a number of additional skills and abilities.
        For example, larger creatures have a penalty to Stealth checks (see \pcref{Size and Stealth}).
        The effects of unusual size are described in those skills and abilities.

        Unusually large or small creatures also have other special rules apply to them, as described below.
        In addition, larger creatures deal more damage with weapons, and smaller creatures deal less damage with weapons, as described in \tref{Size in Combat}.

        \subsubsection{Very Small Creatures}
            \parhead{Space} If a creature takes up less than a single square of space, you can fit multiple creatures in that square. You can fit four Tiny creatures in a square, twenty-five Diminuitive creatures, or 100 Fine creatures.

            \parhead{Reach} Creatures that take up less than 1 square of space typically have a natural reach of 0 feet, meaning they can't reach into adjacent squares. They must enter an opponent's square to attack in melee. You can attack into your own square if you need to, so you can attack such creatures normally. Since they have no natural reach, they do not threaten the squares around them.

            If a creature without a natural reach uses a reach weapon, it gains no benefits or penalties (see \pcref{Reach Weapon}).

            \parhead{Movement} Creatures three size categories smaller than you are not considered obstacles and do not hinder your movement.

        \subsubsection{Very Large Creatures}
            \parhead{Space} Very large creatures take up multiple squares. Anything which affects a single square the creature occupies affects the creature.

            \parhead{Reach} Creatures that take up more than 1 square typically have a natural reach of 10 feet or more, meaning that they can reach targets even if they aren't in adjacent squares. Creatures with a large natural reach can attack anyone within their reach, including adjacent foes.

            Creatures with a large natural reach using reach weapons can strike at up to double their natural reach but can't strike at their natural reach or less, just like Medium sized creatures.

            \parhead{Movement} Creatures three size categories larger than you are not considered obstacles and do not hinder your movement.

            \parhead{Immunities} Creatures at least three size categories larger than you are difficult to fight. You cannot get a \glossterm{critical success} with \glossterm{strikes} or contribute to overwhelm penalties against such creatures. If you can reach a vulnerable point on the creature, such as by flying or by using ranged weapons, you can get critical successes and contribute to overwhelm penalties normally.

    \subsection{Stacking Rules}\label{Stacking Rules}
        Usually, modifiers stack with each other, meaning that you add or subtract all of the modifiers to get the final result. However, some modifiers do not stack with each other, as described below. When bonuses don't stack with each other, you only apply the largest bonus. Likewise, when penalties don't stack with each ather, you only apply the largest penalty.

        \parhead{Special Exceptions}

        \begin{itemize}
            \item Effects from the same source do not stack. Any spell or ability with the same name has the same source.
            \item \glossterm{Sizing} effects do not stack.
                If multiple effects both increase and decrease size, size increases offset size decreases on a one-for-one basis to determine the creature's final size.
            \item Effects that grant extra \glossterm{strikes} (such as the \spell{haste} spell with the Empowered augment) do not stack.
            \item Temporary hit points do not stack.
            \item If a character has two separate abilities which let them add the same attribute to a given roll or statistic, the attribute is still only added once.
        \end{itemize}

        \subsubsection{Maximum Bonuses}\label{Ability Limits}
            Some bonuses specify that they cannot increase the value beyond a given point.
            For example, the \spell{totemic power} spell cannot increase a physical attribute to be higher than the caster's spellpower.
            These bonuses must always be applied last, and cannot be combined with other bonuses to exceed the maximum value.
            If multiple bonuses specify different maximum values, use the lower maximum value.
            If a bonus with a maximum value is applied to a value that already exceeds the maximum value the bonus can provide, simply ignore the bonus and its maximum value.

    \subsection{Doubling}\label{Doubling}
        If you double any in-game value twice, it becomes three times as large. An additional doubling would make it four times as large, and so on. For example, if you make an attack that deals double damage and you get a critical hit, you deal triple damage.

        \parhead{Real-World Values} Values that are ``real'', such as movement and distance, are an exception.
        Any real value has a unit that it measures, such as feet.
        Abstract values, such as bonuses and penalties to attacks and checks, do not have units.
        If you double a real-world value twice, it becomes four times as large.

    \subsection{Changing Statistics}

        Your modifiers and defenses can change for many reasons.
        In general, all changes take effect immediately, though some side effects of those changes may not happen until you rest or level up.

        \parhead{Numerical Modifiers} Changes to numerical modifiers always take effect immediately.
        For example, if a barbarian enters a rage, his damage and defenses are all adjusted immediately.

        \parhead{Ability Prerequisites} Changes to prerequisites for abilities always take effect immmediately.
        For example, if a paladin's Strength is reduced to 0 by a ghost, she immediately loses the benefits of all feats she has that require a high Strength, such as Reaper (see \featpcref{Reaper}).

        \parhead{Ability Use Limits} Effects that change a character's maximum uses of an ability take effect immediately.
        However, increasing a character's maximum uses does not immediately grant the character additional uses.
        They must recovered in the normal fashion, such as by resting.
        If a character's maximum uses is decreased below their currently available uses, they immediately lose ability uses until their current value is equal to their maximum value.

        \parhead{Skill Points} Effects that change a character's skill points take effect immediately.
        However, the character cannot spend additional skill points on new skills until they level up.
        If a character's total skill points are decreased below their currently spent skill points, they immediately lose training from skills until their spent skill points are equal to their total skill points.

        \parhead{Hit Points} Effects that change a character's maximum hit points take effect immediately.
        However, increasing a character's maximum hit points does not immediately grant the character additional hit points.
        They must be recovered in the normal fashion, such as by resting.
        If a character's maximum hit points are decreased below their current hit points, they immediately lose hit points until their current hit points are equal to their maximum hit points.

\section{Injury, Death, and Healing}\label{Injury, Death, and Healing}

    \subsection{Hit Points}\label{Hit Points}
        Your hit points measure how hard you are to kill.
        When you take damage, you subtract that damage from your hit points.
        No matter how many hit points you lose, your character isn't significantly hindered until your hit points drop to 0.
        When you run out of hit points, your actions are limited and you might die.

        Your hit points are calculated as follows:

        \begin{figure}[h]
            \centering Level \x (5 \add starting Constitution)
        \end{figure}

        \parhead{What Hit Points Represent} Hit points represent a combination of durability, luck, divine providence, and sheer determination, depending on the nature of your character.
        When you take 10 damage from an orc with a greataxe, the axe did not literally carve into your skin without affecting your ability to fight.
        Instead, you avoided the worst of the blow, but it bruised you through your armor, the effort to dodge the blow fatigued you, or it barely nicked you through sheer luck -- and everyone's luck runs out eventually.

    \subsection{Vital Damage}\label{Vital Damage}
        When you take damage in excess of your hit points while \glossterm{wounded}, that damage represents serious physical injury to your body.
        This is called \glossterm{vital damage}.
        For every 4 points of \glossterm{vital damage} you have, you take a \minus1 penalty to \glossterm{accuracy}, \glossterm{checks}, and \glossterm{defenses}.

    \subsection{Overkill Damage}\label{Overkill Damage}
        Normally, when you take damage that would reduce your hit points to 0, any excess damage from the attack is wasted.
        However, some attacks deal such massive damage that you begin dying immediately, rather than just becoming staggered.
        If you take damage in excess of your \glossterm{bloodied} hit point total in a single round, any damage past what would reduce your hit points to 0 is dealt as \glossterm{vital damage}.

    \subsection{Stages of Injury}

        \parhead{Healthy} When you are above half hit points, you suffer no significant effects from losing hit points.
        If you take damage, you can become bloodied (see Bloodied, below).

        \parhead{Bloodied} When you drop to half your hit points or below, you are \bloodied.
        A bloodied creature takes a \minus4 penalty to Fortitude and Mental defense.
        The first time you become \glossterm{bloodied} in an encounter, you become \glossterm{staggered} until the end of the next round.
        If you take additional damage, you can become \glossterm{wounded} (see Wounded, below).

        \parhead{Staggered}\label{Staggered}
        While \glossterm{staggered}, you take a \minus4 penalty to \glossterm{accuracy}, checks, and defenses.
        These penalties do not stack with the penalties for being \glossterm{bloodied}.
        % TODO: clarify when this ``resets''
        The first time you become bloodied in an encounter, you become staggered until the end of the next round.
        In addition, whenever you become \glossterm{wounded}, you become staggered until the end of the next round.

        \parhead{Wounded}\label{Wounded} At the end of each round, if you have no hit points remaining after resolving all other effects in the round, you become \wounded.
        In addition, you are wounded if you have any \glossterm{vital damage}.
        If you take additional damage while wounded, that damage can become \glossterm{vital damage}.

        At the end of each round that you are \glossterm{wounded}, if you have taken damage in excess of your hit points, you take that damage as \glossterm{vital damage}.
        When determining whether your damage exceeds your hit points, count all healing and any other effects that modify your hit points, even if they occurred after you took damage.
        If you take \glossterm{vital damage}, you begin dying (see Dying, below).
        If you instead have hit points remaining at the end of the round, you stop being wounded.

        \parhead{Dying}\label{Dying} While you are dying, you must make an attack against your own Fortitude defense at the end of every round.
        This is called a \glossterm{stabilization roll}.
        No bonuses or penalties apply to the attack roll, but \glossterm{vital damage} and other effects can penalize your Fortitude defense.
        If you fail to resist the attack once, you fall unconscious.
        If you fail to resist the attack three times, you die.
        If you resist the attack three times, you stabilize.
        When you stabilize, you no longer need to make attacks against yourself each round, and any failures that brought you closer to death are negated.
        If you take additional \glossterm{vital damage} while dying, any successes you rolled to resist death are negated, but any failures remain unchanged.

        An ally can make a Heal check to tend to you while you are dying.
        The Heal check result can be used in place of your Fortitude defense.
        However, any \glossterm{vital damage penalties} you suffer apply to the Heal check in the same way that they apply to your Fortitude defense.

    \subsection{Healing}
        After taking damage, you can recover hit points through natural healing or through magical healing.
        In any case, you can't regain hit points past your full normal hit point total.

        \parhead{Natural Healing} With half an hour of rest, you recover one quarter of your hit points.
        Any significant interruption (such as combat or the like) during your rest prevents you from healing.
        If you rest for two hours, you recover all your hit points.

        \parhead{Magical Healing} Various abilities and spells can restore hit points.
        However, only certain spells can heal vital damage, as specified in the spell description.
        Unless a spell says it can cure vital damage, it cannot -- though it can still stabilize dying characters.
        Magical healing has no effect on the hit points of creatures with vital damage.

        \parhead{Healing Vital Damage} Vital damage takes much longer to heal than hit point damage.
        Resting for 8 hours restores an amount of vital damage equal to 1 \add half the character's Constitution (minimum 1).
        A character can have both hit points and vital damage.
        As long as a character has vital damage, he is staggered, even if he is at full hit points.

    \subsection{Nonlethal Damage}\label{Nonlethal Damage}
        Some attacks and environmental effects deal nonlethal damage.
        Nonlethal damage is not subtracted from your hit points.
        Instead, it is tracked separately.
        If your nonlethal damage exceeds your hit points, you become staggered, just as if you were at 0 hit points.
        If you take additional damage while staggered, you fall unconscious.
        However, you do not begin dying unless your hit points are actually below 0.

        \parhead{Healing Nonlethal Damage}
        You heal half your hit points in nonlethal damage with 1 hour of rest.
        When a spell or a magical ability cures hit point damage, it also removes an equal amount of nonlethal damage.

    \subsection{Temporary Hit Points}\label{Temporary Hit Points}
        Certain effects give a character temporary hit points which act as a protection against damage.
        Whenever a character takes damage, if he has temporary hit points, the damage is applied to his temporary hit points first.
        Any excess damage is then applied to his hit points as normal.
        Temporary hit points are not ``real'' hit points, and cannot be healed.
        If a character has temporary hit points from multiple effects, only the highest value is used.

% TODO: This is a bad name; organize these better
\section{Special Rules}

    \subsection{Power}\label{Power}
        Many abilities have a \glossterm{power} which represents the overall strength of the ability.
        For example, your \glossterm{power} with \glossterm{spells} is called your \glossterm{spellpower}.
        Your \glossterm{power} with an ability usually determines your \glossterm{accuracy} and \glossterm{damage} with that ability, if applicable.

        Most abilities explicitly specify how to calculate their power.
        If an ability does not have an explicitly stated power, use your level as its power when necessary.

    \subsection{Attunement}\label{Attunement}
        Many abilities last as long as you \glossterm{attune} to them.
        As long as you are attuned to an ability, you cannot regain the \glossterm{action point} spent to use that ability.
        If the ability does not normally require spending an action point to use, you can spend an action point to attune to it.
        That spent action point still counts against the number of action points you recover with a \glossterm{short rest} or \glossterm{long rest}.
        You must release your attunement to abilities before beginning to rest in order to recover the spent action points.
        Attuning to multiple abilities can prevent you from regaining any action points when you rest.

        You can choose to stop attuning to an ability as a \glossterm{free action}.
        If you do, the ability ends, and you can regain the action point spent to use it normally.

        Attuned abilities continue to work across any distance, but not across planar boundaries.
        At the end of each round, your attunement to all abilities active on a different plane than your current plane ends.
        Planar travel that does not last a full round, such as teleportation, does not interrupt your attunement.

        \subsubsection{Shared Attunement}\label{Shared Attunement}
            Some abilities also require the targets of the ability to spend action points to gain the ability's effects.
            A shared attunement ability lasts as long as the creature using the ability and all targets of the ability \glossterm{attune} to it.
            If you target yourself with a shared attunement ability, you only need to spend one action point, not two.

    \subsection{Grappling}\label{Grappling}
        Grappling creatures are physically struggling with each other. While grappled, you suffer certain penalties and restrictions, as described below.

        \subsubsection{Being In A Grapple}
            While grappling, you suffer certain penalties and restrictions, as described below. Other than these restrictions, you can act normally. You can also take certain actions in a grapple, as described in \pcref{Grapple Actions}
            \begin{itemize}
                \item You must use a free hand (or equivalent limbs) to grapple, preventing you from taking any actions which would require having two free hands. If you cannot free a hand, you suffer a \minus10 penalty to accuracy on all physical attacks, including grapple attacks, until you have a free hand.
                \item You are \defenseless against creatures you are not grappling with.
                \item You take a \minus4 penalty to accuracy with weapons that are not light, since they are too large and cumbersome to be used effectively in a grapple.
                \item Spellcasting is extremely difficult. You cannot cast spells with somatic components. Casting a spell without somatic components requires a Concentration check with a DR equal to 20 \add double spell level.
                \item You cannot move normally (but see Move the Grapple, below).
            \end{itemize}

        \subsubsection{Grapple Actions}\label{Grapple Actions}
            While grappled, you can take four special actions to try to affect the grapple.

            \parhead{Bind Foe} As a standard action, you can make a grapple attack to bind your foe with ropes or other restraints. You must have the restraints in hand (in addition to the free hand required to grapple). Apply the result against the Fortitude and Reflex defenses of a creature grappling you. Success against both defenses means the creature is bound, rendering it helpless and effectively paralyzed. The only physical actions a bound creature can take are to esccape or break the bindings. Escaping the bindings requires a grapple attack or Escape Artist check which beats the grapple attack made to bind the creature. Breaking the bindings requires making a Strength check sufficient to break the item used to bind the creature. If you have the time, you can \glossterm{take 10} on your grapple attack to secure bindings on a helpless foe, including a foe rendered helpless through this attack.
            \parhead{Escape the Grapple} As a standard action, you can make a grapple attack or Escape Artist check to escape from the grapple. Apply the result against the Reflex defense of each creature grappling you. Success against an individual creature means you are no longer grappling that creature.
            \parhead{Move the Grapple} As a move action, you can make a grapple attack to move the grapple. Apply the result against the Fortitude defense of each creature grappling you. If you beat every creature's Fortitude defense, you can move yourself and all other creatures in the grapple up to half your speed. At the end of your movement, you choose which spaces creatures grappling you are in, as long as they stay adjacent to you.
            \parhead{Pin} As a standard action, you can make a grapple attack against the Fortitude and Reflex defenses of a creature you are grappling with. Success against both defenses means the creature becomes \glossterm{pinned} (in addition to being grappled), while you remain only grappled. The only physical action a pinned creature can take is to escape the grapple, though it can take mental actions. At your discretion, you can cover the mouth of a pinned creature to prevent it from speaking.

    \subsection{Unarmed Combat}\label{Unarmed Combat}
        Every creature can attack with its body using an unarmed attack.
        You are not proficient with your unarmed attack, so you are usually \defenseless while unarmed.
        In addition, an unarmed attack always deals nonlethal damage.
        You may use any appropriate part of your body to make an unarmed attack -- fists, feet, elbows, and so on.
        However, you only have one unarmed attack.
        You cannot dual-wield unarmed attacks as if you were fighting with two weapons at once (see \pcref{Dual Strike}).

        An unarmed attack is a type of natural weapon.
        Spells and abilities that affect natural weapons can affect your unarmed attack.
        Gauntlets can also be worn to increase the power of an unarmed attack (see \pcref{Unarmed Weapons}).

    \subsection{Mounted Combat}\label{Mounted Combat}
        \parhead{Horses in Combat} Warhorses and warponies can serve readily as combat steeds. Light horses, ponies, and heavy horses, however, are frightened by combat. If you don't dismount, you must make a DR 20 Ride check each round as a move action to control such a horse. If you succeed, you can perform a standard action after the move action. If you fail, you can't do anything else until your next turn.

        Your mount acts on your initiative count as you direct it. You move at its speed, but the mount uses its action to move.

        \parhead{Space} A horse (not a pony) is a Large creature, and thus takes up a space 10 feet (2 squares) across. While mounted, you share your mount's space completely. Anyone threatening your mount can attack either you or your mount. However, your reach is still that of a creature of your normal size. Thus, a Medium paladin would threaten all squares adjacent to his Large horse with a longsword, and all squares 10 feet away from his mount with a lance.

        In the case of abnormally large mounts (two or more size categories larger than you), you may not completely share space. Such situations should be handled on a case-by-case basis, depending on the nature of the mount.

        \parhead{Combat while Mounted} With a DR 5 Ride check, you can guide your mount with your knees so as to use both hands to attack or defend yourself. This is a free action.

        If your mount moves more than its normal speed when charging, you also take the physical defense penalty associated with a charge. If you make an attack at the end of such a charge, you receive the bonus gained from the charge. When charging on horse-back, you deal double damage with a lance (see \pcref{Charge}).

        You can use ranged weapons while your mount is moving in the same phase, but at a \minus2 penalty to accuracy.
        If your mount is also sprinting (see \pcref{Sprint}), this penalty increases to \minus4.

        \parhead{Casting Spells while Mounted} You can cast a spell normally if your mount moves up to a normal move (its speed) either before or after you cast. If you have your mount move both before and after you cast a spell, then you're casting the spell while the mount is moving, and you have to make a Concentration check due to the vigorous motion (DR 10 \add double spell level) or lose the spell. If the mount is sprinting, your Concentration check is more difficult due to the violent motion (DR 15 \add double spell level).

        \parhead{If Your Mount Falls in Battle} If your mount falls, you have to succeed on a DR 15 Ride check to make a soft fall and take no damage. If the check fails, you take 1d6 points of damage.

        \parhead{If You Are Dropped} If you are knocked unconscious, you have a 50\% chance to stay in the saddle (or 75\% if you're in a military saddle). Otherwise you fall and take 1d6 points of damage. Without you to guide it, most mounts avoid combat.

    \subsection{Drowning}\label{Drowning}
        You can hold your breath for a number of rounds equal to 5 \add your Constitution.
        After that time, you must roll 1d10.
        This attack gains a \plus5 bonus for each round you hold your breath beyond your limit.
        If the result exceeds your Fortitude defense, you take \glossterm{vital damage} equal to the difference.

    \subsection{Ability Timing}
        % TODO: add more content here as necessary
        Some reactive abilities can be used at times where actions can't normally be taken.
        For example, many abilities specify that you can use them ``when you are hit''.
        This section defines more precisely when such abilities can be used.

        \parhead{When You Are Hit} These abilities are used after the success or failure of all attacks within that phase has been declared, but before any effects of those attacks are declared.
        That means you can activate the ability after you know all of the attacks that hit you during that phase.
        You would also know which attacks were critical hits, allowing you to use the ability to affect those attacks specifically.

\section{Daily Resources}

    \subsection{Action Points}\label{Action Points}
        You can perform a wide variety of special actions by spending \glossterm{action points}.
        You have four action points at 1st level.
        At 5th level, and every 4 levels thereafter, you gain an additional action point.
        You can gain additional action points from your class, from your Willpower, and from some special abilities.

        After a \glossterm{short rest}, you regain all action points you spent since your last rest, up to a maximum equal to half of your maximum action points.
        After a \glossterm{long rest}, you recover all spent action points.

    \subsection{Legend Points}\label{Legend Points}

        As your character gains power and influence in the world, you may gain \glossterm{legend points}.
        Legend points allow you to change fate to ensure your character succeeds.

        \subsubsection{Using Legend Points}
            You can use a legend point to automatically roll a 10 on any \glossterm{attack} or \glossterm{check} you make.
            On attack rolls, this allows you to roll again, just as if you had rolled a 10 normally (see \pcref{Exploding Attacks}).
            Alternately, you can use a legend point to make any \glossterm{attack} or \glossterm{check} against you roll a 1.

            Using a legend point is not an action, and can be done at any time.
            You can decide to use a legend point after you learn whether the original roll succeeded or failed.
            You can even use a legend point after you learn what the effects of a successful attack or check would be, if that is information you could normally learn if it succeeded.
            However, you must use the legend point before the effects have completely resolved.

            If an attack affects multiple targets, your legend point only affects the roll against you, and does not change the attack's effects against the other targets.
            If you are \glossterm{unaware} of an attack or check, you cannot use a legend point on it.

        \subsubsection{Gaining Legend Points}

            You gain legend points as you gain levels, as described in \pcref{Character Advancement}.
            In addition, some abilities can grant additional legend points.

        \subsubsection{Restoring Legend Points}

            After a \glossterm{long rest}, you regain all spent legend points, up to your maximum number of legend points.
            It is possible to gain additional legend points during the day by performing extraordinary actions worthy of legends.

        \subsubsection{Legendary Foes}
            Some monsters and humanoid enemies you encounter may have their own legend points.
            In addition, some monsters have such legendary might that they can prevent characters from using legend points near them.

    \subsection{Resting}
        When you have a moment to relax, you can rest to regain some of your expended abilities.
        There are two main types of rests: a \glossterm{short rest} and a \glossterm{long rest}.

        \subsubsection{Short Rest}\label{Short Rest}
            Resting for five minutes is considered a \glossterm{short rest}.
            When you take a short rest, you gain the following benefits.
            \begin{itemize}
                \item You heal hit points equal to your starting Constitution \add half your level, to a minimum of 1 hit point of healing.
                    The Heal skill can increase this healing (see \pcref{Heal}).
                    You can also continue resting to heal additional hit points.
                \item You regain all action points you spent since your last rest, up to a maximum equal to half of your maximum action points.
                    Being \glossterm{attuned} to effects at the start of your rest can reduce the number of action points you recover in this way (see \pcref{Attunement}).
                \item You remove all \glossterm{conditions} affecting you (unless they cannot be removed normally).
                \item Some other abilities have specific effects that last until you take a short rest.
                    For example, a barbarian cannot use her \textit{rage} ability again after raging until after she takes a short rest (see Rage, page \pref{Bbn:Rage}.
            \end{itemize}

        \subsubsection{Long Rest}\label{Long Rest}
            Resting for eight hours is considered a \glossterm{long rest}.
            When you take a long rest, you gain the following benefits.
            \begin{itemize}
                \item You heal vital damage equal to your starting Constitution \add half your level, to a minimum of 1 vital damage.
                \item You regain all spent \glossterm{action points}.
                    Being \glossterm{attuned} to effects at the start of your rest can reduce the number of action points you recover in this way (see \pcref{Attunement}).
                \item You regain all spent \glossterm{legend points}.
                \item You regain all spent \glossterm{item slots}.
                \item Some other abilities have specific effects that last until you take a long rest.
            \end{itemize}

\section{Character Advancement}\label{Character Advancement}

    As your character accomplishes challenges and defeats foes, they gain experience.
    If your character has enough experience, they gain a level.
    When you gain a level, you may gain new abilities from your class and the feats you have chosen.
    You also gain some abilities at specific levels, as described in \trefnp{Character Advancement}.

    A character that increases in level gains additional benefits.
    \begin{itemize}
        \item At 1st, 3rd, 6th, and 10th level, you gain a feat (see \pcref{Feats}).
        \item At 2nd level, and every 4 levels thereafter, you gain an additional \glossterm{item slot}.
        \item At 3rd level, and every 4 levels thereafter, you gain a \glossterm{legend point} (see \pcref{Legend Points}).
        \item At 4th level, and every 4 levels thereafter, you gain a \glossterm{legacy item} upgrade (see \pcref{Legacy Items}).
        \item At 5th level, and every 4 levels thereafter, you gain an \glossterm{action point} (see \pcref{Action Points}).
    \end{itemize}

    \begin{dtable}
        \lcaption{Character Advancement}
        \begin{dtabularx}{\columnwidth}{*{3}{>{\lcol}l} *{4}{>{\lcol}X}}
            \tb{Level} & \tb{XP} & \tb{Feats} & \tb{Action Points} & \tb{Legend Points} & \tb{Legacy Item}\fn{1} & \tb{Item Slots}
            \\\bottomrule
            1st     & 0      & 1\fn{2}& 4      & \tdash & \tdash & 1
            \\ 2nd  & 20     & \tdash & \tdash & \tdash & \tdash & 2
            \\ 3rd  & 50     & 2      & \tdash & 1      & \tdash & \tdash
            \\ 4th  & 90     & \tdash & \tdash & \tdash & 1      & \tdash
            \\ 5th  & 150    & \tdash & 5      & \tdash & \tdash & \tdash
            \\ 6th  & 230    & 3      & \tdash & \tdash & \tdash & 3
            \\ 7th  & 350    & \tdash & \tdash & 2      & \tdash & \tdash
            \\ 8th  & 510    & \tdash & \tdash & \tdash & 2      & \tdash
            \\ 9th  & 750    & \tdash & 6      & \tdash & \tdash & \tdash
            \\ 10th & 1,050  & 4      & \tdash & \tdash & \tdash & 4
            \\ 11th & 1,550  & \tdash & \tdash & 3      & \tdash & \tdash
            \\ 12th & 2,200  & \tdash & \tdash & \tdash & 3      & \tdash
            \\ 13th & 3,150  & \tdash & 7      & \tdash & \tdash & \tdash
            \\ 14th & 4,450  & \tdash & \tdash & \tdash & \tdash & 5
            \\ 15th & 6,350  & \tdash & \tdash & 4      & \tdash & \tdash
            \\ 16th & 8,900  & \tdash & \tdash & \tdash & 4      & \tdash
            \\ 17th & 13,000 & \tdash & 8      & \tdash & \tdash & \tdash
            \\ 18th & 18,000 & \tdash & \tdash & \tdash & \tdash & 6
            \\ 19th & 25,500 & \tdash & \tdash & 5      & \tdash & \tdash
            \\ 20th & 36,000 & \tdash & \tdash & \tdash & 5      & \tdash
        \end{dtabularx}
        1. This is the number of abilities you gain with your \glossterm{legacy item} (see \pcref{Legacy Items}). \\
        2. All races also grant a bonus feat at 1st level. The feat must be chosen from a specific list of racial bonus feats (see \pcref{Races}). \\
    \end{dtable}
