\chapter{Core Mechanics}

This chapter describes the core mechanics of Rise.

\section{Defining the Undefined}
    This book does not attempt to include specific rules for every aspect of a realistic world.
    Unless defined otherwise - or if it's not worth the effort to look up Rise's exact rules in the flow of a game - you should assume that the universe works more or less like the real world does, and as long as everyone agrees that something is reasonable, it's not worth worrying about in more detail.
    For example, Rise does not have specific rules for how long it takes to eat a meal, the arc that a thrown ball takes through the air, or how much extra weight a well-made chandelier can hold without breaking.
    It's possible to imagine situations where each of those might be important to a game, however, so you'll have to guess what would be reasonable as obscure situations arise.
    The Game Master has the final word when defining ambiguities like this.

\section{Making Attacks}\label{Attacks}
    Anything that affects another creature in a potentially harmful way, such as striking a creature with a sword, is an attack.
    Many abilities are always considered attacks, even if you use them in a way that you believe is not harmful.
    To make an attack, you must make an \glossterm{attack roll}.

    \subsection{Attack Rolls}\label{Attack Rolls}
        To make an attack roll, roll 1d10 and add your \glossterm{accuracy} with the attack.
        The sum of your die roll and your accuracy is called your \glossterm{attack result}.
        You compare your attack result to a \glossterm{defense} that your \glossterm{target} has (see \pcref{Defenses}).
        All attacks specify which defense they are compared to.

        If your result is equal to or higher than your target's defense, the attack hits.
        This almost always means the target suffers some harmful effect, such as taking \glossterm{damage} (see \pcref{Damage}).
        If your result is 1 or 2 lower than the defender's defense, you get a \glossterm{glancing blow} (see \pcref{Glancing Blows}).
        Otherwise, the attack misses.
        When an attack misses, the defender almost always suffers no effects from the attack.
        Even if the attack had no obvious physical or visual effects, a creature that resists an attack still feels a hostile force or a tingle, but cannot usually deduce the exact nature of the attack.

        Creatures can voluntarily lower their defenses against attacks that they are aware of.
        When they do, their defense is treated as 0 against the attack.

        \subsubsection{Exploding Attacks}\label{Exploding Attacks}
            When you make an attack roll, if you roll a 10 on the d10, the die \glossterm{explodes}.
            In addition, some effects can cause your roll to \glossterm{explode} without rolling a 10.

            When an attack roll \glossterm{explodes}, you roll it again and add the second result to the original result before applying your \glossterm{accuracy}.
            If you roll a 10 on the extra roll, you keep rolling until you stop rolling a 10 and add all of the rolls together.

        \subsubsection{Critical Hits}\label{Critical Hits}
            If your attack result is at least 10 higher than your target's defense, your attack is a \glossterm{critical hit}.
            Some attacks have specific effects when you get a critical hit, as indicated in their descriptions or in other abilities.
            In addition, all damaging attacks have a standard critical hit effect.
            For every increment of 10 by which you beat the target's defense, you double the number of damage dice you roll.
            As normal, two doublings become a tripling, so if you beat your opponent's defense by 20, you roll triple the damage dice.
            This does not increase the damage from your \glossterm{power} or any other non-dice damage modifiers.

            Objects are not normally subject to critical hits.
            Some creatures are also not subject to critical hits, as noted in their descriptions.

        \subsubsection{Glancing Blows}\label{Glancing Blows}
            When you miss on an attack by 2 or less, it is called a glancing blow.
            All damaging attacks have a standard glancing blow effect.
            Non-damaging attacks do not have any special effects on a glancing blow.
            If you get a glancing blow with a damaging attack, you roll no damage dice.
            This does not reduce the damage from your \glossterm{power} or any other non-dice damage modifiers.
            If you would not normally add your \glossterm{power} to the attack, it deals no damage.

    \subsection{Dealing Damage}\label{Dealing Damage}
        Many \glossterm{attacks} deal damage to their targets.
        In general, most damaging attacks deal an amount of damage determined by rolling some number of dice and adding some multiplier of your \glossterm{power} with that attack.
        The details are given in each attack's description.

        When a creature is dealt damage, the damage first reduces that creature's \glossterm{damage resistance} (see \pcref{Damage Resistance}).
        Any damage in excess of the creature's remaining damage resistance causes it to lose that many \glossterm{hit points} (see \pcref{Hit Points}).
        If you take damage that would reduce your hit points below 0, you gain one or more \glossterm{vital wounds} (see \pcref{Negative Hit Points}).
        Monsters typically do not gain vital wounds like player characters do.
        Instead, they simply die or fall unconscious when they reach 0 hit points.

        \parhead{Dealing Damage, Taking Damage, and Losing Hit Points}
        You deal damage whenever you hit with a damaging attack, regardless of whether the target loses hit points or only \glossterm{damage resistance}.
        Likewise, you take damage whenever anything deals damage to you.
        However, you only lose hit points if the damage is not mitigated by your damage resistance.
        Many active abilities require the target to lose hit points from a damaging attack, which does not happen if the damage is resisted.

        \subsubsection{Dice Pools}\label{Dice Pools}
            Almost all attacks deal damage based on a \glossterm{dice pool}.
            Likewise, healing abilities usually heal hit points based on a dice pool.
            Dice pools are written with the number of dice, followed by ``d'', followed by the size of dice to roll.
            For example, 2d6 means you roll two six-sided dice.
            You always sum the roll of all dice rolled in a dice pool to determine the total result.

            Some modifiers add or subtract flat values from the result of the dice pool.
            Others add or subtract \glossterm{dice increments}.

            A die increment is a single increase or decrease in the value of a dice pool.
            Increasing by one die increment is written as \plus1d, and decreasing by one die increment is written as \minus1d.
            Damage dice change in size according to the following pattern:
            \begin{itemize}
                \item 1 damage (minimum)
                \item 1d2
                \item 1d3
                \item 1d4
                \item 1d6
                \item 1d8
                \item 1d10
                \item 2d6
                \item 2d8
                \item 2d10
                \item 4d6
                \item 4d8
                \item 4d10
                \item 5d10
                \item 6d10
                \item 7d10
                \item 8d10
            \end{itemize}

            For each die increment that increases the damage, move one space down the list.
            Likewise, for each die increment that decreases the damage, move one space up the list.
            After the dice pool reaches 8d10, each additional die increment adds an additional 1d10.
            In practice, you are unlikely to ever roll that many dice.

            \parhead{Dice Bonuses From Attributes}\label{Dice Bonuses From Attributes} Your attributes can add or subtract dice increments from all dice pools you roll.
            Whenever you use a \glossterm{mundane} ability that has a dice pool, you add half your Strength in dice increments to the dice pool.
            If your Strength is negative, this can reduce your damage or healing.
            Likewise, you add half your Willpower in dice increments to your dice pools with \glossterm{magical} abilities.
            For example, if you are using a spell that normally deals 1d8 damage, and your Willpower is 2 or 3, you would deal 1d10 damage with that ability instead.

            Items are an exception to this rule.
            Some items specify their own dice pools, like a \mitem{firebomb} or a \mitem{vampiric} weapon.
            Your Strength and Willpower do not modify the dice pools specified by items.

    % TODO: different section?
    \subsection{Damage Types}\label{Damage Types}
        All damage falls into one of two categories: \glossterm{energy damage} or \glossterm{physical damage}.
        Physical damage is the most common type of damage.
        Energy damage is usually caused by \glossterm{magical} effects.

        \subsubsection{Damage Subtypes}\label{Damage Subtypes}
            Physical damage has three subtypes: bludgeoning damage, piercing damage, and slashing damage.
            Energy damage has five subtypes: acid damage, cold damage, electricity damage, fire damage, and sonic damage.
            Damage of a particular subtype is also considered damage of its primary type.
            For example, if you are \trait{impervious} to \glossterm{physical damage}, that applies against bludgeoning damage because bludgeoning damage is a subtype of \glossterm{physical damage}.

            Some damage types have special properties, as described below.
            \parhead{Cold} Abilities that deal cold damage can freeze liquids and have similar effects appropriate to a sudden drop in temperature.
            \parhead{Electricity} Abilities that deal electricity damage can ignite nonmagical fires if they damage combustible objects.
            \parhead{Fire} Abilities that deal fire damage provide light equivalent to a torch for their duration.
            % TODO: what does this actually mean
            Abilities without a duration create a brief burst of torchlight.
            While underwater, they deal half damage and have no nondamaging effects.

        \subsubsection{Multiple Damage Types}\label{Multiple Damage Types}
            Some attacks deal damage that has multiple damage types.
            Defensive abilities such as defense bonuses or damage immunities apply against an attack only if they apply to all damage types dealt by the attack.

    \subsubsection{Special Damage Types}\label{Special Damage Types}

        These special damage types are separate from the standard damage types, like fire damage or energy damage.

        \subsubsection{Subdual Damage}\label{Subdual Damage}
            Some attacks and environmental effects deal subdual damage.
            Subdual damage works in the same way as normal damage, except it cannot inflict \glossterm{vital wounds}.
            If an attack that deals subdual damage would inflict a vital wound, the target increases its \glossterm{fatigue level} by three instead.
            Whenever you make a \glossterm{strike}, you can choose to deal subdual damage instead of normal damage.
            If you do, you deal half damage with the strike.

        \subsubsection{Environmental Damage}\label{Environmental Damage}
            Some abilities and environmental effects deal environmental damage.
            Environmental damage is never dealt as the result of a successful attack roll.
            Environmental damage works in the same way as normal damage, except that environmental damage is reduced by your \glossterm{damage resistance} without subtracting from its remaining value.
            Any environmental damage in excess of a creature's damage resistance is causes the creature to lose hit points just like normal damage.

            It is possible for damage to be both environmental damage and subdual damage.

\section{Taking Damage}\label{Taking Damage}
    Taking damage from attacks reduces your \glossterm{damage resistance}, and then your \glossterm{hit points}, before finally inflicting \glossterm{vital wounds}.
    To calculate your damage resistance and hit points, see \pcref{Character Statistics}.
    This section explains those concepts in more detail.

    Whenever you take damage, you reduce your \glossterm{damage resistance} by an amount equal to the damage you took.
    Any damage in excess of your remaining damage resistance reduces your \glossterm{hit points}.
    There are two main difference between hit points and damage resistance.
    First, losing \glossterm{hit points} makes you vulnerable to many debilitating debuffs.
    Second, many abilities can restore hit points lost during combat, but damage resistance is extremely difficult to recover without taking a \glossterm{short rest}.

    \subsection{Negative Hit Points}
        You can have negative hit points, but only briefly.
        At the end of each \glossterm{phase}, if your hit points are negative, you gain a \glossterm{vital wound} and your hit points are set to 0 (see \pcref{Vital Wounds}).
        % TODO: does this still need to exist? all "end of round" damage should ideally be removed
        % Damage you take at the end of the round is considered to be part of the delayed action phase for this purpose.
        If your negative hit points exceed half your maximum hit points, you gain an additional vital wound.
        You gain an additional vital wound for each increment of half your maximum hit points that you have in negative hit points.

    \subsection{Resolving Simultaneous Damage}\label{Resolving Simultaneous Damage}
        Many attacks have special effects when they cause you to lose \glossterm{hit points}.
        In addition, some attacks have special effects based on other triggers, like whether you resisted all damage from the attack.
        When you take damage from multiple sources during the same phase, it may not be obvious how to determine whether any individual attack caused you to lose hit points.
        There is a rule for this situation: each attack's special effects are triggered as if all damage you took in the current phase was dealt by that attack.

        For example, assume that you started a phase with some damage resistance remaining.
        You are hit by two attacks: the \spell{organ failure} spell and the \maneuver{strip the armor} maneuver.
        The combined damage from the two attacks exceeds your damage resistance, so you lose some hit points.
        The \spell{organ failure} spell would cause you to be \stunned, because it would be treated as if it made you lose hit points, even if your damage resistance at the start of the phase exceeded the damage dealt by that spell.
        Likewise, the \maneuver{strip the armor} maneuver would not have its bonus effect that triggers if you resist all damage from the attack, because you lost hit points from that attack as well.

        Non-damaging attacks are simpler.
        They always care about your current state at the time the attack is made, regardless of any damage you may take during the phase.

        \parhead{Simultaneous Damage and Healing}\label{Simultaneous Damage and Healing}
        If you regain hit points and take damage in the same phase, apply all healing effects after applying any damaging effects.
        The healing still applies before checking for \glossterm{vital wounds}, so the order is mostly irrelevant, except that this avoids the risk of conflicting with your maximum hit point total.

    \subsection{Noticing Damage Resistance}
        In general, it is impossible to determine whether a creature has damage resistance simply by observing them unless there are obvious visual cues like bleeding injuries.
        However, when a creature takes damage from an attack, an observer can determine the result of the attack with an Awareness check with a base \glossterm{difficulty value} of 10 (before applying the normal modifiers for distance, visibility, and so on).
        The creature dealing the damage gains a \plus10 bonus to this check.
        Success on this check allows an observer to distinguish between the following three possibilities:
        \begin{itemize}
            \item The creature resisted all damage from your attack.
            \item The creature resisted some damage from your attack, but also lost some hit points.
                This means that the target's damage resistance was reduced to 0 during this phase.
            \item The creature did not resist any damage from your attack, and took all damage from the attack from their hit points.
        \end{itemize}
        Most of the time, you can simplify this to simply ``if you attack someone, you know if you beat their damage resistance''.
        The Awareness calculation is useful to handle unusual situations like maximum range longbow shots, where you might not be able to observe the exact result of your attack.

\section{Vital Wounds}\label{Vital Wounds}
    A \glossterm{vital wound} represents serious damage to your body.
    Each \glossterm{vital wound} has a specific detrimental effect on you.
    You gain vital wounds by taking damage in excess of your remaining \glossterm{hit points} (see \pcref{Negative Hit Points}).

    To determine the effect of a \glossterm{vital wound}, make a \glossterm{vital roll} and find the corresponding effect in \trefnp{Vital Wound Effects}.
    The effect of the vital wound lasts until you remove that vital wound.
    The effects of vital wounds stack with each other, even if you roll the same effect twice for different \glossterm{vital wounds}.

    \subsection{Vital Rolls}\label{Vital Rolls}
        To make a \glossterm{vital roll}, roll 1d10 \sub the number of \glossterm{vital wounds} you already had, ignoring the vital wound you are rolling for.
        This includes vital wounds that have no specific vital wound effect.
        The result determines the effect of the \glossterm{vital wound}, as listed in \tref{Vital Wound Effects}.
        Vital wound effects from vital rolls below 1 are lethal if untreated, but the Medicine skill can be used to prevent you from dying (see \pcref{Medicine}).

        \parhead{Delaying Death} Vital wounds with a vital roll below 1 can kill you.
        While you are dying in this way, if you receive healing that causes you to regain hit points, you delay your death by one round.
        This benefit applies even if you are already at full hit points.
        You cannot delay your death in this way by more than 5 rounds.

        \begin{dtable}
            \lcaption{Vital Wound Effects}
            \begin{dtabularx}{\textwidth}{l X}
                \tb{Vital Roll}  & \tb{Effect} \tableheaderrule
                \minus6 or less  & You immediately die                                                              \\
                \minus1--\minus5 & You are unconscious, and you die at the end of the next round                    \\
                0                & You are unconscious, and you die after one minute                                \\
                1                & You are unconscious while you have less than full hit points                     \\
                2                & Your maximum \glossterm{hit points} and \glossterm{damage resistance} are halved \\
                3                & You take a \minus2 penalty to \glossterm{accuracy}                               \\
                4                & You take a \minus2 penalty to all \glossterm{defenses}                           \\
                5                & You take a \minus1 penalty to future \glossterm{vital rolls}                     \\
                6                & You move at half speed while you have less than full hit points                  \\
                7                & Your maximum \glossterm{damage resistance} is halved                             \\
                8                & You take a \minus1 penalty to \glossterm{accuracy}                               \\
                9                & You take a \minus1 penalty to all \glossterm{defenses}                           \\
                10 or more       & No extra vital wound effect                                                      \\
            \end{dtabularx}
        \end{dtable}

    \subsection{Removing Vital Wounds}\label{Removing Vital Wounds}
        Vital wounds take time to heal.
        Whenever you take a \glossterm{long rest}, you remove one of your vital wounds.
        If you have multiple vital wounds, you may choose the order in which your vital wounds are removed.

\section{Making Checks}\label{Checks}\label{Making Checks}
    Checks are required to perform actions that have a chance of failure where the difficulty is not measured by the defense of another creature or object.
    For example, climbing a wall or remembering an obscure piece of trivia may require a check.

    To make a check, roll 1d10 and add your modifier with the check.
    You compare that result to a \glossterm{difficulty value} that represents the difficulty of the task.
    The more difficult the task, the higher the \glossterm{difficulty value} will be.
    If your result is equal to or higher than the \glossterm{difficulty value}, the check succeeds.
    This usually means you accomplish a task successfully.
    Normal Difficulty values are described in \tref{Difficulty Value}.

    \begin{dtable}
        \lcaption{Difficulty Values}
        \begin{dtabularx}{\columnwidth}{p{8em} X}
            \tb{Difficulty Value} & \tb{Example (Skill Used)} \tableheaderrule
            Trivial (0)      & Hear a coversation from 10 feet away (Awareness)                          \\
            Average (5)      & Tie or untie a typical knot (Devices)                                     \\
            Tough (10)       & Swim in rough water (Swim)                                                \\
            Challenging (15) & Balance on a one-inch wide wood beam (Balance)                         \\
            Heroic (20)      & Open a high quality lock (Devices)                                        \\
            Legendary (25)   & Leap across a 30-foot chasm with a running start (Jump)                   \\
            Epic (30)        & Convince a wise mayor her husband is secretly a werewolf (Persuasion)     \\
            Godlike (40)     & Track three orcs across firm ground after 24 hours of rainfall (Survival) \\
        \end{dtabularx}
    \end{dtable}

    \subsection{Critical Success}
        If your check result is at least 10 higher than the \glossterm{difficulty value}, your check is a \glossterm{critical success}.
        Some checks have a special effect on a critical success.
        For example, a critical success while climbing means you move twice as quickly (see \pcref{Climb}).

    \subsection{Critical Failure}
        If your check result is at least 6 lower than the \glossterm{difficulty value}, your check is a \glossterm{critical failure}.
        Some checks have a special effect on a critical failure, which is usually bad for the character making the check.
        For example, a critical failure while climbing means you fall (see \pcref{Climb}).

    \subsection{Types of Checks}
        There are two types of checks: attribute checks and skill checks.
        Your bonus with an attribute check is normally equal to your total value for that attribute.
        Your bonus with a skill check is based on your training with that skill, as well as your value for any relevant attribute (see \pcref{Skills}).

        Some abilities give you bonuses to checks based on a particular attribute, such as ``Strength-based checks''.
        Those bonuses apply to both attribute checks and checks with skills based on those attributes.

    \subsection{Opposed Checks}
        Sometimes, you're competing with another creature, such as when you are trying to hold a door closed and a ferocious ogre is trying to shove it open.
        In that case, you both make a \glossterm{check}, and the creature with the higher result wins.
        This is called an opposed check.
        If you both get the same result, roll again to break the tie.

\section{Resting}\label{Resting}
    When you have a moment to relax, you can rest to regain some of your expended resources.
    There are two main types of rests: a \glossterm{short rest} and a \glossterm{long rest}.
    Resting is not actually an ability in the same sense as most other abilities.
    You do not declare that you are using the ``short rest'' ability, and you do not have to differentiate between whether you intend to take a short rest or a long rest.
    The benefits of taking a short rest or long rest happen automatically after you spend enough time avoiding strenuous activity.
    Resting at night is often combined with sleeping, but you can rest at any time without sleeping.

    % TODO: clarify interrupted rests

    \subsection{Short Rest}\label{Short Rest}
        Resting for ten minutes is considered a \glossterm{short rest}.
        When you take a short rest, you gain the following benefits.
        \begin{itemize}
            \item Your \glossterm{hit points} become equal to your maximum hit points.
            \item Your current \glossterm{damage resistance} becomes equal to your maximum damage resistance.
            \item You regain any \glossterm{attunement points} you released from \glossterm{attuned} effects (see \pcref{Attuned Abilities}).
            \item You remove all \glossterm{conditions} affecting you (unless they cannot be removed normally).
            \item Some other abilities have specific effects that last until you take a short rest.
                For example, a barbarian cannot use their \textit{rage} ability again after raging until after they take a short rest (see Rage, page \pref{Bbn:Rage}).
        \end{itemize}

    \subsection{Long Rest}\label{Long Rest}
        Resting for eight hours is considered a \glossterm{long rest}.
        When you take a long rest, you gain the following benefits.
        \begin{itemize}
            \item You remove one of your vital wounds (see \pcref{Removing Vital Wounds}).
                The Medicine skill can increase this healing (see \pcref{Accelerate Recovery}).
            \item Your \glossterm{fatigue level} becomes 0.
            \item Some other abilities have specific effects that last until you take a long rest.
        \end{itemize}

        You can take multiple long rests consecutively to recover from extensive vital wounds.

\section{Size Categories}\label{Size Categories}
    Your size affects your \glossterm{space} and \glossterm{reach} in combat, your speed with any \glossterm{movement modes} that depend on your size category's \glossterm{base speed}, your attributes, and how noticeable you are (see \pcref{Stealth}).
    These effects are shown on \trefnp{Size Categories}.

    \begin{dtable*}
        \lcaption{Size Categories}
        \begin{dtabularx}{\textwidth}{l l l l l l X}
            \tb{Size}         & \tb{Space}\fn{1} & \tb{Reach}\fn{1} & \tb{Base Speed} & \tb{Weight Limits}\fn{2} & \tb{Reflex Defense} & \tb{Example Creature} \tableheaderrule
            Fine              & 1/4 ft.          & 0                & 5 ft.           & \minus4 Str        & \plus4               & Fly                      \\
            Diminuitive       & 1/2 ft.          & 0                & 10 ft.          & \minus3 Str        & \plus3               & Mouse                    \\
            Tiny              & 1 ft.            & 0                & 15 ft.          & \minus2 Str        & \plus2               & Rat                      \\
            Small             & 2-1/2 ft.        & 5 ft.            & 20 ft.          & \minus1 Str        & \plus1               & Cat                      \\
            Medium            & 5 ft.            & 5 ft.            & 30 ft.          & \tdash             & \tdash               & Human                    \\
            Large (tall)      & 10 ft.           & 10 ft.           & 40 ft.          & \plus1 Str         & \minus1              & Ogre                     \\
            Large (long)      & 10 ft.           & 5 ft.            & 40 ft.          & \plus1 Str         & \minus1              & Horse                    \\
            Huge (tall)       & 20 ft.           & 15 ft.           & 50 ft.          & \plus2 Str         & \minus2              & Cloud giant              \\
            Huge (long)       & 20 ft.           & 10 ft.           & 50 ft.          & \plus2 Str         & \minus2              & Bulette                  \\
            Gargantuan (tall) & 40 ft.           & 40 ft.           & 60 ft.          & \plus3 Str         & \minus3              & 50-ft.\ animated statue  \\
            Gargantuan (long) & 40 ft.           & 20 ft.           & 60 ft.          & \plus3 Str         & \minus3              & Kraken                   \\
            Colossal (tall)   & 80\add ft.       & 80\add ft.       & 80 ft.          & \plus4 Str         & \minus4              & Colossal animated object \\
            Colossal (long)   & 80\add ft.       & 40\add ft.       & 80 ft.          & \plus4 Str         & \minus4              & Great wyrm red dragon    \\
        \end{dtabularx}
        1 Creatures can vary in space and reach.  These are simply typical values. \\
        2 This modifies Strength only for the purpose of determining a creature's \glossterm{weight limits} (see \pcref{Weight Limits}). \\
    \end{dtable*}

    \subsection{Space}\label{Space}
        A creature's \glossterm{space} is the area its body occupies while fighting.
        All humanoid species take up a 5-ft.\ by 5-ft.\ space in combat, which is a single \glossterm{square}.
        Normally, other creatures can't be in the space you occupy.
        Most creatures have a space significantly larger than the physical space their body occupies because they need room to maneuver in combat.

    \subsection{Reach}\label{Reach}
        A creature's \glossterm{reach} is the distance that its \glossterm{melee} attacks can reach.

    \subsection{Base Speed}\label{Base Speed}
        Each size category has a \glossterm{base speed} that indicates how far creatures of that size category can generally move.
        Most \glossterm{movement modes} use a speed equal to the base speed for a creature's size category.
        For details about other speeds, see \pcref{Movement Modes}.

    \subsection{Other Effects}
        % TODO: record all of these and add them here
        A creature's size affects some additional skills and abilities.
        For example, larger creatures have a penalty to the Stealth skill (see \pcref{Size and Stealth}).
        The effects of unusual size are described in those skills and abilities.
        Unusually large or small creatures also have other special rules apply to them, as described below.

    \subsection{Very Small Creatures}
        \parhead{Space} If a creature takes up less than a single square of space, you can fit multiple creatures in that square.
        Ignoring flight, you can fit four Small creatures in a square, twenty-five Tiny creatures, 100 Diminuitive creatures, or 400 Fine creatures.
        If the creatures can fly, the number of creatures that can fit into a space increases drastically.

        \parhead{Reach} Creatures that take up less than 1 square of space typically have a natural reach of 0 feet, meaning they can't reach into adjacent squares. They must enter an opponent's square to attack in melee. You can attack into your own square if you need to, so you can attack such creatures normally.

        If a creature without a natural reach uses a Long weapon, it gains no benefits or penalties (see \pcref{Long Weapon}).

        \parhead{Movement} Creatures two size categories smaller than you are not considered obstacles and do not hinder your movement.

    \subsection{Very Large Creatures}\label{Very Large Creatures}
        \parhead{Space} Very large creatures take up multiple squares. Anything which affects a single square the creature occupies affects the creature.

        \parhead{Reach} Creatures that take up more than 1 square typically have a natural reach of 10 feet or more, meaning that they can reach targets even if they aren't in adjacent squares. Creatures with a large natural reach can attack anyone within their reach, including adjacent foes.

        Creatures with a large natural reach using Long weapons can strike at up to double their natural reach but can't strike at their natural reach or less, just like Medium sized creatures (see \pcref{Long Weapon}).

        \parhead{Movement} Creatures two size categories larger than you are not considered obstacles and do not hinder your movement.

        % Should this be two size categories larger, or be more dependent on creature anatomy?
        \parhead{Immunities} Creatures at least two size categories larger than you are difficult to fight.
        You cannot get a \glossterm{critical hit} with melee \glossterm{strikes} against such creatures.
        If you can reach a vulnerable point on the creature, such as by flying, climbing on the creature, or knocking it prone, you can get critical hits normally.

    \subsection{Weight Categories}\label{Weight Categories}
        Weight is generally measured in \glossterm{weight categories} rather than pounds or kilograms.
        Weight categories use the same terms as \glossterm{size categories}, as shown in \tref{Weight Categories}.
        In general, a creature's weight category is the same as its size category.

        Objects and creatures can also be either \glossterm{lightweight} or \glossterm{heavyweight}.
        Lightweight objects and creatures have a weight category that is one category lighter than their size category.
        Heavyweight objects and creatures have a weight category that is one category heavier than their size category.

        Objects that occupy only a small percentage of the space appropriate for their size category, such as swords, are usually lightweight.
        Objects that fully occupy the space appropriate for their size category, like boulders, are usually heavyweight.

        \begin{dtable}
            \lcaption{Weight Categories}
            \begin{dtabularx}{\textwidth}{l X}
                \tb{Weight Category} & \tb{Average Weight} \tableheaderrule
                Fine        & 1 oz.       \\
                Diminuitive & 1/2 lb.     \\
                Tiny        & 2 lb.       \\
                Small       & 15 lb.      \\
                Medium      & 125 lb.     \\
                Large       & 1,000 lb.   \\
                Huge        & 8,000 lb.   \\
                Gargantuan  & 64,000 lb.  \\
                Colossal    & 512,000 lb. \\
            \end{dtabularx}
        \end{dtable}

\section{Weight Limits}\label{Weight Limits}

    \begin{dtable}
        \lcaption{Weight Limits by Strength}
        \setlength{\tabcolsep}{4pt}
        \begin{dtabularx}{\columnwidth}{X X X}
            \tb{Strength} & \tb{Carrying Capacity} & \tb{Push/Drag} \tableheaderrule
            -9            & Fine x8                & Tiny          \\
            -8            & Diminuitive x2         & Tiny x2       \\
            -7            & Diminuitive x4         & Tiny x4       \\
            -6            & Diminuitive x8         & Tiny x8       \\
            -5            & Tiny x2                & Small x2      \\
            -4            & Tiny x4                & Small x4      \\
            -3            & Tiny x8                & Small x8      \\
            -2            & Small x2               & Medium x2     \\
            -1            & Small x4               & Medium x4     \\
            0             & Small x8               & Medium x8     \\
            1             & Medium x2              & Large x2      \\
            2             & Medium x4              & Large x4      \\
            3             & Medium x8              & Large x8      \\
            4             & Large x2               & Huge x2       \\
            5             & Large x4               & Huge x4       \\
            6             & Large x8               & Huge x8       \\
            7             & Huge x2                & Gargantuan x2 \\
            8             & Huge x4                & Gargantuan x4 \\
            9             & Huge x8                & Gargantuan x8 \\
            10            & Gargantuan x2          & Colossal x2   \\
            11            & Gargantuan x4          & Colossal x4   \\
            12            & Gargantuan x8          & Colossal x8   \\
            13            & Colossal x2            & Colossal x16  \\
            14\plus\fn{1} & \tdash                 & \tdash        \\
        \end{dtabularx}
        1 To calculate the weight limits for a creature with epic Strength, double the number of objects it can carry and drag for every point of Strength beyond 13.
    \end{dtable}

    Your Strength determines how much you can carry or push, as shown in \trefnp{Weight Limits by Strength}.
    Your weight limits are measured in terms of how many objects or creatures of a given \glossterm{weight category} that you can carry or push at once.
    Instead of carrying one object of a given weight category, you can carry eight objects that are one weight category lighter.
    In general, it is not meaningful to consider the weight of any objects with a weight category lighter than your maximum weight category.

    You can carry objects or creatures up to your maximum carrying capacity without any penalty.
    Beyond that, you can push or drag objects or creatures up your pushing and dragging limit as a standard action.
    When you do, you move the weight 5 feet.

    \parhead{Large Creatures} Unusually large or small creatures gain a bonus to their Strength for the purpose of determining their weight limits.
    For details, see \tref{Size Categories}.

    \parhead{Multi-Legged Creatures} The figures on \trefnp{Weight Limits by Strength} are for bipedal creatures.
    A creature with four or more legs can carry, push, or drag twice as many objects as a bipedal creature of the same Strength.

\section{Communication and Languages}\label{Languages}\label{Communication and Languages}

    \parhead{Literacy}
    All characters with an Intelligence of \minus2 or higher are presumed to be literate, allowing them to read and write any language they speak. Each language has an alphabet, though sometimes several spoken languages share a single alphabet.

    \parhead{Language Rarity}\label{Language Rarity}
    Some languages are widely spoken in the world, while others are only encountered in unusual circumstances.
    Common languages are summarized on \trefnp{Common Languages}, below.
    Rare languages are summarized on \trefnp{Rare Languages}, below.
    Rare languages are more difficult to learn, and are usually only spoken by unusual creatures.

    \parhead{Learning Languages}\label{Learning Languages}
    You can spend one \glossterm{insight point} to learn two \glossterm{common languages} or one \glossterm{rare language}.
    In addition, you can learn two common languages or one rare language by mastering the Linguistics skill (see \pcref{Linguistics}).

    \begin{dtable}
        \lcaption{Common Languages}
        \begin{dtabularx}{\columnwidth}{l >{\lcol}X l}
            \tb{Language} & \tb{Typical Speakers} & \tb{Alphabet} \tableheaderrule
            Common        & Civilized creatures   & Common   \\
            Draconic      & Dragons, kobolds      & Draconic \\
            Dwarven       & Dwarves               & Dwarven  \\
            Elven         & Elves                 & Elven    \\
            Giant         & Ogres, giants         & Dwarven  \\
            Gnoll         & Gnolls                & Common   \\
            Gnome         & Gnomes                & Dwarven  \\
            Goblin        & Goblins, hobgoblins   & Dwarven  \\
            Halfling      & Halflings             & Common   \\
            Orc           & Orcs                  & Dwarven  \\
        \end{dtabularx}
    \end{dtable}

    \begin{dtable}
        \lcaption{Rare Languages}
        \begin{dtabularx}{\columnwidth}{l >{\lcol}X l}
            \tb{Language}  & \tb{Typical Speakers}  & \tb{Alphabet} \tableheaderrule
            Abyssal     & Evil planeforged      & Abyssal  \\
            Aquan       & Water-based creatures & Elemental \\
            Auran       & Air-based creatures   & Elemental \\
            Celestial   & Good planeforged      & Celestial \\
            Ignan       & Fire-based creatures  & Elemental \\
            Sylvan      & Dryads, faeries       & Elven     \\
            Terran      & Earth-based creatures & Elemental \\
            Undercommon & Drow                  & Elven
        \end{dtabularx}
    \end{dtable}

    \subsection{Telepathy}\label{Telepathy}
        Some creatures have the ability to telepathically communicate with other creatures.
        All telepathy abilities have a defined \glossterm{range}.
        Unless otherwise specified, a telepathic creature can only communicate with one creature at a time.

        As a \glossterm{free action}, a telepathic creature can open a telepathic communication channel with one creature it sees within the range of its telepathy ability.
        The target does not have to be willing to receive telepathic communication in this way.
        While this channel is open, the telepathic creature can cause the target to ``hear'' the telepathic creature's voice inside the target's head.
        If the target attempts to mentally reply while the channel is open, the telepathic creature can similarly ``hear'' the reply in its head as if the target was speaking.
        This does not generally grant the ability to detect any other thoughts, though exceptionally stupid targets may accidentally broadcast their private thoughts.

        Telepathic communication uses words, so it still requires a shared language to be intelligible, even though the words are only imagined.
        A telepathic creature may attempt to telepathically communicate with creatures without a language, though this is generally unproductive.
        A skilled telepath can customize the mental ``voice'' it projects in the same way that a creature can attempt to disguise or alter its voice when speaking.

% Is this the right location?
\section{Planes}\label{Planes}
    The universe of Rise is divided into \glossterm{planes}.
    A plane is a distinct realm of existence.
    Except for the connections between planes through \glossterm{planar rifts}, each plane is effectively an isolated universe, and different planes can obey different fundamental laws.
    For example, the Material Plane has gravity that exerts a consistent acceleration in a single absolute direction.
    However, the Astral Plane has subjective gravity, where each creature on the plane chooses the direction that gravity pulls it in, if any.

    \subsection{General Cosmology}
        The planes of Rise are divided up into groups.

        \parhead{Primal Planes} The primal planes are manifestations of the basic building blocks of the universe.
        Each plane in this group is predominantly composed of a single element or type of energy.
        There are four primal planes: Air, Earth, Fire, and Water.

        \parhead{Aligned Planes} The aligned planes are manifestations of the nine alignments that define the morality of the universe.
        Each plane in this group is strongly associated with a particular alignment.
        The souls of creatures with the corresponding alignment often spend their afterlife in the Aligned Planes.
        There are nine aligned planes, one for each alignment combination (see \pcref{Alignment}).

        \parhead{Nexus Planes} The nexus planes are composite planes with a number of distinct environments and filled with creatures of myriad alignments.
        Nexus planes comprise the majority of civilization across all planes.
        They do not have their own unique planar essence, and no planar creatures are native to nexus planes.
        There are two nexus planes: the Material Plane and the Astral Plane.

        \parhead{Demiplanes} These planes are small, fragmentary realms that are greatly limited in their scope.
        There is no specific list of demiplanes, and they share few common properties.
        Most demiplanes were created for particular purposes by beings of great power, though some simply came into existence through unknown means.

    \subsection{Planar Rifts}\label{Planar Rifts}
        Normally, there are boundaries between different planes that prevent direct passage between them.
        However, \glossterm{planar rifts} are places where these boundaries have weakened, making interplanar travel easier.
        A planar rift joins a specific location on one plane to a specific location on a different plane.
        Most planar rifts lead to and from the Astral Plane, which is the space between the other planes (see \pcref{The Astral Plane}).

        Most planar rifts still require the use of magic, such as the \spell{plane shift} ritual, to actually cross between planes.
        Some especially large rifts enable physical travel between planes without the use of any magic.

    \subsection{Planar Traits}
        \subsubsection{Gravity Direction}
            The direction of gravity on a plane can take one of the following forms:
            \begin{itemize}
                \item Fixed Gravity: Gravity points in a fixed direction and with a fixed strength at all locations on the plane.
                    Almost all planes with a fixed gravity have a perfectly flat surface.
                \item Absolute Directional Gravity: Gravity points in a consistent direction according to a rule that applies equally to everything on the plane, but which is not in a fixed direction.
                    For example, a plane filled with floating spheres where gravity always points towards the closest sphere has absolute directional gravity.
                \item Subjective Gravity: Each creature on the plane chooses the direction of gravity for that creature.
                    The plane has no gravity for unattended objects and nonsentient creatures.
                    A creature on the plane can make use the \textit{control gravity} ability as a \glossterm{minor action}.
                    \begin{freeability}{Control Gravity}
                        Make a Willpower check with a \glossterm{difficulty value} of 10.
                        Success means that you choose the direction of gravity that applies to you on the current plane.
                        Alternately, you can choose for gravity to not apply to you.

                        Failure means you gain a \plus2 bonus to the next \textit{control gravity} ability you use on this plane.
                        This bonus stacks with itself and lasts until you succeed at a \textit{control gravity} ability on this plane.
                    \end{freeability}
            \end{itemize}

        \subsubsection{Gravity Strength} The strength of gravity on a plane can take one of the following forms:
            \begin{itemize}
                \item Normal Gravity: Gravity is about the strength of Earth.
                \item No Gravity: There is no gravity on the plane.
                    The \glossterm{range limits} of ranged weapons are quadrupled.
                    % TODO: what additional effects are there?
                \item Light Gravity: Gravity is about half the strength of Earth.
                    % Should there be check penalties?
                    The weight of all items is halved.
                    The \glossterm{range limits} of ranged weapons are doubled.
                \item Heavy Gravity: Gravity is about twice the strength of Earth.
                    Creatures take a \minus2 penalty to Strength and Dexterity-based checks.
                    The weight of all items is doubled.
                    The \glossterm{range limits} of ranged weapons are halved, to a minimum of 5 feet.
                    % TODO: falling damage
                \item Extreme Gravity: Gravity is about four times the strength of Earth.
                    Creatures take a \minus4 penalty to Strength and Dexterity-based checks.
                    The weight of all items is quadrupled.
                    The \glossterm{range limits} of ranged weapons are reduced to one quarter of the normal value, to a minimum of 5 feet.
                    % TODO: falling damage
            \end{itemize}

        \subsubsection{Light} Various planes are illuminated in different ways.
            \begin{itemize}
                \item Fixed Source: There is a single constant source of light on the plane.
                \item Mobile Source: There is a single source of light on the plane that moves around it, illuminating different parts of the plane at different times.
                \item None: There is no natural source of light on the plane.
                    Other sources of light, such as torches, function normal.
            \end{itemize}

        \subsubsection{Limits} The behavior of a plane at its limits can vary widely.
            Some planes have different behaviors at different limits depending on their shape.
            \begin{itemize}
                \item Astral Gate: If you reach the limits of the plane, you find a planar gate to the Astral Plane.
                \item Barrier: If you reach the limits of the plane, you find an impassable barrier.
                    The barrier takes the form of a substance relevant to the plane's nature.
                    It may be possible to dig tunnels into the barrier to some depth, but there is nothing behind the barrier.
                    As you progress past the limit, the barrier becomes increasingly difficult to break through, and eventually it becomes completely impenetrable.
                \item Looped: If you go beyond the limits of the plane, you wrap around to the opposite side of the plane.
                    There is no obvious transition point or perception of transportation when this occurs - the shape of the plane simply connects to itself.
                    On very small planes, this can allow you to see your own back, though looped planes of that size are rare.
                \item Infinite: The plane has no limits. This is extremely rare.
            \end{itemize}

        \subsubsection{Planar Connectivity}
            Different planes have different degrees of connection to other planes.
            \begin{itemize}
                \item Isolated: The plane is difficult to reach or leave.
                    It has no permanent \glossterm{planar rifts}, and temporary rifts are rare or nonexistent.
                \item Stable Connected: The plane has multiple permanent \glossterm{planar rifts}.
                    However, temporary rifts are rare.
                \item Unstable Connected: The plane has no permanent \glossterm{planar rifts}, but temporary rifts are common.
                \item Conduit: The plane has a large number of permanent \glossterm{planar rifts}, and temporary rifts are common.
            \end{itemize}

        \subsubsection{Shape} The shape of a plane defines the shape of its core surface, and what happens if you travel beyond that surface.

            \begin{itemize}
                \item Flat Surface: The plane consists of a flat surface generally made of earth or similar material.
                    Most activity and civilization on the plane happens on this surface.
                    It is usually possible to construct tunnels into a flat surface plane to some depth, depending on the size of the plane.
                \item Hollow Sphere: The plane consists of a hollow sphere with an outer boundary generally made of earth or similar material.
                    Most activity and civilization on the plane happens on the inner surface of the sphere.
                    Some hollow sphere planes have an outer surface that can also be accessed, but most have a limit before any outer surface can be reached.
                \item Solid Sphere: The plane consists of a solid sphere generally made of earth or similar material.
                    Most activity and civilization on the plane happens on the surface of the sphere.
                    It is possible to construct tunnels into a solid sphere plane, but it may become increasingly difficult to traverse the plane as you approach the center of the sphere.
                    In general, the limit of a solid sphere plane is located at ten times the radius of the plane's primary sphere.
                \item Uniform: The plane has no well-defined surface or ground layer.
                    Some uniform planes have no ground or solid obstacles, while others are composed almost entirely of ground and firmament.
                    Uniform planes almost always still have limits of some kind.
            \end{itemize}

            % TODO: morphicness, alignment, magic

    \subsection{Plane Descriptions}

        \subsubsection{Primal Planes}

            \parhead{The Plane of Air}
            The Plane of Air is a a soaring landscape unencumbered by gravity or ground.
            The vast expanses of empty air are littered with clouds and unpredictable winds.
            Any inhabitants of the plane must adapt to a highly mobile lifestyle.
            A number of towns and structures have been built in the plane using raw materials brought from other planes.
            They sail through the air at the whims of the wind, and are occasionally battered by intersections with other wind streams.

            The Plane of Air has the following planar traits:
            \begin{itemize}
                \item Gravity strength: No Gravity
                \item Light: Fixed Source, from a sun outside the limits of the plane
                \item Limits: Looped
                \item Planar connectivity: Unstable Connected
                \item Shape: Uniform, in a sphere with a radius of about 2,000 miles.
            \end{itemize}

            \parhead{The Plane of Earth}
            The Plane of Earth is a titanically large body of earth and stone.
            A labyrinthine series of mostly airless tunnels weave their way through the plane, connecting the few cities.
            The plane is a major source of valuable gems, diamonds, and rare metals like mithral, but the dense rock and airless environment make successful mining difficult.
            In addition, earthquakes periodically reshape the environment by collapsing old tunnel systems and constructing new ones.
            Some cities have been carved out in vast underground rooms reinforced to survive the earthquakes.

            The Plane of Earth has the following planar traits:
            \begin{itemize}
                \item Gravity direction: Fixed
                \item Gravity strength: Normal
                \item Light: None
                \item Limits: Barrier, formed from increasingly dense rock that eventually becomes so hard that no known material or magic can damage it.
                \item Planar connectivity: Stable Connected
                \item Shape: Uniform, in a sphere with a radius of about 500 miles.
            \end{itemize}

            \parhead{The Plane of Fire}
            The Plane of Fire is an endless searing inferno.
            The plane's essence is highly combustible, allowing fires to burn indefinitely without any obvious fuel.
            However, the intensity of flames on the plane are highly uneven, as the plane generates fuel in various locations that shift over time.
            Some pockets on the surface are devoid of natural fuel, allowing the allow the construction of trading hubs where the few inhabitants of the plane who are not naturally immune to fire can survive.
            A variety of large tunnels and magma flows run through the sphere, and the intensity of the heat generally increases as you approach the center.

            The Plane of Fire has the following planar traits:
            \begin{itemize}
                \item Gravity direction: Absolute Directional, pointing to the center of the sphere
                \item Gravity strength: Normal
                \item Light: None, though the constant fires provide sufficient illumination in most locations on the plane
                \item Limits: Looped
                \item Planar connectivity: Unstable Connected
                \item Shape: Solid Sphere, with a radius of about 1,000 miles.
            \end{itemize}

            \parhead{The Plane of Water}
            The Plane of Water is an impossibly vast ocean.
            Powerful currents sweep through the ocean, but much of it is calm, and many forms of aquatic life abound in the water.
            The Plane of Water is the most densely populated Primal Plane, both by sentient creatures and monsters.
            Magnificant underwater cities are carved from huge rocks that float peacefully suspended in the water.
            Though there is no sun, simple creatures akin to plankton form the base of the food chain by feeding directly on the plane's essence.

            The Plane of Earth has the following planar traits:
            \begin{itemize}
                \item Gravity strength: No Gravity
                \item Light: None, though bioluminescent creatures like plankton are extremely common, making many parts of the plane well-lit
                \item Limits: Looped
                \item Planar connectivity: Stable Connected
                \item Shape: Uniform, in a sphere with a radius of about 1,000 miles.
            \end{itemize}

        \subsubsection{Nexus Planes}

            \parhead{The Material Plane}\label{The Material Plane}
            The Material Plane is the plane that most Rise adventures begin on.
            The surface of the plane is a massive sphere with a radius of about 4,000 miles.
            It is the most familiar to most humanoid creatures.

            The Material Plane has the following planar traits:
            \begin{itemize}
                \item Gravity direction: Absolute Directional, pointing to the center of the sphere
                \item Gravity strength: Normal
                \item Light: Mobile Source, from a sun and moon outside of the plane's limits
                \item Limits: Looped
                \item Planar connectivity: Isolated
                \item Shape: Solid Sphere, with a radius of about 4,000 miles.
            \end{itemize}

        \parhead{The Astral Plane}\label{The Astral Plane}
            The Astral Plane is the space between the other planes.
            It is a necessary intermediate destination for virtually all planar journeys, as all \glossterm{planar rifts} lead to and from the Astral Plane.
            Most activity on the Astral Plane occurs in a space called the Inner Astral Plane, a massive but finite region where all planar rifts on the Astral Plane appear.
            % Are there any other infinite planes?
            However, unlike all other planes, the Astral Plane has no known limits to its extent, and may in fact be infinite.
            The area outside the Inner Astral Plane is known as the Deep Astral Plane, and few venture into those sparsely populated realms.
            % This should be more clearly defined
            The Deep Astral Plane has magical turbulence that interferes with long-range communication and transportation magic, making exploration difficult.

            The Astral Plane has the following planar traits:
            \begin{itemize}
                \item Directional gravity: Subjective
                \item Gravity strength: Normal
                \item Light: Fixed Source, from the infinite reaches of the Deep Astral Plane
                \item Limits: Infinite
                \item Planar connectivity: Conduit
                \item Shape: Uniform
            \end{itemize}

\section{Creatures and Objects}
    In the world of Rise, creatures and objects are meaningfully different.
    Many abilities affect only creatures or only objects.
    The difference between a creature and an object is defined as being agency.
    Creatures have agency, and objects do not.
    This is not the same as sentience or life, which either creatures or objects may have.

    For example, zombies are nonsapient, nonliving creatures.
    Conversely, trees are a nonsapient, living objects.
    Some rare magic items can be made intelligent by magic, making them sapient, nonliving objects.
    Some unintelligent animals and magical beasts like ants and giant spiders are nonsapient, living creatures.

    \subsection{Animates}
        One type of entity in the world is both an object and a creature.
        Animates are a type of creature that are made of nonsapient matter given a semblance of life and sentience by some form of magic.
        Fire elementals, clay golems, and plant creatures like treants are all animates.
        Animates are considered to be both creatures and objects, and are affected fully by abilities that affect both.
