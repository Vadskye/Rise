
\small
\subsection{Arcane Magic}\label{Arcane Magic}
\subsubsection{Arcane Spells}\label{Arcane Spells}
\begin{spelllist}
\spellhead{Astromancy} Transport creatures through the Astral Plane.
\spellhead{Barrier} Shield allies from hostile forces.
\spellhead{Chronomancy} Manipulate the passage of time to inhibit foes and aid allies.
\spellhead{Compel} Bend creatures to your will by controlling their actions.
\spellhead{Corruption} Weaken the life force of foes, reducing their combat prowess.
\spellhead{Cryomancy} Drain heat to injure and freeze foes.
\spellhead{Delusion} Instill false emotions to influence creatures.
\spellhead{Electromancy} Create electricity to injure and stun foes.
\spellhead{Fabrication} Create objects to damage and impair foes.
\spellhead{Glamer} Change how creatures and objects are perceived.
\spellhead{Photomancy} Create bright light to blind foes and illuminate your surroundings.
\spellhead{Polymorph} Change the physical forms of objects and creatures.
\spellhead{Pyromancy} Create fire to incinerate foes.
\spellhead{Revelation} Share visions of the present and future, granting insight or combat prowess.
\spellhead{Scry} See and hear at great distances.
\spellhead{Summon} Summon creatures to fight with you.
\spellhead{Telekinesis} Manipulate creatures and objects at a distance.
\spellhead{Thaumaturgy} Suppress and manipulate magical effects.
\spellhead{Weaponcraft} Create and manipulate weapons to attack foes.
\end{spelllist}
\subsubsection{Arcane Rituals}\label{Arcane Rituals}
\begin{spelllist}
\spellhead[1]{Endure Elements} TODO
\spellhead[1]{Light} TODO
\spellhead[1]{Magic Mouth} TODO
\spellhead[1]{Purify Sustenance} TODO
\spellhead[1]{Read Magic} TODO
\spellhead[2]{Gentle Repose} TODO
\spellhead[2]{Mystic Lock} TODO
\spellhead[2]{Seek Legacy} TODO
\spellhead[2]{Water Breathing} TODO
\spellhead[3]{Animate Dead} TODO
\spellhead[3]{Binding} TODO
\spellhead[3]{Create Object} TODO
\spellhead[3]{Create Sustenance} TODO
\spellhead[3]{Explosive Runes} TODO
\spellhead[3]{Mount} TODO
\spellhead[3]{Purge Curse} TODO
\spellhead[3]{Scryward} TODO
\spellhead[4]{Retrieve Legacy} TODO
\spellhead[4]{Sending} TODO
\spellhead[4]{Telepathic Bond} TODO
\spellhead[5]{Discern Location} TODO
\spellhead[5]{Overland Teleportation} TODO
\spellhead[5]{Plane Shift} TODO
\spellhead[5]{Private Sanctum} TODO
\spellhead[8]{Soul Bind} TODO
\spellhead[9]{Gate} TODO
\end{spelllist}




\small
\subsection{Divine Magic}\label{Divine Magic}
\subsubsection{Divine Spells}\label{Divine Spells}
\begin{spelllist}
\spellhead{Bless} Grant divine blessings to aid allies and improve combat prowess.
\spellhead{Compel} Bend creatures to your will by controlling their actions.
\spellhead{Corruption} Weaken the life force of foes, reducing their combat prowess.
\spellhead{Divine Judgment} Smite foes with divine power.
\spellhead{Photomancy} Create bright light to blind foes and illuminate your surroundings.
\spellhead{Revelation} Share visions of the present and future, granting insight or combat prowess.
\spellhead{Scry} See and hear at great distances.
\spellhead{Summon} Summon creatures to fight with you.
\spellhead{Thaumaturgy} Suppress and manipulate magical effects.
\spellhead{Vital Surge} Alter life energy to cure or inflict wounds.
\end{spelllist}
\subsubsection{Divine Rituals}\label{Divine Rituals}
\begin{spelllist}
\spellhead[1]{Bless Water} TODO
\spellhead[1]{Curse Water} TODO
\spellhead[1]{Endure Elements} TODO
\spellhead[1]{Light} TODO
\spellhead[1]{Purify Sustenance} TODO
\spellhead[1]{Read Magic} TODO
\spellhead[2]{Gentle Repose} TODO
\spellhead[2]{Mystic Lock} TODO
\spellhead[2]{Seek Legacy} TODO
\spellhead[2]{Water Breathing} TODO
\spellhead[3]{Animate Dead} TODO
\spellhead[3]{Binding} TODO
\spellhead[3]{Create Object} TODO
\spellhead[3]{Create Sustenance} TODO
\spellhead[3]{Purge Curse} TODO
\spellhead[3]{Scryward} TODO
\spellhead[4]{Regeneration} TODO
\spellhead[4]{Retrieve Legacy} TODO
\spellhead[4]{Sending} TODO
\spellhead[5]{Discern Location} TODO
\spellhead[5]{Plane Shift} TODO
\spellhead[8]{Soul Bind} TODO
\spellhead[9]{Gate} TODO
\end{spelllist}




\small
\subsection{Nature Magic}\label{Nature Magic}
\subsubsection{Nature Spells}\label{Nature Spells}
\begin{spelllist}
\spellhead{Aeromancy} Command air to protect allies and blast foes.
\spellhead{Aquamancy} Command water to crush and drown foes.
\spellhead{Corruption} Weaken the life force of foes, reducing their combat prowess.
\spellhead{Cryomancy} Drain heat to injure and freeze foes.
\spellhead{Electromancy} Create electricity to injure and stun foes.
\spellhead{Photomancy} Create bright light to blind foes and illuminate your surroundings.
\spellhead{Polymorph} Change the physical forms of objects and creatures.
\spellhead{Pyromancy} Create fire to incinerate foes.
\spellhead{Revelation} Share visions of the present and future, granting insight or combat prowess.
\spellhead{Scry} See and hear at great distances.
\spellhead{Summon} Summon creatures to fight with you.
\spellhead{Thaumaturgy} Suppress and manipulate magical effects.
\spellhead{Vital Surge} Alter life energy to cure or inflict wounds.
\end{spelllist}
\subsubsection{Nature Rituals}\label{Nature Rituals}
\begin{spelllist}
\spellhead[1]{Endure Elements} TODO
\spellhead[1]{Light} TODO
\spellhead[1]{Purify Sustenance} TODO
\spellhead[1]{Read Magic} TODO
\spellhead[2]{Gentle Repose} TODO
\spellhead[2]{Mystic Lock} TODO
\spellhead[2]{Seek Legacy} TODO
\spellhead[2]{Water Breathing} TODO
\spellhead[3]{Create Object} TODO
\spellhead[3]{Create Sustenance} TODO
\spellhead[3]{Fertility} TODO
\spellhead[3]{Infertility} TODO
\spellhead[3]{Ironwood} TODO
\spellhead[3]{Purge Curse} TODO
\spellhead[3]{Scryward} TODO
\spellhead[4]{Regeneration} TODO
\spellhead[4]{Resurrection} TODO
\spellhead[4]{Retrieve Legacy} TODO
\spellhead[4]{Sending} TODO
\spellhead[5]{Discern Location} TODO
\spellhead[5]{Lifeweb Transit} TODO
\spellhead[5]{Plane Shift} TODO
\spellhead[5]{Reincarnation} TODO
\spellhead[7]{Awaken} TODO
\spellhead[9]{Gate} TODO
\end{spelllist}

