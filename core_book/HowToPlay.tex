\chapter{How to Play}

\section{Saying What Your Character Does}
  There are two basic modes that you can use to describe how your character acts.
  You can describe in general terms what you want them to do, and let the GM figure out how to translate that into game mechanics.
  Alternately, you can say that you're using a specific ability or game mechanic, and let the GM figure out how that affects the narrative universe.

  Either approach is generally reasonable.
  Some people tend to prefer using one mode more often, and some GMs generally prefer to hear one mode.
  When in doubt, communicate at your table!

  \subsection{Describing Actions}
    With this style of communication, you describe what you want your character to do.
    For example, you can say that your character steps out of their room in the inn and walks over to knock on a friend's door.
    Although Rise has rules that could govern some aspects of that scenario, such as an Awareness \glossterm{check} to see if your friend notices you knocking, you wouldn't usually reference those rules explicitly.
    Even in the unlikely scenario that your friend doesn't notice you knock the first time, you can just knock again, so there's no point in worrying about the details.
    If something seems reasonable, it probably is, and you don't need to worry about the fiddly bits.

    Sometimes, when you describe what your character tries to do, the action has a narratively relevant chance of failure.
    Instead of knocking on the door to say hi, you might only have time to bang on it once to warn your sleeping friend about an attack from assassins.
    In that case, there's some chance that your friend is sleeping too deeply to notice the noise the first time you knock.
    You could try knocking again, just like in the first scenario, but in this scenario that failure would cost you valuable time to survive the attack.
    In that scenario, you would roll a die to determine whether you succeed in your action - or in this case, whether your friend would succeed in their attempt to notice you.

  \subsection{Using Specific Abilities}
    Instead of describing broadly what you want to have happen, you might choose one of a list of clearly defined abilities that your character can use.
    Every character has specific abilities unique to them, such as a wizard's spells or a fighter's maneuvers.
    There are also a number of simple abilities that anyone can use, such as the \ability{grapple} or \ability{trip} abilities.
    These universal abilities attempt to adequately describe a wide variety of reasonable improvised actions that you might try to use in combat.

    Explicitly defined abilities have rules for determining what happens when you use them.
    Some abilities, such as attacks in combat, require rolling dice to determine how effective they are.
    You can use your character's abilities at any time, not just in combat.
    Abilities such as the \spell{create water} or \spell{distant hand} spells can be used to solve other kinds of problems entirely.

\section{Rolling Dice}
  When you need to determine whether something succeeds or fails, you roll a die.
  This can happen as part of using a specific ability that tells you exactly what to roll, or because you tried to narrate your character taking an action that has a dramatically relevant chance of failure.
  In either case, you'll roll a single ten-sided die, also known as 1d10.
  You'll add some modifier that represents how skilled your character is at the particular thing that they are trying to do.
  At the GM's discretion, they may also give the roll an extra bonus or penalty based on the circumstances that your character is in.
  If your die roll is high enough, your character succeeds at whatever they were trying to do.
  Otherwise, your character fails, which may sometimes have additional consequences.

  In Rise, it's entirely possible for characters to be so skilled that they succeed at what they are trying to do even if you roll a 1.
  Likewise, there are tasks that are so obviously impossible for your character that they cannot possibly succeed.
  In those cases, there's no reason to roll!
  Of course, the GM is the final arbiter of whether rolling is necessary.
  They may have information that the players do not.

  In some cases, you roll multiple dice at once.
  This generally happens when you deal damage in combat.
  A collection of dice is called a \glossterm{dice pool}.
  Dice pools are written with the number of dice, followed by ``d'', followed by the size of dice to roll.
  For example, 2d6 means you roll two six-sided dice.

\section{Making Checks}\label{Checks}\label{Making Checks}
  A \glossterm{check} is a roll that you make to try to accomplish a task.
  For example, climbing a wall or remembering an obscure piece of trivia may require a check.
  Unlike an \glossterm{attack}, the difficulty of a check is not measured by the defense of another creature or object.
  Instead, it depends on the task itself.

  To make a check, roll 1d10 and add your modifier with the check.
  You compare that result to a \glossterm{difficulty value} that represents the difficulty of the task.
  The more difficult the task, the higher the \glossterm{difficulty value} will be.
  If your result is equal to or higher than the \glossterm{difficulty value}, the check succeeds.
  This usually means you accomplish a task successfully.
  Otherwise, the check fails.
  This usually means that nothing happens, though sometimes there are specific consequences for failure.

  \subsection{Critical Success}
    If your check result is at least 10 higher than the \glossterm{difficulty value}, your check is a \glossterm{critical success}.
    Some checks have a special effect on a critical success.
    For example, a critical success while climbing means you move twice as quickly.

  \subsection{Standard Difficulty Values}\label{Standard Difficulty Values}
    Most checks are made against a fixed \glossterm{difficulty value} that represents how hard the task is.
    Detailed rules for determining difficulty values in specific circumstances can be found in the Expanded Skills chapter from the Tome of Guidance.
    However, most of the time, it's not worth the effort to consult charts and tables to figure out how hard a task is.
    Instead, you can estimate it based on the guidelines below.

    \begin{raggeditemize}
      \item Easy (DV 0): Only an exceptionally incompetent or impaired person could possibly fail a DV 0 check. For example, this includes walking on rough ground without tripping (Balance) or noticing that a yelling, red-faced person is angry (Social Insight).
      \item Average (DV 5): A typical human with no relevant skills should still succeed at a DV 5 check without much issue. However, it would be possible to fail in a stressful situation where time is limited if the person had no relevant training. For example, this includes climbing a ladder (Climb) or hearing the topic of a nearby conversation in a crowded bar (Awareness).
      \item Hard (DV 10): A typical human with no relevant skills might succeed at a DV 10 check, but only if they were very lucky or had a lot of time on their hands. An experienced practicioner might fail infrequently in stressful circumstances, but a world-class expert would never fail. For example, this includes swimming in fast-moving water (Swim) or providing first aid to mitigate a barely lethal wound (Medicine).
      \item Very Hard (DV 15): Only an experienced practicioner could succeed at a DV 15 check, and they would still need to get lucky if they were in a rush. Even a world-class expert at the peak of real-world human potential could fail, but only rarely. For example, this includes picking a well-made lock (Devices) or holding your breath for eight minutes while staying still (Endurance).
      \item Almost Impossible (DV 20): A world-class expert like an Olympic medalist could succeed at a DV 20 check if they were lucky or patient. Succeeding consistently at tasks of this difficulty requires superhuman capabilities. For example, this includes climbing a weathered natural rock wall without equipment (Climb) or squeezing through a space with a diameter of only half a foot (Flexibility).
      \item Impossible (DV 25\add): No real-world human can succeed at a DV 25 check. This sort of feat is only possible for high-level Rise characters who have explicitly surpassed ordinary limitations. For example, this includes running at full speed along a slack rope (Balance) or climbing a sheer glass pane (Climb).
    \end{raggeditemize}

  \subsection{Trying Again}
    You can think of checks as being broadly divided into two categories: checks that give you information, and checks that cause a change in the world around you.
    In general, you can retry checks that change your environment indefinitely until you succeed.
    The only major limiting factor to those checks is that failure sometimes also changes your environment in ways that may punish your failure or make it impossible to retry the check.
    For example, if you are trying to climb a cliff, you can keep trying until you succeed, but you may take \glossterm{falling damage} from falling off while halfway up the cliff.

    You generally cannot retry checks that give you information unless the situation changes in a way that is relevant to your check.
    This generally means that you must learn new information before making the check again.
    For example, if you've already examined a creature to determine whether they are disguised, you can't keep just keep staring that creature to make sure.
    However, if you splash the creature with water which washes away some makeup, you can try again now that you have more information.

    % TODO: it would be nice if this wasn't necessary
    In addition, checks that require a free action to make can never be made more than once for the same purpose within a round.

  \subsection{Opposed Checks}
    An opposed check involves multiple creatures competing to get the highest result.
    In case of a tie, all tied creatures roll again to break the tie.
    Usually, the creature with the highest result succeeds, while all other creatures either fail completely or simply succeed less effectively depending on the situation.

    Some opposed checks involve multiple creatures using the same skill to see who does the best job.
    For example, a climbing race up a wall might involve each participant rolling a Climb check, or you might make a Strength check to hold a door closed while another creature tries to shove it open.
    Alternately, it can involve creatures rolling opposite skills.
    For example, if you are trying to hide, you roll a Stealth check opposed by the Awareness check of any creatures who could notice you.

    Not all opposed checks require all participants to roll at the same time.
    For example, a creature who creates a disguise rolls the Disguise check at the time that the disguise is created.
    A creature who tries to notice the disguise would roll their Awareness check at the time they see the disguised creature.

  \subsection{Extended Checks}\label{Extended Checks}
    An \glossterm{extended check} is a \glossterm{check} that represents your character taking some action over a prolonged period of time.
    You cannot use abilities like \ability{desperate exertion} to modify the results of an extended check.
    If your modifier changes over the course of the task, use your lowest modifier at any point during the task.

  \subsection{Hidden Checks}
    The GM can always make checks on your character's behalf without telling you.
    Generally, this is used for observation-based skills.
    For example, it's very suspicious if the GM tells you to make an Awareness check and then tells you that you don't see anything interesting.
    One of the ways a GM can avoid that is by simply rolling a check on behalf of your character and only telling you the result if you succeed.

  \subsection{Helping On Checks}
    You can help an \glossterm{ally} make a check.
    To help an ally, you make a check of the same type against a \glossterm{difficulty value} that is 5 lower than the regular difficulty value.
    This has the same requirements, including time and physical contact, as the check would have if you made it yourself.
    For example, to help an ally climb a cliff, you must be able to touch your ally to guide them up.
    Success means that the ally gains a \plus2 bonus to the check.

    Multiple creatures can try to help the same person.
    At the GM's discretion, there may be a practical limit to how many people can assist with the same task.
    The bonus from multiple creatures helping does not stack.
    It just makes it more likely that the helping attempt will succeed.

  \subsection{Checks for Timed Tasks}
    For every 5 points by which you beat the \glossterm{difficulty value} to accomplish a timed task, the time required is usually halved.
    This only applies for tasks that have a base time requirement of at least one minute, if the GM agrees that it is relevant, and if there are no other specific ways in which your result is improved with higher check results.

\section{Defining the Undefined}
  This book does not attempt to include specific rules for every aspect of a realistic world.
  Unless defined otherwise - or if it's not worth the effort to look up Rise's exact rules in the flow of a game - you should assume that the universe works more or less like the real world does, and as long as everyone agrees that something is reasonable, it's not worth worrying about in more detail.

  For example, Rise does not have specific rules for how long it takes to eat a meal, the arc that a thrown ball takes through the air, or how much extra weight a well-made chandelier can hold without breaking.
  It's possible to imagine situations where each of those might be important to a game, however, so you'll have to guess what would be reasonable as obscure situations arise.
  The Game Master has the final word when defining ambiguities like this.

  \subsection{Resolving Ambiguity}\label{Resolving Ambiguity}
    When the rules are ambiguous about how they apply to you and no other creature, you decide how to resolve that ambiguity.
    For example, if an ability causes you to remove one of your \glossterm{vital wounds}, and you have more than one vital wound, you choose which vital wound is removed.
    When the rules are ambiguous in any other situation, the GM decides how to resolve that ambiguity.
    This includes situations where multiple creatures are relevant and situations where no particular creature is relevant.
