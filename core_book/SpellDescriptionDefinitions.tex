\ExplSyntaxOn

\newcommand{\tempderp}{''}

\def\replace#1#2#3{%
 \def\tmp##1#2{##1#3\tmp}%
   \tmp#1\stopreplace#2\stopreplace}
\def\stopreplace#1\stopreplace{}

% args:
%   '\spellname', 'Spell Name', pre-spell header,
%   spell short description, spell level
\DeclareDocumentCommand{\createspell}{m m o m o}
{
    \DeclareDocumentCommand{#1}{s o}
    {
        \IfBooleanTF{##1}
        {
            \spellhead[##2]*{#2}[#3]~#4
        }
        {
            \spellhead[##2]{#2}[#3]~#4
        }
    }
}

\newcommand{\makemacro}[2]
{
    \lowercase{
        \expandafter
        \newcommand\csname#1\endcsname
    }{#1:~#2}
}

\ExplSyntaxOff

% cantrips
\createspell{\SLacidorb}{Acid Orb}
    {Fling acid to deal damage}
\createspell{\SLaugmentattack}{Augment Attack}
    {Grant damage bonus on next attack}
\createspell{\SLcombattelekinesis}{Combat Telekinesis}
    {Wield weapon telekinetically.}
\createspell{\SLconfusionlesser}{Confusion}[Lesser]
    {Compel foe to briefly act randomly.}
\createspell{\SLconjureprojectile}{Conjure Projectile}
    {Fire projectile at foe to deal damage.}
\createspell{\SLdisplacementlesser}{Displacement}[Lesser]
    {Grant 20\% miss chance against physical attacks}
\createspell{\SLdrainingtouch}{Draining Touch}
    {Steal vital energy to deal damage and heal.}
\createspell{\SLexhaustion}{Exhaustion}
    {Fatigue or exhaust foe.}
\createspell{\SLfearlesser}{Fear}[Lesser]
    {Frighten foe briefly.}
\createspell{\SLflare}{Flare}
    {Create flash of light to impair foe's vision.}
\createspell{\SLmagicray}{Magic Ray}
    {Fire force ray to deal damage.}
\createspell{\SLresistdamage}{Resist Damage}
    {Grant brief damage reduction.}
\createspell{\SLslowlesser}{Slow}[Lesser]
    {Force foe to skip one movement phase.}
\createspell{\SLtwistfate}{Twist Fate}
    {Learn foe's future action, possibly apply penalty.}

% regular spells
\createspell{\SLablativeshield}{Ablative Shield}
    {Immediately reduce damage from attacks.}
\createspell{\SLacidsplash}{Acid Splash}
    {Fling acid to deal damage.}
\createspell{\SLacidarrow}{Acid Arrow}
    {Propel acid to deal lingering damage.}
\createspell{\SLagony}{Agony}
    {Inflict pain to increase damage taken.}
\createspell{\SLaid}{Aid}
    {Grant temporary hit points and immunity to fear.}
\createspell{\SLairwalk}{Air Walk}
    {Grant ability to walk on air.}
\createspell{\SLantilifeshell}{Antilife Shell}
    {Prevent living creatures from entering zone.}
\createspell{\SLantimagicfield}{Antimagic Field}
    {Negate all magic in emanation.}
\createspell{\SLaqueousblade}{Aqueous Blade}
    {Transform weapon to attack Reflex defense.}
\createspell{\SLassimilate}{Assimilate}
    {Absorb creature into your body.}
\createspell{\SLaversion}{Aversion}
    {Compel creature to avoid something.}
\createspell{\SLbane}{Bane}
    {Impair foe.}
\createspell{\SLbarkskin}{Barkskin}
    {Grant damage reduction.}
\createspell{\SLblacktentacles}{Black Tentacles}
    {Grapple foes with tentacles from the ground.}
\createspell{\SLbladebarrier}{Blade Barrier}
    {Create wall of whirling blades to deal damage.}
\createspell{\SLblasphemy}{Blasphemy}
    {Damage and stagger nearby nonevil creatures.}
\createspell{\SLbless}{Bless}
    {Grant a legend point.}
\createspell{\SLblink}{Blink}
    {Gain 50\% miss chance by disappearing and reappearing.}
\createspell{\SLblur}{Blur}
    {Grant concealment by blurring outline.}
\createspell{\SLburninghands}{Burning Hands}
    {Create damaging cone of fire.}
\createspell{\SLcacaphonicword}{Cacaphonic Word}
    {Make incoherent noise to damage and disorient non-chaotic creatures.}
\createspell{\SLcalllightning}{Call Lightning}
    {Call multiple lightning bolts from sky to deal damage.}
\createspell{\SLcalmemotions}{Calm Emotions}
    {Calm multiple creatures to avoid violence.}
\createspell{\SLchainlightning}{Chain Lightning}
    {Create lightning that jumps between foes to deal damage.}
\createspell{\SLchaoshammer}{Chaos Hammer}
    {Damage and disorient nonchaotic creature.}
\createspell{\SLcharmperson}{Charm Person}
    {Delude person into believing you are its ally.}
\createspell{\SLcolorspray}{Color Spray}
    {Create cone of random colors to impair vision.}
\createspell{\SLcommand}{Command}
    {Speak command that creature must obey.}
\createspell{\SLconeofcold}{Cone of Cold}
    {Create frigid cone to deal damage and inhibit movement.}
\createspell{\SLconfusion}{Confusion}
    {Compel foe to act randomly.}
\createspell{\SLcreatesound}{Create Sound}
    {Create sounds from nowhere.}
\createspell{\SLcripple}{Cripple}
    {Cripple limbs to deal damage and stagger or paralyze.}
\createspell{\SLcurewounds}{Cure Wounds}
    {Heal hit points.}
\createspell{\SLcurseofbloodandbone}{Curse of Blood and Bone}
    {Inflict curse to deal damage and reduce total hit points.}
\createspell{\SLcurseofthewaywardmind}{Curse of the Wayward Mind}
    {Inflict curse to confuse or disorient foe.}
\createspell{\SLdancinglights}{Dancing Lights}
    {Create floating lights you control.}
\createspell{\SLdarkvision}{Darkvision}
    {Grant ability to see in total darkness.}
\createspell{\SLdeathknell}{Death Knell}
    {Inflict damage, accelerate death, and gain life if foe dies.}
\createspell{\SLdeathward}{Death Ward}
    {Grant immunity to negative energy and death effects.}
\createspell{\SLdeepslumber}{Deep Slumber}
    {Compel foe to close eyes and possibly sleep forever.}
\createspell{\SLdeflection}{Deflection}
    {Gain 50\% miss chance against physical attacks.}
\createspell{\SLdelaydamage}{Delay Damage}
    {Delay half the damage you take until later.}
\createspell{\SLdivinejudgment}{Divine Judgment}
    {Deal damage, possibly instantly kill foe.}
\createspell{\SLdetectalignment}{Detect Alignment}
    {Reveal entities with selected alignment within cone.}
\createspell{\SLdictum}{Dictum}
    {Damage and slow nearby nonlawful creatures.}
\createspell{\SLdimensiondoor}{Dimension Door}
    {Teleport anywhere within 1,000 feet.}
\createspell{\SLdimensionslide}{Dimension Slide}
    {Teleport creature short distance.}
\createspell{\SLdimensionalanchor}{Dimensional Anchor}
    {Block extradimensional movement.}
\createspell{\SLdiscernlies}{Discern Lies}
    {Reveal deliberate lies within cone.}
\createspell{\SLdiscernvulnerability}{Discern Vulnerability}
    {Quickly find weaknesses in foe's defenses.}
\createspell{\SLdiscordantsong}{Discordant Song}
    {Create music to compel creatures to act randomly.}
\createspell{\SLdisintegrate}{Disintegrate}
    {Deal damage, possibly transforming target into dust.}
\createspell{\SLdispelmagic}{Dispel Magic}
    {Negate magical effects on target.}
\createspell{\SLdisjoinmagic}{Disjoin Magic}
    {Unconditionally negate magical effects.}
\createspell{\SLdivinefavor}{Divine Favor}
    {Gain legend point.}
\createspell{\SLdominateperson}{Dominate Person}
    {Compel person to obey you completely.}
\createspell{\SLdrainlife}{Drain Life}
    {Steal vital energy to deal damage and heal.}
\createspell{\SLearthspull}{Earth's Pull}
    {Increase force of gravity on foe.}
\createspell{\SLearthenblade}{Earthen Blade}
    {Create magical weapon from the ground.}
\createspell{\SLearthglide}{Earth Glide}
    {Grant ability to glide through earth.}
\createspell{\SLearthquake}{Earthquake}
    {Shake ground to knock prone and immobilize foes.}
\createspell{\SLearthspike}{Earthspike}
    {Animate earthen spike to deal damage.}
\createspell{\SLenervation}{Enervation}
    {Inflict two negative levels.}
\createspell{\SLenlarge}{Enlarge}
    {Double size of creature.}
\createspell{\SLentangle}{Entangle}
    {Slow foe with nearby plants.}
\createspell{\SLentropicshield}{Entropic Shield}
    {Grant 50\% miss chance against ranged attacks.}
\createspell{\SLetherealjaunt}{Ethereal Jaunt}
    {Travel to Ethereal Plane.}
\createspell{\SLeyebite}{Eyebite}
    {Inflict foe's eyes to deal damage and impair sight.}
\createspell{\SLfaeriefire}{Faerie Fire}
    {Create lights to negate concealment and stealth.}
\createspell{\SLfear}{Fear}
    {Terrify foe.}
\createspell{\SLfeatherfall}{Feather Fall}
    {Arrest fall of object or ally.}
\createspell{\SLfeeblemind}{Feeblemind}
    {Reduce foe's Intelligence to \minus9.}
\createspell{\SLfingerofdeath}{Finger of Death}
    {Deal damage, possibly instantly kill foe.}
\createspell{\SLfireball}{Fireball}
    {Create burst of fire to deal damage.}
\createspell{\SLfireseeds}{Fire Seeds}
    {Infuse berries with fiery energy, detonate to deal damage.}
\createspell{\SLfireshield}{Fire Shield}
    {Gain cold resistance and retributive damage against attacks.}
\createspell{\SLfirestorm}{Fire Storm}
    {Create storm of fire to deal damage to foes.}
\createspell{\SLfissure}{Fissure}
    {Deal damage, possibly instantly kill foe.}
\createspell{\SLflameblade}{Flame Blade}
    {Wield fire as scimitar.}
\createspell{\SLflamestrike}{Flame Strike}
    {Smite foes with divine fire.}
\createspell{\SLfly}{Fly}
    {Grant ability to fly.}
\createspell{\SLfogcloud}{Fog Cloud}
    {Create zone of fog that obscures vision.}
\createspell{\SLforcecage}{Forcecage}
    {Imprison foe in prison of force.}
\createspell{\SLforget}{Forget}
    {Delude creature into forgetting something.}
\createspell{\SLfreedom}{Freedom}
    {Grant immunity to movement impediments.}
\createspell{\SLgaseousform}{Gaseous Form}
    {Transform ally into gas, granting flight.}
\createspell{\SLgentledescent}{Gentle Descent}
    {Grant ability to glide.}
\createspell{\SLghoultouch}{Ghoul Touch}
    {Sicken or nauseate creauture.}
\createspell{\SLglitterdust}{Glitterdust}
    {Create dust to negate concealment and stealth.}
\createspell{\SLgrease}{Grease}
    {Create grease to make area or object slippery.}
\createspell{\SLgustofwind}{Gust of Wind}
    {Create wind to shove creatures away.}
\createspell{\SLharm}{Harm}
    {Inflict damage or vital damage.}
\createspell{\SLhaste}{Haste}
    {Grant double movement speed.}
\createspell{\SLheal}{Heal}
    {Cure ally of critical injuries and almost all afflictions.}
\createspell{\SLholdperson}{Hold Person}
    {Immobilize a person.}
\createspell{\SLholysmite}{Holy Smite}
    {Damage and daze nongood creature.}
\createspell{\SLholyword}{Holy Word}
    {Damage and daze nearby nongood creatures.}
\createspell{\SLhorridwilting}{Horrid Wilting}
    {Create dessicating burst to deal damage.}
\createspell{\SLhypnoticpattern}{Hypnotic Pattern}
    {Create lights to fascinate creatures.}
\createspell{\SLicestorm}{Ice Storm}
    {Create storm to deal damage.}
\createspell{\SLimplosion}{Implosion}
    {Concentrate to damage and possibly kill multiple creatures.}
\createspell{\SLimprisonment}{Imprisonment}
    {Entomb foe beneath the earth permanently.}
\createspell{\SLinertialshield}{Inertial Shield}
    {Grant damage reduction.}
\createspell{\SLinflictwounds}{Inflict Wounds}
    {Inflict injuries on foe.}
\createspell{\SLinvisibility}{Invisibility}
    {Grant ally invisibility until it attacks.}
\createspell{\SLirresistibledance}{Irresistible Dance}
    {Force foe to dance.}
\createspell{\SLknock}{Knock}
    {Open locked or barred object.}
\createspell{\SLlevitate}{Levitate}
    {Gain ability to move ally vertically.}
\createspell{\SLlifeseekingmissile}{Lifeseeking Missile}
    {Fire homing missiles to deal damage.}
\createspell{\SLlightningbolt}{Lightning Bolt}
    {Create line of lightning to deal damage.}
\createspell{\SLlongstrider}{Longstrider}
    {Gain small speed bonus for prolonged period.}
\createspell{\SLmagearmor}{Mage Armor}
    {Create force armor to increase defenses.}
\createspell{\SLmagehand}{Mage Hand}
    {Telekinetically move objects.}
\createspell{\SLmagicmissile}{Magic Missile}
    {Fire unerring missiles to deal damage.}
\createspell{\SLmaze}{Maze}
    {Trap foe in extradimensional maze.}
\createspell{\SLmeldintostone}{Meld into Stone}
    {Become one with block of stone.}
\createspell{\SLmessage}{Message}
    {Gain ability to whisper conversation at range.}
\createspell{\SLmeteorswarm}{Meteor Swarm}
    {Create huge hail of meteors to deal damage and knock foes prone.}
\createspell{\SLmirrorimage}{Mirror Image}
    {Create decoy duplicates of you.}
\createspell{\SLmissilestorm}{Missile Storm}
    {Create swarm of homing missiles to damage multiple foes.}
\createspell{\SLmomentofprescience}{Moment of Prescience}
    {Immediately gain a legend point.}
\createspell{\SLorderswrath}{Order's Wrath}
    {Damage and slow nonlawful creature.}
\createspell{\SLphantasmalkiller}{Phantasmal Killer}
    {Frighten foe, possibly to death.}
\createspell{\SLplanardisruption}{Planar Disruption}
    {Damage foe, possibly return it to its native plane.}
\createspell{\SLpoison}{Poison}
    {Inflict deadly poison on foe.}
\createspell{\SLpolarray}{Polar Ray}
    {Fire frigid ray to deal damage and slow.}
\createspell{\SLpowerwordblind}{Power Word Blind}
    {Speak word to blind or visually impair foe.}
\createspell{\SLpowerwordfear}{Power Word Fear}
    {Speak word to frighten foe.}
\createspell{\SLpowerwordimpair}{Power Word Impair}
    {Speak word to impair foe.}
\createspell{\SLpowerwordstagger}{Power Word Stagger}
    {Speak word to stagger foe.}
\createspell{\SLpowerwordstun}{Power Word Stun}
    {Speak word to stun or daze foe.}
\createspell{\SLprecognition}{Precognition}
    {Gain a legend point.}
\createspell{\SLprismaticbeam}{Prismatic Beam}
    {Fire multicolored ray with random effects.}
\createspell{\SLprismaticspray}{Prismatic Spray}
    {Fire multicolored cone with random effects.}
\createspell{\SLprismaticstorm}{Prismatic Storm}
    {Create multicolored blast with random effects.}
\createspell{\SLprismaticwall}{Prismatic Wall}
    {Create multicolored wall with random effects.}
\createspell{\SLprohibition}{Prohibition}
    {Punish nearby creatures that take specific action.}
\createspell{\SLprotectionfromalignment}{Protection from Alignment}
    {Grant damage reduction against aligned foes.}
\createspell{\SLreadmind}{Read Mind}
    {Learn a creature's surface thoughts.}
\createspell{\SLregeneration}{Regeneration}
    {Grant automatic healing each round.}
\createspell{\SLrepulsion}{Repulsion}
    {Prevent creatures from approaching you.}
\createspell{\SLresistenergy}{Resist Energy}
    {Grant damage reduction against energy.}
\createspell{\SLretrieveobject}{Retrieve Object}
    {Teleport unattended object to your hand.}
\createspell{\SLrevelation}{Revelation}
    {Grant target vision of one of three futures.}
\createspell{\SLreversegravity}{Reverse Gravity}
    {Reverse gravity in a zone.}
\createspell{\SLrevivify}{Revivify}
    {Resurrect recently deceased creature.}
\createspell{\SLdivinemight}{Divine Might}
    {Double in size and gain damage reduction.}
\createspell{\SLsanctuary}{Sanctuary}
    {Grant immunity to attack until it attacks.}
\createspell{\SLscorchingray}{Scorching Ray}
    {Fire ray of fire to deal damage and ignite.}
\createspell{\SLsearinglight}{Searing Light}
    {Fire ray of light to deal damage and blind.}
\createspell{\SLseeinvisibility}{See Invisibility}
    {Grant ability to see invisible creatures and objects.}
\createspell{\SLsharepain}{Share Pain}
    {Split damage between two allies.}
\createspell{\SLshockinggrasp}{Shocking Grasp}
    {Touch foe with electricity to deal damage and stagger.}
\createspell{\SLshout}{Shout}
    {Shout loudly to deal damage and deafen in cone.}
\createspell{\SLshrink}{Shrink}
    {Halve size of creature.}
\createspell{\SLshrinkitem}{Shrink Item}
    {Shrink object to one-sixteenth size.}
\createspell{\SLsilence}{Silence}
    {Prevent target from making noise.}
\createspell{\SLcreateimage}{Create Image}
    {Creates figment to fool eyes.}
\createspell{\SLsleep}{Sleep}
    {Tire a creature, possibly putting it to sleep.}
\createspell{\SLslow}{Slow}
    {Force foe to skip movement phase.}
\createspell{\SLsoundburst}{Sound Burst}
    {Create blast of sound to deal damage.}
\createspell{\SLresistmagic}{Resist Magic}
    {Grant magic resistance.}
\createspell{\SLspiderclimb}{Spider Climb}
    {Grant ability to climb easily on walls and ceilings.}
\createspell{\SLspiritualweapon}{Spiritual Weapon}
    {Create floating weapon that attacks on its own.}
\createspell{\SLstormlord}{Stormlord}
    {Gain whirlwind that blocks ranged attacks and deals retributive damage.}
\createspell{\SLstormofvengeance}{Storm of Vengeance}
    {Storm rains acid, lightning, and hail.}
\createspell{\SLstriptheflesh}{Strip the Flesh}
    {Rend foe's skin from its body.}
\createspell{\SLsuggestion}{Suggestion}
    {Convince creature to obey suggestion.}
\createspell{\SLsummonmonster}{Summon Monster}
    {Call extraplanar creature to fight for you.}
\createspell{\SLsummonnaturesally}{Summon Nature's Ally}
    {Call creature to fight for you.}
\createspell{\SLsunbeam}{Sunbeam}
    {Create beam of light to deal damage and possibly blind.}
\createspell{\SLsunburst}{Sunburst}
    {Create burst of light to deal damage and possibly blind.}
\createspell{\SLtelepathy}{Telepathy}
    {Grant ability to communicate mentally.}
\createspell{\SLtemporalstasis}{Temporal Stasis}
    {Stop time for creature, possibly permanently.}
\createspell{\SLtimestop}{Time Stop}
    {Gain immense speed to take extra actions.}
\createspell{\SLtransmuteanyobject}{Transmute Any Object}
    {Transform objects or creatures into new forms.}
\createspell{\SLtransmutefleshandstone}{Transmute Flesh and Stone}
    {Transform creature to or from stone}
\createspell{\SLtreeshape}{Tree Shape}
    {Transform into a tree.}
\createspell{\SLtremorsense}{Tremorsense}
    {Grant ability to ``see'' through the ground.}
\createspell{\SLtrueseeing}{True Seeing}
    {Grant ability to see through all illusions and transformations.}
\createspell{\SLtruestrike}{True Strike}
    {Quickly grant brief legend point.}
\createspell{\SLunholyblight}{Unholy Blight}
    {Damage and stagger nonevil creature.}
\createspell{\SLunlivingeyes}{Unliving Eyes}
    {Grant ability to perfectly see living creatures.}
\createspell{\SLunlivingheart}{Unliving Heart}
    {Gain temporary hit points, become undead.}
\createspell{\SLventriloquism}{Ventriloquism}
    {Gain ability to speak from anywhere.}
\createspell{\SLwailofthebanshee}{Wail of the Banshee}
    {Scream to damage and possibly kill nearby creatures.}
\createspell{\SLwalloffire}{Wall of Fire}
    {Create flaming wall that deals damage.}
\createspell{\SLwallofforce}{Wall of Force}
    {Create force wall that is immune to damage.}
\createspell{\SLwallofthorns}{Wall of Thorns}
    {Create thorny wall that blocks sight and deals damage.}
\createspell{\SLwaterwalk}{Water Walk}
    {Grant ability to walk on water.}
\createspell{\SLwavesoffatigue}{Waves of Fatigue}
    {Fatigue creatures in large cone.}
\createspell{\SLweb}{Web}
    {Create spiderwebs that immobilize.}
\createspell{\SLwindstrike}{Windstrike}
    {Bludgeon foe with wind to deal damage and shove.}
\createspell{\SLwordofchaos}{Word of Chaos}
    {Damage and disorient nearby nonchaotic creatures.}
\createspell{\SLwordofrecall}{Word of Recall}
    {Teleport back to designated place.}
\createspell{\SLzephyrblade}{Zephyr Blade}
    {Melee weapon can strike from a short distance.}
\createspell{\SLtotemicmind}{Totemic Mind}
    {Grant \plus2 to mental attribute.}
\createspell{\SLtotemicpower}{Totemic Power}
    {Grant \plus2 to physical attribute.}
\createspell{\SLexcreteslime}{Excrete Slime}
    {Excrete slime to deal retributive damage.}
\createspell{\SLrockblast}{Rock Blast}
    {Fire rocks to deal damage.}
\createspell{\SLrottinggrasp}{Rotting Grasp}
    {Rot flesh of adjacent foe to deal lingering damage.}
\createspell{\SLrotburst}{Rotburst}
    {Rot flesh of nearby foes to deal lingering damage.}
\createspell{\SLfungalgrowth}{Fungal Growth}
    {Grow fungus to sicken and damage foe.}
\createspell{\SLswarmofbats}{Swarm of Bats}
    {Summon bats to impair vision of foes.}
\createspell{\SLlongeye}{Longeye}
    {Reduce penalties for attacking at long range.}
\createspell{\SLmarkoftracking}{Mark of Tracking}
    {Know marked target's approximate location.}
\createspell{\SLmarkofscrying}{Mark of Scrying}
    {Gain ability to scry on marked target.}
\createspell{\SLbleed}{Bleed}
    {Deal damage, plus one vital damage.}
\createspell{\SLanimatecorpse}{Animate Corpse}
    {Briefly raise corpse to fight for you.}
\createspell{\SLrapidreversal}{Rapid Reversal}
    {Quickly teleport creature back to its original position.}
\createspell{\SLboonofknowledge}{Boon of Knowledge}
    {Grant \plus5 to Knowledge skills.}
\createspell{\SLboonofmanyeyes}{Boon of Many Eyes}
    {Grant reduced overwhelm penalties.}
\createspell{\SLboonofmastery}{Boon of Mastery}
    {Grant ability to reroll all skills.}
\createspell{\SLboonofperception}{Boon of Perception}
    {Grant ability to reroll perceptual skills.}
\createspell{\SLboonofprecision}{Boon of Precision}
    {Grant increased critical range and multiplier.}
\createspell{\SLinvulnerability}{Invulnerability}
    {Gain damage reduction.}
\createspell{\SLcreateballista}{Create Ballista}
    {Create a ballista that fires automatically.}
\createspell{\SLtelekineticshove}{Telekinetic Shove}
    {Mentally shove creatures each round.}
\createspell{\SLtelekineticblast}{Telekinetic Blast}
    {Damage nearby foes and blast them away from you.}
\createspell{\SLmaskofthedeceiver}{Mask of the Deceiver}
    {Grant ability to reroll Bluff, Persuasion.}
\createspell{\SLshadowduplicate}{Shadow Duplicate}
    {Create illusory double that can talk and cast spells.}
\createspell{\SLshadowwall}{Shadow Wall}
    {Create illusory wall that only blocks foes.}
\createspell{\SLshadowbolt}{Shadowbolt}
    {Fire invisible bolt that deals cold damage.}
\createspell{\SLshadowshield}{Shadowshield}
    {Grant 50\% chance to subtly negate attacks.}
\createspell{\SLshadowstep}{Shadowstep}
    {Become invisible and create illusory double.}
\createspell{\SLshadowstorm}{Shadowstorm}
    {Create invisible storm that deals cold damage to foes.}
\createspell{\SLlivingmissile}{Living Missile}
    {Throw ally at foe to deal damage.}
\createspell{\SLresistpoison}{Resist Poison}
    {Quickly grant immunity to poison.}
\createspell{\SLavatarofwind}{Avatar of Wind}
    {Gain ability to strike foes with wind at will.}
\createspell{\SLavatarofhealing}{Avatar of Healing}
    {Gain ability to heal allies at will.}
\createspell{\SLcircleofdeath}{Circle of Death}
    {Deal damage to nearby foes over time.}
\createspell{\SLcircleofhealing}{Circle of Healing}
    {Heal nearby allies over time.}
\createspell{\SLavatarofdeath}{Avatar of Death}
    {Gain ability to damage or kill foes at will.}
\createspell{\SLavatarofsuffering}{Avatar of Suffering}
    {Gain ability to stagger and impair foes at will.}
\createspell{\SLgolemheart}{Golem Heart}
    {Gain golem resistances and immunities.}
\createspell{\SLinferno}{Inferno}
    {Deal damage to nearby foes over time.}
\createspell{\SLimmolation}{Immolation}
    {Deal damage, possibly burn foe to ash.}
\createspell{\SLmartyrsgift}{Martyr's Gift}
    {Gain ability to take damage of nearby allies.}
\createspell{\SLholyavatar}{Holy Avatar}
    {Gain ability to damage and daze nongood foes at will.}
\createspell{\SLunholyavatar}{Unholy Avatar}
    {Gain ability to damage and stagger nonevil foes at will.}
\createspell{\SLavatarofchaos}{Avatar of Chaos}
    {Gain ability to damage and disorient nonchaotic foes at will.}
\createspell{\SLavataroforder}{Avatar of Order}
    {Gain ability to damage and slow nonlawful foes at will.}
\createspell{\SLscrybolt}{Scrybolt}
    {Deal damage to foe you unambiguously identify.}
\createspell{\SLthirdeye}{Third Eye}
    {Gain blindsight, cannot be caught unaware.}
\createspell{\SLavataroffealty}{Avatar of Fealty}
    {Gain ability to command foes at will.}
\createspell{\SLavatarofmissiles}{Avatar of Missiles}
    {Gain ability to summon damaging missiles at will.}
\createspell{\SLspellsight}{Spellsight}
    {Gain ability to see and understand all magic.}
\createspell{\SLsevermagic}{Sever Magic}
    {Remove foe's connection to magic.}
\createspell{\SLavatarofshielding}{Avatar of Shielding}
    {Grant damage reduction at will.}
\createspell{\SLauraofimmunity}{Aura of Immunity}
    {Grant allies immunity to chosen damage type.}
\createspell{\SLmightythrow}{Mighty Throw}
    {Throw foe a great distance, dealing damage.}
\createspell{\SLseismicslam}{Seismic Slam}
    {Deal damage and immobilize foe.}
\createspell{\SLavataroftranslocation}{Avatar of Translocation}
    {Gain ability to teleport allies at will.}
\createspell{\SLrestoration}{Restoration}
    {Remove conditions from ally.}
\createspell{\SLdimensionalarmy}{Dimensional Army}
    {Teleport allies to distant location.}
\createspell{\SLavataroflife}{Avatar of Life}
    {Gain ability to resurrect allies at will.}
\createspell{\SLavatarofmight}{Avatar of Might}
    {Gain ability to throw foes at will.}
\createspell{\SLlifebound}{Lifebound}
    {Grant ally immunity to death while another lives.}
\createspell{\SLblessedblade}{Blessed Blade}
    {Transform weapon to attack Mental defense.}
\createspell{\SLavatarofblades}{Avatar of Blades}
    {Gain ability to summon rending blades at will.}
\createspell{\SLbladestorm}{Bladestorm}
    {Create storm of blades to deal damage to foes.}
\createspell{\SLdrown}{Drown}
    {Create water to drown foe.}
\createspell{\SLaquaticsphere}{Aquatic Sphere}
    {Create bubble of water that surrounds you.}
\createspell{\SLicespike}{Ice Spike}
    {Throw ice shard to deal damage and slow.}
\createspell{\SLavatarofgrowth}{Avatar of Growth}
    {Gain ability to entangle foes at will.}
\createspell{\SLlifegivingroots}{Lifegiving Roots}
    {Create roots that immobilize and heal ally.}
\createspell{\SLmindblank}{Mind Blank}
    {Grant immunity to Mind effects.}


% RITUALS
\createspell{\SLalarm}{Alarm}
    {Ward area with alarm that senses motion.}
\createspell{\SLendureelements}{Endure Elements}
    {Grant heat and cold tolerance.}
\createspell{\SLmount}{Mount}
    {Summon riding horse.}
\createspell{\SLunseenservant}{Unseen Servant}
    {Create invisible force that obeys commands.}
\createspell{\SLappraisal}{Appraisal}
    {Learn value of item.}
\createspell{\SLidentify}{Identify}
    {Learn magical properties of item.}
\createspell{\SLreadmagic}{Read Magic}
    {Gain ability to read magical writing.}
\createspell{\SLfloatingdisk}{Floating Disk}
    {Create floating disk to hold objects.}
\createspell{\SLdisguiseself}{Disguise Self}
    {Change your appearance.}
\createspell{\SLlight}{Light}
    {Create light.}
\createspell{\SLmagicaura}{Magic Aura}
    {Change magic aura of item.}
\createspell{\SLerase}{Erase}
    {Remove text from writing.}
\createspell{\SLmending}{Mending}
    {Fix damaged object.}
\createspell{\SLarcanemark}{Arcane Mark}
    {Inscribe personal rune.}
\createspell{\SLpurifysustenance}{Purify Sustenance}
    {Remove impurities from food or water.}
\createspell{\SLarcanelock}{Arcane Lock}
    {Lock openable object.}
\createspell{\SLinvisibilitypurge}{Invisibility Purge}
    {Suppress nearby invisibility.}
\createspell{\SLundetectablealignment}{Undetectable Alignment}
    {Conceal ally's alignment.}
\createspell{\SLcreatesustenance}{Create Sustenance}
    {Create food and water.}
\createspell{\SLcomprehendlanguages}{Comprehend Languages}
    {Grant ability to understand all languages.}
\createspell{\SLfindtraps}{Find Traps}
    {Grant improved trap-finding abilities.}
\createspell{\SLzoneoftruth}{Zone of Truth}
    {Prevent lies from being spoken in area.}
\createspell{\SLdarkness}{Darkness}
    {Create darkness in area around object.}
\createspell{\SLmagicmouth}{Magic Mouth}
    {Create mouth that delivers a message.}
\createspell{\SLenhancearmor}{Enhance Armor}
    {Improve enhancement bonus of armor.}
\createspell{\SLenhanceweapon}{Enhance Weapon}
    {Improve enhancement bonus of weapon.}
\createspell{\SLshapeweapon}{Shape Weapon}
    {Change shape of weapon.}
\createspell{\SLshaping}{Shaping}
    {Craft object into different shape.}
\createspell{\SLgentlerepose}{Gentle Repose}
    {Prevent corpse from decaying.}
\createspell{\SLbinding}{Binding}
    {Create magic circle that imprisons creature.}
\createspell{\SLnondetection}{Nondetection}
    {Grant immunity to magical detection.}
\createspell{\SLcreateobject}{Create Object}
    {Create object from nothing.}
\createspell{\SLremotesenses}{Remote Senses}
    {Gain ability to see or hear at a distance.}
\createspell{\SLtelepathicbond}{Telepathic Bond}
    {Grant allies ability to communicate at a distance.}
\createspell{\SLexplosiverunes}{Explosive Runes}
    {Create explosive trap in writing.}
\createspell{\SLfiretrap}{Fire Trap}
    {Create explosive trap in openable object.}
\createspell{\SLtinyhut}{Tiny Hut}
    {Create shelter from weather conditions.}
\createspell{\SLsecretpage}{Secret Page}
    {Alter writing.}
\createspell{\SLwaterbreathing}{Water Breathing}
    {Grant ability to breathe water.}
\createspell{\SLanimatedead}{Animate Dead}
    {Create skeleton or zombie from corpse.}
\createspell{\SLspeakwithdead}{Speak with Dead}
    {Ask questions of corpse.}
\createspell{\SLitemattunement}{Item Attunement}
    {Attune to item to increase its power.}
\createspell{\SLremovecurse}{Remove Curse}
    {Remove a curse.}
\createspell{\SLarcaneeye}{Arcane Eye}
    {Create mobile sensor you can see through.}
\createspell{\SLdetectscrying}{Detect Scrying}
    {Gain ability to notice scrying attempts.}
\createspell{\SLtongues}{Tongues}
    {Grant ability to speak and understand all languages.}
\createspell{\SLhallucinatoryterrain}{Hallucinatory Terrain}
    {Disguise terrain with illusion.}
\createspell{\SLbreakenchantment}{Break Enchantment}
    {Dispel magical effects and curses.}
\createspell{\SLdimensionallock}{Dimensional Lock}
    {Prevent dimensional travel in area.}
\createspell{\SLprivatesanctum}{Private Sanctum}
    {Prevent sight or scrying into area.}
\createspell{\SLcontactotherplane}{Contact Other Plane}
    {Ask questions of extraplanar entity.}
\createspell{\SLdream}{Dream}
    {Send message via dreams.}
\createspell{\SLlegendlore}{Legend Lore}
    {Learn about matters of legendary importance.}
\createspell{\SLscrying}{Scrying}
    {Gain ability to see and hear at any distance.}
\createspell{\SLsending}{Sending}
    {Send message at any distance.}
\createspell{\SLsensorswarm}{Sensor Swarm}
    {Create many mobile magic sensors.}
\createspell{\SLfalsevision}{False Vision}
    {Prevent scrying from working correctly in area.}
\createspell{\SLfabricate}{Fabricate}
    {Craft finished items from raw materials.}
\createspell{\SLpasswall}{Passwall}
    {Create passage through walls.}
\createspell{\SLplaneshift}{Plane Shift}
    {Travel to another plane.}
\createspell{\SLteleport}{Teleport}
    {Teleport allies across vast distance.}
\createspell{\SLfindthepath}{Find the Path}
    {Gain ability to find chosen location.}
\createspell{\SLgeasquest}{Geas/Quest}
    {Command creature to perform task.}
\createspell{\SLoverlandflight}{Overland Flight}
    {Grant ability to fly long distances.}
\createspell{\SLsequester}{Sequester}
    {Place ally into invisible temporal stasis.}
\createspell{\SLclone}{Clone}
    {Create clone of deceased creature.}
\createspell{\SLmagnificentmansion}{Magnificent Mansion}
    {Create portal to extradimensional mansion.}
\createspell{\SLphasedoor}{Phase Door}
    {Create extradimensional passage through walls.}
\createspell{\SLteleportobject}{Teleport Object}
    {Teleport object across vast distance.}
\createspell{\SLcontrolweather}{Control Weather}
    {Control local weather.}
\createspell{\SLprogrammedimage}{Programmed Image}
    {Create illusory image in response to trigger.}
\createspell{\SLdiscernlocation}{Discern Location}
    {Learn exact location of anything.}
\createspell{\SLemancipation}{Emancipation}
    {Remove all movement-impairing effects from creature.}
\createspell{\SLantipathy}{Antipathy}
    {Compel creatures to leave area.}
\createspell{\SLsympathy}{Sympathy}
    {Compel creatures to come to area.}
\createspell{\SLsoulbind}{Soul Bind}
    {Trap soul of dead creature to prevent resurrection.}

\createspell{\SLblesswater}{Bless Water}
    {Create holy water.}
\createspell{\SLcreatewater}{Create Water}
    {Create drinkable water.}
\createspell{\SLaugury}{Augury}
    {Learn whether action will be good or bad.}
\createspell{\SLcontagion}{Contagion}
    {Infect distant creature with disease.}
\createspell{\SLrestoresenses}{Restore Senses}
    {Restore a missing sense, such as sight.}
\createspell{\SLremovedisease}{Remove Disease}
    {Cure diseases afflicting creature.}
\createspell{\SLdivination}{Divination}
    {Learn advice about proposed action.}
\createspell{\SLatonement}{Atonement}
    {Grant creature forgiveness for misdeeds.}
\createspell{\SLcommune}{Commune}
    {Ask questions of deity.}
\createspell{\SLcommunewithnature}{Commune with Nature}
    {Learn about surrounding environment.}
\createspell{\SLresurrection}{Resurrection}
    {Restore dead creature to life.}
\createspell{\SLmoveearth}{Move Earth}
    {Slowly control nearby terrain.}
\createspell{\SLinstantrefuge}{Instant Refuge}
    {Grant item ability to teleport bearer to you.}

\createspell{\SLanimalmessenger}{Animal Messenger}
    {Compel animal to deliver a message.}
\createspell{\SLpasswithouttrace}{Pass without Trace}
    {Grant ability to move without leaving tracks.}
\createspell{\SLironwood}{Ironwood}
    {Transform wood to be as strong as steel.}
\createspell{\SLreincarnate}{Reincarnate}
    {Restore dead creature to life in new body.}
\createspell{\SLtreestride}{Tree Stride}
    {Teleport yourself across vast distance.}
\createspell{\SLtransportviaplants}{Transport via Plants}
    {Teleport allies across vast distance.}
\createspell{\SLawaken}{Awaken}
    {Grant animal sentience.}
\createspell{\SLdivineshield}{Divine Shield}
    {Grant damage reduction and fast healing.}
\createspell{\SLmartialtransformation}{Martial Transformation}
    {Grant immense martial prowess.}
\createspell{\SLwallofantimagic}{Wall of Antimagic}
    {Create wall that selectively blocks magic.}
\createspell{\SLboulderdrop}{Boulder Drop}
    {Create falling boulder to damage foes.}
\createspell{\SLshiningbeacon}{Shining Beacon}
    {Radiate blinding light.}
\createspell{\SLtelekinesis}{Telekinesis}
    {Manipulate objects at a distance.}
\createspell{\SLtransferimbuement}{Transfer Imbuement}
    {Transfer magic between two objects.}





