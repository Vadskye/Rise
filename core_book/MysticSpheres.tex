\chapter{Mystic Spheres}\label{Mystic Spheres}

\section{Categories of Magic}
    % TODO: description

    \subsection{Magic Sources}
        There are four \glossterm{magic sources} that characters can use to cast spells and perform rituals: arcane (cast by mages), divine (cast by clerics and paladins), nature (cast by druids), and pact (cast by warlocks).
        Each magic source has a set of associated \glossterm{mystic spheres} (see Mystic Spheres, below).
        % TODO: more description

        \subsubsection{Characters with Multiple Magic Sources}
            A character can have access to multiple sources of magic through the use of abilities like the Class Versatility feat (see \featpref{Class Versatility}).
            The \glossterm{mystic spheres}, spells, and rituals that character knows are tracked separately for each source of magic that character has access to.
            If you have access to the same spell or ritual from multiple sources, the two versions of the ability are generally considered to be the same ability.
            When you cast the spell or perform the ritual, you choose which source you are using for the ability.

    \subsection{Mystic Spheres}
        A \glossterm{mystic sphere} is a collection of thematically related magical effects that includes both \glossterm{spells} and \glossterm{rituals}.
        Each \glossterm{mystic sphere} is associated with a single school of magic and any number of \glossterm{magic sources}.
        The mystic spheres are listed at \pcref{Mystic Sphere Lists}.

    \subsection{Magic Schools}
        There are nine schools of magic.
        Each school of magic has a set of thematically related effects.
        Every spell and ritual belongs to at least one school of magic.
        % TODO: wording.
        When you cast a spell or perform a ritual, that ability is considered to be from the source you used to gain access to the ability.

        The schools of magic are described below.

        \subsubsection{Abjuration}
            Abjuration spells and rituals reduce or negate damage, magic, and other effects.
            They can be used to protect allies and remove harmful magic.

        \subsubsection{Channeling}
            Channeling spells and rituals call upon the power of deities or other supernatural entities.
            They can be used to do anything those entities could do.
            Only divine spellcasters have access to Channeling spells.

        \subsubsection{Conjuration}
            Conjuration spells and rituals create and transport objects and creatures.
            They can be used to summon allies, transport creatures, and create objects from thin air.

        \subsubsection{Divination}
            Divination spells and rituals grant knowledge.
            They can be used to reveal hidden truths, predict the future, or communicate at great distances.

        \subsubsection{Enchantment}
            Enchantment spells and rituals alter the minds of creatures.
            They can be used to influence, control, or debilitate creatures.
            Almost all enchantment spells and rituals have the \glossterm{Compulsion} or \glossterm{Emotion} tags,
            and some have the \glossterm{Subtle} tag as well (see \pcref{Ability Tags}).

        \subsubsection{Evocation}
            Evocation spells and rituals create and manipulate energy.
            They can be used to inflict damage with energy blasts or manipulate the environment.

        \subsubsection{Illusion}
            Illusion spells and rituals create or manipulate sensory impressions.
            They can be used to create or remove light, conceal things that exist, or cause creatures to perceive things that do not exist.

        \subsubsection{Transmutation}
            Transmutation spells and rituals change the properties of creatures and objects.
            They can be used to grant new abilities, enhance existing abilities, change a target's form, or even alter the flow of time itself.

        \subsubsection{Vivimancy}
            Vivimancy spells and rituals manipulate the power of life and death, as well as souls.
            They can be used to heal or inflict wounds, resurrect the dead, create undead monsters, and cripple the bodies of creatures.

\section{Spell and Ritual Mechanics}\label{Spell and Ritual Mechanics}

    Spells and rituals share many common properties, defined here.

    \subsection{Casting Components}\label{Casting Components}
        Unless otherwise noted, all spells and rituals require both \glossterm{verbal components} to cast or perform.
        In addition, arcane spells, pact spells, and all rituals require \glossterm{somatic components}.
        You cannot start casting a spell or performing a ritual without all required components.
        If you lose those components before the ability resolves, it is \glossterm{miscast}.

        To provide the verbal component for a spell or ritual, you must speak in a strong voice with a volume at least as loud as ordinary conversation.
        To provide the somatic component for a spell or ritual, you must make a measured and precise movement of at least one free hand.

        \subsubsection{Somatic Component Failure}\label{Somatic Component Failure}
            Encumbrance from armor interferes with the \glossterm{somatic components} required to perform arcane spells, pact spells, and all rituals.
            When you cast a spell or perform a ritual that requires \glossterm{somatic components} while you have an \glossterm{encumbrance}, you must roll 1d10.
            If your result is less than or equal to your \glossterm{encumbrance}, you \glossterm{miscast} the ability (see \pcref{Miscasting}).
            When you perform a ritual, this roll must be repeated at the end of each action phase during the ritual.


    \subsection{Concentration}\label{Concentration}
        Some abilities, such as spells and rituals, require focused thought to use successfully.
        If your focus is tested, such as by casting a spell defensively (see \pcref{Defensive Casting}), you may need to make a \glossterm{concentration} check.
        Your bonus with a concentration check is equal to your level or your Willpower, whichever is higher.
        You apply any \glossterm{overwhelm penalties} you suffer as penalties to concentration checks.

        Success on a concentration check generally means you successfully use the ability that triggered the check.
        Failure generally means that you do not use the ability successfully.

    \subsection{Miscasting}\label{Miscasting}

        If you start using a spell or ritual and fail to complete it successfully for any reason, you miscast it.
        A miscast spell or ritual does not have its normal effect.
        Instead, a wave of magical energy causes a \glossterm{miscast backlash}.
        When a mystic backlash occurs, make an attack against the Mental defense of yourself and all creatures in a 5 foot radius from you.
        Your \glossterm{power} with this ability is equal to your \glossterm{power} with the spell or ritual you tried to use.
        % damage type?
        On a hit, each target takes \glossterm{standard damage} \minus1d.

        \parhead{Impossible Spells and Rituals} When you try to use a spell or ritual in an impossible way, the ability is \glossterm{miscast} instead.
        This most commonly happens if you attempt to declare an invalid target for a spell.
        For example, if you try to cast a spell that only affects living creatures on a creature that is undead, the spell would be miscast.

        % TODO: This may be more clear if minor + no miscast becomes a tag.
        % Also, would it be better to prohibit miscasts on spells that don't cost action points?
        \parhead{Minor Action Spells} Spells that require a \glossterm{minor action} to cast are not miscast if your concentration is broken while casting the spell.
        Instead, the spell simply fizzles without any effect.

    \subsection{Focused and Defensive Casting}\label{Focused and Defensive Casting}
        The concentration required to use spells and rituals can impair your ability to defend yourself.
        When you use a spell or ritual, you choose whether to use \glossterm{focused casting} or \glossterm{defensive casting}.
        For rituals, you can change your choice during each phase that you spend performing the ritual.
        % This is a hack to prevent playing games involving delaying casting until the delayed action phase
        % to avoid being disruptible without taking defense penalties.
        You cannot use \glossterm{focused casting} if you are distracted, as your concentration is divided.
        You are distracted if you are experiencing violent motion, such as if you are on a galloping horse, or if you already took damage during the current round.

        \subsubsection{Focused Casting}\label{Focused Casting}
            If you use a spell or ritual with \glossterm{focused casting}, you suffer a \minus4 penalty to all defenses during each \glossterm{phase} where you concentrate on the ability.

        \subsubsection{Defensive Casting}\label{Defensive Casting}
            If you use a spell or ritual using \glossterm{defensive casting}, you suffer no penalty to defenses.
            In exchange, you have to make a \glossterm{concentration} check to use the ability successfully (see \pcref{Concentration}).
            The \glossterm{difficulty rating} of the check is equal to 4 if the spell is a cantrip.
            Otherwise, it is equal to 5 \add three times the level of the spell you are casting, as shown on \tref{Concentration DRs}.
            Success means that you use the ability successfully.
            Failure means that you \glossterm{miscast} the spell or ritual (see Miscasting, below).

            When you use defensive casting to perform a ritual, you must make a new check during each phase.

            \begin{dtable}
                \lcaption{Concentration DRs}
                \begin{dtabularx}{\columnwidth}{l X}
                    \tb{Spell Level} & \tb{DR} \\
                    \bottomrule
                    Cantrip & 4 \\
                    1st level & 8 \\
                    2nd level & 11 \\
                    3rd level & 15 \\
                    4th level & 18 \\
                    5th level & 21 \\
                    6th level & 24 \\
                    7th level & 27 \\
                \end{dtabularx}
            \end{dtable}


    \subsection{Augments}\label{Augments}
        There are a number of \glossterm{augments} that can be applied to spells and rituals to increase their power.
        Each augment has a name, a level, and an effect.
        When you cast a spell or perform a ritual, can choose to apply any number of augments you know to the spell or ritual.
        For each augment you apply, you increase the spell or ritual's level by an amount equal to the augment's level.
        In exchange, the ability gains the effects of that augment.
        If an augment would increase the spell or ritual's level beyond the maximum level you can cast, you cannot apply the augment to that ability.

        \parhead{Augments and Cantrips}
        You can apply augments to \glossterm{cantrips}.
        Cantrips are considered to be 1st level spells for this purpose.

        \subsubsection{Augment Descriptions}\label{Augment Descriptions}

            \augment{1}{Cryptic} The spell's visual effects and magical aura changes to mimic a different spell of your choice.
            You may choose any combination of spells you know, along with any other augments, that result in a spell of the same level or lower as the spell you are casting.
            This affects inspection of the spell itself by any means, such as with the Spellcraft skill (see \pcref{Spellcraft}).
            However, it does not alter the mechanical effects of the spell in any way.
            If the spell's effects depend on visual components, the spell may fail to work if you alter the spell's visuals too much. 

            \augment{1}{Selective} You may freely exclude any areas from the spell's effect.
            However, all squares in the final area of the spell must be contiguous.
            You cannot create split a spell's area into multiple completely separate areas.
            \par This augment can be applied to any spell or ritual that affects an area.

            \augment{1}{Silent} You do not need to use \glossterm{verbal components} to cast the spell.
            \par This augment can be applied to any spell.

            \augment{1}{Stilled} You do not need to use \glossterm{somatic components} to cast the spell.
            \par This augment can be applied to any spell.

            \augment{2}{Accelerated} The ritual takes half the normal amount of time to perform.
            \par This augment can be applied to any ritual.

            \augment{2}{Empowered} The ability's \glossterm{power} increases by 2.
            This augment can be applied multiple times.
            Its effects stack.
            \par This augment can be applied to any spell or ritual.

            % TODO: Too weak at 2?
            \augment{2}{Extended} The ability's range increases by one step, to a maximum of \rngext.
            The steps are, in order: \rngclose, \rngmed, \rnglong, and \rngext.
            This augment can be applied multiple times.
            Each time, the ability's range increases by an additional step.
            \par This augment can be applied to any spell or ritual with a range that is one of the above ranges.

            % Too weak at 2, weird narrative at 1?
            \augment{2}{Giant} The ability can affect a target one size category larger.
            This augment can be applied multiple times.
            Its effects stack.
            \par This augment can be applied to any spell or ritual that that has a maximum size category of targets that it can affect.

            \augment{2}{Precise} You gain a \plus1 bonus to accuracy with the ability.
            This augment can be applied multiple times.
            Its effects stack.
            \par This augment can be applied to any spell or ritual that has an attack roll.

            \augment{2}{Quickened} You can cast the spell as a \glossterm{minor action}.
            In exchange, you cannot take any actions during the \glossterm{action phase} or \glossterm{delayed action phase} of the next round.
            \par This augment can be applied to any spell.

            \augment{2}{Reach} When you cast the spell, choose a location within \rngclose range.
            The spell takes effect as if you were in the chosen location.
            This affects your \glossterm{line of effect} for the spell, but not your \glossterm{line of sight} (since you still see from your normal location).
            % Wording?
            Since a spell's range is measured from your location, this augment can allow you to affect targets outside your normal range.
            For example, a cone that normally bursts out from you would instead originate from your chosen location, potentially avoiding an obstacle between you and your target.
            % Are there other edge cases that could cause problems?
            \par This augment can be applied to any spell that does not create an \glossterm{emanation}.

            % too weak at 2?
            \augment{2}{Widened} The ability's area increases by one step, to a maximum of \areahuge.
            The steps are, in order: \areasmall, \areamed, \arealarge, and \areahuge.
            Normally, a Small or Medium line is 5 ft.\ wide, while a Large or Huge line is 10 ft.\ wide.
            A line used to define a wall does not have a width.
            This augment can be applied multiple times.
            Each time, the ability's area increases by an additional step.
            \par This augment can be applied to any spell or ritual with an area that is one of the above areas.

            % too weak at 3?
            \augment{3}{Phasing} When determining whether you have \glossterm{line of sight} and \glossterm{line of effect} to a particular location with the spell, you can ignore a single solid obstacle up to five feet thick.
            This can allow you to cast spells through solid walls, though it does not grant you the ability to see through the wall.
            \par This augment can be applied to any spell with a range.

            % \augment{4}{Dual} The spell targets an additional creature within range.
            % This augment can be applied multiple times.
            % Its effects stack.
            % \par This augment can be applied to any spell that has a range and affects a single target of the caster's choice.
            % It cannot be applied to spells that affect a single specific target, such as the caster.
            % In addition, it cannot be applied to \glossterm{Attune} spells.

            % too weak at 4?
            \augment{4}{Echoing} During the \glossterm{delayed action phase} of the next round, the spell's effect occurs again.
            All choices you made for the original casting of the spell are made identically for the repeat casting.
            It affects the same area, targets, and so on.

            If the spell is now invalid, such as if all of its targets are out of range, the additional casting has no effect.
            \par This augment can be applied to any spell that does not have the \glossterm{Attune} tag.

            % Are there any exploitable spells?
            \augment{4}{Innate} The spell loses the \glossterm{AP} tag, allowing you to cast it without spending an \glossterm{action point}.
            \par This augment can be applied to any spell with the \glossterm{AP} tag that requires a standard action to cast and which does not have the \glossterm{Sustain} tag.

    \subsection{Dismissal}
        As a \glossterm{minor action}, you can dismiss any spells or rituals you used that have lasting effects.
        This requires the same casting components (verbal and somatic) as casting the spell or performing the ritual normally.
        Spells and rituals can also be dismissed in other ways, such as after their effects have finished.
        When a spell or ability is dismissed, all of its lingering effects immediately end.
        % Should a spell be dismissed for all targets simultaneously, or per target?

    \subsection{Spell Power}

        The \glossterm{power} of your spells and rituals is the same as your \glossterm{power} with other magical abilities, which is equal to the higher of your level and Willpower.
        You can voluntarily reduce the \glossterm{power} of the spells you cast.
        % Keeping the ``twice spell level'' progression is fine despite that no longer matching spell acquisition
        The minimum power you can use to cast a spell is equal to twice the spell's level.

    \subsection{Resurrecting the Dead}\label{Resurrecting the Dead}
        Several rituals have the power to restore dead characters to life.

        When a living creature dies, its soul departs its body, travels through the Astral Plane, and goes to abide on the plane where the creature's deity resides.
        If the creature did not worship a deity, its soul departs to the plane corresponding to its alignment.
        Bringing a creature back from the dead means retrieving their soul and returning it to their body.

        \subparhead{Death and Old Age} While a creature is dead, it still tracks that time towards its maximum age.
        A creature's maximum age is largely determined by the strength of its soul, not the condition of its body.
        No magic can return a creature to life when it has passed its maximum age.

        \subparhead{Preventing Revivification} Enemies can take steps to make it more difficult for a character to be returned from the dead.
        Except for \spell{true resurrection}, every ritual to raise the dead requires a body, so keeping or destroying the body is an effective deterrent.
        The \spell{soul bind} ritual prevents any sort of revivification unless the soul is first released.

        \subparhead{Revivification against One's Will} A soul cannot be returned to life if it does not wish to be.
        A soul infallibly knows the name, alignment, and patron deity (if any) of the character attempting to revive it and may refuse to return on that basis.

\section{Spells}\label{Spells}
    % TODO: better description
    A \glossterm{spell} is a discrete magical effect with a name, a level, and an effect.
    Each \glossterm{mystic sphere} has a number of spells associated with it.
    An ability that gives you access to \glossterm{mystic spheres} will define how many spells you know from that sphere.
    It will also define the maximum level of spell you can cast.

    When you cast a spell, you can apply any number of \glossterm{augments} to the spell (see \pcref{Augments}).
    You cannot learn or cast spells whose spell level exceeds your maximum spell level.

    \subsection{Action Points}
        Most spells have the \glossterm{AP} tag, which means they cost an \glossterm{action point} to cast.
        However, some spells have the \glossterm{Attune} tag instead, and do not cost action points to cast.
        Instead, they require a creature to attune to the spell's effect, which costs an action point.
        For details, see \pcref{Attunement}.

    \subsection{Cantrips}\label{Cantrips}
        Each \glossterm{mystic sphere} has a \glossterm{cantrip}.
        A cantrip is a minor spell that does not require an \glossterm{action point} to cast.
        Cantrips are considered to be 1st level spells for the purpose of abilities that care about spell level, such as \glossterm{augments} (see \pcref{Augments}).
        Because cantrips are so simple, the \glossterm{difficulty rating} to maintain \glossterm{concentration} on a cantrip is 4 lower than it is for other spells (see \pcref{Concentration}).

\section{Rituals}\label{Rituals}
    Each \glossterm{mystic sphere} has a number of \glossterm{rituals}.
    Some spellcasting characters can learn and perform rituals.
    Rituals are ceremonies that create magical effects.
    Like spells, each ritual has a name, a level, and an effect.
    Although rituals are similar to spells, abilities that affect spells do not affect rituals unless they say they do in their descriptions.
    % Oddly located
    An ability that gives you access to rituals will define the maximum level of ritual that you can perform.

    You don't memorize a ritual as you would a normal spell.
    Rituals are too complex for all but the most knowledgeable sages to commit to memory.
    To perform a ritual, you need to read from a book or a scroll containing it.
    You must have access to the \glossterm{mystic sphere} a ritual is from in order to perform the ritual.

    \subsection{Ritual Descriptions}
        % TODO: proper chapter references
        Rituals are described in the body of the \glossterm{mystic sphere} they are associated with, following the description of spells from that mystic sphere.

    \subsection{Ritual Books}
        A ritual book contains one or more rituals that you can use as frequently as you want, as long as you can spend the time and \glossterm{action points} to perform the ritual.
        Scribing a ritual in a ritual book costs an amount of precious inks.

    \subsection{Ritual Costs}\label{Ritual Costs}
        The costs to scribe rituals are described on \trefnp{Ritual Costs}.
        \begin{dtable}
            \lcaption{Ritual Costs}
            \begin{dtabularx}{\columnwidth}{X l l}
                \tb{Ritual Level} & \tb{Cost to Scribe} & \tb{Item Level} \\
                \bottomrule
                1st-Level & 125 gp     & 2nd  \\
                2nd-Level & 800 gp     & 5th  \\
                3rd-Level & 2,750 gp   & 8th  \\
                4th-Level & 10,000 gp  & 11th \\
                5th-Level & 37,000 gp  & 14th \\
                6th-Level & 125,000 gp & 17th \\
                7th-Level & 400,000 gp & 20th \\
            \end{dtabularx}
        \end{dtable}

    \subsection{Performing Rituals}
        To perform a ritual, you must have a ritual book containing the ritual and the material components required for the ritual.
        Unless otherwise specified, performing a ritual requires spending a single \glossterm{action point}.
        Some rituals require multiple action points to complete.
        Other creatures can supply action points to help you perform rituals; see Ritual Participants, below.

        % The action point cost for 24 hour rituals is equal to (ritual level ^ 2) * 2.
        % Should this be specified explicitly?

        Rituals with the \glossterm{Attune} (ritual) tag do not cost action points to perform, but require a ritual participant to attune to the ritual's effect, which costs an action point.
        For details, see \pcref{Attunement}.

        \subsubsection{Ritual Participants}
            Creatures can assist in the performance of rituals even if they are unable to perform rituals themselves.
            A creature that helps perform a ritual is called a ritual participant, and the creature performing the ritual is called the ritual leader.
            A ritual participant may spend an action point in place of or in addition to the action point spent by the creature performing the ritual.
            It may also \glossterm{attune} to the effect of the ritual in place of the creature performing the ritual.
            Only one creature may attune to the ritual's effect in this way.
            If multiple creatures are willing to spend action points or attune to effects, the ritual leader decides which creatures spend action points or attune to the ritual's effects.

            The steps required to participate in rituals can be complex.
            Ritual participants must be given specific instructions for the actions they must perform during a ritual by a creature who knows how to perform the ritual.
            This instruction generally takes half the time required to perform the ritual.
            A creature cannot participate in rituals unless it has an Intelligence of at least 0, can speak at least one language, and has the fine motor control required to perform the \glossterm{somatic components} of rituals.

            Normally, a ritual participant can only contribute one action point.
            If the participant has access to the same \glossterm{magic source} as the ritual, they can contribute any number of action points.

            \parhead{Changing Ritual Participation}
            Rituals are deeply complex magic, and they cannot be abandoned or paused partway through.
            If the number of ritual participants in a ritual decreases below its initial value, the ritual fails at the end of the next round if the number of participants is not restored.
            However, ritual participants can transfer their participation to other creatures without disrupting the ritual.

            In order to transfer ritual participation, the new creature must be able to participate in the ritual, and must immediately spend the same number of action points as the creature that it is taking over from.
            Similarly, the ritual leader can transfer their leadership to another creature.
            In addition to the requirements for transferring ritual participation, the new leader must know the ritual and be able to perform it themselves.

            Changing ritual participation and leadership is usually done when performing extraordinarily long or demanding rituals.

    \subsection{Magical Writings}
        To record a spell in written form, a character uses complex notation that describes the magical forces involved in the spell.
        The notation constitutes a universal language that spellcasters have discovered, not invented.
        Each writer uses this universal system regardless of their native language or culture.
        However, each character uses the system in their own way.
        Another person's magical writing remains incomprehensible to even the most powerful spellcaster until they take the time to study and decipher it.

        You can identify magical writing using the Spellcraft skill (see \pcref{Identify Magical Writing}).


\section{Mystic Sphere Lists}\label{Mystic Sphere Lists}

    
\small
\subsection{Arcane Magic}\label{Arcane Magic}
\subsubsection{Arcane Mystic Spheres}\label{Arcane Mystic Spheres}
\begin{spelllist}
\spellhead{Astromancy} Transport creatures and objects instantly through space.
\spellhead{Barrier} Shield allies and areas from hostile forces.
\spellhead{Chronomancy} Manipulate the passage of time to inhibit foes and aid allies.
\spellhead{Compel} Bend creatures to your will by controlling their actions.
\spellhead{Corruption} Weaken the life force of foes, reducing their combat prowess.
\spellhead{Cryomancy} Drain heat to injure and freeze foes.
\spellhead{Delusion} Instill false emotions to influence creatures.
\spellhead{Electromancy} Create electricity to injure and stun foes.
\spellhead{Fabrication} Create objects to damage and impair foes.
\spellhead{Glamer} Change how creatures and objects are perceived.
\spellhead{Photomancy} Create bright light to blind foes and illuminate your surroundings.
\spellhead{Polymorph} Change the physical forms of objects and creatures.
\spellhead{Pyromancy} Create fire to incinerate foes.
\spellhead{Revelation} Share visions of the present and future, granting insight or combat prowess.
\spellhead{Scry} See and hear at great distances.
\spellhead{Summon} Summon creatures to fight with you.
\spellhead{Telekinesis} Manipulate creatures and objects at a distance.
\spellhead{Terramancy} Manipulate earth to crush foes.
\spellhead{Thaumaturgy} Suppress and manipulate magical effects.
\spellhead{Weaponcraft} Create and manipulate weapons to attack foes.
\end{spelllist}



\small
\subsection{Divine Magic}\label{Divine Magic}
\subsubsection{Divine Mystic Spheres}\label{Divine Mystic Spheres}
\begin{spelllist}
\spellhead{Barrier} Shield allies and areas from hostile forces.
\spellhead{Bless} Grant divine blessings to aid allies and improve combat prowess.
\spellhead{Channel Divinity} Invoke divine power to smite foes and gain power.
\spellhead{Compel} Bend creatures to your will by controlling their actions.
\spellhead{Corruption} Weaken the life force of foes, reducing their combat prowess.
\spellhead{Delusion} Instill false emotions to influence creatures.
\spellhead{Photomancy} Create bright light to blind foes and illuminate your surroundings.
\spellhead{Revelation} Share visions of the present and future, granting insight or combat prowess.
\spellhead{Scry} See and hear at great distances.
\spellhead{Summon} Summon creatures to fight with you.
\spellhead{Thaumaturgy} Suppress and manipulate magical effects.
\spellhead{Vital Surge} Alter life energy to cure or inflict wounds.
\spellhead{Weaponcraft} Create and manipulate weapons to attack foes.
\end{spelllist}



\small
\subsection{Nature Magic}\label{Nature Magic}
\subsubsection{Nature Mystic Spheres}\label{Nature Mystic Spheres}
\begin{spelllist}
\spellhead{Aeromancy} Command air to protect allies and blast foes.
\spellhead{Aquamancy} Command water to crush and drown foes.
\spellhead{Barrier} Shield allies and areas from hostile forces.
\spellhead{Corruption} Weaken the life force of foes, reducing their combat prowess.
\spellhead{Cryomancy} Drain heat to injure and freeze foes.
\spellhead{Electromancy} Create electricity to injure and stun foes.
\spellhead{Photomancy} Create bright light to blind foes and illuminate your surroundings.
\spellhead{Polymorph} Change the physical forms of objects and creatures.
\spellhead{Pyromancy} Create fire to incinerate foes.
\spellhead{Revelation} Share visions of the present and future, granting insight or combat prowess.
\spellhead{Scry} See and hear at great distances.
\spellhead{Summon} Summon creatures to fight with you.
\spellhead{Terramancy} Manipulate earth to crush foes.
\spellhead{Verdamancy} Animate and manipulate plants.
\spellhead{Vital Surge} Alter life energy to cure or inflict wounds.
\end{spelllist}



\small
\subsection{Pact Magic}\label{Pact Magic}
\subsubsection{Pact Mystic Spheres}\label{Pact Mystic Spheres}
\begin{spelllist}
\spellhead{Astromancy} Transport creatures and objects instantly through space.
\spellhead{Chronomancy} Manipulate the passage of time to inhibit foes and aid allies.
\spellhead{Compel} Bend creatures to your will by controlling their actions.
\spellhead{Corruption} Weaken the life force of foes, reducing their combat prowess.
\spellhead{Cryomancy} Drain heat to injure and freeze foes.
\spellhead{Delusion} Instill false emotions to influence creatures.
\spellhead{Electromancy} Create electricity to injure and stun foes.
\spellhead{Fabrication} Create objects to damage and impair foes.
\spellhead{Photomancy} Create bright light to blind foes and illuminate your surroundings.
\spellhead{Polymorph} Change the physical forms of objects and creatures.
\spellhead{Pyromancy} Create fire to incinerate foes.
\spellhead{Telekinesis} Manipulate creatures and objects at a distance.
\spellhead{Weaponcraft} Create and manipulate weapons to attack foes.
\end{spelllist}


\section{Mystic Sphere Descriptions}\label{Mystic Sphere Descriptions}

    
\begin{spellsection}{Aeromancy}

\begin{spellheader}
\spelldesc{Command air to protect allies and blast foes.}
\end{spellheader}


\parhead{Schools} Transmutation

\parhead{Mystic Sphere Lists} Nature

\subsubsection{Cantrips}


\begin{freeability}{Minor Windstrike}[\glossterm{Air}]
Make an attack vs. Armor against a creature or object within \rngmed range.
\hit The target takes bludgeoning \glossterm{standard damage}.
\end{freeability}


\begin{freeability}{Soften Landing}[\glossterm{Air}]
Choose a willing creature in \rngmed range.
Until the end of the round, the target treats all falls as if they were 5 feet shorter per \glossterm{power} for the purpose of determining \glossterm{falling damage}.
\end{freeability}

\end{spellsection}


\subsubsection{Spells}


\lowercase{\hypertarget{spell:Cyclone}{}}\label{spell:Cyclone}
\begin{apability}[\nth{1}]{\hypertarget{spell:Cyclone}{Cyclone}}[\glossterm{Air}]
Make an attack vs. Armor against everything in a \areasmall radius within \rngmed range.
\hit Each target takes bludgeoning \glossterm{standard damage}.
\end{apability}
\vspace{0.25em}



\lowercase{\hypertarget{spell:Propulsion}{}}\label{spell:Propulsion}
\begin{apability}[\nth{1}]{\hypertarget{spell:Propulsion}{Propulsion}}[\glossterm{Air}, \glossterm{Swift}]
Choose a willing creature in \rngclose range.
You move the target up to 50 feet in any direction.
You cannot change direction partway through the movement.
Moving the target upwards cost twice the normal movement cost.
\end{apability}
\vspace{0.25em}



\lowercase{\hypertarget{spell:Wind Screen}{}}\label{spell:Wind Screen}
\begin{attuneability}[\nth{1}]{\hypertarget{spell:Wind Screen}{Wind Screen}}[\glossterm{Air}, \glossterm{Attune} (target), \glossterm{Shielding}]
Choose a willing creature in \rngclose range.
The target gains a \plus1 \glossterm{magic bonus} to Armor defense.
This bonus is increased to \plus5 against ranged \glossterm{physical attacks} from weapons or projectiles that are Small or smaller.

You can cast this spell as a \glossterm{minor action}.
Any effect which increases the size of creature this spell can affect also increases the size of ranged weapon it defends against by the same amount.
\end{attuneability}
\vspace{0.25em}



\lowercase{\hypertarget{spell:Windblade}{}}\label{spell:Windblade}
\begin{attuneability}[\nth{1}]{\hypertarget{spell:Windblade}{Windblade}}[\glossterm{Air}, \glossterm{Attune} (target), \glossterm{Shaping}]
Choose a willing creature within \rngclose range.
Melee weapons wielded by the target gain an additional five feet of \glossterm{reach}.
This has no effect on ranged attacks the target makes.

You can cast this spell as a \glossterm{minor action}.
\end{attuneability}
\vspace{0.25em}



\lowercase{\hypertarget{spell:Windstrike}{}}\label{spell:Windstrike}
\begin{apability}[\nth{1}]{\hypertarget{spell:Windstrike}{Windstrike}}[\glossterm{Air}]
Make an attack vs. Armor against a creature or object within \rngmed range.
\hit The target takes bludgeoning \glossterm{standard damage} \plus2d.
\end{apability}
\vspace{0.25em}



\lowercase{\hypertarget{spell:Gentle Descent}{}}\label{spell:Gentle Descent}
\begin{attuneability}[\nth{2}]{\hypertarget{spell:Gentle Descent}{Gentle Descent}}[\glossterm{Air}, \glossterm{Attune} (target)]
Choose a willing, Large or smaller creature in \rngclose range.
The target gains a 30 foot \glossterm{glide speed} (see \pcref{Gliding}).
\end{attuneability}
\vspace{0.25em}



\lowercase{\hypertarget{spell:Greater Propulsion}{}}\label{spell:Greater Propulsion}
\begin{apability}[\nth{2}]{\hypertarget{spell:Greater Propulsion}{Greater Propulsion}}[\glossterm{Air}, \glossterm{Swift}]
This spell functions like the \spell{propulsion} spell, except that the distance you can move the target is increased to 100 feet.
In addition, the target gains a \plus1d bonus to damage with melee \glossterm{strikes} during the same phase.
\end{apability}
\vspace{0.25em}



\lowercase{\hypertarget{spell:Gust of Wind}{}}\label{spell:Gust of Wind}
\begin{apability}[\nth{2}]{\hypertarget{spell:Gust of Wind}{Gust of Wind}}[\glossterm{Air}]
Make an attack vs. Armor against everything in a \arealarge, 10 ft. wide line from you.
\hit Each target takes bludgeoning \glossterm{standard damage}.
\end{apability}
\vspace{0.25em}



\lowercase{\hypertarget{spell:Stripping Cyclone}{}}\label{spell:Stripping Cyclone}
\begin{apability}[\nth{2}]{\hypertarget{spell:Stripping Cyclone}{Stripping Cyclone}}[\glossterm{Air}]
This spell functions like the \spell{cyclone} spell, except that the attack result is also compared to each target's Reflex defense.
\hit Each target drops all items it is holding that are not well secured (such as a ring) or held in two hands.
\end{apability}
\vspace{0.25em}



\lowercase{\hypertarget{spell:Stripping Windstrike}{}}\label{spell:Stripping Windstrike}
\begin{apability}[\nth{2}]{\hypertarget{spell:Stripping Windstrike}{Stripping Windstrike}}[\glossterm{Air}]
This spell functions like the \spell{windstrike} spell, except that the attack result is also compared to the target's Reflex defense.
% Clarify: this can hit even if the damaging effect misses
\hit The target drops all items it is holding that are not well secured (such as a ring) or held in two hands.
\end{apability}
\vspace{0.25em}



\lowercase{\hypertarget{spell:Greater Cyclone}{}}\label{spell:Greater Cyclone}
\begin{apability}[\nth{3}]{\hypertarget{spell:Greater Cyclone}{Greater Cyclone}}[\glossterm{Air}]
This spell functions like the \spell{cyclone} spell, except that it affects everything in a \areamed radius within \rnglong range.
\end{apability}
\vspace{0.25em}



\lowercase{\hypertarget{spell:Greater Windstrike}{}}\label{spell:Greater Windstrike}
\begin{apability}[\nth{3}]{\hypertarget{spell:Greater Windstrike}{Greater Windstrike}}[\glossterm{Air}]
This spell functions like the \spell{windstrike} spell, except that it affects a target within \rnglong range and you gain a \plus1d bonus to damage.
\end{apability}
\vspace{0.25em}



\lowercase{\hypertarget{spell:Stormlord}{}}\label{spell:Stormlord}
\begin{attuneability}[\nth{3}]{\hypertarget{spell:Stormlord}{Stormlord}}[\glossterm{Air}, \glossterm{Attune} (target), \glossterm{Shielding}]
This spell functions like the \spell{wind screen} spell, except that the air also retaliates against creatures that attack the target.
When a creature within \rngclose range of the target attacks it, make an attack vs. Armor against the attacking creature.
A hit deals bludgeoning \glossterm{standard damage} \minus1d.
Any individual creature can only be dealt damage in this way once per round.

Any effect which increases this spell's range increases the range of this retaliation by the same amount.
\end{attuneability}
\vspace{0.25em}



\lowercase{\hypertarget{spell:Stripping Gust of Wind}{}}\label{spell:Stripping Gust of Wind}
\begin{apability}[\nth{3}]{\hypertarget{spell:Stripping Gust of Wind}{Stripping Gust of Wind}}[\glossterm{Air}]
This spell functions like the \spell{gust of wind} spell, except that the attack result is also compared to each target's Reflex defense.
\hit Each target drops all items it is holding that are not well secured (such as a ring) or held in two hands.
\end{apability}
\vspace{0.25em}



\lowercase{\hypertarget{spell:Air Walk}{}}\label{spell:Air Walk}
\begin{attuneability}[\nth{4}]{\hypertarget{spell:Air Walk}{Air Walk}}[\glossterm{Air}, \glossterm{Attune} (target)]
Choose a willing creature in \rngclose range.
The target can walk on air as if it were solid ground.
The magic only affects the target's legs and feet.
By choosing when to treat the air as solid, it can traverse the air with ease.
\end{attuneability}
\vspace{0.25em}



\lowercase{\hypertarget{spell:Control Weather}{}}\label{spell:Control Weather}
\begin{attuneability}[\nth{4}]{\hypertarget{spell:Control Weather}{Control Weather}}[\glossterm{Air}, \glossterm{Attune} (self)]
When you cast this spell, you choose a new weather pattern.
You can only choose weather which would be possible in the climate and season of the area you are in.
For example, you can normally create a thunderstorm, but not if you are in a desert.

When you complete the spell, the weather begins to take effect in a two mile radius cylinder-shaped zone centered on from your location.
After five minutes, your chosen weather pattern fully takes effect.

You can control the general tendencies of the weather, such as the direction and intensity of the wind.
You cannot control specific applications of the weather -- where lightning strikes, for example, or the exact path of a tornado.
Contradictory weather conditions are not possible simultaneously.

After the spell's effect ends, the weather continues on its natural course, which may cause your chosen weather pattern to end.
% TODO: This should be redundant with generic spell mechanics
If another ability would magically manipulate the weather in the same area, the most recently used ability takes precedence.
\end{attuneability}
\vspace{0.25em}



\lowercase{\hypertarget{spell:Greater Wind Screen}{}}\label{spell:Greater Wind Screen}
\begin{attuneability}[\nth{4}]{\hypertarget{spell:Greater Wind Screen}{Greater Wind Screen}}[\glossterm{Air}, \glossterm{Attune} (target), \glossterm{Shielding}]
This spell functions like the \spell{wind screen} spell, except that the Armor defense bonus increases to \plus2 and the defense bonus against ranged attacks increases to \plus10.
\end{attuneability}
\vspace{0.25em}



\lowercase{\hypertarget{spell:Greater Windblade}{}}\label{spell:Greater Windblade}
\begin{attuneability}[\nth{4}]{\hypertarget{spell:Greater Windblade}{Greater Windblade}}[\glossterm{Air}, \glossterm{Attune} (target), \glossterm{Shaping}]
Choose a willing creature within \rngclose range.
Melee weapons wielded by the target gain an additional ten feet of \glossterm{reach}.
In addition, the target gains a \plus1d \glossterm{magic bonus} to damage with melee \glossterm{strikes}.
This has no effect on ranged attacks the target makes.

You can cast this spell as a \glossterm{minor action}.
\end{attuneability}
\vspace{0.25em}



\lowercase{\hypertarget{spell:Supreme Propulsion}{}}\label{spell:Supreme Propulsion}
\begin{apability}[\nth{4}]{\hypertarget{spell:Supreme Propulsion}{Supreme Propulsion}}[\glossterm{Air}, \glossterm{Swift}]
This spell functions like the \spell{propulsion} spell, except that the distance you can move the target is increased to 300 feet.
In addition, the target gains a \plus2d bonus to damage with melee \glossterm{strikes} during the same phase.
\end{apability}
\vspace{0.25em}



\lowercase{\hypertarget{spell:Greater Gust of Wind}{}}\label{spell:Greater Gust of Wind}
\begin{apability}[\nth{5}]{\hypertarget{spell:Greater Gust of Wind}{Greater Gust of Wind}}[\glossterm{Air}]
This spell functions like the \spell{gust of wind} spell, except that it affects everything in a \areahuge, 10 ft. wide line from you and you gain a \plus1d bonus to damage.
\end{apability}
\vspace{0.25em}



\lowercase{\hypertarget{spell:Supreme Windstrike}{}}\label{spell:Supreme Windstrike}
\begin{apability}[\nth{5}]{\hypertarget{spell:Supreme Windstrike}{Supreme Windstrike}}[\glossterm{Air}]
This spell functions like the \spell{windstrike} spell, except that it affects a target within \rngext range and you gain a \plus2d bonus to damage.
\end{apability}
\vspace{0.25em}



\lowercase{\hypertarget{spell:Greater Stormlord}{}}\label{spell:Greater Stormlord}
\begin{attuneability}[\nth{6}]{\hypertarget{spell:Greater Stormlord}{Greater Stormlord}}[\glossterm{Air}, \glossterm{Attune} (target), \glossterm{Shielding}]
This spell functions like the \spell{stormlord} spell, except that you gain a \plus2d bonus to damage.
\end{attuneability}
\vspace{0.25em}



\lowercase{\hypertarget{spell:Supreme Cyclone}{}}\label{spell:Supreme Cyclone}
\begin{apability}[\nth{6}]{\hypertarget{spell:Supreme Cyclone}{Supreme Cyclone}}[\glossterm{Air}]
This spell functions like the \spell{cyclone} spell, except that it affects everything in a \arealarge radius and you gain a \plus1d bonus to damage.
\end{apability}
\vspace{0.25em}


\newpage
\begin{spellsection}{Aquamancy}

\begin{spellheader}
\spelldesc{Command water to crush and drown foes.}
\end{spellheader}


\parhead{Schools} Conjuration

\parhead{Mystic Sphere Lists} Nature

\subsubsection{Cantrips}


\lowercase{\hypertarget{spell:Create Water}{}}\label{spell:Create Water}
\begin{freeability}[\nth{1}]{\hypertarget{spell:Create Water}{Create Water}}[\glossterm{Creation}, \glossterm{Water}]
You create up to one gallon per \glossterm{power} of wholesome, drinkable water anywhere within \rngclose range.
The water can be created at multiple locations within the ritual's range, allowing you to fill multiple small water containers.
You must create a minimum of one ounce of water in each location.
\end{freeability}
\vspace{0.25em}



\begin{freeability}{Whelming Wave}[\glossterm{Manifestation}, \glossterm{Water}]
Make an attack vs. Fortitude against everything in a \areamed, 5 ft.\ wide line from you.
\hit Each target takes bludgeoning \glossterm{standard damage} \minus1d.
\end{freeability}

\end{spellsection}


\subsubsection{Spells}


\lowercase{\hypertarget{spell:Crushing Wave}{}}\label{spell:Crushing Wave}
\begin{apability}[\nth{1}]{\hypertarget{spell:Crushing Wave}{Crushing Wave}}[\glossterm{Manifestation}, \glossterm{Water}]
Make an attack vs. Fortitude against everything in a \arealarge, 10 ft.\ wide line from you.
\hit Each target takes bludgeoning \glossterm{standard damage}.
\end{apability}
\vspace{0.25em}



\lowercase{\hypertarget{spell:Underwater Freedom}{}}\label{spell:Underwater Freedom}
\begin{attuneability}[\nth{1}]{\hypertarget{spell:Underwater Freedom}{Underwater Freedom}}[\glossterm{Attune} (target)]
Choose a willing creature within \rngclose range.
The target suffers no penalties for acting underwater, except for those relating to using ranged weapons.

You can cast this spell as a \glossterm{minor action}.
\end{attuneability}
\vspace{0.25em}



\lowercase{\hypertarget{spell:Water Jet}{}}\label{spell:Water Jet}
\begin{apability}[\nth{1}]{\hypertarget{spell:Water Jet}{Water Jet}}[\glossterm{Manifestation}, \glossterm{Water}]
Make an attack vs. Armor against a creature within \rngclose range.
\hit The target takes bludgeoning \glossterm{standard damage} \plus2d.
\end{apability}
\vspace{0.25em}



\lowercase{\hypertarget{spell:Aqueous Sphere}{}}\label{spell:Aqueous Sphere}
\begin{apability}[\nth{2}]{\hypertarget{spell:Aqueous Sphere}{Aqueous Sphere}}[\glossterm{Manifestation}, \glossterm{Water}]
This spell functions like the \spell{crushing wave} spell, except that it targets everything in a \areasmall radius within \rngmed range.
\end{apability}
\vspace{0.25em}



\lowercase{\hypertarget{spell:Geyser}{}}\label{spell:Geyser}
\begin{apability}[\nth{2}]{\hypertarget{spell:Geyser}{Geyser}}[\glossterm{Manifestation}, \glossterm{Water}]
Make an attack vs. Armor against everything in a \arealarge, 5 ft.\ wide vertical line within \rngmed range.
If this spell has its area increased, such as with the Widened \glossterm{augment}, only the length of the line increases.
\hit Each target takes takes bludgeoning \glossterm{standard damage} \plus2d.
\end{apability}
\vspace{0.25em}



\lowercase{\hypertarget{spell:Greater Underwater Freedom}{}}\label{spell:Greater Underwater Freedom}
\begin{attuneability}[\nth{3}]{\hypertarget{spell:Greater Underwater Freedom}{Greater Underwater Freedom}}[\glossterm{Attune} (target)]
This spell functions like the \spell{underwater freedom} spell, except that the target can also breathe water as if it was air.
\end{attuneability}
\vspace{0.25em}



\lowercase{\hypertarget{spell:Overpowering Wave}{}}\label{spell:Overpowering Wave}
\begin{apability}[\nth{3}]{\hypertarget{spell:Overpowering Wave}{Overpowering Wave}}[\glossterm{Manifestation}, \glossterm{Water}]
This spell functions like the \spell{crushing wave} spell, except that it attacks Reflex defense instead of Fortitude defense.
\end{apability}
\vspace{0.25em}



\lowercase{\hypertarget{spell:Raging River}{}}\label{spell:Raging River}
\begin{apability}[\nth{3}]{\hypertarget{spell:Raging River}{Raging River}}[\glossterm{Manifestation}, \glossterm{Water}]
This spell functions like the \spell{crushing wave} spell, except that it gains the \glossterm{Sustain} (standard) tag.
The area affected by the spell becomes a \glossterm{zone} that is continuously filled with rushing water.
Each struck target in the area suffers penalties appropriate for fighting underwater, and may be unable to breathe.
In addition, at the end of each \glossterm{action phase} in subsequent rounds, the attack is repeated in that area.
\end{apability}
\vspace{0.25em}



\lowercase{\hypertarget{spell:Wall of Water}{}}\label{spell:Wall of Water}
\begin{apability}[\nth{3}]{\hypertarget{spell:Wall of Water}{Wall of Water}}[\glossterm{Manifestation}, \glossterm{Water}]
You create a wall of water in a 20 ft.\ high, \arealarge line within \rngmed range.
The wall is four inches thick, and blocks \glossterm{line of effect} for abilities.
Sight through the wall is possible, though distorted.
The wall provides both \glossterm{passive cover} and \glossterm{concealment} to targets on the opposite side of the wall, for a total of a \plus4 bonus to Armor defense.
Creatures can pass through the wall unharmed, though it costs five extra feet of movement to move through the wall.

Each five-foot square of wall has hit points equal to four times your \glossterm{power}, and all of its defenses are 0.
It is immune to most forms of attack, but it can be destroyed by \glossterm{fire damage} and similar effects that can destroy water.
\end{apability}
\vspace{0.25em}



\lowercase{\hypertarget{spell:Greater Aqueous Sphere}{}}\label{spell:Greater Aqueous Sphere}
\begin{apability}[\nth{4}]{\hypertarget{spell:Greater Aqueous Sphere}{Greater Aqueous Sphere}}[\glossterm{Manifestation}, \glossterm{Water}]
This spell functions like the \spell{crushing wave} spell, except that it targets everything in a \areamed radius within \rnglong range.
\end{apability}
\vspace{0.25em}



\lowercase{\hypertarget{spell:Greater Raging River}{}}\label{spell:Greater Raging River}
\begin{apability}[\nth{5}]{\hypertarget{spell:Greater Raging River}{Greater Raging River}}[\glossterm{Manifestation}, \glossterm{Water}]
This spell functions like the \textit{raging river} spell, except that the spell gains the \glossterm{Sustain} (minor) tag instead of the \glossterm{Sustain} (standard) tag.
\end{apability}
\vspace{0.25em}



\subsubsection{Rituals}


\lowercase{\hypertarget{spell:Dampen}{}}\label{spell:Dampen}
\begin{attuneability}[\nth{1}]{\hypertarget{spell:Dampen}{Dampen}}[\glossterm{Attune} (ritual)]
Choose up to five willing ritual participants.
Each target gains a \glossterm{magic bonus} equal to your \glossterm{power} to damage reduction against fire damage.

You can cast this spell as a \glossterm{minor action}.
\end{attuneability}
\vspace{0.25em}



\lowercase{\hypertarget{spell:Water Breathing}{}}\label{spell:Water Breathing}
\begin{attuneability}[\nth{2}]{\hypertarget{spell:Water Breathing}{Water Breathing}}[\glossterm{Attune} (ritual)]
Choose a Medium or smaller willing ritual participant.
The target can breathe water as easily as a human breathes air, preventing it from drowning or suffocating underwater.

This ritual takes one minute to perform.
\end{attuneability}
\vspace{0.25em}


\newpage
\begin{spellsection}{Astromancy}

\begin{spellheader}
\spelldesc{Transport creatures and objects instantly through space.}
\end{spellheader}


\parhead{Schools} Conjuration

\parhead{Mystic Sphere Lists} Arcane, Pact

\subsubsection{Cantrips}


\begin{freeability}{Dimensional Disruption}[\glossterm{Planar}, \glossterm{Teleportation}]
Make an attack vs. Mental against a creature within \rngmed range.
\hit The target takes physical \glossterm{standard damage}.
\end{freeability}


\begin{freeability}{Minor Translocation}[\glossterm{Teleportation}]
Choose a Tiny or smaller unattended object within \rngclose range.
The target teleports into an unoccupied location on a stable surface within range that can support the weight of the target.
If the destination is invalid, the ability fails without effect.
\end{freeability}

\end{spellsection}


\subsubsection{Spells}


\lowercase{\hypertarget{spell:Dimensional Jaunt}{}}\label{spell:Dimensional Jaunt}
\begin{apability}[\nth{1}]{\hypertarget{spell:Dimensional Jaunt}{Dimensional Jaunt}}[\glossterm{Planar}, \glossterm{Teleportation}]
Make an attack vs. Mental against a creature within \rngmed range.
\hit The target takes physical \glossterm{standard damage} \plus2d.
\end{apability}
\vspace{0.25em}



\lowercase{\hypertarget{spell:Teleport}{}}\label{spell:Teleport}
\begin{apability}[\nth{1}]{\hypertarget{spell:Teleport}{Teleport}}[\glossterm{Teleportation}]
Choose a Medium or smaller willing creature or unattended object within \rngclose range.
The target teleports into an unoccupied destination within range.
If the destination is invalid, this spell is \glossterm{miscast}.
\end{apability}
\vspace{0.25em}



\lowercase{\hypertarget{spell:Banishment}{}}\label{spell:Banishment}
\begin{apability}[\nth{2}]{\hypertarget{spell:Banishment}{Banishment}}[\glossterm{Planar}, \glossterm{Teleportation}]
This spell functions like the \spell{dimensional jaunt} spell, except that it gains a \plus2 bonus to \glossterm{accuracy} against \glossterm{outsiders} not on their home planes and creatures created by \glossterm{Manifestation} abilities.
\crit The target takes double damage.
In addition, if it is an outsider not on its home plane, it is teleported to a random location on its home plane.
If it is a creature created by a \glossterm{Manifestation} ability, it immediately disappears.
\end{apability}
\vspace{0.25em}



\lowercase{\hypertarget{spell:Dimensional Jaunt -- Plane of Earth}{}}\label{spell:Dimensional Jaunt -- Plane of Earth}
\begin{apability}[\nth{2}]{\hypertarget{spell:Dimensional Jaunt -- Plane of Earth}{Dimensional Jaunt -- Plane of Earth}}[\glossterm{Planar}, \glossterm{Teleportation}]
This spell functions like the \spell{dimensional jaunt} spell, except that the target is partially teleported into the Plane of Earth.
The damage becomes bludgeoning damage, and a struck target is \glossterm{slowed} as a \glossterm{condition}.
\end{apability}
\vspace{0.25em}



\lowercase{\hypertarget{spell:Dimensional Shuffle}{}}\label{spell:Dimensional Shuffle}
\begin{apability}[\nth{2}]{\hypertarget{spell:Dimensional Shuffle}{Dimensional Shuffle}}[\glossterm{Teleportation}]
Choose up to five willing creatures within \rngmed range.
Each target teleports into the location of a different target.
\end{apability}
\vspace{0.25em}



\lowercase{\hypertarget{spell:Dimension Door}{}}\label{spell:Dimension Door}
\begin{apability}[\nth{3}]{\hypertarget{spell:Dimension Door}{Dimension Door}}[\glossterm{Teleportation}]
You teleport to a location within \rngext range of you.
You must clearly visualize the destination's appearance, but you do not need \glossterm{line of sight} or \glossterm{line of effect} to your destination.
\end{apability}
\vspace{0.25em}



\lowercase{\hypertarget{spell:Dimensional Jaunt -- Plane of Fire}{}}\label{spell:Dimensional Jaunt -- Plane of Fire}
\begin{apability}[\nth{3}]{\hypertarget{spell:Dimensional Jaunt -- Plane of Fire}{Dimensional Jaunt -- Plane of Fire}}[\glossterm{Planar}, \glossterm{Teleportation}]
This spell functions like the \spell{dimensional jaunt} spell, except that the target is partially teleported into the Plane of Fire.
The damage becomes fire damage and increases by \plus1d.
In addition, a struck target is \glossterm{ignited} until it puts out the fire.
This condition can also be removed if the target makes a \glossterm{DR} 10 Dexterity check as a \glossterm{move action} to put out the flames.
Dropping \glossterm{prone} as part of this action gives a \plus5 bonus to this check.
\end{apability}
\vspace{0.25em}



\lowercase{\hypertarget{spell:Greater Teleport}{}}\label{spell:Greater Teleport}
\begin{apability}[\nth{3}]{\hypertarget{spell:Greater Teleport}{Greater Teleport}}[\glossterm{Teleportation}]
This spell functions like the \textit{teleport} spell, except that the range is increased to \rngext.
\end{apability}
\vspace{0.25em}



\lowercase{\hypertarget{spell:Blink}{}}\label{spell:Blink}
\begin{attuneability}[\nth{4}]{\hypertarget{spell:Blink}{Blink}}[\glossterm{Attune} (target), \glossterm{Planar}, \glossterm{Teleportation}]
Choose a willing creature within \rngclose range.
The target randomly blinks between its current plane and the Astral Plane.
This blinking stops if the target takes actions on its current plane.
In any phase where it does not take any actions, the target has a 50\% chance to completely ignore any effect that targets it directly.
It is still affected normally by abilities that affect an area.
\end{attuneability}
\vspace{0.25em}



\lowercase{\hypertarget{spell:Dimensional Jaunt -- Deep Astral Plane}{}}\label{spell:Dimensional Jaunt -- Deep Astral Plane}
\begin{apability}[\nth{5}]{\hypertarget{spell:Dimensional Jaunt -- Deep Astral Plane}{Dimensional Jaunt -- Deep Astral Plane}}[\glossterm{Planar}, \glossterm{Teleportation}]
This spell functions like the \spell{dimensional jaunt} spell, except that the target is partially teleported into the deep Astral Plane.
The damage increases by \plus1d.
In addition, a struck target is \glossterm{stunned} as a \glossterm{condition}.
\end{apability}
\vspace{0.25em}



\lowercase{\hypertarget{spell:Dimensional Jaunt -- Myriad}{}}\label{spell:Dimensional Jaunt -- Myriad}
\begin{apability}[\nth{6}]{\hypertarget{spell:Dimensional Jaunt -- Myriad}{Dimensional Jaunt -- Myriad}}[\glossterm{Planar}, \glossterm{Teleportation}]
This spell functions like the \spell{dimensional jaunt} spell, except that the target is partially teleported through a dizzying array of planes.
The damage increases by \plus3d and becomes damage of all types.
\end{apability}
\vspace{0.25em}



\lowercase{\hypertarget{spell:Greater Blink}{}}\label{spell:Greater Blink}
\begin{attuneability}[\nth{7}]{\hypertarget{spell:Greater Blink}{Greater Blink}}[\glossterm{Attune} (target), \glossterm{Planar}, \glossterm{Teleportation}]
This spell functions like the \spell{blink} spell, except that the target also has a 20\% chance to completely ignore any effect that targets it directly during phases where it takes an action.
\end{attuneability}
\vspace{0.25em}



\subsubsection{Rituals}


\lowercase{\hypertarget{spell:Plane Shift}{}}\label{spell:Plane Shift}
\begin{apability}[\nth{3}]{\hypertarget{spell:Plane Shift}{Plane Shift}}[\glossterm{Planar}, \glossterm{Teleportation}]
Choose up to five Large or smaller willing ritual participants.
In addition, you choose a \glossterm{planar rift} within \rngmed range to travel through.
The targets teleport to the unoccupied spaces closest to the other side of the planar rift.
For details about \glossterm{planar rifts}, see \pcref{Planar Rifts}.

% TODO: Is this planar cosmology correct?
The Astral Plane connects to every plane, but transit from other planes is usually more limited.
From the Material Plane, you can only reach the Astral Plane.

This ritual takes 24 hours to perform, and requires 18 action points from its participants.
\end{apability}
\vspace{0.25em}



\lowercase{\hypertarget{spell:Retrieve Legacy}{}}\label{spell:Retrieve Legacy}
\begin{apability}[\nth{3}]{\hypertarget{spell:Retrieve Legacy}{Retrieve Legacy}}[\glossterm{Teleportation}]
Choose a willing ritual participant.
If the target's \glossterm{legacy item} is on the same plane and \glossterm{unattended}, it is teleported into the target's hand.

This ritual takes 24 hours to perform, and requires 18 action points from its ritual participants.
\end{apability}
\vspace{0.25em}



\lowercase{\hypertarget{spell:Astral Projection}{}}\label{spell:Astral Projection}
\begin{apability}[\nth{4}]{\hypertarget{spell:Astral Projection}{Astral Projection}}[\glossterm{Planar}, \glossterm{Teleportation}]
Choose up to five Large or smaller willing ritual participants.
The targets teleport to a random location within the Inner Astral Plane (see \pcref{The Astral Plane}).

In addition, a localized \glossterm{planar rift} appears at the destination area on the Astral Plane which leads back to the location where this ritual was performed.
The rift can only be passed through by the targets of this effect.
It lasts for one week before disappearing permanently, potentially stranding the targets in the Astral Plane if they have not yet returned.

This ritual takes 24 hours to perform, and requires 32 action points from its participants.
\end{apability}
\vspace{0.25em}



\lowercase{\hypertarget{spell:Overland Teleportation}{}}\label{spell:Overland Teleportation}
\begin{apability}[\nth{4}]{\hypertarget{spell:Overland Teleportation}{Overland Teleportation}}[\glossterm{Teleportation}]
Choose up to five willing, Medium or smaller ritual participants.
In addition, choose a destination up to 100 miles away from you on your current plane.
Each target is teleported to the chosen destination.

You must specify the destination with a precise mental image of its appearance.
The image does not have to be perfect, but it must unambiguously identify the destination.
If you specify its appearance incorrectly, or if the area has changed its appearance, the destination may be a different area than you intended.
The new destination will be one that more closely resembles your mental image.
If no such area exists, the ritual simply fails.
% TODO: does this need more clarity about what teleportation works?

This ritual takes 24 hours to perform and requires 32 action points from its ritual participants.
\end{apability}
\vspace{0.25em}



\lowercase{\hypertarget{spell:Homeward Shift}{}}\label{spell:Homeward Shift}
\begin{apability}[\nth{5}]{\hypertarget{spell:Homeward Shift}{Homeward Shift}}[\glossterm{Planar}, \glossterm{Teleportation}]
This ritual can only be performed on the Astral Plane.
Choose up to five Large or smaller willing ritual participants.
The targets teleport to the last spaces they occupied on their home planes.

This ritual takes 24 hours to perform, and requires 50 action points from its participants.
\end{apability}
\vspace{0.25em}



\lowercase{\hypertarget{spell:Gate}{}}\label{spell:Gate}
\begin{apability}[\nth{7}]{\hypertarget{spell:Gate}{Gate}}[\glossterm{Planar}, \glossterm{Sustain} (standard), \glossterm{Teleportation}]
Choose a plane that connects to your current plane, and a location within that plane.
This ritual creates an interdimensional connection between your current plane and the location you choose, allowing travel between those two planes in either direction.
The gate takes the form of a \areasmall radius circular disk, oriented a direction you choose (typically vertical).
It is a two-dimensional window looking into the plane you specified when casting the spell, and anyone or anything that moves through it is shunted instantly to the other location.
The gate cannot be \glossterm{sustained} for more than 5 rounds, and is automatically dismissed at the end of that time.

You must specify the gate's destination with a precise mental image of its appearance.
The image does not have to be perfect, but it must unambiguously identify the location.
Incomplete or incorrect mental images may result in the ritual leading to an unintended destination within the same plane, or simply failing entirely.

% TODO: Is this planar cosmology correct?
The Astral Plane connects to every plane, but transit from other planes is usually more limited.
From the Material Plane, you can only reach the Astral Plane.

This ritual takes one week to perform, and requires 98 action points from its participants.
\end{apability}
\vspace{0.25em}


\newpage
\begin{spellsection}{Barrier}

\begin{spellheader}
\spelldesc{Shield allies from hostile forces.}
\end{spellheader}


\parhead{Schools} Abjuration

\parhead{Mystic Sphere Lists} Arcane, Divine, Nature

\subsubsection{Cantrips}


\begin{freeability}{Kinetic Shield}[\glossterm{Sustain} (standard)]
Choose a willing creature in \rngclose range.
The target gains a \glossterm{magic bonus} equal to your \glossterm{power} to \glossterm{damage reduction} against \glossterm{physical} damage.
\end{freeability}

\end{spellsection}


\subsubsection{Spells}


\lowercase{\hypertarget{spell:Ablative Shield}{}}\label{spell:Ablative Shield}
\begin{attuneability}[\nth{1}]{\hypertarget{spell:Ablative Shield}{Ablative Shield}}[\glossterm{Attune} (target), \glossterm{Shielding}]
Choose a willing creature in \rngclose range.
The target gains a \glossterm{magic bonus} equal to your \glossterm{power} to \glossterm{damage reduction} against all damage except for \glossterm{energy damage}.

You can cast this spell as a \glossterm{minor action}.
\end{attuneability}
\vspace{0.25em}



\lowercase{\hypertarget{spell:Resist Energy}{}}\label{spell:Resist Energy}
\begin{attuneability}[\nth{1}]{\hypertarget{spell:Resist Energy}{Resist Energy}}[\glossterm{Attune} (target), \glossterm{Shielding}]
Choose a willing creature in \rngclose range.
The target gains a \glossterm{magic bonus} equal to your \glossterm{power} to \glossterm{damage reduction} against \glossterm{energy damage}.

You can cast this spell as a \glossterm{minor action}.
\end{attuneability}
\vspace{0.25em}



\lowercase{\hypertarget{spell:Complete Shield}{}}\label{spell:Complete Shield}
\begin{attuneability}[\nth{2}]{\hypertarget{spell:Complete Shield}{Complete Shield}}[\glossterm{Attune} (target), \glossterm{Shielding}]
This spell functions like the \spell{ablative shield} spell, except that the damage reduction applies against all damage.
\end{attuneability}
\vspace{0.25em}



\lowercase{\hypertarget{spell:Repulsion Field}{}}\label{spell:Repulsion Field}
\begin{apability}[\nth{2}]{\hypertarget{spell:Repulsion Field}{Repulsion Field}}[\glossterm{Sustain} (minor)]
This spell creates a repulsive field in a \areamed radius zone from your location.
When an enemy makes physical contact with the spell's area for the first time, you make an attack vs. Mental against it.
\hit The target is unable to enter the spell's area with any part of its body.
The rest of its movement in the current phase is cancelled.

Creatures in the area at the time that the spell is cast are unaffected by the spell.
\end{apability}
\vspace{0.25em}



\lowercase{\hypertarget{spell:Deflective Shield}{}}\label{spell:Deflective Shield}
\begin{attuneability}[\nth{3}]{\hypertarget{spell:Deflective Shield}{Deflective Shield}}[\glossterm{Attune} (target), \glossterm{Shielding}]
This spell functions like the \spell{ablative shield} spell, except that the target also gains a \plus1 \glossterm{magic bonus} to Armor defense.
\end{attuneability}
\vspace{0.25em}



\lowercase{\hypertarget{spell:Immunity}{}}\label{spell:Immunity}
\begin{attuneability}[\nth{3}]{\hypertarget{spell:Immunity}{Immunity}}[\glossterm{Attune} (target), \glossterm{Shielding}]
Choose a willing creature in \rngclose range, and a type of damage that is not a kind of physical damage (see \pcref{Damage Types}).
The target becomes immune to damage of the chosen type.
Attacks that deal damage of multiple types still inflict damage normally unless the target is immune to all types of damage dealt.
\end{attuneability}
\vspace{0.25em}



\lowercase{\hypertarget{spell:Retributive Shield}{}}\label{spell:Retributive Shield}
\begin{attuneability}[\nth{3}]{\hypertarget{spell:Retributive Shield}{Retributive Shield}}[\glossterm{Attune} (target), \glossterm{Shielding}]
This spell functions like the \spell{ablative shield} spell, except that damage resisted by this spell is dealt back to the attacker as life damage.
If the attacker is beyond \rngclose range of the target, this reflection fails.

Any effect which increases this spell's range increases the range of this effect by the same amount.
This spell is from both the Abjuration and Vivimancy schools and gains the \glossterm{Life} tag in addition to the tags from the \spell{ablative shield} spell.
\end{attuneability}
\vspace{0.25em}



\lowercase{\hypertarget{spell:Empowered Shield}{}}\label{spell:Empowered Shield}
\begin{attuneability}[\nth{4}]{\hypertarget{spell:Empowered Shield}{Empowered Shield}}[\glossterm{Attune} (target), \glossterm{Shielding}]
This spell functions like the \spell{ablative shield} spell, except that the bonus is equal to twice your \glossterm{power}.
\end{attuneability}
\vspace{0.25em}



\lowercase{\hypertarget{spell:Antilife Shell}{}}\label{spell:Antilife Shell}
\begin{apability}[\nth{5}]{\hypertarget{spell:Antilife Shell}{Antilife Shell}}[\glossterm{Sustain} (minor)]
This effect functions like the \spell{repulsion field} spell, except that you gain a \plus10 bonus to accuracy with the attack against living creatures.
\end{apability}
\vspace{0.25em}



\subsubsection{Rituals}


\lowercase{\hypertarget{spell:Endure Elements}{}}\label{spell:Endure Elements}
\begin{attuneability}[\nth{1}]{\hypertarget{spell:Endure Elements}{Endure Elements}}[\glossterm{Attune} (ritual), \glossterm{Shielding}]
Choose a willing creature or unattended object within \rngclose range.
The target suffers no harm from being in a hot or cold environment.
It can exist comfortably in conditions between \minus50 and 140 degrees Fahrenheit.
Its equipment, if any, is also protected.
This does not protect the target from fire or cold damage.

This ritual takes one minute to perform.
\end{attuneability}
\vspace{0.25em}



\lowercase{\hypertarget{spell:Mystic Lock}{}}\label{spell:Mystic Lock}
\begin{attuneability}[\nth{2}]{\hypertarget{spell:Mystic Lock}{Mystic Lock}}[\glossterm{Attune} (ritual)]
Choose a Large or smaller closable, nonmagical object within \rngclose range, such as a door or box.
The target object becomes magically locked.
It can be unlocked with a Devices check against a DR equal to 20 \add your \glossterm{power}.
The DR to break it open forcibly increases by 10.

You can freely pass your own \ritual{arcane lock} as if the object were not locked.
This effect lasts as long as you \glossterm{attune} to it.
If you use this ability multiple times, you can attune to it each time.

This ritual takes one minute to perform.
\end{attuneability}
\vspace{0.25em}



\lowercase{\hypertarget{spell:Scryward}{}}\label{spell:Scryward}
\begin{apability}[\nth{2}]{\hypertarget{spell:Scryward}{Scryward}}[\glossterm{Mystic}]
This ritual creates a ward against scrying in a \arealarge radius zone centered on your location.
All \glossterm{Scrying} effects fail to function in the area.
This effect is permanent.

This ritual takes 24 hour to perform, and requires 8 action points from its participants.
\end{apability}
\vspace{0.25em}



\lowercase{\hypertarget{spell:Explosive Runes}{}}\label{spell:Explosive Runes}
\begin{attuneability}[\nth{3}]{\hypertarget{spell:Explosive Runes}{Explosive Runes}}[\glossterm{Attune} (ritual), \glossterm{Trap}]
Choose a Small or smaller unattended object with writing on it within \rngclose range.
% TODO: clarify how to identify that this is Explosive Runes instead of bad handwriting
The writing on the target is altered by the runes in subtle ways, making it more difficult to read.
To read the writing, a creature must concentrate on reading it, which requires a standard action.
If a creature reads the target, the target explodes.
You make an attack vs. Armor against everything within a \areamed radius from the target.
Each struck target takes bludgeoning \glossterm{standard damage} from the explosion.

After the target object explodes in this way, the ritual is \glossterm{dismissed}.
If the target is destroyed or rendered illegible, the ritual is dismissed without exploding.
This ritual takes one hour to perform.
\end{attuneability}
\vspace{0.25em}



\lowercase{\hypertarget{spell:Private Sanctum}{}}\label{spell:Private Sanctum}
\begin{apability}[\nth{4}]{\hypertarget{spell:Private Sanctum}{Private Sanctum}}[\glossterm{Mystic}]
This ritual creates a ward against any external perception in a \arealarge radius zone centered on your location.
This effect is permanent.
Everything in the area is completely imperceptible from outside the area.
Anyone observing the area from outside sees only a dark, silent void, regardless of darkvision and similar abilities.
In addition, all \glossterm{Scrying} effects fail to function in the area.
Creatures inside the area can see within the area and outside of it without any difficulty.

This ritual takes 24 hours to perform, and requires 32 action points from its participants.
\end{apability}
\vspace{0.25em}



\lowercase{\hypertarget{spell:Resilient Lock}{}}\label{spell:Resilient Lock}
\begin{attuneability}[\nth{4}]{\hypertarget{spell:Resilient Lock}{Resilient Lock}}[\glossterm{Attune} (ritual)]
This ritual functions like the \ritual{mystic lock} ritual, except that the DR to unlock the target with a Devices check is instead equal to 30 + your \glossterm{power}.
In addition, the DR to break it open increases by 20 instead of by 10.
\end{attuneability}
\vspace{0.25em}


\newpage
\begin{spellsection}{Bless}

\begin{spellheader}
\spelldesc{Grant divine blessings to aid allies and improve combat prowess.}
\end{spellheader}


\parhead{Schools} Channeling

\parhead{Mystic Sphere Lists} Divine

\subsubsection{Cantrips}


\begin{freeability}{Minor Cleansing Blessing}
Choose a willing creature within \rngclose range.
The target removes one \glossterm{condition} affecting it.
\end{freeability}

\end{spellsection}


\subsubsection{Spells}


\lowercase{\hypertarget{spell:Blessing of Protection}{}}\label{spell:Blessing of Protection}
\begin{attuneability}[\nth{1}]{\hypertarget{spell:Blessing of Protection}{Blessing of Protection}}[\glossterm{Attune} (target)]
Choose a willing creature within \rngclose range.
The target gains a \plus1 \glossterm{magic bonus} to Armor defense and Mental defense.

You can cast this spell as a \glossterm{minor action}.
\end{attuneability}
\vspace{0.25em}



\lowercase{\hypertarget{spell:Cleansing Blessing}{}}\label{spell:Cleansing Blessing}
\begin{apability}[\nth{1}]{\hypertarget{spell:Cleansing Blessing}{Cleansing Blessing}}
All allies within \arealarge radius from you can remove one \glossterm{condition} affecting them.
\end{apability}
\vspace{0.25em}



\lowercase{\hypertarget{spell:Battle Blessing}{}}\label{spell:Battle Blessing}
\begin{attuneability}[\nth{2}]{\hypertarget{spell:Battle Blessing}{Battle Blessing}}[\glossterm{Attune} (target)]
Choose a willing creature within \rngclose range.
The target gains a \plus1 \glossterm{magic bonus} to \glossterm{accuracy} with all attacks.

You can cast this spell as a \glossterm{minor action}.
\end{attuneability}
\vspace{0.25em}



\lowercase{\hypertarget{spell:Blessing of Might}{}}\label{spell:Blessing of Might}
\begin{attuneability}[\nth{3}]{\hypertarget{spell:Blessing of Might}{Blessing of Might}}[\glossterm{Attune} (target)]
The target gains a \plus1d \glossterm{magic bonus} to damage with all abilities.

You can cast this spell as a \glossterm{minor action}.
\end{attuneability}
\vspace{0.25em}



\lowercase{\hypertarget{spell:Blessing of Resilience}{}}\label{spell:Blessing of Resilience}
\begin{attuneability}[\nth{3}]{\hypertarget{spell:Blessing of Resilience}{Blessing of Resilience}}[\glossterm{Attune} (target)]
Choose a willing creature within \rngclose range.
The target ignores the next two \glossterm{conditions} it would receive.
After resisting two conditions in this way, this spell ends.

You can cast this spell as a \glossterm{minor action}.
\end{attuneability}
\vspace{0.25em}



\lowercase{\hypertarget{spell:Blessing of Supremacy}{}}\label{spell:Blessing of Supremacy}
\begin{attuneability}[\nth{4}]{\hypertarget{spell:Blessing of Supremacy}{Blessing of Supremacy}}[\glossterm{Attune} (target)]
Choose a willing creature within \rngclose range.
The target gains a \plus1 \glossterm{magic bonus} to \glossterm{accuracy} and a \plus1d \glossterm{magic bonus} to \glossterm{damage} with all abilities.

You can cast this spell as a \glossterm{minor action}.
\end{attuneability}
\vspace{0.25em}



\lowercase{\hypertarget{spell:Greater Blessing of Protection}{}}\label{spell:Greater Blessing of Protection}
\begin{attuneability}[\nth{4}]{\hypertarget{spell:Greater Blessing of Protection}{Greater Blessing of Protection}}[\glossterm{Attune} (target)]
This spell functions like the \spell{blessing of protection} spell, except that bonus increases to \plus2.
\end{attuneability}
\vspace{0.25em}



\lowercase{\hypertarget{spell:Greater Cleansing Blessing}{}}\label{spell:Greater Cleansing Blessing}
\begin{apability}[\nth{4}]{\hypertarget{spell:Greater Cleansing Blessing}{Greater Cleansing Blessing}}
This spell functions like the \spell{cleansing blessing} spell, except that each ally can remove two conditions instead of one.
\end{apability}
\vspace{0.25em}



\lowercase{\hypertarget{spell:Greater Battle Blessing}{}}\label{spell:Greater Battle Blessing}
\begin{attuneability}[\nth{5}]{\hypertarget{spell:Greater Battle Blessing}{Greater Battle Blessing}}[\glossterm{Attune} (target)]
This spell functions like the \spell{battle blessing} spell, except that the bonus increases to \plus2.
\end{attuneability}
\vspace{0.25em}



\lowercase{\hypertarget{spell:Greater Blessing of Resilience}{}}\label{spell:Greater Blessing of Resilience}
\begin{attuneability}[\nth{5}]{\hypertarget{spell:Greater Blessing of Resilience}{Greater Blessing of Resilience}}[\glossterm{Attune} (target)]
This spell functions like the \textit{blessing of resilience} spell, except that the spell does not end until it resists three \glossterm{conditions}.
\end{attuneability}
\vspace{0.25em}



\lowercase{\hypertarget{spell:Greater Blessing of Might}{}}\label{spell:Greater Blessing of Might}
\begin{attuneability}[\nth{6}]{\hypertarget{spell:Greater Blessing of Might}{Greater Blessing of Might}}[\glossterm{Attune} (target)]
The target gains a \plus2d \glossterm{magic bonus} to damage with all abilities.
\end{attuneability}
\vspace{0.25em}



\lowercase{\hypertarget{spell:Supreme Blessing of Protection}{}}\label{spell:Supreme Blessing of Protection}
\begin{attuneability}[\nth{7}]{\hypertarget{spell:Supreme Blessing of Protection}{Supreme Blessing of Protection}}[\glossterm{Attune} (target)]
This spell functions like the \spell{blessing of protection} spell, except that bonus increases to \plus3.
\end{attuneability}
\vspace{0.25em}



\lowercase{\hypertarget{spell:Supreme Blessing of Resilience}{}}\label{spell:Supreme Blessing of Resilience}
\begin{attuneability}[\nth{7}]{\hypertarget{spell:Supreme Blessing of Resilience}{Supreme Blessing of Resilience}}[\glossterm{Attune} (target)]
This spell functions like the \textit{blessing of resilience} spell, except that the spell does not end until it resists four \glossterm{conditions}.
\end{attuneability}
\vspace{0.25em}



\subsubsection{Rituals}


\lowercase{\hypertarget{spell:Bless Water}{}}\label{spell:Bless Water}
\begin{attuneability}[\nth{1}]{\hypertarget{spell:Bless Water}{Bless Water}}[\glossterm{Attune} (ritual)]
Choose one pint of unattended, nonmagical water within \rngclose range.
The target becomes holy water.
Holy water can be can be thrown as a splash weapon, dealing 1d8 points of damage to a struck undead creature or an evil outsider.

This ritual takes one minute to perform.
\end{attuneability}
\vspace{0.25em}



\lowercase{\hypertarget{spell:Blessing of Fortification}{}}\label{spell:Blessing of Fortification}
\begin{attuneability}[\nth{1}]{\hypertarget{spell:Blessing of Fortification}{Blessing of Fortification}}[\glossterm{Attune} (ritual)]
Choose an unattended, nonmagical object or part of an object of up to Large size.
Unlike most abilities, this ritual can affect individual parts of a whole object.

% How should this affect Strength break DRs?
The target gains a \plus5 \glossterm{magic bonus} to \glossterm{hardness}.
If the target is moved, this effect ends.
Otherwise, it lasts for one year.

This ritual takes one hour to perform.
\end{attuneability}
\vspace{0.25em}



\lowercase{\hypertarget{spell:Blessing of Purification}{}}\label{spell:Blessing of Purification}
\begin{apability}[\nth{1}]{\hypertarget{spell:Blessing of Purification}{Blessing of Purification}}[\glossterm{Shaping}]
All food and water in a single square within \rngclose range is purified.
Spoiled, rotten, poisonous, or otherwise contaminated food and water becomes pure and suitable for eating and drinking.
This does not prevent subsequent natural decay or spoiling.

This ritual takes one hour to perform.
\end{apability}
\vspace{0.25em}



\lowercase{\hypertarget{spell:Curse Water}{}}\label{spell:Curse Water}
\begin{attuneability}[\nth{1}]{\hypertarget{spell:Curse Water}{Curse Water}}[\glossterm{Attune} (ritual)]
Choose one pint of unattended, nonmagical water within \rngclose range.
The target becomes unholy water.
Unholy water can be can be thrown as a splash weapon, dealing 1d8 points of damage to a struck good outsider.

This ritual takes one minute to perform.
\end{attuneability}
\vspace{0.25em}



\lowercase{\hypertarget{spell:Permanent Bless Water}{}}\label{spell:Permanent Bless Water}
\begin{apability}[\nth{2}]{\hypertarget{spell:Permanent Bless Water}{Permanent Bless Water}}
This ritual functions like the \spell{bless water} ritual, except that it loses the \glossterm{Attune} (ritual) tag and the effect lasts permanently.
This ritual takes one hour to perform.
\end{apability}
\vspace{0.25em}



\lowercase{\hypertarget{spell:Permanent Curse Water}{}}\label{spell:Permanent Curse Water}
\begin{apability}[\nth{2}]{\hypertarget{spell:Permanent Curse Water}{Permanent Curse Water}}
This ritual functions like the \spell{curse water} ritual, except that it loses the \glossterm{Attune} (ritual) tag and the effect lasts permanently.
This ritual takes one hour to perform.
\end{apability}
\vspace{0.25em}



\lowercase{\hypertarget{spell:Enduring Fortification}{}}\label{spell:Enduring Fortification}
\begin{apability}[\nth{3}]{\hypertarget{spell:Enduring Fortification}{Enduring Fortification}}
This ritual functions like the \spell{blessing of fortification} ritual, except that the effect lasts for one hundred years.
\end{apability}
\vspace{0.25em}



\lowercase{\hypertarget{spell:Greater Fortification}{}}\label{spell:Greater Fortification}
\begin{attuneability}[\nth{3}]{\hypertarget{spell:Greater Fortification}{Greater Fortification}}[\glossterm{Attune} (ritual)]
This ritual functions like the \spell{blessing of fortification} ritual, except that the \glossterm{hardness} bonus increases to 10.
\end{attuneability}
\vspace{0.25em}



\lowercase{\hypertarget{spell:Greater Enduring Fortification}{}}\label{spell:Greater Enduring Fortification}
\begin{apability}[\nth{5}]{\hypertarget{spell:Greater Enduring Fortification}{Greater Enduring Fortification}}
This ritual functions like the \spell{greater fortification} ritual, except that the effect lasts for one hundred years.
\end{apability}
\vspace{0.25em}



\lowercase{\hypertarget{spell:Supreme Fortification}{}}\label{spell:Supreme Fortification}
\begin{attuneability}[\nth{6}]{\hypertarget{spell:Supreme Fortification}{Supreme Fortification}}[\glossterm{Attune} (ritual)]
This ritual functions like the \spell{blessing of fortification} ritual, except that the \glossterm{hardness} bonus increases to 15.
\end{attuneability}
\vspace{0.25em}


\newpage
\begin{spellsection}{Channel Divinity}

\begin{spellheader}
\spelldesc{Invoke divine power to smite foes and gain power.}
\end{spellheader}


\parhead{Schools} Channeling

\parhead{Mystic Sphere Lists} Divine

\subsubsection{Cantrips}


\begin{freeability}{Divine Judgment}
Make an attack vs. Mental against a creature within \rngmed range.
\hit The target takes divine \glossterm{standard damage}.
\end{freeability}

\end{spellsection}


\subsubsection{Spells}


\lowercase{\hypertarget{spell:Judge Unworthy}{}}\label{spell:Judge Unworthy}
\begin{apability}[\nth{1}]{\hypertarget{spell:Judge Unworthy}{Judge Unworthy}}
Make an attack vs. Mental against a creature within \rngmed range.
\hit The target takes divine \glossterm{standard damage} \plus2d.
\end{apability}
\vspace{0.25em}



\lowercase{\hypertarget{spell:Mantle of Faith}{}}\label{spell:Mantle of Faith}
\begin{attuneability}[\nth{1}]{\hypertarget{spell:Mantle of Faith}{Mantle of Faith}}[\glossterm{Attune} (self)]
You gain a \glossterm{magic bonus} to equal to your \glossterm{power} to \glossterm{damage reduction} against \glossterm{physical} damage.

You can cast this spell as a \glossterm{minor action}.
\end{attuneability}
\vspace{0.25em}



\lowercase{\hypertarget{spell:Complete Mantle of Faith}{}}\label{spell:Complete Mantle of Faith}
\begin{attuneability}[\nth{2}]{\hypertarget{spell:Complete Mantle of Faith}{Complete Mantle of Faith}}[\glossterm{Attune} (self)]
You gain a \glossterm{magic bonus} equal to your \glossterm{power} to \glossterm{damage reduction} against all damage.

You can cast this spell as a \glossterm{minor action}.
\end{attuneability}
\vspace{0.25em}



\lowercase{\hypertarget{spell:Word of Faith}{}}\label{spell:Word of Faith}
\begin{apability}[\nth{2}]{\hypertarget{spell:Word of Faith}{Word of Faith}}
Make an attack vs. Mental against all enemies in a \areamed radius from you.
\hit Each target takes divine \glossterm{standard damage}.
\end{apability}
\vspace{0.25em}



\lowercase{\hypertarget{spell:Divine Might}{}}\label{spell:Divine Might}
\begin{attuneability}[\nth{3}]{\hypertarget{spell:Divine Might}{Divine Might}}[\glossterm{Attune} (self), \glossterm{Shaping}, \glossterm{Sizing}]
You increase your size by one size category.
This increases your \glossterm{overwhelm value}, \glossterm{overwhelm resistance}, and usually increases your \glossterm{reach} (see \pcref{Size in Combat}).
However, your muscles are not increased fully to match its new size, and your Strength is unchanged.

You can cast this spell as a \glossterm{minor action}.
\end{attuneability}
\vspace{0.25em}



\lowercase{\hypertarget{spell:Greater Mantle of Faith}{}}\label{spell:Greater Mantle of Faith}
\begin{attuneability}[\nth{4}]{\hypertarget{spell:Greater Mantle of Faith}{Greater Mantle of Faith}}[\glossterm{Attune} (self)]
This spell functions like the \spell{mantle of faith} spell, except that the bonus is equal to twice your \glossterm{power}.
\end{attuneability}
\vspace{0.25em}



\lowercase{\hypertarget{spell:Divine Might, Greater}{}}\label{spell:Divine Might, Greater}
\begin{attuneability}[\nth{5}]{\hypertarget{spell:Divine Might, Greater}{Divine Might, Greater}}[\glossterm{Attune} (self), \glossterm{Shaping}, \glossterm{Sizing}]
This spell functions like the \textit{divine might} spell, except that you also gain a \plus2 bonus to Strength.
\end{attuneability}
\vspace{0.25em}



\lowercase{\hypertarget{spell:Greater Complete Mantle of Faith}{}}\label{spell:Greater Complete Mantle of Faith}
\begin{attuneability}[\nth{5}]{\hypertarget{spell:Greater Complete Mantle of Faith}{Greater Complete Mantle of Faith}}[\glossterm{Attune} (self)]
This spell functions like the \spell{complete mantle of faith} spell, except that the bonus is equal to twice your \glossterm{power}.
\end{attuneability}
\vspace{0.25em}



\lowercase{\hypertarget{spell:Divine Might, Supreme}{}}\label{spell:Divine Might, Supreme}
\begin{attuneability}[\nth{7}]{\hypertarget{spell:Divine Might, Supreme}{Divine Might, Supreme}}[\glossterm{Attune} (self), \glossterm{Shaping}, \glossterm{Sizing}]
This spell functions like the \spell{divine might} spell, except that your size is increased by two size categories.
You gain a \plus2 bonus to Strength to partially match your new size.
\end{attuneability}
\vspace{0.25em}



\subsubsection{Rituals}


\lowercase{\hypertarget{spell:Consecrate}{}}\label{spell:Consecrate}
\begin{attuneability}[\nth{2}]{\hypertarget{spell:Consecrate}{Consecrate}}[\glossterm{Attune} (self)]
The area within an \arealarge radius \glossterm{zone} from your location becomes sacred to your deity.
% TODO: what cares about consecration?
This has no tangible effects by itself, but some special abilities and monsters behave differently in consecrated areas.

This ritual takes 24 hours to perform and requires 8 action points from its ritual participants.
\end{attuneability}
\vspace{0.25em}



\lowercase{\hypertarget{spell:Divine Transit}{}}\label{spell:Divine Transit}
\begin{apability}[\nth{4}]{\hypertarget{spell:Divine Transit}{Divine Transit}}[\glossterm{Teleportation}]
Choose up to five willing, Medium or smaller ritual participants.
In addition, choose a destination up to 100 miles away from you on your current plane.
Each target is teleported to the temple or equivalent holy site to your deity that is closest to the chosen destination.

You must specify the destination with a precise mental image of its appearance.
The image does not have to be perfect, but it must unambiguously identify the destination.
If you specify its appearance incorrectly, or if the area has changed its appearance, the destination may be a different area than you intended.
The new destination will be one that more closely resembles your mental image.
If no such area exists, the ritual simply fails.
% TODO: does this need more clarity about what teleportation works?

This ritual takes 24 hours to perform and requires 32 action points from its ritual participants.
It is from the Conjuration school in addition to the Channeling school.
\end{apability}
\vspace{0.25em}


\newpage
\begin{spellsection}{Chronomancy}

\begin{spellheader}
\spelldesc{Manipulate the passage of time to inhibit foes and aid allies.}
\end{spellheader}


\parhead{Schools} Transmutation

\parhead{Mystic Sphere Lists} Arcane, Pact

\subsubsection{Cantrips}


\begin{freeability}{Minor Slow}[\glossterm{Temporal}]
Make an attack vs. Mental against a creature within \rngmed range.
\hit The target is \glossterm{slowed} as a \glossterm{condition}.
\end{freeability}

\end{spellsection}


\subsubsection{Spells}


\lowercase{\hypertarget{spell:Haste}{}}\label{spell:Haste}
\begin{attuneability}[\nth{1}]{\hypertarget{spell:Haste}{Haste}}[\glossterm{Attune} (target), \glossterm{Temporal}]
Choose a willing creature within \rngmed range.
The target gains a \plus10 foot \glossterm{magic bonus} to its \glossterm{base speed}.

You can cast this spell as a \glossterm{minor action}.
\end{attuneability}
\vspace{0.25em}



\lowercase{\hypertarget{spell:Slow}{}}\label{spell:Slow}
\begin{apability}[\nth{1}]{\hypertarget{spell:Slow}{Slow}}[\glossterm{Temporal}]
Make an attack vs. Mental with a \plus2 bonus to \glossterm{accuracy} against a creature within \rngmed range.
\miss You regain the \glossterm{action point} spent to cast this spell.
\hit The target is \glossterm{slowed} as a \glossterm{condition}.
\crit the target is \glossterm{decelerated} as a \glossterm{condition}.
\end{apability}
\vspace{0.25em}



\lowercase{\hypertarget{spell:Mental Lag}{}}\label{spell:Mental Lag}
\begin{apability}[\nth{2}]{\hypertarget{spell:Mental Lag}{Mental Lag}}[\glossterm{Temporal}]
Make an attack vs. Mental against a creature within \rngmed range.
\miss You regain the \glossterm{action point} spent to cast this spell.
\hit The target is \glossterm{slowed} and \glossterm{dazed} as two separate \glossterm{conditions}.
\crit the target is \glossterm{decelerated} and \glossterm{dazed} as two separate \glossterm{conditions}.
\end{apability}
\vspace{0.25em}



\lowercase{\hypertarget{spell:Time Hop}{}}\label{spell:Time Hop}
\begin{apability}[\nth{2}]{\hypertarget{spell:Time Hop}{Time Hop}}[\glossterm{Temporal}]
Choose a Medium or smalller willing creature or unattended object within \rngmed range.
You send the target into the future, causing it to temporarily cease to exist.
When you cast this spell, you choose how many rounds the target ceases to exist for, up to a maximum of five rounds.
At the end of the last round, it reappears in the same location where it disappeared.

The area the target occupied can be physically crossed, but it is treated as an invalid destination for teleportation and other similar magic.
When the target reappears, all of its surroundings are adjusted as if the object had retroactively always existed in its space.
For example, if the location is occupied by a creature that walked into the area, the creature is relocated to the closest unoccupied space along the path it took to reach the target.

You can cast this spell as a \glossterm{minor action}.
\end{apability}
\vspace{0.25em}



\lowercase{\hypertarget{spell:Decelerate}{}}\label{spell:Decelerate}
\begin{apability}[\nth{3}]{\hypertarget{spell:Decelerate}{Decelerate}}[\glossterm{Temporal}]
Make an attack vs. Mental against a creature within \rngmed range.
\miss You regain the \glossterm{action point} spent to cast this spell.
\hit The target is \glossterm{decelerated} as a \glossterm{condition}.
\crit the target is \glossterm{decelerated} twice as two separate \glossterm{conditions}.
\end{apability}
\vspace{0.25em}



\lowercase{\hypertarget{spell:Delay Damage}{}}\label{spell:Delay Damage}
\begin{apability}[\nth{3}]{\hypertarget{spell:Delay Damage}{Delay Damage}}[\glossterm{Sustain} (minor), \glossterm{Temporal}]
When you take damage, half of the damage (rounded down) is not dealt to you immediately.
This damage is tracked separately.
When the ends, you take all of the delayed damage at once.
This damage has no type, and ignores all effects that reduce or negate damage.

You can cast this spell as a \glossterm{minor action}.
\end{apability}
\vspace{0.25em}



\lowercase{\hypertarget{spell:Greater Haste}{}}\label{spell:Greater Haste}
\begin{attuneability}[\nth{3}]{\hypertarget{spell:Greater Haste}{Greater Haste}}[\glossterm{Attune} (target), \glossterm{Temporal}]
Choose a willing creature within \rngmed range.
The target gains a \plus30 foot \glossterm{magic bonus} to its \glossterm{base speed}, up to a maximum of double its \glossterm{base speed}.
In addition, it gains a \plus2 \glossterm{magic bonus} to Reflex defense.

You can cast this spell as a \glossterm{minor action}.
\end{attuneability}
\vspace{0.25em}



\lowercase{\hypertarget{spell:Temporal Stasis}{}}\label{spell:Temporal Stasis}
\begin{attuneability}[\nth{3}]{\hypertarget{spell:Temporal Stasis}{Temporal Stasis}}[\glossterm{Attune} (self), \glossterm{Temporal}]
Choose a Medium or smaller willing creature within \rngmed range.
The target is placed into stasis, rendering it unconscious.
While in stasis, it cannot take any actions and cannot be targeted, moved, damaged, or otherwise affected by outside forces in any way.

% TODO: wording
This effect normally lasts as long as you \glossterm{attune} to it, and until the end of the round when you release the attunement.
If you use this ability on yourself, it instead lasts for a number of rounds you choose when you cast the spell, up to a maximum of five rounds.

You can cast this spell as a \glossterm{minor action}.
\end{attuneability}
\vspace{0.25em}



\lowercase{\hypertarget{spell:Temporal Duplicate}{}}\label{spell:Temporal Duplicate}
\begin{apability}[\nth{4}]{\hypertarget{spell:Temporal Duplicate}{Temporal Duplicate}}[\glossterm{Temporal}]
Choose a willing creature within \rngmed range.
You reach into a possible future and create a duplicate of the target.
The duplicate is identical in all ways to the target when the spell resolves, except that it has no \glossterm{legend points}.
The target and its duplicate can act during the next round.
At the end of that round, the target and its duplicate cease to exist.
At the end of the following round, the target reappears in the place where it ceased to exist.
If that space is occupied, it appears in the closest unoccupied space.

When the target reappears, its condition is unchanged from when it left, except that it loses all action points, spell points, and all similar resources equal to the amount used by its duplicate.
Its hit points, conditions, and all other statistics are unaffected, regardless of any damage or other negative effects suffered by the duplicate.
If this would reduce any of the target's resources below 0, it takes physical \glossterm{standard damage} \plus4d from the paradox and becomes \glossterm{stunned} as a \glossterm{condition}.

You can cast this spell as a \glossterm{minor action}.
\end{apability}
\vspace{0.25em}



\lowercase{\hypertarget{spell:Time Lock}{}}\label{spell:Time Lock}
\begin{apability}[\nth{4}]{\hypertarget{spell:Time Lock}{Time Lock}}[\glossterm{Sustain} (minor), \glossterm{Temporal}]
Choose a willing creature within \rngmed range.
You lock the state of the target's body in time.
Note the target's hit points, vital damage, and active conditions.
If the target dies, this effect ends immediately.

As a \glossterm{standard action}, you can reach through time to restore the target's state.
If you do, the target's hit points, vital damage, and active conditions become identical to what they were when you cast this spell.
This does not affect any other properties of the target, such as any resources expended.
After you restore the target's state in this way, the spell ends.

You can cast this spell as a \glossterm{minor action}.
\end{apability}
\vspace{0.25em}



\lowercase{\hypertarget{spell:Greater Temporal Duplicate}{}}\label{spell:Greater Temporal Duplicate}
\begin{apability}[\nth{7}]{\hypertarget{spell:Greater Temporal Duplicate}{Greater Temporal Duplicate}}[\glossterm{Temporal}]
This spell functions like the \spell{temporal duplicate} spell, except that you can reach up to five minutes into the future to summon the duplicate.
When you cast the spell, you choose the length of time before the target disappears.
The duplicate still only exists for a single round.
\end{apability}
\vspace{0.25em}



\lowercase{\hypertarget{spell:Greater Time Lock}{}}\label{spell:Greater Time Lock}
\begin{attuneability}[\nth{7}]{\hypertarget{spell:Greater Time Lock}{Greater Time Lock}}[\glossterm{Attune} (self), \glossterm{Temporal}]
This spell functions like the \textit{time lock} spell, except that the effect is not ended if the target dies, and restoring the target's state can also restore it to life.
If the target is restored to life in this way, all of its properties not locked by this spell, such as any resources expended, are identical to what they were when the target died.
In addition, this spell has the \glossterm{Attune} (self) tag instead of the \glossterm{Sustain} (minor) tag.
\end{attuneability}
\vspace{0.25em}



\lowercase{\hypertarget{spell:Time Stop}{}}\label{spell:Time Stop}
\begin{apability}[\nth{7}]{\hypertarget{spell:Time Stop}{Time Stop}}[\glossterm{Temporal}]
You can take two full rounds of actions immediately.
During this time, all other creatures and objects are fixed in time, and cannot be targeted, moved, damaged, or otherwise affected by outside forces in any way.
You can still affect yourself and create areas or new effects.

You are still vulnerable to danger, such as from heat or dangerous gases.
However, you cannot be detected by any means while you travel.

After casting this spell, you cannot cast it again until you take a \glossterm{short rest}.
\end{apability}
\vspace{0.25em}



\subsubsection{Rituals}


\lowercase{\hypertarget{spell:Gentle Repose}{}}\label{spell:Gentle Repose}
\begin{attuneability}[\nth{2}]{\hypertarget{spell:Gentle Repose}{Gentle Repose}}[\glossterm{Attune} (ritual), \glossterm{Temporal}]
Choose an unattended, nonmagical object within \rngclose range.
Time does not pass for the target, preventing it from decaying or spoiling.
This can extend the time a poison or similar item lasts before becoming inert.
% What effects have an explicit time limit?
If used on a corpse, this effectively extends the time limit for effects that require a fresh or intact body.
Additionally, this can make transporting a fallen comrade more pleasant.

% Does this need to explicitly clarify that it doesn't stop time from passing for the creature's soul?

This ritual takes one minute to perform.
\end{attuneability}
\vspace{0.25em}


\newpage
\begin{spellsection}{Compel}

\begin{spellheader}
\spelldesc{Bend creatures to your will by controlling their actions.}
\end{spellheader}


\parhead{Schools} Enchantment

\parhead{Mystic Sphere Lists} Arcane, Divine, Pact

\subsubsection{Cantrips}


\begin{freeability}{Fall}[\glossterm{Compulsion}, \glossterm{Mind}]
Make an attack vs. Mental against a creature within \rngmed range.
\hit The target falls \glossterm{prone}.
\end{freeability}

\end{spellsection}


\subsubsection{Spells}


\lowercase{\hypertarget{spell:Collapse}{}}\label{spell:Collapse}
\begin{apability}[\nth{1}]{\hypertarget{spell:Collapse}{Collapse}}[\glossterm{Compulsion}, \glossterm{Mind}]
Make an attack vs. Mental against all enemies in a \areamed radius from you.
\hit Each target falls \glossterm{prone}.
\crit As above, and as a \glossterm{condition}, each target is unable to stand up.
If a target is somehow brought into a standing position, it will immediately fall and become prone again.
\end{apability}
\vspace{0.25em}



\lowercase{\hypertarget{spell:Dance}{}}\label{spell:Dance}
\begin{apability}[\nth{1}]{\hypertarget{spell:Dance}{Dance}}[\glossterm{Compulsion}, \glossterm{Mind}]
Make an attack vs. Mental against a creature within \rngmed range.
\miss You regain the \glossterm{action point} spent to cast this spell.
\hit As a \glossterm{condition}, the target is compelled to dance.
It can spend a \glossterm{move action} to dance, if it is physically capable of dancing.
At the end of each round, if the target did not dance during that round, it takes a \minus2 penalty to \glossterm{accuracy}, \glossterm{checks}, and \glossterm{defenses} as the compulsion intensifies.
This penalty stacks each round until the target dances, which resets the penalties to 0.
\crit As above, except that the target must dance as a \glossterm{standard action} to reset the penalties, instead of as a move action.
\end{apability}
\vspace{0.25em}



\lowercase{\hypertarget{spell:Stay}{}}\label{spell:Stay}
\begin{apability}[\nth{1}]{\hypertarget{spell:Stay}{Stay}}[\glossterm{Compulsion}, \glossterm{Mind}]
Make an attack vs. Mental against a creature within \rngmed range.
\miss You regain the \glossterm{action point} spent to cast this spell.
\hit The target falls \glossterm{prone} and is \glossterm{slowed} as a \glossterm{condition}.
\crit The target falls prone and is \glossterm{decelerated} as a \glossterm{condition}.
\end{apability}
\vspace{0.25em}



\lowercase{\hypertarget{spell:Confusion}{}}\label{spell:Confusion}
\begin{apability}[\nth{3}]{\hypertarget{spell:Confusion}{Confusion}}[\glossterm{Compulsion}, \glossterm{Mind}]
Make an attack vs. Mental against a creature within \rngmed range.
\miss You regain the \glossterm{action point} spent to cast this spell.
\hit The target is \disoriented as a \glossterm{condition}.
\crit The target is \confused as a \glossterm{condition}.
\end{apability}
\vspace{0.25em}



\lowercase{\hypertarget{spell:Sleep}{}}\label{spell:Sleep}
\begin{apability}[\nth{3}]{\hypertarget{spell:Sleep}{Sleep}}[\glossterm{Compulsion}, \glossterm{Mind}]
Make an attack vs. Mental against a creature within \rngclose range.
\miss You regain the \glossterm{action point} spent to cast this spell.
\hit The target is \blinded as a \glossterm{condition}.
\crit The target falls asleep.
It cannot be awakened by any means while the spell lasts.
After that time, it can wake up normally, though it continues to sleep until it would wake up naturally.
% Awkward to sustain without the Sustain tag
This effect lasts as long as you \glossterm{sustain} it as a \glossterm{minor action}.
However, it is a \glossterm{condition}, and can be removed by effects which remove conditions.
\end{apability}
\vspace{0.25em}



\lowercase{\hypertarget{spell:Discordant Song}{}}\label{spell:Discordant Song}
\begin{apability}[\nth{4}]{\hypertarget{spell:Discordant Song}{Discordant Song}}[\glossterm{Compulsion}, \glossterm{Mind}]
Make an attack vs. Mental against all enemies in a \areamed radius from you.
\hit Each target is \disoriented as a \glossterm{condition}.
\crit Each target is \confused as a \glossterm{condition}.
\end{apability}
\vspace{0.25em}



\lowercase{\hypertarget{spell:Dominate}{}}\label{spell:Dominate}
\begin{apability}[\nth{4}]{\hypertarget{spell:Dominate}{Dominate}}[\glossterm{Compulsion}, \glossterm{Mind}]
Make an attack vs. Mental against a creature within \rngmed range.
\miss You regain the \glossterm{action point} spent to cast this spell.
\hit The target is \glossterm{confused} as a \glossterm{condition}.
\crit The target is \glossterm{stunned} as a \glossterm{condition}.
As a standard action, you can make an additional attack vs. Mental against the target as long as it remains stunned in this way and is within \rngmed range of you.
On a hit, the target becomes stunned in the same way as an additional condition, continuing the effect even if the target removed the original condition in the same phase.
On a critical hit, the target becomes \glossterm{dominated} by you as long as you \glossterm{attune} to this ability.
\end{apability}
\vspace{0.25em}



\lowercase{\hypertarget{spell:Irresistible Dance}{}}\label{spell:Irresistible Dance}
\begin{apability}[\nth{6}]{\hypertarget{spell:Irresistible Dance}{Irresistible Dance}}[\glossterm{Compulsion}, \glossterm{Mind}]
This spell functions like the \textit{dance} spell, except that you gain a \plus4 bonus to accuracy on the attack.
\end{apability}
\vspace{0.25em}


\newpage
\begin{spellsection}{Corruption}

\begin{spellheader}
\spelldesc{Weaken the life force of foes, reducing their combat prowess.}
\end{spellheader}


\parhead{Schools} Vivimancy

\parhead{Mystic Sphere Lists} Arcane, Divine, Nature, Pact

\subsubsection{Cantrips}


\begin{freeability}{Sicken}[\glossterm{Life}]
Make an attack vs. Fortitude against a living creature within \rngclose range.
\hit The target is \glossterm{sickened} as a \glossterm{condition}.
\end{freeability}

\end{spellsection}


\subsubsection{Spells}


\lowercase{\hypertarget{spell:Miasma}{}}\label{spell:Miasma}
\begin{apability}[\nth{1}]{\hypertarget{spell:Miasma}{Miasma}}[\glossterm{Life}]
Make an attack vs. Fortitude against all living enemies within an \areamed radius from you.
\hit Each target is \glossterm{sickened} as a \glossterm{condition}.
\crit Each target is \glossterm{nauseated} as a \glossterm{condition}.
\end{apability}
\vspace{0.25em}



\lowercase{\hypertarget{spell:Sickening Decay}{}}\label{spell:Sickening Decay}
\begin{apability}[\nth{1}]{\hypertarget{spell:Sickening Decay}{Sickening Decay}}[\glossterm{Life}]
Make an attack vs. Fortitude against a living creature within \rngclose range.
\miss You regain the \glossterm{action point} spent to cast this spell.
\hit The target is \glossterm{sickened} as a \glossterm{condition}.
% TODO: clarify when exactly this damage is taken (should be at the end of the phase)
In addition, it takes life \glossterm{standard damage} \minus2d when it takes a \glossterm{standard action}.
It can only take damage in this way once per round.
\crit The target is \glossterm{nauseated} as a \glossterm{condition}.
In addition, it takes life \glossterm{standard damage} when it takes a \glossterm{standard action}.
It can only take damage in this way once per round.
\end{apability}
\vspace{0.25em}



\lowercase{\hypertarget{spell:Pernicious Sickness}{}}\label{spell:Pernicious Sickness}
\begin{apability}[\nth{2}]{\hypertarget{spell:Pernicious Sickness}{Pernicious Sickness}}[\glossterm{Life}]
Make an attack vs. Fortitude with a \plus2 bonus to \glossterm{accuracy} against a living creature within \rngmed range.
\hit The target is \glossterm{sickened} as a \glossterm{condition}.
\crit The target is \glossterm{nauseated} as a \glossterm{condition}.
\end{apability}
\vspace{0.25em}



\lowercase{\hypertarget{spell:Bleed}{}}\label{spell:Bleed}
\begin{apability}[\nth{3}]{\hypertarget{spell:Bleed}{Bleed}}[\glossterm{Life}]
This spell functions like the \spell{sickening decay} spell, except that a struck target also begins bleeding as an additional \glossterm{condition}.
At the end of every \glossterm{action phase} in subsequent rounds, the target takes slashing \glossterm{standard damage} \minus1d.
\end{apability}
\vspace{0.25em}



\lowercase{\hypertarget{spell:Corruption of Blood and Bone}{}}\label{spell:Corruption of Blood and Bone}
\begin{apability}[\nth{3}]{\hypertarget{spell:Corruption of Blood and Bone}{Corruption of Blood and Bone}}[\glossterm{Life}]
This spell functions like the \spell{sickening decay} spell, except that it gains a \plus1d bonus to damage.
In addition, damage from the spell reduces the target's maximum hit points by the same amount.
This hit point reduction is part of the same \glossterm{condition} as the spell's other effects.
When the condition is removed, the target's maximum hit points are restored.
\end{apability}
\vspace{0.25em}



\lowercase{\hypertarget{spell:Crippling Decay}{}}\label{spell:Crippling Decay}
\begin{apability}[\nth{3}]{\hypertarget{spell:Crippling Decay}{Crippling Decay}}[\glossterm{Life}]
This spell functions like the \spell{sickening decay} spell, except that a struck target is also \glossterm{decelerated} as an additional \glossterm{condition}.
\end{apability}
\vspace{0.25em}



\lowercase{\hypertarget{spell:Eyebite}{}}\label{spell:Eyebite}
\begin{apability}[\nth{3}]{\hypertarget{spell:Eyebite}{Eyebite}}[\glossterm{Life}]
Make an attack vs. Fortitude against a living creature within \rngclose range.
\miss You regain the \glossterm{action point} spent to cast this spell.
\hit The target is \glossterm{blinded} as a \glossterm{condition}.
\crit The target is \glossterm{blinded} twice by two separate \glossterm{conditions}.
Both conditions must be removed before the target can see again.
\end{apability}
\vspace{0.25em}



\lowercase{\hypertarget{spell:Greater Miasma}{}}\label{spell:Greater Miasma}
\begin{apability}[\nth{3}]{\hypertarget{spell:Greater Miasma}{Greater Miasma}}[\glossterm{Life}]
Make an attack vs. Fortitude against all living enemies within an \areamed radius from you.
\hit Each target is \glossterm{nauseated} as a \glossterm{condition}.
\crit Each target is \glossterm{nauseated} twice as two separate \glossterm{conditions}.
\end{apability}
\vspace{0.25em}



\lowercase{\hypertarget{spell:Curse of Decay}{}}\label{spell:Curse of Decay}
\begin{apability}[\nth{4}]{\hypertarget{spell:Curse of Decay}{Curse of Decay}}[\glossterm{Curse}]
This spell functions like the \spell{sickening decay} spell, except that the attack is made against Mental defense instead of Fortitude defense.
In addition, if the attack critically hits, the spell's effect becomes a permanent curse.
It is no longer a condition, and cannot be removed by abilities that remove conditions.
\end{apability}
\vspace{0.25em}



\lowercase{\hypertarget{spell:Greater Pernicious Sickness}{}}\label{spell:Greater Pernicious Sickness}
\begin{apability}[\nth{5}]{\hypertarget{spell:Greater Pernicious Sickness}{Greater Pernicious Sickness}}[\glossterm{Life}]
This spell functions like the \spell{pernicious sickness} spell, except that the accuracy bonus is increased to \plus4.
\end{apability}
\vspace{0.25em}



\subsubsection{Rituals}


\lowercase{\hypertarget{spell:Animate Dead}{}}\label{spell:Animate Dead}
\begin{attuneability}[\nth{2}]{\hypertarget{spell:Animate Dead}{Animate Dead}}[\glossterm{Attune} (ritual)]
Choose any number of corpses within \rngclose range.
The combined levels of all targets cannot exceed your \glossterm{power}.
The target becomes an undead creature that obeys your spoken commands.
You choose whether to create a skeleton or a zombie.
Creating a zombie require a mostly intact corpse, including most of the flesh.
Creating a skeleton only requires a mostly intact skeleton.
If a skeleton is made from an intact corpse, the flesh quickly falls off the animated bones.

This ritual takes one hour to perform.
\end{attuneability}
\vspace{0.25em}


\newpage
\begin{spellsection}{Cryomancy}

\begin{spellheader}
\spelldesc{Drain heat to injure and freeze foes.}
\end{spellheader}


\parhead{Schools} Evocation

\parhead{Mystic Sphere Lists} Arcane, Nature, Pact

\subsubsection{Cantrips}


\begin{freeability}{Chill}[\glossterm{Cold}]
Make an attack vs. Fortitude against one creature or object within \rngmed range.
\hit The target takes cold \glossterm{standard damage}.
\end{freeability}

\end{spellsection}


\subsubsection{Spells}


\lowercase{\hypertarget{spell:Cone of Cold}{}}\label{spell:Cone of Cold}
\begin{apability}[\nth{1}]{\hypertarget{spell:Cone of Cold}{Cone of Cold}}[\glossterm{Cold}]
Make an attack vs. Fortitude against everything in a \areamed cone from you.
\hit Each target takes cold \glossterm{standard damage}, and is \glossterm{fatigued} as a \glossterm{condition}.
\end{apability}
\vspace{0.25em}



\lowercase{\hypertarget{spell:Frostbite}{}}\label{spell:Frostbite}
\begin{apability}[\nth{1}]{\hypertarget{spell:Frostbite}{Frostbite}}[\glossterm{Cold}]
Make an attack vs. Fortitude against one creature or object within \rngmed range.
\hit The target takes cold \glossterm{standard damage} \plus2d.
\end{apability}
\vspace{0.25em}



\lowercase{\hypertarget{spell:Icecraft}{}}\label{spell:Icecraft}
\begin{attuneability}[\nth{1}]{\hypertarget{spell:Icecraft}{Icecraft}}[\glossterm{Attune} (self), \glossterm{Cold}]
Choose a pool of unattended, nonmagical water within \rngclose range.
This spell creates an icy weapon or a suit of icy armor from the target pool of water.
You can create any weapon, shield, or body armor that you are proficient with, and which would normally be made entirely from metal, except for heavy body armor.
The pool of water targeted must be at least as large as the item you create.

The item functions like a normal item of its type, except that it is more fragile.
It has hit points equal to twice your \glossterm{power}, does not have any \glossterm{hardness}, and is \glossterm{vulnerable} to fire damage.
If the item would take cold damage, it instead heals that many hit points.

When a creature wearing armor created in this way takes physical damage, cold damage, or fire damage, that damage is also dealt to the armor.
Likewise, when a creature wielding a weapon created in this way deals damage with the weapon, that damage is also dealt to the weapon.
If the item loses all of its hit points, this effect is \glossterm{dismissed}.
\end{attuneability}
\vspace{0.25em}



\lowercase{\hypertarget{spell:Blizzard}{}}\label{spell:Blizzard}
\begin{apability}[\nth{2}]{\hypertarget{spell:Blizzard}{Blizzard}}[\glossterm{Cold}]
This spell functions like the \spell{cone of cold} spell, except that the area becomes a \areamed radius from you.
\end{apability}
\vspace{0.25em}



\lowercase{\hypertarget{spell:Cold Snap}{}}\label{spell:Cold Snap}
\begin{apability}[\nth{2}]{\hypertarget{spell:Cold Snap}{Cold Snap}}[\glossterm{Cold}]
This spell functions like the \spell{cone of cold} spell, except that it gains the \glossterm{Sustain} (standard) tag.
The area affected by the spell becomes a \glossterm{zone} that is supernaturally chilled.
At the end of each \glossterm{action phase} in subsequent rounds, the attack is repeated in that area.
\end{apability}
\vspace{0.25em}



\lowercase{\hypertarget{spell:Sturdy Icecraft}{}}\label{spell:Sturdy Icecraft}
\begin{attuneability}[\nth{2}]{\hypertarget{spell:Sturdy Icecraft}{Sturdy Icecraft}}[\glossterm{Attune} (self), \glossterm{Cold}]
This spell functions like the \spell{icecraft} spell, except that the item created has hit points equal to four times your \glossterm{power}.
In addition, you can create heavy body armor.
\end{attuneability}
\vspace{0.25em}



\lowercase{\hypertarget{spell:Deep Freeze}{}}\label{spell:Deep Freeze}
\begin{apability}[\nth{3}]{\hypertarget{spell:Deep Freeze}{Deep Freeze}}[\glossterm{Cold}]
This spell functions like the \spell{cone of cold} spell, except that it attacks Reflex defense instead of Fortitude defense.
\end{apability}
\vspace{0.25em}



\lowercase{\hypertarget{spell:Freezing Cone}{}}\label{spell:Freezing Cone}
\begin{apability}[\nth{3}]{\hypertarget{spell:Freezing Cone}{Freezing Cone}}[\glossterm{Cold}]
This spell functions like the \spell{cone of cold} spell, except that you gain a \plus1d bonus to damage and each struck target is \glossterm{exhausted} instead of \glossterm{fatigued}.
\end{apability}
\vspace{0.25em}



\lowercase{\hypertarget{spell:Greater Cold Snap}{}}\label{spell:Greater Cold Snap}
\begin{apability}[\nth{3}]{\hypertarget{spell:Greater Cold Snap}{Greater Cold Snap}}[\glossterm{Cold}]
This spell functions like the \textit{cold snap} spell, except that the spell gains the \glossterm{Sustain} (minor) tag instead of the \glossterm{Sustain} (standard) tag.
\end{apability}
\vspace{0.25em}



\lowercase{\hypertarget{spell:Greater Frostbite}{}}\label{spell:Greater Frostbite}
\begin{apability}[\nth{3}]{\hypertarget{spell:Greater Frostbite}{Greater Frostbite}}[\glossterm{Cold}]
This spell functions like the \spell{frostbite} spell, except that a struck target is also \glossterm{exhausted} as a \glossterm{condition}.
\end{apability}
\vspace{0.25em}



\lowercase{\hypertarget{spell:Enhanced Icecraft}{}}\label{spell:Enhanced Icecraft}
\begin{attuneability}[\nth{4}]{\hypertarget{spell:Enhanced Icecraft}{Enhanced Icecraft}}[\glossterm{Attune} (self), \glossterm{Cold}]
This spell functions like the \spell{sturdy icecraft} spell, except that the item created is magically enhanced.
A weapon gains a \plus1d \glossterm{magic bonus} to damage with \glossterm{strikes}, and armor grants a \plus1 \glossterm{magic bonus} to the defenses it improves.
\end{attuneability}
\vspace{0.25em}



\lowercase{\hypertarget{spell:Greater Cone of Cold}{}}\label{spell:Greater Cone of Cold}
\begin{apability}[\nth{4}]{\hypertarget{spell:Greater Cone of Cold}{Greater Cone of Cold}}[\glossterm{Cold}]
This spell functions like the \spell{cone of cold} spell, except it affects everything in a \arealarge cone from you and you gain a \plus1d bonus to damage.
\end{apability}
\vspace{0.25em}



\lowercase{\hypertarget{spell:Supreme Cone of Cold}{}}\label{spell:Supreme Cone of Cold}
\begin{apability}[\nth{7}]{\hypertarget{spell:Supreme Cone of Cold}{Supreme Cone of Cold}}[\glossterm{Cold}]
This spell functions like the \spell{cone of cold} spell, except it affects everything in a \areahuge cone from you and you gain a \plus2d bonus to damage.
\end{apability}
\vspace{0.25em}


\newpage
\begin{spellsection}{Delusion}

\begin{spellheader}
\spelldesc{Instill false emotions to influence creatures.}
\end{spellheader}


\parhead{Schools} Enchantment

\parhead{Mystic Sphere Lists} Arcane, Divine, Pact

\subsubsection{Cantrips}


\begin{freeability}{Cause Fear}[\glossterm{Emotion}, \glossterm{Mind}]
Make an attack vs. Mental against a creature within \rngmed range.
\hit The target is \glossterm{shaken} by you as a \glossterm{condition}.
\crit The target is \glossterm{frightened} by you as a \glossterm{condition}.
\end{freeability}

\end{spellsection}


\subsubsection{Spells}


\lowercase{\hypertarget{spell:Agony}{}}\label{spell:Agony}
\begin{apability}[\nth{1}]{\hypertarget{spell:Agony}{Agony}}[\glossterm{Emotion}, \glossterm{Mind}]
Make an attack vs. Mental against a creature within \rngmed range.
\miss You regain the \glossterm{action point} spent to cast this spell.
\hit The target is inflicted with agonizing pain as a \glossterm{condition}.
It suffers a \minus2 penalty to Mental defense.
% Does this need to clarify that it takes effect in the round the spell was cast?
In addition, at the end of each \glossterm{delayed action phase}, if the target took damage that round, it takes \glossterm{standard damage} \minus1d.
This damage is of all damage types that the target was damaged by during that round.
\end{apability}
\vspace{0.25em}



\lowercase{\hypertarget{spell:Enrage}{}}\label{spell:Enrage}
\begin{apability}[\nth{1}]{\hypertarget{spell:Enrage}{Enrage}}[\glossterm{Emotion}, \glossterm{Mind}]
Make an attack vs. Mental with a \plus2 bonus to \glossterm{accuracy} against a creature within \rngmed range.
\hit As a \glossterm{condition}, the target is unable to take any \glossterm{standard actions} that do not cause it to make an attack.
For example, it could make a \glossterm{strike} or cast an offensive spell, but it could not heal itself or summon an ally.
This cannot prevent it from taking the \textit{recover} or \textit{desperate recovery} actions.
\end{apability}
\vspace{0.25em}



\lowercase{\hypertarget{spell:Fearsome Aura}{}}\label{spell:Fearsome Aura}
\begin{attuneability}[\nth{1}]{\hypertarget{spell:Fearsome Aura}{Fearsome Aura}}[\glossterm{Attune} (self), \glossterm{Emotion}, \glossterm{Mind}]
You radiate an aura of fear in a \arealarge radius emanation.
When you attune to this spell, and at the end of each \glossterm{action phase} in subsequent rounds, make an attack vs. Mental against all creatures in the area that you did not already attack with this spell.
\hit Each target is \glossterm{shaken} by you as a \glossterm{condition}.
\end{attuneability}
\vspace{0.25em}



\lowercase{\hypertarget{spell:Terror}{}}\label{spell:Terror}
\begin{apability}[\nth{1}]{\hypertarget{spell:Terror}{Terror}}[\glossterm{Emotion}, \glossterm{Mind}]
Make an attack vs. Mental against a creature within \rngmed range.
\miss You regain the \glossterm{action point} spent to cast this spell.
\hit The target is \frightened by you as a \glossterm{condition}.
\crit The target is \panicked by you as a \glossterm{condition}.
\end{apability}
\vspace{0.25em}



\lowercase{\hypertarget{spell:Charm}{}}\label{spell:Charm}
\begin{attuneability}[\nth{2}]{\hypertarget{spell:Charm}{Charm}}[\glossterm{Attune} (self), \glossterm{Emotion}, \glossterm{Mind}, \glossterm{Subtle}]
Make an attack vs. Mental against a creature within \rnglong range.
If the target thinks that you or your allies are threatening it, you take a \minus5 penalty to accuracy on the attack.
\hit The target is \charmed by you.
Any act by you or your apparent allies that threatens or damages the \spell{charmed} person breaks the effect.
This effect is automatically \glossterm{dismissed} after one hour.
\crit As above, except that the effect is not automatically dismissed.
\end{attuneability}
\vspace{0.25em}



\lowercase{\hypertarget{spell:Redirected Terror}{}}\label{spell:Redirected Terror}
\begin{apability}[\nth{2}]{\hypertarget{spell:Redirected Terror}{Redirected Terror}}[\glossterm{Emotion}, \glossterm{Mind}]
This spell functions like the \spell{terror} spell, except that you also choose a willing ally within the spell's range.
The target is afraid of the chosen ally instead of being afraid of you.
\end{apability}
\vspace{0.25em}



\lowercase{\hypertarget{spell:Calm Emotions}{}}\label{spell:Calm Emotions}
\begin{apability}[\nth{3}]{\hypertarget{spell:Calm Emotions}{Calm Emotions}}[\glossterm{Emotion}, \glossterm{Mind}, \glossterm{Sustain} (standard)]
Make an attack vs. Mental against all creatures within a \arealarge radius from you.
\hit Each target has its emotions calmed.
The effects of all other \glossterm{Emotion} abilities on that target are \glossterm{suppressed}.
It cannot take violent actions (although it can defend itself) or do anything destructive.
If the target takes damage or feels that it is in danger, this effect is \glossterm{dismissed}.
\end{apability}
\vspace{0.25em}



\lowercase{\hypertarget{spell:Greater Fearsome Aura}{}}\label{spell:Greater Fearsome Aura}
\begin{attuneability}[\nth{4}]{\hypertarget{spell:Greater Fearsome Aura}{Greater Fearsome Aura}}[\glossterm{Attune} (self), \glossterm{Emotion}, \glossterm{Mind}]
This spell functions like the \spell{fearsome aura} spell, except that a struck target is \glossterm{frightened} instead of \glossterm{shaken}.
\end{attuneability}
\vspace{0.25em}



\lowercase{\hypertarget{spell:Inevitable Doom}{}}\label{spell:Inevitable Doom}
\begin{apability}[\nth{4}]{\hypertarget{spell:Inevitable Doom}{Inevitable Doom}}[\glossterm{Emotion}, \glossterm{Mind}]
This spell functions like the \spell{terror} spell, except that you gain a \plus2 bonus to \glossterm{accuracy}.
\end{apability}
\vspace{0.25em}



\lowercase{\hypertarget{spell:Mass Enrage}{}}\label{spell:Mass Enrage}
\begin{apability}[\nth{4}]{\hypertarget{spell:Mass Enrage}{Mass Enrage}}[\glossterm{Emotion}, \glossterm{Mind}]
This spell functions like the \spell{enrage} spell, except that it affects all enemies within a \areamed radius.
\end{apability}
\vspace{0.25em}



\lowercase{\hypertarget{spell:Amnesiac Charm}{}}\label{spell:Amnesiac Charm}
\begin{attuneability}[\nth{5}]{\hypertarget{spell:Amnesiac Charm}{Amnesiac Charm}}[\glossterm{Attune} (self), \glossterm{Emotion}, \glossterm{Mind}, \glossterm{Subtle}]
This spell functions like the \spell{charm} spell, except that when the spell ends, an affected target forgets all events that transpired during the spell's duration.
It becomes aware of its surroundings as if waking up from a daydream.
The target is not directly aware of any magical influence on its mind, though unusually paranoid or perceptive creatures may deduce that their minds were affected.
\end{attuneability}
\vspace{0.25em}


\newpage
\begin{spellsection}{Electromancy}

\begin{spellheader}
\spelldesc{Create electricity to injure and stun foes.}
\end{spellheader}


\parhead{Schools} Evocation

\parhead{Mystic Sphere Lists} Arcane, Nature, Pact

\subsubsection{Cantrips}


\begin{freeability}{Spark}[\glossterm{Electricity}]
Make an attack vs. Fortitude against everything in a \areamed, 5 ft.\ wide line from you.
\hit Each target takes electricity \glossterm{standard damage} \minus1d.
\end{freeability}

\end{spellsection}


\subsubsection{Spells}


\lowercase{\hypertarget{spell:Lightning Bolt}{}}\label{spell:Lightning Bolt}
\begin{apability}[\nth{1}]{\hypertarget{spell:Lightning Bolt}{Lightning Bolt}}[\glossterm{Electricity}]
Make an attack vs. Fortitude against everything in a \arealarge, 10 ft.\ wide line from you.
\hit Each target takes electricity \glossterm{standard damage}.
\end{apability}
\vspace{0.25em}



\lowercase{\hypertarget{spell:Shocking Grasp}{}}\label{spell:Shocking Grasp}
\begin{apability}[\nth{1}]{\hypertarget{spell:Shocking Grasp}{Shocking Grasp}}[\glossterm{Electricity}]
Make an attack vs. Fortitude against one creature or object you \glossterm{threaten}.
You gain a \plus4 bonus to \glossterm{concentration} checks to cast this spell.
\hit The target takes electricity \glossterm{standard damage} \plus2d.
\end{apability}
\vspace{0.25em}



\lowercase{\hypertarget{spell:Call Lightning}{}}\label{spell:Call Lightning}
\begin{apability}[\nth{2}]{\hypertarget{spell:Call Lightning}{Call Lightning}}[\glossterm{Electricity}]
Make an attack vs. Fortitude against everything in a \arealarge, 5 ft.\ wide vertical line within \rngmed range.
If you are outdoors in cloudy or stormy weather, you gain a \plus2 bonus to \glossterm{accuracy} with the attack.
If this spell has its area increased, such as with the Widened \glossterm{augment}, only the length of the line increases.
\hit Each target takes takes electricity \glossterm{standard damage} \plus2d.
\end{apability}
\vspace{0.25em}



\lowercase{\hypertarget{spell:Dynamo}{}}\label{spell:Dynamo}
\begin{apability}[\nth{2}]{\hypertarget{spell:Dynamo}{Dynamo}}[\glossterm{Electricity}]
This spell functions like the \spell{lightning bolt} spell, except that it gains the \glossterm{Sustain} (standard) tag.
The area affected by the spell becomes a \glossterm{zone} that is continuously filled with electrical pulses.
At the end of each \glossterm{action phase} in subsequent rounds, the attack is repeated in that area.
\end{apability}
\vspace{0.25em}



\lowercase{\hypertarget{spell:Magnetic}{}}\label{spell:Magnetic}
\begin{apability}[\nth{2}]{\hypertarget{spell:Magnetic}{Magnetic}}[\glossterm{Electricity}]
This spell functions like the \spell{lightning bolt} spell, except that you gain a \plus2 bonus to accuracy against targets wearing metal armor or otherwise carrying or composed of a significant amount of metal.
\end{apability}
\vspace{0.25em}



\lowercase{\hypertarget{spell:Uncontrolled Discharge}{}}\label{spell:Uncontrolled Discharge}
\begin{apability}[\nth{2}]{\hypertarget{spell:Uncontrolled Discharge}{Uncontrolled Discharge}}[\glossterm{Electricity}]
Make an attack vs. Fortitude against everything in a \areamed radius from you.
\hit Each target takes electricity \glossterm{standard damage}.
\end{apability}
\vspace{0.25em}



\lowercase{\hypertarget{spell:Forked Lightning}{}}\label{spell:Forked Lightning}
\begin{apability}[\nth{3}]{\hypertarget{spell:Forked Lightning}{Forked Lightning}}[\glossterm{Electricity}]
This spell functions like the \spell{lightning bolt} spell, except that you gain a \plus1d bonus to damage.
In addition, you create two separate line-shaped areas instead of one.
The two areas can overlap, but targets in the overlapping area are only affected once.
\end{apability}
\vspace{0.25em}



\lowercase{\hypertarget{spell:Greater Dynamo}{}}\label{spell:Greater Dynamo}
\begin{apability}[\nth{3}]{\hypertarget{spell:Greater Dynamo}{Greater Dynamo}}[\glossterm{Electricity}]
This spell functions like the \textit{dynamo} spell, except that the spell gains the \glossterm{Sustain} (minor) tag instead of the \glossterm{Sustain} (standard) tag.
\end{apability}
\vspace{0.25em}



\lowercase{\hypertarget{spell:Magnetic Blade}{}}\label{spell:Magnetic Blade}
\begin{attuneability}[\nth{3}]{\hypertarget{spell:Magnetic Blade}{Magnetic Blade}}[\glossterm{Attune} (target), \glossterm{Electricity}]
Choose a willing creature within \rngclose range.
Metal weapons wielded by the target gain a \plus2 \glossterm{magic bonus} to \glossterm{accuracy} against targets wearing metal armor or otherwise carrying or composed of a significant amount of metal.

You can cast this spell as a \glossterm{minor action}.
\end{attuneability}
\vspace{0.25em}



\lowercase{\hypertarget{spell:Shocking Bolt}{}}\label{spell:Shocking Bolt}
\begin{apability}[\nth{3}]{\hypertarget{spell:Shocking Bolt}{Shocking Bolt}}[\glossterm{Electricity}]
This spell functions like the \spell{lightning bolt} spell, except that each struck target is also \glossterm{dazed} as a \glossterm{condition}.
\end{apability}
\vspace{0.25em}



\lowercase{\hypertarget{spell:Chain Lightning}{}}\label{spell:Chain Lightning}
\begin{apability}[\nth{4}]{\hypertarget{spell:Chain Lightning}{Chain Lightning}}[\glossterm{Electricity}]
Make an attack vs. Fortitude against one creature or object within \rngmed range.
\hit The target takes electricity \glossterm{standard damage} \plus3d.
In addition, make an additional attack vs. Fortitude against any number of creatures in a \areamed radius from the struck target.
\hit Each secondary target takes electricity \glossterm{standard damage} \plus1d.
\end{apability}
\vspace{0.25em}



\lowercase{\hypertarget{spell:Greater Magnetic Blade}{}}\label{spell:Greater Magnetic Blade}
\begin{attuneability}[\nth{6}]{\hypertarget{spell:Greater Magnetic Blade}{Greater Magnetic Blade}}[\glossterm{Attune} (self), \glossterm{Electricity}]
This spell functions like the \spell{magnetic blade} spell, except that the bonus is increased to \plus3.
\end{attuneability}
\vspace{0.25em}



\lowercase{\hypertarget{spell:Stunning Bolt}{}}\label{spell:Stunning Bolt}
\begin{apability}[\nth{6}]{\hypertarget{spell:Stunning Bolt}{Stunning Bolt}}[\glossterm{Electricity}]
This spell functions like the \spell{lightning bolt} spell, except that each struck target is also \glossterm{stunned} as a \glossterm{condition}.
\end{apability}
\vspace{0.25em}


\newpage
\begin{spellsection}{Fabrication}

\begin{spellheader}
\spelldesc{Create objects to damage and impair foes.}
\end{spellheader}


\parhead{Schools} Conjuration

\parhead{Mystic Sphere Lists} Arcane, Pact

\subsubsection{Cantrips}


\begin{freeability}{Acid Splash}[\glossterm{Acid}, \glossterm{Manifestation}]
Make an attack vs. Armor against one creature or object within \rngmed range.
\hit The target takes acid \glossterm{standard damage}.
\end{freeability}


\begin{attuneability}{Fabricate Trinket}[\glossterm{Attune} (self), \glossterm{Manifestation}]
You make a Craft check to create an object of Tiny size or smaller.
The object appears in your hand or at your feet.
It must be made of nonliving, nonmagical, nonreactive vegetable matter, such as wood or cloth.
\end{attuneability}

\end{spellsection}


\subsubsection{Spells}


\lowercase{\hypertarget{spell:Acid Orb}{}}\label{spell:Acid Orb}
\begin{apability}[\nth{1}]{\hypertarget{spell:Acid Orb}{Acid Orb}}[\glossterm{Acid}, \glossterm{Manifestation}]
Make an attack vs. Armor against one creature or object within \rngmed range.
\hit The target takes acid \glossterm{standard damage} \plus2d.
\end{apability}
\vspace{0.25em}



\lowercase{\hypertarget{spell:Forge}{}}\label{spell:Forge}
\begin{attuneability}[\nth{1}]{\hypertarget{spell:Forge}{Forge}}[\glossterm{Attune} (self)]
Choose a type of weapon or shield that you are proficient with.
You create a normal item of that type anywhere within \rngclose range.

The item cannot be constructed of any magical or extraordinary material.
% This should allow the Giant augment; is this worded to allow that?
It is sized appropriately for you, up to a maximum of a Medium size item.
\end{attuneability}
\vspace{0.25em}



\lowercase{\hypertarget{spell:Greater Forge}{}}\label{spell:Greater Forge}
\begin{attuneability}[\nth{2}]{\hypertarget{spell:Greater Forge}{Greater Forge}}[\glossterm{Attune} (self)]
This spell functions like the \spell{forge} spell, except that you can also create any type of body armor you are proficient with.
If you create body armor, you can create it already equipped to a willing creature within range.
\end{attuneability}
\vspace{0.25em}



\lowercase{\hypertarget{spell:Poison}{}}\label{spell:Poison}
\begin{apability}[\nth{2}]{\hypertarget{spell:Poison}{Poison}}[\glossterm{Manifestation}, \glossterm{Poison}]
Make an attack vs. Fortitude against a creature within \rngmed range.

\hit The target takes poison \glossterm{standard damage} \plus1d and is poisoned as a \glossterm{condition}.
If the target is poisoned, repeat this attack at the end of each \glossterm{action phase} after the first round.
On the second hit, the target takes poison \glossterm{standard damage} and becomes \glossterm{sickened}.
On the third hit, the target takes poison \glossterm{standard damage} \plus2d and becomes \glossterm{nauseated} instead of sickened.
After the third hit, no further attacks are made, but the target remains nauseated until the condition is removed.
\end{apability}
\vspace{0.25em}



\lowercase{\hypertarget{spell:Web}{}}\label{spell:Web}
\begin{apability}[\nth{2}]{\hypertarget{spell:Web}{Web}}[\glossterm{Manifestation}, \glossterm{Sustain} (minor)]
You fill a \areasmall radius zone in \rngclose range with webs.
The webs make the area \glossterm{difficult terrain}.
Each 5-ft.\ square of webbing has hit points equal to your \glossterm{power}, and is \glossterm{vulnerable} to fire.

In addition, you make an attack vs. Reflex against all Large or smaller creatures in the area when the spell is cast.
\hit Each target is \glossterm{immobilized} as long as it has webbing from this ability in its space.
\end{apability}
\vspace{0.25em}



\lowercase{\hypertarget{spell:Corrosive Orb}{}}\label{spell:Corrosive Orb}
\begin{apability}[\nth{3}]{\hypertarget{spell:Corrosive Orb}{Corrosive Orb}}[\glossterm{Acid}, \glossterm{Manifestation}]
This spell functions like the \spell{acid orb} spell, except that you gain a \plus1d bonus to damage and it deals double damage to objects.
\end{apability}
\vspace{0.25em}



\lowercase{\hypertarget{spell:Lingering Acid Orb}{}}\label{spell:Lingering Acid Orb}
\begin{apability}[\nth{3}]{\hypertarget{spell:Lingering Acid Orb}{Lingering Acid Orb}}[\glossterm{Acid}, \glossterm{Manifestation}]
This spell functions like the \spell{acid orb} spell, except that the acid lingers on a struck target.
At the end of each \glossterm{action phase} in subsequent rounds, the target takes acid \glossterm{standard damage}.
This is a \glossterm{condition}, and lasts until removed.
\end{apability}
\vspace{0.25em}



\lowercase{\hypertarget{spell:Meteor}{}}\label{spell:Meteor}
\begin{apability}[\nth{3}]{\hypertarget{spell:Meteor}{Meteor}}[\glossterm{Manifestation}]
You create a meteor in midair within \rngmed range that falls to the ground, crushing foes in its path.
The meteor takes up a \areamed radius, and must be created in unoccupied space.
After being summoned, it falls up to 100 feet before disappearing.
Make an attack vs. Armor against everything in its path.
\hit Each target takes bludgeoning and fire \glossterm{standard damage}.
\end{apability}
\vspace{0.25em}



\lowercase{\hypertarget{spell:Reinforced Webbing}{}}\label{spell:Reinforced Webbing}
\begin{apability}[\nth{3}]{\hypertarget{spell:Reinforced Webbing}{Reinforced Webbing}}[\glossterm{Manifestation}, \glossterm{Sustain} (minor)]
This spell functions like the \textit{web} spell, except that each 5-ft.\ square of webbing gains additional hit points equal to your \glossterm{power}.
In addition, the webs are no longer \glossterm{vulnerable} to fire damage.
\end{apability}
\vspace{0.25em}



\lowercase{\hypertarget{spell:Meteor Storm}{}}\label{spell:Meteor Storm}
\begin{apability}[\nth{5}]{\hypertarget{spell:Meteor Storm}{Meteor Storm}}[\glossterm{Manifestation}]
This spell functions like the \textit{meteor} spell, except that you can create up to five different meteors within \rnglong range.
The areas affected by two different meteors cannot overlap.
If one of the meteors is created in an invalid area, that meteor is not created, but the others are created and dealt their damage normally.
\end{apability}
\vspace{0.25em}



\subsubsection{Rituals}


\lowercase{\hypertarget{spell:Create Sustenance}{}}\label{spell:Create Sustenance}
\begin{apability}[\nth{2}]{\hypertarget{spell:Create Sustenance}{Create Sustenance}}[\glossterm{Creation}]
Choose an unoccupied square within \rngclose range.
This ritual creates food and drink in that square that is sufficient to sustain two Medium creatures per \glossterm{power} for 24 hours.
The food that this ritual creates is simple fare of your choice -- highly nourishing, if rather bland.

This ritual takes one hour to perform.
\end{apability}
\vspace{0.25em}



\lowercase{\hypertarget{spell:Manifest Object}{}}\label{spell:Manifest Object}
\begin{attuneability}[\nth{2}]{\hypertarget{spell:Manifest Object}{Manifest Object}}[\glossterm{Attune} (ritual), \glossterm{Manifestation}]
Make a Craft check to create an object of Small size or smaller.
The object appears out of thin air in an unoccupied square within \rngclose range.
% TODO: add ability to create objects of other sizes/materials
It must be made of nonliving, nonmagical, nonreactive vegetable matter, such as wood or cloth.

This ritual takes one hour to perform.
\end{attuneability}
\vspace{0.25em}


\newpage
\begin{spellsection}{Glamer}

\begin{spellheader}
\spelldesc{Change how creatures and objects are perceived.}
\end{spellheader}


\parhead{Schools} Illusion

\parhead{Mystic Sphere Lists} Arcane

\subsubsection{Cantrips}


\begin{freeability}{Assist Disguise}[\glossterm{Sensation}, \glossterm{Visual}]
Choose a willing creature within \rngclose range.
% TODO: wording?
If the target is disguised as another creature, it gains a \plus2 \glossterm{magic bonus} to the result of the disguise.
\end{freeability}


\begin{freeability}{Blur Weapon}[\glossterm{Sensation}, \glossterm{Visual}]
Choose a willing creature within \rngclose range.
The target's weapons become blurred and harder to see, making its attacks harder to avoid.
On the next melee \glossterm{strike} the target makes, it rolls twice and takes the higher result.
This effect ends at the end of the next round if the target has not made a strike by that time.

This effect provides no offensive benefit against creatures immune to \glossterm{Visual} abilities.
\end{freeability}

\end{spellsection}


\subsubsection{Spells}


\lowercase{\hypertarget{spell:Blur}{}}\label{spell:Blur}
\begin{attuneability}[\nth{1}]{\hypertarget{spell:Blur}{Blur}}[\glossterm{Attune} (target), \glossterm{Sensation}, \glossterm{Visual}]
Choose a willing creature within \rngmed range.
The target's physical outline is distorted so it appears blurred, shifting, and wavering.
It gains a \plus1 \glossterm{magic bonus} to Armor defense and Stealth (see \pcref{Stealth}).
This effect provides no defensive benefit against creatures immune to \glossterm{Visual} abilities.

You can cast this spell as a \glossterm{minor action}.
\end{attuneability}
\vspace{0.25em}



\lowercase{\hypertarget{spell:Hidden Blade}{}}\label{spell:Hidden Blade}
\begin{apability}[\nth{1}]{\hypertarget{spell:Hidden Blade}{Hidden Blade}}[\glossterm{Sensation}, \glossterm{Visual}]
You can only cast this spell during the \glossterm{action phase}.
Choose a willing creature within \rngclose range.
The target's weapons become invisible, and its hands are blurred.
On the next melee \glossterm{strike} the target makes,
the attack roll automatically \glossterm{explodes},
as if the target was \glossterm{unaware} of the attack.
This effect ends at the end of the current round if the target has not made a strike by that time.
% TODO: wording
The target is not actually \glossterm{unaware} of the attack, and this does not work with abilities that have effects if the target is unaware of attacks.

This effect provides no offensive benefit against creatures immune to \glossterm{Visual} abilities.
\end{apability}
\vspace{0.25em}



\lowercase{\hypertarget{spell:Suppress Light}{}}\label{spell:Suppress Light}
\begin{attuneability}[\nth{1}]{\hypertarget{spell:Suppress Light}{Suppress Light}}[\glossterm{Attune} (self), \glossterm{Light}, \glossterm{Sensation}]
Choose a Small or smaller unattended object within \rngclose range.
This spell suppresses light in a \areamed radius emanation from the target.
Light within or passing through the area is dimmed to be no brighter than shadowy illumination.
Any object or effect which blocks light also blocks this spell's emanation.
\end{attuneability}
\vspace{0.25em}



\lowercase{\hypertarget{spell:Disguise Image}{}}\label{spell:Disguise Image}
\begin{attuneability}[\nth{2}]{\hypertarget{spell:Disguise Image}{Disguise Image}}[\glossterm{Attune} (target), \glossterm{Sensation}, \glossterm{Visual}]
Choose a willing creature within \rngclose range.
You make a Disguise check to alter the target's appearance (see \pcref{Disguise Creature}).
You gain a \plus5 bonus on the check, and you can freely alter the appearance of the target's clothes and equipment, regardless of their original form.
However, this effect is unable to alter the sound, smell, texture, or temperature of the target or its clothes and equipment.
\end{attuneability}
\vspace{0.25em}



\lowercase{\hypertarget{spell:Mirror Image}{}}\label{spell:Mirror Image}
\begin{attuneability}[\nth{2}]{\hypertarget{spell:Mirror Image}{Mirror Image}}[\glossterm{Attune} (target), \glossterm{Sensation}, \glossterm{Visual}]
Choose a willing creature within \rngclose range.
Four illusory duplicates appear around the target that mirror its every move.
The duplicates shift chaotically in its space, making it difficult to identify the real creature.

All \glossterm{targeted} \glossterm{physical attacks} against the target have a 50\% miss chance.
When an attack misses in this way, it affects an image, destroying it.
This ability provides no defensive benefit against creatures immune to \glossterm{Visual} abilities.
\end{attuneability}
\vspace{0.25em}



\lowercase{\hypertarget{spell:Shadow Mantle}{}}\label{spell:Shadow Mantle}
\begin{attuneability}[\nth{3}]{\hypertarget{spell:Shadow Mantle}{Shadow Mantle}}[\glossterm{Attune} (target), \glossterm{Sensation}, \glossterm{Visual}]
This spell functions like the \spell{blur} spell, except that the spell's deceptive nature extends beyond altering light to affect the nature of reality itself.
The defense bonus it provides applies to all defenses.
In addition, the spell loses the \glossterm{Visual} tag, and can protect against attacks from creatures immune to Visual abilities.
\end{attuneability}
\vspace{0.25em}



\lowercase{\hypertarget{spell:Greater Mirror Image}{}}\label{spell:Greater Mirror Image}
\begin{attuneability}[\nth{4}]{\hypertarget{spell:Greater Mirror Image}{Greater Mirror Image}}[\glossterm{Attune} (target), \glossterm{Sensation}, \glossterm{Visual}]
This spell functions like the \textit{mirror image} spell, except that destroyed images can reappear.
At the end of each \glossterm{action phase}, one destroyed image reappears, to a maximum of four images.
\end{attuneability}
\vspace{0.25em}



\lowercase{\hypertarget{spell:Displacement}{}}\label{spell:Displacement}
\begin{attuneability}[\nth{6}]{\hypertarget{spell:Displacement}{Displacement}}[\glossterm{Attune} (target), \glossterm{Sensation}, \glossterm{Visual}]
Choose a willing creature within \rngmed range.
The target's image appears to be two to three feet from its real location.
\glossterm{Targeted} \glossterm{physical attacks} against the target suffer a 50\% miss chance.
This ability provides no defensive benefit against creatures immune to \glossterm{Visual} abilities.
\end{attuneability}
\vspace{0.25em}



\subsubsection{Rituals}


\lowercase{\hypertarget{spell:Magic Mouth}{}}\label{spell:Magic Mouth}
\begin{attuneability}[\nth{1}]{\hypertarget{spell:Magic Mouth}{Magic Mouth}}[\glossterm{Attune} (ritual), \glossterm{Sensation}]
Choose a Large or smaller willing creature or unattended object within \rngclose range.
In addition, choose a triggering condition and a message of twenty-five words or less.
The condition must be something that a typical human in the target's place could detect.

When the triggering condition occurs, the target appears to grow a magically animated mouth.
The mouth speaks the chosen message aloud.
After the message is spoken, this effect is \glossterm{dismissed}.

This ritual takes one minute to perform.
\end{attuneability}
\vspace{0.25em}


\newpage
\begin{spellsection}{Photomancy}

\begin{spellheader}
\spelldesc{Create bright light to blind foes and illuminate your surroundings.}
\end{spellheader}


\parhead{Schools} Illusion

\parhead{Mystic Sphere Lists} Arcane, Divine, Nature, Pact

\subsubsection{Cantrips}


\begin{freeability}{Flash}[\glossterm{Light}, \glossterm{Sensation}, \glossterm{Visual}]
Make an attack vs. Fortitude against one creature, object, or location within \rngmed range.
Bright light illuminates a 100 foot radius around the target until the end of the round.
\hit The target is \dazzled as a \glossterm{condition}.
\crit As above, and target is also \dazed as an additional \glossterm{condition}.
\end{freeability}


\begin{freeability}{Illuminate}[\glossterm{Light}, \glossterm{Sensation}, \glossterm{Visual}]
Choose a location within \rngmed range.
A glowing light appears in midair in the chosen location.
It casts bright light in a 20 foot radius and dim light in a 40 foot radius.
This effect lasts until you use it again or until you \glossterm{dismiss} it as a \glossterm{free action}.
\end{freeability}

\end{spellsection}


\subsubsection{Spells}


\lowercase{\hypertarget{spell:Flare}{}}\label{spell:Flare}
\begin{apability}[\nth{1}]{\hypertarget{spell:Flare}{Flare}}[\glossterm{Light}, \glossterm{Sensation}, \glossterm{Visual}]
A burst of light light fills a \areasmall radius within \rngmed range of you.
Bright light illuminates a 100 foot radius around the area until the end of the round.
Make an attack vs. Fortitude against all creatures in the source area.
\hit Each target is \dazzled as a \glossterm{condition}.
\crit As above, and target is also \dazed as an additional \glossterm{condition}.
\end{apability}
\vspace{0.25em}



\lowercase{\hypertarget{spell:Faerie Fire}{}}\label{spell:Faerie Fire}
\begin{apability}[\nth{2}]{\hypertarget{spell:Faerie Fire}{Faerie Fire}}[\glossterm{Light}, \glossterm{Sensation}, \glossterm{Visual}]
This spell functions like the \spell{flare} spell, except that each struck target is surrounded with a pale glow made of hundreds of ephemeral points of light.
This causes the struck target to radiate bright light in a 5 foot radius, as a candle.
The lights impose a \minus10 penalty to Stealth checks.
In addition, they reveal the outline of the creatures if they become \glossterm{invisible}.
This allows observers to see their location, though not to see them perfectly.
\end{apability}
\vspace{0.25em}



\lowercase{\hypertarget{spell:Illuminating}{}}\label{spell:Illuminating}
\begin{apability}[\nth{2}]{\hypertarget{spell:Illuminating}{Illuminating}}[\glossterm{Light}, \glossterm{Sensation}, \glossterm{Visual}]
This spell functions like the \spell{flare} spell, except that it gains the \glossterm{Sustain} (minor) tag.
The area affected by the spell becomes an illuminated \glossterm{zone}.
At the end of each \glossterm{action phase} in subsequent rounds, the attack is repeated in that area.
\end{apability}
\vspace{0.25em}



\lowercase{\hypertarget{spell:Flashbang}{}}\label{spell:Flashbang}
\begin{apability}[\nth{3}]{\hypertarget{spell:Flashbang}{Flashbang}}[\glossterm{Light}, \glossterm{Sensation}, \glossterm{Visual}]
This spell functions like the \spell{flare} spell, except that an intense sound accompanies the flash of light caused by the spell.
Each struck target is also \glossterm{deafened} as an additional \glossterm{condition}.
This spell gains the \glossterm{Auditory} tag in addition to the tags from the \spell{flare} spell.
\end{apability}
\vspace{0.25em}



\lowercase{\hypertarget{spell:Pillars of Light}{}}\label{spell:Pillars of Light}
\begin{apability}[\nth{3}]{\hypertarget{spell:Pillars of Light}{Pillars of Light}}[\glossterm{Light}, \glossterm{Sensation}, \glossterm{Visual}]
This spell functions like the \spell{flare} spell, except that you gain a \plus1 bonus to \glossterm{accuracy}.
In addition, it affects up to five different \areasmall radius, 50 ft. tall cylinders within range.
The areas can overlap, but targets in the overlapping area are only affected once.
\end{apability}
\vspace{0.25em}



\lowercase{\hypertarget{spell:Blinding}{}}\label{spell:Blinding}
\begin{apability}[\nth{4}]{\hypertarget{spell:Blinding}{Blinding}}[\glossterm{Light}, \glossterm{Sensation}, \glossterm{Visual}]
This spell functions like the \spell{flare} spell, except that each struck target is \glossterm{blinded} instead of \glossterm{dazzled}.
\end{apability}
\vspace{0.25em}



\lowercase{\hypertarget{spell:Kaleidoscopic Pattern}{}}\label{spell:Kaleidoscopic Pattern}
\begin{apability}[\nth{4}]{\hypertarget{spell:Kaleidoscopic Pattern}{Kaleidoscopic Pattern}}[\glossterm{Light}, \glossterm{Mind}, \glossterm{Sensation}, \glossterm{Visual}]
This spell creates a brilliant, rapidly shifting rainbow of lights in a \areasmall radius within \rngmed range of you.
They illuminate a 100 foot radius around the area with bright light until the end of the round.
Make an attack vs. Mental against all creatures in the source area.
\hit Each target is \disoriented as a \glossterm{condition}.
\crit Each target is \confused as a \glossterm{condition}.
\end{apability}
\vspace{0.25em}



\lowercase{\hypertarget{spell:Solar Flare}{}}\label{spell:Solar Flare}
\begin{apability}[\nth{4}]{\hypertarget{spell:Solar Flare}{Solar Flare}}[\glossterm{Light}, \glossterm{Sensation}, \glossterm{Visual}]
This spell functions like the \spell{flare} spell, except that you gain a \plus2 bonus to \glossterm{accuracy}.
In addition, the light is treated as being natural sunlight for the purpose of abilities.
This can allow it to destroy vampires and have similar effects.
\end{apability}
\vspace{0.25em}



\lowercase{\hypertarget{spell:Greater Solar Flare}{}}\label{spell:Greater Solar Flare}
\begin{apability}[\nth{7}]{\hypertarget{spell:Greater Solar Flare}{Greater Solar Flare}}[\glossterm{Light}, \glossterm{Sensation}, \glossterm{Visual}]
This spell functions like the \spell{solar flare} spell, except that the accuracy bonus is increased to \plus4.
\end{apability}
\vspace{0.25em}



\subsubsection{Rituals}


\lowercase{\hypertarget{spell:Mobile Light}{}}\label{spell:Mobile Light}
\begin{attuneability}[\nth{1}]{\hypertarget{spell:Mobile Light}{Mobile Light}}[\glossterm{Attune} (ritual), \glossterm{Light}, \glossterm{Sensation}]
Choose a Medium or smaller willing creature or unattended object within \rngclose range.
The target glows like a torch, shedding bright light in a \areamed radius (and dim light for an additional 20 feet).

This ritual takes one minute to perform.
\end{attuneability}
\vspace{0.25em}



\lowercase{\hypertarget{spell:Permanent Light}{}}\label{spell:Permanent Light}
\begin{apability}[\nth{2}]{\hypertarget{spell:Permanent Light}{Permanent Light}}[\glossterm{Light}, \glossterm{Sensation}]
This ritual functions like the \spell{light} ritual, except that it loses the \glossterm{Attune} (ritual) tag and the effect lasts permanently.
This ritual takes 24 hours to perform, and it requires 8 action points from its participants.
\end{apability}
\vspace{0.25em}


\newpage
\begin{spellsection}{Polymorph}

\begin{spellheader}
\spelldesc{Change the physical forms of objects and creatures.}
\end{spellheader}


\parhead{Schools} Transmutation

\parhead{Mystic Sphere Lists} Arcane, Nature, Pact

\subsubsection{Cantrips}


\begin{freeability}{Twist Body}[\glossterm{Shaping}]
Make an attack vs. Fortitude against a creature within \rngmed range.
\hit The target takes physical \glossterm{standard damage}.
\end{freeability}


\begin{freeability}{Alter Object}[\glossterm{Shaping}]
Choose an unattended, nonmagical object you can touch.
You make a Craft check to alter the target (see \pcref{Craft}), except that you do not need any special tools to make the check (such as an anvil and furnace).
The maximum hardness of a material you can affect with this ability is equal to your \glossterm{power}.

% too short?
Each time you use this ability, you can accomplish work that would take up to five minutes with a normal Craft check.
\end{freeability}

\end{spellsection}


\subsubsection{Spells}


\lowercase{\hypertarget{spell:Baleful Polymorph}{}}\label{spell:Baleful Polymorph}
\begin{apability}[\nth{1}]{\hypertarget{spell:Baleful Polymorph}{Baleful Polymorph}}[\glossterm{Shaping}]
Make an attack vs. Fortitude against a creature within \rngmed range.
\hit The target takes physical \glossterm{standard damage} \plus2d.
\end{apability}
\vspace{0.25em}



\lowercase{\hypertarget{spell:Shrink}{}}\label{spell:Shrink}
\begin{attuneability}[\nth{1}]{\hypertarget{spell:Shrink}{Shrink}}[\glossterm{Attune} (target), \glossterm{Shaping}, \glossterm{Sizing}]
Choose a willing creature within \rngclose range.
The target's size decreases by one size category, to a minimum of Tiny.
This decreases its Strength by 2 and usually decreases its \glossterm{reach} (see \pcref{Size in Combat}).

You can cast this spell as a \glossterm{minor action}.
\end{attuneability}
\vspace{0.25em}



\lowercase{\hypertarget{spell:Spider Climb}{}}\label{spell:Spider Climb}
\begin{attuneability}[\nth{1}]{\hypertarget{spell:Spider Climb}{Spider Climb}}[\glossterm{Attune} (target)]
Choose a willing creature within \rngclose range.
The target gains a \glossterm{climb speed} equal to its \glossterm{base speed}.
In addition, it gains a \plus5 bonus to Climb checks to climb on ceilings and similar surfaces.
\end{attuneability}
\vspace{0.25em}



\lowercase{\hypertarget{spell:Alter Appearance}{}}\label{spell:Alter Appearance}
\begin{attuneability}[\nth{2}]{\hypertarget{spell:Alter Appearance}{Alter Appearance}}[\glossterm{Attune} (target), \glossterm{Shaping}]
Choose a Large or smaller willing creature within \rngclose range.
You make a Disguise check to alter the target's appearance (see \pcref{Disguise Creature}).
You gain a \plus5 bonus on the check, and you ignore penalties for changing the target's gender, species, subtype, or age.
However, this effect is unable to alter the target's clothes or equipment in any way.

You can cast this spell as a \glossterm{minor action}.
\end{attuneability}
\vspace{0.25em}



\lowercase{\hypertarget{spell:Barkskin}{}}\label{spell:Barkskin}
\begin{attuneability}[\nth{2}]{\hypertarget{spell:Barkskin}{Barkskin}}[\glossterm{Attune} (target)]
Choose a willing creature within \rngclose range.
The target gains a \glossterm{magic bonus} equal to your \glossterm{power} to \glossterm{damage reduction} against damage dealt by \glossterm{physical attacks}.
In addition, it is \glossterm{vulnerable} to fire damage.

You can cast this spell as a \glossterm{minor action}.
\end{attuneability}
\vspace{0.25em}



\lowercase{\hypertarget{spell:Craft Object}{}}\label{spell:Craft Object}
\begin{apability}[\nth{3}]{\hypertarget{spell:Craft Object}{Craft Object}}[\glossterm{Shaping}]
Choose any number of unattended, nonmagical objects within \rngclose range.
You make a Craft check to transform the targets into a new item (or items) made of the same materials.
You require none of the tools or time expenditure that would normally be necessary.
The total size of all targets combined must be Large size or smaller.

You can apply the Giant \glossterm{augment} to this spell.
If you do, it increases the maximum size of all targets combined.
\end{apability}
\vspace{0.25em}



\lowercase{\hypertarget{spell:Enlarge}{}}\label{spell:Enlarge}
\begin{attuneability}[\nth{3}]{\hypertarget{spell:Enlarge}{Enlarge}}[\glossterm{Attune} (target), \glossterm{Shaping}, \glossterm{Sizing}]
Choose a Large or smaller willing creature within \rngclose range.
The target's size increases by one size category.
This increases its \glossterm{overwhelm value}, \glossterm{overwhelm resistance}, and usually increases its \glossterm{reach} (see \pcref{Size in Combat}).
However, the target's muscles are not increased fully to match its new size, and its Strength is unchanged.

You can cast this spell as a \glossterm{minor action}.
\end{attuneability}
\vspace{0.25em}



\lowercase{\hypertarget{spell:Stoneskin}{}}\label{spell:Stoneskin}
\begin{attuneability}[\nth{3}]{\hypertarget{spell:Stoneskin}{Stoneskin}}[\glossterm{Attune} (target)]
Choose a willing creature within \rngclose range.
The target gains a \glossterm{magic bonus} equal to your \glossterm{power} to \glossterm{damage reduction} against damage dealt by \glossterm{physical attacks}, except for damage from adamantine weapons.

You can cast this spell as a \glossterm{minor action}.
\end{attuneability}
\vspace{0.25em}



\lowercase{\hypertarget{spell:Greater Baleful Polymorph}{}}\label{spell:Greater Baleful Polymorph}
\begin{apability}[\nth{4}]{\hypertarget{spell:Greater Baleful Polymorph}{Greater Baleful Polymorph}}[\glossterm{Shaping}]
This spell functions like the \spell{baleful polymorph} spell, except that you gain a \plus1 bonus to \glossterm{accuracy} and a struck target is \glossterm{sickened} as a \glossterm{condition}.
\end{apability}
\vspace{0.25em}



\lowercase{\hypertarget{spell:Greater Shrink}{}}\label{spell:Greater Shrink}
\begin{attuneability}[\nth{4}]{\hypertarget{spell:Greater Shrink}{Greater Shrink}}[\glossterm{Attune} (target), \glossterm{Shaping}, \glossterm{Sizing}]
This spell functions like the \spell{shrink} spell, except that the target's size decreases by two size categories, to a minimum of Diminuitive.
\end{attuneability}
\vspace{0.25em}



\lowercase{\hypertarget{spell:Regeneration}{}}\label{spell:Regeneration}
\begin{attuneability}[\nth{4}]{\hypertarget{spell:Regeneration}{Regeneration}}[\glossterm{Attune} (target)]
Choose a willing creature within \rngclose range.
A the end of each round, the target heals hit points equal to your \glossterm{power}.
\end{attuneability}
\vspace{0.25em}



\lowercase{\hypertarget{spell:Disintegrate}{}}\label{spell:Disintegrate}
\begin{apability}[\nth{5}]{\hypertarget{spell:Disintegrate}{Disintegrate}}[\glossterm{Shaping}]
Make an attack vs. Fortitude against a creature within \rngmed range.
\hit The target takes physical \glossterm{standard damage} \plus4d.
In addition, if the target has no hit points remaining, it dies.
Its body is completely disintegrated, leaving behind only a pinch of fine dust.
Its equipment is unaffected.
\end{apability}
\vspace{0.25em}



\lowercase{\hypertarget{spell:Enlarge, Greater}{}}\label{spell:Enlarge, Greater}
\begin{attuneability}[\nth{5}]{\hypertarget{spell:Enlarge, Greater}{Enlarge, Greater}}[\glossterm{Attune} (target), \glossterm{Shaping}, \glossterm{Sizing}]
This spell functions like the \textit{enlarge} spell, except that the target's Strength is increased by 2 to match its new size.
\end{attuneability}
\vspace{0.25em}



\lowercase{\hypertarget{spell:Ironskin}{}}\label{spell:Ironskin}
\begin{attuneability}[\nth{6}]{\hypertarget{spell:Ironskin}{Ironskin}}[\glossterm{Attune} (target)]
This spell functions like the \textit{stoneskin} spell, except that the bonus is equal to twice your \glossterm{power}.
\end{attuneability}
\vspace{0.25em}



\lowercase{\hypertarget{spell:Enlarge, Supreme}{}}\label{spell:Enlarge, Supreme}
\begin{attuneability}[\nth{7}]{\hypertarget{spell:Enlarge, Supreme}{Enlarge, Supreme}}[\glossterm{Attune} (target), \glossterm{Shaping}, \glossterm{Sizing}]
This spell functions like the \spell{enlarge} spell, except that the target's size is increased by two size categories.
Its Strength is increased by 2 to partially match its new size.
\end{attuneability}
\vspace{0.25em}



\lowercase{\hypertarget{spell:Greater Regeneration}{}}\label{spell:Greater Regeneration}
\begin{attuneability}[\nth{7}]{\hypertarget{spell:Greater Regeneration}{Greater Regeneration}}[\glossterm{Attune} (target)]
This spell functions like the \textit{regeneration} spell, except that the healing is equal to twice your \glossterm{power}.
\end{attuneability}
\vspace{0.25em}



\lowercase{\hypertarget{spell:Supreme Baleful Polymorph}{}}\label{spell:Supreme Baleful Polymorph}
\begin{apability}[\nth{7}]{\hypertarget{spell:Supreme Baleful Polymorph}{Supreme Baleful Polymorph}}[\glossterm{Shaping}]
This spell functions like the \spell{baleful polymorph} spell, except that you gain a \plus2 bonus to \glossterm{accuracy} and a struck target is \glossterm{nauseated} as a \glossterm{condition}.
\end{apability}
\vspace{0.25em}



\subsubsection{Rituals}


\lowercase{\hypertarget{spell:Fortify}{}}\label{spell:Fortify}
\begin{attuneability}[\nth{1}]{\hypertarget{spell:Fortify}{Fortify}}[\glossterm{Attune} (ritual)]
Choose an unattended, nonmagical object or part of an object of up to Large size.
Unlike most abilities, this ritual can affect individual parts of a whole object.

% How should this affect Strength break DRs?
The target gains a \plus5 \glossterm{magic bonus} to \glossterm{hardness}.
If the target is moved, this effect ends.
Otherwise, it lasts for one year.

This ritual takes one hour to perform.
\end{attuneability}
\vspace{0.25em}



\lowercase{\hypertarget{spell:Mending}{}}\label{spell:Mending}
\begin{apability}[\nth{1}]{\hypertarget{spell:Mending}{Mending}}[\glossterm{Shaping}]
% TODO: unattended or attended by a willing creature
Choose an unattended object within \rngclose range.
The target is healed for hit points equal to \glossterm{standard damage} \plus2d.

This ritual takes one minute to perform.
\end{apability}
\vspace{0.25em}



\lowercase{\hypertarget{spell:Purify Sustenance}{}}\label{spell:Purify Sustenance}
\begin{apability}[\nth{1}]{\hypertarget{spell:Purify Sustenance}{Purify Sustenance}}[\glossterm{Shaping}]
All food and water in a single square within \rngclose range is purified.
Spoiled, rotten, poisonous, or otherwise contaminated food and water becomes pure and suitable for eating and drinking.
This does not prevent subsequent natural decay or spoiling.

This ritual takes one hour to perform.
\end{apability}
\vspace{0.25em}



\lowercase{\hypertarget{spell:Enduring Fortify}{}}\label{spell:Enduring Fortify}
\begin{apability}[\nth{3}]{\hypertarget{spell:Enduring Fortify}{Enduring Fortify}}
This ritual functions like the \spell{fortify} ritual, except that the effect lasts for one hundred years.
\end{apability}
\vspace{0.25em}



\lowercase{\hypertarget{spell:Greater Fortify}{}}\label{spell:Greater Fortify}
\begin{attuneability}[\nth{3}]{\hypertarget{spell:Greater Fortify}{Greater Fortify}}[\glossterm{Attune} (ritual)]
This ritual functions like the \spell{fortify} ritual, except that the \glossterm{hardness} bonus increases to 10.
\end{attuneability}
\vspace{0.25em}



\lowercase{\hypertarget{spell:Ironwood}{}}\label{spell:Ironwood}
\begin{apability}[\nth{3}]{\hypertarget{spell:Ironwood}{Ironwood}}[\glossterm{Shaping}]
Choose a Small or smaller unattended, nonmagical wooden object within \rngclose range.
The target is transformed into ironwood.
While remaining natural wood in almost every way, ironwood is as strong, heavy, and resistant to fire as iron.
Metallic armor and weapons, such as full plate, can be crafted from ironwood.

% Should this have an action point cost? May be too rare...
This ritual takes 24 hours to perform.
\end{apability}
\vspace{0.25em}



\lowercase{\hypertarget{spell:Awaken}{}}\label{spell:Awaken}
\begin{apability}[\nth{5}]{\hypertarget{spell:Awaken}{Awaken}}
Choose a Large or smaller willing animal within \rngclose range.
The target becomes sentient.
Its Intelligence becomes 1d6 \sub 5.
Its type changes from animal to magical beast.
It gains the ability to speak and understand one language that you know of your choice.
Its maximum age increases to that of a human (rolled secretly).
This effect is permanent.

This ritual takes 24 hours to perform, and requires 50 action points from its participants.
It can only be learned with the nature \glossterm{magic source}.
\end{apability}
\vspace{0.25em}



\lowercase{\hypertarget{spell:Greater Enduring Fortify}{}}\label{spell:Greater Enduring Fortify}
\begin{apability}[\nth{5}]{\hypertarget{spell:Greater Enduring Fortify}{Greater Enduring Fortify}}
This ritual functions like the \spell{greater fortify} ritual, except that the effect lasts for one hundred years.
\end{apability}
\vspace{0.25em}



\lowercase{\hypertarget{spell:Supreme Fortify}{}}\label{spell:Supreme Fortify}
\begin{attuneability}[\nth{6}]{\hypertarget{spell:Supreme Fortify}{Supreme Fortify}}[\glossterm{Attune} (ritual)]
This ritual functions like the \spell{fortify} ritual, except that the \glossterm{hardness} bonus increases to 15.
\end{attuneability}
\vspace{0.25em}


\newpage
\begin{spellsection}{Pyromancy}

\begin{spellheader}
\spelldesc{Create fire to incinerate foes.}
\end{spellheader}


\parhead{Schools} Evocation

\parhead{Mystic Sphere Lists} Arcane, Fire, Nature, Pact

\subsubsection{Cantrips}


\begin{freeability}{Personal Torch}[\glossterm{Fire}]
You create a flame in your hand.
You can create it at any intensity, up to a maximum heat equivalent to a burning torch.
At it most intense, it sheds bright light in a 20 foot radius and dim light in an 40 foot radius.
If you touch a creature or object with it, the target takes fire \glossterm{standard damage} \minus2d.
This effect lasts until you use it again or until you \glossterm{dismiss} it as a \glossterm{free action}.
\end{freeability}


\begin{freeability}{Scorch}[\glossterm{Fire}]
Make an attack vs. Armor against one creature or object within \rngmed range.
\hit The target takes fire \glossterm{standard damage}.
\end{freeability}

\end{spellsection}


\subsubsection{Spells}


\lowercase{\hypertarget{spell:Firebolt}{}}\label{spell:Firebolt}
\begin{apability}[\nth{1}]{\hypertarget{spell:Firebolt}{Firebolt}}[\glossterm{Fire}]
Make an attack vs. Armor against one creature within \rngmed range.
\hit The target takes fire \glossterm{standard damage} \plus2d.
\end{apability}
\vspace{0.25em}



\lowercase{\hypertarget{spell:Fireburst}{}}\label{spell:Fireburst}
\begin{apability}[\nth{1}]{\hypertarget{spell:Fireburst}{Fireburst}}[\glossterm{Fire}]
Make an attack vs. Armor against everything in a \areasmall radius within \rngmed range.
\hit Each target takes fire \glossterm{standard damage}.
\end{apability}
\vspace{0.25em}



\lowercase{\hypertarget{spell:Blast Furnace}{}}\label{spell:Blast Furnace}
\begin{apability}[\nth{2}]{\hypertarget{spell:Blast Furnace}{Blast Furnace}}[\glossterm{Fire}]
This spell functions like the \spell{fireburst} spell, except that it gains the \glossterm{Sustain} (standard) tag.
The area affected by the spell becomes a \glossterm{zone} that is continuously engulfed in flames.
At the end of each \glossterm{action phase} in subsequent rounds, the attack is repeated in that area.
\end{apability}
\vspace{0.25em}



\lowercase{\hypertarget{spell:Burning Hands}{}}\label{spell:Burning Hands}
\begin{apability}[\nth{2}]{\hypertarget{spell:Burning Hands}{Burning Hands}}[\glossterm{Fire}]
Make an attack vs. Armor against everything in a \arealarge cone from you.
\hit Each target takes fire \glossterm{standard damage}.
\end{apability}
\vspace{0.25em}



\lowercase{\hypertarget{spell:Fearsome Flame}{}}\label{spell:Fearsome Flame}
\begin{apability}[\nth{2}]{\hypertarget{spell:Fearsome Flame}{Fearsome Flame}}[\glossterm{Emotion}, \glossterm{Fire}, \glossterm{Mind}]
This spell functions like the \spell{fireburst} spell, except that the attack result is also compared to each target's Mental defense.
\hit Each target is \glossterm{shaken} by you as a \glossterm{condition}.
\end{apability}
\vspace{0.25em}



\lowercase{\hypertarget{spell:Flame Blade}{}}\label{spell:Flame Blade}
\begin{attuneability}[\nth{2}]{\hypertarget{spell:Flame Blade}{Flame Blade}}[\glossterm{Attune} (target), \glossterm{Fire}]
Choose a willing creature within \rngclose range.
% Is this clear enough at not stacking with magic bonuses intrinsic to the creature?
Weapons wielded by the target gain a \plus1d \glossterm{magic bonus} to damage with \glossterm{strikes}.
In addition, all damage dealt with strikes using its weapons becomes fire damage in addition to the attack's normal damage types.

You can cast this spell as a \glossterm{minor action}.
\end{attuneability}
\vspace{0.25em}



\lowercase{\hypertarget{spell:Ignition}{}}\label{spell:Ignition}
\begin{apability}[\nth{2}]{\hypertarget{spell:Ignition}{Ignition}}[\glossterm{Fire}]
This spell functions like the \spell{fireburst} spell, except that each struck target is also \glossterm{ignited} as a \glossterm{condition}.
This condition can be removed if the target makes a \glossterm{DR} 10 Dexterity check as a \glossterm{move action} to put out the flames.
Dropping \glossterm{prone} as part of this action gives a \plus5 bonus to this check.
\end{apability}
\vspace{0.25em}



\lowercase{\hypertarget{spell:Wall of Fire}{}}\label{spell:Wall of Fire}
\begin{attuneability}[\nth{2}]{\hypertarget{spell:Wall of Fire}{Wall of Fire}}[\glossterm{Attune} (self), \glossterm{Fire}]
You create a wall of fire in a 10 ft.\ high, \arealarge line within \rngmed range.
The flames and heat make it diffcult to see through the wall, granting \glossterm{concealment} to targets on the opposite side of the wall.
When a creature passes through the wall, you make an attack vs. Armor against that creature.
You can only make an attack in this way against a given creature once per \glossterm{phase}.
\hit The target takes fire \glossterm{standard damage}.

Each five-foot square of wall has hit points equal to twice your \glossterm{power}, and all of its defenses are 0.
It is immune to most forms of attack, but it can be destroyed by \glossterm{cold damage} and similar effects that can destroy water.
\end{attuneability}
\vspace{0.25em}



\lowercase{\hypertarget{spell:Fireball}{}}\label{spell:Fireball}
\begin{apability}[\nth{3}]{\hypertarget{spell:Fireball}{Fireball}}[\glossterm{Fire}]
Make an attack vs. Armor against everything in a \areamed radius within \rnglong range.
\hit Each target takes fire \glossterm{standard damage}.
\end{apability}
\vspace{0.25em}



\lowercase{\hypertarget{spell:Flame Serpent}{}}\label{spell:Flame Serpent}
\begin{apability}[\nth{3}]{\hypertarget{spell:Flame Serpent}{Flame Serpent}}[\glossterm{Fire}]
Make an attack vs. Armor against everything in a \arealarge, 5 ft.\ wide shapeable line within \rngmed range.
\hit Each target takes fire \glossterm{standard damage}.
\end{apability}
\vspace{0.25em}



\lowercase{\hypertarget{spell:Greater Blast Furnace}{}}\label{spell:Greater Blast Furnace}
\begin{apability}[\nth{3}]{\hypertarget{spell:Greater Blast Furnace}{Greater Blast Furnace}}[\glossterm{Fire}]
This spell functions like the \spell{blast furnace} spell, except that the spell gains the \glossterm{Sustain} (minor) tag instead of the \glossterm{Sustain} (standard) tag.
\end{apability}
\vspace{0.25em}



\lowercase{\hypertarget{spell:Inferno}{}}\label{spell:Inferno}
\begin{apability}[\nth{3}]{\hypertarget{spell:Inferno}{Inferno}}[\glossterm{Fire}]
Make an attack vs. Armor against everything in a \arealarge radius from you.
\hit Each target takes fire \glossterm{standard damage}.
\end{apability}
\vspace{0.25em}



\lowercase{\hypertarget{spell:Superheated Fireburst}{}}\label{spell:Superheated Fireburst}
\begin{apability}[\nth{3}]{\hypertarget{spell:Superheated Fireburst}{Superheated Fireburst}}[\glossterm{Fire}]
This spell functions like the \spell{fireburst} spell, except that it attacks Reflex defense instead of Armor defense.
\end{apability}
\vspace{0.25em}



\lowercase{\hypertarget{spell:Flame Aura}{}}\label{spell:Flame Aura}
\begin{attuneability}[\nth{4}]{\hypertarget{spell:Flame Aura}{Flame Aura}}[\glossterm{Attune} (target), \glossterm{Fire}]
Choose a willing creature within \rngclose range.
Heat constantly radiates in a \areamed radius emanation from the target.
At the end of each \glossterm{action phase}, make an attack vs. Armor against everything in the area.
\hit Each target takes fire \glossterm{standard damage} \minus2d.

You can cast this spell as a \glossterm{minor action}.
In addition, you can apply the Widened \glossterm{augment} to this spell.
If you do, it increases the area of the emanation.
\end{attuneability}
\vspace{0.25em}



\lowercase{\hypertarget{spell:Greater Ignition}{}}\label{spell:Greater Ignition}
\begin{apability}[\nth{4}]{\hypertarget{spell:Greater Ignition}{Greater Ignition}}[\glossterm{Fire}]
This spell functions like the \spell{fireburst} spell, except that each target hit is also \glossterm{ignited} as a \glossterm{condition}.
In addition, the ignited effect deals fire \glossterm{standard damage} \minus2d instead of the normal 1d6 fire damage each round.
\end{apability}
\vspace{0.25em}



\lowercase{\hypertarget{spell:Greater Inferno}{}}\label{spell:Greater Inferno}
\begin{apability}[\nth{5}]{\hypertarget{spell:Greater Inferno}{Greater Inferno}}[\glossterm{Fire}]
This spell functions like the \textit{inferno} spell, except that it affects everything in a 200 ft.\ radius from you.
\end{apability}
\vspace{0.25em}



\lowercase{\hypertarget{spell:Greater Fireball}{}}\label{spell:Greater Fireball}
\begin{apability}[\nth{6}]{\hypertarget{spell:Greater Fireball}{Greater Fireball}}[\glossterm{Fire}]
This spell functions like the \spell{fireball} spell, except that it affects everything in a \arealarge radius and you gain a \plus1d bonus to damage.
\end{apability}
\vspace{0.25em}



\lowercase{\hypertarget{spell:Supreme Ignition}{}}\label{spell:Supreme Ignition}
\begin{apability}[\nth{6}]{\hypertarget{spell:Supreme Ignition}{Supreme Ignition}}[\glossterm{Fire}]
This spell functions like the \textit{greater ignition} spell, except that the condition must be removed twice before the effect ends.
\end{apability}
\vspace{0.25em}


\newpage
\begin{spellsection}{Revelation}

\begin{spellheader}
\spelldesc{Share visions of the present and future, granting insight or combat prowess.}
\end{spellheader}


\parhead{Schools} Divination

\parhead{Mystic Sphere Lists} Arcane, Divine, Nature

\subsubsection{Cantrips}


\begin{freeability}{Precognitive Strike}
You can only cast this spell during the \glossterm{action phase}.
Choose a willing creature within \rngclose range.
On the next \glossterm{strike} the target makes, it rolls twice and takes the higher result.
If you cast this spell on another creature, the effect ends at the end of the current round if the target has not made a strike by that time.
If you cast this spell on yourself, it lasts until the end of the next round.
\end{freeability}

\end{spellsection}


\subsubsection{Spells}


\lowercase{\hypertarget{spell:Boon of Mastery}{}}\label{spell:Boon of Mastery}
\begin{attuneability}[\nth{1}]{\hypertarget{spell:Boon of Mastery}{Boon of Mastery}}[\glossterm{Attune} (target)]
Choose a willing creature within \rngclose range.
The target gains a \plus2 \glossterm{magic bonus} to all skills.
\end{attuneability}
\vspace{0.25em}



\lowercase{\hypertarget{spell:Precognitive Defense}{}}\label{spell:Precognitive Defense}
\begin{attuneability}[\nth{1}]{\hypertarget{spell:Precognitive Defense}{Precognitive Defense}}[\glossterm{Attune} (target)]
Choose a willing creature within \rngclose range.
The target gains a \plus1 \glossterm{magic bonus} to Armor defense and Reflex defense.
\end{attuneability}
\vspace{0.25em}



\lowercase{\hypertarget{spell:True Strike}{}}\label{spell:True Strike}
\begin{apability}[\nth{1}]{\hypertarget{spell:True Strike}{True Strike}}
Choose a willing creature within \rngclose range.
On the next \glossterm{strike} the target makes, it gains a \plus4 bonus to \glossterm{accuracy} and rolls twice and takes the higher result.
This effect ends at the end of the next round if the target has not made a strike by that time.
\end{apability}
\vspace{0.25em}



\lowercase{\hypertarget{spell:Boon of Knowledge}{}}\label{spell:Boon of Knowledge}
\begin{attuneability}[\nth{2}]{\hypertarget{spell:Boon of Knowledge}{Boon of Knowledge}}[\glossterm{Attune} (target)]
Choose a willing creature within \rngclose range.
The target gains a \plus4 \glossterm{magic bonus} to all Knowledge skills (see \pcref{Knowledge}).
\end{attuneability}
\vspace{0.25em}



\lowercase{\hypertarget{spell:Boon of Many Eyes}{}}\label{spell:Boon of Many Eyes}
\begin{attuneability}[\nth{2}]{\hypertarget{spell:Boon of Many Eyes}{Boon of Many Eyes}}[\glossterm{Attune} (target)]
Choose a willing creature within \rngclose range.
The target gains a \plus1 \glossterm{magic bonus} to \glossterm{overwhelm resistance}.
\end{attuneability}
\vspace{0.25em}



\lowercase{\hypertarget{spell:Discern Lies}{}}\label{spell:Discern Lies}
\begin{attuneability}[\nth{2}]{\hypertarget{spell:Discern Lies}{Discern Lies}}[\glossterm{Attune} (self), \glossterm{Detection}]
Make an attack vs. Mental against a creature within \rngmed range.
\hit You know when the target deliberately and knowingly speaks a lie.
This ability does not reveal the truth, uncover unintentional inaccuracies, or necessarily reveal evasions.
\end{attuneability}
\vspace{0.25em}



\lowercase{\hypertarget{spell:Precognitive Offense}{}}\label{spell:Precognitive Offense}
\begin{attuneability}[\nth{2}]{\hypertarget{spell:Precognitive Offense}{Precognitive Offense}}[\glossterm{Attune} (target)]
Choose a willing creature within \rngclose range.
The target gains a \plus1 \glossterm{magic bonus} to \glossterm{accuracy} with all attacks.

You can cast this spell as a \glossterm{minor action}.
\end{attuneability}
\vspace{0.25em}



\lowercase{\hypertarget{spell:Greater True Strike}{}}\label{spell:Greater True Strike}
\begin{apability}[\nth{3}]{\hypertarget{spell:Greater True Strike}{Greater True Strike}}
This spell functions like the \textit{true strike} spell, except that the bonus is increased to \plus6.
\end{apability}
\vspace{0.25em}



\lowercase{\hypertarget{spell:Third Eye}{}}\label{spell:Third Eye}
\begin{attuneability}[\nth{3}]{\hypertarget{spell:Third Eye}{Third Eye}}[\glossterm{Attune} (target)]
Choose a willing creature within \rngclose range.
The target gains \glossterm{blindsight} with a 50 foot range.
This can allow it to see perfectly without any light, regardless of concealment or invisibility.
\end{attuneability}
\vspace{0.25em}



\lowercase{\hypertarget{spell:Greater Boon of Mastery}{}}\label{spell:Greater Boon of Mastery}
\begin{attuneability}[\nth{4}]{\hypertarget{spell:Greater Boon of Mastery}{Greater Boon of Mastery}}[\glossterm{Attune} (target)]
This spell functions like the \spell{boon of mastery} spell, except that the bonus is increased to \plus4.
\end{attuneability}
\vspace{0.25em}



\lowercase{\hypertarget{spell:Greater Precognitive Defense}{}}\label{spell:Greater Precognitive Defense}
\begin{attuneability}[\nth{4}]{\hypertarget{spell:Greater Precognitive Defense}{Greater Precognitive Defense}}[\glossterm{Attune} (target)]
This spell functions like the \spell{precognitive defense} spell, except that the bonus is increased to \plus2.
\end{attuneability}
\vspace{0.25em}



\lowercase{\hypertarget{spell:Greater Precognitive Offense}{}}\label{spell:Greater Precognitive Offense}
\begin{attuneability}[\nth{5}]{\hypertarget{spell:Greater Precognitive Offense}{Greater Precognitive Offense}}[\glossterm{Attune} (target)]
This spell functions like the \spell{precognitive offense} spell, except that the bonus is increased to \plus2.
\end{attuneability}
\vspace{0.25em}



\lowercase{\hypertarget{spell:Supreme True Strike}{}}\label{spell:Supreme True Strike}
\begin{apability}[\nth{5}]{\hypertarget{spell:Supreme True Strike}{Supreme True Strike}}
This spell functions like the \textit{true strike} spell, except that the bonus is increased to \plus8.
\end{apability}
\vspace{0.25em}



\lowercase{\hypertarget{spell:Supreme Precognitive Defense}{}}\label{spell:Supreme Precognitive Defense}
\begin{attuneability}[\nth{7}]{\hypertarget{spell:Supreme Precognitive Defense}{Supreme Precognitive Defense}}[\glossterm{Attune} (target)]
This spell functions like the \spell{precognitive defense} spell, except that bonus is increased to \plus3.
\end{attuneability}
\vspace{0.25em}



\subsubsection{Rituals}


\lowercase{\hypertarget{spell:Read Magic}{}}\label{spell:Read Magic}
\begin{attuneability}[\nth{1}]{\hypertarget{spell:Read Magic}{Read Magic}}[\glossterm{Attune} (ritual)]
You gain the ability to decipher magical inscriptions that would otherwise be unintelligible.
This can allow you to read ritual books and similar objects created by other creatures.
After you have read an inscription in this way, you are able to read that particular writing without the use of this ritual.

This ritual takes one minute to perform.
\end{attuneability}
\vspace{0.25em}



\lowercase{\hypertarget{spell:Seek Legacy}{}}\label{spell:Seek Legacy}
\begin{apability}[\nth{2}]{\hypertarget{spell:Seek Legacy}{Seek Legacy}}
Choose a willing ritual participant.
The target learns the precise distance and direction to their \glossterm{legacy item}, if it is on the same plane.

This ritual takes 24 hours to perform, and requires 8 action points from its ritual participants.
\end{apability}
\vspace{0.25em}



\lowercase{\hypertarget{spell:Sending}{}}\label{spell:Sending}
\begin{apability}[\nth{3}]{\hypertarget{spell:Sending}{Sending}}[\glossterm{Sustain} (standard)]
Choose a creature on the same plane as you.
You do not need \glossterm{line of sight} or \glossterm{line of effect} to the target.
However,  must specify your target with a precise mental image of its appearance.
The image does not have to be perfect, but it must unambiguously identify the target.
If you specify its appearance incorrectly, or if the target has changed its appearance, you may accidentally target a different creature, or the ritual may simply fail.

You send the target a short verbal message.
The message must be twenty-five words or less, and speaking the message must not take longer than five rounds.

After the the target receives the message, it may reply with a message of the same length as long as the ritual's effect continues.
Once it speaks twenty-five words, or you stop sustaining the effect, the ritual is \glossterm{dismissed}.

This ritual takes one hour to perform.
\end{apability}
\vspace{0.25em}



\lowercase{\hypertarget{spell:Telepathic Bond}{}}\label{spell:Telepathic Bond}
\begin{attuneability}[\nth{3}]{\hypertarget{spell:Telepathic Bond}{Telepathic Bond}}[\glossterm{Attune} (ritual; see text)]
Choose up to five willing ritual participants.
Each target can communicate mentally through telepathy with each other target.
This communication is instantaneous, though it cannot reach more than 100 miles or across planes.

% Is this grammatically correct?
Each target must attune to this ritual independently.
If a target breaks its attunement, it stops being able to send and receive mental messages with other targets.
However, the effect continues as long as at least one target attunes to it.
If you \glossterm{dismiss} the ritual, the effect ends for all targets.

This ritual takes one minute to perform.
\end{attuneability}
\vspace{0.25em}



\lowercase{\hypertarget{spell:Discern Location}{}}\label{spell:Discern Location}
\begin{apability}[\nth{4}]{\hypertarget{spell:Discern Location}{Discern Location}}
Choose a creature or object on the same plane as you.
You do not need \glossterm{line of sight} or \glossterm{line of effect} to the target.
However, you must specify your target with a precise mental image of its appearance.
The image does not have to be perfect, but it must unambiguously identify the target.
You learn the location (place, name, business name, or the like), community, country, and continent where the target lies.
% Wording?
If there is no corresponding information about an aspect of the target's location, such as if the target is in a location which is not part of a recognized country,
you learn only that that that aspect of the information is missing.

This ritual takes 24 hours to perform, and it requires 32 action points from its participants.
\end{apability}
\vspace{0.25em}



\lowercase{\hypertarget{spell:Long-Distance Bond}{}}\label{spell:Long-Distance Bond}
\begin{attuneability}[\nth{5}]{\hypertarget{spell:Long-Distance Bond}{Long-Distance Bond}}[\glossterm{Attune} (ritual; see text)]
This ritual functions like the \ritual{telepathic bond} ritual, except that the effect works at any distance.
The communication still does not function across planes.
\end{attuneability}
\vspace{0.25em}



\lowercase{\hypertarget{spell:Interplanar Discern Location}{}}\label{spell:Interplanar Discern Location}
\begin{apability}[\nth{6}]{\hypertarget{spell:Interplanar Discern Location}{Interplanar Discern Location}}
This ritual functions like the \ritual{discern location} ritual, except that the target does not have to be on the same plane as you.
It gains the \glossterm{Planar} tag in addition to the tags from the \ritual{discern location} ritual.

This ritual takes 24 hours to perform, and it requires 72 action points from its participants.
\end{apability}
\vspace{0.25em}



\lowercase{\hypertarget{spell:Interplanar Sending}{}}\label{spell:Interplanar Sending}
\begin{apability}[\nth{6}]{\hypertarget{spell:Interplanar Sending}{Interplanar Sending}}[\glossterm{Sustain} (standard)]
This ritual functions like the \ritual{sending} ritual, except that the target does not have to be on the same plane as you.
It gains the \glossterm{Planar} tag in addition to the tags from the \ritual{sending} ritual.
\end{apability}
\vspace{0.25em}



\lowercase{\hypertarget{spell:Planar Bond}{}}\label{spell:Planar Bond}
\begin{attuneability}[\nth{7}]{\hypertarget{spell:Planar Bond}{Planar Bond}}[\glossterm{Attune} (ritual; see text)]
This ritual functions like the \ritual{telepathic bond} ritual, except that the effect works at any distance and across planes.
It gains the \glossterm{Planar} tag in addition to the tags from the \ritual{telepathic bond} ritual.
\end{attuneability}
\vspace{0.25em}


\newpage
\begin{spellsection}{Scry}

\begin{spellheader}
\spelldesc{See and hear at great distances.}
\end{spellheader}


\parhead{Schools} Divination

\parhead{Mystic Sphere Lists} Arcane, Divine, Nature

\subsubsection{Cantrips}


\begin{freeability}{Remote Sensing}[\glossterm{Scrying}, \glossterm{Sustain} (minor)]
This cantrip functions like the \textit{arcane eye} spell, except that it gains the \glossterm{Sustain} (minor) tag in place of the \glossterm{Attune} (self) tag.",
In addition, the sensor cannot be moved after it is originally created.
\end{freeability}

\end{spellsection}


\subsubsection{Spells}


\lowercase{\hypertarget{spell:Alarm}{}}\label{spell:Alarm}
\begin{attuneability}[\nth{1}]{\hypertarget{spell:Alarm}{Alarm}}[\glossterm{Attune} (self), \glossterm{Scrying}]
A \glossterm{scrying sensor} appears floating in the air in an unoccupied square within \rngmed range.
The sensor passively observes its surroundings.
If it sees a creature or object of Tiny size or larger moving within 50 feet of it, it will trigger a mental "ping" that only you can notice.
You must be within 1 mile of the sensor to receive this mental alarm.
This mental sensation is strong enough to wake you from normal sleep, but does not otherwise disturb concentration.
\end{attuneability}
\vspace{0.25em}



\lowercase{\hypertarget{spell:Arcane Eye}{}}\label{spell:Arcane Eye}
\begin{attuneability}[\nth{1}]{\hypertarget{spell:Arcane Eye}{Arcane Eye}}[\glossterm{Attune} (self), \glossterm{Scrying}]
A \glossterm{scrying sensor} appears floating in the air in an unoccupied square within \rngmed range.
At the start of each round, you choose whether you see from this sensor or from your body.

While viewing through the sensor, your visual acuity is the same as your normal body, except that it does not share the benefits of any \glossterm{magical} effects that improve your vision.
You otherwise act normally, though you may have difficulty moving or taking actions if the sensor cannot see your body or your intended targets, effectively making you \blinded.

If undisturbed, the sensor floats in the air in its position.
As a \glossterm{minor action}, you can concentrate to move the sensor up to 30 feet in any direction, even vertically.
At the end of each round, if the sensor is does not have \glossterm{line of effect} from you, it is destroyed.
\end{attuneability}
\vspace{0.25em}



\lowercase{\hypertarget{spell:Accelerated Eye}{}}\label{spell:Accelerated Eye}
\begin{attuneability}[\nth{2}]{\hypertarget{spell:Accelerated Eye}{Accelerated Eye}}[\glossterm{Attune} (self), \glossterm{Scrying}]
This spell functions like the \spell{arcane eye} spell, except that the sensor moves up to 100 feet when moved instead of up to 30 feet.
\end{attuneability}
\vspace{0.25em}



\lowercase{\hypertarget{spell:Auditory Eye}{}}\label{spell:Auditory Eye}
\begin{attuneability}[\nth{2}]{\hypertarget{spell:Auditory Eye}{Auditory Eye}}[\glossterm{Attune} (self), \glossterm{Scrying}]
This spell functions like the \spell{arcane eye} spell, except that you can you can also hear through the sensor.
At the start of each round, you can choose whether you hear from the sensor or from your body.
This choice is made independently from your sight.
The sensor's auditory acuity is the same as your own, except that it does not share the benefits of any \glossterm{magical} effects that improve your hearing.
\end{attuneability}
\vspace{0.25em}



\lowercase{\hypertarget{spell:Greater Alarm}{}}\label{spell:Greater Alarm}
\begin{attuneability}[\nth{2}]{\hypertarget{spell:Greater Alarm}{Greater Alarm}}[\glossterm{Attune} (self), \glossterm{Scrying}]
This spell functions like the \textit{alarm} spell, except that the sensor gains 100 ft.\ \glossterm{darkvision} and its Awareness bonus is equal to your \glossterm{power}.
\end{attuneability}
\vspace{0.25em}



\lowercase{\hypertarget{spell:Reverse Scrying}{}}\label{spell:Reverse Scrying}
\begin{attuneability}[\nth{2}]{\hypertarget{spell:Reverse Scrying}{Reverse Scrying}}[\glossterm{Attune} (self), \glossterm{Scrying}]
Choose a magical sensor within \rngmed range.
A new scrying sensor appears at the location of the source of the the ability that created the target sensor.
This sensor functions like the sensor created by the \spell{autonomous eye} spell, except that the sensor cannot move.
\end{attuneability}
\vspace{0.25em}



\lowercase{\hypertarget{spell:Autonomous Eye}{}}\label{spell:Autonomous Eye}
\begin{attuneability}[\nth{3}]{\hypertarget{spell:Autonomous Eye}{Autonomous Eye}}[\glossterm{Attune} (self), \glossterm{Scrying}]
This spell functions like the \spell{arcane eye} spell, except that the sensor is not destroyed when it loses \glossterm{line of effect} to you.
\end{attuneability}
\vspace{0.25em}



\lowercase{\hypertarget{spell:Twin Eye}{}}\label{spell:Twin Eye}
\begin{attuneability}[\nth{3}]{\hypertarget{spell:Twin Eye}{Twin Eye}}[\glossterm{Attune} (self), \glossterm{Scrying}]
This spell functions like the \spell{arcane eye} spell, except that you constantly receive sensory input from both your body and the sensor.
This allows you to see simultaneously from your body and from the sensor.
\end{attuneability}
\vspace{0.25em}



\lowercase{\hypertarget{spell:Penetrating Eye}{}}\label{spell:Penetrating Eye}
\begin{attuneability}[\nth{4}]{\hypertarget{spell:Penetrating Eye}{Penetrating Eye}}[\glossterm{Attune} (self), \glossterm{Scrying}]
This spell functions like the \spell{autonomous eye} spell, except that you do not need \glossterm{line of sight} or \glossterm{line of effect} to target a location.
You must specify a distance and direction to target a location you cannot see.
This can allow you to cast the spell beyond walls and similar obstacles.
As normal, if the intended location is occupied or otherwise impossible, the spell is \glossterm{miscast}.
\end{attuneability}
\vspace{0.25em}



\subsubsection{Rituals}


\lowercase{\hypertarget{spell:Scry Creature}{}}\label{spell:Scry Creature}
\begin{apability}[\nth{4}]{\hypertarget{spell:Scry Creature}{Scry Creature}}[\glossterm{Scrying}]
Make an attack vs. Mental against a creature on the same plane as you.
You do not need \glossterm{line of sight} or \glossterm{line of effect} to the target.
However,  must specify your target with a precise mental image of its appearance.
The image does not have to be perfect, but it must unambiguously identify the target.
If you specify its appearance incorrectly, or if the target has changed its appearance, you may accidentally target a different creature, or the spell may simply be \glossterm{miscast}.
This attack roll cannot \glossterm{explode}.
\hit A scrying sensor appears in the target's space.
This sensor functions like the sensor created by the \spell{arcane eye} spell, except that you cannot move the sensor manually.
Instead, it automatically tries to follow the target to stay in its space.
At the end of each phase, if the sensor is not in the target's space, this effect is \glossterm{dismissed}.

This ritual takes one hour to perform.
\end{apability}
\vspace{0.25em}



\lowercase{\hypertarget{spell:Interplanar Scry Creature}{}}\label{spell:Interplanar Scry Creature}
\begin{apability}[\nth{7}]{\hypertarget{spell:Interplanar Scry Creature}{Interplanar Scry Creature}}[\glossterm{Scrying}]
This ritual functions like the \ritual{scry creature} ritual, except that the target does not have to be on the same plane as you.
It gains the \glossterm{Planar} tag in addition to the tags from the \ritual{scry creature} ritual.
\end{apability}
\vspace{0.25em}


\newpage
\begin{spellsection}{Summon}

\begin{spellheader}
\spelldesc{Summon creatures to fight with you.}
\end{spellheader}


\parhead{Schools} Conjuration

\parhead{Mystic Sphere Lists} Arcane, Divine, Nature

\subsubsection{Cantrips}


\begin{freeability}{Sustained Summoning}[\glossterm{Manifestation}, \glossterm{Sustain} (standard)]
This cantrip functions like the \spell{summon monster} spell, except that it has the \glossterm{Sustain} (standard) tag instead of the \glossterm{Attune} (self) tag.
\end{freeability}

\end{spellsection}


\subsubsection{Spells}


\lowercase{\hypertarget{spell:Summon Monster}{}}\label{spell:Summon Monster}
\begin{attuneability}[\nth{1}]{\hypertarget{spell:Summon Monster}{Summon Monster}}[\glossterm{Attune} (self), \glossterm{Manifestation}]
You summon a creature in an unoccupied square on stable ground within \rngmed range.
It visually appears to be a common Small or Medium animal of your choice, though in reality it is a manifestation of magical energy.
Regardless of the appearance and size chosen, the creature's statistics are unchanged.
It has hit points equal to twice your \glossterm{power}.
% Has to be level instead of power because power can't scale directly with d10s ever
All of its defenses are equal to your 4 \add your level, and its \glossterm{land speed} is equal to 30 feet.
It does not have any \glossterm{action points}.

Each round, you can choose the creature's actions by mentally commanding it.
There are only two actions it can take.
As a \glossterm{move action}, it can move as you direct.
As a standard action, it can make a melee \glossterm{strike} against a creature it threatens.
Its accuracy is equal to your \glossterm{accuracy}.
If it hits, it deals \glossterm{standard damage} \minus1d.
The type of damage dealt by this attack depends on the creature's appearance.
Most animals bite or claw their foes, which deals bludgeoning and slashing damage.

If you do not command the creature's actions, it will continue to obey its last instructions if possible or do nothing otherwise.
\end{attuneability}
\vspace{0.25em}



\lowercase{\hypertarget{spell:Summon Bear}{}}\label{spell:Summon Bear}
\begin{attuneability}[\nth{2}]{\hypertarget{spell:Summon Bear}{Summon Bear}}[\glossterm{Attune} (self), \glossterm{Manifestation}]
This spell functions like the \spell{summon monster} spell, except that the creature appears to be a Medium bear.
As a standard action, it can make a \glossterm{grapple} attack against a creature it threatens.
Its accuracy is the same as its accuracy with \glossterm{strikes}.
While grappling, the manifested creature can either make a strike or attempt to escape the grapple.
\end{attuneability}
\vspace{0.25em}



\lowercase{\hypertarget{spell:Summon Mount}{}}\label{spell:Summon Mount}
\begin{attuneability}[\nth{2}]{\hypertarget{spell:Summon Mount}{Summon Mount}}[\glossterm{Attune} (target), \glossterm{Manifestation}]
This spell functions like the \spell{summon monster} spell, except that you must also choose a willing creature within \rngmed range to ride the summoned creature.
In addition, the summoned creature appears to be either a Large horse or a Medium pony.
It comes with a bit and bridle and a riding saddle, and will only accept the target as a rider.
It has the same statistics as a creature from the \spell{summon monster} spell, except that it follows its rider's directions to the extent that a well-trained horse would and it cannot attack.
\end{attuneability}
\vspace{0.25em}



\lowercase{\hypertarget{spell:Summon Bird}{}}\label{spell:Summon Bird}
\begin{attuneability}[\nth{3}]{\hypertarget{spell:Summon Bird}{Summon Bird}}[\glossterm{Attune} (self), \glossterm{Manifestation}]
This spell functions like the \spell{summon monster} spell, except that the creature appears to be a bird.
It has a 30 foot \glossterm{fly speed}.
\end{attuneability}
\vspace{0.25em}



\lowercase{\hypertarget{spell:Summon Flying Mount}{}}\label{spell:Summon Flying Mount}
\begin{attuneability}[\nth{4}]{\hypertarget{spell:Summon Flying Mount}{Summon Flying Mount}}[\glossterm{Attune} (target), \glossterm{Manifestation}]
This spell functions like the \spell{summon mount} spell, except that the summoned creature appears to be either a Large or Medium pegasus.
It has a 30 foot \glossterm{fly speed}.
\end{attuneability}
\vspace{0.25em}



\lowercase{\hypertarget{spell:Summon Wolfpack}{}}\label{spell:Summon Wolfpack}
\begin{attuneability}[\nth{4}]{\hypertarget{spell:Summon Wolfpack}{Summon Wolfpack}}[\glossterm{Attune} (self), \glossterm{Manifestation}]
This spell functions like the \spell{summon monster} spell, except that it summons a pack of four wolf-shaped creatures instead of a single creature.
Each creature has a \minus2 penalty to \glossterm{accuracy} and \glossterm{defenses} compared to a normal summoned creature.
% TODO: wording?
You must command the creatures as a group, rather than as individuals.
Each creature obeys your command to the extent it can.
\end{attuneability}
\vspace{0.25em}



\subsubsection{Rituals}


\lowercase{\hypertarget{spell:Ritual Mount}{}}\label{spell:Ritual Mount}
\begin{attuneability}[\nth{2}]{\hypertarget{spell:Ritual Mount}{Ritual Mount}}[\glossterm{Attune} (ritual), \glossterm{Manifestation}]
Choose a willing creature within \rngclose range.
This ritual summons your choice of a Large light horse or a Medium pony to serve as a mount.
The creature appears in an unoccupied location within \rngclose range.
It comes with a bit and bridle and a riding saddle, and will only accept the target as a rider.
It has the same statistics as a creature from the \spell{summon monster} spell, except that it follows its rider's directions to the extent that a well-trained horse would and it cannot attack.
\end{attuneability}
\vspace{0.25em}


\newpage
\begin{spellsection}{Telekinesis}

\begin{spellheader}
\spelldesc{Manipulate creatures and objects at a distance.}
\end{spellheader}


\parhead{Schools} Evocation

\parhead{Mystic Sphere Lists} Arcane, Pact

\subsubsection{Cantrips}


\begin{freeability}{Distant Hand}[\glossterm{Sustain} (standard)]
Choose a Medium or smaller unattended object within \rngclose range.
You can move it up to five feet in any direction within range, using your \glossterm{power} instead of your Strength to determine your maximum carrying capacity.

In addition, you can manipulate the target as if you were holding it in your hands.
Any attacks you make with the object or checks you make to manipulate the object have a maximum bonus equal to your \glossterm{power}.
\end{freeability}


\begin{freeability}{Telekinetic Compression}
Make an attack vs. Mental against one creature or object within \rngmed range.
\hit The target takes bludgeoning \glossterm{standard damage}.
\end{freeability}

\end{spellsection}


\subsubsection{Spells}


\lowercase{\hypertarget{spell:Telekinetic Crush}{}}\label{spell:Telekinetic Crush}
\begin{apability}[\nth{1}]{\hypertarget{spell:Telekinetic Crush}{Telekinetic Crush}}
Make an attack vs. Mental against one creature or object within \rngmed range.
\hit The target takes bludgeoning \glossterm{standard damage} \plus2d.
\end{apability}
\vspace{0.25em}



\lowercase{\hypertarget{spell:Telekinetic Lift}{}}\label{spell:Telekinetic Lift}
\begin{attuneability}[\nth{1}]{\hypertarget{spell:Telekinetic Lift}{Telekinetic Lift}}[\glossterm{Attune} (target)]
Choose a Medium or smaller willing creature or unattended object within \rngclose range.
The target is reduced to half of its normal weight.
This gives it a \plus4 bonus to Jump, if applicable, and makes it easier to lift and move.
\end{attuneability}
\vspace{0.25em}



\lowercase{\hypertarget{spell:Telekinetic Throw}{}}\label{spell:Telekinetic Throw}
\begin{apability}[\nth{1}]{\hypertarget{spell:Telekinetic Throw}{Telekinetic Throw}}
Make an attack vs. Mental against a Medium or smaller creature or object within \rngmed range.
\hit You move the target up to 30 feet in any direction.
You can change direction partway through the movement.
Moving the target upwards costs twice the normal movement cost.

% Wording?
If the target is willing, you can move it up to 100 feet.
\end{apability}
\vspace{0.25em}



\lowercase{\hypertarget{spell:Binding Crush}{}}\label{spell:Binding Crush}
\begin{apability}[\nth{2}]{\hypertarget{spell:Binding Crush}{Binding Crush}}
This spell functions like the \spell{telekinetic crush} spell, except that the struck creature is also \glossterm{slowed} as a \glossterm{condition} if it is Large or smaller.
\end{apability}
\vspace{0.25em}



\lowercase{\hypertarget{spell:Greater Telekinetic Lift}{}}\label{spell:Greater Telekinetic Lift}
\begin{attuneability}[\nth{3}]{\hypertarget{spell:Greater Telekinetic Lift}{Greater Telekinetic Lift}}[\glossterm{Attune} (target)]
This spell functions like the \spell{telekinetic lift} spell, except that the target is reduced to one quarter of its normal weight.
This increases the Jump bonus to \plus8.
\end{attuneability}
\vspace{0.25em}



\lowercase{\hypertarget{spell:Greater Telekinetic Throw}{}}\label{spell:Greater Telekinetic Throw}
\begin{apability}[\nth{3}]{\hypertarget{spell:Greater Telekinetic Throw}{Greater Telekinetic Throw}}
This spell functions like the \textit{telekinetic throw} spell, except that you can move the target up to 100 feet.
If the target is willing, you can move it up to 200 feet.
\end{apability}
\vspace{0.25em}



\lowercase{\hypertarget{spell:Wall of Force}{}}\label{spell:Wall of Force}
\begin{attuneability}[\nth{3}]{\hypertarget{spell:Wall of Force}{Wall of Force}}[\glossterm{Attune} (self)]
You create a wall of telekinetic force in a 10 ft.\ high, \arealarge line within \rngmed range.
The wall is transparent, but blocks physical passage and \glossterm{line of effect}.
Each five-foot square of wall has hit points equal to twice your \glossterm{power}, and all of its defenses are 0.
\end{attuneability}
\vspace{0.25em}



\lowercase{\hypertarget{spell:Levitate}{}}\label{spell:Levitate}
\begin{attuneability}[\nth{4}]{\hypertarget{spell:Levitate}{Levitate}}[\glossterm{Attune} (self)]
Choose a Medium or smaller willing creature or unattended object within \rngclose range.
% TODO: Wording
As long as the target remains within 50 feet above a surface that could support its weight, it floats in midair, unaffected by gravity.
During the movement phase, you can move the target up to ten feet in any direction as a \glossterm{free action}.
\end{attuneability}
\vspace{0.25em}



\lowercase{\hypertarget{spell:Greater Binding Crush}{}}\label{spell:Greater Binding Crush}
\begin{apability}[\nth{5}]{\hypertarget{spell:Greater Binding Crush}{Greater Binding Crush}}
This spell functions like the \spell{telekinetic crush} spell, except that the struck creature is also \glossterm{decelerated} as a \glossterm{condition} if it is Large or smaller.
\end{apability}
\vspace{0.25em}



\lowercase{\hypertarget{spell:Forcecage}{}}\label{spell:Forcecage}
\begin{attuneability}[\nth{7}]{\hypertarget{spell:Forcecage}{Forcecage}}[\glossterm{Attune} (self)]
You create a 10 ft.\ cube of telekinetic force within \rngmed range.
You can create the cube around a sufficiently small creature to trap it inside.
Each wall is transparent, but blocks physical passage and \glossterm{line of effect}.
Each five-foot square of wall has hit points equal to twice your \glossterm{power}, and all of its defenses are 0.
\end{attuneability}
\vspace{0.25em}


\newpage
\begin{spellsection}{Terramancy}

\begin{spellheader}
\spelldesc{Manipulate earth to crush foes.}
\end{spellheader}


\parhead{Schools} Conjuration, Transmutation

\parhead{Mystic Sphere Lists} Arcane, Nature

\subsubsection{Cantrips}


\begin{freeability}{Minor Earthspike}[\glossterm{Earth}, \glossterm{Physical}]
You create a spike of earth from the ground that quickly retracts, leaving the surface unchanged.
Make an attack vs. Armor against a creature or object within \rngmed range.
The target must be within 5 feet of a Small or larger body of earth or stone.
\hit The target takes piercing \glossterm{standard damage}.
\end{freeability}

\end{spellsection}


\subsubsection{Spells}


\lowercase{\hypertarget{spell:Earthcraft}{}}\label{spell:Earthcraft}
\begin{attuneability}[\nth{1}]{\hypertarget{spell:Earthcraft}{Earthcraft}}[\glossterm{Attune} (self), \glossterm{Earth}]
You create a weapon or suit of armor from a body of earth or unworked stone within 5 feet of you.
You can create any weapon, shield, or body armor that you are proficient with, and which would normally be made entirely from metal, except for heavy body armor.
The body targeted must be at least as large as the item you create.

The item functions like a normal item of its type, except that it is twice as heavy.
If the item loses all of its hit points, this effect is \glossterm{dismissed}.
\end{attuneability}
\vspace{0.25em}



\lowercase{\hypertarget{spell:Earthspike}{}}\label{spell:Earthspike}
\begin{apability}[\nth{1}]{\hypertarget{spell:Earthspike}{Earthspike}}[\glossterm{Earth}, \glossterm{Physical}]
You create a spike of earth from the ground.
Make an attack vs. Armor against a creature or object within \rngmed range.
The target must be within 5 feet of a Small or larger body of earth or stone.
\hit The target takes piercing \glossterm{standard damage} \plus1d and is \glossterm{slowed} as a \glossterm{condition}.
\end{apability}
\vspace{0.25em}



\lowercase{\hypertarget{spell:Rock Throw}{}}\label{spell:Rock Throw}
\begin{apability}[\nth{1}]{\hypertarget{spell:Rock Throw}{Rock Throw}}[\glossterm{Earth}, \glossterm{Physical}]
% TODO: define maximum hardness?
You extract a Tiny chunk from a body of earth or unworked stone within 5 feet of you and throw it at a foe.
If no such chunk can be extracted, this spell is \glossterm{miscast}.
Otherwise, make an attack vs. Armor against a creature or object within \rngmed range.
\hit The target takes bludgeoning \glossterm{standard damage} \plus2d.
\end{apability}
\vspace{0.25em}



\lowercase{\hypertarget{spell:Earthbind}{}}\label{spell:Earthbind}
\begin{apability}[\nth{2}]{\hypertarget{spell:Earthbind}{Earthbind}}[\glossterm{Earth}]
Make an attack vs. Fortitude against a creature within \rngmed range that is within 50 feet of the ground.
\hit As a \glossterm{condition}, the target is pulled towards the ground with great force, approximately quadrupling the gravity it experiences.
This imposes a \minus4 penalty to \glossterm{accuracy}, physical \glossterm{checks}, and \glossterm{defenses}.
In addition, most flying creatures are unable to fly with this increased gravity and crash to the ground.
\end{apability}
\vspace{0.25em}



\lowercase{\hypertarget{spell:Earthen Fortification}{}}\label{spell:Earthen Fortification}
\begin{attuneability}[\nth{2}]{\hypertarget{spell:Earthen Fortification}{Earthen Fortification}}[\glossterm{Attune} (self), \glossterm{Earth}, \glossterm{Manifestation}]
You construct a fortification made of packed earth within \rngmed range.
This takes the form of up to ten contiguous 5-foot squares, each of which is four inches thick.
The squares can be placed at any angle and used to form any structure as long as that structure is stable.
Since the fortifications are made of packed earth, their maximum weight is limited, and structures taller than ten feet high are usually impossible.
% TODO: define hit points and hardness of earth

The fortifications form slowly, rather than instantly.
The structure becomes complete at the end of the action phase in the next round after this spell is cast.
This makes it difficult to trap creatures within structures formed.
\end{attuneability}
\vspace{0.25em}



\lowercase{\hypertarget{spell:Meld into Stone}{}}\label{spell:Meld into Stone}
\begin{attuneability}[\nth{2}]{\hypertarget{spell:Meld into Stone}{Meld into Stone}}[\glossterm{Attune} (self), \glossterm{Earth}]
Choose a stone object you can touch that is at least as large as your body.
You and up to 100 pounds of nonliving equipment meld into the stone.
If you try to bring excess equipment into the stone, the spell is \glossterm{miscast}.

As long as the spell lasts, you can move within the stone as if it was thick water.
However, at least part of you must remain within one foot of the place you originally melded with the stone.
You gain no special ability to breathe or see while embedded the stone, and you cannot speak if your mouth is within the stone.
The stone muffles sound, but very loud noises may reach your ears within it.
If you fully exit the stone, this spell ends.

If this spell ends before you exit the stone, or if the stone stops being a valid target for the spell (such as if it is broken into pieces), you are forcibly expelled from the stone.
When you are forcibly expelled from the stone, you take 4d10 bludgeoning damage and become \glossterm{nauseated} as a \glossterm{condition}.
\end{attuneability}
\vspace{0.25em}



\lowercase{\hypertarget{spell:Quagmire}{}}\label{spell:Quagmire}
\begin{attuneability}[\nth{2}]{\hypertarget{spell:Quagmire}{Quagmire}}[\glossterm{Attune} (self), \glossterm{Earth}, \glossterm{Physical}]
% TODO: define maximum hardness
Choose a \areamed radius within \rngmed range.
All earth and unworked stone within the area is softened into a thick sludge, creating a quagmire that is difficult to move through.
The movement cost required to move out of each affected square within the area is quadrupled.
This does not affect objects under significant structural stress, such as walls and support columns.
\end{attuneability}
\vspace{0.25em}



\lowercase{\hypertarget{spell:Reinforced Earthcraft}{}}\label{spell:Reinforced Earthcraft}
\begin{attuneability}[\nth{2}]{\hypertarget{spell:Reinforced Earthcraft}{Reinforced Earthcraft}}[\glossterm{Attune} (self), \glossterm{Earth}]
This spell functions like the \spell{earthcraft} spell, except that the item is the same weight as a normal item of its type.
In addition, you can create heavy body armor.
\end{attuneability}
\vspace{0.25em}



\lowercase{\hypertarget{spell:Shrapnel Blast}{}}\label{spell:Shrapnel Blast}
\begin{apability}[\nth{2}]{\hypertarget{spell:Shrapnel Blast}{Shrapnel Blast}}[\glossterm{Earth}, \glossterm{Physical}]
You extract a Tiny chunk from a body of earth or unworked stone within 5 feet of you and blast it at your foes.
If no such chunk can be extracted, this spell is \glossterm{miscast}.
Otherwise, make an attack vs. Armor against everything in a \arealarge cone from you.
\hit Each target takes bludgeoning and piercing \glossterm{standard damage}.
\end{apability}
\vspace{0.25em}



\lowercase{\hypertarget{spell:Tremor}{}}\label{spell:Tremor}
\begin{apability}[\nth{2}]{\hypertarget{spell:Tremor}{Tremor}}[\glossterm{Earth}, \glossterm{Physical}]
You create an highly localized tremor that rips through the ground.
Make an attack vs. Reflex against all Large or smaller creatures other than yourself standing on the ground in a \areamed radius within \rngmed range.
\hit Each target is knocked \glossterm{prone}.
\end{apability}
\vspace{0.25em}



\lowercase{\hypertarget{spell:Fissure}{}}\label{spell:Fissure}
\begin{apability}[\nth{3}]{\hypertarget{spell:Fissure}{Fissure}}[\glossterm{Earth}, \glossterm{Physical}]
You open up a rift in the ground that swallows and traps a foe.
Make an attack vs. Reflex against a Large or smaller creature standing on earth or unworked stone within \rngmed range.
\miss You regain the \glossterm{action point} spent to cast this spell.
\hit The target is \glossterm{immobilized}.
As long as the target is immobilized in this way,
it takes bludgeoning \glossterm{standard damage} \minus1d at the end of each \glossterm{action phase} in subsequent rounds.
This immobilization can be removed by climbing out of the fissure, which requires a \glossterm{DR} 10 Climb check as a \glossterm{move action}.
Alternately, an ally that can reach the target can make a Strength check against the same DR to pull the target out.
Special movement abilities such as teleportation can also remove the target from the fissure.
\end{apability}
\vspace{0.25em}



\lowercase{\hypertarget{spell:Stone Fortification}{}}\label{spell:Stone Fortification}
\begin{attuneability}[\nth{3}]{\hypertarget{spell:Stone Fortification}{Stone Fortification}}[\glossterm{Attune} (self), \glossterm{Earth}, \glossterm{Manifestation}]
This spell functions like the \spell{earthen fortification} spell, except that the fortifications are made of stone instead of earth.
This makes them more resistant to attack and allows the construction of more complex structures.
% TODO: define hit points and hardness of stone
\end{attuneability}
\vspace{0.25em}



\lowercase{\hypertarget{spell:Impaling Earthspike}{}}\label{spell:Impaling Earthspike}
\begin{apability}[\nth{4}]{\hypertarget{spell:Impaling Earthspike}{Impaling Earthspike}}[\glossterm{Earth}, \glossterm{Physical}]
This spell functions like the \spell{earthspike} spell, except that a struck target is \glossterm{decelerated} instead of \glossterm{slowed}.
\end{apability}
\vspace{0.25em}



\lowercase{\hypertarget{spell:Earthquake}{}}\label{spell:Earthquake}
\begin{apability}[\nth{5}]{\hypertarget{spell:Earthquake}{Earthquake}}[\glossterm{Earth}, \glossterm{Physical}]
You create an intense but highly localized tremor that rips through the ground.
Make an attack vs. Reflex against all creatures other than yourself standing on the ground in a \arealarge radius within \rnglong range.
\hit Each target takes bludgeoning \glossterm{standard damage}.
If a target is Huge or smaller, it is also knocked \glossterm{prone}.
\end{apability}
\vspace{0.25em}



\lowercase{\hypertarget{spell:Fissure Swarm}{}}\label{spell:Fissure Swarm}
\begin{apability}[\nth{6}]{\hypertarget{spell:Fissure Swarm}{Fissure Swarm}}[\glossterm{Earth}, \glossterm{Physical}]
This spell functions like the \spell{fissure} spell, except that it affects all enemies in a \areamed radius within \rngmed range.
\end{apability}
\vspace{0.25em}


\newpage
\begin{spellsection}{Thaumaturgy}

\begin{spellheader}
\spelldesc{Suppress and manipulate magical effects.}
\end{spellheader}


\parhead{Schools} Abjuration

\parhead{Mystic Sphere Lists} Arcane, Divine

\subsubsection{Cantrips}


\begin{freeability}{Minor Suppression}[\glossterm{Mystic}, \glossterm{Sustain} (standard)]
Make an attack against one creature within \rngmed range.
The attack result is applied to every \glossterm{magical} effect on the target.
The DR for each effect is equal to the \glossterm{power} of that effect.
\hit Each effect is \glossterm{suppressed}.
\end{freeability}

\end{spellsection}


\subsubsection{Spells}


\lowercase{\hypertarget{spell:Alter Magic Aura}{}}\label{spell:Alter Magic Aura}
\begin{attuneability}[\nth{1}]{\hypertarget{spell:Alter Magic Aura}{Alter Magic Aura}}[\glossterm{Attune} (self), \glossterm{Mystic}]
Make an attack vs. Mental against one Large or smaller magical object in \rngmed range.
\hit One of the target's magic auras is altered (see \pcref{Spellcraft}).
You can change the school and descriptors of the aura.
In addition, you can decrease the \glossterm{power} of the aura by up to half your power, or increase the power of the aura up to a maximum of your power.
\end{attuneability}
\vspace{0.25em}



\lowercase{\hypertarget{spell:Enhance Magic}{}}\label{spell:Enhance Magic}
\begin{attuneability}[\nth{1}]{\hypertarget{spell:Enhance Magic}{Enhance Magic}}[\glossterm{Attune} (target), \glossterm{Mystic}]
Choose a willing creature within \rngmed range.
The target gains a \plus1 \glossterm{magic bonus} to \glossterm{power} with spells.
\end{attuneability}
\vspace{0.25em}



\lowercase{\hypertarget{spell:Suppress Item}{}}\label{spell:Suppress Item}
\begin{apability}[\nth{1}]{\hypertarget{spell:Suppress Item}{Suppress Item}}[\glossterm{Mystic}, \glossterm{Sustain} (minor)]
Make an attack vs. Mental against one Large or smaller magical object in \rngmed range.
\hit All magical properties the target has are \glossterm{suppressed}.
\end{apability}
\vspace{0.25em}



\lowercase{\hypertarget{spell:Suppress Magic}{}}\label{spell:Suppress Magic}
\begin{apability}[\nth{1}]{\hypertarget{spell:Suppress Magic}{Suppress Magic}}[\glossterm{Mystic}, \glossterm{Sustain} (standard)]
Make an attack against one creature, object, or magical effect within \rngmed range.
If you target a creature or object, the attack result is applied to every \glossterm{magical} effect on the target.
% Is this clear enough?
This does not affect the passive effects of any magic items the target has equipped.
If you target a magical effect directly, the attack result is applied against the effect itself.
The DR for each effect is equal to the \glossterm{power} of that effect.
\hit Each effect is \glossterm{suppressed}.
\end{apability}
\vspace{0.25em}



\lowercase{\hypertarget{spell:Dimensional Anchor}{}}\label{spell:Dimensional Anchor}
\begin{apability}[\nth{2}]{\hypertarget{spell:Dimensional Anchor}{Dimensional Anchor}}[\glossterm{Mystic}, \glossterm{Sustain} (minor), \glossterm{Swift}]
Make an attack vs. Mental against a creature or object within \rngmed range.
\hit The target is unable to travel extradimensionally.
This prevents all \glossterm{Manifestation}, \glossterm{Planar}, and \glossterm{Teleportation} effects.
\end{apability}
\vspace{0.25em}



\lowercase{\hypertarget{spell:Dismissal}{}}\label{spell:Dismissal}
\begin{apability}[\nth{2}]{\hypertarget{spell:Dismissal}{Dismissal}}[\glossterm{Mystic}]
Make an attack against one creature or object within \rngmed range.
If the target is an effect of an ongoing \glossterm{magical} ability, such as a summoned monster or created object, the DR is equal to the \glossterm{power} of the ability.
Otherwise, this spell has no effect.
\hit The target is treated as if the ability that created it was \glossterm{dismissed}.
This usually causes the target to disappear.
\end{apability}
\vspace{0.25em}



\lowercase{\hypertarget{spell:Dispel Magic}{}}\label{spell:Dispel Magic}
\begin{apability}[\nth{2}]{\hypertarget{spell:Dispel Magic}{Dispel Magic}}[\glossterm{Mystic}, \glossterm{Sustain} (standard)]
This spell functions like the \spell{suppress magic} spell, except that a hit against an effect causes it to be \glossterm{dismissed} instead of suppressed.
\end{apability}
\vspace{0.25em}



\lowercase{\hypertarget{spell:Malign Transferance}{}}\label{spell:Malign Transferance}
\begin{apability}[\nth{2}]{\hypertarget{spell:Malign Transferance}{Malign Transferance}}
Choose a willing ally within \rngmed range.
The ally must be currently affected by a \glossterm{magical} \glossterm{condition}.
In addition, make an attack vs. Mental against a creature within \rngmed range.
\miss You regain the \glossterm{action point} spent to cast this spell.
\hit One magical condition of your choice is removed from the chosen ally and applied to the struck creature.
\crit As above, except that you can transfer any number of magical conditions in this way.
\end{apability}
\vspace{0.25em}



\lowercase{\hypertarget{spell:Dimensional Lock}{}}\label{spell:Dimensional Lock}
\begin{attuneability}[\nth{4}]{\hypertarget{spell:Dimensional Lock}{Dimensional Lock}}[\glossterm{Attune} (self), \glossterm{Mystic}]
This spell creates a dimensional lock in a \arealarge radius zone from your location.
Extraplanar travel into or out of the area is impossible.
This prevents all \glossterm{Manifestation}, \glossterm{Planar}, and \glossterm{Teleportation} effects.
\end{attuneability}
\vspace{0.25em}



\lowercase{\hypertarget{spell:Greater Enhance Magic}{}}\label{spell:Greater Enhance Magic}
\begin{attuneability}[\nth{4}]{\hypertarget{spell:Greater Enhance Magic}{Greater Enhance Magic}}[\glossterm{Attune} (target), \glossterm{Mystic}]
This spell functions like the \textit{enhance magic} spell, except that the bonus is increased to \plus2.
\end{attuneability}
\vspace{0.25em}



\lowercase{\hypertarget{spell:Greater Malign Transferance}{}}\label{spell:Greater Malign Transferance}
\begin{apability}[\nth{5}]{\hypertarget{spell:Greater Malign Transferance}{Greater Malign Transferance}}
Choose any number of willing allies within \rngmed range.
Each ally must be currently affected by a \glossterm{magical} \glossterm{condition}.
In addition, make an attack vs. Mental against a creature within \rngmed range.
\miss You regain the \glossterm{action point} spent to cast this spell.
\hit Up to two magical conditions of your choice are removed from the chosen allies and applied to the struck creature.
\crit As above, except that you can transfer any number of magical conditions in this way.
\end{apability}
\vspace{0.25em}



\lowercase{\hypertarget{spell:Antimagic Field}{}}\label{spell:Antimagic Field}
\begin{apability}[\nth{7}]{\hypertarget{spell:Antimagic Field}{Antimagic Field}}[\glossterm{Mystic}, \glossterm{Sustain} (minor)]
All other magical abilities and objects are \glossterm{suppressed} within a \areamed radius emanation from you.
% How much of this is redundant with suppression?
Creatures within the area cannot activate, sustain, or dismiss magical abilities.
% TODO: wording
This does not affect aspects of creatures that cannot be suppressed, such as the knowledge of abilities.
You cannot exclude yourself from this emanation.
However, this spell does not prevent you from sustaining or dismissing this spell.
\end{apability}
\vspace{0.25em}


\newpage
\begin{spellsection}{Verdamancy}

\begin{spellheader}
\spelldesc{Animate and manipulate plants.}
\end{spellheader}


\parhead{Schools} Transmutation

\parhead{Mystic Sphere Lists} Nature

\subsubsection{Cantrips}


\begin{freeability}{Minor Embedded Growth}
You throw a seed that embeds itself in a foe and grows painfully.
Make an attack vs. Armor at a creature within \rngclose range.
\hit As a \glossterm{condition}, the target takes \glossterm{standard damage} \minus1d at the end of each \glossterm{action phase}.
This condition can be removed if the target or a creature that can reach the target makes a \glossterm{DR} 5 Heal check as a standard action to remove the seed.
\end{freeability}

\end{spellsection}


\subsubsection{Spells}


\lowercase{\hypertarget{spell:Embedded Growth}{}}\label{spell:Embedded Growth}
\begin{apability}[\nth{1}]{\hypertarget{spell:Embedded Growth}{Embedded Growth}}
You throw a seed that embeds itself in a foe and grows painfully.
Make an attack vs. Armor at a creature within \rngclose range.
\miss You regain the \glossterm{action point} spent to cast this spell.
\hit As a \glossterm{condition}, the target takes \glossterm{standard damage} \plus1d at the end of each \glossterm{action phase}.
This condition can be removed if the target or a creature that can reach the target makes a \glossterm{DR} 5 Heal check as a standard action to remove the seed.
\end{apability}
\vspace{0.25em}



\lowercase{\hypertarget{spell:Entangle}{}}\label{spell:Entangle}
\begin{apability}[\nth{1}]{\hypertarget{spell:Entangle}{Entangle}}
You cause plants to grow and trap a foe.
Make an attack vs. Reflex against a Large or smaller creature within \rngmed range.
The target must be within 5 feet of earth or plants.
You gain a \plus2 bonus to \glossterm{accuracy} with this attack if the target is in standing in \glossterm{undergrowth}.
\miss You regain the \glossterm{action point} spent to cast this spell.
\hit The target is \glossterm{immobilized} as a \glossterm{condition}.
This condition can be removed if the target or a creature that can reach the target makes a \glossterm{DR} 5 Strength check as a standard action to pull the target free of the plants.
\end{apability}
\vspace{0.25em}



\lowercase{\hypertarget{spell:Blight}{}}\label{spell:Blight}
\begin{apability}[\nth{2}]{\hypertarget{spell:Blight}{Blight}}[\glossterm{Life}]
Make an attack vs. Fortitude against a living creature or plant within \rngmed range.
\hit The target takes life \glossterm{standard damage} \plus1d.
This damage is doubled if the target is a plant, including plant creatures.
\end{apability}
\vspace{0.25em}



\lowercase{\hypertarget{spell:Fire Seed}{}}\label{spell:Fire Seed}
\begin{attuneability}[\nth{2}]{\hypertarget{spell:Fire Seed}{Fire Seed}}[\glossterm{Attune} (self), \glossterm{Fire}]
% Does "seed structure" make sense?
You transform an unattended acorn or similar seed structure into a small bomb.
As a standard action, you or another creature can throw the acorn with a \glossterm{range increment} of 20 feet.
On impact, the acorn detonates, and you make an attack vs. Armor against all creatures within a \areasmall radius of the struck creature or object.
\hit Each target takes fire \glossterm{standard damage}.
\end{attuneability}
\vspace{0.25em}



\lowercase{\hypertarget{spell:Plant Growth}{}}\label{spell:Plant Growth}
\begin{attuneability}[\nth{2}]{\hypertarget{spell:Plant Growth}{Plant Growth}}[\glossterm{Attune} (self)]
Choose a \arealarge radius within \rnglong range.
In addition, choose whether you want plants within the area to grow or diminish.

If you choose for plants to grow, all arable earth within the area becomes \glossterm{light undergrowth}.
Light undergrowth within the area is increased in density to \glossterm{heavy undergrowth}.
If you choose for plants to diminish, all \glossterm{heavy undergrowth} in the area is reduced to \glossterm{light undergrowth}, and all \glossterm{light undergrowth} is removed.

When this spell's duration ends, the plants return to their natural size.
\end{attuneability}
\vspace{0.25em}



\lowercase{\hypertarget{spell:Wall of Thorns}{}}\label{spell:Wall of Thorns}
\begin{attuneability}[\nth{2}]{\hypertarget{spell:Wall of Thorns}{Wall of Thorns}}[\glossterm{Attune} (self)]
You create a wall of thorns in a 10 ft.\ high, \areamed line within \rngmed range.
The base of at least half of the wall must be in arable earth.
The wall is four inches thick, but permeable.
It provides \glossterm{passive cover} to attacks made through the wall.
Creatures can pass through the wall, though it costs five extra feet of movement to move through the wall.
When a creature moves through the wall, make an attack vs. Armor against it.
You can only make an attack in this way against a given creature once per \glossterm{phase}.
\hit The target takes piercing \glossterm{standard damage} \minus1d.

Each five-foot square of wall has hit points equal to three times your \glossterm{power}, and all of its defenses are 0.
It is \glossterm{vulnerable} to fire damage.
\end{attuneability}
\vspace{0.25em}



\lowercase{\hypertarget{spell:Greater Fire Seed}{}}\label{spell:Greater Fire Seed}
\begin{attuneability}[\nth{4}]{\hypertarget{spell:Greater Fire Seed}{Greater Fire Seed}}[\glossterm{Attune} (self), \glossterm{Fire}]
This spell functions like the \spell{fire seed} spell, except that you can transform up to four bombs.
In addition, the detonation affects a \areamed radius instead of an \areasmall radius.
\end{attuneability}
\vspace{0.25em}



\lowercase{\hypertarget{spell:Greater Plant Growth}{}}\label{spell:Greater Plant Growth}
\begin{attuneability}[\nth{4}]{\hypertarget{spell:Greater Plant Growth}{Greater Plant Growth}}[\glossterm{Attune} (self)]
This spell functions like the \spell{plant growth} spell, except that its effects are intensified.
If you choose for plants to grow, all arable earth within the area becomes \glossterm{heavy undergrowth}.
If you choose for plants to diminish, all \glossterm{undergrowth} within the area is removed.
\end{attuneability}
\vspace{0.25em}



\lowercase{\hypertarget{spell:Greater Wall of Thorns}{}}\label{spell:Greater Wall of Thorns}
\begin{attuneability}[\nth{4}]{\hypertarget{spell:Greater Wall of Thorns}{Greater Wall of Thorns}}[\glossterm{Attune} (self)]
This spell functions like the \spell{wall of thorns} spell, except that the wall is an \arealarge shapeable line.
\end{attuneability}
\vspace{0.25em}



\subsubsection{Rituals}


\lowercase{\hypertarget{spell:Fertility}{}}\label{spell:Fertility}
\begin{apability}[\nth{2}]{\hypertarget{spell:Fertility}{Fertility}}
This ritual creates an area of bountiful growth in a one mile radius zone from your location.
Normal plants within the area become twice as productive as normal for the next year.
This ritual does not stack with itself.
If the \ritual{infertility} ritual is also applied to the same area, the most recently performed ritual takes precedence.

This ritual takes 24 hours to perform, and requires 8 action points from its participants.
\end{apability}
\vspace{0.25em}



\lowercase{\hypertarget{spell:Infertility}{}}\label{spell:Infertility}
\begin{apability}[\nth{2}]{\hypertarget{spell:Infertility}{Infertility}}
This ritual creates an area of death and decay in a one mile radius zone from your location.
Normal plants within the area become half as productive as normal for the next year.
This ritual does not stack with itself.
If the \ritual{fertility} ritual is also applied to the same area, the most recently performed ritual takes precedence.

This ritual takes 24 hours to perform, and requires 8 action points from its participants.
\end{apability}
\vspace{0.25em}



\lowercase{\hypertarget{spell:Lifeweb Transit}{}}\label{spell:Lifeweb Transit}
\begin{apability}[\nth{4}]{\hypertarget{spell:Lifeweb Transit}{Lifeweb Transit}}[\glossterm{Teleportation}]
Choose up to five willing, Medium or smaller ritual participants and a living plant that all ritual participants touch during the ritual.
The plant must be at least one size category larger than the largest target.
In addition, choose a destination up to 100 miles away from you on your current plane.
By walking through the chosen plant, each target is teleported to the closest plant to the destination that is at least one size category larger than the largest target.

You must specify the destination with a precise mental image of its appearance.
The image does not have to be perfect, but it must unambiguously identify the destination.
If you specify its appearance incorrectly, or if the area has changed its appearance, the destination may be a different area than you intended.
The new destination will be one that more closely resembles your mental image.
If no such area exists, the ritual simply fails.
% TODO: does this need more clarity about what teleportation works?

This ritual takes 24 hours to perform and requires 32 action points from its ritual participants.
It is from from the Conjuration school in addition to the Transmutation school.
\end{apability}
\vspace{0.25em}


\newpage
\begin{spellsection}{Vital Surge}

\begin{spellheader}
\spelldesc{Alter life energy to cure or inflict wounds.}
\end{spellheader}


\parhead{Schools} Vivimancy

\parhead{Mystic Sphere Lists} Divine, Nature

\subsubsection{Cantrips}


\begin{freeability}{Cure Minor Wounds}[\glossterm{Life}]
Choose a willing creature within \rngmed range.
The target heals hit points equal to \glossterm{standard damage}.
\end{freeability}


\begin{freeability}{Inflict Minor Wounds}[\glossterm{Life}]
Make an attack vs. Fortitude against a creature within \rngmed range.
\hit The target takes life damage equal to \glossterm{standard damage}.
\end{freeability}

\end{spellsection}


\subsubsection{Spells}


\lowercase{\hypertarget{spell:Cure Wounds}{}}\label{spell:Cure Wounds}
\begin{apability}[\nth{1}]{\hypertarget{spell:Cure Wounds}{Cure Wounds}}[\glossterm{Life}]
Choose a willing creature within \rngmed range.
The target heals hit points equal to \glossterm{standard damage} \plus2d.
\end{apability}
\vspace{0.25em}



\lowercase{\hypertarget{spell:Inflict Wounds}{}}\label{spell:Inflict Wounds}
\begin{apability}[\nth{1}]{\hypertarget{spell:Inflict Wounds}{Inflict Wounds}}[\glossterm{Life}]
Make an attack vs. Fortitude against a creature within \rngmed range.
\hit The target takes life damage equal to \glossterm{standard damage} \plus2d.
\end{apability}
\vspace{0.25em}



\lowercase{\hypertarget{spell:Death Knell}{}}\label{spell:Death Knell}
\begin{apability}[\nth{2}]{\hypertarget{spell:Death Knell}{Death Knell}}[\glossterm{Life}]
This spell functions like the \spell{inflict wounds} spell, except that a struck target suffers a death knell as a \glossterm{condition}.
At the end of each round, if the target has 0 hit points, it immediately dies.

% TODO: wording
If the target dies while the condition is active, you heal hit points equal to twice your \glossterm{power}.
\end{apability}
\vspace{0.25em}



\lowercase{\hypertarget{spell:Vital Persistence}{}}\label{spell:Vital Persistence}
\begin{attuneability}[\nth{2}]{\hypertarget{spell:Vital Persistence}{Vital Persistence}}[\glossterm{Attune} (target), \glossterm{Life}]
Choose a willing creature within \rngclose range.
The target reduces its \glossterm{vital damage penalties} by an amount equal to your \glossterm{power}.
\end{attuneability}
\vspace{0.25em}



\lowercase{\hypertarget{spell:Circle of Death}{}}\label{spell:Circle of Death}
\begin{attuneability}[\nth{3}]{\hypertarget{spell:Circle of Death}{Circle of Death}}[\glossterm{Attune} (self), \glossterm{Life}]
You are surrounded by an aura of death in a \areamed radius emanation from you.
When this spell resolves, and the end of each \glossterm{action phase} in subsequent rounds, make an attack vs. Fortitude against all enemies in the area.
\hit Each target takes life \glossterm{standard damage} \minus2d.
\end{attuneability}
\vspace{0.25em}



\lowercase{\hypertarget{spell:Circle of Healing}{}}\label{spell:Circle of Healing}
\begin{attuneability}[\nth{3}]{\hypertarget{spell:Circle of Healing}{Circle of Healing}}[\glossterm{Attune} (self), \glossterm{Life}]
You are surrounded by an aura of healing in a \areamed radius emanation from you.
When this spell resolves, and the end of each \glossterm{action phase} in subsequent rounds, all allies in the area heal hit points equal to half your \glossterm{power}.
\end{attuneability}
\vspace{0.25em}



\lowercase{\hypertarget{spell:Drain Life}{}}\label{spell:Drain Life}
\begin{apability}[\nth{3}]{\hypertarget{spell:Drain Life}{Drain Life}}[\glossterm{Life}]
This spell functions like the \spell{inflict wounds} spell, except that you gain a \plus1d bonus to damage.
In addition, you heal hit points equal to your \glossterm{power} if you deal damage.
\end{apability}
\vspace{0.25em}



\lowercase{\hypertarget{spell:Greater Cure Wounds}{}}\label{spell:Greater Cure Wounds}
\begin{apability}[\nth{3}]{\hypertarget{spell:Greater Cure Wounds}{Greater Cure Wounds}}[\glossterm{Life}]
This spell functions like the \spell{cure wounds} spell, except that you gain a \plus1d bonus to healing.
In addition, for every 5 points of healing you provide, you can instead heal one point of \glossterm{vital damage}.
\end{apability}
\vspace{0.25em}



\lowercase{\hypertarget{spell:Greater Inflict Wounds}{}}\label{spell:Greater Inflict Wounds}
\begin{apability}[\nth{3}]{\hypertarget{spell:Greater Inflict Wounds}{Greater Inflict Wounds}}[\glossterm{Life}]
This spell functions like the \spell{inflict wounds} spell, except that you gain a \plus1 bonus to \glossterm{accuracy}.
In addition, a struck target takes a \minus2 penalty to Fortitude defense as a \glossterm{condition}.
\end{apability}
\vspace{0.25em}



\lowercase{\hypertarget{spell:Greater Vital Persistence}{}}\label{spell:Greater Vital Persistence}
\begin{attuneability}[\nth{4}]{\hypertarget{spell:Greater Vital Persistence}{Greater Vital Persistence}}[\glossterm{Attune} (target), \glossterm{Life}]
This spell functions like the \spell{vital persistence} spell, except that the penalty reduction increases to be equal to twice your \glossterm{power}.
\end{attuneability}
\vspace{0.25em}



\lowercase{\hypertarget{spell:Life Exchange}{}}\label{spell:Life Exchange}
\begin{apability}[\nth{4}]{\hypertarget{spell:Life Exchange}{Life Exchange}}[\glossterm{Life}]
Choose a willing ally within \rngmed range.
Make an attack vs. Fortitude against a creature within \rngmed range.
\hit The target takes life damage equal to \glossterm{standard damage} \plus3d.
In addition, the chosen ally heals hit points equal to the damage dealt in this way.
\crit This spell does not deal additional damage on a critical hit.
\end{apability}
\vspace{0.25em}



\lowercase{\hypertarget{spell:Finger of Death}{}}\label{spell:Finger of Death}
\begin{apability}[\nth{5}]{\hypertarget{spell:Finger of Death}{Finger of Death}}[\glossterm{Life}]
Make an attack vs. Fortitude against a living creature within \rngclose range.
\hit The target takes life \glossterm{standard damage} \plus4d.
\crit The target immediately dies.
\end{apability}
\vspace{0.25em}



\lowercase{\hypertarget{spell:Greater Drain Life}{}}\label{spell:Greater Drain Life}
\begin{apability}[\nth{5}]{\hypertarget{spell:Greater Drain Life}{Greater Drain Life}}[\glossterm{Life}]
This spell functions like the \spell{inflict wounds} spell, except that gain a \plus2d bonus to damage.
In addition, you heal hit points equal to twice your \glossterm{power} if you deal damage.
\end{apability}
\vspace{0.25em}



\lowercase{\hypertarget{spell:Heal}{}}\label{spell:Heal}
\begin{apability}[\nth{5}]{\hypertarget{spell:Heal}{Heal}}[\glossterm{Life}]
This spell functions like the \spell{cure wounds} spell, except that you gain a \plus2d bonus to healing.
In addition, it heals \glossterm{vital damage} as easily as it heals hit points.
\end{apability}
\vspace{0.25em}



\lowercase{\hypertarget{spell:Supreme Inflict Wounds}{}}\label{spell:Supreme Inflict Wounds}
\begin{apability}[\nth{5}]{\hypertarget{spell:Supreme Inflict Wounds}{Supreme Inflict Wounds}}[\glossterm{Life}]
This spell functions like the \spell{inflict wounds} spell, except that you gain a \plus2 bonus to \glossterm{accuracy}.
In addition, a struck target takes a \minus4 penalty to Fortitude defense as a \glossterm{condition}.
\end{apability}
\vspace{0.25em}



\subsubsection{Rituals}


\lowercase{\hypertarget{spell:Purge Curse}{}}\label{spell:Purge Curse}
\begin{apability}[\nth{2}]{\hypertarget{spell:Purge Curse}{Purge Curse}}[\glossterm{Mystic}]
Choose a willing creature within \rngclose range.
All curses affecting the target are removed.
This ritual cannot remove a curse that is part of the effect of an item the target has equipped.
However, it can allow the target to remove any cursed items it has equipped.

This ritual takes 24 hours to perform, and requires 8 action points from its participants.
\end{apability}
\vspace{0.25em}



\lowercase{\hypertarget{spell:Remove Disease}{}}\label{spell:Remove Disease}
\begin{apability}[\nth{2}]{\hypertarget{spell:Remove Disease}{Remove Disease}}[\glossterm{Life}]
Choose a willing creature within \rngmed range.
All diseases affecting the target are removed.
\end{apability}
\vspace{0.25em}



\lowercase{\hypertarget{spell:Restore Senses}{}}\label{spell:Restore Senses}
\begin{apability}[\nth{2}]{\hypertarget{spell:Restore Senses}{Restore Senses}}[\glossterm{Life}]
Choose a willing creature within \rngmed range.
One of the target's physical senses, such as sight or hearing, is restored to full capacity.
This can heal both magical and mundane effects, but it cannot completely replace missing body parts required for a sense to function (such as missing eyes).
\end{apability}
\vspace{0.25em}



\lowercase{\hypertarget{spell:Restoration}{}}\label{spell:Restoration}
\begin{apability}[\nth{3}]{\hypertarget{spell:Restoration}{Restoration}}[\glossterm{Flesh}]
Choose a willing creature within \rngclose range.
All of the target's hit points, \glossterm{subdual damage}, and \glossterm{vital damage} are healed.
In addition, any of the target's severed body parts or missing organs grow back by the end of the next round.

This ritual takes 24 hours to perform, and requires 18 action points from its participants.
\end{apability}
\vspace{0.25em}



\lowercase{\hypertarget{spell:Resurrection}{}}\label{spell:Resurrection}
\begin{apability}[\nth{3}]{\hypertarget{spell:Resurrection}{Resurrection}}[\glossterm{Flesh}, \glossterm{Life}]
Choose one intact humanoid corpse within \rngclose range.
The target returns to life.
It must not have died due to old age.

The creature has 0 hit points when it returns to life.
It is cured of all \glossterm{vital damage} and other negative effects, but the body's shape is unchanged.
Any missing or irreparably damaged limbs or organs remain missing or damaged.
The creature may therefore die shortly after being resurrected if its body is excessively damaged.

Coming back from the dead is an ordeal.
All of the creature's action points and other daily abilities are expended when it returns to life.
In addition, its maximum action points are reduced by 1.
This penalty lasts for thirty days, or until the creature gains a level.
If this would reduce a creature's maximum action points below 0, the creature cannot be resurrected.

This ritual takes 24 hours to perform, and requires 18 action points from its participants.
It is from the Conjuration school in addition to the Vivimancy school.
In addition, it can only be learned through the divine \glossterm{magic source}.
\end{apability}
\vspace{0.25em}



\lowercase{\hypertarget{spell:Reincarnation}{}}\label{spell:Reincarnation}
\begin{apability}[\nth{4}]{\hypertarget{spell:Reincarnation}{Reincarnation}}[\glossterm{Creation}, \glossterm{Flesh}, \glossterm{Life}]
Choose one Diminuitive or larger piece of a humanoid corpse.
The target must have been part of the original creature's body at the time of death.
The creature the target corpse belongs to returns to life in a new body.
It must not have died due to old age.

This ritual creates an entirely new body for the creature's soul to inhabit from the natural elements at hand.
During the ritual, the body ages to match the age of the original creature at the time it died.
The creature has 0 hit points when it returns to life.

A reincarnated creature is identical to the original creature in all respects, except for its species.
The creature's species is replaced with a random species from \tref{Humanoid Reincarnations}.
Its appearance changes as necessary to match its new species, though it retains the general shape and distinguishing features of its original appearance.
The creature loses all attribute modifiers and abilities from its old species, and gains those of its new species.
If its species bonus feat is invalid for its new species, it must choose a new species bonus feat.
However, its languages are unchanged.

Coming back from the dead is an ordeal.
All of the creature's action points and other daily abilities are expended when it returns to life.
In addition, its maximum action points are reduced by 1.
This penalty lasts for thirty days, or until the creature gains a level.
If this would reduce a creature's maximum action points below 0, the creature cannot be resurrected.

This ritual takes 24 hours to perform, and requires 32 action points from its participants.
It is from the Conjuration school in addition to the Vivimancy school.
In addition, it can only be learned through the nature \glossterm{magic source}.
\end{apability}
\vspace{0.25em}
\begin{dtable}
\lcaption{Humanoid Reincarnations}
\begin{dtabularx}{\columnwidth}{l X}
d\% & Incarnation \\
\bottomrule
01-\minus13 & Dwarf \\
14-\minus26 & Elf \\
27-\minus40 & Gnome \\
41-\minus52 & Half-elf \\
53-\minus62 & Half-orc \\
63-\minus74 & Halfling \\
75-\minus100 & Human \\
\end{dtabularx}
\end{dtable}


\lowercase{\hypertarget{spell:Complete Resurrection}{}}\label{spell:Complete Resurrection}
\begin{apability}[\nth{5}]{\hypertarget{spell:Complete Resurrection}{Complete Resurrection}}[\glossterm{Creation}, \glossterm{Flesh}, \glossterm{Life}]
This ritual functions like the \ritual{resurrection} ritual, except that it does not have to target a fully intact corpse.
Instead, it targets a Diminuitive or larger piece of a humanoid corpse.
The target must have been part of the original creature's body at the time of death.
The resurrected creature's body is fully restored to its healthy state before dying, including regenerating all missing or damaged body parts.

This ritual takes 24 hours to perform, and requires 50 action points from its participants.
It is from the Conjuration school in addition to the Vivimancy school.
In addition, it can only be learned through the divine \glossterm{magic source}.
\end{apability}
\vspace{0.25em}



\lowercase{\hypertarget{spell:Fated Reincarnation}{}}\label{spell:Fated Reincarnation}
\begin{apability}[\nth{5}]{\hypertarget{spell:Fated Reincarnation}{Fated Reincarnation}}[\glossterm{Creation}, \glossterm{Flesh}, \glossterm{Life}]
This ritual functions like the \ritual{reincarnation} ritual, except that the target is reincarnated as its original species instead of as a random species.

This ritual takes 24 hours to perform, and requires 50 action points from its participants.
It is from the Conjuration school in addition to the Vivimancy school.
In addition, it can only be learned through the nature \glossterm{magic source}.
\end{apability}
\vspace{0.25em}



\lowercase{\hypertarget{spell:Soul Bind}{}}\label{spell:Soul Bind}
\begin{apability}[\nth{5}]{\hypertarget{spell:Soul Bind}{Soul Bind}}[\glossterm{Life}]
Choose one intact corpse within \rngclose range.
% Is this clear enough that you can't use the same gem for this ritual twice?
In addition, choose a nonmagical gem you hold that is worth at least 1,000 gp.
A fragment of the soul of the creature that the target corpse belongs to is imprisoned in the chosen gem.
This does not remove the creature from its intended afterlife.
However, it prevents the creature from being resurrected, and prevents the corpse from being used to create undead creatures, as long as the gem is intact.
A creature holding the gem may still resurrect or reanimate the creature.
If the gem is shattered, the fragment of the creature's soul returns to its body.

This ritual takes one hour to perform.
\end{apability}
\vspace{0.25em}



\lowercase{\hypertarget{spell:True Resurrection}{}}\label{spell:True Resurrection}
\begin{apability}[\nth{7}]{\hypertarget{spell:True Resurrection}{True Resurrection}}[\glossterm{Creation}, \glossterm{Flesh}, \glossterm{Life}]
This ritual functions like the \ritual{resurrection} ritual, except that it does not require any piece of the corpse.
Instead, you must explicitly and unambiguously specify the identity of the creature being resurrected.
The resurrected creature's body is fully restored to its healthy state before dying, including regenerating all missing or damaged body parts.

This ritual takes 24 hours to perform, and requires 98 action points from its participants.
It is from the Conjuration school in addition to the Vivimancy school.
In addition, it can only be learned through the divine \glossterm{magic source}.
\end{apability}
\vspace{0.25em}


\newpage
\begin{spellsection}{Weaponcraft}

\begin{spellheader}
\spelldesc{Create and manipulate weapons to attack foes.}
\end{spellheader}


\parhead{Schools} Conjuration, Transmutation

\parhead{Mystic Sphere Lists} Arcane, Divine, Pact

\subsubsection{Cantrips}


\begin{freeability}{Fire Projectile}[\glossterm{Manifestation}]
Make an attack vs. Armor against one creature or object within \rngmed range.
\hit The target takes piercing \glossterm{standard damage}.
\end{freeability}


\begin{freeability}{Personal Weapon}[\glossterm{Manifestation}]
Choose a type of weapon that you are proficient with.
You create a normal item of that type in your hand.
If the item stops touching you, it disappears, and this effect ends.

If you create a projectile weapon, you can fire it without ammunition by creating projectiles as you fire.
The projectiles disappear after the attack is complete.

% Strange duration for a cantrip
This spell lasts until you use it again, or until you \glossterm{dismiss} it as a \glossterm{free action}.
\end{freeability}

\end{spellsection}


\subsubsection{Spells}


\lowercase{\hypertarget{spell:Blade Barrier}{}}\label{spell:Blade Barrier}
\begin{apability}[\nth{1}]{\hypertarget{spell:Blade Barrier}{Blade Barrier}}[\glossterm{Sustain} (minor)]
A wall of whirling blades appears within \rngmed range.
The wall takes the form of a 10 ft.\ high, \arealarge line.
The wall provides \glossterm{active cover} (20\% miss chance) against attacks made through it.
Attacks that miss in this way harmlessly strike the wall.
When a creature or object passes through the wall, make an attack vs. Armor against it.
\hit The target takes slashing \glossterm{standard damage}.
\end{apability}
\vspace{0.25em}



\lowercase{\hypertarget{spell:Mystic Bow}{}}\label{spell:Mystic Bow}
\begin{apability}[\nth{1}]{\hypertarget{spell:Mystic Bow}{Mystic Bow}}[\glossterm{Manifestation}]
Make an attack vs. Armor against one creature or object within \rngmed range.
\hit The target takes piercing \glossterm{standard damage} \plus2d.
\end{apability}
\vspace{0.25em}



\lowercase{\hypertarget{spell:Summon Weapon}{}}\label{spell:Summon Weapon}
\begin{apability}[\nth{1}]{\hypertarget{spell:Summon Weapon}{Summon Weapon}}[\glossterm{Manifestation}, \glossterm{Sustain} (minor)]
A melee weapon that you are proficient with appears in an unoccupied square within \rngmed range.
The weapon floats about three feet off the ground, and is sized appropriately for a creature of your size.
The specific weapon you choose affects the type of damage it deals.
Regardless of the weapon chosen, it has hit points equal to twice your \glossterm{power}.
All of its defenses are equal to 3 \add your level, and it has a 30 foot fly speed with good maneuverability, though it cannot travel farther than five feet above the ground.

Each round, the weapon automatically moves towards the creature closest to it during the \glossterm{movement phase}.
During the \glossterm{action phase}, it makes a melee \glossterm{strike} against a random creature adjacent to it.
Its accuracy is equal to your \glossterm{accuracy}.
If it hits, it deals \glossterm{standard damage}.
\end{apability}
\vspace{0.25em}



\lowercase{\hypertarget{spell:Aerial Weapon}{}}\label{spell:Aerial Weapon}
\begin{apability}[\nth{2}]{\hypertarget{spell:Aerial Weapon}{Aerial Weapon}}[\glossterm{Manifestation}, \glossterm{Sustain} (minor)]
This spell functions like the \spell{summon weapon} spell, except that the weapon's height above the ground is not limited.
This allows the weapon to fly up to fight airborne foes.
\end{apability}
\vspace{0.25em}



\lowercase{\hypertarget{spell:Blade Perimeter}{}}\label{spell:Blade Perimeter}
\begin{apability}[\nth{2}]{\hypertarget{spell:Blade Perimeter}{Blade Perimeter}}[\glossterm{Sustain} (minor)]
This spell functions like the \spell{blade barrier} spell, except that the wall is an 20 ft.\ high, \areamed radius circle.
\end{apability}
\vspace{0.25em}



\lowercase{\hypertarget{spell:Create Ballista}{}}\label{spell:Create Ballista}
\begin{apability}[\nth{2}]{\hypertarget{spell:Create Ballista}{Create Ballista}}[\glossterm{Manifestation}, \glossterm{Sustain} (minor)]
This spell functions like the \spell{summon weapon} spell, except that it creates a fully functional Large ballista instead of a weapon of your choice.
The ballista functions like any other weapon, with the following exceptions.

It cannot move, and makes ranged \glossterm{strikes} instead of melee strikes.
Its attacks have a maximum range of 100 feet.
Its attacks deal piercing damage, and its hit points are equal to three times your \glossterm{power}.
In addition, the ballista attacks the creature farthest from it, instead of the creature closest to it.
\end{apability}
\vspace{0.25em}



\lowercase{\hypertarget{spell:Blade Barrier, Dual}{}}\label{spell:Blade Barrier, Dual}
\begin{apability}[\nth{3}]{\hypertarget{spell:Blade Barrier, Dual}{Blade Barrier, Dual}}[\glossterm{Sustain} (minor)]
This spell functions like the \spell{blade barrier} spell, except that the area must be a line.
In addition, the spell creates two parallel walls of the same length, five feet apart.
\end{apability}
\vspace{0.25em}



\lowercase{\hypertarget{spell:Contracting Blade Perimeter}{}}\label{spell:Contracting Blade Perimeter}
\begin{apability}[\nth{3}]{\hypertarget{spell:Contracting Blade Perimeter}{Contracting Blade Perimeter}}[\glossterm{Sustain} (minor)]
This spell functions like the \spell{blade perimeter} spell, except that the wall's radius shrinks by 5 feet at the end of every \glossterm{action phase}, dealing damage to everything it moves through.
% Clarify interaction with solid obstacles that block contraction?
\end{apability}
\vspace{0.25em}



\lowercase{\hypertarget{spell:Giant Blade}{}}\label{spell:Giant Blade}
\begin{apability}[\nth{3}]{\hypertarget{spell:Giant Blade}{Giant Blade}}[\glossterm{Manifestation}, \glossterm{Sustain} (minor)]
This spell functions like the \spell{summon weapon} spell, except that the weapon takes the form of a Large greatsword.
The weapon's attacks hit everything in a \areasmall cone from it.
It aims the cone to hit as many creatures as possible.
\end{apability}
\vspace{0.25em}



\lowercase{\hypertarget{spell:Create Ballista, Dual Track}{}}\label{spell:Create Ballista, Dual Track}
\begin{apability}[\nth{4}]{\hypertarget{spell:Create Ballista, Dual Track}{Create Ballista, Dual Track}}[\glossterm{Manifestation}, \glossterm{Sustain} (minor)]
This spell functions like the \spell{create ballista} spell, except that the ballista is created with two separate bolt tracks.
This allows it to fire at two different targets in the same round when you command it to fire.
It cannot fire at the same target twice.
Each round, it attacks the two creatures farthest from it.
\end{apability}
\vspace{0.25em}



\lowercase{\hypertarget{spell:Titan Blade}{}}\label{spell:Titan Blade}
\begin{apability}[\nth{6}]{\hypertarget{spell:Titan Blade}{Titan Blade}}[\glossterm{Manifestation}, \glossterm{Sustain} (minor)]
This spell functions like the \spell{summon weapon} spell, except that the weapon takes the form of a Gargantuan greatsword.
The weapon's attacks hit everything in a \areamed cone from it.
It aims the cone to hit as many creatures as possible.
\end{apability}
\vspace{0.25em}



\lowercase{\hypertarget{spell:Paired Weapons}{}}\label{spell:Paired Weapons}
\begin{apability}[\nth{7}]{\hypertarget{spell:Paired Weapons}{Paired Weapons}}[\glossterm{Manifestation}, \glossterm{Sustain} (minor)]
This spell functions like the \spell{summon weapon} spell, except that you summon two weapons instead of one.
Each weapon attacks independently.
\end{apability}
\vspace{0.25em}


