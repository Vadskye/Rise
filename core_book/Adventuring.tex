\chapter{Adventuring}

\section{Encumbrance}\label{Encumbrance}
Encumbrance rules determine how much a character's armor and equipment hinder his movement. There are several reasons a character can be encumbered. Encumbered characters may be unable to use certain abilities which require free motion.

\subsection{Encumbrance by Armor}

A character's armor affects his or her Dexterity, \glossterm{encumbrance penalty}, and speed. A character wearing medium or heavy armor is encumbered. Unless your character is weak or carrying a lot of gear, that's all you need to know. The extra gear your character carries won't slow him or her down any more than the armor already does.

If your character is weak or carrying a lot of gear, however, then you'll need to calculate encumbrance by weight. Doing so is most important when your character is trying to carry some heavy object.

\begin{dtable}
    \lcaption{Weight Limits}
    \setlength{\tabcolsep}{4pt}
    \begin{dtabularx}{\columnwidth}{X l l l l}
        \tb{Strength} & \tb{Unencumbered} & \tb{Maximum} & \tb{Overloaded} & \tb{Push/Drag} \\
        \bottomrule
        -9 & 6 lb. & 12 lb. & 18 lb. & 60 lb. \\
        -8 & 7     & 14     & 21     & 70     \\
        -7 & 9     & 18     & 27     & 90     \\
        -6 & 12    & 24     & 36     & 120    \\
        -5 & 15    & 30     & 45     & 150    \\
        -4 & 20    & 40     & 60     & 200    \\
        -3 & 25    & 50     & 75     & 250    \\
        -2 & 30    & 60     & 90     & 300    \\
        -1 & 40    & 80     & 120    & 400    \\
        0  & 50    & 100    & 150    & 500    \\
        1  & 60    & 120    & 180    & 600    \\
        2  & 80    & 160    & 240    & 800    \\
        3  & 100   & 200    & 300    & 1,000  \\
        4  & 120   & 240    & 360    & 1,200  \\
        5  & 160   & 320    & 480    & 1,600  \\
        6  & 200   & 400    & 600    & 2,000  \\
        7  & 250   & 500    & 750    & 2,500  \\
        8  & 320   & 640    & 960    & 3,200  \\
        9  & 400   & 800    & 1,200  & 4,000  \\
        10 & 500   & 1,000  & 1,500  & 5,000  \\
        11 & 630   & 1,260  & 1,890  & 6,300  \\
        12 & 800   & 1,600  & 2,400  & 8,000  \\
        13 & 1,000 & 2,000  & 3,000  & 10,000 \\
        14 & 1,300 & 2,600  & 3,900  & 13,000 \\
        15 & 1,600 & 3,200  & 4,800  & 16,000 \\
        16 & 2,000 & 4,000  & 6,000  & 20,000 \\
        17 & 2,500 & 5,000  & 7,500  & 25,000 \\
        18 & 3,200 & 6,400  & 9,600  & 32,000 \\
        19 & 4,000 & 8,000  & 12,000 & 40,000 \\
        20 & 5,100 & 10,200 & 15,300 & 51,000 \\
        21\plus\fn{1} & \tdash & \tdash & \tdash & \tdash \\
    \end{dtabularx}
    1 To calculate weight limits for a creature with epic Strength, double its weight limit every 3 Strength.
\end{dtable}

\subsection{Encumbrance by Weight}
A creature's Strength determines how much it can carry, as shown on \trefnp{Weight Limits}. A creature can carry up to its unencumbered weight limit without any penalty. If it carries more than that, but less than its maximum weight limit, it is encumbered. A creature encumbered by weight halves its Dexterity (if positive), takes a \minus4 encumbrance penalty, and moves at two-thirds speed (as if it were in heavy armor). This encumbrance penalty does not stack with the encumbrance penalty from any armor the creature is wearing; only apply the higher of the two.

\parhead{Lifting and Dragging} A character can lift as much as his or her maximum weight limit over his or her head.

A character can lift as much as 1-1/2 his or her maximum weight limit off the ground (the sum of his unencumbered and maximum weight limits). While overloaded in this way, the character takes a \minus10 penalty to physical accuracy and physical checks, and can only move by spending a full-round action to move 5 feet.

A character can generally push or drag along the ground as much as five times his or her maximum load. Favorable conditions can double these numbers, and bad circumstances can reduce them to one-half or less.

\parhead{Bigger and Smaller Creatures} The figures on \trefnp{Weight Limits} are for Medium bipedal creatures. A larger bipedal creature can carry more weight depending on its size category, as follows: Large \mult2, Huge \mult4, Gargantuan \mult8, Colossal \mult16. A smaller creature can carry less weight depending on its size category, as follows: Small \mult3/4, Tiny \mult1/2, Diminutive \mult1/4, Fine \mult1/8.

Quadrupeds can carry heavier loads than characters can. Instead of the multipliers given above, multiply the value corresponding to the creature's Strength score from Table 9-1 by the appropriate modifier, as follows: Fine \mult1/4, Diminutive \mult1/2, Tiny \mult3/4, Small \mult1, Medium \mult1-1/2, Large \mult3, Huge \mult6, Gargantuan \mult12, Colossal \mult24.

\parhead{Tremendous Strength} For Strength scores not shown on \trefnp{Weight Limits}, subtract 3 from its Strength until you find a Strength value shown on the chart. For each time you subtracted in this way, double the weight limits listed on the chart.

\section{Movement}

\begin{dtable}
\lcaption{Movement and Distance}
\begin{dtabularx}{\columnwidth}{>{\lcol}X c c c c}
 & \multicolumn{4}{c}{\tdash\tdash\tdash Speed \tdash\tdash\tdash} \\
\bottomrule
 & 15 feet & 20 feet & 30 feet & 40 feet \\
One Round (Tactical) &  &  &  &  \\
Walk & 15 ft. & 20 ft. & 30 ft. & 40 ft. \\
Hustle & 30 ft. & 40 ft. & 60 ft. & 80 ft. \\
One Minute (Local) &  &  &  &  \\
Walk & 150 ft. & 200 ft. & 300 ft. & 400 ft. \\
Hustle & 300 ft. & 400 ft. & 600 ft. & 800 ft. \\
One Hour (Overland) &  &  &  &  \\
Walk & 1-1/2 miles & 2 miles & 3 miles & 4 miles \\
Hustle & 3 miles & 4 miles & 6 miles & 8 miles \\
One Day (Overland) &  &  &  &  \\
Walk & 15 miles & 20 miles & 30 miles & 40 miles \\
Hustle & \tdash & \tdash & \tdash & \tdash \\
\end{dtabularx}
\end{dtable}

\begin{dtable}
\lcaption{Hampered Movement}
\begin{dtabularx}{\columnwidth}{l >{\lcol}X >{\ccol}p{8em}}
Condition & Example Extra Movement Cost \\
\bottomrule
Difficult terrain & Rubble, undergrowth, steep slope, ice, cracked and pitted surface, uneven floor & \mult2 \\
Obstacle\fn{1} & Low wall, deadfall, broken pillar & \mult2 \\
Poor visibility & Darkness or fog & \mult2 \\
Impassable & Floor-to-ceiling wall, closed door, blocked passage & \tdash \\
\end{dtabularx}
1 May require a skill check
\end{dtable}

There are three movement scales in the game, as follows.
\begin{itemize}
\item Tactical, for combat, measured in feet (or squares) per round.
\item Local, for exploring an area, measured in feet per minute.
\item Overland, for getting from place to place, measured in miles per
hour or miles per day.
\end{itemize}

\parhead{Modes of Movement} While moving at the different movement scales, creatures generally walk, hustle.
\subparhead{Walk} A walk represents unhurried but purposeful movement at 3 miles per hour for an unencumbered human.
\subparhead{Hustle} A hustle is a jog at about 6 miles per hour for an unencumbered human. A character moving his or her
speed twice in a single round, or moving that speed in the same round that he or she performs a standard action or another move action is hustling when he or she moves.

\subsection{Tactical Movement}
Use tactical movement for combat. Characters generally don't walk during combat -- they hustle. A character who
moves his or her speed and takes some action, such as attacking or casting a spell, is hustling for about half the round and doing something else for the other half.

\parhead{Minimum Movement} In some situations, your movement may be so hampered that you don't have sufficient speed even to move 5 feet (1 square). In such a case, you may use a full-round action to move 5 feet (1 square) in any direction, even diagonally. (You can't take advantage of this rule to move through impassable terrain or to move when all movement is prohibited to you, such as while paralyzed.)

\subsection{Local Movement}
Characters exploring an area use local movement, measured in feet per minute.
\parhead{Walk} A character can walk without a problem on the local scale.
\parhead{Hustle} A character can hustle without a problem on the local scale. See \trefnp{Terrain and Overland Movement}, below, for movement measured in miles per hour.

\subsection{Overland Movement}

\begin{dtable}
\lcaption{Terrain and Overland Movement}
\begin{dtabularx}{\columnwidth}{>{\lcol}X c c c}
\tb{Terrain}  & \tb{Highway} & \tb{Road or Trail} & \tb{Trackless} \\
\bottomrule
Desert, sandy & \mult1 & \mult1/2 & \mult1/2 \\
Forest & \mult1 & \mult1 & \mult1/2 \\
Hills & \mult1 & \mult3/4 & \mult1/2 \\
Jungle & \mult1 & \mult3/4 & \mult1/4 \\
Moor & \mult1 & \mult1 & \mult3/4 \\
Mountains & \mult3/4 & \mult3/4 & \mult1/2 \\
Plains & \mult1 & \mult1 & \mult3/4 \\
Swamp & \mult1 & \mult3/4 & \mult1/2 \\
Tundra, frozen & \mult1 & \mult3/4 & \mult3/4
\end{dtabularx}
\end{dtable}

\begin{dtable}
\lcaption{Mounts and Vehicles}
\begin{dtabularx}{\columnwidth}{>{\lcol}X l l}
\tb{Mount/Vehicle} & \tb{Per Hour} & \tb{Per Day} \\
\bottomrule
Mount (carrying load) &  &  \\
\tind Light horse or light warhorse & 6 miles & 60 miles \\
\tind Light horse (151-450 lb.)\fn{1} & 4 miles & 40 miles \\
\tind Light warhorse (231-690 lb.)\fn{1} & 4 miles & 40 miles \\
\tind Heavy horse or heavy warhorse & 5 miles & 50 miles \\
\tind Heavy horse (201-600 lb.)\fn{1} & 3-1/2 miles & 35 miles \\
\tind Heavy warhorse (301-900 lb.)\fn{1} & 3-1/2 miles & 35 miles \\
\tind Pony or warpony & 4 miles & 40 miles \\
\tind Pony (76-225 lb.)\fn{1} & 3 miles & 30 miles \\
\tind Warpony (101-300 lb.)\fn{1} & 3 miles & 30 miles \\
\tind Donkey or mule & 3 miles & 30 miles \\
\tind Donkey (51-150 lb.)\fn{1} & 2 miles & 20 miles \\
\tind Mule (231-690 lb.)\fn{1} & 2 miles & 20 miles \\
\tind Dog, riding & 4 miles & 40 miles \\
\tind Dog, riding (101-300 lb.)\fn{1} & 3 miles & 30 miles \\
\tind Cart or wagon & 2 miles & 20 miles \\
\tb{Ship} &  &  \\
\tind Raft or barge (poled or towed)\fn{2} & 1/2 mile & 5 miles \\
\tind Keelboat (rowed)\fn{2} & 1 mile & 10 miles \\
\tind Rowboat (rowed)\fn{2} & 1-1/2 miles & 15 miles \\
\tind Sailing ship (sailed) & 2 miles & 48 miles \\
\tind Warship (sailed and rowed) & 2-1/2 miles & 60 miles \\
\tind Longship (sailed and rowed) & 3 miles & 72 miles \\
\tind Galley (rowed and sailed) & 4 miles & 96 miles \\
\end{dtabularx}
1 Quadrupeds, such as horses, can carry heavier loads than characters can. See \trefnp{Weight Limits}, above, for more information. \\
2 Rafts, barges, keelboats, and rowboats are used on lakes and rivers.
If going downstream, add the speed of the current (typically 3 miles per hour) to the speed of the vehicle. In addition to 10 hours of being rowed, the vehicle can also float an additional 14 hours, if someone can guide it, so add an additional 42 miles to the daily distance traveled. These vehicles can't be rowed against any significant current, but they can be pulled upstream by draft animals on the shores.
\end{dtable}

Characters covering long distances cross-country use overland movement. Overland movement is measured in miles per hour or miles per day. A day represents 10 hours of actual travel time. For rowed watercraft, a day represents 10 hours of rowing. For a sailing ship, it represents 24 hours.
\parhead{Walk} A character can walk 10 hours in a day of travel without a problem. Walking for longer than that can wear him or her out (see Forced March, below).
\parhead{Hustle} A character can hustle for 1 hour without a problem. Hustling for a second hour in between sleep cycles deals 1 point of \glossterm{nonlethal damage}, and each additional hour deals twice the damage taken during the previous hour of hustling. A character who takes any nonlethal damage from hustling becomes fatigued.
\parhead{Terrain} The terrain through which a character travels affects how much distance he or she can cover in an hour or a day (see \trefnp{Terrain and Overland Movement}). A highway is a straight, major, paved road. A road is typically a dirt track. A trail is like a road, except that it allows only single-file travel and does not benefit a party traveling with vehicles. Trackless terrain is a wild area with no significant paths.
\parhead{Forced March} In a day of normal walking, a character walks for 10 hours. The rest of the daylight time is spent making and breaking camp, resting, and eating.

A character can walk for more than 10 hours in a day by making a forced march. For each hour of marching beyond 10 hours, a Constitution check (DR 10, \plus2 per extra hour) is required. If the check fails, the character takes 1d6 points of nonlethal damage. A character who takes any nonlethal damage from a forced march becomes fatigued. Eliminating the nonlethal damage also eliminates the fatigue. It's possible for a character to march into unconsciousness by pushing himself too hard.

\parhead{Mounted Movement} A mount bearing a rider can move at a hustle. The damage it takes when doing so, however, is lethal damage, not nonlethal damage. The creature can also be ridden in a forced march, but its Constitution checks automatically fail, and, again, the damage it takes is lethal damage. Mounts also become fatigued when they take any damage from hustling or forced marches.

See \trefnp{Mounts and Vehicles} for mounted speeds and speeds for vehicles pulled by draft animals.

\parhead{Waterborne Movement} See \trefnp{Mounts and Vehicles} for speeds for water vehicles.

\section{Exploration}
\subsection{Vision and Light}
Dwarves and half-orcs have darkvision, but everyone else needs light to see by.  In an area of bright light, all characters can see clearly. A creature can't hide in an area of bright light unless it is invisible or has cover.

In an area of shadowy illumination, a character can see dimly. Creatures within this area have \concealment relative to that character. A creature in an area of shadowy illumination can make a Hide check to conceal itself.

In areas of darkness, creatures without darkvision are effectively \blinded.

Characters with low-light vision (elves, gnomes, and half-elves) treat sources of light as if they had double their normal illumination range.

Characters with darkvision (dwarves and half-orcs) can see lit areas normally as well as dark areas within a radius defined by the ability -- usually, 50 feet. A creature can't hide within that range of a character using darkvision unless it is invisible or has cover. Darkvision does not function if the character is in bright light or is dazzled, and does not resume functioning until 1 round the character leaves the area of bright light or stops being dazzled.

A character with darkvision can see dimly in totally dark areas beyond the range indicated by their ability. They can see in such areas as if the areas were in shadowy illumination.

\subsection{Breaking And Entering}
When attempting to break an object, you have two choices: smash it with a weapon or break it with sheer strength.

\subsubsection{Smashing an Object}
Smashing an object is accomplished by simply attacking the object as you would any other target. If it is attended, this is done using the disarm special attack. Generally, you can smash an object only with a bludgeoning or slashing weapon.

\parhead{Armor Defense} Objects are easier to hit than creatures because they usually don't move, but many are tough enough to shrug off some damage from each blow. An object's Armor defense is equal to 10 \add its size modifier \add its Dexterity. An inanimate object has a Dexterity of \minus10 (\minus10 penalty to physical defenses).

\parhead{Hardness} Each object has hardness -- a number that represents how well it resists damage. Whenever an object takes damage, subtract its hardness from the damage. Only damage in excess of its hardness is deducted from the object's hit points.
%None of these tables exist currently. Also object hit points are obnoxious.
%(see \trefnp{Common Armor, Weapon, and Shield Hardness and Hit Points}; \trefnp{Substance Hardness and Hit Points}; and \trefnp{Object Hardness and Hit Points}).
\parhead{Hit Points} An object's hit point total depends on what it is made of and how big it is. When an object's hit points reach 0, it's ruined.

Very large objects may have separate hit point totals for different sections.

\subparhead{Energy Attacks} Sonic attacks deal damage to most objects just as they do to creatures; roll damage and apply it normally after a successful hit. Electricity and fire attacks deal half damage to most objects; divide the damage dealt by 2 before applying the hardness. Cold attacks deal one-quarter damage to most objects; divide the damage dealt by 4 before applying the hardness.

\subparhead{Ranged Weapon Damage} Objects take half damage from ranged weapons (unless the weapon is a siege engine or something similar). Divide the damage dealt by 2 before applying the object's hardness.

\subparhead{Ineffective Weapons} Certain weapons just can't effectively deal damage to certain objects.

\subparhead{Immunities} Objects are immune to \glossterm{nonlethal damage} and to critical hits. Even animated objects, which are otherwise considered creatures, have these immunities because they are constructs.

\subparhead{Magic Armor, Shields, and Weapons} Each \plus1 of enhancement bonus adds 2 to the hardness of armor, a weapon, or a shield and \plus10 to the item's hit points.

\subparhead{Vulnerability to Certain Attacks} Certain attacks are especially successful against some objects. In such cases, attacks deal double their normal damage and may ignore the object's hardness.

\subparhead{Damaged Objects} A damaged object remains fully functional until the item's hit points are reduced to 0, at which point it is broken. Further damage applies as negative hit points. If an item has more negative hit points than its maximum hit points, it is destroyed.

Broken objects cannot be used for their intended purpose, but still retain enough of their original form to be repaired without too much work. For example, a broken wall lies in pieces on the ground and no longer blocks passage, but can be repaired with no more effort than damaged items. Magic items that are broken retain their magical properties once fixed. Damaged and broken (but not destroyed) objects can be repaired with the Craft skill.

\parhead{Saving Throws} Nonmagical, unattended items never make saving throws. They are considered to have failed their saving throws, so they always are affected by spells. An item attended by a character (being grasped, touched, or worn) makes saving throws as the character (that is, using the character's saving throw bonus).

\par Magic items always get saving throws. A magic item's Fortitude, Reflex, and Mental defenses are equal to 10 \add one-half its spellpower. An attended magic item either makes saving throws as its owner or uses its own saving throw bonus, whichever is better.

\subparhead{Animated Objects} Animated objects count as creatures for purposes of determining their Armor Defense (do not treat them as inanimate objects).

\subsubsection{Breaking Items}
When a character tries to break something with sudden force rather than by dealing damage, use a Strength check (rather than an attack roll and damage roll, as with the disarm special attack) to see whether he or she succeeds. The DR depends more on the
construction of the item than on the material.

If an item has no more than half its hit points remaining, the DR to break it drops by 5.

A crowbar or portable ram improves a character's chance of breaking open a door.
\section{Wealth And Money}

\subsection{Coins}
The most common coin is the gold piece (gp). A gold piece is worth 10 silver pieces. Each silver piece is worth 10 copper pieces (cp). In addition to copper, silver, and gold coins, there are also platinum pieces (pp), which are each worth 10 gp.

The standard coin weighs about a third of an ounce (fifty to the pound).

\begin{dtable}
\lcaption{Coin Exchange Values}
\begin{dtabularx}{\columnwidth}{l c *{4}{>{\ccol}X}}
& & \tb{CP} & \tb{SP} & \tb{GP} & \tb{PP} \\
\bottomrule
Copper piece (cp) & = & 1 & 1/10 & 1/100 & 1/1,000 \\
Silver piece (sp) & = & 10 & 1 & 1/10 & 1/100 \\
Gold piece (gp) & = & 100 & 10 & 1 & 1/10 \\
Platinum piece (pp) & = & 1,000 & 100 & 10 & 1
\end{dtabularx}
\end{dtable}

\subsection{Wealth Other Than Coins}
Merchants commonly exchange trade goods without using currency. As a means of comparison, some trade goods are detailed below.

\begin{dtable}
\lcaption{Trade Goods}
\begin{dtabularx}{\columnwidth}{l >{\lcol}X}
\tb{Cost} & \tb{Item} \\
\bottomrule
1 cp & One pound of wheat \\
2 cp & One pound of flour \\
1 sp & One pound of iron, or one chicken \\
5 sp & One pound of tobacco or copper \\
1 gp & One pound of cinnamon, or one goat \\
2 gp & One pound of ginger or pepper, or one sheep \\
3 gp & One pig \\
4 gp & One square yard of linen \\
5 gp & One pound of salt or silver \\
10 gp & One square yard of silk, or one cow \\
15 gp & One pound of saffron or cloves, or one ox \\
50 gp & One pound of gold \\
500 gp & One pound of platinum
\end{dtabularx}
\end{dtable}

\subsection{Selling Items}
In general, a character can sell something for a quarter its listed price.

Trade goods, such as gems, are the exception to this rule and can be sold for their full value.
A trade good, in this sense, is a valuable good that can be easily exchanged almost as if it were cash itself.

\section{Goods And Services}

\begin{dtable!*}
\lcaption{Goods and Services}
\begin{dtabularx}{\textwidth}{>{\lcol}X c c >{\lcol}X c c >{\lcol}X c c}
\tb{Item} & \tb{Cost} & \tb{Weight} & \tb{Item} & \tb{Cost} & \tb{Weight} & \tb{Item} & \tb{Cost} & \tb{Weight} \\
\bottomrule
\multicolumn{3}{>{\columncolor{tbrown}}c}{\tb{Adventuring Gear}} & \tind Average & 40 gp & 1 lb. & \multicolumn{3}{>{\columncolor{tbrown}}c}{\tb{Special Substances and Items}} \\
Backpack (empty) & 2 gp & 2 lb.\fn{1} & \tind Good & 80 gp & 1 lb. & Acid (flask) & 5 gp & 1 lb. \\
Barrel (empty) & 2 gp & 30 lb. & \tind Amazing & 150 gp & 1 lb. & Alchemist's fire (flask) & 5 gp & 1 lb. \\
Basket (empty) & 4 sp & 1 lb. & Manacles & 15 gp & 2 lb. & Antitoxin (vial) & 10 gp & \tdash \\
Bedroll & 1 sp & 5 lb.\fn{1} & Manacles, masterwork & 50 gp & 2 lb. & Everburning torch & 100 gp & 1 lb. \\
Bell & 1 gp & \tdash & Mirror, small steel & 10 gp & 1/2 lb. & Holy water (flask) & 10 gp & 1 lb. \\
Blanket, winter & 5 sp & 3 lb.\fn{1} & Mug/Tankard, clay & 2 cp & 1 lb. & Smokestick & 5 gp & 1/2 lb. \\
Block and tackle & 5 gp & 5 lb. & Oil (1-pint flask) & 1 sp & 1 lb. & Sunrod & 2 gp & 1 lb. \\
Bottle, wine, glass & 2 gp & \tdash & Paper (sheet) & 4 sp & \tdash & Tanglefoot bag & 15 gp & 4 lb. \\
Bucket (empty) & 5 sp & 2 lb. & Parchment (sheet) & 2 sp & \tdash & Thunderstone & 10 gp & 1 lb. \\
Caltrops & 1 gp & 2 lb. & Pick, miner's & 3 gp & 10 lb. & Tindertwig & 5 sp & \tdash \\
Candle & 1 cp & \tdash & Pitcher, clay & 2 cp & 5 lb. & \multicolumn{3}{c}{\tb{Tools and Skill Kits}}  \\
Canvas (sq.\ yd.) & 1 sp & 1 lb. & Piton & 1 sp & 1/2 lb. & Item & Cost & Weight \\
Case, map or scroll & 1 gp & 1/2 lb. & Pole, 10 foot & 5 cp & 8 lb. & Alchemist's lab & 500 gp & 40 lb. \\
Chain (10 ft.) & 30 gp & 2 lb. & Pot, iron & 5 sp & 10 lb. & Artisan's tools & 5 gp & 5 lb. \\
Chalk, 1 piece & 1 cp & \tdash & Pouch, belt (empty) & 1 gp & 1/2 lb.\fn{1} & Artisan's tools, masterwork & 55 gp & 5 lb. \\
Chest (empty) & 2 gp & 25 lb. & Ram, portable & 10 gp & 20 lb. & Climber's kit & 80 gp & 5 lb.\fn{1} \\
Crowbar & 2 gp & 5 lb. & Rations, trail (per day) & 5 sp & 1 lb.\fn{1} & Disguise kit & 50 gp & 8 lb.\fn{1} \\
Firewood (per day) & 1 cp & 20 lb. & Rope, hempen (50 ft.) & 1 gp & 10 lb. & Healer's kit & 50 gp & 1 lb. \\
Fishhook & 1 sp & \tdash & Rope, silk (50 ft.) & 10 gp & 5 lb. & Holly and mistletoe & \tdash & \tdash \\
Fishing net, 25 sq.\ ft. & 4 gp & 5 lb. & Sack (empty) & 1 sp & 1/2 lb.\fn{1} & Holy symbol, wooden & 1 gp & \tdash \\
Flask (empty) & 3 cp & 1-1/2 lb. & Sealing wax & 1 gp & 1 lb. & Holy symbol, silver & 25 gp & 1 lb. \\
Flint and steel & 1 gp & \tdash & Sewing needle & 5 sp & \tdash & Hourglass & 25 gp & 1 lb. \\
Grappling hook & 1 gp & 4 lb. & Signal whistle & 8 sp & \tdash & Magnifying glass & 100 gp & \tdash \\
Hammer & 5 sp & 2 lb. & Signet ring & 5 gp & \tdash & Musical instrument, common & 5 gp & 3 lb.\fn{1} \\
Ink (1 oz.\ vial) & 8 gp & \tdash & Sledge & 1 gp & 10 lb. & Musical instrument, masterwork & 100 gp & 3 lb.\fn{1} \\
Inkpen & 1 sp & \tdash & Soap (per lb.) & 5 sp & 1 lb. & Scale, merchant's & 2 gp & 1 lb. \\
Jug, clay & 3 cp & 9 lb. & Spade or shovel & 2 gp & 8 lb. & Spell component pouch & 5 gp & 2 lb. \\
Ladder, 10 foot & 2 sp & 20 lb. & Spyglass & 1,000 gp & 1 lb. & Spellbook, wizard's (blank) & 15 gp & 3 lb. \\
Lamp, common & 1 sp & 1 lb. & Tent & 10 gp & 20 lb.\fn{1} & Thieves' tools & 30 gp & 1 lb. \\
Lantern, bullseye & 12 gp & 3 lb. & Torch & 1 cp & 1 lb. & Thieves' tools, masterwork & 100 gp & 2 lb. \\
Lantern, hooded & 7 gp & 2 lb. & Vial, ink or potion & 1 gp & 1/10 lb. & Tool, masterwork & 50 gp & 1 lb. \\
Lock &   & 1 lb. & Waterskin & 1 gp & 4 lb.\fn{1} & Water clock & 1,000 gp & 200 lb. \\
\tind Very simple & 20 gp & 1 lb. & Whetstone & 2 cp & 1 lb. &  &  &  \\
\end{dtabularx}
1 These items weigh one-quarter this amount when made for Small characters. Containers for Small characters also carry one-quarter the normal amount. \\
\tdash No weight, or no weight worth noting.	
\end{dtable!*}

\begin{dtable!*}
\begin{dtabularx}{\textwidth}{>{\lcol}X c c >{\lcol}X c c >{\lcol}X c c}
\tb{Clothing} &  &  & \tb{Mounts and Related Gear} &  &  & \tb{Transport} &  & \\
\bottomrule
\tb{Item} & \tb{Cost} & \tb{Weight} & \tb{Item} & \tb{Cost} & \tb{Weight} & \tb{Item} & \tb{Cost} & \tb{Weight} \\
Artisan's outfit & 1 gp & 4 lb. & Barding &  &  & Carriage & 100 gp & 600 lb. \\
Cleric's vestments & 5 gp & 6 lb. & \tind Medium creature & \mult2\fn{2} & \mult1\fn{2}  & Cart & 15 gp & 200 lb. \\
Cold weather outfit & 8 gp & 7 lb. & \tind Large creature & \mult4\fn{2} & \mult2\fn{2}  & Galley & 30,000 gp & \tdash \\
Courtier's outfit & 30 gp & 6 lb. & Bit and bridle & 2 gp & 1 lb. & Keelboat & 3,000 gp & \tdash \\
Entertainer's outfit & 3 gp & 4 lb. & Dog, guard & 25 gp & \tdash & Longship & 10,000 gp & \tdash \\
Explorer's outfit & 10 gp & 8 lb. & Dog, riding & 150 gp & \tdash & Rowboat & 50 gp & 100 lb. \\
Monk's outfit & 5 gp & 2 lb. & Donkey or mule & 8 gp & \tdash & Oar & 2 gp & 10 lb. \\
Noble's outfit & 75 gp & 10 lb. & Feed (per day) & 5 cp & 10 lb. & Sailing ship & 10,000 gp & \tdash \\
Peasant's outfit & 1 sp & 2 lb. & Horse &  &  & Sled & 20 gp & 300 lb. \\
Royal outfit & 200 gp & 15 lb. & \tind Horse, heavy & 200 gp & \tdash & Wagon & 35 gp & 400 lb. \\
Scholar's outfit & 5 gp & 6 lb. & \tind Horse, light & 75 gp & \tdash & Warship & 25,000 gp & \tdash \\
Traveler's outfit & 1 gp & 5 lb. & \tind Pony & 30 gp & \tdash & \tb{Spellcasting and Services} &  &  \\
\tb{Food, Drink, and Lodging} &  &  & \tind Warhorse, heavy & 400 gp & \tdash & Service & \multicolumn{2}{l}{Cost} \\
Item & Cost & Weight & \tind Warhorse, light & 150 gp & \tdash & Coach cab & \multicolumn{2}{l}{3 cp per mile} \\
Ale &  &  & \tind Warpony & 100 gp & \tdash & Hireling, trained & \multicolumn{2}{l}{3 sp per day} \\
\tind Gallon & 2 sp & 8 lb. & Saddle &  &  & Hireling, untrained & \multicolumn{2}{l}{1 sp per day} \\
\tind Mug & 4 cp & 1 lb. & \tind Military & 20 gp & 30 lb. & Messenger & \multicolumn{2}{l}{2 cp per mile} \\
Banquet (per person) & 10 gp & \tdash & \tind Pack & 5 gp & 15 lb. & Road or gate toll & \multicolumn{2}{l}{1 cp} \\
Bread, per loaf & 2 cp & 1/2 lb. & \tind Riding & 10 gp & 25 lb. & Ship's passage & \multicolumn{2}{l}{1 sp per mile} \\
Cheese, hunk of & 1 sp & 1/2 lb. & \tb{Saddle, Exotic} &  &  &  & & \\
Inn stay (per day) &  &  & \tind Military & 60 gp & 40 lb. & Spell, 1st-level & \multicolumn{2}{l}{Spellpower x 10 gp\fn{1}} \\
\tind Good & 2 gp & \tdash & \tind Pack & 15 gp & 20 lb. & Spell, 2nd-level & \multicolumn{2}{l}{Spellpower x 20 gp\fn{1}} \\
\tind Common & 5 sp & \tdash & \tind Riding & 30 gp & 30 lb. & Spell, 3rd-level & \multicolumn{2}{l}{Spellpower x 30 gp\fn{1}} \\
\tind Poor & 2 sp & \tdash & Saddlebags & 4 gp & 8 lb. & Spell, 4th-level & \multicolumn{2}{l}{Spellpower x 40 gp\fn{1}} \\
Meals (per day) &  &  & Stabling (per day) & 5 sp & \tdash & Spell, 5th-level & \multicolumn{2}{l}{Spellpower x 50 gp\fn{1}} \\
\tind Good & 5 sp & \tdash & &  &  & Spell, 6th-level & \multicolumn{2}{l}{Spellpower x 60 gp\fn{1}} \\
\tind Common & 3 sp & \tdash & &  &  & Spell, 7th-level & \multicolumn{2}{l}{Spellpower x 70 gp\fn{1}} \\
\tind Poor & 1 sp & \tdash & &  &  & Spell, 8th-level & \multicolumn{2}{l}{Spellpower x 80 gp\fn{1}} \\
Meat, chunk of & 3 sp & 1/2 lb. & &  &  & Spell, 9th-level & \multicolumn{2}{l}{Spellpower x 90 gp\fn{1}} \\
Wine &  &  & &  &  &  &  & \\
\tind Common (pitcher) & 2 sp & 6 lb. & &  &  &  &  & \\
\tind Fine (bottle) & 10 gp & 1-1/2 lb. & & & & & & \\
\end{dtabularx}
1 See spell description for additional costs. If the additional costs put the spell's total cost above 2,000 gp, that spell is not generally available. \\
2 Relative to normal armor of the same type
\end{dtable!*}

\subsection{Adventuring Gear}
A few of the pieces of adventuring gear found on Table: Goods and Services are described below, along with any special benefits they confer on the user (``you'').
\parhead{Caltrops} A caltrop is a four-pronged iron spike crafted so that one prong faces up no matter how the caltrop comes to rest. You scatter caltrops on the ground in the hope that your enemies step on them or are at least forced to slow down to avoid them. One 2- pound bag of caltrops covers an area 5 feet square.
\par Each time a creature moves into an area covered by caltrops (or spends a round fighting while standing in such an area), it might step on one. The caltrops make an attack roll (with an accuracy of \plus0) against the creature's Armor defense. If the caltrops succeed on the attack, the creature has stepped on one. The caltrop deals 1 point of damage, and the creature's speed is reduced by one-half because its foot is wounded. This movement penalty lasts for 24 hours, or until the creature is successfully treated with a DR 15 Heal check, or until it receives at least 1 point of healing. Any creature moving at half speed or slower can pick its way through a bed of caltrops without stepping on any.
\par Caltrops may not be effective against unusual opponents.
\parhead{Candle} A candle dimly illuminates a 5 foot radius and burns for 1 hour.
\parhead{Chain} Chain has hardness 10 and 5 hit points. It can be burst with a DR 26 Strength check.
\parhead{Crowbar} A crowbar it grants a \plus2 bonus on Strength checks made for such purposes. If used in combat, treat a crowbar as a improvised weapon equivalent to a club.
\parhead{Flint and Steel} Lighting a torch with flint and steel is a full-round action, and lighting any other fire with them takes at least that long.
\parhead{Grappling Hook} Throwing a grappling hook successfully requires a ranged attack (DR 10, \plus2 per 10 feet of distance thrown).
\parhead{Hammer} If a hammer is used in combat, treat it as a medium improvised weapon that deals bludgeoning damage equal to that of a spiked gauntlet of its size.
\parhead{Ink} This is black ink. You can buy ink in other colors, but it costs twice as much.
\parhead{Jug, Clay} This basic ceramic jug is fitted with a stopper and holds 1 gallon of liquid.
\parhead{Lamp, Common} A lamp clearly illuminates a 20 foot radius, provides shadowy illumination out to a 40 foot radius, and burns for 6 hours on a pint of oil. You can carry a lamp in one hand.
\parhead{Lantern, Bullseye} A bullseye lantern provides bright illumination in a 50 foot cone and shadowy illumination in a 100 foot cone. It burns for 6 hours on a pint of oil. You can carry a bullseye lantern in one hand.
\parhead{Lantern, Hooded} A hooded lantern clearly illuminates a 20 foot radius and provides shadowy illumination in a 40 foot radius. It burns for 6 hours on a pint of oil. You can carry a hooded lantern in one hand.
\parhead{Lock} The DR to open a lock with the Open Lock skill depends on the lock's quality: simple (DR 20), average (DR 25), good (DR 30), or superior (DR 40).
\parhead{Manacles and Manacles, Masterwork} Manacles can bind a Medium creature. A manacled creature can use the Escape Artist skill to slip free (DR 30, or DR 35 for masterwork manacles). Breaking the manacles requires a Strength check (DR 26, or DR 28 for masterwork manacles). Manacles have hardness 10 and 10 hit points.
\par Most manacles have locks; add the cost of the lock you want to the cost of the manacles.
\par For the same cost, you can buy manacles for a Small creature.
\par For a Large creature, manacles cost ten times the indicated amount, and for a Huge creature, one hundred times this amount. Gargantuan, Colossal, Tiny, Diminutive, and Fine creatures can be held only by specially made manacles.
\parhead{Oil} A pint of oil burns for 6 hours in a lantern. You can use a flask of oil as a splash weapon. Use the rules for alchemist's fire, except that it takes a full round action to prepare a flask with a fuse. Once it is thrown, there is a 50\% chance of the flask igniting successfully.
\par You can pour a pint of oil on the ground to cover an area 5 feet square, provided that the surface is smooth. If lit, the oil burns for 2 rounds and deals 1d3 points of fire damage to each creature in the area.
\parhead{Ram, Portable} This iron-shod wooden beam gives you a \plus2 bonus on Strength checks made to break open a door and it allows a second person to help you without having to roll, increasing your bonus by 2.
\parhead{Rope, Hempen} This rope has 2 hit points and can be burst with a DR 23 Strength check.
\parhead{Rope, Silk} This rope has 4 hit points and can be burst with a DR 24 Strength check.
\parhead{Spyglass} Objects viewed through a spyglass are magnified to twice their size.
\parhead{Torch} A torch burns for 1 hour, clearly illuminating a 20 foot radius and providing shadowy illumination out to a 40 foot radius. If a torch is used in combat, treat it as a light improvised weapon that deals bludgeoning damage equal to that of a gauntlet of its size, plus 1 point of fire damage. Wielding a torch gives a \plus2 bonus to dirty trick attempts made to dazzle an opponent. Lighting a torch with an open flame or tindertwig is a standard action, while using flint and steel requires a full-round action.
\parhead{Vial} A vial holds 1 ounce of liquid. The stoppered container usually is no more than 1 inch wide and 3 inches high.

\subsection{Special Substances And Items}
Any of these substances except for the everburning torch and holy water can be made by a character with the Craft (alchemy) skill.
\parhead{Acid} You can throw a flask of acid as a splash weapon. Treat this attack as a ranged touch attack with a \glossterm{range increment} of 10 feet. A direct hit deals 1d6 points of acid damage. Every creature within 5 feet of the point where the acid hits takes 1 point of acid damage from the splash.
\parhead{Alchemist's Fire} You can throw a flask of alchemist's fire as a splash weapon. Treat this attack as a ranged touch attack with a range increment of 10 feet.
\par A direct hit deals 1d6 points of fire damage and makes the creature \ignited for 1 round. Every creature within 5 feet of the point where the flask hits takes 1 point of fire damage from the splash.
\parhead{Antitoxin} If you drink antitoxin, you get a \plus5 bonus to Fortitude defenses against poison for 1 hour.
\parhead{Everburning Torch} This otherwise normal torch has a continual flame spell cast upon it. An everburning torch clearly illuminates a 20 foot radius and provides shadowy illumination out to a 40 foot radius.
\parhead{Holy Water} Holy water damages undead creatures and evil outsiders almost as if it were acid. A flask of holy water can be thrown as a splash weapon.
\par Treat this attack as a ranged touch attack with a \glossterm{range increment} of 10 feet. A flask breaks if thrown against the body of a corporeal creature, but to use it against an incorporeal creature, you must open the flask and pour the holy water out onto the target. Thus, you can douse an incorporeal creature with holy water only if you are adjacent to it. Doing so is a ranged touch attack.
\par A direct hit by a flask of holy water deals 2d4 points of damage to an undead creature or an evil outsider. Each such creature within 5 feet of the point where the flask hits takes 1 point of damage from the splash.
\par Temples to good deities sell holy water at cost (making no profit).
\parhead{Smokestick} This alchemically treated wooden stick instantly creates thick, opaque smoke when ignited. The smoke fills a 10 foot cube (treat the effect as a fog cloud spell, except that a moderate or stronger wind dissipates the smoke in 1 round). The stick is consumed after 1 round, and the smoke dissipates naturally.
\parhead{Sunrod} This 1 foot long, gold-tipped, iron rod glows brightly when struck. It clearly illuminates a 20 foot radius and provides shadowy illumination in a 40 foot radius. It glows for 6 hours, after which the gold tip is burned out and worthless.
\parhead{Tanglefoot Bag} When you throw a tanglefoot bag at a creature (as a ranged touch attack with a \glossterm{range increment} of 10 feet), the bag comes apart and the goo bursts out, causing the target to become \entangled.
\par The goo becomes tough and resilient upon exposure to air, making it difficult to escape. A creature can break the effect by making a DR 15 Strength check or by dealing 5 points of damage to the goo. Unless the attacker uses a slashing weapon, damage dealt to the goo is also dealt to the creature the goo is on. The goo does not have an Armor defense, and can be hit automatically.
\par The goo becomes brittle and fragile after 2d4 rounds, cracking apart and losing its effectiveness. An application of universal solvent to a stuck creature dissolves the goo immediately.
\parhead{Thunderstone} You can throw this stone as a ranged attack with a \glossterm{range increment} of 20 feet. When it strikes a hard surface (or is struck hard), it creates a deafening bang that is treated as a sonic attack. You make a Fortitude attack against each creature within a \areasmall radius spread to deafen them for 5 minutes. Your accuracy on this attack is \plus5. A deafened creature automatically fails Listen checks, takes a \minus2 penalty to any checks which involve hearing, and has a 20\% chance of spell failure when casting spells with verbal components.
\par Since you don't need to hit a specific target, you can simply aim at a particular 5 foot square. Treat the target square as if its physical defense was 5.
\parhead{Tindertwig} The alchemical substance on the end of this small, wooden stick ignites when struck against a rough surface. Creating a flame with a tindertwig is much faster than creating a flame with flint and steel (or a magnifying glass) and tinder. Lighting a torch with a tindertwig is a standard action (rather than a full-round action), and lighting any other fire with one is at least a standard action.

\subsection{Tools and Skill Kits}
\parhead{Alchemist's Lab} An alchemist's lab always has the perfect tool for making alchemical items, so it provides a \plus2 bonus on Craft (alchemy) checks. It has no bearing on the costs related to the Craft (alchemy) skill. Without this lab, a character with the Craft (alchemy) skill is assumed to have enough tools to use the skill but not enough to get the \plus2 bonus that the lab provides.
\parhead{Artisan's Tools} These special tools include the items needed to pursue any craft. Without them, you have to use improvised tools (\minus2 penalty on Craft checks), if you can do the job at all (see \pcref{Craft}).
\parhead{Artisan's Tools, Masterwork} These tools serve the same purpose as artisan's tools (above), but masterwork artisan's tools are the perfect tools for the job, so you get a \plus2 bonus on Craft checks made with them.
\parhead{Climber's Kit} This is the perfect tool for climbing and gives you a \plus2 bonus on Climb checks.
\parhead{Disguise Kit} The kit is the perfect tool for disguise and provides a \plus2 bonus on Disguise checks. A disguise kit is exhausted after ten uses.
\parhead{Healer's Kit} It is the perfect tool for healing and provides a \plus2 bonus on Heal checks. A healer's kit is exhausted after ten uses.
\parhead{Holy Symbol, Silver or Wooden} A holy symbol focuses good energy. A cleric or paladin uses it as the focus for his spells and as a tool for turning undead. Each religion has its own holy symbol.
\parhead{Unholy Symbols} An unholy symbol is like a holy symbol except that it focuses evil energy and is used by evil clerics (or by neutral clerics who want to cast evil spells or command undead).
\parhead{Magnifying Glass} This simple lens allows a closer look at small objects. It is also useful as a substitute for flint and steel when starting fires. Lighting a fire with a magnifying glass requires light as bright as sunlight to focus, tinder to ignite, and at least a full-round action. A magnifying glass grants a \plus2 bonus on Appraise checks
involving any item that is small or highly detailed.
\parhead{Musical Instrument, Common or Masterwork} A masterwork instrument grants a \plus2 bonus on Perform checks involving its use.
\parhead{Pitons} You can make your own handholds and footholds by pounding pitons into a wall. Doing so takes 1 minute per piton, and one piton is needed per 3 feet of distance. As with any surface that offers handholds and footholds, a wall with pitons in it has a DR of 15.
\parhead{Spell Component Pouch} A spellcaster with a spell component pouch is assumed to have all the material components and focuses needed for spellcasting, except for those components that have a specific cost, divine focuses, and focuses that wouldn't fit in a pouch.
\parhead{Spellbook, Wizard's (Blank)} A spellbook has 100 pages of parchment, and each spell takes up one page per spell level (one page each for 0-level spells).
\parhead{Thieves' Tools} This kit contains the tools you need to use the Disable Device and Open Lock skills. Without these tools, you must improvise tools, and you take a \minus2 circumstance penalty on Disable Device and Open Locks checks.
\parhead{Thieves' Tools, Masterwork} This kit contains extra tools and tools of better make, which grant a \plus2 bonus on Disable Device and Open Lock checks.
\parhead{Tool, Masterwork} This well-made item is the perfect tool for the job. It grants a \plus2 bonus on a related skill check (if any). Bonuses provided by multiple masterwork items used toward the same skill check do not stack.
\parhead{Water Clock} This large, bulky contrivance gives the time accurate to within half an hour per day since it was last set. It requires a source of water, and it must be kept still because it marks time by the regulated flow of droplets of water.

\subsection{Clothing}
\parhead{Artisan's Outfit} This outfit includes a shirt with buttons, a skirt or pants with a drawstring, shoes, and perhaps a cap or hat. It may also include a belt or a leather or cloth apron for carrying tools.
\parhead{Cleric's Vestments} These ecclesiastical clothes are for performing priestly functions, not for adventuring.
\parhead{Cold Weather Outfit} A cold weather outfit includes a wool coat, linen shirt, wool cap, heavy cloak, thick pants or skirt, and
boots. This outfit grants a \plus5 bonus to Fortitude defenses against exposure to cold weather.
\parhead{Courtier's Outfit} This outfit includes fancy, tailored clothes in whatever fashion happens to be the current style in the courts of the nobles. If you wear this outfit without jewelry (costing an additional 50 gp), you look like an out-of-place commoner.
\parhead{Entertainer's Outfit} This set of flashy, perhaps even gaudy, clothes is for entertaining. While the outfit looks whimsical, its practical design lets you tumble, dance, walk a tightrope, or just run (if the audience turns ugly).
\parhead{Explorer's Outfit} This is a full set of clothes for someone who never knows what to expect. It includes sturdy boots, leather breeches or a skirt, a belt, a shirt (perhaps with a vest or jacket), gloves, and a cloak. Rather than a leather skirt, a leather overtunic may be worn over a cloth skirt. The clothes have plenty of pockets (especially the cloak). The outfit also includes any extra items you might need, such as a scarf or a wide-brimmed hat.
\parhead{Monk's Outfit} This simple outfit includes sandals, loose breeches, and a loose shirt, and is all bound together with sashes. The outfit is designed to give you maximum mobility, and it's made of high-quality fabric. You can hide small weapons in pockets hidden in the folds, and the sashes are strong enough to serve as short ropes.
\parhead{Noble's Outfit} This set of clothes is designed specifically to be expensive and to show it. Precious metals and gems are worked into the clothing. To fit into the noble crowd, every would-be noble also needs a signet ring (see Adventuring Gear, above) and jewelry (worth at least 100 gp).
\parhead{Peasant's Outfit} This set of clothes consists of a loose shirt and baggy breeches, or a loose shirt and skirt or overdress. Cloth wrappings are used for shoes.
\parhead{Royal Outfit} This is just the clothing, not the royal scepter, crown, ring, and other accoutrements. Royal clothes are ostentatious, with gems, gold, silk, and fur in abundance.
\parhead{Scholar's Outfit} Perfect for a scholar, this outfit includes a robe, a belt, a cap, soft shoes, and possibly a cloak.
\parhead{Traveler's Outfit} This set of clothes consists of boots, a wool skirt or breeches, a sturdy belt, a shirt (perhaps with a vest or jacket), and an ample cloak with a hood.

\subsection{Food, Drink, and Lodging}
\parhead{Inn} Poor accommodations at an inn amount to a place on the floor near the hearth. Common accommodations consist of a place on a raised, heated floor, the use of a blanket and a pillow. Good accommodations consist of a small, private room with one bed, some amenities, and a covered chamber pot in the corner.
\parhead{Meals} Poor meals might be composed of bread, baked turnips, onions, and water. Common meals might consist of bread, chicken stew, carrots, and watered-down ale or wine. Good meals might be composed of bread and pastries, beef, peas, and ale or wine.

\subsection{Mounts and Related Gear}
\parhead{Barding, Medium Creature and Large Creature} Barding is a type of armor that covers the head, neck, chest, body, and possibly legs of a horse or other mount. Barding made of medium or heavy armor provides better protection than light barding, but at the expense of speed. Barding can be made of any of the armor types found on Table: Armor and Shields.
\par Armor for a horse (a Large nonhumanoid creature) costs four times as much as armor for a human (a Medium humanoid creature) and also weighs twice as much as the armor found on Table: Armor and Shields (see \pref{Armor for Unusual Creatures}). If the barding is for a pony or other Medium mount, the cost is only double, and the weight is the same as for Medium armor worn by a humanoid. Medium or heavy barding slows a mount that wears it, as shown on the table below.

\begin{dtable}
\begin{dtabularx}{\columnwidth}{>{\lcol}X *{3}{>{\ccol}X}}
 & \multicolumn{3}{c}{\tdash\tdash\tdash Base Speed\tdash\tdash\tdash} \\
\bottomrule
Barding & (40 ft.) & (50 ft.) & (60 ft.) \\
Medium & 30 ft. & 35 ft. & 40 ft. \\
Heavy & 30 ft.\fn{1} & 35 ft.\fn{1} & 40 ft.\fn{1} \\
\end{dtabularx}
1 A mount wearing heavy armor moves at only triple its normal speed when running instead of quadruple.	
\end{dtable}

\par Flying mounts can't fly in medium or heavy barding.
\par Removing and fitting barding takes five times as long as the figures given on Table: Donning Armor. A barded animal cannot be used to carry any load other than the rider and normal saddlebags.
\parhead{Dog, Riding} This Medium dog is specially trained to carry a Small humanoid rider. It is brave in combat like a warhorse. You take no damage when you fall from a riding dog.
\parhead{Donkey or Mule} Donkeys and mules are stolid in the face of danger, hardy, surefooted, and capable of carrying heavy loads over vast distances. Unlike a horse, a donkey or a mule is willing (though not eager) to enter dungeons and other strange or threatening places.
\parhead{Feed} Horses, donkeys, mules, and ponies can graze to sustain themselves, but providing feed for them is much better. If you have a riding dog, you have to feed it at least some meat.
\parhead{Horse} A horse (other than a pony) is suitable as a mount for a human, dwarf, elf, half-elf, or half-orc. A pony is smaller than a horse and is a suitable mount for a gnome or halfling.
\par Warhorses and warponies can be ridden easily into combat. Light horses, ponies, and heavy horses are hard to control in combat.
\parhead{Saddle, Exotic} An exotic saddle is like a normal saddle of the same sort except that it is designed for an unusual mount. Exotic saddles come in military, pack, and riding styles.
\parhead{Saddle, Military} A military saddle braces the rider, providing a \plus2 bonus on Ride checks related to staying in the saddle. If you're knocked unconscious while in a military saddle, you have a 75\% chance to stay in the saddle (compared to 50\% for a riding saddle).
\parhead{Saddle, Pack} A pack saddle holds gear and supplies, but not a rider. It holds as much gear as the mount can carry.
\parhead{Saddle, Riding} The standard riding saddle supports a rider.

\subsection{Transport}
\parhead{Carriage} This four-wheeled vehicle can transport as many as four people within an enclosed cab, plus two drivers. In general, two horses (or other beasts of burden) draw it. A carriage comes with the harness needed to pull it.
\parhead{Cart} This two-wheeled vehicle can be drawn by a single horse (or other beast of burden). It comes with a harness.
\parhead{Galley} This three-masted ship has seventy oars on either side and requires a total crew of 200. A galley is 130 feet long and 20 feet wide, and it can carry 150 tons of cargo or 250 soldiers. For 8,000 gp more, it can be fitted with a ram and castles with firing platforms fore, aft, and amidships. This ship cannot make sea voyages and sticks to the coast. It moves about 4 miles per hour when being rowed or under sail.
\parhead{Keelboat} This 50- to 75 foot long ship is 15 to 20 feet wide and has a few oars to supplement its single mast with a square sail. It has a crew of eight to fifteen and can carry 40 to 50 tons of cargo or 100 soldiers. It can make sea voyages, as well as sail down rivers (thanks to its flat bottom). It moves about 1 mile per hour.
\parhead{Longship} This 75 foot long ship with forty oars requires a total crew of 50. It has a single mast and a square sail, and it can carry 50 tons of cargo or 120 soldiers. A longship can make sea voyages. It moves about 3 miles per hour when being rowed or under sail.
\parhead{Rowboat} This 8- to 12 foot long boat holds two or three Medium passengers. It moves about one and a half miles per hour.
\parhead{Sailing Ship} This larger, seaworthy ship is 75 to 90 feet long and 20 feet wide and has a crew of 20. It can carry 150 tons of cargo. It has square sails on its two masts and can make sea voyages. It moves about 2 miles per hour.
\parhead{Sled} This is a wagon on runners for moving through snow and over ice. In general, two horses (or other beasts of burden) draw it. A sled comes with the harness needed to pull it.
\parhead{Wagon} This is a four-wheeled, open vehicle for transporting heavy loads. In general, two horses (or other beasts of burden) draw it. A wagon comes with the harness needed to pull it.
\parhead{Warship} This 100 foot long ship has a single mast, although oars can also propel it. It has a crew of 60 to 80 rowers. This ship can carry 160 soldiers, but not for long distances, since there isn't room for supplies to support that many people. The warship cannot make sea voyages and sticks to the coast. It is not used for cargo. It moves about 2-1/2 miles per hour when being rowed or under sail.

\subsection{Spellcasting And Services}
Sometimes the best solution for a problem is to hire someone else to take care of it.
\parhead{Coach Cab} The price given is for a ride in a coach that transports people (and light cargo) between towns. For a ride in a cab that transports passengers within a city, 1 copper piece usually takes you anywhere you need to go.
\parhead{Hireling, Trained} The amount given is the typical daily wage for mercenary warriors, masons, craftsmen, scribes, teamsters, and other trained hirelings. This value represents a minimum wage; many such hirelings require significantly higher pay.
\parhead{Hireling, Untrained} The amount shown is the typical daily wage for laborers, porters, cooks, maids, and other menial workers.
\parhead{Messenger} This entry includes horse-riding messengers and runners. Those willing to carry a message to a place they were going anyway may ask for only half the indicated amount.
\parhead{Road or Gate Toll} A toll is sometimes charged to cross a well-trodden, well-kept, and well-guarded road to pay for patrols on it and for its upkeep. Occasionally, a large walled city charges a toll to enter or exit (or sometimes just to enter).
\parhead{Ship's Passage} Most ships do not specialize in passengers, but many have the capability to take a few along when transporting cargo. Double the given cost for creatures larger than Medium or creatures that are otherwise difficult to bring aboard a ship.
\parhead{Spell} The indicated amount is how much it costs to get a spellcaster to cast a spell for you. This cost assumes that you can go to the spellcaster and have the spell cast at his or her convenience. If you want to bring the spellcaster somewhere to cast a spell you need to negotiate with him or her, and the default answer is no.
\par The cost given is for a spell with no cost for a material component or focus component. If the spell includes a material component, add the cost of that component to the cost of the spell.
\par If the spell has a focus component (other than a divine focus), add 1/10 the cost of that focus to the cost of the spell.
\par Furthermore, if a spell has dangerous consequences, the spellcaster will certainly require proof that you can and will pay for dealing with any such consequences (that is, assuming that the spellcaster even agrees to cast such a spell, which isn't certain). In the case of spells that transport the caster and characters over a distance, you will likely have to pay for two castings of the spell, even if you aren't returning with the caster.
\par In addition, not every town or village has a spellcaster of sufficient level to cast any spell. In general, you must travel to a small town (or larger settlement) to be reasonably assured of finding a spellcaster capable of casting 1st-level spells, a large town for 2nd-level spells, a small city for 3rd- or 4th-level spells, a large city for 5th- or 6th-level spells, and a metropolis for 7th- or 8th-level spells. Even a metropolis isn't guaranteed to have a local spellcaster able to cast 9th-level spells. Additionally, while you may find spellcasters able to cast some spells, there is no guarantee that they are able to cast the spell you desire. In general, the more common the spell, the more likely you are to find a spellcaster able to cast it.
