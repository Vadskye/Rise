\chapter{Adventuring}

\section{Carrying Capacity}\label{Carrying Capacity}

    \begin{dtable}
        \lcaption{Carrying Capacity by Strength}
        \setlength{\tabcolsep}{4pt}
        \begin{dtabularx}{\columnwidth}{X l l l l}
            \tb{Strength} & \tb{Light} & \tb{Maximum} & \tb{Overloaded} & \tb{Push/Drag} \tableheaderrule
            -9 & 6 lb. & 12 lb. & 18 lb. & 60 lb. \\
            -8 & 7     & 14     & 21     & 70     \\
            -7 & 9     & 18     & 27     & 90     \\
            -6 & 12    & 24     & 36     & 120    \\
            -5 & 15    & 30     & 45     & 150    \\
            -4 & 20    & 40     & 60     & 200    \\
            -3 & 25    & 50     & 75     & 250    \\
            -2 & 30    & 60     & 90     & 300    \\
            -1 & 40    & 80     & 120    & 400    \\
            0  & 50    & 100    & 150    & 500    \\
            1  & 60    & 120    & 180    & 600    \\
            2  & 80    & 160    & 240    & 800    \\
            3  & 100   & 200    & 300    & 1,000  \\
            4  & 120   & 240    & 360    & 1,200  \\
            5  & 160   & 320    & 480    & 1,600  \\
            6  & 200   & 400    & 600    & 2,000  \\
            7  & 250   & 500    & 750    & 2,500  \\
            8  & 320   & 640    & 960    & 3,200  \\
            9  & 400   & 800    & 1,200  & 4,000  \\
            10 & 500   & 1,000  & 1,500  & 5,000  \\
            11 & 630   & 1,260  & 1,890  & 6,300  \\
            12 & 800   & 1,600  & 2,400  & 8,000  \\
            13 & 1,000 & 2,000  & 3,000  & 10,000 \\
            14 & 1,300 & 2,600  & 3,900  & 13,000 \\
            15 & 1,600 & 3,200  & 4,800  & 16,000 \\
            16 & 2,000 & 4,000  & 6,000  & 20,000 \\
            17 & 2,500 & 5,000  & 7,500  & 25,000 \\
            18 & 3,200 & 6,400  & 9,600  & 32,000 \\
            19 & 4,000 & 8,000  & 12,000 & 40,000 \\
            20 & 5,000 & 10,000 & 15,000 & 50,000 \\
            21\plus\fn{1} & \tdash & \tdash & \tdash & \tdash \\
        \end{dtabularx}
        1 To calculate the carrying capacity for a creature with epic Strength, double its carrying capacity every 3 Strength.
    \end{dtable}

    A creature's Strength determines how much weight it can carry, as shown in \trefnp{Carrying Capacity by Strength}.
    A creature can carry weight up to its light carrying capacity without any penalty.
    If it carries more than that, but less than its maximum carrying capacity, it increases its \glossterm{encumbrance} by 4.
    This stacks with the encumbrance from any armor the creature wears.

    \parhead{Lifting and Dragging} You can lift as much as your maximum carrying capacity over your head.

    You can lift as much as 1-1/2 your maximum carrying capacity off the ground (the sum of your light and maximum weight limits).
    While overloaded in this way, you increase your \glossterm{encumbrance} by 10, you take a \minus10 penalty to \glossterm{accuracy} with \glossterm{mundane} attacks, and you can only move by spending a \glossterm{standard action} to move 5 feet.
    This replaces the encumbrance from carrying more than your light carrying capacity.

    You can generally push or drag along the ground as much as five times your maximum carrying capacity.

    \parhead{Multi-Legged Creatures} The figures on \trefnp{Carrying Capacity by Strength} are for bipedal creatures. A creature with four or more legs can carry 50\% more weight than a bipedal creature of the same Strength.

    \parhead{Tremendous Strength} For Strength scores not shown on \trefnp{Carrying Capacity by Strength}, subtract 3 from its Strength until you find a Strength value shown on the chart. For each time you subtracted in this way, double the weight limits listed on the chart.

\section{Movement}

    \begin{dtable}
        \lcaption{Movement and Distance}
        \begin{dtabularx}{\columnwidth}{>{\lcol}X c c c c}
            & \multicolumn{4}{c}{\tdash\tdash\tdash Speed \tdash\tdash\tdash} \tableheaderrule
            & 15 feet & 20 feet & 30 feet & 40 feet \\
            One Round (Tactical) &  &  &  &  \\
            Walk & 15 ft. & 20 ft. & 30 ft. & 40 ft. \\
            Hustle & 30 ft. & 40 ft. & 60 ft. & 80 ft. \\
            One Minute (Local) &  &  &  &  \\
            Walk & 150 ft. & 200 ft. & 300 ft. & 400 ft. \\
            Hustle & 300 ft. & 400 ft. & 600 ft. & 800 ft. \\
            One Hour (Overland) &  &  &  &  \\
            Walk & 3/4 mile & 1 mile & 1-1/2 miles & 2 miles \\
            Hustle & 1-1/2 miles & 2 miles & 3 miles & 4 miles \\
            One Day (Overland) &  &  &  &  \\
            Walk & 7-1/2 miles & 10 miles & 15 miles & 20 miles \\
            Hustle & \tdash & \tdash & \tdash & \tdash \\
        \end{dtabularx}
    \end{dtable}

    \begin{dtable}
        \lcaption{Hampered Movement}
        \begin{dtabularx}{\columnwidth}{l >{\lcol}X >{\ccol}p{8em}}
            Condition & Example Extra Movement Cost \tableheaderrule
            Difficult terrain & Rubble, undergrowth, steep slope, ice, cracked and pitted surface, uneven floor & \mult2 \\
            Obstacle\fn{1} & Low wall, deadfall, broken pillar & \mult2 \\
            Poor visibility & Darkness or fog & \mult2 \\
            Impassable & Floor-to-ceiling wall, closed door, blocked passage & \tdash \\
        \end{dtabularx}
        1 May require a skill check
    \end{dtable}

    There are three movement scales in the game, as follows.
    \begin{itemize}
        \item Tactical, for combat, measured in feet (or squares) per round.
        \item Local, for exploring an area, measured in feet per minute.
        \item Overland, for getting from place to place, measured in miles per
            hour or miles per day.
    \end{itemize}

    \subsection{Tactical Movement}
        Use tactical movement for combat.

        \parhead{Minimum Movement} In some situations, your movement may be so hampered that you don't have sufficient speed even to move 5 feet (1 square). In such a case, you may use a standard action to move 5 feet (1 square) in any direction, even diagonally. (You can't take advantage of this rule to move through impassable terrain or to move when all movement is prohibited to you, such as while paralyzed.)

    \subsection{Local Movement}
        Characters exploring an area use local movement, measured in feet per minute.
        \parhead{Walk} A character can walk without a problem on the local scale.
        \parhead{Hustle} A character can hustle without a problem on the local scale. See \trefnp{Terrain and Overland Movement}, below, for movement measured in miles per hour.

    \subsection{Overland Movement}\label{Overland Movement}

        \begin{dtable}
            \lcaption{Terrain and Overland Movement}
            \begin{dtabularx}{\columnwidth}{>{\lcol}X c c c}
                \tb{Terrain}   & \tb{Highway} & \tb{Road or Trail} & \tb{Trackless} \tableheaderrule
                Desert, sandy  & \mult1       & \mult1/2           & \mult1/2 \\
                Forest         & \mult1       & \mult1             & \mult1/2 \\
                Hills          & \mult1       & \mult3/4           & \mult1/2 \\
                Jungle         & \mult1       & \mult3/4           & \mult1/4 \\
                Moor           & \mult1       & \mult1             & \mult3/4 \\
                Mountains      & \mult3/4     & \mult3/4           & \mult1/2 \\
                Plains         & \mult1-1/2   & \mult1             & \mult3/4 \\
                Swamp          & \mult1       & \mult3/4           & \mult1/2 \\
                Tundra, frozen & \mult1       & \mult3/4           & \mult3/4
            \end{dtabularx}
        \end{dtable}

        \begin{dtable}
            \lcaption{Mounts and Vehicles}
            \begin{dtabularx}{\columnwidth}{>{\lcol}X l l}
                \tb{Mount/Vehicle} & \tb{Per Hour} & \tb{Per Day} \tableheaderrule
                Mount (carrying load) &  &  \\
                \tind Light horse or light warhorse & 6 miles & 60 miles \\
                \tind Light horse (151-450 lb.)\fn{1} & 4 miles & 40 miles \\
                \tind Light warhorse (231-690 lb.)\fn{1} & 4 miles & 40 miles \\
                \tind Heavy horse or heavy warhorse & 5 miles & 50 miles \\
                \tind Heavy horse (201-600 lb.)\fn{1} & 3-1/2 miles & 35 miles \\
                \tind Heavy warhorse (301-900 lb.)\fn{1} & 3-1/2 miles & 35 miles \\
                \tind Pony or warpony & 4 miles & 40 miles \\
                \tind Pony (76-225 lb.)\fn{1} & 3 miles & 30 miles \\
                \tind Warpony (101-300 lb.)\fn{1} & 3 miles & 30 miles \\
                \tind Donkey or mule & 3 miles & 30 miles \\
                \tind Donkey (51-150 lb.)\fn{1} & 2 miles & 20 miles \\
                \tind Mule (231-690 lb.)\fn{1} & 2 miles & 20 miles \\
                \tind Dog, riding & 4 miles & 40 miles \\
                \tind Dog, riding (101-300 lb.)\fn{1} & 3 miles & 30 miles \\
                \tind Cart or wagon & 2 miles & 20 miles \\
                \tb{Ship} &  &  \\
                \tind Raft or barge (poled or towed)\fn{2} & 1/2 mile & 5 miles \\
                \tind Keelboat (rowed)\fn{2} & 1 mile & 10 miles \\
                \tind Rowboat (rowed)\fn{2} & 1-1/2 miles & 15 miles \\
                \tind Sailing ship (sailed) & 2 miles & 48 miles \\
                \tind Warship (sailed and rowed) & 2-1/2 miles & 60 miles \\
                \tind Longship (sailed and rowed) & 3 miles & 72 miles \\
                \tind Galley (rowed and sailed) & 4 miles & 96 miles \\
            \end{dtabularx}
            1 Quadrupeds, such as horses, can carry heavier loads than characters can. See \tref{Carrying Capacity by Strength}, for more information. \\
            2 Rafts, barges, keelboats, and rowboats are used on lakes and rivers.
            If going downstream, add the speed of the current (typically 3 miles per hour) to the speed of the vehicle. In addition to 10 hours of being rowed, the vehicle can also float an additional 14 hours, if someone can guide it, so add an additional 42 miles to the daily distance traveled. These vehicles can't be rowed against any significant current, but they can be pulled upstream by draft animals on the shores.
        \end{dtable}

        Characters covering long distances cross-country use overland movement. Overland movement is measured in miles per hour or miles per day. A day represents 10 hours of actual travel time. For rowed watercraft, a day represents 10 hours of rowing. For a sailing ship, it represents 24 hours.

        \parhead{Walk} A character can walk 10 hours in a day of travel without a problem. Walking for longer than that, or hustling faster than that, requires an Endurance check (see \pcref{Overland Exertion}).
        \parhead{Terrain} The terrain through which a character travels affects how much distance they can cover in an hour or a day (see \trefnp{Terrain and Overland Movement}).
        A highway is a straight, major, paved road.
        A road is typically a dirt track.
        A trail is like a road, except that it allows only single-file travel and does not benefit a party traveling with vehicles.
        Trackless terrain is a wild area with no significant paths.
        \parhead{Mounted Movement} A mount bearing a rider can move at a hustle. The damage it takes when doing so, however, is not subdual damage. The creature can also be ridden in a forced march, but its Constitution checks automatically fail, and, again, the damage it takes is lethal damage. Mounts also become fatigued when they take any damage from hustling or forced marches.

        See \trefnp{Mounts and Vehicles} for mounted speeds and speeds for vehicles pulled by draft animals.

        \parhead{Waterborne Movement} See \trefnp{Mounts and Vehicles} for speeds for water vehicles.

\section{Vision and Light}
    Some creatures have \glossterm{darkvision}, but most creatures need light to see by. 
    In an area of bright light, all characters can see clearly. A creature can't hide in an area of bright light unless it is invisible or has cover.

    In an area with shadowy illumination, creatures can see dimly.
    Creatures within this area have \concealment, which can allow them to make Stealth checks to hide (see \pcref{Stealth}).

    In areas of darkness, creatures without \glossterm{darkvision} or some other form of supernatural vision are \blinded.

    Characters with low-light vision (elves, gnomes, and half-elves) treat sources of light as if they had double their normal illumination range.

    Characters with \glossterm{darkvision} can see lit areas normally as well as dark areas within a radius defined by the ability -- usually, 50 feet. A creature can't hide within that range of a character using darkvision unless it is invisible or has cover. Darkvision does not function if the character is in bright light, and does not resume functioning until the end of the next round after the character leaves the area of bright light.

\section{Breaking Objects}
    There are two main ways of breaking objects.
    You can deal damage to objects with attacks, similarly to how you can deal damage to creatures.
    Alternately, you can attempt to sunder the object with sheer strength.

    \subsection{Damaging Objects}
        An object's size and primary material determines the number of \glossterm{vital wounds} it can suffer before being destroyed.
        The primary material it is constructed from determines its \glossterm{resistances}, and can modify the number of vital wounds it can take.
        An object breaks it takes \glossterm{vital wounds} in excess of its maximum.
        Objects do not have \glossterm{hit points} or a \glossterm{wound resistance}, cannot be \glossterm{bloodied}, and do not make \glossterm{vital rolls}.
        Objects are also not normally subject to \glossterm{critical hits}.

    \subsection{Sundering Objects}
        As a standard action, you can attempt to sunder an object you can touch.
        This requires two hands.
        An object's size and primary material determines the \glossterm{difficulty rating} of the check.
        Success means that the object breaks.
        Failure by 5 or less means you inflict a \glossterm{vital wound} on the object, but it does not break.
        Failure by 6 or more means nothing happens.

        \begin{dtable}
            \lcaption{Object Defenses by Size}
            \begin{dtabularx}{\textwidth}{l X X}
                \tb{Size}  & \tb{Max Vital Wounds} & \tb{Sunder Difficulty Rating} \tableheaderrule
                Fine       & 1                     & 1\fn{1} \\
                Diminutive & 1                     & 2       \\
                Tiny       & 2                     & 5       \\
                Small      & 4                     & 10      \\
                Medium     & 8                     & 15      \\
                Large      & 16                    & 20      \\
                Huge       & 32                    & 25      \\
                Gargantuan & 64                    & 30      \\
                Colossal   & 128                   & 35      \\
            \end{dtabularx}
            1. Extremely small objects may be difficult to grip effectively, which can significantly increase the difficulty to sunder them.
        \end{dtable}

        \begin{dtable}
            \lcaption{Object Defenses by Material}
            \begin{dtabularx}{\textwidth}{l X X X}
                \tb{Material}   & \tb{Vital Resistance} & \tb{Max Vital Wounds} & \tb{Sunder Modifier}  & \tableheaderrule
                Adamantine      & 30                    & \mult3                & \plus20              \\
                Glass           & 5                     & \mult1/2              & \tdash               \\
                Ice             & 1                     & \mult1/2              & \minus5              \\
                Iron or steel   & 12                    & \mult2                & \plus10              \\
                Leather or hide & 3                     & \tdash                & \tdash               \\
                Mithral         & 15                    & \mult2                & \plus10              \\
                Paper or cloth  & 1                     & \mult1/2              & \minus5              \\
                Rope            & 2                     & \tdash                & \tdash               \\
                Stone           & 8                     & \mult2                & \plus5               \\
                Wood            & 5                     & \tdash                & \tdash               \\
            \end{dtabularx}
        \end{dtable}

    \subsection{Broken and Destroyed Objects}
        An object that reaches its maximum vital wounds or is sundered becomes \glossterm{broken}.
        You can destroy an object by inflicting ten times its maximum \glossterm{vital wounds}, or by succeeding at a check to sunder the object by 20.

        \parhead{Broken Objects}\label{Broken Objects}
        Broken objects cannot be used for their intended purpose, but still retain enough of their original form to be repaired without too much work.
        For example, a broken wall lies in pieces on the ground and no longer blocks passage, but can be repaired with far less effort than would be required to create a wall from scratch.
        Magic items that are broken retain their magical properties once fixed.
        Broken (but not destroyed) objects can be repaired with the Craft skill for a cost equal to 10\% of their value (see \pcref{Craft}).

        \parhead{Destroyed Objects}\label{Destroyed Objects}
        Destroyed object have been damaged beyond hope of any sort of repair short of crafting the object again from raw materials.
        For example, a destroyed wall is reduced to dust or small, useless chunks of rubble.
        Magic items that are destroyed irrevocably lose their magical properties.
        The remains of a destroyed object generally occupy a space one size category smaller than the original object.

    \subsection{Relative Vital Resistances}\label{Relative Vital Resistances}
        When an object would take damage, if the \glossterm{vital resistance} of the attacking object or creature is lower than the \glossterm{vital resistance} of the defender, the attacking object or creature takes the damage instead.
        For example, if you try to break a stone wall with a wooden club, the club will break instead of the wall.
        % TODO: define hardness for creatures and their natural weapons; natural weapons should generally have higher hardness than creatures to avoid hardness reflection being common

\section{Wealth And Money}

    \subsection{Coins}
        The most common coin is the gold piece (gp). A gold piece is worth 10 silver pieces. Each silver piece is worth 10 copper pieces (cp). In addition to copper, silver, and gold coins, there are also platinum pieces (pp), which are each worth 10 gp.

        The standard coin weighs about a third of an ounce (fifty to the pound).

        \begin{dtable}
            \lcaption{Coin Exchange Values}
            \begin{dtabularx}{\columnwidth}{l c *{4}{>{\ccol}X}}
                & & \tb{CP} & \tb{SP} & \tb{GP} & \tb{PP} \tableheaderrule
                Copper piece (cp) & = & 1 & 1/10 & 1/100 & 1/1,000 \\
                Silver piece (sp) & = & 10 & 1 & 1/10 & 1/100 \\
                Gold piece (gp) & = & 100 & 10 & 1 & 1/10 \\
                Platinum piece (pp) & = & 1,000 & 100 & 10 & 1
            \end{dtabularx}
        \end{dtable}

    \subsection{Wealth Other Than Coins}
        Merchants commonly exchange trade goods without using currency. As a means of comparison, some trade goods are detailed below.

        \begin{dtable}
            \lcaption{Trade Goods}
            \begin{dtabularx}{\columnwidth}{l >{\lcol}X}
                \tb{Cost} & \tb{Item} \tableheaderrule
                1 cp & One pound of wheat \\
                2 cp & One pound of flour \\
                1 sp & One pound of iron, or one chicken \\
                5 sp & One pound of tobacco or copper \\
                1 gp & One pound of cinnamon, or one goat \\
                2 gp & One pound of ginger or pepper, or one sheep \\
                3 gp & One pig \\
                4 gp & One square yard of linen \\
                5 gp & One pound of salt or silver \\
                10 gp & One square yard of silk, or one cow \\
                15 gp & One pound of saffron or cloves, or one ox \\
                50 gp & One pound of gold \\
                500 gp & One pound of platinum
            \end{dtabularx}
        \end{dtable}

    \subsection{Selling Items}
        In general, a character can sell something for a quarter its listed price.

        Trade goods, such as gems, are the exception to this rule and can be sold for their full value.
        A trade good, in this sense, is a valuable good that can be easily exchanged almost as if it were cash itself.
