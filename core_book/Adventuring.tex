\chapter{Adventuring}

\section{Weight Limits}\label{Weight Limits}

    \subsection{Weight Categories}\label{Weight Categories}
        Weight is generally measured in \glossterm{weight categories} rather than pounds or kilograms.
        Weight categories use the same terms as \glossterm{size categories}, as shown in \tref{Weight Categories}.
        In general, a creature's weight category is the same as its size category.

        Objects and creatures can also be either \glossterm{lightweight} or \glossterm{heavyweight}.
        Lightweight objects and creatures have a weight category that is one category lighter than their size category.
        Heavyweight objects and creatures have a weight category that is one category heavier than their size category.

        Objects that occupy only a small percentage of the space appropriate for their size category, such as swords, are usually lightweight.
        Objects that fully occupy the space appropriate for their size category, like boulders, are usually heavyweight.

        \begin{dtable}
            \lcaption{Weight Categories}
            \begin{dtabularx}{\textwidth}{l X}
                \tb{Weight Category} & \tb{Average Weight} \tableheaderrule
                Fine        & 1 oz.       \\
                Diminuitive & 1/2 lb.     \\
                Tiny        & 2 lb.       \\
                Small       & 15 lb.      \\
                Medium      & 125 lb.     \\
                Large       & 1,000 lb.   \\
                Huge        & 8,000 lb.   \\
                Gargantuan  & 64,000 lb.  \\
                Colossal    & 512,000 lb. \\
            \end{dtabularx}
        \end{dtable}

    \begin{dtable}
        \lcaption{Weight Limits by Strength}
        \setlength{\tabcolsep}{4pt}
        \begin{dtabularx}{\columnwidth}{X X X}
            \tb{Strength} & \tb{Carrying Capacity} & \tb{Push/Drag} \tableheaderrule
            -9            & Diminuitive            & Tiny          \\
            -8            & Diminuitive x2         & Tiny x2       \\
            -7            & Diminuitive x4         & Tiny x4       \\
            -6            & Tiny                   & Small         \\
            -5            & Tiny x2                & Small x2      \\
            -4            & Tiny x4                & Small x4      \\
            -3            & Small                  & Medium        \\
            -2            & Small x2               & Medium x2     \\
            -1            & Small x4               & Medium x4     \\
            0             & Medium                 & Large         \\
            1 -- 2        & Medium x2              & Large x2      \\
            3 -- 4        & Medium x4              & Large x4      \\
            5 -- 6        & Large                  & Huge          \\
            7 -- 8        & Large x2               & Huge x2       \\
            9 -- 10       & Large x4               & Huge x4       \\
            11 -- 12      & Huge                   & Gargantuan    \\
            13 -- 14      & Huge x2                & Gargantuan x2 \\
            15 -- 16      & Huge x4                & Gargantuan x4 \\
            17 -- 18      & Gargantuan             & Colossal      \\
            19 -- 20      & Gargantuan x2          & Colossal x2   \\
            21 -- 22      & Gargantuan x4          & Colossal x4   \\
            23 -- 24      & Colossal               & Colossal x8   \\
            25 -- 26      & Colossal x2            & Colossal x16  \\
            27 -- 28      & Colossal x4            & Colossal x32  \\
            29\plus\fn{1} & \tdash                 & \tdash        \\
        \end{dtabularx}
        1 To calculate the weight limits for a creature with epic Strength, double the number of objects it can carry and drag for every 2 Strength beyond 30.
    \end{dtable}

    Your Strength determines how much you can carry or push, as shown in \trefnp{Weight Limits by Strength}.
    Your weight limits are measured in terms of how many objects or creatures of a given \glossterm{weight category} that you can carry or push at once.
    Instead of carrying one object of a given weight category, you can carry eight objects that are one weight category lighter.
    In general, it is not meaningful to consider the weight of any objects two weight categories lighter than your maximum weight category.

    You can carry objects or creatures up to your maximum carrying capacity without any penalty.
    This is called your \glossterm{carrying capacity}.
    Beyond that, you can push or drag objects or creatures up your pushing and dragging limit as a standard action.
    When you do, you move the weight 5 feet.

    \parhead{Multi-Legged Creatures} The figures on \trefnp{Weight Limits by Strength} are for bipedal creatures.
    A creature with four or more legs can carry, push, or drag twice as many objects as a bipedal creature of the same Strength.

\section{Movement}

    \begin{dtable}
        \lcaption{Movement and Distance}
        \begin{dtabularx}{\columnwidth}{>{\lcol}X c c c c}
            & \multicolumn{4}{c}{\tdash\tdash\tdash Speed \tdash\tdash\tdash} \tableheaderrule
                                 & 15 feet     & 20 feet  & 30 feet     & 40 feet  \\
            One Round (Tactical) &             &          &             &          \\
            Walk                 & 15 ft.      & 20 ft.   & 30 ft.      & 40 ft.   \\
            Hustle               & 30 ft.      & 40 ft.   & 60 ft.      & 80 ft.   \\
            One Minute (Local)   &             &          &             &          \\
            Walk                 & 150 ft.     & 200 ft.  & 300 ft.     & 400 ft.  \\
            Hustle               & 300 ft.     & 400 ft.  & 600 ft.     & 800 ft.  \\
            One Hour (Overland)  &             &          &             &          \\
            Walk                 & 3/4 mile    & 1 mile   & 1-1/2 miles & 2 miles  \\
            Hustle               & 1-1/2 miles & 2 miles  & 3 miles     & 4 miles  \\
            One Day (Overland)   &             &          &             &          \\
            Walk                 & 7-1/2 miles & 10 miles & 15 miles    & 20 miles \\
            Hustle               & \tdash      & \tdash   & \tdash      & \tdash   \\
        \end{dtabularx}
    \end{dtable}

    \begin{dtable}
        \lcaption{Hampered Movement}
        \begin{dtabularx}{\columnwidth}{l >{\lcol}X >{\ccol}p{8em}}
            \tb{Condition} & \tb{Example} & \tb{Extra Movement Cost} \tableheaderrule
            Difficult terrain & Rubble, undergrowth, steep slope, ice, cracked and pitted surface, uneven floor & \mult2 \\
            Obstacle\fn{1} & Low wall, deadfall, broken pillar & \mult2 \\
            Poor visibility & Darkness or fog & \mult2 \\
            Impassable & Floor-to-ceiling wall, closed door, blocked passage & \tdash \\
        \end{dtabularx}
        1 May require a skill check
    \end{dtable}

    There are three movement scales in the game, as follows.
    \begin{itemize}
        \item Tactical, for combat, measured in feet (or squares) per round.
        \item Local, for exploring an area, measured in feet per minute.
        \item Overland, for getting from place to place, measured in miles per
            hour or miles per day.
    \end{itemize}

    \subsection{Tactical Movement}
        Use tactical movement for combat.

        \parhead{Minimum Movement} In some situations, your movement may be so hampered that you don't have sufficient speed even to move 5 feet (1 square). In such a case, you may use a standard action to move 5 feet (1 square) in any direction, even diagonally. (You can't take advantage of this rule to move through impassable terrain or to move when all movement is prohibited to you, such as while paralyzed.)

    \subsection{Local Movement}
        Characters exploring an area use local movement, measured in feet per minute.
        \parhead{Walk} A character can walk without a problem on the local scale.
        \parhead{Hustle} A character can hustle without a problem on the local scale. See \trefnp{Terrain and Overland Movement}, below, for movement measured in miles per hour.

    \subsection{Overland Movement}\label{Overland Movement}

        \begin{dtable}
            \lcaption{Terrain and Overland Movement}
            \begin{dtabularx}{\columnwidth}{>{\lcol}X c c c}
                \tb{Terrain}   & \tb{Highway} & \tb{Road or Trail} & \tb{Trackless} \tableheaderrule
                Desert, sandy  & \mult1       & \mult1/2           & \mult1/2 \\
                Forest         & \mult1       & \mult1             & \mult1/2 \\
                Hills          & \mult1       & \mult3/4           & \mult1/2 \\
                Jungle         & \mult1       & \mult3/4           & \mult1/4 \\
                Moor           & \mult1       & \mult1             & \mult3/4 \\
                Mountains      & \mult3/4     & \mult3/4           & \mult1/2 \\
                Plains         & \mult1-1/2   & \mult1             & \mult3/4 \\
                Swamp          & \mult1       & \mult3/4           & \mult1/2 \\
                Tundra, frozen & \mult1       & \mult3/4           & \mult3/4
            \end{dtabularx}
        \end{dtable}

        \begin{dtable}
            \lcaption{Mounts and Vehicles}
            \begin{dtabularx}{\columnwidth}{>{\lcol}X l l}
                \tb{Mount/Vehicle} & \tb{Per Hour} & \tb{Per Day} \tableheaderrule
                Mount (carrying load) &  &  \\
                \tind Light horse or light warhorse & 6 miles & 60 miles \\
                \tind Light horse & 4 miles & 40 miles \\
                \tind Light warhorse & 4 miles & 40 miles \\
                \tind Heavy horse or heavy warhorse & 5 miles & 50 miles \\
                \tind Heavy horse & 3-1/2 miles & 35 miles \\
                \tind Heavy warhorse & 3-1/2 miles & 35 miles \\
                \tind Pony or warpony & 4 miles & 40 miles \\
                \tind Pony & 3 miles & 30 miles \\
                \tind Warpony & 3 miles & 30 miles \\
                \tind Donkey or mule & 3 miles & 30 miles \\
                \tind Donkey & 2 miles & 20 miles \\
                \tind Mule & 2 miles & 20 miles \\
                \tind Dog, riding & 4 miles & 40 miles \\
                \tind Dog, riding & 3 miles & 30 miles \\
                \tind Cart or wagon & 2 miles & 20 miles \\
                \tb{Ship} &  &  \\
                \tind Raft or barge (poled or towed)\fn{1} & 1/2 mile & 5 miles \\
                \tind Keelboat (rowed)\fn{1} & 1 mile & 10 miles \\
                \tind Rowboat (rowed)\fn{1} & 1-1/2 miles & 15 miles \\
                \tind Sailing ship (sailed) & 2 miles & 48 miles \\
                \tind Warship (sailed and rowed) & 2-1/2 miles & 60 miles \\
                \tind Longship (sailed and rowed) & 3 miles & 72 miles \\
                \tind Galley (rowed and sailed) & 4 miles & 96 miles \\
            \end{dtabularx}
            1 Rafts, barges, keelboats, and rowboats are used on lakes and rivers.
            If going downstream, add the speed of the current (typically 3 miles per hour) to the speed of the vehicle. In addition to 10 hours of being rowed, the vehicle can also float an additional 14 hours, if someone can guide it, so add an additional 42 miles to the daily distance traveled. These vehicles can't be rowed against any significant current, but they can be pulled upstream by draft animals on the shores.
        \end{dtable}

        Characters covering long distances cross-country use overland movement. Overland movement is measured in miles per hour or miles per day. A day represents 10 hours of actual travel time. For rowed watercraft, a day represents 10 hours of rowing. For a sailing ship, it represents 24 hours.

        \parhead{Walk} A character can walk 10 hours in a day of travel without a problem. Walking for longer than that, or hustling faster than that, requires an Endurance check (see \pcref{Overland Exertion}).
        \parhead{Terrain} The terrain through which a character travels affects how much distance they can cover in an hour or a day (see \trefnp{Terrain and Overland Movement}).
        A highway is a straight, major, paved road.
        A road is typically a dirt track.
        A trail is like a road, except that it allows only single-file travel and does not benefit a party traveling with vehicles.
        Trackless terrain is a wild area with no significant paths.
        \parhead{Mounted Movement} A mount bearing a rider can move at a hustle. The damage it takes when doing so, however, is not subdual damage. The creature can also be ridden in a forced march, but its Constitution checks automatically fail, and, again, the damage it takes is lethal damage.
        % TODO: does this makes sense?
        % Mounts also become fatigued when they take any damage from hustling or forced marches.

        See \trefnp{Mounts and Vehicles} for mounted speeds and speeds for vehicles pulled by draft animals.

        \parhead{Waterborne Movement} See \trefnp{Mounts and Vehicles} for speeds for water vehicles.

\section{Vision and Light}\label{Vision and Light}
    Some creatures have \trait{darkvision} or other extraordinary senses, but most creatures need light to see by. 
    In an area of \glossterm{bright illumination}, all characters can see clearly.

    Creatures can see only dimly into areas that have \glossterm{shadowy illumination}.
    Everything in the area has \glossterm{concealment}.
    This allows creatures in the area to make Stealth checks to hide even if they don't have \glossterm{cover} (see \pcref{Stealth}).

    In an area with \glossterm{brilliant illumination}, creatures can see clearly just like an area with bright illumination.
    In addition, no shadows exist within an an area of brilliant illumination.
    This makes many effects from the \sphere{umbramancy} mystic sphere difficult or impossible to use.

    In areas of total darkness, creatures without \glossterm{darkvision} or some other form of supernatural vision are \blinded.

    \subsection{Attacking Unseen Foes}
        You can make attacks against creatures and objects you cannot see.
        To do so, you choose a 5-foot square and make the attack against that square.
        You have a 50\% chance to hit nothing at all with the attack and a 50\% chance to hit a random valid target in that square with your attack.

\section{Communication and Languages}\label{Languages}\label{Communication and Languages}

    \parhead{Literacy}
    All characters with an Intelligence of \minus2 or higher are presumed to be literate, allowing them to read and write any language they speak. Each language has an alphabet, though sometimes several spoken languages share a single alphabet.

    \parhead{Language Rarity}\label{Language Rarity}
    Some languages are widely spoken in the world, while others are only encountered in unusual circumstances.
    Common languages are summarized on \trefnp{Common Languages}, below.
    Rare languages are summarized on \trefnp{Rare Languages}, below.
    Rare languages are more difficult to learn, and are usually only spoken by unusual creatures.

    \parhead{Learning Languages}\label{Learning Languages}
    You can spend one \glossterm{insight point} to learn two \glossterm{common languages} or one \glossterm{rare language}.
    In addition, you can learn two common languages or one rare language by mastering the Linguistics skill (see \pcref{Linguistics}).

    \begin{dtable}
        \lcaption{Common Languages}
        \begin{dtabularx}{\columnwidth}{l >{\lcol}X l}
            \tb{Language} & \tb{Typical Speakers} & \tb{Alphabet} \tableheaderrule
            Common        & Civilized creatures   & Common   \\
            Draconic      & Dragons, kobolds      & Draconic \\
            Dwarven       & Dwarves               & Dwarven  \\
            Elven         & Elves                 & Elven    \\
            Giant         & Ogres, giants         & Dwarven  \\
            Gnoll         & Gnolls                & Common   \\
            Gnome         & Gnomes                & Dwarven  \\
            Goblin        & Goblins, hobgoblins   & Dwarven  \\
            Halfling      & Halflings             & Common   \\
            Orc           & Orcs                  & Dwarven  \\
        \end{dtabularx}
    \end{dtable}

    \begin{dtable}
        \lcaption{Rare Languages}
        \begin{dtabularx}{\columnwidth}{l >{\lcol}X l}
            \tb{Language}  & \tb{Typical Speakers}  & \tb{Alphabet} \tableheaderrule
            Abyssal     & Evil planeforged      & Abyssal  \\
            Aquan       & Water-based creatures & Elemental \\
            Auran       & Air-based creatures   & Elemental \\
            Celestial   & Good planeforged      & Celestial \\
            Ignan       & Fire-based creatures  & Elemental \\
            Sylvan      & Dryads, faeries       & Elven     \\
            Terran      & Earth-based creatures & Elemental \\
            Undercommon & Drow                  & Elven
        \end{dtabularx}
    \end{dtable}

    \subsection{Telepathy}\label{Telepathy}
        Some creatures have the ability to telepathically communicate with other creatures.
        All telepathy abilities have a defined \glossterm{range}.
        Unless otherwise specified, a telepathic creature can only communicate with one creature at a time.

        As a \glossterm{free action}, a telepathic creature can open a telepathic communication channel with one creature it sees within the range of its telepathy ability.
        The target does not have to be willing to receive telepathic communication in this way.
        While this channel is open, the telepathic creature can cause the target to ``hear'' the telepathic creature's voice inside the target's head.
        If the target attempts to mentally reply while the channel is open, the telepathic creature can similarly ``hear'' the reply in its head as if the target was speaking.
        This does not generally grant the ability to detect any other thoughts, though exceptionally stupid targets may accidentally broadcast their private thoughts.

        Telepathic communication uses words, so it still requires a shared language to be intelligible, even though the words are only imagined.
        A telepathic creature may attempt to telepathically communicate with creatures without a language, though this is generally unproductive.
        A skilled telepath can customize the mental ``voice'' it projects in the same way that a creature can attempt to disguise or alter its voice when speaking.

\section{Breaking Objects}
    There are two main ways of breaking objects.
    You can deal damage to objects with attacks, similarly to how you can deal damage to creatures.
    Alternately, you can attempt to sunder the object with sheer strength.

    \subsection{Damaging Objects}
        Objects have \glossterm{hit points} and \glossterm{damage resistance} like creatures.
        However, they treat all damage they take as \glossterm{environmental damage} (see \pcref{Environmental Damage}).
        That means that all damage they take is reduced by their \glossterm{damage resistance} without subtracting from the remaining value of their damage resistance.

        An object becomes \glossterm{broken} if its \glossterm{hit points} are reduced to 0 (see \pcref{Broken and Destroyed Objects}).
        Objects cannot gain \glossterm{vital wounds}.
        Objects are also not normally subject to \glossterm{critical hits}.

    \subsection{Object Statistics}
        An object's size primarily influences the number of \glossterm{hit points} it has.
        The primary material it is constructed from determines its \glossterm{damage resistance}, and can modify the number of hit points it has.
        Details are given in \tref{Object Statistics By Size} and \tref{Object Statistics By Material}.

        \begin{dtable}
            \lcaption{Object Statistics By Size}
            \begin{dtabularx}{\textwidth}{l X X}
                \tb{Size}  & \tb{Hit Points} & \tb{Sunder Difficulty Value} \tableheaderrule
                Fine       & 1               & 1\fn{1} \\
                Diminutive & 2               & 2       \\
                Tiny       & 5               & 5       \\
                Small      & 10              & 10      \\
                Medium     & 20              & 15      \\
                Large      & 50              & 20      \\
                Huge       & 100             & 25      \\
                Gargantuan & 200             & 30      \\
                Colossal   & 500             & 35      \\
            \end{dtabularx}
            1. Extremely small objects may be difficult to grip effectively, which can significantly increase the difficulty to sunder them.
        \end{dtable}

        \begin{dtable}
            \lcaption{Object Statistics By Material}
            \begin{dtabularx}{\textwidth}{l X X X}
                \tb{Material}   & \tb{DR}\fn{1} & \tb{Hit Points Multiplier}\fn{2} & \tb{Sunder Difficulty Value Modifier}  \tableheaderrule
                Adamantine      & 30                    & \mult3                & \plus20              \\
                Glass           & 5                     & \mult1/2              & \tdash               \\
                Ice             & 1                     & \mult1/2              & \minus5              \\
                Iron or steel   & 12                    & \mult2                & \plus10              \\
                Leather or hide & 3                     & \tdash                & \tdash               \\
                Mithral         & 15                    & \mult2                & \plus10              \\
                Paper or cloth  & 1                     & \mult1/2              & \minus5              \\
                Rope            & 2                     & \tdash                & \tdash               \\
                Stone           & 8                     & \mult2                & \plus5               \\
                Wood            & 5                     & \tdash                & \tdash               \\
            \end{dtabularx}
            1. See \pcref{Damage Resistance}. \\
            2. Any value here modifies the number of hit points the object would normally have based on its size.
        \end{dtable}

    \subsection{Sundering Objects}
        As a standard action, you can attempt to sunder an object you can touch.
        This requires two hands.
        An object's size and primary material determines the \glossterm{difficulty value} of the check.
        The \glossterm{difficulty value} of this check decreases by 2 if the object is below its maximum \glossterm{hit points}.
        Success means that the object breaks.
        Failure by 5 or less means the object loses a \glossterm{hit point}, but it does not break.
        Failure by 6 or more means nothing happens.

    \subsection{Broken and Destroyed Objects}\label{Broken and Destroyed Objects}
        An object that is reduced to 0 \glossterm{hit points} becomes \glossterm{broken}.
        You can destroy an object by causing it to lose additional hit points equal to ten times its maximum hit points, or by succeeding at a check to sunder the object by 20.

        \parhead{Broken Objects}\label{Broken Objects}
        Broken objects cannot be used for their intended purpose, but still retain enough of their original form to be repaired without too much work.
        For example, a broken wall lies in pieces on the ground and no longer blocks passage, but can be repaired with far less effort than would be required to create a wall from scratch.
        Magic items that are broken retain their magical properties once fixed.
        Broken (but not destroyed) objects can be repaired with the Craft skill for a cost equal to 10\% of their value (see \pcref{Craft}).

        \parhead{Destroyed Objects}\label{Destroyed Objects}
        Destroyed object have been damaged beyond hope of any sort of repair short of crafting the object again from raw materials.
        For example, a destroyed wall is reduced to dust or small, useless chunks of rubble.
        Magic items that are destroyed irrevocably lose their magical properties.
        The remains of a destroyed object generally occupy a space one size category smaller than the original object.

    \subsection{Relative Damage Resistance}\label{Relative Damage Resistance}
        When an object would take damage from a \glossterm{strike}, if the \glossterm{damage resistance} of the attacking object or creature is lower than the damage resistance of the defender, the attacking object or creature takes the damage instead.
        For example, if you try to break a stone wall with a wooden club, the club will break instead of the wall.
        % TODO: define hardness for creatures and their natural weapons; natural weapons should generally have higher hardness than creatures to avoid hardness reflection being common

\section{Poison}\label{Poison}
    Poisons can deal damage, weaken creatures, or even kill them.
    Some effects which are not literally poisonous, such as animal venom or fungal spores, are considered poisons.
    Unless otherwise noted, poisons are not \glossterm{conditions}, and cannot be removed by abilities that remove conditions (see \pcref{Conditions}).
    Common poisons are listed in \pcref{Poisons}.

    \subsection{Poison Transmission}\label{Poison Transmission}\label{Transmission}

        There are three ways that poisons can be contracted.

        \parhead{Contact} A contact poison affects any creature that touches it with bare skin.
        \parhead{Ingestion} An ingestion poison affects any creature that eats, drinks, or breathes it, depending on the type of poison.
        Ingestion poisons have no effect when touched or used to coat weapons.
        \parhead{Injury} An injury poison affects any creature loses \glossterm{hit points} from something bearing the poison.
        Almost all injury poisons take liquid form, and are typically used to coat weapons.

    \subsection{Poison Forms}\label{Poison Forms}

        There are four forms of poison.

        \parhead{Gas} Gaseous poisons are difficult to store, but easy to affect foes with.
        \parhead{Liquid} Liquid poisons are the most common type of poison.
        Liquid poisons can be used to coat weapons, slipped into food, or simply thrown at foes.
        A dose of a liquid poison is usually about one ounce of the poison.
        \parhead{Pellet} Some rare poisons come in small, solid pellets or cubes.
        Typically, these pellets contain a powerful liquid poison that becomes inert quickly after being exposed.
        Pellet poisons cannot be used to coat weapons or thrown at foes, but can be slipped into food.
        \parhead{Powder} Poison in powder form cannot be used to coat weapons, but can be slipped into food or thrown at foes.

    \subsection{Poison Effects}\label{Poison Effects}

        Poisons can have a wide variety of effects, as determined by the type of poison used.
        However, most poison share certain common properties.

        \parhead{Becoming Poisoned}
        All poisons have an base \glossterm{accuracy}.
        When a creature first comes into contact with a poison, the poison makes an attack roll using its accuracy against the Fortitude defense of the poisoned creature.
        On a hit, the target becomes \glossterm{poisoned} and suffers the effects of the first stage of the poison.
        On a critical hit, the target becomes \glossterm{poisoned} and suffers the effects of the two stages of the poison.
        On a miss, the target is not \glossterm{poisoned}.

        Some attacks make the target poisoned if they hit the target.
        In that case, the ability's accuracy defines the poison's accuracy.

        Many poisons have an additional effect when they hit the target for the third time.

        \parhead{Poison Attacks}
        At the end of each subsequent round after the target becomes poisoned, the poison makes an attack roll against the Fortitude defense of the poisoned creature.
        Each hit increases the \glossterm{poison stage} of the poison.
        For every 10 points by which the attack hits, the poison progresses by an additional stage.
        On a miss, the creature gets closer to resisting the poison (see Resisting Poisons, below).

        \parhead{Resisting Poisons}
        If a poison misses a creature three times with its attack at the end of each round, the creature stops being poisoned by that poison.
        % Some poisons may require more attacks to end the poison, as indicated in their description.

        \parhead{Multiple Doses}
        A creature can be affected by multiple doses of the same poison.
        This does not cause the same effect to occur multiple times.
        However, each extra dose increases the accuracy of the poison by 1, up to a maximum bonus of \plus10 more than the poison's normal accuracy.

        A poison is considered the same if it has the same name and comes from the same source.
        For example, a creature bitten multiple times by the same giant spider suffers multiple doses of the same poison.
        A creature bitten multiple times by different giant spiders considers each spider's poison separately.

        \parhead{Poison Quality} Some poisons are unusually high or low quality.

    \subsection{Creating Poisons}\label{Creating Poisons}

        You can use the Craft (poison) skill to create poisons.
        To create a poison, you must make a Craft (poison) check against a \glossterm{difficulty value} equal to 10 \add the poison's base accuracy.
        For every 2 points by which you beat this \glossterm{difficulty value}, the created poison's accuracy gains a \plus1 bonus, up to a maximum bonus of \plus10 more than the poison's base accuracy.

        Creating a poison requires special materials.
        The type of materials required, and how those materials can be acquired, depend on the type of poison.

        \begin{itemize}
            \itemhead{Plant} Plant-based poisons can typically be harvested by making a Survival check to search in appropriate terrain.
                The \glossterm{difficulty value} of this check is usually equal to 10 \add the base accuracy of the poison.
            \itemhead{Venom} Venom requires an appropriate body part from a creature -- often, poison it naturally produces.
            \itemhead{Alchemical} Alchemical poisons require alchemical materials.
                These cannot normally be found in nature.
                In unusual circumstances, these components can be synthesized from natural chemicals or magical materials with a Craft (alchemy) check equal to 10 \add the base accuracy of the poison.
        \end{itemize}

% Is this the right location?
\section{Planes}\label{Planes}
    The universe of Rise is divided into \glossterm{planes}.
    A plane is a distinct realm of existence.
    Except for the connections between planes through \glossterm{planar rifts}, each plane is effectively an isolated universe, and different planes can obey different fundamental laws.
    For example, the Material Plane has gravity that exerts a consistent acceleration in a single absolute direction.
    However, the Astral Plane has subjective gravity, where each creature on the plane chooses the direction that gravity pulls it in, if any.

    \subsection{General Cosmology}
        The planes of Rise are divided up into groups.

        \parhead{Primal Planes} The primal planes are manifestations of the basic building blocks of the universe.
        Each plane in this group is predominantly composed of a single element or type of energy.
        There are four primal planes: Air, Earth, Fire, and Water.

        \parhead{Aligned Planes} The aligned planes are manifestations of the nine alignments that define the morality of the universe.
        Each plane in this group is strongly associated with a particular alignment.
        The souls of creatures with the corresponding alignment often spend their afterlife in the Aligned Planes.
        There are nine aligned planes, one for each alignment combination (see \pcref{Alignment}).

        \parhead{Nexus Planes} The nexus planes are composite planes with a number of distinct environments and filled with creatures of myriad alignments.
        Nexus planes comprise the majority of civilization across all planes.
        They do not have their own unique planar essence, and no planar creatures are native to nexus planes.
        There are two nexus planes: the Material Plane and the Astral Plane.

        \parhead{Demiplanes} These planes are small, fragmentary realms that are greatly limited in their scope.
        There is no specific list of demiplanes, and they share few common properties.
        Most demiplanes were created for particular purposes by beings of great power, though some simply came into existence through unknown means.

    \subsection{Planar Rifts}\label{Planar Rifts}
        Normally, there are boundaries between different planes that prevent direct passage between them.
        However, \glossterm{planar rifts} are places where these boundaries have weakened, making interplanar travel easier.
        A planar rift joins a specific location on one plane to a specific location on a different plane.
        Most planar rifts lead to and from the Astral Plane, which is the space between the other planes (see \pcref{The Astral Plane}).

        Most planar rifts still require the use of magic, such as the \spell{plane shift} ritual, to actually cross between planes.
        Some especially large rifts enable physical travel between planes without the use of any magic.

    \subsection{Planar Traits}
        \subsubsection{Gravity Direction}
            The direction of gravity on a plane can take one of the following forms:
            \begin{itemize}
                \item Fixed Gravity: Gravity points in a fixed direction and with a fixed strength at all locations on the plane.
                    Almost all planes with a fixed gravity have a perfectly flat surface.
                \item Absolute Directional Gravity: Gravity points in a consistent direction according to a rule that applies equally to everything on the plane, but which is not in a fixed direction.
                    For example, a plane filled with floating spheres where gravity always points towards the closest sphere has absolute directional gravity.
                \item Subjective Gravity: Each creature on the plane chooses the direction of gravity for that creature.
                    The plane has no gravity for unattended objects and nonsentient creatures.
                    A creature on the plane can make use the \textit{control gravity} ability as a \glossterm{minor action}.
                    \begin{freeability}{Control Gravity}
                        Make a Willpower check with a \glossterm{difficulty value} of 10.
                        Success means that you choose the direction of gravity that applies to you on the current plane.
                        Alternately, you can choose for gravity to not apply to you.

                        Failure means you gain a \plus2 bonus to the next \textit{control gravity} ability you use on this plane.
                        This bonus stacks with itself and lasts until you succeed at a \textit{control gravity} ability on this plane.
                    \end{freeability}
            \end{itemize}

        \subsubsection{Gravity Strength} The strength of gravity on a plane can take one of the following forms:
            \begin{itemize}
                \item Normal Gravity: Gravity is about the strength of Earth.
                \item No Gravity: There is no gravity on the plane.
                    The \glossterm{range limits} of ranged weapons are quadrupled.
                    % TODO: what additional effects are there?
                \item Light Gravity: Gravity is about half the strength of Earth.
                    % Should there be check penalties?
                    The weight of all items is halved.
                    The \glossterm{range limits} of ranged weapons are doubled.
                \item Heavy Gravity: Gravity is about twice the strength of Earth.
                    Creatures take a \minus2 penalty to Strength and Dexterity-based checks.
                    The weight of all items is doubled.
                    The \glossterm{range limits} of ranged weapons are halved, to a minimum of 5 feet.
                    % TODO: falling damage
                \item Extreme Gravity: Gravity is about four times the strength of Earth.
                    Creatures take a \minus4 penalty to Strength and Dexterity-based checks.
                    The weight of all items is quadrupled.
                    The \glossterm{range limits} of ranged weapons are reduced to one quarter of the normal value, to a minimum of 5 feet.
                    % TODO: falling damage
            \end{itemize}

        \subsubsection{Light} Various planes are illuminated in different ways.
            \begin{itemize}
                \item Fixed Source: There is a single constant source of light on the plane.
                \item Mobile Source: There is a single source of light on the plane that moves around it, illuminating different parts of the plane at different times.
                \item None: There is no natural source of light on the plane.
                    Other sources of light, such as torches, function normal.
            \end{itemize}

        \subsubsection{Limits} The behavior of a plane at its limits can vary widely.
            Some planes have different behaviors at different limits depending on their shape.
            \begin{itemize}
                \item Astral Gate: If you reach the limits of the plane, you find a planar gate to the Astral Plane.
                \item Barrier: If you reach the limits of the plane, you find an impassable barrier.
                    The barrier takes the form of a substance relevant to the plane's nature.
                    It may be possible to dig tunnels into the barrier to some depth, but there is nothing behind the barrier.
                    As you progress past the limit, the barrier becomes increasingly difficult to break through, and eventually it becomes completely impenetrable.
                \item Looped: If you go beyond the limits of the plane, you wrap around to the opposite side of the plane.
                    There is no obvious transition point or perception of transportation when this occurs - the shape of the plane simply connects to itself.
                    On very small planes, this can allow you to see your own back, though looped planes of that size are rare.
                \item Infinite: The plane has no limits. This is extremely rare.
            \end{itemize}

        \subsubsection{Planar Connectivity}
            Different planes have different degrees of connection to other planes.
            \begin{itemize}
                \item Isolated: The plane is difficult to reach or leave.
                    It has no permanent \glossterm{planar rifts}, and temporary rifts are rare or nonexistent.
                \item Stable Connected: The plane has multiple permanent \glossterm{planar rifts}.
                    However, temporary rifts are rare.
                \item Unstable Connected: The plane has no permanent \glossterm{planar rifts}, but temporary rifts are common.
                \item Conduit: The plane has a large number of permanent \glossterm{planar rifts}, and temporary rifts are common.
            \end{itemize}

        \subsubsection{Shape} The shape of a plane defines the shape of its core surface, and what happens if you travel beyond that surface.

            \begin{itemize}
                \item Flat Surface: The plane consists of a flat surface generally made of earth or similar material.
                    Most activity and civilization on the plane happens on this surface.
                    It is usually possible to construct tunnels into a flat surface plane to some depth, depending on the size of the plane.
                \item Hollow Sphere: The plane consists of a hollow sphere with an outer boundary generally made of earth or similar material.
                    Most activity and civilization on the plane happens on the inner surface of the sphere.
                    Some hollow sphere planes have an outer surface that can also be accessed, but most have a limit before any outer surface can be reached.
                \item Solid Sphere: The plane consists of a solid sphere generally made of earth or similar material.
                    Most activity and civilization on the plane happens on the surface of the sphere.
                    It is possible to construct tunnels into a solid sphere plane, but it may become increasingly difficult to traverse the plane as you approach the center of the sphere.
                    In general, the limit of a solid sphere plane is located at ten times the radius of the plane's primary sphere.
                \item Uniform: The plane has no well-defined surface or ground layer.
                    Some uniform planes have no ground or solid obstacles, while others are composed almost entirely of ground and firmament.
                    Uniform planes almost always still have limits of some kind.
            \end{itemize}

            % TODO: morphicness, alignment, magic

    \subsection{Plane Descriptions}

        \subsubsection{Primal Planes}

            \parhead{The Plane of Air}
            The Plane of Air is a a soaring landscape unencumbered by gravity or ground.
            The vast expanses of empty air are littered with clouds and unpredictable winds.
            Any inhabitants of the plane must adapt to a highly mobile lifestyle.
            A number of towns and structures have been built in the plane using raw materials brought from other planes.
            They sail through the air at the whims of the wind, and are occasionally battered by intersections with other wind streams.

            The Plane of Air has the following planar traits:
            \begin{itemize}
                \item Gravity strength: No Gravity
                \item Light: Fixed Source, from a sun outside the limits of the plane
                \item Limits: Looped
                \item Planar connectivity: Unstable Connected
                \item Shape: Uniform, in a sphere with a radius of about 2,000 miles.
            \end{itemize}

            \parhead{The Plane of Earth}
            The Plane of Earth is a titanically large body of earth and stone.
            A labyrinthine series of mostly airless tunnels weave their way through the plane, connecting the few cities.
            The plane is a major source of valuable gems, diamonds, and rare metals like mithral, but the dense rock and airless environment make successful mining difficult.
            In addition, earthquakes periodically reshape the environment by collapsing old tunnel systems and constructing new ones.
            Some cities have been carved out in vast underground rooms reinforced to survive the earthquakes.

            The Plane of Earth has the following planar traits:
            \begin{itemize}
                \item Gravity direction: Fixed
                \item Gravity strength: Normal
                \item Light: None
                \item Limits: Barrier, formed from increasingly dense rock that eventually becomes so hard that no known material or magic can damage it.
                \item Planar connectivity: Stable Connected
                \item Shape: Uniform, in a sphere with a radius of about 500 miles.
            \end{itemize}

            \parhead{The Plane of Fire}
            The Plane of Fire is an endless searing inferno.
            The plane's essence is highly combustible, allowing fires to burn indefinitely without any obvious fuel.
            However, the intensity of flames on the plane are highly uneven, as the plane generates fuel in various locations that shift over time.
            Some pockets on the surface are devoid of natural fuel, allowing the allow the construction of trading hubs where the few inhabitants of the plane who are not naturally immune to fire can survive.
            A variety of large tunnels and magma flows run through the sphere, and the intensity of the heat generally increases as you approach the center.

            The Plane of Fire has the following planar traits:
            \begin{itemize}
                \item Gravity direction: Absolute Directional, pointing to the center of the sphere
                \item Gravity strength: Normal
                \item Light: None, though the constant fires provide sufficient illumination in most locations on the plane
                \item Limits: Looped
                \item Planar connectivity: Unstable Connected
                \item Shape: Solid Sphere, with a radius of about 1,000 miles.
            \end{itemize}

            \parhead{The Plane of Water}
            The Plane of Water is an impossibly vast ocean.
            Powerful currents sweep through the ocean, but much of it is calm, and many forms of aquatic life abound in the water.
            The Plane of Water is the most densely populated Primal Plane, both by sentient creatures and monsters.
            Magnificant underwater cities are carved from huge rocks that float peacefully suspended in the water.
            Though there is no sun, simple creatures akin to plankton form the base of the food chain by feeding directly on the plane's essence.

            The Plane of Earth has the following planar traits:
            \begin{itemize}
                \item Gravity strength: No Gravity
                \item Light: None, though bioluminescent creatures like plankton are extremely common, making many parts of the plane well-lit
                \item Limits: Looped
                \item Planar connectivity: Stable Connected
                \item Shape: Uniform, in a sphere with a radius of about 1,000 miles.
            \end{itemize}

        \subsubsection{Nexus Planes}

            \parhead{The Material Plane}\label{The Material Plane}
            The Material Plane is the plane that most Rise adventures begin on.
            The surface of the plane is a massive sphere with a radius of about 4,000 miles.
            It is the most familiar to most humanoid creatures.

            The Material Plane has the following planar traits:
            \begin{itemize}
                \item Gravity direction: Absolute Directional, pointing to the center of the sphere
                \item Gravity strength: Normal
                \item Light: Mobile Source, from a sun and moon outside of the plane's limits
                \item Limits: Looped
                \item Planar connectivity: Isolated
                \item Shape: Solid Sphere, with a radius of about 4,000 miles.
            \end{itemize}

        \parhead{The Astral Plane}\label{The Astral Plane}
            The Astral Plane is the space between the other planes.
            It is a necessary intermediate destination for virtually all planar journeys, as all \glossterm{planar rifts} lead to and from the Astral Plane.
            Most activity on the Astral Plane occurs in a space called the Inner Astral Plane, a massive but finite region where all planar rifts on the Astral Plane appear.
            % Are there any other infinite planes?
            However, unlike all other planes, the Astral Plane has no known limits to its extent, and may in fact be infinite.
            The area outside the Inner Astral Plane is known as the Deep Astral Plane, and few venture into those sparsely populated realms.
            % This should be more clearly defined
            The Deep Astral Plane has magical turbulence that interferes with long-range communication and transportation magic, making exploration difficult.

            The Astral Plane has the following planar traits:
            \begin{itemize}
                \item Directional gravity: Subjective
                \item Gravity strength: Normal
                \item Light: Fixed Source, from the infinite reaches of the Deep Astral Plane
                \item Limits: Infinite
                \item Planar connectivity: Conduit
                \item Shape: Uniform
            \end{itemize}
