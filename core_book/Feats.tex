\chapter{Feats}\label{Feats}

Feats are special abilities that every character has.
Feats can be used to specialize in a particular area, to gain new abilities, or to change the way you do certain things.

\section{Gaining Feats}
    You gain one feat from your race (see \pcref{Races} at 1st level).
    At 2nd, 5th, and 9th level, you also gain any feat of your choice that you qualify for.
    You cannot gain the same feat twice.

    \subsection{Prerequisites}
        Some feats have prerequisites.
        Unless you meet all of the prerequisites, you cannot take the feat.
        Prerequisites can include a minimum starting attribute score, another feat or feats, a minimum level of training in a skill, or some other property of a character.
        A character can gain a feat at the same level at which he or she gains the prerequisite.

        A character can't use a feat if he or she has lost a prerequisite.

\section{Feat Tags}
    All feats are organized into different groups by tags.

    \parhead{General} General feats can have a wide variety of effects.
    They often grant new abilities or improve your defenses.
    % General feats improve at 1/4/7/10/...

    \parhead{Combat} Combat feats improve your combat capabilities.
    They can increase the damage you deal, grant you new combat abilities, or improve your defenses.
    % Combat feats improve at 1/3/6/9/...

    \parhead{Spell} Spell feats improve your spellcasting abilities.
    Spell feats are useless to characters who cannot cast spells.
    % Spell feats improve at 1/3/6/9/...

    \parhead{Skill} Skill feats improve your skills.
    They can make you more likely to succeed with skill checks and grant you new abilities based on your skills.
    % Skill feats improve at 1/2/5/8/...

    \parhead{Bloodline Feats} Some characters have traces of monstrous blood running in their veins.
    Most of those will never understand the full potential of their unusual heritage.
    Bloodline feats allow characters to explore those posibilities by gaining abilites related to their ancestry.
    You can only have one Bloodline feat.
    % Bloodline feats improve at 1/2/5/8/...

    \parhead{Magical Feats}
    All abilities granted by feats with the [Magical] type are \glossterm{magical} in nature.
    Many feats are not entirely magical, but have specific effects that are magical.

\section{Feat Tables}

% Feat names must follow ``I have'' or ``I am (a)''.
\begin{longtabuwrapper}
    \begin{longtabu}{>{\lcol}p{10em} >{\lcol}p{15em} >{\lcol}X >{\lcol}p{8em} >{\lcol}p{3em}}
        \lcaption{Feats}\\
        \tb{General Feats}\label{General Feats} & \tb{Prerequisites} & \tb{Benefits} & \tb{Feat Types} & \tb{Page} \\
        \featref{Celestial Heritage} & Non-evil & Gain aspects of celestial beings & Bloodline, Magical & \featpref{Celestial Heritage} \\
        \featref{Draconic Heritage} & \tdash & Gain aspects of draconic power & Bloodline & \featpref{Draconic Heritage} \\
        \featref{Iron Will} & Wil 2 & Increase mental resilience & \tdash & \featpref{Iron Will} \\
        \featref{Null} & Wil 2 & Become immune to magic & \tdash & \featpref{Null} \\
        \featref{Regenerator} & Con 2 & Heal from wounds with inhuman speed & \tdash & \featpref{Regenerator} \\
        \featref{Swift} & Dex 2 & Move more quickly & \tdash & \featpref{Swift} \\
        \featref{Toughness} & Con 2 & Increase physical fortitude & \tdash & \featpref{Toughness} \\

        \tb{Class Feats}\label{Class Feats} & \tb{Prerequisites} & \tb{Benefits} & \tb{Feat Types} & \tb{Page} \\
        \featref{Class Versatility} & \tdash & Swap for abilities from additional class & \tdash & \featpref{Class Versatility} \\
        \tind \featref{Class Dedication} & Class Versatility feat & Gain archetype from additional class & \tdash & \featpref{Class Dedication} \\

        \tb{Skill Feats}\label{Skill Feats} & \tb{Prerequisites} & \tb{Benefits} & \tb{Feat Types} & \tb{Page} \\
        \featref{Acrobatics Specialization} & Mastered Acrobatics & \tdash & \tdash & \featpref{Acrobatics Specialization} \\
        \featref{Awareness Specialization} & Mastered Awareness & \tdash & \tdash & \featpref{Awareness Specialization} \\
        \featref{Bardic Exemplar} & Perform Specialization & Mock foes and bolster allies with performances & Magical & \featpref{Bardic Exemplar} \\
        \featref{Bluff Specialization} & Mastered Bluff & \tdash & \tdash & \featpref{Bluff Specialization} \\
        \featref{Climb Specialization} & Mastered Climb & \tdash & \tdash & \featpref{Climb Specialization} \\
        \featref{Craft Specialization} & Mastered Creaft & \tdash & \tdash & \featpref{Craft Specialization} \\
        \featref{Creature Handling Specialization} & Mastered Creature Handling & \tdash & \tdash & \featpref{Creature Handling Specialization} \\
        \featref{Devices Specialization} & Mastered Devices & \tdash & \tdash & \featpref{Devices Specialization} \\
        \featref{Disguise Specialization} & Mastered Disguise & \tdash & \tdash & \featpref{Disguise Specialization} \\
        \featref{Escape Artist Specialization} & Mastered Escape Artist & \tdash & \tdash & \featpref{Escape Artist Specialization} \\
        \featref{Intimidate Specialization} & Mastered Intimidate & \tdash & \tdash & \featpref{Intimidate Specialization} \\
        \featref{Heal Specialization} & Mastered Heal & \tdash & \tdash & \featpref{Heal Specialization} \\
        \featref{Jump Specialization} & Mastered Jump & \tdash & \tdash & \featpref{Jump Specialization} \\
        \featref{Knowledge Specialization} & Mastered Knowledge & \tdash & \tdash & \featpref{Knowledge Specialization} \\
        \featref{Linguistics Specialization} & Mastered Linguistics & \tdash & \tdash & \featpref{Linguistics Specialization} \\
        \featref{Perform Specialization} & Mastered Perform & \tdash & \tdash & \featpref{Perform Specialization} \\
        \featref{Persuasion Specialization} & Mastered Persuasion & \tdash & \tdash & \featpref{Persuasion Specialization} \\
        \featref{Ride Specialization} & Mastered Ride & \tdash & \tdash & \featpref{Ride Specialization} \\
        \featref{Sense Motive Specialization} & Mastered Sense Motive & \tdash & \tdash & \featpref{Sense Motive Specialization} \\
        \featref{Sleight of Hand Specialization} & Mastered Sleight of Hand & \tdash & \tdash & \featpref{Sleight of Hand Specialization} \\
        \featref{Spellcraft Specialization} & Mastered Spellcraft & \tdash & \tdash & \featpref{Spellcraft Specialization} \\
        \featref{Stealth Specialization} & Mastered Stealth & \tdash & \tdash & \featpref{Stealth Specialization} \\
        \featref{Survival Specialization} & Mastered Survival & \tdash & \tdash & \featpref{Survival Specialization} \\
        \featref{Swim Specialization} & Mastered Swim & \tdash & \tdash & \featpref{Swim Specialization} \\

        \tb{Spell Feats}\label{Spell Feats} & \tb{Prerequisites} & \tb{Benefits} & \tb{Feat Types} & \tb{Page} \\
        \featref{Abjurer} & Abjuration spell & Improve Abjuration spells & Magical & \featpref{Abjurer} \\
        \featref{Boongiver} & Any spell & Benefits when casting spells on allies & Magical & \featpref{Boongiver} \\
        \featref{Conjurer} & Conjuration spell & Improve Conjuration spells & Magical & \featpref{Conjurer} \\
        \featref{Diviner} & Divination spell & Improve Divination spells & Magical & \featpref{Diviner} \\
        \featref{Eldritch Knight} & Any spell & Fight with sword and spell together & \tdash & \featpref{Eldritch Knight} \\
        \featref{Enchanter} & Enchantment spell & Improve Enchantment spells & Magical & \featpref{Enchanter} \\
        \featref{Evoker} & Evocation spell & Improve Evocation spells & Magical & \featpref{Evoker} \\
        \featref{Illusionist} & Illusion spell & Improve Illusion spells & Magical & \featpref{Illusionist} \\
        \featref{Miscaster} & Any spell & Miscast spells intentionally and effectively & Magical & \featpref{Miscaster} \\
        \featref{Mystic Archer} & Any spell & Imbue projectiles with magic & Magical & \featpref{Mystic Archer} \\
        \featref{Transmuter} & Transmutation spell & Improve Transmutation spells & Magical & \featpref{Transmuter} \\
        \featref{Vivimancer} & Vivimancy spell & Improve Vivimancy spells & Magical & \featpref{Vivimancer} \\

        \tb{Combat Feats}\label{Combat Feats} & \tb{Prerequisites} & \tb{Benefits} & \tb{Feat Types} & \tb{Page} \\
        \featref{Agility} & Dex 2 & Increase reaction speed & \tdash & \featpref{Agility} \\
        \featref{Blindfighter} & Per 2 & Fight unseen foes better & \tdash & \featpref{Blindfighter} \\
        \featref{Brawler} & Str 1, Dex 1 & Fight better unarmed and in close quarters & \tdash & \featpref{Brawler} \\
        \featref{Duelist} & Dex 1, Int 1 & Fight one-on-one better & \tdash & \featpref{Duelist} \\
        \featref{Executioner} & Str 1, Per 2 & Kill weakened foes more easily & \tdash & \featpref{Executioner} \\
        \featref{Guardian} & Per 1, Wil 1 & Protect nearby allies & \tdash & \featpref{Guardian} \\
        \featref{Leadership} & Wil 2 & Inspire nearby allies & \tdash & \featpref{Leadership} \\
        \featref{Martial Training} & \tdash & Improve combat abilities & \tdash & \featpref{Martial Training} \\
        \featref{Precognition} & Int 2 & React to future events & \tdash & \featpref{Precognition} \\
        \featref{Savage} & Str 2 & Shove and overrun foes to deal damage & \tdash & \featpref{Savage} \\
        \featref{Sniper} & Per 2 & Aim precisely at distant foes & \tdash & \featpref{Sniper} \\
        \featref{Whirlwind Warrior} & Dex 2, Per 1 & Fight hordes with agile ease & \tdash & \featpref{Whirlwind Warrior} \\
    \end{longtabu}
\end{longtabuwrapper}

    \section{Feat Descriptions}
        Each feat has a set of benefits it provides to a character with the feat.
        Some feats also have specific requirements that a character must meet before taking the feat.
        These are listed under a \textbf{Prerequisites} heading.
        If a character loses the prerequisites for a feat, they lose all benefits of the feat until they meet the prerequisites again.

    \begin{feat}{Abjurer}{Magical, Spell}
        \spelldesc{You have great talent with Abjuration spells.}
        \featpre Abjuration spell known.
        \ff{Abjurant Shield} Whenever you cast an Abjuration spell with a duration, you can give one willing creature targeted by the spell a deflective shield.
        The shielded creature gains a \plus1 \glossterm{magic bonus} to Armor defense as long as the spell lasts.
        This is a \glossterm{magical} \glossterm{Shielding} effect.
        You can only shield one creature in this way at a time.
        If you shield another creature, all previous shields are dismissed when the new shield takes effect.

        \ff[3]{Counterspell} As a standard action, you can spend an \glossterm{action point} to use the \textit{counterspell} ability.
        \begin{ability}{Counterspell}
            Choose a creature within \rngmed range of you.
            If the target is casting a spell, and your maximum spell level is at least as high as the target's maximum spell level, their spell has no effect when it resolves.
        \end{ability}

        \ff[6]{Personal Shield} You gain a \plus1 \glossterm{magic bonus} to Armor defense.

        \ff[9]{Greater Abjurant Shield} Your \textit{abjurant shield} also grants \glossterm{damage reduction} against \glossterm{physical damage} equal to your highest \glossterm{spellpower}.

        \ff[12]{Improvised Counterspell} At the start of the \glossterm{delayed action phase}, if you are casting a spell other than a \glossterm{cantrip}, you can use your \textit{counterspell} ability without taking an action.
        If you do, your spell has no effect when it resolves, and you regain the action point spent to cast it (if any).

        \ff[15]{Expanded Protection} The defense bonuses from your \textit{personal shield} and \textit{abjurant shield} abilities apply to all defenses.
        In addition, the damage reduction from your \textit{greater abjurant shield} ability applies against all damage.

        \ff[18]{Punishing Counterspell} When you use your \textit{counterspell} ability, you can make the target \glossterm{miscast} their spell instead of negating the spell's effects.
        In addition, the range of your \textit{counterspell} ability increases to \rnglong.

        \ff[20]{Abjurant Bastion} The defens bonuses from your \textit{personal shield} and \textit{abjurant shield} abilities increase to \plus2.
    \end{feat}

    \begin{feat}{Acrobatics Specialization}{Skill}
        \featpre Acrobatics as a mastered skill.

        \ff{Rapid Balance} Using Acrobatics to balance on slippery or narrow surfaces does not reduce your speed.

        \ff[2]{Specialization} You gain a \plus2 bonus to Acrobatics.

        \ff[5]{Surface Tolerance} You reduce DR modifiers for surface conditions on Acrobatics checks to balance by 2.
        This allows you to ignore minor surface conditions, such as slippery surfaces, when balancing.

        \ff[8]{Greater Specialization} The bonus from your \textit{specialization} ability increases to \plus4.

         % modifier at 10th: 12dex + 2mst + 2ft = 16
        \ff[11]{Air Dancer}[Magical] You can attempt to move on surfaces that cannot support your weight.
        Surfaces that can support at least a quarter of your weight, such as thin tree branches and dense liquids, are DR 20.
        Surfaces that can support at least a tenth of your weight, such as water, are DR 25.
        Surfaces that can support at least a hundredth of your weight, such as tree leaves, are DR 30.
        Surfaces that cannot support your weight at all, such as air, are DR 40.

        Success means you move along the surface at half speed.
        Failure means you fall through the surface.
        The DR increases by 5 for each consecutive round that you spend moving in this way.

        \ff[14]{Supreme Specialization} The bonus from your \textit{specialization} ability increases to \plus6.

        \ff[17]{Greater Surface Tolerance} The DR modifier reduction from your \textit{surface tolerance} ability increases to 5.
        This allows you to ignore almost all surface conditions when balancing.

        \ff[20]{Aerial Supremacy} You can move at full speed while using your \textit{air dancer} ability.
        In addition, for each round that you spend using your \textit{air dancer} ability, the DR increases by 2 instead of by 5.
    \end{feat}

    \begin{feat}{Agility}{Combat}
        \featpre Starting Dexterity of 2.

        \ff{Spring Attack} As a standard action, you can spend an \glossterm{action point} to use the \textit{spring attack} ability.
        \begin{ability}{Spring Attack}
            Move up to your movement speed and make a \glossterm{strike}.
            If you use this ability during the \glossterm{action phase}, you may continue moving during the \glossterm{delayed action phase} if you have remaining movement available.
        \end{ability}

        \ff[3]{Lightning Reflexes} You gain a \plus2 bonus to Reflex defense and \glossterm{initiative} checks.

        \ff[6]{Dodge Roll} When you use your \textit{spring attack} ability, you gain a \plus2 bonus to Armor and Reflex defenses during that phase.
        This is a \glossterm{Swift} ability.

        \ff[9]{Combat Evasion} You gain a \plus1 bonus to Armor defense.

        \ff[12]{Greater Spring Attack} You gain a \plus1d bonus to damage on the strike from your \textit{spring attack} ability.

        \ff[15]{Greater Lightning Reflexes} The bonuses from your \textit{lightning reflexes} ability increase to \plus4.

        \ff[18]{Greater Combat Evasion} The bonus from your \textit{combat evasion} ability increases to \plus2.
    \end{feat}

    \begin{feat}{Awareness Specialization}{Skill}
        \featpre Awareness as a mastered skill.

        % TODO: clarify stacking with existing senses
        \ff{Extraordinary Senses} You gain one of the following senses: \glossterm{blindsense} (50 ft.), \glossterm{darkvision} (100 ft.), \glossterm{scent}, or \glossterm{tremorsense} (50 ft.).

        \ff[2]{Specialization} You gain a \plus2 bonus to Awareness.

        \ff[5]{Distance Tolerance} You reduce DR modifiers for distance on Awareness rolls by 2.
        This usually allows you to ignore up to 100 feet of distance.

        \ff[8]{Greater Specialization} The bonus from your \textit{specialization} ability increases to \plus4.

        \ff[11]{Greater Extraordinary Senses} You gain one of the following senses: \glossterm{blindsense} (200 ft.), \glossterm{blindsight} (50 ft.), \glossterm{darkvision} (500 ft.), \glossterm{tremorsense} (200 ft.), or \glossterm{tremorsight} (50 ft.).

        \ff[14]{Supreme Specialization} The bonus from your \textit{specialization} ability increases to \plus6.

        \ff[17]{Greater Distance Tolerance} The reduction of DR modifiers for distance increases to 5.
        This usually allows you to ignore up to 500 feet of distance.

        \ff[20]{Supreme Extraordinary Senses} You can choose an additional sense from the list given in your \textit{greater extraordinary senses} ability.
        In addition, the range of all senses gained from this feat is doubled.
    \end{feat}

    \begin{feat}{Bardic Exemplar}{Magical, Skill}
        \featpre Perform Specialization feat.

        \ff[1]{Inspire Courage} As a standard action, you can spend an \glossterm{action point} to use the \textit{inspire courage} ability.
        \begin{ability}{Inspire Courage}[\glossterm{Delusion}, \glossterm{Mind}, \glossterm{Sustain} (minor)]
            You begin a performance using one of your Perform skills.
            All willing allies within a \areahuge emanation from you gain a \plus1 bonus to \glossterm{accuracy} and a \plus2 bonus to Mental defense.

            If a target can neither see nor hear your performance, the effect immediately ends for that target.
            This ability may have the \glossterm{Auditory} or \glossterm{Visual} tags, depending on the nature of your performance.
        \end{ability}

        \ff[2]{Vicious Mockery} As a standard action, you can spend an \glossterm{action point} to use the \textit{vicious mockery} ability.
        \begin{ability}{Vicious Mockery}[\glossterm{Delusion}, \glossterm{Mind}]
            You make a brief performance using one of your Perform skills.
            Make a Perform vs. Mental attack against all enemies within a \areahuge radius from you.

            \hit As a \glossterm{condition}, each target takes a \minus1 penalty to \glossterm{accuracy} and a \minus2 penalty to Mental defense.

            This ability may have the \glossterm{Auditory} or \glossterm{Visual} tags, depending on the nature of your performance.
        \end{ability}

        \ff[5]{Song of Serenity} As a standard action, you can spend an \glossterm{action point} to use the \textit{song of serenity} ability.
        \begin{ability}{Song of Serenity}[\glossterm{Delusion}, \glossterm{Mind}, \glossterm{Sustain} (minor)]
            You begin a performance using one of your Perform skills.
            Choose a willing creature within \rngmed range.
            The target is immune to hostile \glossterm{Mind} effects.

            If the target can neither see nor hear your performance, the effect immediately ends.
            This ability may have the \glossterm{Auditory} or \glossterm{Visual} tags, depending on the nature of your performance.
        \end{ability}

        \ff[8]{Greater Inspire Courage} Each target affected by your \textit{inspire courage} ability is immune to \glossterm{Fear} abilities.

        \ff[11]{Greater Vicious Mockery} Your \textit{vicious mockery} ability gains the \glossterm{Sustain} (minor) tag.
        At the end of each action phase after the first round, you make an additional attack against all creatures within a \areahuge radius of you not already affected by the ability.

        \ff[12]{Greater Song of Serenity} When you use your \textit{song of serenity} ability, you can target up to five creatures.

        \ff[15]{Supreme Inspire Courage} A creature affected by your \textit{inspire courage} ability may use your level in place of its accuracy with any attack.

        \ff[18]{Supreme Mocking Performance} The penalties from your \textit{vicious mockery} ability increase to \minus2 accuracy and a \minus4 penalty to Mental defense.

        \ff[20]{Hybrid Performance} You can sustain two different \glossterm{magical} abilities from this feat or the Perform Specialization feat as part of the same \glossterm{minor action}.
    \end{feat}

    \begin{feat}{Blindfighter}{Combat}
        \featpre Starting Perception 2.
        
        \ff[1]{Blind Precision} Whenever you make an attack with a miss chance caused by being unable to see your opponent, you can roll the miss chance twice and take the better result.
        In addition, you are not \defenseless against foes you cannot see if you know their location.

        \ff[3]{Blindsense} You gain \glossterm{blindsense} (50 ft.).

        \ff[6]{Attack the Unseen} If you know the location of a creature you cannot see, and you have \glossterm{line of effect} to that creature, you can target it with targeted abilities.

        \ff[9]{Blindsight} You gain \glossterm{blindsight} (50 ft.).
        In addition, the range of your blindsense improves to 200 feet. 

        \ff[12]{Controlled Sight} You are immune to all abilities that depend on sight to affect you.

        \ff[15]{Greater Blindsight} The range of your blindsight improves to 100 feet.
        In addition, the range of your blindsense improves to 500 feet.

        \ff[18]{Unseeing Precision} You gain a \plus1 bonus to \glossterm{accuracy}.

        \ff[20]{Supreme Sight} The range of your blindsight improves to 200 feet.
        In addition, the range of your blindsense improves to 1,000 feet.
    \end{feat}

    \begin{feat}{Bluff Specialization}{Skill}
        \featpre Bluff as a mastered skill.

        \ff{Sustained Distraction} If you successfully distract a creature with the \textit{distract} ability of the Bluff skill, you can sustain that distraction on that creature with a \glossterm{minor action} as long as you continue to be distracting.
        You must make a new check each round, and the DR increases by 2 for each round you have distracted the creature.
        For details, see \pcref{Distract}.

        \ff[2]{Specialization} You gain a \plus2 bonus to Bluff.

        \ff[5]{Deceive Magic} Any magical abilities which detect lies are unable to detect lies you speak.

        \ff[8]{Greater Specialization} The bonus from your \textit{specialization} ability increases to \plus4.

        \ff[11]{Dual Speech}[Magical] Whenever you speak, you can spend an \glossterm{action point} to use the \textit{dual speech} ability.
        \begin{ability}{Dual Speech}[\glossterm{Sustain} (minor)]
            You speak the same words with two different vocal patterns, such as tone or accent, though you cannot significantly change the volume.
            You can freely choose which creatures hear which pattern, treating all creatures you are not aware of as a single group.

            You can freely choose different vocal patterns each round that you sustain this ability.
        \end{ability}

        \ff[14]{Supreme Specialization} The bonus from your \textit{specialization} ability increases to \plus6.

        \ff[17]{Greater Deceive Magic} Whenever you impersonate a creature, if your Bluff check is high enough, magical abilities treat you as if you were that creature.
        % is this DR high enough?
        The DR is equal to 10 \add the \glossterm{power} of the ability you are deceiving.
        % does this work mechanically?
        For example, if you impersonate an undead creature, the \spell{inflict light wounds} spell would heal you.
        This does not grant you any abilities associated with creatures you impersonate.

        \ff[20]{Greater Dual Speech} When you use your \textit{dual speech} ability, you can speak entirely different words with your two voices.
    \end{feat}

    \begin{feat}{Boongiver}{Spell}
        \featpre Ability to cast a spell.

        \ff[1]{Reciprocal Boon} Whenever you cast a spell with the \glossterm{Attune} tag that targets a single creature other than yourself, you can use the \textit{reciprocal boon} ability.
        \begin{ability}{Reciprocal Boon}
            The spell you are casting targets you in addition to its other target.

            You can only use this ability on one spell at a time.
            If you use the ability while it already affects another spell, you choose which spell is affected by the ability.
        \end{ability}

        \ff[3]{Benevolent Transferance} As a \glossterm{minor action}, you can use the \textit{benevolent transferance} ability.
        \begin{ability}{Benevolent Transferance}
            Choose a creature currently attuning to a spell you cast.
            In addition, choose another creature to transfer the spell to.
            Both targets must within that spell's range of you, and must be valid targets for the spell.
            You cannot target yourself with this ability.

            If both targets are willing, the spell's effect is transferred from the first target to the second.
            The spell's new target must spend an action point to attune to the spell as normal.
        \end{ability}

        \ff[6]{Boon Lore} You learn an additional \glossterm{subspell} of 2nd level or lower.
        The subspell must have the \glossterm{Attune} (shared) tag.

        \ff[9]{Greater Reciprocal Boon} You can use your \textit{reciprocal boon} ability on up to two different spells at once.
        If you use the ability while it already affects another spell, you choose which spells are affected by the ability.

        \ff[12]{Regenerative Transferance} When you use your \textit{benevolent transferance} ability successfully, the original target of the spell regains the action point it spent to attune to the spell.

        \ff[15]{Greater Boon Lore} You learn two additional \glossterm{subspells} of 7th level or lower.
        The subspells must have the \glossterm{Attune} (shared) tag.

        \ff[18]{Supreme Reciprocal Boon} You can use your \textit{reciprocal boon} ability on up to three different spells at once.

        \ff[20]{Myriad Boon} When you use your \textit{reciprocal boon} ability on a spell, you can choose a third target for the spell.
        That target must spend an action point to attune to the spell as normal.
    \end{feat}

    \begin{feat}{Brawler}{Combat}
        \featpre Starting Strength of 1, starting Dexterity of 1.

        \ff[1]{Unarmed Warrior} You are \glossterm{proficient} with your \glossterm{unarmed attack}.
        In addition, you gain a \plus2d bonus to damage with your unarmed attack.
        For details about how to fight while unarmed, see \pcref{Unarmed Combat}.
        This ability does not stack with the ability of the same name from the Unfettered Warrior monk archetype (see \pcref{Unfettered Warrior}).

        \ff[3]{Grapple Expertise} You gain a \plus2 bonus to \glossterm{accuracy} with the \textit{grapple} ability (see \pcref{Grapple}).

        \ff[6]{Takedown} As a standard action, you can spend an \glossterm{action point} to use the \textit{takedown} ability.
        \begin{ability}{Takedown}
            Make an unarmed \glossterm{strike}.
            In addition to the attack's normal effects, if the attack result also beats the target's Fortitude defense, you and the target are \grappled by each other.
        \end{ability}

        \ff[9]{Greater Unarmed Warrior} The damage bonus from your \textit{unarmed warrior} increases to \plus3d.

        \ff[12]{Greater Takedown} You gain a \plus1 bonus to accuracy with your \textit{grapple expertise} ability.

        \ff[15]{Greater Grapple Expertise} The bonus from your \textit{grapple expertise} ability increases to \plus3.

        \ff[18]{Supreme Takedown} The accuracy bonus from your \textit{greater takedown} ability increases to \plus2.
    \end{feat}

    \begin{feat}{Celestial Heritage}{Bloodline, Magical}
        \featpre Non-evil alignment.
        \parhead{Special} You can only have one Bloodline feat.

        \ff[1]{Celestial Power} Your \glossterm{power} with abilities from this feat is equal to your Willpower or your level, whichever is higher.

        \ff[1]{Holy Blessing} As a \glossterm{minor action}, you can spend an \glossterm{action point} to use the \textit{holy blessing} ability.
        \begin{ability}{Holy Blessing}[\glossterm{Attune} (shared)]
            Choose a willing creature within \rngclose range.
            The target gains a \plus1 \glossterm{magic bonus} to \glossterm{accuracy} with all attacks.
        \end{ability}

        \ff[2]{Divine Favor} You gain a \plus1 bonus to Fortitude and Mental defense.

        \ff[5]{Angel Wings} You gain feathery wings that sprout from your back.
        These wings grant you a glide speed equal to your \glossterm{base speed} (see \pcref{Gliding}).
        The wings themselves are physical, but the ability to fly and glide with them is \glossterm{magical}.

        \ff[8]{Greater Holy Blessing} When you use your \textit{holy blessing} ability, the target also gains a \plus1d \glossterm{magic bonus} to all abilities that deal damage or grant healing measured in dice.

        \ff[11]{Angelic Flight} As a \glossterm{free action}, you can spend an \glossterm{action point} to use the \textit{angelic flight} ability.
        \begin{ability}{Angelic Flight}
            You gain a \glossterm{fly speed} equal to your \glossterm{base speed} until the end of the round (see \pcref{Flying}).
        \end{ability}

        \ff[14]{Greater Divine Favor} The bonuses from your \textit{divine favor} ability increase to \plus2.

        \ff[17]{Supreme Holy Blessing} The accuracy bonus from your \textit{holy blessing} ability increases to \plus2.

        \ff[20]{Greater Angelic Flight} You can use your \textit{angelic flight} ability without spending an \glossterm{action point}.
    \end{feat}

    \begin{feat}{Class Dedication}{Class}
        \featpre Class Versatility feat.

        \ff[1]{Additional Archetype} Choose an archetype from a class you have chosen with the \textit{additional class} ability of the Class Versatility feat (see \featpref{Class Versatility}).
        You gain that archetype in addition to your other archetypes.
        Your \glossterm{archetype rank} in that archetype is Rank 1.

        \ff[4]{Archetype Rank} You become Rank 2 in any of your archetypes.

        \ff[7]{Archetype Rank} You become Rank 3 in any of your archetypes.

        \ff[10]{Archetype Rank} You become Rank 4 in any of your archetypes.

        \ff[13]{Archetype Rank} You become Rank 5 in any of your archetypes.

        \ff[16]{Archetype Rank} You become Rank 6 in any of your archetypes.

        \ff[19]{Archetype Rank} You become Rank 7 in any of your archetypes.
    \end{feat}

    \begin{feat}{Class Versatility}{Class}

        \ff[1]{Additional Class} Choose a class.
        You gain the \glossterm{class skills} of that class in addition to your existing class skills.
        You can exchange one class archetype from your class with one class archetype from that class.
        You can also exchange any number of your basic class abilities, such as skill points or weapon proficiencies, for the corresponding abilities of that class.
        If that class has any basic class abilities which are not part of an archetype and do not have abilities of the same on other classes, such as a cleric's \textit{divine power}, you gain those abilities.

        \ff[6]{Versatile Expertise} Choose one of the following benefits.
        You gain that benefit.
        \begin{itemize}
            \item \plus1 bonus to Fortitude defense
            \item \plus1 bonus to Reflex defense
            \item \plus1 bonus to Mental defense
            \item One extra \glossterm{skill point}
            \item Extra \glossterm{hit points} equal to half your level
        \end{itemize}

        \ff[12]{Expanded Versatility} You gain the \textit{versatile expertise} ability again.

        \ff[18]{Expanded Versatility} You gain the \textit{versatile expertise} ability again.
    \end{feat}

    \begin{feat}{Climb Specialization}{Skill}
        \featpre Climb as a mastered skill.

        \ff{Damage Tolerance} Taking damage while climbing does not force you to make an additional Climb check to avoid falling.

        \ff[2]{Specialization} You gain a \plus2 bonus to Climb.

        \ff[5]{Climb Speed} You gain a \glossterm{climb speed} equal to half your \glossterm{base speed}.
        A successful Climb check to move allows you to travel a distance equal to your climb speed.

        \ff[8]{Greater Specialization} The bonus from your \textit{specialization} ability increases to \plus4.

        \ff[11]{Impossible Climber} You can climb surfaces that are perfectly smooth.
        The DR is 30 for perfectly smooth vertical surfaces, and 40 to climb on a perfectly smooth ceiling.

        \ff[14]{Supreme Specialization} The bonus from your \textit{specialization} ability increases to \plus6.

        \ff[17]{Greater Climb Speed} Your climb speed increases to be equal to your \glossterm{base speed}.

        \ff[20]{Personal Gravity} You can wallrun on ceilings in the same way you wallrun on walls.
        In addition, for every round that you spend wallrunning on any surface, the DR increases by 2 instead of by 5.
    \end{feat}

    \begin{feat}{Conjurer}{Magical, Spell}
        \featpre Conjuration spell known.

        \ff[1]{Fortified Manifestation} Objects and creatures you create with \glossterm{Manifestation} abilities have additional hit points equal to your highest \glossterm{spellpower}, and gain a \plus1d \glossterm{magic bonus} to damage with all attacks.

        \ff[3]{Astral Spell Transit} You double the range of all Conjuration spells you cast.
        In addition, all \glossterm{Teleportation} spells you cast can teleport twice their normal distance.

        \ff[6]{Resummon} As a \glossterm{minor action}, you use the \textit{resummon} ability.
        \begin{ability}{Resummon}
            You may teleport one creature or object that you summoned with a \glossterm{Summoning} or \glossterm{Manifestation} ability into an unoccupied space on stable ground within \rngclose range of you.
        \end{ability}

        \ff[9]{Persistent Manifestations} Once per round, when you cast or sustain a spell, you can also sustain a \glossterm{Manifestation} spell as a \glossterm{free action}.
        In addition, objects and creatures you have created with \glossterm{Manifestation} abilities heal hit points equal to your spellpower at the end of each round.

        \ff[12]{Greater Astral Spell Transit} The range doubling from your \textit{astral spell transit} ability applies to all spells you cast, not just Conjuration spells.

        \ff[15]{Greater Resummon} The range of your \textit{resummon} ability increases to \rngext.
        In addition, you can choose up to two creatures or objects to teleport instead of only one.

        \ff[18]{Greater Fortified Manifestation} The hit point bonus from your \textit{fortified manifestation} ability increases to twice your spellpower, and the damage bonus increases to \plus2d.

        \ff[20]{Supreme Astral Spell Transit} When determining whether you have \glossterm{line of effect} to a particular location with a spell, you can ignore all physical obstacles in a single five-foot span.
        This can allow you to cast spells through solid walls, though it does not grant you the ability to see through the wall.
    \end{feat}

    \begin{feat}{Craft Specialization}{Skill}
        \featpre Craft (any) as a mastered skill.

        \ff{Craft Magic Item}[Magical] You can imbue items with magic using your crafting skill.
        Imbuing an item with magic takes material components, as described in \pcref{Magic Item Creation}.
        It takes you one hour per 10 gp of material components to create a item.

        You can also mend a broken magic item if it is one that you could make.
        Doing so costs a tenth of the raw materials and a quarter of the time it would take to craft that item in the first place.
        You cannot mend a \glossterm{destroyed} magic item.

        \ff[2]{Specialization} You gain a \plus2 bonus to all Craft skills.

        \ff[5]{Crafting Savant} You gain two additional \glossterm{skill points} which can only be spent on Craft skills.
        In addition, crafting magic items takes you one hour per 100 gp of material components.

        \ff[8]{Greater Specialization} The bonus to Craft from your \textit{specialization} ability increases to \plus4.

        \ff[11]{Item Attunement} You gain an additional \glossterm{item slot}.
        In addition, crafting magic items takes you one hour per 100 gp of material components.

        \ff[14]{Supreme Specialization} The bonus to Craft from your \textit{specialization} ability increases to \plus6.

        \ff[17]{}
        Crafting magic items takes you one hour per 2,500 gp of material components.

        \ff[20]{Greater Item Attunement} You gain an additional \glossterm{item slot}.
        In addition, crafting magic items takes you one hour per 12,500 gp of material components.
    \end{feat}

    \begin{feat}{Creature Handling Specialization}{Skill}
        \featpre Creature Handling as a mastered skill.

        \ff{Sustained Pacify} You can sustain the \textit{pacify} ability from the Creature Handling skill as a minor action, rather than as a standard action (see \pcref{Pacify}).

        \ff[2]{Specialization} You gain a \plus2 bonus to Creature Handling.

        \ff[5]{Efficient Training} You can teach a creature a trick with 12 hours of work, split as you choose, rather than in a week of 4 hour sessions (see \pcref{Training Creatures}).
        In addition, you can train creatures to learn two bonus tricks beyond their normal maximum (see \pcref{Bonus Tricks}).

        \ff[8]{Greater Specialization} The bonus to Creature Handling from your \textit{specialization} ability increases to \plus4.

        \ff[11]{Battleforged Training} You can teach a creature the Battleforged trick.
        The DR to train the trick is 20.
        A creature with the trick gains the following benefits:
        \begin{itemize}
            \item Its maximum hit points increase by an amount equal to its level.
            \item It gains a \plus1 bonus to accuracy with all attacks.
            \item It gains a \plus1d bonus to \glossterm{strike damage}.
        \end{itemize}
        In addition, the number of bonus tricks you can teach from your \textit{efficient training} ability increases to four.

        \ff[14]{Supreme Specialization} The bonus to Creature Handling from your \textit{specialization} ability increases to \plus6.

        \ff[17]{Rapid Pacify} You can use the \textit{pacify} ability from the Creature Handling skill as a minor action, rather than as a standard action.

        \ff[20]{Greater Battleforged Training} You can teach a creature that has learned the Battleforged trick the Greater Battleforged trick.
        The DR to train the trick is 30.
        A creature with the trick gains the following benefits, which replace the benefits of the Battleforged trick:
        \begin{itemize}
            \item Its maximum hit points increase by an amount equal to twice its level.
            \item It gains a \plus2 bonus to accuracy with all attacks.
            \item It gains a \plus2d bonus to \glossterm{strike damage}.
        \end{itemize}
        In addition, the number of bonus tricks you can teach from your \textit{efficient training} ability increases to six.
    \end{feat}

    \begin{feat}{Devices Specialization}{Skill}
        \featpre Devices as a mastered skill.

        \ff{Rapid Improvisation} As a standard action, you can spend an \glossterm{action point} to use the \textit{improvise} ability to create a device (see \pcref{Improvise}).

        \ff[2]{Specialization}[Magical] You gain a \plus2 bonus to Devices.

        \ff[5]{Lesser Disable Arcana}[Magical] You can disable arcane spell effects on objects or areas as if they were merely complex devices.
        You must be aware of an effect to disable it, either through the Spellcraft skill or because the effect is noticeable.
        You cannot disable effects on creatures.
        The DR to disable an effect is equal to 15 \add the effect's \glossterm{power}.
        Success means the spell is \glossterm{dismissed}, and its effects end.
        This has no effect on abilities that cannot be dismissed.

        \ff[8]{Greater Specialization} The bonus from your \textit{specialization} ability increases to \plus4.

        \ff[11]{Improbable Improvisation} You reduce the DR for using the \textit{improvise} ability to make devices from unsuitable materials by 5.

        \ff[14]{Supreme Specialization} The bonus from your \textit{specialization} ability increases to \plus6.

        \ff[17]{Durable Improvization} Devices you create with the \textit{improvise} ability last for twice as many uses before they break.

        \ff[20]{Greater Disable Arcana}[Magical] You can disable all magical effects on objects or areas, not just spell effects.
    \end{feat}

    \begin{feat}{Disguise Specialization}{Skill}
        \featpre Disguise as a mastered skill.

        \ff{Quick Change} As a standard action, you can spend an \glossterm{action point} to use the \textit{disguise creature} or \textit{emulate creature} ability.

        \ff[2]{Specialization} You gain a \plus2 bonus to Disguise.

        \ff[5]{Disguise Aura}[Magical] Whenever you use the \textit{disguise creature} or \textit{emulate creature} abilities, you can decide how the target and any items on the target appear when examined by Divination spells.
        For example, you could cause all of their equipment to appear nonmagical, or you could cause them to have a strong aura of good alignment.
        The maximum \glossterm{power} you can emulate is equal to your Disguise check result \minus10.

        Anyone using divination magic on the creature must make a \glossterm{spellpower} check with a DR equal to your Disguise check result in order to perceive the truth.
        Regardless of the result of the check, the caster is not aware that the check was made.

        \ff[8]{Greater Specialization} The bonus from your \textit{specialization} ability increases to \plus4.

        \ff[11]{Disguise Size}[Magical] As a \glossterm{standard action}, you can spend an action point to use the \textit{disguise size} ability.
        \begin{ability}{Disguise Size}[\glossterm{Attune}]
            You increase or decrease your size by one \glossterm{size category}.
            This effect lasts as long as you \glossterm{attune} to it.
        \end{ability}

        \ff[14]{Supreme Specialization} The bonus from your \textit{specialization} ability increases to \plus6.
    \end{feat}

    \begin{feat}{Diviner}{Magical, Spell}
        \featpre Divination spell known.

        \ff[1]{Prophesy} As a \glossterm{standard action}, you can spend an \glossterm{action point} to use the \textit{prophesy} ability.
        \begin{ability}{Prophesy}
            When you use this ability, you must visualize an action that a creature (or group of creatures) could take within the next hour.
            This time period is called the \textit{time of prophecy}.

            You see a brief, cryptic vision describing the most likely outcome of the action you visualized.
            This vision does not reveal any consequences that might occur after the \textit{time of prophecy} has ended.
            The vision does not have to be a literally accurate representation of the future.
            For example, if you used this ability to foresee the results of entering a room that had a group of creatures waiting in ambush, you might see a vision of flashing daggers in darkness darting towards your exposed back, regardless of whether the creatures would actually use daggers to attack.

            After using this ability, you cannot use it again until the \textit{time of prophecy} has ended, regardless of whether the action was taken.
        \end{ability}

        \ff[3]{Precognitive Reaction} You gain a \plus1 bonus to Reflex defense and a \plus2 bonus to \glossterm{initiative} checks.

        \ff[6]{Truesight} You gain the \glossterm{truesight} ability with a 50 foot range.

        \ff[9]{Deep Prophecy} When you use your \textit{prophesy} ability, you can increase the \textit{time of prophecy} to eight hours.
        This affects both the distance you can see into the future and the time you must wait before using the ability again.

        \ff[12]{Greater Precognitive Reaction} You are aware of all attacks against you, even those you cannot see, as long as you are conscious.
        This allows you to use abilities to defend yourself, and prevents you from being \unaware.

        \ff[15]{Dual Prophecy} You may use your \textit{prophesy} ability while you have an active \textit{time of prophecy}.
        If you have two active \textit{times of prophecy}, you must wait until one has expired to use your \textit{prophesy} ability again.

        \ff[18]{Supreme Precognitive Reaction} The Reflex defense bonus from your \textit{precognitive reaction} ability increases to \plus2, and the initiative bonus increases to \plus4.

        \ff[20]{Oracle} When you use your \textit{prophesy} ability, the maximum \textit{time of prophesy} you can choose is increased to one week.
        In addition, you may have three active \textit{times of prophecy}, rather than two.
    \end{feat}

    \begin{feat}{Draconic Heritage}{Bloodline}
        \parhead{Special} You can only have one Bloodline feat.

        \ff[1]{Draconic Power} Your \glossterm{power} with abilities from this feat is equal to your level or your Constitution, whichever is higher.

        \ff[1]{Draconic Ancestry} Choose a type of dragon from among the dragons on \trefnp{Dragon Types}.
        You have the blood of that type of dragon in your veins.
        This grants you damage reduction equal to twice your \textit{draconic power} against the damage type that dragon's breath weapon deals.

        \ff[1]{Draconic Weapons} You gain a bite natural weapon and two claw natural weapons, one on each arm.
        For details, see \pcref{Natural Weapons}.

        \ff[2]{Breath Weapon} As a \glossterm{standard action}, you can spend an \glossterm{action point} to use the \textit{breath weapon} ability.
        \begin{ability}{Breath Weapon}
            Make a \textit{draconic power} vs. Reflex attack against everything in the area defined by your \textit{draconic ancestry} ability (see \trefnp{Dragon Types}).
            A hit deals \glossterm{standard damage} against each target.
            The damage type is defined by your \textit{draconic ancestry} ability.
        \end{ability}

        \ff[5]{Draconic Wings} You gain scaly wings that sprout from your back.
        These wings grant you a glide speed equal to your \glossterm{base speed} (see \pcref{Gliding}).
        The wings themselves are physical, but the ability to fly and glide with them is \glossterm{magical}.

        \ff[8]{Greater Breath Weapon} The damage dealt by your \textit{breath weapon} ability increases by \plus1d.
        In addition, the area affected increases.
        A line breath weapon becomes a \areahuge, 10 ft.\ wide line.
        A cone breath weapon becomes a \arealarge cone.

        \ff[11]{Draconic Flight} As a \glossterm{free action}, you can spend an \glossterm{action point} to use the \textit{draconic flight} ability.
        \begin{ability}{Draconic Flight}
            You gain a \glossterm{fly speed} equal to your \glossterm{base speed} until the end of the round (see \pcref{Flying}).
        \end{ability}

        \ff[14]{Supreme Breath Weapon} The damage dealt by your \textit{breath weapon} ability increases by \plus1d.
        In addition, the area affected increases.
        A line breath weapon becomes a \areahuge, 20 ft.\ wide line.
        A cone breath weapon becomes a \areahuge cone.

        \ff[17]{Draconic Scales} You gain a \plus1 bonus to Armor defense. 

        \ff[20]{Greater Draconic Flight}[Magical] You can use your \textit{draconic flight} ability without spending an \glossterm{action point}.
    \end{feat}

    \begin{dtable}
        \lcaption{Dragon Types}
        \begin{dtabularx}{\columnwidth}{l >{\lcol}X >{\lcol}X}
            \tb{Dragon} & \tb{Energy Type} & \tb{Breath Weapon} \\
            \bottomrule
            Black & Acid & \areamed, 5 ft. wide line \\
            Blue & Electricity & \areamed, 5 ft. wide line \\
            Brass & Fire & \areamed, 5 ft. wide line \\
            Bronze & Electricity & \areamed, 5 ft. wide line \\
            Copper & Acid & \areamed, 5 ft. wide line \\
            Gold & Fire & \areamed cone \\
            Green & Acid & \areamed cone \\
            Red & Fire & \areamed cone \\
            Silver & Cold & \areamed cone \\
            White & Cold & \areamed cone \\
        \end{dtabularx}
    \end{dtable}

    \begin{feat}{Duelist}{Combat}
        \featpre Starting Dexterity of 1, starting Intelligence of 1.

        \ff[1]{Focused Strike} As a standard action, you can spend an \glossterm{action point} to use the \textit{focused strike} ability.
        \begin{ability}{Focused Strike}
            Make a melee \glossterm{strike}.
            If you are not threatened by more than one creature, you gain a \plus2 bonus to \glossterm{accuracy} with the strike.
            If you are the only creature threatening your target, you gain a \plus2d bonus to damage with the strike.
        \end{ability}

        \ff[3]{Parry} You gain a \plus1 bonus to Armor defense as long as you are not \glossterm{threatened} by more than one creature.

        \ff[6]{Duel Expertise} You gain a \plus2 bonus to \glossterm{accuracy} with the \textit{disarm} and \textit{feint} abilities (see \pcref{Disarm}, and \pcref{Feint}).

        \ff[6]{Focused Precision} The accuracy bonus from your \textit{focused strike} ability increases to \plus3.

        \ff[9]{Riposte} You gain a \plus1d bonus to damage as long as you are not \glossterm{threatened} by more than one creature.

        \ff[12]{Greater Duel Expertise} The accuracy bonus from your \textit{duel expertise} ability increases to \plus3. 

        \ff[12]{Greater Focused Strike} The damage bonus from your \textit{focused strike} ability increases to \plus3d.

        \ff[15]{Greater Parry} The bonus from your \textit{parry} ability increases to \plus2.

        \ff[18]{Supreme Duel Expertise} The accuracy bonus from your \textit{duel expertise} ability increases to \plus4. 

        \ff[18]{Greater Focused Precision} The accuracy bonus from your \textit{focused strike} ability increases to \plus4.

        \ff[20]{Legendary Duelist} You are treated as being threatened by one fewer creature than you actually are for the purpose of abilities from this feat.
    \end{feat}

    \begin{feat}{Eldritch Knight}{Spell}
        \featpre Ability to cast a spell.

        \ff[1]{Combat Caster} You gain a \plus2 bonus to Concentration checks made to cast spells.
        In addition, you reduce your \glossterm{encumbrance} from \glossterm{armor} by 1.

        \ff[3]{Spellstrike}[Magical] As a standard action during the \glossterm{action phase}, you can spend an \glossterm{action point} to use the \textit{spellstrike} ability.
        \begin{ability}{Spellstrike}
            You imbue magical power into a weapon you wield or your \glossterm{unarmed attack}.
            This requires concentration as if casting a spell, but breaking your concentration does not cause a \glossterm{miscast backlash}.
            During the \glossterm{delayed action phase}, you may make a melee \glossterm{strike} with the chosen weapon.
            If you maintained your concentration, the strike gains a \plus3d bonus to damage.
        \end{ability}

        \ff[6]{Spellsword Rhythm}[Magical] Whenever you hit a creature with a \glossterm{strike}, you gain a \plus1 bonus to \glossterm{accuracy} with spells against that creature during the next round.
        In addition, whenever you hit a creature with a \glossterm{spell}, you gain a \plus1 bonus to \glossterm{accuracy} with \glossterm{physical attacks} against that creature during the next round.

        \ff[9]{Greater Combat Caster} The Concentration bonus from your \textit{combat caster} ability increases to \plus4.
        In addition, the \glossterm{encumbrance} reduction increases to 2.

        \ff[12]{Greater Spellstrike}[Magical] The damage bonus from your \textit{spellstrike} ability increases to \plus4d.

        \ff[15]{Greater Spellsword Rhythm} The accuracy bonus from your \textit{spellsword rhythm} ability increases to \plus2.

        \ff[18]{Supreme Combat Caster} The Concentration bonus from your \textit{combat caster} ability increases to \plus6.
        In addition, the \glossterm{encumbrance} reduction increases to 3.

        \ff[20]{Supreme Spellstrike} The damage bonus from your \textit{spellstrike} ability increases to \plus5d.
    \end{feat}

    \begin{feat}{Enchanter}{Magical, Spell}
        \featpre Enchantment spell known.

        \ff[1]{Mind Fragments} When you use \glossterm{Mind} abilities, you can affect creatures that are immune to \glossterm{Mind} abilities due to not having a mind.
        You take a \minus5 penalty to accuracy on attacks against such creatures.
        This does not allow you to affect creatures who are immune to \glossterm{Mind} abilities for other reasons.

        \ff[3]{Enchanting Presence} You gain a \plus1 bonus to Intimidate and Persuasion.
        In addition, creatures within a 50-foot radius \glossterm{emanation} of you take a \minus1 penalty to Mental defense.
        You may freely exclude creatures you are aware of from this effect.

        \ff[6]{Subtle Influence} You gain a \plus1 bonus to accuracy with \glossterm{Mind} abilities.
        In addition, the DR to identify your \glossterm{Mind} abilities with Spellcraft, and to identify their effects with Sense Motive, increases by 5.

        \ff[9]{Greater Mind Fragments} The accuracy penalty from your \textit{mind fragments} ability decreases to \minus2.

        \ff[12]{Greater Enchanting Presence} The bonuses and penalties from your \textit{enchanting presence} ability increase to \plus2 and \minus2.

        \ff[15]{Greater Subtle Influence} The accuracy bonus from your \textit{subtle influence} ability increases to \plus2.
        In addition, the DR increase from that ability increases to 10.

        \ff[18]{Supreme Mind Fragments} The accuracy penalty from your \textit{mind fragments} ability is removed.

        \ff[20]{Mental Torment} Whenever you target a creature with a Mind ability, you may have that creature take a \minus1 penalty to Mental defense.
        This effect applies regardless of whether the ability's hits, but the target must be a valid target for the ability.
        It is a \glossterm{condition}, and lasts until it is removed.
        The penalty stacks with itself if you target the same creature multiple times.
    \end{feat}

    \begin{feat}{Escape Artist Specialization}{Skill}
        \featpre Escape Artist as a mastered skill.

        \ff{Rapid Escape} You can squeeze and escape bindings and grapples as a move action, rather than as a standard action.

        \ff[2]{Specialization} You gain a \plus2 bonus to Escape Artist.

        \ff[5]{Constraint Tolerance} You reduce your penalties for \glossterm{squeezing} by 1 (see \pcref{Squeezing}).

        \ff[8]{Greater Specialization} The bonus to Escape Artist from your \textit{specialization} ability increases to \plus4.

        \ff[11]{Escape Magic}[Magical] As a standard action, you can spend an \glossterm{action point} to use the \textit{escape magic} ability.
        \begin{ability}{Escape Magic}
            You make an Escape Artist attack against all \glossterm{magical} effects on you.
            You may exclude any number of effects you are aware of from this attack, allowing you to maintain beneficial magical effects.
            The DR for each effect is equal to 10 \add the effect's \glossterm{power}.
            \hit Each effect is \glossterm{dismissed}, if it is an effect that can be dismissed.
        \end{ability}

        You can only dismiss effects with this ability which target you directly, not area effects which include you as a target.
        If an ability targets multiple creatures, you can only remove its effects on you.

        \ff[14]{Supreme Specialization} The bonus to Escape Artist from your \textit{specialization} ability increases to \plus6.

        \ff[17]{Greater Constraint Tolerance} The penalty reduction from your \textit{constraint tolerance} ability increases to 2.
        In addition, your movement speed is not halved while \glossterm{squeezing}.

        \ff[20]{Greater Escape Magic}[Magical] You can use your \textit{escape magic} ability as a \glossterm{minor action}.
    \end{feat}

    \begin{feat}{Evoker}{Magical, Spell}
        \featpre Evocation spell known.

        \ff[1]{Energy Burst} As a standard action, you can spend an \glossterm{action point} to use the \textit{energy burst} ability.
        \begin{ability}{Energy Burst}[(see text)]
            Choose a creature or object within \rngclose range and a type of \glossterm{energy}: cold, electricity, fire, or sonic.
            You make a \glossterm{spellpower} attack against the target's Reflex defense if you chose electricity or fire, or against the target's Fortitude defense if you chose cold or sonic.
            \hit The target takes \glossterm{standard damage} \plus1d of the chosen damage type.

            This ability has the \glossterm{ability tag} corresponding to the type of energy chosen.
        \end{ability}

        \ff[3]{Energy Affinity} Choose a type of \glossterm{energy}: cold, electricity, fire, or sonic.
        You gain a \plus1d bonus to damage with attacks that deal damage of that type measured in dice.
        In addition, you gain damage reduction equal to your highest \glossterm{spellpower} against damage of that type.

        \ff[6]{Devastating Evocation} The area affected by your Evocation spells is doubled.

        \ff[9]{Residual Energy Burst} When your \textit{energy burst} attack hits, the target suffers an additional effect depending on the energy type chosen.
        These effects are \glossterm{conditions}, and last until they are removed.
        On a \glossterm{critical hit}, the attack still deals double damage in addition to the effects given here.
        \begin{itemize}
            \item Cold: If target is a creature, it is \fatigued.
                A \glossterm{critical hit} makes the target \exhausted instead.
            \item Electricity: If the target is a creature, it is \dazed.
                A \glossterm{critical hit} makes the target \stunned instead.
            \item Fire: The target is \ignited. This condition can be removed by putting out the fire.
            \item Sonic: If the target has \glossterm{hardness}, its hardness is reduced by an amount equal to half your spellpower. Otherwise, if it has \glossterm{damage reduction}, its damage reduction is reduced by amount equal to your spellpower.
        \end{itemize}

        \ff[12]{Greater Energy Affinity} The benefits from your \textit{energy affinity} apply to all energy types.

        \ff[15]{Devastating Caster} The area doubling from your \textit{devastating evocation} ability applies to all spells you cast, not just Evocation spells.

        \ff[18]{Greater Energy Burst} You gain a \plus1 bonus to \glossterm{accuracy} and a \plus1d bonus to damage with your \glossterm{energy burst} ability.

        \ff[20]{Supreme Energy Affinity} The damage bonus from your \textit{energy affinity} ability increases to \plus2d.
        In addition, the damage reduction increases to be equal to twice your highest spellpower.
    \end{feat}

    \begin{feat}{Executioner}{Combat}
        \featpres Starting Strength of 1, starting Perception of 2.

        \ff[1]{Execution} As a standard action, you can spend an \glossterm{action point} to use the \textit{execution} ability.
        \begin{ability}{Execution}
            Make a \glossterm{strike}.
            At the end of the current phase, if the target is \glossterm{bloodied}, it takes additional damage equal to the damage you dealt to it with your strike.
        \end{ability}

        \ff[3]{Purge the Weak} You gain a \plus1 bonus to \glossterm{accuracy} against \glossterm{bloodied} creatures.

        \ff[6]{Blood Sense}[Magical] You automatically know the location of all \glossterm{bloodied} living creatures within 50 feet of you, regardless of concealment or invisibility.
        You must have \glossterm{line of effect} to a creature to see it in this way.

        \ff[9]{Greater Execution} You gain a \plus1d bonus to damage with the strike from your \textit{execution} ability.

        \ff[12]{Greater Purge the Weak} The accuracy bonus from your \textit{purge the weak} ability increases to \plus2.

        \ff[15]{Greater Blood Sense}[Magical] Your \textit{blood sense} ability allows you to know the location of all living creatures within range, not just bloodied living creatures.
        You must have \glossterm{line of effect} to a creature to see it.

        \ff[18]{Supreme Execution} The damage bonus from your \textit{greater execution} ability increases to \plus2d.

        \ff[20]{Blood Sight}[Magical] Your \textit{blood sense} ability allows you to see creatures perfectly instead of only knowing their location.
    \end{feat}

    \begin{feat}{Guardian}{Combat}
        \featpre Starting Perception and Willpower of 1.

        \ff[1]{Binding Strike} As a standard action, you can spend an \glossterm{action point} to use the \textit{binding strike} ability.
        \begin{ability}{Binding Strike}
            Make a melee \glossterm{strike} against an adjacent creature.
            In addition to the strike's normal effects, you also compare the attack result against the target's Mental defense.

            \hit The target is \glossterm{immobilized} as a \glossterm{condition}.
            This effect immediately ends if you are \glossterm{defeated} or if you stop being adjacent to the target.
            \crit The target is \glossterm{immobilized} as a \glossterm{condition}.
        \end{ability}

        \ff[3]{Protect} As a \glossterm{minor action}, you can use the \textit{protect} ability.
        \begin{ability}{Protect}[\glossterm{Swift}]
            One ally adjacent to you gains a \plus2 bonus to Armor defense.
            However, you take a \minus2 penalty to Armor defense.
            This effect lasts until the end of the round.
        \end{ability}

        \ff[6]{Certain Bind} You gain a \plus1 bonus to accuracy with your \textit{binding strike} ability.

        \ff[6]{Defender} You gain a \plus2 bonus to \glossterm{threat}.

        \ff[9]{Redirection}[Magical] When you use your \textit{protect} ability, you can spend an \glossterm{action point} to use the \textit{redirection} ability.
        \begin{ability}{Redirection}
            You suffer all effects from \glossterm{strikes} that hit the target in place of the target as long as the \textit{protect} ability lasts.
            Any abilities you have that would make the attack miss or fail have no effect, but your abilities that allow you to reduce or ignore its effects work normally.
        \end{ability}

        \ff[12]{Greater Certain Bind} The accuracy bonus from your \textit{certain bind} ability increases to \plus2.

        \ff[12]{Greater Defender} The threat bonus from your \textit{defender} ability increases to \plus4.

        \ff[15]{Expanded Redirection}[Magical] When you use your \textit{redirection} ability, you redirect the effects of all attacks, not just \glossterm{strikes}.

        \ff[12]{Supreme Certain Bind} The accuracy bonus from your \textit{certain bind} ability increases to \plus3.

        \ff[18]{Supreme Defender} The bonus from your \textit{defender} ability increases to \plus6.

        \ff[20]{Supreme Protector} When you use your \textit{protect} ability, you may target any number of allies adjacent to you.
    \end{feat}

    \begin{feat}{Heal Specialization}{Skill}
        \featpre Heal as a mastered skill.

        \ff{Healing Touch}[Magical] As a standard action, you can spend an \glossterm{action point} to use the \textit{healing touch} ability.
        \begin{ability}{Healing Touch}[\glossterm{Life}]
            Make a Heal check on a willing creature you can touch.
            The target is healed for an amount equal to the Heal check result.
        \end{ability}

        \ff[2]{Specialization} You gain a \plus2 bonus to Heal.

        \ff[6]{Vital Healing} For every five points of healing you would restore with your \textit{healing touch}, you can instead heal a point of \glossterm{vital damage}.

        \ff[9]{Greater Specialization} The bonus from your \textit{specialization} ability increases to \plus4.

        \ff[11]{Purging Touch}[Magical] As a standard action, you can spend an \glossterm{action point} to use the \textit{purging touch} ability.
        \begin{ability}{Purging Touch}
            Make a Heal check on a willing creature you can touch.
            For each poison and disease on the target, if your check result is at least 10 higher than the \glossterm{power} of the effect, the effect is removed.
        \end{ability}

        \ff[14]{Supreme Specialization} The bonus from your \textit{specialization} ability increases to \plus6.

        \ff[17]{Lifesaver} You can stabilize dying creatures as a \glossterm{minor action}.

        \ff[20]{Wellspring of Life} You do not have to spend an \glossterm{action point} to use your \textit{healing touch} ability.
    \end{feat}

    \begin{feat}{Illusionist}{Magical, Spell}
        \featpre Illusion spell known.

        \ff[1]{Create Image} As a standard action, you can spend an \glossterm{action point} to use the \textit{create image} ability.
        \begin{ability}{Create Image}[\glossterm{Sensation}, \glossterm{Sustain} (minor)]
            You create a visual illusion of an object, creature, or force within \rngmed range.
            The figment's size must be no smaller than Tiny, and no larger than Large.
            The figment does not create sound, smell, or temperature.

            During the movement phase, you can move the figment anywhere within the range, with appropriate motions to simulate natural movement.
            For example, if you created the illusion of a squad of human guards, you could cause them to walk realistically across a room.
            The figments otherwise remain motionless, except for minor motions that simulate signs of life (if appropriate).
            If the figment ever leaves this ability's range, the effect immediately ends.

            The maximum intensity of a sensation created by this ability is not enough to have any significant detrimental effects on a human experiencing the sensation.
            For example, it can create a bright light, but not so bright that it would be physically painful to view.
            % should the light be visible outside the size of the illusion? If so, how to clarify?

            When you use this ability, you make a check with a bonus equal to your spellpower \add 5.
            Creatures can recognize the figment is created by illusory magic by interacting with it physically, or by making an Awareness check against a DR equal to your check result when using this ability.
            A creature gets a \plus10 bonus on this Awareness check when using senses which should be present in the figment, but which are missing.
        \end{ability}

        \ff[3]{Reflexive Illusion} You gain a \plus1 bonus to Armor defense, Disguise, Sleight of Hand, and Stealth.

        \ff[6]{Muffled Illusions} You learn the Silent \glossterm{augment}.
        In addition, the level cost to apply the Silent \glossterm{augment} to Illusion spells you cast is reduced by 1, to a minimum of 0 (see \pcref{Augments}).
        % How loud is the sound? How pungent is the smell? etc.

        \ff[9]{Control Image} While your \textit{create image} ability is active, you can use the \textit{control image} ability as a standard action.
        You gain the \textit{control image} ability.
        \begin{ability}{Control Image}[\glossterm{Swift}]
            You can directly control the movement of the figment from your \textit{create image} ability until the end of the round.
            You cannot alter the fundamental shape of the figment, but you can have it perform complex actions, such as pretending to fight other creatures or dancing.
            The figment's ability to simulate physical tasks that require dexterity or training, such as juggling, is limited by your own.
            You may need to make relevant checks to make the figment perform complex actions.
        \end{ability}

        \ff[12]{Hidden Illusions} You learn the Stilled \glossterm{augment}.
        In addition, the level cost to apply the Stilled augment to Illusion spells you cast is reduced by 1, to a minimum of 0 (see \pcref{Augments}).

        \ff[15]{Flexible Image} Your \textit{create image} ability can create figments between Diminuitive and Huge size.
        In addition, choose a sense: sound, smell, or temperature.
        Your \textit{create image} ability can create sensations with the chosen sense.

        \ff[18]{Supreme Reflexive Illusion} The bonus to Armor defense from your \textit{reflexive illusion} ability increases to \plus2, and the bonuses to skills increase to \plus3.

        \ff[20]{Innate Illusions} The level cost to apply both the Silent and Stilled \glossterm{augments} to Illusion spells is removed.
    \end{feat}

    \begin{feat}{Intimidate Specialization}{Skill}
        \featpre Intimidate as a mastered skill.

        \ff{Demoralizing Blow} As a standard action, you can use the \textit{demoralizing blow} ability.
        \begin{ability}{Demoralizing Blow}
            Make a \glossterm{strike}.
            % TODO: Timing is wrong
            If you deal damage to the target, you can immediately use the \textit{demoralize} ability on the target (see \pcref{Demoralize}).
        \end{ability}

        \ff[2]{Specialization} You gain a \plus2 bonus to Intimidate.

        \ff[6]{Critical Demoralization} If you get a \glossterm{critical hit} when you take the \textit{demoralize} action, the target is \frightened by you instead of being shaken.
        See \pcref{Demoralize}, for details.

        \ff[8]{Greater Specialization} The bonus from your \textit{specialization} ability increases to \plus4.

        \ff[14]{Supreme Specialization} The bonus from your \textit{specialization} ability increases to \plus6.
    \end{feat}

    \begin{feat}{Iron Will}{General}
        \featpre Starting Willpower of 2.

        \ff[1]{Mental Discipline} You gain a \plus2 bonus to Mental defense.
        In addition, you gain additional hit points equal to your Willpower.

        \ff[4]{Mind over Matter} You reduce your penalties for being \glossterm{staggered} by 2.

        \ff[7]{Unclouded Mind} You are immune to being \glossterm{dazed}.

        \ff[10]{Greater Mental Discipline} The bonus hit points from your \textit{mental discipline} ability increase to twice your Willpower.

        \ff[13]{Greater Mind over Matter} You do not take penalties for being \glossterm{staggered}.

        \ff[16]{Greater Unclouded Mind} You are immune to being \glossterm{stunned}.

        \ff[19]{Supreme Mental Discipline} The defense bonus from your \textit{mental discipline} ability increases to \plus4.

        \ff[20]{Mental Fortress} You are immune to all hostile \glossterm{Mind} abilities.
    \end{feat}

    \begin{feat}{Jump Specialization}{Skill}
        \featpre Jump as a mastered skill.

        \ff{Instant Leap} You must move at least five feet before jumping to have a running start, rather than twenty feet (see \pcref{Running Start}).

        \ff[2]{Specialization} You gain a \plus2 bonus to Jump.

        \ff[5]{Featherlight Leap} You gain the \textit{featherlight leap} ability.
        \begin{ability}{Featherlight Leap}[\glossterm{Swift}]
            As a \glossterm{free action}, you can spend an \glossterm{action point}.
            If you do, your maximum height for jumps during the current phase is equal to your Jump check result, rather than half your Jump check result.
        \end{ability}

        \ff[8]{Greater Specialization} The bonus from your \textit{specialization} ability increases to \plus4.

        \ff[11]{Greater Featherlight Leap} You do not need to spend an \glossterm{action point} to use your \textit{featherlight leap} ability.

        \ff[14]{Supreme Specialization} The bonus from your \textit{specialization} ability increases to \plus6.

        \ff[17]{Greater Instant Leap} You are always considered to have a running start when jumping, even when rebounding off of objects (see \pcref{Rebounding Leap}).

        \ff[20]{Impact Tolerance} You are immune to \glossterm{falling damage}.
    \end{feat}

    \begin{feat}{Knowledge Specialization}{Skill}
        \featpre Knowledge as a mastered skill.

        \ff{Knowledge Savant} You gain two additional \glossterm{skill points} which can only be spent on Knowledge skills.

        \ff[2]{Specialization} You gain a \plus2 bonus to Knowledge.

        \ff[5]{Greater Knowledge Savant} The number of extra skill points from your \textit{knowledge savant} ability increases to four.

        \ff[8]{Greater Specialization} The bonus from your \textit{specialization} ability increases to \plus4.

        \ff[14]{Supreme Specialization} The bonus from your \textit{specialization} ability increases to \plus6.
    \end{feat}

    \begin{feat}{Leadership}{Combat}
        \featpre Willpower 2.

        \ff[1]{Bolster} As a standard action, you can spend an \glossterm{action point} to use the \textit{bolster} ability.
        \begin{ability}{Bolster}[\glossterm{Mind}]
            All allies other than yourself within a \arealarge radius burst from you can remove one \glossterm{condition}.
            In addition, each target heals hit points equal to your Willpower.
        \end{ability}

        \ff[3]{Battle Command} As a standard action, you can use the \textit{battle command} ability.
        \begin{ability}{Battle Command}[\glossterm{Swift}]
            Choose an ally within \rngmed range of you.
            Until the end of the round, whenever it makes a \glossterm{physical attack}, it can roll the \glossterm{attack roll} twice and take the higher result.
        \end{ability}

        \ff[6]{Inspiring Presence} All allies other than yourself within a \arealarge radius emanation from you gain a \plus1 bonus to Mental defense.
        This ability is \glossterm{suppressed} if you are \glossterm{defeated}.

        \ff[9]{Greater Bolster} The healing from your \textit{bolster} ability increases to twice your Willpower.

        \ff[9]{Potent Command} The target of your \textit{battle command} ability also gains a \plus1d bonus to \glossterm{strike damage}.

        \ff[12]{Greater Inspiring Presence} The defense bonus from your \textit{inspiring presence} ability increases to \plus2.

        \ff[15]{Greater Potent Command} The damage bonus from your \textit{potent command} ability increases to \plus2d.

        \ff[18]{Supreme Inspiring Presence} The defense bonus from your \textit{inspiring presence} ability increases to \plus3.

        \ff[20]{War Leader} The area affected by your \textit{bolster} and \textit{inspiring presence} abilities increases to an \areahuge radius.
    \end{feat}

    \begin{feat}{Linguistics Specialization}{Skill}
        \featpre Linguistics as a mastered skill.

        \ff{Linguistic Savant} You learn two additional \glossterm{common languages}, or one additional \glossterm{rare language}.

        \ff[2]{Specialization} You gain a \plus2 bonus to Linguistics.

        \ff[5]{Language Focus} By spending a day in focused concentration on learning a specific \glossterm{common language}, you can use the \textit{language focus} ability.
        You must have access to either a willing creature fluent in the language or at least a book's worth of material written in the language.
        \begin{ability}{Language Focus}
            If you had access to written material on the language, including from a teacher, you can read or write the language.
            If you had access to a speaker of the language, you can speak and understand the language.
            
            Similarly, if the creature teaching you was unable to commu, you cannot read or write the language.

            This ability's effect lasts until you use this ability again.
        \end{ability}

        \ff[8]{Greater Specialization} The bonus from your \textit{specialization} ability increases to \plus4.

        \ff[11]{Greater Language Focus} You can use your \textit{language focus} ability to learn \glossterm{rare languages} in addition to common languages.

        \ff[14]{Supreme Specialization} The bonus from your \textit{specialization} ability increases to \plus6.
    \end{feat}

    \begin{feat}{Martial Training}{Combat}

        \ff[1]{Equipment Training} You choose one of the following benefits.
        \begin{itemize}
            \item Proficiency with a category of \glossterm{armor}: light, medium, heavy body armor, or shields.
                You must be proficient with light armor to gain proficiency with medium armor, and you must be proficient with medium armor to gain proficiency with heavy armor.
            \item You reduce the \glossterm{encumbrance} of \glossterm{body armor} you wear by 1.
                If you choose this ability multiple times, its effects stack.
            \item Proficiency with an additional \glossterm{weapon group} of your choice.
            \item Proficiency with \glossterm{exotic weapons} from a weapon group of your choice that you are already proficient with.
        \end{itemize}

        \ff[3]{Honed Strike} As a standard action, you can spend an \glossterm{action point} to use the \textit{honed strike} ability.
        \begin{ability}{Honed Strike}
            You make a \glossterm{strike} with a \plus1d bonus to damage.
            If the strike misses, you regain the \glossterm{action point} spent to use this ability.
        \end{ability}

        \ff[6]{Equipment Training} You gain an additional \textit{equipment training} ability of your choice.

        \ff[9]{Practiced Defense} You gain a \plus1 bonus to Armor defense.

        \ff[12]{Equipment Training} You gain an additional \textit{equipment training} ability of your choice.

        \ff[12]{Greater Honed Strike} The damage bonus from your \textit{honed strike} ability increases to \plus2d.

        \ff[15]{Practiced Accuracy} You gain a \plus1 bonus to \glossterm{accuracy} with \glossterm{physical attacks}.

        \ff[18]{Equipment Training} You gain an additional \textit{equipment training} ability of your choice.

        \ff[18]{Supreme Honed Strike} The damage bonus from your \textit{honed strike} ability increases to \plus3d.
    \end{feat}

    \begin{feat}{Miscaster}{Magical, Spell}
        \featpre Ability to cast a spell.

        \ff[1]{Selective Backlash} Your \glossterm{miscast backlash} does not hurt your allies, though it still hurts you.

        \ff[3]{Overchannel} Whenever you cast a spell, you can use the \textit{overchannel} ability.
        \begin{ability}{Overchannel}
            Your concentration on the spell you are casting cannot be disrupted, and you cannot \glossterm{miscast} the spell for any other reason.
            Effects that prevent the spell from having any effect, such as the \textit{counterspell} ability from the Abjurer feat, work normally (see \featpref{Abjurer}).
            In addition, you cause a \glossterm{miscast backlash} when the spell resolves.

            However, using this ability requires channeling excessive magical energy that temporarily limits your connection to magic.
            During the next round, you cannot cast spells other than \glossterm{cantrips}, and you take a \minus2 penalty to all defenses.
        \end{ability}

        \ff[6]{Suppressed Backlash} Whenever you miscast a spell, you can suppress the \glossterm{miscast backlash}.
        If you do, you regain the action point spent to cast the spell (if any).

        \ff[9]{Widened Backlash} The area affected by your \glossterm{miscast backlash} increases to a \areamed radius burst centered on you.

        \ff[12]{Resilient Channeler} In the round after you use your \textit{overchannel} ability, you can cast both cantrips and spells, but you cannot cast \glossterm{subspells} or apply \glossterm{augments} to your spells.
        In addition, the penalty to defenses is reduced to \minus1.

        \ff[15]{Magical Resilience} You gain \glossterm{magic resistance} equal to 5 \add your level.

        \ff[18]{Empowered Channeler} When you use your \textit{overchannel} ability, you gain a \plus2 bonus to spellpower with the spell.
    \end{feat}

    \begin{feat}{Mystic Archer}{Magical, Spell}
        \featpre Ability to cast a spell.

        \ff[1]{Imbue Projectile} Whenever you cast a spell that affects a single target, you can spend an \glossterm{action point} to use the \textit{imbue projectile} ability.
        \begin{ability}{Imbue Projectile}[\glossterm{Attune}]
            The spell does not have its effect immediately, and you regain the action point used to cast the spell (if any).
            Instead, its power is imbued in a \glossterm{projectile} you hold. 

            When you make a \glossterm{strike} using that projectile, the spell affects a target of the strike.
            After the spell takes effect, this ability's duration ends.
        \end{ability}

        \ff[3]{Missile Storm} As a standard action, you can spend an \glossterm{action point} to use the \textit{missile storm} ability.
        \begin{ability}{Missile Storm}
            Make a ranged \glossterm{strike} with a \glossterm{projectile weapon} you wield.
            The strike targets up to five creatures and objects within one \glossterm{range increment} of you with that weapon, except for creatures adjacent to you.
            You take a \minus1d penalty to damage with the strike.
        \end{ability}

        \ff[6]{Guided Projectiles} Your attacks with projectiles ignore \glossterm{cover}, but not \glossterm{total cover}.

        \ff[9]{Imbue Detonating Projectile} Whenever you cast a spell that affects an area, you can spend an \glossterm{action point} to use the \textit{imbue detonating projectile} ability.
        \begin{ability}{Imbue Detonating Projectile}[\glossterm{Attune}]
            The spell does not have its effect immediately, and you regain the action point used to cast the spell (if any).
            Instead, its power is imbued in a \glossterm{projectile} you hold. 

            When you make a \glossterm{strike} using that projectile, the spell takes effect in addition to the normal effects of the strike.
            The spell's \glossterm{point of origin} can be anywhere within the \glossterm{space} of a target of the strike.
            It affects targets within the area as if it had just been cast in that area.
            After the spell takes effect, this ability's duration ends.
        \end{ability}

        \ff[12]{Phasing Projectiles}[Teleportation] When attacking with projectiles, you can ignore all physical obstacles in single five-foot span.
        This can allow you to fire projectiles through solid walls, though it does not grant you the ability to see through the wall.

        \ff[15]{Greater Missile Storm} When you use your \textit{missile storm} ability, you may target any number of creatures.

        \ff[18]{Mystic Precision} You gain a \plus1 bonus to \glossterm{accuracy}.

        \ff[20]{Greater Phasing Projectiles}[Teleportation] Your \textit{phasing projectiles} ability improves, allowing you to ignore obstacles in up to four five-foot spans.
        The spans can be contiguous, which can allow you to ignore a single obstacle up to twenty feet long.
    \end{feat}

    \begin{feat}{Null}{General}
        \featpre Starting Willpower of 2.

        \ff[1]{Nullify Magic} You gain \glossterm{magic resistance} equal to 5 \add your Willpower.
        You cannot lower this magic resistance voluntarily, which can prevent you from being affected by non-hostile magical effects.
        In exchange, you lose the benefits of all \glossterm{magical} abilities you possess.
        In addition, you are unable to \glossterm{attune} to any \glossterm{magical} abilities, such as spells cast by other creatures.

        \ff[4]{Null Severance} As a standard action, you can spend an \glossterm{action point} to use the \textit{null severance} ability.
        \begin{ability}{Null Severance}
            Make a \glossterm{strike}.
            In addition to the strike's normal effects, you also compare the attack result against the target's Mental defense.

            \hit The target breaks its \glossterm{attunement} to a random ability that it is currently attuned to.
            This ability does not affect attunement to magic items.
            \crit The target breaks its attunement to all abilities that it is attuned to.
        \end{ability}

        % clarify that it does not affect spells already in progress in the current phase?
        \ff[7]{Disruptive Presence} All creatures adjacent to you have a 50\% chance to \glossterm{miscast} any spell they cast.

        \ff[10]{Greater Nullify Magic} You gain a \plus2 bonus to the \glossterm{magic resistance} from your \textit{nullify magic} ability.

        \ff[13]{Greater Null Severance} Your \textit{null severance} ability breaks the target's attunement to an additional random ability.

        \ff[16]{Greater Disruptive Presence} Your \textit{disruptive presence} ability affects all creatures in a \areamed radius \glossterm{emanation} from you.

        \ff[19]{Supreme Null Severance} Your \textit{null serverance} ability breaks the target's attunement to two additional random abilities.

        \ff[20]{True Null} You are unaffected by all \glossterm{magical} abilities.
    \end{feat}

    \begin{feat}{Perform Specialization}{Skill}
        \featpre Perform as a mastered skill.

        \ff{Mesmerizing Performance}[Magical] As a standard action, you can spend an \glossterm{action point} to use the \textit{mesmerizing performance} ability.
        \begin{ability}{Mesmerizing Performance}[\glossterm{Delusion}, \glossterm{Mind}, \glossterm{Sustain} (minor)]
            You begin a performance using one of your Perform skills.
            Make a Perform vs. Mental attack against up to five creatures within \rngmed range.
            \hit Each target is \fascinated by you.
            Any act by you or your apparent allies that damages a target or that causes it to feel that it is in danger breaks the effect for that creature.
            An observant target may interpret overt threats to its allies as a threat to itself.

            If a target can neither see nor hear your performance, the effect immediately ends for that target.
            This ability may have the \glossterm{Auditory} or \glossterm{Visual} tags, depending on the nature of your performance.
        \end{ability}

        \ff[2]{Specialization} You gain a \plus2 bonus to Perform.

        \ff[5]{Inspiring Performance}[Magical] As a standard action, you can spend an \glossterm{action point} to use the \textit{inspiring performance} ability.
        \begin{ability}{Inspiring Performance}[\glossterm{Delusion}, \glossterm{Mind}, \glossterm{Sustain} (minor)]
            You begin a performance using one of your Perform skills.
            One other willing creature within \rngmed range gains a \plus2 bonus to \glossterm{checks}.

            If the target can neither see nor hear your performance, the effect immediately ends for that target.
            This ability may have the \glossterm{Auditory} or \glossterm{Visual} tags, depending on the nature of your performance.
        \end{ability}

        \ff[8]{Greater Specialization} The bonus from your \textit{specialization} ability increases to \plus4.

        \ff[11]{Mesmeric Suggestion}[Magical] As a standard action, you can spend an \glossterm{action point} to use the \textit{mesmeric suggestion} ability.
        \begin{ability}{Mesmeric Suggestion}[\glossterm{Delusion}, \glossterm{Mind}, \glossterm{Subtle}, \glossterm{Sustain} (minor)]
            Make a Perform vs. Mental attack against a target within \rngmed range.
            You must also make a verbal suggestion of a particular course of action to the target.
            You can work this suggestion into an active performance without penalty.
            If your suggestion does not seem reasonable, you take a \minus5 penalty to accuracy on this attack.
            Exceptionally unreasonable suggestions can impose even greater penalties, and exceptionally reasonable suggestions can give accuracy bonuses.

            \hit As a \glossterm{condition}, the target thinks your suggestion is a good idea and will try to follow it to the best of its abilities.
            Any act by you or your apparent allies that damages the target or makes it feel that it is in danger breaks the effect.

            If the target is not currently \fascinated by your \textit{mesmerizing performance} ability, this attack automatically fails.
            If the target can neither see nor hear your performance, the effect immediately ends for that target.
            This ability lasts as long as you sustain your \textit{mesmerizing performance}.
        \end{ability}

        \ff[14]{Supreme Specialization} The bonus from your \textit{specialization} ability increases to \plus6.

        \ff[17]{Projected Performance} The range of your \textit{mesmerizing performance}, \textit{inspiring performance}, and \textit{mesmeric suggestion} abilities increases to \rnglong.

        \ff[20]{Mass Performance} When you use your \textit{mesmerizing performance} and \textit{inspiring performance} abilities, you can target any number of creatures within range.
    \end{feat}

    \begin{feat}{Persuasion Specialization}{Skill}
        \featpre Persuasion as a mastered skill.

        \ff{Compel Attention}[Magical] As a standard action, you can spend an \glossterm{action point} to use the \textit{compel attention} ability.
        \begin{ability}{Compel Attention}[\glossterm{Auditory}, \glossterm{Compulsion}, \glossterm{Mind}, \glossterm{Sustain} (minor), \glossterm{Subtle}]
            Make a Persuasion vs. Mental attack against a creature within \rngmed range.
            You must talk loud enough for the target to hear hear to draw its attention.
            \hit The target is \glossterm{fascinated} by you as long as you sustain this ability, which requires maintaining your conversation with it.
            Any act by you or your apparent allies that damages a target or that causes it to feel that it is in danger breaks the effect for that creature.
            An observant target may interpret overt threats to its allies as a threat to itself.
        \end{ability}

        \ff[2]{Specialization} You gain a \plus2 bonus to Persuasion.

        \ff[5]{Center of Attention} When you use your \textit{compel attention} ability, you may target up to five creatures.

        \ff[8]{Greater Specialization} The bonus from your \textit{specialization} ability increases to \plus4.

        \ff[11]{Suggestion} As a standard action, you can spend an \glossterm{action point} to use the \textit{suggestion} ability.
        \begin{ability}{Suggestion}[\glossterm{Delusion}, \glossterm{Mind}, \glossterm{Subtle}, \glossterm{Sustain} (minor)]
            Make a Persuasion vs. Mental attack against a target within \rngmed range.
            You must also make a verbal suggestion of a particular course of action to the target.
            If your suggestion does not seem reasonable, you take a \minus5 penalty to accuracy on this attack.
            Exceptionally unreasonable suggestions can impose even greater penalties, and exceptionally reasonable suggestions can give accuracy bonuses.

            \hit As a \glossterm{condition}, the target thinks your suggestion is a good idea and will try to follow it to the best of its abilities.
            Any act by you or your apparent allies that damages the target or makes it feel that it is in danger breaks the effect.

        \end{ability}

        \ff[14]{Supreme Specialization} The bonus from your \textit{specialization} ability increases to \plus6.
    \end{feat}

    \begin{feat}{Precognition}{Combat}
        \featpre Starting Intelligence of 2.

        \ff[1]{Precognitive Strike} As a standard action, you can use the \textit{precognitive strike} ability.
        \begin{ability}{Precognitive Strike}
            Make a \glossterm{strike}.
            You can use your Intelligence in place of your Perception to determine your accuracy with this strike.
        \end{ability}

        \ff[3]{Combat Prediction} At the start of each phase, you can spend an \glossterm{action point} to use the \textit{combat prediction} ability.
        \begin{ability}{Combat Prediction}
            Make an Intelligence vs. Mental attack against a creature within \rngmed range of you.
            \hit You gain insight into the actions that creature intends to take during the current phase.
                It may change its actions based on your interference if you communicate this information in a way it understands.
            \miss You regain the action point spent to use this ability.

            The insight from this ability allows you to see and hear what actions that creature intends to take.
            You do not gain any knowledge of actions that have no obvious signs, such as purely mental actions, or actions which you are not observant enough to notice.
            In addition, you do not know the results of actions with a chance of failure, such as attacks.
        \end{ability}

        \ff[6]{Greater Precognitive Strike} When you use your \textit{precognitive strike} ability, you can also use your Intelligence to determine your damage with the strike in place of your Strength.

        \ff[9]{Precognitive Reaction} You gain a \plus2 bonus to Reflex defense and \glossterm{initiative} checks.

        \ff[12]{Foresight} During the \glossterm{movement phase}, you choose your action after all other creatures have chosen their actions.
        When you choose your action, you have insight into the actions chosen by any creatures within \rngclose range of you that you can see.
        This insight gives you the same information as the insight from your \textit{combat prediction} ability.
        You choose your actions simultaneously with any other creatures who have a similar ability.

        \ff[15]{Supreme Precognitive Strike} When you use your \textit{precognitive strike} ability, you also compare your attack result against the target's Mental defense.
        \hit You gain insight into the actions that creature intends to take during each phase until the end of the next round.
        This insight gives you the same information as the insight from your \textit{combat prediction} ability.
        You gain a \plus2 bonus to accuracy with your \textit{combat prediction} ability.

        \ff[18]{Precognitive Defense} You gain a \plus1 bonus to Armor defense. 

        \ff[20]{Greater Combat Prediction} When you use your \textit{combat prediction} ability, you can target up to five creatures within range.
    \end{feat}

    \begin{feat}{Regenerator}{General}
        \featpre Starting Constitution of 2.

        \ff[1]{Regenerative Rest} Whenever you take a \glossterm{short rest}, you can heal vital damage equal to half your Constitution in addition to your normal healing.

        \ff[4]{Diehard} You reduce your penalties from \glossterm{vital damage} by an amount equal to half your Constitution.

        \ff[7]{Fast Healing} At the end of each \glossterm{action phase}, you heal hit points equal to half your Constitution.

        \ff[10]{Regeneration} If you have taken \glossterm{vital damage}, the healing from your \textit{fast healing} ability heals vital damage instead of hit points until you have no vital damage remaining.

        \ff[13]{Unkillable} The penalty reduction from your \textit{diehard} ability increases to be equal to your Constitution.

        \ff[16]{Greater Fast Healing} The healing from your \textit{fast healing} ability increases to be equal to your Constitution.

        \ff[19]{Battlescarred Skin} You gain a \plus1 bonus to Armor defense.

        \ff[20]{Indestructible} The penalty reduction from your \textit{diehard} ability increases to be equal to twice your Constitution.
    \end{feat}

    \begin{feat}{Ride Specialization}{Skill}
        \featpre Ride as a mastered skill.

        \ff{Mounted Defense} Your mount gains a \plus2 bonus to all defenses, up to a maximum of your own corresponding defense.

        \ff[2]{Specialization} You gain a \plus2 bonus to Ride.

        \ff[5]{Mounted Warrior} The penalty you take when using a ranged weapon while mounted is decreased by 4.
        In addition, while you are mounted, you gain a \plus1d bonus to damage with Mounted weapons (see \pcref{Mounted Weapon}).

        \ff[8]{Greater Specialization} The bonus from your \textit{specialization} ability increases to \plus4.

        \ff[11]{Greater Mounted Defense} The defense bonus from your \textit{mounted defense} ability increases to \plus4.

        \ff[14]{Supreme Specialization} The bonus from your \textit{specialization} ability increases to \plus6.

        \ff[17]{Greater Mounted Warrior} The penalty reduction from your \textit{mounted warrior} ability increases to \minus8.
        In addition, you gain a \plus1 bonus to Armor defense while mounted.

        \ff[20]{Supreme Mounted Defense} The defense bonus from your \textit{mounted defense} ability increases to \plus6.
    \end{feat}

    \begin{feat}{Savage}{Combat}
        \featpre Starting Strength of 2.

        \ff[1]{Brute Force} You gain a \plus2 bonus to \glossterm{accuracy} with the \textit{shove} and \textit{overrun} abilities (see \pcref{Shove}, and \pcref{Overrun}).
        In addition, if you get a \glossterm{critical hit} with the \textit{shove} ability, you can move the target a maximum distance equal to your movement speed.

        \ff[3]{Wall Slam} If you use the \textit{shove} ability to move a creature, and the creature's movement is interrupted by a solid obstacle, the obstacle and creature both take bludgeoning \glossterm{standard damage} -1d.
        Your \glossterm{power} with this ability is based on your Strength.

        \ff[6]{Trample} If you use the \textit{overrun} ability to move through a creature, it takes bludgeoning \glossterm{standard damage} -3d.
        Your \glossterm{power} with this ability is based on your Strength.

        \ff[9]{Greater Brute Force} The accuracy bonus from your \textit{brute force} ability increases to \plus3.

        \ff[12]{Crushing Maneuvers} You gain a \plus1d bonus to damage with your \textit{wall slam} and \textit{trample} abilities.

        \ff[15]{Inescapable} Whenever you use the \textit{overrun} ability, you may choose not to allow creatures to try to avoid you.

        \ff[18]{Supreme Brute Force} The accuracy bonus from your \textit{brute force} ability increases to \plus4.

        \ff[20]{Greater Crushing Maneuvers} The damage bonus from your \textit{crushing maneuvers} ability increases to \plus2d. 
    \end{feat}

    \begin{feat}{Sense Motive Specialization}{Skill}
        \featpre Sense Motive as a mastered skill.

        \ff{Read Mind}[Magical] As a standard action, you can spend an \glossterm{action point} to use the \textit{read mind} ability.
        \begin{ability}{Read Mind}[\glossterm{Mind}, \glossterm{Sustain} (minor)]
            Make a Sense Motive vs. Mental attack against a creature within \rngclose range.
            \hit You know the target's surface thoughts.
            This grants you a \plus2 bonus to Bluff, Persuasion, and Intimidate attacks and checks against the target.
        \end{ability}

        \ff[2]{Specialization} You gain a \plus2 bonus to Sense Motive.

        \ff[8]{Greater Specialization} The bonus to Sense Motive from your \textit{specialization} ability increases to \plus4.

        \ff[14]{Supreme Specialization} The bonus to Sense Motive from your \textit{specialization} ability increases to \plus6.
    \end{feat}

    \begin{feat}{Sleight of Hand Specialization}{Skill}
        \featpre Sleight of Hand as a mastered skill.

        \ff{}

        \ff[2]{Specialization} You gain a \plus2 bonus to Sleight of Hand.

        \ff[8]{Greater Specialization} The bonus to Sleight from your \textit{specialization} ability increases to \plus4.

        \ff[10]{Extradimensional Concealment}[Magical] Whenever you use the \textit{conceal object} ability, you can spend an \glossterm{action point}.
        If you do, you conceal the object in a pocket dimension that cannot be accessed by nonmagical means.
        You can only hide one object at a time in this way.

        \ff[10]{Extradimensional Retrieval}[Magical] As a standard action, you can spend an \glossterm{action point} to use the \textit{extradimensional retrieval} ability.
        \begin{ability}{Extradimensional Retrieval}
            You reach into your pocket dimension to retrieve the object you stored there previously.

            Alternately, you can reach into the pocket dimension belonging to a creature you are touching to retrieve the object stored there.
            If that creature does not have the \textit{extradimensional concealment} ability, or does not have an object in their pocket dimension, this ability fails.
        \end{ability}

        \ff[14]{Supreme Specialization} The bonus to Sleight from your \textit{specialization} ability increases to \plus6.
    \end{feat}

    \begin{feat}{Sniper}{Combat}
        \featpre Starting Perception of 2.

        \ff[1]{Aim} As a standard action, you can use the \textit{aim} ability.
        \begin{ability}{Aim}[\glossterm{Sustain} (minor)]
            Choose a creature or object within line of sight.
            You gain a \plus2 bonus to accuracy on \glossterm{physical attacks} against the target.

            If you lose sight of the target for a full round, this effect ends.
        \end{ability}

        \ff[3]{Distance Tolerance} You reduce your accuracy penalties from \glossterm{range increments} by 2.

        \ff[6]{Sniper Shot} You gain a \plus2d bonus to damage on \glossterm{strikes} against \unaware creatures that are affected by your \textit{aim} ability.

        \ff[9]{Failure Tolerance} You ignore effects that give you a 20\% miss chance or failure chance with physical ranged attacks, such as \glossterm{concealment}.

        \ff[12]{Greater Distance Tolerance} The penalty reduction from your \textit{distance tolerance} ability increases to 4.

        \ff[15]{Sustained Aim} You can sustain your \textit{aim} ability as a \glossterm{free action}.

        \ff[18]{Greater Sniper Shot} The bonus to damage increases to \plus4d.

        \ff[20]{Rapid Aim} You can spend a \glossterm{action point} to use your \textit{aim} ability as a \glossterm{minor action}.
    \end{feat}

    \begin{feat}{Spellcraft Specialization}{Skill}
        \featpre Spellcraft as a mastered skill.

        \ff{Detect Spellcasting}[Magical] As a standard action, you can spend an \glossterm{action point} to use the \textit{detect spellcasting} ability.
        \begin{ability}{Detect Spellcasting}[\glossterm{Knowledge}, \glossterm{Subtle}]
            Make a Spellcraft attack against the Mental defense of a creature within \rngmed range.
            \hit You know whether the target is capable of casting spells.
            If the target can cast spells, you know what sources the target can cast spells from.
            \crit As above, except that you also know all spells the target is capable of casting.
            This does not grant you knowledge of any subspells or augments the target knows.

            After using this ability on a target, you cannot use it again on the same target for 24 hours regardless of whether you hit or miss.
        \end{ability}

        \ff[2]{Specialization} You gain a \plus2 bonus to Spellcraft.

        \ff[5]{Unweave Magic} As a standard action, you can spend an \glossterm{action point} to use the \textit{unweave magic} ability.
        \begin{ability}{Unweave Magic}[\glossterm{Mystic}]
            Make a Spellcraft check on an active spell effect within \rngmed range.
            The DR is equal to 5 \add the \glossterm{spellpower} of the effect.
            Success means the effect is \glossterm{dismissed} if it is an effect that can be dismissed.
        \end{ability}

        \ff[8]{Greater Specialization} The bonus from your \textit{specialization} ability increases to \plus4.

        \ff[11]{Mystic Tolerance} You gain a \plus1 bonus to Fortitude and Mental defense.

        \ff[14]{Supreme Specialization} The bonus from your \textit{specialization} ability increases to \plus6.

        \ff[17]{Mass Unweave Magic} When you use your \textit{unweave magic} ability, you can target up to five spell effects.

        \ff[20]{Greater Mystic Tolerance} The defense bonus from your \textit{mystic tolerance} ability increases to \plus2.
    \end{feat}

    \begin{feat}{Stealth Specialization}{Skill}
        \featpre Stealth as a mastered skill.

        \ff{}

        \ff[2]{Specialization} You gain a \plus2 bonus to Stealth.

        \ff[5]{Movement Tolerance} Your penalties for moving while hiding are reduced by 2. 

        \ff[8]{Greater Specialization} The bonus from your \textit{specialization} ability increases to \plus4.

        \ff[11]{Hide in Plain Sight} You can use the \textit{hide} ability even while observed.
        You take take a \minus10 penalty to the Stealth check when hiding in this way, and you still need passive cover or concealment to hide.

        \ff[14]{Supreme Specialization} The bonus from your \textit{specialization} ability increases to \plus6.

        \ff[17]{Greater Movement Tolerance} The reduction of penalties for moving increases to 5.
        This allows you to move at half speed without penalty.

        \ff[20]{Greater Hide in Plain Sight} The penalty for hiding while observed is reduced to \minus5.
    \end{feat}

    \begin{feat}{Survival Specialization}{Skill}
        \featpre Survival as a mastered skill.

        \ff{Terrain Tolerance} You ignore \glossterm{difficult terrain} and harmful natural terrain of any kind.
        If a skill check, such as Climb or Swim, would normally be required to move through the terrain, this ability does not help.

        \ff[2]{Specialization} You gain a \plus2 bonus to Survival.

        \ff[6]{Rapid Tracker}
        While following trails with the \textit{track} ability, you can move at your normal speed while following tracks without taking the normal \minus5 penalty.
        In addition, you can use the ability as a \glossterm{minor action} by taking a \minus10 penalty to the check.
        This can allow you to move twice in the same round.

        \ff[8]{Greater Specialization} The bonus from your \textit{specialization} ability increases to \plus4.

        \ff[11]{Planar Tolerance} You are immune to harmful effects imposed by being on other planes.

        \ff[14]{Supreme Specialization} The bonus from your \textit{specialization} ability increases to \plus6.

        \ff[17]{Greater Rapid Tracker} If you use the \textit{track} ability as a \glossterm{minor action}, the check penalty is reduced to \minus5 instead of \minus10.

        \ff[20]{Find the Path} As a standard action, you can spend an \glossterm{action point} to use the \textit{find the path} ability.
        \begin{ability}{Find the Path}[\glossterm{Attune}, \glossterm{Knowledge}]
            When you use this ability, you must unambiguously specify a location on the same plane as you.
            You know exactly what direction you must travel to reach your chosen destination by the most direct physical route.
            You are not always led in the exact direction of the destination -- if there is an impassable obstacle between the target and the destination, this ability will direct you around the obstacle, rather than through it.

            The guidance provided by this ability adjusts to match whatever your current physical capabilities are, including flight and other unusual movement modes. It does not see into the future, and changing circumstances may cause the most direct path to change over time.
            It also does not consider hostile creatures, traps, and other passable dangers which may endanger or slow progress.
        \end{ability}
    \end{feat}

    \begin{feat}{Swift}{General}
        \featpre Starting Dexterity of 2.

        \ff[1]{Rapid Movement} You gain a \plus10 foot bonus to speed in all your movement modes.

        \ff[3]{Sprinter} When you use the \textit{sprint} ability, you move at triple your movement speed.

        \ff[6]{Wall Runner} You gain a \plus5 bonus to checks with the \textit{wallrun} ability (see \pcref{Wallrun}).
        In addition, you can make a Dexterity check in place of a Climb check to use that ability.

        \ff[9]{Water Runner} When you use the \textit{sprint} ability, you can move on water and similar liquids as if they were solid ground.

        \ff[12]{Greater Rapid Movement} The speed bonus from your \textit{rapid movement} ability increases to \plus30 feet.

        \ff[15]{Greater Sprinter} When you use the \textit{sprint} ability, you move at quadruple your movement speed.

        \ff[18]{Cloud Runner} When you use the \textit{sprint} ability, you can move on dense fog and similar gaseous substances as if they were solid ground.

        \ff[20]{Endless Endurance} While you are not using the \textit{sprint} ability, you move at double your movement speed.
    \end{feat}

    \begin{feat}{Swim Specialization}{Skill}
        \featpre Swim as a mastered skill.
        % Maybe add burrow speed? The book needs a description of fighting underwater penalties.

        \ff{Underwater Tolerance} You reduce your penalties for fighting underwater by 2, except for penalties to physical ranged attacks.

        \ff[2]{Specialization} You gain a \plus2 bonus to Swim.

        \ff[5]{Swim Speed} You gain a \glossterm{swim speed} equal to your \glossterm{base speed}.
        A successful Swim check to move allows you to move a distance equal to your swim speed.

        \ff[8]{Greater Specialization} The bonus from your \textit{specialization} ability increases to \plus4.

        \ff[11]{Greater Underwater Tolerance} You do not suffer penalties for fighting underwater, except that you still suffer the normal penalties to physical ranged attacks.

        \ff[14]{Supreme Specialization} The bonus from your \textit{specialization} ability increases to \plus6.

        \ff[17]{Earth Swimmer} You can swim through loose earth and dirt as if it were water.

        \ff[20]{Greater Swim Speed} You gain a \plus30 foot bonus to your swim speed, up to a maximum swim speed equal to twice your \glossterm{base speed}.
    \end{feat}

    \begin{feat}{Toughness}{General}
        \featpre Starting Constitution of 2.

        \ff[1]{Durability} You gain a \plus2 bonus to Fortitude defense.
        In addition, you gain additional hit points equal to your Constitution.

        \ff[4]{Injury Tolerance} You reduce your penalties for being \glossterm{bloodied} by 2.

        \ff[7]{Ailment Tolerance} You are immune to being \glossterm{sickened} and \glossterm{fatigued}.
        This allows you to sleep while you have \glossterm{encumbrance} without penalty.

        \ff[10]{Greater Durability} The bonus hit points from your \textit{durability} ability increase to twice your Constitution.

        \ff[13]{Greater Injury Tolerance} You do not take penalties for being \glossterm{bloodied}.

        \ff[16]{Greater Ailment Tolerance} You are immune to being \glossterm{nauseated} and \glossterm{exhausted}.
        In addition, you need half the normal amount of rest and sleep each day to function normally.
        For example, a human would only need four hours of sleep per night.

        \ff[19]{Supreme Durability} The defense bonus from your \textit{durability} ability increases to \plus4.

        \ff[20]{Unflinching Recovery} Whenever you take the \textit{recover} or \textit{desperate recovery} actions, you take half damage from all attacks until the end of the round.
        This is a \glossterm{Swift} ability.
    \end{feat}

    \begin{feat}{Transmuter}{Magical, Spell}
        \featpre Transmutation spell known.

        \ff[1]{Enhance Body} You gain a \plus1 bonus to Armor defense and Strength-based \glossterm{checks}.

        \ff[3]{Reshaper} As a standard action, you can spend an \glossterm{action point} to use the \textit{alter self} or \textit{alter object} ability.
        \begin{ability}{Alter Self}[\glossterm{Shaping}]
            Make a Disguise check to alter your appearance (see \pcref{Disguise Creature}), except that you can use your spellpower in place of your Disguise skill.
            You can only alter your physical body, not your clothes or equipment.

            This ability lasts until you use it again.
        \end{ability}

        \begin{ability}{Alter Object}[\glossterm{Shaping}]
            Choose an unattended, nonmagical object you can touch.
            You make a Craft check to alter the target (see \pcref{Craft}), except that you can use your spellpower in place of your Craft skill.
            In addition, you do not need any special tools to make the check (such as an anvil and furnace).
            However, the maximum hardness of a material you can affect with this ability is equal to your spellpower.

            % too short?
            Each time you use this ability, you can accomplish work that would take up to five minutes with a normal Craft check.
        \end{ability}

        \ff[6]{Transmutive Vigor} At the end of each \glossterm{action phase}, you heal hit points equal to half your highest \glossterm{spellpower}.

        \ff[9]{Greater Reshaper} When you use your \textit{alter object} ability, you can accomplish work that would take up to an hour with a normal Craft check.
        In addition, as a standard action, you can spend an \glossterm{action point} to use the \textit{alter poison} ability.
        \begin{ability}{Alter Poison}[\glossterm{Shaping}, \glossterm{Sustain} (minor)]
            Make a \glossterm{spellpower} vs. Fortitude attack against a creature within \rngclose range.
            \hit Any poison in the target's system is neutralized.
            It stops suffering any additional effects from poisons in its system.
            As long as the effect lasts, it is immune to all poisons.
            In addition, the target's \glossterm{mundane} poisons, including natural attacks that inflict poison, have no effect.
        \end{ability}

        \ff[12]{Greater Transmutive Vigor} The healing from your \textit{transmutive vigor} ability increases to be equal to your highest \glossterm{spellpower}.

        \ff[15]{Greater Enhance Body} The bonuses from your \textit{enhance body} ability increase to \plus2.

        \ff[18]{Supreme Reshaper} You do not need to spend action points to use your \textit{alter self}, \textit{alter object}, or \textit{alter poison} abilities.

        \ff[20]{Malleable Body} You are immune to \glossterm{critical hits} from \glossterm{strikes}.
        You suffer only the effects of a normal hit instead.
    \end{feat}

    \begin{feat}{Vivimancer}{Magical, Spell}
        \featpre Vivimancy spell known.

        \ff[1]{Restore Life} As a standard action, you can spend an \glossterm{action point} to use the \textit{restore life} ability.
        \begin{ability}{Restore Life}[\glossterm{Life}]
            Choose a willing creature within \rngmed range.
            The target is healed for hit points equal to your \glossterm{standard damage} \plus1d.
            Your \glossterm{power} with this ability is equal to your highest \glossterm{spellpower}.
        \end{ability}

        \ff[3]{Unliving Resilience} You are immune to \glossterm{disease} and hostile \glossterm{Death} effects.

        \ff[6]{Vivimantic Surge} Whenever you cast a Vivimancy spell, you can heal any number of willing targets of the spell for hit points equal to your \glossterm{spellpower} with that spell.
        In addition, you gain a \plus1d bonus to damage with attacks that deal life damage.

        \ff[9]{Greater Restore Life} The healing granted by your \textit{restore life} ability increases by \plus1d.
        In addition, for every 2 points of healing, you can instead heal 1 \glossterm{vital damage}.

        \ff[12]{Greater Unliving Resilience} You are immune to life damage and hostile \glossterm{Life} effects.

        \ff[15]{Greater Vivimantic Surge} The healing granted by your \textit{vivimantic surge} ability increases to three times your \glossterm{spellpower} with the spell.
        In addition, the damage bonus increases to \plus2d.

        \ff[18]{Supreme Restore Life} The healing granted by your \textit{restore life} ability increases by an additional \plus1d.
        In addition, it no longer costs an action point to use.

        \ff[20]{Supreme Unliving Resilience} You gain a \plus4 bonus to Fortitude defense.
    \end{feat}

    \begin{feat}{Whirlwind Warrior}{Combat}
        \featpres Starting Dexterity of 2, starting Perception of 1.

        \ff[1]{Whirlwind Spin} As a standard action, you can spend an \glossterm{action point} to use the \textit{whirlwind spin} ability.
        \begin{ability}{Whirlwind Spin}
            Make a melee \glossterm{strike} against any number of creatures and objects adjacent to you that you \glossterm{threaten}.
            You must use the same weapon to make each strike.
            Use the same attack result and damage against each target.
            You take a \minus2d penalty to damage on each strike.
        \end{ability}

        \ff[3]{Eye of the Storm} You gain a \plus1 bonus to \glossterm{overwhelm resistance}.

        \ff[6]{Greater Whirlwind Spin} The damage penalty on the strike you make with your \textit{whirlwind spin} ability is reduced to \minus1d.

        \ff[9]{Unfettered Movement} During each phase, you may move through one creature's space during movement.
        You move at half speed while in its space.
        After moving through that creature's space, other creatures block you as normal for the remainder of your movement.
        Certain unusual creatures that occupy their entire space, such as gelatinous cubes, may be immune to this ability.

        \ff[12]{Supreme Whirlwind Spin} The damage penalty with your \textit{whirlwind spin} ability is removed.

        \ff[15]{Greater Eye of the Storm} The bonus from your \textit{eye of the storm} ability increases to \plus2.

        \ff[18]{Legendary Whirlwind Spin} You gain a \plus1d bonus to damage with your \textit{whirlwind spin} ability.

        \ff[20]{Greater Unfettered Movement} Using your \textit{unfettered movement} ability does not cause you to move at half speed while in the creature's space.
    \end{feat}

\section{Other Feat Rules}

    \subsection{Retraining Feats}
        At every level, you can choose to retrain an old feat in exchange for a new feat.
