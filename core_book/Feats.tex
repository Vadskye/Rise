\chapter{Feats}\label{Feats}

Feats are special abilities that every character has.
Feats can be used to specialize your character particular area, to grant your character new abilities, or to change the way your character does certain things.

\section{Gaining Feats}
    Your character gains two feats at 1st level, a third at 3rd level, a fourth at 6th level, and a fifth at 10th level.
    % TODO: are there any bonus fetas
    Classes can sometimes grant bonus feats as well, which are in addition to these feats which every character gets.
    A character cannot gain the same feat twice.

    \subsection{Prerequisites}
        Some feats have prerequisites.
        Unless your character has all of the prerequisites, they cannot take the feat.
        Prerequisites can include a minimum starting attribute score, another feat or feats, a minimum level of training in a skill, or some other property of a character.
        A character can gain a feat at the same level at which he or she gains the prerequisite.

        A character can't use a feat if he or she has lost a prerequisite.

\section{Feat Tags}
    All feats are organized into different groups by tags.

    \parhead{General} General feats can have a wide variety of effects.
    They often grant new abilities or improve your defenses.

    \parhead{Combat} Combat feats improve your combat capabilities.
    They can increase the damage you deal, grant you new combat abilities, or improve your defenses.

    \parhead{Spell} Spell feats improve your spellcasting abilities.
    All Spell feats except for the Ritual Caster feat are useless to characters who cannot cast spells (see \featpcref{Ritual Caster}).

    \parhead{Skill} Skill feats improve your skills.
    They can make you more likely to succeed with skill checks and grant you new abilities based on your skills.

    \parhead{Bloodline Feats} Some characters have traces of monstrous blood running in their veins.
    Most of those will never understand the full potential of their unusual heritage.
    Bloodline feats allow characters to explore those posibilities by gaining abilites related to their ancestry.
    You can only have one Bloodline feat.

    \parhead{Magical Feats}
    Magical feats are \glossterm{magical} in nature.
    Many feats are not entirely magical, but have specific effects that are magical.
    These effects are indicated by the \magical tag.

\section{Feat Tables}
    \newpage\onecolumn

% Feat names must follow ``have'', ``are (a)'', or ``can''.
\begin{longtabuwrapper}
    \begin{longtabu}{>{\lcol}p{10em} >{\lcol}p{15em} >{\lcol}X >{\lcol}p{8em} >{\lcol}p{3em}}
        \lcaption{Feats}\\
        \tb{General Feats}\label{General Feats} & \tb{Prerequisites} & \tb{Benefits} & \tb{Feat Types} & \tb{Page} \\
        \featref{Celestial Heritage} & Non-evil & Gain aspects of celestial beings & Bloodline, Magical & \featpref{Celestial Heritage} \\
        \featref{Draconic Heritage} & Gain aspects of draconic power & \tdash & Bloodline & \featpref{Draconic Heritage} \\
        \featref{Iron Will} & Wil 2 & Increase mental resilience & \tdash & \featpref{Iron Will} \\
        \featref{Toughness} & Con 2 & Increase physical fortitude & \tdash & \featpref{Toughness} \\

        \tb{Class Feats}\label{Class Feats} & \tb{Prerequisites} & \tb{Benefits} & \tb{Feat Types} & \tb{Page} \\
        \featref{All Energy Becomes One} & Monk base class, Con 2 & Absorb and redirect damage & Magical & \featpref{All Energy Becomes One} \\

        \tb{Skill Feats}\label{Skill Feats} & \tb{Prerequisites} & \tb{Benefits} & \tb{Feat Types} & \tb{Page} \\
        \featref{Acrobatics Specialization} & Mastered Acrobatics & \tdash & \tdash & \featpref{Acrobatics Specialization} \\
        \featref{Awareness Specialization} & Mastered Awareness & \tdash & \tdash & \featpref{Awareness Specialization} \\
        \featref{Bardic Exemplar} & Perform Specialization & Mock foes and bolster allies with performances & Magical & \featpref{Bardic Exemplar} \\
        \featref{Bluff Specialization} & Mastered Bluff & \tdash & \tdash & \featpref{Bluff Specialization} \\
        \featref{Climb Specialization} & Mastered Climb & \tdash & \tdash & \featpref{Climb Specialization} \\
        \featref{Craft Specialization} & prerequisites & \tdash & \tdash & \featpref{Craft Specialization} \\
        \featref{Creature Handling Specialization} & Mastered Creature Handling & \tdash & \tdash & \featpref{Creature Handling Specialization} \\
        \featref{Devices Specialization} & Mastered Devices & \tdash & \tdash & \featpref{Devices Specialization} \\
        \featref{Disguise Specialization} & Mastered Disguise & \tdash & \tdash & \featpref{Disguise Specialization} \\
        \featref{Escape Artist Specialization} & Mastered Escape Artist & \tdash & \tdash & \featpref{Escape Artist Specialization} \\
        \featref{Intimidate Specialization} & Mastered Intimidate & \tdash & \tdash & \featpref{Intimidate Specialization} \\
        \featref{Heal Specialization} & Mastered Heal & \tdash & \tdash & \featpref{Heal Specialization} \\
        \featref{Jump Specialization} & Mastered Jump & \tdash & \tdash & \featpref{Jump Specialization} \\
        \featref{Knowledge Specialization} & Mastered Knowledge & \tdash & \tdash & \featpref{Knowledge Specialization} \\
        \featref{Linguistics Specialization} & Mastered Linguistics & \tdash & \tdash & \featpref{Linguistics Specialization} \\
        \featref{Perform Specialization} & Mastered Perform & \tdash & \tdash & \featpref{Perform Specialization} \\
        \featref{Persuasion Specialization} & Mastered Persuasion & \tdash & \tdash & \featpref{Persuasion Specialization} \\
        \featref{Ride Specialization} & Mastered Ride & \tdash & \tdash & \featpref{Ride Specialization} \\
        \featref{Sense Motive Specialization} & Mastered Sense Motive & \tdash & \tdash & \featpref{Sense Motive Specialization} \\
        \featref{Sleight of Hand Specialization} & Mastered Sleight of Hand & \tdash & \tdash & \featpref{Sleight of Hand Specialization} \\
        \featref{Spellcraft Specialization} & Mastered Spellcraft & \tdash & \tdash & \featpref{Spellcraft Specialization} \\
        \featref{Sprint Specialization} & Mastered Sprint & \tdash & \tdash & \featpref{Sprint Specialization} \\
        \featref{Stealth Specialization} & Mastered Stealth & \tdash & \tdash & \featpref{Stealth Specialization} \\
        \featref{Survival Specialization} & Mastered Survival & \tdash & \tdash & \featpref{Survival Specialization} \\
        \featref{Swim Specialization} & Mastered Swim & \tdash & \tdash & \featpref{Swim Specialization} \\

        \tb{Spell Feats}\label{Spell Feats} & \tb{Prerequisites} & \tb{Benefits} & \tb{Feat Types} & \tb{Page} \\
        \featref{Abjurer} & Abjuration spell & \tdash & Magical & \featpref{Abjurer} \\
        \featref{Conjurer} & Conjuration spell & \tdash & Magical & \featpref{Conjurer} \\
        \featref{Diviner} & Divination spell & \tdash & Magical & \featpref{Diviner} \\
        \featref{Eldritch Knight} & Any spell & \tdash & \tdash & \featpref{Eldritch Knight} \\
        \featref{Enchanter} & Enchantment spell & \tdash & Magical & \featpref{Enchanter} \\
        \featref{Evoker} & Evocation spell & \tdash & Magical & \featpref{Evoker} \\
        \featref{Illusionist} & Illusion spell & \tdash & Magical & \featpref{Illusionist} \\
        \featref{Miscaster} & Any spell & \tdash & Magical & \featpref{Miscaster} \\
        \featref{Transmuter} & Transmutation spell & \tdash & Magical & \featpref{Transmuter} \\
        \featref{Vivimancer} & Vivimancy spell & \tdash & Magical & \featpref{Vivimancer} \\

        \tb{Combat Feats}\label{Combat Feats} & \tb{Prerequisites} & \tb{Benefits} & \tb{Feat Types} & \tb{Page} \\
        \featref{Agility} & Dex 2 & Increase reaction speed & \tdash & \featpref{Agility} \\
        \featref{Blindfighter} & Per 2 & Fight unseen foes better & \tdash & \featpref{Blindfighter} \\
        \featref{Duelist} & Dex 1, Int 1 & Fight one-on-one better & \tdash & \featpref{Duelist} \\
        \featref{Executioner} & Str 1, Per 2 & Kill weakened foes more easily & \tdash & \featpref{Executioner} \\
        \featref{Guardian} & Per 1, Wil 1 & Protect nearby allies & \tdash & \featpref{Guardian} \\
        \featref{Reaper} & Str 2 & \tdash & Cleave through foes with sweeping strikes & \featpref{Reaper} \\
        \featref{Sniper} & Per 2 & \tdash & Aim precisely at distant foes & \featpref{Sniper} \\
        \featref{Whirlwind Warrior} & Dex 2, Per 1 & Fight hordes with agile ease & \tdash & \featpref{Whirlwind Warrior} \\
    \end{longtabu}
\end{longtabuwrapper}
    \twocolumn

    \section{Feat Descriptions}
        Here is the format for feat descriptions.

    \ssecfake{Feat Name [Type of Feat]}
    \featpre Requirements a character must meet before taking the feat.
    This entry is absent if a feat has no prerequisites.
    \featben What the feat enables the character (``you'' in the feat description) to do.

    \begin{feat}{Abjurer}{Magical, Spell}
        \spelldesc{You have great talent with Abjuration spells.}
        \featpre \glossterm{Abjuration} spell known.
        \featben
        \ff{Abjurant Shield} Whenever you cast an Abjuration spell with a duration, you can give one willing creature targeted by the spell a deflective shield.
        The shielded creature gains a \plus1 bonus to \glossterm{physical defenses} as long as the spell lasts.
        This is a \glossterm{Magical}, \glossterm{Shielding} effect.
        You can only shield one creature in this way at a time.
        If you shield another creature, all previous shields are dismissed when the new shield takes effect.

        \ff[3]{Counterspell} While you are casting a spell, at any time before it resolves, you can take an \glossterm{immediate action} to turn that spell into a counterspell.
        If you do, your spell has no effect when it resolves.
        Instead, choose a creature within \rngmed range of you.
        If that creature is casting a spell, and your spell's level, including all augments, is at least as high as their spell's level, their spell also has no effect when it resolves.

        \ff[5]{Ablative Shield} The shielded creature also gains damage reduction against physical damage equal to your spellpower.
        % which spellpower?

        \ff[7]{Magic Against Magic} You gain a \plus2 bonus to accuracy with the \spell{antimagic} spell.
        In addition, when you use your \textit{counterspell} ability with the \spell{antimagic} spell, it is treated as being two spell levels higher than its actual level for the purpose of determining which spells it can counter.

        \ff[9]{Versatile Shield} The shielded creature's damage reduction applies against all damage, not just physical damage.

        \ff[11]{Pierce Shields} Your skill at creating defenses allows you penetrate defenses more easily.
        You gain a \plus1 bonus to \glossterm{accuracy} with magical attacks.

        \ff[13]{Punishing Counterspell} When you use your \textit{counterspell} ability, you can make the target \glossterm{miscast} their spell instead of negating the spell's effects.
        In addition, the range of your \textit{counterspell} ability increases to \rnglong.

        \ff[15]{Empowered Shield} The shielded creature's defense bonus increases to \plus2.

        \ff[17]{Mass Counterspell} When you use your \textit{counterspell} ability, you may target up to five creatures.

        \ff[19]{Abjuration Master} Your Abjuration spells are easier to cast.
        You can cast Abjuration spells as if they were one level lower than their actual level.
        This does not stack with spell level reductions from other feats.
    \end{feat}

    \begin{feat}{Acrobatics Specialization}{Skill}
        \featpre Acrobatics as a mastered skill.
        \parhead{Benefits}
        
        \ff{Specialization, Lesser} You gain a \plus2 bonus to Acrobatics.

        \ff[2]{Rapid Balance} Using Acrobatics to balance on slippery or narrow surfaces does not reduce your speed.

        \ff[4]{Agile Charge} You can change directions freely while making a \glossterm{charge}.

        \ff[6]{Surface Tolerance} You reduce DR modifiers for surface conditions on Acrobatics checks to balance by 2.
        This allows you to ignore minor surface conditions, such as slippery surfaces, when balancing.

        \ff[8]{Specialization} The bonus to Acrobatics increases to \plus3.

         % modifier at 10th: 12dex + 2mst + 2ft = 16
        \ff[10]{Legendary Airwalker}[Mag] You can attempt to move on surfaces that cannot support your weight.
        Surfaces that can support at least a quarter of your weight, such as thin tree branches and dense liquids, are DR 20.
        Surfaces that can support at least a tenth of your weight, such as water, are DR 25.
        Surfaces that can support at least a hundredth of your weight, such as tree leaves, are DR 30.
        Surfaces that cannot support your weight at all, such as air, are DR 40.
        Success means you move along the surface at half speed.
        Failure means you fall through the surface.
        The DR increases by 2 for each consecutive round that you spend moving in this way.

        \ff[12]{Fall Tolerance} You take half damage from \glossterm{falling damage}.

        \ff[14]{Surface Tolerance, Greater} The reduction of DR modifiers for surface conditions increases to 5.
        This allows you to ignore almost all surface conditions when balancing.

        \ff[16]{Specialization, Greater} The bonus to Acrobatics increases to \plus4.

        \ff[18]{Rapid Airwalker} You can move at full speed with the \textit{legendary airwalker} ability.
    \end{feat}

    \begin{feat}{All Energy Becomes One}{Class, Magical}
        \featpre Monk as a \glossterm{base class}, starting Constitution of 2.
        \parhead{Benefits}

        \ff{Absorb Energy} Whenever you would take \glossterm{energy damage}, you can take an \glossterm{immediate action} to channel the energy into your body.
        If you do, you gain damage reduction against that attack equal to your \ki power.

        \ff[3]{Channel Energy} At the end of each round, if you reduced damage with your \textit{absorb energy} ability, you can channel that energy into your weapons.
        If you do, choose a damage type that you reduced this round with that ability.
        All damage you deal with your weapons gains that damage type until the end of the next round.

        \ff[5]{Energetic Unity} You can use \textit{absorb energy} to reduce any non-physical damage you take, instead of only energy damage.

        \ff[7]{Channel Power} While you are channeling energy with your \textit{channel energy} ability, you deal \plus1d damage with your weapons.

        \ff[9]{Sustained Channeling} You can sustain your \textit{channel energy} ability as a swift action.

        \ff[11]{Absorb Energy, Greater} You increase the damage reduction to twice your \ki power.

        \ff[13]{Kinetic Absorption} You can use \textit{absorb energy} to reduce any damage you take, instead of only non-physical damage.

        \ff[15]{Channel Power, Greater} You increase your damage bonus with your weapons to \plus2d.

        \ff[17]{Attuned Channeling} When you use your \textit{channel energy} ability, you can spend an \glossterm{action point}.
        If you do, you attune to the ability, causing it to last as long as you stay attuned to it.
        For details, see \pcref{Attunement}.

        \ff[19]{Reflexive Absorption} You can use your \textit{greater absorb energy} ability once per round without spending an action.
        You cannot use it twice to affect the same attack.
    \end{feat}

    \begin{feat}{Awareness Specialization}{Skill}
        \featpre Awareness as a mastered skill.
        \parhead{Benefits}

        \ff[1]{Specialization, Lesser} You gain a \plus2 bonus to Awareness.

        \ff[2]{Broad Search} When you take the Search action, you can search a 10-ft.\ square within 30 feet of you (see \pcref{Search}).

        \ff[4]{Extraordinary Senses} You gain one of the following senses: \glossterm{blindsense} (50 ft.), \glossterm{darkvision} (100 ft.), \glossterm{scent}, or \glossterm{tremorsense} (50 ft.).

        \ff[6]{Distance Tolerance} You reduce DR modifiers for distance on Awareness rolls by 2.
        This usually allows you to ignore up to 100 feet of distance.

        % \ff[6]{Attentive} You can pay \glossterm{active attention} to your surroundings for any length of time without becoming fatigued.

        \ff[8]{Specialization} The bonus to Awareness increases to \plus3.

        \ff[10]{Legendary Senses} You gain one of the following senses: \glossterm{blindsense} (200 ft.), \glossterm{blindsight} (50 ft.), \glossterm{darkvision} (500 ft.), \glossterm{tremorsense} (200 ft.), or \glossterm{tremorsight} (50 ft.).

        \ff[12]{Trapmaster} Whenever you come within 50 feet of a trap, you can make an Awareness check to notice it, even if you were not searching for traps.

        \ff[14]{Distance Tolerance, Greater} The reduction of DR modifiers for distance increases to 5.
        This usually allows you to ignore up to 500 feet of distance.

        \ff[16]{Specialization, Greater} The bonus to Awareness increases to \plus4.

        \ff[18]{Supreme Senses} You can choose an additional sense from the \textit{legendary senses} ability.
        Its range is doubled.
    \end{feat}

    \begin{feat}{Eldritch Knight}{General, Spell}
        \featpre Ability to cast a spell.
        \featben

        \ff[1]{Combat Concentration, Lesser} You gain a \plus2 bonus to Concentration checks made to cast spells.

        \ff[1]{Armor Tolerance, Lesser} You reduce your chance of arcane spell failure from wearing armor by 10\%.

        \ff[2]{Spellstrike}[Magical] As a standard action, you can spend an \glossterm{action point} to use this ability.
        \begin{ability}
            \begin{spelltargetinginfo}
                \spellspecial Choose a weapon you wield, or your \glossterm{unarmed attack}.
            \end{spelltargetinginfo}
            \begin{spelleffects}
                \spelleffect You imbue magical power into the weapon.
                This requires concentration as if casting a spell.
                Breaking your concentration does not cause a \glossterm{miscast backlash}, however.
                During the \glossterm{delayed action phase}, you may make a \glossterm{strike} with the weapon.
                If you maintained your concentration, the strike gains a \plus2d bonus to damage.
            \end{spelleffects}
        \end{ability}

        \ff[4]{Combat Concentration} The bonus to Concentration checks increases to \plus5.

        \ff[6]{Armor Tolerance} The reduction of arcane spell failure increases to 20\%.

        \ff[8]{Seeking Spellstrike}[Magical] When you use your \textit{spellstrike} ability, if you maintained your concentration, you gain a \plus1 bonus to accuracy with the \glossterm{strike}.

        \ff[10]{Spellsword Rhythm}[Magical] Whenever you hit a creature with a \glossterm{strike}, you gain a \plus1 bonus to \glossterm{accuracy} with spells against that creature during the next round.
        In addition, whenever you hit a creature with a \glossterm{spell}, you gain a \plus1 bonus to \glossterm{accuracy} with \glossterm{strikes} against that creature during the next round.

        \ff[12]{Combat Concentration, Greater} The bonus to Concentration checks increases to \plus10.

        \ff[14]{Spellstrike, Greater}[Magical] When you use your \textit{spellstrike} ability, the damage bonus if you maintained your concentration increases to \plus3d.

        \ff[16]{Armor Tolerance, Greater} The reduction of arcane spell failure increases to 30\%. 

        \ff[18]{Spellsword Rhythm, Greater}[Magical] The accuracy bonus increases to \plus2.
    \end{feat}

    \begin{feat}{Blindfighter}{Combat}
        \featpre Starting Perception 2.
        \featben
        
        \ff[1]{Blind Precision} Whenever you have a miss chance caused by being unable to see your opponent, you can roll the miss chance twice and take the better result.

        \ff[2]{Unseen Defense} You are not \defenseless against foes you cannot see if you know their location.

        \ff[4]{Blindsense} You gain \glossterm{blindsense} (50 ft.).

        \ff[6]{Attack the Unseen} If you know the location of a creature you cannot see, and you have \glossterm{line of effect} to that creature, you can target it with targeted abilities.

        \ff[8]{Blind Feint} If you are attacked by a creature who incorrectly thinks you are \glossterm{unaware} of the attack, you can take an \glossterm{immediate action} to spend an \glossterm{action point}.
        If you do, you can immediately make a \glossterm{strike} against that creature.

        \ff[10]{Blindsight} You gain \glossterm{blindsight} (50 ft.).
        In addition, the range of your blindsense improves to 200 feet. 

        \ff[12]{Controlled Sight} You are immune to all abilities that depend on sight to affect you.

        \ff[14]{Blindsight, Greater} The range of your blindsight improves to 100 feet.
        In addition, the range of your blindsense improves to 500 feet.

        \ff[16]{Blind Feint, Greater} When you use your \glossterm{blind feint} ability, the target is \glossterm{unaware} of the attack.

        \ff[18]{Supreme Sight} The range of your blindsight improves to 200 feet.
        In addition, the range of your blindsense improves to 1,000 feet.
    \end{feat}

    \begin{feat}{Bluff Specialization}{Skill}
        \featpre Bluff as a mastered skill.
        \featben

        \ff[1]{Specialization, Lesser} You gain a \plus2 bonus to Bluff.

        \ff[2]{Sustained Distraction} If you successfully distract a creature, you can sustain that distraction on that creature with a \glossterm{swift action} as long as you continue to be distracting.
        You must make a new check each round, and the DR increases by 2 for each round you have distracted them.
        For details, see \pcref{Distract}.

        \ff[4]{Deceive Magic} Any magical abilities which detect lies are unable to detect lies you speak.

        % You can make a Bluff attack to \glossterm{feint} instead of using your normal accuracy (see \pcref{Feint}).

        \ff[6]{Impersonation Tolerance} You reduce DR modifiers for details you cannot replicate when impersonating creatures 2 (see \pcref{Impersonate}).

        \ff[8]{Specialization} The bonus to Bluff increases to \plus3.

        \ff[10]{Dual Speech}[Mag] Whenever you speak, you can make a Bluff check to speak in two voices at once.
        The base DR is 15.
        Success means you speak the same words with two different vocal patterns, such as tone or accent, though you cannot significantly change the volume.
        You can freely choose which creatures hear which pattern, treating all creatures you are not aware of as a single group.
        Failure means you choose one of the vocal patterns and speak in that pattern.
        Critical failure means the words come out garbled and incomprehensible.

        If you increase the DR by 5, your vocal patterns can be in different languages.
        If you increase the DR by 10, your vocal patterns can use entirely different words.

        \ff[12]{Deceive Magic, Greater} Whenever you impersonate a creature, if your Bluff check is high enough, magical abilities treat you as if you were that creature.
        % is this DR high enough?
        The DR is equal to 10 \add the \glossterm{power} of the ability you are deceiving.
        % does this work mechanically?
        For example, if you impersonate an undead creature, the \spell{inflict light wounds} spell would heal you.
        This does not grant you any abilities associated with creatures you impersonate.

        \ff[14]{Impersonation Tolerance, Greater} The reduction of DR modifiers for impersonating details you cannot replicate increases to 5.

        \ff[16]{Specialization, Greater} The bonus to Bluff increases to \plus4.

        \ff[18]{Deceive Reality} Whenever you are attacked, after learning whether the attack succeeded or failed, you can spend an \glossterm{action point} as an immediate action.
        If you do, you can make a Bluff check to pretend the attack affected you differently.
        The DR is equal to the result of the attack roll.
        Success means you can choose the result of the attack roll, potentially causing it to miss instead of hitting, or to hit instead of missing.
        However, you cannot make the attack critically fail.
        After using this ability, you cannot use it again until you take a \glossterm{short rest}.
    \end{feat}

    \begin{feat}{Celestial Heritage}{Bloodline, Magical}
        \featpre Non-evil alignment
        \parhead{Special} You can only have one Bloodline feat.
        \featben

        \ff[1]{Celestial Power} Your \glossterm{power} with abilities from this feat is equal to your Willpower or your level, whichever is higher.

        \ff[1]{Holy Blessing} As a standard action, you can spend an \glossterm{action point} to use this ability.
        \begin{ability}
            \begin{spelltargetinginfo}
                \spelltwocol{\spelltgt{One creature}}{\spellrng{\rngclose}}
            \end{spelltargetinginfo}
            \begin{spelleffects}
                \spelleffect The target gains a \plus2d bonus to damage with all attacks.
                \spelldur Attunement
                \spelltags{\glossterm{Good}, \glossterm{Magical}}
            \end{spelleffects}
        \end{ability}

        \ff[3]{Holy Protection} As a standard action, you can spend an \glossterm{action point} to use this ability.
        \begin{ability}
            \begin{spelltargetinginfo}
                \spelltwocol{\spelltgt{One creature}}{\spellrng{\rngclose}}
            \end{spelltargetinginfo}
            \begin{spelleffects}
                \spelleffect The target gains damage reduction equal to your celestial power against \glossterm{Evil} physical effects, and physical attacks made by evil creatures.
                \spelltags{\glossterm{Good}, \glossterm{Magical}}
            \end{spelleffects}
        \end{ability}

        \ff[5]{Angel Wings} You gain feathery wings that sprout from your back.
        You can use these wings to glide at a rate equal to your land speed (see \pcref{Gliding}).
        The wings themselves are \glossterm{mundane}, but the ability to glide and fly with them is \glossterm{magical}.

        \ff[7]{Complete Protection} The damage reduction from your \textit{holy protection} ability also applies against non-physical effects.

        \ff[9]{Empowered Blessing} The damage bonus from your \glossterm{holy blessing} ability increases to \plus3d.

        \ff[11]{Angelic Flight} Your \textit{angel wings} grant you a fly speed equal to your land speed.
        While \unencumbered, you can fly (see \pcref{Flying}).
        You can only fly for a number of rounds equal to half your celestial power.
        After that limit is reached, you must take a \glossterm{short rest} before flying again.

        \ff[13]{Holy Retribution} Whenever an evil creature makes a physical melee attack against the creature protected by your \textit{retributive protection} ability, you make a celestial power vs. Mental attack against the attacking creature.
        Success means the attacker takes 1d4 divine damage \plus1d per celestial power.

        \ff[15]{Empowered Blessing, Greater} The damage bonus from your \glossterm{holy blessing} ability increases to \plus4d.

        \ff[17]{} % another augment?

        \ff[19]{Angelic Flight, Greater} You no longer have a limit on how long you can fly with your \textit{angelic flight} ability.
    \end{feat}

    \begin{feat}{Reaper}{Combat}
        \spelldesc{You can attack with such force that you cleave through your foes.}
        \featpre Starting Strength of 2.

        \ff[1]{Sweeping Strike} As a standard action, you can spend an \glossterm{action point} to use this ability.
        \begin{ability}
            \begin{spelltargetinginfo}
                \spelltwocol{\spellarea{Up to three contiguous squares}}{\spellrng{Threatened}}
                \spelltgts{Everything in the area}
                \spellspecial Choose one or two melee weapons you are wielding that deal slashing or bludgeoning damage.
            \end{spelltargetinginfo}
            \begin{spelleffects}
                \begin{spellattack}{Physical vs. Armor}
                    \spellsuccess The target takes \glossterm{strike damage} from the chosen weapon with a \minus1d penalty (see \pcref{Strike Damage}).
                    \spellcritical As above, but the target takes double damage.
                \end{spellattack}
            \end{spelleffects}
        \end{ability}

        \ff[2]{Cleave} Whenever you get a \glossterm{critical hit} with a strike using a slashing or bludgeoning weapon, you can take an \glossterm{immediate action} to make another strike with the same weapon.
        The target of the new strike must be adjacent to the struck creature.
        You may continue making additional strikes as long as you keep getting critical hits.
        However, you may not attack the same creature more than once.

        \ff[4]{Reaping Charge} As a standard action, you can spend an \glossterm{action point} to use this ability.
        \begin{ability}
            \begin{spelltargetinginfo}
                \spelltgts{Special; see below}
                \spellspecial Choose one or two melee weapons you are wielding that deal slashing or bludgeoning damage.
            \end{spelltargetinginfo}
            \begin{spelleffects}
                \spelleffect You move up to half your movement speed in a straight line.
                Choose either the right or left side of the line.
                You can make a \glossterm{strike} against each creature and object on that side of the line that you threaten at any point during your movement.
                You take a \minus2d penalty to damage on the strike.
            \end{spelleffects}
        \end{ability}

        \ff[6]{Spinning Cleave} When you use your \textit{cleave} ability, the target of your additional attack does not have to be adjacent to the struck creature.

        \ff[8]{Wide Sweep} When you use your \textit{sweeping strike} ability, you can target up to five contiguous squares you threaten.

        \ff[10]{Reaping Whirlwind} When you use your \textit{reaping charge} ability, you do not have to choose a side of the line.
        You can attack any creatures and objects that you threaten at any point during your movement.

        \ff[12]{Mobile Cleave} When you use your \textit{cleave} ability, you can move up to half your movement speed before making the strike.
        This movement counts against your normal movement limit in that phase.

        \ff[14]{Reaping Charge, Greater} You can move up to your full movement speed when you use your \textit{reaping charge} ability, rather than half your movement speed.

        \ff[16]{Whirlwind Sweep} When you use your \textit{sweeping strike} ability, you can target any number of contiguous squares you threaten.

        \ff[18]{Reap the Harvest} When you use your \textit{reaping charge} or and \textit{sweeping strike} abilities, if every creature you attacked is dead at the end of the round, you regain the action point spent to use the ability.

    \end{feat}

    \begin{feat}{Climb Specialization}{Skill}
        \featpre Climb as a mastered skill.
        \featben

        \ff[1]{Specialization, Lesser} You gain a \plus2 bonus to Climb.

        \ff[2]{Damage Tolerance} Taking damage while climbing does not force you to make an additional Climb check to avoid falling.

        \ff[4]{Climb Speed} You gain a \glossterm{climb speed} equal to half your land speed.
        A successful Climb check to move allows you to travel a distance equal to your climb speed.

        \ff[6]{Scale the Beast}
        You gain a \plus2 bonus to Climb attacks you make to climb on other creatures (see \pcref{Creature Climb}).
        In addition, you gain a \plus1 bonus to accuracy with physical attacks against creatures you are climbing on.

        \ff[8]{Specialization} The bonus to Climb increases to \plus3.

        \ff[10]{Climb Speed, Greater} Your climb speed increases to be equal to your land speed.

        \ff[12]{Impossible Climber} You can climb surfaces that are perfectly smooth.
        The DR is 30 for perfectly smooth vertical surfaces, and 40 to climb on a perfectly smooth ceiling.
        In addition, you are treated as one size category smaller than normal for the purpose of determining which creatures you can climb on.

        \ff[14]{Scale the Beast, Greater} The bonus to Climb attack to climb on creatures increases to \plus4.
        In addition, the bonus to physical accuracy against creatures you are climbing on increases to \plus2.

        \ff[16]{Specialization, Greater} The bonus to Climb increases to \plus4.

        \ff[18]{Impossible Climber, Greater} You can wallrun on ceilings in the same way you wallrun on walls.
        In addition, the size category decrease for the purpose of climbing on creatures improves to two size categories smaller than normal.
    \end{feat}

    \begin{feat}{Conjurer}{Magical, Spell}
        \featpre \glossterm{Conjuration} spell known.
        \featben

        \ff[1]{Astral Spirit} When you cast the \glossterm{summon monster} spell, you can manifest an astral spirit instead of an animal.
        An astral spirit is a floating, spirit-like creature with a translucent body.
        Its size is Medium, and it is vaguely humanoid in shape.
        It has a physical form, and occupies space like any other creature.

        An astral spirit does not have a land speed, but it has a 30 foot \glossterm{fly speed} with good maneuverability.
        In addition, it can teleport any distance as a move action as long as its destination is within \rngmed range of you.
        If an astral spirit hits with its strike, it deals arcane damage.

        \ff[3]{Astral Spell Transit, Lesser} Your attacks with spells ignore \glossterm{cover}, but not \glossterm{total cover}.

        \ff[5]{Fortified Manifestations} Objects and creatures you create with \glossterm{Manifestation} abilities have additional hit points equal to your spellpower.

        \ff[7]{Astral Echo} Whenever you teleport, you drift between your plane and the Astral Plane until the end of the next round.
        During this time, all attacks against you have a 20\% failure chance.

        \ff[9]{Regenerating Manifestations} Whenever you cast a spell, objects and creatures you have created with \glossterm{Manifestation} abilities heal hit points equal to your spellpower.

        \ff[11]{Astral Spell Transit} You double the range of all spells you cast.
        In addition, all \glossterm{Teleportation} spells you cast can teleport twice their normal distance.

        \ff[13]{Sustained Manifestations} Once per round, when you cast or sustain a spell, you can also sustain a \glossterm{Manifestation} spell as a \glossterm{free action}.

        \ff[15]{Fortified Manifestations, Greater} Objects and creatures you create with \glossterm{Manifestation} abilities have damage reduction equal to your spellpower.

        \ff[17]{Astral Spell Transit, Greater} When determining whether you have \glossterm{line of effect} to a particular location with a spell, you can ignore a single solid obstacle up to five feet thick.
        This can allow you to cast spells through solid walls, though it does not grant you the ability to see through the wall.

        \ff[19]{Astral Echo, Greater} You constantly drift between your plane and the Astral Plane.
        All attacks against you have a 20\% failure chance.
        You can suppress or resume this ability as a \glossterm{swift action}.
        In addition, whenever you teleport, your connection to the Astral Plane is strengthened until the end of the next round.
        During this time, all attacks against you have a 50\% failure chance.
    \end{feat}

    \begin{feat}{Duelist}{Combat}
        \featpre Starting Dexterity of 1, starting Intelligence of 1.
        \featben

        \ff[1]{Parry} Whenever a creature \glossterm{initiates} a \glossterm{strike} against you, you can take an \glossterm{immediate action} to attempt to parry its attack.
        If you do, you gain a \plus2 bonus to physical defenses against the attack.

        \ff[2]{Defensive Stance} As a \glossterm{swift action}, you can spend an \glossterm{action point} to use this ability.
        \begin{ability}
            \begin{spelltargetinginfo}
                \spellquicktargeting{One creature}{\rngmed}
                \spellspecial You can change the target of this ability to a new creature within range as a \glossterm{swift action}.
            \end{spelltargetinginfo}
            \begin{spelleffects}
                \spelleffect You gain a \plus2 bonus to physical defenses against the target.
                \spelldur Attunement
            \end{spelleffects}
        \end{ability}

        \ff[4]{} 

        \ff[6]{Overwhelm Tolerance} Your \textit{defensive stance} ability prevents you from suffering \glossterm{overwhelm penalties} against attacks from the target creature.

        \ff[8]{Riposte} Whenever you use your \textit{parry} ability, you can spend an \glossterm{action point}.
        If you do, you \glossterm{initiate} a \glossterm{strike} against the attacking creature.

        \ff[10]{Focused Defense} You gain a \plus1 bonus to physical defenses against attacks you do not suffer \glossterm{overwhelm penalties} on.

        \ff[14]{Rapid Parry} Once per round, you can \textit{parry} an attack without spending an immediate action.
    \end{feat}

    \begin{feat}{Craft Specialization}{Skill}
        \featpre Craft (any) as a mastered skill.
        \featben

        \ff[1]{Specialization, Lesser} You gain a \plus2 bonus to all Craft skills.

        \ff[2]{Craft Magic Item}[Magical] You can imbue items with magic using your crafting skill.
        Imbuing an item with magic takes material components, as described in \pcref{Magic Item Creation}.
        It takes you one hour per 10 gp of material components to create a item.

        You can also mend a broken magic item if it is one that you could make.
        Doing so costs a tenth of the raw materials and a quarter of the time it would take to craft that item in the first place.
        You cannot mend a \glossterm{destroyed} magic item.

        \ff[4]{Crafting Savant} You gain two additional \glossterm{skill points} which can only be spent on Craft skills.

        \ff[6]{Rapid Creation} Crafting magic items takes you one hour per 100 gp of material components.

        \ff[8]{Specialization} The bonus to Craft skills increases to \plus3.

        \ff[10]{Rapid Creation} Crafting magic items takes you one hour per 500 gp of material components.

        \ff[12]{Crafting Savant, Greater} The number of extra skill points increases to four.

        \ff[14]{Rapid Creation} Crafting magic items takes you one hour per 2,500 gp of material components.

        \ff[16]{Specialization, Greater} The bonus to Craft skills increases to \plus4.

        \ff[18]{Rapid Creation} Crafting magic items takes you one hour per 12,500 gp of material components.
    \end{feat}

    \begin{feat}{Creature Handling Specialization}{Skill}
        \featpre Creature Handling as a mastered skill.
        \featben

        \ff[1]{Specialization, Lesser} You gain a \plus2 bonus to Creature Handling.

        \ff[2]{Sustained Pacify} You can sustain the \textit{pacify} ability from the Creature Handling skill as a swift action, rather than as a standard action (see \pcref{Pacify}).

        \ff[4]{Compressed Training} You can teach a creature a trick with 12 hours of work, split as you choose, rather than in a week of 4 hour sessions (see \pcref{Training Creatures}).

        \ff[6]{Species Tolerance} You reduce Creature Handling DR modifiers for handling non-animals by 2.

        \ff[8]{Specialization} The bonus to Creature Handling increases to \plus3.
        In addition, you can train creatures to learn one bonus trick beyond their normal maximum (see \pcref{Bonus Tricks}).

        \ff[10]{Battleforged Training} You can teach a creature the Battleforged trick.
        The DR to train the trick is 20.
        A creature with the trick gains the following benefits:
        \begin{itemize}
            \item Its maximum hit points increase by an amount equal to its level.
            \item It gains a \plus1 bonus to accuracy with all attacks.
            \item It gains a \plus1d bonus to damage with \glossterm{strikes}.
        \end{itemize}

        \ff[12]{Rapid Pacify} You can use the \textit{pacify} ability from the Creature Handling skill as a swift action, rather than as a standard action.

        \ff[14]{Species Tolerance, Greater} The reduction of DR modifiers for handling non-animals increases to 5.
        This usually allows you to ignore penalties for working with non-animals.

        \ff[16]{Specialization, Greater} The bonus to Creature Handling increases to \plus4.
        In addition, the number of bonus tricks you can train creatures to learn increases to two.

        \ff[18]{Battleforged Training, Greater} You can teach a creature that has learned the Battleforged trick the Greater Battleforged trick.
        The DR to train the trick is 30.
        A creature with the trick gains the following benefits, which replace the benefits of the Battleforged trick:
        \begin{itemize}
            \item Its maximum hit points increase by an amount equal to twice its level.
            \item It gains a \plus2 bonus to accuracy with all attacks.
            \item It gains a \plus2d bonus to damage with \glossterm{strikes}.
        \end{itemize}
    \end{feat}

    \begin{feat}{Sniper}{Combat}
        \featpre Starting Perception of 2.
        \featben

        \ff[1]{Aim} As a standard action, you can use this ability.
        \begin{ability}
            \begin{spelltargetinginfo}
                \spellquicktargeting{One creature or object}{Line of sight}
            \end{spelltargetinginfo}
            \begin{spelleffects}
                \spelleffect You gain a \plus2 bonus to accuracy on \glossterm{strikes} against the target.
                \spelldur Sustain (swift). If you lose sight of the target for a full round, this effect ends.
            \end{spelleffects}
        \end{ability}

        \ff[2]{Penetrating Aim} Your physical ranged attacks ignore \glossterm{cover}, except total cover.

        \ff[4]{Distance Tolerance, Lesser} You reduce your accuracy penalties from \glossterm{range increments} by 2.

        \ff[6]{Sniper Shot} You gain a \plus2d bonus to damage on \glossterm{strikes} against \unaware creatures that are affected by your \textit{aim} ability.

        \ff[8]{Failure Tolerance} You ignore effects that give you a 20\% miss chance or failure chance with physical ranged attacks, such as \glossterm{concealment}.

        \ff[10]{Distance Tolerance} The reduction in accuracy penalties for range increments increases to 4.

        \ff[12]{Sustained Aim} You can sustain your \textit{aim} ability as a \glossterm{free action}.

        \ff[14]{Sniper Shot, Greater} The bonus to damage increases to \plus4d.

        \ff[16]{Distance Tolerance, Greater} The reduction in accuracy penalties for range increments increases to 6.

        \ff[18]{Rapid Aim} You can spend a \glossterm{action point} to use your \textit{aim} ability as a \glossterm{swift action}.
    \end{feat}

    \begin{feat}{Devices Specialization}{Skill}
        \featpre Devices as a mastered skill.
        \featben

        \ff[1]{Specialization, Lesser}[Magical] You gain a \plus2 bonus to Devices.

        \ff[2]{Rapid Improvisation} As a standard action, you can spend an \glossterm{action point} to use the \textit{improvise} ability to create a device (see \pcref{Improvise}).

        \ff[4]{Steady Hands} You cannot \glossterm{critically fail} on Devices checks.
        If you would critically fail, you simply fail instead, and suffer the normal penalties for non-critical failure.

        \ff[6]{Disable Arcana, Lesser}[Magical] You can disable arcane spell effects on objects or areas as if they were merely complex devices.
        You must be aware of an effect to disable it, either through the Spellcraft skill or because the effect is noticeable.
        You cannot disable effects on creatures.
        The DR to disable an effect is equal to 15 \add the effect's \glossterm{power}.
        Success means the spell is \glossterm{dispelled}.
        This has no effect on abilities that cannot be dispelled.

        \ff[8]{Specialization} The bonus to Devices increases to \plus3.

        \ff[10]{Improbable Improvisation} You reduce the DR for using the \textit{improvise} ability to make devices from unsuitable materials by 5.

        \ff[12]{Disable Arcana}[Magical] You can disable spell effects from any source, not just arcane spell effects.

        \ff[14]{Durable Improvization} Devices you create with the \textit{improvise} ability last for twice as many uses before they break.

        \ff[16]{Specialization, Greater} The bonus to Devices increases to \plus4.

        \ff[18]{Disable Arcana, Greater}[Magical] You can disable all magical effects on objects or areas, not just spell effects.
    \end{feat}

    \begin{feat}{Disguise Specialization}{Skill}
        \featpre Disguise as a mastered skill.
        \featben

        \ff[1]{Specialization, Lesser} You gain a \plus2 bonus to Disguise.

        \ff[2]{Quick Change} As a standard action, you can spend an \glossterm{action point} to use the \textit{disguise creature} or \textit{emulate creature} ability.

        \ff[4]{Disguise Aura}[Magical] Whenever you use the \textit{disguise creature} or \textit{emulate creature} abilities, you can decide how the target and any items on the target appear when examined by \glossterm{Divination} spells.
        For example, you could cause all of their equipment to appear nonmagical, or you could cause them to have a strong aura of good when examined with the \spell{detect alignment} spell.
        % Will this spell still exist?
        The maximum \glossterm{power} you can emulate is equal to your Disguise check result \minus15.

        Anyone using divination magic on the creature must make a spellpower check with a DR equal to your Disguise check result in order to perceive the truth.
        Regardless of the result of the check, the caster is not aware that the check was made.

        \ff[6]{Mismatch Tolerance} You reduce Disguise penalties for differences between the target's normal appearance and its intended appearance by 2.
        This allows you to ignore minor mismatches, such as if the target is a different gender than its intended appearance.

        \ff[8]{Specialization} The bonus to Disguise increases to \plus3.

        \ff[10]{Disguise Size}[Magical] As a \glossterm{standard action}, you can spend an action point to use this ability.
        \begin{ability}
            \begin{spelltargetinginfo}
                \spelltgt{You}
            \end{spelltargetinginfo}
            \begin{spelleffects}
                \spelleffect You increase or decrease your size by one \glossterm{size category}.
                \spelldur Attunement
            \end{spelleffects}
        \end{ability}

        \ff[14]{Mismatch Tolerance, Greater} The reduction of Disguise penalties for appearance differences increases to 5.
        This can allow you to ignore significant appearance differences.

        \ff[16]{Specialization, Greater} The bonus to Disguise increases to \plus4.
    \end{feat}

    \begin{feat}{Diviner}{Magical, Spell}
        \featpre \glossterm{Divination} spell known.
        \featben

        \ff[1]{Prophesy} As a \glossterm{standard action}, you can spend an \glossterm{action point} to use this ability.
        \begin{ability}
            \begin{spelleffects}
                \spellspecial The scope of this ability is limited to the next hour.
                This time period is called the \textit{time of prophecy}.
                When you use this ability, you visualize an action that a creature (or group of creatures) could take within the \textit{time of prophecy}.
                \spelleffect You see a brief, cryptic vision describing the most likely outcome of the action you visualized.
                This vision does not reveal any consequences that might occur after the \textit{time of prophecy} has ended.

                The vision does not have to be a literally accurate representation of the future.
                For example, if you used this ability to foresee the results of entering a room that had a group of creatures waiting in ambush, you might see a vision of flashing daggers in darkness darting towards your exposed back, regardless of whether the creatures would actually use daggers to attack.

                After using this ability, you cannot use it again until the \textit{time of prophecy} has ended, regardless of whether the action was taken.
            \end{spelleffects}
        \end{ability}

        \ff[3]{Precognitive Reaction, Lesser} You gain a \plus1 bonus to Reflex defense.
        In addition, you gain a \plus2 bonus to \glossterm{initiative} checks.

        \ff[5]{Deep Prophecy} When you use your \textit{prophesy} ability, you can increase the \textit{time of prophecy} to eight hours.
        This affects both the distance you can see into the future and the time you must wait before using the ability again.

        \ff[7]{Truesight} You gain the \glossterm{truesight} ability with a 50 foot range.

        \ff[9]{Precognitive Reaction} The bonus to initiative checks increases to \plus4.
        In addition, you are aware of all attacks against you, even those you cannot see, as long as you are conscious.
        This allows you to use abilities to defend yourself, and prevents you from being \unaware.

        \ff[11]{Dual Prophecy} You may use your \textit{prophesy} ability while you have an active \textit{time of prophecy}.
        If you have two active \textit{times of prophecy}, you must wait until one has expired to use your \textit{prophesy} ability again.

        \ff[13]{Truesight, Greater} The range of your \glossterm{truesight} increases to 500 feet.

        \ff[15]{Flexible Prophecy} When you use your \textit{prophesy} ability, you can choose the \textit{time of prophecy} to be any five-minute increment of time, to a minimum of thirty minutes and a maximum of eight hours.

        \ff[17]{Precognitive Reaction, Greater} The bonus to Reflex defense increases to \plus2, and the bonus to initiative checks increases to \plus6.
        In addition, you can never be surprised in combat.
        Whenever there are \glossterm{surprise phases}, you can act in them.

        \ff[19]{Oracle} When you use your \textit{prophesy} ability, the maximum \textit{time of prophesy} you can choose is increased to one week.
        In addition, you may have three active \textit{times of prophecy}, rather than two.
    \end{feat}

    \begin{feat}{Draconic Heritage}{Bloodline}
        \parhead{Special} You can only have one Bloodline feat.
        \featben

        \ff[1]{Draconic Power} Your \glossterm{power} with abilities from this feat is equal to your level or your Constitution, whichever is higher.

        \ff[1]{Draconic Ancestry} Choose a type of dragon from among the dragons on \trefnp{Dragon Types}.
        You have the blood of that type of dragon in your veins.
        This grants you damage reduction equal to twice your \textit{draconic power} against the damage type that dragon's breath weapon deals.

        \ff[1]{Low-Light Vision} You gain \glossterm{low-light vision}.
        If you already have low-light vision, you instead double the benefit, allowing you to quadruple the illumination range of light sources.

        \ff[2]{Draconic Weapons} You gain a bite natural weapon and two claw natural weapons, one on each arm.
        For details, see \pcref{Natural Weapons}.

        \ff[2]{Darkvision} You gain \glossterm{darkvision} with a 50 foot range.
        If you already have darkvision, you instead increase the range of your existing darkvision by 50 feet.

        \ff[4]{Draconic Wings} You gain scaly wings that sprout from your back.
        These wings grant you a glide speed equal to your land speed (see \pcref{Gliding}).
        The wings themselves are physical, but the ability to glide with them is \glossterm{magical}.

        \ff[6]{Breath Weapon} As a \glossterm{standard action}, you can spend an \glossterm{action point} to use this ability.
        \begin{ability}
            \begin{spelltargetinginfo}
                \spellspecial The area affected by this ability depends on your \textit{draconic ancestry}, as described in \trefnp{Dragon Types}.
                \spellburst{\areamed cone or \arealarge line}
                \spelltgts{All creatures in the area}
            \end{spelltargetinginfo}
            \begin{spelleffects}
                \begin{spellattack}{Draconic power vs. Reflex}
                    \spellsuccess The target takes 1d6 damage \plus1d per two draconic power.
                    The damage type depends on your \textit{draconic ancestry}, as described in \trefnp{Dragon Types}.
                    \spellcritical As above, but double damage.
                \end{spellattack}
                \spellspecial After using this ability, you must wait one round before using it again.
            \end{spelleffects}
        \end{ability}

        \ff[8]{Draconic Flight, Lesser}[Magical] 
        As a standard action, if you are \unencumbered, you can use your draconic wings to fly with a \glossterm{fly speed} equal to your land speed.

        \ff[10]{Widened Breath} The area affected by of your \textit{breath weapon} increases.
        A line breath weapon becomes a \areahuge, 10 ft.\ wide line.
        A cone breath weapon becomes a \arealarge cone.

        \ff[12]{Draconic Weapons, Greater}[Magical] The natural weapons gain an \glossterm{enhancement bonus} equal to one quarter of your \textit{draconic power}.

        \ff[14]{Draconic Flight}[Magical] You can use your \textit{draconic flight} ability to fly as a \glossterm{swift action}.

        \ff[16]{Devastating Breath} The damage dealt by your \textit{breath weapon} increases by \plus1d.

        \ff[18]{Draconic Flight, Greater}[Magical] You can use your \textit{draconic flight} ability to fly as a \glossterm{free action}.
    \end{feat}

    \begin{dtable}
        \lcaption{Dragon Types}
        \begin{dtabularx}{\columnwidth}{>{\lcol}X >{\lcol}X >{\lcol}X}
            \tb{Dragon} & \tb{Energy Type} & \tb{Breath Weapon} \\
            \bottomrule
            Black & Acid & Line \\
            Blue & Electricity & Line \\
            Brass & Fire & Line \\
            Bronze & Electricity & Line \\
            Copper & Acid & Line \\
            Gold & Fire & Cone \\
            Green & Acid & Cone \\
            Red & Fire & Cone \\
            Silver & Cold & Cone \\
            White & Cold & Cone \\
        \end{dtabularx}
    \end{dtable}

    \begin{feat}{Enchanter}{Magical, Spell}
        \featpre \glossterm{Enchantment} spell known.
        \featben

        \ff[1]{Mind Fragments} When you use \glossterm{Mind} abilities, you can affect creatures with a \glossterm{mundane} immunity to \glossterm{Mind} abilities.
        You take a \minus5 penalty to accuracy on attacks against such creatures.
        This does not allow you to affect creatures with a \glossterm{magical} immunity to \glossterm{Mind} abilities.

        \ff[3]{Mind Scour} As a \glossterm{standard action}, you can spend an \glossterm{action point} to use this ability.
        \begin{ability}
            \begin{spelltargetinginfo}
                \spellquicktargeting{One creature}{\rngmed}
            \end{spelltargetinginfo}
            \begin{spelleffects}
                \begin{spellattack}{Spellpower vs. Mental} % Which spellpower?
                    % No good damage type for this yet
                    \spellsuccess The target takes \spelldamage{physical}[d6].
                    In addition, its Mental defense is lowered by 2.
                    This is a \glossterm{condition}, and lasts until it is removed.
                    \spellcritical As above, but double damage, and its Mental defense is lowered by 5 instead of 2.
                \end{spellattack}
                \spelltags{\glossterm{Mind}}
            \end{spelleffects}
        \end{ability}

        \ff[5]{Subtle Influence} The DR to identify your \glossterm{Mind} abilities with Spellcraft, and to identify their effects with Sense Motive, increases by 5.

        \ff[7]{Enchanting Presence} You gain a \plus1 bonus to Intimidate and Persuasion.
        In addition, creatures within a 50-foot radius \glossterm{emanation} of you take a \minus1 penalty to Mental defense.
        You may freely exclude creatures you are aware of from this effect.

        \ff[9]{Brutal Scouring} You gain a \plus1d bonus to damage with your \textit{mind scour} ability. 

        \ff[11]{Mind Fragments, Greater} The penalty to accuracy against creatures with \glossterm{mundane} immunity to \glossterm{Mind} effects decreases to \minus2.

        \ff[13]{Enchanting Presence, Greater} The bonus to Intimidate and Persuasion increases to \plus2.
        In addition, the penalty to Mental defense against creatures near you is increased to \minus2.

        \ff[15]{Mental Breach} You gain a \plus1 bonus to accuracy with your \textit{mind scour} ability.

        \ff[17]{Subtle Influence, Greater} The DR increase to identify your \glossterm{Mind} abilities increases to 10.

        \ff[19]{Mental Torment} Whenever you target a creature with a Mind ability, that creature takes a \minus1 penalty to Mental defense.
        This is a \glossterm{condition}, and lasts until it is removed.
        This penalty stacks with itself if you target the same creature multiple times.
    \end{feat}

    \begin{feat}{Toughness}{General}
        \featpre Starting Constitution of 2.
        \featben

        \ff[1]{Durable} You gain additional hit points equal to your Constitution.

        \ff[2]{Great Fortitude, Lesser} You gain a \plus1 bonus to Fortitude defense.

        \ff[4]{Fatigue Tolerance} You ignore effects which would make you \fatigued.
        This allows you to sleep in heavy or medium armor without penalty.
        In addition, any effect which would make you \exhausted makes you \fatigued instead.
        This ability does not allow you to ignore this fatigue.

        \ff[6]{Injury Tolerance} You take a \minus2 penalty for being \glossterm{bloodied} instead of a \minus5 penalty.
        In addition, having \glossterm{vital damage} causes you to sufffer a penalty to accuracy, checks, and defenses equal to half the amount of vital damage you have.

        \ff[8]{Great Fortitude} The bonus to Fortitude defense increases to \plus2.

        \ff[10]{Fatigue Tolerance, Greater} You ignore effects which would make you \exhausted, and which compel you to sleep.
        In addition, you need half the normal amount of rest and sleep each day to function normally.
        For example, a human would only need four hours of sleep per night.

        \ff[12]{Durable, Greater} The increase to hit points increases to twice your Constitution.

        \ff[14]{Great Fortitude, Greater} The bonus to Fortitude defense increases to \plus3.

        \ff[16]{Injury Tolerance, Greater} You do not take penalties for being \glossterm{bloodied}.
        In addition, having \glossterm{vital damage} imposes one quarter of the normal penalties, rather than half.

        \ff[18]{Deathless} You cannot take more than ten damage per level during a single round.
        Any excess damage is ignored.
        In addition, you are immune to \glossterm{Death} effects.
    \end{feat}

    \begin{feat}{Escape Artist Specialization}{Skill}
        \featpre Escape Artist as a mastered skill.
        \featben

        \ff[1]{Specialization, Lesser} You gain a \plus2 bonus to Escape Artist.

        \ff[2]{Rapid Escape} You can squeeze and escape bindings and grapples as a move action, rather than as a standard action.

        \ff[4]{Constraint Tolerance, Lesser} You reduce your penalties to accuracy and defenses for \glossterm{squeezing} by 2.

        \ff[6]{Accelerated Squeeze} Your movement speed is not reduced while squeezing.

        \ff[8]{Specialization} The bonus to Escape Artist increases to \plus3.

        \ff[10]{Escape Magic}[Magical] As a standard action, you can spend an \glossterm{action point} to use this ability.
        \begin{ability}
            \begin{spelleffects}
                \spelleffect You make an Escape Artist attack against all \glossterm{magical} effects on you.
                The DR for each effect is equal to 10 \add the effect's \glossterm{power}.
                Success means the effect is \glossterm{dispelled}, if it is an effect that can be dispelled.
                \par You can only dispel effects which target you directly, not area effects which include you as a target.
                If an ability targets multiple creatures, you can only remove its effects on you.
            \end{spelleffects}
        \end{ability}

        \ff[12]{Constraint Tolerance, Greater} The penalty reduction for squeezing increases to 4.
        This normally allows you to ignore all penalties for squeezing.

        \ff[14]{Rapid Escape, Greater} You can escape bindings and grapples as a \glossterm{swift action}.

        \ff[16]{Specialization, Greater} The bonus to Escape Artist increases to \plus4.

        \ff[18]{Escape Magic, Greater}[Magical] You can use your \textit{escape magic} ability as a \glossterm{swift action}.
    \end{feat}

    \begin{feat}{Evoker}{Magical, Spell}
        \featpre \glossterm{Evocation} spell known.
        \featben

        \ff[1]{Energy Burst} As a standard action, you can spend an \glossterm{action point} to use this ability.
        \begin{ability}
            \begin{spelltargetinginfo}
                \spellspecial When you use this ability, you choose a type of energy: cold, electricity, fire, or sonic.
                \spellquicktargeting{One creature or object}{\rngclose}
            \end{spelltargetinginfo}
            \begin{spelleffects}
                \begin{spellattack}{Spellpower vs. Special}
                    \spellspecial If you chose electricity or fire energy, this attack is made against Reflex defense. If you chose cold or sonic energy, this attack is made against Fortitude defense.
                    \spellsuccess The target takes 1d8 damage \plus1d per two spellpower.
                    The type of energy you choose determines the type of damage dealt.
                    \spellcritical As above, but double damage.
                \end{spellattack}
                \spelltags{As the energy type chosen}
            \end{spelleffects}
        \end{ability}

        % Which spellpower?
        \ff[3]{Energy Resistance} You gain damage reduction against \glossterm{energy damage} equal to your spellpower.

        \ff[5]{Mighty Telekinesis} You gain a \plus1 bonus to accuracy with the \spell{telekinesis} spell.

        \ff[7]{Residual Energy Burst} When your attack succeeds with your \textit{energy burst} ability, the target suffers an additional effect depending on the energy type chosen.
        These effects are \glossterm{conditions}, and last until they are removed.
        \begin{itemize}
            \item Cold: If target is a creature, it is \fatigued.
            \item Electricity: If the target is a creature, it is \impaired with attacks and checks.
            \item Fire: The target is \ignited.
            \item Sonic: If the target has \glossterm{hardness}, its hardness is reduced by an amount equal to half your spellpower. Otherwise, if it has \glossterm{damage reduction}, its damage reduction is reduced by amount equal to your spellpower.
        \end{itemize}

        \ff[9]{Potent Evocation} You gain a \plus1d bonus to damage with \glossterm{Evocation} abilities that deal damage measured in a dice pool, as well as with the \textit{energy burst} ability.

        \ff[11]{Energy Resistance, Greater} The damage reduction against \glossterm{energy damage} increases to twice your spellpower.

        \ff[15]{Devastating Evocation} The area affected by your Evocation spells that affect areas doubles.
    \end{feat}

    \begin{feat}{Executioner}{Combat}
        \featpres Starting Strength of 1, starting Perception of 2.
        \featben

        \ff[1]{Strip the Flesh} As a standard action, you can spend an \glossterm{action point} to use this ability.
        \begin{ability}
            \begin{spelltargetinginfo}
                \spellquicktargeting{One creature or object}{As weapon}
            \end{spelltargetinginfo}
            \begin{spelleffects}
                \spelleffect You make a \glossterm{strike} against the target.
                At the end of the round, if the target is not \glossterm{bloodied}, it takes additional damage equal to the damage you dealt to it with your strike.
            \end{spelleffects}
        \end{ability}

        \ff[2]{Purge the Weak, Lesser} You gain a \plus1 bonus to accuracy with physical attacks against \glossterm{bloodied} creatures.

        \ff[4]{Final Blow} Whenever you deal \glossterm{vital damage} to a creature with a \glossterm{strike}, the creature immediately dies.

        \ff[6]{Bloodfeeder}[Life, Magical] Whenever a creature dies, if you dealt damage to it that round with a \glossterm{strike}, you heal hit points equal to its level.

        \ff[8]{Salt the Wound} When you use your \textit{strip the flesh} ability, the target continues taking damage at the end of each round until it becomes \glossterm{bloodied}.
        This is a \glossterm{condition}, and can be removed by abilities that remove conditions.

        \ff[10]{Purge the Weak} The accuracy bonus against \glossterm{bloodied} creatures increases to \plus2.

        \ff[12]{Final Blow, Greater} Whenever a creature's hit points drop to zero, if you dealt damage to it that round with a \glossterm{strike}, it immediately dies.
        In addition, whenever you deal damage to a creature that has no hit points remaining with a \glossterm{strike}, it immediately dies.

        \ff[14]{Bloodfeeder, Greater}[Life, Magical] Whenever a creature dies, if you used your \textit{strip the flesh} ability on it since the last time it took a \glossterm{short rest}, you regain an \glossterm{action point}.

        \ff[18]{Purge the Weak, Greater} The accuracy bonus against \glossterm{bloodied} creatures increases to \plus3.
    \end{feat}

    \begin{feat}{Whirlwind Warrior}{Combat}
        \featpres Starting Dexterity of 2, starting Perception of 1.
        \featben

        \ff[1]{Whirlwind Strike} As a standard action, you can spend an \glossterm{action point} to use this ability.
        \begin{ability}
            \begin{spelleffects}
                \spelleffect You make a melee \glossterm{strike} against any number of creatures and objects adjacent to you that you \glossterm{threaten}.
                You must use the same weapon to make each strike.
                Use the same attack result and damage against each target.
                You take a \minus2d penalty to damage on the strike.
            \end{spelleffects}
        \end{ability}

        \ff[2]{Eye of the Storm, Lesser} You reduce your \glossterm{overwhelm penalties} by 1.
        If your overwhelm penalty is reduced to 0, you are not considered to be \glossterm{overwhelmed}.

        \ff[4]{Unfettered Movement} During each phase, you may move through one creature's space during movement.
        You move at half speed while in its space.
        After moving through that creature's space, other creatures block you as normal for the remainder of your movement.
        Certain unusual creatures that occupy their entire space, such as gelatinous cubes, may be immune to this ability.

        \ff[6]{Spring Attack} As a standard action, you can spend an \glossterm{action point} to use this ability.
        If you do, move up to your movement speed and make a \glossterm{strike}.
        During the \glossterm{delayed action phase}, you may continue moving if you have remaining movement available in the phase.

        \ff[8]{Eye of the Storm} The penalty reduction increases to 2.

        \ff[10]{Hurricane Strike} The damage penalty on the strike you make with your \textit{whirlwind strike} ability is reduced to \minus1d.

        \ff[12]{Unfettered Movement, Greater} Moving through spaces occupied by creatures with your \textit{unfettered movement} ability does not reduce your movement speed.

        \ff[14]{Eye of the Storm, Greater} The penalty reduction increases to 3.

        \ff[16]{Spring Attack, Greater} When you use your \textit{spring attack} ability, you can make two \glossterm{strikes} instead of one.
        The two strikes cannot be made against the same creature.

        \ff[18]{Hurricane Defense} When you use your \textit{hurricane strike} ability, you ignore all \glossterm{overwhelm penalties} until the end of the round, including during the same phase.
    \end{feat}

    \begin{feat}{Intimidate Specialization}{Skill}
        \featpre Intimidate as a mastered skill.
        \featben

        \ff[1]{Specialization, Lesser} You gain a \plus2 bonus to Intimidate.

        \ff[2]{Critical Demoralization} If you get a \glossterm{critical hit} when you take the \textit{demoralize} action, the target is \frightened by you instead of being shaken.
        See \pcref{Demoralize}, for details.

        \ff[6]{Demoralizing Blow} Whenever you deal damage to a creature, you can spend an \glossterm{action point} to use the \textit{demoralize} ability as an \glossterm{immediate action}.

        \ff[8]{Specialization} The bonus to Intimidate increases to \plus3.

        \ff[16]{Specialization, Greater} The bonus to Intimidate increases to \plus4.
    \end{feat}

    \begin{feat}{Guardian}{Combat}
        \featpre Starting Perception and Willpower of 1.
        \featben

        \ff[1]{Redirection} Whenever an ally adjacent to you is hit by a \glossterm{strike}, you may use this ability as an \glossterm{immediate action}.
        If you do, you suffer all effects of the attack in place of the target.
        Any abilities you have that would make the strike miss or fail have no effect, but your abilities that allow you to reduce or ignore its effects work normally.
        You can learn which strikes hit the target in the current phase before deciding which strike to redirect, but not how much damage they would do (or any other effects they might have).

        \ff[2]{Defend the Weak, Lesser} You reduce the \glossterm{overwhelm penalties} of allies adjacent to you by 1.
        If an ally's overwhelm penalty is reduced to 0, it not considered to be \glossterm{overwhelmed}.

        \ff[4]{Binding Strike} As a standard action, you can spend an \glossterm{action point} to use this ability.
        \begin{ability}
            \begin{spelltargetinginfo}
                \spellquicktargeting{One creature}{Adjacent}
            \end{spelltargetinginfo}
            \begin{spelleffects}
                \spelleffect You make a \glossterm{strike} against the target.
                If the target takes damage from this strike, it is \immobilized.
                This effect is immediately broken if you stop being adjacent to the target.
                \spelldur Condition
            \end{spelleffects}
        \end{ability}

        \ff[6]{Expanded Redirection} You can use your \textit{redirection} ability to redirect the effects of any \glossterm{mundane} attack, not just \glossterm{strikes}.

        \ff[8]{Defend the Weak} The penalty reduction increases to 2.

        \ff[10]{Certain Bind} You gain a \plus1 bonus to accuracy with your \textit{binding strike} ability.

        \ff[12]{Martyr's Boon}[Magical] You can use an \glossterm{action point} to use your \textit{redirection} ability on any ally within \rnglong range of you.

        \ff[14]{Defend the Weak, Greater} The penalty reduction increases to 3.

        \ff[16]{Inescapable} Enemies you threaten must pay four times the normal movement cost to move out of squares you threaten.
        This replaces the normal penalties for moving through threatened squares (see \pcref{Moving Near Foes}).

        \ff[18]{Expanded Redirection, Greater}[Magical] You can use your \textit{redirection} ability to redirect the effects of any attack, not just \glossterm{mundane} attacks. 
    \end{feat}

    \begin{feat}{Heal Specialization}{Skill}
        \featpre Heal as a mastered skill.
        \featben

        \ff[1]{Specialization, Lesser} You gain a \plus2 bonus to Heal.

        \ff[2]{Healing Touch}[Magical] As a standard action, you can spend an \glossterm{action point} to use this ability.
        \begin{ability}
            \begin{spelltargetinginfo}
                \spellquicktargeting{One willing creature}{\rngtouch}
            \end{spelltargetinginfo}
            \begin{spelleffects}
                \spelleffect Make a Heal check. The target is healed for an amount equal to the Heal check result.
                % 10th level: 1d8 +1d per two levels is 4d6, or avg of 14 healing.
                % heal check average is 5 + 10 + 2 + 3 = 20 healing
                % 20th level: 1d8 +1d per two levels is 7d10, or avg of 38 healing.
                % heal check average is 5 + 20 + 2 + 4 = 31 healing
            \end{spelleffects}
        \end{ability}

        \ff[4]{Lifesaver} You gain a \plus5 bonus to Heal checks to stabilize dying creatures (see \pcref{Dying}).

        \ff[6]{Vital Healing} For every five points of healing you would restore with your \textit{healing touch}, you can instead heal a point of \glossterm{vital damage}.

        \ff[8]{Specialization} The bonus to Heal increases to \plus3.

        \ff[10]{Purging Touch}[Magical] As a standard action, you can spend an \glossterm{action point} to use this ability.
        \begin{ability}
            \begin{spelltargetinginfo}
                \spellquicktargeting{One willing creature}{\rngtouch}
            \end{spelltargetinginfo}
            \begin{spelleffects}
                \begin{spellattack}{Heal vs. Special}
                    \spellspecial The attack result is applied to every poison and disease on the target.
                    The DR for each effect is equal to 10 \add the \glossterm{power} of the effect.
                    \spellsuccess Success against a poison or disease causes it to be removed from the target, along with all of its lingering effects.
                \end{spellattack}
            \end{spelleffects}
        \end{ability}

        \ff[12]{Empowered Healing} When you use your \textit{healing touch} ability, you can heal additional hit points equal to your level.

        \ff[14]{Lifesaver, Greater} You can stabilize dying creatures as a \glossterm{swift action}.

        \ff[16]{Specialization, Greater} The bonus to Heal increases to \plus4.

        \ff[18]{Wellspring of Life} You do not have to spend an \glossterm{action point} to use your \textit{healing touch} ability.
    \end{feat}

    \begin{feat}{Illusionist}{Magical, Spell}
        \featpre \glossterm{Illusion} spell known.
        \featben

        \ff[1]{Create Image} As a standard action, you can spend an \glossterm{action point} to use this ability.
        \begin{ability}
            \begin{spelltargetinginfo}
                \spellrng{\rngmed}
            \end{spelltargetinginfo}
            \begin{spelleffects}
                \spelleffect This ability creates the visual illusion of an object, creature, or force, as determined by you.
                The figment's size must be no smaller than Tiny, and no larger than Large.
                The figment does not create sound, smell, or temperature.

                During the movement phase, you can move the figment anywhere within the range, with appropriate motions to simulate natural movement.
                For example, if you created the illusion of a squad of human guards, you could cause them to walk realistically across a room.
                The figments otherwise remain motionless, except for minor motions that simulate signs of life (if appropriate).
                If the figment ever leaves this ability's range, the effect immediately ends.

                The maximum intensity of a sensation created by this ability is not enough to have any significant detrimental effects on a human experiencing the sensation.
                For example, it can create a bright light, but not so bright that it would be physically painful to view.
                % should the light be visible outside the size of the illusion? If so, how to clarify?

                When you use this ability, you make a check with a bonus equal to your spellpower \add 5.
                Creatures can recognize the figment is created by illusory magic by interacting with it physically, or by making an Awareness check against a DR equal to your check result when using this ability.
                A creature gets a \plus10 bonus on this Awareness check when using senses which should be present in the figment, but which are missing.
                \spelldur Sustain (swift)
                \spelltags{Figment}
            \end{spelleffects}
        \end{ability}

        \ff[3]{Reflexive Illusion, Lesser} You gain a \plus1 bonus to Disguise, Sleight of Hand, and Stealth.

        \ff[5]{Controlled Image} While your \textit{create image} ability is active, you can concentrate on the figment as a standard action.
        If you do, you can directly control the figment's movement for the rest of the round, including their actions during the action phase.
        You cannot alter the fundamental shape of the figment, but you can have it perform complex actions, such as pretending to fight other creatures or engaging in a vigorous dance.
        The figment's ability to simulate physical tasks that require dexterity or training, such as juggling, is limited by your own.
        You may need to make relevant checks to make the figment perform complex actions.

        \ff[7]{Muffled Illusions} The level cost to apply the Silent standard augment to Illusion spells you cast is reduced by 1, to a minimum of 0 (see \pcref{Standard Augments}).
        In addition, choose a sense: sound, smell, or temperature.
        Your \textit{create image} ability can create sensations with the chosen sense.
        % How loud is the sound? How pungent is the smell? etc.

        \ff[9]{Flexible Image} Your \textit{create image} ability can create figments between Diminuitive and Huge size.

        \ff[11]{Reflexive Illusion} The bonus to skill checks increases to \plus2.

        \ff[13]{Subtle Illusions} The level cost to apply the Stilled standard augment to Illusion spells you cast is reduced by 1, to a minimum of 0 (see \pcref{Standard Augments}).
        In addition, choose a sense: sound, smell, or temperature.
        Your \textit{create image} ability can create sensations with the chosen sense.

        \ff[15]{Flexible Image, Greater} Your \textit{create image} ability can create figments between Fine and Gargantian size.

        \ff[17]{Reflexive Illusion} The bonus to skill checks increases to \plus3.
        In addition, choose a sense: sound, smell, or temperature.
        Your \textit{create image} ability can create sensations with the chosen sense.
    \end{feat}

    \begin{feat}{Perform Specialization}{Skill}
        \featpre Perform as a mastered skill.
        \featben

        \ff[1]{Specialization, Lesser} You gain a \plus2 bonus to Perform.

        \ff[2]{Inspiring Performance}[Magical] As a standard action, you can spend an \glossterm{action point} to use this ability.
        When you do, you begin a performance using one of your Perform skills.
        \begin{ability}
            \begin{spelltargetinginfo}
                \spellquicktargeting{One creature}{\rngmed}
            \end{spelltargetinginfo}
            \begin{spelleffects}
                \spelleffect Make a Perform check to create an performance that can inspire the target.
                The target gains temporary hit points equal to your check result.
                \spelldur Sustain (swift). Sustaining this ability requires continuing the performance you started when you used this ability. If the target can neither see nor hear your performance, the effect immediately ends.
                \spelltags{\glossterm{Delusion}, \glossterm{Mind}. In addition, this may have the \glossterm{Auditory} or \glossterm{Visual} tags, depending on the nature of the performance.}
            \end{spelleffects}
        \end{ability}

        \ff[4]{Mesmerizing Performance}[Magical] As a standard action, you can spend an \glossterm{action point} to use this ability.
        When you do, you begin a performance using one of your Perform skills.
        \begin{ability}
            \begin{spelltargetinginfo}
                \spellquicktargeting{Up to five creatures}{\rngmed}
            \end{spelltargetinginfo}
            \begin{spelleffects}
                \begin{spellattack}{Perform vs. Mental}
                    \spellspecial If the target thinks that you or your allies are threatening it, you take a \minus5 penalty to accuracy on the attack.
                    \spellsuccess The target is \fascinated by you.
                    Any act by you or your apparent allies that threatens or damages the target breaks the effect.
                    An observant target may interpret overt threats to its allies as a threat to itself.
                \end{spellattack}
                \spelldur Sustain (swift). Sustaining this ability requires continuing the performance you started when you used this ability. If the target can neither see nor hear your performance, the effect immediately ends.
                \spelltags{\glossterm{Compulsion}, \glossterm{Mind}. In addition, this may have the \glossterm{Auditory} or \glossterm{Visual} tags, depending on the nature of the performance.}
            \end{spelleffects}
        \end{ability}

        \ff[6]{Inspire Courage}[Magical] A target affected by your \textit{inspiring performance} is immune to \glossterm{Fear} abilities as long as the effect lasts.

        \ff[8]{Mass Performance}[Magical] You can target up to two creatures with your \textit{inspiring performance} ability.
        In addition, you can target any number of creatures with your \textit{mesmerizing performance} ability.

        \ff[8]{Specialization} The bonus to Perform increases to \plus3.

        \ff[10]{Mesmeric Suggestion}[Magical] As a standard action, you can spend an \glossterm{action point} to use this ability.
        \begin{ability}
            \begin{spelltargetinginfo}
                \spellquicktargeting{One creature}{\rngmed}
            \end{spelltargetinginfo}
            \begin{spelleffects}
                \spellspecial When you use this ability, you make a verbal suggestion of a particular course of action to the target.
                You can work this suggestion into an active performance without penalty.
                \begin{spellattack}{Perform vs. Mental}
                    \spellspecial If the target is not currently \fascinated by your \textit{fascinating performance} ability, this attack automatically fails.
                    If your suggestion does not seem reasonable, you take a \minus5 penalty to accuracy on this attack.
                    Exceptionally unreasonable suggestions can impose even greater penalties, and exceptionally reasonable suggestions can give accuracy bonuses.
                    \spellsuccess The target thinks your suggestion is a good idea and will try to follow it to the best of its abilities.
                    Any act by you or your apparent allies that threatens or damages the target breaks the effect.
                \end{spellattack}
                \spelldur Sustain (swift). Sustaining this ability requires continuing the performance you started when you used this ability. If the target can neither see nor hear your performance, the effect immediately ends.
                \spelltags{\glossterm{Compulsion}, \glossterm{Mind}. In addition, this may have the \glossterm{Auditory} or \glossterm{Visual} tags, depending on the nature of the performance.}
            \end{spelleffects}
        \end{ability}

        \ff[12]{Endless Fascination}[Magical] You can use your \textit{mesmerizing performance} ability without spending an action point.

        \ff[14]{Irresistible Dance} You ignore all failure chances on Perform attacks and checks you make.

        \ff[16]{Mass Performance, Greater}[Magical] When you use your \textit{inspiring performance} ability, you can target any number of creatures.
        In addition, the range of your \textit{mesmerizing performance} ability is increased to \rnglong.

        \ff[16]{Specialization, Greater} The bonus to Perform increases to \plus4.

        % maybe add ability to remove combat penalty for fascination ability

        \ff[18]{Inspire Heroism} Each creature affected by your \textit{inspiring performance} gains additional temporary hit points equal to your level.
    \end{feat}

    \begin{feat}{Bardic Exemplar}{Magical, Skill}
        \featpre Perform Specialization feat.
        \featben

        \ff[1]{Mocking Performance} As a standard action, you can spend an \glossterm{action point} to use this ability.
        When you do, you begin a performance using one of your Perform skills.
        \begin{ability}
            \begin{spelltargetinginfo}
                \spellquicktargeting{One creature}{\rngmed}
            \end{spelltargetinginfo}
            \begin{spelleffects}
                \begin{spellattack}{Perform vs. Mental}
                    \spellsuccess The target is \taunted by a willing creature of your choice within range of you.
                \end{spellattack}
                \spelldur Sustain (swift). Sustaining this ability requires continuing the performance you started when you used this ability. If the target can neither see nor hear your performance, the effect immediately ends.
                \spelltags{\glossterm{Delusion}, \glossterm{Mind}. In addition, this may have the \glossterm{Auditory} or \glossterm{Visual} tags, depending on the nature of the performance.}
            \end{spelleffects}
        \end{ability}

        \ff[3]{Perform Point} You gain a perform point.
        A perform point can be spent to use \glossterm{magical} abilities from this feat and the Perform Specialization feat in place of an action point.
        You recover all spent perform points after a \glossterm{short rest}.

        \ff[5]{Song of Serenity} As a standard action, you can spend an \glossterm{action point} to use this ability.
        When you do, you begin a performance using one of your Perform skills.
        \begin{ability}
            \begin{spelltargetinginfo}
                \spellquicktargeting{One creature}{\rngmed}
            \end{spelltargetinginfo}
            \begin{spelleffects}
                \spelleffect The target is immune to hostile \glossterm{Mind} effects.
                \spelldur Sustain (swift). Sustaining this ability requires continuing the performance you started when you used this ability. If the target can neither see nor hear your performance, the effect immediately ends.
                \spelltags{\glossterm{Delusion}, \glossterm{Mind}. In addition, this may have the \glossterm{Auditory} or \glossterm{Visual} tags, depending on the nature of the performance.}
            \end{spelleffects}
        \end{ability}

        \ff[7]{Demoralizing Mockery} A creature affected by your \textit{mocking performance} ability takes a \minus2 penalty to Mental defense as long as the effect lasts.

        \ff[9]{Hybrid Performance} You can sustain two different \glossterm{magical} abilities from this feat or the Perform Specialization feat as part of the same \glossterm{swift action}.

        \ff[11]{Mass Performance} When you use your \textit{mocking performance} ability, you can target up to five creatures.
        Each affected creature is taunted to attack the same creature.
        In addition, when you use your \textit{song of serenity} ability, you can target up to two creatures.

        \ff[13]{Battle Cry} As a standard action, you can spend an \glossterm{action point} to use this ability.
        When you do, you begin a performance using one of your Perform skills.
        \begin{ability}
            \begin{spelltargetinginfo}
                \spellquicktargeting*{Any number of creatures}{\rngmed}
            \end{spelltargetinginfo}
            \begin{spelleffects}
                \spelleffect The target gains a \plus2 bonus to accuracy with all attacks.
                \spelldur Sustain (swift). Sustaining this ability requires continuing the performance you started when you used this ability. If the target can neither see nor hear your performance, the effect immediately ends.
                \spelltags{\glossterm{Delusion}, \glossterm{Mind}. In addition, this may have the \glossterm{Auditory} or \glossterm{Visual} tags, depending on the nature of the performance.}
            \end{spelleffects}
        \end{ability}

        \ff[15]{Serene Bliss} A creature affected by your \textit{song of serenity} ability is not \glossterm{bloodied} as long as the effect lasts and it has hit points remaining.
        It suffers the normal penalties for becoming bloodied if the effect ends or it has no hit points remaining.

        \ff[17]{Mass Performance, Greater} When you use your \textit{mocking performance} ability, you can target any number of creatures.
        Each affected creature is taunted to attack the same creature.
        In addition, when you use your \textit{song of serenity} ability, you can target up to five creatures.
        Finally, the range of your \textit{battle cry} ability is increased to \rnglong.

        \ff[19]{Champion's Anthem} A creature affected by your \textit{battle cry} ability may use your level in place of its accuracy with any attack.

        \ff[19]{Hideous Laughter} If you get a \glossterm{critical hit} against a creature with your \textit{mocking performance} ability, it laughs uncontrollably until the end of the next round.
        During this time, it can take no other actions.
    \end{feat}

    \begin{feat}{Iron Will}{General}
        \featpre Starting Willpower of 2.
        \featben

        \ff[1]{Resilient} You gain additional hit points equal to your Willpower.

        \ff[2]{Mental Discipline, Lesser} You gain a \plus1 bonus to Mental defense.

        \ff[4]{Unbending Will} You are immune to \glossterm{Compulsion} effects.

        \ff[6]{Mind over Matter} You take a \minus2 penalty for being \glossterm{bloodied} instead of a \minus5 penalty.
        In addition, having \glossterm{vital damage} causes you to sufffer a penalty to accuracy, checks, and defenses equal to half the amount of vital damage you have.

        \ff[8]{Mental Discipline} The bonus to Mental defense increases to \plus2.

        \ff[10]{Mental Fortress} You are immune to hostile \glossterm{Mind} effects.

        \ff[12]{Resilient, Greater} The increase to hit points increases to twice your Willpower.

        \ff[14]{Mental Discipline, Greater} The bonus to Mental defense increases to \plus3.

        \ff[16]{Mind over Matter, Greater} You do not take penalties for being \glossterm{bloodied}.
        In addition, having \glossterm{vital damage} imposes one quarter of the normal penalties, rather than half.

        \ff[18]{Soulbound} When you die, your soul does not depart your body for an hour, during which time your body is treated as alive.
        If your body is healed of all of its \glossterm{vital damage} during that time, you immediately return to life.
        In addition, you are immune to \glossterm{Death} effects.
    \end{feat}

    \begin{feat}{Jump Specialization}{Skill}
        \featpre Jump as a mastered skill.
        \featben

        \ff[1]{Specialization, Lesser} You gain a \plus2 bonus to Jump.

        \ff[2]{Instant Leap} You must move at least five feet before jumping to have a running start, rather than twenty feet.

        \ff[4]{Leaping Strike} As a standard action, you can spend an \glossterm{action point} to use this ability.
        \begin{ability}
            \begin{spelleffects}
                \spelleffect You make a Jump check to leap (see \pcref{Leap}), and move as normal for the leap, up to a maximum distance equal to your land speed.
                During the \glossterm{delayed action phase}, you can make a \glossterm{strike} from your new location.
            \end{spelleffects}
        \end{ability}

        \ff[6]{Featherlight Jump} When you jump, you can spend an \glossterm{action point} as a \glossterm{free action}.
        If you do, your maximum height for that jump is equal to your Jump check result, rather than half your Jump check result.

        \ff[8]{Specialization} The bonus to Jump increases to \plus3.

        \ff[10]{Death from Above} When you use your \textit{leap strike} ability, you can make the strike during the \glossterm{action phase} instead of the \glossterm{delayed action phase}.
        In addition, when you make the strike, you can be in any location along your jump's path.
        If you are above your target when you make the strike, you gain a \plus1d bonus to damage.

        \ff[12]{Featherlight} You do not need to spend an \glossterm{action point} to use your \textit{featherlight jump} ability.

        \ff[14]{Instant Leap, Greater} You are always considered to have a running start when jumping, even when rebounding off of objects (see \pcref{Rebounding Leap}).

        \ff[16]{Specialization, Greater} The bonus to Jump increases to \plus4.

        \ff[18]{Death from Above, Greater} The bonus to damage for being above your target increases to \plus2d.
    \end{feat}

    \begin{feat}{Knowledge Specialization}{Skill}
        \featpre Knowledge as a mastered skill.
        \featben

        \ff[1]{Specialization, Lesser} You gain a \plus2 bonus to Knowledge.

        \ff[4]{Knowledge Savant} You gain two additional \glossterm{skill points} which can only be spent on Knowledge skills.

        \ff[8]{Specialization} The bonus to Knowledge increases to \plus3.

        \ff[12]{Knowledge Savant, Greater} The number of extra skill points increases to four.

        \ff[16]{Specialization, Greater} The bonus to Knowledge increases to \plus4.
    \end{feat}

    \begin{feat}{Agility}{Combat}
        \featpre Starting Dexterity of 2.
        \featben

        \ff[1]{Dodge} At the start of each phase, you can choose a creature you can see.
        You gain a \plus1 bonus to Armor defense against attacks by that target during that phase.

        \ff[2]{Lightning Reflexes, Lesser} You gain a \plus1 bonus to Reflex defense.

        \ff[4]{Rapid Reaction} You gain a \plus2 bonus to \glossterm{initiative} checks.

        \ff[6]{Mobility} Your movement is not impeded by being threatened by the target of your \textit{dodge} ability (see \pcref{Moving Near Foes}).

        \ff[8]{Lightning Reflexes} The bonus to Reflex defense increases to \plus2.

        \ff[10]{Rapid Reaction, Greater} The bonus to initiative checks increases to \plus5.

        \ff[12]{Focused Dodge} You do not suffer \glossterm{overwhelm penalties} against attacks by the target of your \textit{dodge} ability.
        In addition, when counting the number of creatures threatening you to determine your overwhelm penalties, you ignore that creature.

        \ff[14]{Lightning Reflexes, Greater} The bonus to Reflex defense increases to \plus3.

        \ff[16]{Evasion} Your \glossterm{overwhelm penalties} do not affect your Reflex defense.

        \ff[18]{Supreme Dodge} The bonus to Armor defense granted by your \textit{dodge} ability increases to \plus2.
    \end{feat}

    \begin{feat}{Linguistics Specialization}{Skill}
        \featpre Linguistics as a mastered skill.
        \featben

        \ff[1]{Specialization, Lesser} You gain a \plus2 bonus to Linguistics.

        \ff[2]{Linguistic Savant} You learn two additional \glossterm{common languages}, or one additional \glossterm{rare language}.

        \ff[8]{Specialization} The bonus to Linguistics increases to \plus3.

        \ff[12]{Linguistic Savant, Greater} You learn two additional \glossterm{common languages}, or one additional \glossterm{rare language}.

        \ff[16]{Specialization, Greater} The bonus to Linguistics increases to \plus4.
    \end{feat}

    \begin{feat}{Miscaster}{Magical, Spell}
        \featpre Ability to cast a spell.
        \featben

        \ff[1]{Selective Backlash} Your \glossterm{miscast backlash} does not hurt your allies, though it still hurts you.

        \ff[3]{Overchannel} Whenever you cast a spell, you can use this ability as an \glossterm{immediate action}.
        Your concentration on the spell cannot be disrupted, and you cannot \glossterm{miscast} the spell for any other reason.
        Effects that prevent the spell from having any effect, such as the \textit{counterspell} ability from the Abjurer feat, work normally (see \pcref{Abjurer}).
        In addition, you cause a \glossterm{miscast backlash} when the spell resolves.

        However, using this ability causes a mystic backlash from channeling excess magical energy.
        During the next round, you cannot cast spells other than \glossterm{cantrips}, and you take a \minus2 penalty to all defenses.

        \ff[5]{Widened Backlash} The area affected by your \glossterm{miscast backlash} increases to a \areasmall radius burst centered on you.

        \ff[7]{Suppressed Backlash} Whenever you miscast a spell, you can suppress the \glossterm{miscast backlash}.
        If you do, you regain the action point spent to cast the spell (if any).

        \ff[9]{Selective Backlash, Greater} Your \glossterm{miscast backlash} does not hurt you.

        \ff[11]{Resilient Channeler} Using your \textit{overchannel} ability does not impose a penalty to your defenses.

        \ff[13]{Widened Backlash, Greater} The area increases to a \areamed radius burst centered on you.

        \ff[15]{Magical Resilience} You gain \glossterm{magic resistance} equal to 5 \add your level.

        \ff[19]{Empowered Channeler} When you use your \textit{overchannel} ability, you gain a \plus2 bonus to spellpower with the spell.
    \end{feat}

    \begin{feat}{Ride Specialization}{Skill}
        \featpre Ride as a mastered skill.
        \featben

        \ff[1]{Specialization, Lesser} You gain a \plus2 bonus to Ride.

        \ff[2]{Mounted Warrior} Your mount gains a \plus2 bonus to \glossterm{physical defenses}, up to a maximum of your own defenses.

        \ff[4]{Mounted Archer} The penalty you take when using a ranged weapon while mounted is decreased by 4: \minus0 instead of \minus4 if your mount moves during the same phase, and \minus4 instead of \minus8 if your mount is sprinting during the same phase.

        \ff[6]{Knight's Charge} When you \glossterm{charge} a creature while mounted, you gain a \plus1d bonus to damage with the strike.

        \ff[8]{Specialization} The bonus to Ride increases to \plus3.

        \ff[10]{Mounted Warrior, Greater} The bonus to defenses increases to \plus3.

        \ff[12]{Mounted Archer, Greater} The penalty reduction increases to \minus8.

        \ff[14]{Knight's Charge, Greater} The damage bonus increases to \plus2d.

        \ff[16]{Specialization, Greater} The bonus to Ride increases to \plus4.

        \ff[18]{Loyal Rider} Whenever your mount takes \glossterm{physical} damage, you may choose to take half that damage, rounded down.
        The mount takes the other half.
    \end{feat}

    \begin{feat}{Persuasion Specialization}{Skill}
        \featpre Persuasion as a mastered skill.
        \featben

        \ff[1]{Specialization, Lesser} You gain a \plus2 bonus to Persuasion.

        \ff[2]{}

        \ff[8]{Specialization} The bonus to Persuasion increases to \plus3.

        \ff[10]{Suggestion} As a standard action, you can spend an \glossterm{action point} to use this ability.

        \ff[16]{Specialization, Greater} The bonus to Persuasion increases to \plus4.
    \end{feat}

    \begin{feat}{Sense Motive Specialization}{Skill}
        \featpre Sense Motive as a mastered skill.
        \featben

        \ff[1]{Specialization, Lesser} You gain a \plus2 bonus to Sense Motive.

        \ff[6]{Read Mind}[Magical] As a standard action, you can spend an \glossterm{action point} to use this ability.
        \begin{ability}
            \begin{spelltargetinginfo}
                \spellquicktargeting{One creature}{\rngclose}
            \end{spelltargetinginfo}
            \begin{spelleffects}
                \begin{spellattack}{Sense Motive vs. Mental}
                    \spellsuccess You know the target's surface thoughts.
                    You gain a \plus2 bonus to Bluff, Persuasion, and Intimidate checks against a creature whose mind you are reading.
                \end{spellattack}
                \spelldur Sustain (swift)
                \spelltags{\glossterm{Mind}}
            \end{spelleffects}
        \end{ability}

        \ff[8]{Specialization} The bonus to Sense Motive increases to \plus3.

        \ff[16]{Specialization, Greater} The bonus to Sense Motive increases to \plus4.
    \end{feat}

    \begin{feat}{Sleight of Hand Specialization}{Skill}
        \featpre Sleight of Hand as a mastered skill.
        \featben

        \ff[1]{Specialization, Lesser} You gain a \plus2 bonus to Sleight of Hand.

        \ff[2]{}

        \ff[8]{Specialization} The bonus to Sleight of Hand increases to \plus3.

        \ff[10]{Extradimensional Concealment}[Magical] Whenever you take the \textit{conceal object} action, you can spend an \glossterm{action point}.
        If you do, you conceal the object in a pocket dimension that cannot be accessed by nonmagical means.
        This pocket dimension can only hold one object at a time.

        \ff[10]{Extradimensional Retrieval}[Magical] As a standard action, you can spend an \glossterm{action point} to use this ability.
        You reach into your pocket dimension to retrieve the object you stored there previously.

        Alternately, you can reach into the pocket dimension belonging to a creature you are touching to retrieve the object stored there.
        If that creature does not have the \textit{extradimensional concealment} ability, or does not have an object in their pocket dimension, this ability fails.

        \ff[16]{Specialization, Greater} The bonus to Sleight of Hand increases to \plus4.
    \end{feat}

    \begin{feat}{Spellcraft Specialization}{Skill}
        \featpre Spellcraft as a mastered skill.
        \featben

        \ff[1]{Specialization, Lesser} You gain a \plus2 bonus to Spellcraft.

        \ff[8]{Specialization} The bonus to Spellcraft increases to \plus3.

        \ff[16]{Specialization, Greater} The bonus to Spellcraft increases to \plus4.
    \end{feat}

    \begin{feat}{Sprint Specialization}{Skill}
        \featpre Sprint as a mastered skill.
        \featben

        \ff[1]{Specialization, Lesser} You gain a \plus2 bonus to Sprint.

        \ff[2]{Swift, Lesser} You gain a \plus10 foot bonus to speed in all your movement modes.

        \ff[6]{Endurance Runner} You can sprint for a number of rounds equal to 5 \add half your Constitution, instead of 2 \add half your Constitution.

        \ff[8]{Specialization} The bonus to Sprint increases to \plus3.

        \ff[10]{Swift, Lesser} The speed bonus increases to \plus20 feet.

        \ff[14]{Endurance Runner, Greater} You can sprint for a number of minutes equal to 2 \add half your Constitution.

        \ff[16]{Specialization, Greater} The bonus to Sprint increases to \plus4.
    \end{feat}

    \begin{feat}{Stealth Specialization}{Skill}
        \featpre Stealth as a mastered skill.
        \featben

        \ff[1]{Specialization, Lesser} You gain a \plus2 bonus to Stealth.

        \ff[2]{}

        \ff[6]{Movement Tolerance} Your penalties for moving while hiding are reduced by 2. 

        \ff[8]{Specialization} The bonus to Stealth increases to \plus3.

        \ff[10]{Hide in Plain Sight} You can use the \textit{hide} ability even while observed.
        You take take a \minus10 penalty to the Stealth check when hiding in this way, and you still need passive cover or concealment to hide.

        \ff[14]{Movement Tolerance, Greater} The reduction of penalties for moving increases to 5.
        This allows you to move at half speed without penalty.

        \ff[16]{Specialization, Greater} The bonus to Stealth increases to \plus4.

        \ff[18]{Hide in Plain Sight, Greater} The penalty for hiding while observed is reduced to \minus5.
    \end{feat}

    \begin{feat}{Survival Specialization}{Skill}
        \featpre Survival as a mastered skill.
        \featben

        \ff[1]{Specialization, Lesser} You gain a \plus2 bonus to Survival.

        \ff[2]{Terrain Tolerance} You ignore \glossterm{difficult terrain} and harmful natural terrain of any kind.
        If a skill check, such as Climb or Swim, would normally be required to move through the terrain, this ability does not help.

        \ff[4]{Trackless} You can choose to leave no trace of your passage as you move.
        If you do, tracking you is impossible by any \glossterm{mundane} means.

        \ff[6]{Rapid Tracker} 
        Your ability to track your foes improves.
        You can move at your normal speed while following tracks without taking the normal \minus5 penalty.
        In addition, you take only a \minus10 penalty (instead of the normal \minus20) when moving at up to twice normal speed while tracking.

        \ff[8]{Specialization} The bonus to Survival increases to \plus3.

        \ff[10]{Planar Tolerance} You are immune to harmful planar effects.

        \ff[16]{Specialization, Greater} The bonus to Survival increases to \plus4.
    \end{feat}

    \begin{feat}{Swim Specialization}{Skill}
        \featpre Swim as a mastered skill.
        \featben

        \ff[1]{Specialization, Lesser} You gain a \plus2 bonus to Swim.

        \ff[2]{}

        \ff[4]{Swim Speed} You gain a \glossterm{swim speed} equal to your land speed.
        A successful Swim check to move allows you to move a distance equal to your swim speed.

        \ff[8]{Specialization} The bonus to Swim increases to \plus3.

        \ff[10]{Underwater Tolerance} You do not suffer any penalties to physical melee attacks, checks, or physical defenses for being underwater.
        You still suffer the normal penalty with underwater ranged attacks.

        \ff[16]{Specialization, Greater} The bonus to Swim increases to \plus4.
    \end{feat}

    \begin{feat}{Transmuter}{Magical, Spell}
        \featpre \glossterm{Transmutation} spell known.
        \featben

        \ff[1]{Fortify Body} You gain a \plus1 bonus to Armor defense.

        \ff[3]{Alter Self} As a standard action, you can spend an \glossterm{action point} to use this ability.
        \begin{ability}
            \begin{spelleffects}
                \spelleffect You make a Disguise check to alter your appearance (see \pcref{Disguise creature}), except that you can use your spellpower in place of your Disguise skill.
                You can only alter your physical body, not your clothes or equipment.
            \end{spelleffects}
            \spelldur Permanent; if you use this ability again, any the effects of any previous use immediately ends.
            \spelltags{\glossterm{Alteration}}
        \end{ability}

        \ff[5]{Alter Object} As a standard action, you can spend an \glossterm{action point} to use this ability.
        \begin{ability}
            \begin{spelltargetinginfo}
                \spellquicktargeting{One unattended, nonmagical object}{Adjacent}
            \end{spelltargetinginfo}
            \begin{spelleffects}
                \spelleffect You make a Craft check to alter the target (see \pcref{Craft}), except that you can use your spellpower in place of your Craft skill.
                In addition, you do not need any special tools to make the check (such as an anvil and furnace).
                However, the maximum hardness of a material you can affect with this ability is equal to your spellpower.

                Each time you use this ability, you can accomplish work that would take up to five minutes with a normal Craft check.
                % too short?
            \end{spelleffects}
            \spelldur Instantaneous
            \spelltags{\glossterm{Alteration}}
        \end{ability}

        \ff[7]{Enhance Body} You gain a \plus1 bonus to Strength-based checks.

        \ff[9]{Alter Poison} As a standard action, you can spend an \glossterm{action point} to use this ability.
        \begin{ability}
            \begin{spelltargetinginfo}
                \spellquicktargeting{One creature}{\rngclose}
            \end{spelltargetinginfo}
            \begin{spelleffects}
                \begin{spellattack}{Spellpower vs. Fortitude}
                    \spellsuccess Any poison in the target's system is neutralized.
                    It stops suffering any additional effects from poisons in its system.
                    As long as the effect lasts, it is immune to all poisons.
                    In addition, the target's \glossterm{mundane} poisons, including natural attacks that inflict poison, have no effect.
                \end{spellattack}
                \spelldur Sustain (swift).
                \spelldur{\glossterm{Alteration}}
            \end{spelleffects}
        \end{ability}

        \ff[11]{Alter Object, Greater} When you use your \textit{alter object} ability, you can accomplish work that would take up to an hour with a normal Craft check.

        \ff[13]{Enhance Body, Greater} The bonus to Strength-based checks increases to \plus2.
    \end{feat}

    \begin{feat}{Vivimancer}{Magical, Spell}
        \featpre \glossterm{Vivimancy} spell known.
        \featben

        \ff[1]{Restore Life} As a standard action, you can spend an \glossterm{action point} to use this ability.
        \begin{ability}
            \begin{spelltargetinginfo}
                \spellquicktargeting{One creature}{\rngclose}
            \end{spelltargetinginfo}
            \begin{spelleffects}
                \spelleffect The target is healed for \spelldamage{}.
                \spelltags{\glossterm{Life}}
            \end{spelleffects}
        \end{ability}

        \ff[3]{Unliving Resilience, Lesser} You are immune to \glossterm{disease}.

        \ff[5]{Lifebound Resilience, Lesser} You are immune to hostile \glossterm{Death} effects.

        \ff[7]{Restore Vitality} When you use your \textit{restore life} ability, for every 2 points of healing, you can instead heal 1 \glossterm{vital damage}.

        \ff[9]{Unliving Resilience} You are immune to life damage and hostile \glossterm{Life} effects.

        % Too small? Should this be full spellpower?
        \ff[11]{Lifebound Resilience} At the end of each round, you heal hit points equal to half your spellpower.

        \ff[15]{Unliving Resilience, Greater} You are healed by life damage instead of being immune to it.
        This healing applies before immunity, allowing you to heal from life damage dealt by \glossterm{Life} effects without suffering any other effects..

        \ff[17]{Lifebound Resilience, Greater} You heal hit points equal to your spellpower each round, instead of half your spellpower.
    \end{feat}

\section{Other Feat Rules}

    % TODO: are there any abilities left that grant bonus feats?
    \subsection{Bonus Feats}
        Some abilities grant a character bonus feats.
        Unless otherwise specified, the character must still meet any prerequisites for the feat.
        If the character does not meet the prerequisites at the time the bonus feat is granted, the character does not gain the feat.
        If the character later meets the prerequisites, the character immediately gains the benefit of the bonus feat.

        If a character gains a feat as a bonus feat that he or she has already acquired through other means, the character may instead select any other feat for which she qualifies.

    \subsection{Retraining Feats}
        At every level, your character can choose to retrain an old feat in exchange for a new feat.
