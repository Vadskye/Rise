\chapter{Feats}\label{Feats}

This chapter describes a set of optional rules that the GM can choose to use in a campaign.
If you use these rules, characters gain feats which allow them to further specialize in specific areas, making characters more mechanically distinct from each other.
Feats also make the system more mechanically complex, so they are not necessarily enjoyable for all groups.

\section{Gaining Feats}
    There are two main ways you can use feats in your game.
    You cannot gain the same feat twice.

    \parhead{Species Feat Only}
    One simple option is that characters gain a single feat at 1st level based on their species, and no other feats.
    This makes your choice of species more significant without dramatically increasing character complexity.

    \parhead{Feat Progression}
    If you want characters to be more complex and to have more powerful abilities, you can also use a feat progression system.
    For example, you could gain a feat from your species at 1st level, and an additional feat at 3rd, 6th, and 9th level.
    Alternately, you could gain feats based on the completion of major story events.
    In general, it is inadvisable to gain more than four feats total, or to gain feats after about 10th level.

    \subsection{Species Bonus Feats}\label{Species Bonus Feats}
        If you use this rule, each species grants a bonus feat at 1st level.
        Most species can only choose from a small group of feats.
        The specific feats for each species are listed below.
        A character must meet any prerequisites for these bonus feats, as normal.

        \parhead{Human} Any feat.

        \parhead{Dwarf} Any from the following list: \featref*{Battle Armory}, \featref*{Blindfighter}, \featref*{Craft Specialization} and \featref*{Endurance Specialization}, \featref*{Iron Will}, \featref*{Martial Training}, or \featref*{Toughness}.

        \parhead{Elf} Any Casting feat (see \pcref{Casting Feats}), or any from the following list: \featref*{Awareness Specialization} and \featref*{Balance Specialization}, \featref*{Sniper}, or \featref*{Rapid Reaction}.

        \parhead{Gnome} Any Casting feat (see \pcref{Casting Feats}), or any from the following list: \featref*{Blindfighter}, \featref*{Craft Specialization} and \featref*{Stealth Specialization}, \featref*{Telepath}, or \featref*{Toughness}.

        \parhead{Half-Elf} Any Skill feat (see \pcref{Skill Feats}).

        \parhead{Half-Orc} Any Combat feat (see \pcref{Combat Feats}), or any from the following list: \featref*{Endurance Specialization} and \featref*{Intimidate Specialization}, or \featref*{Toughness}.

        \parhead{Halfling} Any from the following list: \featref*{Balance Specialization} and \featref*{Stealth Specialization}, \featref*{Climb Specialization} and \featref*{Jump Specialization}, \featref*{Iron Will}, or \featref*{Rapid Reaction}.

        \subsubsection{Uncommon Species}
            If you are using uncommon species, the feat lists for each uncommon species are given below.
            Note that uncommon species are normally ineligible for any Bloodline feats.

            \parhead{Animal Hybrid} Any feat strongly associated with the chosen animal. For example, a hybrid shark might choose from Awareness Specialization, Survival Specialization, Swiftrunner, or Swim Specialization. A hybrid wolf might choose from Awareness Specialization, Rapid Reaction, Stealth Specialization, Survival Specialization, or Swiftrunner.

            \parhead{Awakened Animal} Any feat strongly associated with the chosen animal. For example, an awakened cat might choose from Awareness Specialization, Climb Specialization, Flexibility Specialization, Rapid Reaction, Stealth Specialization, or Swiftrunner.

            \parhead{Changeling} Chameleon, or any Skill feat.

            \parhead{Dragon} Iron Will, Toughness, or any Casting feat (see \pcref{Casting Feats}).

            \parhead{Drakkenfel} Draconic Heritage. The type of dragon chosen for the drakkenfel's \textit{draconic ancestry} must match its \textit{draconic essence}.

            \parhead{Dryaidi} Herbalist, Mental Magic, Regenerator, Sphere Focus: Toxicology, Sphere Focus: Verdamancy, or Toughness.

            \parhead{Eladrin} Boongiver, Chameleon, Combat Style Versatility, Deception Specialization and Persuasion Specialization, or Spellwarped.

            \parhead{Kit} Balance Specialization and Stealth Specialization, Deception Specialization and Social Insight Specialization, or Swiftrunner.

            \parhead{Naiadi} Boongiver, Leadership, Mental Magic, Persuasion Specialization and Swim Specialization, or Sphere Focus: Aquamancy.

            \parhead{Orc} Any Combat feat (see \pcref{Combat Feats}), or any from the following list: \featref*{Endurance Specialization} and \featref*{Intimidate Specialization}, or \featref*{Toughness}.

            \parhead{Oozeborn} Blindfighter, Chameleon, Climb Specialization and Flexibility Specialization, Juggernaut, Regenerator, Sphere Focus: Toxicology, or Toughness.

            \parhead{Tiefling} \featref*{Deception Specialization} and \featref*{Intimidate Specialization}, \featref*{Executioner}, \featref*{Spellwarped}, or \featref*{Sphere Focus: Pyromancy}.

        \subsubsection{Changing Species}
            In extraordinary cases, a creature may change its species.
            For example, the \ritual{reincarnation} ritual returns a creature to life as a different species.
            Regardless of its new species, the creature keeps its original species bonus feat.

\section{Feat Mechanics}

    \subsection{Prerequisites}
        Some feats have prerequisites.
        Unless you meet all of the prerequisites, you cannot take the feat.
        Prerequisites can include a minimum base attribute score, another feat or feats, a minimum level of training in a skill, or some other property of a character.
        A character can gain a feat at the same level at which they gain the prerequisite.

        A character who no longer meets all prerequisites for a feat loses all abilities from that feat.

    \subsection{Skill Feats}
        Skill feats are weaker and more narrow in their focus than other feats.
        Whenever a character would choose a single feat, they may instead choose two skill feats.

    \subsection{Feat Tags}
        All feats are organized into different groups by tags.

        \parhead{General} General feats can have a wide variety of effects.
        They often grant new abilities or improve your defenses.

        \parhead{Combat} Combat feats improve your combat capabilities.
        They can increase the damage you deal, grant you new combat abilities, or improve your defenses.

        \parhead{Casting} Casting feats improve your spellcasting abilities.
        Casting feats are useless to characters who cannot cast spells.

        \parhead{Skill} Skill feats improve your skills.
        They can make you more likely to succeed with skill checks and grant you new abilities based on your skills.

        \parhead{Bloodline Feats} Some characters have traces of monstrous blood running in their veins.
        Most of those will never understand the full potential of their unusual heritage.
        Bloodline feats allow characters to explore those posibilities by gaining abilites related to their ancestry.
        You can only have one Bloodline feat.

        \parhead{Magical Feats}
        All abilities granted by feats with the Magical type are \glossterm{magical} in nature.
        Many feats are not entirely magical, but have specific effects that are magical.

% Feat names must follow ``I have'' or ``I am (a)''.
\begin{longtablewrapper}
    \tablebookmark{Feats}{feats}
    \begin{longtable}{>{\lcol}p{11em} >{\lcol}p{12em} l >{\lcol}p{8em} >{\lcol}p{3em}}
        \lcaption{Feats}\\
        \tb{General Feats}\label{General Feats} & \tb{Prerequisites} & \tb{Benefits} & \tb{Feat Types} & \tb{Page} \\
        \featref{Celestial Heritage} & Non-evil                 & Gain aspects of celestial beings    & Bloodline, Magical & \featpref{Celestial Heritage} \\
        \featref{Chameleon}          & Trained Disguise, Int 2  & Adapt your archetypes and abilities & \tdash             & \featpref{Chameleon}          \\
        \featref{Draconic Heritage}  & \tdash                   & Gain aspects of draconic power      & Bloodline          & \featpref{Draconic Heritage}  \\
        \featref{Entropist}          & Wil 2                    & Master chaos and entropy            & \tdash             & \featpref{Entropist}          \\
        \featref{Herbalist}          & Trained Knowledge (nature) & Brew potions with natural ingredients & \tdash          & \featpref{Herbalist}                        \\
        \featref{Iron Will}          & Wil 2                    & Increase mental resilience          & \tdash             & \featpref{Iron Will}          \\
        \featref{Null}               & Wil 2                    & Become immune to magic              & \tdash             & \featpref{Null}               \\
        \featref{Precognition}       & Int 3                    & React to future events              & \tdash             & \featpref{Precognition}       \\
        \featref{Regenerator}        & Con 3                    & Heal wounds with inhuman speed      & \tdash             & \featpref{Regenerator}        \\
        \featref{Rapid Reaction}     & Dex 2                    & Increase reaction speed             & \tdash             & \featpref{Rapid Reaction}     \\
        \featref{Spellwarped}        & Wil 2                    & Gain limited spellcasting           & Magical            & \featpref{Spellwarped}        \\
        \featref{Swiftrunner}        & Dex 3                    & Move more quickly                   & \tdash             & \featpref{Swiftrunner}        \\
        \featref{Telepath}           & Int \add Wil is 3        & Communicate with creatures mentally & Magical            & \featpref{Telepath}          \\
        \featref{Toughness}          & Con 2                    & Increase physical fortitude         & \tdash             & \featpref{Toughness}          \\

        \tb{Skill Feats}\label{Skill Feats}        & \tb{Prerequisites}          & \tb{Benefits}                         & \tb{Feat Types} & \tb{Page}                                   \\
        \featref{Awareness Specialization}         & Trained Awareness          & Improve use of chosen skill           & \tdash          & \featpref{Awareness Specialization}         \\
        \featref{Balance Specialization}           & Trained Balance            & Improve use of chosen skill           & \tdash          & \featpref{Balance Specialization}           \\
        \featref{Climb Specialization}             & Trained Climb              & Improve use of chosen skill           & \tdash          & \featpref{Climb Specialization}             \\
        \featref{Craft Specialization}             & Trained Craft              & Improve use of chosen skill           & \tdash          & \featpref{Craft Specialization}             \\
        \featref{Creature Handling Specialization} & Trained Creature Handling  & Improve use of chosen skill           & \tdash          & \featpref{Creature Handling Specialization} \\
        \featref{Deception Specialization}         & Trained Deception          & Improve use of chosen skill           & \tdash          & \featpref{Deception Specialization}         \\
        \featref{Devices Specialization}           & Trained Devices            & Improve use of chosen skill           & \tdash          & \featpref{Devices Specialization}           \\
        \featref{Disguise Specialization}          & Trained Disguise           & Improve use of chosen skill           & \tdash          & \featpref{Disguise Specialization}          \\
        \featref{Endurance Specialization}         & Trained Endurance          & Improve use of chosen skill           & \tdash          & \featpref{Endurance Specialization}         \\
        \featref{Flexibility Specialization}       & Trained Flexibility        & Improve use of chosen skill           & \tdash          & \featpref{Flexibility Specialization}       \\
        \featref{Intimidate Specialization}        & Trained Intimidate         & Improve use of chosen skill           & \tdash          & \featpref{Intimidate Specialization}        \\
        \featref{Jump Specialization}              & Trained Jump               & Improve use of chosen skill           & \tdash          & \featpref{Jump Specialization}              \\
        \featref{Knowledge Specialization}         & Trained Knowledge          & Improve use of chosen skill           & \tdash          & \featpref{Knowledge Specialization}         \\
        \featref{Linguistics Specialization}       & Trained Linguistics        & Improve use of chosen skill           & \tdash          & \featpref{Linguistics Specialization}       \\
        \featref{Medicine Specialization}          & Trained Medicine           & Improve use of chosen skill           & \tdash          & \featpref{Medicine Specialization}          \\
        \featref{Perform Specialization}           & Trained Perform            & Improve use of chosen skill           & \tdash          & \featpref{Perform Specialization}           \\
        \featref{Persuasion Specialization}        & Trained Persuasion         & Improve use of chosen skill           & \tdash          & \featpref{Persuasion Specialization}        \\
        \featref{Ride Specialization}              & Trained Ride               & Improve use of chosen skill           & \tdash          & \featpref{Ride Specialization}              \\
        \featref{Sleight of Hand Specialization}   & Trained Sleight of Hand    & Improve use of chosen skill           & \tdash          & \featpref{Sleight of Hand Specialization}   \\
        \featref{Social Insight Specialization}    & Trained Social Insight     & Improve use of chosen skill           & \tdash          & \featpref{Social Insight Specialization}    \\
        \featref{Stealth Specialization}           & Trained Stealth            & Improve use of chosen skill           & \tdash          & \featpref{Stealth Specialization}           \\
        \featref{Survival Specialization}          & Trained Survival           & Improve use of chosen skill           & \tdash          & \featpref{Survival Specialization}          \\
        \featref{Swim Specialization}              & Trained Swim               & Improve use of chosen skill           & \tdash          & \featpref{Swim Specialization}              \\

        \tb{Casting Feats}\label{Casting Feats} & \tb{Prerequisites} & \tb{Benefits} & \tb{Feat Types} & \tb{Page} \\
        \featref{Boongiver}                      & Spellcasting ability                    & Improve ability to cast spells on allies  & Magical & \featpref{Boongiver}                      \\
        \featref{Blood Magic}                    & Spellcasting ability, Con 2             & Spend hit points to improve magic         & Magical & \featpref{Blood Magic}                    \\
        \featref{Mental Magic}                   & Spellcasting ability, Wil 3             & Cast spells without words or gestures     & Magical & \featpref{Mental Magic}                   \\
        \featref{Metacaster}                     & Spellcasting ability, Int 2             & Manipulate spell effects in creative ways & Magical & \featpref{Metacaster}                     \\
        \featref{Mystic Archer}                  & Spellcasting ability                    & Imbue projectiles with magic              & Magical & \featpref{Mystic Archer}                  \\
        \featref{Prepared Spellcasting}          & Spellcasting ability, Int 3             & Prepare additional spells each day        & Magical & \featpref{Prepared Spellcasting}                                          \\
        \featref{Spellsword}                     & Spellcasting ability                    & Fight with sword and spell together       & \tdash  & \featpref{Spellsword}                     \\
        \featref{Sphere Focus: Aeromancy}        & \sphere{Aeromancy} sphere access        & Improve casting with chosen sphere        & Magical & \featpref{Sphere Focus: Aeromancy}        \\
        \featref{Sphere Focus: Aquamancy}        & \sphere{Aquamancy} sphere access        & Improve casting with chosen sphere        & Magical & \featpref{Sphere Focus: Aquamancy}        \\
        \featref{Sphere Focus: Astromancy}       & \sphere{Astromancy} sphere access       & Improve casting with chosen sphere        & Magical & \featpref{Sphere Focus: Astromancy}       \\
        \featref{Sphere Focus: Channel Divinity} & \sphere{Channel Divinity} sphere access & Improve casting with chosen sphere        & Magical & \featpref{Sphere Focus: Channel Divinity} \\
        \featref{Sphere Focus: Chronomancy}      & \sphere{Chronomancy} sphere access      & Improve casting with chosen sphere        & Magical & \featpref{Sphere Focus: Chronomancy}      \\
        \featref{Sphere Focus: Cryomancy}        & \sphere{Cryomancy} sphere access        & Improve casting with chosen sphere        & Magical & \featpref{Sphere Focus: Cryomancy}        \\
        \featref{Sphere Focus: Electromancy}     & \sphere{Electromancy} sphere access     & Improve casting with chosen sphere        & Magical & \featpref{Sphere Focus: Electromancy}     \\
        \featref{Sphere Focus: Enchantment}      & \sphere{Enchantment} sphere access      & Improve casting with chosen sphere        & Magical & \featpref{Sphere Focus: Enchantment}      \\
        \featref{Sphere Focus: Fabrication}      & \sphere{Fabrication} sphere access      & Improve casting with chosen sphere        & Magical & \featpref{Sphere Focus: Fabrication}      \\
        \featref{Sphere Focus: Photomancy}       & \sphere{Photomancy} sphere access       & Improve casting with chosen sphere        & Magical & \featpref{Sphere Focus: Photomancy}       \\
        \featref{Sphere Focus: Polymorph}        & \sphere{Polymorph} sphere access        & Improve casting with chosen sphere        & Magical & \featpref{Sphere Focus: Polymorph}        \\
        \featref{Sphere Focus: Prayer}           & \sphere{Prayer} sphere access           & Improve casting with chosen sphere        & Magical & \featpref{Sphere Focus: Prayer}           \\
        \featref{Sphere Focus: Pyromancy}        & \sphere{Pyromancy} sphere access        & Improve casting with chosen sphere        & Magical & \featpref{Sphere Focus: Pyromancy}        \\
        \featref{Sphere Focus: Revelation}       & \sphere{Revelation} sphere access       & Improve casting with chosen sphere        & Magical & \featpref{Sphere Focus: Revelation}       \\
        \featref{Sphere Focus: Summoning}        & \sphere{Summoning} sphere access        & Improve casting with chosen sphere        & Magical & \featpref{Sphere Focus: Summoning}        \\
        \featref{Sphere Focus: Telekinesis}      & \sphere{Telekinesis} sphere access      & Improve casting with chosen sphere        & Magical & \featpref{Sphere Focus: Telekinesis}      \\
        \featref{Sphere Focus: Terramancy}       & \sphere{Terramancy} sphere access       & Improve casting with chosen sphere        & Magical & \featpref{Sphere Focus: Terramancy}       \\
        \featref{Sphere Focus: Thaumaturgy}      & \sphere{Thaumaturgy} sphere access      & Improve casting with chosen sphere        & Magical & \featpref{Sphere Focus: Thaumaturgy}      \\
        \featref{Sphere Focus: Toxicology}         & \sphere{Toxicology} sphere access         & Improve casting with chosen sphere        & Magical & \featpref{Sphere Focus: Toxicology}         \\
        \featref{Sphere Focus: Umbramancy}       & \sphere{Umbramancy} sphere access       & Improve casting with chosen sphere        & Magical & \featpref{Sphere Focus: Umbramancy}       \\
        \featref{Sphere Focus: Verdamancy}       & \sphere{Verdamancy} sphere access       & Improve casting with chosen sphere        & Magical & \featpref{Sphere Focus: Verdamancy}       \\
        \featref{Sphere Focus: Vivimancy}        & \sphere{Vivimancy} sphere access        & Improve casting with chosen sphere        & Magical & \featpref{Sphere Focus: Vivimancy}        \\
        \featref{Twinhand Spellcaster}           & Dex 2        & Cast spells with two hands at once        & Magical & \featpref{Twinhand Spellcaster}        \\
        \featref{Wardweaver}          & \abilitytag{Barrier} spell known          & Create more powerful barriers        & Magical & \featpref{Wardweaver}          \\

        \tb{Combat Feats}\label{Combat Feats} & \tb{Prerequisites} & \tb{Benefits} & \tb{Feat Types} & \tb{Page} \\
        \featref{Battle Armory}            & Str 1, Dex 2        & Switch between different weapons easily    & \tdash  & \featpref{Battle Armory}            \\
        \featref{Blindfighter}             & Per 2               & Fight unseen foes better                   & \tdash  & \featpref{Blindfighter}             \\
        \featref{Brawler}                  & Str 1, Dex 1        & Fight better unarmed and in close quarters & \tdash  & \featpref{Brawler}                  \\
        \featref{Combat Style Versatility} & Int 1, combat style & Use highly varied combat styles            & \tdash  & \featpref{Combat Style Versatility} \\
        \featref{Duelist}                  & Dex 2, Int 1        & Fight one-on-one better                    & \tdash  & \featpref{Duelist}                  \\
        \featref{Executioner}              & Per 2               & Kill weakened foes more easily             & \tdash  & \featpref{Executioner}              \\
        \featref{Greatweapon Warrior}      & Str 3               & Fight better with two-handed weapons       & \tdash  & \featpref{Greatweapon Warrior}      \\
        \featref{Ghostblade}               & Dex 1, Wil 2        & Tap into ghostly powers in combat          & Magical & \featpref{Ghostblade}               \\
        \featref{Juggernaut}               & Str 2, Con 1        & Shove and overrun foes to deal damage      & \tdash  & \featpref{Juggernaut}               \\
        \featref{Leadership}               & Int 2 or Wil 2      & Inspire nearby allies                      & \tdash  & \featpref{Leadership}               \\
        \featref{Maneuverist}              & Int 2               & Gain limited maneuver access               & \tdash  & \featpref{Maneuverist}              \\
        \featref{Martial Training}         & \tdash              & Improve combat abilities                   & \tdash  & \featpref{Martial Training}         \\
        \featref{Shieldbearer}             & Str 2               & Defend better with shields                 & \tdash  & \featpref{Shieldbearer}             \\
        \featref{Sniper}                   & Per 2               & Aim precisely at distant foes              & \tdash  & \featpref{Sniper}                   \\
        \featref{Two-Weapon Fighting}      & Dex 2               & Fight better with two weapons at once      & \tdash  & \featpref{Two-Weapon Fighting}      \\
        \featref{Weapon Focus}             & \tdash              & Fight better with a single type of weapon  & \tdash  & \featpref{Weapon Focus}             \\
        \featref{Whirlwind Warrior}        & Dex 2               & Fight hordes with agile ease               & \tdash  & \featpref{Whirlwind Warrior}        \\
    \end{longtable}
\end{longtablewrapper}

    \section{Feat Descriptions}
        Each feat has a set of benefits it provides to a character with the feat.
        Some feats also have specific requirements that a character must meet before taking the feat.
        These are listed under a \textbf{Prerequisites} heading.
        If a character loses the prerequisites for a feat, they lose all benefits of the feat until they meet the prerequisites again.

    \begin{feat}{Awareness Specialization}{Skill}
        \featpre Awareness as a trained skill.

        \ff[1]{Specialization} You gain a \plus3 bonus to the Awareness skill.

        \ff[3]{Extraordinary Senses} You gain one of the following senses: \trait{blindsense} (30 ft.), \trait{darkvision} (60 ft.), \trait{low-light vision}, \trait{scent}, or \trait{tremorsense} (30 ft.).
        As normal, if you already have the chosen sense from another source, you sum the ranges from both abilities to determine your total range with that sense.

        \ff[9]{Extraordinary Senses+} You gain one of the following senses: \trait{blindsense} (120 ft.), \trait{blindsight} (30 ft.), \trait{darkvision} (240 ft.), \trait{tremorsense} (120 ft.), or \trait{tremorsight} (30 ft.).
        You cannot choose the same sense that you chose with your \textit{extraordinary senses} ability.
        However, you can also change the sense you chose with that ability.

        \ff[15]{Specialization+} The bonus from your \textit{specialization} ability increases to \plus6.

        \ff[21]{Extraordinary Senses+} You can choose an additional sense from the list given in your \textit{greater extraordinary senses} ability, except that the range is doubled.
        You cannot choose the same sense that you chose with your \textit{extraordinary senses} or \textit{greater extraordinary senses} abilities.
        However, you can also change the sense you chose with those abilities.
    \end{feat}

    \begin{feat}{Balance Specialization}{Skill}
        \featpre Balance as a trained skill.

        \ff[1]{Specialization} You gain a \plus3 bonus to the Balance skill.

        \ff[3]{Instant Stand} You can use the \textit{rapid stand} ability as a \glossterm{free action} instead of a \glossterm{minor action} (see \pcref{Rapid Stand}).

        \ff[9]{Specialization+} The bonus from your \textit{specialization} ability increases to \plus6.

         % modifier at 10th: 12dex + 2mst + 2ft = 16
        \ff[15]{Air Dancer}[Magical] You can attempt to move on surfaces that cannot support your weight, as described below.
        \begin{itemize}
            \item Surfaces that can support at least a quarter of your weight, such as thin tree branches and dense liquids, are \glossterm{difficulty value} 20.
            \item Surfaces that can support at least a tenth of your weight, such as water, are \glossterm{difficulty value} 25.
            \item Surfaces that can support at least a hundredth of your weight, such as tree leaves, are \glossterm{difficulty value} 30.
            \item Surfaces that cannot support your weight at all, such as air, are \glossterm{difficulty value} 40.
        \end{itemize}

        Success means you move along the surface at half speed.
        Failure means you fall through the surface.
        The \glossterm{difficulty value} increases by 5 for each consecutive round that you spend moving in this way.

        \ff[21]{Balance On Air+} You can move at full speed while using your \textit{air dancer} ability.
        In addition, for each round that you spend using your \textit{air dancer} ability, the \glossterm{difficulty value} increases by 2 instead of by 5.
    \end{feat}

    \begin{feat}{Battle Armory}{Combat}
        \featpre Strength 1, Dexterity 2.

        \ff[1]{Overburdened Quickdraw} You can draw or sheathe any weapon or shield other than a tower shield as a \glossterm{free action}.

        \ff[3]{Legacy Armory}[Magical] You do not choose an individual item as a \glossterm{legacy item} (see \pcref{Legacy Items}).
        Instead, if you choose weapons as your legacy item category, you choose magic weapon abilities that apply to all nonmagical weapons you wield.
        If you wield a magical weapon, it keeps its own magical effects instead of your chosen legacy item properties.

        \ff[9]{Versatile Force} You gain a \plus3 bonus to your \glossterm{power}.

        \ff[15]{Versatile Force+} The bonus from your \textit{versatile force} ability increases to \plus6.

        \ff[21]{Surprising Quickdraw} When you make a \glossterm{strike} with a weapon against a creature, if you drew that weapon in the same phase that you made the strike and did not wield any weapons at the start of that phase, you can use this ability.
        If you do, the creature is \unaware of the attack.
        After you use this ability, the attacked creature and all creatures that observed the attack are immune to this ability until they take a \glossterm{short rest}.
    \end{feat}

    \begin{feat}{Blindfighter}{Combat}
        \featpre Perception 2.

        \ff[1]{Unerring} You ignore all \glossterm{miss chances}, such as from \glossterm{concealment}.
        This does not prevent your attacks from failing for other reasons.

        \ff[3]{Blindsight} You gain \trait{blindsense} with a 90 foot range, allowing you to sense your surroundings without light (see \pcref{Blindsense}).
        If you already have blindsense, the range of your blindsense increases by 90 feet.
        In addition, you gain \trait{blindsight} with a 30 foot range, allowing you to see without light (see \pcref{Blindsight}).
        If you already have blindsight, the range of your blindsight increases by 30 feet.

        \ff[9]{Unseeing Precision} You gain a \plus1 bonus to \glossterm{accuracy}.

        \ff[15]{Unseeing Precision+} The bonus from your \textit{unseeing precision} ability increases to \plus2.

        \ff[21]{Blindsight+} The range of your blindsense increases by 90 feet.
        In addition, the range of your blindsight increases by 150 feet.
    \end{feat}

    \begin{feat}{Blood Magic}{Casting, Magical}
        \featpre Access to a \glossterm{mystic sphere}, Constitution 2.

        \ff[1]{Bloodspell} Whenever you cast a spell, you may use this ability.
        When you do, you lose \glossterm{hit points} equal to your maximum spell rank.
        In exchange, you gain a power bonus with the spell equal to the hit points you lost this way, and the spell does not require \glossterm{casting components}.

        \ff[3]{Bloodbind}[\glossterm{Magical}] Whenever you make a living creature lose \glossterm{hit points} using a spell, you can choose to bind the target's blood to yours.
        While the target is bound, you can see it through all forms of \glossterm{concealment} and \glossterm{cover}, even if it is \trait{invisible}.
        This binding lasts until you bind another creature with this ability, or until you take a \glossterm{long rest}.

        \ff[9]{Blood-Sworn Allegiance} You can swear a blood oath to your allies.
        You automatically fail any attack rolls you make against a creature that you have sworn this blood oath with.
        In addition, those creatures automatically fail any attack rolls they make against you.
        This allows you to safely use area spells that include your allies as targets, since they will suffer no ill effect from the attack.

        \ff[15]{Bloodbind+} You are always considered to have \glossterm{line of effect} with \glossterm{magical} abilities to a creature bound by your \textit{bloodbind} ability, regardless of intervening obstacles.
        This allows you to cast spells on them through solid walls.
        In addition, you can bind any number of creatures at once with your \textit{bloodbind} ability.

        \ff[21]{Bloodspell+} You can choose to lose 10 hit points when you use your \textit{bloodspell} ability.
        If you do, the power bonus increases to \plus10.
    \end{feat}

    \begin{feat}{Boongiver}{Casting, Magical}
        \featpre Access to a \glossterm{mystic sphere}.

        \ff[1]{Share Boon} When you cast a spell with the \abilitytag{Attune} tag, you can use the \textit{share boon} ability.
        \begin{activeability}{Share Boon}
            \rankline
            The spell's \abilitytag{Attune} tag changes to \abilitytag{Attune} (target).
            Choose one \glossterm{ally} within \rngmed range.
            That ally is the target of the spell, and the spell affects that creature as if it were you instead of affecting you.

            You can only use this ability to affect one spell at a time.
            If you use it again, the original ally's attunement to the old spell is released, as the \textit{release attunement} ability (see \pcref{Attuned Abilities}).
        \end{activeability}

        \ff[3]{Sustain Attunement} Whenever you cast an \abilitytag{Attune} (target) spell that is not a \glossterm{deep attunement}, you can choose to replace its \abilitytag{Attune} tag with the \abilitytag{Sustain (minor)} tag.

        \ff[9]{Versatile Boon Lore} You learn additional \glossterm{spell}.
        The spell must have the \abilitytag{Attune} tag, but you do not need to have access to the \glossterm{mystic sphere} it is from.
        The mystic sphere it is from must still be part of your \glossterm{magic source}.
        As normal, you can exchange this spell for other spells as you gain access to new spell ranks, but the spell must always have the \abilitytag{Attune} tag.

        \ff[15]{Personal Boon} You gain an additional \glossterm{attunement point}.
        You can only use this attunement point to \glossterm{attune} to spells and rituals that you use on yourself.

        \ff[21]{Share Boon+} You can use your \textit{share boon} ability on up to three different spells at once.
        If you use the ability while it already affects another spell, you choose which spells are affected by the ability.
    \end{feat}

    \begin{feat}{Brawler}{Combat}
        \featpre Strength 1, Dexterity 1.

        \ff[1]{Unarmed Warrior} You gain a \plus2 bonus to accuracy and a \plus1d bonus to damage with the punch/kick \glossterm{natural weapon} (see \pcref{Natural Weapons}).
        This ability does not stack with the ability of the same name from the Perfected Form monk archetype (see \pcref{Perfected Form}).

        \ff[1]{Grapple Expertise} You gain a \plus1 bonus to \glossterm{accuracy} with the \textit{grapple} ability (see \pcref{Grapple}), as well as with all grapple actions (see \pcref{Grapple Actions}).

        \ff[3]{Takedown} Whenever you hit a target with the \textit{grapple} ability, the target also takes damage as if you had hit with your unarmed attack or a light \glossterm{natural weapon} you wield.

        \ff[9]{Large Grappler} You reduce your size-based penalties for being smaller than your target with the \textit{grapple} ability by 4.

        \ff[15]{Unarmed Warrior+} The damage bonus from your \textit{unarmed warrior} ability increases to \plus3d.

        \ff[15]{Grapple Expertise+} The bonus from your \textit{grapple expertise} ability increases to \plus3.

        \ff[21]{Grapple Supremacy} When you grapple a target with the \textit{grapple} ability, you do not become \grappled by that target.
    \end{feat}

    \begin{feat}{Celestial Heritage}{Bloodline, Magical}
        \featpre Non-evil alignment.
        \parhead{Special} You can only have one Bloodline feat.

        \ff[1]{Rebuke Evil} You can use the \textit{rebuke evil} ability as a standard action.
        \begin{activeability}{Rebuke Evil}
            \rankline
            Make an attack vs. Mental against one creature or object within \rngmed range.
            You gain a \plus2 accuracy bonus to this attack against any creature that damaged any of your \glossterm{allies} during the previous round.
            If the target has the alignment opposed to your devoted alignment, you double all of your damage bonuses along with your damage dice when you get a \glossterm{critical hit}.
            \hit The target takes 1d6 \add \glossterm{power} energy damage.

            \rankline
            You gain a \plus1 accuracy bonus and a \plus1d damage bonus at 3rd level and every 3 levels thereafter.
        \end{activeability}

        \ff[3]{Healing Light}[Magical] When you use the \textit{recover} ability, you \glossterm{briefly} emit \glossterm{brilliant illumination} in a \medarea radius \glossterm{zone} from you, and \glossterm{shadowy illumination} in twice that radius.
        At the end of each round, each \glossterm{ally} in the radius of brilliant illumination regains \glossterm{hit points} equal to your \glossterm{power}.

        \ff[9]{Angel Wings} You gain feathery wings that sprout from your back.
        These wings grant you a \glossterm{fly speed} equal to the \glossterm{base speed} for your size with a maximum height of 30 feet (see \pcref{Flight}).
        As a \glossterm{free action}, you can increase your \glossterm{fatigue level} by one to ignore this height limit until the end of the round.
        The wings themselves are \glossterm{mundane}, but the ability to fly with them is \glossterm{magical}.

        \ff[15]{Healing Light+} The healing provided by your \textit{healing light} ability increases to twice your power.
        In addition, the radius of brilliant illumination increases to a \hugearea radius.

        \ff[21]{Angel Wings+} Your \glossterm{maneuverability} with the fly speed from your \textit{angelic flight} ability increases to perfect (see \pcref{Flying Maneuverability}).
    \end{feat}

    \begin{feat}{Chameleon}{General}
        \featpre Disguise as a trained skill, Intelligence 2.

        \ff[1]{Adaptive Archetype} Choose one archetype that you currently have, and two archetypes you do not have from among any of your classes.
        You cannot choose an archetype that you have which is a prerequisite for another archetype that you have.
        Whenever you finish a \glossterm{long rest}, you can choose which one of those three archetypes you actually have access to.
        You gain all benefits of your chosen archetype, and temporarily lose all benefits from the archetypes you did not choose in this way.

        You must track which choices you made for archetypes that you lose access to in this way, such as which spells and maneuvers you learned.
        When you regain access to that archetype, you must make the same choices.

        \ff[3]{Adaptive Specialty} Whenever you finish a \glossterm{short rest}, you may choose an effect from the list below.
        Each effect lasts until you take a short rest.
        \begin{itemize}
            \item Martial: You become proficient with an additional \glossterm{weapon group} and an additional armor \glossterm{usage class} of your choice.
                You must be proficient with light armor to become proficient with medium armor, and you must be proficient with medium armor to become proficient with heavy armor.
            \item Mystic: You gain a \plus2 bonus to your Mental defense.
            \item Primal: You gain a \plus2 bonus to your \glossterm{fatigue tolerance}.
            \item Skilled: You gain a \plus1 bonus to all skills.
        \end{itemize}

        \ff[9]{Instant Adaptation} As a \glossterm{minor action}, you can change your choice of \textit{adaptive archetype} and \textit{adaptive specialty}.
        When you do, you increase your \glossterm{fatigue level} by two.

        \ff[15]{Adaptive Specialty+} The effects of your \textit{adaptive specialty} ability improve, as described below.
        \begin{itemize}
            \item Martial: You gain a \plus1 bonus to your Armor defense.
            \item Mystic: You gain an additional \glossterm{attunement point}.
                You can only use this attunement point to attune to spells that you cast on yourself.
            \item Primal: The fatigue tolerance bonus increases to \plus4.
            \item Skilled: The skill bonus increases to \plus2.
        \end{itemize}

        \ff[21]{Adaptive Archetype+} Instead of choosing a single archetype to activate with your \textit{adaptive archetype}, you may choose a blend of two archetypes simultaneously.
        First, choose two archetypes to combine.
        For each rank you have access to, you choose one archetype and gain all abilities of that rank from that archetype and no abilities of that rank from the other archetype.
        You cannot choose abilities from an archetype that reference abilities from that same archetype which you do not have.
        For example, you cannot choose the \textit{supreme wildspell} ability unless you also have the \textit{wildspell} ability.
    \end{feat}

    \begin{feat}{Climb Specialization}{Skill}
        \featpre Climb as a trained skill.

        \ff[1]{Specialization} You gain a \plus3 bonus to the Climb skill.

        \ff[3]{Climb Speed} You gain a \glossterm{climb speed} 10 feet slower than the \glossterm{base speed} for your size.
        If you already have a climb speed, you gain a \plus10 foot bonus to your climb speed.
        A successful Climb check to move allows you to travel a distance equal to your climb speed.

        \ff[9]{Creature Climber} You can use the \textit{creature climb} ability against creatures one or more size categories larger than you instead of two (see \pcref{Creature Climb}).
        This does not cause you or the creature to suffer penalties for \squeezing.

        \ff[15]{Specialization+} The bonus from your \textit{specialization} ability increases to \plus6.

        \ff[21]{Freehand Climb} You only need one \glossterm{free hand} to move while climbing.

        \ff[21]{Rapid Climber} You gain a \plus10 foot bonus to your climb speed.
    \end{feat}

    \begin{feat}{Combat Style Versatility}{Combat}
        \featpre Intelligence 1, access to at least one \glossterm{combat style}.

        \ff[1]{Combat Styles} You gain access to two additional \glossterm{combat styles}.

        \ff[3]{Maneuver} You learn two additional \glossterm{maneuvers}.
        When you gain access to new maneuver ranks, you can change which maneuver you know.

        \ff[9]{Precise Maneuvers} You gain a \plus1 bonus to \glossterm{accuracy}.

        \ff[15]{Precise Maneuvers+} The bonus from your \textit{precise maneuvers} ability increases to \plus2.

        \ff[21]{Maneuvers} You learn two additional \glossterm{maneuvers}.
    \end{feat}

    \begin{feat}{Craft Specialization}{Skill}
        \featpre Any Craft skill as a trained skill.

        \ff[1]{Specialization} You gain a \plus3 bonus to all Craft skills.

        \ff[3]{Craft Magic Item}[Magical] You can imbue items with magic using your crafting skill.
        There are two ways to craft magic items: by sacrificing valuable raw materials or by salvaging other magic items.
        If you sacrifice valuable raw materials, you must destroy trade goods or gold pieces as if you were buying an item one rank lower than the item you are crafting  (see \pcref{Item Ranks}).
        If you salvage another magic item, you must either destroy a non-consumable magic item that is at least one rank higher than the item you are crafting, or destroy a non-consumable magic item with the exact same effect as the item you are crafting.
        As normal, you can treat five items of one rank as being equivalent to a single item of one rank higher for either of these crafting methods.

        Crafting a magic item in this way normally requires 24 hours of continuous work which may be split between any number of crafting sessions.
        You can make weaker items more quickly.
        The time required to craft magic items is halved for every rank by which your highest rank exceeds the item's rank, to a minimum of 15 minutes.

        You can also mend a broken magic item if it is one that you could make, or transfer a magical property from one item to a nonmagical item.
        If you transfer an item property in this way, the magic item ability must be valid for the new item.
        If you do so, you treat the item as if it were two ranks lower than its actual rank for the purpose of determining the cost and crafting time, to a minimum rank of 0.
        You cannot mend a \glossterm{destroyed} magic item.

        \ff[9]{Knowledge Savant} You gain two additional \glossterm{trained skills} which must be Craft skills.

        \ff[15]{Specialization+} The bonus from your \textit{specialization} ability increases to \plus6.

        \ff[21]{Item Attunement} You gain an additional \glossterm{attunement point}.
        You can only use this \glossterm{attunement point} to \glossterm{attune} to the effects of items (see \pcref{Item Attunement}).
    \end{feat}

    \begin{feat}{Creature Handling Specialization}{Skill}
        \featpre Creature Handling as a trained skill.

        \ff[1]{Specialization} You gain a \plus3 bonus to the Creature Handling skill.

        \ff[3]{Binding Command} You can \glossterm{sustain} the effect of the \textit{command} ability from the Creature Handling skill with a \glossterm{minor action} instead of with a standard action.
        For details, see \pcref{Command}.

        \ff[3]{Efficient Training} You can teach a creature a trick with 12 hours of work, split as you choose, rather than in a week of 4 hour sessions (see \pcref{Training Creatures}).
        In addition, you can train creatures to learn two bonus tricks beyond their normal maximum (see \pcref{Bonus Tricks}).

        \ff[9]{Battleforged Companion} You can teach a creature the Battleforged Companion trick.
        This does not work on creatures that are already significantly enhanced or altered from their natural state, such as a druid's \textit{natural servant} or a ranger's \textit{animal companion}.
        The \glossterm{difficulty value} to train the trick is 15.
        A creature with the trick uses your level in place of its level to determine its \glossterm{hit points}, \glossterm{damage resistance}, defenses, and \glossterm{accuracy}.
        This includes any special bonuses that monsters normally gain to those values, such as the \plus1 bonus to all defenses that monsters gain at level 3.
        However, it does not include any scaling bonuses that monsters gain based on their level to other statistics, such as power.

        You can only train one creature with this trick at a time.
        If you train another creature, the previous creature loses the benefit of this training.
        In addition, you must spend at least ten minutes per day training with the creature to maintain the trick.
        Otherwise, it forgets the trick and loses its benefits.

        \ff[15]{Specialization+} The bonus from your \textit{specialization} ability increases to \plus6.

        \ff[21]{Battleforged Companion+} The creature with your Battleforged Companion trick gains the following additional benefits:
        \begin{itemize}
            \item It gains a \plus2 bonus to all defenses.
            \item It gains a \plus2 accuracy bonus with all attacks.
            \item It gains a \plus2d damage bonus with \glossterm{strikes}.
        \end{itemize}
    \end{feat}

    \begin{feat}{Deception Specialization}{Skill}
        \featpre Deception as a trained skill.

        \ff[1]{Specialization} You gain a \plus3 bonus to the Deception skill.

        \ff[3]{Dual Speech}[Magical] When you speak, you can use the \textit{dual speech} ability.
        \begin{sustainability}{Dual Speech}{\abilitytag{Sustain} (minor)}
            \rankline
            You speak the same words with two different vocal patterns, such as tone or accent, though you cannot significantly change the volume.
            You can freely choose which creatures hear which pattern, treating all creatures you are not aware of as a single group. 

            You can freely choose different vocal patterns each round that you sustain this ability.
            \rankline
            \featlevel{9} You can speak entirely different words with your two voices.
            \featlevel{15} You can also speak with a third voice, using separate words and vocal patterns.
            \featlevel{21} You can also speak with a fourth voice, using separate words and vocal patterns.
        \end{sustainability}

        \ff[9]{Undetectable Lies} Any \glossterm{magical} abilities which detect lies are unable to detect lies you speak.

        \ff[15]{Specialization+} The bonus from your \textit{specialization} ability increases to \plus6.

        \ff[21]{Deceive Reality}[\glossterm{Magical}] Once per \glossterm{long rest}, you can use the \ability{deceive reality} ability as a standard action.
        \begin{activeability}{Deceive Reality}
            \rankline
            You tell a lie that becomes true.
            The lie must not directly affect sentient creatures.
            The effects of the lie must be \glossterm{mundane}, so you cannot create artifacts or specific spell effects.
            You cannot affect anything beyond one mile from you, ignoring \glossterm{line of sight} and \glossterm{line of effect}.
            In addition, all effects of this ability end after ten minutes.
            However, within those limits, you can generally alter reality around you.

            Some examples of lies that would be valid to use with this ability are ``I have a bathtub full of diamonds in the other room'', ``Suddenly, all of the torches in the room were extinguished simultaneously'', and ``Don't worry, my basement cellar has always had an escape tunnel for emergencies''.
        \end{activeability}
    \end{feat}

    \begin{feat}{Devices Specialization}{Skill}
        \featpre Devices as a trained skill.

        \ff[1]{Specialization} You gain a \plus3 bonus to the Devices skill.

        % TODO: wording
        \ff[3]{Disable Arcana}[Magical] You can affect spell effects on objects or areas with the Devices skill as if they were merely complex devices.
        You must be aware of an effect to use the Devices skill to affect it.
        You cannot affect effects on creatures.
        The \glossterm{difficulty value} to affect a spell effect is equal to 15 \add twice the effect's \glossterm{rank}.

        \ff[9]{Rapid Improvisation} It takes you only a standard action to make a device of up to Tiny size with the \textit{improvise} ability (see \pcref{Improvise}).

        \ff[15]{Specialization+} The bonus from your \textit{specialization} ability increases to \plus6.

        \ff[21]{Disable Arcana+} You can also affect spell effects on yourself as if they were merely complex devices.
        This can allow you to remove \glossterm{conditions} and other detrimental effects on you that were caused by spells.
    \end{feat}

    \begin{feat}{Disguise Specialization}{Skill}
        \featpre Disguise as a trained skill.

        \ff[1]{Specialization} You gain a \plus3 bonus to the Disguise skill.

        \ff[3]{Quick Change} You reduce the penalties for reducing the creation time of disguises with the \textit{disguise creature} and \textit{emulate creature} abilities by 5.

        \ff[9]{Versatile Disguise} Whenever you use the \textit{disguise creature} and \textit{emulate creature} abilities on yourself, you may simultaneously create two different disguises.
        This takes twice as long as creating a single disguise.
        You can change your appearance between the two chosen disguises as a \glossterm{minor action}.

        \ff[15]{Specialization+} The bonus from your \textit{specialization} ability increases to \plus6.

        \ff[21]{Quick Change+} You can create disguises with the \textit{disguise creature}, \textit{emulate creature}, and \textit{versatile disguise} abilities as a single standard action, regardless of the complexity of the disguise.
    \end{feat}

    \begin{feat}{Draconic Heritage}{Bloodline}
        \parhead{Special} You can only have one Bloodline feat.

        \ff[1]{Draconic Ancestry} Choose a type of dragon from among the dragons on \trefnp{Dragon Types}.
        You have the blood of that type of dragon in your veins.
        You are \glossterm{impervious} to that dragon's damage type.

        \ff[1]{Draconic Scales} You gain a \plus1 bonus to your Armor defense.

        \ff[1]{Draconic Weapons} You gain a bite natural weapon and two claw natural weapons, one on each arm.
        For details, see \pcref{Natural Weapons}.

        % Uses rank 1 area, not rank 2. Also uses weird line vs cone scaling.
        \ff[3]{Breath Weapon} You can use the \textit{breath weapon} ability as a \glossterm{standard action}.
        \begin{activeability}{Breath Weapon}
            \rankline
            Make an attack vs. Reflex against everything in the area defined by the type of dragon from your \textit{draconic ancestry} ability (see \trefnp{Dragon Types}).
            After you use this ability, you \glossterm{briefly} cannot use it again.
            \hit Each target takes damage equal to 1d10 plus half your \glossterm{power}.
            The damage type is defined by your \textit{draconic ancestry} ability.

            \rankline
            \featlevel{6} The damage increases to 2d6.
                In addition, the area affected by your breath weapon increases.
                A line breath weapon becomes a \arealarge, 5 ft.\ wide line.
                A cone breath weapon becomes a \areamed cone.
            \featlevel{9} The damage increases to 2d8.
            \featlevel{12} The damage increases to 2d10.
                In addition, the area affected by your breath weapon increases.
                A line breath weapon becomes a \areahuge, 10 ft.\ wide line.
                A cone breath weapon becomes a \arealarge cone.
            \featlevel{15} The damage increases to 4d6.
            \featlevel{18} The damage increases to 4d8.
                In addition, the area affected by your breath weapon increases.
                A line breath weapon becomes a \areagarg, 15 ft.\ wide line.
                A cone breath weapon becomes a \areahuge cone.
            \featlevel{21} The damage increases to 4d10.
        \end{activeability}

        \ff[9]{Draconic Wings} You gain leathery wings that sprout from your back.
        These wings grant you a \glossterm{fly speed} equal to 10 feet faster than the \glossterm{base speed} for your size (see \pcref{Flight}).
        Your maximum height is 30 feet, and your \glossterm{maneuverability} with this fly speed is poor (see \pcref{Flying Maneuverability}).
        As a \glossterm{free action}, you can increase your \glossterm{fatigue level} by one to ignore this height limit until the end of the round.
        The wings themselves are \glossterm{mundane}, but the ability to fly and glide with them is \glossterm{magical}.

        \ff[15]{Draconic Ancestry+} You become immune to damage of the type dealt by your dragon's breath weapon.

        \ff[15]{Draconic Scales+} The Armor defense bonus from your \textit{draconic scales} ability increases to \plus2.

        \ff[21]{Draconic Wings+} The height limit from your \textit{draconic wings} ability increases to 60 feet.
        In addition, you gain a \plus10 foot bonus to the fly speed.
    \end{feat}

    \begin{dtable}
        \lcaption{Dragon Types}
        \begin{dtabularx}{\columnwidth}{l >{\lcol}X >{\lcol}X}
            \tb{Dragon} & \tb{Damage Type} & \tb{Breath Weapon} \tableheaderrule
            Black       & Acid             & \areamed, 5 ft. wide line \\
            Blue        & Electricity      & \areamed, 5 ft. wide line \\
            Brass       & Fire             & \areamed, 5 ft. wide line \\
            Bronze      & Electricity      & \areamed, 5 ft. wide line \\
            Copper      & Acid             & \areamed, 5 ft. wide line \\
            Gold        & Fire             & \areasmall cone           \\
            Green       & Acid             & \areasmall cone           \\
            Red         & Fire             & \areasmall cone           \\
            Silver      & Cold             & \areasmall cone           \\
            White       & Cold             & \areasmall cone           \\
        \end{dtabularx}
    \end{dtable}

    \begin{feat}{Duelist}{Combat}
        \featpre Dexterity 2, Intelligence 1.

        \ff[1]{Duelist Strike} You can use the \textit{duelist strike} ability as a standard action.
        \begin{activeability}{Duelist Strike}
            \rankline
            Make a melee \glossterm{strike} with a light or medium weapon.
            This strikes only targets a single creature, even if your weapon would normally have the Sweeping tag.
            If you are the creature's only \glossterm{enemy} adjacent to it, you gain a \plus1 accuracy bonus with the strike.
            If that creature is not adjacent to any of its \glossterm{allies}, you gain an additional \plus1 accuracy bonus.

            \rankline
            \featlevel{6} You gain a \plus1 accuracy bonus with the strike.
            \featlevel{12} The automatic accuracy bonus increases to \plus2.
            \featlevel{18} The automatic accuracy bonus increases to \plus3.
        \end{activeability}

        \ff[3]{Defensive Stance} You gain a \plus1 bonus to your Armor defense as long as you wield a non-projectile weapon.

        \ff[9]{Duel Focus} At the start of each round, you may choose a creature you can see.
        During that round, you gain a \plus1 bonus to Armor and Reflex defenses against that creature.

        \ff[15]{Defensive Stance+} The bonus from your \textit{defensive stance} ability increases to \plus2.

        \ff[21]{Duel Focus+} The bonuses from your \textit{duel focus} ability increase to \plus2.
        In addition, you also gain a \plus2 bonus to Mental defense against that creature.

        \ff[21]{Duel Serenity} Your \textit{duelist strike} ability always has its full possible accuracy bonus, regardless of how many allies or enemies are near the target.
    \end{feat}

    \begin{feat}{Endurance Specialization}{Skill}
        \featpre Endurance as a trained skill.

        \ff[1]{Specialization} You gain a \plus3 bonus to the Endurance skill.

        \ff[3]{Delay Condition} Whenever you gain a \glossterm{condition}, you can make an Endurance check.
        The \glossterm{difficulty value} starts at 10 and increases by 5 in each subsequent round.
        Success means that you do not suffer the effects of the condition.
        You must repeat this check at the end of each subsequent round to continue to delay the effects of the condition.
        Failure means that the condition has its normal effect on you.

        You can only delay one of your conditions in this way.
        If you gain a new condition, you can choose to either delay the new condition or continue delaying the old condition.

        \ff[9]{Sleepless Endurance} You need half the normal amount of rest and sleep each day to function normally.
        For example, a human would only need four hours of sleep per night.
        This does not reduce the time required for you to take a \glossterm{long rest}.

        \ff[15]{Specialization+} The bonus from your \textit{specialization} ability increases to \plus6.

        \ff[21]{Ignore Vital Wound} When you use the \textit{delay vital wound} ability, the \glossterm{difficulty value} does not increase for each subsequent round, allowing you to delay the wound indefinitely.
    \end{feat}

    \begin{feat}{Entropist}{General, Magical}
        \featpre Willpower 2.

        \ff[1]{Entropic Defense} Whenever you are hit by a \glossterm{critical hit}, there is a 20\% chance that the attack is treated as a regular hit against you instead of a critical hit.

        \ff[3]{Sudden Entropy} You can use the \textit{sudden entropy} ability as a standard action.
        \begin{activeability}{Sudden Entropy}
            \rankline
            Make an attack vs. Mental against one creature or object within \medrange.
            \hit The target takes 1d10 \add \glossterm{power} damage.
            The damage is of a random damage type from among the following options: physical damage, energy damage, or all damage types simultaneously.

            \rankline
            You gain a \plus1 accuracy bonus and a \plus1d damage bonus at 6th level and every 3 levels thereafter.
        \end{activeability}

        \ff[9]{Friend of Chaos} Whenever you roll for a random effect, such as a \glossterm{miss chance} or a sorcerer's \textit{wild magic} ability, you may roll twice and keep whichever result you prefer.
        This does not affect your own \textit{entropic defense} ability, since the attacker rolls that chance instead of you.

        \ff[15]{Improbable Vulnerability} Whenever you make an attack against an \glossterm{enemy} that is \trait{immune} or \trait{impervious} to some aspect of the attack, you have a 50\% chance to affect them as if they were not immune or impervious to the attack.
        This has no effect on objects.

        \ff[21]{Entropic Defense+} The chance from your \textit{entropic defense} ability increases to 50\%.

        \ff[21]{Master of Chaos} Whenever you roll for a random effect, such as a miss chance or a sorcerer's \textit{wild magic} ability, you may use this ability.
        When you do, you increase your \glossterm{fatigue level} by two, and you may freely choose the random result.
    \end{feat}

    \begin{feat}{Executioner}{Combat}
        \featpres Perception 2.

        \ff[1]{Marked for Execution} You consider living creatures that either have a \glossterm{vital wound}, have less than their maximum \glossterm{hit points}, or have no remaining \glossterm{damage resistance} to be \textit{marked for execution}.
        Several abilities from this feat affect creatures \textit{marked for execution}.

        % This ability is melee-only because it makes some of the questions around ``is this creature marked''
        % less narratively awkward since you can automatically detect them with blood sense.
        \ff[1]{Execution} You can use the \textit{execution} ability as a standard action.
        \begin{activeability}{Execution}
            \rankline
            Make a melee \glossterm{strike}.
            If the target is \textit{marked for execution}, you gain two benefits.
            First, you gain a \plus4 damage bonus with the strike.
            Second, if you get a \glossterm{critical hit}, you double all of your damage bonuses along with your damage dice.

            \rankline
            \featlevel{6} The damage bonus increases to \plus8.
            \featlevel{12} The damage bonus increases to \plus16.
            \featlevel{18} The damage bonus increases to \plus24.
        \end{activeability}

        \ff[3]{Blood Sense}[Magical] You gain \trait{lifesense} with a 60 foot range, allowing you to sense the location of living creatures without light (see \pcref{Lifesense}).
        Your lifesense functions like \trait{lifesight} against creatures who are \textit{marked for execution}, allowing you to see them perfectly.
        You can also automatically identify which creatures within the range of your lifesense are \textit{marked for execution}, even if you can already see them normally.

        \ff[9]{Purge the Weak} You gain a \plus1 bonus to \glossterm{accuracy} against creatures that are \textit{marked for execution}.
        In addition, your attack rolls against creatures that are \textit{marked for execution} \glossterm{explode} on a 9 in addition to the normal explosion on a 10.
        This does not affect additional rolls with exploding dice.

        \ff[15]{Blood Sense+}[Magical] The range of the lifesense from your \textit{blood sense} ability increases by 120 feet.
        In addition, you no longer need \glossterm{line of effect} to see creatures that are \textit{marked for execution}.
        This allows you to see them through solid objects, but grants you no special ability to attack through solid objects, so your attacks are still affected normally by \glossterm{cover}.

        \ff[21]{Greater Purge the Weak} The bonus from your \textit{purge the weak} ability increases to \plus2.
        In addition, the first die you roll for each attack roll against a creature that is \textit{marked for execution} \glossterm{explodes} on an 8 or 9 in addition to the normal explosion on a 10.
    \end{feat}

    \begin{feat}{Flexibility Specialization}{Skill}
        \featpre Flexibility as a trained skill.

        \ff[1]{Specialization} You gain a \plus3 bonus to the Flexibility skill.

        \ff[3]{Rapid Escape} You can squeeze and escape bindings and grapples as a \glossterm{movement}, rather than as a standard action.

        \ff[9]{Constraint Tolerance} You reduce your penalties for \squeezing by 2 (see \pcref{Squeezing}).
        In addition, you movement speed is not reduced while \squeezing.

        \ff[15]{Specialization+} The bonus from your \textit{specialization} ability increases to \plus6.

        \ff[21]{Escape Magic}[Magical] You can use the \textit{escape magic} ability as a standard action.
        \begin{activeability}{Escape Magic}
            \rankline
            You make an Flexibility attack against all \glossterm{magical} effects on you.
            You may exclude any number of effects you are aware of from this attack, allowing you to maintain beneficial magical effects.
            The \glossterm{difficulty value} for each effect is equal to 10 \add twice the effect's \glossterm{rank}.
            \hit Each effect is \glossterm{dismissed}, if it is an effect that can be dismissed.
            Unless otherwise noted, all \glossterm{magical} abilities with a duration can be dismissed, including \glossterm{conditions} (see \pcref{Dismissal}).

            You can only dismiss effects with this ability which target you directly, not area effects which include you as a target.
            If an ability targets multiple creatures, you can only remove its effects on you.
        \end{activeability}
    \end{feat}

    \begin{feat}{Ghostblade}{Combat, Magical}
        \featpre Dexterity 1, Willpower 2.

        \ff[1]{Ghost Step} When you use the \textit{sprint} ability, you can become \trait{invisible} for the duration of the movement (see \pcref{Invisible}, and \pcref{Sprint}).
        This usually makes it impossible for creatures to react to your movement, such as by using the \textit{follow} or \textit{withdraw} abilities (see \pcref{Movement Abilities}).
        This ability has the \abilitytag{Swift} tag, so it affects attacks against you during the current phase.

        \ff[3]{Spectral Armament} The equipment you choose as your \glossterm{legacy item} becomes ghostly and translucent (see \pcref{Legacy Items}).
        If you chose a weapon, all damage dealt with it is cold damage in addition to its other types.
        If you chose body armor or a shield, you are \trait{impervious} to cold damage.

        % as rank 3 maneuver, but scales to rank ?? maneuver
        \ff[3]{Spectral Strike} You can use the \textit{spectral strike} ability as a standard action.
        \begin{activeability}{Spectral Strike}[\abilitytag{Magical}]
            \rankline
            Make a melee \glossterm{strike}.
            The attack is made against each target's Reflex defense instead of its Armor defense.
            You may use the higher of your Strength and your Willpower to determine your damage with this ability (see \pcref{Dice Bonuses From Attributes}).
            Each creature that loses \glossterm{hit points} from this strike is \glossterm{briefly} \slowed.
            After it stops being slowed, it becomes immune to being slowed in this way until it takes a short rest.

            \rankline
            \featlevel{9} Each target is no longer immune to being slowed after the first successful slow effect.
            \featlevel{15} The slowing effect becomes a \glossterm{condition} instead of a brief effect.
            \featlevel{21} The slowing effect works against each creature that took damage from the strike, not just against creatures that lost hit points.
        \end{activeability}

        \ff[9]{Ghost Step+} When you use your \textit{ghost step} ability, you can also become \trait{incorporeal} for the duration of the movement.
        This grants you the defensive benefits of being incorporeal during the current phase.

        \ff[15]{Ghost Step+} You can use your \textit{ghost step} ability to affect any movement you make during the \glossterm{movement phase}, even if you do not use the \textit{sprint} ability.

        \ff[21]{Spectral Armament+} The effect of your \glossterm{legacy item} improves.
        If you chose a weapon, whenever you make a melee \glossterm{strike}, you can make the strike against each target's Reflex defense in place of its Armor defense.
        This has no effect on strikes that are not made against Armor defense.
        If you chose body armor or a shield, you may use your Armor defense in place of your Reflex defense against all attacks.
    \end{feat}

    \begin{feat}{Greatweapon Warrior}{Combat}
        \featpre Strength 3.

        \ff[1]{Power Attack} Whenever you make a non-\glossterm{projectile} strike with a weapon you wield in two hands, you may take a \minus1 penalty to \glossterm{accuracy}.
        If you do, you gain a \plus1d damage bonus.

        \ff[3]{Cleave} Whenever you make a \glossterm{melee} \glossterm{strike} with a weapon you hold in two hands, it gains the Sweeping (1) tag (see \pcref{Sweeping}).
        If the weapon already has the Sweeping tag, you increase the number of secondary targets by 1.
        In addition, you can choose secondary targets within 15 feet of the primary target instead of the normal 10 feet.
        Each secondary target must still be adjacent to you unless you are using a \weapontag{Long} weapon (see \pcref{Weapon Tags}).

        \ff[9]{Power Attack+} The damage bonus from your \textit{power attack} ability increases to \plus2d.
        In addition, \textit{power attacks} deal double damage to objects.

        \ff[15]{Cleave+} The tag granted by your \textit{cleave} ability changes to be Sweeping (2).
        If the weapon already has the Sweeping tag, you instead increase the number of secondary targets by 2.

        \ff[21]{Power Attack+} The damage bonus from your \textit{power attack} ability increases to \plus3d.
        In addition, \textit{power attacks} deal triple damage to objects instead of double damage.
    \end{feat}

    \begin{feat}{Herbalist}{General}
        \featpre Knowledge (nature) as a trained skill.

        \ff[1]{Esoteric Concoction} You can use your Knowledge (nature) skill in place of Craft (alchemy) or Craft (poison) to create poisons and potions.
        This does not help you create other alchemical items, such as alchemist's fire.
        When you do, you must use esoteric natural ingredients in place of the normal ingredients.
        However, all poisons you create with this ability deteriorate and lose their effectiveness after 8 hours.

        The replacement ingredients must be difficult to acquire in large quantities and impossible to acquire in a normal city.
        For example, you can use the tail of a blind mouse or the dew from a four-leafed clover, but you could not use dirt or ordinary tree bark.
        Once you have determined a purpose for a particular replacement ingredient, you cannot use that ingredient as a replacement in any other poison or potion.

        In general, it requires an hour of work and a Knowledge (nature) check equal to 5 \add twice the rank of the item to find ingredients for an item in this way.
        Each time you find ingredients for an item this way, the time required to find ingredients again increases by an hour and the difficulty value increases by 5.
        Whenever you finish a \glossterm{long rest} or enter a different environment with different ingredients, these penalties reset.

        \ff[3]{Potent Poisons} You gain a \plus1 bonus to \glossterm{accuracy} with any poisons you create, including poisonous spells you cast.

        \ff[3]{Tempting Concoction} You can use the \textit{tempting concoction} ability as a \glossterm{standard action}.
        \begin{attuneability}{Tempting Concoction}{\abilitytag{Attune}, \abilitytag{Emotion}, \glossterm{Magical}, \abilitytag{Subtle}}
            \rankline
            Choose one liquid poison or potion you created, or an object containing one of those liquids, within \rngshort range.
            Whenever an \glossterm{enemy} notices the chosen object, make a \glossterm{reactive attack} vs. Mental against it.
            You are not aware of this attack, so you cannot use abilities like \ability{desperate exertion} to affect it.
            If the poison or potion is not concealed inside a less suspicious object, such as a tankard of ale or an apple, you take a \minus4 penalty to \glossterm{accuracy}.
            You cannot make this attack more than once against any individual target during this ability's duration.
            \hit The target is filled with the desire to investigate and try to consume the liquid or the object containing the liquid.
            It will not generally interrupt combat or wander into obvious danger to fulfill its desire, but individual creatures may react more or less strongly.
            This effect lasts until the target consumes the object or until it takes a \glossterm{short rest}.

            \rankline
            You gain a \plus2 accuracy bonus with the attack for every 3 levels beyond 3.
        \end{attuneability}

        \ff[9]{Esoteric Concoction+} The time required to find ingredients with your \textit{esoteric concoction} ability is halved.
        In addition, the \glossterm{difficulty value} of the Knowledge check is reduced by 5.

        \ff[15]{Potent Poisons+} The bonus from your \textit{potent poisons} ability increases to \plus2.

        \ff[21]{Esoteric Concoction+} The time required to find ingredients with your \textit{esoteric concoction} ability is reduced to five minutes, and the duration does not increase when you find ingredients for an item.
    \end{feat}

    \begin{feat}{Intimidate Specialization}{Skill}
        \featpre Intimidate as a trained skill.

        \ff[1]{Specialization} You gain a \plus3 bonus to the Intimidate skill.

        \ff[3]{Enduring Demoralize} When you use the \textit{demoralize} ability, the target is \shaken by you as a \glossterm{condition} instead of being briefly shaken.
        For details, see \pcref{Demoralize}.

        \ff[9]{Threatening Presence} Creatures that are shaken, frightened, or panicked by you suffer a penalty to their Armor defense equal to the penalty they suffer to their Mental defense.

        \ff[15]{Specialization+} The bonus from your \textit{specialization} ability increases to \plus6.

        \ff[21]{Enduring Demoralize+} When you use the \textit{demoralize} ability, the target is \frightened by you instead of being shaken.
    \end{feat}

    \begin{feat}{Iron Will}{General}
        \featpre Willpower 2.

        \ff[1]{Mental Discipline} You gain a \plus2 bonus to Mental defense.
        In addition, you gain a \plus1 bonus to your \glossterm{fatigue tolerance}.

        \ff[3]{Mind over Matter} You may use your Willpower in place of your Constitution to determine your \glossterm{hit points} (see \pcref{Hit Points}).

        \ff[9]{Unclouded Mind} You are immune to being \dazed and \stunned.

        \ff[15]{Mental Discipline+} The defense bonus from your \textit{mental discipline} ability increases to \plus4.
        In addition, the fatigue tolerance bonus increases to \plus2.

        \ff[21]{Unclouded Mind+} You are immune to all \abilitytag{Compulsion} and \abilitytag{Emotion} attacks.
    \end{feat}

    \begin{feat}{Juggernaut}{Combat}
        \featpre Strength 1, Constitution 2.

        \ff[1]{Unstoppable} You are immune to being \slowed and \immobilized.
        In addition, you are unaffected by \glossterm{difficult terrain}.

        \ff[3]{Trample} You can use the \textit{trample} ability as a standard action.
        \begin{ability}{Trample}
            This ability functions like the \textit{overrun} ability, except that it does not cause you to increase your \glossterm{fatigue level} and creatures may not choose to avoid you.
            In addition, if you move through a creature's space, it takes bludgeoning damage equal to 1d6 plus half your \glossterm{power}.

            \rankline
            You gain a \plus1 accuracy bonus and a \plus1d damage bonus at 6th level and every 3 levels thereafter.
        \end{ability}

        \ff[9]{Brute Force} You gain a \plus2 accuracy bonus with the \ability{overrun} ability and all related abilities, such as \ability{trample}.
        In addition, you gain a \plus2 bonus to your \glossterm{power}.

        \ff[15]{Brute Force+} The accuracy bonus from your \textit{brute force} ability increases to \plus3, and the power bonus increases to \plus4.

        \ff[15]{Unceasing} You can use the \textit{overrun} ability without increasing your \glossterm{fatigue level}.

        \ff[21]{Brute Force+} The accuracy bonus from your \textit{brute force} ability increases to \plus4, and the power bonus increases to \plus8.

        \ff[21]{Unblockable} You can move through spaces occupied by enemies as if they were unoccupied.
    \end{feat}

    \begin{feat}{Jump Specialization}{Skill}
        \featpre Jump as a trained skill.

        \ff[1]{Specialization} You gain a \plus3 bonus to the Jump skill.

        \ff[3]{Instant Leap} You are always considered to have a running start when making a long jump (see \pcref{Long Jump}).

        \ff[9]{Featherlight Leap} When you leap, your maximum height is equal to your Jump check result, rather than half your Jump check result.
        This does not affect the forward distance you can reach with your jumps.

        \ff[15]{Specialization+} The bonus from your \textit{specialization} ability increases to \plus6.

        \ff[21]{Featherlight Leap+} Your maximum height when jumping is equal to twice your Jump check result.
        This does not affect the forward distance you can reach with your jumps.
    \end{feat}

    \begin{feat}{Knowledge Specialization}{Skill}
        \featpre Any Knowledge skill as a trained skill.

        \ff[1]{Specialization} You gain a \plus3 bonus to all Knowledge skills.

        \ff[3]{Knowledge Savant} You gain two additional \glossterm{trained skills} which must be Knowledge skills.

        \ff[9]{Studied Defense} You gain \plus1 bonus to your choice of Fortitude, Reflex, and Mental defenses. 

        \ff[15]{Specialization+} The bonus from your \textit{specialization} ability increases to \plus6.

        \ff[21]{Studied Defense+} The bonus from your \textit{studied defense} ability instead applies to all defenses.
    \end{feat}

    \begin{feat}{Leadership}{Combat}
        \featpre Either Intelligence 2 or Willpower 2.

        \ff[1]{Battle Command} You can use the \textit{battle command} ability as a standard action.
        \begin{activeability}{Battle Command}[\abilitytag{Swift}]
            \rankline
            Choose an \glossterm{ally} within \rngmed range.
            During the current phase, the target gains a \plus2 bonus to \glossterm{accuracy} and rolls twice for any \glossterm{strikes} it makes, keeping the better result.

            \rankline
            \featlevel{6} The accuracy bonus increases to \plus3.
            \featlevel{12} The accuracy bonus increases to \plus4.
            \featlevel{18} The accuracy bonus increases to \plus5.
        \end{activeability}

        \ff[3]{Encouraging Presence} As long as you are conscious, your \glossterm{allies} who can see or hear you are immune to being \shaken and \frightened.

        \ff[9]{Bolster} You can use the \textit{bolster} ability as a standard action.
        \begin{activeability}{Bolster}[\abilitytag{Emotion}, \abilitytag{Swift}]
            \rankline
            One \glossterm{ally} within \medrange gains a \plus2 bonus to \glossterm{vital rolls} and all defenses this round.

            \rankline
            \featlevel{15} The bonus increases to \plus3.
            \featlevel{21} The bonus increases to \plus4.
        \end{activeability}

        \ff[15]{Desperate Command} You can use your \textit{battle command} and \textit{bolster} abilities as a single standard action.
        Both abilities must target the same creature.
        When you do, you increase your \glossterm{fatigue level} by one.

        \ff[21]{Unyielding Presence} As long as you are conscious, your \glossterm{allies} who can see or hear you are immune to being \dazed, \stunned, \confused, and \panicked.
    \end{feat}

    \begin{feat}{Linguistics Specialization}{Skill}
        \featpre Linguistics as a trained skill.

        \ff[1]{Specialization} You gain a \plus3 bonus to the Linguistics skill.

        \ff[3]{Linguistic Savant} You learn two additional \glossterm{common languages}, or one additional \glossterm{rare language}.

        \ff[9]{Language Focus} By spending a day in focused concentration on learning a specific \glossterm{common language} or \glossterm{rare language}, you can use the \textit{language focus} ability.
        You must have access to either a creature fluent in the language willing to help you or at least a book's worth of material written in the language.
        \begin{activeability}{Language Focus}
            \rankline
            If you had access to written material on the language, including from a teacher, you can read or write the language.
            If you had access to a speaker of the language, you can speak and understand the language.

            This ability's effect lasts until you use this ability again.
        \end{activeability}

        \ff[15]{Specialization+} The bonus from your \textit{specialization} ability increases to \plus6.

        \ff[21]{Language Focus+} The effect of your \textit{language focus} ability is permanent.
        With time, this can allow you to learn any number of languages.
    \end{feat}

    \begin{feat}{Maneuverist}{Combat}
        \featpre Intelligence 2.

        \ff[1]{Maneuver Access} You gain access to one \glossterm{combat style} that you did not already have access to (see \pcref{Combat Styles}).
        In addition, you learn one rank 1 \glossterm{maneuver} from that combat style.
        You may spend \glossterm{insight points} to learn one additional maneuver from that combat style per insight point.
        Unless otherwise noted in an ability's description, using a maneuver requires a \glossterm{standard action}.

        After you use a maneuver you know from this feat, you \glossterm{briefly} cannot use any maneuver from this feat.

        \ff[3]{Trained Maneuverist} Using a maneuver from this feat does not prevent you from using maneuvers from this feat.

        \ff[6]{Maneuver Rank} You become a rank 2 combat style user.
        This gives you access to maneuvers that require a minimum rank of 2.

        \ff[9]{Maneuver Rank} You become a rank 3 combat style user.
        This gives you access to maneuvers that require a minimum rank of 3 and can improve the effectiveness of your existing maneuvers.

        \ff[9]{Maneuver Knowledge} You learn one maneuver.

        \ff[12]{Maneuver Rank} You become a rank 4 combat style user.
        This gives you access to maneuvers that require a minimum rank of 4 and can improve the effectiveness of your existing maneuvers.

        \ff[15]{Maneuver Rank} You become a rank 5 combat style user.
        This gives you access to maneuvers that require a minimum rank of 5 and can improve the effectiveness of your existing maneuvers.

        \ff[15]{Maneuver Knowledge+} You learn one maneuver.

        \ff[18]{Maneuver Rank} You become a rank 6 combat style user.
        This gives you access to maneuvers that require a minimum rank of 6 and can improve the effectiveness of your existing maneuvers.

        \ff[21]{Maneuver Rank} You become a rank 7 combat style user.
        This gives you access to maneuvers that require a minimum rank of 7 and can improve the effectiveness of your existing maneuvers.

        \ff[21]{Maneuver Knowledge+} You learn one maneuver.
    \end{feat}

    \begin{feat}{Martial Training}{Combat}
        \ff[1]{Trained Strike} You can use the \textit{trained strike} ability as a standard action.
        \begin{activeability}{Trained Strike}
            \rankline
            Make a \glossterm{strike} with a \plus1 bonus to \glossterm{accuracy}.

            \rankline
            \featlevel{6} The accuracy bonus increases to \plus2.
            \featlevel{12} The accuracy bonus increases to \plus3.
            \featlevel{18} The accuracy bonus increases to \plus4.
        \end{activeability}

        \ff[3]{Equipment Training} You choose one of the following benefits.
        \begin{itemize}
            \item You gain proficiency with a \glossterm{usage class} of \glossterm{armor} (light, medium, or heavy).
                You must be proficient with light armor to gain proficiency with medium armor, and you must be proficient with medium armor to gain proficiency with heavy armor.
            \item You gain proficiency with an additional \glossterm{weapon group} of your choice.
            \item You gain proficiency with \glossterm{exotic weapons} from a weapon group of your choice that you are already proficient with.
            \item You reduce the \glossterm{encumbrance} of \glossterm{body armor} you wear by 1.
                If you choose this ability multiple times, its effects stack.
            \item You gain a \plus1 bonus to your \glossterm{accuracy}.
            \item You gain a \plus2 bonus to your \glossterm{power}.
                If you choose this ability multiple times, its effects stack.
        \end{itemize}

        \ff[9]{Equipment Training+} You gain an additional \textit{equipment training} ability of your choice.

        \ff[15]{Equipment Training+} You gain two additional \textit{equipment training} abilities of your choice.

        \ff[21]{Equipment Training+} You gain three additional \textit{equipment training} abilities of your choice.
    \end{feat}

    \begin{feat}{Medicine Specialization}{Skill}
        \featpre Medicine as a trained skill.

        \ff[1]{Specialization} You gain a \plus3 bonus to the Medicine skill.

        \ff[3]{Healing Touch} You can use the \textit{healing touch} ability as a standard action.
        \begin{activeability}{Healing Touch}{\abilitytag{Swift}}
            \rankline
            Choose yourself or an adjacent living \glossterm{ally}.
            The target regains 1d10 \add \glossterm{power} \glossterm{hit points} and increases its \glossterm{fatigue level} by one.
            In addition, make a Medicine check.
            For each poison and disease on the target, if your check result beats a \glossterm{difficulty value} equal to 10 \add twice its \glossterm{rank}, the effect is removed.

            \rankline
            \featlevel{6} The healing increases to 2d6.
            \featlevel{9} The healing increases to 2d10.
            \featlevel{12} The healing increases to 4d6.
            \featlevel{15} The healing increases to 4d10.
            \featlevel{18} The healing increases to 5d10.
            \featlevel{21} The healing increases to 7d10.
        \end{activeability}

        \ff[9]{Vital Touch} When you use your \textit{healing touch} ability, the target can also remove one \glossterm{vital wound}.
        If it does, its increases its \glossterm{fatigue level} by three.

        \ff[15]{Specialization+} The bonus from your \textit{specialization} ability increases to \plus6.

        \ff[21]{Vital Touch+} Your \textit{vital touch} ability only increases the target's fatigue level by two.
    \end{feat}

    \begin{feat}{Mental Magic}{Casting, Magical}
        \featpre Spellcasting ability, Willpower 3.

        \ff[1]{Mental Casting} You connect to the magical essence of the universe differently from other spellcasters, allowing you to cast spells with purely mental effort.
        None of your spells have \glossterm{somatic components} or \glossterm{verbal components}.

        \ff[3]{Fractured Mind} Once per round, you can sustain an ability with the \abilitytag{Sustain} (minor) tag as a \glossterm{free action}.

        \ff[9]{Potent Mind} You gain a \plus2 bonus to your \glossterm{power}.

        \ff[15]{Potent Mind+} The power bonus from your \textit{potent mind} ability increases to \plus4.

        \ff[21]{Fractured Mind+} You can use your \textit{fractured mind} ability to sustain abilities with the \abilitytag{Sustain} (standard) tag.
        Each time you sustain an ability in this way, you increase your \glossterm{fatigue level} by one.

        \ff[21]{Potent Mind+} The power bonus from your \textit{potent mind} ability increases to \plus8.
    \end{feat}

    \begin{feat}{Metacaster}{Casting, Magical}
        \featpre \ability{Metamagic} ability, Intelligence 2.

        \ff[1]{Sphere Access} You gain access to an additional \glossterm{mystic sphere}.
        Each \glossterm{mystic sphere} has a set of \glossterm{spells} associated with it.
        You automatically learn all \glossterm{cantrips} from any mystic sphere you have access to.
        % TODO: wording
        If you have multiple \glossterm{magic sources}, you can cast spells from that sphere with any magic source that the mystic sphere belongs to.

        \ff[3]{Adaptive Metamagic} Choose one spell you know.
        Whenever you cast that spell, you may apply one metamagic ability of your choice to that spell.
        You can choose any of the metamagic abilities offered by your class which are valid for that spell.
        This does not allow you to exceed the normal maximum of two metamagic abilities on a single spell.

        \ff[9]{Spell Fusion} You can use the \textit{spell fusion} ability as a \glossterm{standard action}.
        \begin{activeability}{Spell Fusion}
            \rankline
            Choose two spells that you know which do not have the \abilitytag{Sustain} or \abilitytag{Attune} tags.
            You cast both spells simultaneously.
            Both spells that you fuse in this way must have the same area shape, such as a cone or sphere, and targeting restrictions, such as affecting only enemies or living creatures.
            If one spell affects a strictly larger area or a strictly larger number of targets than the other, you must use the smaller of the two areas or target counts.
            You must choose the same targets and area for both spells, if applicable.
            Roll the attack roll and damage for each spell separately.

            After you use this ability, you are unable to take any actions during the \glossterm{action phase} of the following round.
        \end{activeability}

        \ff[15]{Adaptive Metamagic+} You can choose an additional spell you know with your \textit{adaptive metamagic} ability.
        In addition, you can apply two metamagic abilities to your chosen spells, rather than only one.

        \ff[21]{Spell Fusion+} Using your \textit{spell fusion} ability does not prevent you from taking \glossterm{minor actions} during the following round.
    \end{feat}

    \begin{feat}{Mystic Archer}{Casting}
        \featpre Access to a \glossterm{mystic sphere}.

        \ff[1]{Imbued Shot} You can use the \textit{imbued shot} ability as a standard action.
        \begin{activeability}{Imbued Shot}[\glossterm{Magical}]
            \rankline
            Make a ranged \glossterm{strike} with a \plus1 bonus to \glossterm{accuracy} using a \glossterm{projectile weapon} you wield.
            Because this is a \glossterm{magical} ability, you use your Willpower to determine your damage dice instead of your Strength (see \pcref{Dice Bonuses From Attributes}).

            \rankline
            \featlevel{6} The accuracy bonus increases to \plus2.
            \featlevel{12} The accuracy bonus increases to \plus3.
            \featlevel{18} The accuracy bonus increases to \plus4.
        \end{activeability}

        \ff[1]{Magical Strikes}[\abilitytag{Magical}] Whenever you make a projectile \glossterm{strike}, you can choose to treat that as a \glossterm{magical} ability.
        When you do, you use your Willpower to determine your damage dice instead of your Strength (see \pcref{Dice Bonuses From Attributes}).

        \ff[3]{Imbue Projectile}[Magical] When you cast a spell as a \glossterm{standard action}, if that spell does not have the \abilitytag{Attune} or \abilitytag{Sustain} tags,
            you can use the \textit{imbue projectile} ability.
        \begin{attuneability}{Imbue Projectile}{\abilitytag{Attune}}
            \rankline
            The spell does not have its effect immediately.
            Instead, its power is imbued in a \glossterm{projectile} you hold.
            An individual projectile can only be imbued with this ability once, even if multiple creatures use this ability on the same projectile.

            When you use your \textit{imbued shot} ability to attack with that projectile, the spell takes effect on the target of your \textit{imbued shot} ability.
            You must make any attack rolls required for the spell separately from your attack roll with the strike.
            After the spell takes effect this way, your attunement to this ability ends.
        \end{attuneability}

        \ff[9]{Phasing Projectiles}[\glossterm{Magical}] Your projectile \glossterm{strikes} ignore \glossterm{cover}.
        In addition, you can ignore ignore all physical obstacles in single one-foot span.
        This can allow you to fire projectiles through creatures or solid walls, though it does not grant you the ability to see through a wall.

        \ff[15]{Guided Projectiles} Your projectile \glossterm{strikes} ignore all \glossterm{miss chances}, such as from \glossterm{concealment}.
        This does not prevent your attacks from failing for other reasons.

        \ff[21]{Imbue Projectile+}[\glossterm{Magical}] Your \textit{imbue projectile} ability loses the \abilitytag{Attune} tag.
        Instead, it lasts until you use it again.
    \end{feat}

    \begin{feat}{Null}{General}
        \featpre Willpower 2.

        \ff[1]{Nullify Magic} You gain a \plus4 bonus to your \glossterm{defenses} against \glossterm{magical} abilities.
        In exchange, you lose the benefits of all \glossterm{magical} abilities you possess.
        You are unable to \glossterm{attune} to any \glossterm{magical} abilities, such as magic items or spells cast by other creatures.
        You cannot use \glossterm{potions} or other similar magic items or abilities that affect you personally, even if they do not require attunement.
        In addition, you are never considered an \glossterm{ally} for a \glossterm{magical} ability, even while \unconscious.

        % Much stronger than Spellbreaker because this is a huge part of the whole Null thing
        \ff[1]{Sever Magic} You can use the \textit{sever magic} ability as a standard action.
        \begin{activeability}{Sever Magic}
            \rankline
            Make a \glossterm{strike}.
            If the target takes damage from the strike, it stops being \glossterm{attuned} to one effect of its choice that it is currently attuned to.
            If it has any magical abilities, but has no remaining attuned effects, it becomes \dazed as a \glossterm{condition} instead.
            On a \glossterm{critical hit}, the target takes double damage and it stops being attuned to two abilities of its choice that it is currently attuned to.
            In addition, as a \glossterm{condition}, it stops being able to attune to abilities.

            \rankline
            \featlevel{6} You gain a \plus1 bonus to \glossterm{accuracy} with the strike.
            \featlevel{12} A struck target stops being attuned to an additional effect of its choice.
            \featlevel{18} The accuracy bonus increases to \plus2.
        \end{activeability}

        \ff[3]{Mundane Resilience} You gain a bonus to your \glossterm{hit points} and \glossterm{damage resistance} equal to your level.

        \ff[3]{Personal Legacy} You do not gain any legacy item upgrades (see \pcref{Legacy Items}).
        Instead, each time you would gain a legacy item upgrade, you instead gain a \plus1 bonus to your \glossterm{accuracy}, your \glossterm{fatigue tolerance}, and all \glossterm{defenses}.

        % clarify that it does not affect spells already in progress in the current phase?
        \ff[9]{Disruptive Presence} Whenever an \glossterm{enemy} within an \medarea radius from you uses a \glossterm{magical} ability as a standard action, the ability has a 50\% chance to fail with no effect.
        This cannot cause passive or triggered abilities to fail.

        \ff[15]{Mundane Resilience+} The bonuses from your \textit{mundane resilience} ability increase to twice your level.

        \ff[15]{Nullify Magic+} The bonus to your defenses from your \textit{nullify magic} ability increases to \plus8.

        \ff[21]{Disruptive Presence+} The failure chance from your \textit{disruptive presence} ability increases to 80\%.

        \ff[21]{True Null} You are unaffected by all \glossterm{magical} abilities.
    \end{feat}

    \begin{feat}{Perform Specialization}{Skill}
        \featpre Any Perform skill as a trained skill.

        \ff[1]{Specialization} You gain a \plus3 bonus to all Perform skills.

        \ff[3]{Synergistic Performance} You can use any Perform skill you are \glossterm{trained} with in place of other related skills.
        Each Perform skill has an associated skill that it can be used to replace, as listed below.
        When you replace a skill in this way, you add half your modifier with the Perform skill instead of your full modifier since the two skills do not exactly match.
        \begin{itemize}
            \item Acting: Deception
            \item Comedy: Deception
            \item Dance: Balance
            \item Keyboard instruments: Devices
            \item Oratory: Persuasion
            \item Percussion instruments: Creature Handling
            \item Singing: Persuasion
            \item String instruments: Devices
            \item Wind instruments: Creature Handling
        \end{itemize}

        \ff[9]{Inspiring Performance}[Magical] Whenever you perform with the Perform skill, each \glossterm{ally} that can observe the performance gains a \plus1 bonus to Mental defense.
        % This solves the "always active except on round 1" problem
        This effect has the \abilitytag{Swift} tag, so it protects allies in the same phase that you begin performing.
        This includes both normal performances and any special abilities that require performances.
        This bonus lasts as long as the performance lasts.

        \ff[15]{Specialization+} The bonus from your \textit{specialization} ability increases to \plus5.

        \ff[21]{Inspiring Performance+} The bonus from your \textit{inspiring performance} ability increases to \plus2.
    \end{feat}

    \begin{feat}{Persuasion Specialization}{Skill}
        \featpre Persuasion as a trained skill.

        \ff[1]{Specialization} You gain a \plus3 bonus to the Persuasion skill.

        \ff[3]{Compel Attention}[Magical] You can use the \textit{compel attention} ability as a standard action.
        \begin{sustainability}{Compel Attention}{\abilitytag{Auditory}, \abilitytag{Compulsion}, \abilitytag{Sustain} (minor), \abilitytag{Subtle}}
            \rankline
            Make an attack vs. Mental against a creature within \rngmed range.
            Your \glossterm{accuracy} is equal to your Persuasion skill.
            You take a \minus10 penalty to \glossterm{accuracy} with this attack against creatures who have made an attack or been attacked since the start of the last round.
            You must talk loud enough for the target to hear to draw its attention.
            \hit The target only pays attention to you as long as you sustain this ability, which requires maintaining your conversation with it.
            It takes a \minus20 penalty to Awareness checks to observe anything unrelated to your conversation.
            Any act by you or by creatures that appear to be your ally that damages a target or that causes it to feel that it is in danger breaks the effect for that creature.

            \rankline
            \featlevel{9} You may target up to five creatures within range.
            \featlevel{15} You may target any number of creatures within range.
            \featlevel{21} The range increases to \distrange.
        \end{sustainability}

        \ff[9]{First Impressions} When you first meet creatures, you have an Ally relationship instead of a Just Met relationship (see \tref{Relationship Modifiers}.
        This does not affect your relationship with creatures who would not normally have a Just Met relationship with you.

        \ff[15]{Specialization+} The bonus from your \textit{specialization} ability increases to \plus6.

        \ff[21]{Suggestion}[Magical] You can use the \textit{suggestion} ability as a standard action.
        \begin{sustainability}{Suggestion}{\abilitytag{Emotion}, \abilitytag{Subtle}, \abilitytag{Sustain} (minor)}
            \rankline
            Make an attack vs. Mental against a creature within \rngmed range.
            Your \glossterm{accuracy} is equal to your Persuasion skill.
            You must also make a verbal suggestion of a particular course of action to the target.
            If your suggestion does not seem reasonable, you take a \minus5 accuracy penalty on this attack.
            Exceptionally unreasonable suggestions can impose even greater penalties, and exceptionally reasonable suggestions can give accuracy bonuses.

            \hit As a \glossterm{condition}, the target thinks your suggestion is a good idea and will try to follow it to the best of its abilities.
            Any act by you or by creatures that appear to be your ally that damages the target or makes it feel that it is in danger breaks the effect.
            An observant target may interpret overt threats to its \glossterm{allies} as a threat to itself.
        \end{sustainability}
    \end{feat}

    \begin{feat}{Precognition}{General}
        \featpre Intelligence 3.

        \ff[1]{Precognitive Offense} You can use your Intelligence in place of your Strength or Willpower to determine your dice pools (see \pcref{Dice Bonuses From Attributes}).

        \ff[3]{Combat Prediction} You can use the \textit{combat prediction} ability as a standard action.
        \begin{sustainability}{Combat Prediction}{\abilitytag{Subtle}, \abilitytag{Sustain} (minor)}
            \rankline
            Choose a creature within \medrange of you.
            That creature's intentions become obvious to you as long as you sustain this ability.
            This gives you a \plus1 bonus to \glossterm{accuracy} and \glossterm{defenses} against that creature.

            In addition, when you choose your action during each phase, you can do after learning the result of the actions the target took during that phase.
            This includes learning whether its attacks hit, and any obvious effects of those attacks.
            You do not gain any knowledge of actions that have no obvious signs, such as purely mental actions, or actions which you are not observant enough to notice.
            You cannot communicate your understanding of the creature's actions to your allies before they decide their actions.
        \end{sustainability}

        \ff[9]{Mobile Foresight} During the \glossterm{movement phase}, you choose your action after all other creatures have chosen their actions.
        When you choose your action, you have insight into the actions chosen by any creatures within \longrange of you that you can see.
        This insight gives you the same information as the insight from your \textit{combat prediction} ability, except that it only provides information about their actions during the movement phase.
        You choose your actions simultaneously with any other creatures who have a similar ability.

        Knowing another creature's action does not automatically allow you to interrupt that action.
        If you want to interrupt an action, such as by blocking a creature's intended movement, you must make an \glossterm{initiative} check as normal.

        \ff[15]{Precognitive Precision} You gain a \plus1 bonus to \glossterm{accuracy}.

        \ff[21]{Precognitive Precision+} The bonus from your \textit{precognitive precision} ability increases to \plus2.

        \ff[21]{Combat Prediction+} The bonuses from your \textit{combat prediction} ability increase to \plus2.
    \end{feat}

    \begin{feat}{Prepared Spellcasting}{Magical, Spell}
        \featpre Access to a \glossterm{mystic sphere}, Intelligence 3.

        \ff[1]{Spellbook} Choose up to three spells you do not know from among \glossterm{mystic spheres} you have access to.
        The spells must be of a rank that you know how to cast.
        Whenever you gain access to a new spell rank, you may change the spells in your spellbook for any other spells you can cast.
        You inscribe the knowledge of those spells into a book you carry with you.
        This book is your spellbook.
        
        Whenever you finish a \glossterm{long rest}, you may choose one of the spells in your spellbook.
        You learn how to cast that spell until you choose a different spell with this ability.

        \ff[3]{Ritual Book} You gain the ability to perform rituals to create unique magical effects (see \pcref{Rituals}).
        In addition, you automatically learn one free ritual of each rank you have access to, including new ranks as you gain access to them.

        \ff[9]{Study of Magic} You gain a \plus2 bonus to your \glossterm{power}.

        \ff[15]{Spellbook+} You can choose up to six spells to be in your spellbook instead of only three.
        In addition, whenever you finish a \glossterm{long rest}, you may choose two spells in your spellbook with your \textit{spellbook} ability instead of one.
        You learn how to cast both spells until you choose a different pair of spells in this way.

        \ff[15]{Study of Magic+} The bonus from your \textit{study of magic} ability increases to \plus4.

        \ff[21]{Ritual Book+} You learn an additional free ritual of each rank you have access to.

        \ff[21]{Study of Magic+} The bonus from your \textit{study of magic} ability increases to \plus8.
    \end{feat}

    \begin{feat}{Rapid Reaction}{General}
        \featpre Dexterity 2.

        \ff[1]{Lightning Reflexes} You gain a \plus2 bonus to your Reflex defense.

        \ff[3]{Sidestep} If you have movement remaining with your \glossterm{land speed} after the \glossterm{movement phase}, you may use that movement during the \glossterm{action phase} as a \glossterm{free action}.
        You cannot carry over more than half your land speed in this way.

        \ff[9]{Evasive Reaction} You take half damage from abilities that affect an area and attack your Armor or Reflex defense.
        This does not protect you from any non-damaging effects of those abilities, or from abilities that affect multiple specific targets without affecting an area.
        If you have the \textit{evasion} monk or rogue ability with the same effect as this ability, you reduce the total damage you take to one quarter of the normal value instead.

        \ff[15]{Lightning Reflexes+} The bonus from your \textit{lightning reflexes} ability increases to \plus4.

        \ff[21]{Greater Evasive Reaction} Your \textit{evasive reaction} ability also protects you from area attacks against your Fortitude and Mental defenses.
    \end{feat}

    \begin{feat}{Regenerator}{General}
        \featpre Constitution 3.

        \ff[1]{Diehard} You gain a \plus2 bonus to \glossterm{vital rolls}.

        \ff[3]{Regenerative Recovery} You can use the \textit{regenerative recovery} ability as a standard action.
        \begin{activeability}{Regenerative Recovery}[\abilitytag{Swift}]
            \rankline
            When you use this ability, you increase your \glossterm{fatigue level} by one.

            You regain 1d10 \add \glossterm{power} \glossterm{hit points}.
            Unlike normal, your Strength does not modify this dice pool (see \pcref{Dice Bonuses From Attributes}).
            Instead, you gain a \plus1d bonus to the healing for every 2 Constitution you have.
            If you gained any \glossterm{vital wounds} during the previous round, your healing with this ability is doubled.

            \rankline
            \featlevel{6} The healing increases to 2d6.
            \featlevel{9} The healing increases to 2d10.
            \featlevel{12} The healing increases to 4d6.
            \featlevel{15} The healing increases to 4d10.
            \featlevel{18} The healing increases to 5d10.
            \featlevel{21} The healing increases to 7d10.
        \end{activeability}

        \ff[9]{Regenerative Rest} When you take a \glossterm{short rest}, you can remove any number of \glossterm{vital wounds} affecting you.
        If you do, you increase your \glossterm{fatigue level} by three per vital wound removed this way.
        You can use this ability to remove vital wounds even once your fatigue level would already make you unconscious.
        This allows you to recover any number of vital wounds if you go unconscious to do so, regardless of your maximum fatigue level.

        \ff[15]{Battlefield Regeneration} When you use the \textit{recover} action, you can also remove a single vital wound.
        If you do, you increase your fatigue level by one.

        \ff[15]{Diehard+} The bonus from your \textit{diehard} ability increases to \plus4.

        \ff[21]{Regenerative Rest+} Your \textit{regenerative rest} ability only causes you to increase your fatigue level by one per vital wound.
    \end{feat}

    \begin{feat}{Ride Specialization}{Skill}
        \featpre Ride as a trained skill.

        \ff[1]{Specialization} You gain a \plus3 bonus to the Ride skill.

        \ff[3]{Mounted Defense} Your mount gains a \plus3 bonus to all defenses, up to a maximum of your own corresponding defense.

        \ff[9]{Mounted Warrior} The penalty you take for making ranged \glossterm{strikes} while mounted is reduced by 1.
        In addition, while you are mounted, you gain a \plus1 bonus to \glossterm{accuracy} with Mounted weapons (see \pcref{Mounted Weapon}).

        \ff[15]{Specialization+} The bonus from your \textit{specialization} ability increases to \plus6.

        \ff[21]{Mounted Defense+} The defense bonus from your \textit{mounted defense} ability increases to \plus6.

        \ff[21]{Mounted Warrior+} The penalty reduction from your \textit{mounted warrior} ability increases to 2.
        In addition, the accuracy bonus increases to \plus2.
    \end{feat}

    \begin{feat}{Shieldbearer}{Combat}
        \featpre Strength 2.

        \ff[1]{Shield Expertise} You gain a \plus1 bonus to Armor defense while you wield a shield.

        \ff[3]{Forceful Block} Whenever a creature misses or \glossterm{glances} you with a melee \glossterm{strike}, if you are wielding a shield, that creature \glossterm{briefly} takes a \minus2 penalty to Armor defense.
        As normal, this bonus does not stack with itself, even if the same creature misses you with multiple melee attacks.

        \ff[9]{Arrow Deflection} While you wield a shield, you and each \glossterm{ally} adjacent to you gain a \plus2 bonus to Armor defense against ranged \glossterm{strikes}.
        This bonus is doubled when you take the \ability{total defense} action.

        \ff[15]{Shield Expertise+} The bonus from your \textit{shield expertise} ability increases to \plus2.

        \ff[21]{Forceful Block+} The penalty from your \textit{forceful block} ability increases to \minus4.

        \ff[21]{Shield Expertise+} The bonus from your \textit{shield expertise} ability increases to \plus3.
    \end{feat}

    \begin{feat}{Sleight of Hand Specialization}{Skill}
        \featpre Sleight of Hand as a trained skill.

        \ff[1]{Specialization} You gain a \plus3 bonus to the Sleight of Hand skill.

        \ff[3]{Deep Pickpocket} You can use the \textit{pickpocket} ability to retrieve objects that are loose within larger containers, such as backpacks or sacks, even if they are not immediately accessible.
        You must be able to reach at least one of your fingers into the bag, such as through a narrow gap at the opening.
        This does not allow you to retrieve objects from locked containers with no openings.
        The container's size cannot exceed your own size.

        \ff[9]{Extradimensional Concealment}[Magical] When you use the \textit{conceal object} ability, you can use the \textit{extradimensional pocket} ability.
        \begin{attuneability}{Extradimensional Pocket}{\abilitytag{Attune}}
            \rankline
            You conceal the object in a pocket dimension that cannot be accessed by nonmagical means.
            When your attunement to this ability ends, the object appears in a free hand.
            If you have no free hands, it drops to the ground.
        \end{attuneability}

        \ff[15]{Specialization+} The bonus from your \textit{specialization} ability increases to \plus6.

        \ff[21]{Conceal Object+} The maximum size of object you can hide with your \textit{conceal object} ability increases to be equal to your size category.

        \ff[21]{Deep Pickpocket+} The maximum size of the container you can reach into with your \textit{deep pickpocket} ability increases to two size categories larger than your own size.
    \end{feat}

    \begin{feat}{Sniper}{Combat}
        \featpre Perception 2.

        \ff[1]{Aim} You can use the \textit{aim} ability as a standard action.
        \begin{sustainability}{Aim}{\abilitytag{Subtle}, \abilitytag{Sustain} (minor)}
            \rankline
            Choose a creature or object you can see.
            You gain a \plus2 accuracy bonus against the target.

            If you lose sight of the target for a full round, this effect ends.
        \end{sustainability}

        \ff[3]{Sniper's Precision} You gain a \plus1 bonus to \glossterm{accuracy}.

        \ff[9]{Precise Shot} You ignore \glossterm{cover} and \glossterm{concealment} with ranged attacks.

        \ff[15]{Distance Tolerance} You reduce your \glossterm{longshot penalty} by 2.

        \ff[21]{Sniper's Precision+} The accuracy bonus from your \textit{sniper's precision} ability increases to \plus2.
    \end{feat}

    \begin{feat}{Social Insight Specialization}{Skill}
        \featpre Social Insight as a trained skill.

        \ff[1]{Specialization} You gain a \plus3 bonus to the Social Insight skill.

        \ff[3]{Read Emotions}[Magical] You can use the \textit{read emotions} ability as a standard action.
        \begin{sustainability}{Read Emotions}{\abilitytag{Emotion}, \abilitytag{Sustain} (minor), \abilitytag{Subtle}}
            \rankline
            Make an attack vs. Mental against a creature within \rngshort range.
            Your \glossterm{accuracy} is equal to your Social Insight skill.
            \hit You know the target's current emotions.
            In addition to the obvious effects, this grants you a \plus3 bonus to Deception, Persuasion, Intimidate, and Social Insight attacks and checks against the target.
            This bonus does not stack with other effects that allow you access to the target's mind, such as \ability{read mind}.

            \rankline
            \featlevel{12} The range increases to \medrange.
            \featlevel{18} The range increases to \longrange.
        \end{sustainability}

        \ff[9]{Truthsense} Whenever a creature within a \arealarge radius \glossterm{emanation} from you that you can hear and see speaks truth to the best of its knowledge with no attempt at evasion, concealment, or creative wording, you automatically recognize that.
        You do not recognize truth in this way if a creature is using the Deception skill in any way, even if it is speaking the truth.

        \ff[15]{Specialization+} The bonus from your \textit{specialization} ability increases to \plus6.

        \ff[21]{Truthsense+} The area of your \textit{truthsense} ability increases to a \gargarea radius.
        In addition, you automatically recognize the difference between a creatively worded truth and an outright lie.
    \end{feat}

    \begin{feat}{Spellsword}{Magical, Spell}
        \featpre Access to a \glossterm{mystic sphere}.

        \ff[1]{Imbued Blow} You can use the \textit{imbued blow} ability as a standard action.
        \begin{activeability}{Imbued Blow}[\abilitytag{Magical}]
            \rankline
            Make a melee \glossterm{strike} with a \plus1 bonus to \glossterm{accuracy}.
            Because this is a \glossterm{magical} ability, you use your Willpower to determine your damage dice instead of your Strength (see \pcref{Dice Bonuses From Attributes}).

            \rankline
            \featlevel{6} The accuracy bonus increases to \plus2.
            \featlevel{12} The accuracy bonus increases to \plus3.
            \featlevel{18} The accuracy bonus increases to \plus4.
        \end{activeability}

        \ff[1]{Magical Strikes}[\abilitytag{Magical}] Whenever you make a melee \glossterm{strike}, you can choose to treat it as a \glossterm{magical} ability.
        When you do, you use your Willpower to determine your damage dice instead of your Strength (see \pcref{Dice Bonuses From Attributes}).

        \ff[3]{Spellsword Conduit} You can cast spells using a non-projectile weapon as if it were an implement (see \pcref{Implements}).
        In addition, if your legacy item is a weapon, you may choose both weapon and implement magic item effects for it.

        \ff[9]{Imbue Weapon}[Magical] When you cast a spell that does not have the \abilitytag{Attune} or \abilitytag{Sustain} tags,
            you can use the \textit{imbue weapon} ability.
        \begin{attuneability}{Imbue Weapon}{\abilitytag{Attune}}
            \rankline
            The spell does not have its effect immediately.
            Instead, its power is imbued in a non-projectile weapon you hold.
            An individual weapon can only be imbued with this ability once.

            When you use your \textit{imbued blow} ability to make a strike with that weapon, you may choose to activate the spell.
            If you do, the spell takes effect on the target of your \textit{imbued blow} ability as if you had just cast it.
            You must make any attack rolls required for the spell separately from your attack roll with the strike.
            After the spell takes effect this way, your attunement to this ability ends.
        \end{attuneability}

        \ff[15]{Spellsword Conduit+} Whenever you cast a spell as a standard action using a non-projectile weapon as an implement, you \glossterm{briefly} gain a \plus2 bonus to \glossterm{accuracy} with \glossterm{strikes}.

        \ff[21]{Imbue Weapon+} You may imbue two spells with your \textit{imbue weapon} ability instead of only one.
        This only costs a single \glossterm{attunement point}.
        You can only activate one of those spells each time you use the \textit{imbued blow} ability.
    \end{feat}

    \begin{feat}{Spellwarped}{General, Magical}
        \featpre Willpower 2.

        \ff[1]{Mystic Sphere} You gain the ability to use arcane magic.
        You gain access to one arcane \glossterm{mystic sphere}, plus the \sphere{universal} mystic sphere (see \pcref{Arcane Mystic Spheres}).
        Each \glossterm{mystic sphere} has a set of \glossterm{spells} associated with it.
        You automatically learn all \glossterm{cantrips} from the mystic sphere you have access to.

        You require both \glossterm{verbal components} and \glossterm{somatic components} to cast spells from your chosen sphere.
        For details about mystic spheres and casting spells, see \pcref{Spell and Ritual Mechanics}.

        \ff[3]{Spell Rank} You become a rank 1 spellcaster in your chosen \glossterm{mystic sphere}.
        You learn one spell from that mystic sphere.
        In addition, you can spend \glossterm{insight points} to learn one additional arcane spell per \glossterm{insight point}.
        Unless otherwise noted in a spell's description, casting a spell requires a \glossterm{standard action}.

        When you gain access to a spell rank,
            you can exchange any number of spells you know for other spells,
            including spells of the higher rank.

        \ff[6]{Spell Rank} You become a rank 2 spellcaster in your chosen \glossterm{mystic sphere}.
        This gives you access to spells that require a minimum rank of 2.

        \ff[9]{Spell Rank} You become a rank 3 spellcaster in your chosen \glossterm{mystic sphere}.
        This gives you access to spells that require a minimum rank of 3 and can improve the effectiveness of your existing spells.

        \ff[9]{Spell Knowledge} You learn one spell from your \glossterm{mystic sphere}.

        \ff[12]{Spell Rank} You become a rank 4 spellcaster in your chosen \glossterm{mystic sphere}.
        This gives you access to spells that require a minimum rank of 4 and can improve the effectiveness of your existing spells.

        \ff[15]{Spell Rank} You become a rank 5 spellcaster in your chosen \glossterm{mystic sphere}.
        This gives you access to spells that require a minimum rank of 5 and can improve the effectiveness of your existing spells.

        \ff[15]{Spell Knowledge} You learn one spell from your \glossterm{mystic sphere}.

        \ff[18]{Spell Rank} You become a rank 6 spellcaster in your chosen \glossterm{mystic sphere}.
        This gives you access to spells that require a minimum rank of 6 and can improve the effectiveness of your existing spells.

        \ff[21]{Spell Rank} You become a rank 7 spellcaster in your chosen \glossterm{mystic sphere}.
        This gives you access to spells that require a minimum rank of 7 and can improve the effectiveness of your existing spells.

        \ff[21]{Spell Knowledge} You learn one spell from your \glossterm{mystic sphere}.
    \end{feat}

    \begin{feat}{Sphere Focus: Aeromancy}{Casting, Magical}
        \featpre Access to the \sphere{Aeromancy} \glossterm{mystic sphere}.

        \ff[1]{Distant Winds} You gain a \plus30 foot bonus to your range with abilities from the \sphere{Aeromancy} \glossterm{mystic sphere}.

        \ff[3]{Favorable Winds} You gain a \plus4 bonus to the Jump skill.
        In addition, you gain a \plus1 bonus to all defenses against ranged \glossterm{strikes}.

        \ff[9]{Airborne} You gain a \glossterm{fly speed} equal to the \glossterm{base speed} for your size with a maximum height of 30 feet (see \pcref{Flight}).
        As a \glossterm{free action}, you can increase your \glossterm{fatigue level} by one to ignore this height limit until the end of the round.

        \ff[15]{Intrinsic Aeromancy} You learn an additional spell with the \abilitytag{Attune} tag from the \sphere{Aeromancy} mystic sphere.
        If that spell is a \glossterm{deep attunement}, you only need to spend one \glossterm{attunement point} to attune to it.
        Otherwise, you can attune to it without spending an \glossterm{attunement point}.
        When you gain access to new spell ranks, you can change which spell you know with this ability.

        \ff[21]{Distant Winds+} The range bonus from your \textit{distant winds} ability increases to \plus120 feet.
    \end{feat}

    \begin{feat}{Sphere Focus: Aquamancy}{Casting, Magical}
        \featpre Access to the \sphere{Aquamancy} \glossterm{mystic sphere}.

        \ff[1]{Aquatic Propulsion} Once per round, when you make an attack with a spell from the \sphere{Aquamancy} \glossterm{mystic sphere}, you can \glossterm{push} yourself up to 10 feet.
        You can use this ability with \glossterm{strikes} using natural weapons from that sphere, such as from the \spell{aqueous tentacle} spell.
        The attack does not have to hit for you to use this ability.

        \ff[3]{Partially Liquid} You gain a \plus4 bonus to the Flexibility skill.
        In addition, you gain a \plus4 bonus to your defenses when determining whether a \glossterm{strike} gets a \glossterm{critical hit} against you instead of a normal hit.

        \ff[9]{Intrinsic Aquamancy} You learn an additional spell with the \abilitytag{Attune} tag from the \sphere{Aquamancy} mystic sphere.
        If that spell is a \glossterm{deep attunement}, you only need to spend one \glossterm{attunement point} to attune to it.
        Otherwise, you can attune to it without spending an \glossterm{attunement point}.
        When you gain access to new spell ranks, you can change which spell you know with this ability.

        \ff[15]{Aquatic Propulsion+} The distance you can push yourself with your \textit{aquatic propulsion} ability increases to 20 feet.

        \ff[21]{Partially Liquid+} You are immune to \glossterm{critical hits} from \glossterm{strikes}.
    \end{feat}

    \begin{feat}{Sphere Focus: Astromancy}{Casting, Magical}
        \featpre Access to the \sphere{Astromancy} \glossterm{mystic sphere}.

        \ff[1]{Astral Spell Transit} You gain a \plus30 foot bonus to your range with abilities from the \sphere{Astromancy} \glossterm{mystic sphere}.

        \ff[3]{Efficient Transit} You learn how to transport creatures and objects more smoothly between planes.
        The \glossterm{difficulty value} to hear noise caused by creatures and objects you \glossterm{teleport} increases by 10 (see \pcref{Teleportation Noise}).
        In addition, whenever you fail to teleport a creature or object due to specifying an invalid destination, you may immediately specify a different destination for that ability.
        If that second destination is also invalid, the ability fails normally.

        \ff[9]{Intrinsic Astromancy} You learn an additional spell with the \abilitytag{Attune} tag from the \sphere{Astromancy} mystic sphere.
        If that spell is a \glossterm{deep attunement}, you only need to spend one \glossterm{attunement point} to attune to it.
        Otherwise, you can attune to it without spending an \glossterm{attunement point}.
        When you gain access to new spell ranks, you can change which spell you know with this ability.

        \ff[15]{Astral Spell Transit+} The range increase from your \textit{astral spell transit} ability increases to \plus60 feet.

        \ff[21]{Astral Spell Transit+} The range increase from your \textit{astral spell transit} ability increases to \plus90 feet.
        In addition, you gain a \plus30 foot bonus to your range with all \glossterm{magical} abilities that are not from the Astromancy mystic sphere.
    \end{feat}

    \begin{feat}{Sphere Focus: Channel Divinity}{Casting, Magical}
        \featpre Access to the \sphere{Channel Divinity} \glossterm{mystic sphere}.

        \ff[1]{Aspect of Divinity} You gain a \glossterm{magic bonus} equal to your level to your \glossterm{hit points} and \glossterm{damage resistance}.

        \ff[3]{Font of Divinity} Choose a spell with the \abilitytag{Attune} tag from the \sphere{channel divinity} mystic sphere that is not a \glossterm{deep attunement}.
        When you attune to that spell, you may also choose one \glossterm{ally} within \medrange.
        That ally can also choose to attune to the spell, and you both gain its benefits.
        When you stop attuning to that spell, your ally is also forced to stop attuning to the spell.

        Since you cannot attune to the same spell more than once, you cannot share the effects of the spell with more than one ally at a time in this way.
        You can change which spell you choose with this ability whenever you learn a new spell or gain access to a new spell rank.

        \ff[9]{Intrinsic Channeling} You learn an additional spell with the \abilitytag{Attune} tag from the \sphere{Channel Divinity} mystic sphere.
        If that spell is a \glossterm{deep attunement}, you only need to spend one \glossterm{attunement point} to attune to it.
        Otherwise, you can attune to it without spending an \glossterm{attunement point}.
        When you gain access to new spell ranks, you can change which spell you know with this ability.

        \ff[15]{Greater Aspect of Divinity} The bonus from your \textit{aspect of divinity} ability increases to twice your level.

        \ff[21]{Font of Divinity+} When you use your \textit{font of divinity} ability, you may choose up to five allies within \medrange instead of only one.
        In addition, you can choose an additional spell with that ability.
    \end{feat}

    \begin{feat}{Sphere Focus: Chronomancy}{Casting, Magical}
        \featpre Access to the \sphere{Chronomancy} \glossterm{mystic sphere}.

        \ff[1]{Thief of Time} Whenever you cast a spell from the \sphere{Chronomancy} mystic sphere that deals damage to another creature or causes another creature to be \slowed, you \glossterm{briefly} gain a \plus10 foot bonus to your land speed.

        \ff[3]{Accelerated Mind} You can perform primarily mental tasks more quickly as normal.
        \glossterm{Mundane} mental actions that would normally take a \glossterm{standard action} instead take a \glossterm{minor action}.
        Long-term activities can be done twice as quickly as normal.
        This includes reading books, searching areas, and other similar activities.
        It does not affect spellcasting, performing rituals, or other similar magical abilities.

        \ff[9]{Intrinsic Chronomancy} You learn an additional spell with the \abilitytag{Attune} tag from the \sphere{Chronomancy} mystic sphere.
        If that spell is a \glossterm{deep attunement}, you only need to spend one \glossterm{attunement point} to attune to it.
        Otherwise, you can attune to it without spending an \glossterm{attunement point}.
        When you gain access to new spell ranks, you can change which spell you know with this ability.

        \ff[15]{Thief of Time+} The bonus from your \textit{thief of time} ability increases to \plus20 feet.

        \ff[21]{Accelerated Mind+} Once per round, you can perform a purely mental action that would normally require a \glossterm{minor action} as a \glossterm{free action}.
        This can be used to sustain spells or perform other magical feats.
        However, you cannot take the same minor action twice in the same round.
        In addition, the speed increase for long-term tasks from your \textit{accelerated mind} ability increases to five times normal speed.
    \end{feat}

    \begin{feat}{Sphere Focus: Cryomancy}{Casting, Magical}
        \featpre Access to the \sphere{Cryomancy} \glossterm{mystic sphere}.

        \ff[1]{Icy Carapace} You learn the \spell{icy shell} spell.
        In addition, the number of layers you can create with that spell increases by one.

        \ff[3]{Cold Tolerance} You are \trait{impervious} to cold damage.

        \ff[3]{Frozen Blood} You are immune to \glossterm{diseases}.

        \ff[9]{Intrinsic Cryomancy} You learn an additional spell with the \abilitytag{Attune} tag from the \sphere{Cryomancy} mystic sphere.
        If that spell is a \glossterm{deep attunement}, you only need to spend one \glossterm{attunement point} to attune to it.
        Otherwise, you can attune to it without spending an \glossterm{attunement point}.
        When you gain access to new spell ranks, you can change which spell you know with this ability.

        \ff[15]{Cold Immunity} You are \trait{immune} to cold damage.

        \ff[15]{Frozen Blood+} You are immune to \glossterm{poisons}.

        \ff[21]{Icy Carapace+} The number of bonus layers you gain from your \textit{icy carapace} ability increases to three.
    \end{feat}

    \begin{feat}{Sphere Focus: Electromancy}{Casting, Magical}
        \featpre Access to the \sphere{Electromancy} \glossterm{mystic sphere}.

        \ff[1]{Magnetic Attraction} You gain a \plus1 \glossterm{accuracy} bonus against any creature that is wearing metal armor or otherwise carrying or composed of a significant amount of metal.

        \ff[3]{Electricity Tolerance} You are \trait{impervious} to electricity damage.

        \ff[3]{Static Buildup} Whenever you use the \ability{sprint} ability, you gain a \plus1 accuracy bonus with your next spell from the \sphere{Electromancy} mystic sphere.
        This bonus disappears if not used before the end of the next round.

        \ff[9]{Intrinsic Electromancy} You learn an additional spell with the \abilitytag{Attune} tag from the \sphere{Electromancy} mystic sphere.
        If that spell is a \glossterm{deep attunement}, you only need to spend one \glossterm{attunement point} to attune to it.
        Otherwise, you can attune to it without spending an \glossterm{attunement point}.
        When you gain access to new spell ranks, you can change which spell you know with this ability.

        \ff[15]{Electricity Immunity} You are \trait{immune} to electricity damage.

        \ff[15]{Static Buildup+} The bonus from your \textit{static buildup} ability increases to \plus2.

        \ff[21]{Magnetic Attraction+} The bonus from your \textit{magnetic attraction} ability increases to \plus2.
        In addition, it works against any creature that has metal anywhere on them, even as small as a single ring.
    \end{feat}

    \begin{feat}{Sphere Focus: Enchantment}{Casting, Magical}
        \featpre Access to the \sphere{Enchantment} \glossterm{mystic sphere}.

        \ff[1]{Hidden Influence} You gain a \plus4 accuracy bonus with spells from the Enchantment mystic sphere against \unaware creatures.
        In addition, the \glossterm{difficulty value} to observe the effects of your \abilitytag{Emotion} abilities with the Awareness and Social Insight skills increases by 10 (see \pcref{Notice Subtle Effects}, and \pcref{Discern Enchantment}).

        \ff[3]{Mind Fragments} When you use \abilitytag{Compulsion} and \abilitytag{Emotion} abilities, you can affect creatures that are immune to those abilities due to not having a mind.
        You take a \minus5 accuracy penalty on attacks against such creatures.
        This does not allow you to affect creatures who are immune to those abilities for other reasons.

        \ff[9]{Intrinsic Enchantment} You learn an additional spell with the \abilitytag{Attune} tag from the \sphere{Enchantment} mystic sphere.
        If that spell is a \glossterm{deep attunement}, you only need to spend one \glossterm{attunement point} to attune to it.
        Otherwise, you can attune to it without spending an \glossterm{attunement point}.
        When you gain access to new spell ranks, you can change which spell you know with this ability.

        \ff[15]{Mind Fragments+} The accuracy penalty from your \textit{mind fragments} ability is removed.

        \ff[21]{Hidden Influence+} The accuracy bonus from your \textit{subtle influence} ability also applies against \partiallyunaware creatures.
        In addition, the \glossterm{difficulty value} increase from that ability increases to \plus20.
    \end{feat}

    \begin{feat}{Sphere Focus: Fabrication}{Casting, Magical}
        \featpre Access to the \sphere{Fabrication} \glossterm{mystic sphere}.

        \ff[1]{Efficient Forger} You learn the \spell{forge} spell from the Fabrication mystic sphere.
        In addition, you can cast that spell without the \abilitytag{Attune} tag.
        If you do, its effects last until you cast it again.

        \ff[3]{Fabricate Trinket+} The maximum size of the trinket you can create with your \textit{fabricate trinket} cantrip increases by one size category.
        You can cast it with the \abilitytag{Sustain} (minor) tag instead of the \abilitytag{Attune} tag.
        In addition, you can use your \glossterm{power} in place of your Craft skill to create items with spells from the \sphere{Fabrication} mystic sphere.

        \ff[9]{Intrinsic Fabrication} You learn an additional spell with the \abilitytag{Attune} tag from the \sphere{Fabrication} mystic sphere.
        If that spell is a \glossterm{deep attunement}, you only need to spend one \glossterm{attunement point} to attune to it.
        Otherwise, you can attune to it without spending an \glossterm{attunement point}.
        When you gain access to new spell ranks, you can change which spell you know with this ability.

        \ff[15]{Fabricated Armaments} You gain a \plus2 bonus to \glossterm{accuracy} with \glossterm{strikes} using weapons you created with spells from the Fabrication mystic sphere.
        In addition, you gain a \plus1 bonus to the Armor defense provided by body armor from the Fabrication mystic sphere.

        \ff[21]{Fabricate Trinket+} The size increase from your \textit{greater fabricate trinket} ability increases to two size categories.
        In addition, when you cast the \textit{fabricate trinket} ability, you can cast it with the \abilitytag{Sustain} (free) tag instead of the \abilitytag{Attune} tag.
    \end{feat}

    \begin{feat}{Sphere Focus: Photomancy}{Casting, Magical}
        \featpre Access to the \sphere{Photomancy} \glossterm{mystic sphere}.

        \ff[1]{Scour Vision} Whenever you cause a creature to become \dazzled with a spell from the \sphere{Photomancy} \glossterm{mystic sphere}, you \glossterm{briefly} become \trait{invisible} to that creature.
        This does not affect its ability to see other creatures.

        \ff[3]{Light Tolerance} You are immune to being \dazzled and \blinded.

        \ff[9]{Intrinsic Photomancy} You learn an additional spell with the \abilitytag{Attune} tag from the \sphere{Photomancy} mystic sphere.
        If that spell is a \glossterm{deep attunement}, you only need to spend one \glossterm{attunement point} to attune to it.
        Otherwise, you can attune to it without spending an \glossterm{attunement point}.
        When you gain access to new spell ranks, you can change which spell you know with this ability.

        \ff[15]{Piercing Sight} You can see through solid objects and spell effects up to one inch thick.
        You can perceive the existence of obstacles thinner than that, but they do not inhibit your sight.
        This does not grant you \glossterm{line of effect} to anything you see in this way, since the obstacle still exists.

        \ff[21]{Scour Vision+} Your \textit{scour vision} ability also works whenever you deal damage to a creature with a spell from the \sphere{Photomancy} mystic sphere.
    \end{feat}

    \begin{feat}{Sphere Focus: Polymorph}{Casting, Magical}
        \featpre Access to the \sphere{Polymorph} \glossterm{mystic sphere}.

        \ff[1]{Malleable Flesh} You gain a \plus4 bonus to the Flexibility skill.
        In addition, you gain a \plus4 bonus to your defenses when determining whether a \glossterm{strike} gets a \glossterm{critical hit} against you instead of a normal hit.

        \ff[3]{Augmented Body} You can use the \textit{augmented body} ability whenever you finish a \glossterm{short rest}.
        \begin{attuneability}{Augmented Body}{\abilitytag{Attune} (deep)}
            Choose a physical attribute: Strength, Dexterity, or Constitution.
            You gain a \plus1 \glossterm{magic bonus} to that attribute, up to a maximum of 4.
        \end{attuneability}

        \ff[9]{Intrinsic Polymorph} You learn an additional spell with the \abilitytag{Attune} tag from the \sphere{Polymorph} mystic sphere.
        If that spell is a \glossterm{deep attunement}, you only need to spend one \glossterm{attunement point} to attune to it.
        Otherwise, you can attune to it without spending an \glossterm{attunement point}.
        When you gain access to new spell ranks, you can change which spell you know with this ability.

        \ff[15]{Malleable Flesh+} You are immune to \glossterm{critical hits} from \glossterm{strikes}.

        \ff[21]{Augmented Body+} The bonus from your \textit{augmented body} increases to \plus2.
        In addition, the attribute maximum increases to 5.
    \end{feat}

    \begin{feat}{Sphere Focus: Prayer}{Casting, Magical}
        \featpre Access to the \sphere{Prayer} \glossterm{mystic sphere}.

        \ff[1]{Shared Boon} When you cast a targeted spell from the \sphere{Prayer} mystic sphere that does not have the \abilitytag{Sustain} or \abilitytag{Attune} tags on a single \glossterm{ally}, you can target an additional ally within range.

        \ff[3]{Sustained Blessing} Whenever you cast a spell from the \sphere{Prayer} mystic sphere with the \abilitytag{Attune} (target) tag that is not a \glossterm{deep attunement}, you can choose to replace that tag with the \abilitytag{Sustain} (minor) tag.
        When you do, you must cast the spell as a \glossterm{standard action}, even if it could normally be cast as a \glossterm{minor action}.
        You can only apply this ability to one spell at a time.

        \ff[9]{Intrinsic Blessing} You learn an additional spell with the \abilitytag{Attune} tag from the \sphere{Prayer} mystic sphere.
        If that spell is a \glossterm{deep attunement}, you only need to spend one \glossterm{attunement point} to attune to it.
        Otherwise, you can attune to it without spending an \glossterm{attunement point}.
        When you gain access to new spell ranks, you can change which spell you know with this ability.

        \ff[15]{Sustained Blessing+} When you used your \textit{sustained blessing} ability, you can replace the tag with \abilitytag{Sustain} (free) instead of \abilitytag{Sustain} (minor).
        You can still only sustain one spell in this way at a time.

        \ff[21]{Shared Boon+} When you use your \textit{shared boon} ability, you can also target yourself or a third ally within range.
    \end{feat}

    \begin{feat}{Sphere Focus: Pyromancy}{Casting, Magical}
        \featpre Access to the \sphere{Pyromancy} \glossterm{mystic sphere}.

        \ff[1]{Spreading Flame} Whenever you cast a spell from the \sphere{Pyromancy} \glossterm{mystic sphere}, you can double its area.
        If you do, your \glossterm{power} with that spell is halved.

        \ff[3]{Fire Tolerance} You are \trait{impervious} to fire damage.

        \ff[3]{Friendly Fire} Whenever you deal fire damage to your \glossterm{allies}, you deal half damage.

        \ff[9]{Intrinsic Pyromancy} You learn an additional spell with the \abilitytag{Attune} tag from the \sphere{Pyromancy} mystic sphere.
        If that spell is a \glossterm{deep attunement}, you only need to spend one \glossterm{attunement point} to attune to it.
        Otherwise, you can attune to it without spending an \glossterm{attunement point}.
        When you gain access to new spell ranks, you can change which spell you know with this ability.

        \ff[15]{Fire Immunity} You are \trait{immune} to fire damage.

        \ff[15]{Friendly Fire+} Your \glossterm{allies} are immune to fire damage from your abilities.

        \ff[21]{Spreading Flame+} When you use your \textit{spreading flame} ability, you do not reduce your power with that spell.
    \end{feat}

    \begin{feat}{Sphere Focus: Revelation}{Casting, Magical}
        \featpre Access to the \sphere{Revelation} \glossterm{mystic sphere}.

        \ff[1]{Oracle} Once per \glossterm{long rest}, you can gain the benefit of the \ritual{augury} ritual as a standard action.
        Using this ability does not increase your \glossterm{fatigue level}.

        \ff[3]{Blindsight} You gain \trait{blindsense} with a 60 foot range (see \pcref{Blindsense}).
        If you already have blindsense, you increase its range by 60 feet.
        In addition, you gain \trait{blindsight} with a 15 foot range, allowing you to see without light (see \pcref{Blindsight}).
        If you already have blindsight, the range of your blindsight increases by 15 feet.

        \ff[9]{Intrinsic Revelation} You learn an additional spell with the \abilitytag{Attune} tag from the \sphere{Revelation} mystic sphere.
        If that spell is a \glossterm{deep attunement}, you only need to spend one \glossterm{attunement point} to attune to it.
        Otherwise, you can attune to it without spending an \glossterm{attunement point}.
        When you gain access to new spell ranks, you can change which spell you know with this ability.

        \ff[15]{Blindsight+} The range of your \trait{blindsense} ability increases by 60 feet.
        In addition, the range of your \trait{blindsight} ability increases by 15 feet.

        \ff[21]{Oracle+} You can use your \textit{oracle} ability once per \glossterm{short rest} instead of once per long rest.
    \end{feat}

    \begin{feat}{Sphere Focus: Summoning}{Casting, Magical}
        \featpre Access to the \sphere{Summoning} \glossterm{mystic sphere}.

        \ff[1]{Empowered Summons} Once per round, when a creature you create with the \sphere{Summoning} \glossterm{mystic sphere} hits with a \glossterm{strike}, you can add half your \glossterm{power} to the damage dealt.

        \ff[3]{Resummon} You can use the \textit{resummon} ability as a \glossterm{standard action}.
        \begin{activeability}{Resummon}
            \rankline
            After you use this ability, you increase your \glossterm{fatigue level} by two, and you cannot use it again until you take a short rest.

            Choose a spell from the \sphere{Summoning} mystic sphere that you are still \glossterm{attuned} to, even if all creatures created by that spell are dead.
            All effects of that spell, including any active creatures, immediately end.
            Then, you recreate all creatures from that spell as if you had just cast it.
            Those creatures do not act until the next round.

            \rankline
            \featlevel{9} You can use this ability any number of times without taking a short rest.
            \featlevel{15} You can use this ability as a \glossterm{minor action}.
            \featlevel{21} Using this ability only increases your fatigue level by one.
        \end{activeability}

        \ff[9]{Intrinsic Summoning} You learn an additional spell with the \abilitytag{Attune} tag from the \sphere{Summoning} mystic sphere.
        If that spell is a \glossterm{deep attunement}, you only need to spend one \glossterm{attunement point} to attune to it.
        Otherwise, you can attune to it without spending an \glossterm{attunement point}.
        When you gain access to new spell ranks, you can change which spell you know with this ability.

        \ff[15]{Generosity} Whenever you \glossterm{attune} to an item or spell spell that is not a \glossterm{deep attunement}, you may use this ability.
        If you do, all creatures you create with the \sphere{Summoning} mystic sphere gain the benefit of all \glossterm{magic bonuses} from the chosen item or spell.
        It still affects you normally.
        You may only choose one attunement with this ability at a time.

        \ff[21]{Empowered Summons+} The damage from your \textit{empowered summons} ability increases to be equal to your power.
    \end{feat}

    \begin{feat}{Sphere Focus: Telekinesis}{Casting, Magical}
        \featpre Access to the \sphere{Telekinesis} \glossterm{mystic sphere}.

        \ff[1]{Efficient Hand} You can cast the \spell{distant hand} \glossterm{cantrip} as a \glossterm{minor action}, and you can \glossterm{sustain} it as a \glossterm{free action}.
        In addition, the distance you can move the object increases to 30 feet, as if you had that spell's rank 4 upgrade.

        \ff[3]{Telekinetic Strike} You can use the \textit{telekinetic strike} ability as a standard action.
        \begin{activeability}{Telekinetic Strike}[Magical]
            \rankline
            Choose a weapon you are proficient with and currently controlling using the \spell{distant hand} cantrip.
            Make a melee \glossterm{strike} with a \plus1 accuracy bonus using that weapon.
            Because this is a \glossterm{magical} ability, you use your Willpower to determine your damage dice instead of your Strength (see \pcref{Dice Bonuses From Attributes}).

            \rankline
            \featlevel{9} The accuracy bonus increases to \plus2.
            \featlevel{15} The accuracy bonus increases to \plus3.
            \featlevel{21} The accuracy bonus increases to \plus4.
        \end{activeability}

        \ff[9]{Intrinsic Telekinesis} You learn an additional spell with the \abilitytag{Attune} tag from the \sphere{Telekinesis} mystic sphere.
        If that spell is a \glossterm{deep attunement}, you only need to spend one \glossterm{attunement point} to attune to it.
        Otherwise, you can attune to it without spending an \glossterm{attunement point}.
        When you gain access to new spell ranks, you can change which spell you know with this ability.

        \ff[15]{Multiple Hands} When you use the \spell{distant hand} cantrip, you can create two hands instead of one.
        This can allow you to hold multiple objects simultaneously, though you must still take separate actions to manipulate those objects in any way more complicated than simple movement.
        Alternately, you can hold a single object with two hands instead of one.
        This gives you a \plus1 bonus to your Willpower for the purpose of determining your \glossterm{weight limits}, and allows you to make attacks with \glossterm{heavy weapons} using your \ability{telekinetic strike} ability.

        \ff[21]{Agile Levitation} You gain a \glossterm{fly speed} equal to the \glossterm{base speed} for your size with a maximum height of 30 feet (see \pcref{Flight}).
        As a \glossterm{free action}, you can increase your \glossterm{fatigue level} by one to ignore this height limit until the end of the round.
    \end{feat}

    \begin{feat}{Sphere Focus: Terramancy}{Casting, Magical}
        \featpre Access to the \sphere{Terramancy} \glossterm{mystic sphere}.

        \ff[1]{Rocky Carapace} You learn the \spell{rocky shell} spell.
        In addition, the number of layers you can create with that spell increases by one.

        \ff[3]{Earthen Alloys} You may treat iron, steel, and worked stone as if they were stone for the purpose of spells from the \sphere{Terramancy} \glossterm{mystic sphere}.

        \ff[9]{Intrinsic Terramancy} You learn an additional spell with the \abilitytag{Attune} tag from the \sphere{Terramancy} mystic sphere.
        If that spell is a \glossterm{deep attunement}, you only need to spend one \glossterm{attunement point} to attune to it.
        Otherwise, you can attune to it without spending an \glossterm{attunement point}.
        When you gain access to new spell ranks, you can change which spell you know with this ability.

        \ff[15]{Earthen Alloys+} You may treat sand, glass, and all kinds of metal except for cold iron as if they were stone for the purpose of spells from the \sphere{Terramancy} \glossterm{mystic sphere}.

        \ff[21]{Rocky Carapace+} The number of bonus layers you gain from your \textit{rocky carapace} ability increases to three.
    \end{feat}

    \begin{feat}{Sphere Focus: Thaumaturgy}{Casting, Magical}
        \featpre Access to the \sphere{Thaumaturgy} \glossterm{mystic sphere}.

        \ff[1]{Mystic Power} You gain a bonus to your \glossterm{power} equal to your maximum spell rank with the \sphere{Thaumaturgy} mystic sphere.

        \ff[3]{Countermagic} You can use the \textit{countermagic} ability as a standard action.
        \begin{activeability}{Countermagic}[\abilitytag{Swift}]
            \rankline
            Choose a creature within \rngmed range of you.
            If the target uses a \glossterm{magical} ability as a standard action this round, you can make a \glossterm{reactive attack} vs. Mental against that creature.
            On a hit, the target's ability has no effect.

            If a creature is capable of using multiple abilities during a single phase, only the first ability it uses can be countered.
            This is common for \glossterm{elite} monsters.

            \rankline
            \featlevel{9} You gain a \plus2 bonus to accuracy with the attack.
            \featlevel{15} The accuracy bonus increases to \plus4.
            \featlevel{21} The accuracy bonus increases to \plus6.
        \end{activeability}

        \ff[9]{Intrinsic Thaumaturgy} You learn an additional spell with the \abilitytag{Attune} tag from the \sphere{Thaumaturgy} mystic sphere.
        If that spell is a \glossterm{deep attunement}, you only need to spend one \glossterm{attunement point} to attune to it.
        Otherwise, you can attune to it without spending an \glossterm{attunement point}.
        When you gain access to new spell ranks, you can change which spell you know with this ability.

        \ff[15]{Instant Countermagic} You can use your \textit{countermagic} ability as a \glossterm{minor action}.
        When you do, you \glossterm{briefly} cannot use that ability again.
        If you successfully counter a spell with that ability after using it as a minor action, you increase your \glossterm{fatigue level} by two.

        \ff[21]{Mystic Power+} The bonus from your \textit{mystic power} ability increases to \plus12.
    \end{feat}

    \begin{feat}{Sphere Focus: Toxicology}{Casting, Magical}
        \featpre Access to the \sphere{Toxicology} \glossterm{mystic sphere}.

        \ff[1]{Poisonous Blood} Whenever you become poisoned, such as by drinking poison or from an enemy's attack, your body naturally repurposes the poison.
        The poison has no effect on you, but your body gains a dose of natural poison.
        Whenever a creature makes you lose \glossterm{hit points} with a \glossterm{melee} strike using a non-Long weapon, you make a \glossterm{reactive attack} vs. Fortitude against the attacking creature.
        On a hit, it becomes \glossterm{poisoned} by your choice of one of the poisons you store with this ability.
        This expends the dose of that poison.

        Poison that you carry in your body with this ability automatically decays after 24 hours, regardless of the normal duration of the poison.
        You can store up to 3 doses in your body with this ability at a time.

        \ff[3]{Cleanse Toxins} You can use the \textit{cleanse toxins} ability as a standard action.
        \begin{activeability}{Cleanse Toxins}
            \label{Cleanse Toxins}
            You remove all \glossterm{poisons} and \glossterm{diseases} affecting yourself or one \glossterm{ally} within \medrange.
        \end{activeability}

        \ff[9]{Intrinsic Toxicology} You learn an additional spell with the \abilitytag{Attune} tag from the \sphere{Toxicology} mystic sphere.
        If that spell is a \glossterm{deep attunement}, you only need to spend one \glossterm{attunement point} to attune to it.
        Otherwise, you can attune to it without spending an \glossterm{attunement point}.
        When you gain access to new spell ranks, you can change which spell you know with this ability.

        \ff[15]{Poisonous Venom} You can also inflict the poison you store with your \textit{poisonous blood} ability on other creatures with attacks.
        Once per round, when you make a creature lose \glossterm{hit points} with a \glossterm{natural weapon} or a spell from the \sphere{toxicology} mystic sphere, you can cause the creature to become poisoned with your choice of one of the poisons you store.
        This expends the dose of that poison.

        \ff[21]{Poisonous Blood+} You can store up to 10 poison doses with your \textit{innate poison} ability.
    \end{feat}

    \begin{feat}{Sphere Focus: Umbramancy}{Casting, Magical}
        \featpre Access to the \sphere{Umbramancy} \glossterm{mystic sphere}.

        \ff[1]{Darksight} You gain \trait{darkvision} with a 90 foot range, allowing you to see in complete darkness (see \pcref{Darkvision}).
        If you already have darkvision, you increase its range by 90 feet.
        In addition, your darkvision is not disabled in \glossterm{bright illumination}, though it is still disabled in \glossterm{brilliant illumination}.

        \ff[3]{Lightbane} You can cast the \spell{suppress light} \glossterm{cantrip} from the Umbramancy mystic sphere as a \glossterm{minor action}, and you can \glossterm{sustain} it as a \glossterm{free action}.

        \ff[9]{Intrinsic Umbramancy} You learn an additional spell with the \abilitytag{Attune} tag from the \sphere{Umbramancy} mystic sphere.
        If that spell is a \glossterm{deep attunement}, you only need to spend one \glossterm{attunement point} to attune to it.
        Otherwise, you can attune to it without spending an \glossterm{attunement point}.
        When you gain access to new spell ranks, you can change which spell you know with this ability.

        \ff[15]{Darksight+} The range of your \trait{darkvision} increases by 90 feet.
        In addition, your darkvision is not disabled in \glossterm{brilliant illumination} or when you become \dazzled.

        \ff[21]{Lightbane+} When you cast your \textit{suppress light} cantrip, you can choose to completely block all light in the area instead of dimming it to be \glossterm{shadowy illumination}.
        If you do, the maximum area is reduced to a \medarea radius, and you \glossterm{briefly} cannot cast it in this way again.
    \end{feat}

    \begin{feat}{Sphere Focus: Verdamancy}{Casting, Magical}
        \featpre Access to the \sphere{Verdamancy} \glossterm{mystic sphere}.

        \ff[1]{Verdant Allies} Your speed is not reduced when moving in heavy \glossterm{undergrowth}.
        In addition, you can ignore \glossterm{cover} and \glossterm{concealment} from plants whenever doing so would be beneficial to you, as the plants move out of the way to help you.
        This prevents you from suffering a miss chance on your attacks, and also prevents creatures from using cover or concealment from plants to hide from you.

        \ff[3]{Residual Undergrowth} Whenever you cast a spell from the \textit{verdamancy} sphere, you may create either \glossterm{light undergrowth} or \glossterm{heavy undergrowth} in the area of the spell.
        The undergrowth persists \glossterm{briefly}.
        It appears on the ground within the area for area spells, or on the ground in all spaces occupied by each target of the spell for targeted spells.

        \ff[9]{Intrinsic Verdamancy} You learn an additional spell with the \abilitytag{Attune} tag from the \sphere{Verdamancy} mystic sphere.
        If that spell is a \glossterm{deep attunement}, you only need to spend one \glossterm{attunement point} to attune to it.
        Otherwise, you can attune to it without spending an \glossterm{attunement point}.
        When you gain access to new spell ranks, you can change which spell you know with this ability.

        \ff[15]{Verdant Army} Your \glossterm{allies} within a \largearea radius \glossterm{emanation} from you also gain the benefit of your \textit{verdant allies} ability.

        \ff[21]{Residual Undergrowth+} Whenever you create undergrowth with your \textit{residual undergrowth} ability, you can \glossterm{sustain} that undergrowth as a \glossterm{minor action} instead of having it last \glossterm{briefly}.
    \end{feat}

    \begin{feat}{Sphere Focus: Vivimancy}{Casting, Magical}
        \featpre Access to the \sphere{Vivimancy} \glossterm{mystic sphere}.

        % Should this be more specific, like "animates"?
        \ff[1]{Hidden Life} You can treat nonliving creatures other than undead as if they were living creatures for the purpose of your abilities from the \sphere{Vivimancy} \glossterm{mystic sphere}.

        \ff[3]{Personal Vitality} You gain a bonus equal to your level to your \glossterm{hit points}.
        In addition, you gain a \plus1 bonus to your Fortitude defense.

        \ff[9]{Intrinsic Vivimancy} You learn an additional spell with the \abilitytag{Attune} tag from the \sphere{Vivimancy} mystic sphere.
        If that spell is a \glossterm{deep attunement}, you only need to spend one \glossterm{attunement point} to attune to it.
        Otherwise, you can attune to it without spending an \glossterm{attunement point}.
        When you gain access to new spell ranks, you can change which spell you know with this ability.

        \ff[15]{Life Suppression} You are no longer considered a living creature for the purpose of attacks against you.
        This means that attacks which only affect living creatures have no effect against you.

        \ff[21]{Personal Vitality+} The bonus to Fortitude defense from your \textit{personal vitality} ability increases to \plus2.
        In addition, the hit point bonus increases to twice your level.
    \end{feat}

    \begin{feat}{Stealth Specialization}{Skill}
        \featpre Stealth as a trained skill.

        \ff[1]{Specialization} You gain a \plus3 bonus to the Stealth skill.

        \ff[3]{Mobile Stealth} Your penalties for moving while hiding are reduced by 5.
        This allows you to move at half speed without penalty.

        \ff[9]{Ambush the Unwary} You gain a \plus1 accuracy bonus against \unaware and \partiallyunaware creatures.

        \ff[15]{Specialization+} The bonus from your \textit{specialization} ability increases to \plus6.

        \ff[21]{Ambush the Unwary+} The bonus from your \textit{ambush the unwary} ability increases to \plus2.
    \end{feat}

    \begin{feat}{Survival Specialization}{Skill}
        \featpre Survival as a trained skill.

        \ff[1]{Specialization} You gain a \plus3 bonus to the Survival skill.

        \ff[3]{Terrain Tolerance} You ignore \glossterm{difficult terrain} and harmful natural terrain of any kind.
        If a skill check, such as Climb or Swim, would normally be required to move through the terrain, this ability does not help.

        \ff[9]{Planar Survival}[Magical] You are immune to damage and \glossterm{conditions} imposed by being on other planes.
        In addition, you gain a \plus4 bonus to checks and defenses related to planar effects, such as checks required to manipulate subjective gravity.

        \ff[15]{Specialization+} The bonus from your \textit{specialization} ability increases to \plus6.

        \ff[21]{Find the Path}[Magical] You can use the \textit{find the path} ability with ten minutes of concentration.
        \begin{attuneability}{Find the Path}{\abilitytag{Attune}}
            \rankline
            When you use this ability, you must unambiguously specify a location on the same plane as you.
            You know exactly what direction you must travel to reach your chosen destination by the most direct physical route.
            You are not always led in the exact direction of the destination -- if there is an impassable obstacle between the target and the destination, this ability will direct you around the obstacle, rather than through it.

            The guidance provided by this ability adjusts to match whatever your current physical capabilities are, including flight and other unusual movement modes.
            It does not consider teleportation spells or any other activated abilities you may have which could allow you to bypass physical obstacles.
            It does not see into the future, and changing circumstances may cause the most direct path to change over time.
            It also does not consider hostile creatures, traps, and other passable dangers which may endanger or slow progress.
        \end{attuneability}
    \end{feat}

    \begin{feat}{Swiftrunner}{General}
        \featpre Dexterity 3.

        \ff[1]{Sprinter} When you use the \textit{sprint} ability, you can move up to triple your movement speed.
        In addition, you gain a \plus2 bonus to your \glossterm{fatigue tolerance}.

        \ff[3]{Water Runner} During your movement with the \textit{sprint} ability, you can move on water and similar liquids as if they were solid ground.

        \ff[9]{Rapid Movement} You gain a \plus10 foot bonus to your land speed.

        \ff[15]{Cloud Runner} During your movement with the \textit{sprint} ability, you can move on clouds, fog, and similar gaseous substances as if they were solid ground.

        \ff[21]{Rapid Movement+} The speed bonus from your \textit{rapid movement} ability increases to \plus20 feet.
    \end{feat}

    \begin{feat}{Swim Specialization}{Skill}
        \featpre Swim as a trained skill.

        \ff[1]{Specialization} You gain a \plus3 bonus to the Swim skill.

        \ff[3]{Swim Speed} You gain a \glossterm{swim speed} 10 feet slower than the \glossterm{base speed} for your size.
        If you already have a swim speed, you gain a \plus10 foot bonus to your swim speed.
        A successful Swim check to move allows you to move a distance equal to your swim speed.

        \ff[9]{Swimming Blitz} You can use the \textit{sprint} ability during the \glossterm{movement phase} without increasing your \glossterm{fatigue level} if you swim for the entire duration of the movement.
        After you use this ability, you \glossterm{briefly} cannot use it again.

        \ff[15]{Specialization+} The bonus from your \textit{specialization} ability increases to \plus6.

        \ff[21]{Earth Swimmer} You can swim through loose earth and dirt as if it were water.
        Your swim speed in earth is 10 feet, regardless of any bonuses or penalties that would normally apply to your swim speed.
        In addition, you take a \minus4 penalty to \glossterm{accuracy} and your Armor and Reflex defenses while swimming in this way.
        The earth and dirt around you blocks line of sight and line of effect, so you usually cannot used ranged attacks of any kind.
    \end{feat}

    \begin{feat}{Telepath}{General, Magical}
        \featpre Intelligence and Willpower sum to at least 3

        \ff[1]{Telepathy} You gain \glossterm{telepathy} with a 120 foot range (see \pcref{Telepathy}).
        If you already have telepathy, the range of your telepathy increases by 120 feet.

        % This is very loosely a rank 2 spell effect??
        \ff[3]{Mental Assault} You can use the \textit{mental assault} ability as a standard action.
        \begin{activeability}{Mental Assault}[\abilitytag{Compulsion}]
            \rankline
            Make an attack vs. Mental against one creature within half the maximum range of your \glossterm{telepathy}.
            You use your full Intelligence or Willpower in place of half your Perception to determine your accuracy with this attack (see \pcref{Accuracy}).
            \crit Double damage as normal, and the target is dazed if it takes damage instead of if it loses hit points.
            \hit The target takes 1d8 + \glossterm{power} psychic damage.
            If it loses \glossterm{hit points} from this damage, it becomes \dazed as a \glossterm{condition}.

            \rankline
            You gain a \plus1 accuracy bonus and a \plus1d damage bonus at 6th level and every 3 levels thereafter.
        \end{activeability}

        \ff[9]{Read Mind}[Magical] You can use the \textit{read mind} ability as a standard action.
        \begin{sustainability}{Read Mind}{\abilitytag{Emotion}, \abilitytag{Subtle}, \abilitytag{Sustain} (standard)}
            \rankline
            Make an attack vs. Mental against a creature within half the maximum range of your \glossterm{telepathy}.
            You use your full Intelligence or Willpower in place of half your Perception to determine your accuracy with this attack (see \pcref{Accuracy}).
            Whether you hit or miss, you cannot attack the target with this ability again until it takes a \glossterm{short rest}.
            \hit You know the target's current thoughts and emotions.
            In addition to the obvious effects, this grants you a \plus5 bonus to Deception, Persuasion, Intimidate, and Social Insight attacks and checks against the target.
            This bonus does not stack with other effects that allow you access to the target's mind, such as \ability{read emotions}.
            You cannot directly search the target's mind for arbitrary thoughts or information.
            However, creatures often think about questions they are asked, and their thoughts may reveal much more than their words.
            \crit You can also delve through the target's mind to answer a specific question.
            You can pose a question to it mentally and search its mind to know the exact answer to that question.
            This takes five rounds of continuous concentration, and you can only get answers to one such question each time you use this ability.
            The process of searching a creature's mind in this way is no easier to notice than normal for a \abilitytag{Subtle} ability.

            \rankline
            You gain a \plus2 bonus to \glossterm{accuracy} with the attack at 9th level and every 3 levels thereafter.
        \end{sustainability}

        \ff[9]{Telepathy+} You can maintain mental channels with up to 5 creatures at once with your telepathy.
        You can send separate thoughts to each creature.

        \ff[15]{Mindsight} You can perfectly see intelligent creatures within the maximum range of your telepathy.
        This functions like \trait{blindsight}, except that it only allows you to see creatures with an Intelligence of 0 or higher.

        \ff[21]{Mental Domination} Whenever a creature that is dazed by your \textit{mental assault} ability reaches 0 hit points, you can \glossterm{attune} to this ability.
        When you do, that creature becomes \dominated by you as long as you maintain that attunement.
        As normal, you can only maintain one instance of this attunement at a time.

        \ff[21]{Telepathy+} The range of your telepathy increases by 120 feet.
    \end{feat}

    \begin{feat}{Toughness}{General}
        \featpre Constitution 2.

        \ff[1]{Fortified Body} You gain a \plus2 bonus to Fortitude defense.
        In addition, you can sleep while you have \glossterm{encumbrance} without penalty (see \pcref{Encumbrance}).

        \ff[3]{Durability} You gain a bonus equal to your level to your \glossterm{hit points}.

        \ff[9]{Unwavering} You are immune to being \dazed and \stunned.

        \ff[15]{Fortified Body+} The defense bonus from your \textit{fortified body} ability increases to \plus4.
        In addition, you need half the normal amount of rest and sleep each day to function normally.
        For example, a human would only need four hours of sleep per night.
        This does not reduce the time required for you to take a \glossterm{long rest}.

        \ff[21]{Durability+} The bonus from your \textit{durability} ability increases to twice your level.
    \end{feat}

    \begin{feat}{Twinhand Spellcaster}{Casting, Magical}
        \featpre Dexterity 2.

        \ff[1]{Twinhand Precision} You can always choose to use \glossterm{somatic components} to cast your spells (see \pcref{Casting Components}).
        As long as you have two \glossterm{free hands}, you gain a \plus1 \glossterm{accuracy} bonus with spells that you cast using \glossterm{somatic components}.

        \ff[3]{Freehand Implement} You can gain the benefits of up to two magical implements, such as staves or wands, without having to hold them in your hands.
        You must still have them on your person, such as in a pocket or strapped to your back, and you must still be attuned to them to gain their benefits.
        In addition, if your legacy item is an apparel item, you may choose both apparel and implement magic item effects for it.

        \ff[9]{Twinspell} You can use the \textit{twinspell} ability as a \glossterm{standard action}.
        \begin{activeability}{Twinspell}
            \rankline
            You can only use this ability if you have two \glossterm{free hands}.
            After you use this ability, you increase your \glossterm{fatigue level} by two.

            Choose two spells that you know.
            You cast both spells simultaneously, one with each hand.
            This gives the spells \glossterm{somatic components}, regardless of any other effects which would would normally prevent you from requiring somatic components.
            Both spells must affect completely different targets, with no overlap between their targets or areas (if any).
        \end{activeability}

        \ff[15]{Twinhand Precision+} The bonus from your \textit{twinhand precision} ability increases to \plus2.

        \ff[21]{Twinhand Precision+} The bonus from your \textit{twinhand precision} ability increases to \plus3.
    \end{feat}

    \begin{feat}{Two-Weapon Fighting}{Combat}
        \featpre Dexterity 2.

        \ff[1]{Dual-Wielding} You are considered to be dual-wielding while two of your hands are each wielding a non-shield weapon that you are proficient with.
        Each hand must hold a different weapon, rather than simply holding a heavy weapon in two hands.
        You can use \glossterm{natural weapons} to meet this condition, such as claws, as long as your hand is exclusively dedicated to using the natural weapon and not holding or being used for anything else.
        Several abilities from this feat only function while you are dual-wielding.

        \ff[1]{Offhand Freedom} You can use the \textit{offhand strike} ability as a \glossterm{free action} instead of as a \glossterm{minor action}.
        However, you cannot use the \textit{offhand strike} ability more than once per round.
        In addition, you can use the \textit{offhand strike} ability during any phase that you take a \glossterm{standard action}, regardless of whether that standard action causes you to make a \glossterm{strike}.

        \ff[3]{Two-Weapon Stance} At the start of each round, you can enter one of the stances below or change which stance you are in.
        You can maintain that stance as long as you are dual-wielding.
        \begin{itemize}
            \item Balanced: You gain a \plus1 bonus to your Armor defense.
            \item Deflecting Offhand: You gain a \plus3 bonus to your Armor defense.
                However, you cannot use the \ability{offhand strike} ability.
            \item Offhand Precision: When you use the \ability{offhand strike} ability, you can roll the attack roll twice and keep the higher result.
            \item Two Weapons As One: You gain a \plus2 bonus to your \glossterm{accuracy} with \glossterm{strikes}.
                However, you cannot use the \ability{offhand strike} ability.
        \end{itemize}

        \ff[9]{Dual Precision} You gain a \plus1 bonus to \glossterm{accuracy} with \glossterm{strikes} while dual-wielding.

        \ff[15]{Dual Precision+} The bonus from your \textit{dual precision} ability increases to \plus2.

        \ff[21]{Offhand Flurry} You can use the \textit{offhand strike} ability twice per round.
        If you are in a \textit{two-weapon stance} that would normally prevent you from using that ability, you can still use it once per round.
    \end{feat}

    \begin{feat}{Wardweaver}{Casting, Magical}
        \featpre Knowledge of a spell with the \abilitytag{Barrier} tag.

        \ff[1]{Hardened Barriers} Objects you create using abilities with the \abilitytag{Barrier} \glossterm{ability tag} gain a bonus equal to your \glossterm{power} to their \glossterm{damage resistance}.
        In addition, they treat all damage as being \glossterm{environmental damage}, so their damage resistance is never reduced (see \pcref{Environmental Damage}).
        For objects with multiple separate hit point values, such as walls, this bonus applies independently to each section.

        \ff[3]{Defensive Barriers} You gain a \plus1 bonus to your Armor defense.
        In addition, objects you create with \abilitytag{Barrier} abilities have minimum defenses equal to 5 \add half your level.
        If the object already has specific defenses listed, it gains a \plus2 bonus to those defenses.

        \ff[9]{Barrier Spell} You learn a spell with the \abilitytag{Barrier} ability tag from any \glossterm{mystic sphere} that your \glossterm{magic source} gives access to, even if you do not have access to that mystic sphere.
        When you gain access to new spell ranks, you can change which Barrier spell you know.

        \ff[15]{Defensive Barriers+} Your Armor defense bonus from your \textit{defensive barriers} ability increases to \plus2.
        In addition, you gain a \plus2 bonus to the defenses of objects you create with \abilitytag{Barrier} abilities.

        \ff[15]{Hardened Barriers+} The damage resistance bonus from your \textit{hardened barriers} ability increases to twice your \glossterm{power}.

        \ff[21]{Limitless Barriers} Using an ability with the \abilitytag{Barrier} ability tag does not briefly prevent you from using another ability with the \abilitytag{Barrier} ability tag.
    \end{feat}

    \begin{feat}{Weapon Focus}{Combat}
        \ff[1]{Focused Weapon} Choose one type of weapon, such as a broadsword.
        This is your focused weapon, and many abilities from this feat give you benefits with your focused weapon.

        \ff[1]{Perfect Strike} You can use the \textit{perfect strike} ability as a standard action.
        \begin{activeability}{Perfect Strike}
            \rankline
            Make a \glossterm{strike} using your focused weapon.
            You gain your choice of either a \plus1 accuracy bonus or a \plus2 damage bonus with the strike.

            \rankline
            \featlevel{6} The accuracy bonus increases to \plus2, and the damage bonus increases to \plus4.
            \featlevel{12} The accuracy bonus increases to \plus3, and the damage bonus increases to \plus8.
            \featlevel{18} The accuracy bonus increases to \plus4, and the damage bonus increases to \plus16.
        \end{activeability}

        \ff[3]{Firm Grip} You are immune to any effect which would steal focused your weapon or force you to drop it.
        This does not protect you from any other effects of that attack, such as damage to yourself.

        \ff[9]{True Focus} You gain a \plus1 accuracy bonus and a \plus2 power bonus with attacks using your focused weapon.

        \ff[15]{True Focus+} The accuracy bonus from your \textit{true focus} ability increases to \plus2, and the power bonus increases to \plus4.

        \ff[21]{True Focus+} The accuracy bonus from your \textit{true focus} ability increases to \plus3, and the power bonus increases to \plus8.
    \end{feat}

    \begin{feat}{Whirlwind Warrior}{Combat}
        \featpre Dexterity 2.

        \ff[1]{Unfettered Movement} During each phase, you may move through one creature's space during movement.
        After moving through that creature's space, other creatures block you as normal for the remainder of your movement.
        Certain unusual creatures that occupy their entire space, such as gelatinous cubes, may be immune to this ability.
        If you end your movement in a creature's space with this ability, you and that creature are \squeezing if you are no more than one size category larger or smaller than it.

        \ff[3]{Cyclone} You can use the \textit{cyclone} ability as a standard action.
        \begin{sustainability}{Cyclone}{\abilitytag{Sustain} (standard)}
            \rankline
            When you use this ability, make a melee \glossterm{strike} with a slashing weapon.
            Your \glossterm{power} with the strike is halved.
            The strike targets any number of creatures adjacent to you.
            Whenever you sustain this ability, you can move up to half your speed and make a melee \glossterm{strike} with a slashing weapon.
            The strike targets any number of creatures adjacent to you at any point during your movement.

            \rankline
            \featlevel{9} You gain a \plus1 accuracy bonus with the strike.
            \featlevel{15} The accuracy bonus increases to \plus2.
            \featlevel{21} The accuracy bonus increases to \plus3.
            In addition, your \glossterm{power} with the strike is not halved.
        \end{sustainability}

        \ff[9]{Eye of the Storm} You take no penalties for \squeezing with other creatures.
        This does not reduce your penalties for squeezing in tight spaces.

        \ff[15]{Unfettered Movement+} You can move through spaces occupied by enemies as if they were unoccupied.

        \ff[21]{Eye of the Storm+} You gain a \plus2 bonus to Armor defense against creatures that you are \glossterm{squeezing} with.
    \end{feat}

\section{Other Feat Rules}

    \subsection{Retraining Feats}
        At every level, you can choose to retrain an old feat in exchange for a new feat.
