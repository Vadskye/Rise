\chapter{Feats}\label{Feats}

This chapter describes a set of optional rules that you can use in a campaign.
If you use these rules, characters gain feats which allow them to further specialize in specific areas, making characters more mechanically distinct from each other.
Feats also make the system more mechanically complex, so they are not necessarily enjoyable for all groups.

\section{Gaining Feats}
    There are two main ways you can use feats in your game.

    \parhead{Species Feat Only}
    One simple option is that characters gain a single feat at 1st level based on their species, and no other feats.
    This makes your choice of species more significant without dramatically increasing character complexity.

    \parhead{Feat Progression}
    If you want characters to be more complex and to have more powerful abilities, you can also use the feat progression system.
    With this variant, you gain a feat from your species at 1st level.
    In addition, you gain an additional feat at 3rd, 6th, and 9th level, for a total of four feats.
    You cannot gain the same feat twice.

    \subsection{Species Bonus Feats}\label{Species Bonus Feats}
        Each species grants a bonus feat at 1st level. Most species can only choose from a small group of feats, listed in the description of the species. A character must meet any prerequisites for these bonus feats, as normal.

        \parhead{Human} Any feat.

        \parhead{Dwarf} Any from the following list: \featref*{Blindfighter}, \featref*{Craft Specialization}, \featref*{Guardian}, \featref*{Iron Will}, \featref*{Martial Training}, \featref*{Regenerator}, \featref*{Toughness}.

        \parhead{Elf} Any Casting feat (see \pcref{Casting Feats}), or any from the following list: \featref*{Agility Specialization}, \featref*{Awareness Specialization}, \featref*{Sniper}, \featref*{Rapid Reaction}.

        \parhead{Gnome} Any Casting feat (see \pcref{Casting Feats}), or any from the following list: \featref*{Blindfighter}, \featref*{Craft Specialization}, \featref*{Stealth Specialization}, \featref*{Toughness}.

        \parhead{Half-Elf} Any Skill feat (see \pcref{Skill Feats}).

        \parhead{Half-Orc} Any Combat feat (see \pcref{Combat Feats}), or any from the following list: \featref*{Intimidate Specialization}, \featref*{Toughness}.

        \parhead{Halfling} Any from the following list: \featref*{Agility Specialization}, \featref*{Climb Specialization}, \featref*{Iron Will}, \featref*{Jump Specialization}, \featref*{Rapid Reaction}, \featref*{Stealth Specialization}.

        \subsubsection{Changing Species}
            In extraordinary cases, a creature may change its species.
            For example, the \ritual{reincarnation} ritual returns a creature to life as a different species.
            Regardless of its new species, the creature keeps its original species bonus feat.

\section{Feat Mechanics}

    \subsection{Prerequisites}
        Some feats have prerequisites.
        Unless you meet all of the prerequisites, you cannot take the feat.
        Prerequisites can include a minimum base attribute score, another feat or feats, a minimum level of training in a skill, or some other property of a character.
        A character can gain a feat at the same level at which they gain the prerequisite.

        A character who no longer meets all prerequisites for a feat loses all abilities from that feat.

    \subsection{Feat Tags}
        All feats are organized into different groups by tags.

        \parhead{General} General feats can have a wide variety of effects.
        They often grant new abilities or improve your defenses.

        \parhead{Combat} Combat feats improve your combat capabilities.
        They can increase the damage you deal, grant you new combat abilities, or improve your defenses.

        \parhead{Casting} Casting feats improve your spellcasting abilities.
        Casting feats are useless to characters who cannot cast spells.

        \parhead{Skill} Skill feats improve your skills.
        They can make you more likely to succeed with skill checks and grant you new abilities based on your skills.

        \parhead{Bloodline Feats} Some characters have traces of monstrous blood running in their veins.
        Most of those will never understand the full potential of their unusual heritage.
        Bloodline feats allow characters to explore those posibilities by gaining abilites related to their ancestry.
        You can only have one Bloodline feat.

        \parhead{Magical Feats}
        All abilities granted by feats with the Magical type are \glossterm{magical} in nature.
        Many feats are not entirely magical, but have specific effects that are magical.

\section{Feat Tables}

% Feat names must follow ``I have'' or ``I am (a)''.
\begin{longtablewrapper}
    \begin{longtable}{>{\lcol}p{11em} >{\lcol}p{12em} l >{\lcol}p{8em} >{\lcol}p{3em}}
        \lcaption{Feats}\\
        \tb{General Feats}\label{General Feats} & \tb{Prerequisites} & \tb{Benefits} & \tb{Feat Types} & \tb{Page} \\
        \featref{Celestial Heritage} & Non-evil & Gain aspects of celestial beings & Bloodline, Magical & \featpref{Celestial Heritage} \\
        \featref{Draconic Heritage}  & \tdash   & Gain aspects of draconic power   & Bloodline          & \featpref{Draconic Heritage}  \\
        \featref{Iron Will}          & Wil 1    & Increase mental resilience       & \tdash             & \featpref{Iron Will}          \\
        \featref{Null}               & Wil 2    & Become immune to magic           & \tdash             & \featpref{Null}               \\
        \featref{Precognition}       & Int 2    & React to future events           & \tdash             & \featpref{Precognition}       \\
        \featref{Regenerator}        & Con 2    & Heal wounds with inhuman speed   & \tdash             & \featpref{Regenerator}        \\
        \featref{Rapid Reaction}     & Dex 1    & Increase reaction speed          & \tdash             & \featpref{Rapid Reaction}     \\
        \featref{Spellwarped}        & Wil 1    & Gain limited spellcasting        & Magical            & \featpref{Spellwarped}        \\
        \featref{Swift}              & Dex 1    & Move more quickly                & \tdash             & \featpref{Swift}              \\
        \featref{Toughness}          & Con 1    & Increase physical fortitude      & \tdash             & \featpref{Toughness}          \\

        \tb{Skill Feats}\label{Skill Feats}        & \tb{Prerequisites}          & \tb{Benefits}                         & \tb{Feat Types} & \tb{Page}                                   \\
        \featref{Agility Specialization}           & Mastered Agility            & Improve use of chosen skill           & \tdash          & \featpref{Agility Specialization}           \\
        \featref{Awareness Specialization}         & Mastered Awareness          & Improve use of chosen skill           & \tdash          & \featpref{Awareness Specialization}         \\
        \featref{Climb Specialization}             & Mastered Climb              & Improve use of chosen skill           & \tdash          & \featpref{Climb Specialization}             \\
        \featref{Craft Specialization}             & Mastered Craft              & Improve use of chosen skill           & \tdash          & \featpref{Craft Specialization}             \\
        \featref{Creature Handling Specialization} & Mastered Creature Handling  & Improve use of chosen skill           & \tdash          & \featpref{Creature Handling Specialization} \\
        \featref{Deception Specialization}         & Mastered Deception          & Improve use of chosen skill           & \tdash          & \featpref{Deception Specialization}         \\
        \featref{Devices Specialization}           & Mastered Devices            & Improve use of chosen skill           & \tdash          & \featpref{Devices Specialization}           \\
        \featref{Disguise Specialization}          & Mastered Disguise           & Improve use of chosen skill           & \tdash          & \featpref{Disguise Specialization}          \\
        \featref{Endurance Specialization}         & Mastered Endurance          & Improve use of chosen skill           & \tdash          & \featpref{Endurance Specialization}         \\
        \featref{Flexibility Specialization}       & Mastered Flexibility        & Improve use of chosen skill           & \tdash          & \featpref{Flexibility Specialization}       \\
        \featref{Intimidate Specialization}        & Mastered Intimidate         & Improve use of chosen skill           & \tdash          & \featpref{Intimidate Specialization}        \\
        \featref{Herbalist}                        & Mastered Knowledge (nature) & Brew potions with natural ingredients & \tdash          & \featpref{Herbalist}                        \\
        \featref{Jump Specialization}              & Mastered Jump               & Improve use of chosen skill           & \tdash          & \featpref{Jump Specialization}              \\
        \featref{Knowledge Specialization}         & Mastered Knowledge          & Improve use of chosen skill           & \tdash          & \featpref{Knowledge Specialization}         \\
        \featref{Linguistics Specialization}       & Mastered Linguistics        & Improve use of chosen skill           & \tdash          & \featpref{Linguistics Specialization}       \\
        \featref{Medicine Specialization}          & Mastered Medicine           & Improve use of chosen skill           & \tdash          & \featpref{Medicine Specialization}          \\
        \featref{Perform Specialization}           & Mastered Perform            & Improve use of chosen skill           & \tdash          & \featpref{Perform Specialization}           \\
        \featref{Persuasion Specialization}        & Mastered Persuasion         & Improve use of chosen skill           & \tdash          & \featpref{Persuasion Specialization}        \\
        \featref{Ride Specialization}              & Mastered Ride               & Improve use of chosen skill           & \tdash          & \featpref{Ride Specialization}              \\
        \featref{Sleight of Hand Specialization}   & Mastered Sleight of Hand    & Improve use of chosen skill           & \tdash          & \featpref{Sleight of Hand Specialization}   \\
        \featref{Social Insight Specialization}    & Mastered Social Insight     & Improve use of chosen skill           & \tdash          & \featpref{Social Insight Specialization}    \\
        \featref{Spellsense Specialization}        & Mastered Spellsense         & Improve use of chosen skill           & \tdash          & \featpref{Spellsense Specialization}        \\
        \featref{Stealth Specialization}           & Mastered Stealth            & Improve use of chosen skill           & \tdash          & \featpref{Stealth Specialization}           \\
        \featref{Survival Specialization}          & Mastered Survival           & Improve use of chosen skill           & \tdash          & \featpref{Survival Specialization}          \\
        \featref{Swim Specialization}              & Mastered Swim               & Improve use of chosen skill           & \tdash          & \featpref{Swim Specialization}              \\

        \tb{Casting Feats}\label{Casting Feats} & \tb{Prerequisites} & \tb{Benefits} & \tb{Feat Types} & \tb{Page} \\
        \featref{Boongiver}                      & Spellcasting ability                    & Improve ability to cast spells on allies  & Magical                 & \featpref{Boongiver}                      \\
        \featref{Blood Magic}                    & Spellcasting ability                    & Spend hit points to improve magic         & \featpref{Blood Magic} \\
        \featref{Metacaster}                     & Spellcasting ability                    & Manipulate spell effects in creative ways & Magical                 & \featpref{Metacaster}                     \\
        \featref{Mental Magic}                   & Spellcasting ability, Wil 2             & Cast spells without words or gestures     & Magical                 & \featpref{Mental Magic}                   \\
        % \featref{Miscaster}                      & Spellcasting ability                    & Miscast spells intentionally and effectively & Magical & \featpref{Miscaster}                      \\
        \featref{Mystic Archer}                  & Spellcasting ability                    & Imbue projectiles with magic              & Magical                 & \featpref{Mystic Archer}                  \\
        \featref{Prepared Spellcasting} & Spellcasting ability, Int 2 & Prepare additional spells each day & Magical & \\
        \featref{Spellsword}                     & Spellcasting ability                    & Fight with sword and spell together       & \tdash                  & \featpref{Spellsword}                     \\
        \featref{Sphere Focus: Aeromancy}        & \sphere{Aeromancy} sphere access        & Improve casting with chosen sphere        & Magical                 & \featpref{Sphere Focus: Aeromancy}        \\
        \featref{Sphere Focus: Aquamancy}        & \sphere{Aquamancy} sphere access        & Improve casting with chosen sphere        & Magical                 & \featpref{Sphere Focus: Aquamancy}        \\
        \featref{Sphere Focus: Astromancy}       & \sphere{Astromancy} sphere access       & Improve casting with chosen sphere        & Magical                 & \featpref{Sphere Focus: Astromancy}       \\
        \featref{Sphere Focus: Barrier}          & \sphere{Barrier} sphere access          & Improve casting with chosen sphere        & Magical                 & \featpref{Sphere Focus: Barrier}          \\
        \featref{Sphere Focus: Biomancy}        & \sphere{Biomancy} sphere access        & Improve casting with chosen sphere        & Magical                 & \featpref{Sphere Focus: Biomancy}        \\
        \featref{Sphere Focus: Bless}            & \sphere{Bless} sphere access            & Improve casting with chosen sphere        & Magical                 & \featpref{Sphere Focus: Bless}            \\
        \featref{Sphere Focus: Channel Divinity} & \sphere{Channel Divinity} sphere access & Improve casting with chosen sphere        & Magical                 & \featpref{Sphere Focus: Channel Divinity} \\
        \featref{Sphere Focus: Chronomancy}      & \sphere{Chronomancy} sphere access      & Improve casting with chosen sphere        & Magical                 & \featpref{Sphere Focus: Chronomancy}      \\
        \featref{Sphere Focus: Compel}           & \sphere{Compel} sphere access           & Improve casting with chosen sphere        & Magical                 & \featpref{Sphere Focus: Compel}           \\
        \featref{Sphere Focus: Cryomancy}        & \sphere{Cryomancy} sphere access        & Improve casting with chosen sphere        & Magical                 & \featpref{Sphere Focus: Cryomancy}        \\
        \featref{Sphere Focus: Delusion}         & \sphere{Delusion} sphere access         & Improve casting with chosen sphere        & Magical                 & \featpref{Sphere Focus: Delusion}         \\
        \featref{Sphere Focus: Electromancy}     & \sphere{Electromancy} sphere access     & Improve casting with chosen sphere        & Magical                 & \featpref{Sphere Focus: Electromancy}     \\
        \featref{Sphere Focus: Fabrication}      & \sphere{Fabrication} sphere access      & Improve casting with chosen sphere        & Magical                 & \featpref{Sphere Focus: Fabrication}      \\
        \featref{Sphere Focus: Photomancy}       & \sphere{Photomancy} sphere access       & Improve casting with chosen sphere        & Magical                 & \featpref{Sphere Focus: Photomancy}       \\
        \featref{Sphere Focus: Polymorph}        & \sphere{Polymorph} sphere access        & Improve casting with chosen sphere        & Magical                 & \featpref{Sphere Focus: Polymorph}        \\
        \featref{Sphere Focus: Pyromancy}        & \sphere{Pyromancy} sphere access        & Improve casting with chosen sphere        & Magical                 & \featpref{Sphere Focus: Pyromancy}        \\
        \featref{Sphere Focus: Revelation}       & \sphere{Revelation} sphere access       & Improve casting with chosen sphere        & Magical                 & \featpref{Sphere Focus: Revelation}       \\
        \featref{Sphere Focus: Summoning}           & \sphere{Summoning} sphere access           & Improve casting with chosen sphere        & Magical                 & \featpref{Sphere Focus: Summoning}           \\
        \featref{Sphere Focus: Telekinesis}      & \sphere{Telekinesis} sphere access      & Improve casting with chosen sphere        & Magical                 & \featpref{Sphere Focus: Telekinesis}      \\
        \featref{Sphere Focus: Terramancy}       & \sphere{Terramancy} sphere access       & Improve casting with chosen sphere        & Magical                 & \featpref{Sphere Focus: Terramancy}       \\
        \featref{Sphere Focus: Thaumaturgy}      & \sphere{Thaumaturgy} sphere access      & Improve casting with chosen sphere        & Magical                 & \featpref{Sphere Focus: Thaumaturgy}      \\
        \featref{Sphere Focus: Umbramancy}           & \sphere{Umbramancy} sphere access           & Improve casting with chosen sphere        & Magical                 & \featpref{Sphere Focus: Umbramancy}           \\
        \featref{Sphere Focus: Verdamancy}       & \sphere{Verdamancy} sphere access       & Improve casting with chosen sphere        & Magical                 & \featpref{Sphere Focus: Verdamancy}       \\
        \featref{Sphere Focus: Vivimancy}        & \sphere{Vivimancy} sphere access        & Improve casting with chosen sphere        & Magical                 & \featpref{Sphere Focus: Vivimancy}        \\

        \tb{Combat Feats}\label{Combat Feats} & \tb{Prerequisites} & \tb{Benefits} & \tb{Feat Types} & \tb{Page} \\
        \featref{Blindfighter}        & Per 2        & Fight unseen foes better                   & \tdash & \featpref{Blindfighter}        \\
        \featref{Brawler}             & Str 1, Dex 1 & Fight better unarmed and in close quarters & \tdash & \featpref{Brawler}             \\
        \featref{Duelist}             & Dex 1, Int 1 & Fight one-on-one better                    & \tdash & \featpref{Duelist}             \\
        \featref{Executioner}         & Str 1, Per 2 & Kill weakened foes more easily             & \tdash & \featpref{Executioner}         \\
        \featref{Guardian}            & Per 1, Wil 1 & Protect nearby allies                      & \tdash & \featpref{Guardian}            \\
        \featref{Leadership}          & Wil 2        & Inspire nearby allies                      & \tdash & \featpref{Leadership}          \\
        \featref{Martial Training}    & \tdash       & Improve combat abilities                   & \tdash & \featpref{Martial Training}    \\
        \featref{Savage}              & Str 2        & Shove and overrun foes to deal damage      & \tdash & \featpref{Savage}              \\
        \featref{Sniper}              & Per 2        & Aim precisely at distant foes              & \tdash & \featpref{Sniper}              \\
        \featref{Two-Weapon Fighting} & Dex 2        & Fight better with two weapons at once      & \tdash & \featpref{Two-Weapon Fighting} \\
        \featref{Whirlwind Warrior}   & Dex 2, Per 1 & Fight hordes with agile ease               & \tdash & \featpref{Whirlwind Warrior}   \\
    \end{longtable}
\end{longtablewrapper}

    \section{Feat Descriptions}
        Each feat has a set of benefits it provides to a character with the feat.
        Some feats also have specific requirements that a character must meet before taking the feat.
        These are listed under a \textbf{Prerequisites} heading.
        If a character loses the prerequisites for a feat, they lose all benefits of the feat until they meet the prerequisites again.

    \begin{feat}{Agility Specialization}{Skill}
        \featpre Agility as a mastered skill.

        \ff[1]{Specialization} You gain a \plus3 bonus to the Agility skill.

        \ff[4]{Combat Tumble} During each phase, you may move through one creature's space during movement.
        You treat its space as \glossterm{difficult terrain}.
        After moving through that creature's space, other creatures block you as normal for the remainder of your movement.
        Certain unusual creatures that occupy their entire space, such as gelatinous cubes, may be immune to this ability.

        \ff[7]{Acrobatic Fall} You take half damage from \glossterm{falling damage}.

        \ff[10]{Greater Specialization} The bonus from your \textit{specialization} ability increases to \plus6.

         % modifier at 10th: 12dex + 2mst + 2ft = 16
        \ff[13]{Air Dancer}[Magical] You can attempt to move on surfaces that cannot support your weight.
        Surfaces that can support at least a quarter of your weight, such as thin tree branches and dense liquids, are \glossterm{difficulty rating} 20.
        Surfaces that can support at least a tenth of your weight, such as water, are \glossterm{difficulty rating} 25.
        Surfaces that can support at least a hundredth of your weight, such as tree leaves, are \glossterm{difficulty rating} 30.
        Surfaces that cannot support your weight at all, such as air, are \glossterm{difficulty rating} 40.

        Success means you move along the surface at half speed.
        Failure means you fall through the surface.
        The \glossterm{difficulty rating} increases by 5 for each consecutive round that you spend moving in this way.

        \ff[16]{Supreme Specialization} The bonus from your \textit{specialization} ability increases to \plus9.

        \ff[19]{Greater Air Dancer} You can move at full speed while using your \textit{air dancer} ability.
        In addition, for each round that you spend using your \textit{air dancer} ability, the \glossterm{difficulty rating} increases by 2 instead of by 5.
    \end{feat}

    \begin{feat}{Awareness Specialization}{Skill}
        \featpre Awareness as a mastered skill.

        \ff[1]{Specialization} You gain a \plus3 bonus to the Awareness skill.

        % TODO: clarify stacking with existing senses
        \ff[4]{Extraordinary Senses} You gain one of the following senses: \glossterm{blindsense} (20 ft.), \glossterm{darkvision} (50 ft.), \glossterm{low-light vision}, \glossterm{scent}, or \glossterm{tremorsense} (20 ft.).

        \ff[7]{Quick Scan} When you use the \textit{search} ability, you can notice things in a \areasmall radius within \rngclose range (see \pcref{Search}).

        \ff[10]{Greater Specialization} The bonus from your \textit{specialization} ability increases to \plus6.

        \ff[13]{Greater Extraordinary Senses} You gain one of the following senses: \glossterm{blindsense} (100 ft.), \glossterm{blindsight} (20 ft.), \glossterm{darkvision} (200 ft.), \glossterm{tremorsense} (100 ft.), or \glossterm{tremorsight} (20 ft.).

        \ff[16]{Supreme Specialization} The bonus from your \textit{specialization} ability increases to \plus9.

        \ff[19]{Supreme Extraordinary Senses} You can choose an additional sense from the list given in your \textit{greater extraordinary senses} ability.
        In addition, the range of all senses gained from this feat is doubled.
    \end{feat}

    \begin{feat}{Blindfighter}{Combat}
        \featpre Base Perception 2.

        \ff[1]{Blind Precision} When you make an attack with a miss chance caused by being unable to see your opponent, you can roll the miss chance twice and take the better result.
        In addition, you are not \defenseless against foes you cannot see if you know their location.

        \ff[4]{Blindsense} You gain \glossterm{blindsense} (50 ft.).

        \ff[7]{Attack the Unseen} If you know the location of a creature you cannot see, and you have \glossterm{line of effect} to that creature, you can target it with targeted abilities.

        \ff[10]{Blindsight} You gain \glossterm{blindsight} (50 ft.).
        In addition, the range of your blindsense increases by 150 feet.

        \ff[13]{Controlled Sight} You are immune to all abilities that depend on sight to affect you.

        \ff[16]{Greater Blindsight} The range of your blindsight increases by 50 feet.
        In addition, the range of your blindsense increases by 300 feet.

        \ff[19]{Unseeing Precision} You gain a \plus1 bonus to \glossterm{accuracy}.

        \ff[20]{Supreme Sight} The range of your blindsight improves to 200 feet.
        In addition, the range of your blindsense improves to 1,000 feet.
    \end{feat}

    \begin{feat}{Blood Magic}{Casting, Magical}
        \featpre Ability to cast a spell, base Constitution 2.

        \ff[1]{Bloodspell} Whenever you cast a spell, you may use this ability.
        When you do, you lose a \glossterm{hit point}.
        In exchange, you gain a \plus2 bonus to \glossterm{power} with the spell, the spell does not require \glossterm{components}, and it loses the \glossterm{Focus} tag (if it had it).

        \ff[4]{Fortified Blood} You increase your maximum \glossterm{hit points} by 2.
        In addition, you become \glossterm{bloodied} when you have one fewer \glossterm{hit point} remaining than normal.

        \ff[7]{Bloodbind} Whenever you \glossterm{wound} a living creature using a spell, you can choose to bind the target's blood to yours.
        While the target is bound, you can see it through all forms of \glossterm{concealment} and \glossterm{invisibility} (but not through \glossterm{cover}).
        In addition, you constantly know the exact direction and distance to the target bound by your \textit{bloodbind} ability.
        This binding lasts until you bind another creature with this ability.

        \ff[10]{Greater Bloodspell} The bonus to \glossterm{power} from your \textit{bloodspell} ability increases to \plus4.

        \ff[13]{Greater Fortified Blood} The number of additional \glossterm{hit points} you gain from your \textit{fortified blood} ability increases to 3.
        In addition, you become \glossterm{bloodied} when you have two fewer hit points remaining than normal instead of one fewer.

        \ff[16]{Greater Bloodbind} You are always considered to have \glossterm{line of effect} to the target bound by your \textit{bloodbind} ability, regardless of intervening obstacles.
        The target must still be within the normal \glossterm{range} of your spells.

        \ff[19]{Supreme Bloodspell} The bonus to \glossterm{power} from your \textit{bloodspell} ability increases to \plus6.
    \end{feat}

    \begin{feat}{Boongiver}{Casting, Magical}
        \featpre Ability to cast a spell.

        \ff[1]{Boon Lore} You learn an additional \glossterm{spell}.
        The spell must have the \glossterm{Attune} tag.
        You can exchange this spell for other spells as you gain access to new spell levels, but the spell must always have the \glossterm{Attune} tag.

        \ff[4]{Share Boon} When you cast a spell with the \glossterm{Attune} (self) tag, you can use the \textit{share boon} ability.
        \begin{freeability}{Share Boon}
            The spell's \glossterm{Attune} tag changes to \glossterm{Attune} (target).
            Choose one \glossterm{ally} within \rngmed range.
            That ally is the target of the spell, and the spell affects that creature as if it were you instead of affecting you.

            You can only use this ability to affect one spell at a time.
            If you use it again, the original ally's attunement to the old spell is released, as the \textit{release attunement} ability (see \pcref{Attunement}).
        \end{freeability}

        \ff[7]{Benevolent Transferance} You can use the \textit{benevolent transferance} ability as a \glossterm{standard action}.
        \begin{freeability}{Benevolent Transferance}
            Choose a creature currently \glossterm{attuned} to a spell you cast.
            In addition, choose another creature to transfer the spell to.
            Both targets must within that spell's range of you, and must be valid targets for the spell.
            You cannot target yourself with this ability.

            If both targets are willing, and the new target spends an \glossterm{action point} to attune to the spell, the spell's effect is transferred from the first target to the second.
            The spell's old target regains the \glossterm{action point} it spent to attune to the spell.

            \rankline
            \featlevel{13} You can use this ability as a \glossterm{minor action}.
            \featlevel{19} This ability gains the \glossterm{Swift} tag.
        \end{freeability}

        \ff[10]{Greater Share Boon} You can use your \textit{share boon} ability on up to two different spells at once.
        If you use the ability while it already affects another spell, you choose which spells are affected by the ability.

        \ff[16]{Personal Boon} When you cast a spell with the \glossterm{Attune} (target) tag on yourself, you can use the \textit{personal boon} ability.
        \begin{freeability}{Personal Boon}
            You can attune to the spell you are casting without spending an \glossterm{action point}.

            You can only use this ability on one spell at a time.
            If you use this ability while it already affects another spell, you choose which spell is affected by the ability.
            % This has a bizarre edge case involving you being one of multiple targets for a multi-attunement effect,
            % but there is only one of those in the game and it's a ritual, so it's not worth worrying about.
            Your attunement to the other spell is released, as the \textit{release attunement} ability (see \pcref{Attunement}).
        \end{freeability}

        \ff[19]{Supreme Share Boon} You can use your \textit{share boon} ability on up to three different spells at once.
    \end{feat}

    \begin{feat}{Brawler}{Combat}
        \featpre Base Strength of 1, base Dexterity of 1.

        \ff[1]{Unarmed Warrior} You become \glossterm{proficient} with the unarmed weapons \glossterm{weapon group} (see \pcref{Weapon Groups}).
        In addition, you gain a \plus2d bonus to damage with unarmed weapons.
        For details about how to fight while unarmed, see \pcref{Unarmed Combat}.
        This ability does not stack with the ability of the same name from the Esoteric Warrior monk archetype (see \pcref{Esoteric Warrior}).

        \ff[1]{Grapple Expertise} You gain a \plus2 bonus to \glossterm{accuracy} with the \textit{grapple} ability (see \pcref{Grapple}), as well as with all grapple actions (see \pcref{Grapple Actions}).

        \ff[4]{Takedown} Whenever you hit a target with the \textit{grapple} ability, the target also takes damage as if you had hit with your unarmed attack.

        \ff[7]{Greater Grapple Expertise} The bonus from your \textit{grapple expertise} ability increases to \plus3.

        \ff[10]{Pin Mastery} You can use the \textit{pin} ability with only one free hand (see \pcref{Pin}).
        In addition, you can keep the target immobilized with only one free hand if you hit.

        \ff[13]{Greater Grapple Expertise} The bonus from your \textit{grapple expertise} ability increases to \plus4.

        \ff[16]{Greater Pin Mastery} If you get a \glossterm{critical hit} with the \textit{grapple} ability, you can immediately pin the target as if you had hit with the \textit{pin} ability.

        \ff[19]{Grapple Supremacy} When you grapple a target with the \textit{grapple} ability, you do not become \glossterm{grappled} by that target.
    \end{feat}

    \begin{feat}{Celestial Heritage}{Bloodline, Magical}
        \featpre Non-evil alignment.
        \parhead{Special} You can only have one Bloodline feat.

        \ff[1]{Holy Smite} You can use the \textit{holy smite} ability as a standard action.
        \begin{freeability}{Smite}[Magical]
            Make a \glossterm{strike}.
            Because this is a \glossterm{magical} ability, you use your \glossterm{power} with \glossterm{magical} abilities to determine your damage instead of your power with \glossterm{mundane} abilities.
            If your target is good, the strike deals no damage.
            Otherwise, the strike gains a \plus1d bonus to damage.

            \rankline
            \featlevel{7} The damage bonus increases to \plus2d.
            \featlevel{13} The damage bonus increases to \plus3d.
            \featlevel{19} The damage bonus increases to \plus4d.
        \end{freeability}

        \ff[4]{Celestial Soul} You gain a bonus to \glossterm{resistances} against \glossterm{energy damage} equal to half your level.

        \ff[7]{Angel Wings} You gain feathery wings that sprout from your back.
        These wings grant you a glide speed equal to your \glossterm{base speed} (see \pcref{Gliding}).
        The wings themselves are \glossterm{mundane}, but the ability to fly and glide with them is \glossterm{magical}.

        \ff[10]{Celestial Might} You gain a \plus1 bonus to \glossterm{magical power}.

        \ff[13]{Angelic Flight} You can use the \textit{angelic flight} ability as a \glossterm{free action}.
        \begin{freeability}{Angelic Flight}[Magical]
            You gain a \glossterm{fly speed} equal to your \glossterm{base speed} until the end of the round (see \pcref{Flying}).
            If you have used this ability since you last landed on the ground, you also lose a \glossterm{hit point}.
        \end{freeability}

        \ff[16]{Greater Celestial Soul} The bonus from your \textit{celestial soul} ability increases to be equal to your level.

        \ff[19]{Greater Angelic Flight} Your \textit{angelic flight} ability lasts until the end of the next round.
    \end{feat}

    \begin{feat}{Climb Specialization}{Skill}
        \featpre Climb as a mastered skill.

        \ff[1]{Specialization} You gain a \plus3 bonus to the Climb skill.

        \ff[4]{Climb Speed} You gain a \glossterm{climb speed} equal to half your \glossterm{base speed}.
        A successful Climb check to move allows you to travel a distance equal to your climb speed.

        \ff[7]{Damage Tolerance} Taking damage while climbing does not force you to make an additional Climb check to avoid falling.

        \ff[10]{Greater Specialization} The bonus from your \textit{specialization} ability increases to \plus6.

        \ff[13]{Greater Climb Speed} Your \glossterm{climb speed} increases to be equal to your \glossterm{base speed}.

        \ff[13]{Impossible Climber} You can climb surfaces that are perfectly smooth.
        The \glossterm{difficulty rating} is 30 for perfectly smooth vertical surfaces, and 40 to climb on a perfectly smooth ceiling.
        In addition, you can climb with no free hands if you take a \minus10 penalty to the Climb check.

        \ff[16]{Supreme Specialization} The bonus from your \textit{specialization} ability increases to \plus9.

        \ff[19]{Greater Impossible Climber} You take no penalty for climbing with no free hands.
    \end{feat}

    \begin{feat}{Craft Specialization}{Skill}
        \featpre Any Craft skill as a mastered skill.

        \ff[1]{Specialization} You gain a \plus3 bonus to all Craft skills.

        \ff[4]{Craft Magic Item}[Magical] You can imbue items with magic using your crafting skill.
        Imbuing an item with magic takes material components, as described in \pcref{Magic Item Creation}.
        You can craft an item with an item level equal to your level or one item level lower than your level with 24 hours of continuous work.
        You can make weaker items more quickly.
        For every two item levels lower than your level, the time required to craft an item is halved, to a minimum of 15 minutes.

        You can also mend a broken magic item if it is one that you could make.
        Doing so costs a tenth of the raw materials and a quarter of the time it would take to craft that item in the first place.
        You cannot mend a \glossterm{destroyed} magic item.

        \ff[7]{Crafting Savant} You gain two additional \glossterm{skill points} which can only be spent on Craft skills.

        \ff[10]{Greater Specialization} The bonus from your \textit{specialization} ability increases to \plus6.

        \ff[13]{Item Attunement} You gain an additional \glossterm{action point}.
        You can only use this \glossterm{action point} to \glossterm{attune} to the effects of items (see \pcref{Item Attunement}).

        \ff[16]{Supreme Specialization} The bonus from your \textit{specialization} ability increases to \plus9.

        \ff[19]{Greater Item Attunement} You gain an additional \glossterm{action poin}.
        You can only use this \glossterm{action point} to \glossterm{attune} to the effects of items (see \pcref{Item Attunement}).
    \end{feat}

    \begin{feat}{Creature Handling Specialization}{Skill}
        \featpre Creature Handling as a mastered skill.

        \ff[1]{Specialization} You gain a \plus3 bonus to the Creature Handling skill.

        \ff[4]{Battleforged Training} You can teach a creature the Battleforged trick.
        The \glossterm{difficulty rating} to train the trick is 15.
        A creature with the trick gains the following benefits:
        \begin{itemize}
            \item It gains a bonus equal to half your level to \glossterm{resistances} against \glossterm{physical damage}.
            \item It gains a \plus1 bonus to accuracy with all attacks.
            \item It gains a \plus1d bonus to damage with \glossterm{strikes}.
        \end{itemize}

        \ff[7]{Greater Command} You can \glossterm{sustain} the effect of the \textit{command} ability from the Creature Handling skill with a \glossterm{minor action} instead of with a standard action.
        For details, see \pcref{Command}.

        \ff[7]{Efficient Training} You can teach a creature a trick with 12 hours of work, split as you choose, rather than in a week of 4 hour sessions (see \pcref{Training Creatures}).
        In addition, you can train creatures to learn two bonus tricks beyond their normal maximum (see \pcref{Bonus Tricks}).

        \ff[10]{Greater Specialization} The bonus from your \textit{specialization} ability increases to \plus6.

        \ff[13]{Greater Battleforged Training} You can teach a creature that has learned the Battleforged trick the Greater Battleforged trick.
        The \glossterm{difficulty rating} to train the trick is 25.
        A creature with the trick gains the following benefits, which replace the benefits of the Battleforged trick:
        \begin{itemize}
            \item It gains a bonus equal to your level to \glossterm{resistances} against \glossterm{physical damage}.
            \item It gains a \plus2 bonus to accuracy with all attacks.
            \item It gains a \plus2d bonus to damage with \glossterm{strikes}.
        \end{itemize}

        \ff[16]{Supreme Specialization} The bonus from your \textit{specialization} ability increases to \plus9.

        \ff[19]{Greater Efficient Training} You can teach a creature with 4 hours of work, split as you choose (see \pcref{Training Creatures}).
        In addition, the number of bonus tricks you can teach from your \textit{efficient training} ability increases to four.

        \ff[19]{Supreme Command} You can \glossterm{sustain} the effect of the \textit{command} ability from the Creature Handling skill with a \glossterm{free action} instead of with a standard action.
    \end{feat}

    \begin{feat}{Deception Specialization}{Skill}
        \featpre Deception as a mastered skill.

        \ff[1]{Specialization} You gain a \plus3 bonus to the Deception skill.

        \ff[4]{Dual Speech}[Magical] When you speak, you can use the \textit{dual speech} ability.
        \begin{freeability}{Dual Speech}[\glossterm{Sustain} (minor)]
            You speak the same words with two different vocal patterns, such as tone or accent, though you cannot significantly change the volume.
            You can freely choose which creatures hear which pattern, treating all creatures you are not aware of as a single group. 

            You can freely choose different vocal patterns each round that you sustain this ability.
            \rankline
            \featlevel{10} You can speak entirely different words with your two voices.
            \featlevel{16} You can also speak with a third voice, using separate words and vocal patterns.
        \end{freeability}

        \ff[7]{Undetectable Lies} Any \glossterm{magical} abilities which detect lies are unable to detect lies you speak.

        \ff[10]{Greater Specialization} The bonus from your \textit{specialization} ability increases to \plus6.

        \ff[13]{Deceive Magic}[\glossterm{Magical}] When you would be hit by a \glossterm{magical} attack, you can use this ability.
        \begin{apability}{Deceive Magic}[\glossterm{Swift}]
            The attack takes a \minus10 penalty to accuracy.
            This can cause the attack to hit you instead of getting a \glossterm{critical hit}, or it can cause the attack to miss entirely.
        \end{apability}

        \ff[16]{Supreme Specialization} The bonus from your \textit{specialization} ability increases to \plus9.
    \end{feat}

    \begin{feat}{Devices Specialization}{Skill}
        \featpre Devices as a mastered skill.

        \ff[1]{Specialization} You gain a \plus3 bonus to the Devices skill.

        % TODO: wording
        \ff[4]{Disable Arcana}[Magical] You can affect spell effects on objects or areas with the Devices skill as if they were merely complex devices.
        You must be aware of an effect to use the Devices skill to affect it, either through the Spellsense skill or because the effect is noticeable.
        You cannot affect effects on creatures.
        The \glossterm{difficulty rating} to affect an arcane spell effect is equal to 15 \add the effect's \glossterm{power}.

        \ff[7]{Rapid Improvisation} It takes you only a standard action to make a device of up to Diminuitive size with the \textit{improvise} ability (see \pcref{Improvise}).

        \ff[10]{Greater Specialization} The bonus from your \textit{specialization} ability increases to \plus6.
        
        \ff[13]{Greater Disable Arcana}[Magical] You can affect all \glossterm{magical} effects on objects or areas with the Devices skill, not just spell effects.

        \ff[16]{Supreme Specialization} The bonus from your \textit{specialization} ability increases to \plus9.

        \ff[19]{Greater Rapid Improvization} It takes you only a standard action to make a device of up to Small size with the \textit{improvise} ability (see \pcref{Improvise}).
    \end{feat}

    \begin{feat}{Disguise Specialization}{Skill}
        \featpre Disguise as a mastered skill.

        \ff[1]{Specialization} You gain a \plus3 bonus to the Disguise skill.

        \ff[4]{Quick Change} You reduce the penalties for reducing the creation time of disguises with the \textit{disguise creature} and \textit{emulate creature} abilities by 5.

        \ff[7]{Disguise Aura}[Magical] When you use the \textit{disguise creature} or \textit{emulate creature} abilities, you can decide how the target and any items on the target appear when examined by Divination spells.
        For example, you could cause all of their equipment to appear nonmagical, or you could cause them to have a strong aura of good alignment.
        The maximum \glossterm{power} you can emulate is equal to your Disguise check result \minus10.

        Anyone using divination magic on the creature must make a check with a bonus equal to the creature's \glossterm{power} with the ability to perceive the truth.
        The \glossterm{difficulty rating} is equal to your Disguise check result.
        Regardless of the result of the check, the caster is not aware that the check was made.

        \ff[10]{Greater Specialization} The bonus from your \textit{specialization} ability increases to \plus6.

        \ff[13]{Disguise Size}[Magical] You can use the \textit{disguise size} ability as a \glossterm{standard action}.
        \begin{attuneability}{Disguise Size}[\glossterm{Attune} (self)]
            You increase or decrease your size by one \glossterm{size category}.
            This effect lasts as long as you \glossterm{attune} to it.
        \end{attuneability}

        \ff[16]{Supreme Specialization} The bonus from your \textit{specialization} ability increases to \plus9.

        \ff[19]{Greater Quick Change} You do not suffer penalties for reducing the creation time of disguises with the \textit{disguise creature} and \textit{emulate creature} abilities.
    \end{feat}

    \begin{feat}{Draconic Heritage}{Bloodline}
        \parhead{Special} You can only have one Bloodline feat.

        \ff[1]{Draconic Ancestry} Choose a type of dragon from among the dragons on \trefnp{Dragon Types}.
        You have the blood of that type of dragon in your veins.
        You gain a bonus equal to your level to your \glossterm{resistances} against the damage type dealt by that dragon's breath weapon.

        \ff[1]{Draconic Weapons} You gain a bite natural weapon and two claw natural weapons, one on each arm.
        For details, see \pcref{Natural Weapons}.

        \ff[4]{Breath Weapon} You can use the \textit{breath weapon} ability as a \glossterm{standard action}.
        \begin{freeability}{Breath Weapon}
            Make an attack vs. Armor against everything in the area defined by your \textit{draconic ancestry} ability (see \trefnp{Dragon Types}).
            You can use your Constitution in place of your Strength to determine your \glossterm{power} with this ability.
            \hit Each target takes \glossterm{standard damage}.
            The damage type is defined by your \textit{draconic ancestry} ability.

            \rankline
            \featlevel{10} The area affected by your breath weapon increases.
                A line breath weapon becomes a \areahuge, 15 ft.\ wide line.
                A cone breath weapon becomes a \arealarge cone.
            \featlevel{16} You gain a \plus1d bonus to damage.
                In addition, the area affected by your breath weapon increases again.
                A line breath weapon becomes a \areahuge, 20 ft.\ wide line.
                A cone breath weapon becomes a \areahuge cone.
        \end{freeability}

        \ff[7]{Draconic Wings} You gain scaly wings that sprout from your back.
        These wings grant you a glide speed equal to your \glossterm{base speed} (see \pcref{Gliding}).
        The wings themselves are \glossterm{mundane}, but the ability to fly and glide with them is \glossterm{magical}.

        \ff[10]{Greater Draconic Ancestry} The bonus to \glossterm{resistances} from your \textit{draconic ancestry} ability increases to be equal to twice your level.

        \ff[13]{Draconic Flight}[Magical] You can use the \textit{draconic flight} ability as a \glossterm{free action}.
        \begin{freeability}{Draconic Flight}
            You gain a \glossterm{fly speed} equal to your \glossterm{base speed} until the end of the round (see \pcref{Flying}).
            If you have used this ability since you last landed on the ground, you also lose a \glossterm{hit point}.
        \end{freeability}

        \ff[16]{Draconic Scales} You gain a \plus1 bonus to Armor defense.

        \ff[19]{Greater Draconic Flight} Your \textit{draconic flight} ability lasts until the end of the next round.
    \end{feat}

    \begin{dtable}
        \lcaption{Dragon Types}
        \begin{dtabularx}{\columnwidth}{l >{\lcol}X >{\lcol}X}
            \tb{Dragon} & \tb{Damage Type} & \tb{Breath Weapon} \tableheaderrule
            Black & Acid & \areamed, 10 ft. wide line \\
            Blue & Electricity & \areamed, 10 ft. wide line \\
            Brass & Fire & \areamed, 10 ft. wide line \\
            Bronze & Electricity & \areamed, 10 ft. wide line \\
            Copper & Acid & \areamed, 10 ft. wide line \\
            Gold & Fire & \areamed cone \\
            Green & Acid & \areamed cone \\
            Red & Fire & \areamed cone \\
            Silver & Cold & \areamed cone \\
            White & Cold & \areamed cone \\
        \end{dtabularx}
    \end{dtable}

    \begin{feat}{Duelist}{Combat}
        \featpre Base Dexterity of 1, base Intelligence of 1.

        \ff[1]{Duelist Strike} You can use the \textit{duelist strike} ability as a standard action.
        \begin{freeability}{Duelist Strike}
            Make a melee \glossterm{strike}.
            If you are not \glossterm{threatened} by more than one creature, you gain a \plus1 bonus to \glossterm{accuracy} with the strike.
            If you are the only creature threatening your target, you gain a \plus1d bonus to damage with the strike.

            \rankline
            \featlevel{7} The damage bonus increases to \plus2d.
            \featlevel{13} The accuracy bonus increases to \plus2.
            \featlevel{19} The damage bonus increases to \plus4d.
        \end{freeability}

        \ff[4]{Parry} You gain a \plus1 bonus to Armor defense as long as you wield a melee weapon and are not \glossterm{threatened} by more than one creature.

        \ff[7]{Riposte} You gain a \plus1 bonus to \glossterm{accuracy} with melee \glossterm{strikes} as long as you are not \glossterm{threatened} by more than one creature.

        \ff[10]{Duel Focus} You are treated as being \glossterm{threatened} by one fewer creature than you actually are for the purpose of abilities from this feat.

        \ff[13]{Greater Parry} The bonus from your \textit{parry} ability increases to \plus2.

        \ff[16]{Greater Riposte} The bonus from your \textit{riposte} ability increases to \plus2.

        \ff[19]{Greater Duel Focus} The reduction in number of creatures threatening you from your \textit{duel focus} ability increases to two creatures.
    \end{feat}

    \begin{feat}{Endurance Specialization}{Skill}
        \featpre Endurance as a mastered skill.

        \ff[1]{Specialization} You gain a \plus3 bonus to the Endurance skill.

        \ff[4]{Delay Condition} Whenever you gain a \glossterm{condition}, you can make an Endurance check.
        The \glossterm{difficulty rating} starts at 10 and increases by 5 in each subsequent round.
        Success means that you do not suffer the effects of the codition.
        You must repeat this check at the end of each subsequent round to continue to delay the effects of the condition.
        Failure means that the condition has its normal effect on you.

        You can only delay one of your conditions in this way.
        If you gain a new condition, you can choose to either delay the new condition or continue delaying the old condition.

        \ff[7]{Endurance Sprinter} Whenever you use the \textit{sprint} ability, you can make an Endurance check.
        The \glossterm{difficulty rating} is equal to 10 \add 5 for each other time you have used the \textit{sprint} ability since your last \glossterm{short rest}.
        Success means that you do not lose a \glossterm{hit point} for using the ability.
        Failure means you lose a \glossterm{hit point} for using the ability like normal.

        \ff[10]{Greater Specialization} The bonus from your \textit{specialization} ability increases to \plus6.

        \ff[13]{Delay Vital Wound} Whenever you receive a \glossterm{vital wound}, you can make an Endurance check.
        The \glossterm{difficulty rating} starts at 20 and increases by 5 in each subsequent round.
        Success means that you do not suffer the effects of the vital wound.
        It still counts against your total number of vital wounds for determining your penalties on \glossterm{vital rolls} (see \pcref{Vital Roll}).
        You must repeat this check at the end of each subsequent round to continue to delay the effects of the vital wound.
        Failure means that the vital wound has its normal effect on you.

        You can only delay one of your vital wound in this way.
        If you gain a new vital wound, you can choose to either delay the new vital wound or continue delaying the old vital wound.
        You can make this choice after learning the \glossterm{vital roll} for the new vital wound.

        \ff[16]{Supreme Specialization} The bonus from your \textit{specialization} ability increases to \plus9.

        \ff[19]{Multiple Delay} You can delay two \glossterm{vital wounds} and \glossterm{conditions} with your \textit{delay vital wound} and \textit{delay condition} abilities.
    \end{feat}

    \begin{feat}{Executioner}{Combat}
        \featpres Base Strength of 1, base Perception of 2.

        \ff[1]{Marked for Execution} You consider living creatures that either have a \glossterm{vital wound} or have less than their \glossterm{maximum hit points} to be \textit{marked for execution}.
        Several abilities from this feat affect creatures \textit{marked for execution}.

        % This ability is melee-only because it makes some of the questions around ``is this creature marked''
        % less narratively awkward since you can automatically detect them with blood sense.
        \ff[1]{Execution} You can use the \textit{execution} ability as a standard action.
        \begin{freeability}{Execution}
            Make a melee \glossterm{strike}.
            If the target is \textit{marked for execution}, you gain a \plus1d bonus to damage.
            If the target has a \glossterm{vital wound}, you gain a \plus1d bonus to damage.
            These bonuses stack.

            \rankline
            \featlevel{7} You gain a \plus1d bonus to damage with the strike.
            \featlevel{13} Both circumstantial damage bonuses increase to \plus2d.
            \featlevel{19} The automatic damage bonus increases to \plus2d.
        \end{freeability}

        \ff[4]{Blood Sense}[Magical] You automatically know the location of all creatures that are \textit{marked for execution} within 50 feet of you, regardless of concealment or invisibility.
        You must have \glossterm{line of effect} to a creature to sense it in this way, but you do not need \glossterm{line of sight}.

        \ff[7]{Purge the Weak} You gain a \plus1 bonus to \glossterm{accuracy} against creatures that are \textit{marked for execution}.

        \ff[10]{Greater Blood Sense}[Magical] Your \textit{blood sense} ability allows you to know the location of all living creatures within range.
        In addition, you can see creatures that are \textit{marked for death} perfectly instead of only knowing their location.
        Youc an still automatically identify which creatures within range are \textit{marked for execution}.

        \ff[13]{Bloody Resilience} You gain a \plus2 bonus to Fortitude defense.

        \ff[16]{Greater Purge the Weak} The bonus from your \textit{purge the weak} ability increases to \plus2.

        \ff[16]{Supreme Blood Sense}[Magical] The range of your \textit{blood sense} ability increases to 200 feet.
    \end{feat}

    \begin{feat}{Flexibility Specialization}{Skill}
        \featpre Flexibility as a mastered skill.

        \ff[1]{Specialization} You gain a \plus3 bonus to the Flexibility skill.

        \ff[4]{Rapid Escape} You can squeeze and escape bindings and grapples as a \glossterm{move action}, rather than as a standard action.

        \ff[7]{Constraint Tolerance} You reduce your penalties for \glossterm{squeezing} by 2 (see \pcref{Squeezing}).

        \ff[10]{Greater Specialization} The bonus from your \textit{specialization} ability increases to \plus6.

        \ff[13]{Escape Magic}[Magical] You can use the \textit{escape magic} ability as a standard action.
        \begin{freeability}{Escape Magic}
            You make an Flexibility attack against all \glossterm{magical} effects on you.
            You may exclude any number of effects you are aware of from this attack, allowing you to maintain beneficial magical effects.
            The \glossterm{difficulty rating} for each effect is equal to 10 \add the effect's \glossterm{power}.
            \hit Each effect is \glossterm{dismissed}, if it is an effect that can be dismissed.
        \end{freeability}

        You can only dismiss effects with this ability which target you directly, not area effects which include you as a target.
        If an ability targets multiple creatures, you can only remove its effects on you.

        \ff[16]{Supreme Specialization} The bonus from your \textit{specialization} ability increases to \plus9.

        \ff[19]{Greater Constraint Tolerance} The penalty reduction from your \textit{constraint tolerance} ability increases to 4.
        In addition, your movement speed is not halved while \glossterm{squeezing}.
    \end{feat}

    \begin{feat}{Guardian}{Combat}
        \featpre Base Perception and Willpower of 1.

        \ff[1]{Binding Strike} You can use the \textit{binding strike} ability as a standard action.
        \begin{freeability}{Binding Strike}
            Make a melee \glossterm{strike} against an adjacent creature with a \minus1d penalty to damage.
            If the attack result hits the target's Mental defense,
                it is \glossterm{decelerated} as a \glossterm{condition}.
                At the end of each round, this condition ends if the target is not \glossterm{threatened} by you.

            \rankline
            \featlevel{7} On a \glossterm{critical hit}, the target is \glossterm{immobilized} instead of \glossterm{decelerated}.
            \featlevel{13} The condition does not end if you stop threatening the target.
            \featlevel{19} The target is \glossterm{immobilized} on a normal hit.
        \end{freeability}

        \ff[4]{Threatening Guardian} Your \glossterm{enemies} treat each space adjacent to you as \glossterm{difficult terrain}.

        \ff[7]{Protect} You can use the \textit{protect} ability as a \glossterm{minor action}.
        \begin{freeability}{Protect}[\glossterm{Sustain} (free), \glossterm{Swift}]
            Allies adjacent to you gain a \plus2 bonus to Armor defense until the end of the round.
            However, you take a \minus2 penalty to Armor defense this round.
            Because this ability has the \glossterm{Swift} tag, these bonuses and penalties apply against attacks made in the current phase.

            \rankline
            \featlevel{13} The Armor defense penalty is reduced to \minus1.
            \featlevel{19} The Armor defense penalty is removed.
        \end{freeability}

        \ff[10]{Resolute Defender} You gain a \plus1 bonus to Armor defense and Mental defense.

        \ff[16]{Greater Threatening Guardian} Your \textit{threatening guardian} ability applies \glossterm{difficult terrain} twice, causing enemies to move at one quarter speed.

        \ff[19]{Greater Resolute Defender} The bonuses from your \textit{resolute defender} ability increase to \plus2.
    \end{feat}

    \begin{feat}{Herbalist}{Skill}
        \featpre Knowledge (nature) as a mastered skill.

        \ff[1]{Esoteric Concoction} You can use your Knowledge (nature) skill in place of Craft (alchemy) or Craft (poison) to create poisons and potions.
        This does not help you create other alchemical items, such as alchemist's fire.
        When you do, you must use esoteric natural ingredients in place of the normal ingredients.
        The replacement ingredients must be difficult to acquire in large quantities and impossible to acquire in a normal city.
        For example, you can use the tail of a blind mouse or the dew from a four-leafed clover, but you could not use dirt or ordinary tree bark.
        Once you have determined a purpose for a particular replacement ingredient, you cannot use that ingredient as a replacement in any other poison or potion.

        In general, it requires an hour of work and a Knowledge (nature) check equal to 5 \add the level of the item to find ingredients for an item in this way.
        Each time you find ingredients for an item this way, the time required to find ingredients again increases by an hour and the difficulty rating increases by 5.
        Whenever you finish a \glossterm{long rest} or enter a different environment with different ingredients, these penalties reset.

        \ff[4]{Potent Poisons} You gain a \plus1 bonus to \glossterm{accuracy} with any poisons you make.

        \ff[7]{Tempting Concoction} You can use the \textit{tempting concoction} ability as a \glossterm{standard action}.
        \begin{attuneability}{Tempting Concoction}[\glossterm{Emotion}, \glossterm{Magical}, \glossterm{Subtle}]
            \targets{See text}
            Choose one liquid poison or potion you created, or an object containing one of those liquids, within \rngclose range.
            Whenever an \glossterm{enemy} notices the chosen object, make an attack vs. Mental against it.
            If the poison or potion is not concealed inside a less suspicious object, such as a tankard of ale or an apple, you take a \minus4 penalty to \glossterm{accuracy}.
            You cannot make this attack more than once against any individual target during this ability's duration.
            \hit The target is filled with the desire to investigate and try to consume the liquid or the object containing the liquid.
            It will not generally interrupt combat or wander into obvious danger to fulfill its desire, but individual creatures may react more or less strongly.
            This effect lasts until the target consumes the object or until it takes a \glossterm{short rest}.

            \rankline
            \featlevel{13} You gain a \plus1 bonus to \glossterm{accuracy} with the attack.
            \featlevel{19} The accuracy bonus increases to \plus2.
        \end{attuneability}

        \ff[10]{Efficient Concoction} The time required to find ingredients with your \textit{esoteric concoction} ability is halved.

        \ff[13]{Greater Potent Poisons} The bonus from your \textit{potent concoction} ability increases to \plus2.

        \ff[16]{Poison Tolerance} You gain a \plus2 bonus to Fortitude defense.

        \ff[19]{Blended Poison} You can create poisons that combine two poison effects into a single dose.
        This requires twice the normal time to create a poison, and requires all ingredients required to make both poisons.
        A creature affected by the blended poison suffers the full effects of both poisons.
    \end{feat}

    \begin{feat}{Intimidate Specialization}{Skill}
        \featpre Intimidate as a mastered skill.

        \ff[1]{Specialization} You gain a \plus3 bonus to the Intimidate skill.

        \ff[4]{Greater Demoralize} When you use the \textit{demoralize} ability, the target is \glossterm{shaken} by you as a \glossterm{condition} instead of being shaken the end of the next round.
        For details, see \pcref{Demoralize}.

        \ff[7]{Threatening Presence} You are considered to be \glossterm{threatening} each \glossterm{enemy} within a \arealarge emanation from you for the purpose of determining \glossterm{overwhelm penalties}. 

        \ff[10]{Greater Specialization} The bonus from your \textit{specialization} ability increases to \plus6.

        \ff[13]{Supreme Demoralize} When you use the \textit{demoralize} ability, the target is \frightened by you instead of being shaken.

        \ff[16]{Supreme Specialization} The bonus from your \textit{specialization} ability increases to \plus9.

        \ff[19]{Fearsome Foe} You are considered an additional creature for the purpose of determining \glossterm{overwhelm penalties} you inflict on other creatures.
        This can allow you to overwhelm creatures by yourself.
    \end{feat}

    \begin{feat}{Iron Will}{General}
        \featpre Base Willpower of 1.

        \ff[1]{Mental Discipline} You gain a \plus2 bonus to Mental defense.
        In addition, you gain a \plus5 bonus to notice \glossterm{Subtle} abilities that affect your mind.

        \ff[4]{Mind over Matter} You increase your maximum \glossterm{hit points} by 2.

        \ff[7]{Unclouded Mind} You are immune to being \glossterm{dazed} and \glossterm{stunned}.

        \ff[10]{Greater Mental Discipline} The defense bonus from your \textit{mental discipline} ability increases to \plus3.
        In addition, the bonus to identify \glossterm{Subtle} abilities increases to \plus10.

        \ff[13]{Greater Mind over Matter} The number of additional \glossterm{hit points} from your \textit{mind over matter} ability increases to 3.

        \ff[16]{Greater Unclouded Mind} You are immune to being \glossterm{disoriented} and \glossterm{confused}.

        \ff[19]{Supreme Mental Discipline} The defense bonus from your \textit{mental discipline} ability increases to \plus4.
        In addition, you automatically notice all \glossterm{Subtle} abilities that affect your mind.
    \end{feat}

    \begin{feat}{Jump Specialization}{Skill}
        \featpre Jump as a mastered skill.

        \ff[1]{Specialization} You gain a \plus3 bonus to the Jump skill.

        \ff[4]{Featherlight Leap} You can use the \textit{featherlight leap} ability as a \glossterm{free action}.
        \begin{freeability}{Featherlight Leap}[\glossterm{Swift}]
            When you use this ability, you lose a \glossterm{hit point}.
            Your maximum height for jumps during the current phase is equal to your Jump check result, rather than half your Jump check result.
            This does not affect the forward distance you can reach with your jumps.
        \end{freeability}

        \ff[7]{Instant Leap} You suffer no penalty for jumping without a running start (see \pcref{Running Start}).

        \ff[10]{Greater Specialization} The bonus from your \textit{specialization} ability increases to \plus6.

        \ff[13]{Impact Tolerance} You are immune to \glossterm{falling damage}.

        \ff[16]{Supreme Specialization} The bonus from your \textit{specialization} ability increases to \plus9.

        \ff[19]{Greater Featherlight Leap} Your maximum height for jumps is always equal to your Jump check result.
        When you use your \textit{featherlight leap} ability, your maximum height increases to be equal to twice your Jump check result.
    \end{feat}

    \begin{feat}{Knowledge Specialization}{Skill}
        \featpre Any Knowledge skill as a mastered skill.

        \ff[1]{Specialization} You gain a \plus3 bonus to all Knowledge skills.

        \ff[4]{Knowledge Savant} You gain two additional \glossterm{skill points} which can only be spent on Knowledge skills.

        \ff[7]{Greater Knowledge Savant} The number of extra skill points from your \textit{knowledge savant} ability increases to four.

        \ff[10]{Greater Specialization} The bonus from your \textit{specialization} ability increases to \plus6.

        \ff[13]{Experienced Defense} You gain a \plus1 bonus to Fortitude, Reflex, and Mental defenses. 

        \ff[16]{Supreme Specialization} The bonus from your \textit{specialization} ability increases to \plus9.

        \ff[19]{Greater Experienced Defense} You gain a \plus1 bonus to Armor defense.
    \end{feat}

    \begin{feat}{Leadership}{Combat}
        \featpre Base Willpower of 2.

        \ff[1]{Battle Command} You can use the \textit{battle command} ability as a standard action.
        \begin{freeability}{Battle Command}[\glossterm{Swift}]
            Choose an \glossterm{ally} within \rngmed range.
            During the current phase, the target gains a \plus2 bonus to \glossterm{accuracy} and rolls twice for any attacks it makes, keeping the better result.

            \rankline
            \featlevel{7} The accuracy bonus increases to \plus3.
            \featlevel{13} The accuracy bonus increases to \plus4.
            \featlevel{19} The accuracy bonus increases to \plus5.
        \end{freeability}

        \ff[4]{Encouraging Presence} Your \glossterm{allies} within a \arealarge \glossterm{emanation} from you are immune to being \glossterm{shaken}, \glossterm{frightened}, and \glossterm{panicked}.

        \ff[7]{Bolster} You can use the \textit{bolster} ability as a standard action.
        \begin{freeability}{Bolster}[\glossterm{Emotion}]
            One \glossterm{ally} within \rngmed range can remove a \glossterm{condition}.
            This cannot remove a condition applied during the current round.

            \rankline
            \featlevel{13} You may target an additional \glossterm{ally} within range.
            \featlevel{19} Each target may remove an additional \glossterm{condition}.
        \end{freeability}

        \ff[10]{Inspiring Presence} Your \glossterm{allies} within a \arealarge \glossterm{emanation} from you gain a \plus2 bonus to Mental defense.

        \ff[13]{Brave Leader} You are immune to being \glossterm{shaken}, \glossterm{frightened}, and \glossterm{panicked}.

        \ff[16]{Sustaining Presence} Your \glossterm{allies} within a \arealarge \glossterm{emanation} from you gain a \plus2 bonus to \glossterm{vital rolls}.

        \ff[19]{Supreme Presence} The area of your \textit{presence} abilities from this feat increases to a \areahuge \glossterm{emanation} from you.
    \end{feat}

    \begin{feat}{Linguistics Specialization}{Skill}
        \featpre Linguistics as a mastered skill.

        \ff[1]{Specialization} You gain a \plus3 bonus to the Linguistics skill.

        \ff[4]{Linguistic Savant} You learn two additional \glossterm{common languages}, or one additional \glossterm{rare language}.

        \ff[7]{Language Focus} By spending a day in focused concentration on learning a specific \glossterm{common language}, you can use the \textit{language focus} ability.
        You must have access to either a creature fluent in the language willing to help you or at least a book's worth of material written in the language.
        \begin{freeability}{Language Focus}
            If you had access to written material on the language, including from a teacher, you can read or write the language.
            If you had access to a speaker of the language, you can speak and understand the language.

            This ability's effect lasts until you use this ability again.
        \end{freeability}

        \ff[10]{Greater Specialization} The bonus from your \textit{specialization} ability increases to \plus6.

        \ff[13]{Greater Language Focus} You can use your \textit{language focus} ability to learn \glossterm{rare languages} in addition to common languages.

        \ff[16]{Supreme Specialization} The bonus from your \textit{specialization} ability increases to \plus9.
    \end{feat}

    \begin{feat}{Martial Training}{Combat}
        \ff[1]{Trained Strike} You can use the \textit{trained strike} ability as a standard action.
        \begin{freeability}{Trained Strike}
            You make a \glossterm{strike} with a \plus1 bonus to \glossterm{accuracy}.

            \rankline
            \featlevel{7} The accuracy bonus increases to \plus2.
            \featlevel{13} The accuracy bonus increases to \plus3.
            \featlevel{19} The accuracy bonus increases to \plus4.
        \end{freeability}

        \ff[4]{Equipment Training} You choose one of the following benefits.
        \begin{itemize}
            \item You gain proficiency with a \glossterm{usage class} of \glossterm{armor} (light, medium, or heavy).
                You must be proficient with light armor to gain proficiency with medium armor, and you must be proficient with medium armor to gain proficiency with heavy armor.
            \item You gain proficiency with an additional \glossterm{weapon group} of your choice.
            \item You gain proficiency with \glossterm{exotic weapons} from a weapon group of your choice that you are already proficient with.
            \item You reduce the \glossterm{encumbrance} of \glossterm{body armor} you wear by 1.
                If you choose this ability multiple times, its effects stack.
        \end{itemize}

        \ff[7]{Martial Power} You gain a \plus1 bonus to \glossterm{power} with \glossterm{mundane} abilities.

        \ff[10]{Equipment Training} You gain an additional \textit{equipment training} ability of your choice.

        \ff[13]{Practiced Accuracy} You gain a \plus1 bonus to \glossterm{accuracy}.

        \ff[16]{Equipment Training} You gain an additional \textit{equipment training} ability of your choice.

        \ff[19]{Greater Martial Power} The bonus from your \textit{martial power} ability increases to \plus2.
    \end{feat}

    \begin{feat}{Medicine Specialization}{Skill}
        \featpre Medicine as a mastered skill.

        \ff[1]{Specialization} You gain a \plus3 bonus to the Medicine skill.

        \ff[4]{Healing Touch} You can use the \textit{healing touch} ability as a standard action.
        \begin{freeability}{Healing Touch}[\glossterm{Life}]
            \target{Yourself or a living \glossterm{ally} within \glossterm{reach}}
            The target regains a \glossterm{hit point}.
            If it is \glossterm{bloodied}, it regains two hit points instead of one.

            \rankline
            \featlevel{10} The number of hit points regained increases to two.
            \featlevel{16} If the target is bloodied, it regains three hit points instead of two.
        \end{freeability}

        \ff[7]{Purging Touch} You can use the \textit{purging touch} ability as a standard action.
        \begin{freeability}{Purging Touch}
            Make a Medicine check on an \glossterm{ally} you can touch.
            For each poison and disease on the target, if your check result is at least 10 higher than the \glossterm{power} of the effect, the effect is removed.

            \rankline
            \featlevel{13} You can target any number of \glossterm{allies} within your reach.
            \featlevel{19} You gain a \plus5 bonus to the check.
        \end{freeability}

        \ff[10]{Greater Specialization} The bonus from your \textit{specialization} ability increases to \plus6.

        \ff[13]{Lifesaver} You can use the \textit{first aid} ability as a \glossterm{minor action} (see \pcref{First Aid}).

        \ff[16]{Supreme Specialization} The bonus from your \textit{specialization} ability increases to \plus9.

        \ff[19]{Greater Lifesaver} You can use the \textit{first aid} ability as a \glossterm{free action}.
    \end{feat}

    \begin{feat}{Mental Magic}{Casting, Magical}
        \featpre Spellcasting ability, Willpower 2.

        \ff[1]{Mental Casting} You connect to the magical essence of the universe differently from other spellcasters, allowing you to cast spells with purely mental effort.
        None of your spells have \glossterm{somatic components} or \glossterm{verbal components}.
        However, casting spells without components is more challenging.
        You increase your \glossterm{focus penalty} by 1.

        \ff[4]{Potent Mind} You gain a \plus1 bonus to \glossterm{power} with \glossterm{magical} abilities.

        \ff[7]{Innate Creativity} You learn an additional \glossterm{spell}.
        When you gain access to new spell levels, you can change which spell you know.

        \ff[10]{Fractured Mind} Once per round, you can sustain an ability with the \glossterm{Sustain} (minor) tag as a \glossterm{free action}.

        \ff[13]{Greater Potent Mind} The bonus from your \textit{potent mind} ability increases to \plus2.

        \ff[16]{Greater Innate Creativity} You learn an additional \glossterm{spell}.
        When you gain access to new spell levels, you can change which spell you know.

        \ff[19]{Greater Fractured Mind} You can use your \textit{fractured mind} ability on abilities with the \glossterm{Sustain} (standard) tag in addition to the \glossterm{Sustain} (minor) tag.
    \end{feat}

    \begin{feat}{Metacaster}{Casting, Magical}
        \featpre Ability to cast a spell.

        \ff[1]{Sphere Access} You gain access to an additional \glossterm{mystic sphere}.
        Each \glossterm{mystic sphere} has a set of \glossterm{spells} associated with it.
        You automatically learn all \glossterm{cantrips} from any mystic sphere you have access to.
        % TODO: wording
        If you have multiple \glossterm{magic sources}, you can cast spells from that sphere with any magic source that the mystic sphere belongs to.

        \ff[4]{Alter Damage} Whenever you cast a spell that deals damage, you can change the type of damage it deals based on the \glossterm{mystic spheres} you have access to.
        You can use this ability to affect both spells that deal damage directly and spells that cause effects or summon creatures that later deal damage.

        The damage types for each mystic sphere are given in \trefnp{Mystic Sphere Damage Types}.
        Not all mystic spheres have associated damage types.

        \begin{dtable}
            \lcaption{Mystic Sphere Damage Types}
            \begin{dtabularx}{\columnwidth}{l >{\lcol}X}
                \tb{Mystic Sphere} & \tb{Damage Type} \tableheaderrule
                Aeromancy & Bludgeoning \\
                Aquamancy & Bludgeoning \\
                Astromancy & Energy \\
                Barrier & \tdash \\
                Biomancy & Physical \\
                Bless & \tdash \\
                Channel Divinity & Energy \\
                Chronomancy & \tdash \\
                Compel & \tdash \\
                Cryomancy & Cold \\
                Delusion & \tdash \\
                Electromancy & Electricity \\
                Fabrication & Physical \\
                Photomancy & Energy \\
                Polymorph & Physical \\
                Pyromancy & Fire \\
                Revelation & \tdash \\
                Summoning & \tdash \\
                Telekinesis & Physical \\
                Terramancy & Bludgeoning \\
                Thaumaturgy & Energy \\
                Umbramancy & Cold \\
                Verdamancy & \tdash \\
                Vivimancy & \tdash \\
            \end{dtabularx}
        \end{dtable}

        \ff[7]{Augment} You learn one \glossterm{augment} (see \pcref{Augments}).
        You can apply that augment to spells you cast and rituals you perform.
        When you gain access to a new spell level, you may change which augments you know.

        % TODO: make this based on spheres known?
        \ff[10]{Alter Conditions} When you cast a spell that inflicts a \glossterm{condition} with a standard effect, you can change that effect to another effect of the same rank.
        Condition effect ranks are described in \trefnp{Condition Effect Ranks}.
        To change the spell to inflict a particular effect, you must know another spell that inflicts that effect.
        \begin{dtable}
            \lcaption{Condition Effect Ranks}
            \begin{dtabularx}{\columnwidth}{l >{\lcol}X}
                \tb{Rank} & \tb{Condition effects} \tableheaderrule
                1 & Dazed, dazzled, fatigued, sickened, slowed \\
                2 & Decelerated, frightened, ignited (removable)\fn{1}, nauseated, stunned \\
                3 & Blinded, confused, decelerated, disoriented, ignited (non-removable), immobilized, panicked \\
                4 & Asleep\fn{2}, paralyzed \\
            \end{dtabularx}
            1. The rank 2 version of the \glossterm{ignited} condition can be removed with a \glossterm{difficulty rating} 10 Dexterity check. The rank 3 version can only be removed by removing the condition, as normal for conditions.
            2. The target wakes up if it takes a \glossterm{vital wound}, but cannot otherwise wake up during the condition.
        \end{dtable}

        \ff[13]{Augment} You learn an additional \glossterm{augment}.

        \ff[16]{Blend Damage} When you cast a spell that deals damage, you can change the damage type to have any combination of damage types provided by the \glossterm{mystic spheres} you have access to and the original spell's damage type.
        For example, if you cast a spell that deals fire damage and you have access to the \textit{cryomancy} \glossterm{mystic sphere}, you can change that spell's damage type to be either fire damage, cold damage, or both fire and cold damage simultaneously.
        You cannot remove a spell's damage type entirely.

        \ff[19]{Augment} You learn an additional \glossterm{augment}.
    \end{feat}

    % TODO: replace with Wild Magic mage archetype
    % \begin{feat}{Miscaster}{Casting, Magical}
    %     \featpre Ability to cast a spell.

    %     \ff[1]{Selective Backlash} You can freely exclude creatures other than yourself from the effects of your \glossterm{miscast backlash}.

    %     \ff[4]{Overchannel} When you cast a spell, you can use the \textit{overchannel} ability.
    %     \begin{freeability}{Overchannel}
    %         Your concentration on the spell you are casting cannot be disrupted, and you cannot \glossterm{miscast} the spell for any other reason.
    %         Effects that prevent the spell from having any effect, such as the \textit{counterspell} ability from the Sphere Focus: Thaumaturgy feat, work normally (see \featpref{Sphere Focus: Thaumaturgy}).
    %         In addition, you cause a \glossterm{miscast backlash} when the spell resolves.

    %         However, using this ability requires channeling excessive magical energy that temporarily limits your connection to magic.
    %         During the next round, you take a \minus2 penalty to all defenses and any spell you cast is automatically \glossterm{miscast}.
    %     \end{freeability}

    %     \ff[7]{Widened Backlash} The area affected by your \glossterm{miscast backlash} increases to a \areamed radius centered on you.

    %     \ff[10]{Toughened Body} You gain an additional \glossterm{hit point}.

    %     \ff[13]{Resilient Channeler} The penalty to defenses from your \textit{overchannel} ability is removed.

    %     \ff[16]{Empowered Miscast} You gain a \plus2 bonus to \glossterm{power} with your \glossterm{miscast backlash}.

    %     \ff[19]{Greater Widened Backlash} The area affected by your \glossterm{miscast backlash} increases to a \areamed radius centered on you.
    % \end{feat}

    \begin{feat}{Mystic Archer}{Casting}
        \featpre Ability to cast a spell.

        \ff[1]{Imbued Shot} You can use the \textit{imbued shot} ability as a standard action.
        \begin{freeability}{Imbued Shot}[\glossterm{Magical}]
            Make a ranged \glossterm{strike} with a \plus1 bonus to \glossterm{accuracy} using a \glossterm{projectile weapon} you wield.
            Because this is a \glossterm{magical} ability, you use your \glossterm{power} with \glossterm{magical} abilities to determine your damage instead of your power with \glossterm{mundane} abilities.

            \rankline
            \featlevel{7} The accuracy bonus increases to \plus2.
            \featlevel{13} The accuracy bonus increases to \plus3.
            \featlevel{19} The accuracy bonus increases to \plus4.
        \end{freeability}

        \ff[4]{Guided Projectiles}[Magical] Your attacks with projectiles ignore \glossterm{cover}, but not \glossterm{total cover}.

        \ff[7]{Imbue Projectile} When you cast a spell that does not have the \glossterm{AP}, \glossterm{Attune}, or \glossterm{Sustain} tags,
            you can use the \textit{imbue projectile} ability.
        \begin{attuneability}{Imbue Projectile}[\glossterm{Attune} (self), \glossterm{Magical}]
            The spell does not have its effect immediately.
            Instead, its power is imbued in a \glossterm{projectile} you hold.
            An individual projectile can only be imbued with this ability once.

            When you use your \textit{imbued shot} ability to attack with that projectile, the spell takes effect on the target of your \textit{imbued shot} ability.
            After the spell takes effect this way, your attunement to this ability ends.
            The next time you take a \glossterm{short rest}, you regain the \glossterm{action point} you spent to attune to this ability.
        \end{attuneability}

        % \ff[7]{Missile Storm} You can use the \textit{missile storm} ability as a standard action.
        % \begin{freeability}{Missile Storm}[\glossterm{Magical}]
        %     Make a ranged \glossterm{magical strike} with a \glossterm{projectile weapon} you wield.
        %     The strike targets up to five creatures or objects within one \glossterm{range increment} of you with that weapon, except for creatures adjacent to you.

        %     \rankline
        %     \featlevel{13} The strike targets up to ten creatures or objects within range.
        %     \featlevel{19} The strike targets any number of creatures or objects within range.
        % \end{freeability}

        \ff[10]{Phasing Projectiles}[\glossterm{Magical}] When attacking with projectiles, you can ignore all physical obstacles in single one-foot span.
        This can allow you to fire projectiles through solid walls, though it does not grant you the ability to see through the wall.

        \ff[13]{Imbued Connection} You gain a \plus1 bonus to \glossterm{accuracy} with spells that you store with your \textit{imbue projectile} ability.

        \ff[16]{Greater Phasing Projectiles}[\glossterm{Magical}] Your \textit{phasing projectiles} ability improves, allowing you to ignore obstacles in up to five one-foot spans.
        The spans can be contiguous or independent, which can allow you to ignore a single obstacle up to five feet deep.

        \ff[19]{Greater Imbued Connection} The bonus from your \textit{imbued connection} ability increases to \plus2.
    \end{feat}

    \begin{feat}{Null}{General}
        \featpre Base Willpower of 2.

        \ff[1]{Nullify Magic} You gain a \plus4 bonus to \glossterm{defenses} against \glossterm{magical} abilities.
        In addition, you are never considered an \glossterm{ally} for a \glossterm{magical} ability, even while \glossterm{unconscious}.
        In exchange, you lose the benefits of all \glossterm{magical} abilities you possess.
        In addition, you are unable to \glossterm{attune} to any \glossterm{magical} abilities, such as magic items or spells cast by other creatures.

        \ff[1]{Sever Magic} You can use the \textit{sever magic} ability as a standard action.
        \begin{freeability}{Sever Magic}
            Make a \glossterm{strike}.
            On a hit, the target stops being \glossterm{attuned} to one effect of its choice that it is currently attuned to.

            On a \glossterm{critical hit}, the target stops being attuned to two abilities of its choice that it is currently attuned to.
            In addition, as a \glossterm{condition}, it stops being able to attune to abilities.

            \rankline
            \featlevel{7} You gain a \plus1 bonus to \glossterm{accuracy} with the strike.
            \featlevel{13} The accuracy bonus increases to \plus2.
            \featlevel{19} The accuracy bonus increases to \plus2.
        \end{freeability}

        \ff[4]{Personal Legacy} You do not gain any legacy item upgrades (see \pcref{Legacy Items}).
        Instead, each time you would gain a legacy item upgrade, you instead increase your \glossterm{action points} by 1 and increase the number of times you can use the \textit{recover} ability between \glossterm{short rests} by 1 (see \pcref{Recover}).

        % clarify that it does not affect spells already in progress in the current phase?
        \ff[7]{Disruptive Presence} All enemies within an \areamed radius from you have a 50\% chance to \glossterm{miscast} any spell they cast.

        \ff[7]{Greater Nullify Magic} The bonus to defenses from your \textit{nullify magic} ability increases to \plus6.

        \ff[10]{Itembane} Whenever you touch a \glossterm{magical} item or hit it with a melee weapon, such as with the \textit{disarm} ability, it loses all magical abilities until the end of the next round (see \pcref{Disarm}).
        This does not prevent you from suffering the normal effects of the item's initial hit, if the item was used to strike you.
        Under normal circumstances, removes the abilities of items that hit you with melee \glossterm{strikes}, but does not affect magical projectile weapons.
        Items with an intrinsic \glossterm{power} at least 10 higher than your level are immune to this effect.
        The \glossterm{power} of the item's wielder, if any, does not affect whether the item can be affected in this way.

        \ff[13]{Supreme Nullify Magic} The bonus to defenses from your \textit{nullify magic} ability increases to \plus8.

        \ff[16]{Greater Disruptive Presence} Your \textit{disruptive presence} ability affects all enemies in a \arealarge radius \glossterm{emanation} from you.

        \ff[19]{True Null} You are unaffected by all \glossterm{magical} abilities.
    \end{feat}

    \begin{feat}{Perform Specialization}{Skill}
        \featpre Any Perform skill as a mastered skill.

        \ff[1]{Specialization} You gain a \plus3 bonus to all Perform skills.

        \ff[4]{Synergistic Performance} You can use your Perform skills in place of other related skills.
        Each Perform skill has an associated skill that it can be used to replace, as listed below.
        When you replace a skill in this way, you take a \minus4 penalty to the Perform check since the two skills do not exactly match.
        \begin{itemize}
            \item Acting: Deception
            \item Comedy: Deception
            \item Dance: Agility
            \item Keyboard instruments: Devices
            \item Oratory: Persuasion
            \item Percussion instruments: Creature Handling
            \item Singing: Persuasion
            \item String instruments: Devices
            \item Wind instruments: Creature Handling
        \end{itemize}

        \ff[7]{Lingering Performance}[Magical] Whenever you stop performing, the effects of the performance linger until the end of the next round as if you were still performing.
        This affects both normal peformances and any special abilities that require performances to sustain them, allowing you to sustain those abilities without taking any actions.

        \ff[10]{Greater Specialization} The bonus from your \textit{specialization} ability increases to \plus6.

        \ff[13]{Greater Synergistic Performance} The penalty from your \textit{synergistic performance} ability is removed.

        \ff[16]{Supreme Specialization} The bonus from your \textit{specialization} ability increases to \plus9.

        \ff[19]{Endless Performance} You can sustain performances for any length of time.
        This affects both normal performances and any special abilities that require performances to sustain them, allowing you to sustain those abilities beyond the normal 5 minute limit.
    \end{feat}

    \begin{feat}{Persuasion Specialization}{Skill}
        \featpre Persuasion as a mastered skill.

        \ff[1]{Specialization} You gain a \plus3 bonus to the Persuasion skill.

        \ff[4]{Compel Attention}[Magical] You can use the \textit{compel attention} ability as a standard action.
        \begin{freeability}{Compel Attention}[\glossterm{Auditory}, \glossterm{Compulsion},\\\glossterm{Sustain} (minor), \glossterm{Subtle}]
            Make an attack vs. Mental against a creature within \rngmed range.
            Your \glossterm{accuracy} is equal to your Persuasion skill.
            You must talk loud enough for the target to hear to draw its attention.
            \hit The target is \glossterm{fascinated} by you as long as you sustain this ability, which requires maintaining your conversation with it.
            Any act by you or by creatures that appear to be your ally that damages a target or that causes it to feel that it is in danger breaks the effect for that creature.
            An observant target may interpret overt threats to its \glossterm{allies} as a threat to itself.

            \rankline
            \featlevel{10} You may target up to five creatures within range.
            \featlevel{16} You may target any number of creatures within range.
        \end{freeability}

        \ff[7]{First Impressions} When you first meet creatures, you have an Ally relationship instead of a Just Met relationship.
        This does not improve your relationship with creatures who already have an impression of you, whether positive or negative.

        \ff[10]{Greater Specialization} The bonus from your \textit{specialization} ability increases to \plus6.

        \ff[13]{Suggestion}[Magical] You can use the \textit{suggestion} ability as a standard action.
        \begin{freeability}{Suggestion}[\glossterm{Emotion}, \glossterm{Subtle}, \glossterm{Sustain} (minor)]
            Make an attack vs. Mental against a target within \rngmed range.
            Your \glossterm{accuracy} is equal to your Persuasion skill.
            You must also make a verbal suggestion of a particular course of action to the target.
            If your suggestion does not seem reasonable, you take a \minus5 penalty to accuracy on this attack.
            Exceptionally unreasonable suggestions can impose even greater penalties, and exceptionally reasonable suggestions can give accuracy bonuses.

            \hit As a \glossterm{condition}, the target thinks your suggestion is a good idea and will try to follow it to the best of its abilities.
            Any act by you or by creatures that appear to be your ally that damages the target or makes it feel that it is in danger breaks the effect.
            An observant target may interpret overt threats to its \glossterm{allies} as a threat to itself.

            \rankline
            \featlevel{19} You may target up to five creatures within range.
        \end{freeability}

        \ff[16]{Supreme Specialization} The bonus from your \textit{specialization} ability increases to \plus9.

        \ff[19]{Greater First Impressions} When you first meet creatures, you have a Friend relationship instead of a Just Met relationship
        This does not improve your relationship with creatures who already have an impression of you, whether positive or negative.
    \end{feat}

    \begin{feat}{Precognition}{General}
        \featpre Base Intelligence of 2.

        \ff[1]{Precognitive Precision} You can use your Intelligence in place of your Perception to determine your \glossterm{accuracy} with \glossterm{mundane} abilities.

        \ff[4]{Combat Prediction} You can use the \textit{combat prediction} ability as a standard action.
        \begin{freeability}{Combat Prediction}
            Make an attack vs. Mental against a creature within \rngmed range of you.
            \hit As a \glossterm{condition}, that creature's intentions become obvious to you.
            At the start of each phase, you can see and hear what actions that creature intends to take.
            You do not gain any knowledge of actions that have no obvious signs, such as purely mental actions, or actions which you are not observant enough to notice.
            In addition, you do not know the results of actions with a chance of failure, such as attacks.

            The creature may change its actions based on your interference if you communicate your insight in a way it understands.

            \rankline
            \featlevel{10} You may target two creatures within range.
            \featlevel{16} You may target three creatures within range.
        \end{freeability}

        \ff[7]{Precognitive Reaction} You gain a \plus2 bonus to Reflex defense and \glossterm{initiative} checks.

        \ff[10]{Foresight} During the \glossterm{movement phase}, you choose your action after all other creatures have chosen their actions.
        When you choose your action, you have insight into the actions chosen by any creatures within \rngclose range of you that you can see.
        This insight gives you the same information as the insight from your \textit{combat prediction} ability.
        You choose your actions simultaneously with any other creatures who have a similar ability.

        \ff[13]{Greater Precognitive Precision} You gain a \plus1 bonus to \glossterm{accuracy}.

        \ff[16]{Greater Precognitive Reaction} The bonuses from your \textit{precognitive reaction} ability increase to \plus4.

        \ff[19]{Greater Foresight} The range of your \textit{foresight} ability increases to \rnglong range.
    \end{feat}

    \begin{feat}{Prepared Spellcaster}{Magical, Spell}
        \featpre Ability to cast a spell, base Intelligence of 2.

        \ff[1]{Spellbook} Choose up to three spells you do not know from among \glossterm{mystic spheres} you have access to.
        The spells in your spellbook can come from any combination of \glossterm{magic sources} you can cast spells with.
        The spells must be of a rank that you know how to cast.
        Whenever you gain access to a new spell rank, you may change the spells in your spellbook for any other spells you can cast.
        You inscribe the knowledge of those spells into a book you carry with you.
        This book is your spellbook.
        
        Whenever you finish a \glossterm{long rest}, you may choose one of the spells in your spellbook.
        You learn how to cast that spell until you choose a different spell with this ability.

        \ff[4]{Study of Magic} You gain a \plus1 bonus to \glossterm{power} with \glossterm{magical} abilities.

        \ff[7]{Studious Learning} You gain a \plus2 bonus to all Knowledge skills.

        \ff[10]{Expanded Spellbook} You can choose up to five spells to be in your spellbook instead of only three.

        \ff[13]{Greater Study of Magic} The bonus from your \textit{study of magic} ability increases to \plus2.

        \ff[16]{Greater Spellbook} Whenever you finish a \glossterm{long rest}, you may choose two spells in your spellbook with your \textit{spellbook} ability instead of one.
        You learn how to cast both spells until you choose a different pair of spells in this way.

        \ff[19]{Greater Expanded Spellbook} You can choose up to seven spells to be in your spellbook instead of only three.
    \end{feat}

    \begin{feat}{Rapid Reaction}{General}
        \featpre Base Dexterity of 1.

        \ff[1]{Lightning Reflexes} You gain a \plus2 bonus to Reflex defense and \glossterm{initiative} checks.

        \ff[4]{Evasion} When you are attacked by an ability that affects an area, you can use your Reflex defense in place of your Armor defense against that attack.

        \ff[7]{Sidestep} If you have at least five feet of movement remaining after the \glossterm{movement phase}, you may move up to five feet during the \glossterm{action phase} or the \glossterm{delayed action phase} as a \glossterm{free action}.

        \ff[10]{Greater Lightning Reflexes} The bonuses from your \textit{lightning reflexes} ability increase to \plus4.

        \ff[13]{Greater Evasion} You can use your Reflex defense in place of any other defense with your \textit{evasion} ability.

        \ff[16]{Greater Sidestep} The movement you can carry over with your \textit{sidestep} ability increases to half your \glossterm{base speed}.

        \ff[19]{Supreme Lightning Reflexes} The bonuses from your \textit{lightning reflexes} ability increase to \plus6.
    \end{feat}

    \begin{feat}{Regenerator}{General}
        \featpre Base Constitution of 2.

        \ff[1]{Diehard} You gain a \plus2 bonus to \glossterm{vital rolls}.

        \ff[4]{Regenerative Rest} When you take a \glossterm{short rest}, you can spend an \glossterm{action point}. If you do, you heal one \glossterm{vital wound}.

        \ff[7]{Regenerative Recovery} You can use the \textit{regenerative recovery} ability as a standard action.
        \begin{freeability}{Regenerative Recovery}
            You regain a \glossterm{hit point}.
            After you use this ability a number of times equal to your base Constitution, you cannot use it again until you take a \glossterm{short rest}.

            \rankline
            \featlevel{13} You regain an additional \glossterm{hit point}.
            \featlevel{19} You may use this ability a number of times equal to twice your base Constitution.
        \end{freeability}

        \ff[10]{Greater Diehard} The bonus from your \textit{diehard} ability increases to \plus4.

        \ff[13]{Greater Regenerative Rest} Your \textit{regenerative rest} ability removes an additional \glossterm{vital wound}.

        \ff[16]{Enduring Constitution} You gain a \plus1 bonus to your base Constitution.

        \ff[19]{Supreme Diehard} The bonus from your \textit{diehard} ability increases to \plus6.
    \end{feat}

    \begin{feat}{Ride Specialization}{Skill}
        \featpre Ride as a mastered skill.

        \ff[1]{Specialization} You gain a \plus3 bonus to the Ride skill.

        \ff[4]{Mounted Defense} Your mount gains a \plus2 bonus to all defenses, up to a maximum of your own corresponding defense.

        \ff[7]{Mounted Warrior} The penalty you take when using a ranged weapon while mounted is decreased by 4.
        In addition, while you are mounted, you gain a \plus1 bonus to \glossterm{accuracy} with Mounted weapons (see \pcref{Mounted Weapon}).

        \ff[10]{Greater Specialization} The bonus from your \textit{specialization} ability increases to \plus6.

        \ff[13]{Greater Mounted Defense} The defense bonus from your \textit{mounted defense} ability increases to \plus4.

        \ff[16]{Supreme Specialization} The bonus from your \textit{specialization} ability increases to \plus9.

        \ff[19]{Greater Mounted Warrior} The penalty reduction from your \textit{mounted warrior} ability increases to \minus8.
        In addition, you gain a \plus1 bonus to Armor defense while mounted.
    \end{feat}

    \begin{feat}{Savage}{Combat}
        \featpre Base Strength of 2.

        \ff[1]{Brute Force} You gain a \plus2 bonus to \glossterm{accuracy} with the \textit{shove} and \textit{overrun} abilities (see \pcref{Shove}, and \pcref{Overrun}).
        If you get a \glossterm{critical hit} with the \textit{shove} ability, you can move the target a maximum distance equal to your movement speed.
        In addition, if you get a \glossterm{critical hit} with the \textit{overrun} ability, the target is knocked \glossterm{prone}.

        \ff[4]{Wall Slam} If you use the \textit{shove} ability to move a creature, and the creature's movement is interrupted by a solid obstacle, the obstacle and creature both take bludgeoning \glossterm{standard damage}.

        \ff[7]{Trample} If you use the \textit{overrun} ability to move through a creature, it takes bludgeoning \glossterm{standard damage} \minus2d.

        \ff[10]{Greater Brute Force} The accuracy bonus from your \textit{brute force} ability increases to \plus4.

        \ff[13]{Crush} You gain a \plus1d bonus to damage with your \textit{wall slam} and \textit{trample} abilities.

        \ff[16]{Inescapable} When you use the \textit{overrun} ability, you may choose not to allow creatures to try to avoid you.

        \ff[19]{Greater Crush} The damage bonuses from your \textit{crush} ability increase to \plus2d.

        \ff[19]{Supreme Brute Force} The accuracy bonus from your \textit{brute force} ability increases to \plus6.
    \end{feat}

    \begin{feat}{Sleight of Hand Specialization}{Skill}
        \featpre Sleight of Hand as a mastered skill.

        \ff[1]{Specialization} You gain a \plus3 bonus to the Sleight of Hand skill.

        \ff[4]{todo}

        \ff[7]{Extradimensional Concealment}[Magical] When you use the \textit{conceal object} ability, you can use the \textit{extradimensional pocket} ability.
        \begin{attuneability}{Extradimensional Pocket}[\glossterm{Attune} (self), \glossterm{Magical}]
            You conceal the object in a pocket dimension that cannot be accessed by nonmagical means.
            When your attunement to this ability ends, the object appears in a free hand.
            If you have no free hands, it drops to the ground.
        \end{attuneability}

        \ff[10]{Greater Specialization} The bonus from your \textit{specialization} ability increases to \plus6.

        \ff[13]{Greater Conceal Object} The size of object you can hide with your \textit{conceal object} ability increases by one size category.

        \ff[16]{Supreme Specialization} The bonus from your \textit{specialization} ability increases to \plus9.

        \ff[19]{todo} 
    \end{feat}

    \begin{feat}{Sniper}{Combat}
        \featpre Base Perception of 2.

        \ff[1]{Aim} You can use the \textit{aim} ability as a standard action.
        \begin{freeability}{Aim}[\glossterm{Focus}, \glossterm{Sustain} (minor)]
            Choose a creature or object within line of sight.
            You gain a \plus2 bonus to accuracy against the target.

            If you lose sight of the target for a full round, this effect ends.

            \rankline
            \featlevel{7} You also gain a \plus4 bonus to \glossterm{power} against the target if it is \glossterm{unaware} of you.
            \featlevel{13} The \glossterm{accuracy} bonus increases to \plus3.
            \featlevel{19} The \glossterm{power} bonus increases to \plus8.
        \end{freeability}

        \ff[4]{Distance Tolerance} You reduce your accuracy penalties from \glossterm{range increments} by 1.

        \ff[7]{Precise Shot} You ignore \glossterm{cover} (but not \glossterm{total cover}) with ranged attacks.

        \ff[10]{Greater Distance Tolerance} The penalty reduction from your \textit{distance tolerance} ability increases to 2.

        \ff[13]{Greater Precise Shot} You ignore \glossterm{concealment} with ranged attacks.

        \ff[16]{Supreme Distance Tolerance} The penalty reduction from your \textit{distance tolerance} ability increases to 3.

        \ff[19]{Sniper's Precision} You gain a \plus1 bonus to \glossterm{accuracy}.
    \end{feat}

    \begin{feat}{Social Insight Specialization}{Skill}
        \featpre Social Insight as a mastered skill.

        \ff[1]{Specialization} You gain a \plus3 bonus to the Social Insight skill.

        \ff[4]{Social Intuition} You take no penalty for making a social assessment ability after only a single round of observation (see \pcref{Social Assessment}).

        \ff[7]{Read Mind}[Magical] You can use the \textit{read mind} ability as a standard action.
        \begin{freeability}{Read Mind}[\glossterm{Emotion}, \glossterm{Sustain} (minor)]
            Make an attack vs. Mental against a creature within \rngclose range.
            Your \glossterm{accuracy} is equal to your Social Insight skill.
            \hit You know the target's current emotions.
            This grants you a \plus2 bonus to Deception, Persuasion, and Intimidate attacks and checks against the target.
            \crit You know the target's surface thoughts.
            This grants you a \plus5 bonus to Deception, Persuasion, and Intimidate attacks and checks against the target.

            \rankline
            \featlevel{13} The range increases to \rnglong range.
            \featlevel{19} You can use this ability as a \glossterm{minor action}.
        \end{freeability}

        \ff[10]{Greater Specialization} The bonus from your \textit{specialization} ability increases to \plus6.

        \ff[13]{Truthsense} Whenever a creature within a \arealarge radius \glossterm{emanation} from you that you can hear and see speaks the truth with no attempt at evasion, concealment, or creative wording, you automatically recognize that.
        You do not recognize truth in this way if a creature is using the Deception skill in any way, even if it is speaking the truth.

        \ff[16]{Supreme Specialization} The bonus from your \textit{specialization} ability increases to \plus9.

        \ff[19]{Greater Truthsense} The area of your \textit{truthsense} ability increases to a 200 foot radius.
    \end{feat}

    \begin{feat}{Spellsense Specialization}{Skill}
        \featpre Spellsense as a mastered skill.

        \ff[1]{Specialization} You gain a \plus3 bonus to the Spellsense skill.

        \ff[4]{Detect Spellcasting}[Magical] You can use the \textit{detect spellcasting} ability as a standard action.
        \begin{freeability}{Detect Spellcasting}[\glossterm{Knowledge}, \glossterm{Subtle}]
            Make a Spellsense attack against the Mental defense of a creature within \rngmed range.
            \hit You know whether the target is capable of casting spells.
            If the target can cast spells, you know what sources the target can cast spells from.
            \crit As above, except that you also know all \glossterm{mystic spheres} the target is capable of casting.
            This does not grant you knowledge of any specific spells the target knows.

            After using this ability on a target, you cannot use it again on the same target for 24 hours regardless of whether you hit or miss.

            \rankline
            \featlevel{10} The range increases to \rnglong.
            \featlevel{16} You gain the critical hit effect on any hit.
        \end{freeability}

        \ff[7]{Unweave Magic}[Magical] You can use the \textit{unweave magic} ability as a standard action.
        \begin{freeability}{Unweave Magic}[\glossterm{Mystic}]
            Make a Spellsense check on an active spell effect within \rngmed range.
            This can affect spells with the \glossterm{Sustain} tag, but it cannot affect spells with the \glossterm{Attune} tag.
            The \glossterm{difficulty rating} is equal to 5 \add the \glossterm{power} of the effect.
            Success means the effect is \glossterm{dismissed} if it is an effect that can be dismissed.

            \rankline
            \featlevel{13} You can target up to two spell effects within range.
            \featlevel{19} You can affect spells with the \glossterm{Attune} tag.
            The \glossterm{difficulty rating} to affect those spells is increased by 10.
        \end{freeability}

        \ff[10]{Greater Specialization} The bonus from your \textit{specialization} ability increases to \plus6.

        \ff[13]{Mystic Power} You gain a \plus1 bonus \glossterm{power} with \glossterm{magical} abilities.

        \ff[16]{Supreme Specialization} The bonus from your \textit{specialization} ability increases to \plus9.

        \ff[19]{Greater Mystic Power}[Magical] The bonus from your \textit{mystic power} ability increases to \plus2.
    \end{feat}

    \begin{feat}{Spellsword}{Spell}
        \featpre Ability to cast a spell.

        \ff[1]{Imbued Blow} You can use the \textit{imbued blow} ability as a standard action.
        \begin{freeability}{Imbued Blow}[\glossterm{Magical}]
            Make a melee \glossterm{strike} with a \plus1 bonus to \glossterm{accuracy}.
            Because this is a \glossterm{magical} ability, you use your \glossterm{power} with \glossterm{magical} abilities to determine your damage instead of your power with \glossterm{mundane} abilities.

            \rankline
            \featlevel{7} The accuracy bonus increases to \plus2.
            \featlevel{13} The accuracy bonus increases to \plus3.
            \featlevel{19} The accuracy bonus increases to \plus4.
        \end{freeability}

        \ff[4]{Spellsword Conduit}[\glossterm{Magical}] You can cast spells using a melee weapon as if it were an implement (see \pcref{Implements}).
        When you do, you reduce your \glossterm{focus penalty} by 2.

        \ff[7]{Imbue Weapon} When you cast a spell that does not have the \glossterm{AP}, \glossterm{Attune}, or \glossterm{Sustain} tags,
            you can use the \textit{imbue weapon} ability.
        \begin{attuneability}{Imbue Weapon}[\glossterm{Attune} (self), \glossterm{Magical}]
            The spell does not have its effect immediately.
            Instead, its power is imbued in a melee weapon you hold.
            An individual weapon can only be imbued with this ability once.

            When you use your \textit{imbued blow} ability to attack with that weapon, the spell takes effect on the target of your \textit{imbued blow} ability.
            After the spell takes effect this way, your attunement to this ability ends.
            The next time you take a \glossterm{short rest}, you regain the \glossterm{action point} you spent to attune to this ability.
        \end{attuneability}

        \ff[10]{Greater Spellsword Conduit}[\glossterm{Magical}] The penalty reduction from your \textit{spellsword conduit} ability increases to 2.

        \ff[13]{Powerful Imbuement} You gain a \plus2 bonus to \glossterm{power} with spells that you store with your \textit{imbue projectile} ability.

        \ff[16]{Greater Spellsword Conduit}[Magical] The penalty reduction from your \textit{spellsword conduit} ability increases to 3.

        \ff[19]{Greater Powerful Imbuement} The bonus from your \textit{powerful imbuement} ability increases to \plus4.
    \end{feat}

    \begin{feat}{Spellwarped}{General, Magical}
        \featpre Base Willpower of 1.

        \ff[1]{Mystic Sphere} You gain the ability to use arcane magic.
        You gain access to one arcane \glossterm{mystic sphere} (see \pcref{Arcane Mystic Spheres}).
        Each \glossterm{mystic sphere} has a set of \glossterm{spells} associated with it.

        You automatically learn all \glossterm{cantrips} from any mystic sphere you have access to.

        Arcane spells require both \glossterm{verbal components} and \glossterm{somatic components} to cast (see \pcref{Casting Components}).
        For details about mystic spheres and casting spells, see \pcref{Spell and Ritual Mechanics}.

        \ff[4]{Spell} You learn one 1st level \glossterm{spell} from your chosen \glossterm{mystic sphere}.
        You can also spend \glossterm{insight points} to learn one additional arcane spell per \glossterm{insight point}.
        Unless otherwise noted in a spell's description, casting a spell requires a \glossterm{standard action}.

        When you gain access to a spell level,
            you can exchange any number of spells you know for other spells,
            including spells of the higher level.

        \ff[7]{Spell Level} You gain the ability to cast 2nd level spells from your chosen \glossterm{mystic sphere}.

        \ff[10]{Spell Level} You gain the ability to cast 3rd level spells from your chosen \glossterm{mystic sphere}.

        \ff[13]{Spell Level} You gain the ability to cast 4th level spells from your chosen \glossterm{mystic sphere}.

        \ff[16]{Spell Level} You gain the ability to cast 5th level spells from your chosen \glossterm{mystic sphere}.

        \ff[19]{Spell Level} You gain the ability to cast 6th level spells from your chosen \glossterm{mystic sphere}.
    \end{feat}

    \begin{feat}{Sphere Focus: Aeromancy}{Casting, Magical}
        \featpre Access to the \sphere{Aeromancy} \glossterm{mystic sphere}.

        \ff[1]{Spell} You learn a spell from the \sphere{Aeromancy} \glossterm{mystic sphere}.
        When you gain access to a new spell level, you can change which spell you know from that \glossterm{mystic sphere}.

        \ff[4]{Favorable Winds} You gain a \plus1 \glossterm{magic bonus} to \glossterm{accuracy} with ranged \glossterm{strikes}.

        \ff[4]{Personal Updraft} You gain a \plus4 \glossterm{magic bonus} to the Jump skill.

        \ff[7]{Aeromancy Focus} You gain a \plus1 bonus to \glossterm{accuracy} with abilities from the \sphere{Aeromancy} \glossterm{mystic sphere}.

        \ff[10]{Greater Personal Updraft} You gain a \glossterm{glide speed} equal to your \glossterm{base speed} (see \pcref{Gliding}).
        If you already have a \glossterm{glide speed}, you can increase or decrease your glide speed whenever you glide by up to 20 feet (to a minimum of 10 feet).

        \ff[13]{Greater Favorable Winds} The bonus from your \textit{favorable winds} ability increases to \plus2.

        \ff[16]{Greater Aeromancy Focus} The bonus from your \textit{aeromancy focus} ability increases to \plus2.

        \ff[19]{Supreme Personal Updraft} As long as you are within 100 feet of the ground, you gain a \glossterm{fly speed} equal to your \glossterm{base speed} (see \pcref{Flying}).
    \end{feat}

    \begin{feat}{Sphere Focus: Aquamancy}{Casting, Magical}
        \featpre Access to the \sphere{Aquamancy} \glossterm{mystic sphere}.

        \ff[1]{Spell} You learn a spell from the \sphere{Aquamancy} \glossterm{mystic sphere}.
        When you gain access to a new spell level, you can change which spell you know from that \glossterm{mystic sphere}.

        \ff[4]{Swim Familiarity} You gain a \plus2 bonus to the Swim skill.
        In addition, you reduce your penalties for acting underwater by 2 (see \pcref{Underwater Combat}).

        \ff[7]{Aquamancy Focus} You gain a \plus1 bonus to \glossterm{accuracy} with abilities from the \sphere{Aquamancy} \glossterm{mystic sphere}.

        \ff[10]{Slippery Escapist} You gain a \plus2 bonus to the Flexibility skill.
        In addition, you gain a \plus2 bonus to defenses against the \textit{grapple} ability (see \pcref{Grapple}).

        \ff[13]{Greater Swim Familiarity} The Swim bonus from your \textit{swim familiarity} ability increases to \plus4.
        In addition, you take no penalties for acting underwater, except for those relating to using ranged weapons.

        \ff[16]{Greater Aquamancy Focus} The bonus from your \textit{aquamancy focus} ability increases to \plus2.

        \ff[19]{Create Flood} When you use the \spell{create water} cantrip, you can create up to ten gallons of water per \glossterm{power}.
    \end{feat}

    \begin{feat}{Sphere Focus: Astromancy}{Casting, Magical}
        \featpre Access to the \sphere{Astromancy} \glossterm{mystic sphere}.

        \ff[1]{Spell} You learn a spell from the \sphere{Astromancy} \glossterm{mystic sphere}.
        When you gain access to new spell levels, you can change which spell you know from that \glossterm{mystic sphere}.

        \ff[4]{Astral Spell Transit} You double your range with abilities from the \sphere{Astromancy} \glossterm{mystic sphere}.

        \ff[7]{Astromancy Focus} You gain a \plus1 bonus to \glossterm{accuracy} with abilities from the \sphere{Astromancy} \glossterm{mystic sphere}.

        \ff[10]{Personal Translocation} You can use the \textit{personal translocation} ability as a \glossterm{move action}.
        \begin{freeability}{Personal Translocation}[\glossterm{Teleportation}]
            You teleport to an unoccupied destination within \rngclose range.
        \end{freeability}

        \ff[13]{Greater Astral Spell Transit} When determining whether you have \glossterm{line of effect} to a particular location with abilites from the \sphere{Astromancy} \glossterm{mystic sphere}, you can ignore all physical obstacles in a single five-foot span.
        This can allow you to use abilities through solid walls, though it does not grant you the ability to see through the wall.

        \ff[16]{Greater Astromancy Focus} The bonus from your \textit{astromancy focus} ability increases to \plus2.

        \ff[19]{Greater Personal Translocation} You can use the \textit{personal translocation} ability as a \glossterm{minor action}.
    \end{feat}

    \begin{feat}{Sphere Focus: Barrier}{Casting, Magical}
        \featpre Access to the \sphere{Barrier} \glossterm{mystic sphere}.

        \ff[1]{Spell} You learn a spell from the \sphere{Barrier} \glossterm{mystic sphere}.
        When you gain access to new spell levels, you can change which spell you know from that \glossterm{mystic sphere}.

        \ff[4]{Burst Ward} You can use the \textit{burst ward} ability as a standard action.
        \begin{freeability}{Burst Ward}[\glossterm{Swift}]
            You take half damage from all attacks this round.
            This halving is applied before \glossterm{resistances} and similar abilities.

            \rankline
            \featlevel{10} You also gain a \plus1 bonus to all defenses.
            \featlevel{16} The defense bonus increases to \plus2.
        \end{freeability}

        \ff[7]{Potent Barrier} You gain a \plus1 bonus to \glossterm{power} with abilities from the \sphere{Barrier} \glossterm{mystic sphere}.

        \ff[10]{Personal Shield} You gain a \plus1 bonus to Armor defense.

        \ff[13]{Greater Burst Ward} Your \textit{burst ward} ability also gives you a \plus2 bonus to all defenses.

        \ff[16]{Greater Potent Barrier} The bonus from your \textit{potent barrier} ability increases to \plus2.

        \ff[19]{Greater Personal Shield} The bonus from your \textit{personal shield} ability increases to \plus2.
    \end{feat}

    \begin{feat}{Sphere Focus: Bless}{Casting, Magical}
        \featpre Access to the \sphere{Bless} \glossterm{mystic sphere}.

        \ff[1]{Spell} You learn a spell from the \sphere{Bless} \glossterm{mystic sphere}.
        When you gain access to a new spell level, you can change which spell you know from that \glossterm{mystic sphere}.

        \ff[4]{Inspiring Blessing} Each creature that is \glossterm{attuned} to a spell you cast from the \sphere{Bless} \glossterm{mystic sphere} increases its maximum \glossterm{hit points} by 1.
        When a creature's maximum hit points increase in this way, it also increases its current \glossterm{hit points} by one.
        When this effect ends, the creature loses one of its \glossterm{hit points}.

        \ff[7]{Simple Blessing} Spells you cast from the Bless mystic sphere do not have the \glossterm{Focus} tag (see \pcref{Focus}).

        \ff[10]{Personal Blessing} You gain an additional \glossterm{action point}.
        You can only use this action point to attune to spells you cast from the \sphere{Bless} mystic sphere.

        \ff[16]{Greater Inspiring Blessing} The number of hit points granted by your \textit{inspiring blessing} ability increases to two.
        This affects the maximum hit points granted, the current hit points gained, and the current hit points lost.

        \ff[19]{Greater Personal Blessing} The number of additional action points granted by your \textit{personal blessing} ability increases to two.
    \end{feat}

    \begin{feat}{Sphere Focus: Biomancy}{Casting, Magical}
        \featpre Access to the \sphere{Biomancy} \glossterm{mystic sphere}.

        \ff[1]{Spell} You learn a spell from the \sphere{biomancy} \glossterm{mystic sphere}.
        When you gain access to new spell levels, you can change which spell you know from that \glossterm{mystic sphere}.

        \ff[4]{Biological Control} You gain a \plus2 bonus to Fortitude defense and are immune to \glossterm{diseases}.
        In addition, you need half the normal amount of rest and sleep each day to function normally.
        For example, a human would only need four hours of sleep per night.
        This does not reduce the time required for you to take a \glossterm{long rest}.

        \ff[7]{Biomancy Focus} You gain a \plus1 bonus to \glossterm{accuracy} with abilities from the \sphere{Biomancy} \glossterm{mystic sphere}.

        \ff[10]{Mutable Self} You gain an additional \glossterm{action point}.
        You can only use this action point to \glossterm{attune} to spells from the \sphere{biomancy} mystic sphere.

        \ff[13]{Greater Biological Control} You are immune to \glossterm{poisons}.
        In addition, the amount of rest and sleep you need each day is reduced to a quarter of the normal value.
        For example, a human would only need two hours of sleep per night.

        \ff[16]{Greater Biomancy Focus} The bonus from your \textit{biomancy focus} ability increases to \plus2.

        \ff[19]{Greater Mutable Self} The number of action points you gain from your \textit{mutable self} ability increases to 2.
    \end{feat}

    \begin{feat}{Sphere Focus: Channel Divinity}{Casting, Magical}
        \featpre Access to the \sphere{Channel Divinity} \glossterm{mystic sphere}.

        \ff[1]{Spell} You learn a spell from the \sphere{Channel divinity} \glossterm{mystic sphere}.
        When you gain access to a new spell level, you can change which spell you know from that \glossterm{mystic sphere}.

        \ff[4]{Divine Intervention} You gain a \plus3 bonus to any roll that you use the \textit{desperate exertion} ability on (see \pcref{Desperate Exertion}).
        This bonus stacks with the normal \plus2 bonus provided by that ability.

        \ff[7]{Channel Divinity Focus} You gain a \plus1 bonus to \glossterm{accuracy} with abilities from the \sphere{Channel divinity} \glossterm{mystic sphere}.

        % This is pretty weird
        \ff[10]{Divine Servant} Once per \glossterm{long rest}, if you pray to your deity in an area sacred to your deity, you can regain a spent \glossterm{action point}.
        This takes a \glossterm{minor action}.

        \ff[13]{Greater Divine Intervention} The bonus from your \textit{divine intervention} ability increases to \plus5.

        \ff[16]{Greater Channel Divinity Focus} The bonus from your \textit{channel divinity focus} ability increases to \plus2.

        \ff[19]{Greater Divine Champion} The bonus from your \textit{divine champion} ability increases to \plus3.
    \end{feat}

    \begin{feat}{Sphere Focus: Chronomancy}{Casting, Magical}
        \featpre Access to the \sphere{Chronomancy} \glossterm{mystic sphere}.

        \ff[1]{Spell} You learn a spell from the \sphere{Chronomancy} \glossterm{mystic sphere}.
        When you gain access to new spell levels, you can change which spell you know from that \glossterm{mystic sphere}.

        \ff[4]{Accelerated Movement} You gain a \plus5 foot bonus to your \glossterm{base speed}.

        \ff[7]{Chronomancy Focus} You gain a \plus1 bonus to \glossterm{accuracy} with abilities from the \sphere{Chronomancy} \glossterm{mystic sphere}.

        \ff[10]{Accelerated Mind} You can perform primarily mental tasks more quickly as normal.
        Actions that would normally take a \glossterm{standard action} instead take a \glossterm{minor action}.
        Long-term activities can be done twice as quickly as normal.
        This includes reading books, searching areas, identifying magical effects with the Spellsense skill, and other similar activities.
        It does not affect spellcasting, performing rituals, or other similar magical abilities.

        \ff[13]{Greater Accelerated Movement} The speed bonus from your \textit{accelerated movement} ability increases to \plus10 feet.

        \ff[16]{Greater Chronomancy Focus} The bonus from your \textit{chronomancy focus} ability increases to \plus2.

        \ff[19]{Greater Accelerated Mind} Once per round, you can perform one primarily mental task that would normally take a \glossterm{standard action} as a \glossterm{free action}.
        In addition, the speed increase for long-term tasks from your \textit{accelerated mind} ability increases to five times normal speed.
    \end{feat}

    \begin{feat}{Sphere Focus: Compel}{Casting, Magical}
        \featpre Access to the \sphere{Compel} \glossterm{mystic sphere}.

        \ff[1]{Spell} You learn a spell from the \sphere{Compel} \glossterm{mystic sphere}.
        When you gain access to new spell levels, you can change which spell you know from that \glossterm{mystic sphere}.

        \ff[4]{Mind Fragments} When you use \glossterm{Compulsion} abilities, you can affect creatures that are immune to \glossterm{Compulsion} abilities due to not having a mind.
        You take a \minus5 penalty to accuracy on attacks against such creatures.
        This does not allow you to affect creatures who are immune to \glossterm{Compulsion} abilities for other reasons.

        \ff[7]{Compel Focus} You gain a \plus1 bonus to \glossterm{accuracy} with abilities from the \sphere{Compel} \glossterm{mystic sphere}.

        \ff[10]{Mental Dominance} You gain a \plus2 bonus to Mental defense.

        \ff[13]{Greater Mind Fragments} The accuracy penalty from your \textit{mind fragments} ability is reduced to \minus2.

        \ff[16]{Greater Compel Focus} The bonus from your \textit{compel focus} ability increases to \plus2.

        \ff[19]{Greater Mental Dominance} The bonus from your \textit{mental dominance} ability increases to \plus4.
    \end{feat}

    \begin{feat}{Sphere Focus: Cryomancy}{Casting, Magical}
        \featpre Access to the \sphere{Cryomancy} \glossterm{mystic sphere}.

        \ff[1]{Spell} You learn a spell from the \sphere{Cryomancy} \glossterm{mystic sphere}.
        When you gain access to a new spell level, you can change which spell you know from that \glossterm{mystic sphere}.

        \ff[4]{Cold Resistance} You gain a bonus equal to your \glossterm{power} to your \glossterm{resistances} against cold damage.

        \ff[4]{Lingering Chill} Whenever you \glossterm{wound} a creature with a \glossterm{critical hit} using cold damage, that creature is \glossterm{chilled} as a \glossterm{condition}.

        \ff[7]{Cryomancy Focus} You gain a \plus1 bonus to \glossterm{accuracy} with abilities from the \sphere{Cryomancy} \glossterm{mystic sphere}.

        \ff[10]{Potent Cryomancy} You gain a \plus1 bonus to \glossterm{power} with abilities from the \sphere{Cryomancy} \glossterm{mystic sphere}.

        \ff[13]{Greater Cold Resistance} The bonus from your \textit{cold resistance} ability increases to twice your \glossterm{power}.

        \ff[13]{Greater Lingering Chill} Your \textit{lingering chill} ability also makes that creature \glossterm{immobilized} until the end of the next round.

        \ff[16]{Greater Cryomancy Focus} The bonus from your \textit{cryomancy focus} ability increases to \plus2.

        \ff[19]{Greater Potent Cryomancy} The bonus from your \textit{potent cryomancy} ability increases to \plus2.
    \end{feat}

    \begin{feat}{Sphere Focus: Delusion}{Casting, Magical}
        \featpre Access to the \sphere{Delusion} \glossterm{mystic sphere}.

        \ff[1]{Spell} You learn a spell from the \sphere{Delusion} \glossterm{mystic sphere}.
        When you gain access to new spell levels, you can change which spell you know from that \glossterm{mystic sphere}.

        \ff[4]{Subtle Influence} You gain a \plus2 bonus to \glossterm{accuracy} with spells from the Delusion mystic sphere against \glossterm{unaware} creatures.
        In addition, the \glossterm{difficulty rating} to identify your \glossterm{Emotion} abilities with Spellsense, and to identify their effects with the Social Insight skill, increases by 5.

        \ff[7]{Delusion Focus} You gain a \plus1 bonus to \glossterm{accuracy} with abilities from the \sphere{Delusion} \glossterm{mystic sphere}.

        \ff[10]{Enchanting Presence} You gain a \plus2 bonus to the Intimidate and Persuasion skills.

        \ff[13]{Greater Subtle Influence} The accuracy bonus from your \textit{subtle influence} ability increases to \plus4.
        In addition, the \glossterm{difficulty rating} increase from your that ability increases to \plus10.

        \ff[16]{Greater Delusion Focus} The bonus from your \textit{delusion focus} ability increases to \plus2.

        \ff[19]{Greater Enchanting Presence} The bonuses from your \textit{enchanting presence} ability increase to \plus4.
    \end{feat}

    \begin{feat}{Sphere Focus: Electromancy}{Casting, Magical}
        \featpre Access to the \sphere{Electromancy} \glossterm{mystic sphere}.

        \ff[1]{Spell} You learn a spell from the \sphere{Electromancy} \glossterm{mystic sphere}.
        When you gain access to a new spell level, you can change which spell you know from that \glossterm{mystic sphere}.

        \ff[4]{Electricity Resistance} You gain a bonus equal to your \glossterm{power} to your \glossterm{resistances} against electricity damage.

        \ff[4]{Lingering Shock} When you deal electricity damage to a creature with a \glossterm{critical hit}, that creature is \glossterm{dazed} as a \glossterm{condition}.

        \ff[7]{Electromancy Focus} You gain a \plus1 bonus to \glossterm{accuracy} with abilities from the \sphere{Electromancy} \glossterm{mystic sphere}.

        \ff[10]{Potent Electromancy} You gain a \plus1 bonus to \glossterm{power} with abilities from the \sphere{Electromancy} \glossterm{mystic sphere}.

        \ff[13]{Greater Electricity Resistance} Your bonus from the \textit{electricity resistance} ability increases to twice your \glossterm{power}.

        \ff[13]{Greater Lingering Shock} Your \textit{lingering shock} ability makes the creature \glossterm{stunned} instead of \glossterm{dazed}.

        \ff[16]{Greater Electromancy Focus} The bonus from your \textit{electromancy focus} ability increases to \plus2.

        \ff[19]{Greater Potent Electromancy} The bonus from your \textit{potent electromancy} ability increases to \plus2.
    \end{feat}

    \begin{feat}{Sphere Focus: Fabrication}{Casting, Magical}
        \featpre Access to the \sphere{Fabrication} \glossterm{mystic sphere}.

        \ff[1]{Spell} You learn a spell from the \sphere{Fabrication} \glossterm{mystic sphere}.
        When you gain access to a new spell level, you can change which spell you know from that \glossterm{mystic sphere}.

        \ff[4]{Crafting Familiarity} You gain a \plus2 bonus to all Craft skills.
        In addition, any objects you create with abilities from the \sphere{Fabrication} \glossterm{mystic sphere} gain a bonus equal to your \glossterm{power} to their \glossterm{vital resistance}.

        \ff[7]{Fabrication Focus} You gain a \plus1 bonus to \glossterm{accuracy} with abilities from the \sphere{Fabrication} \glossterm{mystic sphere}.

        \ff[10]{Persistent Manifestation} Once per round, you can \glossterm{sustain} a spell from the Fabrication mystic sphere as a \glossterm{free action}.

        \ff[13]{Greater Crafting Familiarity} The Craft bonus from your \textit{crafting familiarity} ability increases to \plus4.
        In addition, the \glossterm{vital resistance} bonus increases to twice your \glossterm{power}.

        \ff[16]{Greater Fabrication Focus} The bonus from your \textit{fabrication focus} ability increases to \plus2.

        \ff[19]{Greater Persistent Manifestation} You can use your \textit{persistent manifestation} ability twice per round.
    \end{feat}

    \begin{feat}{Sphere Focus: Photomancy}{Casting, Magical}
        \featpre Access to the \sphere{Photomancy} \glossterm{mystic sphere}.

        \ff[1]{Spell} You learn a spell from the \sphere{Photomancy} \glossterm{mystic sphere}.
        When you gain access to a new spell level, you can change which spell you know from that \glossterm{mystic sphere}.

        \ff[4]{Augmented Vision} You gain a \plus2 \glossterm{magic bonus} to the Awareness skill.
        In addition, you gain the \glossterm{low-light vision} ability, allowing you to treat sources of light as if they had double their normal illumination range.

        \ff[7]{Photomancy Focus} You gain a \plus1 bonus to \glossterm{accuracy} with abilities from the \sphere{Photomancy} \glossterm{mystic sphere}.

        \ff[10]{Certain Sight} You are immune to being \glossterm{dazzled} and \glossterm{blinded}.

        \ff[13]{Greater Augmented Vision} The bonus from your \textit{augmented vision} ability increases to \plus4.
        In addition, the benefit of your \glossterm{low-light vision} ability doubles, allowing you to treat sources of light as if they had four times their norrmal illumination range.

        \ff[16]{Greater Photomancy Focus} The bonus from your \textit{photomancy focus} ability increases to \plus2.

        \ff[19]{Supreme Augmented Vision} The bonus from your \textit{augmented vision} ability increases to \plus6.
        In addition, you gain the \glossterm{truesight} ability with a 50 foot range.
        If you already have the \glossterm{truesight} ability, you increase its range by 50 feet.
    \end{feat}

    \begin{feat}{Sphere Focus: Polymorph}{Casting, Magical}
        \featpre Access to the \sphere{Polymorph} \glossterm{mystic sphere}.

        \ff[1]{Spell} You learn a spell from the \sphere{Polymorph} \glossterm{mystic sphere}.
        When you gain access to new spell levels, you can change which spell you know from that \glossterm{mystic sphere}.

        \ff[4]{Reshaper} As a standard action, you can use the \textit{alter self} ability.
        In addition, when you use the \spell{alter object} cantrip, you can use your \glossterm{power} in place of your Craft skill.
        \begin{freeability}{Alter Self}[\glossterm{Shaping}]
            Make a Disguise check to alter your appearance (see \pcref{Disguise Creature}), except that you can use your \glossterm{power} in place of your Disguise skill.
            You can only alter your physical body, not your clothes or equipment.

            This ability lasts until you use it again.
        \end{freeability}

        \ff[7]{Polymorph Focus} You gain a \plus1 bonus to \glossterm{accuracy} with abilities from the \sphere{Polymorph} \glossterm{mystic sphere}.

        \ff[10]{Malleable Flesh} You gain a \plus5 bonus to defenses when determining whether a \glossterm{strike} gets a \glossterm{critical hit} against you instead of a normal hit.

        \ff[13]{Greater Reshaper} When you use the \spell{alter object} cantrip, you can accomplish work that would take up to an hour with a normal Craft check.
        In addition, you can use the \textit{alter poison} ability as a standard action.
        \begin{freeability}{Alter Poison}[\glossterm{Shaping}, \glossterm{Sustain} (minor)]
            Make an attack vs. Fortitude against a creature within \rngclose range.
            \hit Any poison in the target's system is neutralized.
            It stops suffering any additional effects from poisons in its system.
            As long as the effect lasts, it is immune to all poisons.
            In addition, the target's \glossterm{mundane} poisons, including natural attacks that inflict poison, have no effect.
        \end{freeability}

        \ff[16]{Greater Polymorph Focus} The bonus from your \textit{polymorph focus} ability increases to \plus2.

        \ff[19]{Greater Malleable Flesh} You are immune to \glossterm{critical hits} from \glossterm{strikes}.
    \end{feat}

    \begin{feat}{Sphere Focus: Pyromancy}{Casting, Magical}
        \featpre Access to the \sphere{Pyromancy} \glossterm{mystic sphere}.

        \ff[1]{Spell} You learn a spell from the \sphere{Pyromancy} \glossterm{mystic sphere}.
        When you gain access to new spell levels, you can change which spell you know from that \glossterm{mystic sphere}.

        \ff[4]{Fire Resistance} You gain a bonus equal to your \glossterm{power} to your \glossterm{resistances} against fire damage.

        \ff[4]{Lingering Flame} Whenever you \glossterm{wound} a creature with a \glossterm{critical hit} using fire damage, that creature takes \glossterm{environmental} fire \glossterm{standard damage} \minus2d at the end of the next round.
        Since this damage is environmental damage, it has no effect if it does not \glossterm{wound} the target (see \pcref{Environmental Damage}).
        This effect can be removed before the target takes damage if it makes a \glossterm{difficulty rating} 10 Dexterity check as a \glossterm{move action} to put out the flames.
        Dropping \glossterm{prone} as part of this action gives a \plus5 bonus to this check.

        \ff[7]{Pyromancy Focus} You gain a \plus1 bonus to \glossterm{accuracy} with abilities from the \sphere{Pyromancy} \glossterm{mystic sphere}.

        \ff[10]{Friendly Fire} Whenever you deal fire damage to your \glossterm{allies}, you deal half damage.

        \ff[13]{Greater Fire Resistance} Your bonus from the \textit{fire resistance} ability increases to twice your \glossterm{power}.

        \ff[13]{Greater Lingering Flame} Your \textit{lingering flame} ability makes the target \glossterm{ignited} as a \glossterm{condition} instead of dealing damage directly.
        This condition can still be removed in the same way.

        \ff[16]{Greater Pyromancy Focus} The bonus from your \textit{pyromancy focus} ability increases to \plus2.

        \ff[19]{Greater Potent Pyromancy} The bonus from your \textit{potent pyromancy} ability increases to \plus2.
    \end{feat}

    \begin{feat}{Sphere Focus: Revelation}{Casting, Magical}
        \featpre Access to the \sphere{Revelation} \glossterm{mystic sphere}.

        \ff[1]{Spell} You learn a spell from the \sphere{Revelation} \glossterm{mystic sphere}.
        When you gain access to new spell levels, you can change which spell you know from that \glossterm{mystic sphere}.

        \ff[4]{Truthsense} You gain a \plus2 \glossterm{magic bonus} to all Knowledge skills and the Social Insight skill.

        \ff[7]{Revelation Focus} You gain a \plus1 bonus to \glossterm{accuracy} with abilities from the \sphere{Revelation} \glossterm{mystic sphere}.

        \ff[10]{Truesight} You gain the \glossterm{truesight} ability with a 50 foot range.
        If you already have the \glossterm{truesight} ability, you increase its range by 50 feet.

        \ff[13]{Greater Truthsense} The bonus from your \textit{truthsense} ability increases to \plus4.

        \ff[16]{Greater Revelation Focus} The bonus from your \textit{revelation focus} ability increases to \plus2.

        \ff[19]{Greater Truesight} The range of your \glossterm{truesight} ability increases by 100 feet.
    \end{feat}

    \begin{feat}{Sphere Focus: Summoning}{Casting, Magical}
        \featpre Access to the \sphere{Summoning} \glossterm{mystic sphere}.

        \ff[1]{Spell} You learn a spell from the \sphere{Summoning} \glossterm{mystic sphere}.
        When you gain access to a new spell level, you can change which spell you know from that \glossterm{mystic sphere}.

        \ff[4]{Fortified Summons} Creatures you create with the \sphere{Summoning} \glossterm{mystic sphere} have their maximum \glossterm{hit points} increased by 2.

        \ff[4]{Resummon} You can use the \textit{resummon} ability as a \glossterm{minor action}.
        \begin{freeability}{Resummon}
            Choose one creature or object that you summoned with an ability from the \sphere{Summoning} \glossterm{mystic sphere}.
            You teleport the target into an unoccupied space on stable ground within \rngmed range of you.
        \end{freeability}

        \ff[7]{Guided Summons} Creatures you create with the \sphere{Summoning} \glossterm{mystic sphere} gain a \plus1 bonus to \glossterm{accuracy}.

        \ff[10]{Augmented Summons} Creatures you create with abilities from the \sphere{Summoning} spell gain an \glossterm{action point}.

        \ff[13]{Greater Fortified Summons} The number of additional \glossterm{hit points} from your \textit{fortified summoning} ability increases to 3.

        \ff[13]{Greater Resummon} The range of your \textit{resummon} ability increases to \rnglong.
        In addition, you can choose up to two creatures to teleport instead of only one.
        Each target can be teleported to a different location within range.

        \ff[16]{Greater Guided Summons} The bonus from your \textit{guided summons} ability increases to \plus2.

        \ff[19]{Greater Augmented Summons} The number of action points granted by your \textit{augmented summons} ability increases to two.
    \end{feat}

    \begin{feat}{Sphere Focus: Telekinesis}{Casting, Magical}
        \featpre Access to the \sphere{Telekinesis} \glossterm{mystic sphere}.

        \ff[1]{Spell} You learn a spell from the \sphere{Telekinesis} \glossterm{mystic sphere}.
        When you gain access to new spell levels, you can change which spell you know from that \glossterm{mystic sphere}.

        \ff[4]{Rapid Distant Hand} You can use the \spell{distant hand} \glossterm{cantrip} as a \glossterm{minor action}, and you can \glossterm{sustain} it as a \glossterm{minor action}.

        \ff[4]{Telekinetic Strike} You can use the \textit{telekinetic strike} ability as a standard action.
        \begin{freeability}{Telekinetic Strike}[Magical]
            Make a \glossterm{strike} with a weapon you are controlling using the \spell{distant hand} cantrip.
            Because this is a \glossterm{magical} ability, you use your \glossterm{power} with \glossterm{magical} abilities to determine your damage instead of your power with \glossterm{mundane} abilities.
        \end{freeability}

        \ff[7]{Telekinesis Focus} You gain a \plus1 bonus to \glossterm{accuracy} with abilities from the \sphere{Telekinesis} \glossterm{mystic sphere}.

        \ff[10]{Personal Levitation} You gain a \plus4 \glossterm{magic bonus} to the Jump skill.
        In addition, as a \glossterm{free action}, you can slow your fall while falling.
        If you do, you fall at a rate of 50 feet per round, preventing you from taking falling damage when you hit the ground.

        \ff[13]{Greater Distant Hand} Your range with the \spell{distant hand} cantrip increases to \rngmed.
        In addition, the distance you can move the target each round increases to 30 feet.

        \ff[13]{Greater Telekinetic Strike} You gain a \plus1d bonus to damage with your \textit{telekinetic strike} ability.

        \ff[16]{Greater Telekinesis Focus} The bonus from your \textit{telekinesis focus} ability increases to \plus2.

        \ff[19]{Greater Personal Levitation} As long as you are within 50 feet above a surface that could support your weight, you can choose to float in midair, unaffected by gravity.
        As a \glossterm{minor action}, you can move yourself up to ten feet in any direction.
    \end{feat}

    \begin{feat}{Sphere Focus: Terramancy}{Casting, Magical}
        \featpre Access to the \sphere{Terramancy} \glossterm{mystic sphere}.

        \ff[1]{Spell} You learn a spell from the \sphere{Terramancy} \glossterm{mystic sphere}.
        When you gain access to a new spell level, you can change which spell you know from that \glossterm{mystic sphere}.

        \ff[4]{Heart of Stone} You gain a \plus2 bonus to Fortitude defense.

        \ff[7]{Terramancy Focus} You gain a \plus1 bonus to \glossterm{accuracy} with abilities from the \sphere{Terramancy} \glossterm{mystic sphere}.

        \ff[10]{Stonemason} You may treat worked stone as if it were earth for the purpose of spells from the \sphere{Terramancy} \glossterm{mystic sphere}.

        \ff[13]{Body of Stone} You gain a \plus1 bonus to Armor defense.

        \ff[16]{Greater Terramancy Focus} The bonus from your \textit{terramancy focus} ability increases to \plus2.

        \ff[19]{Earthen Alloy} You may treat metal as if it were earth for the purpose of spells from the \sphere{Terramancy} \glossterm{mystic sphere}.
    \end{feat}

    \begin{feat}{Sphere Focus: Thaumaturgy}{Casting, Magical}
        \featpre Access to the \sphere{Thaumaturgy} \glossterm{mystic sphere}.

        \ff[1]{Spell} You learn a spell from the \sphere{Thaumaturgy} \glossterm{mystic sphere}.
        When you gain access to new spell levels, you can change which spell you know from that \glossterm{mystic sphere}.

        \ff[4]{Counterspell} You can use the \textit{counterspell} ability as a standard action.
        \begin{freeability}{Counterspell}[\glossterm{Swift}]
            Choose a creature within \rngmed range of you.
            If the target is casting a spell or begins casting a spell this round, you can attempt to counter the spell.
            When you do, if your maximum spell level is at least as high as the target's maximum spell level, their spell has no effect when it resolves.
            Otherwise, make a contested \glossterm{power} check against the target, using your power with this ability against the target's power with the spell it is casting.
            If you win, the target's spell has no effect when it resolves.

            \rankline
            \featlevel{10} You may target an additional creature within range.
            \featlevel{16} You may cause each target to \glossterm{miscast} its spell instead of causing the spell to have no effect.
        \end{freeability}

        \ff[7]{Thaumaturgy Focus} You gain a \plus1 bonus to \glossterm{accuracy} with abilities from the \sphere{Thaumaturgy} \glossterm{mystic sphere}.

        \ff[10]{Magic Tolerance} You gain a \plus1 bonus to \glossterm{defenses} against \glossterm{magical} abilities.

        \ff[13]{Thaumatic Substitution} Whenever you cast a spell that deals damage, you may change the type of damage to energy damage.
        This does not change any other aspects of the spell.

        \ff[16]{Greater Thaumaturgy Focus} The bonus from your \textit{thaumaturgy focus} ability increases to \plus2.

        \ff[19]{Greater Magic Tolerance} The bonus from your \textit{magic tolerance} ability increases to \plus2.
    \end{feat}

    \begin{feat}{Sphere Focus: Umbramancy}{Casting, Magical}
        \featpre Access to the \sphere{Umbramancy} \glossterm{mystic sphere}.

        \ff[1]{Spell} You learn a spell from the \sphere{Umbramancy} \glossterm{mystic sphere}.
        When you gain access to new spell levels, you can change which spell you know from that \glossterm{mystic sphere}.

        \ff[4]{Darkvision} You gain \glossterm{darkvision} with a 50 foot range, allowing you to see in complete darkness clearly.
        If you already have that ability, you increase its range by 50 feet.

        \ff[7]{Umbramancy Focus} You gain a \plus1 bonus to \glossterm{accuracy} with abilities from the \sphere{Umbramancy} \glossterm{mystic sphere}.

        \ff[10]{Greater Suppress Light} You can cast the \spell{suppress light} \glossterm{cantrip} from the Umbramancy mystic sphere as a \glossterm{minor action}.

        \ff[13]{Darksight} The range of your darkvision ability increases by 100 feet.
        In addition, your darkvision is not disabled by being in \glossterm{bright illumination}.

        \ff[13]{Greater Reflexive Concealment} The bonuses from your \textit{reflexive concealment} ability increase to \plus4.

        \ff[16]{Greater Umbramancy Focus} The bonus from your \textit{Umbramancy focus} ability increases to \plus2.

        \ff[19]{Supreme Suppress Light} You can both cast and \glossterm{sustain} the \spell{suppress light} cantrip as a \glossterm{free action}.
    \end{feat}

    \begin{feat}{Sphere Focus: Verdamancy}{Casting, Magical}
        \featpre Access to the \sphere{Verdamancy} \glossterm{mystic sphere}.

        \ff[1]{Spell} You learn a spell from the \sphere{Verdamancy} \glossterm{mystic sphere}.
        When you gain access to a new spell level, you can change which spell you know from that \glossterm{mystic sphere}.

        \ff[4]{Verdant Allies} Your speed is not reduced when moving in light or heavy \glossterm{undergrowth}.
        In addition, you can ignore \glossterm{concealment} from plants when attacking.

        \ff[7]{Verdamancy Focus} You gain a \plus1 bonus to \glossterm{accuracy} with abilities from the \sphere{Verdamancy} \glossterm{mystic sphere}.

        \ff[10]{Creative Genesis} You can cast spells from the \sphere{Verdamancy} \glossterm{mystic sphere} as if non-arable ground of any kind was arable earth.

        \ff[13]{Greater Verdant Allies} You can ignore all \glossterm{cover} and \glossterm{concealment} from plants whenever doing so would be beneficial to you.
        For example, creatures cannot use plants to hide from you.

        \ff[16]{Greater Verdamancy Focus} The bonus from your \textit{verdamancy focus} ability increases to \plus2.

        % Seems weird for an 19th level; should stay in sync with the druid wild aspect
        \ff[19]{Supreme Verdant Allies} The movement penalties from \glossterm{undergrowth} are doubled for enemies within a \areahuge radius emanation from you.
    \end{feat}

    \begin{feat}{Sphere Focus: Vivimancy}{Casting, Magical}
        \featpre Access to the \sphere{Vivimancy} \glossterm{mystic sphere}.

        \ff[1]{Spell} You learn a spell from the \sphere{Vivimancy} \glossterm{mystic sphere}.
        When you gain access to new spell levels, you can change which spell you know from that \glossterm{mystic sphere}.

        \ff[4]{Personal Vitality} You gain a \plus1 bonus to Fortitude defense and are immune to being \glossterm{sickened}.

        \ff[7]{Vivimancy Focus} You gain a \plus1 bonus to \glossterm{accuracy} with abilities from the \sphere{Vivimancy} \glossterm{mystic sphere}.

        \ff[10]{Life Suppression} You gain a \plus4 bonus to defenses against abilities that only affect living creatures.

        \ff[13]{Greater Personal Vitality} The bonus to Fortitude defense from your \textit{fortified life} ability increases to \plus2.
        In addition, you are immune to being \glossterm{nauseated}.

        \ff[16]{Greater Vivimancy Focus} The bonus from your \textit{vivimancy focus} ability increases to \plus2.

        \ff[19]{Greater Life Suppression} You are no longer considered a living creature for the purpose of attacks against you.
        This means that attacks which only affect living creatures have no effect against you.
    \end{feat}

    \begin{feat}{Stealth Specialization}{Skill}
        \featpre Stealth as a mastered skill.

        \ff[1]{Specialization} You gain a \plus3 bonus to the Stealth skill.

        \ff[4]{Ambush the Unwary} You gain a \plus2 bonus to \glossterm{power} against \glossterm{unaware} creatures.

        \ff[7]{Movement Tolerance} Your penalties for moving while hiding are reduced by 5.
        This allows you to move at half speed without penalty.

        \ff[10]{Greater Specialization} The bonus from your \textit{specialization} ability increases to \plus6.

        \ff[13]{Hide in Plain Sight} You can use the \textit{hide} ability even while observed.
        You take take a \minus10 penalty to the Stealth check when hiding in this way, and you still need \glossterm{cover} or \glossterm{concealment} to hide.

        \ff[16]{Supreme Specialization} The bonus from your \textit{specialization} ability increases to \plus9.

        \ff[19]{Greater Hide in Plain Sight} Creatures observing you while you try to hide gain a \plus5 bonus to checks to notice you instead of a \plus10 bonus.
    \end{feat}

    \begin{feat}{Survival Specialization}{Skill}
        \featpre Survival as a mastered skill.

        \ff[1]{Specialization} You gain a \plus3 bonus to the Survival skill.

        \ff[4]{Terrain Tolerance} You ignore \glossterm{difficult terrain} and harmful natural terrain of any kind.
        If a skill check, such as Climb or Swim, would normally be required to move through the terrain, this ability does not help.

        \ff[7]{Rapid Tracker}
        While following trails with the \textit{track} ability, you can move at your normal speed while following tracks without taking the normal \minus5 penalty.

        \ff[10]{Greater Specialization} The bonus from your \textit{specialization} ability increases to \plus6.

        \ff[13]{Planar Tolerance}[Magical] You are immune to harmful effects imposed by being on other planes.

        \ff[16]{Supreme Specialization} The bonus from your \textit{specialization} ability increases to \plus9.

        \ff[19]{Find the Path}[Magical] You can use the \textit{find the path} ability as a standard action.
        \begin{attuneability}{Find the Path}[\glossterm{Attune} (self), \glossterm{Knowledge}]
            When you use this ability, you must unambiguously specify a location on the same plane as you.
            You know exactly what direction you must travel to reach your chosen destination by the most direct physical route.
            You are not always led in the exact direction of the destination -- if there is an impassable obstacle between the target and the destination, this ability will direct you around the obstacle, rather than through it.

            The guidance provided by this ability adjusts to match whatever your current physical capabilities are, including flight and other unusual movement modes.
            It does not consider teleportation spells or any other activated abilities you may have which could allow you to bypass physical obstacles.
            It does not see into the future, and changing circumstances may cause the most direct path to change over time.
            It also does not consider hostile creatures, traps, and other passable dangers which may endanger or slow progress.
        \end{attuneability}
    \end{feat}

    \begin{feat}{Swift}{General}
        \featpre Base Dexterity of 1.

        \ff[1]{Rapid Movement} You gain a \plus10 foot bonus to your \glossterm{base speed}.

        \ff[4]{Sprinter} When you use the \textit{sprint} ability, you move can move up to triple your movement speed.

        \ff[7]{Wall Runner} You gain a \plus5 bonus to checks with the \textit{wallrun} ability (see \pcref{Wallrun}).
        In addition, you can make a Dexterity check in place of a Climb check to use that ability.

        \ff[10]{Water Runner} During your movement with the \textit{sprint} ability, you can move on water and similar liquids as if they were solid ground.

        \ff[13]{Greater Rapid Movement} The speed bonus from your \textit{rapid movement} ability increases to \plus20 feet.

        \ff[16]{Greater Sprinter} When you use the \textit{sprint} ability, you can move up to five times your movement speed.

        \ff[19]{Cloud Runner} During your movement with the \textit{sprint} ability, you can move on dense fog and similar gaseous substances as if they were solid ground.
    \end{feat}

    \begin{feat}{Swim Specialization}{Skill}
        \featpre Swim as a mastered skill.

        \ff[1]{Specialization} You gain a \plus3 bonus to the Swim skill.
        % Maybe add burrow speed? The book needs a description of fighting underwater penalties.

        \ff[4]{Underwater Tolerance} You reduce your penalties for fighting underwater by 2, except for penalties to \glossterm{mundane} ranged attacks (see \pcref{Underwater Combat}).

        \ff[7]{Swim Speed} You gain a \glossterm{swim speed} equal to your \glossterm{base speed}.
        If you already have a swim speed, you gain a \plus10 foot bonus to your swim speed.
        A successful Swim check to move allows you to move a distance equal to your swim speed.

        \ff[10]{Greater Specialization} The bonus from your \textit{specialization} ability increases to \plus6.

        \ff[13]{Greater Underwater Tolerance} You do not suffer penalties for fighting underwater, except that you still suffer the normal penalties to \glossterm{mundane} ranged attacks.

        \ff[16]{Supreme Specialization} The bonus from your \textit{specialization} ability increases to \plus9.

        \ff[19]{Earth Swimmer} You can swim through loose earth and dirt as if it were water.
        Your swim speed is halved while moving in this way.
    \end{feat}

    \begin{feat}{Toughness}{General}
        \featpre Base Constitution of 1.

        \ff[1]{Fortified Body} You gain a \plus2 bonus to Fortitude defense.
        In addition, you can sleep while you have \glossterm{encumbrance} without penalty (see \pcref{Encumbrance}).

        \ff[4]{Durability} You increase your maximum \glossterm{hit points} by 2.

        \ff[7]{Ailment Tolerance} You are immune to being \glossterm{sickened} and \glossterm{nauseated}.

        \ff[10]{Greater Fortified Body} The defense bonus from your \textit{fortified body} ability increases to \plus3.
        In addition, you need half the normal amount of rest and sleep each day to function normally.
        For example, a human would only need four hours of sleep per night.
        This does not reduce the time required for you to take a \glossterm{long rest}.

        \ff[13]{Greater Durability} The number of additional \glossterm{hit points} from your \textit{durability} ability increases to 3.

        \ff[16]{Greater Ailment Tolerance} You are immune to poisons and diseases.

        \ff[19]{Greater Fortified Body} The defense bonus from your \textit{fortified body} ability increases to \plus4.
        In addition, the amount of rest and sleep you need each day is reduced to a quarter of the normal value.
        For example, a human would only need two hours of sleep per night.
    \end{feat}

    \begin{feat}{Two-Weapon Fighting}{Combat}
        \featpre Base Dexterity of 2.

        \ff[1]{Rend} You can use the \textit{rend} ability as a standard action.
        \begin{freeability}{Rend}
            Make a melee \glossterm{strike} using two weapons.
            If you hit with the \glossterm{attack roll} for both weapons, you gain a \plus2d bonus to damage with the strike.

            \rankline
            \featlevel{7} The damage bonus increases to \plus3d.
            \featlevel{13} The damage bonus increases to \plus4d.
            \featlevel{19} The damage bonus increases to \plus5d.
        \end{freeability}

        \ff[4]{Two-Weapon Defense} While wielding two melee weapons, you gain a \plus1 bonus to Armor defense.
        This bonus is considered to come from a shield, and does not stack with the benefits of using a physical shield.

        \ff[7]{Dual Precision} You gain a \plus1 bonus to \glossterm{accuracy} when making strikes with two weapons at once.

        \ff[10]{Overwhelming Flurry} You can use the \textit{overwhelming flurry} ability as a \glossterm{minor action}.
        \begin{attuneability}{Overwhelming Flurry}[\glossterm{Attune} (self)]
            You begin attacking and feinting with incredible speed, dazzling your foes.
            You are considered an additional creature for the purpose of determining \glossterm{overwhelm penalties} you inflict on other creatures.
            This can allow you to overwhelm creatures by yourself.
        \end{attuneability}

        \ff[13]{Greater Two-Weapon Defense} The bonus from your \textit{two-weapon defense} ability increases to \plus2.

        \ff[16]{Greater Dual Precision} The bonus from your \textit{dual precision} ability increases to \plus2.

        \ff[19]{Greater Overwhelming Flurry} Your \textit{overwhelming flurry} ability also causes you to be treated as one \glossterm{size category} larger for the purpose of determining \glossterm{overwhelm penalties} you inflict on other creatures.
    \end{feat}

    \begin{feat}{Whirlwind Warrior}{Combat}
        \featpres Base Dexterity of 2, base Perception of 1.

        % TODO: this would be better if it hit threatened people, but increased weapon range is too common
        \ff[1]{Cyclone} You can use the \textit{cyclone} ability as a standard action.
        \begin{freeability}{Cyclone}[\glossterm{Sustain} (standard)]
            When you use this ability, make a melee \glossterm{strike} with a slashing weapon.
            The strike targets any number of creatures adjacent to you.
            For the duration of this ability, you move at half speed.
            Whenever you sustain this ability, you can move and make a melee \glossterm{strike} with a slashing weapon.
            The strike targets any number of creatures adjacent to you at any point during your movement.

            \rankline
            \featlevel{7} You gain a \plus1d bonus to damage with the strike.
            \featlevel{13} The damage bonus increases to \plus2d.
            \featlevel{19} The damage bonus increases to \plus3d.
        \end{freeability}

        \ff[4]{Wind Dance} You gain a \plus1 bonus to Armor defense.

        \ff[7]{Unfettered Movement} During each phase, you may move through one creature's space during movement.
        You treat its space as \glossterm{difficult terrain}.
        After moving through that creature's space, other creatures block you as normal for the remainder of your movement.
        Certain unusual creatures that occupy their entire space, such as gelatinous cubes, may be immune to this ability.

        \ff[10]{Eye of the Storm} You reduce your \glossterm{overwhelm penalties} by 1.

        \ff[13]{Greater Wind Dance} The bonus from your \textit{wind dance} ability increases to \plus2.

        \ff[16]{Greater Unfettered Movement} Using your \textit{unfettered movement} ability does not cause you to move at half speed while in the creature's space.

        \ff[19]{Greater Eye of the Storm} The penalty reduction from your \textit{eye of the storm} ability increases to 2.
    \end{feat}

\section{Other Feat Rules}

    \subsection{Retraining Feats}
        At every level, you can choose to retrain an old feat in exchange for a new feat.
