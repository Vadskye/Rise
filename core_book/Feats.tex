\chapter{Feats}\label{Feats}

This chapter describes a set of optional rules that you can use in a campaign.
If you use these rules, characters gain feats which allow them to further specialize in specific areas, making characters more mechanically distinct from each other.
Feats also make the system more mechanically complex, so they are not necessarily enjoyable for all groups.
You cannot gain the same feat twice.

\section{Gaining Feats}
    There are two main ways you can use feats in your game.

    \parhead{Species Feat Only}
    One simple option is that characters gain a single feat at 1st level based on their species, and no other feats.
    This makes your choice of species more significant without dramatically increasing character complexity.

    \parhead{Feat Progression}
    If you want characters to be more complex and to have more powerful abilities, you can also use a feat progression system.
    For example, you could gain a feat from your species at 1st level, and an additional feat at 3rd, 6th, and 9th level.
    Alternately, you could gain feats based on the completion of major story events.
    In general, it is inadvisable to gain more than four feats total, or to gain feats after about 10th level.

    \subsection{Species Bonus Feats}\label{Species Bonus Feats}
        Each species grants a bonus feat at 1st level. Most species can only choose from a small group of feats, listed in the description of the species. A character must meet any prerequisites for these bonus feats, as normal.

        \parhead{Human} Any feat.

        \parhead{Dwarf} Any from the following list: \featref*{Blindfighter}, \featref*{Craft Specialization}, \featref*{Iron Will}, \featref*{Martial Training}, \featref*{Regenerator}, \featref*{Toughness}.

        \parhead{Elf} Any Casting feat (see \pcref{Casting Feats}), or any from the following list: \featref*{Awareness Specialization}, \featref*{Balance Specialization}, \featref*{Sniper}, \featref*{Rapid Reaction}.

        \parhead{Gnome} Any Casting feat (see \pcref{Casting Feats}), or any from the following list: \featref*{Blindfighter}, \featref*{Craft Specialization}, \featref*{Stealth Specialization}, \featref*{Toughness}.

        \parhead{Half-Elf} Any Skill feat (see \pcref{Skill Feats}).

        \parhead{Half-Orc} Any Combat feat (see \pcref{Combat Feats}), or any from the following list: \featref*{Intimidate Specialization}, \featref*{Toughness}.

        \parhead{Halfling} Any from the following list: \featref*{Balance Specialization}, \featref*{Climb Specialization}, \featref*{Iron Will}, \featref*{Jump Specialization}, \featref*{Rapid Reaction}, \featref*{Stealth Specialization}.

        \subsubsection{Changing Species}
            In extraordinary cases, a creature may change its species.
            For example, the \ritual{reincarnation} ritual returns a creature to life as a different species.
            Regardless of its new species, the creature keeps its original species bonus feat.

\section{Feat Mechanics}

    \subsection{Prerequisites}
        Some feats have prerequisites.
        Unless you meet all of the prerequisites, you cannot take the feat.
        Prerequisites can include a minimum base attribute score, another feat or feats, a minimum level of training in a skill, or some other property of a character.
        A character can gain a feat at the same level at which they gain the prerequisite.

        A character who no longer meets all prerequisites for a feat loses all abilities from that feat.

    \subsection{Feat Tags}
        All feats are organized into different groups by tags.

        \parhead{General} General feats can have a wide variety of effects.
        They often grant new abilities or improve your defenses.

        \parhead{Combat} Combat feats improve your combat capabilities.
        They can increase the damage you deal, grant you new combat abilities, or improve your defenses.

        \parhead{Casting} Casting feats improve your spellcasting abilities.
        Casting feats are useless to characters who cannot cast spells.

        \parhead{Skill} Skill feats improve your skills.
        They can make you more likely to succeed with skill checks and grant you new abilities based on your skills.

        \parhead{Bloodline Feats} Some characters have traces of monstrous blood running in their veins.
        Most of those will never understand the full potential of their unusual heritage.
        Bloodline feats allow characters to explore those posibilities by gaining abilites related to their ancestry.
        You can only have one Bloodline feat.

        \parhead{Magical Feats}
        All abilities granted by feats with the Magical type are \glossterm{magical} in nature.
        Many feats are not entirely magical, but have specific effects that are magical.

\section{Feat Table}

% Feat names must follow ``I have'' or ``I am (a)''.
\begin{longtablewrapper}
    \begin{longtable}{>{\lcol}p{11em} >{\lcol}p{12em} l >{\lcol}p{8em} >{\lcol}p{3em}}
        \lcaption{Feats}\\
        \tb{General Feats}\label{General Feats} & \tb{Prerequisites} & \tb{Benefits} & \tb{Feat Types} & \tb{Page} \\
        \featref{Celestial Heritage} & Non-evil                 & Gain aspects of celestial beings    & Bloodline, Magical & \featpref{Celestial Heritage} \\
        \featref{Chameleon}          & Mastered Disguise, Int 1 & Adapt your archetypes and abilities & \tdash             & \featpref{Chameleon}          \\
        \featref{Draconic Heritage}  & \tdash                   & Gain aspects of draconic power      & Bloodline          & \featpref{Draconic Heritage}  \\
        \featref{Entropist}          & Wil 1                    & Master chaos and entropy            & \tdash             & \featpref{Entropist}          \\
        \featref{Iron Will}          & Wil 1                    & Increase mental resilience          & \tdash             & \featpref{Iron Will}          \\
        \featref{Null}               & Wil 2                    & Become immune to magic              & \tdash             & \featpref{Null}               \\
        \featref{Precognition}       & Int 2                    & React to future events              & \tdash             & \featpref{Precognition}       \\
        \featref{Regenerator}        & Con 2                    & Heal wounds with inhuman speed      & \tdash             & \featpref{Regenerator}        \\
        \featref{Rapid Reaction}     & Dex 1                    & Increase reaction speed             & \tdash             & \featpref{Rapid Reaction}     \\
        \featref{Spellwarped}        & Wil 1                    & Gain limited spellcasting           & Magical            & \featpref{Spellwarped}        \\
        \featref{Swift}              & Dex 1                    & Move more quickly                   & \tdash             & \featpref{Swift}              \\
        \featref{Telepath}           & Int 1, Wil 1             & Communicate with creatures mentally & \tdash             & \featpref{Telepath}          \\
        \featref{Toughness}          & Con 1                    & Increase physical fortitude         & \tdash             & \featpref{Toughness}          \\

        \tb{Skill Feats}\label{Skill Feats}        & \tb{Prerequisites}          & \tb{Benefits}                         & \tb{Feat Types} & \tb{Page}                                   \\
        \featref{Awareness Specialization}         & Mastered Awareness          & Improve use of chosen skill           & \tdash          & \featpref{Awareness Specialization}         \\
        \featref{Balance Specialization}           & Mastered Balance            & Improve use of chosen skill           & \tdash          & \featpref{Balance Specialization}           \\
        \featref{Climb Specialization}             & Mastered Climb              & Improve use of chosen skill           & \tdash          & \featpref{Climb Specialization}             \\
        \featref{Craft Specialization}             & Mastered Craft              & Improve use of chosen skill           & \tdash          & \featpref{Craft Specialization}             \\
        \featref{Creature Handling Specialization} & Mastered Creature Handling  & Improve use of chosen skill           & \tdash          & \featpref{Creature Handling Specialization} \\
        \featref{Deception Specialization}         & Mastered Deception          & Improve use of chosen skill           & \tdash          & \featpref{Deception Specialization}         \\
        \featref{Devices Specialization}           & Mastered Devices            & Improve use of chosen skill           & \tdash          & \featpref{Devices Specialization}           \\
        \featref{Disguise Specialization}          & Mastered Disguise           & Improve use of chosen skill           & \tdash          & \featpref{Disguise Specialization}          \\
        \featref{Endurance Specialization}         & Mastered Endurance          & Improve use of chosen skill           & \tdash          & \featpref{Endurance Specialization}         \\
        \featref{Flexibility Specialization}       & Mastered Flexibility        & Improve use of chosen skill           & \tdash          & \featpref{Flexibility Specialization}       \\
        \featref{Intimidate Specialization}        & Mastered Intimidate         & Improve use of chosen skill           & \tdash          & \featpref{Intimidate Specialization}        \\
        \featref{Herbalist}                        & Mastered Knowledge (nature) & Brew potions with natural ingredients & \tdash          & \featpref{Herbalist}                        \\
        \featref{Jump Specialization}              & Mastered Jump               & Improve use of chosen skill           & \tdash          & \featpref{Jump Specialization}              \\
        \featref{Knowledge Specialization}         & Mastered Knowledge          & Improve use of chosen skill           & \tdash          & \featpref{Knowledge Specialization}         \\
        \featref{Linguistics Specialization}       & Mastered Linguistics        & Improve use of chosen skill           & \tdash          & \featpref{Linguistics Specialization}       \\
        \featref{Medicine Specialization}          & Mastered Medicine           & Improve use of chosen skill           & \tdash          & \featpref{Medicine Specialization}          \\
        \featref{Perform Specialization}           & Mastered Perform            & Improve use of chosen skill           & \tdash          & \featpref{Perform Specialization}           \\
        \featref{Persuasion Specialization}        & Mastered Persuasion         & Improve use of chosen skill           & \tdash          & \featpref{Persuasion Specialization}        \\
        \featref{Ride Specialization}              & Mastered Ride               & Improve use of chosen skill           & \tdash          & \featpref{Ride Specialization}              \\
        \featref{Sleight of Hand Specialization}   & Mastered Sleight of Hand    & Improve use of chosen skill           & \tdash          & \featpref{Sleight of Hand Specialization}   \\
        \featref{Social Insight Specialization}    & Mastered Social Insight     & Improve use of chosen skill           & \tdash          & \featpref{Social Insight Specialization}    \\
        \featref{Spellsense Specialization}        & Mastered Spellsense         & Improve use of chosen skill           & \tdash          & \featpref{Spellsense Specialization}        \\
        \featref{Stealth Specialization}           & Mastered Stealth            & Improve use of chosen skill           & \tdash          & \featpref{Stealth Specialization}           \\
        \featref{Survival Specialization}          & Mastered Survival           & Improve use of chosen skill           & \tdash          & \featpref{Survival Specialization}          \\
        \featref{Swim Specialization}              & Mastered Swim               & Improve use of chosen skill           & \tdash          & \featpref{Swim Specialization}              \\

        \tb{Casting Feats}\label{Casting Feats} & \tb{Prerequisites} & \tb{Benefits} & \tb{Feat Types} & \tb{Page} \\
        \featref{Boongiver}                      & Spellcasting ability                    & Improve ability to cast spells on allies  & Magical & \featpref{Boongiver}                      \\
        \featref{Blood Magic}                    & Spellcasting ability, Con 2                    & Spend hit points to improve magic         & Magical & \featpref{Blood Magic}                    \\
        \featref{Mental Magic}                   & Spellcasting ability, Wil 1             & Cast spells without words or gestures     & Magical & \featpref{Mental Magic}                   \\
        \featref{Metacaster}                     & Spellcasting ability                    & Manipulate spell effects in creative ways & Magical & \featpref{Metacaster}                     \\
        \featref{Mystic Archer}                  & Spellcasting ability                    & Imbue projectiles with magic              & Magical & \featpref{Mystic Archer}                  \\
        \featref{Prepared Spellcasting}          & Spellcasting ability, Int 2             & Prepare additional spells each day        & Magical & \featpref{Prepared Spellcasting}                                          \\
        \featref{Spellsword}                     & Spellcasting ability                    & Fight with sword and spell together       & \tdash  & \featpref{Spellsword}                     \\
        \featref{Sphere Focus: Aeromancy}        & \sphere{Aeromancy} sphere access        & Improve casting with chosen sphere        & Magical & \featpref{Sphere Focus: Aeromancy}        \\
        \featref{Sphere Focus: Aquamancy}        & \sphere{Aquamancy} sphere access        & Improve casting with chosen sphere        & Magical & \featpref{Sphere Focus: Aquamancy}        \\
        \featref{Sphere Focus: Astromancy}       & \sphere{Astromancy} sphere access       & Improve casting with chosen sphere        & Magical & \featpref{Sphere Focus: Astromancy}       \\
        \featref{Sphere Focus: Barrier}          & \sphere{Barrier} sphere access          & Improve casting with chosen sphere        & Magical & \featpref{Sphere Focus: Barrier}          \\
        \featref{Sphere Focus: Bless}            & \sphere{Bless} sphere access            & Improve casting with chosen sphere        & Magical & \featpref{Sphere Focus: Bless}            \\
        \featref{Sphere Focus: Channel Divinity} & \sphere{Channel Divinity} sphere access & Improve casting with chosen sphere        & Magical & \featpref{Sphere Focus: Channel Divinity} \\
        \featref{Sphere Focus: Chronomancy}      & \sphere{Chronomancy} sphere access      & Improve casting with chosen sphere        & Magical & \featpref{Sphere Focus: Chronomancy}      \\
        \featref{Sphere Focus: Cryomancy}        & \sphere{Cryomancy} sphere access        & Improve casting with chosen sphere        & Magical & \featpref{Sphere Focus: Cryomancy}        \\
        \featref{Sphere Focus: Electromancy}     & \sphere{Electromancy} sphere access     & Improve casting with chosen sphere        & Magical & \featpref{Sphere Focus: Electromancy}     \\
        \featref{Sphere Focus: Enchantment}      & \sphere{Enchantment} sphere access      & Improve casting with chosen sphere        & Magical & \featpref{Sphere Focus: Enchantment}      \\
        \featref{Sphere Focus: Fabrication}      & \sphere{Fabrication} sphere access      & Improve casting with chosen sphere        & Magical & \featpref{Sphere Focus: Fabrication}      \\
        \featref{Sphere Focus: Photomancy}       & \sphere{Photomancy} sphere access       & Improve casting with chosen sphere        & Magical & \featpref{Sphere Focus: Photomancy}       \\
        \featref{Sphere Focus: Polymorph}        & \sphere{Polymorph} sphere access        & Improve casting with chosen sphere        & Magical & \featpref{Sphere Focus: Polymorph}        \\
        \featref{Sphere Focus: Pyromancy}        & \sphere{Pyromancy} sphere access        & Improve casting with chosen sphere        & Magical & \featpref{Sphere Focus: Pyromancy}        \\
        \featref{Sphere Focus: Revelation}       & \sphere{Revelation} sphere access       & Improve casting with chosen sphere        & Magical & \featpref{Sphere Focus: Revelation}       \\
        \featref{Sphere Focus: Summoning}        & \sphere{Summoning} sphere access        & Improve casting with chosen sphere        & Magical & \featpref{Sphere Focus: Summoning}        \\
        \featref{Sphere Focus: Telekinesis}      & \sphere{Telekinesis} sphere access      & Improve casting with chosen sphere        & Magical & \featpref{Sphere Focus: Telekinesis}      \\
        \featref{Sphere Focus: Terramancy}       & \sphere{Terramancy} sphere access       & Improve casting with chosen sphere        & Magical & \featpref{Sphere Focus: Terramancy}       \\
        \featref{Sphere Focus: Thaumaturgy}      & \sphere{Thaumaturgy} sphere access      & Improve casting with chosen sphere        & Magical & \featpref{Sphere Focus: Thaumaturgy}      \\
        \featref{Sphere Focus: Toxicology}         & \sphere{Toxicology} sphere access         & Improve casting with chosen sphere        & Magical & \featpref{Sphere Focus: Toxicology}         \\
        \featref{Sphere Focus: Umbramancy}       & \sphere{Umbramancy} sphere access       & Improve casting with chosen sphere        & Magical & \featpref{Sphere Focus: Umbramancy}       \\
        \featref{Sphere Focus: Verdamancy}       & \sphere{Verdamancy} sphere access       & Improve casting with chosen sphere        & Magical & \featpref{Sphere Focus: Verdamancy}       \\
        \featref{Sphere Focus: Vivimancy}        & \sphere{Vivimancy} sphere access        & Improve casting with chosen sphere        & Magical & \featpref{Sphere Focus: Vivimancy}        \\
        \featref{Twinhand Spellcaster}           & Dex 1        & Cast spells with two hands at once        & Magical & \featpref{Twinhand Spellcaster}        \\

        \tb{Combat Feats}\label{Combat Feats} & \tb{Prerequisites} & \tb{Benefits} & \tb{Feat Types} & \tb{Page} \\
        \featref{Blindfighter}         & Per 2            & Fight unseen foes better                      & \tdash & \featpref{Blindfighter}         \\
        \featref{Brawler}              & Str 1, Dex 1     & Fight better unarmed and in close quarters    & \tdash & \featpref{Brawler}              \\
        \featref{Combat Style Versatility} & Int 1, combat style & Access combat styles from other classes & \tdash & \featpref{Combat Style Versatility} \\
        \featref{Duelist}              & Dex 1, Int 1     & Fight one-on-one better                       & \tdash & \featpref{Duelist}              \\
        \featref{Executioner}          & Per 1            & Kill weakened foes more easily                & \tdash & \featpref{Executioner}          \\
        \featref{Greatweapon Warrior}  & Str 2            & Fight better with two-handed weapons          & \tdash & \featpref{Greatweapon Warrior}  \\
        \featref{Leadership}           & Int 2 or Wil 2   & Inspire nearby allies                         & \tdash & \featpref{Leadership}           \\
        \featref{Maneuverist}        & Int 1    & Gain limited maneuver access     & \tdash             & \featpref{Maneuverist}        \\
        \featref{Martial Training}     & \tdash           & Improve combat abilities                      & \tdash & \featpref{Martial Training}     \\
        \featref{Savage}               & Str 2            & Shove and overrun foes to deal damage         & \tdash & \featpref{Savage}               \\
        \featref{Shieldbearer}         & Str 2            & Defend better with shields                    & \tdash & \featpref{Shieldbearer}         \\
        \featref{Sniper}               & Per 2            & Aim precisely at distant foes                 & \tdash & \featpref{Sniper}               \\
        \featref{Two-Weapon Fighting}  & Dex 2            & Fight better with two weapons at once         & \tdash & \featpref{Two-Weapon Fighting}  \\
        \featref{Weapon Focus}         & \tdash           & Fight better with a single type of weapon     & \tdash & \featpref{Weapon Focus}         \\
        \featref{Whirlwind Warrior}    & Dex 1            & Fight hordes with agile ease                  & \tdash & \featpref{Whirlwind Warrior}    \\
    \end{longtable}
\end{longtablewrapper}

    \section{Feat Descriptions}
        Each feat has a set of benefits it provides to a character with the feat.
        Some feats also have specific requirements that a character must meet before taking the feat.
        These are listed under a \textbf{Prerequisites} heading.
        If a character loses the prerequisites for a feat, they lose all benefits of the feat until they meet the prerequisites again.

    \begin{feat}{Awareness Specialization}{Skill}
        \featpre Awareness as a mastered skill.

        \ff[1]{Specialization} You gain a \plus3 bonus to the Awareness skill.

        % TODO: clarify stacking with existing senses
        \ff[3]{Extraordinary Senses} You gain one of the following senses: \glossterm{blindsense} (30 ft.), \glossterm{darkvision} (60 ft.), \glossterm{low-light vision}, \glossterm{scent}, or \glossterm{tremorsense} (30 ft.).

        \ff[6]{Quick Scan} When you use the \textit{search} ability, you can notice things in a \areasmall radius within \rngshort range (see \pcref{Search}).

        \ff[9]{Greater Specialization} The bonus from your \textit{specialization} ability increases to \plus5.

        \ff[12]{Greater Extraordinary Senses} You gain one of the following senses: \glossterm{blindsense} (120 ft.), \glossterm{blindsight} (30 ft.), \glossterm{darkvision} (240 ft.), \glossterm{tremorsense} (120 ft.), or \glossterm{tremorsight} (30 ft.).

        \ff[15]{Supreme Specialization} The bonus from your \textit{specialization} ability increases to \plus7.

        \ff[18]{Greater Quick Scan} When you use the \textit{search} ability, you can notice things in a \medarea radius within \medrange (see \pcref{Search}).

        \ff[21]{Supreme Extraordinary Senses} You can choose an additional sense from the list given in your \textit{greater extraordinary senses} ability.
        In addition, the range of all senses gained from this feat is doubled.
    \end{feat}

    \begin{feat}{Balance Specialization}{Skill}
        \featpre Balance as a mastered skill.

        \ff[1]{Specialization} You gain a \plus3 bonus to the Balance skill.

        \ff[3]{Combat Tumble} During each phase, you may move through one creature's space during movement.
        You treat its space as \glossterm{difficult terrain}.
        After moving through that creature's space, other creatures block you as normal for the remainder of your movement.
        Certain unusual creatures that occupy their entire space, such as gelatinous cubes, may be immune to this ability.

        \ff[6]{Agile Movement} Whenever you use an ability that causes you to move using one of your movement speeds in a straight line, you can make a single turn of up to 90 degrees during the movement. 
        This only affects voluntary movement, such as the \textit{charge} ability or the \textit{reaping charge} maneuver, and not forced movement imposed on you.
        This ability replaces the \textit{agile charge} ability, and cannot be combined with it (see \pcref{Agile Charge}).

        \ff[6]{Instant Stand} You can use the \textit{rapid stand} ability as a \glossterm{free action} instead of a \glossterm{minor action} (see \pcref{Rapid Stand}).

        \ff[9]{Greater Specialization} The bonus from your \textit{specialization} ability increases to \plus5.

         % modifier at 10th: 12dex + 2mst + 2ft = 16
        \ff[12]{Balance On Air}[Magical] You can attempt to move on surfaces that cannot support your weight.
        Surfaces that can support at least a quarter of your weight, such as thin tree branches and dense liquids, are \glossterm{difficulty rating} 20.
        Surfaces that can support at least a tenth of your weight, such as water, are \glossterm{difficulty rating} 25.
        Surfaces that can support at least a hundredth of your weight, such as tree leaves, are \glossterm{difficulty rating} 30.
        Surfaces that cannot support your weight at all, such as air, are \glossterm{difficulty rating} 40.

        Success means you move along the surface at half speed.
        Failure means you fall through the surface.
        The \glossterm{difficulty rating} increases by 5 for each consecutive round that you spend moving in this way.

        \ff[15]{Supreme Specialization} The bonus from your \textit{specialization} ability increases to \plus7.

        \ff[18]{Greater Agile Movement} Whenever you use an ability that causes you to move one of your movement speeds in a straight line, you can move in any path, not just in a straight line.

        \ff[21]{Greater Balance On Air} You can move at full speed while using your \textit{air dancer} ability.
        In addition, for each round that you spend using your \textit{air dancer} ability, the \glossterm{difficulty rating} increases by 2 instead of by 5.
    \end{feat}

    \begin{feat}{Blindfighter}{Combat}
        \featpre Base Perception 2.

        \ff[1]{Blind Awareness} When you make an attack with a miss chance caused by being unable to see your opponent, you can roll the miss chance twice and take the better result.
        In addition, you are not \partiallyunaware against foes if you know their location.

        \ff[3]{Blindsense} You gain \glossterm{blindsense} (60 ft.).

        \ff[6]{Unseeing Precision} You gain a \plus1 bonus to \glossterm{accuracy}.

        \ff[9]{Controlled Sight} You are immune to all abilities that depend on sight to affect you.

        \ff[12]{Blindsight} The range of your blindsense increases by 60 feet.
        In addition, you gain \glossterm{blindsight} (30 ft.).
        If you already have blindsight, the range of your blindsight increases by 30 feet.

        \ff[15]{Greater Unseeing Precision} The bonus from your \textit{unseeing precision} ability increases to \plus2.

        \ff[18]{Blind Reaction} You are never \unaware or \partiallyunaware.

        \ff[21]{Greater Blindsight} The range of your blindsight increases by 90 feet.
        In addition, the range of your blindsense increases by 360 feet.
    \end{feat}

    \begin{feat}{Blood Magic}{Casting, Magical}
        \featpre Ability to cast a spell, base Constitution 2.

        \ff[1]{Bloodspell} Whenever you cast a spell, you may use this ability.
        When you do, you lose \glossterm{hit points} equal to the spell's rank (minimum 1).
        In exchange, you gain a \plus2 bonus to \glossterm{power} with the spell, the spell does not require \glossterm{casting components}, and it loses the \abilitytag{Focus} tag (if it had it).

        \ff[3]{Spare Blood} You gain a \plus4 bonus to your maximum \glossterm{hit points}.

        \ff[6]{Bloodbind}[\glossterm{Magical}] Whenever you make a living creature lose \glossterm{hit points} using a spell, you can choose to bind the target's blood to yours.
        While the target is bound, you can see it through all forms of \glossterm{concealment} and even if it is \glossterm{invisible} (but not through \glossterm{cover}).
        In addition, you constantly know the exact direction and distance to the target bound by your \textit{bloodbind} ability.
        This binding lasts until you bind another creature with this ability.

        \ff[9]{Greater Bloodspell} The bonus to \glossterm{power} from your \textit{bloodspell} ability increases to \plus4.

        \ff[12]{Greater Spare Blood} The bonus from your \textit{spare blood} ability increases to \plus10.

        \ff[15]{Greater Bloodbind} You are always considered to have \glossterm{line of effect} to the target bound by your \textit{bloodbind} ability, regardless of intervening obstacles.
        The target must still be within the normal \glossterm{range} of your spells.

        \ff[18]{Supreme Bloodspell} The bonus to \glossterm{power} from your \textit{bloodspell} ability increases to \plus8.

        \ff[21]{Supreme Spare Blood} The bonus from your \textit{spare blood} ability increases to \plus25.
    \end{feat}

    \begin{feat}{Boongiver}{Casting, Magical}
        \featpre Ability to cast a spell.

        \ff[1]{Boon Lore} You learn an additional \glossterm{spell}.
        The spell must have the \abilitytag{Attune} tag.
        You can exchange this spell for other spells as you gain access to new spell ranks, but the spell must always have the \abilitytag{Attune} tag.

        \ff[3]{Share Boon} When you cast a spell with the \abilitytag{Attune} (self) tag, you can use the \textit{share boon} ability.
        \begin{freeability}{Share Boon}
            The spell's \abilitytag{Attune} tag changes to \abilitytag{Attune} (target).
            Choose one \glossterm{ally} within \rngmed range.
            That ally is the target of the spell, and the spell affects that creature as if it were you instead of affecting you.

            You can only use this ability to affect one spell at a time.
            If you use it again, the original ally's attunement to the old spell is released, as the \textit{release attunement} ability (see \pcref{Attunement}).
        \end{freeability}

        \ff[6]{Benevolent Transferance} You can use the \textit{benevolent transferance} ability as a \glossterm{standard action}.
        \begin{freeability}{Benevolent Transferance}
            Choose a creature currently \glossterm{attuned} to a spell you cast.
            In addition, choose another creature to transfer the spell to.
            Both targets must within that spell's range of you, and must be valid targets for the spell.
            You cannot target yourself with this ability.

            If both targets are willing, and the new target spends an \glossterm{attunement point} to attune to the spell, the spell's effect is transferred from the first target to the second.
            The spell's old target immediately regains the \glossterm{attunement point} it spent to attune to the spell.

            \rankline
            \featlevel{12} You can use this ability as a \glossterm{minor action}.
            \featlevel{18} The targets can transfer any number of attuned spells between each other.
        \end{freeability}

        \ff[9]{Greater Boon Lore} The number of additional \abilitytag{Attune} spells you gain from your \textit{boon lore} ability increases to two.

        \ff[12]{Greater Share Boon} You can use your \textit{share boon} ability on up to two different spells at once.
        If you use the ability while it already affects another spell, you choose which spells are affected by the ability.

        \ff[15]{Personal Boon} You gain an additional \glossterm{attunement point}.
        You can only use this attunement point to \glossterm{attune} to spells and rituals that you cast on yourself.

        \ff[18]{Supreme Boon Lore} The number of additional \abilitytag{Attune} spells you gain from your \textit{boon lore} ability increases to three.

        \ff[21]{Supreme Share Boon} You can use your \textit{share boon} ability on up to three different spells at once.
    \end{feat}

    \begin{feat}{Brawler}{Combat}
        \featpre Base Strength of 1, base Dexterity of 1.

        \ff[1]{Unarmed Warrior} You become \glossterm{proficient} with the unarmed weapons \glossterm{weapon group} (see \pcref{Weapon Groups}).
        In addition, you gain a \plus2d damage bonus with unarmed weapons.
        For details about how to fight while unarmed, see \pcref{Unarmed Combat}.
        This ability does not stack with the ability of the same name from the Perfected Form monk archetype (see \pcref{Perfected Form}).

        \ff[1]{Grapple Expertise} You gain a \plus1 bonus to \glossterm{accuracy} with the \textit{grapple} ability (see \pcref{Grapple}), as well as with all grapple actions (see \pcref{Grapple Actions}).

        \ff[3]{Takedown} Whenever you hit a target with the \textit{grapple} ability, the target also takes damage as if you had hit with your unarmed attack or a light \glossterm{natural weapon} you wield.

        \ff[6]{Greater Grapple Expertise} The bonus from your \textit{grapple expertise} ability increases to \plus2.

        \ff[9]{Large Grappler} You reduce your size-based penalties for being smaller than your target with the \textit{grapple} ability by 2.

        \ff[12]{Greater Grapple Expertise} The bonus from your \textit{grapple expertise} ability increases to \plus3.

        \ff[15]{Pin Mastery} You can use the \textit{pin} ability with only one free hand (see \pcref{Pin}).
        In addition, you can keep the target immobilized with only one free hand if you hit.

        \ff[18]{Greater Large Grappler} The penalty reduction from your \textit{large grappler} ability increases to 4.

        \ff[21]{Grapple Supremacy} When you grapple a target with the \textit{grapple} ability, you do not become \grappled by that target.
    \end{feat}

    \begin{feat}{Celestial Heritage}{Bloodline, Magical}
        \featpre Non-evil alignment.
        \parhead{Special} You can only have one Bloodline feat.

        \ff[1]{Holy Smite} You can use the \textit{holy smite} ability as a standard action.
        In addition, if you have the \textit{smite} paladin ability and your \textit{devoted alignment} is good, you gain a \plus2 bonus to \glossterm{power} with the \textit{holy smite} ability.
        \begin{freeability}{Holy Smite}[Magical]
            Make a \glossterm{strike} with a \plus1d damage bonus.
            Because this is a \glossterm{magical} ability, you use your \glossterm{magical} \glossterm{power} to determine your damage instead of your \glossterm{mundane} power.
            If your target is good, you take a single point of \glossterm{energy damage} as feedback from the attack warning you that you are persecuting a good creature.

            \rankline
            \featlevel{6} The damage bonus increases to \plus2d.
            \featlevel{12} The damage bonus increases to \plus3d.
            \featlevel{18} The damage bonus increases to \plus4d.
        \end{freeability}

        \ff[3]{Healing Light}[Magical] When you use the \textit{recover} ability, you \glossterm{briefly} emit \glossterm{brilliant illumination} in a \medarea radius from you, and \glossterm{shadowy illumination} in twice that radius.
        Each \glossterm{ally} in the radius of brilliant illumination regains 1d6 \glossterm{hit points}.
        This healing increases by \plus1d at 6th level and every 3 levels thereafter.

        \ff[6]{Angel Wings} You gain feathery wings that sprout from your back.
        These wings grant you a glide speed equal to the \glossterm{base speed} for your size (see \pcref{Gliding}).
        The wings themselves are \glossterm{mundane}, but the ability to fly and glide with them is \glossterm{magical}.

        % TODO: make this more interesting
        \ff[9]{Celestial Soul} You gain a \plus4 bonus to your \glossterm{damage resistance}.

        \ff[12]{Angelic Flight}[Magical] You gain a \glossterm{fly speed} equal to the \glossterm{base speed} for your size with a maximum height of 30 feet (see \pcref{Flying}).
        At the start of each phase, you can increase your \glossterm{fatigue level} by one to ignore this height limit until the end of the round.

        \ff[15]{Greater Healing Light} You add half your magical \glossterm{power} to the healing from your \textit{healing light} ability.

        \ff[18]{Greater Celestial Soul} The bonus from your \textit{celestial soul} ability increases to \plus10.

        \ff[21]{Greater Angelic Flight} Your \glossterm{maneuverability} with the fly speed from your \textit{angelic flight} ability increases to perfect (see \pcref{Flying Maneuverability}).
    \end{feat}

    \begin{feat}{Chameleon}{General}
        \featpre Disguise as a mastered skill, base Intelligence of 1.

        \ff[1]{Adaptive Archetype} Choose one archetype that you currently have, and two archetypes you do not have from among any of your classes.
        You cannot choose an archetype that you have which is a prerequisite for another archetype that you have.
        Whenever you finish a \glossterm{long rest}, you can choose which one of those three archetypes you actually have access to.
        You gain all benefits of your chosen archetype, and temporarily lose all benefits from the archetypes you did not choose in this way.

        You must track which choices you made for archetypes that you lose access to in this way, such as which spells and maneuvers you learned.
        When you regain access to that archetype, you must make the same choices.

        \ff[3]{Versatile Disguise} Whenever you use the \textit{disguise creature} and \textit{emulate creature} abilities on yourself, you may simultaneously create two different disguises.
        This takes twice as long as creating a single disguise, and you take a \minus2 penalty to the Disguise check.
        You can change your appearance between the two chosen disguises as a \glossterm{minor action}.

        \ff[6]{Adaptive Specialty} Whenever you finish a \glossterm{short rest}, you may choose an effect from the list below.
        Each effect lasts until you take a short rest.
        \begin{itemize}
            \item Martial: You gain a \plus1 bonus to Armor defense and a \plus2 bonus to your \glossterm{mundane} \glossterm{power}.
            \item Mystic: You gain a \plus2 bonus to Mental defense and a \plus2 bonus to your \glossterm{magical} \glossterm{power}.
            \item Primal: You gain a \plus2 bonus to Fortitude defense and \plus1 bonus to your \glossterm{fatigue tolerance}.
            \item Skilled: You gain a \plus1 bonus to all skills.
        \end{itemize}

        \ff[9]{Instant Adaptation} As a standard action, you can change your choice of \textit{adaptive archetype} and \textit{adaptive specialty}.
        When you do, you increase your \glossterm{fatigue level} by two.

        \ff[12]{Greater Adaptive Specialty} The effects of your \textit{adaptive specialty} ability improve, as described below.
        \begin{itemize}
            \item Martial: The defense bonus increases to \plus2, and the power bonus increases to \plus4.
            \item Mystic: The defense bonus increases to \plus3, and the power bonus increases to \plus4.
            \item Primal: The defense bonus increases to \plus3, and the fatigue tolerance bonus increases to \plus2.
            \item Skilled: The bonus increases to \plus2.
        \end{itemize}

        \ff[15]{Greater Instant Adaptation} When you use your \textit{instant adaptation} ability,
        you only increase your \glossterm{fatigue level} by one.

        \ff[15]{Greater Versatile Disguise} When you use your \textit{versatile disguise} ability, the time required to create the disguise is not increased, and the penalty to the Disguise check is removed.

        \ff[18]{Supreme Adaptive Specialty} The effects of your \textit{adaptive specialty} ability improve, as described below.
        \begin{itemize}
            \item Martial: The defense bonus increases to \plus3, and the power bonus increases to \plus8.
            \item Mystic: The defense bonus increases to \plus4, and the power bonus increases to \plus8.
            \item Primal: The defense bonus increases to \plus4, and the fatigue tolerance bonus increases to \plus3.
            \item Skilled: The bonus increases to \plus3.
        \end{itemize}

        \ff[21]{Supreme Adaptive Archetype} Instead of choosing a single archetype to activate with your \textit{adaptive archetype}, you may choose a blend of two archetypes simultaneously.
        For each rank you have access to, you choose one archetype, and gain all abilities of that rank from that archetype and no abilities of that rank from the other archetype.
        You cannot choose abilities from an archetype that reference abilities from that same archetype which you do not have.
        For example, you cannot choose the \textit{supreme wildspell} ability unless you also have the \textit{wildspell} ability.
    \end{feat}

    \begin{feat}{Climb Specialization}{Skill}
        \featpre Climb as a mastered skill.

        \ff[1]{Specialization} You gain a \plus3 bonus to the Climb skill.

        \ff[3]{Climb Speed} You gain a \glossterm{climb speed} equal to the \glossterm{base speed} for your size.
        If you already have a climb speed, you gain a \plus5 foot bonus to your climb speed.
        A successful Climb check to move allows you to travel a distance equal to your climb speed.

        \ff[6]{Creature Climber} You can use the \textit{creature climb} ability against creatures one or more size categories larger than you instead of two (see \pcref{Creature Climb}).
        This does not cause you or the creature to suffer penalties for \squeezing.

        \ff[9]{Greater Specialization} The bonus from your \textit{specialization} ability increases to \plus5.

        \ff[12]{Rapid Climber} You gain a \plus5 foot bonus to your climb speed.

        \ff[15]{Supreme Specialization} The bonus from your \textit{specialization} ability increases to \plus7.

        \ff[18]{Greater Creature Climber} You can use the \textit{creature climb} ability against creatures of the same size category as you or larger.

        \ff[21]{Greater Rapid Climber} The speed bonus from your \textit{rapid climber} ability increases to \plus15 feet.
    \end{feat}

    \begin{feat}{Combat Style Versatility}{Combat}
        \featpre Base Intelligence of 1, access to at least one \glossterm{combat style}.

        \ff[1]{Maneuver} You learn an additional \glossterm{maneuver}.
        When you gain access to new maneuver ranks, you can change which maneuver you know.

        \ff[3]{Combat Style Flexibility} You gain access to three additional \glossterm{combat styles}.

        \ff[6]{Maneuver} You learn an additional \glossterm{maneuver}.
        When you gain access to new maneuver ranks, you can change which maneuver you know.

        \ff[9]{Precise Maneuvers} You gain a \plus1 bonus to \glossterm{accuracy}.

        \ff[12]{Maneuver} You learn an additional \glossterm{maneuver}.
        When you gain access to new maneuver ranks, you can change which maneuver you know.

        \ff[15]{Greater Combat Style Flexibility} You gain access to all combat styles.

        \ff[18]{Maneuver} You learn an additional \glossterm{maneuver}.
        When you gain access to new maneuver ranks, you can change which maneuver you know.

        \ff[21]{Greater Precise Maneuvers} The bonus from your \textit{precise maneuvers} ability increases to \plus2.
    \end{feat}

    \begin{feat}{Craft Specialization}{Skill}
        \featpre Any Craft skill as a mastered skill.

        \ff[1]{Specialization} You gain a \plus3 bonus to all Craft skills.

        \ff[3]{Craft Magic Item}[Magical] You can imbue items with magic using your crafting skill.
        Imbuing an item with magic takes material components, as described in \pcref{Magic Item Creation}.
        You can craft an item with an item level equal to your level or one item level lower than your level with 24 hours of continuous work.
        You can make weaker items more quickly.
        For every two item levels lower than your level, the time required to craft an item is halved, to a minimum of 15 minutes.

        You can also mend a broken magic item if it is one that you could make.
        Doing so costs a tenth of the raw materials and a quarter of the time it would take to craft that item in the first place.
        You cannot mend a \glossterm{destroyed} magic item.

        \ff[6]{Crafting Savant} You gain two additional \glossterm{skill points} which can only be spent on Craft skills.

        \ff[9]{Greater Specialization} The bonus from your \textit{specialization} ability increases to \plus5.

        \ff[12]{Item Attunement} You gain an additional \glossterm{attunement point}.
        You can only use this \glossterm{attunement point} to \glossterm{attune} to the effects of items (see \pcref{Item Attunement}).

        \ff[15]{Supreme Specialization} The bonus from your \textit{specialization} ability increases to \plus7.

        \ff[18]{Greater Crafting Savant} The number of additional skill points from your \textit{crafting savant} ability increases to five.

        \ff[21]{Greater Item Attunement} You gain an additional \glossterm{attunement point}.
        You can only use this \glossterm{attunement point} to \glossterm{attune} to the effects of items (see \pcref{Item Attunement}).
    \end{feat}

    \begin{feat}{Creature Handling Specialization}{Skill}
        \featpre Creature Handling as a mastered skill.

        \ff[1]{Specialization} You gain a \plus3 bonus to the Creature Handling skill.

        \ff[3]{Battleforged Training} You can teach a creature the Battleforged trick.
        The \glossterm{difficulty rating} to train the trick is 15.
        A creature with the trick gains the following benefits:
        \begin{itemize}
            \item It gains a bonus equal to your level to its \glossterm{damage resistance}.
            \item It gains a \plus1 bonus to all defenses.
            \item It gains a \plus1 accuracy bonus with all attacks.
            \item It gains a \plus1d damage bonus with \glossterm{strikes}.
        \end{itemize}

        \ff[6]{Greater Command} You can \glossterm{sustain} the effect of the \textit{command} ability from the Creature Handling skill with a \glossterm{minor action} instead of with a standard action.
        For details, see \pcref{Command}.

        \ff[6]{Efficient Training} You can teach a creature a trick with 12 hours of work, split as you choose, rather than in a week of 4 hour sessions (see \pcref{Training Creatures}).
        In addition, you can train creatures to learn two bonus tricks beyond their normal maximum (see \pcref{Bonus Tricks}).

        \ff[9]{Greater Specialization} The bonus from your \textit{specialization} ability increases to \plus5.

        \ff[12]{Greater Battleforged Training} You can teach a creature that has learned the Battleforged trick the Greater Battleforged trick.
        The \glossterm{difficulty rating} to train the trick is 25.
        A creature with the trick gains the following benefits, which replace the benefits of the Battleforged trick:
        \begin{itemize}
            \item It gains a bonus equal to twice your level to its \glossterm{damage resistance}.
            \item It gains a \plus2 bonus to all defenses.
            \item It gains a \plus2 accuracy bonus with all attacks.
            \item It gains a \plus2d damage bonus with \glossterm{strikes}.
        \end{itemize}

        \ff[15]{Supreme Specialization} The bonus from your \textit{specialization} ability increases to \plus7.

        \ff[18]{Greater Efficient Training} You can teach a creature with 4 hours of work, split as you choose (see \pcref{Training Creatures}).
        In addition, the number of bonus tricks you can teach from your \textit{efficient training} ability increases to four.

        \ff[18]{Supreme Command} You can \glossterm{sustain} the effect of the \textit{command} ability from the Creature Handling skill with a \glossterm{free action} instead of with a standard action.

        \ff[21]{Supreme Battleforged Training}  You can teach a creature that has learned the Greater Battleforged trick the Supreme Battleforged trick.
        The \glossterm{difficulty rating} to train the trick is 35.
        A creature with the trick gains the following benefits, which replace the benefits of the Greater Battleforged trick:
        \begin{itemize}
            \item It gains a bonus equal to four times your level to its \glossterm{damage resistance}.
            \item It gains a \plus3 bonus to all defenses.
            \item It gains a \plus3 accuracy bonus with all attacks.
            \item It gains a \plus3d damage bonus with \glossterm{strikes}.
        \end{itemize}
    \end{feat}

    \begin{feat}{Deception Specialization}{Skill}
        \featpre Deception as a mastered skill.

        \ff[1]{Specialization} You gain a \plus3 bonus to the Deception skill.

        \ff[3]{Dual Speech}[Magical] When you speak, you can use the \textit{dual speech} ability.
        \begin{freeability}{Dual Speech}[\abilitytag{Sustain} (minor)]
            You speak the same words with two different vocal patterns, such as tone or accent, though you cannot significantly change the volume.
            You can freely choose which creatures hear which pattern, treating all creatures you are not aware of as a single group. 

            You can freely choose different vocal patterns each round that you sustain this ability.
            \rankline
            \featlevel{9} You can speak entirely different words with your two voices.
            \featlevel{15} You can also speak with a third voice, using separate words and vocal patterns.
            \featlevel{21} You can also speak with a fourth voice, using separate words and vocal patterns.
        \end{freeability}

        \ff[6]{Undetectable Lies} As a \glossterm{minor action}, you can use the \textit{undetectable lies} ability.
        \begin{attuneability}{Undetectable Lies}[\abilitytag{Attune} (self)]
            Any \glossterm{magical} abilities which detect lies are unable to detect lies you speak.
        \end{attuneability}

        \ff[9]{Greater Specialization} The bonus from your \textit{specialization} ability increases to \plus5.

        \ff[12]{Deceive Magic}[\glossterm{Magical}] When you would be hit by a \glossterm{magical} attack, you can use this ability in response.
        \begin{freeability}{Deceive Magic}[\abilitytag{Swift}]
            When you use this ability, you increase your \glossterm{fatigue level} by two.
            After you use this ability, you \glossterm{briefly} cannot use it again.

            The attack must be rerolled with a \minus2 accuracy penalty.
            This can cause the attack to hit you instead of getting a \glossterm{critical hit}, or it can cause the attack to miss entirely.
        \end{freeability}

        \ff[15]{Supreme Specialization} The bonus from your \textit{specialization} ability increases to \plus7.

        \ff[18]{Greater Undetectable Lies} Your \textit{undetectable lies} ability loses the \abilitytag{Attune} (self) tag.
        Instead, it lasts indefinitely, and can toggle its effect on or off as a \glossterm{free action}.

        \ff[21]{Greater Deceive Magic} Using your \textit{deceive magic} ability increases your \glossterm{fatigue level} by one instead of two.
    \end{feat}

    \begin{feat}{Devices Specialization}{Skill}
        \featpre Devices as a mastered skill.

        \ff[1]{Specialization} You gain a \plus3 bonus to the Devices skill.

        % TODO: wording
        \ff[3]{Disable Arcana}[Magical] You can affect spell effects on objects or areas with the Devices skill as if they were merely complex devices.
        You must be aware of an effect to use the Devices skill to affect it, either through the Spellsense skill or because the effect is noticeable.
        You cannot affect effects on creatures.
        The \glossterm{difficulty rating} to affect an arcane spell effect is equal to 15 \add the effect's \glossterm{power}.

        \ff[6]{Rapid Improvisation} It takes you only a standard action to make a device of up to Diminuitive size with the \textit{improvise} ability (see \pcref{Improvise}).

        \ff[9]{Greater Specialization} The bonus from your \textit{specialization} ability increases to \plus5.
        
        \ff[12]{Greater Disable Arcana}[Magical] You can affect all \glossterm{magical} effects on objects or areas with the Devices skill, not just spell effects.

        \ff[15]{Supreme Specialization} The bonus from your \textit{specialization} ability increases to \plus7.

        \ff[18]{Greater Rapid Improvization} It takes you only a standard action to make a device of up to Small size with the \textit{improvise} ability (see \pcref{Improvise}).

        \ff[21]{Supreme Disable Arcana} You can affect \glossterm{magical} effects on creatures with the Devices skill.
    \end{feat}

    \begin{feat}{Disguise Specialization}{Skill}
        \featpre Disguise as a mastered skill.

        \ff[1]{Specialization} You gain a \plus3 bonus to the Disguise skill.

        \ff[3]{Quick Change} You reduce the penalties for reducing the creation time of disguises with the \textit{disguise creature} and \textit{emulate creature} abilities by 5.

        \ff[6]{Disguise Aura}[Magical] When you use the \textit{disguise creature} or \textit{emulate creature} abilities, you can decide how the target and any items on the target appear when examined by Divination spells.
        For example, you could cause all of their equipment to appear nonmagical, or you could cause them to have a strong aura of good alignment.
        The maximum \glossterm{power} you can emulate is equal to your Disguise check result \minus10.

        Anyone using divination magic on the creature must make a check with a bonus equal to the creature's \glossterm{power} with the ability to perceive the truth.
        The \glossterm{difficulty rating} is equal to your Disguise check result.
        Regardless of the result of the check, the caster is not aware that the check was made.

        \ff[9]{Greater Specialization} The bonus from your \textit{specialization} ability increases to \plus5.

        \ff[12]{Disguise Size}[Magical] You can use the \textit{disguise size} ability as a \glossterm{standard action}.
        \begin{attuneability}{Disguise Size}
            \abilitytag{Attune} (self)
            \rankline
            You increase or decrease your size by one \glossterm{size category}.
            Your physical form is not altered fully to match your new size, and your Strength and Dexterity are unchanged.
            This effect lasts as long as you \glossterm{attune} to it.
        \end{attuneability}

        \ff[15]{Supreme Specialization} The bonus from your \textit{specialization} ability increases to \plus7.

        \ff[18]{Greater Quick Change} You do not suffer penalties for reducing the creation time of disguises with the \textit{disguise creature} and \textit{emulate creature} abilities.

        \ff[21]{Greater Disguise Size} You can use your \textit{disguise size} ability with the \abilitytag{Sustain} (free) tag instead of the \abilitytag{Attune} (self) tag.
    \end{feat}

    \begin{feat}{Draconic Heritage}{Bloodline}
        \parhead{Special} You can only have one Bloodline feat.

        \ff[1]{Draconic Ancestry} Choose a type of dragon from among the dragons on \trefnp{Dragon Types}.
        You have the blood of that type of dragon in your veins.
        You gain a \plus4 bonus to \glossterm{defenses} against attacks that deal damage of the type dealt by that dragon's breath weapon.

        \ff[1]{Draconic Weapons} You gain a bite natural weapon and two claw natural weapons, one on each arm.
        For details, see \pcref{Natural Weapons}.

        % Uses rank 1 area, not rank 2. Also uses weird line vs cone scaling.
        \ff[3]{Breath Weapon} You can use the \textit{breath weapon} ability as a \glossterm{standard action}.
        \begin{freeability}{Breath Weapon}
            Make an attack vs. Reflex against everything in the area defined by the type of dragon from your \textit{draconic ancestry} ability (see \trefnp{Dragon Types}).
            You may use your Constitution in place of your Strength to determine your \glossterm{power} with this ability.
            After you use this ability, you \glossterm{briefly} cannot use it again.
            \hit Each target takes damage equal to 1d10 plus half your \glossterm{power}.
            The damage type is defined by your \textit{draconic ancestry} ability.

            \rankline
            \featlevel{6} The damage increases to 2d6.
                In addition, the area affected by your breath weapon increases.
                A line breath weapon becomes a \arealarge, 5 ft.\ wide line.
                A cone breath weapon becomes a \areamed cone.
            \featlevel{9} The damage increases to 2d8.
                In addition, if you miss by 2 or less, the target takes half damage.
                This is called a \glossterm{glancing blow}.
            \featlevel{12} The damage increases to 2d10.
                In addition, the area affected by your breath weapon increases.
                A line breath weapon becomes a \areahuge, 10 ft.\ wide line.
                A cone breath weapon becomes a \arealarge cone.
            \featlevel{15} The damage increases to 4d6.
            \featlevel{18} The damage increases to 4d8.
                In addition, the area affected by your breath weapon increases.
                A line breath weapon becomes a \areagarg, 15 ft.\ wide line.
                A cone breath weapon becomes a \areahuge cone.
        \end{freeability}

        \ff[6]{Draconic Wings} You gain leathery wings that sprout from your back.
        These wings grant you a glide speed equal to the \glossterm{base speed} for your size (see \pcref{Gliding}).
        The wings themselves are \glossterm{mundane}, but the ability to fly and glide with them is \glossterm{magical}.

        \ff[9]{Draconic Scales} You gain a \plus1 bonus to Armor defense.

        \ff[12]{Draconic Flight}[Magical] You gain a \glossterm{fly speed} equal to 10 feet faster than the \glossterm{base speed} for your size with a maximum height of 60 feet (see \pcref{Flying}).
        Your \glossterm{maneuverability} with this fly speed is poor (see \pcref{Flying Maneuverability}).
        At the start of each phase, you can increase your \glossterm{fatigue level} by one to ignore this height limit until the end of the round.

        \ff[15]{Greater Draconic Ancestry} You become immune to damage of the type dealt by your dragon's breath weapon.

        \ff[15]{Greater Draconic Scales} The bonus from your \textit{draconic scales} ability increases to \plus2.

        \ff[21]{Greater Draconic Flight} The height limit from your \textit{draconic flight} ability increases to 120 feet.
        In addition, you gain a \plus10 foot bonus to the fly speed.
    \end{feat}

    \begin{dtable}
        \lcaption{Dragon Types}
        \begin{dtabularx}{\columnwidth}{l >{\lcol}X >{\lcol}X}
            \tb{Dragon} & \tb{Damage Type} & \tb{Breath Weapon} \tableheaderrule
            Black       & Acid             & \areamed, 5 ft. wide line \\
            Blue        & Electricity      & \areamed, 5 ft. wide line \\
            Brass       & Fire             & \areamed, 5 ft. wide line \\
            Bronze      & Electricity      & \areamed, 5 ft. wide line \\
            Copper      & Acid             & \areamed, 5 ft. wide line \\
            Gold        & Fire             & \areasmall cone           \\
            Green       & Acid             & \areasmall cone           \\
            Red         & Fire             & \areasmall cone           \\
            Silver      & Cold             & \areasmall cone           \\
            White       & Cold             & \areasmall cone           \\
        \end{dtabularx}
    \end{dtable}

    \begin{feat}{Duelist}{Combat}
        \featpre Base Dexterity of 1, base Intelligence of 1.

        \ff[1]{Duelist Strike} You can use the \textit{duelist strike} ability as a standard action.
        \begin{freeability}{Duelist Strike}
            Make a melee \glossterm{strike}.
            This strikes only targets a single creature, even if your weapon would normally have the Sweeping tag.
            If you are not \surrounded, you gain a \plus1 bonus to \glossterm{accuracy} with the strike.
            If the target is not \surrounded, you gain a \plus1d damage bonus with the strike.

            \rankline
            \featlevel{6} The accuracy bonus increases to \plus2.
            \featlevel{12} The accuracy bonus increases to \plus3.
            \featlevel{18} The accuracy bonus increases to \plus4.
        \end{freeability}

        \ff[3]{Defensive Stance} You gain a \plus1 bonus to Armor defense as long as you wield a melee weapon and are not \surrounded.

        \ff[6]{Duel Focus} At the start of each round, you may choose a creature you can see.
        During that round, you gain a \plus1 bonus to Armor and Reflex defenses against that creature.

        \ff[9]{Riposte} Whenever a creature misses you with an attack, you \glossterm{briefly} gain a \plus1 \glossterm{accuracy} bonus against that creature.

        \ff[12]{Greater Defensive Stance} The bonus from your \textit{defensive stance} ability increases to \plus2.

        \ff[15]{Greater Duel Focus} The bonuses from your \textit{duel focus} ability increase to \plus2.

        \ff[18]{Greater Riposte} The bonus from your \textit{riposte} ability increases to \plus2.

        \ff[21]{Duel Serenity} Your \textit{duelist strike} ability always has its full effects, regardless of whether you or the target are surrounded.
        In addition, the bonus from your \textit{defensive stance} ability also applies to your Mental defense.
    \end{feat}

    \begin{feat}{Endurance Specialization}{Skill}
        \featpre Endurance as a mastered skill.

        \ff[1]{Specialization} You gain a \plus3 bonus to the Endurance skill.

        \ff[3]{Delay Condition} Whenever you gain a \glossterm{condition}, you can make an Endurance check.
        The \glossterm{difficulty rating} starts at 10 and increases by 5 in each subsequent round.
        Success means that you do not suffer the effects of the condition.
        You must repeat this check at the end of each subsequent round to continue to delay the effects of the condition.
        Failure means that the condition has its normal effect on you.

        You can only delay one of your conditions in this way.
        If you gain a new condition, you can choose to either delay the new condition or continue delaying the old condition.

        \ff[6]{Endurance Sprinter} You can use the \textit{sprint} ability without increasing your \glossterm{fatigue level}.
        After you use this ability, you \glossterm{briefly} cannot use it again.

        \ff[9]{Greater Specialization} The bonus from your \textit{specialization} ability increases to \plus5.

        \ff[12]{Greater Delay Vital Wound}
        When you use the \textit{delay vital wound} ability, the \glossterm{difficulty rating} does not increase for each subsequent round (see \pcref{Delay Vital Wound}).

        \ff[15]{Supreme Specialization} The bonus from your \textit{specialization} ability increases to \plus7.

        \ff[18]{Multiple Delay} You can delay up to two \glossterm{vital wounds} and \glossterm{conditions} with your \textit{delay vital wound} and \textit{delay condition} abilities.

        \ff[21]{Greater Endurance Sprinter} After you use your \textit{endurance sprinter} ability, you can use the \textit{sprint} ability again after the end of the current round.
    \end{feat}

    \begin{feat}{Entropist}{General, Magical}
        \featpre Base Willpower of 1.

        \ff[1]{Entropic Defense} Whenever you are hit by a \glossterm{critical hit} from a \glossterm{strike}, you may use this ability.
        When you do, you increase your \glossterm{fatigue level} by two, and the attacker rerolls the attack against you, which may prevent the attack from getting a critical hit against you.
        This does not protect any other targets of the attack.
        You can choose to use this item after you learn the effects that the critical hit would have, but you must do so during the phase that the attack was made.

        \ff[3]{Sudden Entropy} You can use the \textit{sudden entropy} ability as a standard action.
        \begin{instantability}{Sudden Entropy}[Instant]
            \rankline
            Make an attack vs. Mental against one creature or object within \medrange.
            \hit The target takes 1d10 \add \glossterm{power} damage.
            The damage is of of a random damage type from among the following options: physical damage, energy damage, or all damage types simultaneously.

            \rankline
            \featlevel{6} The damage increases to 2d6.
            \featlevel{9} The damage increases to 2d10.
                In addition, if you miss by 2 or less, the target takes half damage.
                This is called a \glossterm{glancing blow}.
            \featlevel{12} The damage increases to 4d6.
            \featlevel{15} The damage increases to 4d10.
            \featlevel{18} The damage increases to 5d10.
            \featlevel{21} The damage increases to 7d10.
        \end{instantability}

        \ff[6]{Improbable Vulnerability} Whenever you make an attack against an \glossterm{enemy} that is \glossterm{immune} or \glossterm{impervious} to some aspect of the attack, you have a 10\% chance to affect them as if they were not immune or impervious to the attack.
        This has no effect on objects.

        \ff[6]{Things Fall Apart} All of your attacks deal double damage to objects.

        \ff[9]{Greater Entropic Defense} You can use your \textit{entropic defense} ability whenever you suffer a critical hit from any attack, not just a strike.

        \ff[12]{Friend of Chaos} Whenever you roll for a random effect, such as a miss chance or a sorcerer's \textit{wild magic} ability, you may roll twice and keep whichever result you prefer.

        \ff[15]{Greater Improbable Vulnerability} The chance from your \textit{improbable vulnerability} ability increases to 20\%.

        \ff[18]{Supreme Entropic Defense} When you use your \textit{entropic defense} ability, you only increase your \glossterm{fatigue level} by one.

        \ff[21]{Master of Chaos} Whenever you roll for a random effect, such as a miss chance or a sorcerer's \textit{wild magic} ability, you may use this ability.
        When you do, you increase your \glossterm{fatigue level} by two, and you may freely choose the random result.
    \end{feat}

    \begin{feat}{Executioner}{Combat}
        \featpres Base Perception of 1.

        \ff[1]{Marked for Execution} You consider living creatures that either have a \glossterm{vital wound}, have less than their maximum \glossterm{hit points}, or have no remaining \glossterm{damage resistance} to be \textit{marked for execution}.
        Several abilities from this feat affect creatures \textit{marked for execution}.

        % This ability is melee-only because it makes some of the questions around ``is this creature marked''
        % less narratively awkward since you can automatically detect them with blood sense.
        \ff[1]{Execution} You can use the \textit{execution} ability as a standard action.
        \begin{freeability}{Execution}
            Make a melee \glossterm{strike}.
            If the target is \textit{marked for execution}, you gain a \plus2d damage bonus.

            \rankline
            \featlevel{6} The damage bonus increases to \plus3d.
            \featlevel{12} The damage bonus increases to \plus4d.
            \featlevel{18} The damage bonus increases to \plus5d.
        \end{freeability}

        \ff[3]{Blood Sense}[Magical] You automatically know the location of all creatures that are \textit{marked for execution} within 120 feet of you, regardless of concealment or invisibility.
        You must have \glossterm{line of effect} to a creature to sense it in this way, but you do not need \glossterm{line of sight}.
        You can automatically identify which creatures within this range are \textit{marked for execution}, even if you can already see them normally.

        \ff[6]{Purge the Weak} You gain a \plus1 bonus to \glossterm{accuracy} against creatures that are \textit{marked for execution}.
        In addition, the first die you roll for each attack roll against a creature that is \textit{marked for execution} \glossterm{explodes} on a 9 in addition to the normal explosion on a 10.

        \ff[9]{Bloody Resilience} You gain a \plus1 bonus to Fortitude defense.
        % TODO: is this the right value?
        In addition, you gain a \plus6 bonus to your maximum \glossterm{hit points}.

        \ff[12]{Greater Blood Sense}[Magical] You gain \glossterm{lifesense} with a 60 foot range.
        In addition, you can see creatures that are \textit{marked for execution} perfectly instead of only knowing their location.

        \ff[15]{Greater Bloody Resilience} The Fortitude defense bonus from your \textit{bloody resilience} ability increases to \plus2.
        In addition, the hit point bonus increases to \plus12.

        \ff[18]{Greater Purge the Weak} The bonus from your \textit{purge the weak} ability increases to \plus2.
        In addition, the first die you roll for each attack roll against a creature that is \textit{marked for execution} \glossterm{explodes} on an 8 or 9 in addition to the normal explosion on a 10.

        \ff[21]{Supreme Blood Sense}[Magical] The range of your \glossterm{lifesense} and \textit{blood sense} abilities increases to 240 feet.
        In addition, you gain \glossterm{lifesight} with a 60 foot range.
    \end{feat}

    \begin{feat}{Flexibility Specialization}{Skill}
        \featpre Flexibility as a mastered skill.

        \ff[1]{Specialization} You gain a \plus3 bonus to the Flexibility skill.

        \ff[3]{Rapid Escape} You can squeeze and escape bindings and grapples as a \glossterm{move action}, rather than as a standard action.

        \ff[6]{Constraint Tolerance} You reduce your penalties for \squeezing by 2 (see \pcref{Squeezing}).

        \ff[9]{Greater Specialization} The bonus from your \textit{specialization} ability increases to \plus5.

        \ff[12]{Escape Magic}[Magical] You can use the \textit{escape magic} ability as a standard action.
        \begin{freeability}{Escape Magic}
            You make an Flexibility attack against all \glossterm{magical} effects on you.
            You may exclude any number of effects you are aware of from this attack, allowing you to maintain beneficial magical effects.
            The \glossterm{difficulty rating} for each effect is equal to 10 \add the effect's \glossterm{power}.
            \hit Each effect is \glossterm{dismissed}, if it is an effect that can be dismissed.
        \end{freeability}

        You can only dismiss effects with this ability which target you directly, not area effects which include you as a target.
        If an ability targets multiple creatures, you can only remove its effects on you.

        \ff[15]{Supreme Specialization} The bonus from your \textit{specialization} ability increases to \plus7.

        \ff[18]{Greater Constraint Tolerance} The penalty reduction from your \textit{constraint tolerance} ability increases to 4.
        In addition, your movement speed is not halved while \squeezing.

        \ff[21]{Greater Escape Magic} You can use your \textit{escape magic} ability as a \glossterm{minor action}.
        If you do, you \glossterm{briefly} cannot use it as a minor action again.
    \end{feat}

    \begin{feat}{Greatweapon Warrior}{Combat}
        \featpre Base Strength of 2.

        \ff[1]{Cleave} Whenever you wield a melee weapon in two hands, it gains the Sweeping (1) tag (see \pcref{Sweeping}).
        If the weapon already has the Sweeping tag, you increase the number of secondary targets by 1.
        In addition, you can choose secondary targets within 10 feet of the primary target instead of the normal 5 feet.
        Each secondary target must still be within your \glossterm{reach} with the weapon.

        \ff[3]{Power Attack} Whenever you make a non-\glossterm{projectile} strike with a weapon you wield in two hands, you may take a \minus1 penalty to \glossterm{accuracy}.
        If you do, you gain a \plus1d damage bonus.

        \ff[6]{Destructive Force} You gain a \plus2 bonus to \glossterm{accuracy} with the \textit{disarm} ability with weapons you wield in two hands (see \pcref{Disarm}).
        In addition, whenever you make a non-\glossterm{projectile} strike with a weapon you wield in two hands, it deals double damage to objects.

        \ff[9]{Greater Cleave} The tag granted by your \textit{cleave} ability changes to be Sweeping (2).
        If the weapon already has the Sweeping tag, you instead increase the number of secondary targets by 2.

        \ff[12]{Greater Power Attack} The damage bonus from your \textit{power attack} ability increases to \plus2d.

        \ff[15]{Greater Destructive Force} The accuracy bonus from your \textit{destructive force} ability increases to \plus4.
        In addition, the damage multiplier from your \textit{destructive force} ability increases to triple damage.

        \ff[18]{Supreme Cleave} The tag granted by your \textit{cleave} ability changes to be Sweeping (3).
        If the weapon already has the Sweeping tag, you instead increase the number of secondary targets by 3.

        \ff[21]{Greater Power Attack} The damage bonus from your \textit{power attack} ability increases to \plus3d.
    \end{feat}

    \begin{feat}{Herbalist}{Skill}
        \featpre Knowledge (nature) as a mastered skill.

        \ff[1]{Esoteric Concoction} You can use your Knowledge (nature) skill in place of Craft (alchemy) or Craft (poison) to create poisons and potions.
        This does not help you create other alchemical items, such as alchemist's fire.
        When you do, you must use esoteric natural ingredients in place of the normal ingredients.
        The replacement ingredients must be difficult to acquire in large quantities and impossible to acquire in a normal city.
        For example, you can use the tail of a blind mouse or the dew from a four-leafed clover, but you could not use dirt or ordinary tree bark.
        Once you have determined a purpose for a particular replacement ingredient, you cannot use that ingredient as a replacement in any other poison or potion.

        In general, it requires an hour of work and a Knowledge (nature) check equal to 5 \add the level of the item to find ingredients for an item in this way.
        Each time you find ingredients for an item this way, the time required to find ingredients again increases by an hour and the difficulty rating increases by 5.
        Whenever you finish a \glossterm{long rest} or enter a different environment with different ingredients, these penalties reset.

        \ff[3]{Potent Poisons} You gain a \plus1 bonus to \glossterm{accuracy} with any poisons you create, including poisonous spells you cast.

        \ff[6]{Tempting Concoction} You can use the \textit{tempting concoction} ability as a \glossterm{standard action}.
        \begin{attuneability}{Tempting Concoction}
            \spelltwocol{\abilitytag{Attune} (self)}{\abilitytag{Emotion}, \glossterm{Magical}, \abilitytag{Subtle}}
            \rankline
            \targets{See text}
            Choose one liquid poison or potion you created, or an object containing one of those liquids, within \rngshort range.
            Whenever an \glossterm{enemy} notices the chosen object, make an attack vs. Mental against it.
            If the poison or potion is not concealed inside a less suspicious object, such as a tankard of ale or an apple, you take a \minus4 penalty to \glossterm{accuracy}.
            You cannot make this attack more than once against any individual target during this ability's duration.
            \hit The target is filled with the desire to investigate and try to consume the liquid or the object containing the liquid.
            It will not generally interrupt combat or wander into obvious danger to fulfill its desire, but individual creatures may react more or less strongly.
            This effect lasts until the target consumes the object or until it takes a \glossterm{short rest}.

            \rankline
            \featlevel{12} You gain a \plus1 bonus to \glossterm{accuracy} with the attack.
            \featlevel{18} The accuracy bonus increases to \plus2.
        \end{attuneability}

        \ff[9]{Efficient Concoction} The time required to find ingredients with your \textit{esoteric concoction} ability is halved.

        \ff[12]{Greater Potent Poisons} The bonus from your \textit{potent concoction} ability increases to \plus2.

        \ff[15]{Poison Tolerance} You gain a \plus2 bonus to Fortitude defense.

        \ff[18]{Blended Poison} You can create poisons that combine two poison effects into a single dose.
        This requires twice the normal time to create a poison, and requires all ingredients required to make both poisons.
        A creature affected by the blended poison suffers the full effects of both poisons.

        \ff[21]{Greater Efficient Concoction} The time required to find ingredients with your \textit{esoteric concoction} ability is reduced to one-tenth of the normal time.
    \end{feat}

    \begin{feat}{Intimidate Specialization}{Skill}
        \featpre Intimidate as a mastered skill.

        \ff[1]{Specialization} You gain a \plus3 bonus to the Intimidate skill.

        \ff[3]{Greater Demoralize} When you use the \textit{demoralize} ability, the target is \shaken by you as a \glossterm{condition} instead of being shaken the end of the next round.
        For details, see \pcref{Demoralize}.

        \ff[6]{Threatening Presence} You are considered to be adjacent to all unoccupied spaces within a 10 foot radius \glossterm{emanation} from you for the purpose of determining whether your \glossterm{enemies} are \surrounded (see \pcref{Being Surrounded})
        This can allow you to surround isolated creatures by yourself.

        \ff[9]{Greater Specialization} The bonus from your \textit{specialization} ability increases to \plus5.

        \ff[12]{Supreme Demoralize} When you use the \textit{demoralize} ability, the target is \frightened by you instead of being shaken.

        \ff[15]{Supreme Specialization} The bonus from your \textit{specialization} ability increases to \plus7.

        \ff[18]{Greater Threatening Presence} The area of your \textit{threatening presence} ability increases to a \areasmall radius \glossterm{emanation}.

        \ff[21]{Mass Demoralize} When you use the \textit{demoralize} ability, it affects all \glossterm{enemies} within a \largearea radius.
    \end{feat}

    \begin{feat}{Iron Will}{General}
        \featpre Base Willpower of 1.

        \ff[1]{Mental Discipline} You gain a \plus2 bonus to Mental defense.
        In addition, you gain a \plus1 bonus to your \glossterm{fatigue tolerance}.

        \ff[3]{Mind over Matter} You may use your Willpower in place of your Constitution to determine your \glossterm{hit points} (see \pcref{Hit Points}).

        \ff[6]{Controlled Self} You gain a \plus10 bonus to notice \abilitytag{Subtle} abilities that affect you.

        \ff[6]{Unclouded Mind} You are immune to being \dazed and \stunned.

        \ff[9]{Greater Mental Discipline} The defense bonus from your \textit{mental discipline} ability increases to \plus4.

        % TODO: figure out right value for this
        \ff[12]{Greater Mind over Matter} You gain a \plus6 bonus to your maximum \glossterm{hit points}.

        \ff[15]{Greater Controlled Self} The bonus from your \textit{controlled self} ability increases to \plus20.

        \ff[15]{Greater Unclouded Mind} You are immune to being \disoriented and \confused.

        \ff[18]{Supreme Mental Discipline} The defense bonus from your \textit{mental discipline} ability increases to \plus6.
        In addition, the fatigue tolerance bonus increases to \plus2.

        \ff[21]{Supreme Unclouded Mind} You are immune to all \abilitytag{Compulsion} and \abilitytag{Emotion} attacks.
    \end{feat}

    \begin{feat}{Jump Specialization}{Skill}
        \featpre Jump as a mastered skill.

        \ff[1]{Specialization} You gain a \plus3 bonus to the Jump skill.

        \ff[3]{Instant Leap} You suffer no penalty for jumping without a running start (see \pcref{Running Start}).

        \ff[6]{Featherlight Leap} When you leap, your maximum height is equal to your Jump check result, rather than half your Jump check result.
        This does not affect the forward distance you can reach with your jumps.

        \ff[9]{Greater Specialization} The bonus from your \textit{specialization} ability increases to \plus5.

        \ff[12]{Impact Tolerance} You take half damage from \glossterm{falling damage}.

        \ff[15]{Supreme Specialization} The bonus from your \textit{specialization} ability increases to \plus7.

        \ff[18]{Greater Rebounding Leap} You take no penalty when using the \textit{rebounding leap} ability.

        \ff[21]{Greater Impact Tolerance} You are immune to \glossterm{falling damage}.
    \end{feat}

    \begin{feat}{Knowledge Specialization}{Skill}
        \featpre Any Knowledge skill as a mastered skill.

        \ff[1]{Specialization} You gain a \plus3 bonus to all Knowledge skills.

        \ff[3]{Knowledge Savant} You gain two additional \glossterm{skill points} which can only be spent on Knowledge skills.

        \ff[6]{Studied Defense} You gain a \plus1 bonus to Fortitude, Reflex, and Mental defenses. 

        \ff[9]{Greater Specialization} The bonus from your \textit{specialization} ability increases to \plus5.

        \ff[12]{Greater Knowledge Savant} The number of extra skill points from your \textit{knowledge savant} ability increases to four.

        \ff[15]{Supreme Specialization} The bonus from your \textit{specialization} ability increases to \plus7.

        \ff[18]{Greater Studied Defense} You gain a \plus1 bonus to Armor defense.

        \ff[21]{Studied Offense} You gain a \plus1 bonus to \glossterm{accuracy}.
    \end{feat}

    \begin{feat}{Leadership}{Combat}
        \featpre Either base Intelligence of 2 or base Willpower of 2.

        \ff[1]{Battle Command} You can use the \textit{battle command} ability as a standard action.
        \begin{freeability}{Battle Command}[\abilitytag{Swift}]
            Choose an \glossterm{ally} within \rngmed range.
            During the current phase, the target gains a \plus2 bonus to \glossterm{accuracy} and rolls twice for any attacks it makes, keeping the better result.

            \rankline
            \featlevel{6} The accuracy bonus increases to \plus3.
            \featlevel{12} The accuracy bonus increases to \plus4.
            \featlevel{18} The accuracy bonus increases to \plus5.
        \end{freeability}

        \ff[3]{Encouraging Presence} Your \glossterm{allies} within a \arealarge \glossterm{emanation} from you are immune to being \shaken, \frightened, and \panicked.

        \ff[6]{Bolster} You can use the \textit{bolster} ability as a standard action.
        \begin{freeability}{Bolster}[\abilitytag{Emotion}]
            One \glossterm{ally} within \rngmed range can remove a \glossterm{brief} effect or \glossterm{condition}.
            This cannot remove effects applied during the current round.

            \rankline
            \featlevel{12} You may target an additional \glossterm{ally} within range.
            \featlevel{18} Each target may remove an additional effect.
        \end{freeability}

        % TODO: nerf after Mozley
        \ff[9]{Inspiring Presence} Your \glossterm{allies} within a \arealarge \glossterm{emanation} from you gain a \plus2 bonus to Mental defense.

        \ff[12]{Brave Leader} You are immune to being \shaken, \frightened, and \panicked.

        \ff[15]{Sustaining Presence} Your \glossterm{allies} within a \arealarge \glossterm{emanation} from you gain a \plus1 bonus to \glossterm{vital rolls}.

        \ff[18]{Greater Inspiring Presence} The bonus from your \textit{inspiring presence} ability increases to \plus3.

        \ff[21]{Supreme Presence} The area of your \textit{presence} abilities from this feat increases to a \gargarea \glossterm{emanation} from you.
    \end{feat}

    \begin{feat}{Linguistics Specialization}{Skill}
        \featpre Linguistics as a mastered skill.

        \ff[1]{Specialization} You gain a \plus3 bonus to the Linguistics skill.

        \ff[3]{Linguistic Savant} You learn two additional \glossterm{common languages}, or one additional \glossterm{rare language}.

        \ff[6]{Language Focus} By spending a day in focused concentration on learning a specific \glossterm{common language}, you can use the \textit{language focus} ability.
        You must have access to either a creature fluent in the language willing to help you or at least a book's worth of material written in the language.
        \begin{freeability}{Language Focus}
            If you had access to written material on the language, including from a teacher, you can read or write the language.
            If you had access to a speaker of the language, you can speak and understand the language.

            This ability's effect lasts until you use this ability again.
        \end{freeability}

        \ff[9]{Greater Specialization} The bonus from your \textit{specialization} ability increases to \plus5.

        \ff[12]{Greater Language Focus} You can use your \textit{language focus} ability to learn \glossterm{rare languages} in addition to common languages.
        In addition, you can maintain two different instances of the ability instead of only one.

        \ff[15]{Supreme Specialization} The bonus from your \textit{specialization} ability increases to \plus7.

        \ff[18]{Greater Linguistic Savant} You learn four additional \glossterm{common languages}, or two additional \glossterm{rare languages}.

        \ff[21]{Supreme Language Focus} The effect of your \textit{language focus} ability is permanent.
    \end{feat}

    \begin{feat}{Maneuverist}{Combat}
        \featpre Base Intelligence of 1.

        \ff[1]{Maneuver Access} You gain access to one \glossterm{combat style} that you did not already have access to (see \pcref{Combat Styles}).
        In addition, you learn one rank 1 \glossterm{maneuver} from that combat style.
        You may spend \glossterm{insight points} to learn to one additional maneuver from that combat style per insight point.
        Unless otherwise noted in an ability's description, using a maneuver requires a \glossterm{standard action}.

        After you use a maneuver you know from this feat, you \glossterm{briefly} cannot use any maneuver from this feat.

        \ff[3]{Trained Maneuverist} Using a maneuver from this feat does not prevent you from using maneuvers from this feat.

        \ff[6]{Maneuver Rank} You become a rank 2 combat style user.
        This gives you access to maneuvers that require a minimum rank of 2.

        \ff[9]{Maneuver Rank} You become a rank 3 combat style user.
        This gives you access to maneuvers that require a minimum rank of 3 and can improve the effectiveness of your existing maneuvers.

        \ff[12]{Maneuver Rank} You become a rank 4 combat style user.
        This gives you access to maneuvers that require a minimum rank of 4 and can improve the effectiveness of your existing maneuvers.

        \ff[12]{Maneuver Knowledge} You learn one maneuver.

        \ff[15]{Maneuver Rank} You become a rank 5 combat style user.
        This gives you access to maneuvers that require a minimum rank of 5 and can improve the effectiveness of your existing maneuvers.

        \ff[18]{Maneuver Rank} You become a rank 6 combat style user.
        This gives you access to maneuvers that require a minimum rank of 6 and can improve the effectiveness of your existing maneuvers.

        \ff[21]{Maneuver Rank} You become a rank 7 combat style user.
        This gives you access to maneuvers that require a minimum rank of 7 and can improve the effectiveness of your existing maneuvers.

        \ff[21]{Maneuver Knowledge} You learn one maneuver.
    \end{feat}

    \begin{feat}{Martial Training}{Combat}
        \ff[1]{Trained Strike} You can use the \textit{trained strike} ability as a standard action.
        \begin{freeability}{Trained Strike}
            Make a \glossterm{strike} with a \plus1 bonus to \glossterm{accuracy}.

            \rankline
            \featlevel{6} The accuracy bonus increases to \plus2.
            \featlevel{12} The accuracy bonus increases to \plus3.
            \featlevel{18} The accuracy bonus increases to \plus4.
        \end{freeability}

        \ff[3]{Equipment Training} You choose one of the following benefits.
        \begin{itemize}
            \item You gain proficiency with a \glossterm{usage class} of \glossterm{armor} (light, medium, or heavy).
                You must be proficient with light armor to gain proficiency with medium armor, and you must be proficient with medium armor to gain proficiency with heavy armor.
            \item You gain proficiency with an additional \glossterm{weapon group} of your choice.
            \item You gain proficiency with \glossterm{exotic weapons} from a weapon group of your choice that you are already proficient with.
            \item You reduce the \glossterm{encumbrance} of \glossterm{body armor} you wear by 1.
                If you choose this ability multiple times, its effects stack.
        \end{itemize}

        \ff[6]{Martial Power} You gain a \plus2 bonus to your \glossterm{mundane} \glossterm{power}.

        \ff[9]{Equipment Training} You gain an additional \textit{equipment training} ability of your choice.

        \ff[12]{Greater Martial Power} The bonus from your \textit{martial power} ability increases to \plus4.

        \ff[15]{Equipment Training} You gain an additional \textit{equipment training} ability of your choice.

        \ff[18]{Greater Martial Power} The bonus from your \textit{martial power} ability increases to \plus8.

        \ff[21]{Martial Precision} You gain a \plus1 bonus to \glossterm{accuracy}.
    \end{feat}

    \begin{feat}{Medicine Specialization}{Skill}
        \featpre Medicine as a mastered skill.

        \ff[1]{Specialization} You gain a \plus3 bonus to the Medicine skill.

        \ff[3]{Healing Touch} You can use the \textit{healing touch} ability as a standard action.
        In addition, if you have the \textit{restoration} cleric ability, you gain a \plus2 bonus to \glossterm{power} with both the \textit{healing touch} and \textit{restoration} abilities.
        \begin{instantability}{Healing Touch}
            \spelltwocol{Instant}{\abilitytag{Healing}}
            \rankline
            Choose yourself or a living \glossterm{ally} within your \glossterm{reach}.
            The target regains 2d6 \add \glossterm{power} \glossterm{hit points}.
            After you use this ability, you \glossterm{briefly} cannot use it or any other \abilitytag{Healing} ability.

            \rankline
            \featlevel{6} The healing increases to 2d8.
            \featlevel{9} The healing increases to 4d6.
            \featlevel{12} The healing increases to 4d8.
            \featlevel{15} The healing increases to 5d10.
            \featlevel{18} The healing increases to 6d10.
            \featlevel{21} The healing increases to 8d10.
        \end{instantability}

        \ff[6]{Purging Touch} You can use the \textit{purging touch} ability as a standard action.
        \begin{instantability}{Purging Touch}
            Instant
            \rankline
            Make a Medicine check on yourself or an \glossterm{ally} you can touch.
            For each poison and disease on the target, if your check result is at least 10 higher than the \glossterm{power} of the effect, the effect is removed.

            \rankline
            \featlevel{12} You can target yourself and any number of \glossterm{allies} within your reach.
            \featlevel{18} You gain a \plus5 bonus to the check.
        \end{instantability}

        \ff[9]{Greater Specialization} The bonus from your \textit{specialization} ability increases to \plus5.

        \ff[12]{Lifesaver} You can use the \textit{first aid} ability as a \glossterm{minor action} (see \pcref{First Aid}).
        If you do, you \glossterm{briefly} cannot use it as a minor action again.

        \ff[15]{Supreme Specialization} The bonus from your \textit{specialization} ability increases to \plus7.

        \ff[18]{Preventative Medicine} You are immune to \glossterm{poisons} and \glossterm{diseases}.

        \ff[21]{Greater Lifesaver} Using the \textit{first aid} ability as a minor action does not prevent you from using it as a minor action again.
        In addition, using the \textit{first aid} ability to affect multiple creatures simultaneously does not cause you to suffer a penalty to the Medicine check.
    \end{feat}

    \begin{feat}{Mental Magic}{Casting, Magical}
        \featpre Spellcasting ability, base Willpower of 1.

        \ff[1]{Mental Casting} You connect to the magical essence of the universe differently from other spellcasters, allowing you to cast spells with purely mental effort.
        None of your spells have \glossterm{somatic components} or \glossterm{verbal components}.
        However, casting spells without components is more challenging.
        You increase your \glossterm{focus penalty} by 1.

        \ff[3]{Potent Mind} You gain a \plus2 bonus to \glossterm{magical} \glossterm{power}.

        \ff[6]{Innate Creativity} You learn an additional \glossterm{spell}.

        \ff[9]{Fractured Mind} Once per round, you can sustain an ability with the \abilitytag{Sustain} (minor) tag as a \glossterm{free action}.

        \ff[12]{Greater Potent Mind} The bonus from your \textit{potent mind} ability increases to \plus4.

        \ff[15]{Greater Innate Creativity} You learn an additional \glossterm{spell}.

        \ff[18]{Greater Fractured Mind} You can use your \textit{fractured mind} ability on abilities with the \abilitytag{Sustain} (standard) tag in addition to the \abilitytag{Sustain} (minor) tag.

        \ff[21]{Supreme Potent Mind} The bonus from your \textit{potent mind} ability increases to \plus8.
    \end{feat}

    \begin{feat}{Metacaster}{Casting, Magical}
        \featpre Ability to cast a spell.

        \ff[1]{Sphere Access} You gain access to an additional \glossterm{mystic sphere}.
        Each \glossterm{mystic sphere} has a set of \glossterm{spells} associated with it.
        You automatically learn all \glossterm{cantrips} from any mystic sphere you have access to.
        % TODO: wording
        If you have multiple \glossterm{magic sources}, you can cast spells from that sphere with any magic source that the mystic sphere belongs to.

        \ff[3]{Alter Damage} Whenever you cast a spell that deals damage, you can change the type of damage it deals based on the \glossterm{mystic spheres} you have access to.
        You can use this ability to affect both spells that deal damage directly and spells that cause effects or summon creatures that later deal damage.
        If you change a spell's damage type in this way, you change all damage done by the spell, even if the spell would originally deal damage of multiple types.

        The damage types for each mystic sphere are given in \trefnp{Mystic Sphere Damage Types}.
        Not all mystic spheres have associated damage types.

        \begin{dtable}
            \lcaption{Mystic Sphere Damage Types}
            \begin{dtabularx}{\columnwidth}{l >{\lcol}X}
                \tb{Mystic Sphere} & \tb{Damage Type} \tableheaderrule
                Aeromancy & Bludgeoning \\
                Aquamancy & Bludgeoning \\
                Astromancy & Energy \\
                Barrier & \tdash \\
                Bless & \tdash \\
                Channel Divinity & Energy \\
                Chronomancy & \tdash \\
                Cryomancy & Cold \\
                Electromancy & Electricity \\
                Enchantment & \tdash \\
                Fabrication & Physical \\
                Photomancy & Energy \\
                Polymorph & Physical \\
                Pyromancy & Fire \\
                Revelation & \tdash \\
                Summoning & \tdash \\
                Telekinesis & Physical \\
                Terramancy & Bludgeoning \\
                Thaumaturgy & Energy \\
                Toxicology & Acid \\
                Umbramancy & Cold \\
                Verdamancy & \tdash \\
                Vivimancy & \tdash \\
            \end{dtabularx}
        \end{dtable}

        \ff[6]{Spell Fusion} You can use the \textit{spell fusion} ability as a \glossterm{standard action}.
        \begin{freeability}{Spell Fusion}
            Choose two spells that you know which do not have the \abilitytag{Sustain} or \abilitytag{Attune} tags.
            You cast both spells simultaneously.
            Both spells that you fuse in this way must have the same area shape, such as a cone or sphere, and targeting restrictions, such as affecting only enemies or living creatures.
            If one spell affects a strictly larger area or a strictly larger number of targets than the other, you must use the smaller of the two areas or target counts.
            You must choose the same targets and area for both spells, if applicable.
            Roll the attack roll and damage for each spell separately.

            After you use this ability, you are unable to take any actions during the following round.
        \end{freeability}

        % TODO: make this based on spheres known?
        \ff[9]{Alter Conditions} When you cast a spell that inflicts a \glossterm{debuff} with a standard effect as a \glossterm{condition}, you can change that effect to another effect of the same rank.
        Debuff effect ranks are described in \trefnp{Debuff Effect Ranks}.
        To change the spell to inflict a particular effect, you must know another spell that inflicts that effect.
        \begin{dtable}
            \lcaption{Debuff Effect Ranks}
            \begin{dtabularx}{\columnwidth}{l >{\lcol}X}
                \tb{Rank} & \tb{Condition effects} \tableheaderrule
                1 & Dazed, dazzled, shaken (by you), sickened, slowed \\
                2 & Decelerated, frightened (by you), nauseated, stunned \\
                3 & Blinded, confused, decelerated, disoriented, immobilized, panicked (by you) \\
                4 & Asleep\fn{1}, paralyzed \\
            \end{dtabularx}
            1. The target wakes up if it takes a \glossterm{vital wound}, but cannot otherwise wake up during the condition.
        \end{dtable}

        \ff[12]{Greater Spell Fusion} Using your \textit{spell fusion} ability does not prevent you from acting during the \glossterm{movement phase} of the following round.

        \ff[15]{Greater Alter Conditions} You can now exchange debuffs for other debuffs of the same rank with all spells and abilities, not just spells and abilities that inflict \glossterm{conditions}.
        In addition, you can exchange a debuff for any debuff of a lower rank.

        \ff[18]{Sphere Access} You gain access to an additional \glossterm{mystic sphere}.
        You automatically learn all \glossterm{cantrips} from that mystic sphere.
        In addition, you may forget spells from your existing mystic spheres in exchange for spells from that mystic sphere.

        \ff[21]{Supreme Spell Fusion} Using your \textit{spell fusion} ability does not prevent you from taking \glossterm{minor actions} during the following round.
    \end{feat}

    \begin{feat}{Mystic Archer}{Casting}
        \featpre Ability to cast a spell.

        \ff[1]{Magical Strikes}[\abilitytag{Magical}] Whenever you make a ranged \glossterm{strike}, you can choose to treat that as a \glossterm{magical} ability.
        This allows you to use your \glossterm{power} with magical abilities to determine your damage.

        \ff[1]{Imbued Shot} You can use the \textit{imbued shot} ability as a standard action.
        \begin{freeability}{Imbued Shot}[\glossterm{Magical}]
            Make a ranged \glossterm{strike} with a \plus1 bonus to \glossterm{accuracy} using a \glossterm{projectile weapon} you wield.
            Because this is a \glossterm{magical} ability, you use your \glossterm{magical} \glossterm{power} to determine your damage instead of your \glossterm{mundane} power.

            \rankline
            \featlevel{6} The accuracy bonus increases to \plus2.
            \featlevel{12} The accuracy bonus increases to \plus3.
            \featlevel{18} The accuracy bonus increases to \plus4.
        \end{freeability}

        \ff[3]{Guided Projectiles}[Magical] Your attacks with projectiles ignore \glossterm{cover}, but not \glossterm{total cover}.

        \ff[6]{Imbue Projectile} When you cast a spell that does not have the \abilitytag{Attune} or \abilitytag{Sustain} tags,
            you can use the \textit{imbue projectile} ability.
        \begin{attuneability}{Imbue Projectile}
            \spelltwocol{\abilitytag{Attune} (self)}{\glossterm{Magical}}
            \rankline
            The spell does not have its effect immediately.
            Instead, its power is imbued in a \glossterm{projectile} you hold.
            An individual projectile can only be imbued with this ability once, even if multiple creatures use this ability on the same projectile.

            When you use your \textit{imbued shot} ability to attack with that projectile, the spell takes effect on the target of your \textit{imbued shot} ability.
            You must make any attack rolls required for the spell separately from your attack roll with the strike.
            After the spell takes effect this way, your attunement to this ability ends.
        \end{attuneability}

        \ff[9]{Phasing Projectiles}[\glossterm{Magical}] When attacking with projectiles, you can ignore all physical obstacles in single one-foot span.
        This can allow you to fire projectiles through creatures or solid walls, though it does not grant you the ability to see through a wall.

        \ff[12]{Greater Guided Projectiles} Your attacks with projectiles ignore \glossterm{concealment}, and you can roll twice for miss chances with projectile attacks (such as when attacking creatures you cannot see).

        \ff[15]{Greater Phasing Projectiles}[\glossterm{Magical}] Your \textit{phasing projectiles} ability improves, allowing you to ignore obstacles in up to five one-foot spans.
        The spans can be contiguous or independent, which can allow you to ignore a single obstacle up to five feet deep.

        \ff[18]{Supreme Guided Projectiles} You gain a \plus1 bonus to \glossterm{accuracy} with projectile attacks.
        In addition, you ignore all miss chances with projectile attacks.

        \ff[21]{Supreme Phasing Projectiles} The distance you can ignore with your \textit{phasing projectiles} ability increases to fifteen feet.
    \end{feat}

    \begin{feat}{Null}{General}
        \featpre Base Willpower of 2.

        \ff[1]{Nullify Magic} You gain a \plus4 bonus to \glossterm{defenses} against \glossterm{magical} abilities.
        In addition, you are never considered an \glossterm{ally} for a \glossterm{magical} ability, even while \unconscious.
        In exchange, you lose the benefits of all \glossterm{magical} abilities you possess.
        In addition, you are unable to \glossterm{attune} to any \glossterm{magical} abilities, such as magic items or spells cast by other creatures.

        % Much stronger than Spellbreaker because this is a huge part of the whole Null thing
        \ff[1]{Sever Magic} You can use the \textit{sever magic} ability as a standard action.
        \begin{freeability}{Sever Magic}
            Make a \glossterm{strike}.
            You take a \minus1d damage penalty with the strike.
            If the target takes damage from the strike, it stops being \glossterm{attuned} to one effect of its choice that it is currently attuned to.
            On a \glossterm{critical hit}, the target takes double damage and it stops being attuned to two abilities of its choice that it is currently attuned to.
            In addition, as a \glossterm{condition}, it stops being able to attune to abilities.

            \rankline
            \featlevel{6} You gain a \plus1 bonus to \glossterm{accuracy} with the strike.
            \featlevel{12} A struck target stops being attuned to an additional effect of its choice.
            \featlevel{18} The accuracy bonus increases to \plus2.
        \end{freeability}

        \ff[1]{Spell Sensitivity} You treat the Spellsense skill as a \glossterm{class skill} (see \pcref{Spellsense}).

        \ff[3]{Mundane Resilience} You gain a \plus4 bonus to \glossterm{damage resistance}.

        \ff[3]{Personal Legacy} You do not gain any legacy item upgrades (see \pcref{Legacy Items}).
        Instead, each time you would gain a legacy item upgrade, you instead gain a \plus1 bonus to \glossterm{accuracy}, all \glossterm{defenses}, and \glossterm{fatigue tolerance}.

        % clarify that it does not affect spells already in progress in the current phase?
        \ff[6]{Disruptive Presence} Whenever an \glossterm{enemy} within an \areamed radius from you casts a spell, the spell has a 50\% chance to fail with no effect.

        \ff[6]{Greater Nullify Magic} The bonus to defenses from your \textit{nullify magic} ability increases to \plus6.

        \ff[9]{Greater Mundane Resilience} The bonus from your \textit{mundane resilience} ability increases to \plus8.

        \ff[9]{Itembane} Whenever you touch a \glossterm{magical} item or hit it with a melee weapon, such as with the \textit{disarm} ability, it \glossterm{briefly} loses all magical abilities (see \pcref{Disarm}).
        This does not prevent you from suffering the normal effects of the item's initial hit, if the item was used to strike you.
        Under normal circumstances, removes the abilities of items that hit you with melee \glossterm{strikes}, but does not affect magical projectile weapons.
        Items with an intrinsic \glossterm{power} at least 10 higher than your level are immune to this effect.
        The \glossterm{power} of the item's wielder, if any, does not affect whether the item can be affected in this way.

        \ff[12]{Supreme Nullify Magic} The bonus to defenses from your \textit{nullify magic} ability increases to \plus8.

        \ff[15]{Supreme Mundane Resilience} The bonus from your \textit{mundane resilience} ability increases to \plus16.

        \ff[15]{Greater Disruptive Presence} Your \textit{disruptive presence} ability affects all enemies in a \areahuge radius \glossterm{emanation} from you.

        \ff[18]{True Null} You are unaffected by all \glossterm{magical} abilities.

        \ff[21]{Legendary Mundane Resilience} The bonus from your \textit{mundane resilience} ability increases to \plus32.

        \ff[21]{Supreme Disruptive Presence} The miscast chance from your \textit{disruptive presence} ability increases to 90\%.
    \end{feat}

    \begin{feat}{Perform Specialization}{Skill}
        \featpre Any Perform skill as a mastered skill.

        \ff[1]{Specialization} You gain a \plus3 bonus to all Perform skills.

        \ff[3]{Synergistic Performance} You can use your Perform skills in place of other related skills.
        Each Perform skill has an associated skill that it can be used to replace, as listed below.
        When you replace a skill in this way, you add half your modifier with the Perform skill instead of your full modifier since the two skills do not exactly match.
        \begin{itemize}
            \item Acting: Deception
            \item Comedy: Deception
            \item Dance: Balance
            \item Keyboard instruments: Devices
            \item Oratory: Persuasion
            \item Percussion instruments: Creature Handling
            \item Singing: Persuasion
            \item String instruments: Devices
            \item Wind instruments: Creature Handling
        \end{itemize}

        % TODO: nerf after Mozley
        \ff[6]{Inspiring Performance}[Magical] Whenever you perform with the Perform skill, each \glossterm{ally} that can observe the performance gains a \plus2 bonus to Mental defense.
        % This solves the "always active except on round 1" problem
        This effect has the \glossterm{Swift} tag, so it protects allies in the same phase that you begin performing.
        This includes both normal performances and any special abilities that require performances.
        This bonus lasts as long as the performance lasts.

        \ff[9]{Greater Specialization} The bonus from your \textit{specialization} ability increases to \plus5.

        \ff[12]{Greater Inspiring Performance} The bonus from your \textit{inspiring performance} ability increases to \plus4.

        \ff[15]{Supreme Specialization} The bonus from your \textit{specialization} ability increases to \plus7.

        \ff[18]{Greater Inspiring Performance} The bonus from your \textit{inspiring performance} ability increases to \plus3.

        \ff[21]{Endless Performance} You can sustain performances for any length of time.
        This affects both normal performances and any special abilities that require performances to sustain them, allowing you to sustain those abilities beyond the normal 5 minute limit.
    \end{feat}

    \begin{feat}{Persuasion Specialization}{Skill}
        \featpre Persuasion as a mastered skill.

        \ff[1]{Specialization} You gain a \plus3 bonus to the Persuasion skill.

        \ff[3]{Compel Attention}[Magical] You can use the \textit{compel attention} ability as a standard action.
        \begin{freeability}{Compel Attention}[\abilitytag{Auditory}, \abilitytag{Compulsion}, \abilitytag{Sustain} (minor), \abilitytag{Subtle}]
            Make an attack vs. Mental against a creature within \rngmed range.
            Your \glossterm{accuracy} is equal to your Persuasion skill.
            You must talk loud enough for the target to hear to draw its attention.
            \hit The target is \fascinated by you as long as you sustain this ability, which requires maintaining your conversation with it.
            Any act by you or by creatures that appear to be your ally that damages a target or that causes it to feel that it is in danger breaks the effect for that creature.
            An observant target may interpret overt threats to its \glossterm{allies} as a threat to itself.

            \rankline
            \featlevel{9} You may target up to five creatures within range.
            \featlevel{15} You may target any number of creatures within range.
            \featlevel{21} The range increases to \distrange.
        \end{freeability}

        \ff[6]{First Impressions} When you first meet creatures, you have an Ally relationship instead of a Just Met relationship (see \tref{Relationship Modifiers}.
        This does not improve your relationship with creatures who already have an impression of you, whether positive or negative.

        \ff[9]{Greater Specialization} The bonus from your \textit{specialization} ability increases to \plus5.

        \ff[12]{Suggestion}[Magical] You can use the \textit{suggestion} ability as a standard action.
        \begin{freeability}{Suggestion}[\abilitytag{Emotion}, \abilitytag{Subtle}, \abilitytag{Sustain} (minor)]
            Make an attack vs. Mental against a target within \rngmed range.
            Your \glossterm{accuracy} is equal to your Persuasion skill.
            You must also make a verbal suggestion of a particular course of action to the target.
            If your suggestion does not seem reasonable, you take a \minus5 accuracy penalty on this attack.
            Exceptionally unreasonable suggestions can impose even greater penalties, and exceptionally reasonable suggestions can give accuracy bonuses.

            \hit As a \glossterm{condition}, the target thinks your suggestion is a good idea and will try to follow it to the best of its abilities.
            Any act by you or by creatures that appear to be your ally that damages the target or makes it feel that it is in danger breaks the effect.
            An observant target may interpret overt threats to its \glossterm{allies} as a threat to itself.

            \rankline
            \featlevel{18} You may target up to five creatures within range.
        \end{freeability}

        \ff[15]{Supreme Specialization} The bonus from your \textit{specialization} ability increases to \plus7.

        \ff[18]{Greater First Impressions} When you first meet creatures, you have a Friend relationship instead of a Just Met relationship
        This does not improve your relationship with creatures who already have an impression of you, whether positive or negative.

        \ff[21]{Rapid Persuasion} You can make a Persuasion check within the first round of a conversation at no penalty instead of the normal requirement to talk for a minute or longer.
    \end{feat}

    \begin{feat}{Precognition}{General}
        \featpre Base Intelligence of 2.

        \ff[1]{Precognitive Reaction} You can use your Intelligence in place of your Perception to determine your your \glossterm{accuracy} with \glossterm{mundane} abilities and your \glossterm{initiative}.

        \ff[3]{Combat Prediction} You can use the \textit{combat prediction} ability as a standard action.
        \begin{freeability}{Combat Prediction}[\abilitytag{Sustain} (free)]
            Make an attack vs. Mental with a \plus3 bonus to \glossterm{accuracy} against a creature within \rngmed range of you.
            \hit That creature's intentions become obvious to you as long as you sustain this ability.
            This gives you a \plus2 bonus to \glossterm{accuracy} and \glossterm{defenses} against that creature.
            At the start of each phase, you can see and hear what actions that creature intends to take.
            You do not gain any knowledge of actions that have no obvious signs, such as purely mental actions, or actions which you are not observant enough to notice.
            In addition, you do not know the results of actions with a chance of failure, such as attacks.

            The creature may change its actions based on your interference if you communicate your insight in a way it understands.
            \rankline
            You gain a \plus1 bonus to \glossterm{accuracy} with the attack at 6th level and every 3 levels thereafter.
        \end{freeability}

        \ff[6]{Foresight} During the \glossterm{movement phase}, you choose your action after all other creatures have chosen their actions.
        When you choose your action, you have insight into the actions chosen by any creatures within \shortrange of you that you can see.
        This insight gives you the same information as the insight from your \textit{combat prediction} ability, except that it only provides information about their actions during the movement phase.
        You choose your actions simultaneously with any other creatures who have a similar ability.

        Knowing another creature's action does not automatically allow you to interrupt that action.
        If you want to interrupt an action, such as by blocking a creature's intended movement, you must make an \glossterm{initiative} check as normal.

        \ff[9]{Precognitive Precision} You gain a \plus1 bonus to \glossterm{accuracy}.

        \ff[12]{Greater Precognitive Reaction} You gain a \plus4 bonus to \glossterm{initiative} checks.

        \ff[15]{Greater Foresight} The range of your \textit{foresight} ability increases to \rnglong range.

        \ff[18]{Greater Precognitive Precision} The bonus from your \textit{precognitive precision} ability increases to \plus2.

        \ff[21]{Supreme Precognitive Reaction} The bonus from your \textit{greater precognitive reaction} ability increases to \plus10.
    \end{feat}

    \begin{feat}{Prepared Spellcasting}{Magical, Spell}
        \featpre Ability to cast a spell, base Intelligence of 2.

        \ff[1]{Spellbook} Choose up to three spells you do not know from among \glossterm{mystic spheres} you have access to.
        The spells in your spellbook can come from any combination of \glossterm{magic sources} you can cast spells with.
        The spells must be of a rank that you know how to cast.
        Whenever you gain access to a new spell rank, you may change the spells in your spellbook for any other spells you can cast.
        You inscribe the knowledge of those spells into a book you carry with you.
        This book is your spellbook.
        
        Whenever you finish a \glossterm{long rest}, you may choose one of the spells in your spellbook.
        You learn how to cast that spell until you choose a different spell with this ability.

        \ff[3]{Study of Magic} You gain a \plus2 bonus to \glossterm{magical} \glossterm{power}.

        \ff[6]{Studious Learning} You gain a \plus2 bonus to all Knowledge skills.

        \ff[9]{Expanded Spellbook} You can choose up to five spells to be in your spellbook instead of only three.

        \ff[12]{Greater Study of Magic} The bonus from your \textit{study of magic} ability increases to \plus4.

        \ff[15]{Greater Spellbook} Whenever you finish a \glossterm{long rest}, you may choose two spells in your spellbook with your \textit{spellbook} ability instead of one.
        You learn how to cast both spells until you choose a different pair of spells in this way.

        \ff[18]{Greater Expanded Spellbook} You can choose up to seven spells to be in your spellbook instead of only three.

        \ff[21]{Supreme Study of Magic} The bonus from your \textit{study of magic} ability increases to \plus8.
    \end{feat}

    \begin{feat}{Rapid Reaction}{General}
        \featpre Base Dexterity of 1.

        \ff[1]{Lightning Reflexes} You gain a \plus2 bonus to Reflex defense and \glossterm{initiative} checks.

        \ff[3]{Sidestep} If you have at least five feet of movement remaining after the \glossterm{movement phase}, you may move up to five feet during the \glossterm{action phase} or the \glossterm{delayed action phase} as a \glossterm{free action}.

        \ff[6]{Evasive Reaction} You take half damage from abilities that affect an area and attack your Armor or Reflex defense.
        This does not protect you from any non-damaging effects of those abilities, or from abilities that affect multiple specific targets without affecting an area.
        If you have the \textit{evasion} monk or rogue ability with the same effect as this ability, you reduce the total damage you take to one quarter of the normal value instead.

        \ff[9]{Greater Lightning Reflexes} The bonuses from your \textit{lightning reflexes} ability increase to \plus4.

        \ff[12]{Greater Sidestep} The movement you can carry over with your \textit{sidestep} ability increases to half your \glossterm{land speed}.

        \ff[15]{Greater Evasive Reaction} Your \textit{evasive reaction} ability also protects you from area attacks against your Fortitude and Mental defenses.

        \ff[18]{Supreme Lightning Reflexes} The bonuses from your \textit{lightning reflexes} ability increase to \plus6.

        \ff[21]{Supreme Sidestep} The movement you can carry over with your \textit{sidestep} ability increases to your full \glossterm{land speed}.
    \end{feat}

    \begin{feat}{Regenerator}{General}
        \featpre Base Constitution of 2.

        \ff[1]{Diehard} You gain a \plus2 bonus to \glossterm{vital rolls}.

        \ff[3]{Regenerative Recovery} You can use the \textit{regenerative recovery} ability as a standard action.
        \begin{instantability}{Regenerative Recovery}
            \spelltwocol{Instant}{\abilitytag{Healing}}
            \rankline
            You must use your Constitution in place of your Strength to determine your \glossterm{power} with this ability.
            You regain 2d6 \add \glossterm{power} \glossterm{hit points}.
            After you use this ability, you \glossterm{briefly} cannot use it or any other \abilitytag{Healing} ability.

            \rankline
            \featlevel{6} The healing increases to 2d8.
            \featlevel{9} The healing increases to 4d6.
            \featlevel{12} The healing increases to 4d8.
            \featlevel{15} The healing increases to 5d10.
            \featlevel{18} The healing increases to 6d10.
            \featlevel{21} The healing increases to 8d10.
        \end{instantability}

        \ff[6]{Regenerative Rest} When you take a \glossterm{short rest}, you can remove any number of \glossterm{vital wounds} affecting you.
        If you do, you increase your \glossterm{fatigue level} by three per vital wound removed this way.
        Once you increase your fatigue level to the point of unconsciousness, you cannot remove additional vital wounds with this ability.

        \ff[9]{Greater Diehard} The bonus from your \textit{diehard} ability increases to \plus3.

        \ff[12]{Greater Regenerative Rest} Your \textit{regenerative rest} ability only causes you to increase your fatigue level by two per vital wound.

        \ff[12]{Battlefield Regeneration} When you use the \textit{recover} action, you can also remove a single vital wound.
        You cannot use this ability to remove a vital wound that you gained during the current round.

        \ff[15]{Deep Rest} You can use your \textit{regenerative rest} ability to remove vital wounds even once your fatigue level would already make you unconscious.
        This allows you to recover any number of vital wounds regardless of your maximum fatigue level if you go unconscious to do so.

        \ff[18]{Supreme Diehard} The bonus from your \textit{diehard} ability increases to \plus4.

        \ff[21]{Supreme Regenerative Rest} Your \textit{regenerative rest} ability only causes you to increase your fatigue level by one per vital wound.
    \end{feat}

    \begin{feat}{Ride Specialization}{Skill}
        \featpre Ride as a mastered skill.

        \ff[1]{Specialization} You gain a \plus3 bonus to the Ride skill.

        \ff[3]{Mounted Defense} Your mount gains a \plus3 bonus to all defenses, up to a maximum of your own corresponding defense.

        \ff[6]{Mounted Warrior} The penalty you take for making ranged \glossterm{strikes} while mounted is decreased by 2.
        In addition, while you are mounted, you gain a \plus1 bonus to \glossterm{accuracy} with Mounted weapons (see \pcref{Mounted Weapon}).

        \ff[9]{Greater Specialization} The bonus from your \textit{specialization} ability increases to \plus5.

        \ff[12]{Greater Mounted Defense} The defense bonus from your \textit{mounted defense} ability increases to \plus6.

        \ff[15]{Supreme Specialization} The bonus from your \textit{specialization} ability increases to \plus7.

        \ff[18]{Greater Mounted Warrior} The penalty reduction from your \textit{mounted warrior} ability increases to 4.
        In addition, the accuracy bonus increases to \plus2.

        \ff[21]{Greater Mounted Defense} The defense bonus from your \textit{mounted defense} ability increases to \plus12.
    \end{feat}

    \begin{feat}{Savage}{Combat}
        \featpre Base Strength of 2.

        \ff[1]{Brute Force} You gain a \plus1 bonus to \glossterm{accuracy} with the \textit{shove} and \textit{overrun} abilities (see \pcref{Shove}, and \pcref{Overrun}).
        In addition, you gain a \plus1 bonus to your \glossterm{mundane} \glossterm{power}.

        \ff[3]{Wall Slam} Whenever you \glossterm{push} a creature with a \glossterm{mundane} ability and the creature's movement is interrupted by a solid object, the object and creature both take bludgeoning damage equal to 1d10 plus half your \glossterm{power}.
        This damage is not doubled when you get a critical hit with the ability that caused the push.
        Any individual creature or object cannot take damage in this way more than once per round.
        This damage increases by \plus1d at 6th level and every 3 levels thereafter.

        \ff[6]{Trample} You can use the \textit{trample} ability as a standard action.
        This ability functions like the \textit{overrun} ability, except that it does not cause you to increase your \glossterm{fatigue level} and creatures may not choose to avoid you.
        In addition, if you move through a creature's space, it takes bludgeoning damage equal to 1d10 plus half your \glossterm{power}.
        This damage is doubled when you get a critical hit on the trample attack.
        This damage increases by \plus1d at 9th level and every 3 levels thereafter.

        \ff[9]{Greater Brute Force} The accuracy bonus from your \textit{brute force} ability increases to \plus2.
        In addition, the power bonus increases to \plus4.

        \ff[12]{Limitless Savagery} Using the \textit{overrun} ability does not cause you to increase your \glossterm{fatigue level}.

        \ff[15]{Greater Trample} Any creature that you deal damage to with your \textit{trample} ability falls \glossterm{prone}.

        \ff[18]{Supreme Brute Force} The accuracy bonus from your \textit{brute force} ability increases to \plus3.
        In addition, the power bonus increases to \plus8.

        \ff[21]{Greater Wall Slam} Your \textit{wall slam} ability also deals damage when you \glossterm{knockback} a creature with a mundane ability.

        \ff[21]{Knockback Force} When you use the \textit{shove} ability, you can \glossterm{knockback} the target up to 20 feet instead of pushing it.
    \end{feat}

    \begin{feat}{Shieldbearer}{Combat}
        \featpre Base Strength of 2.

        \ff[1]{Shield Expertise} You gain a \plus1 bonus to Armor defense while you wield a shield.

        \ff[3]{Forceful Block} Whenever a creature misses you with a melee \glossterm{strike}, if you are wielding a shield, that creature \glossterm{briefly} takes a \minus1 penalty to Armor defense.
        As normal, this bonus does not stack with itself, even if the same creature misses you with multiple melee attacks.

        \ff[6]{Arrow Deflection} While you wield a shield, you and each \glossterm{ally} adjacent to you gain a \plus2 bonus to Armor defense against \glossterm{mundane} ranged attacks from weapons or projectiles that are at least one \glossterm{size category} smaller than you.

        \ff[9]{Greater Shield Expertise} The bonus from your \textit{shield expertise} ability increases to \plus2.

        \ff[12]{Greater Forceful Block} The penalty from your \textit{forceful block} ability increases to \minus2.

        \ff[15]{Greater Arrow Deflection} The bonus from your \textit{arrow deflection} ability increases to \plus4.

        \ff[18]{Greater Shield Expertise} The bonus from your \textit{shield expertise} ability increases to \plus3.

        \ff[21]{Supreme Forceful Block} The penalty from your \textit{forceful block} ability increases to \minus4.
    \end{feat}

    \begin{feat}{Sleight of Hand Specialization}{Skill}
        \featpre Sleight of Hand as a mastered skill.

        \ff[1]{Specialization} You gain a \plus3 bonus to the Sleight of Hand skill.

        \ff[3]{Deep Pickpocket} You can use the \textit{pickpocket} ability to retrieve objects that are loose within larger containers, such as backpacks or sacks, even if they are not immediately accessible.
        You must be able to reach at least one of your fingers into the bag, such as through a narrow gap at the opening.
        This does not allow you to retrieve objects from locked containers with no openings.
        The container's size cannot exceed your own size.

        \ff[6]{Extradimensional Concealment}[Magical] When you use the \textit{conceal object} ability, you can use the \textit{extradimensional pocket} ability.
        \begin{attuneability}{Extradimensional Pocket}
            \spelltwocol{\abilitytag{Attune} (self)}{\glossterm{Magical}}
            \rankline
            You conceal the object in a pocket dimension that cannot be accessed by nonmagical means.
            When your attunement to this ability ends, the object appears in a free hand.
            If you have no free hands, it drops to the ground.
        \end{attuneability}

        \ff[9]{Greater Specialization} The bonus from your \textit{specialization} ability increases to \plus5.

        \ff[12]{Greater Conceal Object} The maximum size of object you can hide with your \textit{conceal object} ability increases to be equal to your size category.
        You take a \minus10 penalty if the object is the same size category as you instead of one size category smaller.

        \ff[15]{Supreme Specialization} The bonus from your \textit{specialization} ability increases to \plus7.

        \ff[18]{Greater Deep Pickpocket} The maximum size of the container you can reach into with your \textit{deep pickpocket} ability increases to two size categories larger than your own size.

        \ff[21]{Supreme Conceal Object} The \minus10 penalty for concealing an object of the same size category as you with your \textit{conceal object} ability is removed.
    \end{feat}

    \begin{feat}{Sniper}{Combat}
        \featpre Base Perception of 2.

        \ff[1]{Aim} You can use the \textit{aim} ability as a standard action.
        \begin{freeability}{Aim}[\abilitytag{Focus}, \abilitytag{Sustain} (minor)]
            Choose a creature or object within line of sight.
            You gain a \plus2 accuracy bonus against the target.

            If you lose sight of the target for a full round, this effect ends.

            \rankline
            \featlevel{6} You also gain a \plus4 bonus to \glossterm{power} against the target if it is \unaware of you.
            \featlevel{12} The \glossterm{accuracy} bonus increases to \plus3.
            \featlevel{18} The \glossterm{power} bonus increases to \plus8.
        \end{freeability}

        \ff[3]{Distance Tolerance} You reduce your \glossterm{longshot penalty} by 1.

        \ff[6]{Precise Shot} You ignore \glossterm{cover} (but not \glossterm{total cover}) with ranged attacks.

        \ff[9]{Sniper's Precision} You gain a \plus1 bonus to \glossterm{accuracy}.

        \ff[12]{Greater Precise Shot} You ignore \glossterm{concealment} with ranged attacks.

        \ff[15]{Greater Distance Tolerance} The penalty reduction from your \textit{distance tolerance} ability increases to 2.

        \ff[18]{Greater Sniper's Precision} The accuracy bonus from your \textit{sniper's precision} ability increases to \plus2.

        \ff[21]{Supreme Distance Tolerance} The penalty reduction from your \textit{distance tolerance} ability increases to 4.
    \end{feat}

    \begin{feat}{Social Insight Specialization}{Skill}
        \featpre Social Insight as a mastered skill.

        \ff[1]{Specialization} You gain a \plus3 bonus to the Social Insight skill.

        \ff[3]{Social Intuition} You reduce the penalty for making a social assessment after only a single round of observation, and the penalty for making a social assessment without understanding the language, by 5 each (see \pcref{Social Assessment}).

        \ff[6]{Read Emotions}[Magical] You can use the \textit{read emotions} ability as a standard action.
        \begin{freeability}{Read Emotions}[\abilitytag{Emotion}, \abilitytag{Sustain} (minor), \abilitytag{Subtle}]
            Make an attack vs. Mental against a creature within \rngshort range.
            Your \glossterm{accuracy} is equal to your Social Insight skill.
            \hit You know the target's current emotions.
            In addition to the obvious effects, this grants you a \plus3 bonus to Deception, Persuasion, Intimidate, and Social Insight attacks and checks against the target.

            \rankline
            \featlevel{12} The range increases to \rnglong range.
            \featlevel{18} You can use this ability as a \glossterm{minor action}.
        \end{freeability}

        \ff[9]{Greater Specialization} The bonus from your \textit{specialization} ability increases to \plus5.

        \ff[12]{Truthsense} Whenever a creature within a \arealarge radius \glossterm{emanation} from you that you can hear and see speaks truth to the best of its knowledge with no attempt at evasion, concealment, or creative wording, you automatically recognize that.
        You do not recognize truth in this way if a creature is using the Deception skill in any way, even if it is speaking the truth.

        \ff[15]{Supreme Specialization} The bonus from your \textit{specialization} ability increases to \plus7.

        \ff[18]{Greater Social Intuition} You take no penalty for making a social assessment after only a single round of observation, and you take no penalty for not knowing the language (see \pcref{Social Assessment}).

        \ff[21]{Greater Truthsense} The area of your \textit{truthsense} ability increases to a \gargarea radius.
        In addition, you automatically recognize the difference between a creatively worded truth and an outright lie.
    \end{feat}

    \begin{feat}{Spellsense Specialization}{Skill}
        \featpre Spellsense as a mastered skill.

        \ff[1]{Specialization} You gain a \plus3 bonus to the Spellsense skill.

        \ff[3]{Sense Spellcasting}[Magical] You can use the \textit{sense spellcasting} ability as a standard action.
        \begin{freeability}{Sense Spellcasting}[\abilitytag{Subtle}]
            \target{One creature within \rngmed range}
            Make a Spellsense attack vs. Mental against the target.
            \hit You know whether the target is capable of casting spells.
            If the target can cast spells, you know what sources the target can cast spells from.
            \crit As above, except that you also know all \glossterm{mystic spheres} the target is capable of casting.
            This does not grant you knowledge of any specific spells the target knows.

            After using this ability on a target, you cannot use it again on the same target for 24 hours regardless of whether you hit or miss.

            \rankline
            You gain a \plus1 bonus to \glossterm{accuracy} with the attack at 6th level and every 3 levels thereafter.
        \end{freeability}

        \ff[6]{Unweave Magic}[Magical] You can use the \textit{unweave magic} ability as a standard action.
        \begin{freeability}{Unweave Magic}
            Make a Spellsense check on an active spell effect within \rngmed range.
            This can affect spells with the \abilitytag{Sustain} tag, but it cannot affect spells with the \abilitytag{Attune} tag.
            The \glossterm{difficulty rating} is equal to 5 \add the \glossterm{power} of the effect.
            Success means the effect is \glossterm{dismissed} if it is an effect that can be dismissed.

            \rankline
            \featlevel{12} You can target up to two spell effects within range.
            \featlevel{18} You can affect spells with the \abilitytag{Attune} tag.
            The \glossterm{difficulty rating} to affect those spells is increased by 10.
        \end{freeability}

        \ff[9]{Greater Specialization} The bonus from your \textit{specialization} ability increases to \plus5.

        \ff[12]{Mystic Power} You gain a \plus4 bonus \glossterm{magical} \glossterm{power}.

        \ff[15]{Supreme Specialization} The bonus from your \textit{specialization} ability increases to \plus7.

        \ff[18]{Greater Sense Spellcasting} You can use your \textit{sense spellcasting} ability as a \glossterm{minor action}.
        In addition, a hit grants you knowledge of all mystic spheres the target is capable of casting.

        \ff[21]{Greater Mystic Power}[Magical] The bonus from your \textit{mystic power} ability increases to \plus10.
    \end{feat}

    \begin{feat}{Spellsword}{Magical, Spell}
        \featpre Ability to cast a spell.

        \ff[1]{Magical Strikes} Whenever you make a melee \glossterm{strike}, you can choose to treat that as a \glossterm{magical} ability.
        This allows you to use your \glossterm{power} with magical abilities to determine your damage.

        \ff[1]{Imbued Blow} You can use the \textit{imbued blow} ability as a standard action.
        \begin{freeability}{Imbued Blow}
            Make a melee \glossterm{strike} with a \plus1 bonus to \glossterm{accuracy}.
            Because this is a \glossterm{magical} ability, you use your \glossterm{magical} \glossterm{power} to determine your damage instead of your \glossterm{mundane} power.

            \rankline
            \featlevel{6} The accuracy bonus increases to \plus2.
            \featlevel{12} The accuracy bonus increases to \plus3.
            \featlevel{18} The accuracy bonus increases to \plus4.
        \end{freeability}

        \ff[3]{Spellsword Conduit} You can cast spells using a melee weapon as if it were an implement (see \pcref{Implements}).
        When you do, you reduce your \glossterm{focus penalty} by 2.
        In addition, if your legacy item is a weapon, you may choose both weapon and implement magic item effects for it.

        \ff[6]{Imbue Weapon} When you cast a spell that does not have the \abilitytag{Attune} or \abilitytag{Sustain} tags,
            you can use the \textit{imbue weapon} ability.
        \begin{attuneability}{Imbue Weapon}
            \spelltwocol{\abilitytag{Attune} (self)}{\glossterm{Magical}}
            \rankline
            The spell does not have its effect immediately.
            Instead, its power is imbued in a melee weapon you hold.
            An individual weapon can only be imbued with this ability once.

            When you use your \textit{imbued blow} ability to make a strike with that weapon, you may choose to activate the spell.
            If you do, the spell takes effect on the target of your \textit{imbued blow} ability as if you had just cast it.
            You must make any attack rolls required for the spell separately from your attack roll with the strike.
            After the spell takes effect this way, your attunement to this ability ends.
        \end{attuneability}

        \ff[9]{Personal Enhancement} You gain a \plus1 \glossterm{magic bonus} to \glossterm{accuracy} and \glossterm{defenses}.
        Because this is a magic bonus, it does not stack with other magic bonuses (see \pcref{Stacking Rules}).

        \ff[12]{Greater Spellsword Conduit} Whenever you cast a spell using a melee weapon as an implement, you gain a \plus1 bonus to \glossterm{accuracy} with the spell.

        \ff[15]{Greater Personal Enhancement} The bonuses from your \textit{personal enhancement} ability increase to \plus2.

        \ff[18]{Greater Imbue Weapon} You may imbue two spells with your \textit{imbue weapon} ability instead of only one.
        This only costs a single \glossterm{attunement point}.
        When you activate a spell imbued in this way, you \glossterm{briefly} cannot activate the other spell.

        \ff[21]{Supreme Spellsword Conduit} The accuracy bonus from your \textit{greater spellsword conduit} ability increases to \plus2.
    \end{feat}

    \begin{feat}{Spellwarped}{General, Magical}
        \featpre Base Willpower of 1.

        \ff[1]{Mystic Sphere} You gain the ability to use arcane magic.
        You gain access to one arcane \glossterm{mystic sphere} (see \pcref{Arcane Mystic Spheres}).
        Each \glossterm{mystic sphere} has a set of \glossterm{spells} associated with it.
        You automatically learn all \glossterm{cantrips} from the mystic sphere you have access to.

        You require both \glossterm{verbal components} and \glossterm{somatic components} to cast spells from your chosen sphere.
        For details about mystic spheres and casting spells, see \pcref{Spell and Ritual Mechanics}.

        \ff[3]{Spell Rank} You become a rank 1 spellcaster in your chosen \glossterm{mystic sphere}.
        You learn one spell from that mystic sphere.
        In addition, you can spend \glossterm{insight points} to learn one additional arcane spell per \glossterm{insight point}.
        Unless otherwise noted in a spell's description, casting a spell requires a \glossterm{standard action}.

        When you gain access to a spell rank,
            you can exchange any number of spells you know for other spells,
            including spells of the higher rank.

        \ff[6]{Spell Rank} You become a rank 2 spellcaster in your chosen \glossterm{mystic sphere}.
        This gives you access to spells that require a minimum rank of 2.

        \ff[9]{Spell Rank} You become a rank 3 spellcaster in your chosen \glossterm{mystic sphere}.
        This gives you access to spells that require a minimum rank of 3 and can improve the effectiveness of your existing spells.

        \ff[12]{Spell Rank} You become a rank 4 spellcaster in your chosen \glossterm{mystic sphere}.
        This gives you access to spells that require a minimum rank of 4 and can improve the effectiveness of your existing spells.

        \ff[12]{Spell Knowledge} You learn one spell from your \glossterm{mystic sphere}.

        \ff[15]{Spell Rank} You become a rank 5 spellcaster in your chosen \glossterm{mystic sphere}.
        This gives you access to spells that require a minimum rank of 5 and can improve the effectiveness of your existing spells.

        \ff[18]{Spell Rank} You become a rank 6 spellcaster in your chosen \glossterm{mystic sphere}.
        This gives you access to spells that require a minimum rank of 6 and can improve the effectiveness of your existing spells.

        \ff[21]{Spell Rank} You become a rank 7 spellcaster in your chosen \glossterm{mystic sphere}.
        This gives you access to spells that require a minimum rank of 7 and can improve the effectiveness of your existing spells.

        \ff[21]{Spell Knowledge} You learn one spell from your \glossterm{mystic sphere}.
    \end{feat}

    \begin{feat}{Sphere Focus: Aeromancy}{Casting, Magical}
        \featpre Access to the \sphere{Aeromancy} \glossterm{mystic sphere}.

        \ff[1]{Spell} You learn a spell from the \sphere{Aeromancy} \glossterm{mystic sphere}.
        When you gain access to a new spell level, you can change which spell you know from that \glossterm{mystic sphere}.

        \ff[3]{Favorable Winds} You gain a \plus1 bonus to \glossterm{accuracy} with ranged \glossterm{strikes}.
        In addition, you gain a \plus1 bonus to all defenses against ranged \glossterm{strikes}.

        \ff[6]{Personal Updraft} You gain a \glossterm{glide speed} equal to the \glossterm{base speed} for your size (see \pcref{Gliding}).
        If you already have a \glossterm{glide speed}, you can increase or decrease your glide speed whenever you glide by up to 20 feet (to a minimum of 10 feet).

        \ff[9]{Personal Aeromancy} You gain an additional \glossterm{attunement point}.
        You can only use this attunement point to \glossterm{attune} to spells and rituals from the \sphere{Aeromancy} \glossterm{mystic sphere}.

        \ff[12]{Greater Personal Updraft} You gain a \glossterm{fly speed} equal to the \glossterm{base speed} for your size with a maximum height of 30 feet (see \pcref{Flying}).
        At the start of each phase, you can increase your \glossterm{fatigue level} by one to ignore this height limit until the end of the round.

        \ff[15]{Spell} You learn a spell from the \sphere{Aeromancy} \glossterm{mystic sphere}.
        When you gain access to a new spell level, you can change which spell you know from that mystic sphere.

        \ff[18]{Greater Favorable Winds} The bonuses from your \textit{favorable winds} ability increase to \plus2.

        \ff[21]{Supreme Personal Updraft} Your \glossterm{maneuverability} with the fly speed from your \textit{angelic flight} ability increases to perfect (see \pcref{Flying Maneuverability}).
    \end{feat}

    \begin{feat}{Sphere Focus: Aquamancy}{Casting, Magical}
        \featpre Access to the \sphere{Aquamancy} \glossterm{mystic sphere}.

        \ff[1]{Spell} You learn a spell from the \sphere{Aquamancy} \glossterm{mystic sphere}.
        When you gain access to a new spell level, you can change which spell you know from that \glossterm{mystic sphere}.

        \ff[3]{Swim Familiarity} You gain a \plus2 bonus to the Swim skill.
        In addition, you reduce your penalties for fighting underwater by 2 (see \pcref{Underwater Combat}).

        \ff[6]{Slippery Escapist} You gain a \plus2 bonus to the Flexibility skill.
        In addition, you gain a \plus2 bonus to defenses against the \textit{grapple} ability, and a \plus2 bonus to the \ability{escape grapple} ability (see \pcref{Grapple}).

        \ff[9]{Personal Aquamancy} You gain an additional \glossterm{attunement point}.
        You can only use this attunement point to \glossterm{attune} to spells and rituals from the \sphere{Aquamancy} \glossterm{mystic sphere}.

        \ff[12]{Greater Swim Familiarity} The Swim bonus from your \textit{swim familiarity} ability increases to \plus4.
        In addition, the penalty reduction from that ability increases to \minus4.

        \ff[15]{Spell} You learn a spell from the \sphere{Aquamancy} \glossterm{mystic sphere}.
        When you gain access to a new spell level, you can change which spell you know from that mystic sphere.

        \ff[18]{Greater Slippery Escapist} The bonuses from your \textit{slippery escapist} ability increase to \plus4.

        \ff[21]{Create Flood} When you use the \spell{create water} cantrip, you can create up to ten gallons of water per \glossterm{power}.
    \end{feat}

    \begin{feat}{Sphere Focus: Astromancy}{Casting, Magical}
        \featpre Access to the \sphere{Astromancy} \glossterm{mystic sphere}.

        \ff[1]{Spell} You learn a spell from the \sphere{Astromancy} \glossterm{mystic sphere}.
        When you gain access to new spell ranks, you can change which spell you know from that \glossterm{mystic sphere}.

        \ff[3]{Efficient Transit} You learn how to transport creatures and objects more smoothly between planes.
        The \glossterm{difficulty rating} to hear noise caused by creatures and objects you \glossterm{teleport} increases by 10 (see \pcref{Teleportation Noise}).
        In addition, whenever you fail to teleport a creature or object due to specifying an invalid destination, you may immediately specify a different destination for that ability.
        If that second destination is also invalid, the ability fails normally.

        \ff[6]{Astral Spell Transit} You double your range with abilities from the \sphere{Astromancy} \glossterm{mystic sphere}.

        \ff[9]{Personal Astromancy} You gain an additional \glossterm{attunement point}.
        You can only use this attunement point to \glossterm{attune} to spells and rituals from the \sphere{Astromancy} \glossterm{mystic sphere}.

        \ff[12]{Greater Astral Spell Transit} The range increase from your \textit{astral spell transit} ability increases to be triple your normal range.

        \ff[15]{Spell} You learn a spell from the \sphere{Astromancy} \glossterm{mystic sphere}.
        When you gain access to a new spell level, you can change which spell you know from that mystic sphere.

        \ff[18]{Greater Efficient Transit} The difficulty rating increase from your \textit{efficient transit} effect increases to 20.
        In addition, whenever you fail to teleport a creature or object due to specifying an invalid destination, you may automatically change the target's destination to the closest valid location to your intended destination.
        This makes it much easier to gain access to areas whose precise location you do not know.

        \ff[21]{Supreme Astral Spell Transit} The range increase from your \textit{astral spell transit} ability increases to be quadruple your normal range.
        In addition, you double your range with all \glossterm{magical} abilities that are not from the Astromancy mystic sphere.
    \end{feat}

    \begin{feat}{Sphere Focus: Barrier}{Casting, Magical}
        \featpre Access to the \sphere{Barrier} \glossterm{mystic sphere}.

        \ff[1]{Spell} You learn a spell from the \sphere{Barrier} \glossterm{mystic sphere}.
        When you gain access to new spell ranks, you can change which spell you know from that \glossterm{mystic sphere}.

        \ff[3]{Innate Shield} You gain a \plus1 bonus to Armor defense.

        \ff[6]{Hardened Barriers} Objects you create with the \sphere{Barrier} \glossterm{mystic sphere} gain a bonus equal to your \glossterm{power} to their \glossterm{hit points}.
        For objects with multiple separate hit point values, such as walls, this bonus applies independently to each section.

        \ff[9]{Personal Barrier} You gain an additional \glossterm{attunement point}.
        You can only use this attunement point to \glossterm{attune} to spells and rituals from the \sphere{Barrier} \glossterm{mystic sphere}.

        \ff[12]{Greater Innate Shield} The bonus from your \textit{innate shield} ability increases to \plus2.

        \ff[15]{Spell} You learn a spell from the \sphere{Barrier} \glossterm{mystic sphere}.
        When you gain access to a new spell level, you can change which spell you know from that mystic sphere.

        \ff[18]{Greater Hardened Barriers} The bonus from your \textit{hardened barriers} ability increases to twice your \glossterm{power}.

        \ff[21]{Supreme Innate Shield} The bonus from your \textit{innate shield} ability increases to \plus3.
    \end{feat}

    \begin{feat}{Sphere Focus: Bless}{Casting, Magical}
        \featpre Access to the \sphere{Bless} \glossterm{mystic sphere}.

        \ff[1]{Spell} You learn a spell from the \sphere{Bless} \glossterm{mystic sphere}.
        When you gain access to a new spell level, you can change which spell you know from that \glossterm{mystic sphere}.

        \ff[3]{Inspiring Blessing} Each creature that is \glossterm{attuned} to a spell you cast from the \sphere{Bless} \glossterm{mystic sphere} gains a bonus equal to half your \glossterm{power} to its maximum \glossterm{hit points} (minimum 1).

        \ff[6]{Simple Blessing} Spells you cast from the Bless mystic sphere do not have the \abilitytag{Focus} tag (see \pcref{Focus}).

        \ff[9]{Personal Blessing} You gain an additional \glossterm{attunement point}.
        You can only use this attunement point to attune to spells you cast from the \sphere{Bless} mystic sphere.

        \ff[12]{Greater Inspiring Blessing} The number of hit points granted by your \textit{inspiring blessing} ability increases to be equal to your power.

        \ff[15]{Spell} You learn a spell from the \sphere{Bless} \glossterm{mystic sphere}.
        When you gain access to a new spell level, you can change which spell you know from that mystic sphere.

        \ff[18]{Greater Personal Blessing} The number of additional attunement points granted by your \textit{personal blessing} ability increases to two.

        \ff[21]{Supreme Inspiring Blessing} The number of hit points granted by your \textit{inspiring blessing} ability increases to be equal to twice your power.
    \end{feat}

    \begin{feat}{Sphere Focus: Channel Divinity}{Casting, Magical}
        \featpre Access to the \sphere{Channel Divinity} \glossterm{mystic sphere}.

        \ff[1]{Spell} You learn a spell from the \sphere{channel divinity} \glossterm{mystic sphere}.
        When you gain access to a new spell level, you can change which spell you know from that \glossterm{mystic sphere}.

        \ff[3]{Divine Intervention} You gain a \plus3 bonus to any roll that you use the \textit{desperate exertion} ability on (see \pcref{Desperate Exertion}).
        This bonus stacks with the normal \plus2 bonus provided by that ability.

        \ff[6]{Divine Channel} Once per \glossterm{long rest}, you can use the \textit{desperate exertion} ability without increasing your \glossterm{fatigue level}.

        \ff[9]{Personal Channeling} You gain an additional \glossterm{attunement point}.
        You can only use this attunement point to \glossterm{attune} to spells and rituals from the \sphere{Channel Divinity} \glossterm{mystic sphere}.

        \ff[12]{Greater Divine Intervention} The bonus from your \textit{divine intervention} ability increases to \plus5.

        \ff[15]{Spell} You learn a spell from the \sphere{Channel Divinity} \glossterm{mystic sphere}.
        When you gain access to a new spell level, you can change which spell you know from that mystic sphere.

        \ff[18]{Greater Personal Channeling} The number of attunement points you gain from your \textit{personal channeling} ability increases to 2.

        \ff[21]{Greater Divine Channel} You can use your \textit{divine channel} ability once per \glossterm{short rest} instead of once per long rest.
    \end{feat}

    \begin{feat}{Sphere Focus: Chronomancy}{Casting, Magical}
        \featpre Access to the \sphere{Chronomancy} \glossterm{mystic sphere}.

        \ff[1]{Spell} You learn a spell from the \sphere{Chronomancy} \glossterm{mystic sphere}.
        When you gain access to new spell ranks, you can change which spell you know from that \glossterm{mystic sphere}.

        \ff[3]{Accelerated Movement} You gain a \plus5 foot bonus to your speed with all of your \glossterm{movement modes}.

        \ff[6]{Accelerated Mind} You can perform primarily mental tasks more quickly as normal.
        Actions that would normally take a \glossterm{standard action} instead take a \glossterm{minor action}.
        Long-term activities can be done twice as quickly as normal.
        This includes reading books, searching areas, identifying magical effects with the Spellsense skill, and other similar activities.
        It does not affect spellcasting, performing rituals, or other similar magical abilities.

        \ff[9]{Personal Aquamancy} You gain an additional \glossterm{attunement point}.
        You can only use this attunement point to \glossterm{attune} to spells and rituals from the \sphere{Aquamancy} \glossterm{mystic sphere}.

        \ff[12]{Greater Accelerated Movement} The speed bonus from your \textit{accelerated movement} ability increases to \plus10 feet.

        \ff[15]{Spell} You learn a spell from the \sphere{Chronomancy} \glossterm{mystic sphere}.
        When you gain access to a new spell level, you can change which spell you know from that mystic sphere.

        \ff[18]{Greater Accelerated Mind} You can perform \glossterm{mundane} mental tasks that would normally take \glossterm{standard actions} as \glossterm{minor actions}.
        In addition, the speed increase for long-term tasks from your \textit{accelerated mind} ability increases to five times normal speed.

        \ff[21]{Supreme Accelerated Movement} The speed bonus from your \textit{accelerated movement} ability increases to \plus15 feet.
    \end{feat}

    \begin{feat}{Sphere Focus: Cryomancy}{Casting, Magical}
        \featpre Access to the \sphere{Cryomancy} \glossterm{mystic sphere}.

        \ff[1]{Spell} You learn a spell from the \sphere{Cryomancy} \glossterm{mystic sphere}.
        When you gain access to a new spell level, you can change which spell you know from that \glossterm{mystic sphere}.

        \ff[3]{Cold Tolerance} You are \glossterm{impervious} to cold damage.

        \ff[3]{Frozen Blood} You are immune to \glossterm{diseases}.

        \ff[6]{Lingering Chill} Whenever you make a creature lose \glossterm{hit points} with cold damage, that creature is \glossterm{briefly} \slowed.
        That creature then becomes immune to this effect until it takes a \glossterm{short rest}.

        \ff[9]{Personal Cryomancy} You gain an additional \glossterm{attunement point}.
        You can only use this attunement point to \glossterm{attune} to spells and rituals from the \sphere{Cryomancy} \glossterm{mystic sphere}.

        \ff[12]{Greater Frozen Blood} You are immune to \glossterm{poisons}.

        \ff[15]{Spell} You learn a spell from the \sphere{Cryomancy} \glossterm{mystic sphere}.
        When you gain access to a new spell level, you can change which spell you know from that mystic sphere.

        \ff[18]{Greater Lingering Chill} The \slowed effect from your \textit{lingering chill} ability becomes a \glossterm{condition} that lasts until it is removed.

        \ff[21]{Cold Immunity} You are \glossterm{immune} to cold damage.

        \ff[21]{Supreme Frozen Blood} You are immune to being \glossterm{slowed} and \glossterm{decelerated}.
    \end{feat}

    \begin{feat}{Sphere Focus: Electromancy}{Casting, Magical}
        \featpre Access to the \sphere{Electromancy} \glossterm{mystic sphere}.

        \ff[1]{Spell} You learn a spell from the \sphere{Electromancy} \glossterm{mystic sphere}.
        When you gain access to a new spell level, you can change which spell you know from that \glossterm{mystic sphere}.

        \ff[3]{Electricity Tolerance} You are \glossterm{impervious} to electricity damage.

        \ff[3]{Energetic Rush} You gain a \plus5 foot bonus to your land speed.

        \ff[6]{Lingering Shock} Whenever you make a creature lose \glossterm{hit points} with electricity damage, that creature is \glossterm{briefly} \dazed.
        That creature then becomes immune to this effect until it takes a \glossterm{short rest}.

        \ff[9]{Personal Electromancy} You gain an additional \glossterm{attunement point}.
        You can only use this attunement point to \glossterm{attune} to spells and rituals from the \sphere{Electromancy} \glossterm{mystic sphere}.

        \ff[12]{Greater Energetic Rush} The bonus from your \textit{energetic rush} ability increases to \plus10 feet.

        \ff[15]{Spell} You learn a spell from the \sphere{Electromancy} \glossterm{mystic sphere}.
        When you gain access to a new spell level, you can change which spell you know from that mystic sphere.

        \ff[18]{Greater Lingering Shock} The \dazed effect from your \textit{lingering shock} ability becomes a \glossterm{condition} that lasts until it is removed.

        \ff[21]{Electricity Immunity} You are \glossterm{immune} to electricity damage.

        \ff[21]{Supreme Energetic Rush} The bonus from your \textit{energetic rush} ability increases to \plus15 feet.
    \end{feat}

    \begin{feat}{Sphere Focus: Enchantment}{Casting, Magical}
        \featpre Access to the \sphere{Enchantment} \glossterm{mystic sphere}.

        \ff[1]{Spell} You learn a spell from the \sphere{Enchantment} \glossterm{mystic sphere}.
        When you gain access to new spell ranks, you can change which spell you know from that \glossterm{mystic sphere}.

        \ff[3]{Subtle Influence} You gain a \plus2 bonus to \glossterm{accuracy} with spells from the Enchantment mystic sphere against \unaware creatures.
        In addition, the \glossterm{difficulty rating} to observe your \abilitytag{Emotion} abilities with Spellsense, and to observe their effects with the Social Insight skill, increases by 10.

        \ff[6]{Mind Fragments} When you use \abilitytag{Compulsion} and \abilitytag{Emotion} abilities, you can affect creatures that are immune to those abilities due to not having a mind.
        You take a \minus5 accuracy penalty on attacks against such creatures.
        This does not allow you to affect creatures who are immune to those abilities for other reasons.

        \ff[9]{Personal Enchantment} You gain an additional \glossterm{attunement point}.
        You can only use this attunement point to \glossterm{attune} to spells and rituals from the \sphere{Enchantment} \glossterm{mystic sphere}.

        \ff[12]{Greater Subtle Influence} The accuracy bonus from your \textit{subtle influence} ability increases to \plus3, and it also applies against \partiallyunaware creatures.
        In addition, the \glossterm{difficulty rating} increase from that ability increases to \plus20.

        \ff[15]{Spell} You learn a spell from the \sphere{Enchantment} \glossterm{mystic sphere}.
        When you gain access to a new spell level, you can change which spell you know from that mystic sphere.

        \ff[18]{Greater Mind Fragments} The accuracy penalty from your \textit{mind fragments} ability is removed.

        \ff[21]{Supreme Subtle Influence} The accuracy bonus from your \textit{subtle influence} ability increases to \plus4, and it also applies against creatures that were \unaware or \partiallyunaware during the previous round.
        In addition, the \glossterm{difficulty rating} increase from that ability increases to \plus30.
    \end{feat}

    \begin{feat}{Sphere Focus: Fabrication}{Casting, Magical}
        \featpre Access to the \sphere{Fabrication} \glossterm{mystic sphere}.

        \ff[1]{Spell} You learn a spell from the \sphere{Fabrication} \glossterm{mystic sphere}.
        When you gain access to a new spell level, you can change which spell you know from that \glossterm{mystic sphere}.

        \ff[3]{Crafting Familiarity} You gain a \plus2 bonus to all Craft skills.
        In addition, you gain a \plus1 bonus to \glossterm{accuracy} with \glossterm{strikes} using weapons you created with spells from the Fabrication mystic sphere.

        \ff[6]{Greater Fabricate Trinket} The maximum size of the trinket you can create with your \textit{fabricate trinket} cantrip increases by one size category.
        In addition, when you cast that spell, you can treat it as if it had the \abilitytag{Sustain} (minor) tag instead of the \abilitytag{Attune} (self) tag.

        \ff[9]{Personal Fabrication} You gain an additional \glossterm{attunement point}.
        You can only use this attunement point to \glossterm{attune} to spells and rituals from the \sphere{Fabrication} \glossterm{mystic sphere}.

        \ff[12]{Greater Forge} You learn the \spell{forge} spell from the Fabrication mystic sphere.
        In addition, the armor you create with that spell can be made of any special material other than dragonscale or dragonhide as long as the total item level of the armor does not exceed your level.

        \ff[15]{Spell} You learn a spell from the \sphere{Fabrication} \glossterm{mystic sphere}.
        When you gain access to a new spell level, you can change which spell you know from that mystic sphere.

        \ff[18]{Greater Crafting Familiarity} The Craft bonus from your \textit{crafting familiarity} ability increases to \plus4.
        In addition, the accuracy bonus increases to \plus2.

        \ff[21]{Supreme Fabricate Trinket} The size increase from your \textit{greater fabricate trinket} ability increases to two size categories.
        In addition, when you cast the \textit{fabricate trinket} ability, you can treat it as if it had the \abilitytag{Sustain} (free) tag instead of the Attune (self) tag.
    \end{feat}

    \begin{feat}{Sphere Focus: Photomancy}{Casting, Magical}
        \featpre Access to the \sphere{Photomancy} \glossterm{mystic sphere}.

        \ff[1]{Spell} You learn a spell from the \sphere{Photomancy} \glossterm{mystic sphere}.
        When you gain access to a new spell level, you can change which spell you know from that \glossterm{mystic sphere}.

        \ff[3]{Augmented Vision} You gain a \plus2 bonus to the Awareness skill.
        In addition, you gain the \glossterm{low-light vision} ability, allowing you to treat sources of light as if they had double their normal illumination range.
        If you already have low-light vision, you double its benefit, allowing you to treat sources of light as if they had four times their normal illumination range.

        \ff[6]{Certain Sight} You are immune to being \dazzled and \blinded.

        \ff[9]{Personal Photomancy} You gain an additional \glossterm{attunement point}.
        You can only use this attunement point to \glossterm{attune} to spells and rituals from the \sphere{Photomancy} \glossterm{mystic sphere}.

        \ff[12]{Greater Augmented Vision} The bonus from your \textit{augmented vision} ability increases to \plus4.
        In addition, you can see through solid objects up to one inch thick.
        You can perceive the existence of obstacles thinner than that, but they do not inhibit your sight.
        This does not grant you \glossterm{line of effect} to anything you see in this way, since the obstacle still exists.

        \ff[15]{Spell} You learn a spell from the \sphere{Photomancy} \glossterm{mystic sphere}.
        When you gain access to a new spell level, you can change which spell you know from that mystic sphere.

        \ff[18]{Greater Certain Sight} You can see through all effects created by the \sphere{Photomancy} and \sphere{Umbramancy} mystic spheres.
        You can see what those effects look like if you focus your eyes on them, but you can also see through them, so they do not block light or \glossterm{line of sight} for you.

        \ff[21]{Supreme Augmented Vision} The bonus from your \textit{augmented vision} ability increases to \plus6.
        In addition, the maximum thickness that you can see through with your \textit{augmented vision} ability increases to one foot.
    \end{feat}

    \begin{feat}{Sphere Focus: Polymorph}{Casting, Magical}
        \featpre Access to the \sphere{Polymorph} \glossterm{mystic sphere}.

        \ff[1]{Spell} You learn a spell from the \sphere{Polymorph} \glossterm{mystic sphere}.
        When you gain access to new spell ranks, you can change which spell you know from that \glossterm{mystic sphere}.

        \ff[3]{Reshaper} As a standard action, you can use the \textit{alter self} ability.
        In addition, when you use the \spell{alter object} cantrip, you can use your \glossterm{power} in place of your Craft skill.
        \begin{freeability}{Alter Self}[\glossterm{Shaping}]
            Make a Disguise check to alter your appearance (see \pcref{Disguise Creature}), except that you can use your \glossterm{power} in place of your Disguise skill.
            You can only alter your physical body, not your clothes or equipment.

            This ability lasts until you use it again.
        \end{freeability}

        \ff[6]{Malleable Flesh} You gain a \plus4 bonus to defenses when determining whether a \glossterm{strike} gets a \glossterm{critical hit} against you instead of a normal hit.

        \ff[9]{Personal Polymorph} You gain an additional \glossterm{attunement point}.
        You can only use this attunement point to \glossterm{attune} to spells and rituals from the \sphere{Polymorph} \glossterm{mystic sphere}.

        \ff[12]{Greater Reshaper} When you use the \spell{alter object} cantrip, you can accomplish work that would take up to an hour with a normal Craft check.
        In addition, you can use the \textit{alter poison} ability as a standard action.
        \begin{freeability}{Alter Poison}[\glossterm{Shaping}, \abilitytag{Sustain} (minor)]
            Make an attack vs. Fortitude with a \plus4 \glossterm{accuracy} bonus against a creature within \rngshort range.
            \hit Any poison in the target's system is neutralized.
            It stops suffering any additional effects from poisons in its system.
            As long as the effect lasts, it is immune to all poisons.
            In addition, the target's \glossterm{mundane} poisons, including natural attacks that inflict poison, have no effect.

            \rankline
            You gain a \plus1 bonus to \glossterm{accuracy} with the attack at 15th level and every 3 levels thereafter.
        \end{freeability}

        \ff[15]{Spell} You learn a spell from the \sphere{Polymorph} \glossterm{mystic sphere}.
        When you gain access to a new spell level, you can change which spell you know from that mystic sphere.

        \ff[18]{Greater Malleable Flesh} The bonus from your \textit{malleable flesh} ability increases to \plus8.

        \ff[21]{Supreme Reshaper} When you use the \spell{alter object} cantrip, you can accomplish work that would take up to 8 hours with a normal Craft check.
    \end{feat}

    \begin{feat}{Sphere Focus: Pyromancy}{Casting, Magical}
        \featpre Access to the \sphere{Pyromancy} \glossterm{mystic sphere}.

        \ff[1]{Spell} You learn a spell from the \sphere{Pyromancy} \glossterm{mystic sphere}.
        When you gain access to new spell ranks, you can change which spell you know from that \glossterm{mystic sphere}.

        \ff[3]{Fire Tolerance} You are \glossterm{impervious} to fire damage.

        \ff[3]{Friendly Fire} Whenever you deal fire damage to your \glossterm{allies}, you deal half damage.

        \ff[6]{Lingering Flame} Whenever you make a creature lose \glossterm{hit points} with fire damage, that creature takes 1d10 fire damage at the end of the next round.
        That creature then becomes immune to this effect until it takes a \glossterm{short rest}.
        This damage increases by \plus1d at 9th level and every 3 levels thereafter.

        \ff[9]{Personal Pyromancy} You gain an additional \glossterm{attunement point}.
        You can only use this attunement point to \glossterm{attune} to spells and rituals from the \sphere{Pyromancy} \glossterm{mystic sphere}.

        \ff[12]{Greater Friendly Fire} Your \glossterm{allies} treat fire damage from your abilities as \glossterm{environmental damage}.

        \ff[15]{Spell} You learn a spell from the \sphere{Pyromancy} \glossterm{mystic sphere}.
        When you gain access to a new spell level, you can change which spell you know from that mystic sphere.

        \ff[18]{Greater Lingering Flame} The effect from your \textit{lingering chill} ability becomes a \glossterm{condition} that lasts until it is removed.
        It deals its damage at the end of each subsequent round until it is removed.
        This effect can be removed if the target makes a \glossterm{difficulty rating} 15 Dexterity check as a \glossterm{move action} to put out the flames.
        Dropping \prone as part of this action gives a \plus5 bonus to this check.

        \ff[21]{Fire Immunity} You are \glossterm{immune} to fire damage.

        \ff[21]{Supreme Friendly Fire} Your \glossterm{allies} are immune to fire damage from your abilities.
    \end{feat}

    \begin{feat}{Sphere Focus: Revelation}{Casting, Magical}
        \featpre Access to the \sphere{Revelation} \glossterm{mystic sphere}.

        \ff[1]{Spell} You learn a spell from the \sphere{Revelation} \glossterm{mystic sphere}.
        When you gain access to new spell ranks, you can change which spell you know from that \glossterm{mystic sphere}.

        \ff[3]{Instinctive Truth} You gain a \plus2 bonus to all Knowledge skills and the Social Insight skill.

        \ff[6]{Truesight} You gain the \glossterm{truesight} ability with a 60 foot range.
        If you already have the \glossterm{truesight} ability, you increase its range by 60 feet.

        \ff[9]{Personal Revelation} You gain an additional \glossterm{attunement point}.
        You can only use this attunement point to \glossterm{attune} to spells and rituals from the \sphere{Revelation} \glossterm{mystic sphere}.

        \ff[12]{Greater Instinctive Truth} The bonus from your \textit{instinctive truth} ability increases to \plus4.

        \ff[15]{Spell} You learn a spell from the \sphere{Revelation} \glossterm{mystic sphere}.
        When you gain access to a new spell level, you can change which spell you know from that mystic sphere.

        \ff[18]{Greater Truesight} The range of your \glossterm{truesight} ability increases by 120 feet.
        In addition, you can \glossterm{suppress} any \abilitytag{Sensation} ability you observe within the range of your truesight ability as a \glossterm{free action}.

        \ff[21]{Supreme Instinctive Truth} The bonus from your \textit{instinctive truth} ability increases to \plus6.
    \end{feat}

    \begin{feat}{Sphere Focus: Summoning}{Casting, Magical}
        \featpre Access to the \sphere{Summoning} \glossterm{mystic sphere}.

        \ff[1]{Spell} You learn a spell from the \sphere{Summoning} \glossterm{mystic sphere}.
        When you gain access to a new spell level, you can change which spell you know from that \glossterm{mystic sphere}.

        \ff[3]{Fortified Summons} Creatures you create with the \sphere{Summoning} \glossterm{mystic sphere} have half their normal \glossterm{hit points}.
        They gain a bonus to their \glossterm{damage resistance} equal to the hit points lost this way.

        \ff[6]{Resummon} You can use the \textit{resummon} ability as a \glossterm{minor action}.
        \begin{freeability}{Resummon}
            Choose one creature or object that you summoned with a currently active ability from the \sphere{Summoning} \glossterm{mystic sphere} with the \abilitytag{Attune} or \abilitytag{Sustain} tags.
            You teleport the target into an unoccupied space on stable ground within \rngmed range of you.

            \rankline
            \featlevel{12} The range increases to \rngdist.
            \featlevel{18} You can choose two creatures or objects to teleport in this way instead of only one.
                Each chosen creature or object can be teleported to a different location within range.
        \end{freeability}

        \ff[9]{Personal Summoning} You gain an additional \glossterm{attunement point}.
        You can only use this attunement point to \glossterm{attune} to spells and rituals from the \sphere{Summoning} \glossterm{mystic sphere}.

        \ff[12]{Greater Fortified Summons} Creatures you create have three quarters of their normal hit points instead of half.

        \ff[15]{Spell} You learn a spell from the \sphere{Summoning} \glossterm{mystic sphere}.
        When you gain access to a new spell level, you can change which spell you know from that mystic sphere.

        \ff[18]{Augmented Summons} Creatures you create with abilities from the \sphere{Summoning} spell gain an \glossterm{attunement point}.

        \ff[21]{Supreme Fortified Summons} Creatures you create have their normal hit points in addition to the damage reduction from your \textit{fortified summons} ability.
    \end{feat}

    \begin{feat}{Sphere Focus: Telekinesis}{Casting, Magical}
        \featpre Access to the \sphere{Telekinesis} \glossterm{mystic sphere}.

        \ff[1]{Spell} You learn a spell from the \sphere{Telekinesis} \glossterm{mystic sphere}.
        When you gain access to new spell ranks, you can change which spell you know from that \glossterm{mystic sphere}.

        \ff[3]{Greater Distant Hand} You can use the \spell{distant hand} \glossterm{cantrip} as a \glossterm{minor action}, and you can \glossterm{sustain} it as a \glossterm{minor action}.

        \ff[3]{Telekinetic Strike} You can use the \textit{telekinetic strike} ability as a standard action.
        \begin{freeability}{Telekinetic Strike}[Magical]
            Make a \glossterm{strike} with a weapon you are controlling using the \spell{distant hand} cantrip.
            Because this is a \glossterm{magical} ability, you use your \glossterm{magical} \glossterm{power} to determine your damage instead of your \glossterm{mundane} power.

            \rankline
            \featlevel{9} You gain a \plus1d damage bonus with the strike.
            \featlevel{15} The damage bonus increases to \plus2d.
        \end{freeability}

        \ff[6]{Partial Levitation} You gain a \plus4 bonus to the Jump skill.
        In addition, as a \glossterm{free action}, you can slow your fall while falling.
        If you do, you fall at a rate of 50 feet per round, preventing you from taking falling damage when you hit the ground.

        \ff[9]{Personal Telekinesis} You gain an additional \glossterm{attunement point}.
        You can only use this attunement point to \glossterm{attune} to spells and rituals from the \sphere{Telekinesis} \glossterm{mystic sphere}.

        \ff[12]{Levitation} You gain a \glossterm{fly speed} equal to the \glossterm{base speed} for your size with a maximum height of 30 feet (see \pcref{Flying}).
        At the start of each phase, you can increase your \glossterm{fatigue level} by one to ignore this height limit until the end of the round.

        \ff[15]{Spell} You learn a spell from the \sphere{Telekinesis} \glossterm{mystic sphere}.
        When you gain access to a new spell level, you can change which spell you know from that mystic sphere.

        \ff[18]{Supreme Distant Hand} You can use the \spell{distant hand} \glossterm{cantrip} as a \glossterm{free action}, and you can \glossterm{sustain} it as a \glossterm{free action}.

        \ff[21]{Greater Levitation} Your \glossterm{maneuverability} with the fly speed from your \textit{levitation} ability increases to perfect (see \pcref{Flying Maneuverability}).
    \end{feat}

    \begin{feat}{Sphere Focus: Terramancy}{Casting, Magical}
        \featpre Access to the \sphere{Terramancy} \glossterm{mystic sphere}.

        \ff[1]{Spell} You learn a spell from the \sphere{Terramancy} \glossterm{mystic sphere}.
        When you gain access to a new spell level, you can change which spell you know from that \glossterm{mystic sphere}.

        \ff[3]{Heart of Stone} You gain a \plus2 bonus to Fortitude defense.

        \ff[6]{Earthen Alloys} You may treat iron, steel, and worked stone as if they earth for the purpose of spells from the \sphere{Terramancy} \glossterm{mystic sphere}.

        \ff[9]{Personal Terramancy} You gain an additional \glossterm{attunement point}.
        You can only use this attunement point to \glossterm{attune} to spells and rituals from the \sphere{Terramancy} \glossterm{mystic sphere}.

        \ff[12]{Body of Stone} You gain a \plus1 bonus to Armor defense.

        \ff[15]{Spell} You learn a spell from the \sphere{Terramancy} \glossterm{mystic sphere}.
        When you gain access to a new spell level, you can change which spell you know from that mystic sphere.

        \ff[18]{Greater Earthen Alloys} You may treat sand, glass, and metal of any kind as if it were earth for the purpose of spells from the \sphere{Terramancy} \glossterm{mystic sphere}.

        \ff[21]{Soul of Stone} The bonus from your \textit{body of stone} ability increases to \plus2.
        In addition, the bonus from your \textit{heart of stone} ability increases to \plus3.
    \end{feat}

    \begin{feat}{Sphere Focus: Thaumaturgy}{Casting, Magical}
        \featpre Access to the \sphere{Thaumaturgy} \glossterm{mystic sphere}.

        \ff[1]{Spell} You learn a spell from the \sphere{Thaumaturgy} \glossterm{mystic sphere}.
        When you gain access to new spell ranks, you can change which spell you know from that \glossterm{mystic sphere}.

        \ff[3]{Mystic Power} You gain a \plus2 bonus to \glossterm{magical} \glossterm{power}.

        \ff[6]{Counterspell} You can use the \textit{counterspell} ability as a standard action.
        \begin{freeability}{Counterspell}[\abilitytag{Swift}]
            Choose a creature within \rngmed range of you.
            If the target is casting a spell or begins casting a spell this round, you can attempt to counter the spell.
            When you do, if your maximum spell level is at least as high as the target's maximum spell level, their spell has no effect when it resolves.
            Otherwise, make a contested \glossterm{power} check against the target, using your power with this ability against the target's power with the spell it is casting.
            If you win, the target's spell has no effect when it resolves.

            If the subject is capable of casting multiple spells each round, you can only counter the first spell it casts.

            \rankline
            \featlevel{12} You may target an additional creature within range.
            \featlevel{18} You may target an additional creature within range.
        \end{freeability}

        \ff[9]{Personal Thaumaturgy} You gain an additional \glossterm{attunement point}.
        You can only use this attunement point to \glossterm{attune} to spells and rituals from the \sphere{Thaumaturgy} \glossterm{mystic sphere}.

        \ff[12]{Greater Mystic Power} The bonus from your \textit{mystic power} ability increases to \plus4.

        \ff[15]{Spell} You learn a spell from the \sphere{Thaumaturgy} \glossterm{mystic sphere}.
        When you gain access to a new spell level, you can change which spell you know from that mystic sphere.

        \ff[18]{Greater Mystic Power} The bonus from your \textit{mystic power} ability increases to \plus8.

        \ff[21]{Greater Counterspell} You can use your \textit{counterspell} ability as a \glossterm{minor action}.
        When you do, you \glossterm{briefly} cannot use that ability again.
        If you actually counter a spell with that ability after using it as a minor action, you increase your \glossterm{fatigue level} by one.
    \end{feat}

    \begin{feat}{Sphere Focus: Toxicology}{Casting, Magical}
        \featpre Access to the \sphere{Toxicology} \glossterm{mystic sphere}.

        \ff[1]{Spell} You learn a spell from the \sphere{toxicology} \glossterm{mystic sphere}.
        When you gain access to new spell ranks, you can change which spell you know from that \glossterm{mystic sphere}.

        \ff[3]{Cleanse Toxins} You can use the \textit{cleanse toxins} ability as a standard action.
        \begin{freeability}{Cleanse Toxins}
            \label{Cleanse Toxins}
            \target{Yourself or one \glossterm{ally} within your \glossterm{reach}}
            You remove all \glossterm{poisons} and \glossterm{diseases} affecting the target.
            This cannot remove a poison or disease applied during the current round.

            \rankline
            \featlevel{9} The range increases to \rngmed.
            \featlevel{15} You can use this ability as a \glossterm{minor action}.
        \end{freeability}

        \ff[6]{Innate Poison} 
        When you become poisoned, either by drinking poison or from an enemy's attack, your body naturally repurposes the poison.
        The poison has no effect on you, but your body gains a dose of natural poison.
        Whenever a creature makes you lose \glossterm{hit points} with a \glossterm{melee} strike using a \glossterm{natural weapon}, you make an attack vs. Fortitude against the attacking creature.
        On a hit, it becomes \glossterm{poisoned} by your choice of one of the poisons you store with this ability.
        This expends the dose of that poison.

        Poison that you carry in your body with this ability automatically decays after 24 hours, regardless of the normal duration of the poison.
        You can store up to 3 doses in your body with this ability at a time.

        \ff[9]{Personal Toxicology} You gain an additional \glossterm{attunement point}.
        You can only use this attunement point to \glossterm{attune} to spells and rituals from the \sphere{Toxicology} \glossterm{mystic sphere}.

        \ff[12]{Innate Venom}
        You can also inflict the poison you store with your \textit{innate poison} ability on other creatures with attacks.
        Once per round, when you make a creature lose \glossterm{hit points} with a \glossterm{natural weapon} or a spell from the \sphere{toxicology} mystic sphere, you can cause the creature to become poisoned with your choice of one of the poisons you store.
        This expends the dose of that poison.

        \ff[15]{Spell} You learn a spell from the \sphere{Toxicology} \glossterm{mystic sphere}.
        When you gain access to a new spell level, you can change which spell you know from that mystic sphere.

        \ff[18]{Greater Personal Toxicology} The number of attunement points you gain from your \textit{personal toxicology} ability increases to 2.

        \ff[21]{Greater Innate Poison} You can store up to 10 poison doses with your \textit{innate poison} ability.
    \end{feat}

    \begin{feat}{Sphere Focus: Umbramancy}{Casting, Magical}
        \featpre Access to the \sphere{Umbramancy} \glossterm{mystic sphere}.

        \ff[1]{Spell} You learn a spell from the \sphere{Umbramancy} \glossterm{mystic sphere}.
        When you gain access to new spell ranks, you can change which spell you know from that \glossterm{mystic sphere}.

        \ff[3]{Reflexive Concealment} You gain a \plus2 bonus to the Sleight of Hand and Stealth skills.

        \ff[3]{Greater Suppress Light} You can cast the \spell{suppress light} \glossterm{cantrip} from the Umbramancy mystic sphere as a \glossterm{minor action}.
        If you do, you \glossterm{briefly} cannot cast it as a minor action again.
        In addition, that cantrip no longer has the \abilitytag{Focus} keyword for you.

        \ff[6]{Darkvision} You gain \glossterm{darkvision} with a 60 foot range, allowing you to see in complete darkness clearly.
        If you already have that ability, you increase its range by 60 feet.

        \ff[9]{Personal Umbramancy} You gain an additional \glossterm{attunement point}.
        You can only use this attunement point to \glossterm{attune} to spells and rituals from the \sphere{Umbramancy} \glossterm{mystic sphere}.

        \ff[12]{Greater Reflexive Concealment} The bonuses from your \textit{reflexive concealment} ability increase to \plus4.

        \ff[12]{Supreme Suppress Light} You can both cast and \glossterm{sustain} the \spell{suppress light} cantrip as a \glossterm{free action}.
        If you cast it as a free action, you \glossterm{briefly} cannot cast it as a free action or as a minor action again.

        \ff[15]{Spell} You learn a spell from the \sphere{Umbramancy} \glossterm{mystic sphere}.
        When you gain access to a new spell level, you can change which spell you know from that mystic sphere.

        \ff[18]{Darksight} The range of your \glossterm{darkvision} ability increases by 240 feet.
        In addition, your darkvision is not disabled by being in \glossterm{bright illumination}.

        \ff[21]{Supreme Reflexive Conealment}  The bonuses from your \textit{reflexive concealment} ability increase to \plus6.

        \ff[21]{Supreme Suppress Light} When you cast your \textit{suppress light} cantrip, you can choose to completely block all light in the area instead of dimming it to be \glossterm{shadowy illumination}.
        If you do, the maximum area is reduced to a \medarea radius.
    \end{feat}

    \begin{feat}{Sphere Focus: Verdamancy}{Casting, Magical}
        \featpre Access to the \sphere{Verdamancy} \glossterm{mystic sphere}.

        \ff[1]{Spell} You learn a spell from the \sphere{Verdamancy} \glossterm{mystic sphere}.
        When you gain access to a new spell level, you can change which spell you know from that \glossterm{mystic sphere}.

        \ff[3]{Verdant Allies} Your speed is not reduced when moving in light or heavy \glossterm{undergrowth}.
        In addition, you can ignore \glossterm{cover} and \glossterm{concealment} (but not \glossterm{total cover}) from plants whenever doing so would be beneficial to you, as the plants move out of the way to help you.
        This prevents you from suffering penalties on your attacks, and also prevents creatures from using cover or concealment from plants to hide from you.

        \ff[6]{Residual Undergrowth} Whenever you cast a spell from the \textit{verdamancy} sphere, you may create \glossterm{light undergrowth} in the area of the spell that persists \glossterm{briefly}.
        The undergrowth appears on the ground within the area for area spells, or on the ground in all spaces occupied by each target of the spell for targeted spells.

        \ff[9]{Personal Verdamancy} You gain an additional \glossterm{attunement point}.
        You can only use this attunement point to \glossterm{attune} to spells and rituals from the \sphere{Verdamancy} \glossterm{mystic sphere}.

        \ff[12]{Greater Verdant Armor} You learn the \spell{verdant armor} spell from the Verdamancy mystic sphere.
        In addition, the armor you create with that spell can be made of any non-metal special material other than dragonhide as long as the total item level of the armor does not exceed your level.

        \ff[15]{Spell} You learn a spell from the \sphere{Verdamancy} \glossterm{mystic sphere}.
        When you gain access to a new spell level, you can change which spell you know from that mystic sphere.

        \ff[18]{Supreme Residual Undergrowth} You can choose to create either \glossterm{heavy undergrowth} or \glossterm{light undergrowth} with your \textit{residual undergrowth} ability.
        In addition, any \glossterm{light undergrowth} you create with that ability does not impede the movement of your \glossterm{allies}.

        % Seems weird for an 18th level; should stay in sync with the druid wild aspect
        \ff[21]{Greater Verdant Allies} You treat all living creatures as if they were plants for the purpose of abilities from this feat and abilities from the \sphere{Verdamancy} \glossterm{mystic sphere}.
    \end{feat}

    \begin{feat}{Sphere Focus: Vivimancy}{Casting, Magical}
        \featpre Access to the \sphere{Vivimancy} \glossterm{mystic sphere}.

        \ff[1]{Spell} You learn a spell from the \sphere{Vivimancy} \glossterm{mystic sphere}.
        When you gain access to new spell ranks, you can change which spell you know from that \glossterm{mystic sphere}.

        \ff[3]{Personal Vitality} You gain a \plus1 bonus to Fortitude defense and are immune to being \sickened.

        % Should this be more specific, like "animates"?
        \ff[6]{Hidden Life} You can treat nonliving creatures other than undead as if they were living creatures for the purpose of your spells from the \sphere{Vivimancy} \glossterm{mystic sphere}.

        \ff[9]{Personal Vivimancy} You gain an additional \glossterm{attunement point}.
        You can only use this attunement point to \glossterm{attune} to spells and rituals from the \sphere{Vivimancy} \glossterm{mystic sphere}.

        \ff[12]{Greater Personal Vitality} The bonus to Fortitude defense from your \textit{personal vitality} ability increases to \plus2.
        In addition, you are immune to being \nauseated.

        \ff[15]{Spell} You learn a spell from the \sphere{Vivimancy} \glossterm{mystic sphere}.
        When you gain access to a new spell level, you can change which spell you know from that mystic sphere.

        \ff[18]{Life Suppression} You are no longer considered a living creature for the purpose of attacks against you.
        This means that attacks which only affect living creatures have no effect against you.

        \ff[21]{Supreme Personal Vitality} The bonus to Fortitude defense from your \textit{personal vitality} ability increases to \plus3.
        In addition, you are immune to any effects which would cause you to die or gain \glossterm{vital wounds} while you still have \glossterm{hit points} remaining.
    \end{feat}

    \begin{feat}{Stealth Specialization}{Skill}
        \featpre Stealth as a mastered skill.

        \ff[1]{Specialization} You gain a \plus3 bonus to the Stealth skill.

        \ff[3]{Movement Tolerance} Your penalties for moving while hiding are reduced by 5.

        \ff[6]{Ambush the Unwary} You gain a \plus2 bonus to \glossterm{power} against \unaware and \partiallyunaware creatures.

        \ff[9]{Greater Specialization} The bonus from your \textit{specialization} ability increases to \plus5.

        \ff[12]{Greater Movement Tolerance} The penalty reduction from your \textit{movement tolerance} ability increases to 10.
        This allows you to move at half speed without penalty.

        \ff[15]{Supreme Specialization} The bonus from your \textit{specialization} ability increases to \plus7.

        \ff[18]{Greater Ambush the Unwary} The bonus from your \textit{ambush the unwary} ability increases to \plus6.

        \ff[21]{Supreme Movement Tolerance} The penalty reduction from your \textit{movement tolerance} ability increases to 20.
        This allows you to move at full speed without penalty.
    \end{feat}

    \begin{feat}{Survival Specialization}{Skill}
        \featpre Survival as a mastered skill.

        \ff[1]{Specialization} You gain a \plus3 bonus to the Survival skill.

        \ff[3]{Terrain Tolerance} You ignore \glossterm{difficult terrain} and harmful natural terrain of any kind.
        If a skill check, such as Climb or Swim, would normally be required to move through the terrain, this ability does not help.

        \ff[6]{Rapid Tracker}
        While following trails with the \textit{track} ability, you can move at your normal speed while following tracks without taking the normal \minus5 penalty.

        \ff[9]{Greater Specialization} The bonus from your \textit{specialization} ability increases to \plus5.

        \ff[12]{Planar Tolerance}[Magical] You are immune to damage and \glossterm{conditions} imposed by being on other planes.
        In addition, you gain a \plus5 bonus to checks and defenses related to planar effects, such as checks required to manipulate subjective gravity.

        \ff[15]{Supreme Specialization} The bonus from your \textit{specialization} ability increases to \plus7.

        \ff[18]{Find the Path}[Magical] You can use the \textit{find the path} ability as a standard action.
        \begin{attuneability}{Find the Path}
            \abilitytag{Attune} (self)
            \rankline
            When you use this ability, you must unambiguously specify a location on the same plane as you.
            You know exactly what direction you must travel to reach your chosen destination by the most direct physical route.
            You are not always led in the exact direction of the destination -- if there is an impassable obstacle between the target and the destination, this ability will direct you around the obstacle, rather than through it.

            The guidance provided by this ability adjusts to match whatever your current physical capabilities are, including flight and other unusual movement modes.
            It does not consider teleportation spells or any other activated abilities you may have which could allow you to bypass physical obstacles.
            It does not see into the future, and changing circumstances may cause the most direct path to change over time.
            It also does not consider hostile creatures, traps, and other passable dangers which may endanger or slow progress.
        \end{attuneability}

        \ff[21]{Greater Planar Tolerance} The bonus from your \textit{planar tolerance} ability increases to \plus20.
        In addition, your \glossterm{allies} within a \hugearea \glossterm{emanation} from you gain a \plus10 bonus to checks and defenses related to planar effects.
    \end{feat}

    \begin{feat}{Swift}{General}
        \featpre Base Dexterity of 1.

        \ff[1]{Rapid Movement} You gain a \plus5 foot bonus to your speed with all of your \glossterm{movement modes}.

        \ff[3]{Sprinter} When you use the \textit{sprint} ability, you move can move up to triple your movement speed.
        In addition, you gain a \plus1 bonus to your \glossterm{fatigue tolerance}.

        \ff[6]{Wall Runner} You gain a \plus5 bonus to checks with the \textit{wallrun} ability (see \pcref{Wallrun}).
        In addition, you can make a Dexterity check in place of a Climb check to use that ability.

        \ff[9]{Water Runner} During your movement with the \textit{sprint} ability, you can move on water and similar liquids as if they were solid ground.

        \ff[12]{Greater Rapid Movement} The speed bonus from your \textit{rapid movement} ability increases to \plus10 feet.

        \ff[15]{Greater Sprinter} When you use the \textit{sprint} ability, you can move up to four times your movement speed.
        In addition, the bonus to your \glossterm{fatigue tolerance} from your \textit{sprinter} ability increases to \plus2.

        \ff[18]{Cloud Runner} During your movement with the \textit{sprint} ability, you can move on dense fog and similar gaseous substances as if they were solid ground.

        \ff[21]{Supreme Rapid Movement} The speed bonus from your \textit{rapid movement} ability increases to \plus15 feet.
    \end{feat}

    \begin{feat}{Swim Specialization}{Skill}
        \featpre Swim as a mastered skill.

        \ff[1]{Specialization} You gain a \plus3 bonus to the Swim skill.

        \ff[3]{Swim Speed} You gain a \glossterm{swim speed} equal to the \glossterm{base speed} for your size.
        If you already have a swim speed, you gain a \plus10 foot bonus to your swim speed.
        A successful Swim check to move allows you to move a distance equal to your swim speed.

        \ff[6]{Underwater Tolerance} You reduce your penalties for fighting underwater by 4 (see \pcref{Underwater Combat}).

        \ff[9]{Greater Specialization} The bonus from your \textit{specialization} ability increases to \plus5.

        \ff[12]{Earth Swimmer} You can swim through loose earth and dirt as if it were water.
        Your swim speed in earth is only 5 feet, and you take a \minus4 penalty to \glossterm{accuracy}, Strength and Dexterity-based \glossterm{checks}, and Armor and Reflex defenses while swimming in this way.
        The earth and dirt around you blocks line of sight and line of effect, so you usually cannot used ranged attacks of any kind.

        \ff[15]{Supreme Specialization} The bonus from your \textit{specialization} ability increases to \plus7.

        \ff[18]{Rapid Swimmer} You gain a \plus10 foot bonus to your swim speed.

        \ff[21]{Greater Earth Swimmer} Your swim speed in earth increases to 15 feet.
        In addition, you reduce your penalties for swimming in earth by 2.
    \end{feat}

    \begin{feat}{Telepath}{General}
        \featpre Base Intelligence and Willpower of 1.

        \ff[1]{Telepathy} You gain \glossterm{telepathy} with a 60 foot range (see \pcref{Telepathy}).

        \ff[3]{Mind Crush} You can use the \textit{mind crush} ability as a standard action.
        \begin{durationability}{Mind Crush}
            \abilitytag{Emotion}
            \rankline
            Make an attack vs. Mental against one creature within the half the maximum range of your \glossterm{telepathy}.
            \hit As a \glossterm{condition}, the subject is \dazed if it has any \glossterm{damage resistance} remaining, and \stunned if it does not.
            \crit The condition must be removed twice before the effect ends.

            \rankline
            You gain a \plus2 bonus to \glossterm{accuracy} with the attack at 6th level and every 3 levels thereafter.
            In addition, at 9th level, the effect lasts \glossterm{briefly} on a \glossterm{glancing blow}.
        \end{durationability}

        \ff[6]{Read Mind}[Magical] You can use the \textit{read mind} ability as a standard action.
        \begin{durationability}{Read Mind}
            \abilitytag{Emotion}, \abilitytag{Sustain} (standard), \abilitytag{Subtle}
            \rankline
            Make an attack vs. Mental against a creature within half the maximum range of your \glossterm{telepathy}.
            Whether you hit or miss, you cannot target the subject with this ability again until it takes a \glossterm{short rest}.
            \hit You know the target's current thoughts and emotions.
            In addition to the obvious effects, this grants you a \plus5 bonus to Deception, Persuasion, Intimidate, and Social Insight attacks and checks against the target.
            This does not allow you to search their mind for arbitrary thoughts or information, but creatures often think about questions they are asked, and their thoughts may reveal much more than their words.

            \rankline
            You gain a \plus2 bonus to \glossterm{accuracy} with the attack at 9th level and every 3 levels thereafter.
            In addition, at 9th level, the effect lasts \glossterm{briefly} on a \glossterm{glancing blow}.
        \end{durationability}

        \ff[9]{Greater Telepathy} The range of your \glossterm{telepathy} ability increases to 120 feet.
        In addition, you automatically know the location of any creature with an Intelligence of 0 or higher within half the maximum range of your telepathy.
        Unlike abilities like \glossterm{blindsense}, the Stealth skill does not prevent you from learning the location of creatures in this way.

        \ff[12]{Fragmented Mind} You gain a \plus2 bonus to Mental defense.
        In addition, you can maintain mental channels with up to 5 creatures at once with your telepathy.
        You can send separate thoughts to each creature.

        \ff[15]{Greater Read Mind} On a \glossterm{critical hit} with your \textit{read mind} ability, you can delve through the subject's mind to answer a specific question.
        You can pose a question to it mentally and search its mind to know the exact answer to that question.
        This takes five rounds of continuous concentration, and you can only get answers to one such question each time you use this ability.
        The process of searching a creature's mind in this way is no easier to notice than normal for a \abilitytag{Subtle} ability.

        \ff[18]{Supreme Telepathy} The range of your \glossterm{telepathy} ability increases to 240 feet.
        In addition, you can see perfectly any creature with an Intelligence of 0 or higher within half the maximum range of your telepathy.
        Unlike abilities like \glossterm{blindsight}, the Stealth skill does not prevent you from seeing creatures in this way.

        \ff[21]{Mental Domination} Whenever a creature \stunned by your \textit{mind crush} ability reaches 0 hit points, you can \glossterm{attune} to this ability.
        When you do, that creature becomes \dominated by you as long as you maintain that attunement.
        As normal, you can only maintain one instance of this attunement at a time.
    \end{feat}

    \begin{feat}{Toughness}{General}
        \featpre Base Constitution of 1.

        \ff[1]{Fortified Body} You gain a \plus2 bonus to Fortitude defense.
        In addition, you can sleep while you have \glossterm{encumbrance} without penalty (see \pcref{Encumbrance}).

        \ff[3]{Durability} You gain a \plus4 bonus to your maximum \glossterm{hit points}.

        \ff[6]{Ailment Tolerance} You are immune to being \sickened and \nauseated.

        \ff[6]{Sleepless} You need half the normal amount of rest and sleep each day to function normally.
        For example, a human would only need four hours of sleep per night.
        This does not reduce the time required for you to take a \glossterm{long rest}.

        \ff[9]{Greater Fortified Body} The defense bonus from your \textit{fortified body} ability increases to \plus4.

        \ff[12]{Greater Durability} The bonus from your \textit{durability} ability increases to \plus10.

        \ff[15]{Greater Ailment Tolerance} You are immune to poisons and diseases.

        \ff[15]{Greater Sleepless} The amount of rest and sleep you need each day is reduced to a quarter of the normal value.
        For example, a human would only need two hours of sleep per night.

        \ff[18]{Greater Fortified Body} The defense bonus from your \textit{fortified body} ability increases to \plus6.

        \ff[21]{Supreme Durability} The bonus from your \textit{durability} ability increases to \plus25.
    \end{feat}

    \begin{feat}{Twinhand Spellcaster}{Casting, Magical}
        \featpre Base Dexterity of 1.

        \ff[1]{Twinhand Focus} You can always choose to use \glossterm{somatic components} to cast your spells (see \pcref{Casting Components}).
        As long as you have two \glossterm{free hands}, you reduce your \glossterm{focus penalty} by 2 with spells that you cast using \glossterm{somatic components}.

        \ff[3]{Freehand Implement} You can gain the benefits of one magical implement, such as a staff or wand, without having to hold it in your hands.
        You must still have it on your person, such as in a pocket or strapped to your back, and you must still be attuned to it to gain its benefits.
        This ability only affects one implement at a time.
        In addition, if your legacy item is an apparel item, you may choose both apparel and implement magic item effects for it.

        \ff[6]{Double Spell} You can use the \textit{double spell} ability as a \glossterm{standard action}.
        \begin{instantability}{Double Spell}
            You can only use this ability if you have two \glossterm{free hands}.

            Choose two spells that you know which do not have the \abilitytag{Sustain} or \abilitytag{Attune} tags.
            You cast both spells simultaneously, one with each hand.
            This gives the spells \glossterm{somatic components}, regardless of any other effects which would would normally prevent you from requiring somatic components.
            Both spells must affect completely different targets, with no overlap between their targets or areas (if any).

            After you use this ability, you are unable to take any actions during the following round.
        \end{instantability}

        \ff[9]{Twinhand Precision} As long as you have two \glossterm{free hands}, you gain a \plus1 \glossterm{accuracy} bonus with spells that you cast using \glossterm{somatic components}.

        \ff[12]{Greater Double Spell} Using your \textit{double spell} ability does not prevent you from acting during the \glossterm{movement phase} of the following round.

        \ff[15]{Greater Freehand Implement} You can use your \textit{freehand implement} ability to affect an additional implement.

        \ff[18]{Greater Twinspell Precision} The bonus from your \textit{twinhand precision} ability increases to \plus2.

        \ff[21]{Supreme Double Spell} Using your \textit{double spell} ability does not prevent you from taking \glossterm{minor actions} during the following round.
    \end{feat}

    \begin{feat}{Two-Weapon Fighting}{Combat}
        \featpre Base Dexterity of 2.

        \ff[1]{Offhand Freedom} You can use the \textit{offhand strike} ability as a \glossterm{free action} instead of as a \glossterm{minor action}.
        However, you cannot use the \textit{offhand strike} ability more than once per round.
        In addition, you can use the \textit{offhand strike} ability during any phase that you take a \glossterm{standard action}, regardless of whether that standard action causes you to make a \glossterm{strike}.

        \ff[3]{Offhand Force} You gain a \plus1d damage bonus with the \textit{offhand strike} ability.

        \ff[6]{Dual Precision} You gain a \plus1 bonus to \glossterm{accuracy} with \glossterm{strikes} while you wield two weapons.

        \ff[9]{Greater Offhand Force} The bonus from your \textit{offhand force} ability increases to \plus2d.

        \ff[12]{Greater Offhand Freedom} When you use your \textit{offhand strike} ability, you treat all other weapons you attack with during that phase as being light for the purpose of determining your accuracy penalties.
        In addition, you may add half your \glossterm{power} to damage with the \glossterm{strike} from your \textit{offhand strike} ability if it is the only \glossterm{strike} you make during that phase.

        \ff[15]{Supreme Offhand Force} The bonus from your \textit{offhand force} ability increases to \plus3d.

        \ff[18]{Greater Dual Precision} The bonus from your \textit{dual precision} ability increases to \plus2.

        \ff[21]{Offhand Flurry} You can use the \textit{offhand strike} ability twice per round.
    \end{feat}

    \begin{feat}{Weapon Focus}{Combat}
        \ff[1]{Focused Weapon} Choose one type of weapon, such as a broadsword.
        This is your focused weapon, and many abilities from this feat give you benefits with your focused weapon.

        \ff[1]{Perfect Strike} You can use the \textit{perfect strike} ability as a standard action.
        \begin{freeability}{Perfect Strike}
            Make a \glossterm{strike} using your focused weapon.
            You gain a \plus1 bonus to \glossterm{accuracy} with the strike and a \plus1d damage bonus.

            \rankline
            \featlevel{6} The damage bonus increases to \plus2d.
            \featlevel{12} The damage bonus increases to \plus3d.
            \featlevel{18} The damage bonus increases to \plus4d.
        \end{freeability}

        \ff[3]{Firm Grip} Your focused weapon is considered to be well-secured whenever you want it to be, making it difficult or impossible for you to be disarmed.

        \ff[6]{Focused Power} You gain a \plus2 bonus to \glossterm{power} with attacks using your focused weapon.

        \ff[9]{Focused Precision} You gain a \plus1 bonus to \glossterm{accuracy} with attacks using your focused weapon.

        \ff[12]{Greater Focused Power} The bonus from your \textit{focused power} ability increases to \plus4.

        \ff[15]{Greater Focused Precision} The bonus from your \textit{focused precision} ability increases to \plus2.

        \ff[18]{Supreme Focused Power} The bonus from your \textit{focused power} ability increases to \plus8.

        \ff[21]{Supreme Focused Precision} The bonus from your \textit{focused precision} ability increases to \plus3.
    \end{feat}

    \begin{feat}{Whirlwind Warrior}{Combat}
        \featpre Base Dexterity of 1.

        % TODO: this would be better if it hit threatened people, but increased weapon range is too common
        \ff[1]{Cyclone} You can use the \textit{cyclone} ability as a standard action.
        \begin{freeability}{Cyclone}[\abilitytag{Sustain} (standard)]
            When you use this ability, make a melee \glossterm{strike} with a slashing weapon.
            Your \glossterm{power} with the strike is halved.
            The strike targets any number of creatures adjacent to you.
            Whenever you sustain this ability, you can move up to half your speed and make a melee \glossterm{strike} with a slashing weapon.
            The strike targets any number of creatures adjacent to you at any point during your movement.

            \rankline
            \featlevel{6} You gain a \plus1d damage bonus with the strike.
            \featlevel{12} The damage bonus increases to \plus2d.
            \featlevel{18} The damage bonus increases to \plus3d.
        \end{freeability}

        \ff[3]{Unfettered Movement} During each phase, you may move through one creature's space during movement.
        You treat its space as \glossterm{difficult terrain}.
        After moving through that creature's space, other creatures block you as normal for the remainder of your movement.
        Certain unusual creatures that occupy their entire space, such as gelatinous cubes, may be immune to this ability.
        If you end your movement in a creature's space with this ability, you and that creature are \glossterm{squeezing} if you are no more than one size category larger or smaller than it.

        \ff[6]{Windrush} You gain a \plus5 foot bonus to your speed with all of your \glossterm{movement modes}.

        \ff[9]{Eye of the Storm} You cannot be \surrounded.

        \ff[12]{Greater Unfettered Movement} Using your \textit{unfettered movement} ability does not cause you to treat spaces occupied by creatures as difficult terrain.

        \ff[15]{Greater Windrush} The bonus from your \textit{wind dance} ability increases to \plus10 feet.

        \ff[18]{Greater Eye of the Storm} You take no penalties for \glossterm{squeezing} with other creatures.
        This does not reduce your penalties for squeezing in tight spaces.

        \ff[21]{Supreme Unfettered Movement} You may move through any number of creatures with your \textit{unfettered movement} ability.
    \end{feat}

\section{Other Feat Rules}

    \subsection{Retraining Feats}
        At every level, you can choose to retrain an old feat in exchange for a new feat.
