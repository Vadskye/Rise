\chapter{Feats}\label{Feats}

Feats are special abilities that every character has.
Feats can be used to specialize your character particular area, to grant your character new abilities, or to change the way your character does certain things.

\section{Gaining Feats}
    Your character gains a feat at 1st level and every 4 levels thereafter: 1st, 5th, 9th, and so on.
    Classes can sometimes grant bonus feats as well, which are in addition to these feats which every character gets.

    \subsection{Prerequisites}
        Some feats have prerequisites.
        Your character must have the indicated attribute score, ability, feat, skill, or other quality designated in order to select or use that feat.
        A character can gain a feat at the same level at which he or she gains the prerequisite.

        A character can't use a feat if he or she has lost a prerequisite.

\section{Types Of Feats}
    All feats belong to one of four broad categories.

    \parhead{General} General feats can have a wide variety of effects.
    They often grant new abilities or improve your defenses.

    \parhead{Combat} Combat feats improve your combat capabilities.
    They can increase the damage you deal, grant you new combat abilities, or improve your defenses.

    \parhead{Spell} Spell feats improve your spellcasting abilities.
    All Spell feats except for the Ritual Caster feat are useless to characters who cannot cast spells (see \featpcref{Ritual Caster}).

    \parhead{Skill} Skill feats improve your skills.
    They can make you more likely to succeed with skill checks and grant you new abilities based on your skills.
    % generic skill feat rank requirements:
    % 1st = 4, 5th = 6, 9th = 8, 13th = 10, 17th = 12
    %    because should be able to get them 1 feat behind if trained
    % mastery feats:
    % 1st = 6, 5th = 10, 9th = 14, 13th = 18, 17th = 23
    %    because should need to have mastered the skill

    \subsection{Feat Tags}

        All feats are tagged according to their category.
        In addition, some feats have more specific tags that describe what the feat does.
        The tags are described below.

        \parhead{Bloodline Feats}

        Some characters have traces of monstrous blood running in their veins.
        Most of those will never understand the full potential of their unusual heritage.
        Bloodline feats allow characters to explore those posibilities by gaining abilites related to their ancestry.
        Each bloodline feat belongs to a specific type of monster, such as ``dragon''.

        You can only have one type of bloodline feat.
        Each type of bloodline has a single feat with ``Heritage'' in the name, which all other feats in the bloodline have as a prerequisite.

        \parhead{Magical Feats}
        Magical feats are \glossterm{magical} in nature.
        Many feats are not entirely magical, but have specific effects that are magical.
        These effects are indicated by the \magical tag.

        \parhead{Style Feats}\label{Style Feats}
        Style feats grant a character the ability to fight or cast spells in a particular style, granting them bonuses while in that style.
        A character can only be in one style at once.
        Once per round, a character can initiate a style, change to a different style, or stop using a style as a \glossterm{free action}.

        Most style feats have requirements.
        If a style requires specific equipment, such as a melee weapon, you must meet the requirements to activate the style.
        If you fail to meet a style's requirements during a round, you leave the style at the end of the round.

\section{Feat Tables}
    \newpage\onecolumn

% Feat names must follow ``have'', ``are (a)'', or ``can''.
        \begin{longtabuwrapper}
            \begin{longtabu}{>{\lcol}p{10em} >{\lcol}p{15em} >{\lcol}X >{\lcol}p{8em} >{\lcol}p{3em}}
                \lcaption{Feats}\\
                \tb{General Feats}\label{General Feats} & \tb{Prerequisites} & \tb{Benefit} & \tb{Feat Types} & \tb{Page} \\
                \featref{Celestial Heritage} & Non-evil & Gain holy abilities & Bloodline & \featpref{Celestial Heritage} \\
                \tind \featref{Celestial Apotheosis} & 9th level, Celestial Heritage, non-evil & Gain wings and holy judgment & Bloodline & \featpref{Celestial Apotheosis} \\
                \featref{Celestial Spell Conduit} & 5th level, 2nd level spells, Celestial Heritage, non-evil & Heal allies when you cast spells & Bloodline & \featpref{Celestial Spell Conduit} \\
                \featref{Draconic Heritage} & \tdash & Gain draconic abilities & Bloodline & \featpref{Draconic Heritage} \\
                \tind \featref{Draconic Apotheosis} & 9th level, Draconic Heritage & Gain wings, improved draconic abilities & Bloodline & \featpref{Draconic Apotheosis} \\
                \featref{Endurance} & Con 3 & Resist fatigue and exhaustion & \tdash & \featpref{Endurance} \\
                \featref{Iron Will} & Wil 3 & Improve Mental defenses & \tdash & \featpref{Iron Will} \\
                \featref{Item Conduit} & 5th level & Improve magic item usage & \tdash & \featpref{Item Conduit} \\
                \featref{Legendary Celerity} & 13th level, Dex 12 & Gain additional swift or immediate action & \tdash & \featpref{Legendary Celerity} \\
                \featref{Legendary Mental Fortitude} & 13th level, Wil 12 & Gain \plus5 Fortitude, damage reduction & \tdash & \featpref{Legendary Mental Fortitude} \\
                \featref{Legendary Might} & 13th level, Str 12 & Increase weight limits and weapon damage & \tdash & \featpref{Legendary Might} \\
                \featref{Lightning Reflexes} & Dex 3 & Improve Reflex defense, initiative & \tdash & \featpref{Lightning Reflexes} \\
                \featref{Rapid Recovery} & Con 3 & Heal damage very quickly & \tdash & \featpref{Rapid Recovery} \\
                \featref{Spellbreaker} & 9th level, Wil 9 & Gain magic resistance & \tdash & \featpref{Spellbreaker} \\
                \featref{Spellgift} & 5 non-spellcaster levels, Wil 6 & Gain spell-like abilities & \tdash & \featpref{Spellgift} \\
                \featref{Swift} & \tdash & Gain speed bonus & \tdash & \featpref{Swift} \\
                \featref{Toughness} & Con 3 & Improve Fortitude defenses & \tdash & \featpref{Toughness} \\

                \tb{Class Feats}\label{Class Feats} & \tb{Prerequisites} & \tb{Benefit} & \tb{Feat Types} & \tb{Page} \\
                \featref{All Energy Becomes One} & Monk 5, Con 6, manifest \ki & Instantly absorb energy attacks & \tdash & \featpref{All Energy Becomes One} \\
                \featref{Arcane Resilience} & Mage 1 & Reduce damage from spells & \tdash & \featpref{Arcane Resilience} \\
                \featref{Chaotic Rage} & Paladin 5 & Gain ability to rage & \tdash & \featpref{Chaotic Rage} \\
                \featref{Combat Leader} & Fighter 9, Int 9 & Grant feats to allies & \tdash & \featpref{Combat Leader} \\
                \featref{Controlling Wild Speech} & Druid or ranger 5, wild speech & Control animals and objects with wild speech & \tdash & \featpref{Controlling Wild Speech} \\
                \featref{Extra Domain} & Cleric 5 & Gain additional domain & \tdash & \featpref{Extra Domain} \\
                \featref{Style Fusion} & Fighter 9, Int 9 & Use multiple style feats at once & \tdash & \featpref{Style Fusion} \\
                \featref{Versatile Wild Speech} & Druid or ranger 1, wild speech & Improve communication with wild speech & \tdash & \featpref{Versatile Wild Speech} \\

                \tb{Spell Feats}\label{Spell Feats} & \tb{Prerequisites} & \tb{Benefit} & \tb{Feat Types} & \tb{Page} \\
                \featref{Abjurer} & 1st level Abjuration spell & Improve Abjuration spells you cast & \tdash & \featpref{Abjurer} \\
                \featref{Battlecaster} & 1st level spells & Improve spellcasting in combat & \tdash & \featpref{Battlecaster} \\
                \featref{Conjurer} & 1st level Conjuration spell & Improve Conjuration spells you cast & \tdash & \featpref{Conjurer} \\
                \featref{Counterspell} & 5th level, 2nd level spells, Spellcraft 6 & Force foes to miscast spells & \tdash & \featpref{Counterspell} \\
                \featref{Devastating Magic} & 13th level, 6th level spells & Spells deal more damage & \tdash & \featpref{Devastating Magic} \\
                \featref{Distant Magic} & 5th level, 2nd level spells & Cast spells at greater distances & \tdash & \featpref{Distant Magic} \\
                \featref{Diviner} & 1st level Divination spell & Improve Divination spells you cast & \tdash & \featpref{Diviner} \\
                \featref{Energetic Magic} & 9th level, 4th level \glossterm{energy} spell & Improve energy spells you cast & \tdash & \featpref{Energetic Magic} \\
                \featref{Empowered Magic} & 9th level, 4th level spells & Empower effects of your spells & \tdash & \featpref{Empowered Magic} \\
                \featref{Enchant Item} & 1st level spells & Gain ability to create magic items & \tdash & \featpref{Enchant Item} \\
                \featref{Enchanter} & 1st level Enchantment spell & Improve Enchantment spells you cast & \tdash & \featpref{Enchanter} \\
                \featref{Evoker} & 1st level Evocation spell & Improve Evocation spells you cast & \tdash & \featpref{Evoker} \\
                \featref{Hidden Magic} & 5th level, 2nd level spells & Cast spells without components & \tdash & \featpref{Hidden Magic} \\
                \featref{Illusionist} & 1st level Illusion spell & Improve Illusion spells you cast & \tdash & \featpref{Illusionist} \\
                \featref{Legendary Spell Alacrity} & 13th level, 6th level spells & Spells take effect more quickly & \tdash & \featpref{Legendary Spell Alacrity} \\
                \featref{Legendary Well of Magic} & 13th level, 6th level spells & Cast spells without spending spell slots & \tdash & \featpref{Legendary Well of Magic} \\
                \featref{Miscaster} & 1st level spells & Improve your miscast spells & \tdash & \featpref{Miscaster} \\
                \featref{Quickened Magic} & 9th level, 4th level spells & Cast spells quickly & \tdash & \featpref{Quickened Magic} \\
                \featref{Ritual Caster} & Int 3 & Gain ability to perform rituals & \tdash & \featpref{Ritual Caster} \\
                \featref{Shaped Magic} & 5th level, 2nd level spells & Control area of spells & \tdash & \featpref{Shaped Magic} \\
                \featref{Somatic Strike} & 9th level, 4th level spells & Attack in place of somatic components & \tdash & \featpref{Somatic Strike} \\
                \featref{Spellwoven Performance} & 5th level, 1st level spells, any Performance feat & Blend spells and performances & Skill & \featpref{Spellwoven Performance} \\
                \featref{Spellcasting Versatility} & 5th level, 1st level spells & Improve spellcasting when multiclassing & \tdash & \featpref{Spellcasting Versatility} \\
                \featref{Spellstrike} & 5th level, 2nd level spells & Channel spell through physical attack  & \tdash & \featpref{Spellstrike} \\
                \featref{Transmuter} & 1st level Transmutation spell & Improve Transmutation spells you cast & \tdash & \featpref{Transmuter} \\
                \featref{Vivimancer} & 1st level Vivimancy spell & Improve Vivimancy spells you cast & \tdash & \featpref{Vivimancer} \\

                \tb{Skill Feats}\label{Skill Feats} & \tb{Prerequisites} & \tb{Benefit} & \tb{Feat Types} & \tb{Page} \\
                \featref{Acrobatics Mastery} & Acrobatics 6 & Improve Acrobatics checks & \tdash & \featpref{Acrobatics Mastery} \\
                \featref{Awareness Mastery} & Awareness 6 & Improve Awareness checks & \tdash & \featpref{Awareness Mastery} \\
                \featref{Bluff Mastery} & Bluff 6 & Improve Bluff checks & \tdash & \featpref{Bluff Mastery} \\
                \featref{Climb Mastery} & Climb 6 & Improve Climb checks & \tdash & \featpref{Climb Mastery} \\
                \featref{Craft Magic Item} & Craft (any) 4 & Gain ability to craft magic items & \tdash & \featpref{Craft Magic Item} \\
                \featref{Craft Mastery} & Craft 6 & Improve Craft checks & \tdash & \featpref{Craft Mastery} \\
                \featref{Creature Handling Mastery} & Creature Handling 6 & Improve Creature Handling checks & \tdash & \featpref{Creature Handling Mastery} \\
                \featref{Device Mastery} & Devices 6 & Improve Devices checks & \tdash & \featpref{Device Mastery} \\
                \featref{Disguise Mastery} & Disguise 6 & Improve Disguise checks & \tdash & \featpref{Disguise Mastery} \\
                \featref{Escape Artist Mastery} & Escape Artist 6 & Improve Escape Artist checks & \tdash & \featpref{Escape Artist Mastery} \\
                \featref{Heal Mastery} & Heal 6 & Improve Heal checks & \tdash & \featpref{Heal Mastery} \\
                \featref{Inspiring Performer} & Perform 4 & Perform to improve abilities of allies & Performance & \featpref{Inspiring Performer} \\
                \featref{Intimidate Mastery} & Intimidate 6 & Improve Intimidate checks & \tdash & \featpref{Intimidate Mastery} \\
                \featref{Jump Mastery} & Jump 6 & Improve Jump checks & \tdash & \featpref{Jump Mastery} \\
                \featref{Knowledge Mastery} & Knowledge (any) 6 & Improve Knowledge checks & \tdash & \featpref{Knowledge Mastery} \\
                \featref{Legendary Skill Savant} & 13th level & Gain ten skill points & \tdash & \featpref{Legendary Skill Savant} \\
                \featref{Linguistic Mastery} & Linguistics 6 & Improve Linguistics checks & \tdash & \featpref{Linguistic Mastery} \\
                \featref{Mesmerizing Performer} & Perform (any) 4 & Perform to influence and distract foes & Performance & \featpref{Mesmerizing Performer} \\
                \featref{Mocking Performer} & Perform (any) 4 & Perform to impair foes & Performance & \featpref{Mocking Performer} \\
                \featref{Persuasion Mastery} & Persuasion 6 & Improve Persuasion checks & \tdash & \featpref{Persuasion Mastery} \\
                \featref{Perform Mastery} & Perform 6 & Improve Perform checks & \tdash & \featpref{Perform Mastery} \\
                \featref{Ride Mastery} & Ride 6 & Improve Ride checks & \tdash & \featpref{Ride Mastery} \\
                \featref{Sense Motive Mastery} & Sense Motive 6 & Improve Sense Motive checks & \tdash & \featpref{Sense Motive Mastery} \\
                \featref{Skill Savant} & \tdash & Gain skill points & \tdash & \featpref{Skill Savant} \\
                \featref{Sleight of Hand Mastery} & Sleight of Hand 6 & Improve Sleight of Hand checks & \tdash & \featpref{Sleight of Hand Mastery} \\
                \featref{Spellcraft Mastery} & Spellcraft 6 & Improve Spellcraft checks & \tdash & \featpref{Spellcraft Mastery} \\
                \featref{Sprint Mastery} & Sprint 6 & Improve Sprint checks & \tdash & \featpref{Sprint Mastery} \\
                \featref{Stealth Mastery} & Stealth 6 & Improve Stealth checks & \tdash & \featpref{Stealth Mastery} \\
                \featref{Sprint Mastery} & Sprint 6 & Improve Sprint checks & \tdash & \featpref{Sprint Mastery} \\
                \featref{Supreme Inspiration} & 9th level, Perform 9 & Perform to greatly inspire allies & \tdash & \featpref{Supreme Inspiration} \\
                \featref{Survival Mastery} & Survival 6 & Improve Survival checks & \tdash & \featpref{Survival Mastery} \\
                \featref{Swim Mastery} & Swim 6 & Improve Swim checks & \tdash & \featpref{Swim Mastery} \\
                \featref{Trapfinder} & Awareness 4 & Improve ability to notice traps & \tdash & \featpref{Trapfinder} \\

                \tb{Combat Feats}\label{Combat Feats} & \tb{Prerequisites} & \tb{Benefit} & \tb{Feat Types} & \tb{Page} \\
                \featref{Armor-Piercing Attack} & 5th level, Per 6 & Gain precision-based special attacks & \tdash & \featpref{Armor-Piercing Attack} \\
                \featref{Battlerage} & 5th level, Wil 6 & Gain ability to rage & \tdash & \featpref{Battlerage} \\
                \featref{Blindfighter} & Per 3 & Fight better while unable to see & \tdash & \featpref{Blindfighter} \\
                \featref{Cleave} & Str 3 & Gain bonus strike after defeating foes & \tdash & \featpref{Cleave} \\
                \featref{Close-Quarters Fighting} & Dex 3 & Fight better while squeezing and grappling & \tdash & \featpref{Close-Quarters Fighting} \\
                \featref{Combat Mobility} & Dex 3 & Move through creatures, even while attacking & \tdash & \featpref{Combat Mobility} \\
                \featref{Counterattack} & 5th level, Dex 6 & Make free attacks when missed & Style & \featpref{Counterattack} \\
                \featref{Covering Fire} & Per 3 & Impair foes with ranged attacks & Style & \featpref{Covering Fire} \\
                \featref{Deadly Aim} & 5th level, Per 6 & Bonus damage with ranged attacks & Style & \featpref{Deadly Aim} \\
                \featref{Defensive Fighting} & \tdash & Trade damage for defense bonus & Style & \featpref{Defensive Fighting} \\
                \featref{Deflect Arrows} & Dex 3 & Instantly deflect ranged attacks & \tdash & \featpref{Deflect Arrows} \\
                \featref{Destructive} & Str 3 & Ignore damage reduction and hardness & \tdash & \featpref{Destructive} \\
                \featref{Executioner} & 5th level & Make free attacks against weakened foes & \tdash & \featpref{Executioner} \\
                \featref{Eye of the Storm} & Dex 3 & Reduce overwhelm penalties & Style & \featpref{Eye of the Storm} \\
                \featref{Far Shot} & Str 3 & Fight better at long range & \tdash & \featpref{Far Shot} \\
                \featref{Fearsome} & 5th level, Intimidate 10 & Intimidate struck foes & Skill & \featpref{Fearsome} \\
                \featref{Frenzied Assault} & 13th level, Wil 12 & Deal and take more physical damage & Style & \featpref{Frenzied Assault} \\
                \featref{Guardian} & \tdash & Reduce overwhelm penalties of allies & \tdash & \featpref{Guardian} \\
                \featref{Heavy Weapon Fighting} & Str 3 & Bonus damage with two-handed attacks & \tdash & \featpref{Heavy Weapon Fighting} \\
                \featref{Improvised Fighting} & \tdash & Fight better with improvised weapons & \tdash & \featpref{Improvised Fighting} \\
                \featref{Inescapable} & 9th level & Immobilize struck foes & Style & \featpref{Inescapable} \\
                \featref{Infuriating} & 5th level & Goad struck foes into attacking you & Style & \featpref{Infuriating} \\
                \featref{Legendary Combat Awareness} & 13th level, Per 12 & Choose actions after other creatures & \tdash & \featpref{Legendary Combat Awareness} \\
                \featref{Legendary Combat Reaction} & 13th level, Dex 12 & Reduce overwhelm penalties & \tdash & \featpref{Legendary Combat Reaction} \\
                \featref{Legendary Combat Tactics} & 13th level, Int 12 & Predict action a creature will take & \tdash & \featpref{Legendary Combat Tactics} \\
                \featref{Legendary Speed} & 13th level, Dex 12 & Make free attacks & Style & \featpref{Legendary Speed} \\
                \featref{Mage Slayer} & 5th level & Fight better against spellcasters & Style & \featpref{Mage Slayer} \\
                \featref{Magical Perception} & 9th level, Per 9 & Improve accuracy of magical attacks & \tdash & \featpref{Magical Perception} \\
                \featref{Maneuver Focus} & \tdash & Improve chosen combat maneuver & \tdash & \featpref{Maneuver Focus} \\
                \featref{Martial Training} & \tdash & Gain armor, weapon proficiencies & \tdash & \featpref{Martial Training} \\
                \featref{Mounted Combat} & Ride 4 & Fight better while mounted & \tdash & \featpref{Mounted Combat} \\
                \featref{Overwhelming Fire} & \tdash & Overwhelm foes with ranged weapons & Style & \featpref{Overwhelming Fire} \\
                \featref{Parry} & Dex 3 & Deflect strikes with martial skill & Style & \featpref{Parry} \\
                \featref{Point Blank Shot} & \tdash & Fight better with ranged weapons at close range & \tdash & \featpref{Point Blank Shot} \\
                \featref{Power Attack} & 5th level, Str 6 & Gain bonus to damage & Style & \featpref{Power Attack} \\
                \featref{Precise Attack} & Per 3 & Trade damage for accuracy bonus & Style & \featpref{Precise Attack} \\
                \featref{Precise Shot} & 5th level, Per 6 & Ignore cover, concealment, miss chances & \tdash & \featpref{Precise Shot} \\
                \featref{Quick Draw} & \tdash & Draw and stow items more quickly & \tdash & \featpref{Quick Draw} \\
                \featref{Reflexive Dodge} & 13th level, Dex 12 & Automatically dodge strikes & \tdash & \featpref{Reflexive Dodge} \\
                \featref{Shielded Fighting} & Shield proficiency & Gain defense bonuses with shields & \tdash & \featpref{Shielded Fighting} \\
                \featref{Tactical Prediction} & 5th level, Int 6 & Predict foe's next action & \tdash & \featpref{Tactical Prediction} \\
                \featref{Two-Weapon Fighting} & Dex 3 & Bonus accuracy, damage while dual wielding & \tdash & \featpref{Two-Weapon Fighting} \\
                \featref{Unarmed Fighting} & \tdash & Fight better with unarmed attacks & \tdash & \featpref{Unarmed Fighting} \\
                \featref{Weapon Focus} & \tdash & Fight better with chosen weapons & \tdash & \featpref{Weapon Focus} \\
            \end{longtabu}
            1. You can gain this feat multiple times. Each time you do, it has a different effect. \\
        \end{longtabuwrapper}
        \twocolumn

\section{Feat Descriptions}
    Here is the format for feat descriptions.

    \ssecfake{Feat Name [Type of Feat]}
    \featpre A minimum attribute score, another feat or feats, a minimum number of ranks in one or more skills, or a class level that a character must have in order to acquire this feat.
    This entry is absent if a feat has no prerequisite.
    A feat may have more than one prerequisite.
    \featben What the feat enables the character (``you'' in the feat description) to do.
    If a character has the same feat more than once, its benefits do not stack unless indicated otherwise in the description.
    \par In general, characters cannot gain the same feat twice.
    \parhead{Normal}
    What a character who does not have this feat is limited to or restricted from doing.
    If not having the feat causes no particular drawback, this entry is absent.
    \parhead{Special}
    Additional facts about the feat that may be helpful when you decide whether to acquire the feat.

    \begin{feat}{Abjurer}{Magical, Spell}
        \spelldesc{You have great talent with Abjuration spells.}
        \featpre \glossterm{Abjuration} spell known.
        \featben
        \ff{Abjurant Shield} Whenever you cast an Abjuration spell with a duration, you can give one willing creature targeted by the spell a deflective shield.
        The shielded creature gains a \plus1 bonus to \glossterm{physical defenses} as long as the spell lasts.
        This is a \glossterm{Magical}, \glossterm{Shielding} effect.
        You can only shield one creature in this way at a time.
        If you shield another creature, all previous shields are dismissed when the new shield takes effect.

        \ff[3]{Counterspell} While you are casting a spell, at any time before it resolves, you can take an \glossterm{immediate action} to turn that spell into a counterspell.
        If you do, your spell has no effect when it resolves.
        Instead, choose a creature within \rngmed range of you.
        If that creature is casting a spell, and your spell's level, including all augments, is at least as high as their spell's level, their spell also has no effect when it resolves.

        \ff[5]{Ablative Shield} The shielded creature also gains damage reduction against physical damage equal to your spellpower.
        % which spellpower?

        \ff[7]{Magic Against Magic} You gain a \plus2 bonus to accuracy with the \spell{antimagic} spell.
        In addition, when you use your \textit{counterspell} ability with the \spell{antimagic} spell, it is treated as being two spell levels higher than its actual level for the purpose of determining which spells it can counter.

        \ff[9]{Versatile Shield} The shielded creature's damage reduction applies against all damage, not just physical damage.

        \ff[11]{Pierce Shields} Your skill at creating defenses allows you penetrate defenses more easily.
        You gain a \plus1 bonus to \glossterm{accuracy} with magical attacks.

        \ff[13]{Punishing Counterspell} When you use your \texit{counterspell} ability, you can make the target \glossterm{miscast} their spell instead of negating the spell's effects.
        In addition, the range of your \textit{counterspell} ability increases to \rnglong.

        \ff[15]{Empowered Shield} The shielded creature's defense bonus increases to \plus2.

        \ff[17]{Mass Counterspell} When you use your \textit{counterspell} ability, you may target up to five creatures.

        \ff[19]{Abjuration Master} Your Abjuration spells are easier to cast.
        You can cast Abjuration spells as if they were one level lower than their actual level.
        This does not stack with spell level reductions from other feats.
    \end{feat}

    \begin{feat}{Skill Specialization, Acrobatics}{Skill}
        \featpre Acrobatics as a mastered skill.
        \parhead{Benefits}
        
        \ff{Specialization, Lesser} You gain a \plus2 bonus to Acrobatics.

        \ff[2]{Rapid Balance} Using Acrobatics to balance on slippery or narrow surfaces does not reduce your speed.

        \ff[4]{Agile Charge} You can change directions freely while making a \glossterm{charge}.

        \ff[6]{Surface Tolerance} You reduce DR modifiers for surface conditions on Acrobatics checks to balance by 2.
        This allows you to ignore minor surface conditions, such as slippery surfaces, when balancing.

        \ff[8]{Specialization} The bonus to Acrobatics increases to \plus3.

         % modifier at 10th: 12dex + 2mst + 2ft = 16
        \ff[10]{Legendary Airwalker}[Mag] You can attempt to move on surfaces that cannot support your weight.
        Surfaces that can support at least a quarter of your weight, such as thin tree branches and dense liquids, are DR 20.
        Surfaces that can support at least a tenth of your weight, such as water, are DR 25.
        Surfaces that can support at least a hundredth of your weight, such as tree leaves, are DR 30.
        Surfaces that cannot support your weight at all, such as air, are DR 40.
        Success means you move along the surface at half speed.
        Failure means you fall through the surface.
        The DR increases by 2 for each consecutive round that you spend moving in this way.

        \ff[12]{Fall Tolerance} You take half damage from \glossterm{falling damage}.

        \ff[14]{Surface Tolerance, Greater} The reduction of DR modifiers for surface conditions increases to 5.
        This allows you to ignore almost all surface conditions when balancing.

        \ff[16]{Specialization, Greater} The bonus to Acrobatics increases to \plus4.

        \ff[18]{Rapid Airwalker} You can move at full speed with the \textit{legendary airwalker} ability.
    \end{feat}

    \begin{feat}{All Energy Becomes One}{Class, Magical}
        \featpre Monk as a \glossterm{base class}, starting Constitution of 2.
        \parhead{Benefits}

        \ff{Absorb Energy} Whenever you would take \glossterm{energy damage}, you can take an \glossterm{immediate action} to channel the energy into your body.
        If you do, you gain damage reduction against that attack equal to your \ki power.

        \ff[3]{Channel Energy} At the end of each round, if you reduced damage with your \textit{absorb energy} ability, you can channel that energy into your weapons.
        If you do, choose a damage type that you reduced this round with that ability.
        All damage you deal with your weapons gains that damage type until the end of the next round.

        \ff[5]{Energetic Unity} You can use \textit{absorb energy} to reduce any non-physical damage you take, instead of only energy damage.

        \ff[7]{Channel Power} While you are channeling energy with your \textit{channel energy} ability, you deal \plus1d damage with your weapons.

        \ff[9]{Sustained Channeling} You can sustain your \texit{channel energy} ability as a swift action.

        \ff[11]{Absorb Energy, Greater} You increase the damage reduction to twice your \ki power.

        \ff[13]{Kinetic Absorption} You can use \textit{absorb energy} to reduce any damage you take, instead of only non-physical damage.

        \ff[15]{Channel Power, Greater} You increase your damage bonus with your weapons to \plus2d.

        \ff[17]{Attuned Channeling} When you use your \textit{channel energy} ability, you can spend an \glossterm{action point}.
        If you do, you attune to the ability, causing it to last as long as you stay attuned to it.
        For details, see \pcref{Attunement}.

        \ff[19]{Reflexive Absorption} You can use your \textit{greater absorb energy} ability once per round without spending an action.
        You cannot use it twice to affect the same attack.
    \end{feat}

    \begin{feat}{Skill Specialization, Awareness}{Skill}
        \featpre Awareness as a mastered skill.
        \parhead{Benefits}

        \ff[1]{Specialization, Lesser} You gain a \plus2 bonus to Awareness.

        \ff[2]{Broad Search} When you take the Search action, you can search a 10-ft.\ square within 30 feet of you (see \pcref{Search}).

        \ff[4]{Extraordinary Senses} You gain one of the following senses: \glossterm{blindsense} (50 ft.), \glossterm{darkvision} (100 ft.), \glossterm{scent}, or \glossterm{tremorsense} (50 ft.).

        \ff[6]{Distance Tolerance} You reduce DR modifiers for distance on Awareness rolls by 2.
        This usually allows you to ignore up to 100 feet of distance.

        % \ff[6]{Attentive} You can pay \glossterm{active attention} to your surroundings for any length of time without becoming fatigued.

        \ff[8]{Specialization} The bonus to Awareness increases to \plus3.

        \ff[10]{Legendary Senses} You gain one of the following senses: \glossterm{blindsense} (200 ft.), \glossterm{blindsight} (50 ft.), \glossterm{darkvision} (500 ft.), \glossterm{tremorsense} (200 ft.), or \glossterm{tremorsight} (50 ft.).

        \ff[12]{Trapmaster} Whenever you come within 50 feet of a trap, you can make an Awareness check to notice it, even if you were not searching for traps.

        \ff[14]{Distance Tolerance, Greater} The reduction of DR modifiers for distance increases to 5.
        This usually allows you to ignore up to 500 feet of distance.

        \ff[16]{Specialization, Greater} The bonus to Awareness increases to \plus4.

        \ff[18]{Supreme Senses} You can choose an additional sense from the \textit{legendary senses} ability.
        Its range is doubled.
    \end{feat}

    \begin{feat}{Battlecaster}{Spell}
        \featpre Ability to cast spells.
        \featben

        \ff[1]{Combat Concentration, Lesser} You gain a \plus2 bonus to Concentration checks made to cast spells.
        In addition, you reduce your chance of arcane spell failure from wearing armor by 10\%.

        \ff[3]{Spellstrike} As a standard action, you can spend an \glossterm{action point} to imbue magical power into your weapon.
        This requires concentration as if casting a spell.
        Whether or not your concentration was broken, you may make a \glossterm{strike} with the weapon as a \glossterm{delayed action}.
        If you maintained your concentration, the strike gains a \plus2d bonus to damage.

        \ff[5]{Combat Concentration} The bonus to Concentration checks increases to \plus5.

        \ff[7]{Armor Tolerance} The reduction of arcane spell failure increases to 20\%.

        \ff[9]{Martial Spell} You gain a \plus1 bonus to \glossterm{accuracy} with spells against creatures you \glossterm{threaten}.

        \ff[11]{Defensive Caster} You gain a \plus1 bonus to physical defenses.

        \ff[13]{Combat Concentration, Greater} The bonus to Concentration checks increases to \plus10.

        \ff[15]{Martial Spell, Greater} The accuracy bonus with spells against threatened creatures increases to \plus2.

        \ff[17]{Armor Tolerance, Greater} The reduction of arcane spell failure increases to 30\%. 

        \ff[19]{Spellstrike, Greater} The damage bonus if you maintained your concentration increases to \plus3d.
    \end{feat}

    \begin{feat}{Blindfighter}{Combat}
        \featpre Starting Perception 2.
        \featben
        
        \ff[1]{Blind Precision} Whenever you have a miss chance caused by being unable to see your opponent, you can roll the miss chance twice and take the better result.

        \ff[2]{Unseen Defense} You are not \defenseless against foes you cannot see if you know their location.

        \ff[4]{Blindsense} You gain \glossterm{blindsense} (50 ft.).

        \ff[6]{Attack the Unseen} If you know the location of a creature you cannot see, and you have \glossterm{line of effect} to that creature, you can target it with targeted abilities.

        \ff[8]{Blind Feint} If you are attacked by a creature who incorrectly thinks you are \glossterm{unaware} of the attack, you can take an \glossterm{immediate action} to spend an \glossterm{action point}.
        If you do, you can immediately make a \glossterm{strike} against that creature.

        \ff[10]{Blindsight} You gain \glossterm{blindsight} (50 ft.).
        In addition, the range of your blindsense improves to 200 feet. 

        \ff[12]{Controlled Sight} You are immune to all abilities that depend on sight to affect you.

        \ff[14]{Blindsight, Greater} The range of your blindsight improves to 100 feet.
        In addition, the range of your blindsense improves to 500 feet.

        \ff[16]{Blind Feint, Greater} When you use your \glossterm{blind feint} ability, the target is \glossterm{unaware} of the attack.

        \ff[18]{Supreme Sight} The range of your blindsight improves to 200 feet.
        In addition, the range of your blindsense improves to 1,000 feet.
    \end{feat}

    % move this into power attack?
    % \feat{Blows of Legendary Force}{Combat, Style}
    % \featpre 13th level, Strength 12.
    % \featben Whenever you make a \glossterm{strike}, you compare your result against the target's Armor and Fortitude defenses.
    % If you hit either defense, the strike hits and deals full damage.

    \begin{feat}{Bluff Specialization}{Skill}
        \featpre Bluff as a mastered skill.
        \featben

        \ff[1]{Specialization, Lesser} You gain a \plus2 bonus to Bluff.

        \ff[2]{Sustained Distraction} If you successfully distract a creature, you can sustain that distraction on that creature with a \glossterm{swift action} as long as you continue to be distracting.
        You must make a new check each round, and the DR increases by 2 for each round you have distracted them.
        For details, see \pcref{Distract}.

        \ff[4]{Deceive Magic} Any magical abilities which detect lies are unable to detect lies you speak.

        % You can make a Bluff attack to \glossterm{feint} instead of using your normal accuracy (see \pcref{Feint}).

        \ff[6]{Impersonation Tolerance} You reduce DR modifiers for details you cannot replicate when impersonating creatures 2 (see \pcref{Impersonate}).

        \ff[8]{Specialization} The bonus to Bluff increases to \plus3.

        \ff[10]{Dual Speech}[Mag] Whenever you speak, you can make a Bluff check to speak in two voices at once.
        The base DR is 15.
        Success means you speak the same words with two different vocal patterns, such as tone or accent, though you cannot significantly change the volume.
        You can freely choose which creatures hear which pattern, treating all creatures you are not aware of as a single group.
        Failure means you choose one of the vocal patterns and speak in that pattern.
        Critical failure means the words come out garbled and incomprehensible.

        If you increase the DR by 5, your vocal patterns can be in different languages.
        If you increase the DR by 10, your vocal patterns can use entirely different words.

        \ff[12]{Deceive Magic, Greater} Whenever you impersonate a creature, if your Bluff check is high enough, magical abilities treat you as if you were that creature.
        % is this DR high enough?
        The DR is equal to 10 \add the \glossterm{power} of the ability you are deceiving.
        % does this work mechanically?
        For example, if you impersonate an undead creature, the \spell{inflict light wounds} spell would heal you.
        This does not grant you any abilities associated with creatures you impersonate.

        \ff[14]{Impersonation Tolerance, Greater} The reduction of DR modifiers for impersonating details you cannot replicate increases to 5.

        \ff[16]{Specialization, Greater} The bonus to Bluff increases to \plus4.

        \ff[18]{Deceive Reality} Whenever you are attacked, after learning whether the attack succeeded or failed, you can spend an \glossterm{action point} as an immediate action.
        If you do, you can make a Bluff check to pretend the attack affected you differently.
        The DR is equal to the result of the attack roll.
        Success means you can choose the result of the attack roll, potentially causing it to miss instead of hitting, or to hit instead of missing.
        However, you cannot make the attack critically fail.
        After using this ability, you cannot use it again until you take a \glossterm{short rest}.
    \end{feat}

    \begin{feat}{Celestial Heritage}{Bloodline, Magical}
        \featpre Non-evil alignment
        \featben

        \ff[1]{Celestial Power} Your \glossterm{power} with abilities from this feat is equal to your Willpower or your level, whichever is higher.

        \ff[1]{Holy Blessing} As a standard action, you can spend an \glossterm{action point} to use this ability.
        \begin{ability}
            \begin{spelltargetinginfo}
                \spelltwocol{\spelltgt{One creature}}{\spellrng{\rngclose}}
            \end{spelltargetinginfo}
            \begin{spelleffects}
                \spelleffect The target gains a \plus2d bonus to damage with all attacks.
                \spelldur Attunement
                \spelltags{\glossterm{Good}, \glossterm{Magical}}
            \end{spelleffects}
        \end{ability}

        \ff[3]{Holy Protection} As a standard action, you can spend an \glossterm{action point} to use this ability.
        \begin{ability}
            \begin{spelltargetinginfo}
                \spelltwocol{\spelltgt{One creature}}{\spellrng{\rngclose}}
            \end{spelltargetinginfo}
            \begin{spelleffects}
                \spelleffect The target gains damage reduction equal to your celestial power against \glossterm{Evil} physical effects, and physical attacks made by evil creatures.
                \spelltags{\glossterm{Good}, \glossterm{Magical}}
            \end{spelleffects}
        \end{ability}

        \ff[5]{Angel Wings} You gain feathery wings that sprout from your back.
        You can use these wings to glide at a rate equal to your land speed (see \pcref{Gliding}).
        The wings themselves are \glossterm{mundane}, but the ability to glide and fly with them is \glossterm{magical}.

        \ff[7]{Complete Protection} The damage reduction from your \textit{holy protection} ability also applies against non-physical effects.

        \ff[9]{Empowered Blessing} The damage bonus from your \glossterm{holy blessing} ability increases to \plus3d.

        \ff[11]{Angelic Flight} Your \textit{angel wings} grant you a fly speed equal to your land speed.
        While \unencumbered, you can fly (see \pcref{Flying}).
        You can only fly for a number of rounds equal to half your celestial power.
        After that limit is reached, you must take a \glossterm{short rest} before flying again.

        \ff[13]{Holy Retribution} Whenever an evil creature makes a physical melee attack against the creature protected by your \textit{retributive protection} ability, you make a celestial power vs. Mental attack against the attacking creature.
        Success means the attacker takes 1d4 divine damage \plus1d per celestial power.

        \ff[15]{Empowered Blessing, Greater} The damage bonus from your \glossterm{holy blessing} ability increases to \plus4d.

        \ff[17]{} % another augment?

        \ff[19]{Angelic Flight, Greater} You no longer have a limit on how long you can fly with your \textit{angelic flight} ability.
    \end{feat}

    \begin{feat}{Reaper}{Combat}
        \spelldesc{You can attack with such force that you cleave through your foes.}
        \featpre Starting Strength of 2.

        \ff[1]{Sweeping Strike} As a standard action, you can spend an \glossterm{action point} to use this ability.
        \begin{ability}
            \begin{spelltargetinginfo}
                \spelltwocol{\spellarea{Up to three contiguous squares}}{\spellrng{Threatened}}
                \spelltgts{Everything in the area}
                \spellspecial Choose one or two melee weapons you are wielding that deal slashing or bludgeoning damage.
            \end{spelltargetinginfo}
            \begin{spelleffects}
                \spelleffect You make a melee \glossterm{strike} against the target with the chosen weapon or weapons.
                You take a \minus1d penalty to damage with the strike.
            \end{spelleffects}
        \end{ability}

        \ff[2]{Cleave} Whenever you get a \glossterm{critical hit} with a strike using a slashing or bludgeoning weapon, you can take an \glossterm{immediate action} to make another strike with the same weapon.
        The target of the new strike must be adjacent to the struck creature.
        You may continue making additional strikes as long as you keep getting critical hits.
        However, you may not attack the same creature more than once.

        \ff[4]{Reaping Charge} As a standard action, you can spend an \glossterm{action point} to use this ability.
        \begin{ability}
            \begin{spelltargetinginfo}
                \spelltgts{Special; see below}
                \spellspecial Choose one or two melee weapons you are wielding that deal slashing or bludgeoning damage.
            \end{spelltargetinginfo}
            \begin{spelleffects}
                \spelleffect You move up to half your movement speed in a straight line.
                Choose either the right or left side of the line.
                You can make a \glossterm{strike} against each creature and object on that side of the line that you threaten at any point during your movement.
                You take a \minus2d penalty to damage on the strike.
            \end{spelleffects}
        \end{ability}

        \ff[6]{Spinning Cleave} When you use your \textit{cleave} ability, the target of your additional attack does not have to be adjacent to the struck creature.

        \ff[8]{Wide Sweep} When you use your \textit{sweeping strike} ability, you can target up to five contiguous squares you threaten.

        \ff[10]{Reaping Whirlwind} When you use your \textit{reaping charge} ability, you do not have to choose a side of the line.
        You can attack any creatures and objects that you threaten at any point during your movement.

        \ff[12]{Mobile Cleave} When you use your \textit{cleave} ability, you can move up to half your movement speed before making the strike.
        This movement counts against your normal movement limit in that phase.

        \ff[14]{Reaping Charge, Greater} You can move up to your full movement speed when you use your \textit{reaping charge} ability, rather than half your movement speed.

        \ff[16]{Whirlwind Sweep} When you use your \textit{sweeping strike} ability, you can target any number of contiguous squares you threaten.

        \ff[18]{Reap the Harvest} When you use your \textit{reaping charge} or and \textit{sweeping strike} abilities, if every creature you attacked is dead at the end of the round, you regain the action point spent to use the ability.

    \end{feat}

    \begin{feat}{Climb Specialization}{Skill}
        \featpre Climb as a mastered skill.
        \featben

        \ff[1]{Specialization, Lesser} You gain a \plus2 bonus to Climb.

        \ff[2]{Damage Tolerance} Taking damage while climbing does not force you to make an additional Climb check to avoid falling.

        \ff[4]{Climb Speed} You gain a \glossterm{climb speed} equal to half your land speed.
        A successful Climb check to move allows you to travel a distance equal to your climb speed.

        \ff[6]{Scale the Beast}
        You gain a \plus2 bonus to Climb attacks you make to climb on other creatures (see \pcref{Creature Climb}).
        In addition, you gain a \plus1 bonus to accuracy with physical attacks against creatures you are climbing on.

        \ff[8]{Specialization} The bonus to Climb increases to \plus3.

        \ff[10]{Climb Speed, Greater} Your climb speed increases to be equal to your land speed.

        \ff[12]{Impossible Climber} You can climb surfaces that are perfectly smooth.
        The DR is 30 for perfectly smooth vertical surfaces, and 40 to climb on a perfectly smooth ceiling.
        In addition, you are treated as one size category smaller than normal for the purpose of determining which creatures you can climb on.

        \ff[14]{Scale the Beast, Greater} The bonus to Climb attack to climb on creatures increases to \plus4.
        In addition, the bonus to physical accuracy against creatures you are climbing on increases to \plus2.

        \ff[16]{Specialization, Greater} The bonus to Climb increases to \plus4.

        \ff[18]{Impossible Climber, Greater} You can wallrun on ceilings in the same way you wallrun on walls.
        In addition, the size category decrease for the purpose of climbing on creatures improves to two size categories smaller than normal.
    \end{feat}

    \begin{feat}{Combat Mobility}{Combat, Mobility}
        \featpre Starting Dexterity of 2.
        \featben

        \ff[1]{Unfettered Movement, Lesser} During each phase, you may move through one creature's space during movement.
        You move at half speed while in its space.
        After moving through that creature's space, other creatures block you as normal for the remainder of your movement.
        Certain unusual creatures that occupy their entire space, such as gelatinous cubes, may be immune to this ability.

        \ff[2]{Spring Attack} As a standard action, you can spend an \glossterm{action point} to use this ability.
        If you do, move up to your movement speed and make a \glossterm{strike}.
        As a \glossterm{delayed action}, you may continue moving if you have remaining movement available in the phase.

        \ff[8]{Unfettered Movement} Moving through spaces occupied by creatures does not reduce your movement speed.

        \ff[14]{Unfettered Movement, Greater} You can move through the spaces of any number of creatures in the same phase, rather than just one creature.
    \end{feat}

    \begin{feat}{Conjurer}{Magical, Spell}
        \featpre \glossterm{Conjuration} spell known.
        \featben

        \ff[1]{Astral Spirit} When you cast the \glossterm{summon monster} spell, you can manifest an astral spirit instead of an animal.
        An astral spirit is a floating, spirit-like creature with a translucent body.
        Its size is Medium, and it is vaguely humanoid in shape.
        It has a physical form, and occupies space like any other creature.

        An astral spirit does not have a land speed, but it has a 30 foot \glossterm{fly speed} with good maneuverability.
        In addition, it can teleport any distance as a move action as long as its destination is within \rngmed range of you.
        If an astral spirit hits with its strike, it deals arcane damage.

        \ff[3]{Astral Spell Transit, Lesser} Your attacks with spells ignore \glossterm{cover}, but not \glossterm{total cover}.

        \ff[5]{Fortified Manifestations} Objects and creatures you create with \glossterm{Manifestation} abilities have additional hit points equal to your spellpower.

        \ff[7]{Astral Echo} Whenever you teleport, you drift between your plane and the Astral Plane until the end of the next round.
        During this time, all attacks against you have a 20\% failure chance.

        \ff[9]{Regenerating Manifestations} Whenever you cast a spell, objects and creatures you have created with \glossterm{Manifestation} abilities heal hit points equal to your spellpower.

        \ff[11]{Astral Spell Transit} You double the range of all spells you cast.
        In addition, all \glossterm{Teleportation} spells you cast can teleport twice their normal distance.

        \ff[13]{Sustained Manifestations} Once per round, when you cast or sustain a spell, you can also sustain a \glossterm{Manifestation} spell as a \glossterm{free action}.

        \ff[15]{Fortified Manifestations, Greater} Objects and creatures you create with \glossterm{Manifestation} abilities have damage reduction equal to your spellpower.

        \ff[17]{Astral Spell Transit, Greater} When determining whether you have \glossterm{line of effect} to a particular location with a spell, you can ignore a single solid obstacle up to five feet thick.
        This can allow you to cast spells through solid walls, though it does not grant you the ability to see through the wall.

        \ff[19]{Astral Echo, Greater} You constantly drift between your plane and the Astral Plane.
        All attacks against you have a 20\% failure chance.
        You can suppress or resume this ability as a \glossterm{swift action}.
        In addition, whenever you teleport, your connection to the Astral Plane is strengthened until the end of the next round.
        During this time, all attacks against you have a 50\% failure chance.
    \end{feat}

    \begin{feat}{Duelist}{Combat}
        \featpre Starting Dexterity of 1, starting Intelligence of 1.
        \featben

        \ff[1]{Parry} Whenever a creature \glossterm{initiates} a \glossterm{strike} against you, you can take an \glossterm{immediate action} to attempt to parry its attack.
        If you do, you gain a \plus2 bonus to physical defenses against the attack.

        \ff[2]{Defensive Stance} As a \glossterm{swift action}, you can spend an \glossterm{action point} to use this ability.
        \begin{ability}
            \begin{spelltargetinginfo}
                \spellquicktargeting{One creature}{\rngmed}
                \spellspecial You can change the target of this ability to a new creature within range as a \glossterm{swift action}.
            \end{spelltargetinginfo}
            \begin{spelleffects}
                \spelleffect You gain a \plus2 bonus to physical defenses against the target.
                \spelldur Attunement
            \end{spelleffects}
        \end{ability}

        \ff[4]{} 

        \ff[6]{Overwhelm Tolerance} Your \textit{defensive stance} ability prevents you from suffering \glossterm{overwhelm penalties} against attacks from the target creature.

        \ff[8]{Riposte} Whenever you use your \textit{parry} ability, you can spend an \glossterm{action point}.
        If you do, you \glossterm{initiate} a \glossterm{strike} against the attacking creature.

        \ff[10]{Focused Defense} You gain a \plus1 bonus to physical defenses against attacks you do not suffer \glossterm{overwhelm penalties} on.

        \ff[14]{Rapid Parry} Once per round, you can \textit{parry} an attack without spending an immediate action.
    \end{feat}

    \begin{feat}{Craft Specialization}{Skill}
        \featpre Craft (any) as a mastered skill.
        \featben
    
        \ff[1]{Specialization, Lesser} You gain a \plus2 bonus to all Craft skills.
    
        \ff[2]{Craft Magic Item}[Magical] You can imbue items with magic using your crafting skill.
        Imbuing an item with magic takes material components, as described in \pcref{Magic Item Creation}.
        It takes you one hour per 10 gp of material components to create a item.

        You can also mend a broken magic item if it is one that you could make.
        Doing so costs a tenth of the raw materials and a quarter of the time it would take to craft that item in the first place.
        You cannot mend a \glossterm{destroyed} magic item.

        \ff[4]{Crafting Savant} You gain two additional \glossterm{skill points} which must be spent on Craft skills.

        \ff[6]{Rapid Creation} Crafting magic items takes you one hour per 100 gp of material components.

        \ff[8]{Specialization} The bonus to Craft skills increases to \plus3.

        \ff[10]{Rapid Creation} Crafting magic items takes you one hour per 500 gp of material components.

        \ff[12]{Crafting Savant, Greater} You gain two additional \glossterm{skill points} which must be spent on Craft skills.

        \ff[14]{Rapid Creation} Crafting magic items takes you one hour per 2,500 gp of material components.

        \ff[16]{Specialization, Greater} The bonus to Craft skills increases to \plus4.

        \ff[18]{Rapid Creation} Crafting magic items takes you one hour per 12,500 gp of material components.
    \end{feat}

    \begin{feat}{Creature Handling Specialization}{Skill}
        \featpre Creature Handling as a mastered skill.
        \featben

        \ff[1]{Specialization, Lesser} You gain a \plus2 bonus to Creature Handling.

        \ff[2]{Sustained Pacify} You can sustain the \textit{pacify} ability from the Creature Handling skill as a swift action, rather than as a standard action (see \pcref{Pacify}).

        \ff[4]{Compressed Training} You can teach a creature a trick with 12 hours of work, split as you choose, rather than in a week of 4 hour sessions (see \pcref{Training Creatures}).

        \ff[6]{Species Tolerance} You reduce Creature Handling DR modifiers for handling non-animals by 2.

        \ff[8]{Specialization} The bonus to Creature Handling increases to \plus3.
        In addition, you can train creatures to learn one bonus trick beyond their normal maximum (see \pcref{Bonus Tricks}).

        \ff[10]{Battleforged Training} You can teach a creature the Battleforged trick.
        The DR to train the trick is 20.
        A creature with the trick gains the following benefits:
        \begin{itemize}
            \item Its maximum hit points increase by an amount equal to its level.
            \item It gains a \plus1 bonus to accuracy with all attacks.
            \item It gains a \plus1d bonus to damage with \glossterm{strikes}.
        \end{itemize}

        \ff[12]{Rapid Pacify} You can use the \textit{pacify} ability from the Creature Handling skill as a swift action, rather than as a standard action.

        \ff[14]{Species Tolerance, Greater} The reduction of DR modifiers for handling non-animals increases to 5.
        This usually allows you to ignore penalties for working with non-animals.

        \ff[16]{Specialization, Greater} The bonus to Creature Handling increases to \plus4.
        In addition, the number of bonus tricks you can train creatures to learn increases to two.

        \ff[18]{Battleforged Training, Greater} You can teach a creature that has learned the Battleforged trick the Greater Battleforged trick.
        The DR to train the trick is 30.
        A creature with the trick gains the following benefits, which replace the benefits of the Battleforged trick:
        \begin{itemize}
            \item Its maximum hit points increase by an amount equal to twice its level.
            \item It gains a \plus2 bonus to accuracy with all attacks.
            \item It gains a \plus2d bonus to damage with \glossterm{strikes}.
        \end{itemize}
    \end{feat}

    \begin{feat}{Sniper}{Combat}
        \featpre Starting Perception of 2.
        \featben

        \ff[1]{Aim} As a standard action, you can use this ability.
        \begin{ability}
            \begin{spelltargetinginfo}
                \spellquicktargeting{One creature or object}{Line of sight}
            \end{spelltargetinginfo}
            \begin{spelleffects}
                \spelleffect You gain a \plus2 bonus to accuracy on \glossterm{strikes} against that target.
                \spelldur Sustain (swift). If you lose sight of the target for a full round, this effect ends.
            \end{spelleffects}
        \end{ability}

        \ff[2]{Penetrating Aim} Your physical ranged attacks ignore \glossterm{cover}, except total cover.

        \ff[4]{Distance Tolerance, Lesser} You reduce your accuracy penalties from \glossterm{range increments} by 2.

        \ff[6]{Sniper Shot} You gain a \plus2d bonus to damage on \glossterm{strikes} against \unaware creatures that are affected by your \textit{aim} ability.

        \ff[8]{Failure Tolerance} You ignore effects that give you a 20\% miss chance or failure chance with physical ranged attacks, such as \glossterm{concealment}.

        \ff[10]{Distance Tolerance} The reduction in accuracy penalties for range increments increases to 4.

        \ff[12]{Sustained Aim} You can sustain your \textit{aim} ability as a \glossterm{free action}.

        \ff[14]{Sniper Shot, Greater} The bonus to damage increases to \plus4d.

        \ff[16]{Distance Tolerance, Greater} The reduction in accuracy penalties for range increments increases to 6.

        \ff[18]{Rapid Aim} You can spend a \glossterm{action point} to use your \textit{aim} ability as a \glossterm{swift action}.
    \end{feat}

    \begin{feat}{Devices Specialization}{Skill}
        \featpre Devices as a mastered skill.
        \featben

        \ff[1]{Specialization, Lesser}[Magical] You gain a \plus2 bonus to Devices.

        \ff[2]{Rapid Improvisation} As a standard action, you can spend an \glossterm{action point} to use the \textit{improvise} ability to create a device (see \pcref{Improvise}).

        \ff[4]{Steady Hands} You cannot \glossterm{critically fail} on Devices checks.
        If you would critically fail, you simply fail instead, and suffer the normal penalties for non-critical failure.

        \ff[6]{Disable Arcana, Lesser}[Magical] You can disable arcane spell effects on objects or areas as if they were merely complex devices.
        You must be aware of an effect to disable it, either through the Spellcraft skill or because the effect is noticeable.
        You cannot disable effects on creatures.
        The DR to disable an effect is equal to 15 \add the effect's \glossterm{power}.
        Success means the spell is \glossterm{dispelled}.
        This has no effect on abilities that cannot be dispelled.

        \ff[8]{Specialization} The bonus to Devices increases to \plus3.

        \ff[10]{Improbable Improvisation} You reduce the DR for using the \textit{improvise} ability to make devices from unsuitable materials by 5.

        \ff[12]{Disable Arcana}[Magical] You can disable spell effects from any source, not just arcane spell effects.

        \ff[14]{Durable Improvization} Devices you create with the \textit{improvise} ability last for twice as many uses before they break.

        \ff[16]{Specialization, Greater} The bonus to Devices increases to \plus4.

        \ff[18]{Disable Arcana, Greater}[Magical] You can disable all magical effects on objects or areas, not just spell effects.
    \end{feat}

    \begin{feat}{Disguise Specialization}{Skill}
        \featpre Disguise as a mastered skill.
        \featben

        \ff[1]{Specialization, Lesser} You gain a \plus2 bonus to Disguise.

        \ff[2]{Quick Change} As a standard action, you can spend an \glossterm{action point} to use the \textit{disguise creature} or \textit{emulate creature} ability.

        \ff[4]{Disguise Aura}[Magical] Whenever you use the \textit{disguise creature} or \textit{emulate creature} abilities, you can decide how the target and any items on the target appear when examined by \glossterm{Divination} spells.
        For example, you could cause all of their equipment to appear nonmagical, or you could cause them to have a strong aura of good when examined with the \spell{detect alignment} spell.
        % Will this spell still exist?
        The maximum \glossterm{power} you can emulate is equal to your Disguise check result \minus15.

        Anyone using divination magic on the creature must make a spellpower check with a DR equal to your Disguise check result in order to perceive the truth.
        Regardless of the result of the check, the caster is not aware that the check was made.

        \ff[6]{Mismatch Tolerance} You reduce Disguise penalties for differences between the target's normal appearance and its intended appearance by 2.
        This allows you to ignore minor mismatches, such as if the target is a different gender than its intended appearance.

        \ff[8]{Specialization} The bonus to Disguise increases to \plus3.

        \ff[10]{Disguise Size}[Magical] As a \glossterm{standard action}, you can spend an action point to use this ability.
        \begin{ability}
            \begin{spelltargetinginfo}
                \spelltgt{You}
            \end{spelltargetinginfo}
            \begin{spelleffects}
                \spelleffect You increase or decrease your size by one \glossterm{size category}.
                \spelldur Attunement
            \end{spelleffects}
        \end{ability}

        \ff[14]{Mismatch Tolerance, Greater} The reduction of Disguise penalties for appearance differences increases to 5.
        This can allow you to ignore significant appearance differences.

        \ff[16]{Specialization, Greater} The bonus to Disguise increases to \plus4.
    \end{feat}

    \begin{feat}{Diviner}{Magical, Spell}
        \featpre \glossterm{Divination} spell known.
        \featben

        \ff[1]{Prophesy} As a \glossterm{standard action}, you can spend an \glossterm{action point} to use this ability.
        \begin{ability}
            \begin{spelleffects}
                \spellspecial The scope of this ability is limited to the next hour.
                This time period is called the \textit{time of prophecy}.
                When you use this ability, you visualize an action that a creature (or group of creatures) could take within the \textit{time of prophecy}.
                \spelleffect You see a brief, cryptic vision describing the most likely outcome of the action you visualized.
                This vision does not reveal any consequences that might occur after the \textit{time of prophecy} has ended.

                The vision does not have to be a literally accurate representation of the future.
                For example, if you used this ability to foresee the results of entering a room that had a group of creatures waiting in ambush, you might see a vision of flashing daggers in darkness darting towards your exposed back, regardless of whether the creatures would actually use daggers to attack.

                After using this ability, you cannot use it again until the \textit{time of prophecy} has ended, regardless of whether the action was taken.
            \end{spelleffects}
        \end{ability}

        \ff[3]{Precognitive Reaction, Lesser} You gain a \plus1 bonus to Reflex defense.
        In addition, you gain a \plus2 bonus to \glossterm{initiative} checks.

        \ff[5]{Deep Prophecy} When you use your \textit{prophesy} ability, you can increase the \textit{time of prophecy} to eight hours.
        This affects both the distance you can see into the future and the time you must wait before using the ability again.

        \ff[7]{Truesight} You gain the \glossterm{truesight} ability with a 50 foot range.

        \ff[9]{Precognitive Reaction} The bonus to initiative checks increases to \plus4.
        In addition, you are aware of all attacks against you, even those you cannot see, as long as you are conscious.
        This allows you to use abilities to defend yourself, and prevents you from being \unaware.

        \ff[11]{Dual Prophecy} You may use your \textit{prophesy} ability while you have an active \textit{time of prophecy}.
        If you have two active \textit{times of prophecy}, you must wait until one has expired to use your \textit{prophesy} ability again.

        \ff[13]{Truesight, Greater} The range of your \glossterm{truesight} increases to 500 feet.

        \ff[15]{Flexible Prophecy} When you use your \textit{prophesy} ability, you can choose the \textit{time of prophecy} to be any five-minute increment of time, to a minimum of thirty minutes and a maximum of eight hours.

        \ff[17]{Precognitive Reaction, Greater} The bonus to Reflex defense increases to \plus2, and the bonus to initiative checks increases to \plus6.
        In addition, you can never be surprised in combat.
        Whenever there are \glossterm{surprise phases}, you can act in them.

        \ff[19]{Oracle} When you use your \textit{prophesy} ability, the maximum \textit{time of prophesy} you can choose is increased to one week.
        In addition, you may have three active \textit{times of prophecy}, rather than two.
    \end{feat}

    \begin{feat}{Draconic Heritage}{Bloodline}
        \featben

        \ff[1]{Draconic Power} Your \glossterm{power} with abilities from this feat is equal to your level or your Constitution, whichever is higher.

        \ff[1]{Draconic Ancestry} Choose a type of dragon from among the dragons on \trefnp{Dragon Types}.
        You have the blood of that type of dragon in your veins.
        This grants you damage reduction equal to twice your \textit{draconic power} against the damage type that dragon's breath weapon deals.

        \ff[1]{Low-Light Vision} You gain \glossterm{low-light vision}.
        If you already have low-light vision, you instead double the benefit, allowing you to quadruple the illumination range of light sources.

        \ff[2]{Draconic Weapons} You gain a bite natural weapon and two claw natural weapons, one on each arm.
        For details, see \pcref{Natural Weapons}.

        \ff[2]{Darkvision} You gain \glossterm{darkvision} with a 50 foot range.
        If you already have darkvision, you instead increase the range of your existing darkvision by 50 feet.

        \ff[4]{Draconic Wings} You gain scaly wings that sprout from your back.
        These wings grant you a glide speed equal to your land speed (see \pcref{Gliding}).
        The wings themselves are physical, but the ability to glide with them is \glossterm{magical}.

        \ff[6]{Breath Weapon} As a \glossterm{standard action}, you can spend an \glossterm{action point} to use this ability.
        \begin{ability}
            \begin{spelltargetinginfo}
                \spellspecial The area affected by this ability depends on your \textit{draconic ancestry}, as described in \trefnp{Dragon Types}.
                \spellburst{\areamed cone or \arealarge line}
                \spelltgts{All creatures in the area}
            \end{spelltargetinginfo}
            \begin{spelleffects}
                \begin{spellattack}{Draconic power vs. Reflex}
                    \spellsuccess The target takes 1d6 damage \plus1d per two draconic power.
                    The damage type depends on your \textit{draconic ancestry}, as described in \trefnp{Dragon Types}.
                    \spellcritical As above, but double damage.
                \end{spellattack}
                \spellspecial After using this ability, you must wait one round before using it again.
            \end{spelleffects}
        \end{ability}

        \ff[8]{Draconic Flight, Lesser}[Magical] 
        As a standard action, if you are \unencumbered, you can use your draconic wings to fly with a \glossterm{fly speed} equal to your land speed.

        \ff[10]{Widened Breath} The area affected by of your \textit{breath weapon} increases.
        A line breath weapon becomes a \areahuge, 10 ft.\ wide line.
        A cone breath weapon becomes a \arealarge cone.

        \ff[12]{Draconic Weapons, Greater}[Magical] The natural weapons gain an \glossterm{enhancement bonus} equal to one quarter of your \textit{draconic power}.

        \ff[14]{Draconic Flight}[Magical] You can use your \textit{draconic flight} ability to fly as a \glossterm{swift action}.

        \ff[16]{Devastating Breath} The damage dealt by your \textit{breath weapon} increases by \plus1d.

        \ff[18]{Draconic Flight, Greater}[Magical] You can use your \textit{draconic flight} ability to fly as a \glossterm{free action}.
    \end{feat}

    \begin{dtable}
        \lcaption{Dragon Types}
        \begin{dtabularx}{\columnwidth}{>{\lcol}X >{\lcol}X >{\lcol}X}
            \tb{Dragon} & \tb{Energy Type} & \tb{Breath Weapon} \\
            \bottomrule
            Black & Acid & Line \\
            Blue & Electricity & Line \\
            Brass & Fire & Line \\
            Bronze & Electricity & Line \\
            Copper & Acid & Line \\
            Gold & Fire & Cone \\
            Green & Acid & Cone \\
            Red & Fire & Cone \\
            Silver & Cold & Cone \\
            White & Cold & Cone \\
        \end{dtabularx}
    \end{dtable}

    \begin{feat}{Enchanter}{Magical, Spell}
        \featpre \glossterm{Enchantment} spell known.
        \featben

        \ff[1]{Mind Fragments} When you use \glossterm{Mind} abilities, you can affect creatures with a \glossterm{mundane} immunity to \glossterm{Mind} abilities.
        You take a \minus5 penalty to accuracy on attacks against such creatures.
        This does not allow you to affect creatures with a \glossterm{magical} immunity to \glossterm{Mind} abilities.

        \ff[3]{Mind Scour} As a \glossterm{standard action}, you can spend an \glossterm{action point} to use this ability.
        \begin{ability}
            \begin{spelltargetinginfo}
                \spellquicktargeting{One creature}{\rngmed}
            \end{spelltargetinginfo}
            \begin{spelleffects}
                \begin{spellattack}{Spellpower vs. Mental} % Which spellpower?
                    % No good damage type for this yet
                    \spellsuccess The target takes \spelldamage{physical}[d6].
                    In addition, its Mental defense is lowered by 2.
                    This is a \glossterm{condition}, and lasts until it is removed.
                    \spellcritical As above, but double damage, and its Mental defense is lowered by 5 instead of 2.
                \end{spellattack}
                \spelltags{\glossterm{Mind}}
            \end{spelleffects}
        \end{ability}

        \ff[5]{Subtle Influence} The DR to identify your \glossterm{Mind} abilities with Spellcraft, and to identify their effects with Sense Motive, increases by 5.

        \ff[7]{Enchanting Presence} You gain a \plus1 bonus to Intimidate and Persuasion.
        In addition, creatures within a 50-foot radius \glossterm{emanation} of you take a \minus1 penalty to Mental defense.
        You may freely exclude creatures you are aware of from this effect.

        \ff[9]{Brutal Scouring} You gain a \plus1d bonus to damage with your \textit{mind scour} ability. 

        \ff[11]{Mind Fragments, Greater} The penalty to accuracy against creatures with \glossterm{mundane} immunity to \glossterm{Mind} effects decreases to \minus2.

        \ff[13]{Enchanting Presence, Greater} The bonus to Intimidate and Persuasion increases to \plus2.
        In addition, the penalty to Mental defense against creatures near you is increased to \minus2.

        \ff[15]{Mental Breach} You gain a \plus1 bonus to accuracy with your \textit{mind scour} ability.

        \ff[17]{Subtle Influence, Greater} The DR increase to identify your \glossterm{Mind} abilities increases to 10.

        \ff[19]{Mental Torment} Whenever you target a creature with a Mind ability, that creature takes a \minus1 penalty to Mental defense.
        This is a \glossterm{condition}, and lasts until it is removed.
        This penalty stacks with itself if you target the same creature multiple times.
    \end{feat}

    \begin{feat}{Toughness}{General}
        \featpre Starting Constitution of 2.
        \featben

        \ff[1]{Durable} You gain additional hit points equal to your Constitution.

        \ff[2]{Great Fortitude, Lesser} You gain a \plus1 bonus to Fortitude defense.

        \ff[4]{Fatigue Tolerance} You ignore effects which would make you \fatigued.
        This allows you to sleep in heavy or medium armor without penalty.
        In addition, any effect which would make you \exhausted makes you \fatigued instead.
        This ability does not allow you to ignore this fatigue.

        \ff[6]{Injury Tolerance} You take a \minus2 penalty for being \glossterm{bloodied} instead of a \minus5 penalty.
        In addition, having \glossterm{vital damage} imposes half the normal penalties.

        \ff[8]{Great Fortitude} The bonus to Fortitude defense increases to \plus2.

        \ff[10]{Fatigue Tolerance, Greater} You ignore effects which would make you \exhausted, and which compel you to sleep.
        In addition, you need half the normal amount of rest and sleep each day to function normally.
        For example, a human would only need four hours of sleep per night.

        \ff[12]{Durable, Greater} The increase to hit points increases to twice your Constitution.

        \ff[14]{Great Fortitude, Greater} The bonus to Fortitude defense increases to \plus3.

        \ff[16]{Injury Tolerance, Greater} You do not take penalties for being \glossterm{bloodied}.
        In addition, having \glossterm{vital damage} imposes one quarter of the normal penalties, rather than half.

        \ff[18]{Deathless} You cannot take more than ten damage per level during a single round.
        Any excess damage is ignored.
        In addition, you are immune to \glossterm{Death} effects.
    \end{feat}

    \begin{feat}{Escape Artist Specialization}{Skill}
        \featpre Escape Artist as a mastered skill.
        \featben

        \ff[1]{Specialization, Lesser} You gain a \plus2 bonus to Escape Artist.

        \ff[2]{Rapid Escape} You can squeeze and escape bindings as a move action, rather than as a standard action.

        \ff[4]{Constraint Tolerance, Lesser} You reduce your penalties to accuracy and defenses for \glossterm{squeezing} by 2.

        \ff[6]{Accelerated Squeeze} Your movement speed is not reduced while squeezing.

        \ff[8]{Specialization} The bonus to Escape Artist increases to \plus3.

        \ff[10]{Escape Magic}[Magical] As a standard action, you can spend an \glossterm{action point} to use this ability.
        \begin{ability}
            \begin{spelleffects}
                \spelleffect You make an Escape Artist attack against all \glossterm{magical} effects on you.
                The DR for each effect is equal to 10 \add the effect's \glossterm{power}.
                Success means the effect is \glossterm{dispelled}, if it is an effect that can be dispelled.
                \par You can only dispel effects which target you directly, not area effects which include you as a target.
                If an ability targets multiple creatures, you can only remove its effects on you.
            \end{spelleffects}
        \end{ability}

        \ff[12]{Constraint Tolerance, Greater} The penalty reduction for squeezing increases to 4.
        This normally allows you to ignore all penalties for squeezing.

        \ff[14]{Rapid Escape, Greater} You can escape bindings as a \glossterm{swift action}.

        \ff[16]{Specialization, Greater} The bonus to Escape Artist increases to \plus4.

        \ff[18]{Escape Magic, Greater}[Magical] You can use your \textit{escape magic} ability as a \glossterm{swift action}.
    \end{feat}

    \begin{feat}{Evoker}{Magical, Spell}
        \featpre \glossterm{Evocation} spell known.
        \featben

        \ff[1]{Energy Burst} As a standard action, you can spend an \glossterm{action point} to use this ability.
        \begin{ability}
            \begin{spelltargetinginfo}
                \spellspecial When you use this ability, you choose a type of energy: cold, electricity, fire, or sonic.
                \spellquicktargeting{One creature or object}{\rngclose}
            \end{spelltargetinginfo}
            \begin{spelleffects}
                \begin{spellattack}{Spellpower vs. Special}
                    \spellspecial If you chose electricity or fire energy, this attack is made against Reflex defense. If you chose cold or sonic energy, this attack is made against Fortitude defense.
                    \spellsuccess The target takes 1d8 damage \plus1d per two spellpower.
                    The type of energy you choose determines the type of damage dealt.
                    \spellcritical As above, but double damage.
                \end{spellattack}
                \spelltags{As the energy type chosen}
            \end{spelleffects}
        \end{ability}

        % Which spellpower?
        \ff[3]{Energy Resistance} You gain damage reduction against \glossterm{energy damage} equal to your spellpower.

        \ff[5]{Mighty Telekinesis} You gain a \plus1 bonus to accuracy with the \spell{telekinesis} spell.

        \ff[7]{Residual Energy Burst} When your attack succeeds with your \textit{energy burst} ability, the target suffers an additional effect depending on the energy type chosen.
        These effects are \glossterm{conditions}, and last until they are removed.
        \begin{itemize}
            \item Cold: If target is a creature, it is \fatigued.
            \item Electricity: If the target is a creature, it is \impaired with attacks and checks.
            \item Fire: The target is \ignited.
            \item Sonic: If the target has \glossterm{hardness}, its hardness is reduced by an amount equal to half your spellpower. Otherwise, if it has \glossterm{damage reduction}, its damage reduction is reduced by amount equal to your spellpower.
        \end{itemize}

        \ff[9]{Potent Evocation} You gain a \plus1d bonus to damage with \glossterm{Evocation} abilities that deal damage measured in a dice pool, as well as with the \textit{energy burst} ability.

        \ff[11]{Energy Resistance, Greater} The damage reduction against \glossterm{energy damage} increases to twice your spellpower.

        \ff[15]{Devastating Evocation} The area affected by your Evocation spells that affect areas doubles.
    \end{feat}

    \begin{feat}{Executioner}{Combat}
        \featpres Starting Strength of 1, starting Perception of 2.
        \featben

        \ff[1]{Strip the Flesh} As a standard action, you can spend an \glossterm{action point} to use this ability.
        \begin{ability}
            \begin{spelltargetinginfo}
                \spellquicktargeting{One creature or object}{As weapon}
            \end{spelltargetinginfo}
            \begin{spelleffects}
                \spelleffect You make a \glossterm{strike} against the target.
                At the end of the round, if the target is not \glossterm{bloodied}, it takes additional damage equal to the damage you dealt to it with your strike.
            \end{spelleffects}
        \end{ability}

        \ff[2]{Purge the Weak, Lesser} You gain a \plus1 bonus to accuracy with physical attacks against \glossterm{bloodied} creatures.

        \ff[4]{Final Blow} Whenever you deal \glossterm{vital damage} to a creature with a \glossterm{strike}, the creature immediately dies.

        \ff[6]{Bloodfeeder}[Life, Magical] Whenever a creature dies, if you dealt damage to it that round with a \glossterm{strike}, you heal hit points equal to its level.

        \ff[8]{Salt the Wound} When you use your \textit{strip the flesh} ability, the target continues taking damage at the end of each round until it becomes \glossterm{bloodied}.
        This is a \glossterm{condition}, and can be removed by abilities that remove conditions.

        \ff[10]{Purge the Weak} The accuracy bonus against \glossterm{bloodied} creatures increases to \plus2.

        \ff[12]{Final Blow, Greater} Whenever a creature's hit points drop to zero, if you dealt damage to it that round with a \glossterm{strike}, it immediately dies.
        In addition, whenever you deal damage to a creature that has no hit points remaining with a \glossterm{strike}, it immediately dies.

        \ff[14]{Bloodfeeder, Greater}[Life, Magical] Whenever a creature dies, if you used your \textit{strip the flesh} ability on it since the last time it took a \glossterm{short rest}, you regain an \glossterm{action point}.

        \ff[18]{Purge the Weak, Greater} The accuracy bonus against \glossterm{bloodied} creatures increases to \plus3.
    \end{feat}

    \begin{feat}{Whirlwind Warrior}{Combat}
        \featpres Starting Dexterity of 2, starting Perception of 1.
        \featben

        \ff[1]{Whirlwind Strike} As a standard action, you can spend an \glossterm{action point} to use this ability.
        \begin{ability}
            \begin{spelleffects}
                \spelleffect You make a melee \glossterm{strike} against any number of creatures and objects you \glossterm{threaten}.
                You must use the same weapon to make each strike.
                Use the same attack result and damage against each target.
                You take a \minus2d penalty to damage on the strike.
            \end{spelleffects}
        \end{ability}

        \ff[2]{Eye of the Storm, Lesser} You reduce your \glossterm{overwhelm penalties} by 1.
        If your overwhelm penalty is reduced to 0, you are not considered to be \glossterm{overwhelmed}.

        \ff[8]{Eye of the Storm} The penalty reduction increases to 2.

        \ff[14]{Eye of the Storm, Greater} The penalty reduction increases to 3.
    \end{feat}

    \begin{feat}{Intimidate Specialization}{Skill}
        \featpre Intimidate as a mastered skill.
        \featben

        \ff[1]{Specialization, Lesser} You gain a \plus2 bonus to Intimidate.

        \ff[2]{Critical Demoralization} If you get a \glossterm{critical hit} when you take the \textit{demoralize} action, the target is \frightened by you instead of being shaken.
        See \pcref{Demoralize}, for details.

        \ff[6]{Demoralizing Blow} Whenever you deal damage to a creature, you can spend an \glossterm{action point} to use the \textit{demoralize} ability as an \glossterm{immediate action}.

        \ff[8]{Specialization} The bonus to Intimidate increases to \plus3.

        \ff[16]{Specialization, Greater} The bonus to Intimidate increases to \plus4.
    \end{feat}

    \begin{feat}{Guardian}{Combat}
        \featpre Starting Perception and Willpower of 1.
        \featben

        \ff[1]{Redirection} Whenever an ally adjacent to you is hit by a \glossterm{strike}, you may use this ability as an \glossterm{immediate action}.
        If you do, you suffer all effects of the attack in place of the target.
        Any abilities you have that would make the strike miss or fail have no effect, but your abilities that allow you to reduce or ignore its effects work normally.
        You can learn which strikes hit the target in the current phase before deciding which strike to redirect, but not how much damage they would do.

        \ff[2]{Defend the Weak, Lesser} You reduce the \glossterm{overwhelm penalties} of allies adjacent to you by 1.
        If an ally's overwhelm penalty is reduced to 0, it not considered to be \glossterm{overwhelmed}.

        \ff[4]{Binding Strike} As a standard action, you can spend an \glossterm{action point} to use this ability.
        \begin{ability}
            \begin{spelltargetinginfo}
                \spellquicktargeting{One creature}{5 feet}
            \end{spelltargetinginfo}
            \begin{spelleffects}
                \spelleffect You make a \glossterm{strike} against the target.
                If the target takes damage from this strike, it is \immobilized.
                This effect is immediately broken if you stop being adjacent to the target.
                \spelldur Condition
            \end{spelleffects}
        \end{ability}

        \ff[6]{Expanded Redirection} You can use your \textit{redirection} ability to redirect the effects of any \glossterm{mundane} attack, not just \glossterm{strikes}.

        \ff[8]{Defend the Weak} The penalty reduction increases to 2.

        \ff[10]{Certain Bind} You gain a \plus1 bonus to accuracy with your \textit{binding strike} ability.

        \ff[12]{Martyr's Boon}[Magical] You can use an \glossterm{action point} to use your \textit{intercession} ability on any ally within \rnglong range of you.

        \ff[14]{Defend the Weak, Greater} The penalty reduction increases to 3.

        \ff[16]{Inescapable} Enemies you threaten must pay four times the normal movement cost to move out of squares you threaten.
        This replaces the normal penalties for moving through threatened squares (see \pcref{Moving Near Foes}).

        \ff[18]{Expanded Redirection, Greater}[Magical] You can use your \textit{redirection} ability to redirect the effects of any attack, not just \glossterm{mundane} attacks. 
    \end{feat}

    \begin{feat}{Heal Specialization}{Skill}
        \featpre Heal as a mastered skill.
        \featben

        \ff[1]{Specialization, Lesser} You gain a \plus2 bonus to Heal.

        \ff[2]{Healing Touch}[Magical] As a standard action, you can spend an \glossterm{action point} to use this ability.
        \begin{ability}
            \begin{spelltargetinginfo}
                \spellquicktargeting{One willing creature}{\rngtouch}
            \end{spelltargetinginfo}
            \begin{spelleffects}
                \spelleffect Make a Heal check. The target is healed for an amount equal to the Heal check result.
                % 10th level: 1d8 +1d per two levels is 4d6, or avg of 14 healing.
                % heal check average is 5 + 10 + 2 + 3 = 20 healing
                % 20th level: 1d8 +1d per two levels is 7d10, or avg of 38 healing.
                % heal check average is 5 + 20 + 2 + 4 = 31 healing
            \end{spelleffects}
        \end{ability}

        \ff[4]{Lifesaver} You gain a \plus5 bonus to Heal checks to stabilize dying creatures (see \pcref{Dying}).

        \ff[6]{Vital Healing} For every five points of healing you would restore with your \textit{healing touch}, you can instead heal a point of \glossterm{vital damage}.

        \ff[8]{Specialization} The bonus to Heal increases to \plus3.

        \ff[10]{Purging Touch}[Magical] As a standard action, you can spend an \glossterm{action point} to use this ability.
        \begin{ability}
            \begin{spelltargetinginfo}
                \spellquicktargeting{One willing creature}{\rngtouch}
            \end{spelltargetinginfo}
            \begin{spelleffects}
                \begin{spellattack}{Heal vs. Special}
                    \spellspecial The attack result is applied to every poison and disease on the target.
                    The DR for each effect is equal to 10 \add the \glossterm{power} of the effect.
                    \spellsuccess Success against a poison or disease causes it to be removed from the target, along with all of its lingering effects.
                \end{spellattack}
            \end{spelleffects}
        \end{ability}

        \ff[12]{Empowered Healing} When you use your \textit{healing touch} ability, you can heal additional hit points equal to your level.

        \ff[14]{Lifesaver, Greater} You can stabilize dying creatures as a \glossterm{swift action}.

        \ff[16]{Specialization, Greater} The bonus to Heal increases to \plus4.

        \ff[18]{Wellspring of Life} You do not have to spend an \glossterm{action point} to use your \textit{healing touch} ability.
    \end{feat}

    \begin{feat}{Illusionist}{Magical, Spell}
        \featpre \glossterm{Illusion} spell known.
        \featben

        \ff[1]{Create Image} As a standard action, you can spend an \glossterm{action point} to use this ability.
        \begin{ability}
            \begin{spelltargetinginfo}
                \spellrng{\rngmed}
            \end{spelltargetinginfo}
            \begin{spelleffects}
                \spelleffect This ability creates the visual illusion of an object, creature, or force, as determined by you.
                The figment's size must be no smaller than Tiny, and no larger than Large.
                The figment does not create sound, smell, or temperature.

                During the movement phase, you can move the figment anywhere within the range, with appropriate motions to simulate natural movement.
                For example, if you created the illusion of a squad of human guards, you could cause them to walk realistically across a room.
                The figments otherwise remain motionless, except for minor motions that simulate signs of life (if appropriate).
                If the figment ever leaves this ability's range, the effect immediately ends.

                The maximum intensity of a sensation created by this ability is not enough to have any significant detrimental effects on a human experiencing the sensation.
                For example, it can create a bright light, but not so bright that it would be physically painful to view.
                % should the light be visible outside the size of the illusion? If so, how to clarify?

                When you use this ability, you make a check with a bonus equal to your spellpower \add 5.
                Creatures can recognize the figment is created by illusory magic by interacting with it physically, or by making an Awareness check against a DR equal to your check result when using this ability.
                A creature gets a \plus10 bonus on this Awareness check when using senses which should be present in the figment, but which are missing.
                \spelldur Sustain (swift)
                \spelltags{Figment}
            \end{spelleffects}
        \end{ability}

        \ff[3]{Reflexive Illusion, Lesser} You gain a \plus1 bonus to Disguise, Sleight of Hand, and Stealth.

        \ff[5]{Controlled Image} While your \textit{create image} ability is active, you can concentrate on the figment as a standard action.
        If you do, you can directly control the figment's movement for the rest of the round, including their actions during the action phase.
        You cannot alter the fundamental shape of the figment, but you can have it perform complex actions, such as pretending to fight other creatures or engaging in a vigorous dance.
        The figment's ability to simulate physical tasks that require dexterity or training, such as juggling, is limited by your own.
        You may need to make relevant checks to make the figment perform complex actions.

        \ff[7]{Muffled Illusions} The level cost to apply the Silent standard augment to Illusion spells you cast is reduced by 1, to a minimum of 0 (see \pcref{Standard Augments}).
        In addition, choose a sense: sound, smell, or temperature.
        Your \textit{create image} ability can create sensations with the chosen sense.
        % How loud is the sound? How pungent is the smell? etc.

        \ff[9]{Flexible Image} Your \textit{create image} ability can create figments between Diminuitive and Huge size.

        \ff[11]{Reflexive Illusion} The bonus to skill checks increases to \plus2.

        \ff[13]{Subtle Illusions} The level cost to apply the Stilled standard augment to Illusion spells you cast is reduced by 1, to a minimum of 0 (see \pcref{Standard Augments}).
        In addition, choose a sense: sound, smell, or temperature.
        Your \textit{create image} ability can create sensations with the chosen sense.

        \ff[15]{Flexible Image, Greater} Your \textit{create image} ability can create figments between Fine and Gargantian size.

        \ff[17]{Reflexive Illusion} The bonus to skill checks increases to \plus3.
        In addition, choose a sense: sound, smell, or temperature.
        Your \textit{create image} ability can create sensations with the chosen sense.
    \end{feat}

    \begin{feat}{Perform Specialization}{Skill}
        \featpre Perform as a mastered skill.
        \featben

        \ff[1]{Specialization, Lesser} You gain a \plus2 bonus to Perform.

        \ff[2]{Inspiring Performance}[Magical] As a standard action, you can spend an \glossterm{action point} to use this ability.
        When you do, you begin a performance using one of your Perform skills.
        \begin{ability}
            \begin{spelltargetinginfo}
                \spellquicktargeting{One creature}{\rngmed}
            \end{spelltargetinginfo}
            \begin{spelleffects}
                \spelleffect Make a Perform check to create an performance that can inspire the target.
                The target gains temporary hit points equal to your check result.
                \spelldur Sustain (swift). Sustaining this ability requires continuing the performance you started when you used this ability. If the target can neither see nor hear your performance, the effect immediately ends.
                \spelltags{\glossterm{Delusion}, \glossterm{Mind}. In addition, this may have the \glossterm{Auditory} or \glossterm{Visual} tags, depending on the nature of the performance.}
            \end{spelleffects}
        \end{ability}

        \ff[4]{Mesmerizing Performance}[Magical] As a standard action, you can spend an \glossterm{action point} to use this ability.
        When you do, you begin a performance using one of your Perform skills.
        \begin{ability}
            \begin{spelltargetinginfo}
                \spellquicktargeting{Up to five creatures}{\rngmed}
            \end{spelltargetinginfo}
            \begin{spelleffects}
                \begin{spellattack}{Perform vs. Mental}
                    \spellspecial If the target thinks that you or your allies are threatening it, you take a \minus5 penalty to accuracy on the attack.
                    \spellsuccess The target is \fascinated by you.
                    Any act by you or your apparent allies that threatens or damages the target breaks the effect.
                    An observant target may interpret overt threats to its allies as a threat to itself.
                \end{spellattack}
                \spelldur Sustain (swift). Sustaining this ability requires continuing the performance you started when you used this ability. If the target can neither see nor hear your performance, the effect immediately ends.
                \spelltags{\glossterm{Compulsion}, \glossterm{Mind}. In addition, this may have the \glossterm{Auditory} or \glossterm{Visual} tags, depending on the nature of the performance.}
            \end{spelleffects}
        \end{ability}

        \ff[6]{Inspire Courage}[Magical] A target affected by your \textit{inspiring performance} is immune to \glossterm{Fear} abilities as long as the effect lasts.

        \ff[8]{Mass Performance}[Magical] You can target up to two creatures with your \textit{inspiring performance} ability.
        In addition, you can target any number of creatures with your \textit{mesmerizing performance} ability.

        \ff[8]{Specialization} The bonus to Perform increases to \plus3.

        \ff[10]{Mesmeric Suggestion}[Magical] As a standard action, you can spend an \glossterm{action point} to use this ability.
        \begin{ability}
            \begin{spelltargetinginfo}
                \spellquicktargeting{One creature}{\rngmed}
            \end{spelltargetinginfo}
            \begin{spelleffects}
                \spellspecial When you use this ability, you make a verbal suggestion of a particular course of action to the target.
                You can work this suggestion into an active performance without penalty.
                \begin{spellattack}{Perform vs. Mental}
                    \spellspecial If the target is not currently \fascinated by your \textit{fascinating performance} ability, this attack automatically fails.
                    If your suggestion does not seem reasonable, you take a \minus5 penalty to accuracy on this attack.
                    Exceptionally unreasonable suggestions can impose even greater penalties, and exceptionally reasonable suggestions can give accuracy bonuses.
                    \spellsuccess The target thinks your suggestion is a good idea and will try to follow it to the best of its abilities.
                    Any act by you or your apparent allies that threatens or damages the target breaks the effect.
                \end{spellattack}
                \spelldur Sustain (swift). Sustaining this ability requires continuing the performance you started when you used this ability. If the target can neither see nor hear your performance, the effect immediately ends.
                \spelltags{\glossterm{Compulsion}, \glossterm{Mind}. In addition, this may have the \glossterm{Auditory} or \glossterm{Visual} tags, depending on the nature of the performance.}
            \end{spelleffects}
        \end{ability}

        \ff[12]{Endless Fascination}[Magical] You can use your \textit{mesmerizing performance} ability without spending an action point.

        \ff[14]{Irresistible Dance} You ignore all failure chances on Perform attacks and checks you make.

        \ff[16]{Mass Performance, Greater}[Magical] When you use your \textit{inspiring performance} ability, you can target any number of creatures.
        In addition, the range of your \textit{mesmerizing performance} ability is increased to \rnglong.

        \ff[16]{Specialization, Greater} The bonus to Perform increases to \plus4.

        % maybe add ability to remove combat penalty for fascination ability

        \ff[18]{Inspire Heroism} Each creature affected by your \textit{inspiring performance} gains additional temporary hit points equal to your level.
    \end{feat}

    \begin{feat}{Bardic Exemplar}{Magical, Skill}
        \featpre Perform Specialization feat.
        \featben

        \ff[1]{Mocking Performance} As a standard action, you can spend an \glossterm{action point} to use this ability.
        When you do, you begin a performance using one of your Perform skills.
        \begin{ability}
            \begin{spelltargetinginfo}
                \spellquicktargeting{One creature}{\rngmed}
            \end{spelltargetinginfo}
            \begin{spelleffects}
                \begin{spellattack}{Perform vs. Mental}
                    \spellsuccess The target is \taunted by a willing creature of your choice within range of you.
                \end{spellattack}
                \spelldur Sustain (swift). Sustaining this ability requires continuing the performance you started when you used this ability. If the target can neither see nor hear your performance, the effect immediately ends.
                \spelltags{\glossterm{Delusion}, \glossterm{Mind}. In addition, this may have the \glossterm{Auditory} or \glossterm{Visual} tags, depending on the nature of the performance.}
            \end{spelleffects}
        \end{ability}

        \ff[3]{Perform Point} You gain a perform point.
        A perform point can be spent to use \glossterm{magical} abilities from this feat and the Perform Specialization feat in place of an action point.
        You recover all spent perform points after a \glossterm{short rest}.

        \ff[5]{Song of Serenity} As a standard action, you can spend an \glossterm{action point} to use this ability.
        When you do, you begin a performance using one of your Perform skills.
        \begin{ability}
            \begin{spelltargetinginfo}
                \spellquicktargeting{One creature}{\rngmed}
            \end{spelltargetinginfo}
            \begin{spelleffects}
                \spelleffect The target is immune to hostile \glossterm{Mind} effects.
                \spelldur Sustain (swift). Sustaining this ability requires continuing the performance you started when you used this ability. If the target can neither see nor hear your performance, the effect immediately ends.
                \spelltags{\glossterm{Delusion}, \glossterm{Mind}. In addition, this may have the \glossterm{Auditory} or \glossterm{Visual} tags, depending on the nature of the performance.}
            \end{spelleffects}
        \end{ability}

        \ff[7]{Demoralizing Mockery} A creature affected by your \textit{mocking performance} ability takes a \minus2 penalty to Mental defense as long as the effect lasts.

        \ff[9]{Hybrid Performance} You can sustain two different \glossterm{magical} abilities from this feat or the Perform Specialization feat as part of the same \glossterm{swift action}.

        \ff[11]{Mass Performance} When you use your \textit{mocking performance} ability, you can target up to five creatures.
        Each affected creature is taunted to attack the same creature.
        In addition, when you use your \textit{song of serenity} ability, you can target up to two creatures.

        \ff[13]{Battle Cry} As a standard action, you can spend an \glossterm{action point} to use this ability.
        When you do, you begin a performance using one of your Perform skills.
        \begin{ability}
            \begin{spelltargetinginfo}
                \spellquicktargeting*{Any number of creatures}{\rngmed}
            \end{spelltargetinginfo}
            \begin{spelleffects}
                \spelleffect The target gains a \plus2 bonus to accuracy with all attacks.
                \spelldur Sustain (swift). Sustaining this ability requires continuing the performance you started when you used this ability. If the target can neither see nor hear your performance, the effect immediately ends.
                \spelltags{\glossterm{Delusion}, \glossterm{Mind}. In addition, this may have the \glossterm{Auditory} or \glossterm{Visual} tags, depending on the nature of the performance.}
            \end{spelleffects}
        \end{ability}

        \ff[15]{Serene Bliss} A creature affected by your \textit{song of serenity} ability is not \glossterm{bloodied} as long as the effect lasts and it has hit points remaining.
        It suffers the normal penalties for becoming bloodied if the effect ends or it has no hit points remaining.

        \ff[17]{Mass Performance, Greater} When you use your \textit{mocking performance} ability, you can target any number of creatures.
        Each affected creature is taunted to attack the same creature.
        In addition, when you use your \textit{song of serenity} ability, you can target up to five creatures.
        Finally, the range of your \textit{battle cry} ability is increased to \rnglong.

        \ff[19]{Champion's Anthem} A creature affected by your \textit{battle cry} ability may use your level in place of its accuracy with any attack.

        \ff[19]{Hideous Laughter} If you get a \glossterm{critical hit} against a creature with your \textit{mocking performance} ability, it laughs uncontrollably until the end of the next round.
        During this time, it can take no other actions.
    \end{feat}

    \feat{Iron Will}{General}
    \featpre Willpower 3.
    \featben You gain a \plus2 bonus to your Mental defense.
    This bonus can increase your hit points (see \pcref{Hit Points}).

    At 3rd level, you become immune to hostile \glossterm{Compulsion} effects.

    At 7th level, if you have 6 Willpower, the defense bonus increases to \plus4.

    At 11th level, if you have 9 Willpower, you become immune to hostile \glossterm{Mind} effects.

    \feat{Item Conduit}{General, Magical}
    \featpre 5th level.
    \featben You gain two additional daily item uses.

    At 7th level, you gain a third additional daily item use.

    At 11th level, you recover one spent item use per hour.

    At 15th level, you gain a fourth additional daily item use.

    \feat{Intimidate Mastery}{Skill}
    \featpre Intimidate 6 ranks.
    \featben Whenever you make a Intimidate check, you may roll twice and take the higher result.

    At 3rd level, critical success when you demoralize a foe means the target is \frightened by you for 2 rounds instead of being shaken.

    At 7th level, if you have 10 ranks in Intimidate, you gain a \plus5 bonus to Intimidate checks.

    At 11th level, if you have 14 ranks in Intimidate, you can demoralize creatures as a \glossterm{swift action}.

    \feat{Jump Mastery}{Skill}
    \featpre Jump 6 ranks.
    \featben Whenever you make a Jump check, you may roll twice and take the higher result.

    At 3rd level, you are always treated as if you had a running start when jumping.

    At 7th level, if you have 10 ranks in Jump, you gain a \plus5 bonus to Jump checks.

    At 11th level, if you have 14 ranks in Jump, your maximum horizontal distance when leaping is equal to twice your check result, rather than being equal to your check result.
    In addition, your maximum height is equal to your check result, rather than a quarter of your check result.

    \feat{Knowledge Mastery}{Skill}
    \featpre Knowledge (any) 6 ranks.
    \featben Whenever you make a Knowledge check, you may roll twice and take the higher result.

    At 3rd level, as long as you have mastered one Knowledge skill, you treat all Knowledge skills you were trained in as if you had mastered them.
    This grants you additional skill ranks and increases your check modifier as normal for mastering a skill.
    However, you cannot use these additional ranks to qualify for any feats or abilities.

    At 7th level, if you have 10 ranks in any Knowledge skill, you gain a \plus5 bonus to all Knowledge checks.

    At 11th level, if you have 14 ranks in any Knowledge skill, you can gain the effect of the \spell{legend lore} ritual as a standard action.
    You can use this ability once per day.

    \feat{Legendary Celerity}{General}
    \featpre 13th level, Dexterity 12.
    \featben You may take an additional \glossterm{swift action} or \glossterm{immediate action} each round.

    \feat{Legendary Combat Awareness}{Combat}
    \featpre 13th level, Perception 12.
    \featben In each phase, you may decide your actions after all other creatures have decided their actions.
    This can allow you to observe and react to the actions that other creatures will take.
    You must decide your actions simultaneously with any other creatures with equivalent abilities.

    \feat{Legendary Combat Reaction}{Combat}
    \featpre 13th level, Dexterity 12.
    \featben You reduce your \glossterm{overwhelm penalties} by 2.
    If your overwhelm penalty is reduced to 0, you are not considered to be overwhelmed.
    As long as you are not overwhelmed, your movement is not impeded by enemies threatening you (see \pcref{Moving Near Foes}).

    \feat{Legendary Combat Tactics}{Combat}
    \featpre 13th level, Intelligence 12.
    \featben As a \glossterm{swift action}, you can predict the action that a target creature within \rngmed range intends to take during the current phase.
    You can attempt to inform other creatures of this information to help them choose their actions, though the target may change its action if it notices this.

    \feat{Legendary Mental Fortitude}{General}
    \featpre 13th level, Willpower 12.
    \featben You gain a \plus5 bonus to Fortitude defense.
    In addition, you gain \glossterm{damage reduction} against all damage equal to your Willpower.

    \feat{Legendary Might}{General}
    \featpre 13th level, Strength 12.
    \featben Your weight limits are five times their normal values (see \tref{Weight Limits}).
    In addition, you increase the damage of weapons you wield by one die increment (see \pcref{Weapon Size}).

    At 19th level, you increase the damage of weapons you wield by two die increments.

    \feat{Legendary Precision}{Combat}
    \featpre 13th level, Perception 12.
    \featben You ignore all miss chances and failure chances on your physical attacks.

    \feat{Legendary Skill Savant}{Skill}
    \featpre 13th level.
    \featben You gain ten skill points.

    \feat{Legendary Speed}{Combat, Style}
    \featpres 13th level, Dexterity 12.
    \featben You can make an additional \glossterm{strike} whenever you make a \glossterm{standard attack} .
    This does not stack with other effects that grant extra strikes.

    At 19th level, if you have 15 Dexterity, you can make a strike as a \glossterm{swift action}.
    \stylereq None.
    \parhead{Special} This is a Style feat, and all of its effects only apply while you are in this style.
    For details, see \pcref{Style Feats}.

    \feat{Legendary Spell Alacrity}{Magical}
    \featpre 13th level, 6th level spells.
    \featben The effects of your spells are no longer \glossterm{delayed}.
    This makes it impossible for simultaneous attacks against you to break your concentration, and allows you to cast spells after delaying your action.
    In addition, you gain a \plus1 bonus to spellpower.

    \feat{Legendary Well of Magic}{Magical}
    \featpres 13th level, 6th level spells known.
    \featben You may cast spells without spending spell slots, up to a maximum spell slot level.
    The maximum is equal to two less than the level of your highest level spell known.
    In addition, you gain a \plus1 bonus to spellpower.

    \feat{Lightning Reflexes}{General}
    \featpre Dexterity 3.
    \featben You gain a \plus2 bonus to your Reflex defense.

    At 3rd level, you gain a \plus5 bonus to initiative checks.

    At 7th level, if you have 6 Dexterity, the defense bonus increases to \plus4.

    At 11th level, if you have 9 Dexterity, the initiative check bonus increases to \plus10.

    \feat{Linguistic Mastery}{Skill}
    \featpre Linguistics 6 ranks.
    \featben Whenever you make a Linguistics check, you may roll twice and take the higher result.

    At 3rd level, \tdash.

    At 7th level, if you have 10 ranks in Linguistics, you gain a \plus5 bonus to Linguistics checks.

    At 11th level, if you have 14 ranks in Linguistics, you can speak, read, and understand all languages.
    This does not allow you to speak with creatures that lack a language.
    Certain extremely obscure languages may be beyond your knowledge.

    \feat{Mage Slayer}{Combat, Style}
    \featpre 5th level.
    \featben While in this style, if you hit a creature with a physical melee attack, that creature automatically fails Concentration checks it makes that round.

    At 7th level, all foes you threaten take a \minus4 penalty to Concentration checks.

    At 11th level, you gain \glossterm{magic resistance} equal to 10 \add your level.

    At 15th level, if you attack a creature with a physical melee attack, that creature automatically fails Concentration checks it makes that round.
    \stylereq Wield a melee weapon. You must make a physical melee attack each round.

    \feat{Magical Perception}{Combat}
    \featpre 5th level, Perception 6.
    \featben You can use your Perception to increase your accuracy with magical attacks other than spells in the same way that it increases your accuracy with physical attacks.

    \feat{Maneuver Focus}{Combat}
    Choose one \glossterm{combat maneuver}.
    \featben You gain an ability based on the maneuver chosen.
    \begin{itemize}
        \item Dirty trick: When you successfully perform a dirty trick, the target is \impaired for 1d4 rounds, rather than 1 round.
        \item Disarm: When you successfully perform a disarm, you can make the disarmed item land up to 15 feet away in a random direction.
        \item Feint: When you critically succeed at a feint, you deal damage with your weapon normally.
        \item Grapple: Grappling does not cause you to be \defenseless.
        \item Shove: When you successfully perform a shove, you can move the target the full distance without needing to move with it.
        \item Trip: When you successfully perform a trip, you can make a \glossterm{strike} against the target as an \glossterm{immediate action}. The target does not suffer prone penalties for the trip against the free attack.
    \end{itemize}

    At 3rd level, you a \plus1 bonus to accuracy with the chosen maneuver.

    At 7th level, the accuracy bonus increases to \plus2.

    At 11th level, whenever you perform the chosen maneuver, if your attack result also beats the target's Armor defense, you deal normal damage to the target with the weapon used to perform the maneuver.
    This damage is in addition to the maneuver's normal effects.
    If the maneuver was performed with a free hand, you deal damage with your unarmed attack (see \pcref{Unarmed Combat}).
    \parhead{Special} This feat can be taken multiple times.
    Each time, you choose a different combat maneuver.

    \feat{Martial Training}{General}
    \featben You are proficient in light and medium body armor, as well as shields.
    In addition, you become proficient in one additional weapon group of your choice.

    At 3rd level, you become proficient in heavy body armor.

    At 7th level, you gain a \plus1 bonus to damage with all physical attacks.

    At 11th level, you gain a \plus1 bonus to \glossterm{physical defenses}.
    \parhead{Normal}
    A character who is wearing armor with which she is not proficient applies its \glossterm{encumbrance penalty} to accuracy with physical attacks.
    The character also suffers double the normal arcane spell failure chance for wearing the armor.

    \feat{Miscaster}{Magical, Spell}
    \featpre 1st level spells.
    \featben Your explosive miscasts do not affect your allies.

    At 3rd level, your explosive miscasts no longer hurt you.
    Instead, they target all enemies in a \areasmall radius burst centered on you.

    At 7th level, if you know 2nd level spells, when you target a random creature with a spell's miscast effect, roll twice to determine which creature is affected.
    You choose which result is used.
    In addition, your explosive miscasts target all enemies in a \areamed radius burst centered on you.

    At 11th level, if you know 4th level spells, you are immune to the effects of all miscast spells, both by you and by other spellcasters.
    They are unable to damage you or affect you in any way.
    You may still choose to be affected by miscast effects, if desired.

    \feat{Mounted Combat}{Combat}
    \featpre Ride 4 ranks.
    \featben Whenever your mount is hit by a physical attack, you can take an \glossterm{immediate action} to made a Ride check.
    Your mount can use your check result in place of its physical defenses against the attack, potentially causing the attack to miss.

    At 3rd level, the penalty you take when using a ranged weapon while mounted is decreased by 4: \minus0 instead of \minus4 if your mount moves during the action phase, and \minus4 instead of \minus8 if your mount is sprinting.
    In addition, whenever you charge a creature while mounted, you increase your weapon damage die by one size increment (see \pcref{Weapon Size}).

    At 7th level, you can make a Ride check to negate one attack per round without spending an action.

    At 11th level, you take no penalties for firing a ranged weapon while mounted.
    In addition, the die size modifier for charging a creature while mounted increases to two increments.

    \feat{Overwhelming Fire}{Combat, Style}
    \featben While wielding a ranged weapon, you contribute to overwhelm penalties against all creatures you declare a physical ranged attack against in that phase.
    You do not contribute to overwhelm penalties against creatures with cover from you, or who are more than one range increment away from you with the weapon you use.

    At 7th level, overwhelmed creatures you are overwhelming with this feat increase their \glossterm{overwhelm penalties} by 1.
    In addition, you can help overwhelm creatures up to two range increments away from you.

    At 11th level, the additional overwhelm penalty increases to 2.
    In addition, you can help overwhelm creatures up to three range increments away from you.
    \stylereq Wield a ranged weapon.
    You must make a physical ranged attack each round.
    \parhead{Special} This is a Style feat, and all of its effects only apply while you are in this style.
    For details, see \pcref{Style Feats}.

    \feat{Parry}{Combat, Style}
    \featpre Dexterity 3.
    \featben While in this style, whenever you are hit by a \glossterm{strike}, you may attempt to parry it as an \glossterm{immediate action}.
    If you do, you make an attack roll with a weapon you wield.
    You may use your attack result in place of your physical defenses against the attack.
    This can cause the attack to miss.

    At 3rd level, if you have a free hand and are not wielding a shield, you gain a \plus2 bonus to your parry attacks.
    If you are wielding a shield, you can add your shield's defense bonus to the attack you make to parry.

    At 7th level, if you have 6 Dexterity, you can parry one strike per round without spending an action.

    At 11th level, if you have 9 Dexterity, you can counterattack more effectively after parrying.
    When you successfully parry an attack from a foe you threaten, if your parry attempt exceeds your foe's attack roll by 10 or more, it is \defenseless against you for 1 round.

    At 15th level, if you have 12 Dexterity, you can parry two strikes per round without spending an action.

    \stylereq Wield a melee weapon.
    You must make a melee attack or take the total defense action each round.
    \parhead{Special} This is a Style feat, and all of its effects only apply while you are in this style.
    For details, see \pcref{Style Feats}.

    \feat{Persuasion Mastery}{Skill}
    \featpre Persuasion 6 ranks.
    \featben Whenever you make a Persuasion check, you may roll twice and take the higher result.

    At 3rd level, \tdash.

    At 7th level, if you have 10 ranks in Persuasion, you gain a \plus5 bonus to Persuasion checks.

    At 11th level, if you have 14 ranks in Persuasion, you can compel creatures to obey your suggestions.
    As an immediate action, when you speak a suggestion aloud, you can make a Persuasion vs. Mental attack against a creature within \rngmed range of you.
    Success means the target it is compelled to obey your suggestion, as the effect of the \spell{suggestion} spell.
    You can use this ability three times per day.
    \magical

    \feat{Point Blank Shot}{Combat}
    \featben You gain a \plus2 bonus to damage with physical ranged attacks against targets within 50 feet of you.

    At 3rd level, you reduce your penalty for firing medium and large ranged weapons at adjacent creatures to \minus2.

    At 7th level, you suffer no penalty for firing ranged weapons at adjacent creatures.

    At 11th level, you are not treated as \defenseless while wielding a ranged weapon you are proficient with.
    In addition, the damage bonus increases to \plus3.
    \stylereq Wield a ranged weapon.

    \feat{Power Attack}{Combat, Style}
    \featpres 5th level, Strength 6.
    \featben You gain a \plus2 bonus to damage with physical melee attacks.

    At 7th level, this bonus increases to \plus3.

    At 11th level, if you have 9 Strength, this bonus increases to \plus4.

    At 15th level, if you have 12 Strength, this bonus increases to \plus5.

    At 19th level, if you have 15 Strength, this bonus increases to \plus6.
    \stylereq Wield a melee weapon.
    \parhead{Special} This is a Style feat, and all of its effects only apply while you are in this style.
    For details, see \pcref{Style Feats}.

    \feat{Precise Attack}{Combat, Style}
    \featpres Perception 3.
    \featben You gain a \plus2 bonus to accuracy with \glossterm{physical attacks}.
    In exchange, you take a \minus2 penalty to damage with physical attacks (to a minimum of 1).

    At 3rd level, the accuracy bonus increases to \plus3.

    At 7th level, if you have 6 Perception, the damage penalty is reduced to \minus1.

    At 11th level, if you have 9 Perception, the damage penalty is removed.
    \parhead{Special} This is a Style feat, and all of its effects only apply while you are in this style.
    For details, see \pcref{Style Feats}.

    \feat{Precise Shot}{Combat}
    \featpres 5th level, Perception 6.
    \featben Your ranged attacks ignore cover and concealment, except total cover and total concealment.

    At 7th level, when you attack a grappling opponent with a physical ranged attack, you do not have a chance to attack the wrong creature in the grapple.

    At 11th level, if you have 9 Perception, you ignore any effects which would give you a 20\% failure chance with your physical ranged attacks.
    This includes being \impaired.

    \feat{Armor-Piercing Attack}{Combat, Style}
    \featpre 5th level, Perception 6.
    \featben Whenever you make a \glossterm{strike}, you compare your result against the target's Armor and Reflex defenses.
    If you miss the target's Armor defense, but hit its Reflex defense, the strike hits and deals half damage.

    At 11th level, if you hit with a strike in this way, it deals full damage.

    \feat{Quick Draw}{Combat}
    \featben You can draw light and medium weapons as a \glossterm{free action}.
    In addition, you can draw heavy weapons and hidden weapons of any type (see the Sleight of Hand skill) as a move action.
    This allows you to throw light weapons at your full normal rate of attacks (much like a character with a bow).

    At 3rd level, you can draw any item of similar size to a light or medium weapon as a free action.

    At 7th level, you can also draw any item of similar size to a heavy weapon as a free action.
    In addition, you can sheathe or similarly put away items of similar size to light and medium weapons as a free action.

    At 11th level, you can also sheathe or similarly put away items of similar size to heavy weapons as a free action.

    \feat{Quickened Magic}{Magical, Spell}
    \featpre 9th level, 4th level spells.
    \featben As a \glossterm{swift action}, you can cast a spell using a spell slot two levels higher than normal.
    The spell must have a casting time of one standard action.
    This is mentally draining, and if you cast a spell in this way, you cannot act during the next action phase.
    You can only cast one spell in this way per round, even if you can take multiple swift actions.
    In addition, you gain a \plus1 bonus to spellpower.

    At 15th level, if you know 6th level spells, you are less draining by casting a quickened spell.
    When you cast a spell with this feat, you cannot cast spells in the next action phase, but you can take other actions normally.

    \feat{Rapid Recovery}{General}
    \featpre Con 3.
    \featben You naturally heal in half the normal time, allowing you to recover a quarter of your hit points with fifteen minutes of rest.
    Likewise, you heal \glossterm{vital damage} after four hours, rather than eight.
    This stacks with the benefits of accelerating recovery with the Heal skill (see \pcref{Accelerate Recovery}).

    At 3rd level, you only need one successfully resisted \glossterm{stabilization roll} to stabilize while dying.

    At 7th level, if you have 6 Constitution, magical healing is more effective on you.
    Whenever you receive magical healing, you increase the healing by an amount equal to your Constitution, up to a maximum of the original healing provided.

    At 11th level, if you have 9 Constitution, you heal hit points equal to your Constitution at the end of every round.
    In addition, you heal \glossterm{vital damage} at a rate of one vital damage per 5 minutes.
    \magical

    \feat{Reflexive Dodge}{Combat}
    \featpres 13th level, Dexterity 12.
    \featben Whenever you are hit by a \glossterm{strike}, you can take an \glossterm{immediate action} to dodge it.
    If you do, the attack misses.
    You must be aware of an attack to dodge it in this way.

    At 19th level, if you have 15 Dexterity, you can dodge one attack per round in this way without spending an action.

    \feat{Ride Mastery}{Skill}
    \featpre Ride 6 ranks.
    \featben Whenever you make a Ride check, you may roll twice and take the higher result.

    At 3rd level, \tdash.

    At 7th level, if you have 10 ranks in Ride, you gain a \plus5 bonus to Ride checks.

    At 11th level, if you have 14 ranks in Ride, you can attempt to ride unwilling creatures.
    As a standard action, you can make a Ride vs. Fortitude and Reflex attack against a creature adjacent to you.
    The creature must be at least one size category larger than you.
    Success means you ride the creature.
    Riding a creature is like grappling the creature, with the following changes.
    \begin{itemize}
        \item You share space with the creature you ride, just like riding a normal mount.
            If your mount breaks the grapple, you move to an adjacent unoccupied square of your choice.
            If there are no adjacent unoccupied squares, you in the same space as your mount, squeezing as necessary.
        \item You make Ride checks instead of grapple attacks to remain in the grapple and take actions in the grapple.
        \item You cannot pin the creature.
    \end{itemize}

    \feat{Ritual Caster}{Spell}
    \featpre Intelligence 3.
    \featben You can learn and perform rituals as if you were an arcane caster with a spellpower equal to your level.
    The maximum level of ritual that you can learn or perform is equal to half your level or your Intelligence, whichever is lower.

    \feat{Sense Motive Mastery}{Skill}
    \featpre Sense Motive 6 ranks.
    \featben Whenever you make a Sense Motive check, you may roll twice and take the higher result.

    At 3rd level, \tdash.

    At 7th level, if you have 10 ranks in Sense Motive, you gain a \plus5 bonus to Sense Motive checks.

    At 11th level, if you have 14 ranks in Sense Motive, you can read the minds of creatures.
    As a standard action, you can make a Sense Motive vs. Mental attack against a creature within \rngmed range of you.
    Success means you read the target's surface thoughts.
    You gain a \plus4 bonus to Bluff, Persuasion, and Intimidate checks against a creature whose mind you are reading.
    You can use this ability three times per day.
    \magical

    \feat{Shaped Magic}{Magical, Spell}
    \featpre 5th level, 2nd level spells.
    \featben If you cast a spell with an area using a spell slot one level higher than normal, you can change its area.
    The spell's area becomes a \areasmall, \areamed, \arealarge, or \areahuge \glossterm{burst}, as described in \trefnp{Shaped Areas}.
    The spell's range does not change.
    A radius burst without a range originates from you, and you may choose whether or not you are affected by it.

    \begin{dtable}
        \lcaption{Shaped Areas}
        \begin{dtabularx}{\columnwidth}{l >{\lcol}X >{\lcol}X >{\lcol}X}
            \tb{Initial Shape} & \tb{Line} & \tb{Cone} & \tb{Radius} \\
            Line & \tdash & One size smaller & Two sizes smaller \\
            Cone & 5 ft.\ wide, same size & \tdash & One size smaller \\
            Radius & 5 ft.\ wide, one size larger & Same size & \tdash \\
        \end{dtabularx}
    \end{dtable}

    At 11th level, you can cast a spell with an area using a spell slot two levels higher than normal.
    If you do, you can exclude any number of 5-foot cubes within the spell's area from its effect.
    This allows you to prevent the spell from affecting your allies, while still allowing it to affect your enemies.
    The area affected by the spell, ignoring all removed cubes, must be contiguous.

    At 15th level, you can change the area of all area spells you cast without spending a spell slot of a higher level.

    \feat{Shielded Fighting}{Combat}
    \featpre Proficiency with shields.
    \featben While wielding a shield, you gain a \plus1 bonus to \glossterm{physical defenses}.

    At 3rd level, you gain a \plus5 bonus to defenses against physical ranged attacks while wielding a shield.

    At 7th level, the bonus to physical defenses increases to \plus2.

    At 11th level, when you are hit by a ranged \glossterm{strike} while wielding a shield, you can make the strike miss automatically as an \glossterm{immediate action}.
    You must be aware of an attack to negate it in this way.

    \feat{Skill Savant}{Skill}
    \featben You gain two skill points.
    You may spend these skill points immediately.

    At 7th level, you gain a third additional skill point.

    At 11th level, you gain a fourth additional skill point.
    % make this better? pretty lame

    \feat{Sleight of Hand Mastery}{Skill}
    \featpre Sleight of Hand 6 ranks.
    \featben Whenever you make a Sleight of Hand check, you may roll twice and take the higher result.

    At 3rd level, \tdash.

    At 7th level, if you have 10 ranks in Sleight of Hand, you gain a \plus5 bonus to Sleight of Hand checks.

    At 11th level, if you have 14 ranks in Sleight of Hand, you can hide objects in impossible ways.
    Whenever you make a Sleight of Hand check to conceal or pickpocket an object, if the result is 30 or higher, you can hide the object into a pocket dimension.
    You can retrieve the item later as a move action.
    You may only have up to three items hidden in this way, none of which can be larger than one size category smaller than you.
    \magical

    \feat{Somatic Strike}{Magical, Spell}
    \featpres 9th level, 4th level spells.
    \featben When you cast spells, you can make a single \glossterm{strike} with a melee weapon in place of the somatic components for the spell.
    The target of the strike does not matter, and you can even attack thin air.
    The spell is otherwise cast as normal, regardless of whether the strike hits or misses.
    In addition, you gain a \plus1 bonus to spellpower.

    At 11th level, you gain a bonus to damage with this strike equal to the spell level of the spell being cast.

    At 15th level, if you know 6th level spells, you can make two strikes instead of one.
    The damage bonus applies to both strikes.

    At 19th level, the bonus to damage increases to be equal to half your spellpower with the spell being cast.

    \feat{Spellbreaker}{General, Magical}
    \featpres 9th level, Willpower 9.
    \featben You gain \glossterm{magic resistance} equal to 10 \add your Willpower.

    At 15th level, if you have 12 Willpower, you can reflect spells.
    When you resist a \glossterm{targeted spell} with this magic resistance, you can choose for the spell to be reflected back at the caster.
    The spell otherise functions normally, including effects on other targets of the spell, except that the caster is treated as a target of the spell instead of you.
    A spell reflected in this way cannot be reflected back at you.

    At 19th level, if you have 15 Willpower, you can completely negate spells cast on you.
    When you resist a spell with this magic resistance, you can choose to negate the spell's effects entirely.
    The spell has no effect, even if it was an area spell or a spell that targeted other creatures.
    If the spell had a duration, it is immediately dispelled, even if resisting it would not normally end or negate the spell's effects.
    This happens before any other magic resistance effects, preventing other creatures from resisting or reflecting it.

    \feat{Spellcasting Versatility}{Spell}
    Choose a spellcasting class you have.
    \featpres 5th level, 1st level spells from the chosen class.
    \featben You gain a bonus when determining your class level for the purpose of spells per day (if any), spells known, and maximum available spell level in your chosen class.
    The bonus is equal to half your class level in your chosen class.
    This cannot increase your effective class level above your character level, and does not improve the power of class abilities or grant you additional class abilities from your chosen class.

    For example, a character with 4 levels in wizard and 4 levels in fighter would have the spells per day and spells known of a 6th level wizard, and she would be able to cast 3rd level spells.
    \parhead{Special} This feat can be taken multiple times.
    Each time, you choose a different spellcasting class you possess.

    \feat{Spellcraft Mastery}{Skill}
    \featpre Spellcraft 6 ranks.
    \featben Whenever you make a Spellcraft check, you may roll twice and take the higher result.

    At 3rd level, \tdash.

    At 7th level, if you have 10 ranks in Spellcraft, you gain a \plus5 bonus to Spellcraft checks.

    At 11th level, if you have 14 ranks in Spellcraft, you gain a \plus2 bonus on accuracy, checks, and defenses against spells and \glossterm{magical} effects you identify with a successful Spellcraft check.

    \feat{Spellgift}{General, Magical}
    \featpres 5 levels in classes without spellcasting, Willpower 6.
    \featben You have inherent magic in your body, granting you magical power.
    When you gain this feat, you choose a \glossterm{spell source}: arcane, divine, or nature.
    You also choose a 1st level spell from the \glossterm{spell list} for that source.
    You can spend a spellgift point to use that spell as a magical ability.
    You have a maximum number of spellgift points equal half your Willpower.
    Your spellpower with spellgifts is equal to your Willpower or your level, whichever is higher.

    At 7 levels in non-spellcasting classes, you can choose a spell of up to 2nd level from the same source. You can also spend a spellgift point to use that spell.

    At 11 levels in non-spellcasting classes, if you have 9 Willpower, you can choose a spell of up to 4th level from the same source. You can also spend a spellgift point to use that spell.

    At 15 levels in non-spellcasting classes, if you have 12 Willpower, you can choose a spell of up to 6th level from the same source. You can also spend a spellgift point to use that spell.

    \feat{Spellstrike}{Magical, Spell}
    \featpre 5th level, 2nd level spells.
    \featben If you cast a \glossterm{targeted spell} using a spell slot one level higher than normal, you can make a single \glossterm{strike} with a weapon you wield as part of casting the spell.
    The spell must have a casting time of a standard action.
    The strike is made at the same time as other physical attacks in the action phase.
    If the strike hits a creature or object, you can spend an \glossterm{immediate action} to activate the spell.
    If you do, the target suffers the effects of the spell.

    If the strike misses, the spell is imbued into the weapon used to make the strike for 2 rounds.
    If the weapon hits a creature or object during that time, you can spend an immediate action to activate the spell, just as if the initial strike had succeeded.
    If you activate the spell, the duration runs out, or if you cast another spell, the spell in the weapon dissipates without effect.

    At 7th level, the imbuement lasts for 5 minutes before fading away.
    This can give you enough time to imbue a spell into a weapon and give the weapon to another creature.

    At 11th level, if you know 4th level spells, casting spells does not cause the imbuement to fade.
    You can still only have one spell imbued into a weapon with this feat at once.

    At 15th level, you can spend a spell slot of the spell's level to gain the benefits of this feat, rather than a spell slot one level higher than normal.

    \feat{Sprint Mastery}{Skill}
    \featpre Sprint 6 ranks.
    \featben Whenever you make a Sprint check, you may roll twice and take the higher result.

    At 3rd level, you gain a \plus10 foot bonus to speed in all your movement modes.

    At 7th level, if you have 10 ranks in Sprint, you gain a \plus5 bonus to Sprint checks.

    At 11th level, if you have 14 ranks in Sprint, you can sprint for a number of minutes equal to twice your Constitution (minimum 2).
    After that time, you must rest for 5 minutes before you can sprint again.

    \feat{Stealth Mastery}{Skill}
    \featpre Stealth 6 ranks.
    \featben Whenever you make a Stealth check, you may roll twice and take the higher result.

    At 3rd level, your penalties for moving while hiding are reduced by 5.
    This means you can move at up to half speed with no penalty, and you can move at full speed with a \minus5 penalty.

    At 7th level, if you have 10 ranks in Stealth, you gain a \plus5 bonus to Stealth checks.

    At 11th level, if you have 14 ranks in Stealth, you can use the Stealth skill to hide even while being observed.
    You take take a \minus10 penalty to the Stealth check when hiding in this way, and you still need passive cover or concealment to hide.

    At 15th level, if you have 11 ranks in Stealth, the penalty for hiding while observed is reduced to \minus5.

    \feat{Style Fusion}{Class, Combat}
    \featpre Fighter level 9, Intelligence 9.
    \featben You can use two styles at once, gaining the benefits of both styles (see \pcref{Style Feats}).
    This allows you to use a single \glossterm{free action} to initiate two styles at once, or change which two styles you are using.
    You may only do this for a number of rounds per day equal to your Intelligence.

    At 15th level, if you have 12 Intelligence, there is no limit on the number of rounds you can use this feat each day.

    \feat{Survival Mastery}{Skill}
    \featpre Survival 6 ranks.
    \featben Whenever you make a Survival check, you may roll twice and take the higher result.

    At 3rd level, you ignore \glossterm{difficult terrain} and harmful natural terrain of any kind.
    If a skill check, such as Climb or Swim, would normally be required to move through the terrain, this ability does not help.
    In addition, you can choose to leave no trace of your passage as you move.
    If you do, tracking you is impossible by any physical means.

    At 7th level, if you have 10 ranks in Survival, you gain a \plus5 bonus to Survival checks.

    At 11th level, if you have 14 ranks in Survival, you are immune to harmful planar effects.
    In addition, as a standard action, you can find your way to any location, as the effect of the \spell{find the path} ritual.
    You may use this ability once per day.
    Finding locations in this way is a \glossterm{magical} ability.

    \feat{Swift}{General}
    \featben You increase your land speed by 5 feet.

    At 3rd level, the speed bonus increases to 10 feet.

    At 7th level, the speed bonus increases to 20 feet.

    At 11th level, the speed bonus increases to 30 feet.

    \feat{Swim Mastery}{Skill}
    \featpre Swim 6 ranks.
    \featben Whenever you make a Swim check, you may roll twice and take the higher result.

    At 3rd level, you gain a \glossterm{swim speed} equal to your land speed.
    This grants several benefits.
    \begin{itemize}
        \item A successful Swim check to move allows you to move a distance equal to your swim speed.
        \item You gain a \plus10 bonus to Swim checks.
    \end{itemize}

    At 7th level, if you have 10 ranks in Swim, you gain a \plus5 bonus to Swim checks.

    At 11th level, if you have 14 ranks in Swim, you do not suffer any penalties to physical melee attacks, checks, or physical defenses for being underwater.
    You still suffer the normal penalty with underwater ranged attacks.

    \feat{Tactical Prediction}{Combat}
    \featpres 5th level, Intelligence 6.
    \featben As a swift action, you can make an Intelligence vs. Mental attack against a creature within \rngmed range of you.
    Success means you learn in general terms what the creature is planning to do during the next phase.
    Of course, it can change its plans, particularly if it hears you tell your allies what it will do.

    At 11th level, if you have 9 Intelligence, you gain a \plus2 bonus to this attack per round you have seen the creature fight.

    At 15th level, if you have 12 Intelligence, once per round you can attempt to predict a creature's actions as a \glossterm{free action}.

    \feat{Transmuter}{Magical, Spell}
    \featpre 1st level or higher Transmutation spell known.
    \featben You gain \glossterm{damage reduction} against \glossterm{physical damage} equal to half your spellpower with the \glossterm{spell source} used to qualify for this feat.
    Adamantine weapons ignore this damage reduction and negate it for 1 round.

    At 7th level, if you know a 2nd level or higher Transmutation spell, your Transmutation spells are more powerful.
    You can cast Transmutation spells as if they were one level lower than their actual level (minimum 1).
    This does not stack with spell level reductions from other feats.

    At 11th level, if you know a 4th level or higher Transmutation spell, your damage reduction increases to be equal to your spellpower, and applies against all damage.

    \feat{Trapfinder}{Skill}
    \spelldesc{You have honed your senses to recognize traps quickly and effectively.}
    \featpre Awareness 4 ranks.
    \featben As a full-round action, you can move up to 10 feet while searching every square within 10 feet of you for traps with the Awareness skill (see \pcref{Awareness}).
    If you detect a trap partway through your movement, you may immediately stop moving.

    At 3rd level, you gain a \plus5 bonus to Awareness checks to find traps.

    At 7th level, whenever you come within 10 feet of a trap, you can immediately make an Awareness check to notice the trap.
    This check should be made secretly, so you do not know whether you failed to notice a trap.
    In addition, when you take a full-round action to search for traps with this feat, the range at which you can notice traps increases to 50 feet.

    At 11th level, the bonus to Awareness checks to find traps increases to \plus10.
    In addition, you can immediately make an Awareness check to notice any traps within 20 feet of you.

    \feat{Two-Weapon Fighting}{Combat}
    \featpre Dexterity 3.
    \featben When making a \glossterm{dual attack} with two weapons at once, you gain a \plus1 bonus to accuracy with physical attacks.

    At 3rd level, you gain a \plus1 bonus to damage while making a dual attack with two weapons at once.

    At 7th level, if you have 6 Dexterity, the accuracy bonus increases to \plus2.

    At 11th level, if you have 9 Dexterity, the damage bonus increases to \plus2.

    \stylereq Wield two weapons at once.
    \parhead{Special} This is a Style feat, and all of its effects only apply while you are in this style.
    For details, see \pcref{Style Feats}.

    \feat{Unarmed Fighting}{Combat}
    \featben You gain proficiency with your unarmed attack.
    This grants you a \plus4 bonus to accuracy with the weapon and allows you to defend yourself with it, just as if you were using another melee weapon you are proficient with.
    In addition, your unarmed attacks can deal lethal or nonlethal damage as you choose.

    At 3rd level, you can attack with multiple parts of your body simultaneously.
    This allows you to make dual attacks with your unarmed strike (see \pcref{Dual Attacking}).

    At 7th level, you gain a \plus1 bonus to damage with unarmed attacks.

    At 11th level, the damage bonus increases to \plus2.

    \feat{Versatile Wild Speech}{Class, Magical}
    \featpres Druid or ranger level 1, wild speech ability.
    \featben You can use your wild speech ability to communicate with animates, such as oozes and animated plants, and ordinary plants. A regular plant's sense of its surroundings is limited, so it won't be able to give (or recognize) detailed descriptions of creatures or answer questions about events outside its immediate vicinity.

    At druid or ranger level 7, you can use your wild speech ability to communicate with magical beasts.

    At druid or ranger level 11, you can use your wild speech ability to communicate with one element of the natural world: air, earth, fire or water.
    You choose which element when you gain this ability.
    A element's sense of its surroundings is limited like a plant's.
    This also allows you to communicate with extraplanar elementals of your chosen element, which are much more intelligent.

    At druid or ranger level 15, you can use your wild speech to communicate with any living creature, even with creatures that do not have languages.

    \feat{Vivimancer}{Magical, Spell}
    \featpres 1st level or higher \glossterm{Vivimancy} spell known.
    \featben You are immune to hostile \glossterm{Life} and \glossterm{Death} effects.

    At 7th level, if you know a 2nd level or higher Vivimancy spell, your Vivimancy spells are more powerful.
    You can cast Vivimancy spells as if they were one level lower than their actual level (minimum 1).
    This does not stack with spell level reductions from other feats.

    At 11th level, if you know a 4th level or higher Vivimancy spell, you heal hit points equal to your spellpower with Vivimancy spells at the end of every round.

    \feat{Weapon Focus}{Combat}
    Choose one weapon group.
    \featpre Proficiency with selected weapon group.
    \featben You gain an ability based on the weapon group chosen.
    \begin{itemize}
        \item Armor weapons: When you perform a shield bash, you still benefit from the shield's defense bonus.
            In addition, armor spikes no longer impose a penalty to your physical defenses.
        \item Axes: You increase your \glossterm{critical multiplier} by 1 when attacking with axes.
        \item Blades, heavy: You increase your \glossterm{critical range} by 1 when attacking with heavy blades.
        \item Blades, light: You increase your critical range by 1 when attacking with light blades.
        \item Blunt weapons: When you deal damage to a creature with a blunt weapon, it takes a \minus2 penalty to Mental defense for 2 rounds.
            This penalty is not cumulative with itself.
        \item Bows: You can arc your shots with bows, allowing you to ignore cover and treat total cover as total concealment (50\% failure chance) if you can shoot above the obstacle to reach your target.
        \item Crossbows: The time required for you to reload crossbows is reduced to a free action (for a hand or light crossbow) or a move action (for a heavy crossbow).
        \item Flexible weapons: You gain a \plus2 accuracy bonus with \glossterm{combat maneuvers} that you perform with flexible weapons.
        \item Headed weapons: You increase your critical multiplier by 1 when attacking with headed weapons.
        \item Monk weapons: You gain a \plus2 accuracy bonus with combat maneuvers that you perform with monk weapons.
        \item Polearms: You can switch grips to short haft or stop short hafting a polearm as a swift action, and you take no penalty while short hafting it.
        \item Spears: When you are charged by a creature, you can brace a spear you wield as an \glossterm{immediate action}. The spear must have the Bracing property (see \pcref{Weapon Properties}).
        \item Thrown weapons: You can defend yourself with thrown weapons as you throw them, preventing you from being \defenseless (see \pcref{Thrown Weapons in Melee}).
    \end{itemize}

    At 3rd level, you gain proficiency with exotic weapons from your chosen weapon group.

    At 7th level, you gain a \plus1 bonus to damage on physical attacks with weapons from your chosen weapon group.

    At 11th level, the damage bonus increases to \plus2.

    \parhead{Special} This feat can be taken multiple times.
    Its effects do not stack.
    Each time, you choose a different weapon group.

\section{Other Feat Rules}

    \subsection{Bonus Feats}
        Some abilities grant a character bonus feats.
        Unless otherwise specified, the character must still meet any prerequisites for the feat.
        If the character does not meet the prerequisites at the time the bonus feat is granted, the character does not gain the feat.
        If the character later meets the prerequisites, the character immediately gains the benefit of the bonus feat.

        If a character gains a feat as a bonus feat that he or she has already acquired through other means, the character may select instead any other feat for which she qualifies.

    \subsection{Retraining Feats}
        At every level, your character can choose to retrain an old feat in exchange for a new feat.
        You can only retrain feats for other feats you could have acquired at the time you took the original feat.
        For example, a 6th level fighter can retrain his 2nd level fighter bonus feat for any other combat feat that he qualified for at his 2nd fighter level.
        This also means you cannot retrain feats gained through class abilities which give you a specific feat, since there were no other feats you could have taken.
