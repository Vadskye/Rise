\chapter{Feats}\label{Feats}

Feats are special abilities that every character has.
Feats can be used to specialize your character particular area, to grant your character new abilities, or to change the way your character does certain things.

\section{Gaining Feats}
    Your character gains two feats at 1st level, a third at 3rd level, a fourth at 6th level, and a fifth at 10th level.
    % TODO: are there any bonus fetas
    Classes can sometimes grant bonus feats as well, which are in addition to these feats which every character gets.
    A character cannot gain the same feat twice.

    \subsection{Prerequisites}
        Some feats have prerequisites.
        Unless your character has all of the prerequisites, they cannot take the feat.
        Prerequisites can include a minimum starting attribute score, another feat or feats, a minimum level of training in a skill, or some other property of a character.
        A character can gain a feat at the same level at which he or she gains the prerequisite.

        A character can't use a feat if he or she has lost a prerequisite.

\section{Feat Tags}
    All feats are organized into different groups by tags.

    \parhead{General} General feats can have a wide variety of effects.
    They often grant new abilities or improve your defenses.

    \parhead{Combat} Combat feats improve your combat capabilities.
    They can increase the damage you deal, grant you new combat abilities, or improve your defenses.

    \parhead{Spell} Spell feats improve your spellcasting abilities.
    Spell feats are useless to characters who cannot cast spells.

    \parhead{Skill} Skill feats improve your skills.
    They can make you more likely to succeed with skill checks and grant you new abilities based on your skills.

    \parhead{Bloodline Feats} Some characters have traces of monstrous blood running in their veins.
    Most of those will never understand the full potential of their unusual heritage.
    Bloodline feats allow characters to explore those posibilities by gaining abilites related to their ancestry.
    You can only have one Bloodline feat.

    \parhead{Magical Feats}
    All abilities granted by feats with the [Magical] type are \glossterm{magical} in nature.
    Many feats are not entirely magical, but have specific effects that are magical.

\section{Feat Tables}

% Feat names must follow ``have'', ``are (a)'', or ``can''.
\begin{longtabuwrapper}
    \begin{longtabu}{>{\lcol}p{10em} >{\lcol}p{15em} >{\lcol}X >{\lcol}p{8em} >{\lcol}p{3em}}
        \lcaption{Feats}\\
        \tb{General Feats}\label{General Feats} & \tb{Prerequisites} & \tb{Benefits} & \tb{Feat Types} & \tb{Page} \\
        \featref{Celestial Heritage} & Non-evil & Gain aspects of celestial beings & Bloodline, Magical & \featpref{Celestial Heritage} \\
        \featref{Draconic Heritage} & \tdash & Gain aspects of draconic power & Bloodline & \featpref{Draconic Heritage} \\
        \featref{Iron Will} & Wil 2 & Increase mental resilience & \tdash & \featpref{Iron Will} \\
        \featref{Null} & Wil 2 & Become immune to magic & \tdash & \featpref{Null} \\
        \featref{Regenerator} & Con 2 & Heal from wounds with inhuman speed & \tdash & \featpref{Regenerator} \\
        \featref{Swift} & Dex 2 & \tdash & \tdash & \featpref{Swift} \\
        \featref{Toughness} & Con 2 & Increase physical fortitude & \tdash & \featpref{Toughness} \\

        \tb{Class Feats}\label{Class Feats} & \tb{Prerequisites} & \tb{Benefits} & \tb{Feat Types} & \tb{Page} \\
        \featref{All Energy Becomes One} & Monk class, Con 2 & Absorb and redirect damage & Magical & \featpref{All Energy Becomes One} \\
        \featref{Class Versatility} & \tdash & Swap for abilities from additional class & \tdash & \featpref{Class Versatility} \\
        \tind \featref{Class Dedication} & Class Versatility feat & Gain archetype from additional class & \tdash & \featpref{Class Dedication} \\

        \tb{Skill Feats}\label{Skill Feats} & \tb{Prerequisites} & \tb{Benefits} & \tb{Feat Types} & \tb{Page} \\
        \featref{Acrobatics Specialization} & Mastered Acrobatics & \tdash & \tdash & \featpref{Acrobatics Specialization} \\
        \featref{Awareness Specialization} & Mastered Awareness & \tdash & \tdash & \featpref{Awareness Specialization} \\
        \featref{Bardic Exemplar} & Perform Specialization & Mock foes and bolster allies with performances & Magical & \featpref{Bardic Exemplar} \\
        \featref{Bluff Specialization} & Mastered Bluff & \tdash & \tdash & \featpref{Bluff Specialization} \\
        \featref{Climb Specialization} & Mastered Climb & \tdash & \tdash & \featpref{Climb Specialization} \\
        \featref{Craft Specialization} & Mastered Creaft & \tdash & \tdash & \featpref{Craft Specialization} \\
        \featref{Creature Handling Specialization} & Mastered Creature Handling & \tdash & \tdash & \featpref{Creature Handling Specialization} \\
        \featref{Devices Specialization} & Mastered Devices & \tdash & \tdash & \featpref{Devices Specialization} \\
        \featref{Disguise Specialization} & Mastered Disguise & \tdash & \tdash & \featpref{Disguise Specialization} \\
        \featref{Escape Artist Specialization} & Mastered Escape Artist & \tdash & \tdash & \featpref{Escape Artist Specialization} \\
        \featref{Intimidate Specialization} & Mastered Intimidate & \tdash & \tdash & \featpref{Intimidate Specialization} \\
        \featref{Heal Specialization} & Mastered Heal & \tdash & \tdash & \featpref{Heal Specialization} \\
        \featref{Jump Specialization} & Mastered Jump & \tdash & \tdash & \featpref{Jump Specialization} \\
        \featref{Knowledge Specialization} & Mastered Knowledge & \tdash & \tdash & \featpref{Knowledge Specialization} \\
        \featref{Linguistics Specialization} & Mastered Linguistics & \tdash & \tdash & \featpref{Linguistics Specialization} \\
        \featref{Perform Specialization} & Mastered Perform & \tdash & \tdash & \featpref{Perform Specialization} \\
        \featref{Persuasion Specialization} & Mastered Persuasion & \tdash & \tdash & \featpref{Persuasion Specialization} \\
        \featref{Ride Specialization} & Mastered Ride & \tdash & \tdash & \featpref{Ride Specialization} \\
        \featref{Sense Motive Specialization} & Mastered Sense Motive & \tdash & \tdash & \featpref{Sense Motive Specialization} \\
        \featref{Sleight of Hand Specialization} & Mastered Sleight of Hand & \tdash & \tdash & \featpref{Sleight of Hand Specialization} \\
        \featref{Spellcraft Specialization} & Mastered Spellcraft & \tdash & \tdash & \featpref{Spellcraft Specialization} \\
        \featref{Stealth Specialization} & Mastered Stealth & \tdash & \tdash & \featpref{Stealth Specialization} \\
        \featref{Survival Specialization} & Mastered Survival & \tdash & \tdash & \featpref{Survival Specialization} \\
        \featref{Swim Specialization} & Mastered Swim & \tdash & \tdash & \featpref{Swim Specialization} \\

        \tb{Spell Feats}\label{Spell Feats} & \tb{Prerequisites} & \tb{Benefits} & \tb{Feat Types} & \tb{Page} \\
        \featref{Abjurer} & Abjuration spell & \tdash & Magical & \featpref{Abjurer} \\
        \featref{Boongiver} & Any spell & Benefits when casting spells on allies & Magical & \featpref{Boongiver} \\
        \featref{Conjurer} & Conjuration spell & \tdash & Magical & \featpref{Conjurer} \\
        \featref{Diviner} & Divination spell & \tdash & Magical & \featpref{Diviner} \\
        \featref{Eldritch Knight} & Any spell & Fight with sword and spell together & \tdash & \featpref{Eldritch Knight} \\
        \featref{Enchanter} & Enchantment spell & \tdash & Magical & \featpref{Enchanter} \\
        \featref{Evoker} & Evocation spell & \tdash & Magical & \featpref{Evoker} \\
        \featref{Illusionist} & Illusion spell & \tdash & Magical & \featpref{Illusionist} \\
        \featref{Miscaster} & Any spell & \tdash & Magical & \featpref{Miscaster} \\
        \featref{Mystic Archer} & Any spell & Imbue projectiles with magic & Magical & \featpref{Mystic Archer} \\
        \featref{Transmuter} & Transmutation spell & \tdash & Magical & \featpref{Transmuter} \\
        \featref{Vivimancer} & Vivimancy spell & \tdash & Magical & \featpref{Vivimancer} \\

        \tb{Combat Feats}\label{Combat Feats} & \tb{Prerequisites} & \tb{Benefits} & \tb{Feat Types} & \tb{Page} \\
        \featref{Agility} & Dex 2 & Increase reaction speed & \tdash & \featpref{Agility} \\
        \featref{Blindfighter} & Per 2 & Fight unseen foes better & \tdash & \featpref{Blindfighter} \\
        \featref{Duelist} & Dex 1, Int 1 & Fight one-on-one better & \tdash & \featpref{Duelist} \\
        \featref{Executioner} & Str 1, Per 2 & Kill weakened foes more easily & \tdash & \featpref{Executioner} \\
        \featref{Guardian} & Per 1, Wil 1 & Protect nearby allies & \tdash & \featpref{Guardian} \\
        \featref{Leadership} & Wil 2 & Inspire nearby allies & \tdash & \featpref{Leadership} \\
        \featref{Martial Training} & \tdash & Improve combat abilities & \tdash & \featpref{Martial Training} \\
        \featref{Precognition} & Int 2 & React to future events & \tdash & \featpref{Precognition} \\
        \featref{Reaper} & Str 2 & Cleave through foes with sweeping strikes & \tdash & \featpref{Reaper} \\
        \featref{Savage} & Str 2 & Shove and overrun foes to deal damage & \featpref{Savage} \\
        \featref{Sniper} & Per 2 & Aim precisely at distant foes & \tdash & \featpref{Sniper} \\
        \featref{Whirlwind Warrior} & Dex 2, Per 1 & Fight hordes with agile ease & \tdash & \featpref{Whirlwind Warrior} \\
    \end{longtabu}
\end{longtabuwrapper}

    \section{Feat Descriptions}
        Here is the format for feat descriptions.

    \ssecfake{Feat Name [Type of Feat]}
    \featpre Requirements a character must meet before taking the feat.
    This entry is absent if a feat has no prerequisites.
    \featben What the feat enables the character (``you'' in the feat description) to do.

    \begin{feat}{Abjurer}{Magical, Spell}
        \spelldesc{You have great talent with Abjuration spells.}
        \featpre Abjuration spell known.
        \featben
        \ff{Abjurant Shield} Whenever you cast an Abjuration spell with a duration, you can give one willing creature targeted by the spell a deflective shield.
        The shielded creature gains a \plus1 bonus to \glossterm{physical defenses} as long as the spell lasts.
        This is a \glossterm{magical} \glossterm{Shielding} effect.
        You can only shield one creature in this way at a time.
        If you shield another creature, all previous shields are dismissed when the new shield takes effect.

        \ff[3]{Counterspell} You gain the \textit{counterspell} ability.
        \begin{ability}{Counterspell}
            As a standard action, you can spend an \glossterm{action point} to use this ability.
            If you do, choose a creature within \rngmed range of you.
            If the target is casting a spell, and your maximum spell level is at least as high as the target's maximum spell level, their spell has no effect when it resolves.
        \end{ability}

        \ff[5]{Personal Shield} You gain a \plus1 bonus to \glossterm{physical defenses}.

        % which spellpower?
        \ff[7]{Greater Abjurant Shield} Your \textit{abjurant shield} also grants \glossterm{damage reduction} against \glossterm{physical damage} equal to your \glossterm{spellpower}.

        \ff[9]{Improvised Counterspell} At the start of the \glossterm{delayed action phase}, if you are casting a spell other than a \glossterm{cantrip}, you can use your \textit{counterspell} ability without taking an action.
        If you do, your spell has no effect when it resolves, and you regain the action point spent to cast it (if any).

        \ff[11]{Expanded Shield} The bonus from your \textit{personal shield} ability applies to all defenses.

        \ff[13]{Supreme Abjurant Shield} The defense bonus from your \textit{abjurant shield} ability applies to all defenses.
        In addition, the damage reduction from your \textit{greater abjurant shield} ability applies against all damage.

        \ff[15]{Punishing Counterspell} When you use your \textit{counterspell} ability, you can make the target \glossterm{miscast} their spell instead of negating the spell's effects.
        In addition, the range of your \textit{counterspell} ability increases to \rnglong.

        \ff[17]{Pierce Shields} Your skill at creating defenses allows you penetrate defenses more easily.
        You gain a \plus1 bonus to \glossterm{accuracy}.

        \ff[19]{Mass Counterspell} When you use your \textit{counterspell} ability, you can target up to five creatures.
        % Alternately, this could take control of the countered spell instead of negating it?
    \end{feat}

    \begin{feat}{Acrobatics Specialization}{Skill}
        \featpre Acrobatics as a mastered skill.
        \parhead{Benefits}

        \ff{Lesser Specialization} You gain a \plus2 bonus to Acrobatics.

        \ff[2]{Agile Charge} You can change directions freely while making a \glossterm{charge}.

        \ff[4]{Rapid Balance} Using Acrobatics to balance on slippery or narrow surfaces does not reduce your speed.

        \ff[6]{Surface Tolerance} You reduce DR modifiers for surface conditions on Acrobatics checks to balance by 2.
        This allows you to ignore minor surface conditions, such as slippery surfaces, when balancing.

        \ff[8]{Specialization} The bonus to Acrobatics increases to \plus4.

         % modifier at 10th: 12dex + 2mst + 2ft = 16
        \ff[10]{Legendary Airwalker}[Mag] You can attempt to move on surfaces that cannot support your weight.
        Surfaces that can support at least a quarter of your weight, such as thin tree branches and dense liquids, are DR 20.
        Surfaces that can support at least a tenth of your weight, such as water, are DR 25.
        Surfaces that can support at least a hundredth of your weight, such as tree leaves, are DR 30.
        Surfaces that cannot support your weight at all, such as air, are DR 40.

        Success means you move along the surface at half speed.
        Failure means you fall through the surface.
        The DR increases by 2 for each consecutive round that you spend moving in this way.

        \ff[12]{Greater Surface Tolerance} The reduction of DR modifiers for surface conditions increases to 5.
        This allows you to ignore almost all surface conditions when balancing.

        \ff[14]{Rapid Airwalker} You can move at full speed with the \textit{legendary airwalker} ability.

        \ff[16]{Greater Specialization} The bonus to Acrobatics increases to \plus6.

        \ff[18]{} 
    \end{feat}

    \begin{feat}{Agility}{Combat}
        \featpre Starting Dexterity of 2.
        \featben

        \ff[1]{Lightning Reflexes} You gain a \plus2 bonus to Reflex defense.

        \ff[2]{Dodge} At the start of each phase, you can choose a creature you can see.
        You gain a \plus1 bonus to Armor defense against attacks by that target during that phase.

        \ff[4]{Rapid Reaction} You gain a \plus2 bonus to \glossterm{initiative} checks.

        \ff[6]{} 

        \ff[8]{Focused Dodge} You do not suffer \glossterm{overwhelm penalties} against attacks by the target of your \textit{dodge} ability.
        In addition, when counting the number of creatures threatening you to determine your overwhelm penalties, you ignore that creature.

        \ff[10]{Greater Rapid Reaction} The bonus to initiative from your \textit{rapid reaction} ability increases to \plus5.

        \ff[12]{Greater Lightning Reflexes} The bonus to Reflex defense from your \textit{lightning reflexes} ability increases to \plus4.

        \ff[14]{Greater Dodge} The bonus to Armor defense from your \textit{dodge} ability increases to \plus2.

        \ff[16]{Tumble} You can move through the space occupied by the target of your \textit{dodge} ability at half speed.

        \ff[18]{Evasion} Your \glossterm{overwhelm penalties} do not affect your Reflex defense.
    \end{feat}

    \begin{feat}{All Energy Becomes One}{Class, Magical}
        \featpre Monk, starting Constitution of 2.
        \parhead{Benefits}

        \ff{Resist Energy} You gain \glossterm{damage reduction} against \glossterm{energy damage} equal to your monk \textit{ki power}.

        \ff[3]{Channel Energy} At the end of each round, if you reduced damage with your \textit{resist energy} ability that round, you can channel that energy into your weapons.
        If you do, choose a damage type that you reduced this round with that ability.
        All damage you deal with \glossterm{strikes} gains that damage type until the end of the next round.

        \ff[5]{Energetic Might} Choose a type of \glossterm{energy}: cold, electricity, fire, or sonic.
        You gain a \plus1d bonus to damage with attacks that deal damage of that type that is measured in dice.

        \ff[7]{Energetic Unity} The damage reduction from your \textit{resist energy} ability applies against all damage from all \glossterm{magical} attacks.

        \ff[9]{Sustained Channeling} You can sustain your \textit{channel energy} ability as a \glossterm{minor action}.

        \ff[11]{Myriad Energetic Might} The damage bonus from your \textit{energetic might} ability applies to damage from all energy types.

        \ff[13]{Greater Resist Energy} The damage reduction from your \textit{resist energy} ability increases to be equal to twice your \ki power.

        \ff[15]{Attuned Channeling} When you use your \textit{channel energy} ability, you can spend an \glossterm{action point}.
        If you do, you \glossterm{attune} to the ability, causing it to last as long as you stay attuned to it.
        For details, see \pcref{Attunement}.

        \ff[17]{Greater Energetic Might} The damage bonus from your \textit{energetic might} ability increases to \plus2d.

        \ff[19]{Kinetic Absorption} The damage reduction from your \textit{resist energy} ability applies to all damage.
    \end{feat}

    \begin{feat}{Awareness Specialization}{Skill}
        \featpre Awareness as a mastered skill.
        \parhead{Benefits}

        \ff[1]{Lesser Specialization} You gain a \plus2 bonus to Awareness.

        \ff[2]{Extraordinary Senses} You gain one of the following senses: \glossterm{blindsense} (50 ft.), \glossterm{darkvision} (100 ft.), \glossterm{scent}, or \glossterm{tremorsense} (50 ft.).

        \ff[4]{Broad Search} When you take the Search action, you can search a 10-ft.\ square within 30 feet of you (see \pcref{Search}).

        \ff[6]{Distance Tolerance} You reduce DR modifiers for distance on Awareness rolls by 2.
        This usually allows you to ignore up to 100 feet of distance.

        % \ff[6]{Attentive} You can pay \glossterm{active attention} to your surroundings for any length of time without becoming fatigued.

        \ff[8]{Specialization} The bonus to Awareness increases to \plus4.

        \ff[10]{Legendary Senses} You gain one of the following senses: \glossterm{blindsense} (200 ft.), \glossterm{blindsight} (50 ft.), \glossterm{darkvision} (500 ft.), \glossterm{tremorsense} (200 ft.), or \glossterm{tremorsight} (50 ft.).

        \ff[12]{Trapmaster} Whenever you come within 50 feet of a trap, you can make an Awareness check to notice it, even if you were not searching for traps.

        \ff[14]{Greater Distance Tolerance} The reduction of DR modifiers for distance increases to 5.
        This usually allows you to ignore up to 500 feet of distance.

        \ff[16]{Greater Specialization} The bonus to Awareness increases to \plus6.

        \ff[18]{Supreme Senses} You can choose an additional sense from the \textit{legendary senses} ability.
        Its range is doubled.
    \end{feat}

    \begin{feat}{Bardic Exemplar}{Magical, Skill}
        \featpre Perform Specialization feat.
        \featben

        \ff[1]{Mocking Performance} You gain the \textit{mocking performance} ability.
        \begin{ability}{Mocking Performance}[\glossterm{Delusion}, \glossterm{Mind}, \glossterm{Sustain} (minor)]
            As a standard action, you can spend an \glossterm{action point} to use this ability.
            When you do, you begin a performance using one of your Perform skills.
            Make a Perform vs. Mental attack against a creature within \rngmed range, and choose a willing creature within the same range.
            As a \glossterm{condition}, the target suffers a \minus3 penalty to \glossterm{accuracy} with attacks that do not include your chosen creature as a target.
            The penalty is increased to \minus6 on a critical hit.

            If the target can neither see nor hear your performance, the effect immediately ends.
            This ability may have the \glossterm{Auditory} or \glossterm{Visual} tags, depending on the nature of your performance.
        \end{ability}

        \ff[3]{Perform Point} You gain a perform point.
        A perform point can be spent to use \glossterm{magical} abilities from this feat and the Perform Specialization feat in place of an action point.
        You recover all spent perform points after a \glossterm{short rest}.

        \ff[5]{Song of Serenity} You gain the \textit{song of serenity} ability.
        \begin{ability}{Song of Serenity}[\glossterm{Delusion}, \glossterm{Mind}, \glossterm{Sustain} (minor)]
            As a standard action, you can spend an \glossterm{action point} to use this ability.
            When you do, you begin a performance using one of your Perform skills.
            Choose a willing creature within \rngmed range.
            The target is immune to hostile \glossterm{Mind} effects.

            If the target can neither see nor hear your performance, the effect immediately ends.
            This ability may have the \glossterm{Auditory} or \glossterm{Visual} tags, depending on the nature of your performance.
        \end{ability}

        \ff[7]{Demoralizing Mockery} A creature affected by your \textit{mocking performance} ability takes a \minus2 penalty to Mental defense as long as the effect lasts.

        \ff[9]{Hybrid Performance} You can sustain two different \glossterm{magical} abilities from this feat or the Perform Specialization feat as part of the same \glossterm{minor action}.

        \ff[11]{Mass Performance} When you use your \textit{mocking performance} ability, you can target up to five creatures.
        In addition, when you use your \textit{song of serenity} ability, you can target up to two creatures.

        \ff[13]{Battle Cry} You gain the \textit{battle cry} ability.
        \begin{ability}{Battle Cry}[\glossterm{Delusion}, \glossterm{Mind}, \glossterm{Sustain} (minor)]
            As a \glossterm{minor action}, you can spend an \glossterm{action point} to use this ability.
            When you do, you begin a performance using one of your Perform skills.
            Choose any number of willing creatures within \rngmed range.
            Each target gains a \plus2 bonus to accuracy with all attacks.

            If a target can neither see nor hear your performance, the effect immediately ends for that creature.
            This ability may have the \glossterm{Auditory} or \glossterm{Visual} tags, depending on the nature of your performance.
        \end{ability}

        \ff[15]{Serene Bliss} A creature affected by your \textit{song of serenity} ability is not \glossterm{bloodied} as long as the effect lasts and it has hit points remaining.
        It suffers the normal penalties for becoming bloodied if the effect ends or it has no hit points remaining.

        \ff[17]{Greater Mass Performance} When you use your \textit{mocking performance} ability, you can target any number of creatures.
        In addition, when you use your \textit{song of serenity} ability, you can target up to five creatures.
        Finally, the range of your \textit{battle cry} ability is increased to \rnglong.

        \ff[19]{Champion's Anthem} A creature affected by your \textit{battle cry} ability may use your level in place of its accuracy with any attack.

        \ff[19]{Hideous Laughter} If you get a \glossterm{critical hit} against a creature with your \textit{mocking performance} ability, it laughs uncontrollably until the end of the next round.
        During this time, it can take no other actions.
    \end{feat}

    \begin{feat}{Blindfighter}{Combat}
        \featpre Starting Perception 2.
        \featben
        
        \ff[1]{Blind Precision} Whenever you have a miss chance caused by being unable to see your opponent, you can roll the miss chance twice and take the better result.

        \ff[2]{Unseen Defense} You are not \defenseless against foes you cannot see if you know their location.

        \ff[4]{Blindsense} You gain \glossterm{blindsense} (50 ft.).

        \ff[6]{Attack the Unseen} If you know the location of a creature you cannot see, and you have \glossterm{line of effect} to that creature, you can target it with targeted abilities.

        \ff[8]{Blind Feint} Whenever you are attacked by a creature who incorrectly thinks you are \glossterm{unaware} of the attack, you gain a \plus1 bonus to \glossterm{accuracy} with \glossterm{physical attacks} against that creature until the end of the next round.

        \ff[10]{Blindsight} You gain \glossterm{blindsight} (50 ft.).
        In addition, the range of your blindsense improves to 200 feet. 

        \ff[12]{Controlled Sight} You are immune to all abilities that depend on sight to affect you.

        \ff[14]{Greater Blindsight} The range of your blindsight improves to 100 feet.
        In addition, the range of your blindsense improves to 500 feet.

        \ff[16]{Greater Blind Feint} The damage bonus from your \textit{blind feint} ability increases to \plus2d.

        \ff[18]{Supreme Sight} The range of your blindsight improves to 200 feet.
        In addition, the range of your blindsense improves to 1,000 feet.
    \end{feat}

    \begin{feat}{Bluff Specialization}{Skill}
        \featpre Bluff as a mastered skill.
        \featben

        \ff[1]{Lesser Specialization} You gain a \plus2 bonus to Bluff.

        \ff[2]{Sustained Distraction} If you successfully distract a creature with the \textit{distract} ability of the Bluff skill, you can sustain that distraction on that creature with a \glossterm{minor action} as long as you continue to be distracting.
        You must make a new check each round, and the DR increases by 2 for each round you have distracted them.
        For details, see \pcref{Distract}.

        \ff[4]{Deceive Magic} Any magical abilities which detect lies are unable to detect lies you speak.

        \ff[6]{Intuitive Impersonation} When you use the \textit{impersonate} ability of the Bluff skill, you reduce your penalties for not knowing how to act by 2.

        % \ff[6]{Deceptive Feint} When you use the \textit{feint} combat maneuver, you can use Bluff in place of your physical accuracy (see \pcref{Feint}).

        \ff[8]{Specialization} The bonus to Bluff increases to \plus4.

        \ff[10]{Dual Speech}[Mag] You gain the \textit{dual speech} ability.
        \begin{ability}{Dual Speech}[\glossterm{Sustain} (minor)]
            Whenever you speak, you can spend an \glossterm{action point} to use this ability.
            If you do, you speak the same words with two different vocal patterns, such as tone or accent, though you cannot significantly change the volume.
            You can freely choose which creatures hear which pattern, treating all creatures you are not aware of as a single group.

            You can freely choose different vocal patterns each round that you sustain this ability.
        \end{ability}

        \ff[12]{Greater Deceive Magic} Whenever you impersonate a creature, if your Bluff check is high enough, magical abilities treat you as if you were that creature.
        % is this DR high enough?
        The DR is equal to 10 \add the \glossterm{power} of the ability you are deceiving.
        % does this work mechanically?
        For example, if you impersonate an undead creature, the \spell{inflict light wounds} spell would heal you.
        This does not grant you any abilities associated with creatures you impersonate.

        \ff[14]{Greater Intuitive Impersonation} The penalty reduction from your \textit{intuitive impersonation} ability increases to 5.

        \ff[16]{Greater Specialization} The bonus to Bluff increases to \plus6.

        \ff[18]{Greater Dual Speech} When you use your \textit{dual speech} ability, you can speak entirely different words with your two voices.
    \end{feat}

    \begin{feat}{Boongiver}{types}
        \featpre Ability to cast a spell.
        \featben

        \ff[1]{Reciprocal Boon} You gain the \textit{reciprocal boon} ability.
        \begin{ability}{Reciprocal Boon}
            Whenever you cast a spell that targets a single creature other than yourself and that requires the target to attune to the effect, you can use this ability.
            If you do, the spell targets you in addition to its other target.

            You can only use this ability on one spell at a time.
            If you use the ability while it already affects another spell, you choose which spell is affected by the ability.
        \end{ability}

        \ff[3]{Benevolent Transferance} You gain the \textit{benevolent transferance} ability.
        \begin{ability}{Benevolent Transferance}
            As a \glossterm{minor action}, you can use this ability.
            If you do, choose a creature currently attuning to a spell you cast.
            In addition, choose another creature to transfer the spell to.
            Both targets must within that spell's range of you.
            You cannot target yourself with this ability.

            If both targets are willing, the spell's effect is transferred from the first target to the second.
            The spell's new target must spend an action point to attune to the spell as normal.
        \end{ability}

        \ff[5]{Boon Lore} You learn an additional \glossterm{subspell} of 2nd level or lower.
        The subspell must have an effect that lasts as long as you and the target attune to it.

        \ff[7]{Greater Reciprocal Boon} You can use your \textit{reciprocal boon} ability on up to two different spells at once.
        If you use the ability while it already affects another spell, you choose which spells are affected by the ability.

        \ff[9]{Myriad Boon} Whenever you use your \textit{reciprocal boon} ability, you can spend an \glossterm{action point}.
        If you do, you can choose a third target for the spell.
        That target must spend an action point to attune to the spell as normal.

        \ff[11]{Boon Lore} You learn an additional \glossterm{subspell} of 5th level or lower.
        The subspell must have an effect that lasts as long as you and the target attune to it.

        \ff[13]{Regenerative Transferance} When you use your \textit{benevolent transferance} ability successfully, the original target of the spell regains the action point it spent to attune to the spell.

        \ff[15]{Supreme Reciprocal Boon} You can use your \textit{reciprocal boon} ability on up to three different spells at once.

        \ff[17]{Boon Lore} You learn an additional \glossterm{subspell} of 8th level or lower.
        The subspell must have an effect that lasts as long as you and the target attune to it.

        \ff[19]{Greater Myriad Boon} You do not have to spend an \glossterm{action point} to use your \textit{myriad boon} ability.

    \end{feat}

    \begin{feat}{Celestial Heritage}{Bloodline, Magical}
        \featpre Non-evil alignment
        \parhead{Special} You can only have one Bloodline feat.
        \featben

        \ff[1]{Celestial Power} Your \glossterm{power} with abilities from this feat is equal to your Willpower or your level, whichever is higher.

        \ff[1]{Holy Blessing} You gain the \textit{holy blessing} ability.
        \begin{ability}{Holy Blessing}[\glossterm{Attune} (shared)]
            As a \glossterm{minor action}, you can spend an \glossterm{action point} to use this ability.
            If you do, choose a willing creature within \rngclose range.
            The target gains a \plus1 bonus to \glossterm{accuracy} with all attacks.
            This effect lasts as long as you and the target \glossterm{attune} to it.
        \end{ability}

        \ff[3]{Holy Protection} You gain the \textit{holy protection} ability.
        \begin{ability}{Holy Protection}[\glossterm{Attune} (shared)]
            As a standard action, you can spend an \glossterm{action point} to use this ability.
            If you do, choose a willing creature within \rngclose range.
            The target gains damage reduction equal to your celestial power against damage from \glossterm{Evil} attacks and from physical attacks made by evil creatures.
        \end{ability}

        \ff[5]{Angel Wings} You gain feathery wings that sprout from your back.
        You can use these wings to glide at a rate equal to your land speed (see \pcref{Gliding}).
        The wings themselves are \glossterm{mundane}, but the ability to glide and fly with them is \glossterm{magical}.

        \ff[7]{Complete Protection} The damage reduction from your \textit{holy protection} ability also applies against non-physical effects.

        % TODO: vague?
        \ff[9]{Aligned Blessing} The target of your \textit{holy blessing} ability treats all of their attacks as being good-aligned.

        \ff[11]{Angelic Flight} Your \textit{angel wings} grant you a fly speed equal to your land speed.
        While \glossterm{unencumbered}, you can fly (see \pcref{Flying}).
        You can only fly for a number of rounds equal to half your celestial power.
        After that limit is reached, you must take a \glossterm{short rest} before flying again.

        \ff[13]{Holy Retribution} Whenever an evil creature makes a \glossterm{strike} against the creature protected by your \textit{holy protection} ability, you make a \textit{celestial power} vs. Mental attack against the attacking creature.
        Success means the attacker takes \glossterm{standard damage} \minus1d.

        \ff[15]{Empowered Blessing} The accuracy bonus from your \textit{holy blessing} ability increases to \plus2.

        \ff[17]{} % another augment?

        \ff[19]{Greater Angelic Flight} You no longer have a limit on how long you can fly with your \textit{angelic flight} ability.
    \end{feat}

    \begin{feat}{Class Dedication}{Class}
        \featpre Class Versatility feat.
        \featben

        \ff[1]{Additional Archetype} Choose an archetype from a class you have chosen with the \textit{additional class} ability of the Class Versatility feat (see \featpref{Class Versatility}).
        You gain all abilities from that archetype.
    \end{feat}

    \begin{feat}{Class Versatility}{Class}
        \featben

        \ff[1]{Additional Class} Choose a class.
        You gain the \glossterm{class skills} of that class in addition to your existing class skills.
        You can exchange one class archetype from your class with one class archetype from that class.
        You can also exchange any number of your basic class abilities, such as skill points or weapon proficiencies, for the corresponding abilities of that class.
        If that class has any basic class abilities which are not part of an archetype and do not have abilities of the same on other classes, such as a cleric's \textit{divine power}, you gain those abilities.

        \ff[6]{Versatile Expertise} Choose one of the following benefits.
        You gain that benefit.
        \begin{itemize}
            \item \plus1 bonus to Fortitude defense
            \item \plus1 bonus to Reflex defense
            \item \plus1 bonus to Mental defense
            \item One extra \glossterm{skill point}
            \item Extra \glossterm{hit points} equal to half your level
        \end{itemize}

        \ff[12]{Expanded Versatility} You gain the \textit{versatile expertise} ability again.

        \ff[18]{Expanded Versatility} You gain the \textit{versatile expertise} ability again.
    \end{feat}

    \begin{feat}{Climb Specialization}{Skill}
        \featpre Climb as a mastered skill.
        \featben

        \ff[1]{Lesser Specialization} You gain a \plus2 bonus to Climb.

        \ff[2]{Damage Tolerance} Taking damage while climbing does not force you to make an additional Climb check to avoid falling.

        \ff[4]{Climb Speed} You gain a \glossterm{climb speed} equal to half your land speed.
        A successful Climb check to move allows you to travel a distance equal to your climb speed.

        % \ff[6]{Scale the Beast}
        % You gain a \plus2 bonus to Climb attacks you make to climb on other creatures (see \pcref{Creature Climb}).
        % In addition, you gain a \plus1 bonus to accuracy with physical attacks against creatures you are climbing on.

        \ff[8]{Specialization} The bonus to Climb increases to \plus4.

        \ff[10]{Greater Climb Speed} Your climb speed increases to be equal to your land speed.

        \ff[12]{Impossible Climber} You can climb surfaces that are perfectly smooth.
        The DR is 30 for perfectly smooth vertical surfaces, and 40 to climb on a perfectly smooth ceiling.
        In addition, you are treated as one size category smaller than normal for the purpose of determining which creatures you can climb on.

        \ff[14]{Greater Scale the Beast} The bonus to Climb attack to climb on creatures increases to \plus4.
        In addition, the bonus to physical accuracy against creatures you are climbing on increases to \plus2.

        \ff[16]{Greater Specialization} The bonus to Climb increases to \plus6.

        \ff[18]{Greater Impossible Climber} You can wallrun on ceilings in the same way you wallrun on walls.
        In addition, the size category decrease for the purpose of climbing on creatures improves to two size categories smaller than normal.
    \end{feat}

    \begin{feat}{Conjurer}{Magical, Spell}
        \featpre Conjuration spell known.
        \featben

        \ff[1]{Fortified Manifestation} Objects and creatures you create with \glossterm{Manifestation} abilities have additional hit points equal to your spellpower, and gain a \plus1d bonus to damage with all attacks.

        \ff[3]{Astral Spirit} When you cast the \spell{summon monster} spell, you can manifest an astral spirit instead of an animal.
        An astral spirit is a floating, spirit-like creature with a translucent body.
        Its size is Medium, and it is vaguely humanoid in shape.
        It has a physical form, and occupies space like any other creature.

        An astral spirit does not have a land speed, but it has a 30 foot \glossterm{fly speed} with good maneuverability.
        In addition, it can teleport any distance as a move action as long as its destination is within \rngmed range of you.
        If an astral spirit hits with its strike, it deals physical damage.

        \ff[5]{Astral Spell Transit} Your attacks with spells ignore \glossterm{cover}, but not \glossterm{total cover}.

        \ff[7]{Distant Spells} You double the range of all spells you cast.
        In addition, all \glossterm{Teleportation} spells you cast can teleport twice their normal distance.

        \ff[9]{Regenerating Manifestations} Whenever you cast a spell, objects and creatures you have created with \glossterm{Manifestation} abilities heal hit points equal to your spellpower.

        \ff[11]{Sustained Manifestations} Once per round, when you cast or sustain a spell, you can also sustain a \glossterm{Manifestation} spell as a \glossterm{free action}.

        \ff[13]{Greater Astral Spell Transit} When determining whether you have \glossterm{line of effect} to a particular location with a spell, you can ignore all physical obstacles in a single five-foot span.
        This can allow you to cast spells through solid walls, though it does not grant you the ability to see through the wall.

        \ff[15]{Greater Fortified Manifestation} The hit point bonus from your \textit{fortified manifestation} ability increases to twice your spellpower, and the damage bonus increases to \plus2d.

        \ff[17]{} 

        \ff[19]{Greater Astral Echo} You constantly drift between your plane and the Astral Plane.
        All attacks against you have a 20\% failure chance.
        You can suppress or resume this ability as a \glossterm{minor action}.
        In addition, whenever you teleport, your connection to the Astral Plane is strengthened until the end of the next round.
        During this time, all attacks against you have a 50\% failure chance.
    \end{feat}

    \begin{feat}{Craft Specialization}{Skill}
        \featpre Craft (any) as a mastered skill.
        \featben

        \ff[1]{Lesser Specialization} You gain a \plus2 bonus to all Craft skills.

        \ff[2]{Craft Magic Item}[Magical] You can imbue items with magic using your crafting skill.
        Imbuing an item with magic takes material components, as described in \pcref{Magic Item Creation}.
        It takes you one hour per 10 gp of material components to create a item.

        You can also mend a broken magic item if it is one that you could make.
        Doing so costs a tenth of the raw materials and a quarter of the time it would take to craft that item in the first place.
        You cannot mend a \glossterm{destroyed} magic item.

        \ff[4]{Crafting Savant} You gain two additional \glossterm{skill points} which can only be spent on Craft skills.

        \ff[6]{Rapid Creation} Crafting magic items takes you one hour per 100 gp of material components.

        \ff[8]{Specialization} The bonus to Craft skills increases to \plus4.

        \ff[10]{Rapid Creation} Crafting magic items takes you one hour per 500 gp of material components.

        \ff[12]{Greater Crafting Savant} The number of extra skill points increases to four.

        \ff[14]{Rapid Creation} Crafting magic items takes you one hour per 2,500 gp of material components.

        \ff[16]{Greater Specialization} The bonus to Craft skills increases to \plus6.

        \ff[18]{Rapid Creation} Crafting magic items takes you one hour per 12,500 gp of material components.
    \end{feat}

    \begin{feat}{Creature Handling Specialization}{Skill}
        \featpre Creature Handling as a mastered skill.
        \featben

        \ff[1]{Lesser Specialization} You gain a \plus2 bonus to Creature Handling.

        \ff[2]{Sustained Pacify} You can sustain the \textit{pacify} ability from the Creature Handling skill as a minor action, rather than as a standard action (see \pcref{Pacify}).

        \ff[4]{Efficient Training} You can teach a creature a trick with 12 hours of work, split as you choose, rather than in a week of 4 hour sessions (see \pcref{Training Creatures}).
        In addition, you can train creatures to learn two bonus tricks beyond their normal maximum (see \pcref{Bonus Tricks}).

        \ff[6]{Species Tolerance} You reduce Creature Handling DR modifiers for handling non-animals by 2.

        \ff[8]{Specialization} The bonus to Creature Handling increases to \plus4.

        \ff[10]{Battleforged Training} You can teach a creature the Battleforged trick.
        The DR to train the trick is 20.
        A creature with the trick gains the following benefits:
        \begin{itemize}
            \item Its maximum hit points increase by an amount equal to its level.
            \item It gains a \plus1 bonus to accuracy with all attacks.
            \item It gains a \plus1d bonus to \glossterm{strike damage}.
        \end{itemize}
        In addition, the number of bonus tricks you can teach from your \textit{efficient training} ability increases to four.

        \ff[12]{Rapid Pacify} You can use the \textit{pacify} ability from the Creature Handling skill as a minor action, rather than as a standard action.

        \ff[14]{Greater Species Tolerance} The reduction of DR modifiers for handling non-animals increases to 5.
        This usually allows you to ignore penalties for working with non-animals.

        \ff[16]{Greater Specialization} The bonus to Creature Handling increases to \plus6.

        \ff[18]{Greater Battleforged Training} You can teach a creature that has learned the Battleforged trick the Greater Battleforged trick.
        The DR to train the trick is 30.
        A creature with the trick gains the following benefits, which replace the benefits of the Battleforged trick:
        \begin{itemize}
            \item Its maximum hit points increase by an amount equal to twice its level.
            \item It gains a \plus2 bonus to accuracy with all attacks.
            \item It gains a \plus2d bonus to \glossterm{strike damage}.
        \end{itemize}
        In addition, the number of bonus tricks you can teach from your \textit{efficient training} ability increases to six.
    \end{feat}

    \begin{feat}{Devices Specialization}{Skill}
        \featpre Devices as a mastered skill.
        \featben

        \ff[1]{Lesser Specialization}[Magical] You gain a \plus2 bonus to Devices.

        \ff[2]{Rapid Improvisation} As a standard action, you can spend an \glossterm{action point} to use the \textit{improvise} ability to create a device (see \pcref{Improvise}).

        \ff[4]{Steady Hands} You cannot get a \glossterm{critical failure} on Devices checks.
        If you would get a \glossterm{critical failure}, you simply fail instead, and suffer the normal penalties for non-critical failure.

        \ff[6]{Lesser Disable Arcana}[Magical] You can disable arcane spell effects on objects or areas as if they were merely complex devices.
        You must be aware of an effect to disable it, either through the Spellcraft skill or because the effect is noticeable.
        You cannot disable effects on creatures.
        The DR to disable an effect is equal to 15 \add the effect's \glossterm{power}.
        Success means the spell is \glossterm{dismissed}, and its effects end.
        This has no effect on abilities that cannot be dismissed.

        \ff[8]{Specialization} The bonus to Devices increases to \plus4.

        \ff[10]{Improbable Improvisation} You reduce the DR for using the \textit{improvise} ability to make devices from unsuitable materials by 5.

        \ff[12]{Disable Arcana}[Magical] You can disable spell effects from any source, not just arcane spell effects.

        \ff[14]{Durable Improvization} Devices you create with the \textit{improvise} ability last for twice as many uses before they break.

        \ff[16]{Greater Specialization} The bonus to Devices increases to \plus6.

        \ff[18]{Greater Disable Arcana}[Magical] You can disable all magical effects on objects or areas, not just spell effects.
    \end{feat}

    \begin{feat}{Disguise Specialization}{Skill}
        \featpre Disguise as a mastered skill.
        \featben

        \ff[1]{Lesser Specialization} You gain a \plus2 bonus to Disguise.

        \ff[2]{Quick Change} As a standard action, you can spend an \glossterm{action point} to use the \textit{disguise creature} or \textit{emulate creature} ability.

        \ff[4]{Disguise Aura}[Magical] Whenever you use the \textit{disguise creature} or \textit{emulate creature} abilities, you can decide how the target and any items on the target appear when examined by Divination spells.
        For example, you could cause all of their equipment to appear nonmagical, or you could cause them to have a strong aura of good when examined with the \spell{detect alignment} spell.
        % Will this spell still exist?
        The maximum \glossterm{power} you can emulate is equal to your Disguise check result \minus15.

        Anyone using divination magic on the creature must make a spellpower check with a DR equal to your Disguise check result in order to perceive the truth.
        Regardless of the result of the check, the caster is not aware that the check was made.

        \ff[6]{Mismatch Tolerance} You reduce Disguise penalties for differences between the target's normal appearance and its intended appearance by 2.
        This allows you to ignore minor mismatches, such as if the target is a different gender than its intended appearance.

        \ff[8]{Specialization} The bonus to Disguise increases to \plus4.

        \ff[10]{Disguise Size}[Magical]
        \begin{ability}
            As a \glossterm{standard action}, you can spend an action point to use this ability.
            If you do, you increase or decrease your size by one \glossterm{size category}.
            This effect lasts as long as you \glossterm{attune} to it.
        \end{ability}

        \ff[14]{Greater Mismatch Tolerance} The reduction of Disguise penalties for appearance differences increases to 5.
        This can allow you to ignore significant appearance differences.

        \ff[16]{Greater Specialization} The bonus to Disguise increases to \plus6.
    \end{feat}

    \begin{feat}{Diviner}{Magical, Spell}
        \featpre Divination spell known.
        \featben

        \ff[1]{Prophesy} You gain the \textit{prophesy} ability.
        \begin{ability}{Prophesy}
            As a \glossterm{standard action}, you can spend an \glossterm{action point} to use this ability.
            To do so, you must visualize an action that a creature (or group of creatures) could take within the next hour.
            This time period is called the \textit{time of prophecy}.

            When you use this ability, you see a brief, cryptic vision describing the most likely outcome of the action you visualized.
            This vision does not reveal any consequences that might occur after the \textit{time of prophecy} has ended.
            The vision does not have to be a literally accurate representation of the future.
            For example, if you used this ability to foresee the results of entering a room that had a group of creatures waiting in ambush, you might see a vision of flashing daggers in darkness darting towards your exposed back, regardless of whether the creatures would actually use daggers to attack.

            After using this ability, you cannot use it again until the \textit{time of prophecy} has ended, regardless of whether the action was taken.
        \end{ability}

        \ff[3]{Lesser Precognitive Reaction} You gain a \plus1 bonus to Reflex defense.
        In addition, you gain a \plus2 bonus to \glossterm{initiative} checks.

        \ff[5]{Deep Prophecy} When you use your \textit{prophesy} ability, you can increase the \textit{time of prophecy} to eight hours.
        This affects both the distance you can see into the future and the time you must wait before using the ability again.

        \ff[7]{Truesight} You gain the \glossterm{truesight} ability with a 50 foot range.

        \ff[9]{Precognitive Reaction} The bonus to initiative checks increases to \plus4.
        In addition, you are aware of all attacks against you, even those you cannot see, as long as you are conscious.
        This allows you to use abilities to defend yourself, and prevents you from being \unaware.

        \ff[11]{Dual Prophecy} You may use your \textit{prophesy} ability while you have an active \textit{time of prophecy}.
        If you have two active \textit{times of prophecy}, you must wait until one has expired to use your \textit{prophesy} ability again.

        \ff[13]{Greater Truesight} The range of your \glossterm{truesight} increases to 500 feet.

        \ff[15]{Flexible Prophecy} When you use your \textit{prophesy} ability, you can choose the \textit{time of prophecy} to be any five-minute increment of time, to a minimum of thirty minutes and a maximum of eight hours.

        \ff[17]{Greater Precognitive Reaction} The bonus to Reflex defense increases to \plus2, and the bonus to initiative checks increases to \plus6.
        In addition, you can never be surprised in combat.
        % Whenever there are \glossterm{surprise phases}, you can act in them.

        \ff[19]{Oracle} When you use your \textit{prophesy} ability, the maximum \textit{time of prophesy} you can choose is increased to one week.
        In addition, you may have three active \textit{times of prophecy}, rather than two.
    \end{feat}

    \begin{feat}{Draconic Heritage}{Bloodline}
        \parhead{Special} You can only have one Bloodline feat.
        \featben

        \ff[1]{Draconic Power} Your \glossterm{power} with abilities from this feat is equal to your level or your Constitution, whichever is higher.

        \ff[1]{Draconic Ancestry} Choose a type of dragon from among the dragons on \trefnp{Dragon Types}.
        You have the blood of that type of dragon in your veins.
        This grants you damage reduction equal to twice your \textit{draconic power} against the damage type that dragon's breath weapon deals.

        \ff[1]{Low-Light Vision} You gain \glossterm{low-light vision}.
        If you already have low-light vision, you instead double the benefit, allowing you to quadruple the illumination range of light sources.

        \ff[2]{Draconic Weapons} You gain a bite natural weapon and two claw natural weapons, one on each arm.
        For details, see \pcref{Natural Weapons}.

        \ff[2]{Darkvision} You gain \glossterm{darkvision} with a 50 foot range.
        If you already have darkvision, you instead increase the range of your existing darkvision by 50 feet.

        \ff[4]{Draconic Wings} You gain scaly wings that sprout from your back.
        These wings grant you a glide speed equal to your land speed (see \pcref{Gliding}).
        The wings themselves are physical, but the ability to glide with them is \glossterm{magical}.

        \ff[6]{Breath Weapon} You gain the \textit{breath weapon} ability.
        \begin{ability}{Breath Weapon}
            As a \glossterm{standard action}, you can spend an \glossterm{action point} to use this ability.
            If you do, make a \textit{draconic power} vs. Reflex attack against everything in the area defined by your \textit{draconic ancestry} ability (see \trefnp{Dragon Types}).
            A hit deals \glossterm{standard damage} against each target.
            The damage type is defined by your \textit{draconic ancestry} ability.
        \end{ability}

        \ff[8]{Lesser Draconic Flight}[Magical] 
        As a standard action, if you are \glossterm{unencumbered}, you can use your draconic wings to fly with a \glossterm{fly speed} equal to your land speed.

        \ff[10]{Widened Breath} The area affected by of your \textit{breath weapon} increases.
        A line breath weapon becomes a \areahuge, 10 ft.\ wide line.
        A cone breath weapon becomes a \arealarge cone.

        % \ff[12]{Greater Draconic Weapons}[Magical] The natural weapons gain an \glossterm{enhancement bonus} equal to one quarter of your \textit{draconic power}.

        \ff[14]{Draconic Flight}[Magical] You can use your \textit{draconic flight} ability to fly as a \glossterm{minor action}.

        \ff[16]{Devastating Breath} The damage dealt by your \textit{breath weapon} increases by \plus1d.

        \ff[18]{Greater Draconic Flight}[Magical] You gain a fly speed equal to your land speed.
    \end{feat}

    \begin{dtable}
        \lcaption{Dragon Types}
        \begin{dtabularx}{\columnwidth}{>{\lcol}X >{\lcol}X >{\lcol}X}
            \tb{Dragon} & \tb{Energy Type} & \tb{Breath Weapon} \\
            \bottomrule
            Black & Acid & Line \\
            Blue & Electricity & Line \\
            Brass & Fire & Line \\
            Bronze & Electricity & Line \\
            Copper & Acid & Line \\
            Gold & Fire & Cone \\
            Green & Acid & Cone \\
            Red & Fire & Cone \\
            Silver & Cold & Cone \\
            White & Cold & Cone \\
        \end{dtabularx}
    \end{dtable}

    \begin{feat}{Duelist}{Combat}
        \featpre Starting Dexterity of 1, starting Intelligence of 1.
        \featben

        \ff[1]{Focused Strike} You gain the \textit{focused strike} ability.
        \begin{ability}{Focused Strike}
            As a standard action, you can spend an \glossterm{action point} to use this ability.
            If you do, make a melee \glossterm{strike}.
            If you are not \glossterm{overwhelmed}, you gain a \plus2d bonus to damage with the strike.
            If your target is not \glossterm{overwhelmed}, you gain a \plus1d bonus to damage with the strike.
            These bonuses stack.
        \end{ability}

        \ff[2]{Parry} At the start of each phase, you can choose a creature you can see.
        If you are wielding a melee weapon, you gain a \plus1 bonus to Armor defense against attacks by that target during that phase.

        \ff[4]{Duel Expertise} You gain a \plus1 bonus to \glossterm{accuracy} with the \textit{disarm} and \textit{feint} abilities (see \pcref{Disarm}).

        \ff[6]{Greater Focused Strike} You gain a \plus1 bonus to \glossterm{accuracy} with your \textit{focused strike} ability.

        \ff[8]{Focused Parry} You do not suffer \glossterm{overwhelm penalties} against attacks from the target of your \textit{parry} ability.
        In addition, when you use your \textit{focused strike} ability against that creature, you are not considered to be \glossterm{overwhelmed}.

        \ff[10]{Greater Duel Expertise} The accuracy bonus from your \textit{duel expertise} ability increases to \plus2. 

        \ff[12]{Supreme Focused Strike} The accuracy bonus from your \textit{greater focused strike} ability increases to \plus2.

        \ff[14]{Riposte} When the target of your \textit{parry} ability attacks you, you gain a \plus1 bonus to \glossterm{accuracy} with \glossterm{physical attacks} against that creature until the end of the next round.

        \ff[16]{Supreme Duel Expertise} The accuracy bonus from your \textit{duel expertise} ability increases to \plus3. 

        \ff[18]{}
    \end{feat}

    \begin{feat}{Eldritch Knight}{Spell}
        \featpre Ability to cast a spell.
        \featben

        \ff[1]{Lesser Combat Concentration} You gain a \plus2 bonus to Concentration checks made to cast spells.

        \ff[1]{Lesser Armor Tolerance} You reduce your \glossterm{encumbrance penalty} from \glossterm{armor} by 1.

        \ff[2]{Spellstrike}[Magical] You gain the \textit{spellstrike} ability.
        \begin{ability}{Spellstrike}
            As a standard action during the \glossterm{action phase}, you can spend an \glossterm{action point} to use this ability.
            If you do, you imbue magical power into a weapon you wield or your \glossterm{unarmed attack}.
            This requires concentration as if casting a spell, but breaking your concentration does not cause a \glossterm{miscast backlash}.
            During the \glossterm{delayed action phase}, you may make a \glossterm{strike} with the chosen weapon.
            If you maintained your concentration, the strike gains a \plus3d bonus to damage.
        \end{ability}

        \ff[4]{Combat Concentration} The bonus to Concentration checks increases to \plus5.

        \ff[6]{Armor Tolerance} The reduction of \glossterm{encumbrance penalty} increases to 2.

        \ff[8]{Seeking Spellstrike}[Magical] When you use your \textit{spellstrike} ability, if you maintained your concentration, you gain a \plus1 bonus to accuracy with the \glossterm{strike}.

        \ff[10]{Spellsword Rhythm}[Magical] Whenever you hit a creature with a \glossterm{strike}, you gain a \plus1 bonus to \glossterm{accuracy} with spells against that creature during the next round.
        In addition, whenever you hit a creature with a \glossterm{spell}, you gain a \plus1 bonus to \glossterm{accuracy} with \glossterm{physical attacks} against that creature during the next round.

        \ff[12]{Greater Combat Concentration} The bonus to Concentration checks increases to \plus10.

        \ff[14]{Greater Seeking Spellstrike}[Magical] The accuracy bonus from your \textit{seeking spellstrike} ability increases to \plus2.

        \ff[16]{Greater Armor Tolerance} The reduction of \glossterm{encumbrance penalty} increases to 3.

        \ff[18]{Greater Spellsword Rhythm}[Magical] The accuracy bonus increases to \plus2.
    \end{feat}

    \begin{feat}{Enchanter}{Magical, Spell}
        \featpre Enchantment spell known.
        \featben

        \ff[1]{Mind Fragments} When you use \glossterm{Mind} abilities, you can affect creatures with a \glossterm{mundane} immunity to \glossterm{Mind} abilities.
        You take a \minus5 penalty to accuracy on attacks against such creatures.
        This does not allow you to affect creatures with a \glossterm{magical} immunity to \glossterm{Mind} abilities.

        \ff[3]{Enchanting Presence} You gain a \plus1 bonus to Intimidate and Persuasion.
        In addition, creatures within a 50-foot radius \glossterm{emanation} of you take a \minus1 penalty to Mental defense.
        You may freely exclude creatures you are aware of from this effect.

        \ff[5]{Subtle Influence} You gain a \plus1 bonus to accuracy with \glossterm{Mind} abilities.
        In addition, the DR to identify your \glossterm{Mind} abilities with Spellcraft, and to identify their effects with Sense Motive, increases by 5.

        \ff[7]{Greater Mind Fragments} The accuracy penalty from your \textit{mind fragments} ability decreases to \minus2.

        \ff[9]{Greater Enchanting Presence} The bonuses and penalties from your \textit{enchanting presence} ability increase to \plus2 and \minus2.

        \ff[11]{Greater Subtle Influence} The accuracy bonus from your \textit{subtle influence} ability increases to \plus2.
        In addition, the DR increase from that ability increases to 10.

        \ff[15]{Supreme Mind Fragments} The accuracy penalty from your \textit{mind fragments} ability is removed.

        \ff[17]{Supreme Enchanting Presence} The bonuses and penalties from your \textit{enchanting presence} increase to \plus2 and \minus2.

        \ff[19]{Mental Torment} Whenever you target a creature with a Mind ability, that creature takes a \minus1 penalty to Mental defense.
        This is a \glossterm{condition}, and lasts until it is removed.
        This penalty stacks with itself if you target the same creature multiple times.
    \end{feat}

    \begin{feat}{Escape Artist Specialization}{Skill}
        \featpre Escape Artist as a mastered skill.
        \featben

        \ff[1]{Lesser Specialization} You gain a \plus2 bonus to Escape Artist.

        \ff[2]{Rapid Escape} You can squeeze and escape bindings and grapples as a move action, rather than as a standard action.

        \ff[4]{Constraint Tolerance} You reduce your penalties for \glossterm{squeezing} by 1 (see \pcref{Squeezing}).

        \ff[6]{Accelerated Squeeze} Your movement speed is not reduced while squeezing.

        \ff[8]{Specialization} The bonus to Escape Artist increases to \plus4.

        \ff[10]{Escape Magic}[Magical] You gain the \textit{escape magic} ability.
        \begin{ability}{Escape Magic}
            As a standard action, you can spend an \glossterm{action point} to use this ability.
            If you do, you make an Escape Artist attack against all \glossterm{magical} effects on you.
            The DR for each effect is equal to 10 \add the effect's \glossterm{power}.
            On a hit, each effect is \glossterm{dismissed}, if it is an effect that can be dismissed.
        \end{ability}

        You can only dismiss effects with this ability which target you directly, not area effects which include you as a target.
        If an ability targets multiple creatures, you can only remove its effects on you.

        \ff[12]{Greater Rapid Escape} You can escape bindings and grapples as a \glossterm{minor action}.

        \ff[14]{Greater Constraint Tolerance} The penalty reduction from your \textit{constraint tolerance} ability increases to 2.
        This typically allows you to ignore all penalties for \glossterm{squeezing}.

        \ff[16]{Greater Specialization} The bonus to Escape Artist increases to \plus6.

        \ff[18]{Greater Escape Magic}[Magical] You can use your \textit{escape magic} ability as a \glossterm{minor action}.
    \end{feat}

    \begin{feat}{Evoker}{Magical, Spell}
        \featpre Evocation spell known.
        \featben

        \ff[1]{Energy Burst} You gain the \textit{energy burst} ability.
        \begin{ability}{Energy Burst}
            As a standard action, you can spend an \glossterm{action point} to use this ability.
            If you do, choose a creature or object within \rngclose range and a type of \glossterm{energy}: cold, electricity, fire, or sonic.
            You make a \glossterm{spellpower} attack against the target's Reflex defense if you chose electricity or fire, or against the target's Fortitude defense if you chose cold or sonic.
            On a hit, the target takes \glossterm{standard damage} \plus1d of the chosen damage type.
            This ability has the \glossterm{ability tag} corresponding to the type of energy chosen.
        \end{ability}

        % Which spellpower?
        \ff[3]{Energy Resistance} You gain damage reduction equal to your spellpower against \glossterm{energy damage}.

        \ff[5]{Energy Affinity} You gain a \plus1d bonus to damage with attacks that deal \glossterm{energy damage} measured in dice.

        \ff[7]{Residual Energy Burst} When your \textit{energy burst} attack hits, the target suffers an additional effect depending on the energy type chosen.
        These effects are \glossterm{conditions}, and last until they are removed.
        \begin{itemize}
            \item Cold: If target is a creature, it is \fatigued.
                A \glossterm{critical hit} makes the target \exhausted instead.
            \item Electricity: If the target is a creature, it is \dazed.
                A \glossterm{critical hit} makes the target \stunned instead.
            \item Fire: The target is \ignited.
            \item Sonic: If the target has \glossterm{hardness}, its hardness is reduced by an amount equal to half your spellpower. Otherwise, if it has \glossterm{damage reduction}, its damage reduction is reduced by amount equal to your spellpower.
        \end{itemize}

        \ff[9]{Potent Evocation} You gain a \plus1d bonus to damage with Evocation abilities that deal damage measured in a dice pool, as well as with the \textit{energy burst} ability.

        \ff[11]{Greater Energy Resistance} The damage reduction from your \textit{energy resistance} ability increases to twice your spellpower.

        \ff[13]{Greater Energy Affinity} The accuracy bonus from your \textit{energy affinity} ability increases to \plus2.

        \ff[15]{Devastating Evocation} The area affected by your Evocation spells that affect areas doubles.
    \end{feat}

    \begin{feat}{Executioner}{Combat}
        \featpres Starting Strength of 1, starting Perception of 2.
        \featben

        \ff[1]{Execution} You gain the \textit{execution} ability.
        \begin{ability}{Execution}
            As a standard action, you can spend an \glossterm{action point} to use this ability.
            If you do, make a \glossterm{strike}.
            At the end of the current phase, if the target is \glossterm{bloodied}, it takes additional damage equal to the damage you dealt to it with your strike.
        \end{ability}

        \ff[2]{Purge the Weak} You gain a \plus1 bonus to accuracy with physical attacks against \glossterm{bloodied} creatures.

        \ff[4]{Final Blow}[Death, Magical] At the end of each round, if you dealt \glossterm{vital damage} to a creature with a \glossterm{strike} that round, the creature immediately dies.

        \ff[6]{Bloodfeeder}[Life, Magical] Whenever a creature dies, if you dealt damage to it that round, you heal hit points equal to its level.

        \ff[8]{Greater Execution} When you use your \textit{execution} ability, you gain a \plus1d bonus to damage with the strike.

        \ff[10]{Greater Purge the Weak} The accuracy bonus from your \textit{purge the weak} ability increases to \plus2.

        \ff[12]{Greater Final Blow}[Death, Magical] At the end of each round, if a creature has no hit points remaining and you damaged it with a \glossterm{strike} that round, the creature immediately dies.

        \ff[14]{Greater Bloodfeeder}[Life, Magical] Whenever a creature dies, if you dealt damage to it that round and you used your \textit{execution} ability on it since the last time it took a \glossterm{short rest}, you regain an \glossterm{action point}.

        \ff[16]{Supreme Execution} The damage bonus with your \textit{execution} ability increases to \plus2d.

        \ff[18]{Supreme Purge the Weak} The accuracy bonus from your \textit{purge the weak} ability increases to \plus3.
    \end{feat}

    \begin{feat}{Guardian}{Combat}
        \featpre Starting Perception and Willpower of 1.
        \featben

        \ff[1]{Binding Strike} You gain the \textit{binding strike} ability.
        \begin{ability}{Binding Strike}
            As a standard action, you can spend an \glossterm{action point} to use this ability.
            If you do, you make a melee \glossterm{strike} against an adjacent creature.
            In addition to the strike's normal effects, you also compare the attack result against the target's Mental defense.

            On a hit, the target is \glossterm{immobilized} as a \glossterm{condition}.
            This effect immediately ends if you are \glossterm{defeated} or if you stop being adjacent to the target.
            On a critical hit, the immobilization does not end if you stop being adjacent to the target.
        \end{ability}

        \ff[2]{Protect} You gain the \textit{protect} ability.
        \begin{ability}{Protect}[\glossterm{Swift}]
            As a \glossterm{minor action}, you can use this ability to focus on defending an ally.
            If you do, one ally adjacent to you gains a \plus2 bonus to Armor defense.
            However, you take a \minus2 penalty to Armor defense.
            This effect lasts until the end of the round.
        \end{ability}

        \ff[4]{Defend the Weak} Allies adjacent to you gain a \plus1 bonus to \glossterm{overwhelm resistance}.

        \ff[6]{Certain Bind} You gain a \plus1 bonus to accuracy with your \textit{binding strike} ability.

        \ff[8]{Redirection}[Magical] You gain the \textit{redirection} ability.
        \begin{ability}{redirection}
            When you use your \textit{protect} ability, you can spend an \glossterm{action point} to use this ability.
            If you do, you suffer all effects from \glossterm{strikes} that hit the target in place of the target as long as the \textit{protect} ability lasts.
            Any abilities you have that would make the attack miss or fail have no effect, but your abilities that allow you to reduce or ignore its effects work normally.
        \end{ability}

        \ff[10]{Greater Defend the Weak} The bonus from your \textit{defend the weak} ability increases to \plus2.

        \ff[12]{Greater Certain Bind} The accuracy bonus with your \textit{binding strike} ability increases to \plus2.

        \ff[14]{Expanded Redirection}[Magical] When you use your \textit{redirection} ability, you redirect the effects of all attacks, not just \glossterm{strikes}.

        \ff[16]{Supreme Defend the Weak} The bonus from your \textit{defend the weak} ability increases to \plus3.

        \ff[18]{Supreme Protector} When you use your \textit{protect} ability, you may target any number of allies adjacent to you.
    \end{feat}

    \begin{feat}{Heal Specialization}{Skill}
        \featpre Heal as a mastered skill.
        \featben

        \ff[1]{Lesser Specialization} You gain a \plus2 bonus to Heal.

        \ff[2]{Healing Touch}[Magical] You gain the \textit{healing touch} ability.
        \begin{ability}{Healing Touch}[\glossterm{Life}]
            As a standard action, you can spend an \glossterm{action point} to use this ability.
            If you do, make a Heal check on a willing creature you can touch.
            The target is healed for an amount equal to the Heal check result.
        \end{ability}

        \ff[4]{Lifesaver} You gain a \plus5 bonus to Heal checks to stabilize dying creatures (see \pcref{Dying}).

        \ff[6]{Vital Healing} For every five points of healing you would restore with your \textit{healing touch}, you can instead heal a point of \glossterm{vital damage}.

        \ff[8]{Specialization} The bonus to Heal increases to \plus4.

        \ff[10]{Purging Touch}[Magical] You gain the \textit{purging touch} ability.
        \begin{ability}{Purging Touch}
            As a standard action, you can spend an \glossterm{action point} to use this ability.
            If you do, make a Heal check on a willing creature you can touch.
            For each poison and disease on the target, if your check result is at least 10 higher than the \glossterm{power} of the effect, the effect is removed.
        \end{ability}

        \ff[12]{Empowered Healing} When you use your \textit{healing touch} ability, you can heal additional hit points equal to your level.

        \ff[14]{Greater Lifesaver} You can stabilize dying creatures as a \glossterm{minor action}.

        \ff[16]{Greater Specialization} The bonus to Heal increases to \plus6.

        \ff[18]{Wellspring of Life} You do not have to spend an \glossterm{action point} to use your \textit{healing touch} ability.
    \end{feat}

    \begin{feat}{Illusionist}{Magical, Spell}
        \featpre Illusion spell known.
        \featben

        \ff[1]{Create Image} You gain the \textit{create image} ability.
        \begin{ability}{Create Image}[\glossterm{Figment}, \glossterm{Sustain} (minor)]
            As a standard action, you can spend an \glossterm{action point} to use this ability.
            If you do, you create a visual illusion of an object, creature, or force within \rngmed range.
            The figment's size must be no smaller than Tiny, and no larger than Large.
            The figment does not create sound, smell, or temperature.

            During the movement phase, you can move the figment anywhere within the range, with appropriate motions to simulate natural movement.
            For example, if you created the illusion of a squad of human guards, you could cause them to walk realistically across a room.
            The figments otherwise remain motionless, except for minor motions that simulate signs of life (if appropriate).
            If the figment ever leaves this ability's range, the effect immediately ends.

            The maximum intensity of a sensation created by this ability is not enough to have any significant detrimental effects on a human experiencing the sensation.
            For example, it can create a bright light, but not so bright that it would be physically painful to view.
            % should the light be visible outside the size of the illusion? If so, how to clarify?

            When you use this ability, you make a check with a bonus equal to your spellpower \add 5.
            Creatures can recognize the figment is created by illusory magic by interacting with it physically, or by making an Awareness check against a DR equal to your check result when using this ability.
            A creature gets a \plus10 bonus on this Awareness check when using senses which should be present in the figment, but which are missing.
        \end{ability}

        \ff[3]{Reflexive Illusion} You gain a \plus1 bonus to Armor defense, Disguise, Sleight of Hand, and Stealth.

        \ff[5]{Muffled Illusions} You learn the Silent \glossterm{augment}.
        In addition, the level cost to apply the Silent \glossterm{augment} to Illusion spells you cast is reduced by 1, to a minimum of 0 (see \pcref{Augments}).
        % How loud is the sound? How pungent is the smell? etc.

        \ff[7]{Control Image} You gain the \textit{control image} ability.
        \begin{ability}{Control Image}{\glossterm{Swift}}
            While your \textit{create image} ability is active, you can use this ability as a standard action.
            If you do, you can directly control the figment's movement for the rest of the round.
            You cannot alter the fundamental shape of the figment, but you can have it perform complex actions, such as pretending to fight other creatures or dancing.
            The figment's ability to simulate physical tasks that require dexterity or training, such as juggling, is limited by your own.
            You may need to make relevant checks to make the figment perform complex actions.
        \end{ability}

        \ff[9]{Greater Reflexive Illusion} The bonuses to skills from your \textit{reflexive illusion} ability increase to \plus2.

        \ff[11]{Hidden Illusions} You learn the Stilled \glossterm{augment}.
        In addition, the level cost to apply the Stilled augment to Illusion spells you cast is reduced by 1, to a minimum of 0 (see \pcref{Augments}).

        \ff[13]{Flexible Image} Your \textit{create image} ability can create figments between Diminuitive and Huge size.
        In addition, choose a sense: sound, smell, or temperature.
        Your \textit{create image} ability can create sensations with the chosen sense.

        \ff[15]{Supreme Reflexive Illusion} The bonus to Armor defense from your \textit{reflexive illusion} ability increases to \plus2, and the bonuses to skills increase to \plus3.

        \ff[17]{Innate Illusions} The level cost to apply both the Silent and Stilled \glossterm{augments} to Illusion spells is removed.

        \ff[19]{Greater Control Image} You can use your \textit{control image} ability as part of the same action you use to sustain your \textit{create image} ability. 
    \end{feat}

    \begin{feat}{Intimidate Specialization}{Skill}
        \featpre Intimidate as a mastered skill.
        \featben

        \ff[1]{Lesser Specialization} You gain a \plus2 bonus to Intimidate.

        \ff[2]{Demoralizing Blow} You gain the \textit{demoralizing blow} ability.
        \begin{ability}{Demoralizing Blow}
            As a standard action, you can use this ability.
            If you do, make a \glossterm{strike}.
            % TODO: Timing is wrong
            If you deal damage to the target, you can immediately use the \textit{demoralize} ability on the target (see \pcref{Demoralize}).
        \end{ability}

        \ff[6]{Critical Demoralization} If you get a \glossterm{critical hit} when you take the \textit{demoralize} action, the target is \frightened by you instead of being shaken.
        See \pcref{Demoralize}, for details.

        \ff[8]{Specialization} The bonus to Intimidate increases to \plus4.

        \ff[16]{Greater Specialization} The bonus to Intimidate increases to \plus6.
    \end{feat}

    \begin{feat}{Iron Will}{General}
        \featpre Starting Willpower of 2.
        \featben

        \ff[1]{Mental Discipline} You gain a \plus2 bonus to Mental defense.

        \ff[2]{Resilience} You gain additional hit points equal to your Willpower.

        \ff[4]{Unclouded Mind} You are immune to being \glossterm{dazed}.

        \ff[6]{Mind over Matter} You reduce your penalties for being \glossterm{staggered} by 2.

        \ff[8]{Greater Resilience} The bonus hit points from your \textit{resilience} ability increase to twice your Willpower.

        \ff[10]{Greater Unclouded Mind} You are immune to being \glossterm{stunned}.

        \ff[12]{Greater Mental Discipline} The defense bonus from your \textit{mental discipline} ability increases to \plus4.

        \ff[14]{Greater Mind over Matter} You do not take penalties for being \glossterm{staggered}.

        \ff[16]{Supreme Resilience} The bonus hit points from your \textit{resilience} ability increase to four times your Willpower.

        \ff[18]{Mental Fortress} You are immune to all hostile \glossterm{Mind} abilities.
    \end{feat}

    \begin{feat}{Jump Specialization}{Skill}
        \featpre Jump as a mastered skill.
        \featben

        \ff[1]{Lesser Specialization} You gain a \plus2 bonus to Jump.

        \ff[2]{Instant Leap} You must move at least five feet before jumping to have a running start, rather than twenty feet.

        \ff[4]{Leaping Strike} You gain the \textit{leaping strike} ability.
        \begin{ability}{Leaping Strike}
            As a standard action, you can spend an \glossterm{action point} to use this ability.
            If you do, make a Jump check to leap and move as normal for the leap, up to a maximum distance equal to your land speed (see \pcref{Leap}).
            If you use this ability during the \glossterm{action phase}, you can also make a \glossterm{strike} from your new location during the \glossterm{delayed action phase}.
        \end{ability}

        \ff[6]{Featherlight Jump} When you jump, you can spend an \glossterm{action point} as a \glossterm{free action}.
        If you do, your maximum height for that jump is equal to your Jump check result, rather than half your Jump check result.

        \ff[8]{Specialization} The bonus to Jump increases to \plus4.

        \ff[10]{Death from Above} When you use your \textit{leap strike} ability, you can make the strike during the \glossterm{action phase} instead of the \glossterm{delayed action phase}.
        In addition, when you make the strike, you can be in any location along your jump's path.
        If you are above your target when you make the strike, you gain a \plus1d bonus to damage.

        \ff[12]{Featherlight} You do not need to spend an \glossterm{action point} to use your \textit{featherlight jump} ability.

        \ff[14]{Greater Instant Leap} You are always considered to have a running start when jumping, even when rebounding off of objects (see \pcref{Rebounding Leap}).

        \ff[16]{Greater Specialization} The bonus to Jump increases to \plus6.

        \ff[18]{Greater Death from Above} The bonus to damage for being above your target increases to \plus2d.
    \end{feat}

    \begin{feat}{Knowledge Specialization}{Skill}
        \featpre Knowledge as a mastered skill.
        \featben

        \ff[1]{Lesser Specialization} You gain a \plus2 bonus to Knowledge.

        \ff[4]{Knowledge Savant} You gain two additional \glossterm{skill points} which can only be spent on Knowledge skills.

        \ff[8]{Specialization} The bonus to Knowledge increases to \plus4.

        \ff[12]{Greater Knowledge Savant} The number of extra skill points increases to four.

        \ff[16]{Greater Specialization} The bonus to Knowledge increases to \plus6.
    \end{feat}

    \begin{feat}{Leadership}{Combat}
        \featpre Willpower 2.
        \featben

        \ff[1]{Bolster} You gain the \textit{bolster} ability.
        \begin{ability}{Bolster}[\glossterm{Mind}]
            As a standard action, you can spend an \glossterm{action point} to use this ability.
            If you do, all allies other than yourself within a \arealarge radius burst from you can remove one \glossterm{condition}.
            In addition, each target heals hit points equal to your Willpower.
        \end{ability}

        \ff[2]{Battle Command} You gain the \textit{battle command} ability.
        \begin{ability}{Battle Command}[\glossterm{Swift}]
            As a standard action, you can use this ability.
            If you do, choose an ally within \rngmed range of you.
            Until the end of the round, whenever it makes a \glossterm{physical attack}, it can roll the \glossterm{attack roll} twice and take the higher result.
        \end{ability}

        \ff[4]{Inspiring Presence} All allies other than yourself within a \arealarge radius emanation from you gain a \plus1 bonus to Mental defense.
        This ability is \glossterm{suppressed} if you are \glossterm{defeated}.

        \ff[6]{Greater Bolster} The healing from your \textit{bolster} ability increases to twice your Willpower.

        \ff[8]{Potent Command} The target of your \textit{battle command} also gains a \plus1d bonus to \glossterm{strike damage}.

        \ff[10]{Greater Inspiring Presence} The defense bonus from your \textit{inspiring presence} ability increases to \plus2.

        \ff[12]{Supreme Bolster} The healing from your \textit{bolster} ability increases to three times your Willpower.

        \ff[14]{Greater Potent Command} The damage bonus from your \textit{potent command} ability increases to \plus2d.

        \ff[16]{Supreme Inspiring Presence} The defense bonus from your \textit{inspiring presence} ability increases to \plus3.

        \ff[18]{War Leader} The area affected by your \textit{bolster} and \textit{inspiring presence} abilities increases to an \areahuge radius.
    \end{feat}

    \begin{feat}{Linguistics Specialization}{Skill}
        \featpre Linguistics as a mastered skill.
        \featben

        \ff[1]{Lesser Specialization} You gain a \plus2 bonus to Linguistics.

        \ff[2]{Linguistic Savant} You learn two additional \glossterm{common languages}, or one additional \glossterm{rare language}.

        \ff[8]{Specialization} The bonus to Linguistics increases to \plus4.

        \ff[12]{Greater Linguistic Savant} You learn two additional \glossterm{common languages}, or one additional \glossterm{rare language}.

        \ff[16]{Greater Specialization} The bonus to Linguistics increases to \plus6.
    \end{feat}

    \begin{feat}{Martial Training}{Combat}
        \featben

        \ff[1]{Equipment Training} You choose one of the following benefits.
        \begin{itemize}
            \item Proficiency with a category of \glossterm{armor}: light, medium, heavy body armor, or shields.
                You must be proficient with light armor to gain proficiency with medium armor, and you must be proficient with medium armor to gain proficiency with heavy armor.
            \item You reduce the \glossterm{encumbrance penalty} of \glossterm{body armor} you wear by 1.
                If you choose this ability multiple times, its effects stack.
            \item Proficiency with an additional \glossterm{weapon group} of your choice.
            \item Proficiency with \glossterm{exotic weapons} from a weapon group of your choice that you are already proficient with.
        \end{itemize}

        \ff[2]{Honed Strike} You gain the \textit{honed strike} ability.
        \begin{ability}{Honed Strike}
            As a standard action, you can spend an \glossterm{action point} to use this ability.
            If you do, you make a \glossterm{strike} with a \plus1d bonus to damage.
            If the strike misses, you regain the \glossterm{action point} spent to use this ability.
        \end{ability}

        \ff[4]{Equipment Training} You gain an additional \textit{equipment training} ability of your choice.

        \ff[6]{Practiced Defense} You gain a \plus1 bonus to Armor defense.

        \ff[8]{Equipment Training} You gain an additional \textit{equipment training} ability of your choice.

        \ff[10]{Greater Honed Strike} The damage bonus from your \textit{honed strike} ability increases to \plus2d.

        \ff[12]{Equipment Training} You gain an additional \textit{equipment training} ability of your choice.

        \ff[14]{Practiced Arruracy} You gain a \plus1 bonus to \glossterm{accuracy} with \glossterm{physical attacks}.

        \ff[16]{Equipment Training} You gain an additional \textit{equipment training} ability of your choice.

        \ff[18]{Supreme Honed Strike} The damage bonus from your \textit{honed strike} ability increases to \plus5d.
    \end{feat}

    \begin{feat}{Miscaster}{Magical, Spell}
        \featpre Ability to cast a spell.
        \featben

        \ff[1]{Selective Backlash} Your \glossterm{miscast backlash} does not hurt your allies, though it still hurts you.

        \ff[3]{Overchannel} You gain the \textit{overchannel} ability.
        \begin{ability}{Overchannel}
            Whenever you cast a spell, you can use this ability.
            Your concentration on the spell cannot be disrupted, and you cannot \glossterm{miscast} the spell for any other reason.
            Effects that prevent the spell from having any effect, such as the \textit{counterspell} ability from the Abjurer feat, work normally (see \featpref{Abjurer}).
            In addition, you cause a \glossterm{miscast backlash} when the spell resolves.

            However, using this ability causes a mystic backlash from channeling excess magical energy.
            During the next round, you cannot cast spells other than \glossterm{cantrips}, and you take a \minus2 penalty to all defenses.
        \end{ability}

        \ff[5]{Widened Backlash} The area affected by your \glossterm{miscast backlash} increases to a \areasmall radius burst centered on you.

        \ff[7]{Suppressed Backlash} Whenever you miscast a spell, you can suppress the \glossterm{miscast backlash}.
        If you do, you regain the action point spent to cast the spell (if any).

        \ff[9]{Greater Selective Backlash} Your \glossterm{miscast backlash} does not hurt you.

        \ff[11]{Resilient Channeler} Using your \textit{overchannel} ability does not impose a penalty to your defenses.

        \ff[13]{Greater Widened Backlash} The area increases to a \areamed radius burst centered on you.

        \ff[15]{Magical Resilience} You gain \glossterm{magic resistance} equal to 5 \add your level.

        \ff[19]{Empowered Channeler} When you use your \textit{overchannel} ability, you gain a \plus2 bonus to spellpower with the spell.
    \end{feat}

    \begin{feat}{Mystic Archer}{Magical, Spell}
        \featpre Ability to cast a spell.
        \featben

        \ff[1]{Imbue Projectile} You gain the \textit{imbue projectile} ability.
        \begin{ability}{Imbue Projectile}[\glossterm{Attune}]
            Whenever you cast a spell that affects a single target, you can \glossterm{action point} to use this ability.
            If you do, the spell does not have its effect immediately, and you regain the action point used to cast the spell (if any).
            Instead, its power is imbued in a \glossterm{projectile} you hold. 

            When you make a \glossterm{strike} using that projectile, the spell affects a target of the strike.
            After the spell takes effect, this ability's duration ends.
        \end{ability}

        \ff[3]{Missile Storm} You gain the \textit{missile storm} ability.
        \begin{ability}{Missile Storm}
            As a standard action, you can spend an \glossterm{action point} to use this ability.
            If you do, you make a ranged \glossterm{strike} with a \glossterm{projectile weapon} you wield.
            The strike targets up to five creatures and objects within one \glossterm{range increment} of you with that weapon, except for creatures adjacent to you.
            You take a \minus1d penalty to damage with the strike.
        \end{ability}

        \ff[5]{Guided Projectiles} Your attacks with projectiles ignore \glossterm{cover}, but not \glossterm{total cover}.

        \ff[7]{Imbue Detonating Projectlie} You gain the \textit{imbue detonating projectile} ability.
        \begin{ability}{Imbue Detonating Projectile}[\glossterm{Attune}]
            Whenever you cast a spell that affects an area, you can spend an \glossterm{action point} to use this ability.
            If you do, the spell does not have its effect immediately, and you regain the action point used to cast the spell (if any).
            Instead, its power is imbued in a \glossterm{projectile} you hold. 

            When you make a \glossterm{strike} using that projectile, the spell takes effect in addition to the normal effects of the strike.
            The spell's \glossterm{point of origin} can be anywhere within the \glossterm{space} of a target of the strike.
            It affects targets within the area as if it had just been cast in that area.
            After the spell takes effect, this ability's duration ends.
        \end{ability}

        \ff[9]{Greater Missile Storm} When you use your \textit{missile storm} ability, you may target a number of creatures equal to your highest \glossterm{spellpower}.

        \ff[11]{Phasing Projectiles}[Teleportation] When attacking with projectiles, you can ignore all physical obstacles in single five-foot span.
        This can allow you to fire projectiles through solid walls, though it does not grant you the ability to see through the wall.

        \ff[13]{Imbue Precision} You gain a \plus1 bonus to accuracy with projectiles that you have used your \textit{imbue projectile} or \textit{greater imbue projectile} abilities on.

        \ff[15]{Supreme Missile Storm} When you use your \textit{missile storm} ability, you may target any number of creatures.

        \ff[17]{Greater Phasing Projectiles}[Teleportation] Your \textit{phasing projectiles} ability improves, allowing you to ignore obstacles in two separate five-foot spans or one ten-foot span.

        \ff[19]{Unbound Imbuement} When you cast a spell and use your \textit{imbue projectile} or \textit{greater imbue projectile} abilities, you regain the action point spent to cast the spell, if any.
    \end{feat}

    \begin{feat}{Null}{General}
        \featpre Starting Willpower of 2.
        \featben

        \ff[1]{Nullify Magic} You gain \glossterm{magic resistance} equal to 5 \add your Willpower.
        You cannot lower this magic resistance voluntarily, which can prevent you from being affected by non-hostile magical effects.
        In exchange, you lose the benefits of all \glossterm{magical} abilities you possess.
        In addition, you are unable to \glossterm{attune} to any \glossterm{magical} abilities, such as spells cast by other creatures.

        \ff[2]{Null Severance} You gain the \textit{null severance} ability.
        \begin{ability}{Null Severance}
            As a standard action, you can spend an \glossterm{action point} to use this ability.
            If you do, make a \glossterm{strike}.
            In addition to the strike's normal effects, you also compare the attack result against the target's Mental defense.
            On a hit, the target breaks its \glossterm{attunement} to a random ability that it is currently attuned to.
            On a critical hit, the target breaks its attunement to all abilities that it is attuned to.
            This ability does not affect attunement to magic items.
        \end{ability}

        % clarify that it does not affect spells already in progress?
        \ff[4]{Disruptive Presence} All creatures that you are adjacent to have a 20\% chance to \glossterm{miscast} any spell they cast.

        \ff[6]{Greater Nullify Magic} You gain a \plus2 bonus to the \glossterm{magic resistance} from your \textit{nullify magic} ability.

        \ff[8]{Greater Null Severance} You gain a \plus1 bonus to \glossterm{accuracy} with your \textit{null severance} ability.

        \ff[10]{Greater Disruptive Presence} The failure chance from your \textit{disruptive presence} ability increases to 50\%.

        \ff[12]{Supreme Nullify Magic} The bonus from your \textit{greater nullify magic} ability increases to \plus4.

        \ff[14]{Supreme Null Severance} The accuracy bonus from your \textit{greater null severance} ability increases to \plus2.

        \ff[16]{Supreme Disruptive Presence} Your \textit{disruptive presence} ability affects all creatures in a \areamed radius \glossterm{emanation} from you.

        \ff[18]{True Null} You are unaffected by all \glossterm{magical} abilities.
    \end{feat}

    \begin{feat}{Perform Specialization}{Skill}
        \featpre Perform as a mastered skill.
        \featben

        \ff[1]{Lesser Specialization} You gain a \plus2 bonus to Perform.

        \ff[2]{Inspiring Performance}[Magical] You gain the \textit{inspiring performance} ability.
        \begin{ability}{Inspiring Performance}[\glossterm{Delusion}, \glossterm{Mind}, \glossterm{Sustain} (minor)]
            As a standard action, you can spend an \glossterm{action point} to use this ability.
            When you do, you begin a performance using one of your Perform skills.
            One other willing creature within \rngmed range gains a \plus2 bonus to \glossterm{checks}.

            If the target can neither see nor hear your performance, the effect immediately ends.
            This ability may have the \glossterm{Auditory} or \glossterm{Visual} tags, depending on the nature of your performance.
        \end{ability}

        \ff[4]{Mesmerizing Performance}[Magical] You gain the \textit{mesmerizing performance} ability.
        \begin{ability}{Mesmerizing Performance}[\glossterm{Delusion}, \glossterm{Mind}, \glossterm{Sustain} (minor)]
            As a standard action, you can spend an \glossterm{action point} to use this ability.
            When you do, you begin a performance using one of your Perform skills.
            Make a Perform vs. Mental attack against up to five creatures within \rngmed range.
            On a hit, each target is \fascinated by you.
            Any act by you or your apparent allies that damages a target or that causes it to feel threatened breaks the effect for that creature.
            An observant target may interpret overt threats to its allies as a threat to itself.

            If a target can neither see nor hear your performance, the effect immediately ends for that target.
            This ability may have the \glossterm{Auditory} or \glossterm{Visual} tags, depending on the nature of your performance.
        \end{ability}

        \ff[6]{Greater Inspirating Performance}[Magical] The bonus from your \textit{inspiring performance} ability increases to \plus3.

        \ff[8]{Mass Performance}[Magical] You can target up to two creatures with your \textit{inspiring performance} ability.
        In addition, you can target any number of creatures with your \textit{mesmerizing performance} ability.

        \ff[8]{Specialization} The bonus to Perform increases to \plus4.

        \ff[10]{Mesmeric Suggestion}[Magical] You gain the \textit{mesmeric suggestion} ability.
        \begin{ability}{Mesmeric Suggestion}[\glossterm{Delusion}, \glossterm{Mind}, \glossterm{Sustain} (minor)]
            As a standard action, you can spend an \glossterm{action point} to use this ability.
            When you do, make a Perform vs. Mental attack against target within \rngmed range.
            You must also make a verbal suggestion of a particular course of action to the target.
            You can work this suggestion into an active performance without penalty.
            On a hit, the target thinks your suggestion is a good idea and will try to follow it to the best of its abilities.
            Any act by you or your apparent allies that threatens or damages the target breaks the effect.

            If the target is not currently \fascinated by your \textit{fascinating performance} ability, this attack automatically fails.
            This ability lasts as long as you sustain your \textit{fascinating performance}.
            If your suggestion does not seem reasonable, you take a \minus5 penalty to accuracy on this attack.
            Exceptionally unreasonable suggestions can impose even greater penalties, and exceptionally reasonable suggestions can give accuracy bonuses.
        \end{ability}

        \ff[12]{Supreme Inspirating Performance}[Magical] The bonus from your \textit{inspiring performance} ability increases to \plus4.

        \ff[14]{Irresistible Dance} You ignore all failure chances on Perform attacks and checks you make.

        \ff[16]{Greater Mass Performance}[Magical] When you use your \textit{inspiring performance} ability, you can target any number of creatures.
        In addition, the range of your \textit{mesmerizing performance} ability is increased to \rnglong.

        \ff[16]{Greater Specialization} The bonus to Perform increases to \plus6.

        \ff[18]{Greater Mesmeric Suggestion} You gain a \plus2 bonus to accuracy with your \textit{mesmeric suggestion} ability.
    \end{feat}

    \begin{feat}{Persuasion Specialization}{Skill}
        \featpre Persuasion as a mastered skill.
        \featben

        \ff[1]{Lesser Specialization} You gain a \plus2 bonus to Persuasion.

        \ff[2]{}

        \ff[8]{Specialization} The bonus to Persuasion increases to \plus4.

        \ff[10]{Suggestion} You gain the \textit{suggestion} ability.
        \begin{ability}{Suggestion}[\glossterm{Delusion}, \glossterm{Mind}]
            As a standard action, you can spend an \glossterm{action point} to use this ability.
        \end{ability}

        \ff[16]{Greater Specialization} The bonus to Persuasion increases to \plus6.
    \end{feat}

    \begin{feat}{Precognition}{Combat}
        \featpre Starting Intelligence of 2.
        \featben

        \ff[1]{Precognitive Strike} You gain the \textit{precognitive strike} ability.
        \begin{ability}{Precognitive Strike}
            As a standard action, you can use this ability.
            If you do, make a \glossterm{strike}.
            You can use your Intelligence in place of your Perception to determine your accuracy with this strike.
        \end{ability}

        \ff[2]{Combat Prediction} You gain the \textit{combat prediction} ability.
        \begin{ability}{Combat Prediction}
            At the start of each phase, you can spend an \glossterm{action point} to use this ability.
            If you do, make an Intelligence vs. Mental attack against a creature within \rngmed range of you.
            A hit means you gain insight into the actions that creature intends to take during that phase.
            It may change its actions based on your interference if you communicate this information in a way it understands.
            A miss means you regain the action point spent to use this ability.

            This insight allows you to see and hear what actions that creature intends to take.
            You do not gain any knowledge of actions that have no obvious signs, such as purely mental actions, or actions which you are not observant enough to notice.
            In addition, you do not know the results of actions with a chance of failure, such as attacks.
        \end{ability}

        \ff[4]{Danger Sense} You receive brief flashes of insight into immediate danger.
        As a result, you are not \glossterm{unaware} when attacked by surprise.
        In addition, you gain a \plus2 bonus to \glossterm{initiative}.

        \ff[6]{Precognitive Reaction} You can use your Intelligence to determine your Reflex defense in place of your Perception.

        \ff[8]{Foresight} During the \glossterm{movement phase}, you choose your action after all other creatures have chosen their actions.
        When you choose your action, you have insight into the actions chosen by any creatures within \rngclose range of you that you can see.
        This insight gives you the same information as the insight from your \textit{combat prediction} ability.
        You choose your actions simultaneously with any other creatures who have a similar ability.

        \ff[10]{Greater Danger Sense} The initiative bonus from your \textit{danger sense} ability increases to \plus5.

        \ff[12]{Precognitive Defense} You can use your Intelligence to determine your Armor defense in place of your Dexterity.

        \ff[14]{Certain Combat Prediction} You gain a \plus2 bonus to accuracy with your \textit{combat prediction} ability.

        \ff[16]{Supreme Danger Sense} The bonus to initiative checks from your \textit{danger sense} ability increases to \plus10.

        \ff[18]{Greater Precognitive Strike} When you use your \textit{precognitive strike} ability, you can also use your Intelligence to determine your damage with the strike in place of your Strength.
    \end{feat}

    \begin{feat}{Reaper}{Combat}
        \spelldesc{You can attack with such force that you cleave through your foes.}
        \featpre Starting Strength of 2.

        \ff[1]{Sweeping Blow} You gain the \textit{sweeping blow} ability.
        \begin{ability}{Sweeping Blow}
            As a standard action, you can spend an \glossterm{action point} to use this ability.
            If you do, you can make a melee \glossterm{strike} with a slashing or bludgeoning weapon.
            The strike targets each of up to three creatures you \glossterm{threaten}.
            You take a \minus1d penalty to \glossterm{strike damage} with the strike.
        \end{ability}

        \ff[2]{Cleave} If you get a \glossterm{critical hit} with a melee strike using a slashing or bludgeoning weapon during the \glossterm{action phase}, you can make an additional strike during the \glossterm{delayed action phase}.
        This additional strike must target a creature adjacent to a target of the original strike.

        \ff[4]{Reaping Assault} You gain the \textit{reaping assault} ability.
        \begin{ability}{Reaping Assault}
            As a standard action, you can spend an \glossterm{action point} to use this ability.
            If you do, you can move up to your movement speed in a straight line and make a \glossterm{strike} with a slashing or bludgeoning weapon.
            The strike targets each creature and object that you \glossterm{threaten} at any point during your movement, except for the space you start in and the space you end in.
            You take a \minus1d penalty to \glossterm{strike damage} with the strike.
        \end{ability}

        \ff[6]{Powerful Sweep} The damage penalty with your \textit{sweeping blow} ability is removed.

        \ff[8]{Mobile Cleave} When you use your \textit{cleave} ability, you can move up to ten feet before making the strike.
        This movement counts against your normal movement limit in that phase.
        In addition, the target of the additional strike does not have to be adjacent to the struck creature.

        \ff[10]{Powerful Reap} The damage penalty with your \textit{reaping assault} ability is removed.

        \ff[12]{Greater Powerful Sweep} You gain a \plus1d bonus to damage with the strike from your \textit{sweeping blow} ability.

        \ff[14]{Greater Mobile Cleave} The maximum distance you can move with your \textit{mobile cleave} ability increases to be equal to your movement speed.

        \ff[16]{Greater Powerful Reap}  You gain a \plus1d bonus to damage with the strike from your \textit{reaping assault} ability.

        \ff[18]{Reap the Harvest} When you use your \textit{reaping charge} or \textit{sweeping blow} abilities, if every creature you attacked is dead at the end of the round, you regain the action point spent to use the ability.
    \end{feat}

    \begin{feat}{Regenerator}{General}
        \featpre Starting Constitution of 2.
        \featben

        \ff[1]{Diehard} You reduce your penalties from \glossterm{vital damage} by an amount equal to half your Constitution.

        \ff[2]{Regenerative Rest} Whenever you take a \glossterm{short rest}, you can heal vital damage equal to half your Constitution in addition to your normal healing.

        \ff[4]{Regenerative Recovery} When you use the \textit{recover} ability, you heal \plus1d hit points.
        In addition, it heals vital damage instead of hit points until you have no vital damage remaining, up to a maximum of half your Constitution in vital damage.

        \ff[6]{Fast Healing} At the end of each \glossterm{action phase}, you heal hit points equal to half your Constitution.

        \ff[8]{Unkillable} The penalty reduction from your \textit{diehard} ability increases to be equal to your Constitution.

        \ff[10]{Regeneration} If you have taken \glossterm{vital damage}, the healing from your \textit{fast healing} ability heals vital damage instead of hit points until you have no vital damage remaining.

        \ff[12]{Greater Fast Healing} The healing from your \textit{fast healing} ability increases to be equal to your Constitution.

        \ff[14]{Greater Regenerative Recovery} The increase to healing from the \textit{recover} ability increases to \plus2d.
        In addition, there is no cap on the vital damage healing you can receive from the \textit{recover} ability.

        \ff[16]{Healing Burst} You gain the \textit{healing burst} ability.
        \begin{ability}{Healing Burst}
            As a \glossterm{minor action}, you can spend an \glossterm{action point} to use this ability.
            If you do, you heal hit points equal to your \glossterm{standard damage} \minus1d.
        \end{ability}

        \ff[18]{Indestructible} The penalty reduction from your \textit{diehard} ability increases to be equal to twice your Constitution.
    \end{feat}

    \begin{feat}{Ride Specialization}{Skill}
        \featpre Ride as a mastered skill.
        \featben

        \ff[1]{Lesser Specialization} You gain a \plus2 bonus to Ride.

        \ff[2]{Mounted Warrior} Your mount gains a \plus2 bonus to \glossterm{physical defenses}, up to a maximum of your own defenses.

        \ff[4]{Mounted Archer} The penalty you take when using a ranged weapon while mounted is decreased by 4: \minus0 instead of \minus4 if your mount moves during the same phase, and \minus4 instead of \minus8 if your mount is \glossterm{sprinting} during the same phase.

        \ff[6]{Knight's Charge} When you \glossterm{charge} a creature while mounted, you gain a \plus1d bonus to damage with the strike.

        \ff[8]{Specialization} The bonus to Ride increases to \plus4.

        \ff[10]{Greater Mounted Warrior} The bonus to defenses increases to \plus3.

        \ff[12]{Greater Mounted Archer} The penalty reduction increases to \minus8.

        \ff[14]{Greater Knight's Charge} The damage bonus increases to \plus2d.

        \ff[16]{Greater Specialization} The bonus to Ride increases to \plus6.

        \ff[18]{Loyal Rider} Whenever your mount takes damage from a \glossterm{physical attack}, you may choose to take half that damage, rounded down.
        The mount takes the other half.
    \end{feat}

    \begin{feat}{Savage}{Combat}
        \featpre Starting Strength of 2.
        \featben

        \ff[1]{Brute Force} You gain a \plus2 bonus to \glossterm{accuracy} with the \textit{shove} and \textit{overrun} abilities (see \pcref{Shove}, and \pcref{Overrun}).

        \ff[2]{Wall Slam} If you use the \textit{shove} ability to move a creature, and the creature's movement is interrupted by a solid obstacle, the obstacle and creature both take bludgeoning \glossterm{standard damage} -1d.
        Your \glossterm{power} with this ability is based on your Strength.

        \ff[4]{Trample} If you use the \textit{overrun} ability to move through a creature, it takes bludgeoning \glossterm{standard damage} -3d.
        Your \glossterm{power} with this ability is based on your Strength.

        \ff[6]{Greater Brute Force} The accuracy bonus from your \textit{brute force} ability increases to \plus3.

        \ff[8]{Greater Wall Slam} If you deal damage to a creature with your \textit{wall slam} ability, it is also \glossterm{dazed} as a \glossterm{condition}.

        \ff[10]{Greater Trample} The damage from your \textit{trample} ability increases to \glossterm{standard damage} -2d.

        \ff[12]{Supreme Brute Force} The accuracy bonus from your \textit{brute force} ability increases to \plus4.

        \ff[14]{Supreme Wall Slam} You gain a \plus1d bonus to damage with your \textit{trample} ability.

        \ff[16]{Supreme Trample} The damage from your \textit{trample} ability increases to \glossterm{standard damage} -1d.

        \ff[18]{Inescapable} Whenever you use the \textit{overrun} ability, you may choose not to allow creatures to try to avoid you.
    \end{feat}

    \begin{feat}{Sense Motive Specialization}{Skill}
        \featpre Sense Motive as a mastered skill.
        \featben

        \ff[1]{Lesser Specialization} You gain a \plus2 bonus to Sense Motive.

        \ff[6]{Read Mind}[Magical] You gain the \textit{read mind} ability.
        \begin{ability}{Read Mind}[\glossterm{Mind}, \glossterm{Sustain} (minor)]
            As a standard action, you can spend an \glossterm{action point} to use this ability.
            If you do, make a Sense Motive vs. Mental attack against a creature within \rngclose range.
            On a hit, you know the target's surface thoughts.
            You gain a \plus2 bonus to Bluff, Persuasion, and Intimidate checks against a creature whose mind you are reading.
        \end{ability}

        \ff[8]{Specialization} The bonus to Sense Motive increases to \plus4.

        \ff[16]{Greater Specialization} The bonus to Sense Motive increases to \plus6.
    \end{feat}

    \begin{feat}{Sleight of Hand Specialization}{Skill}
        \featpre Sleight of Hand as a mastered skill.
        \featben

        \ff[1]{Lesser Specialization} You gain a \plus2 bonus to Sleight of Hand.

        \ff[2]{}

        \ff[8]{Specialization} The bonus to Sleight of Hand increases to \plus4.

        \ff[10]{Extradimensional Concealment}[Magical] Whenever you take the \textit{conceal object} action, you can spend an \glossterm{action point}.
        If you do, you conceal the object in a pocket dimension that cannot be accessed by nonmagical means.
        This pocket dimension can only hold one object at a time.

        \ff[10]{Extradimensional Retrieval}[Magical] You gain the \textit{extradimensional retrieval} ability.
        \begin{ability}{Extradimensional Retrieval}
            As a standard action, you can spend an \glossterm{action point} to use this ability.
            You reach into your pocket dimension to retrieve the object you stored there previously.

            Alternately, you can reach into the pocket dimension belonging to a creature you are touching to retrieve the object stored there.
            If that creature does not have the \textit{extradimensional concealment} ability, or does not have an object in their pocket dimension, this ability fails.
        \end{ability}

        \ff[16]{Greater Specialization} The bonus to Sleight of Hand increases to \plus6.
    \end{feat}

    \begin{feat}{Sniper}{Combat}
        \featpre Starting Perception of 2.
        \featben

        \ff[1]{Aim} You gain the \textit{aim} ability.
        \begin{ability}{Aim}[\glossterm{Sustain} (minor)]
            As a standard action, you can use this ability.
            If you do, choose a creature or object within line of sight.
            You gain a \plus2 bonus to accuracy on \glossterm{physical attacks} against the target.

            If you lose sight of the target for a full round, this effect ends.
        \end{ability}

        \ff[2]{Penetrating Aim} Your physical ranged attacks ignore \glossterm{cover}, except total cover.

        \ff[4]{Lesser Distance Tolerance} You reduce your accuracy penalties from \glossterm{range increments} by 2.

        \ff[6]{Sniper Shot} You gain a \plus2d bonus to damage on \glossterm{strikes} against \unaware creatures that are affected by your \textit{aim} ability.

        \ff[8]{Failure Tolerance} You ignore effects that give you a 20\% miss chance or failure chance with physical ranged attacks, such as \glossterm{concealment}.

        \ff[10]{Distance Tolerance} The reduction in accuracy penalties for range increments increases to 4.

        \ff[12]{Sustained Aim} You can sustain your \textit{aim} ability as a \glossterm{free action}.

        \ff[14]{Greater Sniper Shot} The bonus to damage increases to \plus4d.

        \ff[16]{Greater Distance Tolerance} The reduction in accuracy penalties for range increments increases to 6.

        \ff[18]{Rapid Aim} You can spend a \glossterm{action point} to use your \textit{aim} ability as a \glossterm{minor action}.
    \end{feat}

    \begin{feat}{Spellcraft Specialization}{Skill}
        \featpre Spellcraft as a mastered skill.
        \featben

        \ff[1]{Lesser Specialization} You gain a \plus2 bonus to Spellcraft.

        \ff[2]{Detect Spellcasting}[Magical] You gain the \textit{detect spellcasting} ability.
        \begin{ability}{Detect Spellcasting}[\glossterm{Knowledge}, \glossterm{Subtle}]
            As a standard action, you can spend an \glossterm{action point} to use this ability.
            If you do, make a Spellcraft attack against the Mental defense of a creature within \rngmed range.
            On a hit, you know whether the target is capable of casting spells.
            If the target can cast spells, you know what sources the target can cast spells from.
            On a critical hit, you also know all spells the target is capable of casting.
            This does not grant you knowledge of any subspells or augments the target knows.

            After using this ability on a target, you cannot use it again on the same target for 24 hours.
        \end{ability}

        \ff[4]{Unweave Magic} You gain the \textit{unweave magic} ability.
        \begin{ability}{Unweave Magic}[\glossterm{Thaumaturgy}]
            As a standard action, you can spend an \glossterm{action point} to use this ability.
            If you do, make a Spellcraft check on an active spell effect within \rngmed range.
            The DR is equal to 5 \add the \glossterm{spellpower} of the effect.
            Success means the effect is \glossterm{dismissed} if it is an effect that can be dismissed.
        \end{ability}

        \ff[6]{Mystic Tolerance} You gain a \plus1 bonus to Fortitude and Mental defense.

        \ff[8]{Specialization} The bonus to Spellcraft increases to \plus4.

        \ff[10]{Mass Unweave Magic} When you use your \textit{unweave magic} ability, you can target up to five spell effects.

        \ff[12]{Greater Mystic Tolerance} The defense bonus from your \textit{mystic tolerance} ability increases to \plus2.

        \ff[14]{Greater Detect Spellcasting} You gain a \plus5 bonus to accuracy with your \textit{detect spellcasting} ability.

        \ff[16]{Greater Specialization} The bonus to Spellcraft increases to \plus6.

        \ff[18]{Supreme Mystic Tolerance} The defense bonus from your \textit{mystic tolerance} abilty increases to \plus3.
    \end{feat}

    \begin{feat}{Stealth Specialization}{Skill}
        \featpre Stealth as a mastered skill.
        \featben

        \ff[1]{Lesser Specialization} You gain a \plus2 bonus to Stealth.

        \ff[2]{}

        \ff[6]{Movement Tolerance} Your penalties for moving while hiding are reduced by 2. 

        \ff[8]{Specialization} The bonus to Stealth increases to \plus4.

        \ff[10]{Hide in Plain Sight} You can use the \textit{hide} ability even while observed.
        You take take a \minus10 penalty to the Stealth check when hiding in this way, and you still need passive cover or concealment to hide.

        \ff[14]{Greater Movement Tolerance} The reduction of penalties for moving increases to 5.
        This allows you to move at half speed without penalty.

        \ff[16]{Greater Specialization} The bonus to Stealth increases to \plus6.

        \ff[18]{Greater Hide in Plain Sight} The penalty for hiding while observed is reduced to \minus5.
    \end{feat}

    \begin{feat}{Survival Specialization}{Skill}
        \featpre Survival as a mastered skill.
        \featben

        \ff[1]{Lesser Specialization} You gain a \plus2 bonus to Survival.

        \ff[2]{Terrain Tolerance} You ignore \glossterm{difficult terrain} and harmful natural terrain of any kind.
        If a skill check, such as Climb or Swim, would normally be required to move through the terrain, this ability does not help.

        \ff[4]{Trackless} You can choose to leave no trace of your passage as you move.
        If you do, tracking you is impossible by any \glossterm{mundane} means.

        \ff[6]{Rapid Tracker} 
        Your ability to track your foes improves.
        You can move at your normal speed while following tracks without taking the normal \minus2 penalty.
        In addition, you take only a \minus5 penalty (instead of the normal \minus10) when moving at up to twice normal speed while tracking.

        \ff[8]{Specialization} The bonus to Survival increases to \plus4.

        \ff[10]{Planar Tolerance} You are immune to harmful planar effects.

        \ff[14]{Find the Path} You gain the \textit{find the path} ability.
        \begin{ability}{Find the Path}[\glossterm{Attune}, \glossterm{Knowledge}]
            As a standard action, you can spend an \glossterm{action point} to use this ability.
            When you do, you must unambiguously specify a location on the same plane as you.
            You know exactly what direction you must travel to reach your chosen destination by the most direct physical route.
            You are not always led in the exact direction of the destination -- if there is an impassable obstacle between the target and the destination, this ability will direct you around the obstacle, rather than through it.

            The guidance provided by this ability adjusts to match whatever your current physical capabilities are, including flight and other unusual movement modes. It does not see into the future, and changing circumstances may cause the most direct path to change over time.
            It also does not consider hostile creatures, traps, and other passable dangers which may endanger or slow progress.
        \end{ability}

        \ff[16]{Greater Specialization} The bonus to Survival increases to \plus6.
    \end{feat}

    \begin{feat}{Swift}{General}
        \featpre Starting Dexterity of 2.
        \featben

        \ff[1]{Rapid Movement} You gain a \plus10 foot bonus to speed in all your movement modes.

        \ff[2]{Endurance Runner} You can use the \textit{sprint} ability as a \glossterm{standard action}.
        If you do, it does not cost an \glossterm{action point}.

        \ff[4]{Wall Runner} You gain a \plus5 bonus to checks with the \textit{wallrun} ability (see \pcref{Wallrun}).
        In addition, you can make a Dexterity check in place of a Climb check to use that ability.

        \ff[6]{Greater Rapid Movement} The speed bonus from your \textit{rapid movement} ability increases to \plus20 feet.

        \ff[8]{Sprinter} When you use the \textit{sprint} ability, you move at triple your movement speed.

        \ff[10]{Water Runner} When you use the \textit{sprint} ability, you can move on water and similar liquids as if they were solid ground.

        \ff[12]{Supreme Rapid Movement} The speed bonus from your \textit{rapid movement} ability increases to \plus30 feet.

        \ff[14]{Greater Sprinter} When you ues the \textit{sprint} ability, you move at quadruple your movement speed.

        \ff[16]{Cloud Runner} When you use the \textit{sprint} ability, you can move on dense fog and similar gaseous substances as if they were solid ground.

        \ff[18]{Endless Endurance} While you are not using the \textit{sprint} ability, you move at double your movement speed.
    \end{feat}

    \begin{feat}{Swim Specialization}{Skill}
        \featpre Swim as a mastered skill.
        \featben

        \ff[1]{Lesser Specialization} You gain a \plus2 bonus to Swim.

        \ff[2]{}

        \ff[4]{Swim Speed} You gain a \glossterm{swim speed} equal to your land speed.
        A successful Swim check to move allows you to move a distance equal to your swim speed.

        \ff[8]{Specialization} The bonus to Swim increases to \plus4.

        \ff[10]{Underwater Tolerance} You do not suffer any penalties to physical melee attacks, checks, or physical defenses for being underwater.
        You still suffer the normal penalty with underwater ranged attacks.

        \ff[16]{Greater Specialization} The bonus to Swim increases to \plus6.
    \end{feat}

    \begin{feat}{Toughness}{General}
        \featpre Starting Constitution of 2.
        \featben

        \ff[1]{Fortitude} You gain a \plus2 bonus to Fortitude defense.

        \ff[2]{Durability} You gain additional hit points equal to your Constitution.

        \ff[4]{Ailment Tolerance} You are immune to being \glossterm{sickened} and \glossterm{fatigued}.
        This allows you to sleep in heavy or medium armor without penalty.

        \ff[6]{Injury Tolerance} You reduce your penalties for being \glossterm{bloodied} by 2.

        \ff[8]{Greater Durability} The bonus hit points from your \textit{durability} ability increase to twice your Constitution.

        \ff[10]{Greater Ailment Tolerance} You are immune to being \glossterm{nauseated} and \glossterm{exhausted}.
        In addition, you need half the normal amount of rest and sleep each day to function normally.
        For example, a human would only need four hours of sleep per night.

        \ff[12]{Greater Fortitude} The defense bonus from your \textit{fortitude} ability increases to \plus4.

        \ff[14]{Greater Injury Tolerance} You do not take penalties for being \glossterm{bloodied}.

        \ff[16]{Supreme Durability} The bonus hit points from your \textit{durability} ability increase to four times your Constitution.

        \ff[18]{Unflinching Recovery} Whenever you take the \textit{recover} or \textit{desperate recovery} actions, you take half damage from all attacks until the end of the round.
        This applies to damage from attacks in the same phase.
    \end{feat}

    \begin{feat}{Transmuter}{Magical, Spell}
        \featpre Transmutation spell known.
        \featben

        \ff[1]{Enhance Body} You gain a \plus1 bonus to Armor defense and Strength-based \glossterm{checks}.

        \ff[3]{Reshaper} You gain the \textit{alter self} and \textit{alter object} abilities.
        \begin{ability}{Alter Self}[\glossterm{Shaping}]
            As a standard action, you can spend an \glossterm{action point} to use this ability.
            If you do, you make a Disguise check to alter your appearance (see \pcref{Disguise Creature}), except that you can use your spellpower in place of your Disguise skill.
            You can only alter your physical body, not your clothes or equipment.

            This ability lasts until you use it again.
        \end{ability}

        \begin{ability}{Alter Object}[\glossterm{Shaping}]
            As a standard action, you can spend an \glossterm{action point} to use this ability.
            If you do, choose an unattended, nonmagical object you can touch.
            You make a Craft check to alter the target (see \pcref{Craft}), except that you can use your spellpower in place of your Craft skill.
            In addition, you do not need any special tools to make the check (such as an anvil and furnace).
            However, the maximum hardness of a material you can affect with this ability is equal to your spellpower.

            % too short?
            Each time you use this ability, you can accomplish work that would take up to five minutes with a normal Craft check.
        \end{ability}

        \ff[5]{Transmutive Vigor} At the end of each \glossterm{action phase}, you heal hit points equal to half your highest \glossterm{spellpower}.

        \ff[7]{Greater Enhance Body} The bonus to checks from your \textit{enhance body} ability increases to \plus2.

        \ff[9]{Greater Reshaper} When you use your \textit{alter object} ability, you can accomplish work that would take up to an hour with a normal Craft check.
        In addition, you gain the \textit{alter poison} ability.
        \begin{ability}{Alter Poison}[\glossterm{Shaping}, \glossterm{Sustain} (minor)]
            As a standard action, you can spend an \glossterm{action point} to use this ability.
            If you do, make a \glossterm{spellpower} vs. Fortitude attack against a creature within \rngclose range.
            On a hit, any poison in the target's system is neutralized.
            It stops suffering any additional effects from poisons in its system.
            As long as the effect lasts, it is immune to all poisons.
            In addition, the target's \glossterm{mundane} poisons, including natural attacks that inflict poison, have no effect.
        \end{ability}

        \ff[11]{Greater Transmutive Vigor} The healing from your \textit{transmutive vigor} ability increases to be equal to your highest \glossterm{spellpower}.

        \ff[13]{Supreme Enhance Body} The bonus to Armor defense from your \textit{enhance body} ability increases to \plus2, and the bonus to checks increases to \plus3.

        \ff[15]{Supreme Reshaper} You do not need to spend action points to use your \textit{alter self}, \textit{alter object}, or \textit{alter poison} abilities.

        \ff[17]{Supreme Transmutive Vigor} The healing from your \textit{transmutive vigro} ability increases to be equal to twice your highest \glossterm{spellpower}.

        \ff[19]{Malleable Body} You are immune to \glossterm{critical hits} from \glossterm{strikes}.
        You suffer only the effects of a normal hit instead.
    \end{feat}

    \begin{feat}{Vivimancer}{Magical, Spell}
        \featpre Vivimancy spell known.
        \featben

        \ff[1]{Restore Life} You gain the \textit{restore life} ability.
        \begin{ability}{Restore Life}[\glossterm{Life}]
            As a standard action, you can spend an \glossterm{action point} to use this ability.
            If you do, choose a willing creature within \rngmed range.
            The target is healed for hit points equal to your \glossterm{standard damage} \plus1d.
            Your \glossterm{power} with this ability is equal to your highest \glossterm{spellpower}.
        \end{ability}

        \ff[3]{Unliving Resilience} You are immune to \glossterm{disease} and hostile \glossterm{Death} effects.

        \ff[5]{Vivimantic Surge} Whenever you cast a Vivimancy spell, you can heal any number of willing targets of the spell for hit points equal to your \glossterm{spellpower} with that spell.
        In addition, you gain a \plus1d bonus to damage with attacks that deal life damage.

        \ff[7]{Greater Restore Life} The healing granted by your \textit{restore life} ability increases by \plus1d.
        In addition, for every 2 points of healing, you can instead heal 1 \glossterm{vital damage}.

        \ff[9]{Greater Unliving Resilience} You are immune to life damage and hostile \glossterm{Life} effects.

        \ff[11]{Greater Vivimantic Surge} The healing granted by your \textit{vivimantic surge} ability increases to twice your \glossterm{spellpower} with the spell.

        \ff[13]{Supreme Restore Life} The healing granted by your \textit{restore life} ability increases by an additional \plus1d.
        In addition, it no longer costs an action point to use.

        \ff[15]{Supreme Unliving Resilience} You are healed by life damage instead of being immune to it.
        This healing applies before immunity, allowing you to heal from life damage dealt by \glossterm{Life} effects without suffering any other effects.

        \ff[17]{Supreme Vivimantic Surge} The healing granted by your \textit{vivimantic surge} ability increases to three times your \glossterm{spellpower} with the spell.
        In addition, the damage bonus increases to \plus2d.

        \ff[19]{} 
    \end{feat}

    \begin{feat}{Whirlwind Warrior}{Combat}
        \featpres Starting Dexterity of 2, starting Perception of 1.
        \featben

        \ff[1]{Whirlwind Strike} You gain the \textit{whirlwind strike} ability.
        \begin{ability}{Whirlwind Strike}
            As a standard action, you can spend an \glossterm{action point} to use this ability.
            If you do, you make a melee \glossterm{strike} against any number of creatures and objects adjacent to you that you \glossterm{threaten}.
            You must use the same weapon to make each strike.
            Use the same attack result and damage against each target.
            You take a \minus2d penalty to damage on each strike.
        \end{ability}

        \ff[2]{Spring Attack} You gain the \textit{spring attack} ability.
        \begin{ability}{Spring Attack}
            As a standard action, you can spend an \glossterm{action point} to use this ability.
            If you do, move up to your movement speed and make a \glossterm{strike}.
            During the \glossterm{delayed action phase}, you may continue moving if you have remaining movement available from the action phase.
        \end{ability}

        \ff[4]{Eye of the Storm} You gain a \plus1 bonus to \glossterm{overwhelm resistance}.

        \ff[6]{Greater Whirlwind Strike} The damage penalty on the strike you make with your \textit{whirlwind strike} ability is reduced to \minus1d.

        \ff[8]{Greater Spring Attack} When you use your \textit{spring attack} ability, you can make two \glossterm{strikes} instead of one.
        These strikes can be made at any point during your movement.
        However, they cannot be made against the same creature.

        \ff[10]{Unfettered Movement} During each phase, you may move through one creature's space during movement.
        You move at half speed while in its space.
        After moving through that creature's space, other creatures block you as normal for the remainder of your movement.
        Certain unusual creatures that occupy their entire space, such as gelatinous cubes, may be immune to this ability.

        \ff[12]{Hurricane Strike} The damage penalty with your \textit{whirlwind strike} ability is removed.

        \ff[14]{Supreme Spring Attack} When you use your \textit{spring attack} ability, you can make a third strike.
        It must target a different creature from the other two strikes.

        \ff[16]{Greater Eye of the Storm} The bonus from your \textit{eye of the storm} ability increases to \plus2.

        \ff[18]{Hurricane Defense} When you use your \textit{whirlwind strike} ability, you ignore all \glossterm{overwhelm penalties} until the end of the round.
        This is a \glossterm{swift ability}.
    \end{feat}

\section{Other Feat Rules}

    \subsection{Retraining Feats}
        At every level, your character can choose to retrain an old feat in exchange for a new feat.
