\chapter{Advanced Combat}

\section{Attacks}

    \subsection{Multiple Attacks}\label{Multiple Attacks}
        If your \glossterm{combat prowess} is at least 6, you can make multiple \glossterm{strikes} as part of a \glossterm{standard attack}. This progression is shown on \trefnp{Strikes per Round}.
        \begin{dtable}
            \lcaption{Strikes per Round}
            \begin{dtabularx}{\columnwidth}{*{3}{>{\lcol}X}}
                \tb{Combat Prowess} & \tb{Strikes per Round} \\
                \hline
                1--5     & 1 \\
                6--10    & 2 \\
                11--15   & 3 \\
                16--20   & 4 \\
                21\add & 5 \\
            \end{dtabularx}
        \end{dtable}

        Some abilities also grant you the ability to make multiple strikes. In all cases, making multiple strikes requires making a standard attack.

    \subsection{Special Attacks}

        \parhead{Charge}\label{Charge} As a standard action, you can move up to your speed in the action phase and make a \glossterm{standard attack}.
        Your strikes are delayed (see \pcref{Delayed Actions}).
        However, if you charge, you take a \minus5 penalty to \glossterm{physical defenses} during that phase.

        \par Your movement while charging has special limitations.
        First, you must move entirely in a single straight line.
        Your path must be clear of all obstacles and movement impediments, including \glossterm{difficult terrain}.
        If your movement becomes impeded while charging, you stop moving immediately, though you still suffer the defense penalty for charging and can still attack from your new location.

        \parhead{Touch Attacks}\label{Touch Attacks} A touch attack is an attempt to simply touch a target, rather than to strike it for damage.
        A touch attack targets Reflex defense instead of Armor defense, but deals no damage.
        You can make a touch attack in place of a \glossterm{strike}, or as part of another ability that requires touching the target.

    \subsection{Combat Maneuvers}\label{Combat Maneuvers}
        A combat maneuver is an attempt to physically hinder your foe with your body, such as by shoving or tripping it.
        Unless otherwise noted, combat maneuvers do not deal damage.
        There are two kinds of combat maneuvers: heavy maneuvers and light maneuvers.

        \parhead{Heavy Maneuvers} You can use a heavy maneuver as a standard action.
        You cannot use Dexterity to determine your accuracy with a heavy maneuver, just like attacking with a heavy weapon.

        \parhead{Light Maneuvers} You can use a light maneuver in place of a \glossterm{strike} as part of a \glossterm{standard attack}.
        You can use either Strength or Dexterity to determine your accuracy with a light maneuver, just like attacking with a light weapon.

        \begin{dtable}
            \lcaption{Combat Maneuvers}
            \begin{dtabularx}{\columnwidth}{l l l X}
                \tb{Maneuver}  & \tb{Type} & \tb{Defense} & \tb{Brief Description} \\
                \hline
                Dirty Trick & Light & Any       & Impose penalty on a foe         \\
                Disarm      & Light & Reflex    & Force foe to drop item          \\
                Feint       & Light & Reflex    & Leave foe vulnerable to attacks \\
                Grapple     & Heavy & Special   & Wrestle with a foe              \\
                Shove       & Heavy & Fortitude & Move a foe                      \\
                Trip        & Light & Reflex    & Trip a foe                      \\
            \end{dtabularx}
        \end{dtable}

        \subsubsection{Maneuver Descriptions}

            \parhead{Dirty Trick}\label{Dirty Trick} You strike your foe in a sensitive spot, pull its pants down, or creatively use your environment to attack.
            Depending on the nature of your dirty trick, you can target Fortitude, Reflex, or Mental defense.
            You must use a free hand to perform a dirty trick.
            Success means the target is \impaired with actions relevant to the nature of your trick for 2 rounds.
            Critical success means the target is \severelyimpaired instead.
            Critical failure means you are \impaired with relevant actions instead.
            Unusual dirty tricks using the environment may impose other effects of similar strength.
            Dirty tricks are a light maneuver.

            \parhead{Disarm}\label{Disarm} You can strike an item your foe is wearing or holding to knock it out of their hands or damage it.
            You can perform a disarm with a free hand or any weapon.
            You make a physical attack against the target's Reflex defense.
            Success means you hit the object, and can choose whether or not to deal damage to it.
            In addition, if the creature is not holding the item in its hands, it falls to the ground in the foe's square.
            Critical success means you can also knock an item held directly in a creature's hands to the ground.
            Critical failure means you drop the weapon you used to attack (if any).

            If you disarm a foe using a free hand, you can hold a disarmed item rather than letting it fall to the ground.
            Items that are very well secured, such as worn rings, cannot be knocked to the ground or grabbed.
            Disarming is a light maneuver.

            \parhead{Feint}\label{Feint} You can make a fake attack to leave your foe off-balance.
            You can perform a feint with a free hand or any weapon.
            You make a physical attack against the target's Reflex defense.
            Success means the target takes a \minus5 penalty to physical defenses against your next attack.
            Critical success means the target takes a \minus5 penalty to physical defenses against all of your attacks.
            Critical failure means you take a \minus5 penalty to physical defenses against the next attack by the target.
            In either case, if you stop threatening the target, the effect of the feint immediately ends.
            Feinting is a light maneuver.

            \parhead{Grapple}\label{Grapple} You physically grab and restrain your foe.
            You must use a free hand to perform a grapple.
            You make a physical attack against the target's Reflex and Fortitude defenses.
            Success against both defenses means you and the creature become \grappled.
            See \pcref{Grappling} for more details.
            Grapping is a heavy maneuver.

            \parhead{Shove}\label{Shove} You shove your foe where you want it to go.
            You must use a free hand to perform a shove.
            You make a physical attack against the target's Fortitude defense.
            Success means you move the creature 5 feet in a direction of your choice.
            For every 5 points by which you succeed, you can move it an additional 5 feet.
            You cannot normally move the creature further after moving it outside of your reach.
            If you have movement remaining in the round, you can move with the creature in order to shove it farther.
            If the creature encounters a solid object or creature, you must stop shoving it.
            Shoving is a heavy maneuver.

            \parhead{Trip}\label{Trip} You try to trip your foe.
            You must use a free hand to perform a trip.
            You make a physical attack against the target's Reflex defense.
            Success means the target falls \prone.
            Critical failure means you fall prone instead.
            As with other effects, the target does not suffer penalties for being prone against your other attacks within the same round.
            It is still in the process of falling as you make the subsequent attacks.
            Tripping is a light maneuver.

\section{Movement and Positioning}

    Each round, creatures can move around the battlefield.
    Movement almost always takes place during the \glossterm{movement phase}, as all creatures move around simultaneously.

    \subsection{Standard Movement}
        Every creature has one or more \glossterm{speeds} that indicate how far they can move.
        At the start of the movement phase, each creature designates a location it is trying to move to.
        If there are no conflicts, the moving creatures simply occupy their new locations, and the action phase begins.

    \subsection{Reactive Movements}

        It is possible to declare movement that reacts automatically to the movement of an opponent.
        For example, you can try to follow a creature wherever it goes that round.
        If you declare a reactive movement at the start of the movement phase, you automatically move accordingly.
        In all cases, if you run out of movement speed before accomplishing your intended task, you simply stop where you ran out of movement.
        The two most common types of reactive movements are \glossterm{blocking}, \glossterm{following}, and \glossterm{withdrawing}, but you can come up with other reactive movements.
        The only requirement is that a reactive movement must have a simple criteria for determining how you move.

        \parhead{Blocking}\label{Blocking} You can designate a target creature or object to block, and the area you want to block it from entering.
        When you do, you automatically move to intercept the target as it approaches the blocked area.
        Usually, blocking a target requires an opposed \glossterm{initiative} check against the target.
        Success means you successfully keep ahead of the target as it moves, preventing it from entering the area (unless it can move through you).
        Failure means the target moves around you (if there is room) to enter the area.

        Multiple creatures can coordinate to block a single creature.
        The blocked creature must beat the initiative of all blocking creatures to enter the blocked area.

        \parhead{Following}\label{Following} You can designate a target creature or object to follow, and the maximum distance you want to follow at. When you do, you automatically move such that your distance to the target is no greater than your desired follow distance. For the rest of the round, whenever that creature or object moves, you move with it to stay within that follow distance.

        If the target takes an action that makes it impossible to follow with movement, such as teleporting, you cannot follow it for the rest of the round.

        \parhead{Withdrawing}\label{Withdrawing} Withdrawing functions the same way as following, except that you specify a minimum distance between you and the target instead of a maximum distance.
        In addition, you can specify multiple targets and try to keep away from all of them.

    \subsection{Special Movements}

        Every character can take some special movement-related actions which are described below.

        \parhead{Struggle} As a full-round action, you can move five feet, regardless of movement penalties.
        This does not allow you to pass obstacles unrelated to movement speed penalties, such as walls.
        You can only use this action with a land speed, and not with any other type of speed (see \pcref{Special Movement Modes}).

        \parhead{Overrun}\label{Overrun} As part of movement, you can try to move directly through creatures in your way.
        Each creature in your way can choose to avoid you, allowing you to pass through its square unhindered.
        If a creature does not attempt to avoid you, you make a Strength vs. Fortitude attack against it.
        Success means you move through the creature's space, though you treat it as \glossterm{difficult terrain}.
        Critical success means the creature is knocked prone, and you do not treat its space as difficult terrain.
        Failure means you end your movement immediately.
        Critical failure means you end your movement and fall \prone.

    \subsection{Movement Impediments}

        \subsubsection{Difficult Terrain}\label{Difficult Terrain}
            Some terrain is hard to move through, like thick bushes or a swamp. If a square is difficult terrain, it doubles the movement cost required to move out of the square. That generally means it takes ten feet of movement, or fifteen feet if you are moving diagonally. If a square is considered difficult terrain for multiple reasons, the cost increases stack. For example, a square in a swamp that also has thick bushes blocking your passage would take twenty feet of movement, or thirty feet to move diagonally.

        \subsubsection{Obstacles}
            An obstacle is anything that gets in your way. Enemies and large solid objects like walls are blocking obstacles: they completely block your movement. If you can get past an obstacle, like a low wall, that square is treated as difficult terrain. Some obstacles require a skill check to bypass.

        \subsubsection{Squeezing}\label{Squeezing}

            In some cases, you may have to squeeze into or through an area that isn't as wide as the space you take up. You can squeeze through or into a space that is at least half as wide as your normal space. While squeezing, you move half as fast, and you take a \minus4 penalty to physical accuracy, physical checks, and physical defenses. You can squeeze into tighter spaces with the Escape Artist skill.

            Creatures that take up multiple squares take up half their normal number of squares while squeezing. For example, a Large creature who normally takes up four spaces takes up two spaces while squeezing.

            \parhead{Accidentally Squeezing} Sometimes a character ends its movement while moving through a space where it's not normally allowed to stop. When that happens, the character is squeezing in the space until it can move. If squeezing is impossible, the creature immediately moves to the closest available space. Try not to do this.

    \subsection{Special Movement Modes}\label{Special Movement Modes}
        Some spells and abilities grant creatures the ability to move in unusual ways. These forms of movement are described here.

        \subsubsection{Burrowing}
            A creature with a burrow speed can move through the ground at the indicated speed in any direction, even vertically. Unless otherwise noted, the creature can only burrow through dirt and loose earth, not rock or harder substances. It does not leave behind a usable tunnel for other creatures.

        \subsubsection{Climbing}
            A creature with a \glossterm{climb speed} can move a distance equal to its climb speed with a successful Climb check (see \pcref{Climb}).
            In addition, it gains a \plus10 bonus to any Climb checks it makes.

        \subsubsection{Flying}\label{Flying}
            A creature with a fly speed can fly through the air at the indicated speed. It must be \unencumbered (see \pcref{Encumbrance}). If a creature with a fly speed is encumbered, it is treated as having a glide speed instead (see \pcref{Gliding}), which it can use to glide even though it is encumbered.

            Each creature with a fly speed also has a maneuverability: good, average, poor, or special. Normally, a flying creature must move forward by at least half its fly speed each round. If it does not, it falls. Turning by 90 degrees costs 5 feet of movement, and it can't turn in the same place by more than 90 degrees. It can move up by only one square vertically per square traveled horizontally, but it can fly directly down if it chooses. The creature can fly up at half speed, but can fly down twice as fast.

            \parhead{Maneuverability}\label{Maneuverability} Some creatures have fly speeds with special maneuverability rules.

            \subparhead{Good Maneuverability} If a creature has good maneuverability while flying, it gains three benefits while flying.
            First, it not need to move forward to maintain its flight, allowing it to hover.
            Second, it can turn in place without spending movement.
            Third, it can move up at the same speed as it moves horizontally.

            \parhead{Poor Maneuverability} If a creature has poor maneuverability while flying, it must spend five feet of movement to turn by 45 degrees, and it can't turn in the same place by more than 45 degrees. In addition, it can only descend by up to one square vertically per square traveled horizontally without falling.

            \parhead{Falling} If a flying creature loses control, usually by failing to maintain its minimum forward speed, or loses the ability to fly, it falls just like any other creature would in midair. As long as it still has the ability to fly, it can regain control of its fall as a standard action, causing it to resume flying normally.

    \subsection{Gliding}\label{Gliding}
        A creature with a glide speed can glide through the air at the indicated speed. It must be unencumbered (see \pcref{Encumbrance}).

        While in the air, a creature with a glide speed can control its fall as a move action. This allows it to move up to its speed horizontally in a direction of its choice while moving only five feet down. If it desires, it can move half as far horizontally and fall down twice as fast. It takes no falling damage if it touches the ground while gliding.

\section{Circumstances, Bonuses, and Penalties}\label{Circumstances, Bonuses, and Penalties}

    \subsection{Changing Statistics}

        Your modifiers and defenses can change for many reasons.
        In general, all changes take effect immediately, though some side effects of those changes may not happen until you rest or level up.

        \parhead{Numerical Modifiers} Changes to numerical modifiers always take effect immediately.
        For example, if a barbarian enters a rage, his damage and defenses are all adjusted immediately.

        \parhead{Ability Prerequisites} Changes to prerequisites for abilities always take effect immmediately.
        For example, if a paladin's Strength is reduced to 0 by a ghost, she immediately loses the benefits of all feats she has that require a high Strength, such as Power Attack (see \featpcref{Power Attack}).

        \parhead{Ability Use Limits} Effects that change a character's maximum uses of an ability take effect immediately.
        However, increasing a character's maximum uses does not immediately grant the character additional uses.
        They must recovered in the normal fashion, such as by resting.
        If a character's maximum uses is decreased below their currently available uses, they immediately lose ability uses until their current value is equal to their maximum value.

        \parhead{Skill Points} Effects that change a character's skill points take effect immediately.
        However, the character cannot spend additional skill points on new skills until they level up.
        If a character's total skill points are decreased below their currently spent skill points, they immediately lose training from skills until their spent skill points are equal to their total skill points.

        \parhead{Hit Points} Effects that change a character's maximum hit points take effect immediately.
        However, increasing a character's maximum hit points does not immediately grant the character additional hit points.
        They must be recovered in the normal fashion, such as by resting.
        If a character's maximum hit points are decreased below their current hit points, they immediately lose hit points until their current hit points are equal to their maximum hit points.

    \subsection{Stacking Rules}\label{Stacking Rules}
        Usually, modifiers stack with each other, meaning that you add or subtract all of the modifiers to get the final result. However, some modifiers do not stack with each other, as described below. When bonuses don't stack with each other, you only apply the largest bonus. Likewise, when penalties don't stack with each ather, you only apply the largest penalty.

        \parhead{Special Exceptions}

        \begin{itemize}
            \item Effects from the same source do not stack. Any spell or ability with the same name has the same source.
            \item \glossterm{Sizing} effects do not stack.
                If multiple effects both increase and decrease size, size increases offset size decreases on a one-for-one basis to determine the creature's final size.
            \item Effects that grant extra \glossterm{strikes} (such as the \spell{haste} spell with the Empowered augment) do not stack.
            \item Temporary hit points do not stack.
            \item If a character has two separate abilities which let them add the same attribute to a given roll or statistic, the attribute is still only added once.
        \end{itemize}

    \subsection{Cover}\label{Cover}

        Cover represents any obstacle that physically prevents you from striking your target, such as a tree or intervening creature.
        A creature behind cover is more difficult to attack.
        There are three kinds of cover: \glossterm{active cover}, \glossterm{passive cover}, and \glossterm{total cover}.
        All three types of cover are determined by the presence or absence of physical obstacles.

        \parhead{Active Cover} Active cover is provided by mobile obstacles between you and your target, such as creatures or tree branches blowing in the wind.
        Physical attacks against creatures with active cover suffer a 20\% miss chance.
        If an attack misses due to active cover, the attack is made against the intervening obstacle rather than being negated like normal for miss chances.
        The obstacle takes any damage from a successful attack normally.

        \parhead{Passive Cover} Passive cover is provided by immobile obstacles between you and your target, such as trees and walls.
        Creatures with passive cover gain a \plus4 bonus to \glossterm{physical defenses}.
        In addition, creatures with passive cover can hide (see \pcref{Stealth}).

        \parhead{Measuring Cover}

        When you make an attack, choose a single square within your \glossterm{space} and a single \glossterm{target square} within your target's space.
        If you are making a ranged attack, choose one corner of your space.
        If you are making a melee attack, choose any two corners of your square.
        These corners are called the \glossterm{points of origin} for your attack.
        For the purpose of determining cover, your attack originates from your chosen \glossterm{points of origin} and travels to the \glossterm{target square}.

        First, check if you can attack the target at all.
        You must be able to draw a line from the \glossterm{points of origin} of your attack to any two corners of your attack's \glossterm{target square}.
        That line must not be blocked by solid objects, but it can pass through obstacles that do not take up the entire area within their space (such as most creatures).
        In addition, the line must not be blocked by other squares within the target's space, preventing you from targeting the ``inside'' of large creatures.
        If you cannot draw such a line, the target has \glossterm{total cover} from you.
        This makes all targeted attacks impossible.

        If you can draw an uninterrupted line from the \glossterm{points of origin} of your attack to the center of your attack's \glossterm{target square}, the target does not have cover from you.
        If such a line would be blocked by an obstacle that grants active or passive cover, the target has active or passive cover from you, according to the nature of the obstacle.

        \parhead{Partial Obstacles} Many obstacles, such as trees and low walls, can provide passive cover without normally blocking \glossterm{line of sight} or providing \glossterm{total cover}.
        Unusually small creatures, or creatures who intentionally take cover behind such obstacles, may be able to gain total cover from them.

        \subsubsection{Improved Cover}

            A creature can benefit from both passive and active cover.
            Cover of the same type generally doesn't stack; a creature behind two trees is not substantially more protected than a creature behind a single tree.
            However, exceptionally well covered creatures, such as a creature behind an arrow slit in a castle, may receive additional benefits.
            In that case, each additional major obstacle increases the miss chance by 10\% or grants an additional \plus2 bonus to physical defenses, as appropriate.

    \subsection{Concealment}\label{Concealment}
        Concealment represents anything which makes it more difficult to see your target, such as dim lighting. A creature with concealment from you gains a \plus4 bonus to physical defenses. The concealment bonus does not apply if you can't see your opponent (such as if you close your eyes).
        Determining concealment works similarly to determining cover.
        You must use the same \glossterm{points of origin} and \glossterm{target square} when determining concealment that you would use to determine cover.

        \parhead{Determining Concealment} There are two things that can cause a creature to be concealed: poor lighting, and intervening obstacles that block sight.
        Determining concealment from obstacles that block sight works the same way as determining cover.

        Determining concealment from lighting conditions is simpler, since it ignores lighting conditions between you and the target.
        If your \glossterm{target square} square is in lighting that provides concealment, the target has concealment.
        Otherwise, it does not.

        \parhead{Hiding} A creature with concealment can hide (see \pcref{Hiding}).

    \subsection{Grappling}\label{Grappling}
        Grappling creatures are physically struggling with each other. While grappled, you suffer certain penalties and restrictions, as described below.

        \subsubsection{Being In A Grapple}
            While grappling, you suffer certain penalties and restrictions, as described below. Other than these restrictions, you can act normally. You can also take certain actions in a grapple, as described in \pcref{Grapple Actions}
            \begin{itemize}
                \item You must use a free hand (or equivalent limbs) to grapple, preventing you from taking any actions which would require having two free hands. If you cannot free a hand, you suffer a \minus10 penalty to accuracy on all physical attacks, including grapple attacks, until you have a free hand.
                \item You are \defenseless against creatures you are not grappling with.
                \item You take a \minus4 penalty to accuracy with weapons that are not light, since they are too large and cumbersome to be used effectively in a grapple.
                \item Spellcasting is extremely difficult. You cannot cast spells with somatic components. Casting a spell without somatic components requires a Concentration check with a DR equal to 20 \add double spell level.
                \item You cannot move normally (but see Move the Grapple, below).
            \end{itemize}

        \subsubsection{Grapple Actions}\label{Grapple Actions}
            While grappled, you can take four special actions to try to affect the grapple.

            \parhead{Bind Foe} As a standard action, you can make a grapple attack to bind your foe with ropes or other restraints. You must have the restraints in hand (in addition to the free hand required to grapple). Apply the result against the Fortitude and Reflex defenses of a creature grappling you. Success against both defenses means the creature is bound, rendering it helpless and effectively paralyzed. The only physical actions a bound creature can take are to esccape or break the bindings. Escaping the bindings requires a grapple attack or Escape Artist check which beats the grapple attack made to bind the creature. Breaking the bindings requires making a Strength check sufficient to break the item used to bind the creature. If you have the time, you can \glossterm{take 20} on your grapple attack to secure bindings on a helpless foe, including a foe rendered helpless through this attack.
            \parhead{Escape the Grapple} As a standard action, you can make a grapple attack or Escape Artist check to escape from the grapple. Apply the result against the Reflex defense of each creature grappling you. Success against an individual creature means you are no longer grappling that creature.
            \parhead{Move the Grapple} As a move action, you can make a grapple attack to move the grapple. Apply the result against the Fortitude defense of each creature grappling you. If you beat every creature's Fortitude defense, you can move yourself and all other creatures in the grapple up to half your speed. At the end of your movement, you choose which spaces creatures grappling you are in, as long as they stay adjacent to you.
            \parhead{Pin} As a standard action, you can make a grapple attack against the Fortitude and Reflex defenses of a creature you are grappling with. Success against both defenses means the creature becomes \glossterm{pinned} (in addition to being grappled), while you remain only grappled. The only physical action a pinned creature can take is to escape the grapple, though it can take mental actions. At your discretion, you can cover the mouth of a pinned creature to prevent it from speaking.

    \subsection{Helpless Defenders}
        A helpless creature is completely at an opponent's mercy.
        Its physical defenses are calculated as if it had a Dexterity of \minus10.
        Paralyzed, bound, and unconscious creatures are helpless.

        \parhead{Coup de Grace}\label{Coup de Grace} As a full-round action, you can deliver a coup de grace to a helpless opponent within your \glossterm{reach} that you can see.
        You automatically hit with your weapon and score a critical hit. If the damage exceeds the struck creature's Fortitude defense, it immediately dies.

        Delivering a coup de grace requires focused concentration and methodical action. This leaves you \defenseless. If you take damage in excess of your Fortitude defense while delivering a coup de grace, your attempt fails.
        If your target stops being helpless during your coup de grace attempt for any reason, the attempt fails.
        You can't deliver a coup de grace against a creature that is immune to critical hits.

    \subsection{Invisibility}\label{Invisibility}
        If it is impossible to see your target, you can't attack him normally. However, you can attack a square that you think he occupies. An attack into a square occupied by an invisible enemy has a 50\% miss chance. If an adjacent invisible creature strikes you, you can automatically identify the square he occupied when he struck you.

    \subsection{Surprise Attacks}\label{Surprise Attacks}
        Sometimes, creatures are not aware that combat is taking place when the combat starts. This most commonly happens with ambushes. Any creature that is not aware of the combat continues taking whatever actions it would normally be taking until it becomes aware of the combat. It is \unaware until that point.

\section{Special Actions}

    \subsection{Partial Actions}

        If you are restricted to only taking a move or standard action, but not both, you can spend a standard action to perform a partial version of a full-round action. For most actions, you spend the first round starting the action, and use a second standard action to complete it. Some full-round actions have specific partial versions described below which you take instead. You can only take these partial actions when you cannot take full-round actions.

        \subsubsection{Partial Charge}
            As a standard action, you can move up to your speed and make a single strike against a foe. You must move in a straight line, and your movement must not be impeded in any way. In addition, you take a \minus2 penalty to physical defenses until the start of your next turn. An interrupted partial charge becomes a move action.

\section{Special Rules}

    \subsection{Combat Prowess Progressions}\label{Combat Prowess Progressions}

        A character's \glossterm{combat prowess} measures how skilled they are combat (see \pcref{Combat Prowess}).
        There are three progressions: Good, Average, or Poor.
        The effects of each progression are described on \trefnp{Combat Prowess Progressions}.
        A high combat prowess can grant additional attacks, as described in \pcref{Multiple Attacks}.

        \begin{dtable}
            \lcaption{Combat Prowess Progressions}
            \setlength\tabcolsep{0.45em}%
            \begin{dtabularx}{\columnwidth}{l X}
                \tb{Progression} & \tb{Combat Prowess} \\
                \hline
                Good                & Class level \add 2             \\
                Average             & Four-fifths class level \add 2 \\
                Poor                & Two-thirds class level \add 1  \\
            \end{dtabularx}
        \end{dtable}

        \begin{dtable}
            \lcaption{Combat Prowess Progression Bonuses}
            \begin{dtabularx}{\columnwidth}{l l X X}
                \tb{Level} & \tb{Good} & \tb{Average} & \tb{Poor} \\
                \hline
                \allbabprogressionrow{1}  \\
                \allbabprogressionrow{2}  \\
                \allbabprogressionrow{3}  \\
                \allbabprogressionrow{4}  \\
                \allbabprogressionrow{5}  \\
                \allbabprogressionrow{6}  \\
                \allbabprogressionrow{7}  \\
                \allbabprogressionrow{8}  \\
                \allbabprogressionrow{9}  \\
                \allbabprogressionrow{10} \\
                \allbabprogressionrow{11} \\
                \allbabprogressionrow{12} \\
                \allbabprogressionrow{13} \\
                \allbabprogressionrow{14} \\
                \allbabprogressionrow{15} \\
                \allbabprogressionrow{16} \\
                \allbabprogressionrow{17} \\
                \allbabprogressionrow{18} \\
                \allbabprogressionrow{19} \\
                \allbabprogressionrow{20} \\
            \end{dtabularx}
        \end{dtable}

        Under normal circumstances, these tables will not need to be referenced.
        The tables for each class should be sufficient to determine each character's bonuses and abilities.

    \subsection{Maximum Bonuses}\label{Ability Limits}

        Some bonuses specify that they cannot increase the value beyond a given point.
        For example, the \spell{totemic power} spell cannot increase a physical attribute to be higher than the caster's spellpower.
        These bonuses must always be applied last, and cannot be combined with other bonuses to exceed the maximum value.
        If multiple bonuses specify different maximum values, use the lower maximum value.
        If a bonus with a maximum value is applied to a value that already exceeds the maximum value the bonus can provide, simply ignore the bonus and its maximum value.

    \subsection{Doubling}\label{Doubling}
        If you double any in-game value twice, it becomes three times as large. An additional doubling would make it four times as large, and so on. For example, if you make an attack that deals double damage and you get a critical hit, you deal triple damage.

        \parhead{Real-World Values} Values that are ``real'', such as movement and distance, are an exception. If you double a real-world value twice, it becomes four times as large.

    \subsection{Extraordinary Size Differences}\label{Special Size Rules}

        \subsubsection{Very Small Creatures}
            \parhead{Space} If a creature takes up less than a single square of space, you can fit multiple creatures in that square. You can fit four Tiny creatures in a square, twenty-five Diminuitive creatures, or 100 Fine creatures.

            \parhead{Reach} Creatures that take up less than 1 square of space typically have a natural reach of 0 feet, meaning they can't reach into adjacent squares. They must enter an opponent's square to attack in melee. You can attack into your own square if you need to, so you can attack such creatures normally. Since they have no natural reach, they do not threaten the squares around them.

            If a creature without a natural reach uses a reach weapon, it gains no benefits or penalties.

            \parhead{Movement} Creatures three size categories smaller than you are not considered obstacles and do not hinder your movement.

            \parhead{Stealth} Small creatures gain a bonus to Stealth checks equal to their special size modifier.

        \subsubsection{Very Large Creatures}
            \parhead{Space} Very large creatures take up multiple squares. Anything which affects a single square the creature occupies affects the creature.

            \parhead{Reach} Creatures that take up more than 1 square typically have a natural reach of 10 feet or more, meaning that they can reach targets even if they aren't in adjacent squares. Creatures with a large natural reach can attack anyone within their reach, including adjacent foes.

            Creatures with a large natural reach using reach weapons can strike at up to double their natural reach but can't strike at their natural reach or less, just like Medium sized creatures.

            \parhead{Movement} Creatures three size categories larger than you are not considered obstacles and do not hinder your movement.

            \parhead{Stealth} Large creatures take a penalty to Stealth checks equal to their special size modifier.

            \parhead{Immunities} Creatures at least three size categories larger than you are difficult to fight. You cannot score critical hits or contribute to overwhelm penalties against such creatures. If you can reach a vulnerable point on the creature, such as by flying or by using ranged weapons, you can score critical hits, but you still do not contribute to overwhelm penalties.

    \subsection{Mounted Combat}\label{Mounted Combat}
        \parhead{Horses in Combat} Warhorses and warponies can serve readily as combat steeds. Light horses, ponies, and heavy horses, however, are frightened by combat. If you don't dismount, you must make a DR 20 Ride check each round as a move action to control such a horse. If you succeed, you can perform a standard action after the move action. If you fail, you can't do anything else until your next turn.

        Your mount acts on your initiative count as you direct it. You move at its speed, but the mount uses its action to move.

        \parhead{Space} A horse (not a pony) is a Large creature, and thus takes up a space 10 feet (2 squares) across. While mounted, you share your mount's space completely. Anyone threatening your mount can attack either you or your mount. However, your reach is still that of a creature of your normal size. Thus, a Medium paladin would threaten all squares adjacent to his Large horse with a longsword, and all squares 10 feet away from his mount with a lance.

        In the case of abnormally large mounts (two or more size categories larger than you), you may not completely share space. Such situations should be handled on a case-by-case basis, depending on the nature of the mount.

        \parhead{Combat while Mounted} With a DR 5 Ride check, you can guide your mount with your knees so as to use both hands to attack or defend yourself. This is a free action.

        If your mount moves more than its normal speed when charging, you also take the physical defense penalty associated with a charge. If you make an attack at the end of such a charge, you receive the bonus gained from the charge. When charging on horse-back, you deal double damage with a lance (see \pcref{Charge}).

        You can use ranged weapons while your mount is taking a double move, but at a \minus4 penalty to accuracy. You can use ranged weapons while your mount is sprinting, but at a \minus8 penalty (see \pcref{Sprint}). In either case, you make the attack roll when your mount has completed half its movement. You can make a standard attack with a ranged weapon while your mount is moving. Likewise, you can take move actions normally.

        \parhead{Casting Spells while Mounted} You can cast a spell normally if your mount moves up to a normal move (its speed) either before or after you cast. If you have your mount move both before and after you cast a spell, then you're casting the spell while the mount is moving, and you have to make a Concentration check due to the vigorous motion (DR 10 \add double spell level) or lose the spell. If the mount is sprinting, your Concentration check is more difficult due to the violent motion (DR 15 \add double spell level).

        \parhead{If Your Mount Falls in Battle} If your mount falls, you have to succeed on a DR 15 Ride check to make a soft fall and take no damage. If the check fails, you take 1d6 points of damage.

        \parhead{If You Are Dropped} If you are knocked unconscious, you have a 50\% chance to stay in the saddle (or 75\% if you're in a military saddle). Otherwise you fall and take 1d6 points of damage. Without you to guide it, most mounts avoid combat.

    \subsection{Dual Attacking}\label{Dual Attacking}
        If you are wielding two weapons at once, you can attack with both weapons whenever you attack. This is called dual attacking. Roll a single attack roll for both weapons. If you hit, roll the base damage die for both weapons and add them together. All other damage modifiers are added only once. If the weapons have different damage bonuses, such as if they are magic weapons with different enhancement bonuses, use the lower of the two damage bonuses.

        \parhead{Critical Hits} Normally, when you score a critical hit, you roll damage separately for each weapon and use the higher of the two results when dealing the bonus damage. If your weapons have different \glossterm{critical ranges}, it is possible to only score a critical hit with one of the two weapons. In that case, only use that weapon's damage when dealing extra damage.

        \parhead{Weapon Size} Dual attacking is easiest with light weapons. You take a \minus1 penalty to accuracy if you use a non-light weapon while fighting with two weapons, or a \minus2 penalty if neither weapon is light. This penalty does not apply if you alternate attacks between your weapons, instead of attacking with both at once.

        \parhead{Unarmed Attacks} Normally, you can't make unarmed attacks as if fighting with two weapons. However, a character with the Unarmed Fighting feat gains the ability to make dual attacks with their unarmed attacks (see \featpcref{Unarmed Fighting}).

        \parhead{Example} Felix the 1st-level fighter is wielding two short swords against an evil goblin. The goblin has an Armor defense of 15. Felix has a combat prowess of 3, and is proficient with his swords, so his accuracy is \plus7. If he attacks with both weapons at once, he takes no penalty to his accuracy, because both weapons are light. If he rolls an 8 or higher, he hits the goblin. His damage bonus is \plus1 from his combat prowess, so he rolls 1d6 for each weapon and adds 1, for a total of 2d6\plus1 damage. If he rolled a 2 and a 4, he would deal a total of 7 damage.

    \subsection{Drowning}\label{Drowning}
        You can hold your breath for a number of rounds equal to 5 \add your Constitution.
        After that time, you must roll 1d20.
        This attack gains a \plus5 bonus for each round you hold your breath beyond your limit.
        If the result exceeds your Fortitude defense, you take \glossterm{critical damage} equal to the difference.

    \subsection{Ability Timing}
        Some reactive abilities can be used at times where actions can't normally be taken.
        For example, many abilities specify that you can use them ``when you are hit''.
        This section defines more precisely when such abilities can be used.

        \parhead{When You Are Hit} These abilities are used after the success or failure of all attacks within that phase has been declared, but before any effects of those attacks are declared.
        That means you can activate the ability after you know all of the attacks that hit you during that phase.
        You would also know which attacks were critical hits, allowing you to use the ability to affect those attacks specifically.
