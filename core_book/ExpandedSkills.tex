\chapter{Expanded Skills}

This chapter provides a more detailed explanation of how skills can be used in Rise.
It's generally not worth the time to reference this chapter during an active game session.
However, it can be useful for pre-planning adventures, or for resolving important checks where the players might disagree about how difficult it should be.

There are two main types of information in this chapter.
First, some tasks are simply so rare or esoteric that they aren't worth the space it would take to define them in the core book.
Most campaigns will never need to know exactly how difficult it is to read someone's lips at a distance.
This book has more space to go into detail about infrequently used rules.

Second, the core rules are sometimes vague to allow room for reasonable interpretation.
Your game will inevitably run into situations outside the scope of what can be defined ahead of time in a book, so the core rules have to be flexible.
This chapter provides additional examples and context to help you choose reasonable modifiers for specific or unusual circumstances.
Listing those examples here emphasizes that they are guidelines instead of hard rules.

\section{General Complications}
    In most circumstances, failure and critical failure have no specific detrimental effects defined in the core rules.
    This makes some narratively appropriate consequences of failure impossible in practice.
    For example, the core rules do not provide a way for a character to incorrectly identify a real item as a forgery, or to believe that an undisguised character is wearing a disguise.
    Although those can be interesting developments, and should narratively be possible, it's cumbersome to write rules to make that sort of failure possible without making it overly common.
    Giving completely false information to players should be done sparingly, since it can send them down wild goose chases that take time and effort to resolve for little payoff.

    As a GM, you should feel free to decide that failure in particular circumstances causes additional complications.
    You have the best understanding of whether mistakes or complications will improve the narrative of your game instead of derailing it.

\section{Awareness}

    \subsection{Uncommon Awareness Tasks}
        \parhead{Read Lips} When you see a creature speaking, you can make an sight-based Awareness check to read its lips.
        The \glossterm{difficulty value} is 10 for ordinary conversation, or up to 20 if the speaking creature makes an effort to avoid moving its lips.
        You must be able to understand the language spoken.
        Success means you can understand what is being said.

    \subsection{Awareness and Distance}
        There is no clear and consistent relationship between distances and Awareness modifiers.
        It's basically impossible to read a book from thirty feet away, but a large statue is almost as obvious from that distance as it is from up close.
        On the other hand, a lit torch at night is visible from incredible distances.
        This is very difficult to define in rules, but people generally have good intuitions for what seems reasonable, so this is simply left as an especially tricky area for the GM to determine in the moment.
        
        Keep in mind that different senses can react differently to distances.
        Scent tends to work best when smelling things that are upwind of you, but Rise obviously doesn't have rules for determining the wind speed and direction at any given moment.
        It's fine to keep these details completely abstracted, and just say that a successful Stealth check includes staying downwind without going into too much detail.

        Alternately, you can be very specific about the circumstances.
        If a rogue is trying to sneak up on a wolf, you can tell them that the wind is currently blowing from north to south.
        Based on that wind direction, the wolf will gain an Awareness bonus if they approach from the north or an Awareness penalty if they approach from the south.
        As always, the most important thing is to clearly communicate expectations with your players.

\section{Balance}

    The Balance skill is generally pretty easy to resolve.
    However, when dealing with unusual circumstances or dangerous surfaces, it may not be obvious how much to increase the difficulty of Balance checks.
    You can use \trefnp{Example Balance Modifiers} as a guide.

    \begin{dtable}
        \lcaption{Example Balance Modifiers}
        \begin{dtabularx}{\columnwidth}{l X}
            \tb{Ice}                                  & \tb{DV Modifier} \tableheaderrule
            Rough, hardpacked ice, like a frozen lake & \plus2  \\
            Typical ice                               & \plus5  \\
            Recently frozen or ultra-smooth ice       & \plus10 \\
            \tb{Liquid}                               & \tb{DV Modifier} \tableheaderrule
            Water-covered ground, such as from rain   & \plus2  \\
            Ankle-deep moving stream                  & \plus5  \\
            Knee-deep static water                    & \plus5  \\
            Oil-coated ground                         & \plus5  \\
            Knee-deep moving stream                   & \plus10 \\
            \tb{Sand}                                 & \tb{DV Modifier} \tableheaderrule
            Water-logged beach sand                   & \plus2  \\
            Hard-packed desert sand                   & \plus2  \\
            Typical beach or desert sand              & \plus5  \\
            Quicksand                                 & \plus10 \\
            Unusually smooth, wind-tossed desert sand & \plus10 \\
            \tb{Uneven Ground}                        & \tb{DV Modifier} \tableheaderrule
            Infrequent ankle-high bumps and dips      & \plus2  \\
            Constant ankle-high bumps and dips        & \plus5  \\
            Infrequent knee-high bumps and dips       & \plus5  \\
            Constant knee-high bumps and dips         & \plus10 \\
        \end{dtabularx}
    \end{dtable}

\section{Craft}

    \subsection{Uncommon Craft Tasks}
        \parhead{Craft Disguised Item} You can craft an item that superficially appears to function like a similar, but different, item.
        This functions like creating the item normally, except that you treat the item's \glossterm{rank} as being one higher than it actually is.
        A creature studying the item with the Identify Item task only identifies the item's false purpose unless they get a \glossterm{critical success} on the check.
