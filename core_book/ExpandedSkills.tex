\chapter{Expanded Skills}

This chapter provides a more detailed explanation of how skills can be used in Rise.
It's generally not worth the time to reference this chapter during an active game session.
Instead, you can just use the guidelines for Standard Difficulty Values when unexpected circumstances arise.
Basically, just guess how hard the task seems, choose an appropriate DV, and move on.
However, this chapter can be useful for pre-planning adventures, or for resolving important checks where the players might disagree about how difficult it should be.

There are two main types of information in this chapter.
First, some tasks are simply so rare or esoteric that they aren't worth the space it would take to define them in the core book.
Most campaigns will never need to know exactly how difficult it is to read someone's lips at a distance.
This book has more space to go into detail about infrequently used rules.

Second, the core rules are sometimes vague to allow room for reasonable interpretation.
Your game will inevitably run into situations outside the scope of what can be defined ahead of time in a book, so the core rules have to be flexible.
This chapter provides additional examples and context to help you choose reasonable modifiers for specific or unusual circumstances.
Listing those examples here emphasizes that they are guidelines instead of hard rules.

\section{General Guidance}

  \subsection{Consequences of Failure}
    In most circumstances, failure has no specific detrimental effects defined in the core rules.
    This makes some narratively appropriate consequences of failure impossible in practice.
    For example, the core rules do not provide a way for a character to incorrectly identify a real item as a forgery, or to believe that an undisguised character is wearing a disguise.
    Although those can be interesting developments, and should narratively be possible, it's cumbersome to write rules to make that sort of failure possible without making it overly common.
    Giving completely false information to players should be done sparingly, since it can send them down wild goose chases that take time and effort to resolve for little payoff.

    As a GM, you should feel free to decide that failure in particular circumstances causes additional complications.
    You have the best understanding of whether mistakes or complications will improve the narrative of your game instead of derailing it.

  \subsection{Rushing Skills}
    Many skill tasks are vague about exactly how much time they take to perform.
    This is usually because the time required can vary widely depending on the circumstances.
    For example, throwing on a wig, ashy makeup, and rags to imitate a beggar would take much less time than applying layers of beautifying makeup and donning a formal ball gown, but both may be similarly effective disguises for their intended purpose.

    In most situations, the precise time required to complete tasks isn't critical, and it's reasonable to communicate that explicitly.
    Most players don't mind being fuzzy with the details as long as they know their characters won't suffer negative consequences for being too slow or fast.
    When time is critical, a character can try to rush their task by accepting a penalty of \minus5 or so to their check result.
    As a GM, you will have to use your best judgment about what seems reasonable.

\section{Awareness}
  There is no clear and consistent relationship between distances and Awareness modifiers.
  It's basically impossible to read a book from thirty feet away, but a large statue is almost as obvious from that distance as it is from up close.
  On the other hand, a lit torch at night is visible from incredible distances.
  This is very difficult to define in rules, but people generally have good intuitions for what seems reasonable, so this is simply left as an especially tricky area for the GM to determine in the moment.

  Keep in mind that different senses can react differently to distances.
  Scent tends to work best when smelling things that are upwind of you, but Rise obviously doesn't have rules for determining the wind speed and direction at any given moment.
  It's fine to keep these details completely abstracted, and just say that a successful Stealth check includes staying downwind without going into too much detail.

  Alternately, you can be very specific about the circumstances.
  If a rogue is trying to sneak up on a wolf, you can tell them that the wind is currently blowing from north to south.
  Based on that wind direction, the wolf will gain an Awareness bonus if they approach from the north or an Awareness penalty if they approach from the south.
  As always, the most important thing is to clearly communicate expectations with your players.

\section{Craft}

  Very few items in the book naturally use some of the more obscure Crafting skill options, like bone or ceramics.
  Where it seems plausible, feel free to let players craft alternate versions of common items with different materials.
  If it is a stretch, consider increasing the DV, but still allowing the player to attempt it.
  A galley made of bone instead of wood would be difficult to craft, and is not particularly realistic, but could feel like a very satisfying and appropriate achievement to a player who is skilled with Craft (bone).

\section{Deduction}

  Deduction is a vague skill that can serve multiple purposes, which can make it one of the hardest skills to resolve as a GM.
  Some players specifically dislike solving in-game puzzles, and they want to use Deduction to allow their character to be skilled in that area even if they don't like doing that themselves.
  It's similar to players who want to play socially skilled characters despite not enjoying in-character roleplaying, or not having superhuman social skills in real life.
  That's totally fine!
  If you want this sort of player to have fun, you probably shouldn't be using puzzle-heavy games anyway.
  When puzzles do arise, allowing a sufficiently high Deduction check to basically solve the puzzle on its own will only increase that player's overall fun.

  On the other hand, some players really like puzzle-solving, and they want to use the Deduction skill to give their characters more opportunities to do that.
  For those players, you should generally use their Deduction checks to give them additional clues and allow them to identify evidence that is relevant and discard red herrings.
  However, you should let them draw the final conclusions on their own.

  Of course, you won't always be able to draw those easy divisions.
  Some games will have multiple different player types in it, where some players like puzzles and other players don't.
  You also might not know your players well enough to understand their preferred play styles at first.
  One reliable approach is to say that a successful Deduction check will solve a problem on its own eventually, but the players can try to figure it out first if they want.
  In this context, ``eventually'' can refer either to in-game time or real-life time.

  For example, you might give players a ten-minute time limit in real life to solve a puzzle room on their own.
  If they don't solve it in that time, a successful Deduction check either provides a key hint to get the players unstuck, or simply solves the puzzle completely, depending on whether the players seem to want to spend more time on the puzzle.

  As another example, you can let a character spend an in-game week making a Deduction check to identify the murderer, assuming the party has access to the crime scene and suspects to interrogate.
  The party can accelerate that time frame and lower the difficulty value of the Deduction check by making specific investigations on their own to eliminate suspects or gather additional evidence.
  However, if they don't enjoy that process, it's fine if they just wait a week and roll a die to decide.

\section{Devices}

  The Devices skill is generally easy to resolve as long as you can choose a reasonable difficulty value for the device.
  For examples and common usage, see \tref{Devices Difficulty Values}.
  Of course, you should adjust some of the difficulty values of devices in your games so their difficulty values aren't always perfectly on increments of 5.
  An individual lock might be old and loose, reducing its DV, or unsually well-crafted, increasing its DV.

  It may sometimes be challenging to deal with players who have high Devices checks, since they might be able to bypass any mundane lock they encounter.
  This can it hard to seal off areas that you don't want your players to access yet.
  There are two common ways you can address this.

  First, you can design your game so physical access to arbitrary areas doesn't negate the challenges the party faces.
  This is easiest in narrative-heavy games, since you can keep introducing additional complications.
  Even if the party can break into the paladin's office and find proof of his treachery, that doesn't have to be the end of the story.
  They may not be able to prove the authenticity of the documents, or the documents could be forgeries that were placed there as a ruse by someone who knew the party was likely to snoop around.

  Second, you can make a successful Devices check impossible.
  Magically sealed locks can be useful for this, or at higher levels, magic portals that only activate if the party fulfills specific conditions.
  In dungeon-style games, this is sometimes the only reasonable way to make the dungeon work.
  If you use this approach, make sure to provide other opportunities to reward the player for their high Devices skill, such as middle-tier locks or traps that they can interact with.
  Players will generally be okay with not being able to solve every problem as long as they still feel like they are contributing.

\section{Endurance}\label{Endurance}
  Players should feel free to try to use Endurance to mitigate other narrative challenges beyond the standard set listed in the core book.
  However, you should not allow players to use Endurance to ignore core game mechanics like conditions, encumbrance, or fatigue.
  Skills are primarily intended to serve narrative purposes, not provide raw combat power.

\section{Intimidate}\label{Intimidate}

  Intimidate is much more effective when a character is obviously significantly more powerful than whoever they are trying to intimidate.
  It's reasonable for obviously powerful characters to bully cowardly creatures without making Intimidate checks.
  Of course, there are many kinds of power, so don't just think of this skill as being based on level, physical might, or overt threats.
  Players should be able to use a stone-cold stare (Willpower), enduring pain or injury without reacting (Constitution), or similar creative approaches for their intimidation attempts.

  Sometimes, it might be reasonable to require additional skill checks for specific intimidation approaches.
  If you do require extra skills, you should reward success with a bonus on the Intimidate check.
  It's narratively appropriate, since the character successfully demonstrated superiority in the relevant area.
  In addition, requiring extra skills obviously makes the intimidation attempt harder.
  If you don't provide a corresponding benefit, your players might realize that their odds of success are better if they just vaguely say ``I intimidate them'' without getting too specific.

  Although it's reasonable to give large bonuses for obviously powerful characters, you shouldn't give large penalties to apparently weak characters.
  Part of the skill of intimidation is playing a weak hand well, and having an intimidating presence even when you are at an apparent disadvantage.
  Many stories have been told about characters who have a dangerous presence and inspire fear no matter how bad their situation might seem.
  In some cases, being intimidating despite an obvious disadvantage might be easier if a character can also make a Deception check, but that isn't strictly necessary.

\section{Knowledge}\label{Knowledge}

  \subsection{Monster Identification}
    Although monsters have specific information listed in their descriptions, those generic descriptions might not answer specific questions that are relevant to the players.
    This is especially true if you are making custom modifications to existing monsters or inventing your own monsters from scratch.
    You will have to use your judgment to determine how obvious or well-known specific features are.
    In general, characters in the universe often have a rough understanding of how dangerous monsters are, though they wouldn't use words like ``level''.

    One thing to consider is that it's often good to tell players if their attacks will be useless or very unlikely to succeed ahead of time.
    It can be frustrating for a player to try a particular attack once or twice before they realize that they were wasting their time all along.
    Monsters may also have specific weaknesses that players can try to take advantage of.
    Weak monsters might not have any damage reduction, which can encourage players to use abilities that are more effective when they inflict hit point loss.
    Spellcasters with a wide variety of spells are often particularly interested in learning which of a monster's defenses are lowest, so they can choose the perfect attack.

    Be careful not to get bogged down giving too much specific information to the players before a fight starts.
    Giving players too much information at once can be more confusing than helpful, and ruin any sense of dramatic urgency.
    This is especially true for numeric statistics like a monster's accuracy or defenses.
    Comparative information, like highest or lowest defenses, is generally better than than absolute information, like exact defenses or hit points.

\section{Medicine}
  The Medicine skill cannot be used to regain hit points or remove arbitrary conditions.
  Players often assume that this skill is more directly relevant in an ordinary combat than it is intended to be.
  Instead, the Medicine skill is extremely important for dealing with vital wounds.
  It's almost essential to have at least one person with the Medicine skill in high level groups unless the characters are willing to expend a lot of potions.

\section{Perform}
  The Perform skill does not have as many obvious uses as other skills.
  It primarily exists to provide an opportunity for role-playing, especially for bards.
  If a player finds a reason why this skill might be relevant, it's generally good to let that improvisation work.
  Just don't allow Perform to completely replace other skills.

\section{Persuasion}
  The Persuasion skill is one of the most nuanced, and the most difficult to resolve by simply rolling a die and checking the result.
  For social skills like Persuasion, you need to have a good understanding of what your players enjoy about in-game social interactions.
  Some players enjoy speaking in-character, and want to be rewarded for good role-playing that matches their character.
  This should be generally encouraged where possible, because good role-playing can be fun for everyone.

  However, other players may struggle to speak compellingly as their character, or may simply dislike it.
  No one would require that a player must demonstrate superhuman intelligence to play a wizard, or exceptional strength to play a barbarian.
  For the same reasons, you should not require your players to personally have great social skills in order to play a socially adept character!
  It's completely fine for a player to say ``my character tries to persuade them'', without saying every word that their character says, as long as it's still clear what the objective of the persuasion is.
  You can also encourage players to give it their best shot at speaking in character, and make it clear that NPCs will react as if the character was far more eloquent than the player.

  \subsection{Persuading Groups}
    Creatures often make decisions together, rather than individually.
    For example, in a king's court, a player generally cannot simply influence the king alone; his trusted advisors must also be persuaded.
    There are two ways that you can represent this: competing Persuasion checks, or shared defenses.
    One way would be to have the players make a competing Persuasion check against the advisors, with the highest result determining the king's decision.
    Alternately, you can have the king and his advisors all collectively treat their defense against the player's Persuasion attempt as the highest value among any individuals within the group.
    In general, competing Persuasion checks makes more sense for loose-knit associations, while shared defenses makes more sense for tight-knit groups.

  \subsection{Specific Persuasion Modifiers}
    The Persuasion skill has unusually large circumstantial modifiers compared to other skills.
    This is because the social context surrounding any given persuasion attempt is of critical importance, and only a GM can reliably determine that.
    There are example modifiers unique to the Compel Belief and Form Agreement tasks listed in the tables below.

    \begin{dtable}
      \lcaption{Compel Belief Modifiers}
      \begin{dtabularx}{\columnwidth}{X l}
        \tb{Believability}                                                           & \tb{Difficulty Modifier}  \tableheaderrule
        Expected to be true (``Nothing interesting happened while I was on patrol'') & \minus5         \\
        Plausible (``The mayor is too busy to see you now.'')                        & \plus0          \\
        Unlikely (``That bloodstain was just an accident I had with a razor.'')                               & \plus5          \\
        Extremely unlikely (``Your neighbor is secretly a werewolf.'')                   & \plus10         \\
        Virtually impossible (``That crime was committed by my identical twin, not me.'')              & \plus15 or more \\
        \tb{Incentive}                                                           & \tb{Difficulty Modifier} \tableheaderrule
        Extremely beneficial (``You have an uncle who died and left you his inheritance.'')              & \minus5 \\
        Somewhat beneficial (``That dress looks lovely on you.'')              & \minus2 \\
        No particular impact (``I'm busy.'')                       & \plus0  \\
        Somewhat detrimental (``You can't come with us to the party.'') & \plus5  \\
        Extremely detrimental (``Your brother is a murderer.'') & \plus10 or more  \\
      \end{dtabularx}
    \end{dtable}

    \begin{dtable}
      \lcaption{Form Agreement Modifiers}
      \begin{dtabularx}{\columnwidth}{X l}
        \tb{Risk vs. Reward}                                                           & \tb{Difficulty Modifier}  \tableheaderrule
        Fantastic: The reward for accepting the deal is very worthwhile; the risk is either acceptable or extremely unlikely. The best-case scenario is a virtual guarantee. Example: An offer to pay 10gp for directions to the well-known local tavern.                                                  & \minus10 or more                                                   \\
        Good: The reward is good and the risk is minimal. The target is very likely to profit from the deal. Example: An offer to pay someone twice their normal daily wage to spend their evening in a seedy tavern and later report on everyone they saw there.                  & \minus5                                                           \\
        Favorable: The reward is appealing, but there's risk involved. If all goes according to plan, though, the deal will end up benefiting the target. Example: A request for a mercenary to aid the party in battle against a weak goblin tribe in return for a cut of the money and first pick of the magic items. & \minus2                                                            \\
        Even: The reward and risk more of less even out; or the deal involves neither reward nor risk. Example: A request for directions to a place that isn't a secret.                                                                                                                                                                     & \plus0 \\
        Unfavorable: The reward is not enough compared to the risk involved. Even if all goes according to plan, chances are it will end badly for the target. Example: A request to free a prisoner the target is guarding for a small amount of money.                                                                 & \plus5                                                             \\
        Bad: The reward is poor and the risk is high. The target is very likely to get the raw end of the deal. Example: A request for a mercenary to aid the party in battle against an fearsome dragon for a small cut of any non-magical treasure.                                                                    & \plus10                                                            \\
        Horrible: There is no conceivable way that the proposed plan could end up with the target ahead or the worst-case scenario is guaranteed to occur. Example: An offer to trade a broken sword hilt for a shiny new longsword.                                                                                      & \plus15 or more                                                    \\
      \end{dtabularx}
    \end{dtable}

\section{Profession}
  This skill mostly exists to support the narrative universe surrounding the players, rather than as a tool for players.
  If players do take Profession, it's good to think about when their profession is relevant and reward them for their choice.
  However, make sure not to let Profession completely replace multiple other skills.
  In general, Profession can only ever replace a small subset of an existing skill.
  One way to reward players without making Profession too powerful is by applying a bonus or penalty for Profession-based checks based on the profession's relevance.

  For example, a player with Profession (sailor) shouldn't be able to use that skill to completely replace the Awareness, Devices, and Swim skills.
  It's true that sailors are often good at those things, but players should typically represent that by also having the relevant skills.
  You could allow Profession (sailor) to be used instead of Awareness (because the player might have spent some time in the crow's nest as a lookout), but at a penalty since the skill isn't perfectly relevant.
  You can also get more specific if you want, such as by saying that Profession (sailor) helps you see distant objects better, but it doesn't help your hearing at all.
  This is more detail than you will need most of the time, but it may help you improvise when you need it.

\section{Ride}
  Mounts can be logistically challenging, though little of that complexity comes from the Ride skill itself.
  Characters without the Ride skill are unlikely to ride mounts into combat.
  On the other hand, characters trained in the Ride skill can generally ride mounts well within their skill level with no risk of randomly falling off.
  This means that Ride checks are rare in practice.

  At higher levels, ordinary horses stop being a viable option in combat, both narratively and mechanically.
  % TODO: More examples
  Instead, you should provide players with the opportunity to use more level-appropriate mounts like unicorns.
  Be careful about introducing flying mounts, since flight introduces a great deal of complexity to the game.

\section{Sleight of Hand}
  Don't forget that the target of a pickpocket attempt isn't the only one that can notice it.
  Anyone nearby can notice that something fishy is happening.
  This can be useful if your players start trying to pickpocket everything in sight.
  Yes, the commoners they're stealing from don't have a high Awareness skill, but some guards and passing travellers might.

  % Nothing to say?
  % \section{Social Insight}

\section{Stealth}
  The Stealth skill has a number of specific circumstances that can make it less likely to succeed.
  In general, players are intended to be able to scout and avoid notice out of combat, but it should be nearly impossible to maintain stealth within an active combat.
  Keep in mind that special senses like \trait{blindsight} are powerful anti-Stealth tools, but they are not intended to be a complete defense.
  High-level characters can have high enough bonuses to overcome those penalties, allowing them to sneak up on creatures with blindsense or even blindsight.

\section{Survival}
  The Survival skill is a fairly broad skill.
  It is intended to encompass all of the small details that make it possible to navigate and thrive in the wilderness.

  Survival has some overlap with Knowledge (nature), which generally represents a more abstract book knowledge of the natural world.
  For example, both Survival and Knowledge (nature) could be used to identify whether an individual plant is safe to eat.
  However, only Survival would be used to actively forage for that plant.
  Foraging involves a variety of other practical skills, such as choosing a reasonable search pattern and keeping track of where you have already checked, that fall within the purview of Survival.

  \subsection{Overland Activities}
    There are no standard rules specifying exactly how overland navigation works and what the consequences are for failure, because it depends heavily on the geography in the universe.
    In general, you should require a check to navigate a wilderness when you have a specific obstacle in mind that the party needs to avoid.
    For example, failure might mean that the party stumbles into the territory of a powerful monster.
    If the only real outcome of failure is that the characters waste in-game time getting lost and finding their way again, try not to spend too much real-life time resolving the situation unless it serves your narrative.
    Wasting in-game time can still be an important consequence, especially if the players are on a specific time table to accomplish a goal.
    Just be careful not to waste real-life time on unimportant events.

\section{Swim}
  The Swim skill is one of the most narratively narrow skills, but it can still be very important in aquatic environments.
  There are no specific rules for catastrophic failure, because the consequences depend on the context.
  For example, muddy water with many weeds might very hard to swim through quickly, but you wouldn't be much more likely to drown in it by swimming across the surface than you would in calm water.
  A creature in full plate might also sink like a stone without active effort, while an unarmored creature can usually stay above water by simply floating still.
  Feel free to decide that a creature who fails a Swim check by a large enough amount sinks underwater if it seems appropriate.
