\chapter{Expanded Skills}

This chapter provides a more detailed explanation of how skills can be used in Rise.
It's generally not worth the time to reference this chapter during an active game session.
However, it can be useful for pre-planning adventures, or for resolving important checks where the players might disagree about how difficult it should be.

There are two main types of information in this chapter.
First, some tasks are simply so rare or esoteric that they aren't worth the space it would take to define them in the core book.
Most campaigns will never need to know exactly how difficult it is to read someone's lips at a distance.
This book has more space to go into detail about infrequently used rules.

Second, the core rules are sometimes vague to allow room for reasonable interpretation.
Your game will inevitably run into situations outside the scope of what can be defined ahead of time in a book, so the core rules have to be flexible.
This chapter provides additional examples and context to help you choose reasonable modifiers for specific or unusual circumstances.
Listing those examples here emphasizes that they are guidelines instead of hard rules.

\section{General Guidance}

    \subsection{Consequences of Failure}
        In most circumstances, failure and critical failure have no specific detrimental effects defined in the core rules.
        This makes some narratively appropriate consequences of failure impossible in practice.
        For example, the core rules do not provide a way for a character to incorrectly identify a real item as a forgery, or to believe that an undisguised character is wearing a disguise.
        Although those can be interesting developments, and should narratively be possible, it's cumbersome to write rules to make that sort of failure possible without making it overly common.
        Giving completely false information to players should be done sparingly, since it can send them down wild goose chases that take time and effort to resolve for little payoff.

        As a GM, you should feel free to decide that failure in particular circumstances causes additional complications.
        You have the best understanding of whether mistakes or complications will improve the narrative of your game instead of derailing it.

    \subsection{Rushing Skills}
        Many skill tasks are vague about exactly how much time they take to perform.
        This is usually because the time required can vary widely depending on the circumstances.
        For example, throwing on a wig, ashy makeup, and rags to imitate a beggar would take much less time than applying layers of beautifying makeup and donning a formal ball gown, but both may be similarly effective disguises for their intended purpose.

        In most situations, the precise time required to complete tasks isn't critical, and it's reasonable to communicate that explicitly.
        Most players don't mind being fuzzy with the details as long as they know their characters won't suffer negative consequences for being too slow or fast.
        When time is critical, a character can try to rush their task by accepting a penalty of \minus5 or so to their check result.
        As a GM, you will have to use your best judgment about what seems reasonable.

\section{Awareness}

    \subsection{Uncommon Awareness Tasks}
        \parhead{Read Lips} When you see a creature speaking, you can make an sight-based Awareness check to read its lips.
        The \glossterm{difficulty value} is 10 for ordinary conversation, or up to 20 if the speaking creature makes an effort to avoid moving its lips.
        You must be able to understand the language spoken.
        Success means you can understand what is being said.

    \subsection{Awareness and Distance}
        There is no clear and consistent relationship between distances and Awareness modifiers.
        It's basically impossible to read a book from thirty feet away, but a large statue is almost as obvious from that distance as it is from up close.
        On the other hand, a lit torch at night is visible from incredible distances.
        This is very difficult to define in rules, but people generally have good intuitions for what seems reasonable, so this is simply left as an especially tricky area for the GM to determine in the moment.
        
        Keep in mind that different senses can react differently to distances.
        Scent tends to work best when smelling things that are upwind of you, but Rise obviously doesn't have rules for determining the wind speed and direction at any given moment.
        It's fine to keep these details completely abstracted, and just say that a successful Stealth check includes staying downwind without going into too much detail.

        Alternately, you can be very specific about the circumstances.
        If a rogue is trying to sneak up on a wolf, you can tell them that the wind is currently blowing from north to south.
        Based on that wind direction, the wolf will gain an Awareness bonus if they approach from the north or an Awareness penalty if they approach from the south.
        As always, the most important thing is to clearly communicate expectations with your players.

\section{Balance}

    The Balance skill is generally pretty easy to resolve.
    However, when dealing with unusual circumstances or dangerous surfaces, it may not be obvious how much to increase the difficulty of Balance checks.
    You can use \trefnp{Example Balance Modifiers} as a guide.

    \begin{dtable}
        \lcaption{Example Balance Modifiers}
        \begin{dtabularx}{\columnwidth}{l X}
            \tb{Ice}                                  & \tb{DV Modifier} \tableheaderrule
            Rough, hardpacked ice, like a frozen lake & \plus2  \\
            Typical ice                               & \plus5  \\
            Recently frozen or ultra-smooth ice       & \plus10 \\
            \tb{Liquid}                               & \tb{DV Modifier} \tableheaderrule
            Water-covered ground, such as from rain   & \plus2  \\
            Ankle-deep moving stream                  & \plus5  \\
            Knee-deep static water                    & \plus5  \\
            Oil-coated ground                         & \plus5  \\
            Knee-deep moving stream                   & \plus10 \\
            \tb{Narrow Surface}                       & \tb{DV Modifier} \tableheaderrule
            About two feet wide                       & \plus2  \\
            About one foot wide                       & \plus5  \\
            About six inches wide                     & \plus10 \\
            About two inches wide                     & \plus15 \\
            Less than than two inches wide            & \plus20 \\
            \tb{Sand}                                 & \tb{DV Modifier} \tableheaderrule
            Water-logged beach sand                   & \plus2  \\
            Hard-packed desert sand                   & \plus2  \\
            Typical beach or desert sand              & \plus5  \\
            Quicksand                                 & \plus10 \\
            Unusually smooth, wind-tossed desert sand & \plus10 \\
            \tb{Uneven Ground}                        & \tb{DV Modifier} \tableheaderrule
            Infrequent ankle-high bumps and dips      & \plus2  \\
            Constant ankle-high bumps and dips        & \plus5  \\
            Infrequent knee-high bumps and dips       & \plus5  \\
            Constant knee-high bumps and dips         & \plus10 \\
        \end{dtabularx}
    \end{dtable}

\section{Craft}

    \subsection{Uncommon Craft Tasks}
        \parhead{Craft Disguised Item} You can craft an item that superficially appears to function like a similar, but different, item.
        This functions like creating the item normally, except that you treat the item's \glossterm{rank} as being one higher than it actually is.
        A creature studying the item with the Identify Item task only identifies the item's false purpose unless they get a \glossterm{critical success} on the check.

\section{Creature Handling}

    \subsection{Uncommon Creature Handling Tasks}
        \parhead{Rear a Wild Creature} A character can make a Creature Handling check to raise a wild creature from infancy so that it becomes domesticated. 
        The time required depends on how long it takes the creature in question to reach adulthood.
        The \glossterm{difficulty value} for this check is equal to 5 \add twice the creature's level in its adult form.
        This check must be repeated once per year during the process of raising the creature, and when that process is complete.
        Failure means that an additional year of training is required.
        A successfully domesticated creature can be taught tricks at the same time it's being raised, or it can be taught as a domesticated creature later.

    \subsection{Teaching Tricks}
        Generally speaking, teaching a creature a new trick requires spending at least four hours a day in training over the course of a week.
        It is not generally possible to accelerate the process by spending more time each day; the creature must take time to learn the new behavior.
        If a creature is taught more tricks than its Intelligence allows it to retain, it will forget one of its old tricks during the course of learning the new trick.
        The trainer can choose which old trick will be replaced in this way.

        A list of specific tricks that creatures can be taught is given below.
        Of course, players should feel free to define new tricks to accomplish more specific goals.
        However, complicated tricks are probably more difficult for an animal to learn, so the difficulty value to teach a custom trick might be 15 or higher.

        \parhead{Attack (DV 10)} The creature attacks apparent enemies. You may point to a particular creature that you wish the creature to attack, and it will comply if able. This trick includes teaching the creature how to stop attacking if you give it a command to relent.
        \parhead{Come (DV 5)} The creature comes to you.
        \parhead{Defend (DV 10)} The creature defends you (or is ready to defend you if no threat is present), even without any command being given. Alternatively, you can command the creature to defend a specific other character.
        \parhead{Down (DV 5)} The creature breaks off from combat or otherwise backs down. A creature that doesn't know this trick continues to fight until it must flee (due to injury, a fear effect, or the like) or its opponent is defeated.
        \parhead{Fetch (DV 5)} The creature goes and gets something. If you do not point out a specific item, the creature fetches some random object.
        \parhead{Guard (DV 10)} The creature stays in place and prevents others from approaching.
        \parhead{Heel (DV 5)} The creature follows you closely, even to places where it normally wouldn't go.
        \parhead{Messenger (DV 15)} The creature carries a small item to a destination.
        Once it arrives, it waits for up to 24 hours for someone to take the item from it.
        The destination must be known to the creature.
        \par When you instruct the creature to deliver the item, you must communicate the destination to the creature.
        This normally requires a DV 20 Creature Handling check as a standard action.
        The DV of this check is lowered to 15 for locations the creature is extremely familiar with, such as its home.
        If you have other means of communicating the destination to the creature, such as the \textit{animal speech} druid ability (see Animal Speech, page \pref{Drd:Animal Speech}), that check is unnecessary.
        \parhead{Perform (DV 10)} The creature performs a variety of simple tricks, such as sitting up, rolling over, roaring or barking, and so on.
        \parhead{Seek (DV 5)} The creature moves into an area and looks around for anything that is obviously alive or animate.
        \parhead{Stay (DV 5)} The creature stays in place, waiting for you to return. It does not challenge other creatures that come by, though it still defends itself if it needs to.
        \parhead{Track (DV 10)} The creature tracks the scent presented to it. (This requires the creature to have the scent ability)
        \parhead{Work (DV 5)} The creature pulls or pushes a medium or heavy load.

        \subsubsection{Bonus Tricks}\label{Bonus Tricks}
            Some trainers can teach creatures bonus tricks in addition to their normal maximum number of tricks known.
            Once a creature has learned a bonus trick, that trick may not be retrained into a different trick by a trainer who does not have the same ability to grant bonus tricks.

\section{Deduction}

    Deduction is a vague skill that can serve multiple purposes, which can make it one of the hardest skills to resolve as a GM.
    Some players specifically dislike solving in-game puzzles, and they want to use Deduction to allow their character to be skilled in that area even if they don't like doing that themselves.
    It's similar to players who want to play socially skilled characters despite not enjoying in-character roleplaying, or not having superhuman social skills in real life.
    That's totally fine!
    If you want this sort of player to have fun, you probably shouldn't be using puzzle-heavy games anyway.
    When puzzles do arise, allowing a sufficiently high Deduction check to basically solve the puzzle on its own will only increase that player's overall fun.

    On the other hand, some players really like puzzle-solving, and they want to use the Deduction skill to give their characters more opportunities to do that.
    For those players, you should generally use their Deduction checks to give them additional clues and allow them to identify evidence that is relevant and discard red herrings.
    However, you should let them draw the final conclusions on their own.

    Of course, you won't always be able to draw those easy divisions.
    Some games will have multiple different player types in it, where some players like puzzles and other players don't.
    You also might not know your players well enough to understand their preferred play styles at first.
    One reliable approach is to say that a successful Deduction check will solve a problem on its own eventually, but the players can try to figure it out first if they want.
    In this context, ``eventually'' can refer either to in-game time or real-life time.

    For example, you might give players a ten-minute time limit in real life to solve a puzzle room on their own.
    If they don't solve it in that time, a successful Deduction check either provides a key hint to get the players unstuck, or simply solves the puzzle completely, depending on whether the players seem to want to spend more time on the puzzle.

    As another example, you can let a character spend an in-game week making a Deduction check to identify the murderer, assuming the party has access to the crime scene and suspects to interrogate.
    The party can accelerate that time frame and lower the difficulty value of the Deduction check by making specific investigations on their own to eliminate suspects or gather additional evidence.
    However, if they don't enjoy that process, it's fine if they just wait a week and roll a die to decide.

\section{Devices}

    The Devices skill is generally easy to resolve as long as you can choose a reasonable difficulty value for the device.
    For examples and common usage, see \tref{Devices Difficulty Values}.
    Of course, you should adjust some of the difficulty values of devices in your games so their difficulty values aren't always perfectly on increments of 5.
    An individual lock might be old and loose, reducing its DV, or unsually well-crafted, increasing its DV.

    It may sometimes be challenging to deal with players who have high Devices checks, since they might be able to bypass any mundane lock they encounter.
    This can it hard to seal off areas that you don't want your players to access yet.
    There are two common ways you can address this.

    First, you can design your game so physical access to arbitrary areas doesn't negate the challenges the party faces.
    This is easiest in narrative-heavy games, since you can keep introducing additional complications.
    Even if the party can break into the paladin's office and find proof of his treachery, that doesn't have to be the end of the story.
    They may not be able to prove the authenticity of the documents, or the documents could be forgeries that were placed there as a ruse by someone who knew the party was likely to snoop around.

    Second, you can make a successful Devices check impossible.
    Magically sealed locks can be useful for this, or at higher levels, magic portals that only activate if the party fulfills specific conditions.
    In dungeon-style games, this is sometimes the only reasonable way to make the dungeon work.
    If you use this approach, make sure to provide other opportunities to reward the player for their high Devices skill, such as middle-tier locks or traps that they can interact with.
    Players will generally be okay with not being able to solve every problem as long as they still feel like they are contributing.

    \begin{dtable}
        \lcaption{Devices Difficulty Values}
        \begin{dtabularx}{\columnwidth}{>{\lcol}X c}
            \tb{Device Type}                          & \tb{Devices Difficuty Value} \tableheaderrule
            Simple device (wagon wheel, typical knot) & 5                              \\
            Average device (door hinge, complex knot) & 10                             \\
            Difficult device (typical lock)           & 15                             \\
            Extraordinary device (expert lock)        & 20                             \\
            Impossible device (magically sealed lock) & 25                             \\
            Mundane trap                              & 10 \add twice \glossterm{rank} \\
            Magic trap                                & 15 \add twice \glossterm{rank} \\
        \end{dtabularx}
    \end{dtable}

\section{Disguise}
    The key complexity of the Diguise skill involves applying appropriate penalties for drastic body changes.
    It is generally easier to enlarge a creature or add new features than it is to shrink a creature or remove existing features.
    You can use \tref{Example Disguise Modifiers} as a guide, and improvise as necessary.
    If a creature makes multiple major alterations, the penalties stack.

    \begin{dtable}
        \lcaption{Example Balance Modifiers}
        \begin{dtabularx}{\columnwidth}{l X}
            \tb{Age Change}                              & \tb{Disguise Penalty} \\
            Per age category of difference & \minus2 \\
            \tb{Body Shape Change}                              & \tb{Disguise Penalty} \\
            To a different gender                               & \minus2               \\
            Per removed limb                                    & \minus5               \\
            Per additional limb                                 & \minus10              \\
            From bipedal to quadrupedal                         & \minus5               \\
            From quadrupedal to bipedal                         & \minus10              \\
            \tb{Species Change}                                 & \tb{Disguise Penalty} \tableheaderrule
            To a similar-size species (human to elf)            & \minus2               \\
            To a noticeably larger species (halfling to human)  & \minus5               \\
            To a noticeably smaller species (human to halfling) & \minus15              \\
            To a larger size category (human to ogre)           & \minus15              \\
        \end{dtabularx}
    \end{dtable}

\section{Endurance}\label{Endurance}
    Players should feel free to try to use Endurance to mitigate other narrative challenges beyond the standard set listed in the core book.
    However, you should not allow players to use Endurance to ignore core game mechanics like conditions, encumbrance, or fatigue.
    Skills are primarily intended to serve narrative purposes, and Endurance's ability to ignore vital wounds already makes it an unusually powerful combat skill.
    If you increase its power further, it could easily become virtually mandatory for combat-focused characters.

    \subsection{Uncommon Endurance Tasks}
        \parhead{Overland Exertion} You can make an Endurance check while travelling overland to cover more round (see \pcref{Overland Movement}).
        There are two ways that you can exert yourself: hustling, which doubles your distance travelled during a given hour, and making a forced march, which allows you to travel for an extra hour beyond the normal travel time.
        Exerting yourself for an hour requires a \glossterm{difficulty value} 5 Endurance check.
        The \glossterm{difficulty value} increases by 2 for every hour you spend exerting yourself between \glossterm{long rests}.
        If you combine both forms of exertion, you increase the \glossterm{difficulty value} of the check by 4 for that hour instead of by 2.
        Failure means you gain a \glossterm{vital wound} from \glossterm{subdual damage}.
