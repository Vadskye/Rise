\chapter{Expanded Skills}

This chapter provides a more detailed explanation of how skills can be used in Rise.
It's generally not worth the time to reference this chapter during an active game session.
However, it can be useful for pre-planning adventures, or for resolving important checks where the players might disagree about how difficult it should be.

There are two main types of information in this chapter.
First, some tasks are simply so rare or esoteric that they aren't worth the space it would take to define them in the core book.
Most campaigns will never need to know exactly how difficult it is to read someone's lips at a distance.
This book has more space to go into detail about infrequently used rules.

Second, the core rules are sometimes vague to allow room for reasonable interpretation.
Your game will inevitably run into situations outside the scope of what can be defined ahead of time in a book, so the core rules have to be flexible.
This chapter provides additional examples and context to help you choose reasonable modifiers for specific or unusual circumstances.
Listing those examples here emphasizes that they are guidelines instead of hard rules.

\section{General Guidance}

    \subsection{Consequences of Failure}
        In most circumstances, failure and critical failure have no specific detrimental effects defined in the core rules.
        This makes some narratively appropriate consequences of failure impossible in practice.
        For example, the core rules do not provide a way for a character to incorrectly identify a real item as a forgery, or to believe that an undisguised character is wearing a disguise.
        Although those can be interesting developments, and should narratively be possible, it's cumbersome to write rules to make that sort of failure possible without making it overly common.
        Giving completely false information to players should be done sparingly, since it can send them down wild goose chases that take time and effort to resolve for little payoff.

        As a GM, you should feel free to decide that failure in particular circumstances causes additional complications.
        You have the best understanding of whether mistakes or complications will improve the narrative of your game instead of derailing it.

    \subsection{Rushing Skills}
        Many skill tasks are vague about exactly how much time they take to perform.
        This is usually because the time required can vary widely depending on the circumstances.
        For example, throwing on a wig, ashy makeup, and rags to imitate a beggar would take much less time than applying layers of beautifying makeup and donning a formal ball gown, but both may be similarly effective disguises for their intended purpose.

        In most situations, the precise time required to complete tasks isn't critical, and it's reasonable to communicate that explicitly.
        Most players don't mind being fuzzy with the details as long as they know their characters won't suffer negative consequences for being too slow or fast.
        When time is critical, a character can try to rush their task by accepting a penalty of \minus5 or so to their check result.
        As a GM, you will have to use your best judgment about what seems reasonable.

\section{Awareness}

    \subsection{Uncommon Awareness Tasks}
        \parhead{Read Lips} When you see a creature speaking, you can make an sight-based Awareness check to read its lips.
        The \glossterm{difficulty value} is 10 for ordinary conversation, or up to 20 if the speaking creature makes an effort to avoid moving its lips.
        You must be able to understand the language spoken.
        Success means you can understand what is being said.

    \subsection{Awareness and Distance}
        There is no clear and consistent relationship between distances and Awareness modifiers.
        It's basically impossible to read a book from thirty feet away, but a large statue is almost as obvious from that distance as it is from up close.
        On the other hand, a lit torch at night is visible from incredible distances.
        This is very difficult to define in rules, but people generally have good intuitions for what seems reasonable, so this is simply left as an especially tricky area for the GM to determine in the moment.
        
        Keep in mind that different senses can react differently to distances.
        Scent tends to work best when smelling things that are upwind of you, but Rise obviously doesn't have rules for determining the wind speed and direction at any given moment.
        It's fine to keep these details completely abstracted, and just say that a successful Stealth check includes staying downwind without going into too much detail.

        Alternately, you can be very specific about the circumstances.
        If a rogue is trying to sneak up on a wolf, you can tell them that the wind is currently blowing from north to south.
        Based on that wind direction, the wolf will gain an Awareness bonus if they approach from the north or an Awareness penalty if they approach from the south.
        As always, the most important thing is to clearly communicate expectations with your players.

\section{Balance}

    The Balance skill is generally pretty easy to resolve.
    However, when dealing with unusual circumstances or dangerous surfaces, it may not be obvious how much to increase the difficulty of Balance checks.
    You can use \trefnp{Example Balance Modifiers} as a guide.

    \begin{dtable}
        \lcaption{Example Balance Modifiers}
        \begin{dtabularx}{\columnwidth}{l X}
            \tb{Ice}                                  & \tb{DV Modifier} \tableheaderrule
            Rough, hardpacked ice, like a frozen lake & \plus2  \\
            Typical ice                               & \plus5  \\
            Recently frozen or ultra-smooth ice       & \plus10 \\
            \tb{Liquid}                               & \tb{DV Modifier} \tableheaderrule
            Water-covered ground, such as from rain   & \plus2  \\
            Ankle-deep moving stream                  & \plus5  \\
            Knee-deep static water                    & \plus5  \\
            Oil-coated ground                         & \plus5  \\
            Knee-deep moving stream                   & \plus10 \\
            \tb{Narrow Surface}                       & \tb{DV Modifier} \tableheaderrule
            About two feet wide                       & \plus2  \\
            About one foot wide                       & \plus5  \\
            About six inches wide                     & \plus10 \\
            About two inches wide                     & \plus15 \\
            Less than than two inches wide            & \plus20 \\
            \tb{Sand}                                 & \tb{DV Modifier} \tableheaderrule
            Water-logged beach sand                   & \plus2  \\
            Hard-packed desert sand                   & \plus2  \\
            Typical beach or desert sand              & \plus5  \\
            Quicksand                                 & \plus10 \\
            Unusually smooth, wind-tossed desert sand & \plus10 \\
            \tb{Uneven Ground}                        & \tb{DV Modifier} \tableheaderrule
            Infrequent ankle-high bumps and dips      & \plus2  \\
            Constant ankle-high bumps and dips        & \plus5  \\
            Infrequent knee-high bumps and dips       & \plus5  \\
            Constant knee-high bumps and dips         & \plus10 \\
        \end{dtabularx}
    \end{dtable}

\section{Craft}

    \subsection{Uncommon Craft Tasks}
        \parhead{Craft Disguised Item} You can craft an item that superficially appears to function like a similar, but different, item.
        This functions like creating the item normally, except that you treat the item's \glossterm{rank} as being one higher than it actually is.
        A creature studying the item with the Identify Item task only identifies the item's false purpose unless they get a \glossterm{critical success} on the check.

\section{Creature Handling}

    \subsection{Uncommon Creature Handling Tasks}
        \parhead{Rear a Wild Creature} A character can make a Creature Handling check to raise a wild creature from infancy so that it becomes domesticated. 
        The time required depends on how long it takes the creature in question to reach adulthood.
        The \glossterm{difficulty value} for this check is equal to 5 \add twice the creature's level in its adult form.
        This check must be repeated once per year during the process of raising the creature, and when that process is complete.
        Failure means that an additional year of training is required.
        A successfully domesticated creature can be taught tricks at the same time it's being raised, or it can be taught as a domesticated creature later.

    \subsection{Teaching Tricks}
        Generally speaking, teaching a creature a new trick requires spending at least four hours a day in training over the course of a week.
        It is not generally possible to accelerate the process by spending more time each day; the creature must take time to learn the new behavior.
        If a creature is taught more tricks than its Intelligence allows it to retain, it will forget one of its old tricks during the course of learning the new trick.
        The trainer can choose which old trick will be replaced in this way.

        A list of specific tricks that creatures can be taught is given below.
        Of course, players should feel free to define new tricks to accomplish more specific goals.
        However, complicated tricks are probably more difficult for an animal to learn, so the difficulty value to teach a custom trick might be 15 or higher.

        \parhead{Attack (DV 10)} The creature attacks apparent enemies. You may point to a particular creature that you wish the creature to attack, and it will comply if able. This trick includes teaching the creature how to stop attacking if you give it a command to relent.
        \parhead{Come (DV 5)} The creature comes to you.
        \parhead{Defend (DV 10)} The creature defends you (or is ready to defend you if no threat is present), even without any command being given. Alternatively, you can command the creature to defend a specific other character.
        \parhead{Down (DV 5)} The creature breaks off from combat or otherwise backs down. A creature that doesn't know this trick continues to fight until it must flee (due to injury, a fear effect, or the like) or its opponent is defeated.
        \parhead{Fetch (DV 5)} The creature goes and gets something. If you do not point out a specific item, the creature fetches some random object.
        \parhead{Guard (DV 10)} The creature stays in place and prevents others from approaching.
        \parhead{Heel (DV 5)} The creature follows you closely, even to places where it normally wouldn't go.
        \parhead{Messenger (DV 15)} The creature carries a small item to a destination.
        Once it arrives, it waits for up to 24 hours for someone to take the item from it.
        The destination must be known to the creature.
        \par When you instruct the creature to deliver the item, you must communicate the destination to the creature.
        This normally requires a DV 20 Creature Handling check as a standard action.
        The DV of this check is lowered to 15 for locations the creature is extremely familiar with, such as its home.
        If you have other means of communicating the destination to the creature, such as the \textit{animal speech} druid ability (see Animal Speech, page \pref{Drd:Animal Speech}), that check is unnecessary.
        \parhead{Perform (DV 10)} The creature performs a variety of simple tricks, such as sitting up, rolling over, roaring or barking, and so on.
        \parhead{Seek (DV 5)} The creature moves into an area and looks around for anything that is obviously alive or animate.
        \parhead{Stay (DV 5)} The creature stays in place, waiting for you to return. It does not challenge other creatures that come by, though it still defends itself if it needs to.
        \parhead{Track (DV 10)} The creature tracks the scent presented to it. (This requires the creature to have the scent ability)
        \parhead{Work (DV 5)} The creature pulls or pushes a medium or heavy load.

        \subsubsection{Bonus Tricks}\label{Bonus Tricks}
            Some trainers can teach creatures bonus tricks in addition to their normal maximum number of tricks known.
            Once a creature has learned a bonus trick, that trick may not be retrained into a different trick by a trainer who does not have the same ability to grant bonus tricks.

\section{Deduction}

    Deduction is a vague skill that can serve multiple purposes, which can make it one of the hardest skills to resolve as a GM.
    Some players specifically dislike solving in-game puzzles, and they want to use Deduction to allow their character to be skilled in that area even if they don't like doing that themselves.
    It's similar to players who want to play socially skilled characters despite not enjoying in-character roleplaying, or not having superhuman social skills in real life.
    That's totally fine!
    If you want this sort of player to have fun, you probably shouldn't be using puzzle-heavy games anyway.
    When puzzles do arise, allowing a sufficiently high Deduction check to basically solve the puzzle on its own will only increase that player's overall fun.

    On the other hand, some players really like puzzle-solving, and they want to use the Deduction skill to give their characters more opportunities to do that.
    For those players, you should generally use their Deduction checks to give them additional clues and allow them to identify evidence that is relevant and discard red herrings.
    However, you should let them draw the final conclusions on their own.

    Of course, you won't always be able to draw those easy divisions.
    Some games will have multiple different player types in it, where some players like puzzles and other players don't.
    You also might not know your players well enough to understand their preferred play styles at first.
    One reliable approach is to say that a successful Deduction check will solve a problem on its own eventually, but the players can try to figure it out first if they want.
    In this context, ``eventually'' can refer either to in-game time or real-life time.

    For example, you might give players a ten-minute time limit in real life to solve a puzzle room on their own.
    If they don't solve it in that time, a successful Deduction check either provides a key hint to get the players unstuck, or simply solves the puzzle completely, depending on whether the players seem to want to spend more time on the puzzle.

    As another example, you can let a character spend an in-game week making a Deduction check to identify the murderer, assuming the party has access to the crime scene and suspects to interrogate.
    The party can accelerate that time frame and lower the difficulty value of the Deduction check by making specific investigations on their own to eliminate suspects or gather additional evidence.
    However, if they don't enjoy that process, it's fine if they just wait a week and roll a die to decide.

\section{Devices}

    The Devices skill is generally easy to resolve as long as you can choose a reasonable difficulty value for the device.
    For examples and common usage, see \tref{Devices Difficulty Values}.
    Of course, you should adjust some of the difficulty values of devices in your games so their difficulty values aren't always perfectly on increments of 5.
    An individual lock might be old and loose, reducing its DV, or unsually well-crafted, increasing its DV.

    It may sometimes be challenging to deal with players who have high Devices checks, since they might be able to bypass any mundane lock they encounter.
    This can it hard to seal off areas that you don't want your players to access yet.
    There are two common ways you can address this.

    First, you can design your game so physical access to arbitrary areas doesn't negate the challenges the party faces.
    This is easiest in narrative-heavy games, since you can keep introducing additional complications.
    Even if the party can break into the paladin's office and find proof of his treachery, that doesn't have to be the end of the story.
    They may not be able to prove the authenticity of the documents, or the documents could be forgeries that were placed there as a ruse by someone who knew the party was likely to snoop around.

    Second, you can make a successful Devices check impossible.
    Magically sealed locks can be useful for this, or at higher levels, magic portals that only activate if the party fulfills specific conditions.
    In dungeon-style games, this is sometimes the only reasonable way to make the dungeon work.
    If you use this approach, make sure to provide other opportunities to reward the player for their high Devices skill, such as middle-tier locks or traps that they can interact with.
    Players will generally be okay with not being able to solve every problem as long as they still feel like they are contributing.

    \begin{dtable}
        \lcaption{Devices Difficulty Values}
        \begin{dtabularx}{\columnwidth}{>{\lcol}X c}
            \tb{Device Type}                          & \tb{Difficulty Value} \tableheaderrule
            Simple device (wagon wheel, typical knot) & 5                              \\
            Average device (door hinge, complex knot) & 10                             \\
            Difficult device (typical lock)           & 15                             \\
            Extraordinary device (expert lock)        & 20                             \\
            Impossible device (magically sealed lock) & 25                             \\
            Mundane trap                              & 10 \add twice \glossterm{rank} \\
            Magic trap                                & 15 \add twice \glossterm{rank} \\
        \end{dtabularx}
    \end{dtable}

\section{Disguise}
    The key complexity of the Diguise skill involves applying appropriate penalties for drastic body changes.
    It is generally easier to enlarge a creature or add new features than it is to shrink a creature or remove existing features.
    You can use \tref{Example Disguise Modifiers} as a guide, and improvise as necessary.
    If a creature makes multiple major alterations, the penalties stack.

    \begin{dtable}
        \lcaption{Example Balance Modifiers}
        \begin{dtabularx}{\columnwidth}{l X}
            \tb{Age Change}                              & \tb{Disguise Penalty} \\
            Per age category of difference & \minus2 \\
            \tb{Body Shape Change}                              & \tb{Disguise Penalty} \\
            To a different gender                               & \minus2               \\
            Per removed limb                                    & \minus5               \\
            Per additional limb                                 & \minus10              \\
            From bipedal to quadrupedal                         & \minus5               \\
            From quadrupedal to bipedal                         & \minus10              \\
            \tb{Species Change}                                 & \tb{Disguise Penalty} \tableheaderrule
            To a similar-size species (human to elf)            & \minus2               \\
            To a noticeably larger species (halfling to human)  & \minus5               \\
            To a noticeably smaller species (human to halfling) & \minus15              \\
            To a larger size category (human to ogre)           & \minus15              \\
        \end{dtabularx}
    \end{dtable}

\section{Endurance}\label{Endurance}
    Players should feel free to try to use Endurance to mitigate other narrative challenges beyond the standard set listed in the core book.
    However, you should not allow players to use Endurance to ignore core game mechanics like conditions, encumbrance, or fatigue.
    Skills are primarily intended to serve narrative purposes, and Endurance's ability to ignore vital wounds already makes it an unusually powerful combat skill.
    If you increase its power further, it could easily become virtually mandatory for combat-focused characters.

    \subsection{Uncommon Endurance Tasks}
        \parhead{Overland Exertion} You can make an Endurance check while travelling overland to cover more round (see \pcref{Overland Movement}).
        There are two ways that you can exert yourself: hustling, which doubles your distance travelled during a given hour, and making a forced march, which allows you to travel for an extra hour beyond the normal travel time.
        Exerting yourself for an hour requires a \glossterm{difficulty value} 5 Endurance check.
        The \glossterm{difficulty value} increases by 2 for every hour you spend exerting yourself between \glossterm{long rests}.
        If you combine both forms of exertion, you increase the \glossterm{difficulty value} of the check by 4 for that hour instead of by 2.
        Failure means you gain a \glossterm{vital wound} from \glossterm{subdual damage}.

\section{Flexibility}\label{Flexibility}

    The Flexibility skill has a fairly small narrative space, but it can provide significant combat utility.

        \begin{dtable}
            \lcaption{Flexibility Difficulty Values}
            \begin{dtabularx}{\columnwidth}{>{\lcol}X l}
                \tb{Restraint}         & \tb{Difficulty Value} \tableheaderrule
                Net                    & 5  \\
                Common manacles        & 15 \\
                High-quality manacles  & 20 \\
                Extraordinary manacles & 25 \\
            \end{dtabularx}
        \end{dtable}

\section{Intimidate}\label{Intimidate}

    Intimidate is massively more effective when a character is obviously significantly more powerful than whoever they are trying to intimidate.
    It's okay for high-level, obviously powerful characters to bully cowardly creatures without making Intimidate checks.
    Of course, there are many kinds of power, so don't just think of this skill as being based on level, physical might, or overt threats.
    Players should be able to use a stone-cold stare (Willpower), acting menacingly sober after ingesting a significant amount of alcohol (Constitution), or similar creative approaches for their intimidation attempts.

    Sometimes, it might be reasonable to require additional skill checks for specific intimidation approaches.
    If you do require extra skills, you should reward success with a bonus on the Intimidate check.
    It's narratively appropriate, since the character successfully demonstrated superiority in the relevant area.
    In addition, requiring extra skills obviously makes the intimidation attempt harder.
    If you don't provide a corresponding benefit, your players might realize that their odds of success are better if they just vaguely say ``I intimidate them'' without getting too specific.

    Although it's reasonable to give large bonuses for obviously powerful characters, you shouldn't give large penalties to apparently weak characters.
    Part of the skill of intimidation is playing a weak hand well, and having an intimidating presence even when you are at an apparent disadvantage.
    Many stories have been told about characters who have a dangerous presence and inspire fear no matter how bad their situation might seem.
    In some cases, being intimidating despite an obvious disadvantage might be easier if a character can also make a Deception check, but that isn't strictly necessary.

\section{Jump}\label{Jump}

    \subsection{Falling After Jumping}
        It is possible to make a Jump check that leaves in a creature in midair at the end of its movement.
        At the end of each phase after it jumped into midair, it falls as normal.
        This can allow a creature to take actions in midair during the phase immediately after its jump.
        For example, a fighter fighting against an airborne foe can jump into the air during the \glossterm{movement phase} and act in midair during the \glossterm{action phase}.

        % Is this necessary? Seems implied?
        If an insufficiently long jump would cause a creature to fall into a gap, it can attempt to stop its fall with the Grab Surface use of the Climb skill if it can reach the wall.

    \subsection{Jump Arcs}
        Generally, it's way too complicated to deal with the exact path that a jumping creature takes during its jump.
        According to the Rise rules, some of the jumping arcs look very strange.
        For example, if a creature jumps forward at a 45 degree angle into thin air, it will fall to the ground in a straight line at the end of its jump, which creates an unrealistic triangle-shaped trajectory.
        Try not to get bogged down in the details of exactly what space creatures occupy in midair, or the exact arc.

        If you really want to be more detailed, you can say that a creature's maximum height during a long jump must happen in the middle of its jump.
        That maximum height jump must be no greater than half of the Jump check result, and no less than a quarter of the forward distance travelled.
        Those rules generate more realistic outcomes if a creature tries to make a thirty-foot long jump in a room with a five-foot ceiling, since the creature should hit its head on the ceiling and be unable to complete its jump.
        However, although this creates more realistic results, it is much more convoluted to resolve, which can take time.
        In addition, it weakens the mobility of mundane characters, which makes magical forms of mobility like flight even more powerful than they already are.
        As always, use the rules and conventions that keep you and your player group happiest.

\section{Knowledge}\label{Knowledge}
    
    \subsection{Monster Identification}
        Although monsters have specific information listed in their descriptions, those generic descriptions might not answer specific questions that are relevant to the players.
        This is especially true if you are making custom modifications to existing monsters or inventing your own monsters from scratch.
        You will have to use your judgment to determine how obvious or well-known specific features are.
        In general, characters in the universe often have a rough understanding of how dangerous monsters are, though they wouldn't use words like ``level'' or ``challenge rating''.

        One thing to consider is that it's often good to tell players if their attacks will be useless or very unlikely to succeed ahead of time.
        It can be frustrating for a player to try a particular attack once or twice before they realize that they were wasting their time all along.
        Monsters may also have specific weaknesses that players can try to take advantage of.
        Weak monsters might not have any damage reduction, which can encourage players to use abilities that are more effective when they inflict hit point loss.
        Spellcasters with a wide variety of spells are often particularly interested in learning which of a monster's defenses are lowest, so they can choose the perfect attack.

        Be careful not to get bogged down giving too much specific information to the players before a fight starts.
        Giving players too much information at once can be more confusing than helpful, and ruin any sense of dramatic urgency.
        This is especially true for numeric statistics like a monster's accuracy or defenses.
        Comparative information, like highest or lowest defenses, is generally better than than absolute information, like exact defenses or hit points.

\section{Linguistics}

    For the most part, Rise mechanics assume that you either do or don't know a language, with no middle ground.
    This obviously doesn't match the real world, where people often know small snippets from other languages, and learning a language is a gradual process that can take years.
    These nuances often don't matter, so it's not worth the effort to formalize a complex language proficiency system.
    In addition, the Linguistics skill can help smooth out oddities.
    It's reasonable to allow characters to make Linguistics checks with a low DV, such as 5 or 10, to understand particularly simple messages like ``Help'' from unfamiliar languages.

\section{Medicine}
    The Medicine skill cannot be used to regain hit points or remove arbitrary conditions.
    Players often assume that this skill is more directly relevant in an ordinary combat than it is intended to be.
    Instead, the Medicine skill is extremely important for dealing with vital wounds.
    It's almost essential to have at least one person with the Medicine skill in high level groups unless the characters are willing to expend a lot of potions.

\section{Perform}
    The Perform skill does not have as many obvious uses as other skills.
    It primarily exists to provide an opportunity for role-playing, especially for bards.

\section{Persuasion}
    The Persuasion skill is one of the most nuanced, and the most difficult to resolve by simply rolling a die and checking the result.
    For social skills like Persuasion, you need to have a good understanding of what your players enjoy about in-game social interactions.
    Some players enjoy speaking in-character, and want to be rewarded for good role-playing that matches their character.
    This should be generally encouraged where possible, because good role-playing can be fun for everyone.

    However, other players may struggle to speak compellingly as their character, or may simply dislike it.
    No one would require that a player must demonstrate superhuman intelligence to play a wizard, or exceptional strength to play a barbarian.
    For the same reasons, you should not require your players to personally have great social skills in order to play a socially adept character!
    It's completely fine for a player to say ``my character tries to persuade them'', without saying every word that their character says, as long as it's still clear what the objective of the persuasion is.
    You can also encourage players to give it their best shot at speaking in character, and make it clear that NPCs will react as if the character was far more eloquent than the player.

    \subsection{Persuading Groups}
        Creatures often make decisions together, rather than individually.
        For example, in a queen's court, a player generally cannot simply influence the queen alone; her trusted advisors must also be persuaded.
        There are two ways that you can represent this: competing Persuasion checks, or shared defenses.
        One way would be to have the players make a competing Persuasion check against the advisors, with the highest result determining the queen's decision.
        Alternately, you can have the queen and her advisors all collectively treat their defense against the player's Persuasion attempt as the highest value among any individuals within the group.
        In general, competing Persuasion checks makes more sense for loose-knit associations, while shared defenses makes more sense for tight-knit groups.

    \subsection{Specific Persuasion Modifiers}
        The Persuasion skill has unusually large circumstantial modifiers compared to other skills.
        This is because the social context surrounding any given persuasion attempt is of critical importance, and only a GM can reliably determine that.
        There are example modifiers unique to the Compel Belief and Form Agreement tasks listed in the tables below.

        \begin{dtable}
            \lcaption{Compel Belief Modifiers}
            \begin{dtabularx}{\columnwidth}{X l}
                \tb{Believability}                                                           & \tb{Difficulty Modifier}  \tableheaderrule
                Expected to be true (``Nothing interesting happened while I was on patrol'') & \minus5         \\
                Plausible (``The mayor is too busy to see you now.'')                        & \plus0          \\
                Unlikely (``That bloodstain was just an accident I had with a razor.'')                               & \plus5          \\
                Extremely unlikely (``Your neighbor is secretly a werewolf.'')                   & \plus10         \\
                Virtually impossible (``That crime was committed by my identical twin, not me.'')              & \plus15 or more \\
                \tb{Incentive}                                                           & \tb{Difficulty Modifier} \tableheaderrule
                Extremely beneficial (``You have an uncle who died and left you his inheritance.'')              & \minus5 \\
                Somewhat beneficial (``That dress looks lovely on you.'')              & \minus2 \\
                No particular impact (``I'm busy.'')                       & \plus0  \\
                Somewhat detrimental (``You can't come with us to the party.'') & \plus5  \\
                Extremely detrimental (``Your brother is a murderer.'') & \plus10 or more  \\
            \end{dtabularx}
        \end{dtable}

        \begin{dtable}
            \lcaption{Form Agreement Modifiers}
            \begin{dtabularx}{\columnwidth}{X l}
                \tb{Risk vs. Reward}                                                           & \tb{Difficulty Modifier}  \tableheaderrule
                Fantastic: The reward for accepting the deal is very worthwhile; the risk is either acceptable or extremely unlikely. The best-case scenario is a virtual guarantee. Example: An offer to pay 10gp for directions to the well-known local tavern.                                                  & \minus10 or more                                                   \\
                Good: The reward is good and the risk is minimal. The target is very likely to profit from the deal. Example: An offer to pay someone twice their normal daily wage to spend their evening in a seedy tavern and later report on everyone they saw there.                  & \minus5                                                           \\
                Favorable: The reward is appealing, but there's risk involved. If all goes according to plan, though, the deal will end up benefiting the target. Example: A request for a mercenary to aid the party in battle against a weak goblin tribe in return for a cut of the money and first pick of the magic items. & \minus2                                                            \\
                Even: The reward and risk more of less even out; or the deal involves neither reward nor risk. Example: A request for directions to a place that isn't a secret.                                                                                                                                                                     & \plus0 \\
                Unfavorable: The reward is not enough compared to the risk involved. Even if all goes according to plan, chances are it will end badly for the target. Example: A request to free a prisoner the target is guarding for a small amount of money.                                                                 & \plus5                                                             \\
                Bad: The reward is poor and the risk is high. The target is very likely to get the raw end of the deal. Example: A request for a mercenary to aid the party in battle against an fearsome dragon for a small cut of any non-magical treasure.                                                                    & \plus10                                                            \\
                Horrible: There is no conceivable way that the proposed plan could end up with the target ahead or the worst-case scenario is guaranteed to occur. Example: An offer to trade a broken sword hilt for a shiny new longsword.                                                                                      & \plus15 or more                                                    \\
            \end{dtabularx}
        \end{dtable}
