\chapter{Reference}\label{Reference}

\section{Circumstances and Debuffs}\label{Circumstances and Debuffs}

\debuffdef{blinded} A blinded creature cannot see.
It automatically fails at actions which depend on vision, including simply seeing the locations of other objects and creatures (but see \pcref{Awareness}).
In addition, it has a 50\% miss chance with all attacks.
It is at least \partiallyunaware of all attacks against it, and it can be fully \unaware as normal depending on its level of awareness.

\debuffdef{charmed} A charmed creature is mentally influenced to like another creature.
It always sees the words and actions of the creature that charmed it in the most favorable way, as a close friend or trusted ally.
A charmed creature cannot be controlled like an automaton, but can be persuaded to take particular actions with the Persuasion skill (see \pcref{Persuasion}).
It treats the creature that charmed it as a friend (a \plus10 relationship modifier) for the purpose of Persuasion checks.

\debuffdef{confused} A confused creature takes a \minus4 penalty to all defenses and is unable to independently control its actions.
This penalty does not stack with the \dazed or \stunned effects.
\confusionexplanation

\debuffdef{crouching} A crouching creature is ducking down instead of standing normally.
It takes a \minus2 accuracy penalty and it moves at half speed.
Melee \glossterm{strikes} against it gain a \plus2 bonus to \glossterm{accuracy}, while ranged \glossterm{strikes} against it take a \minus2 accuracy penalty.
You cannot crouch while flying or gliding.

\debuffdef{dazed} A dazed creature takes a \minus2 penalty to all defenses.
This does not stack with the \stunned or \confused effects.

\debuffdef{dazzled} A dazzled creature has difficulty seeing.
It loses the benefits of the \glossterm{darkvision} and \glossterm{low-light vision} abilities if it has them.
In addition, it treats everything as if it had \glossterm{concealment}.
Among other effects, this gives its attacks a 25\% \glossterm{miss chance}.

\debuffdef{deafened} A deafened creature cannot hear. It automatically fails at actions which depend on hearing. In addition, it has a 20\% failure chance when casting any spell with verbal components.

\debuffdef{decelerated} A decelerated creature moves at one quarter speed and takes a \minus4 penalty to Reflex defense.
This does not stack with the \slowed or \immobilized effects.

\debuffdef{disoriented} A disoriented creature cannot control the direction of its movement.
If it tries to use one of its movement speeds to move a given distance, it moves that distance in a random direction instead of in its intended direction.
A disoriented creature can remain in the same location without penalty.
This does not affect abilities that change location without normal movement, such as \glossterm{teleportation}.

\debuffdef{dominated}[dominate] A dominated creature is mentally compelled to obey another creature.
It obeys the commands of the creature of the dominated it unquestioningly, as an automaton.
If it does not understand the language of the creature that dominated it, it still attempts to obey as much as possible, and simple commmands (such as ``attack'' or ``follow'') can usually be communicated successfully.

\debuffdef{fascinated} A fascinated creature can take no actions. It remains in place, giving its total attention to some object, creature, or effect. It takes a \minus5 penalty to all checks made to observe anything other than the object of its fascination.
If the creature notices any threat against it, such as an approaching enemy, it is no longer fascinated.

\debuffdef{frightened} A frightened creature takes a \minus4 penalty to \glossterm{accuracy} and Mental defense while it is within \rngmed range of the source of its fear.
This does not stack with the \shaken or \panicked effects.
If the source of a frightened creature's fear is \glossterm{defeated}, this effect is broken.

\debuffdef{goaded} A goaded creature takes a \minus2 penalty to \glossterm{accuracy} against creatures other than the creature that goaded it it as long as it is within \rngmed range of of that creature.
If the goading creature is \glossterm{defeated}, this effect is broken.
If a creature is goaded by multiple different creatures simultaneously, it suffers the accuracy penalty on all of its attacks.

\debuffdef{grappled} A grappled creature is wrestling or in some other form of hand-to-hand struggle with at least one other creature.
While grappled, you suffer certain penalties and restrictions, as described below.
\begin{itemize}
    \item You are unable to use one of your hands for any purposes other than grappling.
        This prevents humanoid creatures from taking any actions which would require having two free hands, such as attacking with heavy weapons.
    \item You take a \minus2 penalty to Armor and Reflex defenses.
    \item Abilities that have \glossterm{somatic components} have a 25\% \glossterm{failure chance}.
    \item You cannot move unless you \glossterm{push} all creatures grappling you, such as with the \textit{shove} ability (see \pcref{Shove}).
\end{itemize}

Other than the restrictions listed above, you can act normally. You can also try to move the grapple, escape the grapple, or pin your opponent. For details, see \pcref{Grapple Actions}.

\debuffdef{helpless} A helpless creature is completely at an opponent's mercy.
It takes a \minus10 penalty to its Armor and Reflex defenses.
In addition, it is \unaware of all attacks against it, but the penalty for being unaware does not stack with the penalty for being helpless.
Paralyzed, bound, and unconscious creatures are helpless.

\debuffdef{immobilized} An immobilized creature takes a \minus4 penalty to Reflex defense and can't use any of its movement speeds.
Immobilized flying creatures that have the ability to hover can maintain their initial altitude.
% TODO: Fix flying interaction; safest way to descend for non-good maneuverability
All other flying creatures subjected to this condition descend at a rate of 20 feet per round until they reach the ground, taking no falling damage.
This does not stack with the \slowed or \decelerated effects.

\debuffdef{nauseated} A nauseated creature takes a \minus4 penalty to all defenses.
This does not stack with the \sickened effect.

\debuffdef{panicked} While a panicked creature is within \rngmed range of the source of its fear, it takes a \minus4 penalty to Mental defense and must flee from the source of its fear by any means necessary.
If unable to flee, it must do nothing other than use the \textit{total defense} ability every round (see \pcref{Total Defense}).
The penalty from this effect does not stack with the \frightened or \panicked effects.

If the source of a panicked creature's fear is \glossterm{defeated}, this effect is broken.

\debuffdef{paralyzed} A paralyzed creature is unable to take physical actions. It is \helpless, but can take purely mental actions. This can cause flying creatures to crash, swimming creatures to drown, and so on. Any creature can move through a space occupied by a paralyzed creature without slowing down, and creatures can stand in a square with a paralyzed creature without \squeezing.

\debuffdef{partially unaware} An creature that is partially unaware of an attack knows that it is in danger, but is missing information about the exact location or nature of the attack.
Creatures take a \minus2 penalty to Armor and Reflex defenses against attacks that they are partially unaware of.
For details, see \pcref{Awareness and Surprise}.

\debuffdef{prone} A prone creature is lying on the ground, rather than standing normally.
It takes a \minus2 penalty to Armor and Reflex defenses, though it gains a \plus4 bonus to all defenses against ranged \glossterm{strikes}.
In addition, it takes a \minus2 accuracy penalty.
It moves at a quarter of its normal speed until it stands up.
A creature can stand up from being prone during the movement phase.
This generally requires one free hand.

\debuffdef{shaken} A shaken creature takes a \minus2 penalty to \glossterm{accuracy} and Mental defense while it is within \rngmed range of the source of its fear.
This does not stack with the \frightened or \panicked effects.

If the source of a shaken creature's fear is \glossterm{defeated}, this effect is broken.

\debuffdef{sickened} A sickened creature takes a \minus2 penalty to all defenses.
This does not stack with the \nauseated effect.

\debuffdef{slowed} A slowed creature moves at half speed and takes a \minus2 penalty to Reflex defense.
This does not stack with the \decelerated or \immobilized effects.

\debuffdef{squeezing} A squeezing creature is trying to move though an area too small for it to fight in normally.
While squeezing, a creature moves at half speed and takes a \minus2 penalty to \glossterm{accuracy}, as well as Armor and Reflex defenses.
For details, see \pcref{Squeezing}.

\debuffdef{stunned} A stunned creature takes a \minus4 penalty to all defenses.
This does not stack with the \dazed or \confused effects.

\debuffdef{surrounded} An creature is surrounded if every space adjacent to it either contains an \glossterm{enemy} or is both empty and adjacent to an \glossterm{enemy}.
A surrounded creature suffers a \minus2 penalty to Armor and Reflex defenses.

\debuffdef{unaware} An creature that is unaware of an attack makes no attempt to defend itself.
Creatures take a \minus5 penalty to Armor and Reflex defenses against attacks that they are unaware of.
For details, see \pcref{Awareness and Surprise}.

\debuffdef{unconscious} While you are unconscious, you are \helpless and completely unable to take any actions.
Some sensory abilities, such as the Awareness and Spellsense skills, can be used while you are asleep, but not while you are forcibly knocked unconscious.

\newpage
\section{Ability Tags}\label{Ability Tags}

\abilitytagdef{Attune} Attune abilities require an \glossterm{attunement point} to maintain.
For details, see \pcref{Attunement}.

\abilitytagdef{Auditory} Auditory abilities use sound to cause their effects.
Creatures and objects that cannot hear the effect are immune to it.
However, they ignore any \glossterm{miss chance} from \glossterm{concealment} or being \glossterm{blinded}.

\abilitytagdef{Compulsion} Compulsion abilities forcibly alter a creature's actions, but do not necessarily affect its opinions or personality.
They have no effect on objects or creatures without minds.

\abilitytagdef{Creation} Creation abilities create permanent physical objects.
Objects created with Creation abilities are identical to objects created through more mundane means.

\abilitytagdef{Curse} Curse abilities lay supernatural curses on their targets.
They cannot be \glossterm{dismissed}, but can be removed with the \spell{dispel curse} spell.

\abilitytagdef{Detection}\label{Detection} Detection abilities reveal magical auras or information within an area.
They can penetrate up to 1 foot of stone, 1 inch of common metal, a thin sheet of lead, or 3 feet of wood or dirt.
For its ability to penetrate other materials, use the most similar substance from the list above.

\abilitytagdef{Emotion} Emotion abilities alter a creature's opinons or personality, but do not necessarily affect their actions.
They have no effect on objects or creatures without minds.

\abilitytagdef{Focus}\label{Focus} Focus abilities require concentration to use.
When you use a Focus ability, you take a \minus4 penalty to Armor and Reflex defenses until the end of the round.
This penalty is called your \glossterm{focus penalty}.
It applies immediately, affecting attacks made against you during the current phase.
Some abilities can reduce or increase your \glossterm{focus penalty}.
Most \glossterm{spells} are \abilitytag{Focus} abilities.

When you use a Focus ability during the \glossterm{delayed action phase}, the ability has a 50\% \glossterm{failure chance} if you already took damage in the current round.
If your \glossterm{focus penalty} is reduced to 0, this failure chance no longer applies.

If a \abilitytag{Focus} ability has a \abilitytag{Sustain} tag, sustaining the ability does not require focus and does not cause the sustaining creature to suffer a focus penalty.

\abilitytagdef{Healing} This tag indicates that an ability restores hit points.
When you use most Healing abilities, they \glossterm{briefly} prevent you from using any other Healing abilities.

\abilitytagdef{Magical} This tag indicates that an ability is \glossterm{magical}, which means that its origin derives from magic.
For details, see \pcref{Magical Abilities}.

\abilitytagdef{Manifestation} Manifestation abilities create temporary constructs formed from raw magical energy.
Objects and creatures created with manifestation abilities seem real on the surface, but they have no internal structure.
When an object or creature created by a Manifestation ability is destroyed or killed, or when the duration of the ability that created it ends, it disappears without a trace.

\abilitytagdef{Ritual} Ritual abilities have a number of shared properties.
For details, see \pcref{Spell and Ritual Mechanics}.

\abilitytagdef{Scrying} Scrying abilities create one or more invisible magical sensors that send you information.
Unless otherwise noted, the sensor created has the same powers of sensory acuity that you possess.
This includes the effect of any abilities which target you personally, such as spells to increase your visual acuity, but not abilities which affect an area around you.
However, the sensor is treated as a separate, independent sensory organ, and it functions normally even if you have been blinded, deafened, or otherwise suffered sensory impairment.
\par Any creature trained in Spellsense can notice the sensor by making a \glossterm{difficulty value} 20 Spellsense check.
The sensor can be dismissed as if it were an active spell.
You cannot create a sensor in a location with lead sheeting between you and the location, and you sense that the effect is blocked in this way.

\abilitytagdef{Sensation} Sensation abilities create or manipulate light, sound, or other sensations.
You can only create sensations you understand.
For example, you cannot create an illusory figment which speaks coherently in a language you do not understand.
\par If a Sensation ability appears to create a physical object or creature, its defenses are equal to 0.

\abilitytagdef{Speech} Speech abilities use words to achieve their ends.
You must specify a language when using a Speech effect, and the language must be one you know (or have memorized the correct words to say). They have no effect on objects or creatures that do not understand the chosen language.

\abilitytagdef{Spell} Spell abilities have a number of shared properties.
For details, see \pcref{Spell and Ritual Mechanics}.

\abilitytagdef{Subtle} Subtle abilities have no visual or otherwise perceivable manifestation.
Creatures successfully affected by Subtle abilities do not generally know that they are being influenced.
However, a creature that successfully resists a Subtle ability can generally notice that it resisted a special effect of some kind, just like a non-Subtle ability.
Subtle magical abilities can still be identified with the Spellsense skill (see \pcref{Spellsense}), but the \glossterm{difficulty value} is equal to 15 \add the ability's \glossterm{power}.

\abilitytagdef{Sustain} Sustain abilities require an action to maintain.
The tag includes an action type, such as (minor), which indicates the type of action required to sustain the ability.
For details, see \pcref{Sustained Abilities}.

\abilitytagdef{Swift} Swift abilities take effect before non-Swift abilities used during the same phase.
For details, see \pcref{Swift Abilities}.

\abilitytagdef{Visual} Visual abilities use visible objects or forces to cause their effects.
Creatures and objects that cannot see the effect are immune to it.
