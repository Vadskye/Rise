\chapter{Reference}\label{Reference}

\section{Ability Tags}\label{Ability Tags}

    \abilitytagdef{Attune} Attune abilities require an \glossterm{attunement point} to maintain.
    For details, see \pcref{Attuned Abilities}.

    \abilitytagdef{Auditory} Auditory abilities use sound to cause their effects.
    Creatures and objects that cannot hear the effect are immune to it.

    \abilitytagdef{Barrier} Barrier abilities create obstacles that block or punish passage through them.
    They create a powerful magical resonance around themselves that inhibits other barriers from being formed.
    % TODO: awkward wording
    Any part of a Barrier ability that would extend within 5 feet of an already active Barrier ability is not created, unless the new barrier simply extends an already existing wall along its length.

    \abilitytagdef{Brawling} Brawling abililities rely on brute force in hand to hand combat.
    Your accuracy with Brawling abilities is equal to half the sum of your level and Strength.
    Most Brawling abilities are described in \pcref{Special Combat Abilities}.

    \abilitytagdef{Compulsion} Compulsion abilities forcibly alter a creature's actions, but do not necessarily affect its opinions or personality.
    They have no effect on objects or creatures without minds.

    \abilitytagdef{Creation} Creation abilities create permanent physical objects.
    Objects created with Creation abilities are identical to objects created through more mundane means.

    \abilitytagdef{Curse} Curse abilities lay supernatural curses on their targets.
    They cannot be \glossterm{dismissed}, but can be removed with the \spell{dispel curse} spell.

    \abilitytagdef{Detection}\label{Detection} Detection abilities reveal magical auras or information within an area.
    They can penetrate up to 1 foot of stone, 1 inch of common metal, a thin sheet of lead, or 3 feet of wood or dirt.
    For its ability to penetrate other materials, use the most similar substance from the list above.

    \abilitytagdef{Emotion} Emotion abilities alter a creature's opinons or personality, but do not necessarily affect their actions.
    They have no effect on objects or creatures without minds.

    \abilitytagdef{Exertion} Exertion abilities increase your \glossterm{fatigue level}.
    Most Exertion abilities increase your fatigue when used, but some increase fatigue when they end or have some other timing, as indicated in the ability's description.
    If an Exertion ability increases your fatigue level when used, that fatigue is applied after you finish using the ability, so you do not suffer a \glossterm{fatigue penalty} from that extra fatigue while using the ability.

    \abilitytagdef{Manifestation} Manifestation abilities create temporary constructs formed from raw magical energy.
    Objects and creatures created with manifestation abilities seem real on the surface, but they have no internal structure.
    When an object or creature created by a Manifestation ability is destroyed or killed, or when the duration of the ability that created it ends, it disappears without a trace.

    \abilitytagdef{Ritual} Ritual abilities have a number of shared properties.
    For details, see \pcref{Spell and Ritual Mechanics}.

    \abilitytagdef{Scrying} Scrying abilities create one or more invisible magical sensors that send you information.
    Unless otherwise noted, the sensor created has the same powers of sensory acuity that you possess.
    This includes the effect of any abilities which target you personally, such as spells to increase your visual acuity, but not abilities which affect an area around you.
    However, the sensor is treated as a separate, independent sensory organ, and it functions normally even if you have been blinded, deafened, or otherwise suffered sensory impairment.
    \par The sensor can be dismissed as if it were an active spell.
    You cannot create a sensor in a location with lead sheeting between you and the location, and you sense that the effect is blocked in this way.

    \abilitytagdef{Size-Based} Size-Based abilities are limited based on your own size.
    They have no effect on creatures or objects that are two or more size categories larger than you.
    You can mitigate this limitation with the \ability{creature climb} ability (see \pcref{Creature Climb}), or simply by flying close (see \pcref{Flight}).

    \abilitytagdef{Speech} Speech abilities use words to achieve their ends.
    You must specify a language when using a Speech effect, and the language must be one you know (or have memorized the correct words to say). They have no effect on objects or creatures that do not understand the chosen language.

    \abilitytagdef{Spell} Spell abilities have a number of shared properties.
    For details, see \pcref{Spell and Ritual Mechanics}.

    \abilitytagdef{Subtle} Subtle abilities have no visual or otherwise perceivable manifestation.
    Creatures successfully affected by Subtle abilities do not generally know that they are being influenced.
    However, a creature that successfully resists a Subtle ability can generally notice that it resisted a special effect of some kind, just like a non-Subtle ability.
    You can notice the effects of a Subtle ability on yourself with the Awareness skill (see \pcref{Notice Subtle Effect}).

    \abilitytagdef{Sustain} Sustain abilities require an action to maintain.
    The tag includes an action type, such as (minor), which indicates the type of action required to sustain the ability.
    If it also includes ``attuneable'', you can choose to \glossterm{attune} to the effect instead of sustaining it every round.
    For details, see \pcref{Sustained Abilities}.

    \abilitytagdef{Swift} Swift abilities take effect before non-Swift abilities used during the same phase.
    For details, see \pcref{Swift Abilities}.

    \abilitytagdef{Trap} Trap abilities create triggered effects that punish trespassing.
    They create a powerful magical resonance around themselves that inhibits other barriers from being formed.
    Any part of a Trap ability that would extend within 15 feet of an already active Trap ability is not created.

    \abilitytagdef{Visual} Visual abilities use visible objects or forces to cause their effects.
    Creatures and objects that cannot see the effect are immune to it.
    Special vision abilities that replace normal vision entirely, such as \trait{blindsight} and \trait{tremorsense}, are unable to perceive Visual effects.
    A creature that exclusively perceives its surroundings without normal sight, such as by closing its eyes and relying on its blindsight, is unaffected.

\newpage
\section{Circumstances and Debuffs}\label{Circumstances and Debuffs}

    \debuffdef{blinded} A blinded creature cannot see.
    It is at least \partiallyunaware of everything, it may be fully \unaware as normal depending on its non-visual understanding of its surroundings.
    In addition, it automatically fails at actions which depend on vision, including simply seeing the locations of other objects and creatures.

    \debuffdef{charmed} A charmed creature is mentally influenced to like another creature.
    It always sees the words and actions of the creature that charmed it in the most favorable way, as a close friend or trusted ally.
    A charmed creature cannot be controlled like an automaton, but can be persuaded to take particular actions with the Persuasion skill (see \pcref{Persuasion}).
    It treats the creature that charmed it as a friend (a \plus10 relationship modifier) for the purpose of Persuasion checks.

    Any act by the charming creature or by creatures that appear to be its allies that threatens or harms the charmed person breaks the effect.
    Harming a charmed creature is not limited to dealing it damage, but also includes causing it significant subjective discomfort.
    An observant creature may interpret overt threats to its allies as a threat to itself.

    \debuffdef{climbing} A creature that is climbing without a \glossterm{climb speed} takes a \minus2 penalty to its \glossterm{accuracy} and Armor and Reflex defenses.

    \debuffdef{confused} A confused creature takes a \minus2 penalty to all defenses and is unable to independently control its actions.
    This penalty does not stack with the \stunned effect.
    When a creature becomes confused, and at the beginning of each round, it randomly decides to have one of two behaviors that round: attack its \glossterm{enemies}, or protect itself and its \glossterm{allies} without attacking.
    Within those constraints, it can freely choose its actions.
    If it can't carry out the indicated action, it does nothing but babble incoherently.

    % Kind of an awkward hack
    A confused creature automatically stops being confused after ten minutes, even if it was unable to rest due to its confusion.

    \debuffdef{dazzled} A dazzled creature has difficulty seeing.
    It loses the benefits of the \trait{darkvision} and \trait{low-light vision} abilities if it has them.
    In addition, it treats everything as if it had \glossterm{concealment}.
    Among other effects, this gives its \glossterm{targeted} attacks a 20\% \glossterm{miss chance}.

    \debuffdef{deafened} A deafened creature cannot hear. It automatically fails at actions which depend on hearing. In addition, it has a 20\% failure chance when casting any spell with verbal components.

    \debuffdef{dominated}[dominate] A dominated creature is mentally compelled to obey another creature.
    It obeys the commands of the creature of the dominated it unquestioningly, as an automaton.
    If it does not understand the language of the creature that dominated it, it still attempts to obey as much as possible, and simple commmands (such as ``attack'' or ``follow'') can usually be communicated successfully.
    A creature that is both dominated and \confused obeys its dominated orders, ignoring the confusion.

    \debuffdef{enraged} An enraged creature must spend a \glossterm{standard action} to make an attack during each round.
    It can still take other actions normally.
    The creature's attacks do not have to specifically target other creatures, so it can attack inanimate objects.
    If it is unable to take standard actions, such as if it is unconscious, it suffers no extra penalty.

    % Kind of an awkward hack
    An enraged creature automatically stops being enraged after ten minutes, even if it was unable to rest due to its rage.

    \debuffdef{flying} A creature that is flying takes a \minus2 penalty to its Armor and Reflex defenses.
    If it has a poor \glossterm{maneuverability}, this penalty increases to \minus4 (see \pcref{Maneuverability}).

    \debuffdef{frightened} A frightened creature takes a \minus2 penalty to its Mental defense.
    In addition, it takes a \minus2 penalty to \glossterm{accuracy} against the source of its fear.
    This does not stack with the \panicked effect.
    If the source of a frightened creature's fear is \glossterm{defeated}, this effect is broken.
    Being frightened is always an \abilitytag{Emotion} effect, even if it is caused by an ability that does not have that tag.

    \debuffdef{goaded} A goaded creature takes a \minus2 penalty to \glossterm{accuracy} against creatures other than the creature that goaded it it as long as it is within \rngmed range of of that creature.
    If the goading creature is \glossterm{defeated}, this effect is broken.
    If a creature is goaded by multiple different creatures simultaneously, it suffers the accuracy penalty on all of its attacks.
    Being goaded is always an \abilitytag{Emotion} effect, even if it is caused by an ability that does not have that tag.

    \debuffdef{grappled} A grappled creature is wrestling or in some other form of hand-to-hand struggle with at least one other creature.
    While grappled, you suffer certain penalties and restrictions, as described below.
    \begin{itemize}
        \item One of your hands cannot be used for any purposes.
            This prevents humanoid creatures from taking any actions which would require having two free hands, such as attacking with \weapontag{Heavy} weapons.
            This does not affect creatures without hands.
        \item You take a \minus2 penalty to Armor and Reflex defenses.
        \item You cannot move unless you \glossterm{push} all creatures grappling you, such as with the \ability{shove} ability (see \pcref{Shove}).
            In addition, you cannot \glossterm{push} a creature grappling with you so it stops being adjacent to you.
            However, you can use the \ability{shove} ability to affect creatures you are grappling with as a \glossterm{movement} instead of as a standard action.
        \item You can use the \ability{escape grapple} and \ability{maintain grapple} abilities to stop or continue grappling (see \pcref{Special Combat Abilities}).
    \end{itemize}

    \debuffdef{helpless} A helpless creature is completely at an opponent's mercy.
    It is considered to be \unaware of all attacks against it, even if it knows they are coming.
    Paralyzed, bound, and unconscious creatures are helpless.

    \debuffdef{immobilized} An immobilized creature takes a \minus4 penalty to its Armor and Reflex defenses and can't use any of its movement speeds.
    Immobilized flying creatures that have the ability to hover can maintain their initial altitude.
    % TODO: Fix flying interaction; safest way to descend for non-good maneuverability
    All other flying creatures subjected to this condition descend at a rate of 20 feet per round until they reach the ground, taking no falling damage.
    This does not stack with the \slowed effect.

    \debuffdef{panicked} A panicked creature takes a \minus4 penalty to its Mental defense.
    In addition, it is unable to make any attacks that include the source of its fear as a target.
    The penalty from this effect does not stack with the \frightened or \panicked effects.
    If the source of a panicked creature's fear is \glossterm{defeated}, this effect is broken.
    Being panicked is always an \abilitytag{Emotion} effect, even if it is caused by an ability that does not have that tag.

    \debuffdef{paralyzed} A paralyzed creature is unable to take physical actions. It is \helpless, but can take purely mental actions. This can cause flying creatures to crash, swimming creatures to drown, and so on. Any creature can move through a space occupied by a paralyzed creature without slowing down, and creatures can stand in a square with a paralyzed creature without \squeezing.

    \debuffdef{partially unaware} An creature that is partially unaware knows that something is nearby, but is missing information about the exact location or nature of the creature, object, or attack it is partially unaware of.
    Creatures take a \minus2 penalty to Armor and Reflex defenses against attacks that they are partially unaware of.
    They have a 50\% miss chance with \glossterm{targeted} atacks against creatures and objects that they are partially unaware of, and they can only attempt to target creatures and objects that they know the location of.

    These penalties do not stack with the penalties for being \unaware.
    For details, see \pcref{Awareness and Surprise}.

    \debuffdef{prone} A prone creature is lying on the ground, rather than standing normally.
    It takes a \minus2 penalty to Armor and Reflex defenses.
    However, it gains a \plus4 bonus to all defenses against ranged \glossterm{strikes} as long as the attacker is not adjacent.
    It moves at half of its normal speed, and is considered one size category smaller than normal when determining whether it is subject to \abilitytag{Size-Based} effects, including critical hits.
    Creatures that are not on the ground, such as flying or gliding creatures, are immune to being knocked prone.

    If a creature becomes prone while in a precarious situation, such as on a narrow ledge, it may fall.
    Mounted creatures that are knocked prone fall off their mounts.
    Creatures cannot glide or fly while prone.

    A creature can stand up from being prone as part of a \glossterm{movement} using one of their move speeds.
    This generally requires one \glossterm{free hand}.
    Standing up from a prone position costs half of the creature's speed during that movement.

    \debuffdef{slowed} A slowed creature moves at half speed and takes a \minus2 penalty to its Armor and Reflex defenses.
    This does not stack with the \immobilized effect.

    \debuffdef{squeezing} A squeezing creature is trying to move though an area too small for it to fight in normally.
    While squeezing, a creature moves at half speed and takes a \minus2 penalty to its Armor and Reflex defenses.
    For details, see \pcref{Squeezing}.

    \debuffdef{stunned} A stunned creature takes a \minus2 penalty to all defenses.
    This does not stack with the \confused effect.

    \debuffdef{swimming} A creature that is swimming without a \glossterm{swim speed} takes a \minus2 penalty to its \glossterm{accuracy} and Armor and Reflex defenses.

    \debuffdef{unaware} An creature that is unaware makes no attempt to defend itself.
    Creatures take a \minus6 penalty to Armor and Reflex defenses against attacks that they are unaware of.
    They are completely unable to use \glossterm{targeted} abilities against creatures and objects that they are unaware of.

    These penalties do not stack with the penalties for being \partiallyunaware.
    For details, see \pcref{Awareness and Surprise}.

    \debuffdef{unconscious} While you are unconscious, you are \helpless and completely unable to take any actions.
    Some sensory abilities, such as the Awareness skill, can be used while you are asleep, but not while you are forcibly knocked unconscious.

    \debuffdef{underwater} Ranged weapons have difficulty working underwater.
    All ranged weapons have \glossterm{range limits} of 5/15 when used by a creature that is underwater, or when used against a target that is underwater, regardless of the attack's normal range limits or any other modifiers.

\newpage
\section{Traits}\label{Traits}
    % TODO: should there be subgroups within this section, or should it be strictly alphabetical?
    % A possible subgroup would be "sensory traits".

    \traitdef{Blindsense}{blindsense}
        A creature with blindsense can sense the location of everything in its surroundings.
        It does not need to use its eyes to gain this benefit.
        This ability works regardless of concealment, invisibility, or light levels.
        Blindsense always has a specific range limit, and provides no benefit beyond that range.
        However, it only grants knowledge of location, not actual sight, so does not mitigate any \glossterm{miss chances} that would apply.
        It also does not mitigate \glossterm{cover} or otherwise allow sensing through objects that block \glossterm{line of effect}.

    \traitdef{Blindsight}{blindsight}
        A creature with blindsight can perceive its surroundings perfectly regardless of concealment, invisibility, or light levels.
        It does not need to use its eyes to gain this benefit.
        This allows the creature to ignore all \glossterm{miss chances} caused by those effects.
        Blindsight always has a specific range limit, and provides no benefit beyond that range.
        It also does not mitigate \glossterm{cover} or otherwise allow sensing through objects that block \glossterm{line of effect}.

    \traitdef{Darkvision}{darkvision}
        A creature with darkvision can see perfectly in both complete darkness and \glossterm{shadowy illumination} just like a human does in \glossterm{bright illumination}.
        Darkvision always has a specific range limit, and provides no benefit beyond that range.
        As long as a creature with darkvision is in \glossterm{bright illumination} or \glossterm{brilliant illumination}, their darkvision stops working.
        The darkvision \glossterm{briefly} stays disabled even after they leave the lit area.
        Darkvision is disabled while you are \dazzled.

    \traitdef{Impervious}{impervious}
        A creature can be impervious to a particular damage type, ability tag, or other property of an attack.
        Creatures gain a \plus4 bonus to all defenses against attacks that they are impervious to.
        In addition, they take no damage if the attack misses, even if the attack would normally deal half damage on a miss.
        If an attack deals damage of multiple types, a creature is impervious to that attack only if it is impervious to all of the attack's damage types.
        For attacks with random effects, such as the \spell{chromatic orb} spell, determine the random effect before determining if the creature is impervious.

    \traitdef{Immune}{immune}
        A creature that is immune to an attack is completely unaffected by it.
        Creatures and objects can be immune to specific damage types or debuffs.
        It is also possible to be immune to more esoteric concepts, like being \grappled or gaining \glossterm{conditions}.

        Being immune to part of an attack does not grant immunity to other aspects of that attack.
        If an attack deals damage of multiple types, a creature is immune to that attack only if it is impervious to all of the attack's damage types.
        This also applies to more specific immunities that are not related to damage types.
        For example, if you are immune to being \stunned, you still take full damage from an attack that deals damage and stuns you.

    \traitdef{Incorporeal}{incorporeal}
        An incorporeal creature does not have a tangible body.
        It is \trait{immune} to \glossterm{physical damage} and is never considered to be \squeezing.
        It moves silently and ignores the effects of abilities that only work if it has a corporeal body, such as \glossterm{difficult terrain} and the \textit{grapple} or \textit{shove} abilities.
        This includes being \grappled, detected by \trait{tremorsense}, setting off pressure plates, and so on.

        Many incorporeal creatures have no Strength attribute.
        If an incorporeal creature has a Strength attribute, it has some ability to manipulate the physical world despite being incorporeal.
        Unless otherwise noted, an incorporeal creature with a Strength attribute may selectively choose whether it wants to interact with physical objects.

        An incorporeal creature can enter or pass through solid objects, but it must remain adjacent to the object's exterior at all times.
        If it is completely inside an object, it cannot see out or attack.
        It can fight while partially inside an object, which grants it \glossterm{cover} and allows it to attack and see normallly.

    \traitdef{Invisible}{invisible}
        An invisible creature or object cannot be seen with light.
        Creatures unable to see an invisible creature are at least \partiallyunaware of its attacks, and they can be fully \unaware as normal depending on their level of awareness.
        Attackers suffer a 50\% \glossterm{miss chance} with \glossterm{targeted} attacks even if they know the location of the invisible creature.
        See \pcref{Awareness}, and \pcref{Stealth}, for how to identify invisible creatures.

    % This name is awkward, so creatures aren't specifically called out as being 
    \traitdef{Legless}{legless}
        A legless creature has legs.
        Legless creatures are immune to being \prone.

    \traitdef{Lifesense}{lifesense}
        Lifesense functions like \trait{blindsense}, except that it only grants knowledge of the location of living creatures.

    \traitdef{Lifesight}{lifesight}
        Lifesight functions like \trait{blindsight}, except that it can only see living creatures.

    \traitdef{Low-light Vision}{low-light vision}
        A creature with low-light vision can see perfectly in \glossterm{shadowy illumination}, just like a human does in \glossterm{bright illumination}.
        This provides no benefit in areas of complete darkness.
        Low-light vision is disabled while you are \dazzled.

    \traitdef{Mindless}{mindless}
        A mindless creature lacks a mind.
        It does not have Intelligence or Willpower attributes, and has no Mental defense.
        Any attacks against it that would normally use its Mental defense use its Fortitude defense instead.
        It uses its Strength to determine its \glossterm{magical power} instead of its Willpower.
        It has no soul, so if it dies, it cannot be resurrected.
        Mindless creatures are immune to psychic damage and \abilitytag{Compulsion} and \abilitytag{Emotion} abilities.

    \traitdef{Multipedal}{multipedal}
        A multipedal creature uses three or more legs to move.
        Bidepal creatures, like humans, are not multipedal.

        Multipedal creatures gain a \plus10 foot bonus to their \glossterm{land speed} and a \plus5 bonus to the Balance skill.

    \traitdef{Scent}{scent}
        A creature with the scent ability has an unusually good sense of smell.
        It reduces the \glossterm{difficulty value} of scent-based Awareness checks by 10 (see \pcref{Awareness}).

    \traitdef{Simple-Minded}{simple-minded}
        A simple-minded creature has an incompletely functioning mind.
        It can follow simple instructions, but is not fully \glossterm{sentient} or capable of complex reasoning.
        It has no soul, so if it dies, it cannot be resurrected.
        Simple-minded creatures are immune to psychic damage and \abilitytag{Emotion} abilities.
        However, they are \vulnerable to \abilitytag{Compulsion} attacks.

    \traitdef{Telepathy}{telepathy}
        A creature with telepathy has the ability to mentally communicate with other nearby creatures.
        All telepathy abilities have a defined \glossterm{range}.
        Unless otherwise specified, a telepathic creature can only communicate with one creature at a time.

        As a \glossterm{free action}, a telepathic creature can open a telepathic communication channel with one creature it sees within the range of its telepathy ability.
        The target does not have to be willing to receive telepathic communication in this way.
        While this channel is open, the telepathic creature can cause the target to ``hear'' the telepathic creature's voice inside the target's head.
        If the target attempts to mentally reply while the channel is open, the telepathic creature can similarly ``hear'' the reply in its head as if the target was speaking.
        This does not generally grant the ability to detect any other thoughts, though exceptionally stupid targets may accidentally broadcast their private thoughts.

        Telepathic communication uses words, so it still requires a shared language to be intelligible, even though the words are only imagined.
        A telepathic creature may attempt to telepathically communicate with creatures without a language, though this is generally unproductive.
        A skilled telepath can customize the mental ``voice'' it projects in the same way that a creature can attempt to disguise or alter its voice when speaking.

    \traitdef{Tremorsense}{tremorsense}
        Tremorsense functions like \trait{blindsense}, except that it requires an uninterrupted path through solid objects instead of \glossterm{line of effect}.
        This makes it incapable of sensing flying creatures, but it ignores \glossterm{cover} and can even sense through solid obstacles that are no more than half a foot thick.

    \traitdef{Tremorsight}{tremorsight}
        Tremorsense functions like \trait{blindsight}, except that it requires an uninterrupted path through solid objects instead of \glossterm{line of effect}.
        This makes it incapable of seeing flying creatures, but it ignores \glossterm{cover} and can even see through solid obstacles that are no more than half a foot thick.

    \traitdef{Vulnerable}{vulnerable}
        A creature can be vulnerable to a particular damage type or ability tag.
        It takes a \minus4 penalty to all defenses against attacks that would cause it to take damage of that type, or from abilities with that tag.
        For attacks with random effects, such as the \ability{sudden entropy} ability from the Entropist feat, determine the random effect before determining if the creature is vulnerable.

    \traitdef{Undead}{undead}
        Undead are creatures that are made from the corpse or spirit of a dead creature.
        They are animated by some part of the soul of the original creature.

        Although undead are not \glossterm{living}, they are affected in unusual ways by effects that directly manipulate life energy.
        They can be targeted as if they were living \glossterm{allies} by \magical effects that would cause living creatures to regain \glossterm{hit points}.
        Whenever they would regain \glossterm{hit points} from an ability that normally only affects living creatures, they instead take energy damage equal to the hit points that they would have regained, ignoring any hit point maximum the ability would normally have.
        If the ability requires an attack roll, such as an attack vs. Reflex to touch the target, they are \vulnerable against that attack, and a \glossterm{critical hit} doubles the damage they take as normal for damaging attacks.

        Any other effects beyond simple hit point recovery are ignored.
        For example, if a cleric uses their \ability{divine aid} ability to heal an undead creature, the undead would take damage, but it would not gain any bonus to its defenses.

        % TODO: should undead heal from vivimancy effects? Is it worth introducing a new damage type just for that?
        % Similarly, undead are always considered living creatures for the purpose of effects that would normally deal energy damage to living creatures.
        % Whenever they would take energy damage from an ability that normally only affects living creatures, they instead regain hit points equal to the energy damage that they would have taken. 
        % This ability does not have the \abilitytag{Swift} tag, so it resolves after incoming attacks during the current phase.
        % Any other effects beyond simple energy damage are ignored.
