\chapter{The Expanded Universe}

\section{Expanding the Cosmology}
  Rise normally assumes there is only one Material Plane containing a single solar system.
  As a GM, you can decide that there are multiple independent Material Planes.
  You can also add multiple solar systems to a single Material Plane if that suits the needs of your campaign.
  Technically, that should be unstable since the solar systems should be pulled towards each other by gravity over time.
  You can handwave it by saying that magic keeps the systems apart.
  Alternatively, you can simply trust that your players will not scrutinize the physics of that universe too closely.

\section{Creatures and Objects}
  In the world of Rise, creatures and objects are meaningfully different.
  Many abilities only affect creatures or objects.
  The difference between a creature and an object is agency.
  Creatures have agency, and objects do not.
  This is not the same as sentience or life, which either creatures or objects may have.

  For example, skeletons are non-sapient, nonliving creatures.
  Conversely, trees are non-sapient, living objects.
  Some rare magic items can be made intelligent by magic, making them sapient, nonliving objects.
  Some unintelligent animals and magical beasts like ants and giant spiders are non-sapient, living creatures.

  There are two types of creatures that are also objects in unusual ways: \trait{constructs} and \trait{indwelt}.

\section{The Four Elements of Existence}

  The four elements that define existence in Rise are body, life, mind, and soul.
  Body and mind are fundamental elements, with simple and obvious effects.
  Life and soul are energetic elements, with more subtle and cosmic effects.

  \subsection{Physical Elements: Body and Life}
    Something has a body if its existence is physical.
    If something has a body, it is called corporeal.
    This is usually straightforward: trees, humans, and rocks all have bodies.
    Not everything with a body can be easily touched or seen.
    Clouds and gases are still corporeal, though they are \trait{intangible}.

    Life is the ability of the body to change and adapt over time.
    If something has life, it is called alive.
    Humans and trees are alive, but rocks are lifeless, and corpses are dead.
    Living things are constantly changing, and require input and output from their environment to maintain equilibrium.
    They almost always breathe, eat, sleep, and perform similar body maintenance activities.

    Lifeless things do not need to perform those tasks, and their bodies typically persist unchanged without outside intervention.
    Some lifeless things exist in a state of gradual decay instead of permanent stasis.
    In either case, their bodies are defined by inert consistency rather than change and equilibrium.

  \subsection{Mental Elements: Mind and Soul}
    Something has a mind if it can understand aspects of its environment and react to it.
    If something has a mind, it is called intelligent.
    This is defined broadly, and not every intelligent creature is sentient or self-aware.
    Humans and other animals are intelligent, but trees and rocks are mindless.

    A mind is a separate entity from a body.
    When an intelligent living creature dies, its mind can persist after the body's destruction.
    The brain is a tool that anchors and connects a mind to a body, not the fundamental mechanism that creates the mind.
    Damage to a brain can inhibit the connection between a mind and a body, diminishing a creature's functional intelligence, but it does not directly damage or alter the mind.

    Soul is the ability of the mind to change and adapt over time.
    If something has a soul, it is called ensouled.
    Humans and other animals are ensouled, but artificial constructs like golems are soulless.
    Ensouled things are able to change their manner of thinking and personality over time.
    Some soulless entities can learn and retain information, and can be highly intelligent, but they cannot change their method of thought or fundamental opinions.
    If you lock a human and an intelligent artificial construct in a room for twenty years, the human's mind would emerge fundamentally changed and possibly insane, while the construct would simply have gained information that it had been in a room for twenty years.

  \subsection{Energetic Elements: Life and Soul}
    Life cannot exist independently of a body, and soul cannot exist independently of a mind.
    They are elements defined by change and adaptation.
    This may explain why they have such intrinsic power in Rise.
    Life and soul are the underlying power behind most superhuman effects in Rise.

  \subsection{Examples}
    \begin{raggeditemize}
      \item Some creatures are created when creatures die, but their minds refuse to pass on to the appropriate afterlife. These creatures have neither body nor life. They have a mind, and may or may not have a soul. This includes allips, ghosts, and wraiths.
      \item Some creatures are formed from planar essence to act as an embodiment of the plane's identity. These creatures have a body, but no life. They have a mind, but no soul. This includes angels, demons, and elementals.
      \item Some creatures are magically granted a semblance of life by animating an inanimate object. These creatures have a body, but no life. They have a mind, but no soul. This includes golems, skeletons, and zombies.
      \item Some creatures are able to move independently, but their responses to their surroundings are entirely instinctual, without any ability to form thoughts or make decisions. These creatures have a body and are alive. They have neither mind nor soul. This includes oozes, plants, and extremely simple animals like ants.
      \item When creatures die and their minds travel to the appropriate afterlife, they gain new bodies formed from the planar essence of their new home. These creatures have a body, but no life. They have both a mind and a soul.
    \end{raggeditemize}

\sectiongraphic*{Secrets of the Universe}{width=\columnwidth}{the universe/secrets of the universe}

  There are many mysteries in the universe of Rise.
  This section gives a glimpse into some of the underlying truths, though few characters in the universe would understand such details.

  \subsection{Power Ultimately Derives From Life and Soul}
    When most creatures are born, they enter existence with life and a new soul.
    These energetic elements have great intrinsic power.

    The more powerful the life, the more the body can change.
    The body of an elephant is much stronger than that of an ant.
    However, elephants and ants have similarly weak life energy, because the ability of their body to change and adapt to circumstances is limited.
    Humans can have greater life energy, as their bodies can undergo significant changes to adapt to their training and circumstances.
    Dragons represent the pinnacle of immense life energy.
    Their bodies undergo vast changes in power and shape over their lives.
    Similarly, the more powerful the soul, the more the mind can change.

    Mechanically, life energy justifies a creature gaining levels that increase its physical abilities, and soul energy justifies a creature gaining levels that increase its mental abilities.
    Life energy is often associated with mundane abilities, and soul energy is often associated with magical abilities.
    That relationship is not strictly followed, since some mundane abilities are mental and some magical abilities are physical.

    \subsubsection{Variable Intrinsic Power}
      Not all lives and souls are equal in power.
      Most humanoid creatures and magical creatures have only a moderate amount of life and soul energy.
      They can train, fight, and learn, and this will increase their power.
      However, they will eventually reach limits that they cannot surpass.
      Some extraordinary individuals seem to be nearly limitless.
      They can change their minds and bodies to extreme degrees, acquiring vast power in the process.

      Mechanically, this is represented by creatures gaining levels, but only up to a certain point.
      Player characters can gain levels without limit (unless the GM defines a level cap for their specific campaign).
      They also often level up much faster than other characters in the same universe who have similar experiences.
      If everyone in the universe of Rise acquired limitless power at the same rate as player characters, the world would be filled with superhuman demigods and country-destroying monsters.
      The fundamental limits of life and soul provide a narrative justification for a more traditional fantasy universe.
      They allow Rise to have grizzled old war veterans and mass-murdering monsters who are still only level 5 or so.

    \subsubsection{Transferring Power: Death and Sublimation}
      The intrinsic power of life and soul can be transferred.
      Life cannot exist without a body, and soul cannot exist without a mind.
      When a creature dies, its life and soul are shattered and vulnerable in the moments after death.
      With no body to anchor it, the life energy sublimates into pure energy.
      A strong mind can retain control of its soul, and travel intact to the appropriate afterlife.
      However, weak minds quickly break without being anchored to a body, allowing their souls to sublimate as well.

      If creatures present at the death have a life or soul that is strong enough, they can consume part of this energy to enact changes in their own bodies or minds.
      This is a common method of power acquisition for adventurers and monsters, who often slay powerful foes.
      Certain rituals can also be used to feed on the powers of death more effectively.
      This can be used to benefit the ritual participants, or by demons and evil deities who feed on deaths offered to them in ritual sacrifices by their cultists.

      Transferring a power through death is deeply inefficient.
      Under normal circumstances, only a fraction of the soul energy released in this way can be claimed by the killer.
      Some energy infuses the area, which can give rise to magical phenomena at areas of mass death or the death of particularly powerful entities.
      The rest goes to an appropriate afterlife or is claimed by Gaia.

    \subsubsection{Transferring Power: Connection}
      A soul's power can be transferred without the inefficiency of death.
      Commonly, it is simply freely given through love and emotional connection in the form of soul motes.
      Creatures who love each other naturally share small portions of their souls with each other.
      Over time, deeply connected creatures, such as old married couples, can mix their souls so fully that they become virtually indistinguishable.

      Voluntary soul sharing does not have to be perfectly symmetric.
      Tyrants can earn soul motes through the enforced fear and subservience that they create in their underlings.
      Worship is another method of transferring soul motes, and many deities fundamentally derive power from the combined soul motes willingly given by their legions of worshippers.
      In exchange, deities can use their power to protect their worshippers, either through divinely empowered clerics or more rarely through direct intervention.
      More mundanely, adventurers who save a town from a dire threat may earn soul motes freely granted from the gratitude of its inhabitants.

    \subsubsection{Soul Motes and Splinters}
      Souls can be subdivided into lesser pieces.
      There are two forms of lesser soul pieces: motes and splinters.

      Soul motes are emitted from souls unconsciously, like light is emitted from a torch.
      It is possible for a soul that emits a large number of soul motes to diminish if it does not receive any in exchange.
      For example, a minor underling who pledges their life to an uncaring leader might give away far more soul motes than they receive in exchange.
      Most people have enough interpersonal relationships to avoid this danger, but completely isolated people who are neither loved nor hated, but simply ignored, may diminish in this fashion.
      Even with this risk, the process of emitting soul motes is not harmful or individually significant in any way.
      In addition, individual soul motes are far too small to be manipulated or used by magical effects.

      Soul splinters are created in a much more dramatic fashion.
      When a soul undergoes significant trauma that shakes its will and sense of self, it may splinter, losing a chunk of its soul.
      Of course, death is one of the greatest traumas of all, and almost all souls splinter to some degree when they die.

      Soul splinters can be consumed or manipulated in a variety of ways.
      For example, skeletons and zombies are animated by splintering a soul that originally inhabited a corpse.
      The splinter is used to give the corpse a crude imitation of sentience - just enough to obey orders, but not enough to think for itself.

    \subsubsection{Potential and Acquired Power}
      The strength of a creature's life and soul determine the limits of its ability to progress.
      Many monsters reach the limit of their potential by the time they reach adulthood, and are unable to develop further.
      However, many humanoid creatures never discover their true limits, because they have never had the necessary experiences to develop their potential.
      A well-trained soldier will easily defeat a commoner in battle, but this does not mean that the soldier's life or soul is necessarily stronger.
      It simply means they have progressed farther towards their potential.

      In a typical campaign setting, it's reasonable to assume that 20\% of the humanoid population can reach 5th level, 2\% can reach 10th level, and 0.2\% can reach 15th level.
      Of course, most people don't have the life experiences necessary to reach their maximum potential.
      All player characters are assumed to have exceptional potential, and are able to reach 21st level, unless the GM says otherwise.
      Legendary monsters of epic proportions may even be able to surpass that limit.

    \subsubsection{Mysteries of the Soul}
      The mysteries of differing soul strength have no clear and consistent explanation.
      In broad terms, the strength of a creature's soul usually correlates to its emotional and intellectual potential, as well as its force of will.
      Humanoid creatures and dragons are unusually mentally capable - not just in raw intelligence, but also in empathy, determination, and capacity for belief - and correspondingly have unusually strong souls.
      There are individual exceptions that suggest that this is not the entire dimension of what causes strong and weak souls.
      Some animals have unusually strong souls for no known reason, causing them to develop over time into their ``dire'' variants.
      Dire animals, who have gained levels by feeding on soul splinters, are more aggressively malevolent than normal animals, though they show no greater signs of general intelligence.
      Perhaps there is simply an element of randomness in the creation of each new soul.

      The fundamental mysteries of souls and their sharing are not widely known in the universe of Rise.
      Individual elements of this truth are widely known, such as the observation that people can become stronger by slaying monsters, but monsters do not seem to grow dramatically in power by killing people.
      Strange phenomena can occur where death occurred, and old battlegrounds are often haunted by naturally occurring undead.
      Learned scholars may understand that the civilized species like humans seem to have unusually strong souls, and that this is related to their capacity for drastic personal growth.
      They may identify the general phenomena surrounding soul splinters, but not soul motes.

      Some powerful and unusual entities, such as deities and demon princes, know particular elements of how life energy and soul energy can be transferred.
      All demons are generally aware that they can feed on sublimating soul splinters from souls in evil afterlife planes as they break down over time, though they do not understand the exact mechanics of this transfer.
      They attempt to torment weaker souls to accelerate this breakdown, and avoid souls that are too strong to break.
      However, they are unaware of the subtler aspects of soul sharing, such as willing soul mote transfer between loved ones.
      Powerful deities know more about souls than any other entities as a result of being worshipped and maintaining the existence of their personal afterlife planes.
      In exceptionally rare occasions they may deem it appropriate to share that knowledge if it serves their purposes.

  \subsectiongraphic*{Gaia}{width=\columnwidth}{the universe/nature}
    Gaia is the embodiment of life.
    Although some people worship her, she is not a deity, and does not grant power in that way.
    While deities draw power from souls, Gaia draws power from life and death.

    Every time a living thing dies on the Material Plane, Gaia claims a portion of its life force.
    This includes creatures with tiny souls and plants with no souls at all.
    This amounts to trillions of deaths every day sustaining her, and she is unambiguously the strongest entity in Rise.
    She resides in a demiplane in the Astral Expanse, and is one of the few entities that even precursors fear to challenge.

    Gaia lacks a coherent anthropomorphic representation, and her will is almost never brought to bear in any organized way.
    The use of female pronouns to describe Gaia is more of a social convention than a reflection of her true nature.
    Druids are granted power by Gaia, but they are not required to agree to any particular ideology, and their usage of that power is virtually never policed or revoked by Gaia herself in the way that an errant cleric might be punished by their deity.
    Gaia welcomes a diversity of viewpoints, for nature is almost infinitely diverse.
    She possesses a wealth of power, so she does not need to jealously hoard her gifts like deities must.
    % Can this actually happen?
    % The only druids who have had their powers revoked were a rare few who turned their powers to the explicit and intentional destruction of Gaia itself.

    People who worship nature do not have any special territory in an afterlife reserved for them, since Gaia exists separately from the Spiritual Planes and has no relationship to souls.
    The Spiritual Planes are where Gaia's power is weakest, and she can claim no tithe of any deaths there, since the planes themselves absorb any energy released.
    Instead, devoted worshippers of Gaia may have their souls reincarnated instead of going to a normal afterlife.
    This gift is not granted to all worshippers, and indeed many would prefer to go to a normal afterlife.

  \subsection{Soul-Fuelled Phenomena}

    The peculiar nature of souls causes a wide variety of strange and unique effects in the Rise universe.

    \subsubsection{Deities}
      Deities are among the most well-known phenomena that are fundamentally created by the energy of souls.
      When hordes of living creatures pay homage to the same entity, that entity can feed on that outpouring of worship and become incredibly powerful if it has a strong enough soul.
      The history of Rise is full of minor deities and demigods who either lack a sufficient base of worshippers to become a true deity or who lack a strong enough soul to effectively use the worship they receive.

      Not every powerful entity with an immensely powerful soul is a deity.
      Deities are, by definition, worshipped.
      They depend on receiving energy through soul motes granted by worship.
      Since all deities share this constraint, they are all vulnerable to anything which would disrupt this flow of power.
      This encourages deities to find common ground with each other to ensure that they all succeed, even when their own ideologies and personal beliefs are violently opposed.
      As a result, deities have developed a shared body of elaborate conventions and constraints that govern their interventions in mortal realms.
      Any deity that violates these rules risks being declared anathema by all other deities, which would devastate their ability to acquire worshippers.

      The most consistent constraint that deities operate under is that their primary method of intervention in mortal affairs must come through their followers, rather than independently.
      They freely share a portion of their power with their most dedicated followers, granting them extraordinary abilities.
      In most societies, these empowered worshippers are called clerics.

      A deity with enough power can claim territory within the afterlife plane associated with its alignment.
      The claimed territory becomes a Divine Realm under the deity's control.
      Deities have extraordinary power within their Divine Realm, and can reshape it as they see fit.
      However, they must expend a significant amount of energy to maintain their territory against entropy.
      As a result, deities are always hungry to gain additional followers, and only successful deities expend the effort to claim any territory at all.

      Any souls that worship a deity will be reborn within that deity's Divine Realm in the appropriate afterlife plane, even if that plane does not match their personal alignment.
      This is both a reward for worshippers and a way for deities to accumulate energy.
      When a soul in an afterlife eventually loses the will to maintain its individual existence, its mind breaks and the soul sublimates into energy.
      Normally, this energy becomes part of the afterlife plane.
      However, if this sublimation happens within a deity's Divine Realm, they can claim that power for themselves.
      This allows deities to eventually reclaim the energy they invested in their clerics.
      For deities with the power to maintain a Divine Realm, the energy they gain in this way significantly exceeds the energy they gain from ordinary soul motes.
      However, since this often requires centuries before breaking even, new and minor deities are unable to maintain their own Divine Realms.

    \subsubsection{Pact Magic}
      Entities of great power can make pacts with mortals.
      In these pacts, the mortals offer their soul to the entity for a period of time after death, making the entity their soulkeeper.
      In exchange, the soulkeeper grants the mortal energy from its own supply.
      The soulkeeper's goal is to have the mortal accumulate a substantial amount of its own energy in its life, and then to break the will of the soul while it is in the soulkeeper's clutches.
      If the soulkeeper succeeds, it gains the ability to harvest the energy released by the mortal's entire soul, similar to how deities claim the energy of decaying souls  within their Divine Realms.
      This is a vast wealth of energy compared to the normal splinters and motes gained from death and worship.
      However, the process completely consumes the mortal's soul.

      Successful soulkeepers can therefore amass great power.
      However, it is a risky business, much like adventuring is for mortals.
      If the mortal resists the soulkeeper's torments during its time in the afterlife, it may take its entire soul intact to its normal afterlife.
      When this happens, the soulkeeper loses the bounty of the soul, all of the energy it originally invested in the mortal, and time it wasted trying to break the mortal's spirit.
      This is particularly likely if the mortal dies soon after making the pact, so soulkeepers must choose their mortal partners wisely.

      Failing to break a mortal's spirit is not the worst thing that can happen to a soulkeeper.
      It may attract attention from more powerful entities within its own plane.
      When a soulkeeper is killed, ownership of the soul is transferred to whatever killed it.
      This means that soulkeepers with active contracts - especially active contracts with mortals who are nearing death after a long life - are extremely attractive targets for anyone who wants to steal the reward of the soul.

      Demons are the most common soulkeepers.
      They are more likely than any other type of creature to meet the four main prerequisites for offering soul pacts.
      First, they have sufficient raw energy to make soul pacts.
      Second, they have enough understanding of magic and souls to transfer power through the pact.
      Third, they have the patience to wait until the mortal dies to claim their reward.
      Fourth, they have the ambition and risk tolerance to take the gamble of being a soulkeeper and risk losing the energy they invest.

      There is nothing that prevents a deity from becoming a soulkeeper.
      On very rare occasions, deities may make a pact and become a soulkeeper for a non-worshipper.
      Mortals that gain power in this way are called favored souls.
      However, being a soulkeeper is risky.
      Few deities would risk the possibility of losing their energy entirely when they could instead use that energy to more safely empower a cleric.
      In addition, being known for making soul pacts can discourage people from voluntarily worshipping the deity.

    % TODO: clarify how much of this is life vs soul
    \subsubsection{Ambient Magic and Magical Creatures}
      The world of Rise is full of strange creatures that have superhuman strength or magical abilities, like minotaurs and manticores.
      It is common knowledge that such creatures are typically found only in distant wilderness or in deep dungeons.
      In general, the farther you get from civilization, the more powerful the monsters in the area become, and the more likely you are to encounter strange magical phenomena.
      Small towns seem to cause a subtle warding effect, and powerful monsters in the area will typically avoid them.
      Even monsters that lack the intellectual capacity to understand complex causation chains like ``if I attack the town, they may send powerful warriors to hunt me down'' will typically avoid interacting with civilization unless necessary.

      All of this can be explained by the release of life energy when things die.
      The constant cycle of life and death in nature produces a great wealth of energy.
      Most of the life energy is claimed by Gaia, but some lingers at the location of each death.
      This energy can build up over time in the form of ambient magic.
      Many monsters can instinctively feed on this ambient energy.
      In addition, predatory monsters can feed on the energy released by their own kills.
      This naturally allows them to build their power to near the limit of their potential by the time they are adults.

      Civilization disrupts the natural cycles of life and death, reducing the soul energy present in an area.
      Although humanoid creatures have powerful souls, they die less frequently, and the vast majority of the soul energy of their death moves with them to their afterlife.
      From the perspective of creatures that feed on ambient magic, civilized areas stand out as a dead zone.

      Since educated people in the universe of Rise can observe that monsters tend to avoid civilization if they study the phenomenon, they may have their own theories about why this is true.
      Reasonable theories that might have truth to them in some contexts could include ``monsters have evolved to instinctively avoid civilization to avoid death from monster hunters'', ``druids magically discourage monsters from entering civilization so they don't get killed'', or ``monsters have to kill other strong monsters to get stronger, so they try to avoid areas that don't have any powerful prey''.
