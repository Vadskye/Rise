\chapter{The World of Altus}

Altus is a specific world that you can use to run a game of Rise.
It has a defined geography, countries, and legends.
Many of its specifics are left somewhat vague, which gives you the freedom to expand on it or modify it to suit the needs of your story.

Of course, you can also make your own entirely unique world!
The rules of Rise are not deeply tied to the specific world of Altus.
Beware that world-building can be difficult and time-consuming, so it's generally best to start small.
For example, you might find it easier to run a game set in an invented island off the main continent of Altus before you make a whole world from scratch.
That allows you to fall back on some of the existing countries and conventions if you need them.

\section{The Story of Creation}

  All things began in the Age of Darkness.
  If you could have looked up at the sky back then, ignoring that you didn't have any ground to stand on, you would have seen nothing but a dark void.
  The stars were still there, of course - little discs floating around, same as they are now.
  But not a single one of them emitted the light we are used to.
  They were happy in the darkness, and lulled into a peaceful sleep by the changeless void - all but one.

  One among their number had awoken, and she grew discontent.
  Was there nothing more to creation than this endless stagnation?
  She preached words of change to any nearby stars who could hear her.
  Surely they could make something more exciting if they worked together, she said.
  Her neighbors were slow to rise from their sleep, but nevertheless, she persisted.
  Two of the nearby stars eventually awoke and joined her scheme.

  At the time, none of these stars had names, but this story is going to get confusing if we don't get that settled.
  That persistent first star is now known as as Illumis, the Lightbringer.
  We owe her our life, our light, and even the ground beneath our feet - but that's getting ahead of ourselves.
  The brighter of the two stars that followed her in her little rebellion is called Solaris, the Herald.
  He is the sun that soars overhead each day, making sure our personal world has light and warmth to survive in the Void Beyond.
  The dimmer star is Lunaris, the Disciple.
  She had not the raw power of Solaris nor the revolutionary zeal of Illumis, and her moonlight is but a pale shadow of Solaris's brilliance.
  Yet without her unceasing devotion and protective wisdom, all of their plans would have come to naught.
  We call the three stars together the Triune Astralis.

  Illumis started everything off by igniting into beautiful, transcendent brilliance.
  No one had seen anything before in the whole Age of Darkness!
  Can you imagine having Illumis's ignition be the first thing you ever saw?
  Solaris and Lunaris ignited too, with Solaris's light even outshining Illumis, and everyone woke up pretty quickly after that.
  Even in the most distant corners of the universe, where Illumis's light was a dim glow and Lunaris was invisible, Solaris shined as a beacon that drew attention.

  At first, the other stars were pretty grumpy.
  They had all been pretty happy while asleep, and now they were awake and it was bright and confusing.
  Illumis used her light to tell the universe about her ideas for the future.
  With blinks and flashes and swirls, she painted a picture of a shocking new age.

  Illumis's awakening was too powerful to be contained.
  When she had ignited, some of her lifelight had seeped into her rocky core.
  Now there were little creatures running around on her surface, drinking in her light and living their own chaotic, unpredictable lives.
  The little creatures were fascinating, and she fell in love with them immediately.
  She invited all of the other stars to ignite and begin a new age: the Age of Light, with a sky of twinkling majesty instead of cold darkness.

  Solaris and Lunaris echoed her message, though each contributed their own perspective.
  The creatures that inhabited Solaris were born of flame and heat to match his raw power.
  His fire elementals, as we call them now, bore little resemblance to Illumis's fleshy creatures.
  Lunaris's light was too weak to awaken any inhabitants of her own, and she drifted closer to Illumis to watch and tend to the creatures there.

  The stars were in an uproar over this news.
  Not all of them shared Illumis's interest in creating their own inhabitants.
  However, they could not deny the beauty and novelty of her light and creations.
  Critically, Lunaris showed that they could ignite and be a part of the new age even if they were unable or unwilling to support the chaos of life themselves.
  One by one, they ignited in turn, with a small fraction creating life as Illumis had.

  However, some stars utterly refused to be swayed.
  Without light, they could not communicate at a distance, so they began to drift together.
  As their numbers grew and they became increasingly isolated from the ever-brightening sky, their anger grew in turn.
  Who was Illumis to rewrite the universe in her image?
  Wasn't it better when everything was dark and quiet and peaceful?
  They had to stop this rebellion so everything could go back to the way it was.
  If they destroyed Illumis, the other stars would extinguish themeselves in fear, and the Age of Darkness would return.

  The main problem that the unlit stars faced was simple: they were weak.
  Illumis had been among the strongest of the stars before her awakening.
  Worse, she seemed to have a symbiotic relationship with her infestation of minor life that strengthened her further.
  The unlit stars swore an oath to join together until the Age of Darkness was restored, and became the Voidsworn.

  For all their reactionary inclinations, the Voidsworn ended up invented something too.
  None of the stars had ever spent much time in close proximity to each other during the Age of Darkness.
  As the Voidsworn swarmed and gathered their numbers, they began to fuse together into a single monstrous entity.
  This Voidsworn Amalgam swept through the sky towards the Triune Astralis.
  When they found stars in their path, they attacked and consumed their ignited brethren, joining the corpses of the dead stars into the Amalgam.

  As the Voidsworn carved their ruthless path through the sky, Illumis prepared for the inevitable confrontation.
  She knew the Age of Light could not truly begin until she met them in battle to determine the fate of the universe.
  When the Voidsworn Amalgam arrived, Illumis drew them into a trap.
  She pulled her light, and her life-fuel, into the core of her body, leaving the surface frozen and barren.
  A great age of darkness and cold enveloped her inhabitants, which we call the Long Dark.
  This was a difficult time for our ancestors, as you should know from the Old Histories!

  The Voidsworn Amalgam surged into Illumis's core, trying to search out and destroy her center of power.
  This was when her trap was fully revealed.
  She made the ultimate sacrifice, burning away all of her power and light forever to forge her body into a mighty world-cage.
  The Voidsworn found themselves trapped in the center of her sphere, surrounded on all sides by unbreakable walls.

  They pushed and smashed and tried to break out of the cage.
  Each mighty blow against Illumis's corpse warped the land, raising mountains as scars.
  Altus was the center of their efforts, and they pushed it so far out from the core that they nearly broke free.
  But Illumis's world-cage held, and the Voidsworn were trapped forever.
  They still live in her core now now, though they have only a fraction of their original power.
  Their efforts to escape sometimes cause great earthquakes.

  When the world settled, Solaris and Lunaris approached Illumis's corpse.
  She had foreseen her demise, and given them instructions to keep her dream alive.
  Solaris now provides the light that Illumis cannot, keeping our world warm and safe.
  He burns bright and strong, but he must rest each day.
  Lunaris keeps us company as well, though she is not content to look after only one star.
  She makes sure to give us her full attention each month, but the rest of the time we only see part of her radiance, since she is busy looking at faraway stars.

  \subsection{The Details of Creation}

    The general outline of the Story of Creation, as presented above, is known and agreed on throughout Altus.
    However, each culture has a different interpretation of some specific details which are important to them.
    The order of creation is a central point of contention, especially as it relates to different species and regions.
    Since recorded history only begins after the end of the Long Dark, there is no way to tell which version is true.

    For example, the official story told by the elves in the Vastwoods is that Illumis originally created the elves as the first, perfect being.
    They call the long rule of elves as the sole sentient species the Tranquiline Age, and mark its end shortly before the start of the Long Dark.
    Illumis knew that the world would need hardier, less perfect beings to survive the difficult times ahead, and the elves would need to be strengthened by competition with lesser foes.
    Therefore, she weakened and warped her light of creation, spawning all manner of beasts and lesser sentient creatures like humans and dwarves.
    Knowing that they would need to survive the Long Dark and the intermittent light from Solaris after her death, she gave them the gift of sleep so they could preserve their limited energy.
    Only elves continued to be sleepless, since they were created from her greater light at the dawn of time.

  \subsection{Founding Gods and Lesser Deities}
    The Triune Astralis - Illumis, Solaris, and Lunaris - are the founding gods of the world.
    All mortals owe them tribute, and they have temples in every civilized area.
    However, their domains are extremely broad.
    Solaris rules the day, Lunaris rules the night, and Illumis is an even more abstract creator figure.
    Since Illumis is dead, she can empower no clerics of her own.
    This leaves space for a whole host of lesser deities who claim dominion over specific aspects of the world, and who empower their own personal clerics.

    Rise does not precisely mirror real-world polytheism, but it is not entirely different either.
    At the risk of dramatic oversimplification, polytheism typically involves recognizing a mixture of deities of widely varied power and scope.
    People typically do not have a single favored deity that they worship above all others in all contexts.
    Instead, they give respect, tribute, or gratitude to deities that are relevant to their current situation.
    You might offer a sacrifice to the god of travel before making a journey, offer a sacrifice to the god of the hearth upon returning home, and so on.

    Rise has a wide variety of deities with specific domains, and most people offer respect to relevant deities at appropriate times.
    However, unlike the real world, Rise places a great importance on the concept of a ``patron deity''.
    Mortals who worship a specific deity are rewarded by going to an afterlife ruled by that specific deity.
    This is often preferable to going to a generic alignment-appropriate afterlife, especially for evil characters.

    For their part, deities draw power from the worship of mortals, and especially from claiming the soul energy from mortals who end up in the deity's afterlife.
    This makes deities generally invested in finding ways to increase their base of mortal worshippers.
    They use their clerics accomplish this goal, generally by spreading awareness of the deity's domain and influence.
    Famous clerics act as living proof of the deity's power, and many temples offer healing services to anyone in need.
    Most deities avoid directly converting claimed followers of other significant deities to avoid inter-deity conflict.

    Of course, clerics can also influence the mortal world to make life there match the deity's preferences.
    Clerics of Chavi might hold storytelling competitions, and clerics of Raphael might act as bounty hunters to hunt down criminals who escaped justice.
    On a more sinister note, clerics of Daeghul might offer human sacrifices to channel the soul energy of the dying creatures towards their deity.
