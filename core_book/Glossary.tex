\chapter{Glossary}\label{Glossary}

\glossdef{ability}[abilities] An ability is a generic term for any special action a creature can perform or effect that a creature or object can cause.
Spells, feats, and the benefits from class \glossterm{archetypes} can all be called abilities.

\glossdef{ability tag}[ability tags] An ability tag describes the effects of an ability.
For details, see \pcref{Ability Tags}.

\glossdef{acid}[Acid] Acid damage is very effective against most objects. For the Acid ability tag, see \pcref{Ability Tags}.

\glossdef{action phase} The action phase is the second of two \glossterm{phases} in a combat \glossterm{round}.
During the action phase, creatures can \glossterm{attack}, cast \glossterm{spells}, and take other major combat actions.

\glossdef{action point}[action points] Action points allow you to perform special actions that you have access to.
For details, see \pcref{Action Points}.

\glossdef{active cover} Active cover is a type of \glossterm{cover} that can block \glossterm{physical attacks}.
It represent mobile obstacles that move unpredictably, imposing a 20\% \glossterm{miss chance} on physical attacks.
For details, see \pcref{Active Cover}.

\glossdef{accuracy} The bonus added to a \glossterm{attack roll}.

\glossdef{Air} See \pcref{Ability Tags}.

\glossdef{alignment}[alignments] Your alignment represents your general morality in broad terms.
For details, see \pcref{Alignment}.

\glossdef{ally}[allies] Many abilities affect allies.
An ally is any creature you designate who is willing to be considered an ally, not including yourself.

\glossdef{arcane spell failure} Wearing \glossterm{armor} makes it more difficult to cast \glossterm{arcane} spells.
For details, see Arcane Spell Failure, page \pref{Mge:Arcane Spell Failure}.

\glossdef{archetype}[archetypes] An archetype is a collection of related abilities from a particular class.
Each class has three archetypes that members of that class normally have.
For details, see \pcref{Archetypes}.

\glossdef{armor} Armor is a form of equipment that protects your body from harm.
There are two kinds of armor: \glossterm{body armor}, which you wear on your body, and \glossterm{shields}, which you wield in a hand.
For details, see \pcref{Armor}.

\glossdef{attack}[attacks] Anything that affects another creature in a potentially harmful way. There are two kinds of attacks: \glossterm{physical attacks} and \glossterm{magical} attacks.

\glossdef{attack result} An attack result is the total you get on an \glossterm{attack roll}, after taking to account any bonuses or penalties that apply to the roll.

\glossdef{attack roll}[attack rolls] A roll required to succeed with an attack.
To make an attack roll, roll 1d10 \add your \glossterm{accuracy} with the attack.
If the result of the attack roll equals or exceeds the target's \glossterm{defense}, the attack succeeds.
Some attacks, especially magical attacks, have effects even if the attack roll fails.

\glossdef{attended} An attended item is an item currently being held or carried by a creature.
Some abilities can only affect \glossterm{unattended} items.

\glossdef{attribute}[attributes] A core representation of a character's capacity in a wide range of areas. There are six attributes: \glossterm{Strength}, \glossterm{Dexterity}, \glossterm{Constitution}, \glossterm{Intelligence}, \glossterm{Perception}, and \glossterm{Willpower}.

\glossdef{attune}[attunement] Some abilities last as long as you attune to them.
As long as you are attuned to an ability, you cannot regain the \glossterm{action point} spent to use that ability.
For details, see \pcref{Attunement}.

\glossdef{Attune} An ability with this \glossterm{ability tag} lasts as long as you attune to it.
If the tag includes (shared), then it lasts as long as you and all targets attune to it.
For details, see \pcref{Attunement}.

\glossdef{attuned} If you are attuned to an ability, you have invested an action point in it to prolong its effect.
For details, see \pcref{Attunement}.

\glossdef{Auditory} See \pcref{Ability Tags}.

\glossdef{augment}[augments] Many spells have augments.
Each augment on a spell has a level and an effect.
When casting a spell, you add the augment's level to the spell's level.
If you do, the spell gains the effect of the augment.
You can apply any number of augments to a spell in this way, increasing the spell's level for each augment.

\glossdef{Barrier} See \pcref{Ability Tags}.

\glossdef{base bonus} A base bonus is the value of a bonus before any temporary modifiers or effects. For example, your base Fortitude defense bonus is equal to the Fortitude defense bonus granted by your class.

\glossdef{blinded} A blinded creature cannot see. It automatically fails at actions which depend on vision, including simply seeing the locations of other objects and creatures (but see \pcref{Awareness}). It has a 50\% miss chance with \glossterm{strikes} and vision-related checks, even if it knows the location of its target. Finally, it is \defenseless.

\glossdef{blindsense} A creature with blindsense can sense its surroundings without any light, regardless of concealment or invisiblity.
It knows the location of everything around it, but it still takes normal \glossterm{miss chances} for concealment, invisibility, and so on.
It still needs line of effect to see its surroundings.
Blindsense always has a range, and grants no benefits beyond that range.

\glossdef{blindsight} A creature with blindsight can ``see'' its surroundings perfectly without any light, regardless of concealment or invisibility.
It still needs line of effect to see its surroundings.
Blindsight always has a range, and grants no benefits beyond that range.

\glossdef{block}[blocking] The \textit{block} ability allows you to try to prevent other creatures from entering an area.
For details, see \pcref{Block}.

\glossdef{bloodied} At or below half hit points. Bloodied creatures take a \minus4 penalty to Fortitude and Mental defense.

\glossdef{body armor} Body armor is a form of \glossterm{armor} that you wear on your body.
For details, see \pcref{Armor}.

\glossdef{broken} A broken object is damaged and unsuitable for use, though it retains its general structure and can be repaired.
For details, see \pcref{Broken Objects}.

An object that reaches 0 hit points is broken. If an object takes additional damage equal to its maximum hit points, it is \glossterm{destroyed}. A destroyed object cannot be repaired by any means.

\glossdef{burst} A burst is a type of area that an ability can have (see \pcref{Area Types}).
A burst ability has an immediate effect on all valid targets within an area.

\glossdef{cantrip}[cantrips] Every \glossterm{spell} can be cast as a cantrip.
A cantrip is a weaker version of the spell that does not cost an \glossterm{action point} to use.
For details, see \pcref{Cantrips}.

\glossdef{Chaos} See \pcref{Ability Tags}.

\glossdef{charge}[charging] Charging is a combat action that consists of running directly at a foe to attack it.
It is described at \pcref{Charge}.

\glossdef{Charm} See \pcref{Ability Tags}.

\glossdef{character level} Your character level is your total level, including levels from all of your classes.
Whenever text refers to your ``level'', without specifying a particular kind of level, it means your character level.

\glossdef{charmed} A charmed creature is mentally influenced to like another creature.
It always sees the words and actions of the creature that charmed it in the most favorable way, as a close friend or trusted ally.
A charmed creature cannot be controlled like an automaton, but can be persuaded to take particular actions with the Persuasion skill (see \pcref{Persuasion}).
It treats the creature that charmed it as a friend (a \plus10 relationship modifier) for the purpose of Persuasion checks.

\glossdef{check}[checks] A check is a d10 roll required to accomplish an action that has a chance of failure that is not an attack.
If the result of your roll, including your \glossterm{check modifier}, is high enough, you succeed.
Otherwise, you fail.
For details, see \pcref{Checks}.

\glossdef{check modifier}[check modifiers] A check modifier is a number that you add to or subtract from your d10 roll when you make a \glossterm{check}.
For details, see \pcref{Checks}.

\glossdef{class}[classes] Your class represents your fundamental source of power and the type of abilities you have.
For example, barbarians draw power from the primal energy found deep within all living things, while clerics draw power from their worship of mighty deities.
For details, see \pcref{Classes}.

\glossdef{class skill}[class skills] A class skill is a skill that a class is particularly good at using.
Each class has a specific set of class skills given in its description.
Normally, it costs 3 \glossterm{skill points} to make a skill \glossterm{mastered}.
It only costs 2 skill points to make a class skill \glossterm{mastered}.
For details, see \pcref{Skill Training}.

\glossdef{climb speed} A creature with a climb speed can climb as easily as a human walks on land.
The effects of a climb speed are described at \pcref{Climb Speed}.

\glossdef{cold}[Cold] A kind of \glossterm{energy}. For the Cold ability tag, see \pcref{Ability Tags}.

\glossdef{combat maneuver}[combat maneuvers] A combat maneuver is an attempt to physically hinder your foe with your body, such as by shoving or tripping it.
For details, see \pcref{Combat Maneuvers}.

\glossdef{common language}[common languages] Common languages are languages that are widely spoken.
They are described in \tref{Common Languages}.

\glossdef{Compulsion}[Compulsions] See \pcref{Ability Tags}.

\glossdef{concealment} Concealment represents effects which make a target harder to see, such as shadowy lighting.
You take a \minus4 penalty to accuracy with \glossterm{physical attacks} against creatures and objects that have concealment from you.

\glossdef{condition}[conditions] A condition is a negative effect on a creature.
Conditions last until they are removed, such as by the \textit{recover} ability (see \pcref{Recover}).

\glossdef{confused} A confused creature is unable to independently control its actions. \confusionexplanation

\glossdef{cover} Cover represents any obstacle that physically prevents you from striking your target, such as a tree or intervening creature.
There are three kinds of cover: \glossterm{active cover}, \glossterm{passive cover}, and \glossterm{total cover}.
For details, see \pcref{Cover}.

\glossdef{Creation} See \pcref{Ability Tags}.

\glossdef{crouching} A crouching creature is ducking down instead of standing normally.
\glossterm{Melee attacks} against it gain a \plus2 bonus to \glossterm{accuracy}, while physical ranged attacks against it take a \minus2 penalty to accuracy.
In addition, it takes a \minus2 penalty to accuracy with melee attacks and moves at half speed.

\glossdef{Curse} See \pcref{Ability Tags}.

\glossdef{critical failure} When you make a check, if your result failed to beat the DR by 10 or more, you get a critical failure.
Some abilities have special effects on critical failures.

\glossdef{critical hit}[critical hits] When you make an attack, if your result beat the target's defense by 10 or more, you get a critical hit.
Unless otherwise noted, damaging attacks deal double damage on a critical hit.
Some abilities have special effects on critical hits.

\glossdef{critical success} When you make a check, if your result beat the DR by 10 or more, you get a critical success.
Some abilities have special effects on critical successes.

\glossdef{damage} Some attacks deal damage to you when they hit.
When you take damage, you reduce your \glossterm{hit points} by that amount.
If you have no hit points remaining, you may take that damage as \glossterm{vital damage} instead, which represents potentially life-threatening injuries.
For details, see \pcref{Damage}.

\glossdef{damage reduction} Damage reduction allows you to ignore a certain amount of incoming damage.
Each \glossterm{round}, you ignore the first points of damage you would take.
Damage reduction always specifies an amount of damage it reduces.
Once it reduces that much damage, it stops functioning until the start of the next round.

Most sources of damage reduction only apply against a specific type of attack.
For example, a barbarian's damage reduction only applies against damage dealt by \glossterm{physical attacks}.
If an attack deals multiple types of damage, you must have damage reduction against every type of damage dealt.
For example, damage reduction against piercing damage would not help if you are struck by a morningstar, since it deals both bludgeoning and piercing damage.

\glossdef{darkvision} A creature with darkvision can see in the dark clearly up to a given range.
Beyond that, it can see dimly, treating areas of darkness as \glossterm{shadowy illumination}.
Darkvision does not function if a creature is in a brightly lit area, and does not resume functioning until 1 round after the creature leaves the brightly lit area.

\glossdef{dazed} A dazed creature cannot act during the \glossterm{action phase}.
In addition, it takes a \minus2 penalty to \glossterm{accuracy}, \glossterm{checks}, and \glossterm{defenses}.
It can take its normal actions during the \glossterm{delayed action phase}.

A creature that normally takes separate actions during the action phase and the delayed action phase instead takes all of its actions during the delayed action phase.

\glossdef{dazzled} A dazzled creature has difficulty seeing.
It loses any special vision abilities it has, such as \glossterm{darkvision} or \glossterm{low-light vision}.
In addition, it takes a \minus2 penalty to \glossterm{accuracy} with \glossterm{physical attacks} and vision-related checks.

\glossdef{dead} A dead creature's soul leaves its body. Dead creatures cannot benefit from normal or magical healing, but they can be restored to life via magic (see \pcref{Resurrecting the Dead}). A dead body decays normally unless magically preserved.

\glossdef{deafened} A deafened creature cannot hear. It automatically fails at actions which depend on hearing. In addition, it has a 20\% failure chance when casting any spell with verbal components.

\glossdef{defenseless} A defenseless creature is unable to defend itself in melee combat.
\glossterm{Melee attacks} against it gain a \plus2 bonus to accuracy.
Any creature not capable of using a weapon or shield to defend itself, such as most unarmed creatures, is defenseless.

\glossdef{Death} See \pcref{Ability Tags}.

\glossdef{defeated} An enemy is defeated if it dies, surrenders or is incapacitated for an extended period of time (such as by being knocked unconscious).
Some abilities, such as a ranger's \textit{quarry} ability (see \pcref{Quarry}), last until an enemy is defeated.
If there is ambiguity about whether a surrendering or seemingly incapacitated enemy still poses a threat, you choose whether you consider the enemy to be defeated.

\glossdef{defense}[defenses] A defense is a static number which represents how difficult you are to affect with attacks. See \glossterm{attack rolls}.

\glossdef{delayed action phase} The delayed action phase is a \glossterm{phase} that occurs after the \glossterm{action phase}.
It is not always necessary, because most actions are not delayed.
For details, see \pcref{The Delayed Action Phase}.

\glossdef{destroyed} A destroyed object has been damaged to the point where it is completely beyond repair.
For details, see \pcref{Destroyed Objects}.

\glossdef{Detection} See \pcref{Ability Tags}.

\glossdef{die increment}[die increments] A die increment is a single increase or decrease of the die size of a pool of dice.
For example, a 1d8 that is increased by one die increment becomes a 1d10 die.
Similarly, a 2d6 dice pool that is decreased by one die increment also becomes a 1d10 die.
For details, see \pcref{Die Increments}.

\glossdef{difficult terrain} Difficult terrain costs double the normal movement cost to move out of.
For details, see \pcref{Difficult Terrain}.

\glossdef{difficulty rating} The difficulty rating of a \glossterm{check} is the check result required to succeed.
In general, attacks are rolled to beat \glossterm{defenses}, and checks are rolled to beat difficulty ratings.

\glossdef{DR} A shorthand for \glossterm{difficulty rating}.

\glossdef{dirty trick} A dirty trick is a light \glossterm{combat maneuver} that allows you to impair a foe with your environment.
For details, see \pcref{Dirty Trick}.

\glossdef{disarm} A disarm is a \glossterm{combat maneuver} that allows you to strike items held or worn by a creature.
For details, see \pcref{Disarm}.

\glossdef{disease} An affliction of the body, causing a steady deterioration over time. For the Disease ability tag, see \pcref{Ability Tags}.

\glossdef{dismiss}[dismissed] When you dismiss an ability, it ends, and all of its lingering effects are removed.
Unless otherwise noted, all abilities with a \glossterm{duration} can be dismissed.

\glossdef{disoriented} During each movement phase, a disoriented creature is compelled to move its full speed in a random direction.
It moves as far as it can, but will not sprint or take similar strenuous actions to increase its speed.

\glossdef{dominated} A dominated creature is mentally compelled to obey another creature.
It obeys the commands of the creature of the dominated it unquestioningly, as an automaton.
If it does not understand the language of the creature that dominated it, it still attempts to obey as much as possible, and simple commmands (such as ``attack'' or ``follow'') can usually be communicated successfully.

% Does this need more description?
\glossdef{duration} An ability's duration determines how long that ability lasts.

\glossdef{dying} A dying creature is unconscious and near death. See \pcref{Dying}.

\glossdef{Earth} See \pcref{Ability Tags}.

\glossdef{effect} The result of using an \glossterm{ability}.

\glossdef{electricity}[Electricity] A kind of \glossterm{energy}. For the Electricity ability tag, see \pcref{Ability Tags}.

\glossdef{emanation} An emanation is a type of area that an ability can have (see \pcref{Area Types}).
An emanation ability has effects within an area for the \glossterm{duration} of the ability.
It emanates from a specific creature or object, rather than a location.
If that creature or object moves, the emanation moves with it.

\glossdef{Emotion} See \pcref{Ability Tags}.

\glossdef{encumbrance} Your encumbrance is a value that represents how much you are burdened by armor and weight.
For details, see \pcref{Encumbrance}.

\glossdef{energy}[energy damage] There are four types of energy: cold, electricity, fire, and sonic. Energy effects often deal damage.

% \glossdef{Enchantment} Enchantment spells affect the minds of others, influencing or controlling their behavior or mental capabilities. Almost all enchantment spells are \glossterm{Mind} spells, and many of them are \glossterm{Subtle} as well.

\glossdef{exhausted} An exhausted creature moves at half speed, cannot sprint (see \pcref{Sprint}), and takes a \minus2 penalty to \glossterm{accuracy}, checks, and defenses.

\glossdef{exotic weapon}[exotic weapons] A rare few weapons are considered exotic weapons.
They are unusually difficult to wield, and even being \glossterm{proficient} with the associated \glossterm{weapon group} does not grant you the ability to use an exotic ewapon.
To use an exotic weapon, you must take the Martial Training feat (see \featpref{Martial Training}).

\glossdef{falling damage} For every 10 feet you fall, you take 1d6 bludgeoning damage, to a maximum of 20d6 damage.
If you control your fall with a successful Acrobatics or Jump check, you can reduce the falling damage you take (see \pcref{Acrobatics}, and \pcref{Jump}).

\glossdef{fascinated} A fascinated creature can take no actions. It remains in place, giving its total attention to some object, creature, or effect. It takes a \minus4 penalty to skill checks made as reactions, such as Awareness checks. If the creature notices any threat against it, such as an approaching enemy, it is no longer fascinated.

\glossdef{fast healing} A creature with fast healing automatically heals hit points at the end of every round.
Like other healing, this healing offsets damage taken during the round for the purposes of taking \glossterm{vital damage} and becoming \wounded.

\glossdef{fatigued} A fatigued creature moves at half speed and cannot sprint (see \pcref{Sprint}).

\glossdef{Fear} See \pcref{Ability Tags}.

\glossdef{feint} A feint is a \glossterm{combat maneuver} that allows you to trick a creature into lowering its defenses.
For details, see \pcref{Feint}.

\glossdef{fire}[Fire] A kind of \glossterm{energy}. For the Fire ability tag, see \pcref{Ability Tags}.

\glossdef{Figment} See \pcref{Ability Tags}.

\glossdef{Flesh} See \pcref{Ability Tags}.

\glossdef{fly speed} A creature with a fly speed has the ability to fly through the air.
Its speed is the distance it covers in a single \glossterm{move action}.
For details, see \pcref{Flying}.

\glossdef{Fog} See \pcref{Ability Tags}.

\glossdef{follow} The \textit{follow} ability allows you to follow another creature to match their movements during the \glossterm{movement phase}.
For details, see \pcref{Follow}.

\glossdef{Force} See \pcref{Ability Tags}.

\glossdef{free action}[free actions] Each round, you can any number of free actions.
Free actions can be taken in any phase.
For details, see \pcref{Free Actions}.

\glossdef{frightened} A frightened creature takes a \minus4 penalty to \glossterm{accuracy}, checks, and defenses as long as it is within \rngmed range of the source of its fear.
These penalties do not stack with the penalties for being \shaken or \panicked.

If the source of a frightened creature's fear is a creature and is \glossterm{defeated}, this effect is broken.

\glossdef{Glamer} See \pcref{Ability Tags}.

\glossdef{glide speed} A creature with a glide speed can glide through the air.
It cannot fly upwards, but it can travel forward while it descends, and it descends at a significantly reduced rate.
For details, see \pcref{Gliding}.

\glossdef{good}[Good] One of the four \glossterm{alignment} components. For the Good ability tag, see \pcref{Ability Tags}.

\glossdef{grapple} A grapple is a \glossterm{combat maneuver} that allows you to physically restrain a creature.
For details, see \pcref{Grapple}.

\glossdef{grappled} A grappled creature is wrestling or in some other form of hand-to-hand struggle with at least one other creature.
While grappled, you suffer certain penalties and restrictions, as described below.
\begin{itemize}
    \item You must use a free hand (or equivalent limbs) to grapple, preventing you from taking any actions which would require having two free hands.
        If you cannot free a hand, you suffer a \minus10 penalty to accuracy on all \glossterm{physical attacks} until you have a free hand.
    \item You are \defenseless against creatures who are not grappled by you.
    \item You take a \minus4 penalty to accuracy with weapons that are not light, since they are too large and cumbersome to be used effectively in a grapple.
    \item Spellcasting is extremely difficult. You cannot cast spells with somatic components.
        Casting a spell without somatic components requires a Concentration check with a DR equal to 20 \add double spell level.
    \item You cannot normally move from your location. 
\end{itemize}

Other than the restrictions listed above, you can act normally. You can also try to move the grapple, escape the grapple, or pin your opponent. For details, see \pcref{Grapple Actions}.

\glossdef{hardness} An object's hardness indicates how durable it is.
Whenever a creature or object with hardness takes damage, it reduces that damage by an amount equal to its hardness.

\glossdef{heavy weapon} A heavy weapon is a type of \glossterm{weapon} that requires two hands to wield properly.
For details, see \pcref{Weapon Encumbrance}.

\glossdef{helpless} A helpless creature is completely at an opponent's mercy.
Its Dexterity is treated as \minus10.
Paralyzed, bound, and unconscious creatures are helpless.
Any \glossterm{physical attack} against a helpless creature automatically \glossterm{explodes} on the first die.

\glossdef{hidden task}[hidden tasks] Any checks for a hidden \glossterm{task} should be rolled secretly by the GM.\@
You should not know the result of your check, or even that a check was made.
For details, see \pcref{Hidden Tasks}.

\glossdef{hit point}[hit points] Your hit points measure how hard you are to kill.
When you take damage, you subtract that damage from your hit points.
For details, see \pcref{Hit Points}.

\glossdef{hit value} Your hit value is the number of hit points you gain per level.
Your hit value is normally equal to half your Fortitude defense or half your Mental defense, whichever is higher.

\glossdef{hunting party} A hunting party is the group of allies affected by a ranger's \textit{quarry} ability (see \pcref{Quarry}).

\glossdef{ignited} An ignited creature has been set on fire.
It takes 1d6 fire damage at the end of each \glossterm{action phase}, and takes a \minus2 penalty to \glossterm{accuracy}, checks, and defenses.
As a move action, an ignited creature can make a DR 10 Dexterity check to put out the flames.
This action requires a free hand.
Dropping prone as part of the action gives a \plus5 bonus on this check.

\glossdef{Imbuement} See \pcref{Ability Tags}.

\glossdef{immobilized} An immobilized creature can't move out of the space it was in when it became immobilized. Immobilized flying creatures that have the ability to hover can maintain their initial altitude. All other flying creatures subjected to this condition descend at a rate of 20 feet per round until they reach the ground, taking no falling damage.

\glossdef{improvised weapon}[improvised weapons] An improvised weapon is an object which could conceivably be used as a weapon, but which was not designed for that purpose.
Common examples include doors and wine bottles.
For details, see \pcref{Improvised Weapons}.

\glossdef{incorporeal} An incorporeal creature does not have a body.
It has no Strength or Constitution attributes.
It cannot take any action that requires having a body, and is immune to all such effects.
This includes suffering critical hits, moving objects, grappling, setting off pressure traps, and so on.

An incorporeal creature is immune to all nonmagical effects.
Even magical effects, including spells and attacks with magic weapons, have a 50\% chance to fail.

An incorporeal creature can enter or pass through solid objects, but it must remain adjacent to the object's exterior at all times.
If it is completely inside an object, it cannot see out or attack.
It can fight while partially inside an object, which grants it passive \glossterm{cover} and allows it to attack and see normallly.

\glossdef{initiative} When multiple creatures take mutually impossible actions simultaneously, such as racing to be the first one to a door, they must roll initiative checks to determine who completes the action first.
For details, see \pcref{Initiative}.

\glossdef{Instantaneous} See \pcref{Ability Tags}.

\glossdef{invisible} An invisible creature or object cannot be seen. Creatures unable to see an invisible creature are \defenseless against its attacks. Attackers suffer a 50\% miss chance even if they know the location of the invisible creature. See \pcref{Awareness}, and \pcref{Stealth}, for how to identify invisible creatures.

\glossdef{item slot}[item slots] Item slots are a resource that you can use to \glossterm{attune} to items in place of \glossterm{action points}.
If you use an item slot to attune to an item, you gain its effects without reducing your available action points.
For details, see \pcref{Item Slots}.

\glossdef{key attribute} The key attribute for a skill is the attribute associated with that skill.
For example, Climb is a Strength-based skill.
Some skills, such as Persuasion, do not have a key attribute.

\glossdef{legacy item}[legacy items] A legacy item is an item magically bonded to its bearer.
As its bearer gains levels, it increases in power as well.
For details, see \pcref{Legacy Items}.

\glossdef{legend point}[legend points] Legend points can be used to reroll failed rolls, or force your foes to reroll successful rolls against you. See \pcref{Legend Points}, for details.

\glossdef{Life} See \pcref{Ability Tags}.

\glossdef{Light} See \pcref{Ability Tags}.

\glossdef{light weapon}[light weapons] A light weapon is a type of \glossterm{weapon} that is relatively small and easy to use.
You can use your Dexterity to determine your \glossterm{accuracy} when making \glossterm{physical attacks} with light weapons.
For details, see \pcref{Weapon Encumbrance}.

\glossdef{line of effect} You cannot target something that you do not have line of effect to.
Line of sight is blocked by solid obstacles.
For details, see \pcref{Line of Effect}.

\glossdef{line of sight} You cannot target something that you do not have line of sight to.
Line of sight is blocked by any obstacle that blocks sight.
For details, see \pcref{Line of Sight}.

\glossdef{long rest} A long rest represents eight hours of relaxation or sleep.
It allows you to recover all of your spent \glossterm{action points} and \glossterm{legend points}.
For details, see \pcref{Long Rest}.

\glossdef{low-light vision} A creature with low-light vision can see more clearly in conditions of dim light.
It treats sources of light as if they had double their normal illumination range.
In addition, the creature treats environments with ambient dim light, such as a moonlit night, as if they were brightly lit when doing so is beneficial for it.

\glossdef{magic resistance} A creature with magic resistance can automatically resist magical abilities.
It functions like any other defense, except that it only works against magical effects.
To affect a magic resistant creature with a magical ability, you must make an additional magical attack against the creature's magic resistance value.
Your accuracy is equal to your \glossterm{power} with the ability you using, such as your spellpower with spells.
If your attack result beats the creature's magic resistance, the ability works normally.
Otherwise, the ability has no effect on the creature.
For details, see \pcref{Magic Resistance}.

\glossdef{magic source}[magic sources] A magic source defines where a creature's spells and rituals come from.
There are three magic sources: arcane, divine, and nature.
Mages cast arcane spells, clerics cast divine spells, and druids cast nature spells.

\glossdef{magical} A magical ability is an ability that has no physical explanation.
Examples include spells, a medusa's petrifying gaze, and a cleric's domain invocations.
Magical attacks often target Fortitude and Mental defenses, and can be resisted by \glossterm{magic resistance}.
For details, see \pcref{Magical Abilities}.

\glossdef{maneuverability} While flying, your maneuverability determines how easily you can change directions and perform aerial feats.
There are three types of manueverability: good, normal, and poor.
For details, see \pcref{Maneuverability}.

\glossdef{Manifestation} See \pcref{Ability Tags}.

\glossdef{mastered} If you have \glossterm{mastered} a skill, you have learned to use it to its maximum potential.
Your modifier with a mastered skill is equal to 3 \add either the skill's key attribute (if any) or your level, whichever is higher.
For details, see \pcref{Skill Training}.

\glossdef{medium weapon} A medium weapon is a type of \glossterm{weapon} that can be wielded in either one or two hands.
For details, see \pcref{Weapon Encumbrance}.

\glossdef{melee attack}[melee attacks] A melee attack is a \glossterm{physical attack} using your body or a weapon that does not leave your grasp.
You can only make melee attacks against targets within your \glossterm{reach}.

\glossdef{Mind} See \pcref{Ability Tags}.

\glossdef{minor action}[minor actions] Each round, you can take a single minor action in addition to your other actions that round.
Minor actions can be taken in either the \glossterm{action phase} or the \glossterm{delayed action phase}.
They are declared and resolved simultaneously with any other actions you take during that phase.
For details, see \pcref{Minor Actions}.

\glossdef{miscast} If your concentration is disrupted while casting a \glossterm{spell}, you miscast the spell instead.
The spell does not have its normal effect.
Instead, a damaging \glossterm{miscast backlash} occurs.

\glossdef{miscast backlash} When you \glossterm{miscast} a spell, you deal damage to yourself and creatures around you.
For details, see \pcref{Miscasting}.

\glossdef{miss chance}[miss chances] If you have a miss chance with an \glossterm{attack}, you have a random chance to miss with the attack regardless of the result of your attack roll.
If you have multiple miss chances, only the highest one applies.

\glossdef{Morale} See \pcref{Ability Tags}.

\glossdef{move} When you move, you usually travel a distance equal to your speed.
See \pcref{Movement and Positioning}, for details.
For specific actions that involve movement, see \glossterm{move action}.

\glossdef{move action}[move actions] A move action is a minor action that requires motion, such as drawing a sword.
You can take move actions during the \glossterm{movement phase}.
For the act of moving from one place to another, see \glossterm{move}.

\glossdef{movement phase} The movement phase is the first of two \glossterm{phases} in a combat \glossterm{round}.
During the movement phase, creatures can \glossterm{move} and take \glossterm{move actions}.
The movement phase is followed by the \glossterm{action phase}.

\glossdef{mundane} Most abilities are considered mundane abilities.
Mundane abilities have a tangible component and some form of natural explanation.
Examples include \glossterm{strikes}, \glossterm{combat maneuvers}, a dragon's breath weapon, and a barbarian's rage.
Unless otherwise indicated, all abilities are mundane in nature.

\glossdef{Mystic} See \pcref{Ability Tags}.

\glossterm{natural weapon} A natural weapon is a \glossterm{weapon} that is part of a creature's body.
For details, see \pcref{Natural Weapons}.

\glossdef{nauseated} A nauseated creature takes a \minus4 penalty to \glossterm{accuracy}, checks, and defenses.
These penalties do not stack with the penalties for being \sickened.

\glossdef{Negative} See \pcref{Ability Tags}.

\glossdef{opposed alignment} Each \glossterm{alignment} has an opposed alignment that is antethical to its principles and goals.
Good and Evil are opposed alignments, and Chaos and Law are opposed alignments.
For details, see \pcref{Alignment}.

\glossdef{overkill damage} If you take damage in excess of your \glossterm{bloodied} hit point total in a single round, the excess damage is dealt as \glossterm{vital damage}.
This excess damage is called overkill damage.
For details, see \pcref{Overkill Damage}.

\glossdef{overrun} An overrun is a special movement that allows you to move directly through creatures.
For details, see \pcref{Overrun}.

\glossdef{overwhelm penalties} An \glossterm{overwhelmed} creature suffers overwhelm penalties equal to half the combined \glossterm{overwhelm value} of all creatures threatening it.
Overwhelm penalties apply to Armor and Reflex defenses.
For details, see \pcref{Overwhelm}.

\glossdef{overwhelm resistance} A creature with \glossterm{overwhelm resistance} reduces the effective \glossterm{overwhelm value} of creatures threatening it.
This can reduce or remove their \glossterm{overwhelm penalties}.
For details, see \pcref{Overwhelm Resistance}.

\glossdef{overwhelm value}[overwhelm values] Your overwhelm value determines how much you contribute to \glossterm{overwhelm penalties} against creature you \glossterm{threaten}.
Most Small and Medium creatures have a overwhelm value of 1.
For details, see \pcref{Overwhelm Value}.

\glossdef{overwhelmed} A creature is overwhelmed if the combined \glossterm{overwhelm value} of all creatures that \glossterm{threaten} it is at least 2.
An overwhelmed creature suffers \glossterm{overwhelm penalties}.
For details, see \pcref{Overwhelm}.

\glossdef{outsider} An outsider is a type of creature.
Outsiders are composed of planar material from a plane other than the Material Plane.

\glossdef{panicked} A panicked creature takes a \minus4 penalty to \glossterm{accuracy}, checks, and defenses as long as it is within \rngmed range of the source of its fear.
In addition, it must flee from the source of its fear by any means necessary if it is within that range.
If unable to flee, it must do nothing other than use the \textit{total defense} ability every round (see \pcref{Total Defense}).
These penalties do not stack with the penalties for being \shaken or \panicked.

If the source of a panicked creature's fear is a creature and is \glossterm{defeated}, this effect is broken.

\glossdef{paralyzed} A paralyzed creature is unable to take physical actions. It has effective Dexterity and Strength scores of \minus10 and is \helpless, but can take purely mental actions. This can cause flying creatures to crash, swimming creatures to drown, and so on. A creature can move through a space occupied by a paralyzed creature -- ally or otherwise. Each square occupied by a paralyzed creature, however, counts as 2 squares.

\glossdef{passive cover} Passive cover is a type of \glossterm{cover} that can block \glossterm{physical attacks}.
It represent immobile obstacles that make it harder to aim, imposing a \minus2 penalty to \glossterm{accuracy}.
For details, see \pcref{Passive Cover}.

\glossdef{petrified} A petrified creature has been turned to stone. It is neither alive nor dead, but is unconscious and unable to take actions, and its body is an inanimate statue. If the statue is broken or damaged before the creature is restored to its original state, the creature has equivalent damage or deformities.

\glossdef{phase}[phases] A phase is part of the combat \glossterm{round}.
There are two phases: the \glossterm{movement phase} and the \glossterm{action phase}.
A phase does not represent a fixed span of time.
It is an abstract concept designed to represent a variety of actions that all take place nearly simultaneously.

\glossdef{Physical} See \pcref{Ability Tags}.

\glossdef{physical attack}[physical attacks] A physical attack is an \glossterm{attack} made with a creature's body.
The most common type of physical attack is a \glossterm{strike}, but there are other physical attacks, such as \glossterm{combat maneuvers}.
All physical attacks share a common way of determining their \glossterm{accuracy} (see \pcref{Physical Accuracy}).
Most physical attacks target Armor defense.

\glossdef{pinned} A pinned creature is held completely immobile in a grapple.
The only physical actions it can make are to escape the grapple (see \pcref{Grappling}).
Like a \glossterm{helpless} creature, its Dexterity is treated as \minus10.

\glossdef{Planar} See \pcref{Ability Tags}.

\glossdef{origin}[origin] An origin is the grid intersection, creature, or object that an area originates from.
For details, see \pcref{Area}.

\glossdef{poison}[poisons] For a description of poisons and how they work, see \pcref{Poisons}.

\glossdef{Poison} See \pcref{Ability Tags}.

\glossdef{Positive} See \pcref{Ability Tags}.

\glossdef{potency} The potency of a poison, disease, or similar effect determines its attack bonus.

\glossdef{power} The power of an \glossterm{ability} represents how strong the ability is.
This usually determines your \glossterm{accuracy} with the ability and the ability's \glossterm{standard damage}.
For details, see \pcref{Power}.

\glossdef{proficient}[proficiency] A creature can be proficient with weapons and armor.
If you try to attack with a weapon you are not proficient with, you take a \minus2 penalty to accuracy (see \pcref{Weapon Proficiency}).
If you try to use armor you are not proficient with, it is less effective and your \glossterm{accuracy} with \glossterm{physical attacks} is reduced (see \pcref{Armor Proficiency}).

\glossdef{projectile} A projectile is an object fired from a projectile weapon.
Arrows and bolts are projectiles.

\glossdef{prone} A prone creature is lying on the ground, rather than standing normally.
\glossterm{Melee attacks} against it gain a \plus2 bonus to \glossterm{accuracy}, while physical ranged attacks against it take a \minus2 penalty to accuracy.
In addition, it takes a \minus2 penalty to accuracy with melee attacks and is unable to move until it stands up.
A creature can stand up from being prone during the movement phase.
This generally requires one free hand.

\glossdef{random effect}[random effects] Random effects change what they do based on a specific die roll.
This does not include effects which require a successful attack or similar roll.
The \spell{prismatic beam} spell is an example of a random effect.
In addition, the random retargeting of certain miscast spells, such as \spell{scorching ray}, is a random effect.

\glossdef{rage bonus} The bonus a character with the rage ability adds to their damage, Fortitude, Willpower, and more.
For details, see \pcref{Rage}.

\glossdef{range} The range of an ability determines how far away it can be used.
You can't use abilities on a target outside of the ability's range.

\glossdef{range increment}[range increments] Physical ranged attacks often have a specific range increment.
A range increment is always measured in feet.
You take a \minus1 penalty to accuracy with the ranged attack for each full range increment between you and your target.

\glossdef{rare language}[rare languages] Rare languages are languages that are only spoken by rare or distant creatures or cultures.
They are described in \tref{Rare Languages}.

\glossdef{Retributive} See \pcref{Ability Tags}.

\glossdef{reach} Your reach is how far away from your body you can make melee attacks.
A typical Medium creature has a five-foot reach.

\glossdef{ritual}[rituals] A ritual is a \glossterm{magical} ability with esoteric effects that includes multiple \glossterm{subrituals}.
For details, see \pcref{Rituals}.

\glossdef{roll}[rolls] 

\glossdef{round}[rounds] Combat takes place in a series of rounds, which represent about six seconds of action.
Rounds are divided into two \glossterm{phases}: the \glossterm{movement phase}, and the \glossterm{action phase}.

\glossdef{scent} A creature with the scent ability has an unusually good sense of smell.
It gains a \plus10 bonus to scent-based Awareness checks (see \pcref{Senses}).

\glossdef{Scrying} See \pcref{Ability Tags}.

\glossdef{shadowy illumination} In an area with shadowy illumination, creatures can see dimly.
Creatures within this area have \concealment, which can allow them to make Stealth checks to hide (see \pcref{Stealth}).

\glossdef{shaken} A shaken creature takes a \minus2 penalty to \glossterm{accuracy}, checks, and defenses as long as it is within 100 feet of the source of its fear.
These penalties do not stack with the penalties for being \frightened or \panicked.

If the source of a shaken creature's fear is a creature and is \glossterm{defeated}, this effect is broken.

\glossdef{Shaping} See \pcref{Ability Tags}.

\glossdef{shield}[shields] Shields are a form of \glossterm{armor} that you wield in a hand to protect you from harm.
For details, see \pcref{Armor}.

\glossdef{Shielding} See \pcref{Ability Tags}.

\glossdef{short rest} A short rest represents five minutes of relaxation.
It allows you to recover a small amount of \glossterm{hit points} and some of your spent \glossterm{action points}.
For details, see \pcref{Short Rest}.

\glossdef{shove} A shove is a \glossterm{combat maneuver} that allows you to move a creature.
For details, see \pcref{Shove}.

\glossdef{sickened} A sickened creature takes a \minus2 penalty to \glossterm{accuracy}, checks, and defenses.
These penalties do not stack with the penalties for being \nauseated.

\glossdef{size category} A creature's size category indicates how large it is.
There are nine size categories, from smallest to largest: Fine, Diminuitive, Tiny, Small, Medium, Large, Huge, Gargantuan, Colossal.
For details, see \pcref{Size in Combat}.

\glossdef{Sizing} See \pcref{Ability Tags}.

\glossdef{skill}[skills] A skill represents your degree of talent with a particular non-combat aspect of the world.
For example, the Climb skill represents how skilled you are at climbing.
For details, see \pcref{Skills}.

\glossdef{skill point}[skill points] You can spend skill points to gain training in skills (see \pcref{Skill Training}).
You gain skill points from your class, from having a high Intelligence, and from taking penalties to your starting attributes (see \pcref{Impaired Attributes}).
For details, see \pcref{Skill Points}.

\glossdef{slowed} A slowed creature cannot act during the movement phase, and moves at half speed.

\glossdef{somatic components}[somatic] Somatic components are hand motions required to cast most spells.
For details, see \pcref{Components}.

\glossdef{Sonic} See \pcref{Ability Tags}.

\glossdef{space} Your space is the area that your physical body occupies.
For convenience, your space is measured in five-foot \glossterm{squares}.
Small and Medium creatures occupy space equal to a single five-foot square.

\glossdef{Speech} See \pcref{Ability Tags}.

\glossdef{speed} Your speed represents the number of feet you can move with a single movement (see \pcref{The Movement Phase}).

\glossdef{spell}[spells] A spell is a \glossterm{magical} ability with combat-relevant effects that includes multiple \glossterm{subspells}.
For details, see \pcref{Spells}.

\glossdef{spell list} The list of spells you can cast from a particular \glossterm{spell source}.
Each spell source has a specific spell list which is described at \pcref{Spells}.
Most characters with the same spell sources have the same spell lists.
However, some effects, such as a cleric's domains, can add spells to a character's individual spell list.

\glossdef{spellpower} Your spellpower represents how powerful the spells you cast are (see \pcref{Magic}).

\glossdef{sprinting} A sprinting creature is currently using the \textit{sprint} action to move faster (see \pcref{Sprint}).

\glossdef{square}[squares] A square represents a single 5-ft.\ by 5-ft.\ space.
Many areas are measured in squares for convenience.

\glossdef{squeezing} A squeezing creature is trying to move though an area too small for it to fight in normally.
While squeezing, a creature moves at half speed and takes a \minus2 penalty to physical accuracy and Armor and Reflex defenses.
For details, see \pcref{Squeezing}.

\glossdef{stabilization roll}[stabilization rolls] A roll made when a creature is \glossterm{dying} to see if it stabilizes or dies. For details, see \pcref{Injury, Death, and Healing}.

\glossdef{staggered} A staggered creature is temporarily overwhelmed by physical trauma.
It takes a \minus4 penalty to \glossterm{accuracy} and \glossterm{checks}.
Becoming \glossterm{bloodied} or \glossterm{wounded} can cause you to become staggered (see \pcref{Staggered}).

\glossdef{standard action} You can use a standard action to attack with a weapon, cast a spell, drink a potion, and do most other things that take concentration and effort.

\glossdef{standard damage} A common damage value for abilities.
For details, see \pcref{Standard Damage}.

\glossdef{strike}[strikes] A strike is a single physical attack with a weapon.
It is the most common type of attack.
You can make a strike as a \glossterm{standard action} in the \glossterm{action phase}.
For details, see \pcref{Strikes}.

\glossdef{strike damage} The damage you deal with a single \glossterm{strike} from a weapon is called your \glossterm{strike damage}.
For details on calculating your strike damage, see \pcref{Strike Damage}.
Some abilities other than \glossterm{strikes} deal damage based on your \glossterm{strike damage}.

\glossdef{stunned} A stunned creature cannot take any actions during the \glossterm{action phase} or \glossterm{delayed action phase} except the \textit{recover} and \textit{desperate recovery} actions (see \pcref{Recover}, and \pcref{Desperate Recovery}).
In addition, it takes a \minus2 penalty to \glossterm{accuracy}, \glossterm{checks}, and \glossterm{defenses}.

\glossdef{subdual damage} Subdual damage is a special kind of damage that can't kill you.
If you take subdual damage while \glossterm{wounded}, you fall unconscious.
For details, see \pcref{Subdual Damage}.

\glossdef{Subtle} See \pcref{Ability Tags}.

\glossdef{subritual}[subrituals] A subritual is a particular variant of a \glossterm{ritual} that is cast at a higher level.
In exchange, a subritual may have more powerful effects, or it may have completely different effects that are thematically related to the base ritual.
For details, see \pcref{Subrituals}.

\glossdef{subspell}[subspells] A subspell is a particular variant of a \glossterm{spell} that is cast at a higher level.
In exchange, a subspell may have more powerful effects, or it may have completely different effects that are thematically related to the base spell.
For details, see \pcref{Subspells}.

\glossdef{suppressed} A suppressed ability has temporarily ceased to function.
It has no effect for as long as it remains suppressed.
Time spent while suppressed counts against the ability's \glossterm{duration}, and it may expire while suppressed if it lasts for a specific amount of time.
Only \glossterm{magical} abilities can be suppressed.
Mundane results of magical abilities that have already occured, such as the water created by a \ritual{create water} ritual, cannot themselves be suppressed, and do not disappear if they enter an area that suppresses magical abilities.

\glossdef{sustain}[sustained] Some abilities last as long as you sustain them.
Each ability specifies a particular action that is required to sustain the ability, such as a \glossterm{minor action}
At the end of each round, the ability is dismissed unless you used the ability that round or took the action to sustain the ability that round.
For details, see \pcref{Sustained Abilities}.

\glossdef{Sustain} An ability with this \glossterm{ability tag} lasts as long as you sustain it each round.
The tag includes an action type, such as (minor), which indicates the type of action required to sustain the ability.
For details, see \pcref{Sustained Abilities}.

\glossdef{Swift} An ability with this \glossterm{ability tag} resolves its effects before other actions in the same phase.
For details, see \pcref{Swift Abilities}.

\glossdef{swim speed} A creature with a swim speed can swim as easily as a human walks on land.
The effects of a swim speed are described at \pcref{Swim Speed}.

\glossdef{take 10}[taking 10] If you have plenty of time to accomplish a task that requires a \glossterm{check}, and there are no meaningful consequences for failure, you can take 10 to accomplish the task.
If you do, the task takes ten times as long, but you treat your roll for the check as if you had rolled a 10.
For details, see \pcref{Taking 10}.

\glossdef{target}[targets] A target is a creature or object directly affected by an ability.
Many abilities only affect a single target, and some affect a specific number of targets.

\glossdef{target square} A target square is a particular \glossterm{square} that an attack is made against.
A target square is chosen to determine \glossterm{cover} and \glossterm{concealment} (see \pcref{Cover}).

\glossdef{targeted} A \glossterm{targeted} ability is an ability that allows you to directly choose which targets the ability affects.
A spell that affects an area is not a targeted ability, because you choose the area affected instead of choosing the targets directly.

\glossdef{task} A task is a particular way to use a \glossterm{skill}.
For example, balancing on slippery ground is a task that you can use the Acrobatics skill for (see \pcref{Balance}).
For details, see \pcref{Tasks}.

\glossdef{Telekinesis} See \pcref{Ability Tags}.

\glossdef{Teleportation} See \pcref{Ability Tags}.

\glossdef{Temporal} See \pcref{Ability Tags}.

\glossdef{threat} A creature's threat represents how threatening it is.
Many monsters choose the targets of their attacks based on the threat of their foes.
For details, see \pcref{Threat}.

\glossdef{threaten}[threatened] If you are using a \glossterm{melee} weapon, all enemies within your \glossterm{reach} with that weapon are \glossterm{threatened}.
A threatened creature may suffer \glossterm{overwhelm penalties} if there are multiple creatures threatening it.

\glossdef{total cover} Total cover is a type of \glossterm{cover}.
If a creature is completely behind a physical object that blocks sight, it has \glossterm{total cover} from attacks.
A creature with total cover cannot be targeted by any attacks.
For details, see \pcref{Total Cover}.

\glossdef{trained} If you are trained in a skill, you have learned how to use it well, but you have not \glossterm{mastered} it.
Your modifier with a trained skill is equal to either half your level \add 1 or the skill's \glossterm{key attribute} (if any), whichever is higher.
For details, see \pcref{Skill Training}.

\glossdef{Trap} See \pcref{Ability Tags}.

\glossdef{trip} A trip is a \glossterm{combat maneuver} that allows you to knock a foe off its feet.
For details, see \pcref{Trip}.

\glossdef{tremorsight} A creature with tremorsight can ``see'' its surroundings perfectly without any light, regardless of concealment or invisibility.
It needs an uninterrupted path through solid objects to sense its surroundings, but does not require line of effect.
Tremorsight always has a range, and grants no benefits beyond that range.

\glossdef{tremorsense} A creature with tremorsense can sense its surroundings without any light, regardless of concealment or invisiblity.
It knows the location of everything around it, but it still takes normal failure chances for concealment, invisibility, and so on.
It needs an uninterrupted path through solid objects to sense its surroundings, but does not require line of effect.
Tremorsense always has a range, and grants no benefits beyond that range.

\glossdef{truesight} A creature with truesight can ignore all \glossterm{Figment} and \glossterm{Glamer} effects within a given range.
Despite the name of the ability, it affects all senses, not merely sight.

\glossdef{unarmed attack}[unarmed attacks] Every corporeal creature is capable of making an attack using its bare fists (or similar appendages).
For details, see \pcref{Unarmed Combat}.

% TODO: expand definition to include ``attended by a willing creature''
% Or just expand ``willing'' to include objects
\glossdef{unattended} An unattended item is an item not being held or carried by a creature.
Some abilities can only affect unattended items.

\glossdef{unaware} An unaware creature does not know that it is being attacked.
Any \glossterm{physical attack} against an unaware creature automatically \glossterm{explodes} on the first die.
After being attacked, an unaware creature typically stops being unaware of future attacks, even if cannot see or identify its attacker.

\glossdef{verbal components}[verbal] Verbal components are words required to cast most spells.
For details, see \pcref{Components}.

\glossdef{Visual} See \pcref{Ability Tags}.

\glossdef{vital damage} If you take damage when you have no \glossterm{hit points} remaining, that damage is dealt as vital damage.
Vital damage inflicts debilitating \glossterm{vital damage penalties}.
For details, see \pcref{Vital Damage}.

\glossdef{vital damage penalties} For every 4 points of \glossterm{vital damage} you have, you take a \minus1 penalty to \glossterm{accuracy}, \glossterm{checks}, and \glossterm{defenses}.
For details, see \pcref{Vital Damage}.

\glossdef{vulnerable} A creature can be vulnerable to a type of damage or a special weapon material.
It takes double damage from sources it is vulnerable to.
If it takes damage from a damage source with multiple types or multiple materials, it takes double damage if it is vulnerable to any of those types or materials.
Vulnerability is calculated before applying \glossterm{damage reduction}.
\par If a creature would become vulnerable to the same thing multiple times, it still only takes double damage from damage of that type.

\glossdef{Water} See \pcref{Ability Tags}.

\glossdef{weapon}[weapons] A weapon is an object used to inflict damage.
Some creatures can treat parts of their body as weapons.
For details, see \pcref{Weapons}.

\glossdef{weapon group}[weapon groups] A weapon group is a category of \glossterm{weapons} with a similar design and fighting style.
You have proficiency with some number of weapon groups based on your \glossterm{class}.
For details, see \pcref{Weapon Groups}.

\glossdef{willing} Some abilities can only affect willing targets. You can choose to be a willing target at any time. Unconscious creatures and objects are automatically considered willing, but a character who is conscious but immobile or helpless (such as one who is bound or paralyzed) is not automatically willing.

\glossdef{withdraw} The \textit{withdraw} ability allows you to stay away from a creature, preventing it from coming too close to you.
For details, see \pcref{Withdraw}.

\glossdef{wounded} A wounded creature has no hit points remaining, or has taken \glossterm{vital damage}.
If you take damage in excess of your hit points while wounded, that damage is dealt as vital damage.
For details, see \pcref{Wounded}.

\glossdef{zone} A zone is a type of area that an ability can have (see \pcref{Area Types}).
A zone ability has effects within an area for the \glossterm{duration} of the ability.
Unless otherwise noted, it does not move after being created.
