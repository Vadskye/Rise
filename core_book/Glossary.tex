\chapter{Glossary}\label{Glossary}

\glossdef{ability} An ability is a generic term for any unusual property a creature has or any special actions it can take to cause particular effects.
Spells, racial traits, and the benefits from class \glossterm{archetypes} can all be called abilities.

\glossdef{ability tag} An ability tag describes the effects of an ability.
For details, see \pcref{Ability Tags}.

\glossdef{action phase} The action phase is the second of two \glossterm{phases} in a combat \glossterm{round}.
During the action phase, creatures can \glossterm{attack}, cast \glossterm{spells}, and take other major combat actions.

\glossdef{alchemical item} An alchemical item is any item created using the Craft (alchemy) skill.
This includes firebombs, potions, and many other items.

\glossdef{attunement point} Attunement points allow you to \glossterm{attune} to effects such as spells or items (see \pcref{Attunement}).
In addition, you can use some special abilities by spending \glossterm{attunement points}.
For details, see \pcref{Attunement Points}.

\glossdef{accuracy} The bonus added to an \glossterm{attack roll}.
For details, see \pcref{Accuracy}.

\glossdef{alignment} Your alignment represents your general morality in broad terms.
For details, see \pcref{Alignment}.

\glossdef{ally}[allies] Some beneficial abilities affect allies.
An ally is any creature you consider an ally who also considers you an ally, not including yourself.
For details, see \pcref{Allies and Enemies}.

\glossdef{archetype}[archetypes] An archetype is a collection of related abilities from a particular class.
Each class has three archetypes that members of that class normally have.
For details, see \pcref{Archetypes}.

\glossdef{archetype rank} Each ability from an \glossterm{archetype} has a minimum rank required to gain the ability.
For details, see \pcref{Archetype Ranks}.

\glossdef{armor} Armor is a form of equipment that protects your body from harm.
There are two kinds of armor: \glossterm{body armor}, which you wear on your body, and \glossterm{shields}, which you wield in a hand.
For details, see \pcref{Armor}.

\glossdef{attack}[attacks] Anything that affects another creature in a potentially harmful way. There are two kinds of attacks: \glossterm{mundane} attacks and \glossterm{magical} attacks.

\glossdef{attack result} An attack result is the total you get on an \glossterm{attack roll}, after taking to account any bonuses or penalties that apply to the roll.

\glossdef{attack roll}[attack rolls] A roll required to succeed with an attack.
To make an attack roll, roll 1d10 \add your \glossterm{accuracy} with the attack.
If the result of the attack roll equals or exceeds the target's \glossterm{defense}, the attack succeeds.
Some attacks, especially magical attacks, have effects even if the attack roll fails.
For details, see \pcref{Attack Rolls}.

\glossdef{attended} An attended item is an item currently being held or carried by a creature.
Some abilities can only affect \glossterm{unattended} items.

\glossdef{attribute} A core representation of a character's capacity in a wide range of areas. There are six attributes: \glossterm{Strength}, \glossterm{Dexterity}, \glossterm{Constitution}, \glossterm{Intelligence}, \glossterm{Perception}, and \glossterm{Willpower}.

\glossdef{attune}[attunement] Some abilities last as long as you attune to them.
Attuning to an ability costs an \glossterm{attunement point} that you cannot recover as long as you maintain your attunement to that ability.
For details, see \pcref{Attunement}.

\glossdef{Attune} An ability with this \glossterm{ability tag} lasts as long as a creature attunes to it.
For details, see \pcref{Attunement}.

\glossdef{attuned} If you are attuned to an ability, you have invested an \glossterm{attunement point} in it to maintain its effect.
For details, see \pcref{Attunement}.

\glossdef{Auditory} See \pcref{Ability Tags}.

\glossdef{base attribute}[base] Your base attribute is the value 

\glossdef{base speed} Each size category has a base speed that indicates how far creatures of that size category can generally move.
For details, see \pcref{Base Speed}.

\glossdef{briefly}[brief] An effect that lasts briefly, or a brief effect, lasts until after the end of the next round after the effect was applied.
As normal, unless the effect has the \abilitytag{Swift} tag, it does not have any effect during the phase that it is applied.

\glossdef{blinded} See \pcref{Circumstances and Debuffs}.

\glossdef{blindsense}[Blindsense] A creature with blindsense can sense the location of everything nearby, regardless of light levels.
For details, see \pcref{Blindsense}.

\glossdef{blindsight} A creature with blindsight see perfectly without any light.
For details, see \pcref{Blindsight}.

\glossdef{bright illumination} In an area with bright illumination, creatures can see clearly.
A creature can't hide in an area with bright illumination unless it is invisible or has cover.
For details, see \pcref{Vision and Light}.

\glossdef{brilliant illumination} In an area with brilliant illumination, creatures can see clearly.
A creature can't hide in an area with brilliant illumination unless it is invisible or has cover.
In addition, no shadows exist within an area of brilliant illumination.
For details, see \pcref{Vision and Light}.

\glossdef{body armor} Body armor is a form of \glossterm{armor} that you wear on your body.
For details, see \pcref{Armor}.

\glossdef{broken} A broken object is damaged and unsuitable for use, though it retains its general structure and can be repaired.
For details, see \pcref{Broken Objects}.

\glossdef{burst} A burst is a type of area that an ability can have (see \pcref{Area Types}).
A burst ability has an immediate effect on all valid targets within an area.

\glossdef{cantrip}[cantrips] Some \glossterm{mystic spheres} have minor spells called cantrips.
Anyone who has access to a mystic sphere knows all cantrips from that sphere.
For details, see \pcref{Cantrips}.

\glossdef{carrying capacity} Your carrying capacity defines the amount of weight you can carry without penalty.
For details, see \pcref{Weight Limits}.

\glossdef{casting components} Spells generally require specific casting components.
There are two types of casting components: \glossterm{somatic components} and \glossterm{verbal components}.
Somatic components are only used by arcane and pact spellcasters while \glossterm{verbal components} are used by all spellcasters.
For details, see \pcref{Casting Components}.

\glossdef{challenge rating} The challenge rating of a monster indicates its approximate strength within its level.
For details, see \pcref{Challenge Rating}.

\glossdef{character level} Your character level is your total level, including levels from all of your classes.
Whenever text refers to your ``level'', without specifying a particular kind of level, it means your character level.

\glossdef{charmed} See \pcref{Circumstances and Debuffs}.

\glossdef{check}[checks] A check is a d10 roll required to accomplish an action that has a chance of failure that is not an attack.
If the result of your roll, including your modifier, is high enough, you succeed.
Otherwise, you fail.
For details, see \pcref{Checks}.

\glossdef{class}[classes] Your class represents your fundamental source of power and the type of abilities you have.
For example, barbarians draw power from the primal energy found deep within all living things, while clerics draw power from their worship of mighty deities.
For details, see \pcref{Classes}.

\glossdef{class skill}[class skills] A class skill is a skill that a class is particularly good at using.
Each class has a specific set of class skills given in its description.
Each class also gives a particular number of \glossterm{trained skills} from among your class skills.
For details, see \pcref{Trained Skills}.

\glossdef{climb speed} A creature with a climb speed can climb as easily as a human walks on land.
The effects of a climb speed are described at \pcref{Climb Speed}.

\glossdef{close range} Weapons have two \glossterm{range limits}: close range and \glossterm{long range}.
Attacks within a weapon's close range have no penalty.
For details, see \pcref{Weapon Range Limits}.

\glossdef{combat style} A combat style is a collection of \glossterm{maneuvers} that some classes gain access to.
For details, see \pcref{Combat Styles}.

\glossdef{common language}[common languages] Common languages are languages that are widely spoken.
They are described in \tref{Common Languages}.

\glossdef{Compulsion}[Compulsions] See \pcref{Ability Tags}.

\glossdef{concealment} Concealment represents effects which make a target harder to see, such as shadowy lighting.
All attacks against a creature or object with concealment from you have a 25\% miss chance.
For details, see \pcref{Concealment}.

\glossdef{condition}[conditions] A condition is an effect that lasts on a creature until it is removed by effects that remove conditions.
All conditions are detrimental, and most are standard \glossterm{debuffs}.
Player characters can remove conditions with the \ability{recover} ability or by taking a \glossterm{short rest}, as well as with various special abilities (see \pcref{Recover}).
For details, see \pcref{Ability Durations}.

\glossdef{confused} See \pcref{Circumstances and Debuffs}.

\glossdef{Constitution} Constitution is an \glossterm{attribute} that measures your health and stamina.
For details, see \pcref{Constitution}.

\glossdef{cover} Cover represents any obstacle that physically prevents you from striking your target, such as a tree or intervening creature.
For details, see \pcref{Cover}.

\glossdef{Creation} See \pcref{Ability Tags}.

\glossdef{Curse} See \pcref{Ability Tags}.

\glossdef{critical failure} When you make a check, if your result failed to beat the \glossterm{difficulty value} by 10 or more, you get a critical failure.
Some abilities have special effects on critical failures.

\glossdef{critical hit}[critical hits] When you make an attack, if your result beat the target's defense by 10 or more, you get a critical hit.
Unless otherwise noted, damaging attacks deal double damage on a critical hit.
Some abilities have special effects on critical hits.

\glossdef{critical success} When you make a check, if your result beat the \glossterm{difficulty value} by 10 or more, you get a critical success.
Some abilities have special effects on critical successes.

\glossdef{damage} Many attacks deal damage to you when they hit.
For details, see \pcref{Damage}.

\glossdef{damage resistance} Whenever you take damage, you first apply that damage to your damage resistance applying it to your \glossterm{hit points}.
For details, see \pcref{Damage Resistance}.

\glossdef{darkvision}[Darkvision] A creature with darkvision can see perfectly in complete darkness.
For details, see \pcref{Darkvision}.

\glossdef{dazed} See \pcref{Circumstances and Debuffs}.

\glossdef{dazzled} See \pcref{Circumstances and Debuffs}.

\glossdef{dead} A dead creature's soul leaves its body. Dead creatures cannot benefit from normal or magical healing, but they can be restored to life via magic (see \pcref{Resurrecting the Dead}). A dead body decays normally unless magically preserved.

\glossdef{deafened} See \pcref{Circumstances and Debuffs}.

\glossdef{debuff} A debuff is a negative effect on a creature.
Many debuffs are applied as \glossterm{conditions}, but some last for longer or shorter times.
For a list of debuffs, see \pcref{Circumstances and Debuffs}.

\glossdef{defeat} You defeat a creature if you personally cause it to become \glossterm{defeated}.
Abilities that trigger when you defeat a creature generally activate if you deal damage to it in a phase when it dies or is knocked unconscious.
This often means multiple creatures are considered to have defeated the same enemy.
For narrative purposes, you can choose to give credit to the creature who dealt the most damage in the last phase, but you shouldn't use that method for determining whether creatures gain the benefit of effects like an \mitem{onslaught} weapon.

\glossdef{defeated} A creature is defeated if it dies, surrenders or is incapacitated for an extended period of time (such as by being knocked unconscious).
Some abilities, such as a ranger's \textit{quarry} ability (see \pcref{Quarry}), last until their target is defeated.
If there is ambiguity about whether a surrendering or seemingly incapacitated enemy still poses a threat, you choose whether you consider the enemy to be defeated.

\glossdef{defense}[defenses] A defense is a static number which represents how difficult you are to affect with attacks.
There are four defenses: Armor, Fortitude, Reflex, and Mental.
For details, see \pcref{Defenses}.

\glossdef{delayed action phase} The delayed action phase is a \glossterm{phase} that occurs after the \glossterm{action phase}.
It is not always necessary, because most actions are not delayed.
For details, see \pcref{The Delayed Action Phase}.

\glossdef{destroyed} A destroyed object has been damaged to the point where it is completely beyond repair.
For details, see \pcref{Destroyed Objects}.

\glossdef{Detection} See \pcref{Ability Tags}.

\glossdef{Dexterity} Dexterity is an \glossterm{attribute} that measures your hand-eye coordination, agility, and reflexes.
For details, see \pcref{Dexterity}.

\glossdef{dice increment}[dice increments] A die increment is a single increase or decrease in the value of a dice pool.
For example, a 1d8 that is increased by one die increment becomes a 1d10 die.
Similarly, a 2d6 dice pool that is decreased by one die increment also becomes a 1d10 die.
For details, see \pcref{Dice Pools}.

\glossdef{difficult terrain} Difficult terrain costs double the normal movement cost to move out of.
For details, see \pcref{Difficult Terrain}.

\glossdef{difficulty value} The difficulty value of a \glossterm{check} is the check result required to succeed.
It is often abbreviated as DV.
In general, attacks are rolled to beat \glossterm{defenses}, and checks are rolled to beat a given difficulty value.

\glossdef{dirty trick} You can use the \textit{dirty trick} ability to impair a foe by using your environment.
For details, see \pcref{Dirty Trick}.

\glossdef{disarm} You can use the \textit{disarm} ability to strike items held or worn by a creature.
For details, see \pcref{Disarm}.

\glossdef{disease} An affliction of the body, causing a steady deterioration over time.

\glossdef{dismiss}[dismissed] When you dismiss an ability, it ends, and all of its lingering effects are removed.
Unless otherwise noted, all abilities with a duration can be dismissed.
You can dismiss abilities as a \glossterm{free action} (see \pcref{Dismissal}).

\glossdef{dominated} See \pcref{Circumstances and Debuffs}.

\glossdef{emanation} An emanation is a type of area that an ability can have (see \pcref{Area Types}).
An emanation ability has effects within an area for the duration of the ability.
It emanates from a specific creature or object, rather than a location.
If that creature or object moves, the emanation moves with it.

\glossdef{Emotion} See \pcref{Ability Tags}.

\glossdef{encumbrance} Your encumbrance is a value that represents how much you are burdened by armor and weight.
For details, see \pcref{Encumbrance}.

\glossdef{enemy}[enemies] Some harmful abilities affect enemies.
An enemy is any creature you consider to be an enemy.
For details, see \pcref{Allies and Enemies}.

\glossdef{energy damage}[energy] There are five types of energy damage: acid, cold, electricity, fire, and sonic.
For details, see \pcref{Damage Types}.

\glossdef{environmental damage}[environmental] Environmental damage is a type of damage.
Environmental damage does not reduce the \glossterm{damage resistance} of creatures or objects, making small amounts of environmental damage irrelevant to healthy creatures.
For details, see \pcref{Environmental Damage}.

\glossdef{exotic weapon}[exotic weapons] A rare few weapons are considered exotic weapons.
They are unusually difficult to wield, and even being \glossterm{proficient} with the associated \glossterm{weapon group} does not grant you the ability to use an exotic ewapon.
Some class abilities grant proficiency with exotic weapons.

\glossdef{explode}[explodes] When you roll a 10 on an \glossterm{attack roll}, the die can explode.
If it does, you roll it again and add the two results together to determine the total.
For details, see \pcref{Exploding Attacks}.

\glossdef{failure chance}[failure chance] If you have a failure chance with an \glossterm{attack}, you have a random chance to miss with the attack regardless of the result of your attack roll.
If you have multiple failure chances, only the highest one applies.
Failure chances are rolled independently from \glossterm{miss chances}, and they are not affected by abilities that mitigate miss chances.
They are less common than a miss chance, and reflect circumstances that no amount of skill can mitigate.

% Falling damage ignores the normal dice pool mechanics because adding large chains
% of +1d modifiers on the fly is a huge hassle.
\glossdef{falling damage} If you fall at least 5 feet, you and the object you land on take bludgeoning \glossterm{environmental damage}.
This damage is called falling damage, and it is equal to 1d6 per 10 feet you fell, up to a maximum of 20d6 damage.
If you control your fall with the \textit{mitigate fall} ability, you can reduce the falling damage you take (see \pcref{Jump}).

\glossdef{fatigue level} Your fatigue level measures how fatigued you are.
You take a \glossterm{fatigue penalty} if your fatigue level exceeds your \glossterm{fatigue tolerance}.
For details, see \pcref{Fatigue}.

\glossdef{fatigue penalty} You take a penalty to \glossterm{accuracy} and \glossterm{checks} equal to your \glossterm{fatigue level} \sub your \glossterm{fatigue tolerance}.
If you have a fatigue penalty of at least \minus1, you are considered \glossterm{fatigued}.
When your fatigue penalty reaches \minus5, you fall \unconscious until your fatigue penalty is reduced below \minus5.
For details, see \pcref{Fatigue Penalty}.

\glossdef{fatigue tolerance} Your fatigue tolerance measures the maximum \glossterm{fatigue level} you can reach before you suffer a \glossterm{fatigue penalty}.
For details, see \pcref{Fatigue Tolerance}.

\glossdef{fly speed} A creature with a fly speed has the ability to fly through the air.
Its speed is the distance it covers in a single \glossterm{move action}.
For details, see \pcref{Flying}.

\glossdef{forced movement} A forced movement ability can cause a creature to move unwillingly.
There are two types of forced movement: \glossterm{knockback} and \glossterm{push}.
Although \glossterm{teleportation} can cause a creature's location to change unwillingly, it is not considered a type of forced movement.

\glossdef{free action}[free actions] Each round, you take can any number of free actions.
Free actions can be taken in any phase.
For details, see \pcref{Free Actions}.

\glossdef{free hand} A free hand is a hand or similarly dexterous appendage that is not currently being used for any purpose.
Many abilities require a free hand to use.
You cannot use the same hand for two different purposes in the same \glossterm{phase}.

\glossdef{frightened} See \pcref{Circumstances and Debuffs}.

\glossdef{glancing blow} When you miss on an attack by 2 or less, it is called a glancing blow.
Some attacks have effects when you get a glancing blow, as indicated in their descriptions or in other abilities.
In addition, whenever you get a glancing blow with a damaging attack, you roll no damage dice.
You still add your power to the attack.
For details, see \pcref{Glancing Blows}.

\glossdef{glide speed} A creature with a glide speed can glide through the air.
It cannot fly upwards, but it can travel forward while it descends, and it descends at a significantly reduced rate.
For details, see \pcref{Gliding}.

\glossdef{goaded} See \pcref{Circumstances and Debuffs}.

\glossdef{grappled} See \pcref{Circumstances and Debuffs}.

\glossdef{heavy undergrowth} A space overrun with thick bushes, vines, and similar natural obstacles has heavy undergrowth.
Heavy undergrowth quadruples the movement cost required to move out of each square and provides \glossterm{concealment}.

\glossdef{heavy weapon} A heavy weapon is a type of \glossterm{weapon} that requires two hands to wield properly.
For details, see \pcref{Weapon Usage Classes}.

\glossdef{heavyweight} A heavyweight object has a \glossterm{weight category} that is one category larger than the object's \glossterm{size category}.
For details, see \pcref{Weight Categories}.

\glossdef{helpless} See \pcref{Circumstances and Debuffs}.

\glossdef{hidden task}[hidden tasks] Any checks for a hidden \glossterm{task} should be rolled secretly by the GM.\@
You should not know the result of your check, or even that a check was made.
For details, see \pcref{Hidden Tasks}.

\glossdef{hit point} Your hit points measure how hard you are to seriously injure or kill.
You lose hit points when you take damage.
If you run out of hit points, you gain \glossterm{vital wounds} when you take damage instead, which can cause you to die quickly.
For details, see \pcref{Hit Points}.

\glossdef{immobilized} See \pcref{Circumstances and Debuffs}.

\glossdef{impervious} A creature can be impervious to a particular damage type.
It gains a \plus5 bonus to all defenses against attacks that would cause it to take damage of that type.
If an attack deals damage of multiple types, a creature is impervious to that attack only if it is impervious to all of the attack's damage types.
For attacks with random effects, such as the \spell{chromatic orb} spell, determine the random effect before determining if the creature is impervious.
An impervious creature gains no defensive benefit against attacks that do not deal damage.

\glossdef{improvised weapon}[improvised weapons] An improvised weapon is an object which could conceivably be used as a weapon, but which was not designed for that purpose.
Common examples include doors and wine bottles.
For details, see \pcref{Improvised Weapons}.

\glossdef{incorporeal} An incorporeal creature does not have a tangible body.
It is \glossterm{immune} to \glossterm{physical damage}.
It moves silently and ignores the effects of abilities that only work if it has a corporeal body, such as \glossterm{difficult terrain} and the \textit{grapple} or \textit{shove} abilities.
This includes being \grappled, detected by \glossterm{tremorsense}, setting off pressure plates, and so on.

Many incorporeal creatures have no Strength attribute.
If an incorporeal creature has a Strength attribute, it has some ability to manipulate the physical world despite being incorporeal.
Unless otherwise noted, an incorporeal creature with a Strength attribute may selectively choose whether it wants to interact with physical objects.

An incorporeal creature can enter or pass through solid objects, but it must remain adjacent to the object's exterior at all times.
If it is completely inside an object, it cannot see out or attack.
It can fight while partially inside an object, which grants it \glossterm{cover} and allows it to attack and see normallly.

\glossdef{initiative} When multiple creatures take mutually impossible actions simultaneously, such as racing to be the first one to a door, they must roll initiative checks to determine who completes the action first.
For details, see \pcref{Initiative}.

\glossdef{insight point} Insight points can be spent to gain additional abilities or proficiencies.
For details, see \pcref{Insight Points}.

\glossdef{Intelligence} Intelligence is an \glossterm{attribute} that represents how well you learn and reason.
For details, see \pcref{Intelligence}.

\glossdef{invisible} An invisible creature or object cannot be seen.
Creatures unable to see an invisible creature are at least \partiallyunaware of its attacks, and they can be fully \unaware as normal depending on their level of awareness.
Attackers suffer a 50\% miss chance even if they know the location of the invisible creature.
See \pcref{Awareness}, and \pcref{Stealth}, for how to identify invisible creatures.

\glossdef{item rank} Items have ranks indicating their approximate value and rarity.
For details, see \pcref{Item Ranks}.

\glossdef{key attribute} The key attribute for a skill is the attribute associated with that skill.
For example, Climb is a Strength-based skill.
Some skills, such as Persuasion, do not have a key attribute.

\glossdef{knockback}[knocked back] Knockback is a type of \glossterm{forced movement}.
It represents being thrown backwards by a single large impact.
If a creature or object being knocked back encounters an obstacle, it and the obstacle each take 1d6 bludgeoning \glossterm{environmental damage} per 10 feet of movement remaining, up to a maximum of 20d6 damage.

\glossdef{land speed} A creature's land speed is a \glossterm{movement mode} that determines how fast it can walk on land.
For details, see \pcref{Movement Modes}.

\glossdef{legacy item} A legacy item is an item magically bonded to its bearer.
As its bearer gains levels, it increases in power as well.
For details, see \pcref{Legacy Items}.

\glossdef{lifesense} A creature with lifesense can sense the location of living creatures nearby, regardless of light levels.
For details, see \pcref{Lifesense}.

\glossdef{lifesight} A creature with lifesight see living creatures perfectly without any light.
For details, see \pcref{Lifesight}.

\glossdef{light undergrowth} A space with passable bushes, vines, and similar natural obstacles has light undergrowth.
Light undergrowth is \glossterm{difficult terrain} and provides \glossterm{concealment}.

\glossdef{light weapon}[light weapons] A light weapon is a type of \glossterm{weapon} that is relatively small and easy to use.
For details, see \pcref{Weapon Usage Classes}.

\glossdef{lightweight} A lightweight object has a \glossterm{weight category} that is one category smaller than the object's \glossterm{size category}.
For details, see \pcref{Weight Capacity}.

\glossdef{line} A line is an area shape that an ability can have (see \pcref{Area Shapes}).
A line-shaped area has a given length, width, and height.
Unless otherwise stated, a line's height is equal to its width.

\glossdef{line of effect} You cannot target something that you do not have line of effect to.
Line of effect is blocked by solid obstacles, even invisible ones.
For details, see \pcref{Line of Effect}.

\glossdef{line of sight} You cannot target something that you do not have line of sight to.
Line of sight is blocked by any obstacle that blocks sight, even if that obstacle does not block physical passage.
For details, see \pcref{Line of Sight}.

\glossdef{long range} Weapons have two \glossterm{range limits}: \glossterm{close range} and long range.
Attacks beyond a weapon's \glossterm{close range}, but within its long range, have a \minus4 \glossterm{longshot penalty}.
For details, see \pcref{Weapon Range Limits}.

\glossdef{long rest} A long rest represents eight hours of relaxation or sleep.
It allows you to remove all of your \glossterm{fatigue levels} and make progress towards healing a \glossterm{vital wound}.
For details, see \pcref{Long Rest}.

\glossdef{longshot penalty} A longshot penalty is the penalty that you take for attacking outside of a weapon's \glossterm{close range}.
It is normally a \minus4 \glossterm{accuracy} penalty.
For details, see \pcref{Weapon Range Limits}.

\glossdef{low-light vision} A creature with low-light vision can see perfectly in \glossterm{shadowy illumination}.
For details, see \pcref{Low-light Vision}.

\glossdef{magic bonus}[magic bonuses] Some abilities provide a magic bonus instead of a regular bonus.
Magic bonuses function like normal bonuses except that they do not stack with each other, even if the magic bonuses come from different sources.
For details, see \pcref{Stacking Rules}.

\glossdef{magic source}[magic sources] A magic source defines where a creature's \glossterm{mystic spheres} come from.
There are four magic sources: arcane, divine, nature, and pact.
Sorcerers and wizards cast arcane spells, clerics and paladins cast divine spells, druids cast nature spells, and warlocks cast pact spells.

\glossdef{magical}[Magical] A magical ability is an ability whose origin derives from magic.
Examples include \glossterm{spells}, a dragon's ability to fly, and a paladin's ability to smite foes.
For details, see \pcref{Magical Abilities}.

\glossdef{maneuver} A maneuver is a type \glossterm{mundane} ability that some classes grant access to through particular combat styles.
For details, see \pcref{Combat Styles}.

\glossdef{maneuverability} While flying, your maneuverability determines how easily you can change directions and perform aerial feats.
There are three types of maneuverability: good, average, and poor.
Unless otherwise stated, a creature with a fly speed has aveage maneuverability.
For details, see \pcref{Flying Maneuverability}.

\glossdef{Manifestation} See \pcref{Ability Tags}.

% TODO: ``medium'' is an annoying name
\glossdef{medium weapon} A medium weapon is a type of \glossterm{weapon} that can be wielded in either one or two hands.
For details, see \pcref{Weapon Usage Classes}.

\glossdef{melee}[Melee] A melee attack is an attack using your body or a weapon that does not leave your grasp.
You can only make melee attacks against targets within your \glossterm{reach}.

\glossdef{mindless} A mindless creature lacks a normally functioning mind.
Mindless creatures do not have an Intelligence attribute.
They are immune to \abilitytag{Compulsion} and \abilitytag{Emotion} abilities.

\glossdef{minor action}[minor actions] Each round, you can take a single minor action in addition to your other actions that round.
Minor actions can be taken in either the \glossterm{action phase} or the \glossterm{delayed action phase}.
They are declared and resolved simultaneously with any other actions you take during that phase.
For details, see \pcref{Minor Actions}.

\glossdef{miss chance}[miss chances] If you have a miss chance with an \glossterm{attack}, you have a random chance to miss with the attack regardless of the result of your attack roll.
If you have multiple miss chances, only the highest one applies.

\glossdef{move} When you move, you usually travel a distance equal to your speed.
See \pcref{Movement and Positioning}, for details.
For specific actions that involve movement, see \glossterm{move action}.

\glossdef{move action}[move actions] A move action is one of the types of actions you can take each \glossterm{round}.
Abilities that require a move action typically move you around the battlefield, and are usually used in the \glossterm{movement phase}.
For details, see \pcref{Movement and Positioning}.

\glossdef{movement mode}[movement modes] A movement mode is a method of moving from one location to another.
The most common mode is a \glossterm{land speed}.
For details, see \pcref{Movement Modes}.

\glossdef{movement phase} The movement phase is the first of two \glossterm{phases} in a combat \glossterm{round}.
During the movement phase, creatures can \glossterm{move} and take \glossterm{move actions}.
The movement phase is followed by the \glossterm{action phase}.

\glossdef{multiclass} A multiclass character can gain access to \glossterm{archetypes} and other abilities from multiple classes.
For details, see \pcref{Multiclass Characters}.

\glossdef{mundane}[Mundane] Most abilities are considered mundane abilities.
Mundane abilities have some form of natural explanation and do not fundamentally originate from a magical source.
Examples include weapon attacks, a dragon's frightful presence, and a barbarian's rage.
Unless otherwise indicated, all abilities are mundane in nature.

\glossdef{mystic sphere} A mystic sphere is a collection of thematically related magical effects that includes both \glossterm{spells} and \glossterm{rituals}.
For details, see \pcref{Mystic Spheres}.

\glossdef{natural weapon} A natural weapon is a \glossterm{weapon} that is part of a creature's body.
For details, see \pcref{Natural Weapons}.

\glossdef{neutral party}[neutral parties] A neutral party is any creature who is neither an \glossterm{ally} nor an \glossterm{enemy}.
For details, see \pcref{Allies and Enemies}.

\glossdef{opposed alignment} Each \glossterm{alignment} has an opposed alignment that is antethical to its principles and goals.
Good and Evil are opposed alignments, and Chaos and Law are opposed alignments.
For details, see \pcref{Alignment}.

\glossdef{overrun} An overrun is a special movement that allows you to move directly through creatures.
For details, see \pcref{Overrun}.

\glossdef{panicked} See \pcref{Circumstances and Debuffs}.

\glossdef{paralyzed} See \pcref{Circumstances and Debuffs}.

\glossdef{partially unaware} See \pcref{Circumstances and Debuffs}.

\glossdef{Perception} Perception is an \glossterm{attribute} that describes your ability to observe and be aware of your surroundings.
For details, see \pcref{Perception}.

\glossdef{phase}[phases] A phase is part of the combat \glossterm{round}.
There are two phases: the \glossterm{movement phase} and the \glossterm{action phase}.
A phase does not represent a fixed span of time.
It is an abstract concept designed to represent a variety of actions that all take place nearly simultaneously.

\glossdef{physical damage} There are three types of physical damage: bludgeoning, piercing, and slashing.
For details, see \pcref{Damage Types}.

\glossdef{planar rift}[planar rifts] A planar rift is a location where the boundaries between planes are unusually thin.
Planar rifts can be used to travel between planes using the appropriate rituals.
For details, see \pcref{Planar Rifts}.

\glossdef{plane} A plane is a distinct realm of existence.
Except for the connections between planes through \glossterm{planar rifts}, each plane is effectively an isolated universe, and different planes can obey different fundamental laws.
For details, see \pcref{Planes}.

\glossdef{planeforged} Planeforged creatures are entirely composed of planar material from a single plane.
For details, see \pcref{Planeforged}, and \pcref{Planes}.

\glossdef{point of origin}[points of origin] A point of origin is the grid intersection, creature, or object that an area originates from.
For details, see \pcref{Area}.

% TODO: better short description
\glossdef{poison}[poisoned] For a description of poisons and how they work, see \pcref{Poison}.

\glossdef{poison stage} Each \glossterm{poison} progresses in a series of stages.
Each stage inflicts a particular negative effect on the poisoned creature according to the poison's description.
For details, see \pcref{Poison}.

\glossdef{potion} A potion is a magical liquid that is typically contained in a Fine vial.
In general, drinking a potion requires a standard action.
Potions cannot be safely mixed together without diluting their magic, so you cannot consume two potions with the same action.

\glossdef{power} The power of an \glossterm{ability} represents how strong the ability is.
Many abilities add your \glossterm{power} to the damage they deal, and it may also determine other effects of the ability.
Your power with an ability depends on whether the ability is \glossterm{magical} or \glossterm{mundane}.
Your power with magical abilities, or your magical power, is normally equal to half your Willpower.
Similarly, your power with mundane abilities, or your mundane power, is normally equal to half your Strength.
For details, see \pcref{Power}.

\glossdef{proficient}[proficiency] A creature can be proficient with weapons and armor.
If you try to attack with a weapon you are not proficient with, you take a \minus2 accuracy penalty (see \pcref{Weapon Proficiency}).
If you try to use armor you are not proficient with, it is less effective and your \glossterm{accuracy} is reduced (see \pcref{Armor Proficiency}).

\glossdef{projectile} A projectile is an object fired from a weapon at a target.
Arrows and bolts are projectiles.

\glossdef{projectile weapon} A projectile weapon is a weapon designed to fire \glossterm{projectiles}.
For details about how to attack with projectile weapons, see \pcref{Projectile Strike}.

\glossdef{prone} See \pcref{Circumstances and Debuffs}.

\glossdef{push}[pushed] A push is a type of \glossterm{forced movement}.
It represents being pushed by a constant force.
If a creature being pushed encounters an obstacle, it stops moving with no negative consequences.

\glossdef{range} The range of an ability determines how far away it can be used.
Unless otherwise noted, all abilities with a range require both \glossterm{line of sight} and \glossterm{line of effect} to the point of origin or to all targets.
For details, see \pcref{Range}.

\glossdef{range limits} Ranged weapons have two \glossterm{range limits} listed, with a slash between them, such as 120/480.
The first number indicates the maximum range for a weapon's \glossterm{close range}.
The second number indicates the maximum range for a weapon's \glossterm{long range}.
For details, see \pcref{Weapon Range Limits}.

\glossdef{rank} Spells and rituals have a rank.
The rank defines the minimum \glossterm{archetype rank} you must have to learn and use them.

\glossdef{rare language}[rare languages] Rare languages are languages that are only spoken by rare or distant creatures or cultures.
They are described in \tref{Rare Languages}.

\glossdef{reach} Your reach is how far away from your body you can make \glossterm{melee} attacks.
A typical Medium creature has a five-foot reach.
Long weapons may change your reach (see \pcref{Weapons}).

\glossdef{resource} A resource is something that a character can lose during play or expend to gain a benefit.
Most resources are shared between all types of characters, though different characters can use them differently.
There are two resources that are used during the character creation and leveling process: \glossterm{insight points} and \glossterm{trained skills}.
In addition, there are are five resources that are used during gameplay: \glossterm{attunement points}, \glossterm{damage resistance}, \glossterm{fatigue level}, \glossterm{hit points}, and \glossterm{vital wounds}.

\glossdef{ritual}[rituals] A ritual is a discrete \glossterm{magical} ability with esoteric effects.
For details, see \pcref{Rituals}.

\glossdef{round}[rounds] Combat takes place in a series of rounds, which represent about six seconds of action.
Rounds are divided into two \glossterm{phases}: the \glossterm{movement phase}, and the \glossterm{action phase}.

\glossdef{scent}[Scent] A creature with the scent ability has an unusually good sense of smell.
It gains a \plus10 bonus to scent-based Awareness checks (see \pcref{Senses}).

\glossdef{Scrying} See \pcref{Ability Tags}.

\glossdef{scrying sensor} A scrying sensor is a magical construct created by some magical abilities.
Scrying sensors are Fine objects resembling a human eye in size and shape, though they are \glossterm{invisible}.
Scrying sensors typically float in a fixed position in the air.
They cannot normally be moved by external forces without destroying the sensor.
Unless otherwise specified, a scrying sensor's visual acuity is the same as that of a normal human, giving it a \plus0 bonus to the Awareness skill and similar checks.

\glossdef{Sensation} See \pcref{Ability Tags}.

\glossdef{shadowy illumination} In an area with shadowy illumination, creatures can see dimly.
Creatures and objects within this area have \glossterm{concealment}, which can allow creatures to make Stealth checks to hide (see \pcref{Stealth}).
For details, see \pcref{Vision and Light}.

\glossdef{shaken} See \pcref{Circumstances and Debuffs}.

\glossdef{Shaping} See \pcref{Ability Tags}.

\glossdef{shield}[shields] Shields are a form of \glossterm{armor} that you wield in a hand to protect you from harm.
For details, see \pcref{Armor}.

\glossdef{Shielding} See \pcref{Ability Tags}.

\glossdef{short rest}[short rests] A short rest represents ten minutes of relaxation.
It allows you to regain lost \glossterm{hit points} and any \glossterm{attunement points} you released from \glossterm{attunement}.
For details, see \pcref{Short Rest}.

\glossdef{shove} You can use the \textit{shove} ability to forcibly move a creature.
For details, see \pcref{Shove}.

\glossdef{size category}[size categories] A creature's size category indicates how large it is.
There are nine size categories, from smallest to largest: Fine, Diminuitive, Tiny, Small, Medium, Large, Huge, Gargantuan, Colossal.
For details, see \pcref{Size in Combat}.

\glossdef{Sizing} See \pcref{Ability Tags}.

\glossdef{skill}[skills] A skill represents your degree of talent with a particular non-combat aspect of the world.
For example, the Climb skill represents how skilled you are at climbing.
For details, see \pcref{Skills}.

\glossdef{slowed} See \pcref{Circumstances and Debuffs}.

\glossdef{somatic components}[somatic] Somatic components are hand motions required to cast arcane and pact spells.
For details, see \pcref{Casting Components}.

\glossdef{somatic component failure} If you have any \glossterm{encumbrance}, you may fail to successfully perform the intricate gestures required to cast spells with \glossterm{somatic components}.
For details, see \pcref{Somatic Component Failure}.

\glossdef{space} Your space is the area that your physical body occupies.
For convenience, your space is measured in five-foot \glossterm{squares}.
Medium creatures occupy space equal to a single five-foot square.
For details, see \pcref{Size in Combat}.

\glossdef{Speech} See \pcref{Ability Tags}.

\glossdef{speed} Your speed represents the number of feet you can move with a single movement (see \pcref{The Movement Phase}).

\glossdef{spell}[Spell] A spell is a disrete \glossterm{magical} ability with combat-relevant effects.
For details, see \pcref{Spells}.

\glossdef{spell list} The list of spells you can cast from a particular \glossterm{magic source}.
Each spell source has a specific spell list which is described at \pcref{Spells}.
Most characters with the same spell sources have the same spell lists.
However, some effects, such as a cleric's domains, can add spells to a character's individual spell list.

\glossdef{square}[squares] A square represents a single 5-ft.\ by 5-ft.\ space.
Many areas are measured in squares for convenience.

\glossdef{squeezing} See \pcref{Circumstances and Debuffs}.

\glossdef{standard action}[standard actions] You can use a standard action to attack with a weapon, cast a spell, and do most other things that take concentration and effort.

\glossdef{Strength} Strength is an \glossterm{attribute} that measures your muscle and physical power.
For details, see \pcref{Strength}.

\glossdef{strike}[strikes] A strike is a single physical attack with a weapon.
It is the most common type of attack.
You can make a strike as a \glossterm{standard action} in the \glossterm{action phase}.
For details, see \pcref{Strikes}.

\glossdef{stunned} See \pcref{Circumstances and Debuffs}.

\glossdef{subdual damage} Subdual damage is a special kind of damage that can't kill you.
If you would gain a \glossterm{vital wound} from subdual damage, you increase your \glossterm{fatigue level} by three instead.
For details, see \pcref{Subdual Damage}.

\glossdef{Subtle} See \pcref{Ability Tags}.

\glossdef{suppressed}[suppress] A suppressed ability has temporarily ceased to function.
It has no effect for as long as it remains suppressed.
Time spent while suppressed counts against the ability's duration, and it may expire while suppressed if it lasts for a specific amount of time.
Only \glossterm{magical} abilities can be suppressed.
Mundane results of magical abilities that have already occured, such as the water created by a \ritual{create water} ritual, cannot themselves be suppressed, and do not disappear if they enter an area that suppresses magical abilities.

\glossdef{sustain}[sustained] Some abilities last as long as you sustain them.
Each ability specifies a particular action that is required to sustain the ability, such as a \glossterm{minor action}.
At the end of each round the ability is dismissed unless you used the ability that phase or took the action to sustain the ability that phase.
For details, see \pcref{Sustained Abilities}.

\glossdef{Swift} An ability with this \glossterm{ability tag} resolves its effects before other actions in the same phase.
For details, see \pcref{Swift Abilities}.

\glossdef{swim speed} A creature with a swim speed can swim as easily as a human walks on land.
The effects of a swim speed are described at \pcref{Swim Speed}.

\glossdef{take 10}[taking 10] If you have plenty of time to accomplish a task that requires a \glossterm{check}, and there are no meaningful consequences for failure, you can take 10 to accomplish the task.
If you do, the task takes ten times as long, but you treat your roll for the check as if you had rolled a 10.
For details, see \pcref{Taking 10}.

\glossdef{target} A target is a creature or object directly affected by an ability.
Many abilities only affect a single target, and some affect a specific number of targets.

\glossdef{target square} A target square is a particular \glossterm{square} that an attack is made against.
A target square is chosen to determine \glossterm{cover} and \glossterm{concealment} (see \pcref{Cover}).

\glossdef{targeted}[Targeted] A \glossterm{targeted} ability is an ability that allows you to directly choose which targets the ability affects.
A spell that affects an area is not a targeted ability, because you choose the area affected instead of choosing the targets directly.

\glossdef{task} A task is a particular way to use a \glossterm{skill}.
For example, balancing on slippery ground is a task that you can use the Balance skill for (see \pcref{Balance}).
For details, see \pcref{Tasks}.

\glossdef{telepathy} A creature with telepathy can mentally communicate with other creatures within a given range.
For details, see \pcref{Telepathy}.

\glossdef{teleportation}[teleport] A creature or object that is teleported instantly leaves one location and arrives at another.
For deatils, see \pcref{Teleportation}.

\glossdef{thrown weapon}[thrown weapons] A thrown weapon is a weapon designed to be thrown at a target.
For details about attacking with thrown weapons, see \pcref{Thrown Strike}.

\glossdef{total cover} Total cover is a type of \glossterm{cover}.
If a creature is completely behind a physical object that blocks sight, it has \glossterm{total cover} from attacks.
A creature with total cover cannot be targeted by any attacks.
For details, see \pcref{Total Cover}.

\glossdef{trained} If you are trained in a \glossterm{skill}, you have learned how to use it well.
Your modifier with a trained skill is equal to 3 \add the higher of its associated attribute (if any) and half your level.
For details, see \pcref{Trained Skills}.

\glossdef{tremorsense} A creature with tremorsense can sense the location of everything nearby, as long as there is a path through solid objects.
For details, see \pcref{Tremorsense}.

\glossdef{tremorsight} A creature with tremorsight see perfectly without any light, as long as there is a path through solid objects.
For details, see \pcref{Tremorsight}.

\glossdef{unarmed attack}[unarmed attacks] Every corporeal creature is capable of making an attack using its bare fists (or similar appendages).
For details, see \pcref{Unarmed Combat}.

\glossdef{unattended} An unattended item is an item not being held or carried by a creature, or that is being held or carried by an \glossterm{ally}.
Some abilities can only affect unattended items.

\glossdef{unaware} See \pcref{Circumstances and Debuffs}.

\glossdef{undead} Undead creatures are animated by necromantic magic.
For details, see \pcref{Undead}.

\glossdef{unconscious} See \pcref{Circumstances and Debuffs}.

\glossdef{undergrowth} The presence of a significant amount of roots, bushes, and similar plants that can obstruct movement is called undergrowth.
There are two kinds of undergrowth: \glossterm{light undergrowth} and \glossterm{heavy undergrowth}.
For details, see \pcref{Undergrowth}.

\glossdef{untrained} If you are untrained in a \glossterm{skill}, you have no particular ability with it.
You are untrained with all skills by default.
Your modifier with a trained skill is equal to half of its associated attribute (if any).
For details, see \pcref{Trained Skills}.

\glossdef{usage class} The \glossterm{usage class} of a weapon or armor is a measure of how much effort it takes to use the item.
For details, see \pcref{Weapon Usage Classes} and \pcref{Armor Usage Classes}.

\glossdef{verbal components}[verbal] Verbal components are words required to cast most spells.
For details, see \pcref{Casting Components}.

\glossdef{Visual} See \pcref{Ability Tags}.

\glossdef{vital wound} A \glossterm{vital wound} is a serious injury that inflicts negative effects on you.
You gain one or more \glossterm{vital wounds} when you take damage in excess of your hit points (see \pcref{Reaching Zero Hit Points}).
For details, see \pcref{Vital Wounds}.

\glossdef{vulnerable} A creature can be vulnerable to a particular damage type or debuff.
It takes a \minus5 penalty to all defenses against attacks that would cause it to take damage of that type, or that would cause it suffer that debuff.
This penalty applies against the whole attack even if the attack would only inflict the debuff or damage under specific circumstances, such as if the attack gets a critical hit or if the attack causes the creature to lose hit points.
For attacks with random effects, such as the \spell{chromatic orb} spell, determine the random effect before determining if the creature is vulnerable.

\glossdef{wall} A wall is an area shape that an ability can have (see \pcref{Area Shapes}).
A wall-shaped area has a length and height, but its width is not measured in squares.

\glossdef{weapon}[weapons] A weapon is an object used to inflict damage.
Some creatures can treat parts of their body as weapons.
For details, see \pcref{Weapons}.

\glossdef{weapon group} A weapon group is a category of \glossterm{weapons} with a similar design and fighting style.
You have proficiency with some number of weapon groups based on your \glossterm{class}.
For details, see \pcref{Weapon Groups}.

\glossdef{weapon tag} A weapon tag describes the special effects of a weapon.
For details, see \pcref{Weapon Tags}.

\glossdef{weight limit} Your weight limits define the amount of weight you can carry or push without penalty.
For details, see \pcref{Weight Limits}.

\glossdef{weight category}[weight categories] The weight category of an object or creature is a broad measurement of how much it weighs.
Weight categories are closely related to \glossterm{size categories}.
For details, see \tref{Weight Categories}.

\glossdef{Willpower} Willpower is an \glossterm{attribute} that represents your ability to endure mental hardships.
For details, see \pcref{Willpower}.

\glossdef{vital roll} When you gain a \glossterm{vital wound}, you make a \glossterm{vital roll} to determine the detrimental effect of the \glossterm{vital wound}.
To make a \glossterm{vital roll}, roll 1d10 \sub the number of \glossterm{vital wounds} you already had, ignoring the vital wound you are rolling for.
For details, see \pcref{Vital Wounds}.

\glossdef{zone} A zone is a type of area that an ability can have (see \pcref{Area Types}).
A zone ability has effects within an area for the duration of the ability.
Unless otherwise noted, it does not move after being created.
