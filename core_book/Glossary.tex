\chapter{Glossary}\label{Glossary}

\glossdef{ability}[abilities] An ability is a generic term for any special action a creature can perform or effect that a creature or object can cause. Spells, feats, and class abilities can all be called abilities.

\glossdef{ability tag}[ability tags] An ability tag describes the effects of an ability.
For details, see \pcref{Ability Tags}.

\glossdef{acid}[Acid] A type of damage. Acid damage is very effective against most objects. For the Acid spell tag, see \pcref{Ability Tags}.

\glossdef{action phase} The action phase is the second of two \glossterm{phases} in a combat \glossterm{round}.
During the action phase, creatures can \glossterm{attack}, cast \glossterm{spells}, and take other major combat actions.

\glossdef{accuracy} The bonus added to a \glossterm{attack roll}.

\glossdef{Air} See \pcref{Ability Tags}.

\glossdef{attack}[attacks] Anything that affects another creature in a potentially harmful way. There are two kinds of attacks: \glossterm{physical} attacks and \glossterm{magical} attacks.

\glossdef{attack roll}[attack rolls] A roll required to succeed with an attack.
To make an attack roll, roll 1d20 \add your \glossterm{accuracy} with the attack.
If the result of the attack roll equals or exceeds the target's \glossterm{defense}, the attack succeeds.
Some attacks, especially magical attacks, have effects even if the attack roll fails.

\glossdef{attribute} A core representation of a character's capacity in a wide range of areas. There are six attributes: \glossterm{Strength}, \glossterm{Dexterity}, \glossterm{Constitution}, \glossterm{Intelligence}, \glossterm{Perception}, and \glossterm{Willpower}.

\glossdef{Auditory} See \pcref{Ability Tags}.

\glossdef{augment}[augments] Many spells have augments.
Each augment on a spell has a level and an effect.
When casting a spell, you add the augment's level to the spell's level.
This affects the spell slot required to cast the spell, and similar effects.
If you do, the spell gains the effect of the augment.
You can apply any number of augments to a spell in this way, increasing the spell's level for each augment.

\glossdef{Barrier} See \pcref{Ability Tags}.

\glossdef{base bonus} A base bonus is the value of a bonus before any temporary modifiers or effects. For example, your base Fortitude defense bonus is equal to the Fortitude defense bonus granted by your class.

\glossdef{base class} Every character has one base class.
You may choose any class you have at least one level in as a base class.
Whenever you gain a level, you can change your base class to a different class you have.
Your choice of base class affects your \glossterm{combat prowess}, \glossterm{defenses}, \glossterm{skill points}, and \glossterm{class skills}.
In addition, every class grants you special abilities if you choose that class as a base class, as given in the class description.

\glossdef{blindsense} A creature with blindsense can sense its surroundings without any light, regardless of concealment or invisiblity.
It knows the location of everything around it, but it still takes normal failure chances for concealment, invisibility, and so on.
It still needs line of effect to see its surroundings.
Blindsense always has a range, and grants no benefits beyond that range.

\glossdef{blindsight} A creature with blindsight can ``see'' its surroundings perfectly without any light, regardless of concealment or invisibility.
It still needs line of effect to see its surroundings.
Blindsight always has a range, and grants no benefits beyond that range.

\glossdef{broken} A broken object is damaged and unsuitable for use, though it retains its general structure. Both magical and mundane objects can be repaired for a cost equal to 10\% of their value. You must be able to craft the item originally to repair it.

An object that reaches 0 hit points is broken. If an object takes additional damage equal to its maximum hit points, it is destroyed. A destroyed object cannot be repaired by any means.

\glossdef{burst} A burst is a type of area that an ability can have (see \pcref{Area Types}).
A burst ability has an immediate effect on all valid targets within an area.

\glossdef{charge}[charging] Charging is a combat action that consists of running directly at a foe to attack it.
It is described at \pcref{Charge}.

\glossdef{Charm} See \pcref{Ability Tags}.

\glossdef{chaotic}[Chaotic] Relating to chaos, one of the four \glossterm{alignment} components. For the Chaotic spell tag, see \pcref{Ability Tags}.

\glossdef{character level} Your character level is your total level, including levels from all of your classes.
Whenever text refers to your ``level'', without specifying a particular kind of level, it means your character level.

\glossdef{class skill}[class skills] A class skill is a skill which you can train with using \glossterm{skill points} from your class. For details, see \pcref{Skill Training}.

\glossdef{climb speed} A creature with a climb speed can climb as easily as a human walks on land.
The effects of a climb speed are described at \pcref{Climb Speed}.

\glossdef{cold}[Cold] A kind of \glossterm{energy}. For the Cold spell tag, see \pcref{Ability Tags}.

\glossdef{combat maneuver}[combat maneuvers] A combat maneuver is an attempt to physically hinder your foe with your body, such as by shoving or tripping it.
Most combat maneuvers are made in place of a \glossterm{strike}.

\glossdef{combat prowess} Your combat prowess represents how skilled you are in physical combat.
You can add your combat prowess to your accuracy and damage with physical attacks (see \pcref{Physical Accuracy}, and \pcref{Physical Damage}).
You can also add your combat prowess to your physical defenses (see \pcref{Defense Values}).
In addition, your combat prowess may grant you additional \glossterm{strikes} during a round (see \pcref{Multiple Attacks}).

\glossdef{common language}[common languages] Common languages are languages that are widely spoken.
They are described in \tref{Common Languages}.

\glossdef{Compulsion}[Compulsions] See \pcref{Ability Tags}.

\glossdef{concealment} Concealment represents effects which make a target harder to see, such as shadowy lighting.
You take a \minus4 penalty to accuracy with physical attacks against creatures and objects that have concealment from you.

\glossdef{coup de grace} A coup de grace is a powerful attack that you can use on \helpless creatures.
It requires a full-round action, but can instantly kill the target.
For details, see \pcref{Coup de Grace}.

\glossdef{cover} Cover represents any obstacle that physically prevents you from striking your target, such as a tree or intervening creature.
There are three kinds of cover: \glossterm{active cover}, \glossterm{passive cover}, and \glossterm{total cover}.
For details, see \pcref{Cover}.

\glossdef{Creation} See \pcref{Ability Tags}.

\glossdef{Curse} See \pcref{Ability Tags}.

\glossdef{critical multiplier}[critical multipliers] Your critical multiplier is the multiplier you add to your damage when you score a critical hit.
Your critical multiplier is normally 2, which means your critical hits deal double damage.

\glossdef{critical range}[critical ranges] Your critical range is the number of die rolls that you can score a critical hit on.
Your critical range is normally 1, which means you score a critical hit by rolling a 20.

\glossdef{critical failure} Critical failure on a check or attack means your result failed to beat the DR by 10 or more.
Some skills and abilities have special effects on critical failures.

\glossdef{critical success} Critical success on a check or attack means your result beat the DR by 10 or more.
Some skills and abilities have special effects on critical successes.

\glossdef{damage reduction} Damage reduction allows you to ignore a certain amount of incoming damage.
Each \glossterm{round}, you ignore the first points of damage you would take.
Damage reduction always specifies an amount of damage it reduces.
Once it reduces that much damage, it stops functioning until the end of the round.

Most sources of damage reduction only apply against a specific type of attack.
For example, a barbarian's damage reduction only applies against physical damage.
If an attack deals multiple types of damage, you must have damage reduction against every type of damage dealt.
For example, damage reduction against piercing damage would not help if you are struck by a morningstar, since it deals both bludgeoning and piercing damage.

Many sources of damage reduction can be ignored and negated by a specific type of attack.
For example, the \spell{barkskin} spell grants damage reduction that can be ignored and negated by fire and slashing damage.
If you are hit an attack that negates your damage reduction, you cannot apply your damage reduction against any other attacks that round.
This includes other attacks that resolve simultaneously, but not attacks that resolved earlier in the round.
For example, if you had the \spell{barkskin} spell active, and you were hit by a club (bludgeoning damage) and a longsword (slashing damage), you would take full damage both attacks.
However, if you were instead hit by a club and a \spell{fireball} spell (fire damage), you would reduce the damage from the club, because the spell resolves later in the round.

\glossdef{darkvision} A creature with darkvision can see in the dark clearly up to a given range.
Beyond that, it can see dimly, treating areas of darkness as shadowy illumination.
Darkvision does not function if a creature is in a brightly lit area, and does not resume functioning until 1 round after the creature leaves the brightly lit area.

\glossdef{Death} See \pcref{Ability Tags}.

\glossdef{defense}[defenses] A defense is a static number which represents how difficult you are to affect with attacks. See \glossterm{attack rolls}.

\glossdef{Delusion} See \pcref{Ability Tags}.

\glossdef{Detection} See \pcref{Ability Tags}.

\glossdef{difficult terrain} Difficult terrain costs double the normal movement cost to move out of.
For details, see \pcref{Difficult Terrain}.

\glossdef{Difficulty Rating} The Difficulty Rating of a \glossterm{check} is the check result required to succeed.
In general, attacks are rolled to beat \glossterm{defenses}, and checks are rolled to beat Difficulty Ratings.

\glossdef{dirty trick} A dirty trick is a light \glossterm{combat maneuver} that allows you to impair a foe with your environment.
For details, see \pcref{Dirty Trick}.

\glossdef{disarm} A disarm is a light \glossterm{combat maneuver} that allows you to strike items held or worn by a creature.
For details, see \pcref{Disarm}.

\glossdef{disease} An affliction of the body, causing a steady deterioration over time. For the Disease spell tag, see \pcref{Ability Tags}.

\glossdef{Earth} See \pcref{Ability Tags}.

\glossdef{effect} The result of using an \glossterm{ability}.

\glossdef{electricity}[Electricity] A kind of \glossterm{energy}. For the Electricity spell tag, see \pcref{Ability Tags}.

\glossdef{emanation} An emanation is a type of area that an ability can have (see \pcref{Area Types}).
An emanation ability has effects within an area for the duration of the ability.
It emanates from a specific creature or object, rather than a location.
If that creature or object moves, the emanation moves with it.

\glossdef{encumbrance penalty} A character's encumbrance penalty applies to all Strength and Dexterity-based skill checks the character makes.
A character can acquire an encumbrance penalty by wearing armor or by carrying an excessive weight (see \tref{Weight Limits}).

\glossdef{energy}[energy damage] There are four types of energy: cold, electricity, fire, and sonic. Energy effects often deal damage.

\glossdef{Enchantment} Enchantment spells affect the minds of others, influencing or controlling their behavior or mental capabilities. Almost all enchantment spells are \glossterm{Mind} spells, and many of them are \glossterm{Subtle} as well.

\glossdef{enhancement bonus} Magic armor and weapons can have enhancement bonuses.
Each \plus1 of enhancement bonus on magic armor grants temporary hit points equal the item's power, and grants you an additional defensive legend point each day.

You gain a bonus to damage on physical attacks using a magic weapon equal to the weapon's enhancement bonus.
In addition, each \plus1 of enhancement bonus on a weapon grants you an additional offensive legend point each day.

See \pcref{Armor Enhancement Bonuses} and \pcref{Weapon Enhancement Bonuses} for details.

\glossdef{evil}[Evil] One of the four \glossterm{alignment} components. For the Evil spell tag, see \pcref{Ability Tags}.

\glossdef{falling damage} For every 10 feet you fall, you take 1d6 bludgeoning damage, to a maximum of 20d6 damage.
If you control your fall with a successful Acrobatics or Jump check, you can reduce the falling damage you take (see \pcref{Acrobatics}, and \pcref{Jump}).

\glossdef{fast healing} A creature with fast healing automatically heals hit points at the end of every round.
Like other healing, this healing offsets damage taken during the round for the purposes of taking \glossdef{vital damage} and becoming \disabled.

\glossdef{feint} A feint is a light \glossterm{combat maneuver} that allows you to trick a creature into lowering its defenses.
For details, see \pcref{Feint}.

\glossdef{fire}[Fire] A kind of \glossterm{energy}. For the Fire spell tag, see \pcref{Ability Tags}.

\glossdef{Figment} See \pcref{Ability Tags}.

\glossdef{Flesh} See \pcref{Ability Tags}.

\glossdef{Fog} See \pcref{Ability Tags}.

\glossdef{Force} See \pcref{Ability Tags}.

\glossdef{free action}[free actions] Each round, you can any number of free actions.
Free actions can be taken in any phase.
For details, see \pcref{Free Actions}.

\glossdef{Glamer} See \pcref{Ability Tags}.

\glossdef{good}[Good] One of the four \glossterm{alignment} components. For the Evil spell tag, see \pcref{Ability Tags}.

\glossdef{grapple} A grapple is a heavy \glossterm{combat maneuver} that allows you to physically restrain a creature.
For details, see \pcref{Grapple}.

\glossdef{hardness} An object's hardness indicates how durable it is.
Whenever a creature or object with hardness takes damage, it reduces that damage by an amount equal to its hardness.

\glossdef{hit value} Your hit value is the number of hit points you gain per level.
Your hit value is normally equal to half your Fortitude defense or half your Mental defense, whichever is higher.

\glossdef{heavy maneuver} A heavy maneuver is a type of \glossterm{combat maneuver}.
You can perform a heavy maneuver as a standard action, and you cannot use your Dexterity to determine its accuracy.
For details, see \pcref{Combat Maneuvers}.

\glossdef{Imbuement} See \pcref{Ability Tags}.

\glossdef{immediate action}[immediate actions] Each round, you can take a single swift or immmediate action.
You can take immediate actions at any time, even in the middle of another creature's action.
All immediate actions have a specific triggering condition which allows you to take the action.
For details, see \pcref{Swift and Immediate Actions}.

\glossdef{incorporeal} An incorporeal creature does not have a body.
It has no Strength or Constitution attributes.
It cannot take any action that requires having a body, and is immune to all such effects.
This includes suffering critical hits, moving objects, grappling, setting off pressure traps, and so on.

An incorporeal creature is immune to all nonmagical effects.
Even magical effects, including spells and attacks with magic weapons, have a 50\% chance to fail.

An incorporeal creature can enter or pass through solid objects, but it must remain adjacent to the object's exterior at all times.
If it is completely inside an object, it cannot see out or attack.
It can fight while partially inside an object, which grants it passive \glossterm{cover} and allows it to attack and see normallly.

\glossdef{Instantaneous} See \pcref{Ability Tags}.

\glossdef{item power} An item's power represents how strong its effects are.
See \pcref{Item Power}, for details.

\glossdef{lawful}[Lawful] Relating to law, one of the four \glossterm{alignment} components. For the Lawful spell tag, see \pcref{Ability Tags}.

\glossdef{legend point}[legend points] Legend points can be used to reroll failed rolls, or force your foes to reroll successful rolls against you. See \pcref{Legend Points}, for details.

\glossdef{Life} See \pcref{Ability Tags}.

\glossdef{Light} See \pcref{Ability Tags}.

\glossdef{light maneuver} A light maneuver is a type of \glossterm{combat maneuver}.
You can perform a light maneuver in place of a \glossterm{strike}, and you can use your Dexterity to determine its accuracy.
For details, see \pcref{Combat Maneuvers}.

\glossdef{low-light vision} A creature with low-light vision can see more clearly in conditions of dim light.
It treats sources of light as if they had double their normal illumination range.
In addition, the creature treats environments with ambient dim light, such as a moonlit night, as if they were brightly lit when doing so is beneficial for it.

\glossdef{magic resistance} A creature with magic resistance can automatically resist magical abilities.
It functions like any other defense, except that it only works against magical effects.
To affect a magic resistant creature with a magical ability, you must make an additional magical attack against the creature's magic resistance value.
Your accuracy is equal to your \glossterm{power} with the ability you using, such as your spellpower with spells.
If your attack result beats the creature's magic resistance, the ability works normally.
Otherwise, the ability has no effect on the creature.
For details, see \pcref{Magic Resistance}.

\glossdef{magical} A magical ability is an ability that has no physical explanation.
Examples include spells, a medusa's petrifying gaze, and a cleric's domain invocations.
Magical attacks often target Fortitude and Mental defenses, and can be resisted by \glossterm{magic resistance}.
For details, see \pcref{Magical Abilities}.

\glossdef{Mind} See \pcref{Ability Tags}.

\glossdef{melee attack} A melee attack is a physical \glossterm{attack} against a creature within your \glossterm{reach}.

\glossdef{Morale} See \pcref{Ability Tags}.

\glossdef{move} When you move, you usually travel a distance equal to your speed.
See \pcref{Movement and Positioning}, for details.
For specific actions that involve movement, see \glossterm{move action}.

\glossdef{move action}[move actions] A move action is a minor action that requires motion, such as drawing a sword.
You can take move actions during the \glossterm{movement phase}.
For the act of moving from one place to another, see \glossterm{move}.

\glossdef{movement phase} The movement phase is the first of two \glossterm{phases} in a combat \glossterm{round}.
During the movement phase, creatures can \glossterm{move} and take \glossterm{move actions}.
The movement phase is followed by the \glossterm{action phase}.

\glossdef{Negative} See \pcref{Ability Tags}.

\glossdef{nonlethal damage} Nonlethal damage is a special kind of damage that can't kill you.
A creature that takes too much nonlethal damage falls unconscious.
For details, see \pcref{Nonlethal Damage}.

\glossdef{overkill damage} If you take damage in excess of your \glossterm{bloodied} hit point total in a single round, the excess damage is dealt as \glossterm{vital damage}.
This excess damage is called overkill damage.
For details, see \pcref{Overkill Damage}.

\glossdef{overrun} An overrun is a special movement that allows you to move directly through creatures.
For details, see \pcref{Overrun}.

\glossdef{overwhelmed} An overwhelmed creature is suffering \glossterm{overwhelm penalties}.

\glossdef{overwhelm penalties} A creature \glossterm{threatened} by at least two creatures suffers a penalty to physical defenses (Armor, Manuever, Reflex).
The size of the penalty is equal to the number of creatures threatening it, to a maximum of \minus8.
These penalties are called overwhelm penalties.
A creature suffering overwhelm penalties is \glossterm{overwhelmed}.

\glossdef{outsider} An outsider is a type of creature.
Outsiders are composed of planar material from a plane other than the Material Plane.

\glossdef{phase}[phases] A phase is part of the combat \glossterm{round}.
There are two phases: the \glossterm{movement phase} and the \glossterm{action phase}.
A phase does not represent a fixed span of time.
It is an abstract concept designed to represent a variety of actions that all take place nearly simultaneously.

\glossdef{Physical} See \pcref{Ability Tags}.

\glossdef{physical} A physical ability has a tangible component and some form of natural explanation.
Examples include weapon attacks, a dragon's breath weapon, and a barbarian's rage.
Physical attacks often target Armor and Reflex defenses.
Unless otherwise indicated, all abilities are physical in nature.
For details, see \pcref{Physical Abilities}.

\glossdef{physical defenses}[physical defense] Your physical defenses are your Armor and Reflex defenses.
For details, see \pcref{Defenses}.

\glossdef{Planar} See \pcref{Ability Tags}.

\glossdef{point of origin}[points of origin] A point of origin is the point where an ability originates from.
A point of origin is always a grid intersection, not the center of a square.
Spells and similar magical abilities use points of origin to determine their areas.
In addition, points of origin are used to calculate \glossterm{cover} and \glossterm{concealment} (see \pcref{Cover}).

\glossdef{poison} For a description of poisons and how they work, see \pcref{Poisons}. For the Poison spell tag, see \pcref{Ability Tags}.

\glossdef{Positive} See \pcref{Ability Tags}.

\glossdef{potency} The potency of a poison, disease, or similar effect determines its attack bonus.

\glossdef{power} The power of an \glossterm{ability} represents how strong the ability is.
Each ability uses a particular kind of power, which is usually calculated in a unique way.
For example, spells use \glossterm{spellpower}, and class abilities typically use a power specified in the class description.

\glossdef{random effect}[random effects] Random effects change what they do based on a specific die roll.
This does not include effects which require a successful attack or similar roll.
The \spell{prismatic beam} spell is an example of a random effect.
In addition, the random retargeting of certain miscast spells, such as \spell{scorching ray}, is a random effect.

\glossdef{rage bonus} The bonus a character with the rage ability adds to their damage, Fortitude, Willpower, and more.
For details, see \pcref{Rage}.

\glossdef{range} The range of an ability determines how far away it can be used.
You can't use abilities on a target outside of the ability's range.

\glossdef{range increment} Physical ranged attacks often have a specific range increment.
A range increment is always measured in feet.
You take a \minus2 penalty to accuracy with the ranged attack for each full range increment between you and your target.

\glossdef{rare language}[rare languages] Rare languages are languages that are only spoken by rare or distant creatures or cultures.
They are described in \tref{Rare Languages}.

\glossdef{Retributive} See \pcref{Ability Tags}.

\glossdef{reach} Your reach is how far away from your body you can make melee attacks.
A typical Medium creature has a five-foot reach.

\glossdef{round} Combat takes place in a series of rounds, which represent about six seconds of action.
Rounds are divided into two \glossterm{phases}: the \glossterm{movement phase}, and the \glossterm{action phase}.

\glossdef{scent} A creature with the scent ability has an unusually good sense of smell.
It gains a \plus10 bonus to scent-based Awareness checks (see \pcref{Senses}).

\glossdef{Scrying} See \pcref{Ability Tags}.

\glossdef{Shielding} See \pcref{Ability Tags}.

\glossdef{shove} A shove is a heavy \glossterm{combat maneuver} that allows you to move a creature.
For details, see \pcref{Shove}.

\glossdef{Sizing} See \pcref{Ability Tags}.

\glossterm{skill point}[skill points] You can spend skill points to gain training in skills (see \pcref{Skill Training}).
You gain skill points from your \glossterm{base class}, from having a high Intelligence, and from taking penalties to your starting attributes (see \pcref{Impaired Attributes}).
Skill points from your base class can only be spent on your \glossterm{class skills}, but skill points from any other source can be spent on any skill.
For details, see \pcref{Skill Points}.

\glossdef{somatic components}[somatic] Somatic components are hand motions required to cast most spells.
For details, see \pcref{Components}.

\glossdef{Speech} See \pcref{Ability Tags}.

\glossdef{spell list} The list of spells you can cast from a particular \glossterm{spell source}.
Each spell source has a specific spell list which is described at \pcref{Spells}.
Most characters with the same spell sources have the same spell lists.
However, some effects, such as a cleric's domains, can add spells to a character's individual spell list.

\glossdef{spell source} A spell source defines where a creature's spells come from.
There are three spell sources: arcane, divine, and nature.
Sorcerers and wizards cast arcane spells, clerics cast divine spells, and druids cast nature spells.

\glossdef{spellpower} Your spellpower represents how powerful the spells you cast are (see \pcref{Magic}).

\glossdef{square} A square represents a single 5-ft.\ by 5-ft.\ space.
A typical Medium creature occupies a single square in combat.

\glossdef{stabilization roll}[stabilization rolls] A roll made when a creature is \conditionlink{dying} to see if it stabilizes or dies. For details, see \pcref{Injury, Death, and Healing}.

\glossdef{standard action} You can use a standard action to attack with a weapon, cast a spell, drink a potion, and do most other things that take concentration and effort.

\glossterm{standard attack} A standard attack is the most common way to attack with weapons you wield.
During the action phase, you can make a standard attack to \glossterm{strike} foes with your weapons.
If you have a high \glossterm{combat prowess}, you may be able to make multiple strikes during a standard attack (see \pcref{Multiple Attacks}).

\glossdef{strike}[strikes] A strike is a single physical attack with a weapon.
You usually make strikes during a \glossterm{standard attack} in the \glossterm{action phase}.
You can make some special attacks in place of strikes, such as \glossterm{combat maneuvers}.

\glossdef{Subtle} See \pcref{Ability Tags}.

\glossdef{Summoning} See \pcref{Ability Tags}.

\glossdef{swift action}[swift actions] Each round, you can take a single swift or immmediate action.
You can take a swift action during either the movement or action phase.
For details, see \pcref{Swift and Immediate Actions}.

\glossdef{swim speed} A creature with a swim speed can swim as easily as a human walks on land.
The effects of a swim speed are described at \pcref{Swim Speed}.

\glossdef{target square} A target square is a particular \glossterm{square} that an attack is made against.
A target square is chosen to determine \glossterm{cover} and \glossterm{concealment} (see \pcref{Cover}).

\glossdef{targeted spell}[targeted spells] A targeted spell is a spell that affects one or more targets of your choice.
For example, \spell{acid splash} and \spell{magic missile} are targeted spells, but \spell{mage armor} and \spell{fireball} are not.

\glossdef{Teleportation} See \pcref{Ability Tags}.

\glossdef{temporary hit points} Temporary hit points are extra hit points that can exceed your maximum.
If you take damage, temporary hit points are always lost before your ``real'' hit points.
For details, see \pcref{Temporary Hit Points}.

\glossdef{threaten}[threatened] When using a melee weapon, you threaten any creatures within the weapon's \glossterm{reach}.
A typical Medium creature threatens creatures in all adjacent squares.
If you threaten a creature, you can make \glossterm{melee} attacks against it, and you can make it suffer \glossterm{overwhelm penalties}.

\glossdef{threatened area} The area that you can make melee attacks into, as determined by your \glossterm{reach}.
The threatened area of a typical Medium creature consists of all squares adjacent to the creature.

\glossdef{Trap} See \pcref{Ability Tags}.

\glossdef{trip} A trip is a light \glossterm{combat maneuver} that allows you to knock a foe off its feet.
For details, see \pcref{Trip}.

\glossdef{tremorsight} A creature with tremorsight can ``see'' its surroundings perfectly without any light, regardless of concealment or invisibility.
It needs an uninterrupted path through solid objects to sense its surroundings, but does not require line of effect.
Tremorsight always has a range, and grants no benefits beyond that range.

\glossdef{tremorsense} A creature with tremorsense can sense its surroundings without any light, regardless of concealment or invisiblity.
It knows the location of everything around it, but it still takes normal failure chances for concealment, invisibility, and so on.
It needs an uninterrupted path through solid objects to sense its surroundings, but does not require line of effect.
Tremorsense always has a range, and grants no benefits beyond that range.

\glossdef{verbal components}[verbal] Verbal components are words required to cast most spells.
For details, see \pcref{Components}.

\glossdef{vital damage penalties} If you have \glossterm{vital damage}, you take a penalty to accuracy, checks, and defenses equal to the amount of vital damage you have.

\glossdef{vulnerable} A creature can be vulnerable to a type of damage or a special weapon material.
It takes double damage from sources it is vulnerable to.

\glossdef{warp damage} Warp damage is a special kind of damage.
Warp damage counts as damage for the purpose of determining how many hit points a character has remaining, just like other kinds of damage.
In addition, it cannot be cured by effects that restore hit points, effectively reducing the damaged character's maximum hit points.
An hour of rest cures warp damage equal to a character's level.
Warp damage is usually caused by spellwarped class abilities (see \pcref{Spellwarped}).

\glossdef{Water} See \pcref{Ability Tags}.

\glossdef{wild magic roll} Whenever a sorcerer casts a spell, he must make a wild magic roll.
Success means he gains a bonus to his spellpower with the spell.
Failure means the spell's miscast effect occurs in addition to its normal effect, and the sorcerer loses the ability to cast spells of the same level.
See \pcref{Wild Magic}, for details.

\glossdef{willing} Some abilities can only affect willing targets. You can choose to be a willing target at any time. Unconscious creatures and objects are automatically considered willing, but a character who is conscious but immobile or helpless (such as one who is bound or paralyzed) is not automatically willing.

\glossdef{zone} A zone is a type of area that an ability can have (see \pcref{Area Types}).
A zone ability has effects within an area for the duration of the ability.
Unless otherwise noted, it does not move after being created.
