\chapter{Glossary}\label{Glossary}

\glossdef{ability} An ability is a generic term for any unusual property a creature has or any special actions it can take to cause particular effects.
Spells, racial traits, and the benefits from class \glossterm{archetypes} can all be called abilities.

\glossdef{ability tag} An ability tag describes the effects of an ability.
For details, see \pcref{Ability Tags}.

\glossdef{accuracy} The bonus added to an \glossterm{attack roll}.
For details, see \pcref{Accuracy}.

\glossdef{action phase} The action phase is the second of two \glossterm{phases} in a combat \glossterm{round}.
During the action phase, creatures can \glossterm{attack}, cast \glossterm{spells}, and take other major combat actions.

\glossdef{alchemical item} An alchemical item is any item created using the Craft (alchemy) skill.
This includes firebombs, potions, and many other items.

\glossdef{alignment} Your alignment represents your general morality in broad terms.
For details, see \pcref{Alignment}.

\glossdef{allied group} Your allied group is the set of allies that you can coordinate your actions with.
Your whole allied group resolves their actions together, separately from other combatants.
For details, see \pcref{Resolving Actions}.

\glossdef{ally}[allies] Some beneficial abilities affect allies.
An ally is any creature you consider an ally who also considers you an ally, not including yourself.
For details, see \pcref{Allies and Enemies}.

\glossdef{archetype}[archetypes] An archetype is a collection of related abilities from a particular class.
Each class has five archetypes.
For details, see \pcref{Archetypes}.

\glossdef{archetype rank} Each ability from an \glossterm{archetype} has a minimum rank required to gain the ability.
For details, see \pcref{Archetype Ranks}.

\glossdef{area} A area ability affects multiple targets within an area.
Some area abilities are \glossterm{ranged}, while others are centered around their user.
There are five standard area sizes: \smallarea, \medarea, \largearea, \hugearea, and \gargarea.
For details, see \pcref{Area Shapes}, and \pcref{Area Types}.
If an ability is not an area ability, it is either a \glossterm{melee} ability or a \glossterm{ranged} ability.

\glossdef{armor} Armor is a form of equipment that protects your body from harm.
There are two kinds of armor: \glossterm{body armor}, which you wear on your body, and \glossterm{shields}, which you wield in a hand.
For details, see \pcref{Armor}.

\glossdef{astral beacon} An area with an astral beacon is easier to \glossterm{teleport} to using long-distance teleportation abilities.
For details, see \pcref{Astral Beacons}.

\glossdef{attack}[attacks] Anything that affects another creature in a potentially harmful way, such as striking a creature with a sword, is an attack.
All attacks require making an \glossterm{attack roll}.
If an ability requires an attack roll, it is considered to be an attack, even if you use them in a way that you believe is not harmful.

\glossdef{attack result} An attack result is the total you get on an \glossterm{attack roll}, after taking to account any bonuses or penalties that apply to the roll.

\glossdef{attack roll}[attack rolls] A roll required to succeed with an attack.
To make an attack roll, roll 1d10 \add your \glossterm{accuracy} with the attack.
If the result of the attack roll equals or exceeds the target's \glossterm{defense}, the attack succeeds.
Some attacks, especially magical attacks, have effects even if the attack roll fails.
For details, see \pcref{Attack Rolls}.

\glossdef{attended} An attended item is an item currently being held or carried by a creature.
Some abilities can only affect \glossterm{unattended} items.

\glossdef{attribute} A core representation of a character's capacity in a wide range of areas. There are six attributes: \glossterm{Strength}, \glossterm{Dexterity}, \glossterm{Constitution}, \glossterm{Intelligence}, \glossterm{Perception}, and \glossterm{Willpower}.

\glossdef{attune}[attunement] Some abilities last as long as you attune to them.
Attuning to an ability costs an \glossterm{attunement point} that you cannot recover as long as you maintain your attunement to that ability.
For details, see \pcref{Attuned Abilities}.

\glossdef{attuned} If you are attuned to an ability, you have invested an \glossterm{attunement point} in it to maintain its effect.
For details, see \pcref{Attuned Abilities}.

\glossdef{attunement point}[Attunement points] Attunement points allow you to \glossterm{attune} to effects such as spells or items.
For details, see \pcref{Attunement Points}, and \pcref{Attuned Abilities}.

\glossdef{barding} Armor designed for non-humanoid creatures is called barding.
The Armor defense bonus provided by barding is 2 lower than normal.
For details, see \pcref{Barding}.

\glossdef{base class} Your base \glossterm{class} grants you a variety of benefits.
You always have a single base class, even if you are a multiclass character.
For details, see \pcref{Base Class}.

\glossdef{base speed} Each size category has a base speed that indicates how far creatures of that size category can generally move.
For details, see \pcref{Base Speed}.

\glossdef{brawling accuracy} Your brawling accuracy is your \glossterm{accuracy} with \atBrawling abilities.
It uses your Strength instead of your Perception to determine your accuracy.
For details, see \pcref{Brawling Accuracy}.

\glossdef{brawling attack} A brawling attack uses your \glossterm{brawling accuracy} instead of your normal accuracy.
For details, see \pcref{Brawling Accuracy}.

\glossdef{briefly}[brief] An effect that lasts briefly, or a brief effect, lasts until after the end of the next round after the effect was applied.

\glosssynonym{Bright illumination}
\glossdef{bright illumination} In an area with bright illumination, creatures can see clearly.
Any effect which creates bright illumination in an area also creates enough light for \glossterm{shadowy illumination} in twice that area.
For details, see \pcref{Vision and Light}.

\glosssynonym{Brilliant illumination}
\glossdef{brilliant illumination} In an area with brilliant illumination, creatures can see clearly.
No shadows exist within an area of brilliant illumination.
Any effect which creates brilliant illumination in an area also creates enough light for \glossterm{shadowy illumination} in twice that area.
For details, see \pcref{Vision and Light}.

\glossdef{body armor} Body armor is a form of \glossterm{armor} that you wear on your body.
For details, see \pcref{Armor}.

\glossdef{broken} A broken object is damaged and unsuitable for use, though it retains its general structure and can be repaired.
For details, see \pcref{Broken Objects}.

\glossdef{burrow speed} A creature with a burrow speed can move at that speed through solid ground.
For details, see \pcref{Movement Modes}.

\glossdef{burst} A burst is a type of area that an ability can have (see \pcref{Area Types}).
A burst ability has an immediate effect on all valid targets within an area.

\glossdef{cantrip}[cantrips] Some \glossterm{mystic spheres} have minor spells called cantrips.
Anyone who has access to a mystic sphere knows all cantrips from that sphere.

\glossdef{carrying capacity} Your carrying capacity defines the amount of weight you can carry without penalty.
For details, see \pcref{Weight Limits}.

\glossdef{character level} Your character level is your total level, including levels from all of your classes.
Whenever text refers to your ``level'', without specifying a particular kind of level, it means your character level.

% TODO: clarify that you can chain from objects even if a spell only affects creatures
\glossdef{chain} An ability can specify that it chains a certain number of times.
For each time that the ability chains, you may choose an additional secondary target for the ability.
You can't chain back to a creature or object that is already a target of the ability.
Each additional target must be within 15 feet of the previous target in the chain.
The chain starts from one of the ability's primary targets.
These additional targets must have \glossterm{line of sight} to you and \glossterm{line of effect} to the previous target in the chain.
However, they do not need \glossterm{line of effect} to you, and they can be beyond the ability's original range.

Unless otherwise noted in a spell's description, the secondary targets from chaining are affected by the ability in the exact same way as the primary target.
Both creatures and objects are valid targets for chaining, but they have to be reasonably sized.
You can't chain off of the ground.

\glossdef{check}[checks] A check is a d10 roll required to accomplish an action that has a chance of failure that is not an attack.
If the result of your roll, including your modifier, is high enough, you succeed.
Otherwise, you fail.
For details, see \pcref{Making Checks}.

\glossdef{class}[classes] Your class represents your fundamental source of power and the type of abilities you have.
For example, barbarians draw power from the primal energy found deep within all living things, while clerics draw power from their worship of mighty deities.
For details, see \pcref{Classes}.

\glossdef{class skill}[class skills] Each \glossterm{class} has an associated set of skills that members of that class often know.
These are called class skills.
Your \glossterm{base class} automatically grants you training with a specific number of skills from among your class skills.
For details, see \pcref{Skills}.

\glossdef{climb speed} A creature with a climb speed can move that at that speed while climbing, and does not suffer penalties while doing so.
For details, see \pcref{Movement Modes}.

\glossdef{close range} Weapons have two \glossterm{range limits}: close range and \glossterm{long range}.
Attacks within a weapon's close range have no penalty.
For details, see \pcref{Weapon Range Limits}.

\glossdef{combat style} A combat style is a collection of \glossterm{maneuvers} that some classes gain access to.
For details, see \pcref{Combat Styles}.

\glossdef{common language}[common languages] Common languages are languages that are widely spoken.
They are described in \tref{Common Languages}.

\glossdef{concealment} Concealment represents effects which make a target harder to see, such as shadowy lighting.
All \glossterm{targeted} attacks against a creature or object with concealment from you have a 20\% \glossterm{miss chance}.
For details, see \pcref{Concealment}.

\glossdef{condition}[conditions] A condition is an effect that lasts on a creature until it is removed by effects that remove conditions.
All conditions are detrimental, and most are standard \glossterm{debuffs}.
Player characters can remove conditions with the \ability{recover} ability or by taking a \glossterm{short rest}, as well as with various special abilities (see \pcref{Recover}).
For details, see \pcref{Ability Durations}.

\glossdef{Constitution} Constitution is an \glossterm{attribute} that measures your health and stamina.
For details, see \pcref{Constitution}.

\glossdef{corpse} A corpse is the deceased body of a once-living creature.
If a corpse is \glossterm{destroyed}, it can no longer be treated as a corpse.

\glossdef{cover} Cover represents any obstacle that physically prevents you from striking your target, such as a tree or intervening creature.
It grants a \plus2 bonus to Armor, Brawn, and Reflex defenses.
For details, see \pcref{Cover}.

\glossdef{critical hit}[critical hits] When you make an attack, if your result beat the target's defense by 10 or more, you get a critical hit.
Unless otherwise noted, damaging attacks roll twice as many damage dice on a critical hit and double all flat modifiers.
For details, see \pcref{Critical Hits}.

\glossdef{critical success} When you make a check, if your result beat the \glossterm{difficulty value} by 10 or more, you get a critical success.
Some abilities have special effects on critical successes.

\glossdef{damage} Many attacks deal damage to you when they hit.
For details, see \pcref{Taking Damage}.

\glossdef{damage resistance} Whenever you take damage, you first apply that damage to your damage resistance applying it to your \glossterm{hit points}.
For details, see \pcref{Damage Resistance}.

\glossdef{damaging hit} Some abilities have special effects if they get a damaging hit.
If you miss, glance, or hit but fail to deal damage, you do not get a damaging hit.

\glossdef{darkvision}[Darkvision] A creature with darkvision can see perfectly in complete darkness.
For details, see \pcref{Darkvision}.

\glossdef{dead} A dead creature's soul leaves its body. Dead creatures cannot benefit from normal or magical healing, but they can be restored to life via magic (see \pcref{Resurrection}). A dead body decays normally unless magically preserved.

\glossdef{debuff} A debuff is a negative effect on a creature.
Many debuffs are applied as \glossterm{conditions}, but some last for longer or shorter times.
For a list of debuffs, see \pcref{Circumstances and Debuffs}.

\glossdef{deep attunement} Deep attunement abilities are \abilitytag{Attune} abilities with two additional restrictions.
First, they cost extra \glossterm{attunement point} to \glossterm{attune} to.
Second, you can't get back those attunement points until you take a \glossterm{short rest}, even if you release the attunement.
For details, see \pcref{Deep Attunement}.

\glossdef{defeat} You defeat a creature if you kill it or incapacitate it, causing it to be \glossterm{defeated}.

\glossdef{defeated} A creature is defeated if it dies or is incapacitated for an extended period of time (such as by being knocked unconscious).
Defeating a creature generally requires inflicting a \glossterm{vital wound} on it.

\glossdef{defense}[defenses] A defense is a static number which represents how difficult you are to affect with attacks.
There are five defenses: Armor, Brawn, Fortitude, Reflex, and Mental.
For details, see \pcref{Defenses}.

\glossdef{destroyed} A destroyed object has been damaged to the point where it is completely beyond repair.
For details, see \pcref{Destroyed Objects}.

\glossdef{Dexterity} Dexterity is an \glossterm{attribute} that measures your hand-eye coordination, agility, and reflexes.
For details, see \pcref{Dexterity}.

\glossdef{dice pool}[dice pools] A dice pool is a collection of dice that are all rolled together and summed to find a result.
Damage typically uses dice pool.

\glossdef{difficult terrain} Difficult terrain costs an additional 5 feet of movement to move out of.
For details, see \pcref{Difficult Terrain}.

\glossdef{difficulty value} The difficulty value of a \glossterm{check} is the check result required to succeed.
It can be abbreviated as ``DV''.
In general, attacks are rolled to beat \glossterm{defenses}, and checks are rolled to beat a given difficulty value.

\glossdef{disease} An affliction of the body, causing a steady deterioration over time.

\glosssynonym{dismisses}
\glossdef{dismiss}[dismissed] When you dismiss an ability, it ends, and all of its lingering effects are removed.
Unless otherwise noted, all \magical abilities with a duration can be dismissed, but \glossterm{mundane} abilities cannot be dismissed.
This includes \glossterm{conditions}, \glossterm{brief} effects, and other abilities with more specific durations.
You can dismiss abilities as a \glossterm{free action} (see \pcref{Dismissal}).

\glossdef{dual strike} A dual strike is a \glossterm{strike} made with two weapons at once.
You treat both weapons as a single combined weapon, adding together most of their statistics.
For details, see \pcref{Dual Strikes}.

\glossdef{elite} Elite monsters are much more dangerous than standard monsters.
% TODO: link differently depending on whether we are in the Grimoire of Guidance?

\glossdef{elite action}[elite actions] Elite monsters can take a special extra action every round called an elite action.
Every elite monster has at least one special ability which requires an elite action to use.
% TODO: Link differently based on whether we are in the Grimoire of Guidance
% For details, see \pcref{Elite Monsters}.

\glossdef{emanation} An emanation is a type of area that an ability can have (see \pcref{Area Types}).
An emanation ability has effects within an area for the duration of the ability.
It emanates from a specific creature or object, rather than a location.
If that creature or object moves, the emanation moves with it.

\glossdef{encumbrance} Your encumbrance is a value that represents how much you are burdened by armor and weight.
For details, see \pcref{Encumbrance}.

\glossdef{enemy}[enemies] Some harmful abilities affect enemies.
An enemy is any creature you consider to be an enemy.
For details, see \pcref{Allies and Enemies}.

\glossdef{enhancement bonus}[enhancement bonuses] Some abilities provide an enhancement bonus instead of a regular bonus.
Enhancement bonuses function like normal bonuses except that they do not stack with each other, even if the enhancement bonuses come from different sources.
For details, see \pcref{Stacking Rules}.

\glossdef{environmental damage}[environmental] Environmental damage is a type of damage.
Environmental damage does not reduce the \glossterm{damage resistance} of creatures or objects, making small amounts of environmental damage irrelevant to healthy creatures.
For details, see \pcref{Environmental Damage}.

\glossdef{exclude} Some effects allow you to exclude specific targets that would normally be affected by your abilities.
A creature or object excluded from an ability is not considered a target of the ability, even if it is within the ability's area or otherwise would normally be affected by the ability.

\glossdef{exotic weapon}[exotic weapons] A rare few weapons are considered exotic weapons.
They are unusually difficult to wield, and even being \glossterm{proficient} with the associated \glossterm{weapon group} does not grant you the ability to use an exotic weapon.
Some class abilities grant proficiency with exotic weapons.

\glossdef{explode}[explosions] When you roll a 10 on an \glossterm{attack roll}, the die can explode.
If it does, you roll it again and add the two results together to determine the total.
For details, see \pcref{Exploding Attacks}.

\glossdef{extra damage} Some attacks deal extra damage.
This damage is added on top of the normal damage from that attack.
For details, see \pcref{Extra Damage}.

\glossdef{failure chance}[failure chance] If you have a failure chance with an \glossterm{attack}, you have a random chance to miss with the attack regardless of the result of your attack roll.
If you have multiple failure chances, only the highest one applies.
Failure chances are rolled independently from \glossterm{miss chances}, and they are not affected by abilities that mitigate miss chances.
They are less common than a miss chance, and reflect circumstances that no amount of skill can mitigate.

\glossdef{falling damage} If you fall at least 10 feet, you and the object you land on take damage.
This damage is called falling damage.
A creature with a Medium \glossterm{weight category} takes 1d8 falling damage per 10 feet, to a maximum of 30d8.
For details, see \pcref{Falling Damage}.

\glossdef{fatigue level} Your fatigue level measures how fatigued you are.
You take a \glossterm{fatigue penalty} if your fatigue level exceeds your \glossterm{fatigue tolerance}.
For details, see \pcref{Fatigue}.

% Does "fatigued" need to exist as a term?
\glossdef{fatigue penalty} You take a penalty to \glossterm{accuracy} and \glossterm{checks} equal to your \glossterm{fatigue level} \sub your \glossterm{fatigue tolerance}.
When your fatigue penalty reaches \minus5, you fall \unconscious until your fatigue penalty is reduced below \minus5.
For details, see \pcref{Fatigue Penalty}.

\glossdef{fatigue tolerance}[Fatigue tolerance] Your fatigue tolerance measures the maximum \glossterm{fatigue level} you can reach before you suffer a \glossterm{fatigue penalty}.
For details, see \pcref{Fatigue Tolerance}.

\glossdef{fly speed} A creature with a fly speed has the ability to fly through the air.
Its speed is the distance it covers in a single \glossterm{movement}.
Most creatures suffer a \minus4 penalty to their Armor and Reflex defenses while flying.
For details, see \pcref{Aerial Movement}.

\glossdef{forced movement} A forced movement ability can cause a creature to move unwillingly.
There are two types of forced movement: \glossterm{knockback} and \glossterm{push}.
Although \glossterm{teleportation} can cause a creature's location to change unwillingly, it is not considered a type of forced movement.

\glossdef{free action}[Free action] A free action is one of the four action types (see \pcref{Actions}).
Each round, you take can any number of free actions.
Free actions can be taken in any phase.
For details, see \pcref{Free Actions}.

\glossdef{free hand} A free hand is a hand or similarly dexterous appendage that is not currently being used for any purpose.
Many abilities require a free hand to use.
You cannot use the same hand for two different purposes in the same \glossterm{phase}.

\glossdef{glance} When a creature glances another creature with an attack, it means that the attacker scored a \glossterm{glancing blow}.

\glossdef{glancing blow}[Glancing blow] When you miss on any attack by 2 or less, it is called a glancing blow.
Whenever you get a glancing blow with a damaging attack, you deal half damage.
For details, see \pcref{Glancing Blows}.

\glossdef{glide speed}[glide] A creature with a glide speed can glide through the air.
It cannot fly upwards, but it can travel forward while it descends, and it descends at a significantly reduced rate.
Most creatures suffer a \minus4 penalty to their Armor and Reflex defenses while gliding.
For details, see \pcref{Gliding}.

\glossdef{grappling} You are grappling if either a creature is \grappled by you or you are \grappled by a creature.
For details, see \pcref{Grappling}.

\glossdef{grounded} A grounded creature or object is standing on or otherwise supported by a stable surface that can support its weight.
The surface must be at least as large as the creature or object resting on it.
Some effects only work if the creature or object is grounded by a particular material, such as stone.

\glossdef{heavy undergrowth} A space overrun with thick bushes, vines, and similar natural obstacles has heavy undergrowth.
Heavy undergrowth provides \glossterm{concealment} and is considered \glossterm{difficult terrain}.

\glossdef{heavyweight} A heavyweight object has a \glossterm{weight category} that is one category larger than the object's \glossterm{size category}.
For details, see \pcref{Weight Categories}.

\glossdef{height limit} Some abilities have a height limit.
A height limit defines your maximum distance directly above an object at least two size categories larger than you that is free-standing and capable of supporting your weight.
This is common for flying creatures (see \pcref{Flight}).

\glossdef{hit point} Your hit points measure how hard you are to seriously injure or kill.
You lose hit points when you take damage.
If you run out of hit points, you gain \glossterm{vital wounds} when you take damage instead, which can cause you to die quickly.
For details, see \pcref{Hit Points}.

\glossdef{icy terrain} Icy terrain is covered in ice, making it hard to traverse.
For details, see \pcref{Cryomancy}.

\glossdef{immune} A creature that is immune to a particular effect treats that effect as if it did not exist.
% TODO: wording
An immune creature cannot gain \glossterm{conditions} or similar effects like \glossterm{poison} if it is immune to them, or if the only effect of that condition would be to apply a specific debuff that it is immune to.
In addition, a creature that temporarily becomes immune to an effect immediately removes all instances of that effect.
For example, a creature that suddenly becomes immune to poison would remove all poisons currently affecting it, and those poisons would not return once the immunity ends.

\glossdef{improvised weapon}[improvised weapons] An improvised weapon is an object which could conceivably be used as a weapon, but which was not designed for that purpose.
Common examples include doors and wine bottles.
For details, see \pcref{Improvised Weapons}.

\glossdef{initiative} When multiple creatures take mutually impossible actions simultaneously, such as racing to be the first one to a door, they must roll initiative checks to determine who completes the action first.
Your initiative modifier is equal to your Dexterity.
For details, see \pcref{Conflicting Actions}.

\glossdef{insight point}[Insight points] Insight points can be spent to gain additional abilities or proficiencies.
For details, see \pcref{Insight Points}.

\glossdef{Intelligence} Intelligence is an \glossterm{attribute} that represents how well you learn and reason.
For details, see \pcref{Intelligence}.

\glossdef{item rank} Items have ranks indicating their approximate value and rarity.
For details, see \pcref{Item Ranks}.

\glossdef{living} A living thing has life, which means that it can change and adapt over time.
% TODO: something should explain creature types in more detail
Most creatures are living, but animates and undead are not.

\glossdef{loose equipment} Loose equipment is much more vulnerable to damage than ordinary equipment.
For details, see \pcref{Loose Equipment}.

\glossdef{key attribute} The key attribute for a skill is the attribute associated with that skill.
For example, Climb is a Strength-based skill.
Some skills, such as Persuasion, do not have a key attribute.

\glossdef{knockback}[knocked back] Knockback is a type of \glossterm{forced movement}.
It represents being thrown backwards by a single large impact.
If a creature or object being knocked back encounters an obstacle, it and the obstacle each take 1d6 damage per 10 feet of movement remaining.
For details, see \pcref{Knockback Effects}.

\glossdef{legacy item} A legacy item is an item magically bonded to its bearer.
As its bearer gains levels, it increases in power as well.
For details, see \pcref{Legacy Items}.

\glossdef{light undergrowth} A space with passable bushes, vines, and similar natural obstacles has light \glossterm{undergrowth}.
Light undergrowth provides \glossterm{concealment}.

\glossdef{lightweight} A lightweight object has a \glossterm{weight category} that is one category smaller than the object's \glossterm{size category}.
For details, see \pcref{Weight Limits}.

\glossdef{line} A line is an area shape that an ability can have (see \pcref{Area Shapes}).
A line-shaped area has a given length, width, and height.
Unless otherwise stated, a line's height is equal to its width.

\glossdef{line of effect} You cannot target something that you do not have line of effect to.
Line of effect is blocked by solid obstacles, even invisible ones.
For details, see \pcref{Line of Effect}.

\glossdef{line of sight} You cannot target something that you do not have line of sight to.
Line of sight is blocked by any obstacle that blocks sight, even if that obstacle does not block physical passage.
For details, see \pcref{Line of Sight}.

\glossdef{long range} Ranged weapons have two \glossterm{range limits}: \glossterm{close range} and long range.
Attacks beyond a weapon's \glossterm{close range}, but within its long range, have a \minus4 \glossterm{longshot penalty}.
For details, see \pcref{Weapon Range Limits}.

\glossdef{long rest} A long rest represents eight hours of relaxation or sleep.
It allows you to remove all of your \glossterm{fatigue levels} and make progress towards healing a \glossterm{vital wound}.
For details, see \pcref{Long Rest}.

\glossdef{longshot penalty} A longshot penalty is the penalty that you take for attacking outside of a weapon's \glossterm{close range}.
It is normally a \minus4 \glossterm{accuracy} penalty.
For details, see \pcref{Weapon Range Limits}.

\glossdef{magic source}[magic sources] A magic source defines where a creature's \glossterm{mystic spheres} come from.
There are four magic sources: arcane, divine, nature, and pact.
Sorcerers and wizards cast arcane spells, clerics and paladins cast divine spells, druids cast nature spells, and votives cast pact spells.

\glossdef{magical} A magical ability is an ability whose origin derives from magic.
Examples include \glossterm{spells}, a dragon's ability to fly, and a paladin's ability to smite foes.
For details, see \pcref{Magical and Mundane Abilities}.

\glossdef{magical power} Your magical power is your \glossterm{power} with \magical abilities.
It is typically equal to half your level \add your Willpower.
For details, see \pcref{Power}.

\glossdef{maneuver} A maneuver is a type \glossterm{mundane} ability that some classes grant access to through particular combat styles.
For details, see \pcref{Combat Styles}.

\glossdef{manufactured weapon} A manufactured weapon is a \glossterm{weapon} that is external to its user's body.
A \glossterm{natural weapon} is not a manufactured weapon.
Some abilities affect or require manufactured weapons instead of natural weapons.

\glossdef{melee}[Melee] A melee ability affects targets in physical contact with its source.
Typically, this involves touching a target or using a weapon that never leaves your grasp.
Unless you are using a \weapontag{Long} weapon, you can only make melee attacks against targets adjacent to you.
If an ability is not melee, it is either a \glossterm{ranged} ability or an \glossterm{area} ability.

\glossdef{metallic} A creature is considered metallic if it is wearing metal armor or otherwise carrying a significant amount of exposed metal.
This includes any \glossterm{body armor} with a metal material type.
It also includes exposed metal objects or parts of objects that are no more than two size categories smaller than the creature.
This includes most weapons with any metallic components.
It does not include creatures who have small amounts of metal safely stowed in larger containers, such as a common amount of coins or metallic tools stowed in a coin purse or backpack.

Similarly, an object is generally considered metallic if it has an exposed piece made of metal that is no more than two size categories larger than the object as a whole.

\glossdef{midair} A land-based creature typically suffers a \minus4 penalty to its Armor and Reflex defenses while it is in the air and unable to touch the ground and move normally.
This applies even if the creature has a fly speed or glide speed.
However, it does not apply to creatures who are native to the air, such as birds and monsters with no defined walk speed.

\glossdef{minor action}[Minor action] A minor action is one of the four action types (see \pcref{Actions}).
You can take one minor action each \glossterm{round} during the \glossterm{action phase}.
For details, see \pcref{Actions}.

\glossdef{miss chance}[miss chances] If you have a miss chance with an \glossterm{attack}, you have a random chance to miss with the attack.
You roll the miss chance first, and if it causes you to miss, you do not roll an ordinary attack roll.
In general, only \glossterm{targeted} attacks can have a miss chance.
If you have multiple miss chances, only the highest one applies.

\glossdef{move} When you move, you usually travel a distance equal to your speed.
See \pcref{Movement and Positioning}, for details.
For specific \glossterm{move actions}, see \pcref{Movement Abilities}.

\glossdef{move action}[Move action] A move action is one of the four action types (see \pcref{Actions}).
You can use one move action during the \glossterm{movement phase} of each round.
Almost all move actions change your location on the battlefield.
For details, see \pcref{Movement and Positioning}.

\glossdef{movement mode}[movement modes] A movement mode is a method of moving from one location to another.
The most common mode is a \glossterm{walk speed}.
For details, see \pcref{Movement Modes}.

\glossdef{movement phase} The movement phase is the first of two \glossterm{phases} in a combat \glossterm{round}.
During the movement phase, creatures can make \glossterm{movements} (see \pcref{Movement and Positioning}.
The movement phase is followed by the \glossterm{action phase}.

\glossdef{multiclass} A multiclass character can gain access to \glossterm{archetypes} and other abilities from multiple classes.
For details, see \pcref{Multiclass Characters}.

\glossdef{mundane}[Mundane] Most abilities are considered mundane abilities.
Mundane abilities have some form of natural explanation and do not fundamentally originate from a magical source.
Examples include weapon attacks, a dragon's frightful presence, and a barbarian's rage.
Unless otherwise indicated, all abilities are mundane in nature.

\glossdef{mundane power} Your mundane power is your \glossterm{power} with \glossterm{mundane} abilities.
It is typically equal to half your level \add your Strength.
For details, see \pcref{Power}.

\glossdef{mystic sphere} A mystic sphere is a collection of thematically related magical effects that includes both \glossterm{spells} and \glossterm{rituals}.
For details, see \pcref{Mystic Spheres}.

\glossdef{natural weapon} A natural weapon is a \glossterm{weapon} that is part of a creature's body.
For details, see \pcref{Natural Weapons}.

\glossdef{object manipulation} The weight and accessibility of an object determines the action required to manipulate it.
For details, see \pcref{Manipulating Objects}.

\glossdef{obstacle} An obstacle is anything that blocks free movement.
Normally, both large objects and \glossterm{enemies} are obstacles, but \glossterm{allies} are not.
For details, see \pcref{Obstacles}.

\glossdef{Perception} Perception is an \glossterm{attribute} that describes your ability to observe and be aware of your surroundings.
For details, see \pcref{Perception}.

\glossdef{phase}[phases] A phase is part of the combat \glossterm{round}.
There are two phases: the \glossterm{movement phase} and the \glossterm{action phase}.
For details, see \pcref{Phases}.

\glossdef{planar rift}[planar rifts] A planar rift is a location where the boundaries between planes are unusually thin.
Planar rifts can be used to travel between planes using the appropriate rituals.
For details, see the Tome of Guidance.

\glossdef{plane} A plane is a distinct realm of existence.
Except for the connections between planes through \glossterm{planar rifts}, each plane is effectively an isolated universe, and different planes can obey different fundamental laws.
For details, see the Tome of Guidance.

\glossdef{point of origin}[points of origin] A point of origin is the grid intersection, creature, or object that an area originates from.
For details, see \pcref{Point of Origin}.

% TODO: better short description
\glossdef{poison}[poisoned] For a description of poisons and how they work, see \pcref{Poison}.

\glossdef{poison stage} Each \glossterm{poison} progresses in a series of stages.
Each stage inflicts a particular negative effect on the poisoned creature according to the poison's description.
For details, see \pcref{Poison}.

\glossdef{potion} A potion is a magical liquid that is typically contained in a Fine vial.
In general, drinking a potion requires a standard action.
Potions cannot be safely mixed together without diluting their magic, so you cannot consume two potions with the same action.

\glossdef{power} The power of an \glossterm{ability} represents how strong the ability is.
For details, see \pcref{Power}.

\glossdef{primary target} Some abilities that affect multiple targets distinguish between their primary and secondary targets.
For details, see \pcref{Primary and Secondary Targets}.

\glossdef{proficient}[proficiency] A creature can be proficient with weapons and armor.
You take a \minus2 accuracy penalty with weapons you are not proficient with.
If you wear or use armor you are not proficient with, it provides half its normal defense bonus.
In addition, you apply that armor's \glossterm{encumbrance} as a penalty to your \glossterm{accuracy}.

\glossdef{projectile} A projectile is an object fired from a weapon at a target.
Arrows and bolts are projectiles.

\glossdef{push}[pushed] A push is a type of \glossterm{forced movement}.
It represents being pushed by a constant force.
If a creature being pushed encounters an obstacle, it stops moving with no negative consequences.
For details, see \pcref{Push Effects}.

\glossdef{range} The range of an ability determines how far away it can be used.
Unless otherwise noted, all abilities with a range require both \glossterm{line of sight} and \glossterm{line of effect} to the point of origin or to all targets.
There are five standard ranges used for abilities: \shortrange, \medrange, \longrange, \distrange, and \extrange (see \pcref{Ability Range}).
Ranged weapons do not use those standard ranges, and instead use specific \glossterm{range limits} (see \pcref{Weapon Range Limits}).

\glossdef{range limit}[range limits] Ranged weapons have two \glossterm{range limits} listed, with a slash between them, such as 60/180.
The first number indicates the maximum range for a weapon's \glossterm{close range}.
The second number indicates the maximum range for a weapon's \glossterm{long range}.
For details, see \pcref{Weapon Range Limits}.

\glossdef{ranged} A ranged ability affects targets at a distance from its source.
Ranged abilities always have a \glossterm{range} at which they function.
If an ability is not ranged, it is either a \glossterm{melee} ability or an \glossterm{area} ability.

\glossdef{rank} Many abilities have a rank.
This is typically equal to the minimum \glossterm{archetype rank} you need to learn or use the ability.
For abilities with no explicitly defined rank, use one third of the minimum level required to learn or use the ability (minimum 0).

\glossdef{rare language}[rare languages] Rare languages are languages that are only spoken by rare or distant creatures or cultures.
They are described in \tref{Rare Languages}.

\glossdef{reactive attack} A reactive attack is an \glossterm{attack} that you make during the resolution of another creature's actions.
You cannot modify a reactive attack in any way - it happens entirely outside of your control.
For example, you cannot use the \ability{desperate exertion} ability to reroll a reactive attack, or add an extra target with a \weapontag{Sweeping} weapon.
If you would make multiple reactive attacks during the same phase with the same ability against different targets, use the same attack roll for each target.
A reactive attack can never be triggered by a reactive attack or reactive check.

\glossdef{reactive check} A reactive check is a \glossterm{check} that you make during the resolution of another creature's actions.
Just like a \glossterm{reactive attack}, you cannot modify a reactive check in any way.

\glossdef{repeat} Some effects can repeat abilities at a later time.
When an ability repeats, it retains all choices for all decisions as the original ability usage, such as targets and affected area.
All attack rolls made for a repeated ability are \glossterm{reactive attacks}.

Some repeats specify their targets, such as repeating only for a particular creature.
Other repeats affect the entire ability.
If a repeat specifies a target, it works on that target regardless of the ability's original targeting restrictions.
Otherwise, the repeat originates from the creature that originally used the ability, so targeting restrictions and range limits still apply.

\glossdef{reroll} Some abilities allow you to reroll a roll you just made.
The most common ability that allows rerolling is \ability{desperate exertion} (see \pcref{Desperate Exertion}).
You must reroll the entire roll, not just one die from the roll (such as if the original roll \glossterm{explodes}).
It is possible to reroll the same same roll multiple times with different abilities.
Each reroll only grants one extra roll.

\glossdef{resource} A resource is something that a character can lose during play or expend to gain a benefit.
Most resources are shared between all types of characters, though different characters can use them differently.
There are two resources that are used during the character creation and leveling process: \glossterm{insight points} and \glossterm{trained skills}.
In addition, there are are five resources that are used during gameplay: \glossterm{attunement points}, \glossterm{damage resistance}, \glossterm{fatigue level}, \glossterm{hit points}, and \glossterm{vital wounds}.

\glosssynonym{resurrect}
\glossdef{resurrection}[resurrected] When a creature is resurrected, it comes back to life after being dead.
For details, see \pcref{Resurrection}.

\glossdef{ritual}[rituals] A ritual is a complex \magical ceremony that has a specific effect when completed.
For details, see \pcref{Spell and Ritual Mechanics}.

\glossdef{round}[rounds] Combat takes place in a series of rounds, which represent about six seconds of action.
Rounds are divided into two \glossterm{phases}: the \glossterm{movement phase}, and the \glossterm{action phase}.

\glossdef{secondary target} Some abilities that affect multiple targets distinguish between their primary and secondary targets.
For details, see \pcref{Primary and Secondary Targets}.

\glossdef{scent}[Scent] A creature with the scent ability has an unusually good sense of smell.
For details, see \pcref{Scent}.

\glossdef{scrying sensor} A scrying sensor is a magical construct created by some magical abilities.
Scrying sensors are Fine objects resembling a human eye in size and shape, though they are \trait{invisible}.
Scrying sensors typically float in a fixed position in the air.
They normally can't be moved by external forces without destroying the sensor.
Unless otherwise specified, a scrying sensor's visual acuity is the same as that of a normal human, giving it a \plus0 bonus to the Awareness skill and similar checks.

\glossdef{sentient} A sentient creature is capable of experiencing emotions and perceiving its surroundings.
Complex animals are sentient, but trees are not.
Some creatures have incomplete minds that are capable of simulating intelligence without true sentience.
These creatures are called \trait{simple-minded}.

\glossdef{shadowed} A creature or object is shadowed if it is touching its shadow.
That typically means it is in \glossterm{shadowy illumination} or \glossterm{bright illumination}, but not \glossterm{brilliant illumination} or complete darkness.
In addition, it must be \glossterm{grounded} or otherwise touching a surface.

\glossdef{shadowy illumination} In an area with shadowy illumination, creatures can see dimly.
Creatures and objects within this area have \glossterm{concealment}, which can allow creatures to make Stealth checks to hide (see \pcref{Stealth}).
For details, see \pcref{Vision and Light}.

\glossdef{shapeshift}[shapeshifting] Shapeshifting abilities change the physical form and abilities of a creature or object.
For details, see \pcref{Shapeshifting}.

\glossdef{shield}[shields] Shields are a form of \glossterm{armor} that you wield in a hand to protect you from harm.
For details, see \pcref{Armor}.

\glossdef{short rest}[short rests] A short rest represents ten minutes of relaxation.
It allows you to regain lost \glossterm{hit points} and any \glossterm{attunement points} you released from \glossterm{attunement}.
For details, see \pcref{Short Rest}.

\glossdef{size category}[size categories] A creature's size category indicates how large it is.
There are nine size categories, from smallest to largest: Fine, Diminutive, Tiny, Small, Medium, Large, Huge, Gargantuan, Colossal.
For details, see \pcref{Size Categories}.

\glossdef{skill}[skills] A skill represents your degree of talent with a particular non-combat aspect of the world.
For example, the Climb skill represents how skilled you are at climbing.
For details, see \pcref{Skills}.

\glossdef{somatic components}[somatic] Somatic components are hand motions required to cast arcane and pact spells.
For details, see \pcref{Ability Usage Components}.

\glossdef{something} Many abilities say they target ``something'', generally within a \glossterm{range}.
This means they target one creature or object of your choice.

\glossdef{space} Your space is the area that your physical body occupies.
For convenience, your space is measured in five-foot \glossterm{squares}.
Medium creatures occupy space equal to a single five-foot square.
For details, see \pcref{Size Categories}.

\glossdef{speed} Your speed represents the number of feet you can move with a single movement (see \pcref{Movement and Positioning}).

\glossdef{spell}[Spell] A spell is a disrete \magical ability with combat-relevant effects.
For details, see \pcref{Spells}.

\glossdef{spell list} The list of spells you can cast from a particular \glossterm{magic source}.
Each spell source has a specific spell list which is described at \pcref{Spells}.
Most characters with the same spell sources have the same spell lists.
However, some effects, such as a cleric's domains, can add spells to a character's individual spell list.

\glossdef{square}[squares] A square represents a single 5-ft.\ by 5-ft.\ space.
Many areas are measured in squares for convenience.

\glossdef{standard action} A standard action is one of the four action types (see \pcref{Actions}).
You can take one standard action each \glossterm{round} during the \glossterm{action phase}.
For details, see \pcref{Actions}.

\glossdef{Strength} Strength is an \glossterm{attribute} that measures your muscle and physical power.
For details, see \pcref{Strength}.

\glossdef{strike}[strikes] A strike is a single physical attack with a weapon.
It is the most common type of attack.
You can make a strike as a \glossterm{standard action} in the \glossterm{action phase}.
For details, see \pcref{Strikes}.

\glossdef{subdual damage} Subdual damage is a special kind of damage that can't kill you.
If you would gain a \glossterm{vital wound} from subdual damage, you increase your \glossterm{fatigue level} by three instead.
For details, see \pcref{Subdual Damage}.

\glossdef{suppressed}[suppress] A suppressed ability has temporarily ceased to function.
It has no effect for as long as it remains suppressed.
Time spent while suppressed counts against the ability's duration, and it may expire while suppressed if it lasts for a specific amount of time.
Only \magical abilities can be suppressed.
Mundane results of magical abilities that have already occured, such as the water created by a \ritual{create water} ritual, cannot themselves be suppressed, and do not disappear if they enter an area that suppresses magical abilities.

\glossdef{sustain}[sustained] Some abilities last as long as you sustain them.
Each ability specifies a particular action that is required to sustain the ability, such as a \glossterm{minor action}.
When \abilitytag{Swift} abilities resolve during each \glossterm{action phase}, the ability is dismissed unless you take the action to sustain the ability that round.
For details, see \pcref{Sustained Abilities}.

\glossdef{Swift} An ability with this \glossterm{ability tag} resolves its effects before other actions in the same phase.
For details, see \pcref{Swift Abilities}.

\glossdef{swim speed} A creature with a swim speed can move at that speed while swimming, and does not suffer penalties while \submerged.
For details, see \pcref{Movement Modes}.

\glossdef{target} A target is a creature or object directly affected by an ability.
Many abilities only affect a single target, and some affect a specific number of targets.
For details, see \pcref{Ability Targeting}.

\glossdef{target square} A target square is a particular \glossterm{square} that an attack is made against.
A target square is chosen to determine \glossterm{cover} and \glossterm{concealment} (see \pcref{Cover}).

\glossdef{targeted}[Targeted] A targeted ability is an ability that allows you to directly choose which targets the ability affects.
A spell that affects an area is not a targeted ability, because you choose the area affected instead of choosing the targets directly.
A \glossterm{strike} is a targeted ability, and so is a spell or other special ability that causes you to immediately make a single strike.
Adding an extra target to an ability that causes you to make a strike means you hit an extra creature with the strike, not that the extra target also makes a strike.

\glossdef{targeting proxy} When you use an ability through a targeting proxy, you determine its targets as if you were in the targeting proxy's location instead of your own.
This can allow you to affect targets outside your normal range.
For details, see \pcref{Targeting Proxies}.

\glossdef{telepathy} A creature with telepathy can mentally communicate with other creatures within a given range.
For details, see \pcref{Telepathy}.

\glosssynonym{teleports}
\glosssynonym{teleported}
\glossdef{teleportation}[teleport] A creature or object that is teleported instantly leaves one location and arrives at another.
Unless otherwise specified, teleporation requires \glossterm{line of sight}, \glossterm{line of effect}, and an unoccupied destination on stable ground.
For details, see \pcref{Teleportation}.

\glossdef{thrown weapon}[thrown weapons] A thrown weapon is a weapon designed to be thrown at a target.
For details about attacking with thrown weapons, see \pcref{Basic Strike -- Thrown}.

\glosssynonym{touched}
\glossdef{touch}[touches] Some abilities function on creatures you touch, rather than having a range away from you.
You can generally touch an adjacent creature as long as you have a \glossterm{free hand}, even if it is an enemy, though this has no mechanical effect unless an ability says it does.
Hitting someone with a \glossterm{natural weapon} does count as touching them, but it still requires an action, so you can't make a strike as part of using another ability unless it says explicitly that you can.
Some creatures cannot be touched, such as \trait{intangible} creatures.

% TODO: define standard trap rules
% \glossdef{trap}

\glossdef{trained skill}[Trained skills] If you are trained in a \glossterm{skill}, you have learned how to use it well.
Your modifier with a trained skill is equal to 3 \add the higher of its associated attribute (if any) and half your level.
For details, see \pcref{Trained Skills}.

\glossdef{unaffected} If you are unaffected by a particular effect, it doesn't do anything to you.
Unlike being \glossterm{immune}, you do not automatically remove persistent effects that you are unaffected by, such as \glossterm{conditions}c.
This means you may still need to track that the effect is on you in case you stop being unaffected by it.
For example, a barbarian is unaffected by conditions while raging, but those conditions have their full effects when the barbarian stops raging.

\glossdef{unattended} An unattended item is an item not being held or carried by a creature, or that is being held or carried by an \glossterm{ally}.
Some abilities can only affect unattended items.

\glossdef{unaware} See \pcref{Circumstances and Debuffs}.

\glossdef{unconscious} See \pcref{Circumstances and Debuffs}.

\glossdef{undergrowth} The presence of a significant amount of roots, bushes, and similar plants that can obstruct movement is called undergrowth.
There are two kinds of undergrowth: \glossterm{light undergrowth} and \glossterm{heavy undergrowth}.
For details, see \pcref{Undergrowth}.

\glossdef{usage class}[usage classes] The \glossterm{usage class} of armor is a measure of how much effort it takes to use it.
There are three usage classes: light, medium, and heavy.
For details, see \pcref{Armor Usage Classes}.

\glossdef{verbal components}[verbal] Verbal components are words required to cast most spells.
For details, see \pcref{Ability Usage Components}.

\glossdef{Visual} See \pcref{Ability Tags}.

\glossdef{vital wound} A \glossterm{vital wound} is a serious injury that inflicts negative effects on you.
You gain one or more \glossterm{vital wounds} when you take damage in excess of your hit points (see \pcref{Negative Hit Points}).
For details, see \pcref{Vital Wounds}.

\glossdef{vulnerable} A vulnerable creature takes a \minus4 penalty to all defenses against whatever it is vulnerable to.
For details, see \pcref{Vulnerable}.

\glossdef{wall} A wall is an area shape that an ability can have (see \pcref{Area Shapes}).
A wall-shaped area has a length and height, but its width is not measured in squares.

\glossdef{walk speed} A creature's walk speed is a \glossterm{movement mode} that determines how fast it can walk on land (see \pcref{Movement Modes}).
Most creatures have an average walk speed.

\glossdef{weapon}[weapons] A weapon is an object used to inflict damage.
Some creatures can treat parts of their body as weapons.
For details, see \pcref{Weapons}.

\glossdef{weapon damage} Your weapon damage is the damage you deal with weapons.
Typically, weapon damage is dealt by \glossterm{strikes} (see \pcref{Strikes}).
You gain a bonus to your weapon damage equal to half your relevant \glossterm{power} (see \pcref{Power}).
For details, see \pcref{Weapon Damage}.

\glossdef{weapon group} A weapon group is a category of \glossterm{weapons} with a similar design and fighting style.
% TODO: are there actually any special abilities besides proficiency that are specific to weapon groups?
Some abilities grant you proficiency with or special abilities with particular weapon groups.
For details, see \pcref{Weapon Groups}.

\glossdef{weapon tag} A weapon tag describes the special effects of a weapon.
For details, see \pcref{Weapon Tags}.

\glossdef{weight limit} Your weight limits define the amount of weight you can carry or push without penalty.
For details, see \pcref{Weight Limits}.

\glossdef{weight category}[weight categories] The weight category of an object or creature is a broad measurement of how much it weighs.
Weight categories are closely related to \glossterm{size categories}.
For details, see \tref{Weight Categories}.

\glossdef{Willpower} Willpower is an \glossterm{attribute} that represents your ability to endure mental hardships.
For details, see \pcref{Willpower}.

\glosssynonym{Vital rolls}
\glossdef{vital roll} When you gain a \glossterm{vital wound}, you make a \glossterm{vital roll} to determine the detrimental effect of the \glossterm{vital wound}.
To make a \glossterm{vital roll}, roll 1d10 \sub the number of \glossterm{vital wounds} you already had, ignoring the vital wound you are rolling for.
For details, see \pcref{Vital Wounds}.

\glossdef{zone} A zone is a type of area that an ability can have (see \pcref{Area Types}).
A zone ability has effects within an area for the duration of the ability.
Unless otherwise noted, it does not move after being created.
