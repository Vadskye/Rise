\chapter{Classes}\label{Classes}

Your character's class represents the things your character has chosen to train in.
This choice determines a great deal about your character's abilities.

\section{How Classes Work}
    When you first create a character, you choose a class.
    Each class grants some basic class features to all members of that class.
    In addition, each class has a number of \glossterm{archetypes} that grant more powerful and specific abilities.

    \subsection{Archetypes}\label{Archetypes}
        Each class has a number of \glossterm{archetypes}.
        An archetype is a collection of thematically related abilities.
        For examples, barbarians have the Battlerager archetype, which grants abilities related to being angry and flying into a rage in combat.
        You have an \glossterm{archetype rank} associated with each archetype you have.

        \subsubsection{Archetype Ranks}\label{Archetype Ranks}
            Each ability from an archetype has a minimum rank required to gain the ability.
            When you gain a rank in an archetype, you gain all abilities associated with that rank.
            In addition, some of your existing abilities may increase their power based on your rank in that archetype.

            At 1st level, you choose three of the archetypes associated with your class.
            You are Rank 1 in one of those archetypes, and rank 0 in your other two archetypes.
            You have no ranks at all in any other archetypes, and can never gain abilities from archetypes other than your chosen three.

            Every level after 1st level, you increase your rank in one archetype of your choice.
            This gives you the abilities associated with that rank.
            Each \glossterm{archetype rank} has a minimum level, as shown on \trefnp{Archetype Ranks by Level}.
            This minimum level is included in each class table as a reminder.

            \begin{dtable}
                \lcaption{Archetype Ranks by Level}
                \begin{dtabularx}{\columnwidth}{l >{\lcol}X}
                    \tb{Archetype Rank} & \tb{Minimum Level} \tableheaderrule
                    1 & 1  \\
                    2 & 4  \\
                    3 & 7  \\
                    4 & 10 \\
                    5 & 13 \\
                    6 & 16 \\
                    7 & 19 \\
                \end{dtabularx}
            \end{dtable}

        \subsubsection{Duplicate Archetypes}\label{Duplicate Archetypes}
            Some archetypes can be gained by multiple classes.
            For example, both clerics and paladins have the Divine Magic archetype.
            You cannot gain two archetypes with the same name, even if you can choose archetypes from multiple classes.

        \subsection{Multiclass Characters}\label{Multiclass Characters}
            You can spend two \glossterm{insight points} to become a \glossterm{multiclass} character (see \pcref{Insight Points}).
            If you do, choose a class other than your original class.
            You gain the following benefits relating to that class.
            \begin{itemize}
                \item You gain the \glossterm{class skills} of that class in addition to your existing \glossterm{class skills}.
                \item You can choose which of your classes you are considered for the purpose of determining which basic class abilities, such as defense bonuses and weapon proficiencies, you start with.
                    You must choose a single starting class for this purpose, so you cannot gain the defense bonuses from one class and the weapon proficiencies from another.
                \item If that class has any special class abilities which are not part of an archetype, such as a warlock's \textit{soul pact} ability, you gain those abilities.
                \item You may gain one \glossterm{archetype} from your that class in place of one \glossterm{archetype} from your base class.
            \end{itemize}

            You may gain access to multiple classes in this way, spending two \glossterm{insight points} for each class.

\section{Class Introductions}

    There are nine classes in Rise.
    \begin{itemize}
        \item Barbarians are mighty warriors who draw power from their physical prowess.
        \item Clerics are divine spellcasters who draw power from their veneration of a deity.
        \item Druids are nature spellcasters who draw power from their veneration of the natural world.
        \item Fighters are highly disciplined warriors who excel in physical combat of any variety.
        \item Monks are agile masters of ``ki'' who hone their personal abilities to strike down foes and perform supernatural feats.
        \item Paladins are divinely empowered warriors embody a particular alignment.
        \item Rangers are skilled hunters who bridge the divide between nature and civilization.
        \item Rogues are exceptionally skillful characters known for their ability to strike at their foe's weak points in combat.
        \item Sorcerers are arcane spellcasters who draw power from their inherently magical nature.
            % \item Spellwarped wield a unique blend of martial skill and narrowly focused magical abilities.
        \item Warlocks are pact spellcasters who draw their power from a dark pact made with infernal creatures.
        \item Wizards are arcane spellcasters who study magic to unlock its powerful secrets.
    \end{itemize}

    \subsection{Class Description Format}
        Each class is described from the perspective of a member of that class, using ``you'' in the description.

        \parhead{Class Table}
        The class's table describes the special abilities a member of that class gains at each level, assuming they have all of that class's \glossterm{archetypes}.

        \parhead{Alignment}
        Some classes require specific alignments (see \pcref{Alignment}).
        Most classes allow characters of any alignment.

        \parhead{Skills}
        Each class has specific \glossterm{skills} that members of that class are typically good at (see \pcref{Skills}).
        These skills are called \glossterm{class skills}.
        It is easier to become \glossterm{mastered} in class skills than in other skills.
        For details, see \pcref{Skill Training}.

        \parhead{Defenses}
        Each class grants bonuses to specific defenses.

        \parhead{Weapon Proficiencies}
        This indicates the types of weapons that members of this class are proficient with.

        \parhead{Armor Proficiencies}
        This indicates the types of armor that members of this class are proficient with.

        \parhead{Other Special Abilities}
        Some classes have abilities shared by all members of the class that are not part of an archetype, such as a druid's \textit{druidic language} ability.

        \parhead{Archetypes}
        The abilities associated with each of the three archetypes the class has.

\newpage
\section{Barbarian}\label{Barbarian}
    \begin{dtable!*}
        \lcaption{Barbarian Progression}
\begin{dtabularx}{\textwidth}{l l l >{\lcol}X >{\lcol}X >{\lcol}X >{\lcol}X}
    \tb{Rank} & \tb{Min Level} & \tb{Battleforged Resilience}  & \tb{Battlerager}         & \tb{Outland Savage}                        & \tb{Primal Warrior}    \tableheaderrule
    0 & \tdash & Unbattered resilience         & Insensible anger         & Savage rush                                & Primal might                                \\
    1 & 1      & Battle-scarred                & Rage                     & Fast movement                              & Combat styles                         \\
    2 & 4      & Resilient recovery            & Enraged strike           & Savage precision                           & Combat style rank (2), primal force         \\
    3 & 7      & Greater unbattered resilience & Greater insensible anger & Savage force                               & Combat style rank (3), glancing strikes     \\
    4 & 10     & Greater battle-scarred        & Greater rage             & Greater fast movement, greater savage rush & Combat style rank (4), primal maneuver      \\
    5 & 13     & Greater resilient recovery    & Immutable anger          & Greater savage precision                   & Combat style rank (5), greater primal force \\
    6 & 16     & Supreme unbattered resilience & Supreme insensible anger & Greater savage force                       & Combat style rank (6), greater primal might \\
    7 & 19     & Supreme battle-scarred        & Supreme rage             & Supreme fast movement, supreme savage rush & Combat style rank (7), primal maneuver      \\
\end{dtabularx}
    \end{dtable!*}

    Barbarians are primal warriors that draw power from the physical strength of their bodies.
    A typical barbarian is strong, fast, and durable - and a terrifying threat for their unfortunate enemies to behold.
    They are more inclined to chaos than law, and often have little respect for civilized society, though they may respect authorities with the power to back up their claims.
    They tend to forgo defense in favor of offense, trusting their physical durability to keep them in the fight.
    Barbarians have limited utility outside of combat, though their physical prowess often gives them high physical skills.

    \classbasics{Alignment} Any.

    \classbasics{Archetypes} Barbarians have the Battlerager, Battleforged Resilience, Outland Savage, and Primal Warrior \glossterm{archetypes}.

    \subsection{Basic Class Abilities}
        If you are a barbarian, you gain the following abilities.
        % 2 insight, 6 skill, 4 AP, 8 fatigue, 2 armor, 3 weapons = 25 points

        \cf{Bbn}{Defenses}
        You gain the following bonuses to your \glossterm{defenses}: \plus1 Armor, \plus7 Fortitude, \plus5 Reflex, \plus3 Mental.

        \cf{Bbn}{Resources} You have the following \glossterm{resources}:
        \begin{itemize}
            \item One \glossterm{insight point}, which you can spend to gain additional abilities or proficiencies (see \pcref{Insight Points}).
            \item Nine \glossterm{skill points}, which you can spend to learn skills (see \pcref{Skills}).
            \item One \glossterm{attunement point}, which you can use to attune to items and abilities that affect you (see \pcref{Attunement Points}).
            \item A \plus4 bonus to your \glossterm{fatigue tolerance}, which makes it easier for you to use powerful abilities that fatigue you (see \pcref{Fatigue}).
        \end{itemize}

        \cf{Bbn}{Weapon Proficiencies} 
        You are proficient with simple weapons and any two other \glossterm{weapon groups}.

        \cf{Bbn}{Armor Proficiencies} 
        You are proficient with light and medium armor.

        \cf{Bbn}{Skills}
        You have the following \glossterm{class skills}:
        \begin{itemize}
            \item \subparhead{Strength} Climb, Jump, Swim.
            \item \subparhead{Dexterity} Agility, Flexibility, Ride.
            \item \subparhead{Constitution} Endurance.
            \item \subparhead{Intelligence} Craft, Medicine.
            \item \subparhead{Perception} Awareness, Creature Handling, Survival.
            \item \subparhead{Other} Deception, Intimidate, Persuasion, Profession.
        \end{itemize}

    \newpage
    \subsection{Battleforged Resilience}
        This archetype improves your durability in combat.

        \cf{Bbn}[0]{Unbattered Resilience} At the end of each round, if you did not resist any physical damage that round, you regain physical damage resistance equal to your rank in this archetype (minimum 1).

        \cf{Bbn}[1]{Battle-Scarred} You gain a bonus equal to twice your rank in this archetype to your \glossterm{resistance} against \glossterm{physical damage} (see \pcref{Resistances}).

        \cf{Bbn}[2]{Resilient Recovery} You regain half of your maximum physical and energy \glossterm{resistances} when you take the \textit{recover} action (see \pcref{Recover}).
        In addition, you gain a \plus1 bonus to your \glossterm{fatigue tolerance}.

        \cf{Bbn}[3]{Greater Unbattered Resilience} The amount of physical damage resistance you regain with your \textit{unbattered resilience} ability increases to twice your rank in this archetype.

        \cf{Bbn}[4]{Greater Battle-Scarred} The bonus from your \textit{battle-scarred} ability increases to be equal to three times your rank in this archetype.
        In addition, you gain a bonus equal to your rank in this archetype to your \glossterm{resistance} against energy damage.

        \cf{Bbn}[5]{Greater Resilient Recovery} You can use the \textit{recover} action an additional time before you take a \glossterm{short rest}.
        In addition, the bonus to your fatigue tolerance from your \textit{resilient recovery} ability increases to \plus2.

        \cf{Bbn}[6]{Supreme Unbattered Resilience} At the end of each round, if you did not resist any energy damage that round, you regain energy damage resistance equal to your rank in this archetype.

        \cf{Bbn}[7]{Supreme Battle-Scarred} The bonus from your \textit{battle-scarred} ability increases to be equal to four times your rank in this archetype.
        In addition, the bonus from your \textit{greater battle-scarred} ability increases to be equal to twice your rank in this archetype.

    \newpage
    \subsection{Battlerager}\label{Rage}
        This archetype grants you a devastating rage, improving your combat prowess.

        \cf{Bbn}[0]{Insensible Anger} You reduce your maximum hit points by an amount equal to twice your rank in this archetype (minimum 1).
        In exchange, you gain the same bonus to your \glossterm{resistances} against both \glossterm{physical damage} and \glossterm{energy damage} (see \pcref{Resistances}).

        \cf{Bbn}[1]{Rage} You can use the \textit{rage} ability as a \glossterm{free action}.
        For most barbarians, this represents entering a furious rage.
        Some barbarians instead enter a joyous battle trance or undergo a partial physical transformation into a more fearsome form.
        \begin{durationability}{Rage}
            \spelltwocol{\abilitytag{Sustain} (free)}{\abilitytag{Emotion}, \abilitytag{Swift}}
            \rankline
            For the duration of this ability, you gain the following benefits and drawbacks:
            \begin{itemize}
                % This is an aggressive scaling that prevents any direct upgrades later in the archetype
                \item You gain a \plus2 bonus to your \glossterm{mundane} \glossterm{power}.
                \item You gain a \plus2 bonus to \glossterm{vital rolls}.
                \item You take a \minus2 penalty to Armor and Reflex defenses.
                \item You are unable to take \glossterm{standard actions} that do not cause you to make \glossterm{mundane} attacks.
                \item You are unable to use \abilitytag{Focus} abilities of any kind.
                \item At the end of each round, if you did not make a \glossterm{mundane} attack that round, this ability ends.
                \item When this ability ends for any reason, you cannot use it again until after the end of the next round.
            \end{itemize}

            \rankline
            \rank{3} The power bonus increases to \plus4.
            \rank{5} The power bonus increases to \plus8.
            \rank{7} The power bonus increases to \plus16.
        \end{durationability}

        \cf{Bbn}[2]{Enraged Strike} As a standard action, you can use the \textit{enraged strike} ability.
        \begin{instantability}{Enraged Strike}
            Instant
            \rankline
            Make a melee \glossterm{strike}.
            If the target of the strike dealt damage to you during the previous round, you gain a \plus2 bonus to \glossterm{accuracy} with the strike and a \plus2d bonus to damage.

            \rankline
            \rank{4} The damage bonus increases to \plus3d.
            \rank{6} The damage bonus increases to \plus4d.
        \end{instantability}

        \cf{Bbn}[3]{Greater Insensible Anger} The hit point reduction and resistance bonuses from your \textit{insensible anger} ability both increase to be equal to three times your rank in this archetype.

        \cf{Bbn}[4]{Greater Rage} You gain a \plus1 bonus to \glossterm{accuracy} during your \textit{rage} ability.

        \cf{Bbn}[5]{Immutable Anger} You become immune to hostile \abilitytag{Emotion} effects.

        \cf{Bbn}[6]{Supreme Insensible Anger} The hit point reduction and resistance bonuses from your \textit{insensible anger} ability both increase to be equal to four times your rank in this archetype.

        \cf{Bbn}[7]{Supreme Rage} When you use your \textit{rage} ability, you can grow by one \glossterm{size category}.
        In addition, the \glossterm{accuracy} bonus from your \textit{greater rage} ability increases to \plus2.

    \newpage
    \subsection{Outland Savage}
        This archetype improves your mobility and combat prowess with direct, brutal abilities.

        \cf{Bbn}[0]{Savage Rush} When you use the \textit{sprint} ability, you can move through spaces occupied by enemies during that movement.
        You treat those spaces as \glossterm{difficult terrain}, which causes to you to move at half speed.

        \cf{Bbn}[1]{Fast Movement} You gain a \plus10 foot bonus to your \glossterm{base speed}.

        \cf{Bbn}[2]{Savage Precision} You can use your Strength in place of your Perception to determine your \glossterm{accuracy} with the \textit{dirty trick}, \textit{disarm}, \textit{grapple}, \textit{overrun}, and \textit{trip} abilities, as well as with grapple actions (see \pcref{Special Combat Abilities}, and \pcref{Grapple Actions}).
        In addition, you gain a \plus1 bonus to \glossterm{accuracy} with those abilities and with the \textit{shove} ability.

        \cf{Bbn}[3]{Savage Force} You gain a \plus1d bonus to your damage with all weapons.

        \cf{Bbn}[4]{Greater Fast Movement} The speed bonus from your \textit{fast movement} ability increases to \plus20 feet.

        \cf{Bbn}[4]{Greater Savage Rush} Your \textit{savage rush} ability no longer causes you to treat spaces occupied by enemies as difficult terrain.

        \cf{Bbn}[5]{Greater Savage Precision} The accuracy bonus from your \textit{savage precision} ability increases to \plus2.
        In addition, choose one of the following \glossterm{weapon tags} (see \pcref{Weapon Tags}): Disarming, Forceful, Grappling, or Tripping.
        You may treat all weapons you wield as if they had the chosen weapon tag.

        \cf{Bbn}[6]{Greater Savage Force} The bonus from your \textit{savage force} ability increases to \plus2d.

        \cf{Bbn}[7]{Supreme Fast Movement} The speed bonus from your \textit{fast movement} ability increases to \plus30 feet.

        \cf{Bbn}[7]{Supreme Savage Rush} You can move through spaces occupied by enemies at all times.
        You treat spaces occupied by enemies as being \glossterm{difficult terrain} unless your movement is accelerated by the \textit{sprint} ability.

    \newpage
    \subsection{Primal Warrior}
        This archetype grants you abilities to use in combat and improves your physical skills.

        \cf{Bbn}[0]{Primal Might} You gain a \plus1 bonus to Strength-based \glossterm{checks} and Constitution-based \glossterm{checks}.

        {
            \cf{Bbn}[1]{Combat Styles}
            You can channel your primal energy into ferocious attacks.
            You gain access to one of the following \glossterm{combat styles}: \textit{dirty fighting}, \textit{herald of war}, or \textit{unbreakable defense}.
            In addition, you gain access to any two combat styles of your choice (see \pcref{Combat Styles}).
            You may spend \glossterm{insight points} to gain access to one additional combat style per insight point.

            You learn two rank 1 \glossterm{maneuvers} from combat styles you have access to.
            You may spend \glossterm{insight points} to learn to one additional maneuver per insight point.
            Unless otherwise noted in an ability's description, using a maneuver requires a \glossterm{standard action}.

            When you gain access to a new \glossterm{rank} in this archetype,
                you can exchange any number of maneuvers you know for other maneuvers,
                including maneuvers of the higher rank.
        }

        {
            \cf{Bbn}[2]{Combat Style Rank} You become a rank 2 combat style user.
            This gives you access to maneuvers that require a minimum rank of 2.

            \cf{Bbn}[2]{Primal Force} You gain a \plus1d bonus to your damage with all weapons.
        }

        {
            \cf*{Bbn}[3]{Combat Style Rank} You become a rank 3 combat style user.
            This gives you access to maneuvers that require a minimum rank of 3 and can improve the effectiveness of your existing maneuvers.

            \cf{Bbn}[3]{Glancing Strikes} Whenever you miss by 2 or less with a \glossterm{strike}, the target takes half damage from the strike.
            This is called a \glossterm{glancing blow}.
        }

        {
            \cf*{Bbn}[4]{Combat Style Rank} You become a rank 4 combat style user.
            This gives you access to maneuvers that require a minimum rank of 4 and can improve the effectiveness of your existing maneuvers.

            \cf{Bbn}[4]{Primal Maneuver}
            You learn an additional \glossterm{maneuver} from a combat style you have access to (see \pcref{Combat Styles}).
        }

        {
            \cf*{Bbn}[5]{Combat Style Rank} You become a rank 5 combat style user.
            This gives you access to maneuvers that require a minimum rank of 5 and can improve the effectiveness of your existing maneuvers.

            \cf{Bbn}[5]{Greater Primal Force} The bonus from your \textit{primal force} ability increases to \plus2d.
        }

        {
            \cf*{Bbn}[6]{Combat Style Rank} You become a rank 6 combat style user.
            This gives you access to maneuvers that require a minimum rank of 6 and can improve the effectiveness of your existing maneuvers.

            \cf{Bbn}[6]{Greater Primal Might} The bonuses from your \textit{primal might} ability increase to \plus2.
        }

        {
            \cf*{Bbn}[7]{Combat Style Rank} You become a rank 7 combat style user.
            This gives you access to maneuvers that require a minimum rank of 7 and can improve the effectiveness of your existing maneuvers.

            \cf*{Bbn}[7]{Primal Maneuver}
            You learn an additional \glossterm{maneuver} from a combat style you have access to (see \pcref{Combat Styles}).

        }


\newpage
\section{Cleric}\label{Cleric}
    \begin{dtable!*}
        \lcaption{Cleric Progression}
\begin{dtabularx}{\textwidth}{l l >{\lcol}X >{\lcol}X >{\lcol}X >{\lcol}X}
    \tb{Rank} & \tb{Min Level} & \tb{Divine Magic} & \tb{Divine Spell Mastery}   & \tb{Domain Influence}  & \tb{Healer} \tableheaderrule
    0 & \tdash & Cantrips                        & Mystic sphere               & Domains, domain gift & Desperate healing           \\
    1 & 1      & Spellcasting                    & Mystic insight              & Domain gift          & Restoration                 \\
    2 & 4      & Spell rank (2), spell knowledge & Divine guidance             & Domain aspect        & Healer's grace              \\
    3 & 7      & Spell rank (3)                  & Wellspring of power         & Domain aspect        & Divine healing              \\
    4 & 10     & Spell rank (4), spell knowledge & Mystic insight              & Domain essences       & Called to the needy         \\
    5 & 13     & Spell rank (5)                  & Greater divine guidance     & Miracle       & Greater healer's grace      \\
    6 & 16     & Spell rank (6)                  & Greater wellspring of power & Domain mastery              & Revivify                    \\
    7 & 19     & Spell rank (7), spell knowledge & Mystic insight              & Domain mastery     & Greater called to the needy \\
\end{dtabularx}
    \end{dtable!*}

    Clerics are divine spellcasters that draw power from their worship of a specific deity.
    Divine magic tends to be more subtle than other forms of magic, and focuses more on aiding yourself and allies than destroying foes with flashy bursts.
    A cleric's deity often grants them influence over specific aspects of the world that the deity has purview over, aligning the cleric's power more closely with that of the deity.
    Some clerics pursue a path of healing, allowing them to keep themselves and their allies healthy through dangerous trials.
    These clerics are commonly found in temples of their gods, tending to the needs of their fellow followers or the wider community.

    \classbasics{Alignment} Your alignment must be within one step of your deity's (that is, it may be one step away on either the lawful-chaotic axis or the good-evil axis, but not both).

    \classbasics{Archetypes} Clerics have the Divine Magic, Domain Influence, and Divine Spell Mastery \glossterm{archetypes}.

    \subsection{Basic Class Abilities}
        If you are a cleric, you gain the following abilities.
        % 6 insight, 4 skill, 8 attunement, 2 fatigue, 2 armor, 2 weapons = 24 points

        \cf{Clr}{Defenses}
        You gain the following bonuses to your \glossterm{defenses}: \plus1 Armor, \plus5 Fortitude, \plus3 Reflex, \plus7 Mental.

        \cf{Clr}{Resources} You have the following \glossterm{resources}:
        \begin{itemize}
            \item Three \glossterm{insight points}, which you can spend to gain additional abilities or proficiencies (see \pcref{Insight Points}).
            \item Six \glossterm{skill points}, which you can spend to learn skills (see \pcref{Skills}).
            \item Two \glossterm{attunement points}, which you can use to attune to items and abilities that affect you (see \pcref{Attunement Points}).
            \item A \plus1 bonus to your \glossterm{fatigue tolerance}, which makes it easier for you to use powerful abilities that fatigue you (see \pcref{Fatigue Level}).
        \end{itemize}

        \cf{Clr}{Weapon Proficiencies} 
        You are proficient with simple weapons and any one other \glossterm{weapon group}.

        \cf{Clr}{Armor Proficiencies} 
        You are proficient with light and medium armor.

        \cf{Clr}{Skills}
        You have the following \glossterm{class skills}:
        \begin{itemize}
            \item \subparhead{Intelligence} Craft, Deduction, Knowledge (arcana, local, religion, planes), Linguistics, Medicine.
            \item \subparhead{Perception} Awareness, Social Insight, Spellsense.
            \item \subparhead{Other} Deception, Intimidate, Persuasion, Profession.
        \end{itemize}

    \subsection{Special Class Abilities}

        \cf{Clr}{Deity}
        You must worship a specific deity to be a cleric.
        Deities and their associated domains are listed in \trefnp{Deities}.

        \begin{dtable!*}
            \lcaption{Deities}
            \begin{dtabularx}{\textwidth}{X l X}
                \tb{Deity} & \tb{Alignment} & \tb{Domains} \tableheaderrule
                Gregory, warrior god of mundanity     & Lawful good     & Law, Protection, Strength, War         \\
                Guftas, horse god of justice          & Lawful good     & Good, Law, Strength, Travel            \\
                Lucied, paladin god of justice        & Lawful good     & Destruction, Good, Protection, War     \\
                Simor, fighter god of protection      & Lawful good     & Good, Protection, Strength, War        \\
                Ayala, naiad god of water             & Neutral good    & Life, Magic, Water, Wild               \\
                Pabs, dwarf god of drink              & Neutral good    & Good, Life, Strength, Wild             \\
                Rucks, monk god of pragmatism         & Neutral good    & Good, Law, Protection, Travel          \\
                Vanya, centaur god of nature          & Neutral good    & Good, Strength, Travel, Wild           \\
                Brushtwig, pixie god of creativity    & Chaotic good    & Chaos, Good, Trickery, Wild            \\
                Camilla, tiefling god of fire         & Chaotic good    & Fire, Good, Magic, Protection          \\
                Chavi, wandering god of stories       & Chaotic good    & Chaos, Knowledge, Trickery             \\
                Chort, dwarf god of optimism          & Chaotic good    & Good, Life, Travel, Wild               \\
                Ivan Ivanovitch, bear god of strength & Chaotic good    & Chaos, Strength, War, Wild             \\
                Krunch, barbarian god of destruction  & Chaotic good    & Destruction, Good, Strength, War       \\
                Sir Cakes, dwarf god of freedom       & Chaotic good    & Chaos, Good, Strength                  \\
                Mikolash, scholar god of knowledge    & Lawful neutral  & Knowledge, Law, Magic, Protection      \\
                Raphael, monk god of retribution      & Lawful neutral  & Death, Law, Protection, Travel         \\
                Declan, god of fire                   & True neutral    & Destruction, Fire, Knowledge, Magic    \\
                Mammon, golem god of endurance        & True neutral    & Knowledge, Magic, Protection, Strength \\
                Kurai, shaman god of nature           & True neutral    & Air, Earth, Fire, Water                \\
                Amanita, druid god of decay           & Chaotic neutral & Chaos, Destruction, Life, Wild         \\
                Antimony, elf god of necromancy       & Chaotic neutral & Death, Knowledge, Life, Magic          \\
                Clockwork, elf god of time            & Chaotic neutral & Chaos, Magic, Trickery, Travel         \\
                Diplo, doll god of destruction        & Chaotic neutral & Chaos, Destruction, Strength, War      \\
                Lord Khallus, fighter god of pride    & Chaotic neutral & Chaos, Strength, War                   \\
                Celeano, sorcerer god of deception    & Chaotic neutral & Chaos, Magic, Protection, Trickery     \\
                Murdoc, god of mercenaries            & Chaotic neutral & Destruction, Knowledge, Travel, War    \\
                Ribo, halfling god of trickery        & Chaotic neutral & Chaos, Trickery, Water                 \\
                Tak, orc god of war                   & Lawful evil     & Law, Strength, Trickery, War           \\
                Theodolus, sorcerer god of ambition   & Neutral evil    & Evil, Knowledge, Magic, Trickery       \\
                Daeghul, demon god of slaughter       & Chaotic evil    & Destruction, Evil, Magic, War          \\
            \end{dtabularx}
        \end{dtable!*}

    \newpage
    \subsection{Divine Magic}
        This archetype grants you the ability to cast divine spells.
        All abilities from this archetype are \glossterm{magical}.

        \cf{Clr}[0]{Cantrips}
        Your deity grants you the ability to use divine magic.
        You gain access to one divine \glossterm{mystic sphere} (see \pcref{Divine Mystic Spheres}).
        You may spend \glossterm{insight points} to gain access to one additional divine \glossterm{mystic sphere} per two \glossterm{insight points}.
        You automatically learn all \glossterm{cantrips} from any mystic sphere you have access to.
        You do not yet gain access to any other spells from those mystic spheres.

        Divine spells require \glossterm{verbal components} to cast (see \pcref{Casting Components}).
        For details about mystic spheres and casting spells, see \pcref{Spell and Ritual Mechanics}.

        \cf{Clr}[1]{Spellcasting}
        You become a rank 1 divine spellcaster.
        You learn two rank 1 \glossterm{spells} from divine \glossterm{mystic spheres} you have access to.
        You can also spend \glossterm{insight points} to learn one additional rank 1 spell per \glossterm{insight point}.
        Unless otherwise noted in a spell's description, casting a spell requires a \glossterm{standard action}.

        When you gain access to a new \glossterm{mystic sphere} or spell \glossterm{rank},
            you can forget any number of spells you know to learn that many new spells in exchange,
            including spells of the higher rank.
        All of those spells must be from divine mystic spheres you have access to.

        \cf{Clr}[2]{Spell Rank} You become a rank 2 divine spellcaster.
        This gives you access to spells that require a minimum rank of 2.

        \cf{Clr}[2]{Spell Knowledge} You learn an additional divine \glossterm{spell} from a \glossterm{mystic sphere} you have access to.

        \cf*{Clr}[3]{Spell Rank} You become a rank 3 divine spellcaster.
        This gives you access to spells that require a minimum rank of 3 and can improve the effectiveness of your existing spells.

        \cf*{Clr}[4]{Spell Rank} You become a rank 4 divine spellcaster.
        This gives you access to spells that require a minimum rank of 4 and can improve the effectiveness of your existing spells.

        \cf*{Clr}[4]{Spell Knowledge} You learn an additional divine \glossterm{spell} from a \glossterm{mystic sphere} you have access to.

        \cf*{Clr}[5]{Spell Rank} You become a rank 5 divine spellcaster.
        This gives you access to spells that require a minimum rank of 5 and can improve the effectiveness of your existing spells.

        \cf*{Clr}[6]{Spell Rank} You become a rank 6 divine spellcaster.
        This gives you access to spells that require a minimum rank of 6 and can improve the effectiveness of your existing spells.

        \cf*{Clr}[7]{Spell Rank} You become a rank 7 divine spellcaster.
        This gives you access to spells that require a minimum rank of 7 and can improve the effectiveness of your existing spells.

        \cf*{Clr}[7]{Spell Knowledge} You learn an additional divine \glossterm{spell} from a \glossterm{mystic sphere} you have access to.

    \newpage
    \subsection{Divine Spell Mastery}
        This archetype improves the divine spells you cast.
        You must have the Divine Magic archetype from the cleric class to gain the abilities from this archetype.
        All abilities from this archetype are \glossterm{magical}.

        \cf{Clr}[0]{Mystic Sphere} You gain access to an additional divine \glossterm{mystic sphere}, including all \glossterm{cantrips} from that sphere.

        \cf{Clr}[1]{Mystic Insight}
        You gain your choice of one of the following abilities.
        Whenever you increase your rank in this archetype, you may change your choice.
        This can allow you to apply the benefits of insights like \textit{signature spell} to higher rank spells.
        {
            \parhead{Focused Caster} You reduce your \glossterm{focus penalty} by 1.
                You cannot choose this ability multiple times.
            \parhead{Insight Point} You gain an additional \glossterm{insight point}.
                You can choose this ability multiple times, gaining an additional insight point each time.
            \parhead{Rituals} You gain the ability to perform divine rituals to create unique magical effects (see \pcref{Rituals}).
                The maximum \glossterm{rank} of divine ritual you can learn or perform is equal to the maximum \glossterm{rank} of divine spell that you can cast.
                You cannot choose this ability multiple times.
            \parhead{Signature Spell} Choose a divine \glossterm{spell} you know.
                The spell loses the \abilitytag{Focus} tag, allowing you to cast it without lowering your guard in combat.
                In adition, you gain a \plus1 bonus to \glossterm{accuracy} with that spell.
                You can choose this ability multiple times, choosing a different spell each time.
            \parhead{Spell Power} Choose a divine \glossterm{spell} you know.
                You gain a bonus equal to your rank in this archetype to your \glossterm{power} with that spell.
                You can choose this ability multiple times, choosing a different spell each time.
        }

        \cf{Clr}[2]{Divine Guidance} Once per \glossterm{long rest}, you may use the \textit{desperate exertion} ability without increasing your \glossterm{fatigue level} to affect a divine spell you cast (see \pcref{Desperate Exertion}).

        \cf{Clr}[3]{Wellspring of Power}
        You gain a \plus2 bonus to your \glossterm{magical} \glossterm{power}.

        \cf*{Clr}[4]{Mystic Insight}
        You gain an additional \textit{mystic insight} ability.

        \cf{Clr}[5]{Greater Divine Guidance} You can use your \textit{divine guidance} ability once per \glossterm{short rest} instead of once per long rest.

        \cf{Clr}[6]{Greater Wellspring of Power} The bonus from your \textit{wellspring of power} ability increases to \plus6.

        \cf*{Clr}[7]{Mystic Insight}
        You gain an additional \textit{mystic insight} ability.

    \newpage
    \subsection{Domain Influence}
        This archetype grants you divine influence over two domains of your choice.
        All abilities from this archetype are \glossterm{magical}.

        \cf{Clr}[0]{Domains}
        You choose two domains which represent your personal spiritual inclinations.
        You must choose your domains from among those your deity offers.
        The domains are listed below.

        \begin{itemize}
            \item{Air}
            \item{Chaos}
            \item{Death}
            \item{Destruction}
            \item{Earth}
            \item{Evil}
            \item{Fire}
            \item{Good}
            \item{Knowledge}
            \item{Law}
            \item{Life}
            \item{Magic}
            \item{Protection}
            \item{Strength}
            \item{Travel}
            \item{Trickery}
            \item{War}
            \item{Water}
            \item{Wild}
        \end{itemize}

        \cf{Clr}[0]{Domain Gift}
        Each domain has a corresponding \textit{domain gift}.
        A domain gift is a passive ability that reinforces your ability to embody your domain.
        You gain the \textit{domain gift} for one of your domains (see \pcref{Cleric Domain Abilities}).

        \cf*{Clr}[1]{Domain Gift}
        You gain the \textit{domain gift} for another one of your domains.

        \cf{Clr}[2]{Domain Aspect}
        Each domain has a corresponding \textit{domain aspect}.
        A domain aspect is an active ability that allows you to exert the influence of your domain in the world.
        You gain the \textit{domain aspect} ability for one of your domains (see \pcref{Cleric Domain Abilities}).

        \cf*{Clr}[3]{Domain Aspect}
        You gain the \textit{domain aspect} for another one of your domains.

        \cf{Clr}[4]{Domain Essence}
        Each domain has a corresponding \textit{domain essence}.
        You gain the \textit{domain essence} for both of your domains (see \pcref{Cleric Domain Abilities}).

        \cf{Clr}[5]{Miracle}
        Once per week, you can request a miracle as a standard action.
        You mentally specify your request, and your deity fulfills that request in the manner it sees fit.
        This can emulate the effects of any spell or ritual, or have any other effect of a similar power level.
        If the deity has a direct interest in your situation, the miracle may be of even greater power.

        If you perform an extraordinary service for your deity, you can gain the ability to request an additional miracle that week.

        \cf{Clr}[6]{Domain Mastery}
        Each domain has a corresponding \textit{domain mastery}.
        You gain the \textit{domain mastery} for one of your domains (see \pcref{Cleric Domain Abilities}).

        \cf*{Clr}[7]{Domain Mastery}
        You gain the \textit{domain mastery} for another one of your domains.

    \newpage
    \subsection{Healer}
        This archetype grants you healing abilities.

        \cf{Clr}[0]{Desperate Healing} When you use the \textit{desperate exertion} ability to affect a Medicine check, you gain a \plus3 bonus to the check.
        This stacks with the normal +2 bonus from the \textit{desperate exertion} ability.
        In addition, using the \textit{desperate exertion} ability to affect a Medicine check only increases your \glossterm{fatigue level} by one instead of two.

        \cf{Clr}[1]{Restoration} You can use the \textit{restoration} ability as a standard action.
        \begin{instantability}{Restoration}
            \spelltwocol{Instant}{\abilitytag{Healing}, \glossterm{Magical}}
            \rankline
            Choose yourself or one living \glossterm{ally} within your \glossterm{reach}.
            The subject regains 1d8 \glossterm{hit points}.
            If the subject is a creature other than yourself, they also regain hit points equal to your \glossterm{power}.
            After you use this ability, you cannot use it or any other \abilitytag{Healing} ability until after the end of the next round.

            \rankline
            \rank{2} The healing increases to 1d10.
            \rank{3} The healing increases to 2d8.
            \rank{4} The healing increases to 2d10.
            \rank{5} The healing increases to 4d8.
            \rank{6} The healing increases to 4d10.
            \rank{7} The healing increases to 6d10.
        \end{instantability}

        \cf{Clr}[2]{Healer's Grace} You gain a \plus1 bonus to all defenses.
        Whenever you attack or deal damage to a living creature, you lose this bonus until the end of the next round.

        \cf{Clr}[3]{Divine Healing} You can use the \textit{divine healing} ability as a standard action.
        \begin{instantability}{Divine Healing}
            \spelltwocol{Instant}{\glossterm{Magical}}
            \rankline
            When you use this ability, you increase your \glossterm{fatigue level} by two (see \pcref{Fatigue}).

            Choose yourself or one living \glossterm{ally} within your \glossterm{reach}.
            The subject removes one of its \glossterm{vital wounds}.

            \rankline
            \rank{5} If the subject's level is at least two levels lower than your level,
                you do not increase your fatigue level when you use this ability.
            \rank{7} The subject can remove an additional \glossterm{vital wound}.
        \end{instantability}

        \cf{Clr}[4]{Called to the Needy} At the end of each phase, if a living \glossterm{ally} within \distrange of you gained a \glossterm{vital wound} during that phse, you can \glossterm{teleport} into the unoccupied square closest to that creature.

        \cf{Clr}[5]{Greater Healer's Grace} The bonus from your \textit{healer's grace} ability increases to \plus2.

        \cf{Clr}[6]{Revivify} You can use the \textit{revivify} ability as a standard action.
        \begin{instantability}{Revivify}
            Instant
            \rankline
            When you use this ability, you increase your \glossterm{fatigue level} by three (see \pcref{Fatigue}).

            Choose one intact corpse within your \glossterm{reach}.
            If it belongs to a creature that has been dead for no more than 1 minute, that creature is restored to life, as the \ritual{resurrection} ritual.
            After using this ability, you cannot use it again until you take a \glossterm{long rest}.
        \end{instantability}

        \cf{Clr}[7]{Greater Called to the Needy} You can use your \textit{called to the needy} ability to teleport to allies that lost \glossterm{hit points} in addition to allies that gained vital wounds.

    \newpage
    \subsection{Cleric Domain Abilities}\label{Cleric Domain Abilities}
        These domain abilities can be granted by the \textit{domain influence} cleric archetype.
        All cleric domain abilities are \glossterm{magical} unless otherwise specified.

        \subsubsection{Air}
            If you choose this domain, you add the \sphere{aeromancy} \glossterm{mystic sphere} to your list of divine mystic spheres (see \pcref{Mystic Spheres}).
            In addition, you add the Jump skill to your \glossterm{class skill} list.

            \parhead{Gift} You gain a \plus4 bonus to the Jump skill (see \pcref{Jump}).
            In addition, you take half damage from \glossterm{falling damage}.
            \parhead{Aspect} You gain a \glossterm{glide speed} equal to your \glossterm{base speed} (see \pcref{Gliding}).
            \parhead{Essence} You can use the \textit{speak with air} ability as a standard action.
            \begin{attuneability}{Speak with Air}
                \abilitytag{Attune} (self)
                \rankline
                You can speak with and command air within a \areahuge radius \glossterm{zone} from your location.
                You can ask the air simple questions and understand its responses.
                If you command the air to perform a task, it will do so do the best of its ability until this effect ends.
                You cannot compel the air to move faster than 50 mph.

                After you use this ability on a particular area of air, you cannot use it again on that same area for 24 hours.

                \rankline
                \rank{6} The area increases to a \areagarg radius.
            \end{attuneability}
            \parhead{Mastery} You gain a \glossterm{fly speed} equal to your \glossterm{base speed} with a maximum height of 120 feet (see \pcref{Flying}).
            At the start of each phase, you can increase your \glossterm{fatigue level} by one to ignore this height limit until the end of the round.

        \subsubsection{Chaos}
            If you choose this domain, you add the Deception skill to your \glossterm{class skill} list.

            \parhead{Gift} You are immune to \abilitytag{Compulsion} attacks.
            \parhead{Aspect} If you roll a 1 on an attack roll, it explodes (see \pcref{Exploding Attacks}).
            This does not affect bonus dice rolled for exploding attacks.
            \parhead{Essence} You can use the \textit{twist of fate} ability as a standard action.
            \begin{instantability}{Twist of Fate}
                Instant
                \rankline
                An improbable event occurs within \rnglong range.
                You can specify in general terms what you want to happen, such as ``Make the bartender leave the bar''.
                You cannot control the exact nature of the event, though it always beneficial for you in some way.
                After using this ability, you cannot use it again until you take a \glossterm{long rest}.
            \end{instantability}
            \parhead{Mastery} You gain a \plus5 bonus to \glossterm{accuracy} with any attack roll that explodes (see \pcref{Exploding Attacks}).

        \subsubsection{Death}
            % Lame gift
            \parhead{Gift} You gain a \plus2 bonus to Fortitude defense.
            \parhead{Aspect} Whenever you kill a creature, you gain a \plus2 bonus to \glossterm{accuracy} until the end of the next round.
            \parhead{Essence} You can use the \textit{speak with dead} ability as a standard action.
            \begin{attuneability}{Speak with Dead}
                \abilitytag{Attune} (self)
                \rankline
                Choose a corpse within \rngshort range.
                The corpse must have died no more than 24 hours ago.
                It regains a semblance of life, allowing you to speak with it as if it were the creature the corpse belonged to.
                The creature is able to refuse to speak with you, though you can attempt to persuade it to speak normally, and some creatures may be more willing to talk if they know they are already dead.
                The corpse must have an intact mouth to be able to speak.
                This ability ends if 24 hours have passed since the creature died.
            \end{attuneability}
            \parhead{Mastery} Whenever you inflict a \glossterm{vital wound} on a living creature, you may cause that creature to immediately die.

        \subsubsection{Destruction}
            % This ability gets rank upgrades while maneuvers don't
            % because maneuvers are upgraded into higher rank maneuvers,
            % while this has no upgrades and should remain relevant alone.
            \parhead{Gift} You can use the \textit{destructive attack} ability as a standard action.
            \begin{instantability}{Destructive Attack}
                Instant
                \rankline
                Make a \glossterm{strike} with a \minus2 penalty to \glossterm{accuracy}.
                You gain a \plus2d bonus damage with the strike.

                \rankline
                \rank{3} The damage bonus increases to \plus3d.
                \rank{5} The damage bonus increases to \plus4d.
                \rank{7} The damage bonus increases to \plus5d.
            \end{instantability}
            \parhead{Aspect} Your abilities deal double damage to objects.
            \parhead{Essence} You can use the \textit{lay waste} ability as a standard action.
            \begin{instantability}{Lay Waste}
                Instant
                \rankline
                Make an attack vs. Fortitude against all unattended objects in a \areamed radius.
                You may freely exclude any number of 5-ft\. cubes from the area, as long as the resulting area is still contiguous.
                \hit For each target, if its \glossterm{resistance} against \glossterm{physical damage} is lower than your \glossterm{power}, it crumbles into a fine power and is irreparably \glossterm{destroyed}.

                \rankline
                \rank{6} The area increases to a \arealarge radius.
            \end{instantability}
            \parhead{Mastery} You gain a \plus4 bonus to your \glossterm{power} with all abilities.

        \subsubsection{Earth}
            If you choose this domain, you add the \sphere{terramancy} \glossterm{mystic sphere} to your list of divine mystic spheres (see \pcref{Mystic Spheres}).

            \parhead{Gift} You gain a \plus2 bonus to Fortitude defense.
            \parhead{Aspect} You gain a \plus4 bonus to your maximum \glossterm{hit points}.
            \parhead{Essence} You can use the \textit{speak with earth} ability as a standard action.
            \begin{attuneability}{Speak with Earth}
                \abilitytag{Attune} (self)
                \rankline
                You can speak with earth within a \areahuge radius \glossterm{zone} from your location.
                You can ask the earth simple questions and understand its responses.

                After you use this ability on a particular area of earth, you cannot use it again on that same area for 24 hours.

                \rankline
                \rank{6} The area increases to a \areagarg radius.
            \end{attuneability}
            \parhead{Mastery} The bonus from this domain's gift increases to \plus5, and the number of hit points you gain from its aspect increases to 16.

        \subsubsection{Evil}
            \parhead{Gift} At the start of each phase, you may choose an adjacent \glossterm{ally}.
            If you do, the first time you would lose a \glossterm{hit point} that phase, the target loses that hit point instead.
            If the target is unable to lose hit points, such as if it has no hit points remaining, you suffer the hit point loss normally.
            \parhead{Aspect} You can use this domain's domain gift to target any \glossterm{ally} within \rngmed range.
            \parhead{Essence} You can use the \textit{compel evil} ability as a standard action.
            \begin{durationability}{Compel Evil}
                \spelltwocol{Duration}{\abilitytag{Compulsion}}
                \rankline
                Make an attack vs. Mental against a creature within \rngmed range.
                Creatures who have strict codes prohibiting them from taking evil actions, such as paladins devoted to Good, are immune to this ability.
                \hit The target takes an evil action as soon as it can.
                Once it takes the evil action, this effect ends.
                You have no control over the act the creature takes, but circumstances can make the target more likely to take an action you desire.
                After this effect ends, the subject becomes immune to this effect until it takes a \glossterm{short rest}.
                \glance As above, except that the effect ends at the end of the next round.

                \rankline
                You gain a \plus1 bonus to \glossterm{accuracy} with the attack for each rank beyond 4.
            \end{durationability}
            \parhead{Mastery} You can use your domain gift to redirect your hit point loss to an adjacent unwilling creature.
            You cannot target the same unwilling creature more than once with this ability between \glossterm{short rests}.

        \subsubsection{Fire}
            If you choose this domain, you add the \sphere{pyromancy} \glossterm{mystic sphere} to your list of divine mystic spheres (see \pcref{Mystic Spheres}).

            \parhead{Gift} You gain a bonus equal to twice your rank in this archetype to your \glossterm{resistance} against fire damage.
            \parhead{Aspect} Your abilities cannot deal fire damage to your \glossterm{allies}.
            \parhead{Essence} You can use the \textit{speak with fire} ability as a standard action.
            \begin{attuneability}{Speak with Fire}
                \abilitytag{Attune} (self)
                \rankline
                You can speak with and command fire within a \areahuge radius \glossterm{zone} from your location.
                You can ask the fire simple questions and understand its responses.
                If you command the fire to perform a task, it will do so do the best of its ability until this effect ends.
                You cannot compel the fire to move farther than 30 feet in a single round.
                Fire that ends the round on non-combustable materials usually goes out, depending on the circumstances.

                After you use this ability on a particular area of fire, you cannot use it again on that same area for 24 hours.
                % TODO: What does an ``area of fire'' mean?

                \rankline
                \rank{6} The area increases to a \areagarg radius.
            \end{attuneability}
            \parhead{Mastery} Whenever you deal fire damage, you also treat that damage as being pure energy damage.
            This can help you deal damage to enemies that are highly resistant to fire damage.
            In addition, you become immune to fire damage.

        \subsubsection{Good}
            \parhead{Gift} Whenever an adjacent \glossterm{ally} suffers a \glossterm{vital wound}, you may gain a \glossterm{vital wound} instead.
            You gain a \plus2 bonus to the \glossterm{vital roll} of each \glossterm{vital wound} you gain this way.
            The original target suffers any other effects of the attack normally.
            \parhead{Aspect} This domain's domain gift affects any \glossterm{ally} within a \areamed radius \glossterm{emanation} from you.
            \parhead{Essence} You can use the \textit{compel good} ability as a standard action.
            \begin{instantability}{Compel Good}
                \spelltwocol{Instant}{\abilitytag{Compulsion}}
                \rankline
                Make an attack vs. Mental against a creature within \rngmed range.
                Creatures who have strict codes prohibiting them from taking evil actions, such as paladins devoted to Good, are immune to this ability.
                \hit The target takes a good action as soon as it can.
                Once it takes the good action, this effect ends.
                You have no control over the act the creature takes, but circumstances can make the target more likely to take an action you desire.
                After this effect ends, the subject becomes immune to this effect until it takes a \glossterm{short rest}.
                \glance As above, except that the effect ends at the end of the next round.

                \rankline
                You gain a \plus1 bonus to \glossterm{accuracy} with the attack for each rank beyond 4.
            \end{instantability}
            \parhead{Mastery} Once per round, when an \glossterm{ally} within a \areamed radius \glossterm{emanation} from you would lose \glossterm{hit points}, you may lose those hit points instead.
            The target suffers any other effects of the attack normally, though it is not treated as if it lost hit points from the attack for the purpose of special attack effects.

        \subsubsection{Knowledge}
            If you choose this domain, you add all Knowledge skills to your cleric \glossterm{class skill} list.

            \parhead{Gift} You gain two skill points.
            \parhead{Aspect} Your extensive knowledge of all methods of attack and defense grants you a \plus1 bonus to all defenses.
            \parhead{Essence} You can use the \textit{share knowledge} ability as a standard action.
            \begin{instantability}{Share Knowledge}
                Instant
                \rankline
                Make a Knowledge check of any kind.
                You and your \glossterm{allies} within a \arealarge radius learn the results of your check.
                Creatures believe the information gained in this way to be true as if they it had seen it with their own eyes.

                You cannot alter the knowledge you gain with this check in any way, such as by adding or withholding information.

                \rankline
                \rank{6} You gain a \plus3 bonus to the Knowledge check.
            \end{instantability}
            \parhead{Mastery} You gain a \plus1 bonus to \glossterm{accuracy} with all attacks.
            In addition, you can use your \textit{share knowledge} ability to affect all creatures, not just your allies.

        \subsubsection{Law}
            \parhead{Gift} You gain a \plus2 bonus to Mental defense.
            % Clarify - does this apply to exploding dice?
            \parhead{Aspect} When you roll a 1 on an \glossterm{attack roll}, it is treated as if you had rolled a 6.
            \parhead{Essence} You can use the \textit{compel law} ability as a standard action.
            \begin{durationability}{Compel Law}
                \spelltwocol{Duration}{\abilitytag{Compulsion}}
                \rankline
                Make an attack vs. Mental against all creatures within a \arealarge radius from you.
                \hit Each subject is unable to break the laws that apply in the area, and any attempt to do so simply fails.
                The laws which are applied are those which are most appropriate for the area, regardless of whether you or any other creature know those laws.

                % Sufficiently clear that this isn't part of the hit effect?
                When you use this ability, you also gain the condition.
                If this condition is removed from you, it is also removed from all other affected creatures.
                In areas under ambiguous or nonexistent government, this ability may have unexpected effects, or it may have no effect at all.
                \glance As above, except that the effect ends at the end of the next round.

                \rankline
                You gain a \plus1 bonus to \glossterm{accuracy} with the attack for each rank beyond 4.
            \end{durationability}
            % on average, slightly better than +1 accuracy
            \parhead{Mastery} When you roll less than a 6 on an \glossterm{attack roll}, it is treated as if you had rolled a 6.

        \subsubsection{Life}
            \parhead{Gift} You gain a \plus4 bonus to the Medicine skill (see \pcref{Medicine}).
            \parhead{Aspect} You gain a \plus1 bonus to \glossterm{vital rolls} (see \pcref{Vital Rolls}).
            \parhead{Essence} At the end of each round, if you became \unconscious from a \glossterm{vital wound} that round, you can use one \glossterm{magical} ability you have that modifies \glossterm{vital rolls} or removes \glossterm{vital wounds} on yourself without taking an action.
            \parhead{Mastery} You gain a \plus1 bonus to your base Constitution.

        \subsubsection{Magic}
            If you choose this domain, you add the \sphere{thaumaturgy} \glossterm{mystic sphere} to your list of divine mystic spheres (see \pcref{Mystic Spheres}).

            \parhead{Gift} You gain a \plus4 bonus to the Spellsense skill (see \pcref{Spellsense}).
            \parhead{Aspect} You learn an additional divine \glossterm{spell} from a \glossterm{mystic sphere} you have access to.
            \parhead{Essence} You gain a \plus4 bonus to your \glossterm{magical} \glossterm{power}.
            \parhead{Mastery} The power bonus from this domain's essence increases to \plus8.

        \subsubsection{Protection}
            \parhead{Gift} You gain a bonus equal to twice your rank in this archetype to your \glossterm{resistance} against \glossterm{energy damage}.
            \parhead{Aspect} You can use the \textit{divine protection} ability as a \glossterm{minor action}.
            \begin{durationability}{Divine Protection}
                \spelltwocol{Duration}{\abilitytag{Swift}}
                \rankline
                Choose an \glossterm{ally} adjacent to you.
                It gains a \plus2 bonus to all defenses until the end of the round.
                Because this ability has the \abilitytag{Swift} tag, this bonus applies against attacks made in the current phase.

                A creature that sees an attack against an ally protected in this way can observe that you are the cause of the protection with a \glossterm{difficulty rating} 5 Awareness check.
                While this ability is active, you cannot gain a defense bonus from this ability, even if another creature with this ability uses it on you.
            \end{durationability}
            \parhead{Essence} The bonus from this domain's gift increases to be equal to three times your rank in this archetype.
            \parhead{Mastery} The bonus from your \textit{divine protection} ability increases to \plus3.

        \subsubsection{Strength}
            If you choose this domain, you add the Climb, Jump, and Swim skills to your cleric \glossterm{class skill} list.

            \parhead{Gift} You gain two skill points.
            \parhead{Aspect} You gain a \plus4 bonus to Strength for the purpose of checks and determining your carrying capacity.
            \parhead{Essence} You can use the \textit{divine strength} ability as a \glossterm{minor action}.
            \begin{attuneability}{Divine Strength}
                \abilitytag{Attune} (self)
                \rankline
                You gain a \plus4 \glossterm{magic bonus} to Strength.
            \end{attuneability}
            \parhead{Mastery} You gain a \plus2 bonus to your base Strength.

        \subsubsection{Travel}
            If you choose this domain, you add the \sphere{astromancy} \glossterm{mystic sphere} to your list of divine mystic spheres (see \pcref{Mystic Spheres}).
            In addition, you add the Knowledge (geography), Survival, and Swim skills to your cleric \glossterm{class skill} list.

            \parhead{Gift} You gain two skill points.
            \parhead{Aspect} You gain a \plus20 foot \glossterm{magic bonus} to your \glossterm{base speed}, up to a maximum of double your normal speed.
            \parhead{Essence} You can use the \textit{dimensional travel} ability as a standard action.
            \begin{instantability}{Dimensional Travel}
                Instant
                \rankline
                You teleport up to 1 mile in any direction.
                You do not need \glossterm{line of sight} or \glossterm{line of effect} to your destination, but you must be able to clearly visualize it.

                \rankline
                \rank{6} The maximum distance increases to 5 miles.
            \end{instantability}
            \parhead{Mastery} When you would move using one of your movement speeds, you can teleport the same distance instead.
            This does not change the total distance you can move, but you can teleport in any direction, including vertically.
            Being \grappled or otherwise physically constrained does not prevent you from teleporting in this way.

            You can even attempt to move to locations outside of \glossterm{line of sight} and \glossterm{line of effect}, up to the limit of your remaining movement speed.
            If your intended destination is invalid, the distance you tried to teleport is taken from your remaining movement, but you suffer no other ill effects.

        \subsubsection{Trickery}
            If you choose this domain, you add the Deception, Disguise, and Stealth skills to your cleric \glossterm{class skill} list.

            \parhead{Gift} You gain two skill points.
            \parhead{Aspect} You gain a \plus2 bonus to the Deception, Disguise, and Stealth skills.
            \parhead{Essence} You can use the \textit{compel belief} ability as a standard action.
            \begin{durationability}{Compel Belief}
                \spelltwocol{\abilitytag{Sustain} (minor)}{\abilitytag{Compulsion}}
                \rankline
                Make an attack vs. Mental against a creature within \rngmed range.
                You must also choose a belief that the target has.
                The belief may be a lie that you told it, or even a simple misunderstanding (such as believing a hidden creature is not present in a room).
                If the creature does not already hold the chosen belief, this ability automatically fails.
                \hit The subject continues to maintain the chosen belief, regardless of any evidence to the contrary.
                It will interpret any evidence that the falsehood is incorrect to be somehow wrong -- an illusion, a conspiracy to decieve it, or any other reason it can think of to continue believing the falsehood.
                At the end of the effect, the creature can decide whether it believes the falsehood or not, as normal.
                After this effect ends, the subject becomes immune to this effect until it takes a \glossterm{short rest}.
                \glance As above, except that the effect ends at the end of the next round.

                \rankline
                You gain a \plus1 bonus to \glossterm{accuracy} with the attack for each rank beyond 4.
            \end{durationability}
            % This seems like it's a complicated muddle of weird and possibly hilarious edge cases
            \parhead{Mastery} You are undetectable to all \glossterm{magical} abilities.
            They cannot detect your presence, sounds you make, or any actions you take.
            For example, a scrying sensor created by a \abilitytag{Scrying} effect would be unable to detect your presence, and a creature with magical \glossterm{darkvision} would not be able to see you without light.

        \subsubsection{War}
            \parhead{Gift} You gain proficiency with heavy armor and an additional \glossterm{weapon group} of your choice.
            \parhead{Aspect} You deal half damage to your allies with \glossterm{magical} abilities.
            \parhead{Essence} Whenever you miss by 2 or less with a \glossterm{strike}, the target takes half damage from the strike.
            This is called a \glossterm{glancing blow}.
            \parhead{Mastery} You gain a \plus4 bonus to your \glossterm{power} with all abilities.

        \subsubsection{Water}
            If you choose this domain, you add the \sphere{aquamancy} \glossterm{mystic sphere} to your list of divine mystic spheres (see \pcref{Mystic Spheres}).
            In addition, you add the Flexibility and Swim skills to your cleric \glossterm{class skill} list.

            \parhead{Gift} You gain a \plus2 bonus to the Flexibility and Swim skills.
            \parhead{Aspect} You can breathe water as easily as a human breathes air, preventing you from drowning or suffocating underwater.
            \parhead{Essence} You can use the \textit{speak with water} ability as a standard action.
            \begin{attuneability}{Speak with Water}
                \abilitytag{Attune} (self)
                \rankline
                You can speak with and command water within a \areahuge \glossterm{zone} from your location.
                You can ask the water simple questions and understand its responses.
                If you command the water to perform a task, it will do so do the best of its ability until this effect ends.
                You cannot compel the water to move faster than 30 feet per round.

                After you use this ability on a particular area of water, you cannot use it again on that same area for 24 hours.

                \rankline
                \rank{6} The area increases to a \areagarg radius.
            \end{attuneability}
            \parhead{Mastery} Your body becomes partially aquatic, allowing you to manipulate it more easily.
            The bonuses from this domain's gift increase to \plus10.
            In addition, you gain a \plus1 bonus to Armor and Reflex defenses.

        \subsubsection{Wild}
            If you choose this domain, you add the \sphere{verdamancy} \glossterm{mystic sphere} to your list of divine mystic spheres (see \pcref{Mystic Spheres}).
            In addition, you add the Creature Handling, Knowledge (nature), and Survival skills to your cleric \glossterm{class skill} list.

            \parhead{Gift} You gain two skill points.
            % TODO: clarify whether you can have this and the druid ability
            \parhead{Aspect} This ability functions like the \textit{wild aspect} druid ability from the Shifter archetype (see \pcref{Shifter}), except that you cannot spend \glossterm{insight points} to learn additional wild aspects.
            \parhead{Essence} You learn an additional \textit{wild aspect}.
            \parhead{Mastery} You can maintain both of your wild aspects simultaneously.

        \subsubsection{Ex-Clerics}
            If you grossly violate the code of conduct required by your deity, you lose all spells and magical cleric class abilities.
            You cannot regain those abilities until you atone for your transgressions to your deity.

\newpage
\section{Druid}\label{Druid}
    \begin{dtable!*}
        \lcaption{Druid Progression}
\begin{dtabularx}{\textwidth}{l l >{\lcol}X >{\lcol}X l >{\lcol}X >{\lcol}X}
    \tb{Rank} & \tb{Min Level} & \tb{Elementalist}         & \tb{Nature Magic}               & \tb{Nature Spell Mastery}   & \tb{Shifter}                                    & \tb{Wildspeaker}         \tableheaderrule
    0         & \tdash         & Elemental balance         & Cantrips                        & Combat caster               & Shifting defense                                & Animal speech                            \\
    1         & 1              & Elemental magic           & Spellcasting                    & Mystic insight              & Wild aspects                                    & Natural servant                          \\
    2         & 4              & Elemental influence       & Spell rank (2), spell knowledge & Natural guidance            & Shift body                                      & Nature's ally                            \\
    3         & 7              & Greater elemental balance & Spell rank (3)                  & Wellspring of power         & Greater shifting defense, natural force         & Nature's might, plant speech             \\
    4         & 10             & Elemental versatility     & Spell rank (4), spell knowledge & Mystic insight              & Greater wild aspect                             & Greater natural servant                  \\
    5         & 13             & Elemental control         & Spell rank (5)                  & Greater natural guidance    & Greater shift body                              & Greater nature's ally                    \\
    6         & 16             & Elemental versatility     & Spell rank (6)                  & Greater wellspring of power & Greater natural force, supreme shifting defense & Greater nature's might, universal speech \\
    7         & 19             & Supreme elemental balance & Spell rank (7), spell knowledge & Mystic insight              & Wild adaptability                               & Supreme natural servant                  \\
\end{dtabularx}
    \end{dtable!*}

    Druids are nature spellcasters that draw power from their veneration of the natural world.
    Nature magic can be used to help allies or to crush foes with natural elements like air or fire.
    Many druids live outside of organized society, either on their own or in small communes with other druids and similarly minded people.
    Druids have strong ties to the natural world with a variety of abilities other than their spells.
    They can speak with and command animals or elements, or even change their body to take on on aspects of animals themselves.

    \classbasics{Alignment} Neutral good, lawful neutral, neutral, chaotic neutral, or neutral evil.

    \classbasics{Archetypes} Druids have the Elementalist, Nature Magic, Nature Spell Mastery, Shifter, and Wildspeaker \glossterm{archetypes}.

    \subsection{Basic Class Abilities}
        If you are a druid, you gain the following abilities.
        % 6 insight, 6 skill, 4 attunement, 4 fatigue, 2 armor, 2 weapons = 24 points

        \cf{Drd}{Defenses}
        You gain the following bonuses to your \glossterm{defenses}: \plus1 Armor, \plus5 Fortitude, \plus4 Reflex, \plus6 Mental.

        \cf{Drd}{Resources} You have the following \glossterm{resources}:
        \begin{itemize}
            \item Three \glossterm{insight points}, which you can spend to gain additional abilities or proficiencies (see \pcref{Insight Points}).
            \item Nine \glossterm{skill points}, which you can spend to learn skills (see \pcref{Skills}).
            \item One \glossterm{attunement point}, which you can use to attune to items and abilities that affect you (see \pcref{Attunement Points}).
            \item A \plus2 bonus to your \glossterm{fatigue tolerance}, which makes it easier for you to use powerful abilities that fatigue you (see \pcref{Fatigue}).
        \end{itemize}

        \cf{Drd}{Weapon Proficiencies} 
        You are proficient with simple weapons, any one other \glossterm{weapon group}, scimitars, and sickles.

        \cf{Drd}{Armor Proficiencies} 
        You are proficient with light and medium armor.

        \cf{Drd}{Skills}
        You have the following \glossterm{class skills}:
        \begin{itemize}
            \item \subparhead{Strength} Climb, Jump, Swim.
            \item \subparhead{Dexterity} Agility, Ride, Stealth.
            \item \subparhead{Constitution} Endurance.
            \item \subparhead{Intelligence} Craft, Deduction, Knowledge (geography, nature), Medicine.
            \item \subparhead{Perception} Awareness, Creature Handling, Survival.
            \item \subparhead{Other} Deception, Intimidate, Persuasion, Profession.
        \end{itemize}

    \newpage
    \subsection{Elementalist}\label{Elementalist}
        This archetype grants you influence over four elements that define the natural world: air, earth, fire, and water.
        All abilities from this archetype are \glossterm{magical}.

        \cf{Drd}[0]{Elemental Balance} You gain a small benefit from each of the four elements.
        \begin{itemize}
            \item Air: You gain a \plus2 bonus to the Jump skill.
            \item Earth: You gain a \plus1 bonus to Fortitude defense.
            \item Fire: You gain a bonus equal to twice your rank in this archetype to your \glossterm{resistance} against fire damage.
            \item Water: You gain a \plus2 bonus to the Swim skill.
        \end{itemize}

        \cf{Drd}[1]{Elemental Magic} Choose one of the \glossterm{mystic spheres} associated with the four elements: \sphere{aeromancy}, \sphere{aquamancy}, \sphere{pyromancy}, or \sphere{terramancy}.
        If you already have access to that mystic sphere, you learn two spells from that sphere.
        Otherwise, you gain access to that mystic sphere, including all \glossterm{cantrips} from that sphere.

        \cf{Drd}[2]{Elemental Influence} You can use the \textit{elemental influence} ability as a standard action.
        \begin{durationability}{Elemental Influence}
            \abilitytag{Sustain} (standard)
            \rankline
            You can speak with air, earth, fire, and water within a \areahuge \glossterm{zone} from your location.
            You can ask the elements simple questions and understand their responses.
            Each element has different limitations on its memory and awareness, as described below.

            \begin{itemize}
                \item Air: Air can remember events up to an hour ago on a very calm day or only a few minutes ago on a windy day.
                    Moving air is aware of events near where it blew through, not necessarily in your current location.
                \item Earth: Earth can remember events up to a year ago, but its awareness is extremely limited.
                    It can only remember very large events, such as giant creatures tearing up the terrain, earthquakes, or major construction.
                    Earth can tell you whether there exist underground tunnels within the area, but any sort of detailed mapping is beyond its ability to communicate.
                \item Fire: Fire can remember everything it touched and consumed since it started burning.
                    Individual pieces of a very large fire, such as a particular burning tree in a forest fire, are not aware of the behavior of the entirety of the fire.
                    However, the fire on burning tree could tell you how it got to the tree and everything it burned along the way, including the event that started the forest fire.
                \item Water: Water can remember events up to a day ago in a very calm pool or only a few minutes ago in a turbulent river.
                    Moving water is aware of events near where it moved through, not necessarily in your current location.
            \end{itemize}

            % Oddly placed? there must be text between an itemize block and the end of a mdframed env
            Air, earth, and water are only able to give information about what they touch.
            This includes the general shapes, sizes, and locations of creatures and objects they interacted with, but not any details about color or subjective appearance.
            Fire is also able to give information about anything illuminated by its light, allowing it to report more detailed information like color.
            It is still unable to make meaningful subjective judgments like a creature would.
        \end{durationability}

        \cf{Drd}[3]{Greater Elemental Balance} The bonuses from your \textit{elemental balance} ability improve.
        \begin{itemize}
            \item Air: You gain a \glossterm{glide speed} equal to your \glossterm{base speed}.
            \item Earth: The bonus to Fortitude defense increases to \plus2.
            \item Fire: The bonus to your \glossterm{resistance} increases to be equal to three times your rank in this archetype.
            \item Water: You gain a \glossterm{swim speed} equal to your \glossterm{base speed}.
        \end{itemize}

        \cf{Drd}[4]{Elemental Versatility} You learn a spell from any of the spheres associated with the four elements: \sphere{aeromancy}, \sphere{aquamancy}, \sphere{pyromancy}, or \sphere{terramancy}.
        You do not have to have access to that mystic sphere.
        As normal, you can change which spell you learn with this ability as you gain access to new spell ranks.

        \cf{Drd}[5]{Elemental Control} When you use your \textit{elemental influence} ability, you can also command the elements to move as you desire.
        Each element has different limitations on its ability to move, as described below.
        \begin{itemize}
            \item Air: You can change the wind speed of air by up to 50 miles per hour.
                If you reduce the air's speed to 0 and then increase it again, you can change the direction the air blows.
            \item Earth: You can reshape earth or unworked stone at a rate of up to one foot per round.
            \item Fire: You can make fire leap up to 30 feet between combustable materials, suppress fire so it smolders without being extinguished, or snuff out fire entirely.
            \item Water: You can change the speed of water by up to 30 feet per round.
                If you reduce the water's speed to 0 and then increase it again, you can change the direction the water flows.
        \end{itemize}

        \cf*{Drd}[6]{Elemental Versatility} You learn an additional spell with your \textit{elemental versatility} ability.

        \cf{Drd}[7]{Supreme Elemental Balance} The bonuses from your \textit{elemental balance} ability improve.
        \begin{itemize}
            \item Air: You gain a \glossterm{fly speed} equal to your \glossterm{base speed} with a maximum height of 60 feet (see \pcref{Flying}).
            At the start of each phase, you can increase your \glossterm{fatigue level} by one to ignore this height limit until the end of the round.
            \item Earth: The bonus to Fortitude defense increases to \plus4.
            \item Fire: The bonus to your \glossterm{resistance} increases to be equal to four times your rank in this archetype.
            \item Water: You suffer no penalties for fighting underwater, and you can breathe water as if it was air.
        \end{itemize}

    \newpage
    \subsection{Nature Magic}
        This archetype grants you the ability to cast nature spells.
        All abilities from this archetype are \glossterm{magical}.

        \cf{Drd}[0]{Cantrips}
        Your deity grants you the ability to use nature magic.
        You gain access to one nature \glossterm{mystic sphere} (see \pcref{Nature Mystic Spheres}).
        You may spend \glossterm{insight points} to gain access to one additional nature \glossterm{mystic sphere} per two \glossterm{insight points}.
        You automatically learn all \glossterm{cantrips} from any mystic sphere you have access to.
        You do not yet gain access to any other spells from those mystic spheres.

        Nature spells require \glossterm{verbal components} to cast (see \pcref{Casting Components}).
        For details about mystic spheres and casting spells, see \pcref{Spell and Ritual Mechanics}.

        \cf{Drd}[1]{Spellcasting}
        You become a rank 1 nature spellcaster.
        You learn two rank 1 \glossterm{spells} from nature \glossterm{mystic spheres} you have access to.
        You can also spend \glossterm{insight points} to learn one additional rank 1 spell per \glossterm{insight point}.
        Unless otherwise noted in a spell's description, casting a spell requires a \glossterm{standard action}.

        When you gain access to a new \glossterm{mystic sphere} or spell \glossterm{rank},
            you can forget any number of spells you know to learn that many new spells in exchange,
            including spells of the higher rank.
        All of those spells must be from nature mystic spheres you have access to.

        \cf{Drd}[2]{Spell Rank} You become a rank 2 nature spellcaster.
        This gives you access to spells that require a minimum rank of 2.

        \cf{Drd}[2]{Spell Knowledge} You learn an additional nature \glossterm{spell} from a \glossterm{mystic sphere} you have access to.

        \cf*{Drd}[3]{Spell Rank} You become a rank 3 nature spellcaster.
        This gives you access to spells that require a minimum rank of 3 and can improve the effectiveness of your existing spells.

        \cf*{Drd}[4]{Spell Rank} You become a rank 4 nature spellcaster.
        This gives you access to spells that require a minimum rank of 4 and can improve the effectiveness of your existing spells.

        \cf*{Drd}[4]{Spell Knowledge} You learn an additional nature \glossterm{spell} from a \glossterm{mystic sphere} you have access to.

        \cf*{Drd}[5]{Spell Rank} You become a rank 5 nature spellcaster.
        This gives you access to spells that require a minimum rank of 5 and can improve the effectiveness of your existing spells.

        \cf*{Drd}[6]{Spell Rank} You become a rank 6 nature spellcaster.
        This gives you access to spells that require a minimum rank of 6 and can improve the effectiveness of your existing spells.

        \cf*{Drd}[7]{Spell Rank} You become a rank 7 nature spellcaster.
        This gives you access to spells that require a minimum rank of 7 and can improve the effectiveness of your existing spells.

        \cf*{Drd}[7]{Spell Knowledge} You learn an additional nature \glossterm{spell} from a \glossterm{mystic sphere} you have access to.

    \newpage
    \subsection{Nature Spell Mastery}
        This archetype improves the nature spells you cast.
        You must have the Nature Magic archetype from the cleric class to gain the abilities from this archetype.
        All abilities from this archetype are \glossterm{magical}.

        \cf{Drd}[0]{Combat Caster} You reduce your \glossterm{focus penalties} by 2.

        \cf{Drd}[1]{Mystic Insight}
        You gain your choice of one of the following abilities.
        Whenever you increase your rank in this archetype, you may change your choice.
        This can allow you to apply the benefits of insights like \textit{signature spell} to higher rank spells.
        {
            \parhead{Focused Caster} You reduce your \glossterm{focus penalty} by 1.
                You cannot choose this ability multiple times.
            \parhead{Insight Point} You gain an additional \glossterm{insight point}.
                You can choose this ability multiple times, gaining an additional insight point each time.
            \parhead{Rituals} You gain the ability to perform nature rituals to create unique magical effects (see \pcref{Rituals}).
                The maximum \glossterm{rank} of nature ritual you can learn or perform is equal to the maximum \glossterm{rank} of nature spell that you can cast.
                You cannot choose this ability multiple times.
            \parhead{Signature Spell} Choose a nature \glossterm{spell} you know.
                The spell loses the \abilitytag{Focus} tag, allowing you to cast it without lowering your guard in combat.
                In adition, you gain a \plus1 bonus to \glossterm{accuracy} with that spell.
                You can choose this ability multiple times, choosing a different spell each time.
            \parhead{Spell Power} Choose a nature \glossterm{spell} you know.
                You gain a bonus equal to your rank in this archetype to your \glossterm{power} with that spell.
                You can choose this ability multiple times, choosing a different spell each time.
        }

        \cf{Drd}[2]{Natural Guidance} Once per \glossterm{long rest}, you may use the \textit{desperate exertion} ability without increasing your \glossterm{fatigue level} to affect a nature spell you cast (see \pcref{Desperate Exertion}).

        \cf{Drd}[3]{Wellspring of Power}
        You gain a \plus2 bonus to your \glossterm{magical} \glossterm{power}.

        \cf*{Drd}[4]{Mystic Insight}
        You gain an additional \textit{mystic insight} ability.

        \cf{Drd}[5]{Greater Natural Guidance} You can use your \textit{natural guidance} ability once per \glossterm{short rest} instead of once per long rest.

        \cf{Drd}[6]{Greater Wellspring of Power}
        The bonus from your \textit{wellspring of power} ability increases to \plus6.

        \cf*{Drd}[7]{Mystic Insight}
        You gain an additional \textit{mystic insight} ability.

    \newpage
    \subsection{Shifter}\label{Shifter}
        This archetype grants you the ability to embody aspects of the natural world in your own form.

        \cf{Drd}[0]{Shifting Defense} You gain a \plus2 bonus to Fortitude, Reflex, or Mental defense.
        You can change the defense this bonus applies to as a \glossterm{standard action}.

        \cf{Drd}[1]{Wild Aspects}[Magical]
        You gain the ability to embody an aspect of an animal or of nature itself.
        Choose two wild aspects from the list below.
        You can also spend \glossterm{insight points} to learn one additional \textit{wild aspect} per \glossterm{insight point}.

        As a \glossterm{standard action}, you can gain the effects of one wild aspect that you know.
        You cannot change your wild aspect more than once per round.
        That effect lasts until you activate a different wild aspect you know or until you dismiss it as a \glossterm{free action}.

        The abilities in the list below describe the effects of the aspect.
        Your appearance also changes to match the aspect's effects, but the nature of this change is not described.
        Different druids change in different ways.
        For example, one druid might grow brown fur when using the Form of the Bear, while another might instead change their face to become broader and more bear-shaped when embodying the same aspect.
        You choose how your appearance changes when you gain a wild aspect.
        This change cannot be used to gain an additional substantive benefit beyond the effects given in the description of the aspect.

        Many wild aspects grant natural weapons.
        See \pcref{Natural Weapons}, for details about natural weapons.

        {
            \begin{durationability}{Form of the Bear}
                Duration
                \rankline
                You gain a \plus2 bonus to Fortitude defense.
                In addition, your mouth and hands transform, granting you a bite and two claw \glossterm{natural weapons} (see \tref{Natural Weapons}).

                \rankline
                \rank{3} The Fortitude bonus increases to \plus3.
                \rank{5} You gain a \plus1d bonus to your damage with natural weapons.
                \rank{7} The Fortitude bonus increases to \plus4.
            \end{durationability}

            \begin{durationability}{Form of the Bull}
                Duration
                \rankline
                You gain a \plus2 bonus to \glossterm{accuracy} with the \textit{shove} ability (see \pcref{Shove}).
                In addition, your head transforms, granting you a gore \glossterm{natural weapon} (see \tref{Natural Weapons}).

                \rankline
                \rank{3} The accuracy bonus increases to \plus3.
                \rank{5} You can move your full movement speed when you push a creature with the \textit{shove} ability even if you do not get a critical hit.
                \rank{7} When you push a creature with the the \textit{shove} ability, the pushed creature also takes damage as if you had hit it with your gore natural weapon.
                This damage cannot be combined with other effects that deal damage with a shove, such as the \textit{wall slam} ability.
            \end{durationability}

            \begin{durationability}{Form of the Constrictor}
                Duration
                \rankline
                You gain a \plus2 bonus to \glossterm{accuracy} with the \textit{grapple} ability and all grapple actions (see \pcref{Grapple}).
                In addition, you gain a constrict \glossterm{natural weapon} (see \tref{Natural Weapons}).
                This weapon deals 1d10 damage, and it has the Grappling weapon tag (see \pcref{Weapon Tags}).
                It can only be used against a foe you are grappling with.

                \rankline
                \rank{3} The accuracy bonus increases to \plus3.
                \rank{5} You can contort your body, allowing it to act as a free hand for the purpose of using the \textit{grapple} ability and grapple actions even if you do not have a free hand.
                \rank{7} When you grapple a creature with the \textit{grapple} ability, you are not considered to be \grappled (see \pcref{Asymmetric Grappling}).
            \end{durationability}

            \begin{durationability}{Form of the Fish}
                Duration
                \rankline
                You gain a \glossterm{swim speed} equal to your \glossterm{base speed}.
                In addition, you gain a bite \glossterm{natural weapon} (see \tref{Natural Weapons}).

                \rankline
                \rank{3} You can breathe water as easily as a human breathes air, preventing you from drowning or suffocating underwater.
                \rank{5} You suffer no penalties for acting underwater.
                \rank{7} You are immune to \glossterm{magical} effects that restrict your mobility.
                In addition, you gain a \plus4 bonus to defenses against the \textit{grapple} ability and grapple actions (see \pcref{Grapple}).
            \end{durationability}

            \begin{durationability}{Form of the Hawk}
                Duration
                \rankline
                You gain \glossterm{low-light vision}.
                If you already have low-light vision, you double its benefit, allowing you to treat sources of light as if they had four times their normal illumination range.
                In addition, you gain a \plus3 bonus to Awareness.

                \rankline
                \rank{3} You grow wings, granting your a glide speed equal to your \glossterm{base speed} (see \pcref{Gliding}).
                \rank{5} The Awareness bonus increases to \plus6.
                \rank{7} The glide speed is replaced with a \glossterm{fly speed} equal to your \glossterm{base speed} with a maximum height of 60 feet (see \pcref{Flying}).
                At the start of each phase, you can increase your \glossterm{fatigue level} by one to ignore this height limit until the end of the round.
            \end{durationability}

            \begin{durationability}{Form of the Hound}
                Duration
                \rankline
                You gain the ability to move on all four limbs.
                When doing so, you gain a \plus20 foot bonus to your land speed, up to a maximum of double your original speed.
                When not using your hands to move, your ability to use your hands is unchanged.
                You can descend to four legs and rise up to stand on two legs again as part of movement.
                In addition, you gain a bite \glossterm{natural weapon} (see \tref{Natural Weapons}).

                \rankline
                \rank{3} You gain the \glossterm{scent} ability.
                \rank{5} You gain a \plus10 foot bonus to your land speed.
                \rank{7} You gain an additional \plus10 bonus to scent-based Awareness checks (see \pcref{Awareness}).
            \end{durationability}

            % Seems boring? What abilities would make sense?
            \begin{durationability}{Form of the Monkey}
                Duration
                \rankline
                You gain a \glossterm{climb speed} equal to your \glossterm{base speed}.
                In addition, you gain a bite \glossterm{natural weapon} (see \tref{Natural Weapons}).

                \rankline
                \rank{3} You grow a tail that you can use as a free hand for the purpose of climbing.
                \rank{5} You gain a \plus10 foot bonus to your climb speed.
                \rank{7} You can use the \textit{creature climb} ability against creatures only one size category larger than you instead of two size categories.
            \end{durationability}

            \begin{durationability}{Form of the Mouse}
                Duration
                \rankline
                You gain a \plus2 bonus to the Flexibility and Stealth skills.
                In addition, you gain a bite \glossterm{natural weapon} (see \tref{Natural Weapons}).
                
                \rankline
                \rank{3} When you use this wild aspect, you can choose to shrink by one \glossterm{size category}.
                \rank{5} The skill bonuses increases to \plus4.
                \rank{7} When you use this wild aspect, you can choose to shrink by up to two \glossterm{size categories} instead of only one.
            \end{durationability}

            % \begin{durationability}{Form of the Oak}
            %     Duration
            %     \rankline
            %     As long as you have any remaining resistance to physical damage, you are both \glossterm{immobilized} and \glossterm{impervious} to \glossterm{physical damage}.
            %     \rankline
            %     \rank{3} You also gain a \plus1 bonus to Armor defense.
            %     \rank{5} The resistance bonus increases to be equal to three times your rank in this archetype.
            %     \rank{7} The defense bonuse increases to \plus2.
            % \end{durationability}

            \begin{durationability}{Form of the Viper}
                Duration
                \rankline
                You gain a \glossterm{climb speed} equal to half your \glossterm{base speed}.
                You do not need to use your hands to climb in this way.
                In addition, you gain a bite \glossterm{natural weapon} (see \tref{Natural Weapons}).

                \rankline
                \rank{3} When a creature takes damage from your bite \glossterm{natural weapon}, it is poisoned.
                At the end of each round, you make an attack vs. Fortitude against the target.
                If you hit, the target is \sickened until it removes the poison.
                The poison is removed if you miss the target on this attack three times.
                \rank{5} You gain a \plus1d bonus to your damage with natural weapons.
                \rank{7} The poison makes the target \nauseated instead of \sickened.
            \end{durationability}

            \begin{durationability}{Form of the Wolf}
                Duration
                \rankline
                You gain a \plus1 bonus to \glossterm{accuracy} against \surrounded creatures.
                In addition, you gain a bite \glossterm{natural weapon} (see \tref{Natural Weapons}).

                \rankline
                \rank{3} The accuracy bonus increases to \plus2.
                \rank{5} You gain a \plus1d bonus to your damage with natural weapons.
                \rank{7} The accuracy bonus increases to \plus3.
            \end{durationability}

            \begin{durationability}{Myriad Form}
                Duration
                \rankline
                You can use your \glossterm{power} in place of your Disguise skill when making Disguise checks to alter your own appearance.

                \rankline
                \rank{3} When you use this wild aspect, you can choose to grow or shrink by one \glossterm{size category}.
                    Your physical form is not altered fully to match your new size, and your Strength and Dexterity are unchanged.
                \rank{5} You can use the \textit{disguise creature} ability to disguise yourself as a \glossterm{standard action} (see \pcref{Disguise Creature}).
                \rank{7} When you use this wild aspect, you can choose to grow or shrink by up to two \glossterm{size categories} instead of only one.
            \end{durationability}

            \begin{durationability}{Photosynthesis}
                Duration
                \rankline
                As long as you are in natural sunlight, you gain a \plus10 foot bonus to your \glossterm{base speed}.
                \rankline
                \rank{3} As long as you are in natural sunlight, you do not gain hunger or thirst.
                When you leave natural sunlight, you continue gaining hunger or thirst at your normal rate, ignoring any time you spent in natural sunlight.
                \rank{5} The speed bonus increases to \plus20 feet.
                \rank{7} When you take a \glossterm{short rest} while you are in natural sunlight, you remove a \glossterm{vital wound}.
            \end{durationability}

            \begin{durationability}{Plantspeaker}
                Duration
                \rankline
                Your speed is not reduced when moving in light or heavy \glossterm{undergrowth}.
                In addition, you can ignore \glossterm{cover} and \glossterm{concealment} (but not \glossterm{total cover}) from plants whenever doing so would be beneficial to you, as the plants move out of the way to help you.
                This prevents you from suffering penalties on your attacks, and also prevents creatures from using cover or concealment from plants to hide from you.

                \rankline
                \rank{3} You gain a \plus1 bonus to Armor and Reflex defenses while standing in \glossterm{undergrowth}.
                \rank{5} The movement penalties from \glossterm{undergrowth} are doubled for enemies within a \areahuge radius emanation from you.
                \rank{7} The bonus to Armor and Reflex defenses increases to \plus2.
            \end{durationability}
        }

        \cf{Drd}[2]{Shift Body} You can use the \textit{shift body} ability whenever you finish a \glossterm{long rest}.
        \begin{attuneability}{Shift Body}
            \abilitytag{Attune} (self)
            \rankline
            When you use this ability, choose a physical \glossterm{attribute}: Strength, Dexterity, or Constitution (see \pcref{Attributes}).
            You gain a \plus1 bonus to the base value of that attribute.
        \end{attuneability}

        \cf{Drd}[3]{Glancing Natural Strikes} Whenever you miss by 2 or less with a \glossterm{strike} using a \glossterm{natural weapon}, the target takes half damage from the strike.
        This is called a \glossterm{glancing blow}.

        \cf{Drd}[3]{Greater Shifting Defense}
        The bonus from your \textit{shifting defense} ability increases to \plus3.

        \cf{Drd}[4]{Greater Wild Aspect}[Magical] You can change your \textit{wild aspect} as a \glossterm{minor action} instead of as a standard action.
        In addition, you learn an additional \textit{wild aspect}.

        \cf{Drd}[5]{Greater Shift Body} The bonus from your \textit{shift body} ability increases to \plus2.

        \cf{Drd}[6]{Natural Force} You gain a \plus1d bonus to your damage with natural weapons.

        \cf{Drd}[6]{Supreme Shifting Defense}
        The bonus from your \textit{shifting defense} ability increases to \plus4.

        \cf*{Drd}[7]{Supreme Wild Aspect} You can change your \textit{wild aspect} as a \glossterm{free action} instead of as a minor action.
        In addition, you learn an additional \textit{wild aspect}.

        % TODO: move this into a separate concept that is less archetype specific where each character can gain immortality
        % \cf{Drd}[8]{Avatar of Nature}[Magical]
        % If you die, except if by old age, you may choose to have your body and soul become an instrument of nature's will.
        % Your body immediately decomposes or otherwise disappears, and your soul does not travel to an afterlife.
        % You have no physical form, and cannot use any of your normal abilities.
        % Instead, you have a \glossterm{fly speed} of 100 feet, with special maneuverability.
        % As a standard action, you can temporarily possess any living plants or animals within a 10 mile radius of the place of your death.

        % While possessing a living plant or animal, you can see through its senses and control its actions completely.
        % In addition, you may cast spells, and the spells take effect as if the plant or animal had cast them.
        % You use the plant or animal's position to determine range, visible targets, and so on.
        % You do not require \glossterm{verbal components} to cast your spells in this form.
        % % TODO: are there any spells or rituals that have material components or focus objects?
        % % but are unable to cast spells or perform rituals that require material components or focus objects.

        % While not possessing a plant or animal, you can rest, or you can focus on reincarnating your physical form.
        % Creating a new body in this way takes 12 consecutive hours of concentration.
        % At the end of that time, you are reincarnated in a new body in your location, as the effect of the \ritual{reincarnation} ritual, except that you can choose your species from among the species listed (not including the ``Other'' species).

        % While you are an avatar of nature, you do not age and you cannot die of old age.
        % You can continue to exist in this form indefinitely.

    \newpage
    \subsection{Wildspeaker}\label{Wildspeaker}
        All abilities from this archetype are \glossterm{magical}.

        \cf{Drd}[0]{Animal Speech} You can use the \textit{animal speech} ability as a standard action.
        \begin{durationability}{Animal Speech}
            \abilitytag{Sustain} (minor)
            \rankline
            Choose an animal within \rnglong range.
            You can speak to and understand the speech of the target animal, and any other animals of the same species.

            This ability does not make the target any more friendly or cooperative than normal.
            Wary and cunning animals are likely to be terse and evasive, while stupid ones tend to make inane comments and are unlikely to say or understand anything of use.
        \end{durationability}

        \cf{Drd}[1]{Natural Servant}
        You can use the \textit{natural servant} ability.
        This ability requires spending 1 hour performing rituals in a natural area.
        \begin{attuneability}{Natural Servant}
            \abilitytag{Attune} (self)
            \rankline
            An animal native to the local environment appears to help you.
            It follows your directions to the best of its ability as long as you remain in its natural environment.
            If you leave the animal's natural habitat, it remains behind and this effect ends.
            If the animal gains a \glossterm{vital wound} or has no hit points remaining at the end of the round, this effect ends.

            Your magical connection to the animal improves its resilience and strength in combat.
            The animal's statistics use the values below, except that each animal also gains a special ability based on the environment you are in.
            \begin{itemize}
                % TODO: figure out why this is a 2
                \item Its \glossterm{fatigue tolerance} is 2.
                \item Its \glossterm{hit points} are equal to the base value for your level (see \tref{Character Advancement}).
                \item Its \glossterm{resistances} are equal to the base value for your level (see \pcref{Character Advancement}).
                \item Each of its \glossterm{defenses} is equal to 5 \add your level.
                \item Its \glossterm{accuracy} is equal to your level \add half your base Perception \add your \glossterm{magic bonuses} to accuracy.
                \item Its \glossterm{power} with its attacks is 0.
                \item Its \glossterm{base speed} is the normal base speed for its size (see \tref{Size in Combat}).
                \item It has no \glossterm{attunement points}.
                \item The damage dealt by its natural weapons increases by \plus1d for each rank in this archetype beyond 1.
            \end{itemize}

            % Oddly placed? there must be text between an itemize block and the end of a mdframed env
            Animals are unable to understand complex concepts, so their ability to obey convoluted instructions is limited.
        \end{attuneability}

        The special ability of the animal that appears depends on your environment, as described below.
        You may choose a different animal native to that environment that is similar in size and type, but that does not change the animal's statistics.
        For example, your \textit{natural servant} in an aquatic environment may be a fish or seal instead of a shark.
        Unusual environments may have different animals than the standard animals listed below.
        \begin{itemize}
            \item Aquatic: A Medium shark appears that has a 30 foot \glossterm{swim speed} and no land speed.
                It has a bite \glossterm{natural weapon}.
            \item Arctic: An Medium arctic fox appears that has no penalties for being in cold environments.
                It has a bite \glossterm{natural weapon}.
            \item Desert: A Medium hyena appears that has no penalties for being in hot environments.
                It has a bite \glossterm{natural weapon}.
            \item Mountain: A Medium goat appears that can move up or down steep slopes without slowing its movement.
                It has a ram \glossterm{natural weapon}.
            \item Forest: A Medium wolverine appears that has two additional \glossterm{hit points}.
                It has a bite \glossterm{natural weapon}.
            \item Plains: A Medium wolf appears that has the \glossterm{scent} ability.
                It has a bite \glossterm{natural weapon}.
            \item Swamp: A Medium crocodile appears that has a 15 foot \glossterm{land speed} and a 25 foot \glossterm{swim speed}.
                It has a bite \glossterm{natural weapon}.
            \item Underground: A Medium dire rat appears that has \glossterm{low-light vision}.
                It has a bite \glossterm{natural weapon}.
        \end{itemize}

        \cf{Drd}[2]{Nature's Ally} Wild animals will not willingly attack you or your \glossterm{allies} within a \largearea radius \glossterm{emanation} from you.
        They can be compelled to attack despite this protection with a Creature Handling check against a \glossterm{difficulty rating} equal to 10 \add your level.
        If any subject attacks a creature that this ability protects you from, this ability is \glossterm{suppressed} until you take a \glossterm{short rest}.

        \cf{Drd}[3]{Nature's Might} You and your \textit{natural servant} gain a \plus1d bonus to damage with all weapons.

        \cf{Drd}[3]{Plant Speech} When you use your \textit{animal speech} ability, you can choose a plant instead of an animal.
        When you do, you can speak to and understand the speech of the target plant, and any other plants of the same species.

        \cf{Drd}[4]{Greater Natural Servant} Your \textit{natural servant} gains an \glossterm{attunement point}.
        This attunement point is shared among any creatures you summon with your \textit{natural servant} ability, and is only recovered when you take a \glossterm{long rest}.
        In addition, you can summon an alternate \textit{natural servant} based on your local environment, as described below.
        \begin{itemize}
            \item Aquatic: A Medium shark appears that has a 40 foot \glossterm{swim speed} and no land speed.
                It has a bite \glossterm{natural weapon}.
            \item Arctic: An Medium polar bear appears that has a bonus equal to twice your rank in this archetype to its \glossterm{resistance} against cold damage.
                It has a bite \glossterm{natural weapon} and two claw \glossterm{natural weapons}.
            \item Desert: A Medium camel appears that has no penalties for being in hot environments.
                It has a bite \glossterm{natural weapon}.
            \item Mountain: A Medium goat appears that can move up or down steep slopes without slowing its movement.
                It has a ram \glossterm{natural weapon}.
            \item Forest: A Medium bear appears that has a bonus to its \glossterm{hit points} equal to twice your rank in this archetype.
                It has a bite \glossterm{natural weapon} and two claw \glossterm{natural weapons}.
            \item Plains: A Medium wolf appears that has the \glossterm{scent} ability.
                It has a bite \glossterm{natural weapon}.
            % TODO: define shallow water
            \item Swamp: A Medium crocodile appears that has a 20 foot \glossterm{land speed} and a 30 foot \glossterm{swim speed}.
                It has a bite \glossterm{natural weapon}.
            \item Underground: A Medium dire rat appears that has \glossterm{low-light vision} and \glossterm{darkvision}.
                It has a bite \glossterm{natural weapon}.
        \end{itemize}

        \cf{Drd}[5]{Greater Nature's Ally} Your \textit{nature's ally} ability also protects you and your allies from plant creatures and elementals.
        In addition, all creatures that you are protected from with this ability automatically attempt to aid you and your allies if they observe you fighting.
        Finally, the effect can no longer be bypassed with a Creature Handling check or any other form of control that does not first suppress this effect.
        Even creatures summoned by enemies to fight you will immediately turn on their summoners or otherwise avoid attacking you.

        \cf{Drd}[6]{Universal Speech} When you use your \textit{animal speech} ability, you can choose any living creature that knows at least one language.
        When you do, you can speak in and understand that creature's native language.

        \cf{Drd}[6]{Nature's Might} The damage bonus from your \textit{nature's might} ability increases to \plus2d.

        \cf{Drd}[7]{Supreme Natural Servant} Your \textit{natural servant} gains two additional \glossterm{attunement points}.
        In addition, you may choose a natural servant from any environment that is within a 50 mile walking distance from your current location, rather than only gaining a natural servant from your current environment.

        % \subcf{15th -- Air Mantle}
        % You are surrounded by a mantle of air.
        % Thrown and projectile weapons have a 50\% chance to miss your while this effect is active.
        % Unusually large weapons, such as a giant's boulders, may suffer a decreased miss chance as appropriate to their size.
        % \subcf{15th -- Aqueous Step}
        % Wherever the druid moves, you leave a path of animated water that can grab creatures.
        % When a creature crosses the path, the druid makes a Reflex attack to trip the creature, causing it to fall prone and waste the rest of its movement.
        % Your accuracy is equal to your druid level \add your Constitution.
        % \subcf{15th -- Flaming Step}
        % Wherever the druid moves, you leave a path of burning flame behind your that lasts for 1 round.
        % When a creature crosses the path, the druid makes a Reflex attack to deal damage to the creature.
        % The attack deals 1d8 points of fire damage per two druid levels.
        % Your accuracy is equal to your druid level \add your Constitution.
        % A failed attack deals half damage.
        % \subcf{15th -- Lifegiving Step}
        % Wherever the druid moves, you leave a path of small, living plants that entangle foes for 1 round.
        % When a creature crosses the path, the druid makes a Reflex attack to entangle the creature, causing it to waste the rest of its movement.
        % Your accuracy is equal to your druid level \add your Constitution.
        % The plants appear on any surface, and will continue to grow if they can survive, though they may die quickly if they appear on inhospitable terrain.
        % \subcf{17th -- Flaming Soul}
        % You gain the fire subtype, making your immune to fire but giving your a 50\% vulnerability to cold damage.
        % In addition, when you deal fire damage to a creature, the creature is \ignited for 5 rounds.
        % \subcf{17th -- Sunblessed Rejuvenation}
        % You gain fast healing equal to your druid level as long as you remain in sunlight or touches a plant of your size or larger.
        % \subcf{17th -- Sunscour}
        % This aspect functions like the heart of the sun natural aspect, except that it also suppresses shadow effects and the visual components of illusions within the area of bright light.

        % \subcf{17th -- Water's Flow}
        % As a minor action, the druid can transform herself into a rushing flow of water with a volume roughly equal to your normal volume until the end of your turn.
        % In this form, she may move wherever water could go, but she cannot take other actions, such as jumping, attacking, or casting spells.
        % Your speed is halved when moving uphill and doubled when moving downhill.
        % She may move through squares occupied by enemies without penalty.
        % She may return to your normal form as a free action.
        % \par If the water is split, she may reform from anywhere the water has reached, to as little as a single ounce of water.
        % If not even an ounce of water exists contiguously, your body reforms from the largest available parts of water, cut into pieces of appropriate size.
        % This usually causes the druid to die.

    \subsection{Ex-Druids}
        A druid who ceases to revere nature or who changes to a prohibited alignment loses all magical druid class abilities.
        They cannot thereafter gain levels as a druid until they atone for their transgressions.

        % \subsection{Variant Druids}

        %     \subsubsection{Blighter}

        %         Blighters draw power from nature, as do other druids. However, while other druids revere nature and draw power from it gently, blighters steal power from nature forcefully. Wherever a blighter goes, destruction and death surely follows.

        %         \altcf{Blight} Instead of meditating to regain spell slots, a blighter draws power from your environment forcefully.
        %         This affects a \arealarge radius zone centered on your, and the process takes 1 minute of concentration.
        %         At the end of every round, every living thing in the area other than the blighter takes damage equal to your nature power.
        %         All inanimate plants of Huge size or smaller immediately wither and die.
        %         The earth becomes cracked and infertile, and any nutrients from the soil are destroyed.
        %         This ability has no effect on artificial environments or materials, such as metal or worked stone.
        %         At the end of the minute, the blighter regains your spent nature spell slots.

        %         A blighter can only blight your surroundings in this way once per hour.
        %         If your surroundings are already blighted or are not natural terrain, she cannot use this ability to regain your spells.
        %         Instead, she must meditate for 8 hours to slowly draw power from your surroundings, as a normal druid.

        %         \altcf{Spells} As normal, except that a blighter adds all Vivimancy arcane spells to your spell list.

        %         \altcf[2]{Wild Speech} As normal, except that a blighter gains a \plus5 bonus to Intimidate against your wild speech targets, and a \minus5 penalty to Persuasion.

        %         \altcf[10]{Blightcasting}

        %         \altcf[20]{Improved Blightcasting}

        % \subsubsection{Rotbringer}

        %     While most druids seek to emulate and interact with animals, rotbringers focus on the power of fungi, decay, and regeneration.

        %     \altcf{Invoke Rot} Instead of meditating to regain spell slots, a rotbringer accelerates the natural forces of decomposition and decay on your environment.
        %     This affects a \arealarge radius zone centered on your, and the process takes 1 minute of concentration.
        %     All organic objects of Huge size or smaller, such as plants and corpses, decompose.
        %     This decomposition kills inanimate, living plants.
        %     All organic objects, regardless of size, are covered with various fungi.
        %     This ability has no effect on artificial environments or materials, such as metal or worked stone.
        %     At the end of the minute, the rotbringer regains your spent nature spell slots.

        %     If the rotbringer decomposes a Huge object with this ability, or a combination of smaller objects equivalent in size to a Huge object, you gain an bonus nature spell slot of your highest available spell level.
        %     This extra spell slot lasts until it is used, or until you regain your spell slots again.

        %     A rotbringer can only invoke rot on your surroundings in this way once per hour.
        %     If your surroundings are already decomposed or are not natural terrain, she cannot use this ability to regain your spells.
        %     Instead, she must meditate for 8 hours to slowly draw power from your surroundings, as a normal druid.

        %     \altcf[2]{Wild Speech} The rotbringer gains the ability to speak with plants at 2nd level.
        %     You gain the ability to speak with animals at 6th level, instead of at 2nd level.

        %     \altcf[3rd]{Wild Aspect} The rotbringer does not gain this ability.

        %     \altcf[3rd]{Rot Spell} The druid learns an additional spell slot and spell known.
        %     The spell must be taken from the following list of spells.
        %     The spell's level cannot exceed half your druid level.
        %     If she already knows a spell from the list at every spell level you ha access to, she may instead learn any nature spell (see \pcref{Nature Spells}).

        %     At 5th level, and every odd level, the druid may learn a new spell.

        %     \begin{dtable}
        %         \begin{dtabularx}{\columnwidth}{l X}
        %             \tb{Spell level} & \tb{Rotbringer Spells} \\
        %             1st & \spell{excrete slime}, \spell{lesser regeneration} \\
        %             2nd & \spell{fungal growth} \\
        %             3rd & \spell{rotburst} \\
        %             4th & \spell{poison} \\
        %             6th & \spell{regeneration} \\
        %             7th & \spell{greater rotburst} \\
        %         \end{dtabularx}
        %     \end{dtable}

        %     \altcf[7]{Fungal Armor} The rotbringer becomes covered in fungus that protects your from attacks. You gain a \plus1 bonus to Armor and Fortitude defense.

        %     This bonus increases by 1 at your 7th druid level, and every 4 druid levels thereafter.

\newpage
\section{Fighter}\label{Fighter}
    \begin{dtable!*}
        \lcaption{Fighter Progression}
\begin{dtabularx}{\textwidth}{l l >{\lcol}X >{\lcol}X >{\lcol}X >{\lcol}X}
    \tb{Rank} & \tb{Min Level} & \tb{Combat Discipline}       & \tb{Equipment Training}      & \tb{Martial Mastery}                           & \tb{Tactician} \tableheaderrule
    0         & \tdash         & Enduring discipline          & Armor expertise              & Defensive expertise                            & Tactical insight           \\
    1         & 1              & Discipline                   & Weapon training              & Martial maneuvers                              & Battle tactics             \\
    2         & 4              & Greater enduring discipline  & Equipment efficiency         & Combat style rank (2), martial force               & Tactical precision         \\
    3         & 7              & Disciplined force            & Greater armor expertise      & Combat style rank (3), glancing strikes            & Greater tactical insight           \\
    4         & 10             & Disciplined reaction         & Weapon expertise             & Combat style rank (4), martial maneuver            & Reactive tactics           \\
    5         & 13             & Supreme enduring discipline  & Greater equipment efficiency & Combat style rank (5), greater martial force       & Greater tactical precision \\
    6         & 16             & Greater disciplined force    & Supreme armor expertise      & Combat style rank (6), greater defensive expertise & Supreme tactical insight           \\
    7         & 19             & Greater disciplined reaction & Greater weapon expertise     & Combat style rank (7), martial maneuver            & Greater reactive tactics   \\
\end{dtabularx}
    \end{dtable!*}

    Fighters are disciplined warriors that excel in all aspects of physical combat.
    What they lack in utility, they make up for with tactical talent, combat experience, and physical skill.
    Fighters can span all variety of alignments, social stations, and roles in life.

    \classbasics{Alignment} Any.

    \classbasics{Archetypes} Fighters have the Combat Discipline, Equipment Training, Martial Mastery, and Tactician \glossterm{archetypes}.

    \subsection{Basic Class Abilities}
        If you are a fighter, you gain the following abilities.
        % 1 AD, 4 insight, 4 skill, 4 AP, 6 fatigue, 3 armor, 3 weapons = 25 points

        \cf{Ftr}{Defenses}
        You gain the following bonuses to your \glossterm{defenses}: \plus2 Armor, \plus7 Fortitude, \plus3 Reflex, \plus4 Mental.

        \cf{Ftr}{Resources} You have the following \glossterm{resources}:
        \begin{itemize}
            \item Two \glossterm{insight points}, which you can spend to gain additional abilities or proficiencies (see \pcref{Insight Points}).
            \item Six \glossterm{skill points}, which you can spend to learn skills (see \pcref{Skills}).
            \item One \glossterm{attunement point}, which you can use to attune to items and abilities that affect you (see \pcref{Attunement Points}).
            \item A \plus3 bonus to your \glossterm{fatigue tolerance}, which makes it easier for you to use powerful abilities that fatigue you (see \pcref{Fatigue}).
        \end{itemize}

        \cf{Ftr}{Weapon Proficiencies} 
        You are proficient with simple weapons and any two other \glossterm{weapon groups}.

        \cf{Ftr}{Armor Proficiencies} 
        You are proficient with all armor.

        \cf{Ftr}{Skills}
        You have the following \glossterm{class skills}:
        \begin{itemize}
            \item \subparhead{Strength} Climb, Jump, Swim.
            \item \subparhead{Dexterity} Agility, Flexibility, Ride.
            \item \subparhead{Constitution} Endurance.
            \item \subparhead{Intelligence} Craft.
            \item \subparhead{Perception} Awareness.
            \item \subparhead{Other} Deception, Intimidate, Persuasion, Profession.
        \end{itemize}

    \newpage
    \subsection{Combat Discipline}
        This archetype allows you to improve your defenses and resist conditions.

        \cf{Ftr}[0]{Enduring Discipline} You gain a \plus2 bonus to your \glossterm{fatigue tolerance}.

        \cf{Ftr}[1]{Discipline} You can use the \textit{discipline} ability as a \glossterm{standard action}.
        \begin{instantability}{Discipline}
            Instant
            \rankline
            Remove up to two \glossterm{conditions} affecting you.
            This cannot remove a condition applied during the current round.

            \rankline
            \rank{3} This ability gains the \abilitytag{Swift} tag.
            When you use it, the penalties from the removed conditions do not affect you during the current phase.
            In addition, you cannot gain any additional \glossterm{conditions} until after the end of the next round.
            \rank{5} You can use this ability as a \glossterm{minor action}.
            When you do, you increase your \glossterm{fatigue level} by one.
            \rank{7} You can remove any number of \glossterm{conditions}.
        \end{instantability}

        \cf{Ftr}[2]{Greater Enduring Discipline}
        The bonus from your \textit{enduring discipline} ability increases to \plus3.
        In addition, you gain a \plus1 bonus to \glossterm{vital rolls} (see \pcref{Vital Rolls}).

        \cf{Ftr}[3]{Disciplined Force}
        You gain a \plus1d bonus to your damage with all weapons.

        \cf{Ftr}[4]{Disciplined Reaction}
        You do not suffer any effects from \glossterm{conditions} or \glossterm{vital wounds} until the next round after they are applied.
        While a vital wound is delayed in this way, you do not consider it when calculating your penalties to future \glossterm{vital rolls}, and you do not suffer any effects from its specific vital wound effect.
        You suffer their normal effects in the following round.

        \cf{Ftr}[5]{Supreme Enduring Discipline}
        The bonus from your \textit{enduring discipline} ability increases to \plus4.
        In addition, the bonus to vital rolls from your \textit{greater enduring discipline} ability increases to \plus2.

        \cf{Ftr}[6]{Greater Disciplined Force} The bonus from your \textit{disciplined force} ability increases to \plus2d.

        \cf{Ftr}[7]{Greater Disciplined Reaction}
        The delay from your \textit{disciplined} reaction ability increases by an additional round.
        You suffer their normal effects after that time.

    \newpage
    \subsection{Equipment Training}
        This archetype improves your combat prowess with weapons and armor.

        \cf{Ftr}[0]{Armor Expertise} You reduce the \glossterm{encumbrance} of body armor you wear by 1.

        \cf{Ftr}[1]{Weapon Training} You can use the \textit{weapon training} ability by spending an hour training with a weapon.
        You cannot use this ability with an \glossterm{exotic weapon} that is from a \glossterm{weapon group} you are not proficient with.
        \begin{instantability}{Weapon Training}
            Instant
            \rankline
            You become proficient with the weapon you trained with.
            You gain a \plus1 bonus to \glossterm{accuracy} with that weapon unless it is an \glossterm{exotic weapon} that you would not be proficient with without this ability.
            This ability's effect lasts until you use this ability again.

            \rankline
            \rank{4} You can use this ability with only five minutes of training.
            \rank{6} You can use this ability as a \glossterm{minor action}.
        \end{instantability}

        \cf{Ftr}[2]{Equipment Efficiency} You gain an additional \glossterm{attunement point}.
        You can only use this attunement point to \glossterm{attune} to magic weapons and magic armor.

        \cf{Ftr}[3]{Greater Armor Expertise}
        The penalty reduction from your \textit{armor expertise} ability increases to 2.
        In addition, you treat body armor were one usage class lighter than normal when doing so would be beneficial for you (see \pcref{Armor Usage Classes}).

        \cf{Ftr}[4]{Weapon Expertise} You gain a \plus1d bonus to your damage with all weapons.

        \cf{Ftr}[5]{Greater Equipment Efficiency} The number of attunement points you gain from your \textit{efficient equipment} ability increases to two.
        In addition, you can use the attunement points from that ability to attune to any magic item, not just weapons and armor.

        \cf{Ftr}[6]{Supreme Armor Expertise}
        The \glossterm{encumbrance} reduction from your \textit{armor expertise} ability increases to 3.
        In addition, you treat body armor as if it were an additional usage class lighter than normal when doing so would be beneficial for you.

        \cf{Ftr}[7]{Greater Weapon Expertise} The bonus from your \textit{weapon expertise} ability increases to \plus3d.

    \newpage
    \subsection{Martial Mastery}
        This archetype grants you special abilities to use in combat.

        \cf{Ftr}[0]{Defensive Expertise} You gain a \plus1 bonus to Armor defense.

        {
            \cf{Ftr}[1]{Combat Styles}
            You can channel your martial prowess into dangerous attacks.
            You gain access to one of the following \glossterm{combat styles}: \textit{blunt force}, \textit{penetrating precision}, or \textit{rip and tear}.
            In addition, you gain access to any two combat styles of your choice (see \pcref{Combat Styles}).
            You may spend \glossterm{insight points} to gain access to one additional combat style per insight point.

            You learn two rank 1 \glossterm{maneuvers} from combat styles you have access to.
            You may spend \glossterm{insight points} to learn to one additional maneuver per insight point.
            Unless otherwise noted in an ability's description, using a maneuver requires a \glossterm{standard action}.

            When you gain access to a new \glossterm{rank} in this archetype,
                you can exchange any number of maneuvers you know for other maneuvers,
                including maneuvers of the higher rank.
        }

        {
            \cf{Ftr}[2]{Combat Style Rank} You become a rank 2 combat style user.
            This gives you access to maneuvers that require a minimum rank of 2.

            \cf{Ftr}[2]{Martial Force} You gain a \plus1d bonus to your damage with all weapons.
        }

        {
            \cf*{Ftr}[3]{Combat Style Rank} You become a rank 3 combat style user.
            This gives you access to maneuvers that require a minimum rank of 3 and can improve the effectiveness of your existing maneuvers.

            \cf{Ftr}[3]{Glancing Strikes} Whenever you miss by 2 or less with a \glossterm{strike}, the target takes half damage from the strike.
            This is called a \glossterm{glancing blow}.
        }

        {
            \cf*{Ftr}[4]{Combat Style Rank} You become a rank 4 combat style user.
            This gives you access to maneuvers that require a minimum rank of 4 and can improve the effectiveness of your existing maneuvers.

            \cf{Ftr}[4]{Martial Maneuver}
            You learn an additional \glossterm{maneuver} from a combat style you have access to (see \pcref{Combat Styles}).
        }

        {
            \cf*{Ftr}[5]{Combat Style Rank} You become a rank 5 combat style user.
            This gives you access to maneuvers that require a minimum rank of 5 and can improve the effectiveness of your existing maneuvers.

            \cf{Ftr}[5]{Greater Martial Force} The bonus from your \textit{martial force} ability increases to \plus2d.
        }

        {
            \cf*{Ftr}[6]{Combat Style Rank} You become a rank 6 combat style user.
            This gives you access to maneuvers that require a minimum rank of 6 and can improve the effectiveness of your existing maneuvers.

            \cf{Ftr}[6]{Greater Defensive Expertise} The bonus from your \textit{defensive expertise} ability increases to \plus2.
        }

        {
            \cf*{Ftr}[7]{Combat Style Rank} You become a rank 7 combat style user.
            This gives you access to maneuvers that require a minimum rank of 7 and can improve the effectiveness of your existing maneuvers.

            \cf*{Ftr}[7]{Martial Maneuver}
            You learn an additional \glossterm{maneuver} from a combat style you have access to (see \pcref{Combat Styles}).
        }

    \newpage
    \subsection{Tactician}

        \cf{Ftr}[0]{Tactical Insight} You gain an additional \glossterm{insight point}.

        \cf{Ftr}[1]{Battle Tactics}
        You can lead your allies using tactics appropriate for the situation.
        Choose a single battle tactic from the list below.
        You can also spend \glossterm{insight points} to learn one additional \textit{battle tactic} per \glossterm{insight point}.

        You can initiate a \textit{battle tactic} as a \glossterm{minor action}.
        When you initiate a battle tactic, you choose whether to use visual cues like gestures, auditory cues like shouts, or both to communicate your tactic with your allies.
        Your \textit{battle tactics} affect yourself and your \glossterm{allies} within a \areahuge radius \glossterm{emanation} from you who can either see or hear your chosen communication style.

        All \textit{battle tactics} have the \abilitytag{Sustain} (free) tag, so they last as long as you \glossterm{sustain} them (see \pcref{Sustained Abilities}).
        You cannot sustain multiple battle tactics simultaneously.

        {
            \begin{durationability}{Break Through}
                \abilitytag{Sustain} (free)
                \rankline
                Each target gains a \plus2 bonus to \glossterm{accuracy} with the \textit{overrun} and \textit{shove} abilities (see \pcref{Special Combat Abilities}).

                \rankline
                \rank{3} The bonus increases to \plus3.
                \rank{5} The bonus increases to \plus4.
                \rank{7} The bonus increases to \plus4.
            \end{durationability}

            \begin{durationability}{Dogpile}
                \abilitytag{Sustain} (free)
                \rankline
                Each target gains a \plus2 bonus to \glossterm{accuracy} with the \textit{grapple} ability and with all grapple actions (see \pcref{Grapple}, and \pcref{Grapple Actions}).
                This does not affect any other abilities that may have similar effects, such as the Strangle maneuver (see Strangle, page \pref{maneuver:Strangle}).

                \rankline
                \rank{3} The bonus increases to \plus3.
                \rank{5} The bonus increases to \plus4.
                \rank{7} The bonus increases to \plus5.
            \end{durationability}

            \begin{durationability}{Duck and Cover}
                \abilitytag{Sustain} (free)
                \rankline
                Each target gains a \plus1 bonus to Armor defense against non-\glossterm{melee} attacks.

                \rankline
                \rank{3} The bonus increases to \plus2.
                \rank{5} The bonus increases to \plus3.
                \rank{7} The bonus increases to \plus4.
            \end{durationability}

            \begin{durationability}{Group Up}
                \abilitytag{Sustain} (free)
                \rankline
                Each target that is adjacent to at least one other target gains a \plus1 bonus to Armor defense.

                \rankline
                \rank{3} Each target affected by the Armor defense bonus also gains a \plus2 bonus to Mental defense.
                \rank{5} The Armor defense bonus increases to \plus2.
                \rank{7} The Mental defense bonus increases to \plus4.
            \end{durationability}

            \begin{durationability}{Hold The Line}
                \abilitytag{Sustain} (free)
                \rankline
                Your \glossterm{enemies} treat all areas adjacent to any target as \glossterm{difficult terrain}.

                \rankline
                \rank{3} Each area adjacent to any target is doubly difficult terrain, and costs quadruple the normal movement cost to move out of.
                \rank{5} Each area within a 10 foot radius \glossterm{emanation} from each target is difficult terrain.
                \rank{7} Each area within a 10 foot radius \glossterm{emanation} from each target is doubly difficult terrain.
            \end{durationability}

            \begin{durationability}{Hustle}
                \abilitytag{Sustain} (free)
                \rankline
                Each target gains a \plus5 foot bonus to its \glossterm{base speed}.

                \rankline
                \rank{3} The speed bonus increases to \plus10 feet.
                \rank{5} The speed bonus increases to \plus15 feet.
                \rank{7} The speed bonus increases to \plus20 feet.
            \end{durationability}

            \begin{durationability}{Keep Moving}
                \abilitytag{Sustain} (free)
                \rankline
                Each target that ends the \glossterm{movement phase} at least twenty feet away from where it started the round
                    gains a \plus1 bonus to Armor defense until the end of the round.

                \rankline
                \rank{3} Each target affected by the Armor defense bonus also gains a \plus2 bonus to Reflex defense.
                \rank{5} The Armor defense bonus increases to \plus2.
                \rank{7} The Reflex defense bonus increases to \plus4.
            \end{durationability}

            \begin{durationability}{Stand Your Ground}
                \abilitytag{Sustain} (free)
                \rankline
                Each target that ends the \glossterm{movement phase} without changing its location gains a \plus1 bonus to Armor defense until its location changes.

                \rankline
                \rank{3} Each target affected by the Armor defense bonus also gains a \plus2 bonus to Fortitude defense.
                \rank{5} The Armor defense bonus increases to \plus2.
                \rank{7} The Fortitude defense bonus increases to \plus4.
            \end{durationability}
        }

        \cf{Ftr}[2]{Tactical Precision} You gain a \plus1 bonus to \glossterm{accuracy}.

        \cf*{Ftr}[3]{Greater Tactical Insight} You gain an additional \glossterm{insight point}.
        In addition, you learn an additional \textit{battle tactic}.

        \cf{Ftr}[4]{Reactive Tactics} You gain a \plus5 bonus to \glossterm{initiative} checks.
        In addition, you can initiate a new \textit{battle tactic} as a \glossterm{free action} instead of as a \glossterm{minor action}.

        \cf{Ftr}[5]{Greater Tactical Precision} The bonus from your \textit{tactical precision} ability increases to \plus2.

        \cf*{Ftr}[6]{Supreme Tactical Insight} You gain an additional \glossterm{insight point}.
        In addition, you learn an additional \textit{battle tactic}.

        \cf{Ftr}[7]{Greater Reactive Tactics} The bonus from your \textit{reactive tactics} ability increases to \plus10.
        In addition, all of your \textit{battle tactics} abilities gain the \abilitytag{Swift} tag, so their bonuses and penalties take effect in the phase that you active them.

\newpage
\section{Monk}\label{Monk}
    \begin{dtable!*}
        \lcaption{Monk Progression}
\begin{dtabularx}{\textwidth}{l l >{\lcol}X >{\lcol}X >{\lcol}X >{\lcol}X}
    \tb{Rank} & \tb{Min Level} & \tb{Esoteric Warrior}                        & \tb{Ki}                  & \tb{Perfected Form}       & \tb{Transcendent Sage} \tableheaderrule
    0         & \tdash         & Esoteric fluidity                            & Ki barrier, ki energy    & Unarmed warrior           & Transcend frailty             \\
    1         & 1              & Combat styles                           & Ki manifestations        & Fast movement             & Clear the mind                \\
    2         & 4              & Combat style rank (2), esoteric force            & Greater ki barrier       & Perfect precision         & Feel the flow of life         \\
    3         & 7              & Combat style rank (3), glancing strikes          & Ki power                 & Perfect body              & Transcend time                \\
    4         & 10             & Combat style rank (4), esoteric maneuver         & Ki manifestation         & Greater fast movement     & Inner peace                   \\
    5         & 13             & Combat style rank (5), greater esoteric force    & Supreme ki barrier       & Greater perfect precision & Greater feel the flow of life \\
    6         & 16             & Combat style rank (6), greater esoteric fluidity & Greater ki power         & Greater perfect body      & Transcend mortality           \\
    7         & 19             & Combat style rank (7), esoteric maneuver         & Greater ki manifestation & Supreme fast movement     & Inner transcendence           \\
\end{dtabularx}
    \end{dtable!*}

    Monks are agile warriors that use intensive training to move beyond the limits of their physical bodies.
    They are usually lightly armored and use light weapons - if they use weapons or armor at all.
    The intense dedication monks must have to their training means they tend not to be chaotic.

    \classbasics{Alignment} Any.

    \classbasics{Archetypes} Monks have the Ki, Esoteric Warrior, Transcendent Sage, and Perfected Form \glossterm{archetypes}.

    \subsection{Basic Class Abilities}
        If you are a monk, you gain the following abilities.
        % 1 AD, 6 insight, 8 skill, 4 AP, 4 fatigue, 1 armor, 1 weapons = 25 points

        \cf{Mnk}{Defenses}
        You gain the following bonuses to your \glossterm{defenses}: \plus2 Armor, \plus3 Fortitude, \plus6 Reflex, \plus5 Mental.

        \cf{Mnk}{Resources} You have the following \glossterm{resources}:
        \begin{itemize}
            \item Three \glossterm{insight points}, which you can spend to gain additional abilities or proficiencies (see \pcref{Insight Points}).
            \item Twelve \glossterm{skill points}, which you can spend to learn skills (see \pcref{Skills}).
            \item One \glossterm{attunement point}, which you can use to attune to items and abilities that affect you (see \pcref{Attunement Points}).
            \item A \plus2 bonus to your \glossterm{fatigue tolerance}, which makes it easier for you to use powerful abilities that fatigue you (see \pcref{Fatigue}).
        \end{itemize}

        \cf{Mnk}{Weapon Proficiencies} 
        You are proficient with simple weapons and monk weapons.

        \cf{Mnk}{Armor Proficiencies} 
        You are proficient with light armor.

        \cf{Mnk}{Skills}
        You have the following \glossterm{class skills}:
        \begin{itemize}
            \item \subparhead{Strength} Climb, Jump, Swim.
            \item \subparhead{Dexterity} Agility, Flexibility, Ride, Stealth.
            \item \subparhead{Constitution} Endurance.
            \item \subparhead{Intelligence} Craft, Deduction, Medicine.
            \item \subparhead{Perception} Awareness, Creature Handling, Social Insight, Spellsense, Survival.
            \item \subparhead{Other} Deception, Intimidate, Perform, Persuasion, Profession.
        \end{itemize}

    \newpage
    \subsection{Esoteric Warrior}\label{Esoteric Warrior}
        This archetype improves your combat prowess with unusual abilities you can use in combat.

        \cf{Mnk}[0]{Esoteric Fluidity} You gain a \plus1 bonus to Dexterity-based \glossterm{checks}, except \glossterm{initiative} checks.

        {
            \cf{Mnk}[1]{Combat Styles}
            You can perform a wide variety of unusual attacks.
            You gain access to one of the following \glossterm{combat styles}: \textit{ebb and flow}, \textit{flurry of blows}, or \textit{mobile assault}.
            In addition, you gain access to any two combat styles of your choice (see \pcref{Combat Styles}).
            You may spend \glossterm{insight points} to gain access to one additional combat style per insight point.

            You learn two rank 1 \glossterm{maneuvers} from combat styles you have access to.
            You may spend \glossterm{insight points} to learn to one additional maneuver per insight point.
            Unless otherwise noted in an ability's description, using a maneuver requires a \glossterm{standard action}.

            When you gain access to a new \glossterm{rank} in this archetype,
                you can exchange any number of maneuvers you know for other maneuvers,
                including maneuvers of the higher rank.
        }

        {
            \cf{Mnk}[2]{Combat Style Rank} You become a rank 2 combat style user.
            This gives you access to maneuvers that require a minimum rank of 2.

            \cf{Mnk}[2]{Esoteric Force} You gain a \plus1d bonus to your damage with all weapons.
        }

        {
            \cf*{Mnk}[3]{Combat Style Rank} You become a rank 3 combat style user.
            This gives you access to maneuvers that require a minimum rank of 3 and can improve the effectiveness of your existing maneuvers.

            \cf{Mnk}[3]{Glancing Strikes} Whenever you miss by 2 or less with a \glossterm{strike}, the target takes half damage from the strike.
            This is called a \glossterm{glancing blow}.
        }

        {
            \cf*{Mnk}[4]{Combat Style Rank} You become a rank 4 combat style user.
            This gives you access to maneuvers that require a minimum rank of 4 and can improve the effectiveness of your existing maneuvers.

            \cf{Mnk}[4]{Esoteric Maneuver}
            You learn an additional \glossterm{maneuver} from an combat style you have access to (see \pcref{Combat Styles}).
        }

        {
            \cf*{Mnk}[5]{Combat Style Rank} You become a rank 5 combat style user.
            This gives you access to maneuvers that require a minimum rank of 5 and can improve the effectiveness of your existing maneuvers.

            \cf{Mnk}[5]{Greater Esoteric Force} The bonus from your \textit{esoteric force} ability increases to \plus2d.
        }

        {
            \cf*{Mnk}[6]{Combat Style Rank} You become a rank 6 combat style user.
            This gives you access to maneuvers that require a minimum rank of 6 and can improve the effectiveness of your existing maneuvers.

            \cf{Mnk}[6]{Greater Esoteric Fluidity} The bonus from your \textit{esoteric fluidity} ability increases to \plus2.
        }

        {
            \cf*{Mnk}[7]{Combat Style Rank} You become a rank 7 combat style user.
            This gives you access to maneuvers that require a minimum rank of 7 and can improve the effectiveness of your existing maneuvers.

            \cf*{Mnk}[7]{Esoteric Maneuver}
            You learn an additional \glossterm{maneuver} from an combat style you have access to (see \pcref{Combat Styles}).
        }

    \newpage
    \subsection{Ki}
        This archtype grants you unusual abilities based on tapping into your inner ki.
        If you have any \glossterm{encumbrance}, you lose the benefit of all abilities from this archetype.
        All abilities from this archetype are \glossterm{magical}.

        \cf{Mnk}[0]{Ki Barrier}
        While you have no other body armor, you gain a ki barrier around your body.
        This functions like body armor that provides a \plus2 bonus to Armor defense and has no \glossterm{encumbrance}.
        It also provides a \plus4 bonus to \glossterm{resistance} against \glossterm{energy damage}, and a \plus2 bonus to \glossterm{resistance} against \glossterm{physical damage}.
        The armor has no \glossterm{encumbrance}.

        As long as you are not wearing armor and have a free hand, the barrier also manifests as a shield that provides a \plus1 bonus to Armor defense.
        This bonus is considered to come from a shield, and does not stack with the benefits of using any other shield.

        \cf{Mnk}[0]{Ki Energy} Whenever you make a \glossterm{strike}, you can choose to treat that as a \glossterm{magical} ability.
        This allows you to use your \glossterm{power} with magical abilities to determine your damage.
        In addition, that strike does not deal \glossterm{physical damage} or any physical damage subtypes.
        If the strike would normally deal one or more subtype of energy damage, the damage is of those types.
        Otherwise, all damage dealt by the strike is \glossterm{energy damage}.

        \cf{Mnk}[1]{Ki Manifestations}
        You can channel your ki to temporarily enhance your abilities.
        Choose one \textit{ki manifestation} from the list below.
        You can also spend \glossterm{insight points} to learn one additional \textit{ki manifestation} per \glossterm{insight point}.
        You can use any \textit{ki manifestation} ability you know using the type of action indicated in the ability's description.

        After you use a \textit{ki manifestation}, you cannot use a \textit{ki manifestation} until after the end of the next round.
        {
            \begin{durationability}{Abandon the Fragile Self}
                \spelltwocol{Duration}{\abilitytag{Swift}}
                \rankline
                You can use this ability as a \glossterm{free action}.
                You can negate one \glossterm{condition} that would be applied to you this phase.
                In exchange, you take a \minus2 penalty to \glossterm{defenses} this phase.

                \rankline
                \rank{3} You can negate any number of conditions instead of only one condition.
                \rank{5} The defense penalty is reduced to \minus1.
                \rank{7} The defense penalty is removed.
            \end{durationability}

            \begin{durationability}{Burst of Blinding Speed}
                \spelltwocol{Duration}{\abilitytag{Swift}}
                \rankline
                You can use this ability as a \glossterm{free action}.
                You gain a \plus10 foot bonus to your land speed this phase.

                \rankline
                \rank{3} You can also ignore \glossterm{difficult terrain} this phase.
                \rank{5} The speed bonus increases to \plus20 feet.
                \rank{7} You can also move or stand on liquids as if they were solid this phase.
            \end{durationability}

            \begin{durationability}{Elegant Whirl of Fluid Motion}
                \spelltwocol{Duration}{\abilitytag{Swift}}
                \rankline
                You can use this ability as a \glossterm{free action}.
                You gain a \plus5 bonus to the Agility skill this round (see \pcref{Agility}).

                \rankline
                \rank{3} The bonus increases to \plus10.
                \rank{5} The bonus lasts until the end of the next round.
                \rank{7} The bonus increases to \plus20.
            \end{durationability}

            \begin{durationability}{Extend the Flow of Ki}
                \spelltwocol{Duration}{\abilitytag{Swift}}
                \rankline
                You can use this ability as a \glossterm{free action}.
                You gain a \plus5 foot \glossterm{magic bonus} to \glossterm{reach} this phase.

                \rankline
                \rank{3} 
                \rank{5} The bonus to \glossterm{reach} increases to 10 feet.
                \rank{7} 
            \end{durationability}

            \begin{instantability}{Flash Step}
                Instant
                \rankline
                You can use this ability as part of movement.
                % TODO: is 'horizontally' the correct word?
                You teleport horizontally instead of moving normally.
                If your \glossterm{line of effect} to your destination is blocked, or if this teleportation would somehow place you inside a solid object, your teleportation is cancelled and you remain where you are.

                Teleporting a given distance costs movement equal to twice that distance.
                For example, if you have a 30 foot movement speed, you can move 10 feet, teleport 5 feet, and move an additional 10 feet before your movement ends.

                \rankline
                \rank{3} The movement cost to teleport is reduced to be equal to the distance you teleport.
                \rank{5} You can use this ability to move even if you are \immobilized or \grappled.
                \rank{7} You can attempt to teleport to locations outside of \glossterm{line of sight} and \glossterm{line of effect}.
                If your intended destination is invalid, the distance you spent teleporting is wasted, but you suffer no other ill effects.
            \end{instantability}

            \begin{durationability}{Leap of the Heavens}
                \spelltwocol{Duration}{\abilitytag{Swift}}
                \rankline
                You can use this ability as a \glossterm{free action}.
                You gain a \plus5 bonus to the Jump skill this round (see \pcref{Jump}).

                \rankline
                \rank{3} The bonus increases to \plus10.
                \rank{5} The bonus lasts until the end of the next round.
                \rank{7} The bonus increases to \plus20.
            \end{durationability}

            \begin{durationability}{Scale the Highest Tower}
                \spelltwocol{Duration}{\abilitytag{Swift}}
                \rankline
                You can use this ability as a \glossterm{free action}.
                You gain a \plus5 bonus to the Climb skill this round (see \pcref{Climb}).
                % TODO: is this wording correct?

                \rankline
                \rank{3} The Climb bonus increases to \plus10.
                \rank{5} The bonus lasts until the end of the next round.
                \rank{5} The bonus increases to \plus20.
            \end{durationability}

            \begin{durationability}{Sense the Mystic Truth}
                \spelltwocol{Duration}{\abilitytag{Swift}}
                \rankline
                You can use this ability as a \glossterm{free action}.
                You gain a \plus5 bonus to the Spellsense skill this round (see \pcref{Spellsense}).

                \rankline
                \rank{3} The bonus increases to \plus10.
                \rank{5} The bonus lasts until the end of the next round.
                \rank{7} The bonus increases to \plus20.
            \end{durationability}

            \begin{durationability}{Step Between the Mystic Worlds}
                \spelltwocol{Duration}{\abilitytag{Swift}}
                \rankline
                You can use this ability as a \glossterm{free action}.
                You gain a \plus2 bonus to \glossterm{defenses} against \glossterm{magical} abilities this phase.
                After the effect ends, you take a \minus2 penalty to \glossterm{defenses} against \glossterm{magical} attacks until the end of the next round.

                \rankline
                \rank{3} The defense bonus is increased to \plus3.
                \rank{5} The effect lasts until the end of the current round.
                \rank{7} The defense bonus is increased to \plus5.
            \end{durationability}

            \begin{durationability}{Surpass the Mortal Limits}
                \spelltwocol{Duration}{\abilitytag{Swift}}
                \rankline
                You can use this ability as a \glossterm{free action}.
                You can use your \glossterm{power} in place of your Strength, Dexterity, and Constitution when making checks this phase.

                \rankline
                \rank{3} You also gain a \plus2 bonus to checks based on Strength, Dexterity, and Constitution.
                \rank{5} The effect lasts until the end of the current round.
                \rank{7} The bonus increases to \plus4.
            \end{durationability}

            % TODO: add more
        }

        \cf{Mnk}[2]{Greater Ki Barrier} 
        The defense bonus from the body armor created by your \textit{ki barrier} ability increases to \plus3.
        In addition, its bonus to \glossterm{resistance} against \glossterm{energy damage} increases to be equal to three times your rank in this archetype, and its bonus to resistance against \glossterm{physical damage} changes to be equal to your rank in this archetype.

        \cf{Mnk}[3]{Ki Power}
        You gain a \plus2 bonus to your \glossterm{power} with all abilities.

        \cf{Mnk}[4]{Ki Manifestation}
        You learn an additional \textit{ki manifestation}.

        \cf{Mnk}[5]{Supreme Ki Barrier}
        The defense bonus from the body armor created by your \textit{ki barrier} ability increases to \plus4.
        In addition, its bonus to \glossterm{resistance} against \glossterm{energy damage} increases to be equal to four times your rank in this archetype, and its bonus to resistances against \glossterm{physical damage} increases to be equal to twice your rank in this archetype.

        \cf{Mnk}[6]{Greater Ki Power} The bonus from your \textit{ki power} ability increases to \plus6.

        \cf{Mnk}[7]{Greater Ki Manifestation} After using a \textit{ki manifestation} ability, you can use another one after the end of the current round instead of after the end of the next round.

    \newpage
    \subsection{Perfected Form}
        This archetype improves the perfection of your physical body, including your unarmed attacks, through rigorous training.

        \cf{Mnk}[0]{Unarmed Warrior} You become \glossterm{proficient} with the unarmed weapons \glossterm{weapon group} (see \pcref{Weapon Groups}).
        In addition, you gain a \plus2d bonus to damage with weapons that have the Unarmed weapon tag (see \pcref{Unarmed}).
        For details about how to fight while unarmed, see \pcref{Unarmed Combat}.

        \cf{Mnk}[1]{Fast Movement} You gain a \plus10 foot bonus to your \glossterm{base speed}.

        \cf{Mnk}[2]{Perfect Precision} You gain a \plus1 bonus to \glossterm{accuracy} with attacks using weapons from the monk weapons and unarmed weapons \glossterm{weapon groups}, natural weapons, and to any attack using one or more \glossterm{free hands}.

        \cf{Mnk}[3]{Perfect Body} You gain a \plus1 bonus to the base value of one physical \glossterm{attribute} of your choice: Strength, Dexterity, or Constitution.

        \cf{Mnk}[4]{Greater Fast Movement} The speed bonus from your \textit{fast movement} ability increases to \plus20 feet.

        \cf{Mnk}[5]{Greater Perfect Precision} The bonuses from your \textit{perfect precision} ability increase to \plus2.

        \cf{Mnk}[6]{Greater Perfect Body} The bonus from your \textit{perfect body} ability applies to the base value of all physical attributes, not just the one you chose.

        \cf{Mnk}[7]{Supreme Fast Movement} The speed bonus from your \textit{fast movement} ability increases to \plus30 feet.

    \newpage
    \subsection{Transcendent Sage}
        This archetype grants you abilities to resist or remove conditions.

        \cf{Mnk}[0]{Transcend Frailty}
        You are immune to being \sickened and \nauseated.

        \cf{Mnk}[1]{Clear the Mind} You can use the \textit{clear the mind} ability as a standard action.
        \begin{instantability}{Clear the Mind}
            Instant
            \rankline
            Remove up to two \glossterm{conditions} affecting you.
            This cannot remove a condition applied during the current round.

            \rankline
            \rank{3} This ability gains the \abilitytag{Swift} tag.
            When you use it, the penalties from the removed conditions do not affect you during the current phase.
            In addition, you cannot gain any additional \glossterm{conditions} until after the end of the next round.
            \rank{5} You can use this ability as a \glossterm{minor action}.
            When you do, you increase your \glossterm{fatigue level} by one.
            \rank{7} You can remove any number of \glossterm{conditions}.
        \end{instantability}

        \cf{Mnk}[2]{Feel the Flow of Life} You become so attuned to the natural energy of life that you can sense it even when sight fails you.
        You gain the \glossterm{lifesense} ability with a 120 foot range.
        In addition, you gain the \glossterm{lifesight} ability with a 30 foot range.

        \cf{Mnk}[3]{Transcend Time} You are immune to being \slowed and \decelerated.

        \cf{Mnk}[4]{Inner Peace} You are immune to \abilitytag{Compulsion} and \abilitytag{Emotion} attacks.

        \cf{Mnk}[5]{Greater Feel the Flow of Life}
        The range of your \glossterm{lifesense} ability increases by 200 feet.
        In addition, the range of your \glossterm{lifesight} ability increases by 80 feet.

        \cf{Mnk}[6]{Transcend Mortality}
        You are no longer considered a living creature for the purpose of attacks against you.
        This means that attacks which only affect living creatures have no effect against you.
        In addition, you no longer take penalties to your attributes for aging, and cannot be magically aged.
        You still die of old age when your time is up.

        \cf{Mnk}[7]{Inner Transcendence} You are immune to \glossterm{conditions}.

        % TODO: move this out of an archetype and into a different subsystem
        % \cf{Mnk}[8]{Transcend Mortality}[Magical]
        % If you die, you may choose to retain control of your body and soul through sheer force of will.
        % Your body immediately disappears, and your soul does not travel to an afterlife.
        % Instead, your body reforms with no trace of its injuries 8 hours later.
        % The reformed body is in perfect health and can be any age you choose, to a minimum of the age of adulthood for your species.
        % You can reform your body at the place where you died, or in any place on the same plane that is deeply familiar to you.

        % After each time you reform yourself in this way, it takes an additional hour to reform the next time you ``die''.
        % You can only be permanently killed by the direct intervention of a deity.

\newpage
\section{Paladin}\label{Paladin}
    \begin{dtable!*}
        \lcaption{Paladin Progression}
\begin{dtabularx}{\textwidth}{l l >{\lcol}X >{\lcol}X >{\lcol}X >{\lcol}X >{\lcol}X}
    \tb{Rank} & \tb{Min Level} & \tb{Devoted Paragon}     & \tb{Divine Magic}               & \tb{Divine Spell Expertise} & \tb{Stalwart Guardian}      & \tb{Zealous Warrior}       \tableheaderrule
    0 & \tdash & Devoted endurance        & Cantrips                        & Combat caster               & Stalwart defense            & Zealous exertion                           \\
    1 & 1      & Aligned aura             & Spellcasting                    & Insight point               & Lay on hands                & Smite                                      \\
    2 & 4      & Aligned immunity         & Spell rank (2), spell knowledge & Greater combat caster       & Stalwart resilience         & Zealous offense                            \\
    3 & 7      & Paragon power            & Spell rank (3)                  & Wellspring of power         & Greater stalwart defense    & Glancing strikes                           \\
    4 & 10     & Greater aligned aura     & Spell rank (4), spell knowledge & Insight point               & Greater lay on hands        & Forceful zeal, zealous purge               \\
    5 & 13     & Greater aligned immunity & Spell rank (5)                  & Supreme combat caster       & Greater stalwart resilience & Greater zealous offense                    \\
    6 & 16     & Greater paragon power    & Spell rank (6)                  & Greater wellspring of power & Supreme stalwart defense    & Greater zealous exertion, zealous fixation \\
    7 & 19     & Supreme aligned aura     & Spell rank (7), spell knowledge & Insight point               & Supreme lay on hands        & Greater forceful zeal, pass judgment       \\
\end{dtabularx}
    \end{dtable!*}

    \classbasics{Alignment} Any other than true neutral.

    \classbasics{Archetypes} Paladins have the Devoted Paragon, Divine Magic, Stalwart Guardian, and Zealous Warrior \glossterm{archetypes}.

    \subsection{Basic Class Abilities}
        If you are a paladin, you gain the following abilities.
        % 1 AD, 4 insight, 4 skill, 4 AP, 6 fatigue, 3 armor, 3 weapons = 25 points

        \cf{Pal}{Defenses}
        You gain the following bonuses to your \glossterm{defenses}: \plus2 Armor, \plus6 Fortitude, \plus3 Reflex, \plus5 Mental.

        \cf{Pal}{Resources} You have the following \glossterm{resources}:
        \begin{itemize}
            \item Two \glossterm{insight points}, which you can spend to gain additional abilities or proficiencies (see \pcref{Insight Points}).
            \item Six \glossterm{skill points}, which you can spend to learn skills (see \pcref{Skills}).
            \item One \glossterm{attunement point}, which you can use to attune to items and abilities that affect you (see \pcref{Attunement Points}).
            \item A \plus3 bonus to your \glossterm{fatigue tolerance}, which makes it easier for you to use powerful abilities that fatigue you (see \pcref{Fatigue}).
        \end{itemize}

        \cf{Pal}{Weapon Proficiencies}
        You are proficient with simple weapons and any two other \glossterm{weapon groups}.

        \cf{Pal}{Armor Proficiencies} 
        You are proficient with all armor.

        \cf{Pal}{Skills}
        You have the following \glossterm{class skills}:
        \begin{itemize}
            \item \subparhead{Dexterity} Ride.
            \item \subparhead{Constitution} Endurance.
            \item \subparhead{Intelligence} Craft, Deduction, Knowledge (local, religion), Medicine.
            \item \subparhead{Perception} Awareness, Social Insight.
            \item \subparhead{Other} Deception, Intimidate, Persuasion, Profession.
        \end{itemize}

    \subsection{Special Class Abilities}

        \cf{Pal}{Devoted Alignment} 
        You are devoted to a specific alignment.
        You must choose one of your alignment components: good, evil, lawful, or chaotic.
        The alignment you choose is your devoted alignment.
        Your paladin abilities are affected by this choice.
        % seems unnecessary
        % You excel at slaying creatures with alignments opposed to your devoted alignment.
        Your alignment cannot be changed without extraordinary repurcussions.

    \newpage
    \subsection{Devoted Paragon}
        This archetype deepens your connection to your alignment, granting you an aura and improving your combat abilities.

        \cf{Pal}[0]{Devoted Endurance} You gain a \plus2 bonus to your \glossterm{fatigue tolerance}.

        \cf{Pal}[1]{Aligned Aura}[Magical]
        Your devotion to your alignment affects the world around you, bringing it closer to your ideals.
        You constantly radiate an aura in a \areamed radius \glossterm{emanation} from you.
        You can suppress or resume the aura as a \glossterm{minor action}.
        Whenever you resume the aura, you can choose which creatures within the area are affected by aura as any combination of yourself, your \glossterm{allies}, your \glossterm{enemies}, and other creatures.
        The effect of the aura depends on your devoted alignment, as described below.

        \subparhead{Chaos} When a target rolls a 1 on an attack roll with a \glossterm{strike}, it \glossterm{explodes} (see \pcref{Exploding Attacks}.
        This does not affect bonus dice rolled for exploding attacks (see \pcref{Exploding Attacks}).
        \subparhead{Evil} Each target suffers a \minus1 penalty to \glossterm{defenses} as long as it is affected by at least one \glossterm{condition}.
        % TODO: clarify what happens if multiple people try to Good aura the same target
        \subparhead{Good} When a target gains a \glossterm{vital wound}, you may gain a \glossterm{vital wound} instead.
        You gain a \plus2 bonus to the \glossterm{vital roll} of each \glossterm{vital wound} you gain this way.
        The target suffers any other effects of the attack normally.
        \subparhead{Law} When a target rolls a 1 on an attack roll with a \glossterm{strike}, the attack roll is treated as a 6.
        This does not affect bonus dice rolled for exploding attacks (see \pcref{Exploding Attacks}).

        \cf{Pal}[2]{Aligned Immunity}[Magical]
        Your devotion to your alignment grants you immunities.

        \subparhead{Chaos} You are immune to the \slowed, \decelerated, and \immobilized effects.
        \subparhead{Evil} You are immune to poisons and diseases.
        \subparhead{Good} You are immune to the \shaken, \frightened, and \panicked effects.
        \subparhead{Law} You are immune to the \dazed, \stunned, \disoriented, and \confused effects.

        \cf{Pal}[3]{Paragon Power}
        You gain a \plus2 bonus to your \glossterm{power} with all abilities.

        \cf{Pal}[4]{Greater Aligned Aura}[Magical]
        The effect of your \textit{aligned aura} becomes stronger, as described below.

        \subparhead{Chaos} The effect applies to all attacks, not just \glossterm{strikes}.
        % TODO: explain how this works on monsters
        \subparhead{Evil} When a target removes a \glossterm{condition}, it increase its \glossterm{fatigue level} by two.
        \subparhead{Good} When a target would lose \glossterm{hit points}, you may lose those hit points instead.
        The target suffers any other effects of the attack normally, though it is not treated as if it lost hit points from the attack for the purpose of special attack effects.
        \subparhead{Law} The effect applies to all attacks, not just \glossterm{strikes}.

        \cf{Pal}[5]{Greater Aligned Immunity}[Magical]
        The effect of your \textit{aligned immunity} ability is shared with your \glossterm{allies} within the area of your \textit{aligned aura}.

        \cf{Pal}[6]{Greater Paragon Power} The bonus from your \textit{paragon power} ability increases to \plus6.

        \cf{Pal}[7]{Supreme Aligned Aura}[Magical]
        The area affected by your \textit{aligned aura} ability increases to a \gargarea radius \glossterm{emanation} from you.

        % \cf{Pal}[8]{Epic Devotion} The bonus from your \textit{paragon of power} ability increases to \plus6.
        % In addition, the bonus from your \textit{devoted mind} ability increases to \plus5.

        % TODO: move to non-archetype subsystem
        % \cf{Pal}[8]{Aligned Soul}[Magical]
        % While you are dead, you may approach the deity or governing figure of your afterlife and request to be returned to life to continue your mission.
        % Travelling to the relevant entity and making the request takes 12 hours.
        % This request is almost always granted unless there are extenuating circumstances.
        % You are resurrected in a new body at a location of the entity's choice, which is usually a place of power for the entity.
        % This functions like the \ritual{resurrection} ritual, except that no part of the body is required, and a new body is created by the entity.
        % You can be resurrected in this way regardless of the condition of your body, but not if your soul has been trapped or otherwise prevented from going to the correct afterlife.

    \newpage
    \subsection{Divine Magic}
        This archetype grants you the ability to cast divine spells.
        All abilities from this archetype are \glossterm{magical}.

        \cf{Pal}[0]{Cantrips}
        Your deity grants you the ability to use divine magic.
        You gain access to one divine \glossterm{mystic sphere} (see \pcref{Divine Mystic Spheres}).
        You may spend \glossterm{insight points} to gain access to one additional divine \glossterm{mystic sphere} per two \glossterm{insight points}.
        You automatically learn all \glossterm{cantrips} from any mystic sphere you have access to.
        You do not yet gain access to any other spells from those mystic spheres.

        Divine spells require \glossterm{verbal components} to cast (see \pcref{Casting Components}).
        For details about mystic spheres and casting spells, see \pcref{Spell and Ritual Mechanics}.

        \cf{Pal}[1]{Spellcasting}
        You become a rank 1 divine spellcaster.
        You learn two rank 1 \glossterm{spells} from divine \glossterm{mystic spheres} you have access to.
        You can also spend \glossterm{insight points} to learn one additional rank 1 spell per \glossterm{insight point}.
        Unless otherwise noted in a spell's description, casting a spell requires a \glossterm{standard action}.

        When you gain access to a new \glossterm{mystic sphere} or spell \glossterm{rank},
            you can forget any number of spells you know to learn that many new spells in exchange,
            including spells of the higher rank.
        All of those spells must be from divine mystic spheres you have access to.

        \cf{Pal}[2]{Spell Rank} You become a rank 2 divine spellcaster.
        This gives you access to spells that require a minimum rank of 2.

        \cf{Pal}[2]{Spell Knowledge} You learn an additional divine \glossterm{spell} from a \glossterm{mystic sphere} you have access to.

        \cf*{Pal}[3]{Spell Rank} You become a rank 3 divine spellcaster.
        This gives you access to spells that require a minimum rank of 3 and can improve the effectiveness of your existing spells.

        \cf*{Pal}[4]{Spell Rank} You become a rank 4 divine spellcaster.
        This gives you access to spells that require a minimum rank of 4 and can improve the effectiveness of your existing spells.

        \cf*{Pal}[4]{Spell Knowledge} You learn an additional divine \glossterm{spell} from a \glossterm{mystic sphere} you have access to.

        \cf*{Pal}[5]{Spell Rank} You become a rank 5 divine spellcaster.
        This gives you access to spells that require a minimum rank of 5 and can improve the effectiveness of your existing spells.

        \cf*{Pal}[6]{Spell Rank} You become a rank 6 divine spellcaster.
        This gives you access to spells that require a minimum rank of 6 and can improve the effectiveness of your existing spells.

        \cf*{Pal}[7]{Spell Rank} You become a rank 7 divine spellcaster.
        This gives you access to spells that require a minimum rank of 7 and can improve the effectiveness of your existing spells.

        \cf*{Pal}[7]{Spell Knowledge} You learn an additional divine \glossterm{spell} from a \glossterm{mystic sphere} you have access to.

    \newpage
    \subsection{Divine Spell Expertise}
        This archetype improves the divine spells you cast.
        You must have the Divine Magic archetype from the paladin class to gain the abilities from this archetype.

        \cf{Pal}[0]{Combat Caster} You reduce your \glossterm{focus penalty} by 2.

        \cf{Pal}[1]{Insight Point} You gain an additional \glossterm{insight point}.

        \cf{Pal}[2]{Greater Combat Caster} The penalty reduction from your \textit{combat caster} ability increases to 4.

        \cf{Pal}[3]{Wellspring of Power}[Magical]
        You gain a \plus2 bonus to your \glossterm{magical} \glossterm{power}.

        \cf*{Pal}[4]{Insight Point} You gain an additional \glossterm{insight point}.

        \cf{Pal}[5]{Supreme Combat Caster} You gain a \plus1 bonus to Armor defense.

        \cf{Pal}[6]{Greater Wellspring of Power}[Magical]
        The bonus from your \textit{wellspring of power} ability increases to \plus6.

        \cf*{Pal}[7]{Insight Point} You gain an additional \glossterm{insight point}.

    \newpage
    \subsection{Stalwart Guardian}
        This archetype grants you healing abilities and improves your defensive prowess.

        \cf{Pal}[0]{Stalwart Defense} You gain a \plus1 bonus to Fortitude defense and Mental defense.

        \cf{Pal}[1]{Lay on Hands}[Magical] You can use the \textit{lay on hands} ability as a standard action.
        \begin{instantability}{Lay on Hands}
            \spelltwocol{Instant}{\abilitytag{Healing}, \abilitytag{Magical}}
            \rankline
            Choose yourself or a living \glossterm{ally} within your \glossterm{reach}.
            The target regains \glossterm{hit points} equal to 1d8 plus half your \glossterm{power}.
            After you use this ability, you cannot use it or any other \abilitytag{Healing} ability until after the end of the next round.

            \rankline
            \rank{2} The healing increases to 1d10.
            \rank{3} The healing increases to 2d8.
            \rank{4} The healing increases to 2d10.
            \rank{5} The healing increases to 4d8.
            \rank{6} The healing increases to 4d10.
            \rank{7} The healing increases to 6d10.
        \end{instantability}

        \cf{Pal}[2]{Stalwart Resilience} You gain a bonus equal to your rank in this archetype to your \glossterm{resistances} against both \glossterm{physical damage} and \glossterm{energy damage} (see \pcref{Resistances}).

        \cf{Pal}[3]{Greater Stalwart Defense} The bonus from your \textit{stalwart defense} ability increases to \plus2.

        \cf{Pal}[4]{Greater Lay on Hands}[Magical] When you use your \textit{lay on hands} ability, you can choose to remove a \glossterm{condition} of the target's choice from it instead of restoring its hit points.
        Alternately, you can choose to remove a \glossterm{vital wound} of the target's choice from it.
        If a vital wound is removed in this way, you increase your \glossterm{fatigue level} by two.

        \cf{Pal}[5]{Greater Stalwart Resilience} The resistance bonuses from your \textit{stalwart resilience} ability both increase to be equal to twice your rank in this archetype.

        \cf{Pal}[6]{Supreme Stalwart Defense} The bonus from your \textit{stalwart defense} ability increases to \plus3.

        \cf{Pal}[7]{Supreme Lay on Hands}[Magical] When you use your \textit{lay on hands} ability, you and the target each gain a \plus1 bonus to all defenses during the current phase.
        This is a \abilitytag{Swift} effect, so it protects against attacks made during the same phase.

    \newpage
    \subsection{Zealous Warrior}
        This archetype improves your combat prowess, especially against foes who do not share your devoted alignment.
        % The +d and ability ranks are more powerful than some other archetypes, but on par with maneuver archetypes.
        % Since the abilities from this archetype don't stack well with other archetypes, it's not a great dip, so it can be a little stronger.
        % Without +d scaling, it's hard to justify using melee instead of spells at higher levels.

        \cf{Pal}[0]{Zealous Exertion} You gain a \plus2 bonus to any roll that you use the \textit{desperate exertion} ability on.
        This bonus stacks with the normal \plus2 bonus provided by that ability.

        \cf{Pal}[1]{Smite}[Magical] You can use the \textit{smite} ability as a standard action.
        % This ability gets rank upgrades while maneuvers don't
        % because maneuvers are upgraded into higher rank maneuvers,
        % while this has no upgrades and should remain relevant alone.
        \begin{instantability}{Smite}
            Instant
            \rankline
            Make a \glossterm{strike}.
            Because this is a \glossterm{magical} ability, you use your magical \glossterm{power} to determine your damage instead of your \glossterm{mundane} power.
            If your target shares your devoted alignment, the strike deals no damage.
            Otherwise, you gain a \plus1d bonus to damage with the strike.

            \rankline
            \rank{3} The damage bonus increases to \plus2d.
            \rank{5} The damage bonus increases to \plus3d.
            \rank{7} The damage bonus increases to \plus4d.
        \end{instantability}

        \cf{Pal}[2]{Zealous Offense} You gain a \plus1 bonus to \glossterm{accuracy}.

        \cf{Pal}[3]{Glancing Strikes} Whenever you miss by 2 or less with a \glossterm{strike}, the target takes half damage from the strike.
        This is called a \glossterm{glancing blow}.

        \cf{Pal}[4]{Forceful Zeal} You gain a \plus1d bonus to your damage with all weapons.

        \cf{Pal}[4]{Zealous Purge} You can use your \textit{zealous purge} ability as a standard action.
        \begin{instantability}{Zealous Purge}
            \spelltwocol{Instant}{\abilitytag{Magical}}
            \rankline
            Make a \glossterm{strike}.
            You add half your \glossterm{power} to damage with the strike instead of your full power.
            Because this is a \glossterm{magical} ability, you use your \glossterm{magical} \glossterm{power} to determine your damage instead of your \glossterm{mundane} power.

            If the target takes damage from the strike, it stops being \glossterm{attuned} to one effect.
            It can freely choose which effect it releases its attunement to.

            \rankline
            \rank{6} The target stops being attuned to two effects instead of one.
        \end{instantability}

        \cf{Pal}[5]{Greater Zealous Offense} The bonus from your \textit{zealous offense} ability increases to \plus2.

        \cf{Pal}[6]{Greater Zealous Exertion} The bonus from your \textit{zealous exertion} ability increases to \plus5.

        \cf{Pal}[6]{Zealous Fixation} Whenever you hit a creature with a \glossterm{strike}, you ignore \glossterm{concealment} and all \glossterm{miss chances} against that creature with your attacks until you take a \glossterm{short rest} or until you hit a different creature with a strike.
        If you hit multiple creatures with the same strike, you may freely choose which creature to fixate on with this ability.

        \cf{Pal}[7]{Greater Forceful Zeal} The bonus from your \textit{forceful zeal} ability increases to \plus2d.

        \cf{Pal}[7]{Pass Judgment} You can use the \textit{pass judgment} ability as a \glossterm{minor action}.
        \begin{durationability}{Pass Judgment}
            \spelltwocol{Duration}{\abilitytag{Magical}}
            \rankline
            \target{One creature within \rnglong range}
            The target is treated as if it had the alignment opposed to your devoted alignment for the purpose of all abilities.
            This only affects its alignment along the alignment axis your devoted alignment is on.
            For example, if your devoted alignment was evil, a chaotic neutral target would be treated as chaotic good.
            This ability lasts until you \glossterm{dismiss} it as a \glossterm{free action}.

            You can use this ability to do battle against foes who share your alignment, but you should exercise caution in doing so.
            Persecution of those who share your ideals can lead you to fall and become an ex-paladin.
        \end{durationability}

        % TODO: Doesn't work with new spell system
        % \cf{Pal}[20]{Martyr's Retribution}[Mag]
        % If you die in the service of your devoted alignment, you may choose to have your fallen body erupt in an immense burst of divine energy.
        % If you do, your body is almost completely consumed, preventing you from being raised with \ritual{resurrection} and similar effects that require an intact body.
        % This burst has two effects.
        % First, a \spell{sunburst} spell immediately takes effect over the area where you died.
        % Second, a \spell{storm of vengeance} spell begins to take effect, centered on the same area.
        % The spell lasts for 10 rounds, and the lightning strikes target the paladin's enemies.
        % Both of these effects harm only the paladin's foes, and do not harm your allies.
        % However, your allies' vision is still impeded by the \spell{storm of vengeance}.

    \subsection{Ex-Paladins}
        If you cease to follow your devoted alignment, you lose all \glossterm{magical} paladin class abilities.
        If your atone for your misdeeds and resume the service of your devoted alignment, you can regain your abilities.

\newpage
\section{Ranger}\label{Ranger}
    \begin{dtable!*}
        \lcaption{Ranger Progression}
\begin{dtabularx}{\textwidth}{l l >{\lcol}X >{\lcol}X >{\lcol}X > {\lcol}X}
    \tb{Rank} & \tb{Min Level} & \tb{Beastmaster}         & \tb{Huntmaster}          & \tb{Scout}                & \tb{Wilderness Warrior} \tableheaderrule
    0 & \tdash & Beast affinity           & Tracker                  & Keen vision               & Wild senses                                \\
    1 & 1      & Animal companion         & Quarry                   & Skirmisher                & Combat styles                             \\
    2 & 4      & Pack tactics             & Hunting style            & Perceive weakness         & Combat style rank (2), wild force          \\
    3 & 7      & Power of beasts          & Hunter's prowess         & Blindsight                & Combat style rank (3), glancing strikes    \\
    4 & 10     & Greater animal companion & Greater quarry           & Greater skirmisher        & Combat style rank (4), wild maneuver       \\
    5 & 13     & Greater pack tactics     & Hunting style            & Greater perceive weakness & Combat style rank (5), greater wild force  \\
    6 & 16     & Greater power of beasts  & Greater hunter's prowess & Greater blindsight        & Combat style rank (6), greater wild senses \\
    7 & 19     & Supreme animal companion & Supreme quarry           & Supreme skirmisher        & Combat style rank (7), wild maneuver       \\
\end{dtabularx}
    \end{dtable!*}

    \classbasics{Alignment} Any.

    \classbasics{Archetypes} Rangers have the Beastmaster, Huntmaster, Scout, and Wilderness Warrior \glossterm{archetypes}.

    \subsection{Basic Class Abilities}
        If you are a ranger, you gain the following abilities.
        % 1 AD, 4 insight, 6 skill, 4 AP, 6 fatigue, 2 armor, 2.5 weapons = 25.5 points

        \cf{Rgr}{Defenses}
        You gain the following bonuses to your \glossterm{defenses}: \plus2 Armor, \plus5 Fortitude, \plus5 Reflex, \plus4 Mental.

        \cf{Rgr}{Resources} You have the following \glossterm{resources}:
        \begin{itemize}
            \item Two \glossterm{insight points}, which you can spend to gain additional abilities or proficiencies (see \pcref{Insight Points}).
            \item Nine \glossterm{skill points}, which you can spend to learn skills (see \pcref{Skills}).
            \item One \glossterm{attunement point}, which you can use to attune to items and abilities that affect you (see \pcref{Attunement Points}).
            \item A \plus3 bonus to your \glossterm{fatigue tolerance}, which makes it easier for you to use powerful abilities that fatigue you (see \pcref{Fatigue}).
        \end{itemize}

        \cf{Rgr}{Weapon Proficiencies} 
        You are proficient with simple weapons and any one other \glossterm{weapon group}.
        In addition, you are also proficient with your choice of bows, crossbows, or thrown weapons.

        \cf{Rgr}{Armor Proficiencies} 
        You are proficient with light and medium armor.

        \cf{Rgr}{Skills}
        You have the following \glossterm{class skills}:
        \begin{itemize}
            \item \subparhead{Strength} Climb, Jump, Swim.
            \item \subparhead{Dexterity} Agility, Flexibility, Ride, Stealth.
            \item \subparhead{Constitution} Endurance.
            \item \subparhead{Intelligence} Craft, Deduction, Knowledge (dungeoneering, geography, nature), Medicine.
            \item \subparhead{Perception} Awareness, Creature Handling, Survival.
            \item \subparhead{Other} Deception, Intimidate, Persuasion, Profession.
        \end{itemize}

    \newpage
    \subsection{Beastmaster}
        This archetype improves your connection to animals, allowing you to control and command them in battle.

        \cf{Rgr}[0]{Beast Affinity} You gain a \plus3 bonus to the Creature Handling skill (see \pcref{Creature Handling}).
        In addition, you gain a \plus1 bonus to \glossterm{accuracy} and \glossterm{defenses} against animals and magical beasts.

        \cf{Rgr}[1]{Animal Companion}
        You can use the \textit{animal companion} ability.
        This ability requires 8 hours of training and attunement which the target must actively participate in.
        You can compel a wild animal to undergo this training by sustaining the \textit{command} ability from the Creature Handling skill (see \pcref{Command}).
        \begin{attuneability}{Animal Companion}
            \spelltwocol{\abilitytag{Attune} (self)}{\abilitytag{Emotion}, \glossterm{Magical}}
            \rankline
            Choose a Medium or smaller animal \glossterm{ally} within your \glossterm{reach} with a level no higher than your level and a \glossterm{challenge rating} no higher than 1.
            The target serves as a loyal companion to you.
            It follows your directions to the best of its ability.

            Your magical connection to the animal improves its resilience and strength in combat.
            If any of its statistics are higher than the normal values below, the animal uses its own statistics instead.
            All other aspects of the animal, such as its speed and natural weapons, are unchanged.
            % Same as Natural Servant except that it gains resistances since having the animal die is more problematic
            \begin{itemize}
                % TODO: figure out why this is a 2
                \item Its \glossterm{fatigue tolerance} is 2.
                \item Its \glossterm{hit points} are equal to your Constitution \add the base value for your level (see \tref{Character Advancement}).
                \item Its \glossterm{resistance} to \glossterm{physical damage} is equal to your Constitution \add the base value for your level (see \tref{Character Advancement}).
                \item Its \glossterm{resistance} to \glossterm{energy damage} is equal to your Willpower \add the base value for your level.
                \item Each of its \glossterm{defenses} is normally equal to 5 \add your level.
                \item Its \glossterm{accuracy} is equal to your level \add half your base Perception \add your \glossterm{magic bonuses} to accuracy.
                \item Its \glossterm{power} with its attacks is 0.
                \item It has no \glossterm{attunement points}.
                \item The damage dealt by its natural weapons increases by \plus1d for each rank in this archetype beyond 1.
            \end{itemize}

            % Oddly placed? there must be text between an itemize block and the end of a mdframed env
            Animals are unable to understand complex concepts, so their ability to obey convoluted instructions is limited.
        \end{attuneability}

        \cf{Rgr}[2]{Pack Tactics} Any \surrounded \glossterm{enemy} that is adjacent to you or your animal companion takes a \minus1 penalty to \glossterm{accuracy}.

        \cf{Rgr}[3]{Power of Beasts} You and your \textit{animal companion} gain a \plus1d bonus to damage with all weapons.

        \cf{Rgr}[4]{Greater Animal Companion} Your \textit{animal companion} gains an \glossterm{attunement point}.
        In addition, it gains a \plus1 bonus to \glossterm{accuracy}, \glossterm{defenses}, and \glossterm{vital rolls}.

        \cf{Rgr}[5]{Greater Pack Tactics} The penalty from your \textit{pack tactics} ability applies to any creature that is adjacent to you or your animal companion, regardless of whether it is surrounded.

        \cf{Rgr}[6]{Greater Power of Beasts} The bonus from your \textit{power of beasts} ability increases to \plus2d.

        \cf{Rgr}[6]{Greater Beast Affinity} The bonus to the Creature Handling skill from your \textit{beast affinity} ability increases to \plus6.
        In addition, the bonuses to accuracy and defenses from that ability increase to \plus2.

        \cf{Rgr}[7]{Supreme Animal Companion} Your \textit{animal companion} gains an additional \glossterm{attunement point}.
        In addition, the bonuses from your \textit{greater animal companion} ability increase to \plus2.

    \newpage
    \subsection{Huntmaster}
        This archetype grants you and your allies abilities to hunt down specific foes.

        \cf{Rgr}[0]{Tracker} You gain a \plus4 bonus to Survival checks to follow tracks.
        In addition, using the \textit{desperate exertion} ability on a Survival check to follow tracks only causes you to increase your \glossterm{fatigue level} by one instead of two.

        \cf{Rgr}[1]{Quarry}\label{Quarry} You can use the \textit{quarry} ability as a \glossterm{minor action}.
        \begin{attuneability}{Quarry}
            \abilitytag{Attune} (self)
            \rankline
            Choose a creature within \rnglong range.
            The target becomes your quarry.
            You and your \glossterm{allies} within the same range are called your hunting party.
            Your hunting party gains a \plus1 bonus to \glossterm{accuracy} against your quarry.
            If the target is \glossterm{defeated}, you may end this ability and regain the \glossterm{attunement point} you spent to attune to this ability.
        \end{attuneability}

        \cf{Rgr}[2]{Hunting Style}
        You learn specific hunting styles to defeat particular quarries.
        Choose one hunting style from the list below.
        You can also spend \glossterm{insight points} to learn one additional \textit{hunting style} per \glossterm{insight point}.
        When you use your \textit{quarry} ability, you may also use one of your \textit{hunting styles}.
        Each \textit{hunting style} ability lasts as long as the \textit{quarry} ability you used it with.
        {
            \begin{durationability}{Anchoring}
                \spelltwocol{Duration}{Magical}
                \rankline
                As long as your quarry is adjacent to any member of your hunting party, it cannot travel extradimensionally.
                This prevents all \abilitytag{Manifestation} and \glossterm{teleportation} effects.

                \rankline
                \rank{4} This effect instead applies if your quarry is within \medrange of any member of your hunting party.
                \rank{6} This effect instead applies if your quarry is within \distrange of any member of your hunting party.
            \end{durationability}

            \begin{durationability}{Coordinated Stealth}
                Duration
                \rankline
                Your quarry takes a \minus4 penalty to Awareness checks to notice members of your hunting party.

                \rankline
                \rank{4} The Awareness penalty increases to \minus8.
                \rank{6} The Awareness penalty increases to \minus12.
            \end{durationability}

            \begin{durationability}{Cover Weaknesses}
                Duration
                \rankline
                The accuracy bonus against your quarry is replaced with a \plus1 bonus to Armor and Reflex defenses against your quarry's attacks.

                \rankline
                \rank{4} The defense bonus applies to all defenses.
                \rank{6} The defense bonus increases to \plus2.
            \end{durationability}

            \begin{durationability}{Decoy}
                Duration
                \rankline
                If you are adjacent to your quarry, it takes a \minus2 penalty to accuracy on attacks against members of your hunting party other than you.

                \rankline
                \rank{4} The penalty increases to \minus3.
                \rank{6} The penalty increases to \minus4.
            \end{durationability}

            \begin{durationability}{Lifeseal}
                \spelltwocol{Duration}{Magical}
                \rankline
                As long as your quarry is adjacent to any member of your hunting party, it cannot regain \glossterm{hit points}.

                \rankline
                \rank{4} This effect instead applies if the target is within \rngmed range of any member of your hunting party.
                \rank{6} This effect instead applies if your quarry is within \rngdist range of any member of your hunting party.
            \end{durationability}

            \begin{durationability}{Martial Suppression}
                Duration
                \rankline
                As long as your quarry is adjacent to at least two members of your hunting party, it takes a \minus1 penalty to accuracy with \glossterm{mundane} attacks.

                \rankline
                \rank{4} The penalty increases to \minus2.
                \rank{6} The penalty increases to \minus3.
            \end{durationability}

            \begin{durationability}{Mystic Suppression}
                Duration
                \rankline
                As long as your quarry is adjacent to at least two members of your hunting party, it takes a \minus1 penalty to \glossterm{accuracy} with \glossterm{magical} attacks.

                \rankline
                \rank{4} The penalty increases to \minus2. 
                \rank{6} The penalty increases to \minus3.
            \end{durationability}

            \begin{durationability}{Solo Hunter}
                Duration
                \rankline
                Your hunting party other than you gains no benefit from your \textit{quarry} ability.
                In exchange, you gain a \plus1 bonus to defenses against your quarry.

                \rankline
                \rank{4} You gain an additional \plus1 bonus to accuracy against your quarry.
                \rank{6} The defense bonus increases to \plus2.
            \end{durationability}

            \begin{durationability}{Swarm Hunter}
                Duration
                \rankline
                When you use your \textit{quarry} ability, you can target any number of creatures to be your quarry.

                \rankline
                \rank{4} Your hunting party reduces their penalties for being \surrounded by 1.
                \rank{6} The penalty reduction increases to 2.
            \end{durationability}

            \begin{durationability}{Wolfpack}
                Duration
                \rankline
                At the start of each \glossterm{phase}, if your quarry is adjacent to at least two members of your hunting party, it moves at half speed until the end of that phase.

                \rankline
                \rank{4} This effect instead applies if your quarry is adjacent to any member of your hunting party.
                \rank{6} Your quarry is \slowed instead of moving at half speed.
            \end{durationability}
        }

        \cf{Rgr}[3]{Hunter's Prowess} You gain a \plus1d bonus to your damage with all weapons.

        \cf{Rgr}[4]{Greater Quarry} You can use your \textit{quarry} ability with the \abilitytag{Sustain} (free) tag instead of the \abilitytag{Attune} (self) tag.
        If you originally use your \textit{quarry} ability as a sustained ability, you can attune to the same quarry as a free action, even if your quarry is no longer in sight.
        In addition, you gain a \plus10 bonus to follow tracks left by your quarry.

        \cf*{Rgr}[5]{Hunting Style}
        You learn an additional \textit{hunting style}.

        \cf{Rgr}[6]{Greater Hunter's Prowess} The bonus from your \textit{hunter's prowess} ability increases to \plus2d.

        \cf{Rgr}[7]{Supreme Quarry} The accuracy bonus from your \textit{quarry} ability increases to \plus2.

    \newpage
    \subsection{Scout}
        This archetype improves your senses and overall scouting ability.

        \cf{Rgr}[0]{Keen Vision}
        You gain \glossterm{low-light vision}, allowing you to treat sources of light as if they had double their normal illumination range.
        If you already have low-light vision, you double its benefit, allowing you to treat sources of light as if they had four times their normal illumination range.
        In addition, you gain \glossterm{darkvision} with a 60 foot range, allowing you to see in complete darkness clearly.
        If you already have that ability, you increase its range by 60 feet.

        \cf{Rgr}[1]{Skirmisher} You gain a \plus5 foot bonus to your \glossterm{base speed}.
        In addition, you reduce your \glossterm{range increment} penalties for attacking at long range by 1.

        \cf{Rgr}[2]{Perceive Weakness} You gain a \plus1 bonus to \glossterm{accuracy}.

        \cf{Rgr}[3]{Blindsight} Your perceptions are so finely honed that you can sense your enemies without seeing them.
        You gain the \glossterm{blindsense} ability out to 100 feet.
        This ability allows you to sense the presence and location of objects and foes within its range without seeing them.
        If you already have the blindsense ability, you increase its range by 100 feet.
        In addition, you gain the \glossterm{blindsight} ability out to 20 feet.
        With this ability, you can fight just as well with your eyes closed as with them open.
        If you already have the blindsight ability, you increase its range by 20 feet.

        \cf{Rgr}[4]{Greater Skirmisher} The speed bonus from your \textit{skirmisher} ability increases to \plus10 feet.
        In addition, the penalty reduction from that ability increases to 2.

        \cf{Rgr}[5]{Greater Perceive Weakness} The bonus from your \textit{perceive weakness} ability increases to \plus2.

        \cf{Rgr}[6]{Greater Blindsight} The range of your \glossterm{blindsense} ability increases by 200 feet.
        In addition, the range of your \glossterm{blindsight} ability increases by 80 feet.

        \cf{Rgr}[7]{Supreme Skirmisher} The speed bonus from your \textit{skirmisher} ability increases to \plus20 feet.
        In addition, the penalty reduction from that ability increases to 3.

    % Would be nice to call this ``Hunter'' and the maneuvers ``Hunting'' maneuvers,
    % but that conflicts with the naming for Huntmaster
    \newpage
    \subsection{Wild Warrior}
        This archetype grants you abilities to use in combat and improves your physical skills.

        \cf{Rgr}[0]{Wild Senses} You gain a \plus1 bonus to Perception-based checks, except \glossterm{initiative} checks.

        {
            \cf{Rgr}[1]{Combat Styles}
            You can channel your wild energy into ferocious attacks.
            You gain access to one of the following \glossterm{combat styles}: \textit{flurry of blows}, \textit{mobile assault}, or \textit{penetrating precision}.
            In addition, you gain access to any two combat styles of your choice (see \pcref{Combat Styles}).
            You may spend \glossterm{insight points} to gain access to one additional combat style per insight point.

            You learn two rank 1 \glossterm{maneuvers} from combat styles you have access to.
            You may spend \glossterm{insight points} to learn to one additional maneuver per insight point.
            Unless otherwise noted in an ability's description, using a maneuver requires a \glossterm{standard action}.

            When you gain access to a new \glossterm{rank} in this archetype,
                you can exchange any number of maneuvers you know for other maneuvers,
                including maneuvers of the higher rank.
        }

        {
            \cf{Rgr}[2]{Combat Style Rank} You become a rank 2 combat style user.
            This gives you access to maneuvers that require a minimum rank of 2.

            \cf{Rgr}[2]{Wild Force} You gain a \plus1d bonus to your damage with all weapons.
        }

        {
            \cf*{Rgr}[3]{Combat Style Rank} You become a rank 3 combat style user.
            This gives you access to maneuvers that require a minimum rank of 3 and can improve the effectiveness of your existing maneuvers.

            \cf{Rgr}[3]{Glancing Strikes} Whenever you miss by 2 or less with a \glossterm{strike}, the target takes half damage from the strike.
            This is called a \glossterm{glancing blow}.
        }

        {
            \cf*{Rgr}[4]{Combat Style Rank} You become a rank 4 combat style user.
            This gives you access to maneuvers that require a minimum rank of 4 and can improve the effectiveness of your existing maneuvers.

            \cf{Rgr}[4]{Wild Maneuver}
            You learn an additional \glossterm{maneuver} from a combat style you have access to (see \pcref{Combat Styles}).
        }

        {
            \cf*{Rgr}[5]{Combat Style Rank} You become a rank 5 combat style user.
            This gives you access to maneuvers that require a minimum rank of 5 and can improve the effectiveness of your existing maneuvers.

            \cf{Rgr}[5]{Greater Wild Force} The bonus from your \textit{wild force} ability increases to \plus2d.
        }

        {
            \cf*{Rgr}[6]{Combat Style Rank} You become a rank 6 combat style user.
            This gives you access to maneuvers that require a minimum rank of 6 and can improve the effectiveness of your existing maneuvers.

            \cf{Rgr}[6]{Greater Wild Senses} The bonus from your \textit{wild senses} ability increases to \plus2.
        }

        {
            \cf*{Rgr}[7]{Combat Style Rank} You become a rank 7 combat style user.
            This gives you access to maneuvers that require a minimum rank of 7 and can improve the effectiveness of your existing maneuvers.

            \cf*{Rgr}[7]{Wild Maneuver}
            You learn an additional \glossterm{maneuver} from a combat style you have access to (see \pcref{Combat Styles}).
        }

    % \subsection{Boundary Warden}
    %     This archetype grants you abilities improving your ability to guard the boundaries between civilization and nature.

    %     \cf{Rgr}[2]{Favored Terrain} Choose one of the following terrain types: 
    %     You can also spend \glossterm{insight points} to choose one additional creature type per \glossterm{insight point}.
    %     You gain a \plus1 bonus to all \glossterm{defenses} and a \plus10 foot bonus to land speed while in any terrain you chose.

\newpage
\section{Rogue}\label{Rogue}
    \begin{dtable!*}
        \lcaption{Rogue Progression}
\begin{dtabularx}{\textwidth}{l l >{\lcol}X >{\lcol}X >{\lcol}X >{\lcol}X}
    \tb{Rank} & \tb{Min Level} & \tb{Assassin}       & \tb{Bardic Music}                             & \tb{Combat Trickster}                     & \tb{Jack of All Trades} \tableheaderrule
    0 & \tdash & Stealthy instincts  & Bardic lore                            & Tricky finesse                            & Dabbler              \\
    1 & 1      & Sneak attack        & Bardic performances                    & Combat styles           & Skill exemplar          \\
    2 & 4      & Evasion             & Combat inspiration                     & Combat style rank (2), tricky force       & Greater dabbler                 \\
    3 & 7      & Darkstalker         & Bardic performance, loremaster         & Combat style rank (3), glancing strikes         & Versatile power         \\
    4 & 10     & Hide in plain sight & Virtuoso                               & Combat style rank (4), trick maneuver   & Greater skill exemplar  \\
    5 & 13     & Greater evasion     & Greater combat inspiration             & Combat style rank (5), greater tricky force & Supreme dabbler         \\
    6 & 16     & Assassination       & Bardic performance, greater loremaster & Combat style rank (6), greater tricky finesse & Greater versatile power \\
    7 & 19     & Greater darkstalker & Greater virtuoso                       & Combat style rank (7), trick maneuver         & Supreme skill exemplar  \\
\end{dtabularx}
    \end{dtable!*}

    \classbasics{Alignment} Any.

    \classbasics{Archetypes} Rogues have the Assassin, Bardic Music, Combat Trickster, and Jack of All Trades \glossterm{archetypes}.

    \subsection{Basic Class Abilities}
        If you are a rogue, you gain the following abilities.
        % 6 insight, 8 skill, 4 AP, 4 fatigue, 1 armor, 2 weapons = 25 points

        \cf{Rog}{Defenses}
        You gain the following bonuses to your \glossterm{defenses}: \plus1 Armor, \plus3 Fortitude, \plus7 Reflex, \plus5 Mental.

        \cf{Rog}{Resources} You have the following \glossterm{resources}:
        \begin{itemize}
            \item Three \glossterm{insight points}, which you can spend to gain additional abilities or proficiencies (see \pcref{Insight Points}).
            \item Twelve \glossterm{skill points}, which you can spend to learn skills (see \pcref{Skills}).
            \item One \glossterm{attunement point}, which you can use to attune to items and abilities that affect you (see \pcref{Attunement Points}).
            \item A \plus2 bonus to your \glossterm{fatigue tolerance}, which makes it easier for you to use powerful abilities that fatigue you (see \pcref{Fatigue}).
        \end{itemize}

        \cf{Rog}{Weapon Proficiencies} 
        You are proficient with simple weapons, any one other \glossterm{weapon group}, and saps.

        \cf{Rog}{Armor Proficiencies} 
        You are proficient with light armor.

        \cf{Rog}{Skills}
        You have the following \glossterm{class skills}:
        \begin{itemize}
            \item \subparhead{Strength} Climb, Jump, Swim.
            \item \subparhead{Dexterity} Agility, Flexibility, Sleight of Hand, Stealth.
            \item \subparhead{Intelligence} Craft, Deduction, Devices, Disguise, Knowledge (dungeoneering, local), Linguistics.
            \item \subparhead{Perception} Awareness, Social Insight.
            \item \subparhead{Other} Deception, Intimidate, Perform, Persuasion, Profession.
        \end{itemize}

    \newpage
    \subsection{Assassin}
        This archetype improves your agility, stealth, and combat prowess against unaware targets.

        \cf{Rog}[0]{Unseen Assassin} You gain a \plus2 bonus to the Stealth skill.
        In addition, using the \textit{desperate exertion} ability on a Stealth check only causes you to increase your \glossterm{fatigue level} by one instead of two.

        \cf{Rog}[1]{Sneak Attack}
        You can use the \textit{sneak attack} ability as a standard action.
        % This ability gets rank upgrades while maneuvers don't
        % because maneuvers are upgraded into higher rank maneuvers,
        % while this has no upgrades and should remain relevant alone.
        \begin{instantability}{Sneak Attack}
            Instant
            \rankline
            Make a \glossterm{strike} with a \glossterm{light weapon} or a \glossterm{projectile weapon} against a creature within \rngshort range.
            If the target is \unaware or \partiallyunaware of your attack, or if the target is adjacent to one of your \glossterm{allies}, you gain a \plus2d bonus to damage with the strike.
            You do not gain this damage bonus against creatures who that you are unable to score a \glossterm{critical hit} against, such as excessively large creatures or oozes.

            \rankline
            \rank{3} The damage bonus increases to \plus3d.
            \rank{5} The damage bonus increases to \plus4d.
            \rank{7} The damage bonus increases to \plus5d.
        \end{instantability}

        \cf{Rog}[2]{Evasion} You take half damage from abilities that affect an area.
        This does not protect you from any non-damaging effects of those abilities, or from abilities that affect multiple specific targets without affecting an area.

        \cf*{Rog}[3]{Darkstalker} You can use the \textit{darkstalker} ability as a standard action.
        \begin{attuneability}{Darkstalker}
            \spelltwocol{\abilitytag{Attune} (self)}{\abilitytag{Magical}}
            \rankline
            You become completely undetectable by your choice of one of the following sense groups:
            \begin{itemize}
                \item \glossterm{Blindsense} and \glossterm{blindsight}
                \item \glossterm{Darkvision}
                \item \abilitytag{Detection} abilities
                \item \glossterm{Lifesense} and \glossterm{lifesight}
                \item \glossterm{Scent}
                \item \abilitytag{Scrying} abilities
                \item \glossterm{Tremorsense} and \glossterm{tremorsight}
            \end{itemize}
            If you have access to any other more unusual senses, such as the \textit{blood sense} ability from the Executioner feat, you may also choose one of those senses as a separate sense group.
        \end{attuneability}

        \cf{Rog}[4]{Hide in Plain Sight}
        You can use the \textit{hide} ability without moving in a way that causes observers to lose sight of you.
        This does not remove the bonus that observers receive if you have no cover or concealment at all.

        \cf{Rog}[5]{Greater Evasion} You can use your Reflex defense in place of any other defense against abilities that affect an area.
        This does not protect you from abilities that affect multiple specific targets without affecting an area.

        \cf{Rog}[6]{Assassination} You can use the \textit{assassination} ability as a \glossterm{minor action}.
        \begin{durationability}{Assassination}
            \spelltwocol{Duration}{\abilitytag{Swift}}
            \rankline
            You study a creature within \rngmed range, finding weak points you can take advantage of.
            Until the end of the next round, the subject takes maximum damage from your melee \glossterm{strikes} that it is \unaware of.
        \end{durationability}

        \cf{Rog}[7]{Greater Darkstalker}[Magical] When you use your \textit{darkstalker} ability, you become undetectable by any number of the possible sense groups, not just one.

    \newpage
    \subsection{Bardic Music}
        This archetype grants you the ability to inspire your allies and impair your foes with musical performances.

        \cf{Rog}[0]{Bardic Lore} You gain two additional skill points.
        In addition, you gain all Knowledge skills as \glossterm{class skills}.

        \cf{Rog}[1]{Bardic Performances}[Magical]
        You learn two \textit{bardic performances} from the list below.
        You can also spend \glossterm{insight points} to learn one additional bardic performance per \glossterm{insight point}.
        You can use any bardic performance you know as a \glossterm{standard action} unless it specifies that it requires a different type of action to activate.

        All \textit{bardic performances} have the \abilitytag{Auditory} tag.
        When you use a \textit{bardic performance} ability, you begin a performance using one of your Perform skills.
        You must use either an instrumental performance or a vocal performance, and not a visual performance.
        If you use a vocal performance, the bardic performance gains the \abilitytag{Speech} tag, preventing it from affecting creatures that do not speak the language you perform in.
        You must be \glossterm{trained} with a Perform skill capable of making an auditory performance to use a bardic performance ability.
        If you are \glossterm{mastered} with an appropriate Perform skill, you gain a \plus2 bonus to \glossterm{accuracy} with the ability.

        The names of bardic performances do not have to precisely match your actual performance.
        For example, you can use the \textit{palliative poem} ability with a gentle song using Perform (wind instruments) or a distracting joke using Perform (comedy) instead of a poem.

        Many bardic performances require you to sustain the performance as a \glossterm{minor action}.
        If the targets stop being able to see or hear you, depending on the nature of your performance, the effect ends for them as if you had stopped sustaining the performance.
        However, targets do not stop being affected by your performance simply by travelling beyond the initial range of the bardic performance ability.
        Using a bardic performance ability with an immediate effect does not interfere with your ability to sustain other bardic performance abilities.
        {
            \begin{durationability}{Aria of Alacrity}
                \spelltwocol{\abilitytag{Sustain} (minor)}{\abilitytag{Auditory}}
                \par \noindent Usage time: One \glossterm{minor action}.
                \rankline
                For the duration of your performance, you and all \glossterm{allies} within a \arealarge radius from you each gain a \plus5 foot bonus to their \glossterm{base speed}.

                \rankline
                \rank{3} The speed bonus increases to \plus10 feet.
                \rank{5} The speed bonus increases to \plus15 feet.
                \rank{7} The speed bonus increases to \plus20 feet.
            \end{durationability}

            % Bardic performance power guidelines:
            % These generally start from the same rank 1 baseline effect as spells.
            % Since there are no higher rank bardic performances, they need more aggressive rank scalings to ensure
            % that a rank 7 bardic performance is comparable to a rank 7 spell. That is provided by the greater/supreme
            % bardic performance class abilities, and doesn't need to be included in each individual performance.
            % In general, bardsongs are likely to trade damage or accuracy for increased area.
            %
            % Bardsong debuffs are interesting, since they can't be removed like conditions, but also can't be stacked.
            % For now, they're just ranked in the same way as conditions.
            \begin{durationability}{Ballad of Belligerence}
                \spelltwocol{\abilitytag{Sustain} (minor)}{\abilitytag{Auditory}, \abilitytag{Emotion}}
                \rankline
                Make an attack vs. Mental against all \glossterm{enemies} in a \areamed radius from you.
                \hit For the duration of your performance, each subject is unable to take any \glossterm{standard actions} that do not cause it to make an attack.
                For example, a subject could make a \glossterm{strike} or cast an offensive spell, but it could not heal itself or summon a creature.

                \rankline
                You gain a \plus2 bonus to \glossterm{accuracy} with the attack for each rank beyond 1.
            \end{durationability}

            \begin{durationability}{Boastful Bravura}
                \spelltwocol{Duration}{\abilitytag{Auditory}}
                \rankline
                This ability affects all \glossterm{enemies} within a \arealarge radius from you.
                During the next round, you gain a \plus4 bonus to \glossterm{accuracy} with \textit{bardic performance} abilities against each subject.

                \rankline
                \rank{3} The bonus increases to \plus6.
                \rank{5} The bonus increases to \plus8.
                \rank{7} The bonus increases to \plus10.
            \end{durationability}

            \begin{instantability}{Cacaphony}
                \spelltwocol{Instant}{\abilitytag{Auditory}}
                \rankline
                Make an attack vs. Fortitude against all \glossterm{enemies} in a \tinyarea radius from you.
                \hit Each subject takes 1d6 \add half \glossterm{power} sonic damage.

                \rankline
                \rank{2} The damage increases to 1d8, and the area increases to a \smallarea radius.
                \rank{3} The damage increases to 1d10, and the area increases to a \medarea radius.
                    In addition, if you miss by 2 or less, each subject takes half damage.
                    This is called a \glossterm{glancing blow}.
                \rank{4} The damage increases to 2d6.
                \rank{5} The damage increases to 2d8, and the area increases to a \largearea radius.
                \rank{6} The damage increases to 2d10.
                \rank{7} The damage increases to 4d6, and the area increases to a \hugearea radius.
            \end{instantability}

            \begin{durationability}{Cadenza of Courage}
                \spelltwocol{Sustain (minor)}{\abilitytag{Auditory}, \abilitytag{Emotion}}
                \par \noindent Usage time: One \glossterm{minor action}.
                \rankline
                Choose yourself or one \glossterm{ally} within \medrange.
                For the duration of your performance, the subject gains a \plus1 bonus to \glossterm{accuracy}.

                \rankline
                \rank{3} The subject also gains a \plus2 bonus to Mental defense.
                \rank{5} The accuracy bonus increases to \plus2.
                \rank{7} The bonus to Mental defense increases to \plus4.
            \end{durationability}

            \begin{durationability}{Cantata of Caution}
                \spelltwocol{Sustain (minor)}{\abilitytag{Auditory}, \abilitytag{Emotion}}
                \par \noindent Usage time: One \glossterm{minor action}.
                \rankline
                Choose yourself or one \glossterm{ally} within \medrange.
                For the duration of your performance, the subject gains a \plus1 bonus to Armor and Reflex defenses.

                \rankline
                \rank{3} The bonus applies to all defenses.
                \rank{5} The bonus to Armor and Reflex defenses increases to \plus2.
                \rank{7} The bonus to Fortitude and Mental defenses also increases to \plus2.
            \end{durationability}

            \begin{durationability}{Cleansing Counterpoint}
                \spelltwocol{Sustain (minor)}{\abilitytag{Auditory}}
                \rankline
                Choose yourself or one \glossterm{ally} within \rngmed range.
                The subject chooses one of its \glossterm{conditions}.
                It cannot choose a condition applied during the current round.
                For the duration of your performance, the subject ignores all effects of that condition.

                \rankline
                \rank{3} This ability loses the \abilitytag{Sustain} (minor) tag.
                    Instead, the chosen condition is removed entirely.
                \rank{5} You can subject an additional \glossterm{ally} within range.
                \rank{7} Each subject can remove two conditions instead of one.
            \end{durationability}

            \begin{durationability}{Dazzling Discordance}
                \spelltwocol{Duration}{\abilitytag{Auditory}}
                \rankline
                Make an attack vs. Mental against all \glossterm{enemies} within a \areamed radius from you.
                \hit Each subject is \dazzled until the end of the next round.
                \crit Each subject is \dazzled as a \glossterm{condition}.

                \rankline
                \rank{2} You gain a \plus1 \glossterm{accuracy} bonus with the attack, and the area increases to a \largearea radius.
                \rank{3} The accuracy bonus increases to \plus2.
                \rank{4} The accuracy bonus increases to \plus3, and the area increases to a \hugearea radius.
                \rank{5} The accuracy bonus increases to \plus4.
                \rank{6} The accuracy bonus increases to \plus5, and the area increases to a \gargarea radius.
                \rank{7} The accuracy bonus increases to \plus6.
            \end{durationability}

            \begin{instantability}{Dirge of Doom}
                \spelltwocol{Instant}{\abilitytag{Auditory}}
                \rankline
                Make an attack vs. Mental against anything within \medrange.
                \hit The subject takes sonic damage equal to 1d10 plus your \glossterm{power}.

                \rankline
                \rank{2} The damage increases to 2d6.
                \rank{3} The damage increases to 2d10.
                    In addition, if you miss by 2 or less, the subject takes half damage.
                    This is called a \glossterm{glancing blow}.
                \rank{4} The damage increases to 4d6.
                \rank{5} The damage increases to 4d10.
                \rank{6} The damage increases to 5d10.
                \rank{7} The damage increases to 7d10.
            \end{instantability}

            \begin{durationability}{Dizzying Ditty}
                \spelltwocol{\abilitytag{Sustain} (minor)}{\abilitytag{Auditory}, \abilitytag{Compulsion}}
                \rankline
                Make an attack vs. Mental against one creature within \shortrange.
                \hit For the duration of your performance, the subject is \dazed.
                \crit For the duration of your performance, the subject is \stunned.

                \rankline
                You gain a \plus2 bonus to \glossterm{accuracy} with the attack for each rank beyond 1.
            \end{durationability}

            \begin{durationability}{Frightening Fugue}
                \spelltwocol{\abilitytag{Sustain} (minor)}{\abilitytag{Auditory}, \abilitytag{Emotion}}
                \rankline
                Make an attack vs. Mental against one creature within \medrange.
                \hit The subject takes 1d6 sonic damage.
                If it loses \glossterm{hit points} from this damage, it is \frightened by you for the duration of your performance.
                Unlike normal, the subject continues to suffer the penalties of being frightened while it is beyond \rngmed range from you as long as it is still affected by your performance.

                \rankline
                \rank{2} The damage increases to 1d8.
                \rank{3} The damage increases to 2d6.
                    In addition, if you miss by 2 or less, the subject takes half damage.
                    This is called a \glossterm{glancing blow}.
                \rank{4} The damage increases to 2d8.
                \rank{5} The damage increases to 4d6.
                \rank{6} The damage increases to 4d8.
                \rank{7} The damage increases to 5d10.
            \end{durationability}

            \begin{durationability}{Hypnotic Hymn}
                \spelltwocol{\abilitytag{Sustain} (minor)}{\abilitytag{Auditory}, \abilitytag{Emotion}}
                \rankline
                Make an attack vs. Mental against one creature within \medrange.
                You take a \minus10 penalty to \glossterm{accuracy} with this attack against creatures who are engaged in combat during the current phase.
                \hit For the duration of your performance, the subject is \charmed by you.
                Any act by you or by creatures that appear to be your allies that threatens or harms the charmed person breaks the effect.
                Harming the subject is not limited to dealing it damage, but also includes causing it significant subjective discomfort.
                An observant subject may interpret overt threats to its allies as a threat to itself.
                This ability does not have the \abilitytag{Subtle} tag, so an observant subject may notice it is being influenced.

                \rankline
                You gain a \plus2 bonus to \glossterm{accuracy} with the attack for each rank beyond 1.
            \end{durationability}

            \begin{durationability}{Inspiring Intonation}
                \spelltwocol{\abilitytag{Sustain} (minor)}{\abilitytag{Auditory}, \abilitytag{Emotion}}
                \par \noindent Usage time: One \glossterm{minor action}.
                \rankline
                Choose yourself or one \glossterm{ally} within \rngmed range.
                For the duration of your performance, the subject gains a \plus2 bonus to \glossterm{checks}.

                \rankline
                \rank{3} The bonus increases to \plus3.
                \rank{5} The bonus increases to \plus4.
                \rank{7} The bonus increases to \plus5.
            \end{durationability}

            \begin{durationability}{Mesmerizing Melody}
                \spelltwocol{\abilitytag{Sustain} (minor)}{\abilitytag{Auditory}, \abilitytag{Emotion}}
                \rankline
                Make an attack vs. Mental against all creatures in a \largearea radius from you.
                You take a \minus10 penalty to \glossterm{accuracy} with this attack against creatures who are engaged in combat during the current phase.
                \hit For the duration of your performance, each subject is \fascinated by you.
                Any act by you or your apparent allies that harms a subject or that causes it to feel that it is in danger breaks the effect for that creature.
                Harming a subject is not limited to dealing it damage, but also includes causing it significant subjective discomfort.
                An observant subject may interpret overt threats to its allies as a threat to itself.

                \rankline
                You gain a \plus2 bonus to \glossterm{accuracy} with the attack for each rank beyond 1.
            \end{durationability}

            \begin{instantability}{Palliative Poem}
                \spelltwocol{\abilitytag{Sustain} (minor)}{\abilitytag{Auditory}, \abilitytag{Healing}}
                \rankline
                Choose one living \glossterm{ally} within \shortrange.
                The subject regains \glossterm{hit points} equal to 1d8 plus half your \glossterm{power}.
                After you use this ability, you cannot use it or any other \abilitytag{Healing} ability until after the end of the next round.

                \rankline
                \rank{2} The healing increases to 1d10.
                \rank{3} The healing increases to 2d8.
                \rank{4} The healing increases to 2d10.
                \rank{5} The healing increases to 4d8.
                \rank{6} The healing increases to 4d10.
                \rank{7} The healing increases to 6d10.
            \end{instantability}

            \begin{durationability}{Partita of Provocation}
                \spelltwocol{\abilitytag{Sustain} (minor)}{\abilitytag{Auditory}, \abilitytag{Emotion}}
                \rankline
                Make an attack vs. Mental against all \glossterm{enemies} within a \largearea radius from you.
                \hit For the duration of your performance, each subject takes a \minus2 penalty to \glossterm{accuracy} against creatures other than you.
                \crit The penalty increases to \minus4.

                \rankline
                You gain a \plus2 bonus to \glossterm{accuracy} with the attack for each rank beyond 1.
            \end{durationability}

            \begin{durationability}{Serenade of Serenity}
                \spelltwocol{\abilitytag{Sustain} (minor)}{\abilitytag{Auditory}, \abilitytag{Emotion}}
                \par \noindent Usage time: One \glossterm{minor action}.
                \rankline
                For the duration of your performance, you and all \glossterm{allies} within a \largearea radius from you gain a \plus4 bonus to defenses against hostile \abilitytag{Compulsion} and \abilitytag{Emotion} effects.

                \rankline
                \rank{3} At the end of each round, each subject removes all \glossterm{conditions} caused by Compulsion and Emotion effects that were not applied during that round.
                \rank{5} The area increases to a \areahuge radius.
                \rank{7} Each subject is immune to hostile Compulsion and Emotion effects.
            \end{durationability}

            \begin{durationability}{Sickening Staccato}
                \spelltwocol{Duration}{\abilitytag{Auditory}}
                \rankline
                Make an attack vs. Fortitude against all \glossterm{enemies} within a \areamed radius from you.
                \hit Each subject is \sickened until the end of the next round.
                \crit Each subject is \sickened as a \glossterm{condition}.

                \rankline
                \rank{2} You gain a \plus1 \glossterm{accuracy} bonus with the attack, and the area increases to a \largearea radius.
                \rank{3} The accuracy bonus increases to \plus2.
                \rank{4} The accuracy bonus increases to \plus3, and the area increases to a \hugearea radius.
                \rank{5} The accuracy bonus increases to \plus4.
                \rank{6} The accuracy bonus increases to \plus5, and the area increases to a \gargarea radius.
                \rank{7} The accuracy bonus increases to \plus6.
            \end{durationability}

            \begin{durationability}{Vigorous Verse}
                \spelltwocol{Duration}{\abilitytag{Auditory}}
                \par \noindent Usage time: One \glossterm{minor action}.
                \rankline
                Choose yourself or one \glossterm{ally} within \rngmed range.
                For the duration of your performance, the subject gains a \plus4 bonus to its maximum \glossterm{hit points}.
                In addition, it immediately gains that many hit points.
                When this effect ends, the subject loses hit points equal to the hit points it gained this way.

                \rankline
                \rank{3} The bonus increases to \plus8.
                \rank{5} The bonus increases to \plus16.
                \rank{7} The bonus increases to \plus32.
            \end{durationability}
        }

        \cf{Rog}[2]{Combat Inspiration} Once per round, when you make a Perform check, you may choose an \glossterm{ally} that can see or hear your performance.
        During the next round, that ally gains a \plus1 bonus to \glossterm{accuracy}.

        \cf{Rog}[3]{Bardic Performance} You learn an additional bardic performance.

        \cf{Rog}[3]{Loremaster} You gain a \plus2 bonus to all Knowledge skills.

        \cf{Rog}[4]{Virtuoso}[Magical] Once per round, you can \glossterm{sustain} two bardic performances as a single \glossterm{minor action}.

        \cf{Rog}[5]{Greater Combat Inspiration} The bonus from your \textit{combat inspiration} ability increases to \plus2.

        \cf*{Rog}[6]{Bardic Performance} You learn an additional bardic performance.

        \cf{Rog}[6]{Greater Loremaster} The Knowledge skill bonus from your \textit{loremaster} ability increases to \plus4.

        \cf{Rog}[7]{Greater Virtuoso}[Magical] The number of bardic performances you can sustain with your \textit{virtuoso} ability increases to three.

    \newpage
    \subsection{Combat Trickster}
        This archetype grants you abilities to use in combat and improves your combat prowess.

        \cf{Rog}[0]{Tricky Finesse} You gain a \plus1 bonus to Dexterity-based \glossterm{checks}, except \glossterm{initiative} checks.

        {
            \cf{Rog}[1]{Trick Maneuvers}
            You can confuse and confound your foes in combat.
            You gain access to one of the following \glossterm{combat styles}: \textit{dirty fighting}, \textit{ebb and flow}, or \textit{mobile assault}.
            In addition, you gain access to any two combat styles of your choice (see \pcref{Combat Styles}).
            You may spend \glossterm{insight points} to gain access to one additional combat style per insight point.

            You learn two rank 1 \glossterm{maneuvers} from combat styles you have access to.
            You may spend \glossterm{insight points} to learn to one additional maneuver per insight point.
            Unless otherwise noted in an ability's description, using a maneuver requires a \glossterm{standard action}.

            When you gain access to a new \glossterm{rank} in this archetype,
                you can exchange any number of maneuvers you know for other maneuvers,
                including maneuvers of the higher rank.
        }

        {
            \cf{Rog}[2]{Combat Style Rank} You become a rank 2 combat style user.
            This gives you access to maneuvers that require a minimum rank of 2.

            \cf{Rog}[2]{Tricky Force} You gain a \plus1d bonus to your damage with all weapons.
        }

        {
            \cf*{Rog}[3]{Combat Style Rank} You become a rank 3 combat style user.
            This gives you access to maneuvers that require a minimum rank of 3 and can improve the effectiveness of your existing maneuvers.

            \cf{Rog}[3]{Glancing Strikes} Whenever you miss by 2 or less with a \glossterm{strike}, the target takes half damage from the strike.
            This is called a \glossterm{glancing blow}.
        }

        {
            \cf*{Rog}[4]{Combat Style Rank} You become a rank 4 combat style user.
            This gives you access to maneuvers that require a minimum rank of 4 and can improve the effectiveness of your existing maneuvers.

            \cf{Rog}[4]{Trick Maneuver}
            You learn an additional \glossterm{maneuver} from a combat style you have access to (see \pcref{Combat Styles}).
        }

        {
            \cf*{Rog}[5]{Combat Style Rank} You become a rank 5 combat style user.
            This gives you access to maneuvers that require a minimum rank of 5 and can improve the effectiveness of your existing maneuvers.

            \cf{Rog}[5]{Greater Tricky Force} The damage bonus from your \textit{tricky force} ability increases to \plus2d.
        }

        {
            \cf*{Rog}[6]{Combat Style Rank} You become a rank 6 combat style user.
            This gives you access to maneuvers that require a minimum rank of 6 and can improve the effectiveness of your existing maneuvers.

            \cf{Rog}[6]{Greater Tricky Finesse} The bonus from your \textit{tricky finesse} ability increases to \plus2.
        }

        {
            \cf*{Rog}[7]{Combat Style Rank} You become a rank 7 combat style user.
            This gives you access to maneuvers that require a minimum rank of 7 and can improve the effectiveness of your existing maneuvers.

            \cf*{Rog}[7]{Trick Maneuver}
            You learn an additional \glossterm{maneuver} from a combat style you have access to (see \pcref{Combat Styles}).
        }

    \newpage
    \subsection{Jack of All Trades}
        This archetype improves your skills.

        \cf{Rog}[0]{Dabbler} You gain an additional \glossterm{insight point}.
        In addition, choose any two skills.
        You treat those skills as \glossterm{class skills} for you.

        \cf{Rog}[1]{Skill Exemplar} You gain two additional skill points.
        In addition, you gain a \plus1 bonus to all skills.

        \cf{Rog}[2]{Greater Dabbler} You gain an additional \glossterm{insight point}.

        \cf{Rog}[3]{Versatile Power}
        You gain a \plus2 bonus to your \glossterm{power} with all abilities.

        \cf{Rog}[4]{Greater Skill Exemplar} The skill bonus from your \textit{skill exemplar} ability increases to \plus3.

        \cf{Rog}[5]{Supreme Dabbler} You gain an additional \glossterm{insight point}.

        \cf{Rog}[6]{Greater Versatile Power} The bonus from your \textit{versatile power} ability increases to \plus6.

        \cf{Rog}[7]{Supreme Skill Exemplar} The skill bonus from your \textit{skill exemplar} ability increases to \plus5.

    % \subsection{Suave Scoundrel}
    %     This archetype improves your social manipulation abilities.

    %     \cf{Rog}[1]{Confound} You can use the \textit{confound} ability as a standard action.
    %     \begin{freeability}{Confound}[\abilitytag{Compulsion}, \abilitytag{Speech}, \abilitytag{Subtle}]
    %         Make a attack vs. Mental against a creature within \rngshort range.
    %         You can choose to use your Deception skill to attack in place of your normal \glossterm{accuracy}.
    %         \hit As a \glossterm{condition}, the target is \dazed.
    %         That target cannot be affected by this ability again until it takes a \glossterm{short rest}.

    %         \rankline
    %         \rank{3} On a \glossterm{critical hit}, the target is \confused in addition to being \dazed as part of the same \glossterm{condition}.
    %         \rank{5} The range increases to \rngmed.
    %         \rank{7} The target is \confused in addition to being \dazed as part of the same condition.
    %     \end{freeability}

    %     \cf{Rog}[2]{Glibness} You gain two additional \glossterm{skill points}.
    %     In addition, you gain a \plus2 bonus to the Deception, Intimidate, and Persuasion skills.

    %     \cf{Rog}[3]{Distract} You can use the \textit{distract} ability as a standard action.
    %     \begin{freeability}{Distract}[\abilitytag{Compulsion}, \abilitytag{Speech}, \abilitytag{Subtle}]
    %         Make a attack vs. Mental against a creature within \rngshort range.
    %         You can choose to use your Deception skill to attack in place of your normal \glossterm{accuracy}.
    %         \hit Until the end of the next round, the target is \fascinated by a creature or object of your choice within range.
    %         That target cannot be affected by this ability again until it takes a \glossterm{short rest}.

    %         \rankline
    %         \rank{5} The range increases to \rngmed.
    %         \rank{7} The target is also distracted by the creature or object as a \glossterm{condition}.
    %     \end{freeability}

    %     \cf{Rog}[4]{Slippery Mind} You gain a \plus2 bonus to Mental defense.

    %     \cf{Rog}[5]{Greater Glibness} The bonus from your \textit{glibness} ability increases to \plus4.

    %     \cf{Rog}[6]{Compel Belief} You can use the \textit{compel belief} ability as a standard action.
    %     \begin{freeability}{Compel Belief}[\abilitytag{Compulsion}, \glossterm{Magical}, \abilitytag{Speech}, \abilitytag{Subtle}]
    %         Make an attack vs. Mental against a creature within \rngmed range.
    %         You can use your Persuasion skill to attack in place of your normal \glossterm{accuracy}.
    %         You must also choose a belief.
    %         The belief may be a truth the target knows, a lie that you told it, or even a simple misunderstanding (such as believing a hidden creature is not present in a room).
    %         If the creature does not already hold the chosen belief, this ability automatically fails.
    %         \hit As a \glossterm{condition}, the target continues to maintain the chosen belief, regardless of any evidence to the contrary.
    %         It will interpret any evidence that the falsehood is incorrect to be somehow wrong -- an illusion, a conspiracy to decieve it, or any other reason it can think of to continue believing the falsehood.
    %         When the effect ends, the creature can decide whether it believes the falsehood or not, as normal.

    %         \rankline
    %         \rank{8} The first time the condition would be removed, it instead remains.
    %     \end{freeability}

    %     \cf{Rog}[7]{Slippery Mind} The bonus from your \textit{slippery mind} ability increases to \plus4.

\newpage
\section{Sorcerer}\label{Mage}
    \begin{dtable!*}
        \lcaption{Sorcerer Progression}
\begin{dtabularx}{\textwidth}{l l >{\lcol}X >{\lcol}X >{\lcol}X >{\lcol}X >{\lcol}X}
    \tb{Rank} & \tb{Min Level} & \tb{Arcane Magic}                  & \tb{Arcane Spell Mastery}   & \tb{Draconic Magic}                                & \tb{Innate Arcanist}                                   & \tb{Wild Magic} \tableheaderrule
    0         & \tdash         & Cantrips, mage armor               & Combat caster               & Draconic bloodline, draconic focus                 & Innate magic                                           & Chaotic exertion         \\
    1         & 1              & Spellcasting                       & Mystic insight              & Draconic spells                                    & Mystic tolerance                                       & Wildspell                \\
    2         & 4              & Spell rank (2), spell knowledge    & Arcane attunement           & Greater draconic focus, greater draconic bloodline & Personal enhancement                                   & Chaotic insight          \\
    3         & 7              & Spell rank (3), greater mage armor & Wellspring of power         & Draconic hide                                      & Spell absorption                                       & Greater chaotic exertion \\
    4         & 10             & Spell rank (4), spell knowledge    & Mystic insight              & Draconic precision                                 & Greater mystic tolerance, greater personal enhancement & Greater wildspell        \\
    5         & 13             & Spell rank (5)                     & Greater arcane attunement   & Draconic spell, energy immunity                    & Greater spell absorpion                                & Greater chaotic insight  \\
    6         & 16             & Spell rank (6), supreme mage armor & Greater wellspring of power & Greater draconic hide                              & Supreme personal enhancement                           & Supreme chaotic exertion \\
    7         & 19             & Spell rank (7), spell knowledge    & Mystic insight              & Greater draconic precision                         & Supreme mystic tolerance, supreme spell absorption     & Supreme wildspell        \\
\end{dtabularx}
    \end{dtable!*}

    \classbasics{Alignment} Any.

    \classbasics{Archetypes} Sorcerers have the Arcane Magic, Arcane Spell Mastery, Draconic Magic, Innate Arcanist, and Wild Magic \glossterm{archetypes}.

    \subsection{Basic Class Abilities}
        If you are a sorcerer, you gain the following abilities.
        % -1 AD, 6 insight, 4 skill, 12 AP, 4 fatigue = 25 points

        \cf{Sor}{Defenses}
        You gain the following bonuses to your \glossterm{defenses}: \plus0 Armor, \plus3 Fortitude, \plus5 Reflex, \plus8 Mental.

        \cf{Sor}{Resources} You have the following \glossterm{resources}:
        \begin{itemize}
            \item Three \glossterm{insight points}, which you can spend to gain additional abilities or proficiencies (see \pcref{Insight Points}).
            \item Six \glossterm{skill points}, which you can spend to learn skills (see \pcref{Skills}).
            \item Three \glossterm{attunement points}, which you can use to attune to items and abilities that affect you (see \pcref{Attunement Points}).
            \item A \plus2 bonus to your \glossterm{fatigue tolerance}, which makes it easier for you to use powerful abilities that fatigue you (see \pcref{Fatigue}).
        \end{itemize}

        \cf{Sor}{Weapon Proficiencies} 
        You are proficient with simple weapons.

        \cf{Sor}{Armor Proficiencies} 
        You are not proficient with any type of armor.
        Encumbrance from armor interferes with the gestures you make to cast spells, which can cause your spells with \glossterm{somatic components} to fail (see \pcref{Somatic Component Failure}).

        \cf{Sor}{Skills}
        You have the following \glossterm{class skills}:
        \begin{itemize}
            \item \subparhead{Intelligence} Craft, Deduction, Knowledge (arcana, planes), Linguistics.
            \item \subparhead{Perception} Awareness, Spellsense.
            \item \subparhead{Other} Deception, Intimidate, Persuasion, Profession.
        \end{itemize}

    \newpage
    \subsection{Arcane Magic}
        This archetype grants you the ability to cast arcane spells.
        All abilities from this archetype are \glossterm{magical}.

        \cf{Sor}[0]{Cantrips}
        Your deity grants you the ability to use arcane magic.
        You gain access to one arcane \glossterm{mystic sphere} (see \pcref{Arcane Mystic Spheres}).
        You may spend \glossterm{insight points} to gain access to one additional arcane \glossterm{mystic sphere} per two \glossterm{insight points}.
        You automatically learn all \glossterm{cantrips} from any mystic sphere you have access to.
        You do not yet gain access to any other spells from those mystic spheres.

        Arcane spells require both \glossterm{verbal components} and \glossterm{somatic components} to cast (see \pcref{Casting Components}).
        For details about mystic spheres and casting spells, see \pcref{Spell and Ritual Mechanics}.

        \cf{Sor}[0]{Mage Armor} You can use the \textit{mage armor} ability as a standard action.
        \begin{durationability}{Mage Armor}
            Duration
            \rankline
            You create a translucent suit of magical armor on your body and over your hands.
            This functions like body armor that provides a \plus2 bonus to Armor defense and has no \glossterm{encumbrance}.
            It also provides a \plus4 bonus to \glossterm{resistance} against \glossterm{energy damage}, and a \plus2 bonus to \glossterm{resistance} against \glossterm{physical damage}.

            As long as you have a free hand, the barrier also manifests as a shield that provides a \plus1 bonus to Armor defense.
            This bonus is considered to come from a shield, and does not stack with the benefits of using any other shield.

            This ability lasts until you use it again or until you \glossterm{dismiss} it as a free action.
            In addition, it is automatically dismissed if you have other body armor of any kind.
        \end{durationability}

        \cf{Sor}[1]{Spellcasting}
        You become a rank 1 arcane spellcaster.
        You learn two rank 1 \glossterm{spells} from arcane \glossterm{mystic spheres} you have access to.
        You can also spend \glossterm{insight points} to learn one additional rank 1 spell per \glossterm{insight point}.
        Unless otherwise noted in a spell's description, casting a spell requires a \glossterm{standard action}.

        When you gain access to a new \glossterm{mystic sphere} or spell \glossterm{rank},
            you can forget any number of spells you know to learn that many new spells in exchange,
            including spells of the higher rank.
        All of those spells must be from arcane mystic spheres you have access to.

        \cf{Sor}[2]{Spell Rank} You become a rank 2 arcane spellcaster.
        This gives you access to spells that require a minimum rank of 2.

        \cf{Sor}[2]{Spell Knowledge} You learn an additional arcane \glossterm{spell} from a \glossterm{mystic sphere} you have access to.

        \cf*{Sor}[3]{Spell Rank} You become a rank 3 arcane spellcaster.
        This gives you access to spells that require a minimum rank of 3 and can improve the effectiveness of your existing spells.

        \cf{Sor}[3]{Greater Mage Armor}
        The defense bonus from the body armor created by your \textit{mage armor} ability increases to \plus3.
        In addition, its bonus to \glossterm{resistance} against \glossterm{energy damage} increases to be equal to three times your rank in this archetype, and its bonus to resistance against \glossterm{physical damage} increases to be equal to your rank in this archetype.

        \cf*{Sor}[4]{Spell Rank} You become a rank 4 arcane spellcaster.
        This gives you access to spells that require a minimum rank of 4 and can improve the effectiveness of your existing spells.

        \cf*{Sor}[4]{Spell Knowledge} You learn an additional arcane \glossterm{spell} from a \glossterm{mystic sphere} you have access to.

        \cf*{Sor}[5]{Spell Rank} You become a rank 5 arcane spellcaster.
        This gives you access to spells that require a minimum rank of 5 and can improve the effectiveness of your existing spells.

        \cf*{Sor}[6]{Spell Rank} You become a rank 6 arcane spellcaster.
        This gives you access to spells that require a minimum rank of 6 and can improve the effectiveness of your existing spells.

        \cf{Sor}[6]{Supreme Mage Armor}
        The defense bonus from the body armor created by your \textit{mage armor} ability increases to \plus4.
        In addition, its bonus to \glossterm{resistance} against \glossterm{energy damage} increases to be equal to four times your rank in this archetype, and its bonus to resistances against \glossterm{physical damage} increases to be equal to twice your rank in this archetype.

        \cf*{Sor}[7]{Spell Rank} You become a rank 7 arcane spellcaster.
        This gives you access to spells that require a minimum rank of 7 and can improve the effectiveness of your existing spells.

        \cf*{Sor}[7]{Spell Knowledge} You learn an additional arcane \glossterm{spell} from a \glossterm{mystic sphere} you have access to.

    \newpage
    \subsection{Arcane Spell Mastery}
        This archetype improves the arcane spells you cast.
        You must have the Arcane Magic archetype from the sorcerer class to gain the abilities from this archetype.
        All abilities from this archetype are \glossterm{magical}.

        \cf{Sor}[0]{Combat Caster} You reduce your \glossterm{focus penalties} by 2.

        \cf{Sor}[1]{Mystic Insight}
        You gain your choice of one of the following abilities.
        Whenever you increase your rank in this archetype, you may change your choice.
        This can allow you to apply the benefits of insights like \textit{signature spell} to higher rank spells.
        {
            \parhead{Focused Caster} You reduce your \glossterm{focus penalty} by 1.
                You cannot choose this ability multiple times.
            \parhead{Insight Point} You gain an additional \glossterm{insight point}.
                You can choose this ability multiple times, gaining an additional insight point each time.
            \parhead{Signature Spell} Choose an arcane \glossterm{spell} you know.
                The spell loses the \abilitytag{Focus} tag, allowing you to cast it without lowering your guard in combat.
                In adition, you gain a \plus1 bonus to \glossterm{accuracy} with that spell.
                You can choose this ability multiple times, choosing a different spell each time.
            \parhead{Spell Power} Choose an arcane \glossterm{spell} you know.
                You gain a bonus equal to your rank in this archetype to your \glossterm{power} with that spell.
                You can choose this ability multiple times, choosing a different spell each time.
        }

        \cf{Sor}[2]{Arcane Attunement} You gain an additional \glossterm{attunement point}.

        \cf{Sor}[3]{Wellspring of Power}
        You gain a \plus2 bonus to your \glossterm{magical} \glossterm{power}.

        \cf*{Sor}[4]{Mystic Insight}
        You gain an additional \textit{mystic insight} ability.

        \cf{Sor}[5]{Greater Arcane Attunement} The number of attunement points you gain from your \textit{arcane attunement} ability increases to two.

        \cf{Sor}[6]{Greater Wellspring of Power}
        The bonus from your \textit{wellspring of power} ability increases to \plus6.

        \cf*{Sor}[7]{Mystic Insight}
        You gain an additional \textit{mystic insight} ability.

    \newpage
    \subsection{Draconic Magic}
        Not all sorcerers know the reason for their innate connection to magic.
        Some discover that they have draconic blood in their veins, and some of those sorcerers learn how to tap into their heritage.
        This archetype deepens your magical connection to your draconic ancestor and enhances your spellcasting.
        You must have the Arcane Magic archetype from the sorcerer class to gain the abilities from this archetype.

        \cf{Sor}[0]{Draconic Bloodline} Choose a type of dragon from among the dragons on \trefnp{Draconic Bloodline Types}.
        You have the blood of that type of dragon in your veins.
        You gain a bonus equal to twice your rank in this archetype to your \glossterm{resistance} to the damage type dealt by that dragon's breath weapon.

        \begin{dtable}
            \lcaption{Draconic Bloodline Types}
            \begin{dtabularx}{\columnwidth}{l >{\lcol}X >{\lcol}X}
                \tb{Dragon} & \tb{Damage Type} & \tb{Mystic Sphere} \tableheaderrule
                Black       & Acid             & Vivimancy    \\
                Blue        & Electricity      & Electromancy \\
                Brass       & Fire             & Enchantment  \\
                Bronze      & Electricity      & Revelation   \\
                Copper      & Acid             & Terramancy   \\
                Gold        & Fire             & Photomancy   \\
                Green       & Acid             & Compulsion   \\
                Red         & Fire             & Pyromancy    \\
                Silver      & Cold             & Telekinesis  \\
                White       & Cold             & Cryomancy    \\
            \end{dtabularx}
        \end{dtable}

        \cf{Sor}[0]{Draconic Focus} You reduce your \glossterm{focus penalties} by 1.

        \cf{Sor}[1]{Draconic Spells} If you already have access to your dragon's mystic sphere, you learn two spells from that sphere.
        Otherwise, you gain access to that mystic sphere, including all \glossterm{cantrips} from that sphere.

        \cf{Sor}[2]{Greater Draconic Focus} The focus penalty reduction from your \textit{draconic focus} ability increases to 2.

        \cf{Sor}[2]{Greater Draconic Bloodline} The bonus to \glossterm{resistance} from your \textit{draconic bloodline} ability increases to be equal to four times your rank in this archetype.

        \cf{Sor}[3]{Draconic Hide} You gain a bonus equal to your rank in this archetype to your \glossterm{resistances} against both \glossterm{physical damage} and \glossterm{energy damage} (see \pcref{Resistances}).

        \cf{Sor}[4]{Draconic Precision} You gain a \plus1 bonus to \glossterm{accuracy} with any spell that either deals damage of your dragon's damage type or is from your dragon's \glossterm{mystic sphere}.

        \cf{Sor}[5]{Draconic Spell} You learn an additional arcane spell from your dragon's mystic sphere.

        \cf{Sor}[5]{Energy Immunity} You become immune to your dragon's damage type.

        \cf{Sor}[6]{Greater Draconic Hide} The resistance bonuses from your \textit{draconic hide} ability both increase to be equal to twice your rank in this archetype.

        \cf{Sor}[7]{Greater Draconic Precision} The bonus from your \textit{draconic precision} ability increases to \plus2.

    \newpage
    \subsection{Innate Arcanist}
        This archetype deepens your innate connection to arcane magic and improves your ability to defeat other spellcasters.
        You must have the Arcane Magic archetype from the sorcerer class to gain the abilities from this archetype.

        \cf{Sor}[0]{Innate Magic} None of your arcane spells have \glossterm{somatic components} or \glossterm{verbal components}.

        \cf{Sor}[1]{Mystic Tolerance} You gain a bonus equal to twice your rank in this archetype to your \glossterm{resistance} against \glossterm{energy damage}.

        \cf{Sor}[2]{Personal Enhancement} You gain a \plus1 \glossterm{magic bonus} to \glossterm{accuracy} and \glossterm{defenses}.
        Because this is a magic bonus, it does not stack with other magic bonuses (see \pcref{Stacking Rules}).

        \cf{Sor}[3]{Spell Absorption} Whenever another creature uses a spell to attack you, you can choose to absorb its energy.
        This does not reduce the spell's effect on you, but it grants you the ability to cast the spell once.
        When you cast the spell, you use your own \glossterm{accuracy}, \glossterm{power}, and abilities to determine the effects of the spell.
        Once you cast the spell, you expend the absorbed energy, and you cannot cast it again.

        Whenever you are attacked by a new spell, if you already have the ability to cast a spell with this ability, you choose which spell you gain the ability to cast.
        When you take a \glossterm{long rest}, you lose the ability to cast any spells you have stored with this ability.

        \cf{Sor}[4]{Greater Mystic Tolerance} The bonus from your \textit{mystic tolerance} ability increases to be equal to three times your rank in this archetype.

        \cf{Sor}[4]{Greater Personal Enhancement} The bonuses from your \textit{personal enhancement} ability increase to \plus2.

        \cf{Sor}[5]{Greater Spell Absorption} You can store the ability to cast up to three spells with your \textit{spell absorption} ability.
        In addition, you can cast each stored spell twice before you lose the ability to cast it instead of only once.

        \cf{Sor}[6]{Supreme Personal Enhancement} The bonuses from your \textit{personal enhancement} ability increase to \plus3.

        \cf{Sor}[7]{Supreme Mystic Tolerance} The bonus from your \textit{mystic tolerance} ability increases to be equal to four times your rank in this archetype.

        \cf{Sor}[7]{Supreme Spell Absorption} Whenever you absorb a spell with your \textit{spell absorption} ability, you take half damage from the spell.

    \newpage
    \subsection{Wild Magic}
        This archetype makes the magic you cast more chaotic, generally increasing its power at the cost of your control over your magic.
        You must have the Arcane Magic archetype from the sorcerer class to gain the abilities from this archetype.

        \cf{Sor}[0]{Chaotic Exertion} You gain a \plus2 bonus to the roll when you use the \textit{desperate exertion} ability.
        This bonus stacks with the normal \plus2 bonus provided by that ability.

        \cf{Sor}[1]{Wildspell} Whenever you cast a spell that does not have the \abilitytag{Attune} or \abilitytag{Sustain} tags, you may use this ability after making all other decisions for the spell (such as targets, intended area, and so on).
        When you do, you reduce your \glossterm{focus penalties} for casting the spell by 2, and you gain a bonus equal to your rank in this archetype to \glossterm{power} with the spell.
        In addition, roll 1d10 and apply the corresponding wild magic effect from \trefnp{Wild Magic Effects}.
        Some wild magic effects cannot be meaningfully applied to all spells.
        For example, changing the damage dealt by a spell does not affect spells that do not deal damage.
        Any wildspell effects that do not make sense for a particular spell should be ignored.

        \begin{dtable}
            \lcaption{Wild Magic Effects}
            \begin{dtabularx}{\textwidth}{l X}
                \tb{Roll} & \tb{Effect} \tableheaderrule
                1 & The spell fails with no effect \\
                2 & When you attack with the spell this round, you roll twice and take the lower result \\
                3 & When you attack with the spell this round, you are a target of the attack in addition to any other targets \\
                4 & The spell's area is halved this round \\
                5 & The spell's area is doubled this round \\
                6 & Each target that resists damage from the spell this round is also \dazed until the end of the next round \\
                7 & Each target that loses hit points from the spell this round is also \sickened until the end of the next round \\
                8 & When you deal damage with the spell this round, you roll twice for the spell and take the higher result \\
                9 & When you attack with the spell this round, you roll twice and take the higher result \\
                10 or higher & During the \glossterm{action phase} of the next round, the spell takes effect again with the same choices for all decisions, such as targets \\
            \end{dtabularx}
        \end{dtable}

        \cf{Sor}[2]{Chaotic Insight} You learn a spell that does not have the \abilitytag{Attune} or \abilitytag{Sustain} tags from any \glossterm{mystic sphere}, even if you do not have access to that mystic sphere.
        The spell does not have to be from a mystic sphere on the arcane mystic sphere list.
        As normal, you can change which spell you learn with this ability as you gain access to new spell ranks.

        \cf{Sor}[3]{Greater Chaotic Exertion} Once per \glossterm{long rest}, you can use the \textit{desperate exertion} ability without increasing your \glossterm{fatigue level}.

        \cf{Sor}[4]{Greater Wildspell} The bonus to \glossterm{power} from your \textit{wildspell} ability increases to be equal to twice your rank in this archetype.

        \cf{Sor}[5]{Greater Chaotic Insight} You learn an additional spell with your \textit{chaotic insight} ability.

        \cf{Sor}[6]{Supreme Chaotic Exertion} You can use your \textit{chaotic exertion} ability once per \glossterm{short rest} instead of once per long rest.

        \cf{Sor}[7]{Supreme Wildspell} You replace your normal wild magic effects from your \textit{wildspell} ability with the effects from the table below.
        \begin{dtable}
            \lcaption{Epic Wild Magic Effects}
            \begin{dtabularx}{\textwidth}{l X}
                \tb{Roll} & \tb{Effect} \tableheaderrule
                1 or lower & The spell has its normal effect \\
                2 & All damage dealt by the spell is considered to be all damage types \\
                3 & When you attack with the spell this round, you roll twice and take the higher result \\
                4 & When you deal damage with the spell this round, you roll twice for the spell and take the higher result \\
                5 & Any \glossterm{conditions} inflicted by the spell this round become a \abilitytag{Curse} instead of a condition, and are removed when the target takes a \glossterm{short rest} \\
                6 & When the spell would cause a creature to lose hit points this round, that creature loses twice as many hit points \\
                7 & The spell's area is tripled this round \\
                8 & Each target that loses hit points from the spell this round is also \stunned until the end of the next round \\
                9 & You gain a \plus4 bonus to \glossterm{accuracy} with the spell this round, but you take the minimum possible result when the spell would deal damage \\
                10 or higher & During both the \glossterm{action phase} and \glossterm{delayed action phase} of the next round, the spell takes effect again with the same choices for all decisions, such as targets \\
            \end{dtabularx}
        \end{dtable}

\newpage
\section{Warlock}\label{Warlock}
    \begin{dtable!*}
        \lcaption{Warlock Progression}
\begin{dtabularx}{\textwidth}{l l >{\lcol}X >{\lcol}X >{\lcol}X >{\lcol}X}
    \tb{Rank} & \tb{Min Level} & \tb{Blessings of the Abyss} & \tb{Pact Magic}                         & \tb{Pact Spell Mastery}       & \tb{Soulkeeper's Chosen} \tableheaderrule
    0         & \tdash         & Fiendish resistance         & Armor tolerance, cantrips               & Combat caster                 & Empowering whispers         \\
    1         & 1              & Abyssal blast               & Spellcasting                            & Mystic insight                & Possession                  \\
    2         & 4              & Abyssal sphere              & Spell rank (2), spell knowledge         & Soulkeeper's guidance         & Exchange soul fragment      \\
    3         & 7              & Banish to the abyss         & Spell rank (3), greater armor tolerance & Wellspring of power           & Greater empowering whispers \\
    4         & 10             & Hellfire                    & Spell rank (4), spell knowledge         & Mystic insight                & Greater possession          \\
    5         & 13             & Greater fiendish resistance & Spell rank (5)                          & Greater soulkeeper's guidance & Exchange vitality           \\
    6         & 16             & Abyssal curse               & Spell rank (6), supreme armor tolerance & Greater wellspring of power   & Supreme empowering whispers \\
    7         & 19             & Brimstone                   & Spell rank (7), spell knowledge         & Mystic insight                & Supreme possession          \\
\end{dtabularx}
    \end{dtable!*}

    \classbasics{Alignment} Any.

    \classbasics{Archetypes} Warlocks have the Blessings of the Abyss, Pact Magic, Pact Spell Mastery, and Soulkeeper's Chosen \glossterm{archetypes}.

    \subsection{Basic Class Abilities}
        If you are a warlock, you gain the following abilities.
        % 6 insight, 4 skill, 8 AP, 6 fatigue, 1 armor = 25 points

        \cf{War}{Defenses}
        You gain the following bonuses to your \glossterm{defenses}: \plus1 Armor, \plus5 Fortitude, \plus3 Reflex, \plus7 Mental.

        \cf{War}{Resources} You have the following \glossterm{resources}:
        \begin{itemize}
            \item Three \glossterm{insight points}, which you can spend to gain additional abilities or proficiencies (see \pcref{Insight Points}).
            \item Six \glossterm{skill points}, which you can spend to learn skills (see \pcref{Skills}).
            \item Two \glossterm{attunement points}, which you can use to attune to items and abilities that affect you (see \pcref{Attunement Points}).
            \item A \plus3 bonus to your \glossterm{fatigue tolerance}, which makes it easier for you to use powerful abilities that fatigue you (see \pcref{Fatigue}).
        \end{itemize}

        \cf{War}{Weapon Proficiencies}
        You are proficient with simple weapons.

        \cf{War}{Armor Proficiencies}
        You are proficient with light armor.
        Encumbrance from armor interferes with the gestures you make to cast spells, which can cause your spells with \glossterm{somatic components} to fail (see \pcref{Somatic Component Failure}).

        \cf{War}{Skills}
        You have the following \glossterm{class skills}:
        \begin{itemize}
            \item \subparhead{Dexterity} Ride.
            \item \subparhead{Intelligence} Craft, Deduction, Disguise, Knowledge (arcana, planes, religion), Linguistics.
            \item \subparhead{Perception} Awareness, Social Insight, Spellsense.
            \item \subparhead{Other} Deception, Intimidate, Persuasion, Profession.
        \end{itemize}

    \subsection{Special Class Abilities}

        \cf{War}{Soul Pact}
        To become a warlock, you must make a pact with a creature capable of sharing its power with you.
        Generally, such a creature must be at least 20th level, and must be a planeforged from a plane other than your own.
        You must make a sacrifice, the details of which are subject to negotiation, and offer a part of your immortal soul.
        In exchange, you gain the powers of a warlock.
        The creature you make the pact with is called your soulkeeper.
        Most warlocks make pacts with demons or devils, though other soulkeepers are possible.

        Offering your soul to an entity in this way grants it the ability to communicate with you in limited ways.
        This communication typically manifests as unnatural emotional urges or whispered voices audible only to you.
        demon or devil.

        Your pact specifies how much of your soul is granted to your soulkeeper, and the circumstances of the transfer.
        The most common arrangement is for a soulkeeper to gain possession of your soul immediately after you die.
        It will keep the soul for one decade per year of your life that you spend as a warlock.
        During that time, it will not prevent you from being resurrected.
        At the end of that time, if your soul remains intact, your soul will pass on to its intended afterlife.
        However, other arrangements are possible, and each warlock's pact can be unique.

        The longer you spend in an afterlife that is not your own, the more likely you are to lose your sense of self and become subsumed by the plane you are on.
        Only a soul of extraordinary strength can maintain its integrity after decades or centuries in any plane.
        Many warlocks seek power zealously while mortal to gain the mental fortitude necessary to keep their soul intact after death.

        \cf{War}{Whispers of the Lost}[Magical]
        You hear the voices of souls that inhabit your soulkeeper's plane, linked to you through your soulkeeper.
        Choose one of the following types of whispers that you hear.
        {
            \subcf{Mentoring Whispers} You hear the voice of a dead warlock whose soul is bound to the same soulkeeper as yours.

            \subcf{Spiteful Whispers} You hear the voices of cruel souls who berate you for your flaws and mistakes.

            \subcf{Sycophantic Whispers} You hear the voices of adoring souls who praise your talents and everything you do.

            \subcf{Warning Whispers} You hear the voices of paranoid and fearful souls warning you of danger, both real and imagined.

            \subcf{Whispers of the Mighty} Your soulkeeper forges the connection to your soul into a boon granted to any soul in the Abyss strong enough to claim it in battle.
            You hear the voice of whatever soul currently possesses the boon, which may change suddenly and unexpectedly.
        }

    \newpage
    \subsection{Blessings of the Abyss}
        You can only choose this archetype if your soulkeeper is a demon or devil.
        This archetype enhances your connection to the Abyss and allows you to channel its sinister power more directly.
        All abilities from this archetype are \glossterm{magical}.

        \cf{War}[0]{Fiendish Resistance} You gain a bonus equal to twice your rank in this archetype to your \glossterm{resistance} against \glossterm{energy damage}, to a minimum bonus of 1.

        \cf{War}[1]{Abyssal Blast} You can use the \textit{abyssal blast} ability as a standard action.
        \begin{instantability}{Abyssal Blast}
            Instant
            \rankline
            Make an attack vs. Armor against one creature or object within \rngmed range.
            \hit The target takes fire damage equal to 1d10 plus your \glossterm{power}.

            \rankline
            \rank{2} The damage increases to 2d6.
            \rank{3} The damage increases to 2d10.
                In addition, if you miss by 2 or less, the target takes half damage.
                This is called a \glossterm{glancing blow}.
            \rank{4} The damage increases to 4d6.
            \rank{5} The damage increases to 4d10.
            \rank{6} The damage increases to 5d10.
            \rank{7} The damage increases to 7d10.
        \end{instantability}

        \cf{War}[2]{Abyssal Sphere} Choose one of the following \glossterm{mystic spheres}: \sphere{astromancy}, \sphere{enchantment}, \sphere{pyromancy}, or \sphere{summoning}.
        If you already have access to that mystic sphere, you learn two spells from that sphere.
        Otherwise, you gain access to that mystic sphere, including all \glossterm{cantrips} from that sphere.

        \cf{War}[3]{Banish to the Abyss} You can use the \textit{banish to the abyss} ability as a standard action.
        \begin{durationability}{Banish to the Abyss}
            Duration
            \rankline
            Make an attack vs. Mental against one creature within \rngmed range.
            \hit The subject takes 2d6 \add half \glossterm{power} fire damage.
            If it loses \glossterm{hit points} from this damage, it is briefly teleported into the Abyss.
            At the end of the next round, it teleports back to its original location, or into the closest open space if that location is occupied.
            After this effect ends, it becomes immune to this effect until it takes a \glossterm{short rest}.
            \glance Half damage.

            \rankline
            The damage increases by \plus1d for each rank beyond 3.
            \rank{5} You gain a \plus1 bonus to \glossterm{accuracy} with the attack.
            \rank{7} The accuracy bonus increases to \plus2.
        \end{durationability}

        \cf{War}[4]{Hellfire} You gain a \plus2 bonus to your \glossterm{magical} \glossterm{power}.
        In addition, whenever you use an ability that deals fire damage, you can change the type of the damage to be energy damage in place of fire damage.
        Any other aspects of the ability, including damage types other than fire, remain unchanged.

        \cf{War}[5]{Greater Fiendish Resistance}[Magical] The bonus from your \textit{fiendish resistance} ability increases to be equal to three times your rank in this archetype.
        In addition, you gain a bonus equal to your rank in this archetype to your \glossterm{resistance} against physical damage.

        \cf{War}[6]{Abyssal Curse} You can use the \textit{abyssal curse} ability as a standard action.
        \begin{durationability}{Abyssal Curse}
            \spelltwocol{Duration}{\abilitytag{Curse}}
            \rankline
            Make an attack vs. Fortitude against one creature or object within \rngmed range.
            \hit The target is \nauseated until it takes a \glossterm{short rest}.
            \glance The target is nauseated until the end of the next round.
            \crit The target is nauseated until this curse is removed.

            \rankline
            You gain a \plus1 bonus to \glossterm{accuracy} with the attack for each rank beyond 6.
        \end{durationability}

        \cf{War}[7]{Brimstone} The power bonus from your \textit{hellfire} ability increases to \plus4.
        In addition, whenever you cause a creature to lose \glossterm{hit points} with fire damage or energy damage, it becomes \sickened until the end of the next round.
        That creature then becomes immune to this effect until it takes a \glossterm{short rest}.

    \newpage
    \subsection{Pact Magic}
        This archetype grants you the ability to cast pact spells.
        You must have a base Willpower of at least 1 to gain this archetype.
        All abilities from this archetype are \glossterm{magical}.

        \cf{War}[0]{Cantrips}
        Your deity grants you the ability to use pact magic.
        You gain access to one pact \glossterm{mystic sphere} (see \pcref{Pact Mystic Spheres}).
        You may spend \glossterm{insight points} to gain access to one additional pact \glossterm{mystic sphere} per two \glossterm{insight points}.
        You automatically learn all \glossterm{cantrips} from any mystic sphere you have access to.
        You do not yet gain access to any other spells from those mystic spheres.

        Pact spells require both \glossterm{verbal components} and \glossterm{somatic components} to cast (see \pcref{Casting Components}).
        For details about mystic spheres and casting spells, see \pcref{Spell and Ritual Mechanics}.

        \cf{War}[0]{Armor Tolerance} You reduce your \glossterm{encumbrance} by 2 when determining your \glossterm{somatic component failure}.

        \cf{War}[1]{Spellcasting}
        You become a rank 1 pact spellcaster.
        You learn two rank 1 \glossterm{spells} from pact \glossterm{mystic spheres} you have access to.
        You can also spend \glossterm{insight points} to learn one additional rank 1 spell per \glossterm{insight point}.
        Unless otherwise noted in a spell's description, casting a spell requires a \glossterm{standard action}.

        When you gain access to a new \glossterm{mystic sphere} or spell \glossterm{rank},
            you can forget any number of spells you know to learn that many new spells in exchange,
            including spells of the higher rank.
        All of those spells must be from pact mystic spheres you have access to.

        \cf{War}[2]{Spell Rank} You become a rank 2 pact spellcaster.
        This gives you access to spells that require a minimum rank of 2.

        \cf{War}[2]{Spell Knowledge} You learn an additional pact \glossterm{spell} from a \glossterm{mystic sphere} you have access to.

        \cf*{War}[3]{Spell Rank} You become a rank 3 pact spellcaster.
        This gives you access to spells that require a minimum rank of 3 and can improve the effectiveness of your existing spells.

        \cf{War}[3]{Greater Armor Tolerance} The penalty reduction from your \textit{armor tolerance} ability increases to 3.

        \cf*{War}[4]{Spell Rank} You become a rank 4 pact spellcaster.
        This gives you access to spells that require a minimum rank of 4 and can improve the effectiveness of your existing spells.

        \cf*{War}[4]{Spell Knowledge} You learn an additional pact \glossterm{spell} from a \glossterm{mystic sphere} you have access to.

        \cf*{War}[5]{Spell Rank} You become a rank 5 pact spellcaster.
        This gives you access to spells that require a minimum rank of 5 and can improve the effectiveness of your existing spells.

        \cf*{War}[6]{Spell Rank} You become a rank 6 pact spellcaster.
        This gives you access to spells that require a minimum rank of 6 and can improve the effectiveness of your existing spells.

        \cf{War}[6]{Supreme Armor Tolerance} The penalty reduction from your \textit{armor tolerance} ability increases to 4.

        \cf*{War}[7]{Spell Rank} You become a rank 7 pact spellcaster.
        This gives you access to spells that require a minimum rank of 7 and can improve the effectiveness of your existing spells.

        \cf*{War}[7]{Spell Knowledge} You learn an additional pact \glossterm{spell} from a \glossterm{mystic sphere} you have access to.

    \newpage
    \subsection{Pact Spell Mastery}
        This archetype improves your ability to cast spells with the power of your dark pact.
        You must have the Pact Magic archetype to gain the abilities from this archetype.
        All abilities from this archetype are \glossterm{magical}.

        \cf{War}[0]{Combat Caster} You reduce your \glossterm{focus penalties} by 2.

        \cf{War}[1]{Mystic Insight}
        You gain your choice of one of the following abilities.
        Whenever you increase your rank in this archetype, you may change your choice.
        This can allow you to apply the benefits of insights like \textit{signature spell} to higher rank spells.
        {
            \parhead{Focused Caster} You reduce your \glossterm{focus penalty} by 1.
                You cannot choose this ability multiple times.
            \parhead{Insight Point} You gain an additional \glossterm{insight point}.
                You can choose this ability multiple times, gaining an additional insight point each time.
            \parhead{Rituals} You gain the ability to perform pact rituals to create unique magical effects (see \pcref{Rituals}).
                The maximum \glossterm{rank} of pact ritual you can learn or perform is equal to the maximum rank of pact spell that you can cast.
                You cannot choose this ability multiple times.
            \parhead{Signature Spell} Choose a pact \glossterm{spell} you know.
                The spell loses the \abilitytag{Focus} tag, allowing you to cast it without lowering your guard in combat.
                In adition, you gain a \plus1 bonus to \glossterm{accuracy} with that spell.
                You can choose this ability multiple times, choosing a different spell each time.
            \parhead{Spell Power} Choose an arcane \glossterm{spell} you know.
                You gain a bonus equal to your rank in this archetype to your \glossterm{power} with that spell.
                You can choose this ability multiple times, choosing a different spell each time.
        }

        \cf{War}[2]{Soulkeeper's Guidance} Once per \glossterm{long rest}, you may use the \textit{desperate exertion} ability without increasing your \glossterm{fatigue level} to affect a pact spell you cast (see \pcref{Desperate Exertion}).

        \cf{War}[3]{Wellspring of Power}
        You gain a \plus2 bonus to your \glossterm{magical} \glossterm{power}.

        \cf*{War}[4]{Mystic Insight}
        You gain an additional \textit{mystic insight} ability.

        \cf{War}[5]{Greater Soulkeeper's Guidance} You can use your \textit{soulkeeper's guidance} ability once per \glossterm{short rest} instead of once per long rest.

        \cf{War}[6]{Greater Wellspring of Power}
        The bonus from your \textit{wellspring of power} ability increases to \plus6.

        \cf*{War}[7]{Mystic Insight}
        You gain an additional \textit{mystic insight} ability.

    \newpage
    \subsection{Soulkeeper's Chosen}
        This archetype enhances your connection to your soulkeeper, granting you abilities relating to your pact.
        All abilities from this archetype are \glossterm{magical}.

        \cf{War}[0]{Empowering Whispers}
        You gain an ability based on the type of whispers you hear with your \textit{whispers of the lost} ability.
        {
            \subcf{Mentoring Whispers} You gain two additional \glossterm{skill points}.

            \subcf{Spiteful Whispers} Whenever you miss a creature with an attack, you gain a \plus1 bonus to \glossterm{accuracy} against the creature you missed during the next round.

            \subcf{Sycophantic Whispers} You gain a \plus2 bonus to Mental defense.

            \subcf{Warning Whispers} You gain a \plus2 bonus to \glossterm{initiative} checks and Reflex defense.

            \subcf{Whispers of the Mighty} You gain a \plus2 bonus to Fortitude defense.
        }

        \cf{War}[1]{Possession} You can use the \textit{possession} ability as a \glossterm{free action} to allow your soulkeeper a greater influence over your actions.
        \begin{durationability}{Possession}
            \abilitytag{Sustain} (free)
            \rankline
            You gain the following benefits and drawbacks:
            \begin{itemize}
                \item You gain a \plus2 bonus to your \glossterm{magical} \glossterm{power}.
                \item You gain a \plus4 bonus to your \glossterm{fatigue tolerance}.
                    If you would be unconscious due to fatigue without this bonus, your soulkeeper directly controls all of your actions.
                    Your soulkeeper's objectives may differ from your own, but except in very unusual circumstances, your soulkeeper is invested in continuing your life and ensuring your victory in difficult circumstances.
                \item You take a \minus2 penalty to Fortitude and Mental defenses.
                \item You are unable to take \glossterm{standard actions} that do not cause you to make \glossterm{magical} attacks.
                \item At the end of each round, if you did not make a \glossterm{magical} attack that round, this ability ends.
                \item When this ability ends for any reason, you cannot use it again until after the end of the next round.
            \end{itemize}

            \rankline
            \rank{3} The power bonus increases to \plus4.
            \rank{5} The power bonus increases to \plus8.
            \rank{7} The power bonus increases to \plus16.
        \end{durationability}

        \cf{War}[2]{Exchange Soul Fragment} Your connection to your soulkeeper deepens, allowing you to send a fragment of your experiences through the link.
        You can use the \textit{exchange soul fragment} ability as a \glossterm{minor action}.
        \begin{instantability}{Exchange Soul Fragment}
            \spelltwocol{Instant}{\abilitytag{Swift}}
            \rankline
            When you use this ability, you increase your \glossterm{fatigue level} by one.

            Remove a \glossterm{condition} affecting you.
            This cannot remove a condition applied during the current round.
            Because this ability has the \abilitytag{Swift} tag, the penalties from the removed condition do not affect you during the current phase.
        \end{instantability}

        \cf{War}[3]{Greater Empowering Whispers} You gain an additional ability depending on the voices you chose with your \textit{whispers of the lost} ability.
        {
            \subcf{Mentoring Whispers} You gain an additional \glossterm{insight point}.

            \subcf{Spiteful Whispers} The bonus from your \textit{empowering whispers} ability increases to \plus2.

            \subcf{Sycophantic Whispers} The bonus from your \textit{empowering whispers} ability increases to \plus4.

            \subcf{Warning Whispers} The bonuses from your \textit{empowering whispers} ability increases to \plus4.

            \subcf{Whispers of the Mighty} The bonus from your \textit{empowering whispers} ability increases to \plus4.
        }

        \cf{War}[4]{Greater Possession} You gain a \plus1 bonus to \glossterm{accuracy} during your \textit{possession} ability.

        \cf{War}[5]{Exchange Vitality} Your connection to your soulkeeper deepens, allowing you to send a fragment of your vitality through the link.
        You can use the \textit{exchange vitality} ability as a \glossterm{minor action}.
        \begin{instantability}{Exchange Vitality}
            \spelltwocol{Instant}{\abilitytag{Swift}}
            \rankline
            When you use this ability, you increase your \glossterm{fatigue level} by two.

            Remove one of your \glossterm{vital wounds}.
            This cannot remove a vital wound applied during the current round.
            Because this ability has the \abilitytag{Swift} tag, the penalties from the removed vital wound do not affect you during the current phase.
        \end{instantability}

        \cf{War}[6]{Supreme Empowering Whispers} You gain an additional ability depending on the voices you chose with your \textit{whispers of the lost} ability.
        {
            \subcf{Mentoring Whispers} You gain an additional \glossterm{insight point}.

            \subcf{Spiteful Whispers} The bonus from your \textit{empowering whispers} ability increases to \plus3.

            \subcf{Sycophantic Whispers} You are immune to all \abilitytag{Emotion} attacks.

            \subcf{Warning Whispers} You are never \unaware or \partiallyunaware.

            \subcf{Whispers of the Mighty} You gain a bonus to your \glossterm{resistances} equal to your rank in this archetype.
        }

        \cf{War}[7]{Supreme Possession} The bonus from your \textit{greater possession} ability increases to \plus2.

\newpage
\section{Wizard}\label{Wizard}
    \begin{dtable!*}
        \lcaption{Wizard Progression}
\begin{dtabularx}{\textwidth}{l l >{\lcol}X >{\lcol}X >{\lcol}X >{\lcol}X}
    \tb{Rank} & \tb{Min Level} & \tb{Alchemist}              & \tb{Arcane Magic}                  & \tb{Arcane Scholar}     & \tb{Arcane Spell Mastery}   \tableheaderrule
    0 & \tdash & Alchemical infusion          & Cantrips, mage armor               & Deep knowledge       & Mystic sphere               \\
    1 & 1      & Portable workshop            & Spellcasting                       & Ritualist            & Mystic insight              \\
    2 & 4      & Alchemical discovery         & Spell rank (2), spell knowledge    & Scholastic insight   & Arcane attunement           \\
    3 & 7      & Experienced quaffing         & Spell rank (3), greater mage armor & Contingency          & Wellspring of power         \\
    4 & 10     & Alchemical discovery         & Spell rank (4), spell knowledge    & Scholastic insight   & Mystic insight              \\
    5 & 13     & Greater portable workshop    & Spell rank (5)                     & Greater ritualist    & Greater arcane attunement   \\
    6 & 16     & Alchemical discovery         & Spell rank (6), supreme mage armor & Scholastic insight   & Greater wellspring of power \\
    7 & 19     & Greater experienced quaffing & Spell rank (7), spell knowledge    & Multiple contingency & Mystic insight              \\
\end{dtabularx}
    \end{dtable!*}

    \classbasics{Alignment} Any.

    \classbasics{Archetypes} Wizards have the Alchemist, Arcane Magic, Arcane Scholar, and Arcane Spell Mastery \glossterm{archetypes}.

    \subsection{Basic Class Abilities}
        If you are a wizard, you gain the following abilities.
        % -1 AD, 8 insight, 4 skill, 12 AP, 2 fatigue = 25 points

        \cf{Wiz}{Defenses}
        You gain the following bonuses to your \glossterm{defenses}: \plus0 Armor, \plus3 Fortitude, \plus5 Reflex, \plus8 Mental.

        \cf{Wiz}{Resources} You have the following \glossterm{resources}:
        \begin{itemize}
            \item Four \glossterm{insight points}, which you can spend to gain additional abilities or proficiencies (see \pcref{Insight Points}).
            \item Six \glossterm{skill points}, which you can spend to learn skills (see \pcref{Skills}).
            \item Three \glossterm{attunement points}, which you can use to attune to items and abilities that affect you (see \pcref{Attunement Points}).
            \item A \plus1 bonus to your \glossterm{fatigue tolerance}, which makes it easier for you to use powerful abilities that fatigue you (see \pcref{Fatigue}).
        \end{itemize}

        \cf{Wiz}{Weapon Proficiencies} 
        You are proficient with simple weapons.

        \cf{Wiz}{Armor Proficiencies}
        You are not proficient with any type of armor.
        Encumbrance from armor interferes with the gestures you make to cast spells, which can cause your spells with \glossterm{somatic components} to fail (see \pcref{Somatic Component Failure}).

        \cf{Wiz}{Skills}
        You have the following \glossterm{class skills}:
        \begin{itemize}
            \item \subparhead{Intelligence} Craft, Deduction, Knowledge (all kinds, taken individually), Linguistics.
            \item \subparhead{Perception} Awareness, Spellsense.
            \item \subparhead{Other} Deception, Intimidate, Persuasion, Profession.
        \end{itemize}

    \newpage
    \subsection{Alchemist}
        This archetype improves your ability to use alchemy to create unusual concoctions to aid your allies and harm your foes.

        \cf{Wiz}[0]{Alchemical Infusion}[Magical] Whenever you create or use an alchemical item, you may use your \glossterm{magical} \glossterm{power} in place of the item's normal power to determine its effects.
        % TODO: what if alchemical items deal damage over time?
        In addition, you may add half your magical power to the damage dealt by any alchemical items you use.

        \cf{Wiz}[1]{Portable Workshop}[Magical]
        You carry materials necessary to refine low-grade alchemical items wherever you are.
        Where you lack material components, you fill in with some of your own magic, allowing you to create items more easily.
        The items are just as effective when used as items created normally.
        However, they are less durable, since they are partially sustained by your magic.
        Items created with this ability deteriorate and become useless after 24 hours or after you finish a long rest, whichever comes first.

        You can use this ability to create alchemical items with a item level up to your level (see \pcref{Item Levels}).
        Creating an item in this way functions in the same way as crafting alchemical items normally, with the following changes.
        First, you do not require any raw materials.
        Second, the maximum duration of any item created with this ability is five minutes.
        Third, you can only maintain the existence of three items with this ability at once.
        If you try to create a fourth item, you must stop maintaining the existence of another item created.
        You can do this as a \glossterm{free action} regardless of distance.
        This removes any lingering effects from the removed item, such as the protective qualities of an \textit{antitoxin elixir}.

        \cf{Wiz}[2]{Alchemical Discovery} You learn how to create alchemical items more effectively.
        You gain your choice of one of the following benefits.
        Each benefit can only be chosen once.
        {
            \parhead{Aerodynamic Construction} The range of thrown alchemical items you create increases by 30 feet.
                This does not affect alchemical items that are not designed to be thrown.
            \parhead{Complex Construction} You can use your portable workshop ability to create items with an item level up to two levels higher than your level.
            \parhead{Efficient Crafting} It takes you half the normal time and material components to craft alchemical items without using your \textit{portable workshop} ability.
            \parhead{Enduring Construction} The duration of any alchemical item you create is doubled.
                % TODO: wording
                % In addition, alchemical items that last for a fixed number of uses have their number of uses doubled.
            \parhead{Explosive Construction} The area affected by any alchemical item you create is doubled.
            \parhead{Potent Construction} The damage dealt by any alchemical item you create gains a bonus equal to your rank in this archetype.
            \parhead{Repetitive Construction} Whenever you use your \textit{portable workshop} ability, you can create two copies of the same alchemical item.
            This only counts as one item for the purpose of determining the number of items you can maintain with that ability.
        }

        \cf{Wiz}[3]{Experienced Quaffing} You can drink up to two doses of potions, elixirs, and other drinkable alchemical items as part of the same standard action.

        \cf*{Wiz}[4]{Alchemical Discovery} You gain an additional \textit{alchemical discovery} ability.

        \cf{Wiz}[5]{Greater Portable Workshop}[Magical] The number of items you can maintain with your \textit{portable workshop} ability increases to 5.

        \cf*{Wiz}[6]{Alchemical Discovery} You gain an additional \textit{alchemical discovery} ability.

        \cf{Wiz}[7]{Greater Experienced Quaffing} You can drink a single dose of a potion, elixir, or other drinkable alchemical item as a \glossterm{minor action}.

    \newpage
    \subsection{Arcane Magic}
        This archetype grants you the ability to cast arcane spells.
        All abilities from this archetype are \glossterm{magical}.

        \cf{Wiz}[0]{Cantrips}
        Your deity grants you the ability to use arcane magic.
        You gain access to one arcane \glossterm{mystic sphere} (see \pcref{Arcane Mystic Spheres}).
        You may spend \glossterm{insight points} to gain access to one additional arcane \glossterm{mystic sphere} per two \glossterm{insight points}.
        You automatically learn all \glossterm{cantrips} from any mystic sphere you have access to.
        You do not yet gain access to any other spells from those mystic spheres.

        Arcane spells require both \glossterm{verbal components} and \glossterm{somatic components} to cast (see \pcref{Casting Components}).
        For details about mystic spheres and casting spells, see \pcref{Spell and Ritual Mechanics}.

        \cf{Wiz}[0]{Mage Armor} You can use the \textit{mage armor} ability as a standard action.
        \begin{durationability}{Mage Armor}
            Duration
            \rankline
            You create a translucent suit of magical armor on your body and over your hands.
            This functions like body armor that provides a \plus2 bonus to Armor defense and has no \glossterm{encumbrance}.
            It also provides a \plus4 bonus to \glossterm{resistance} against \glossterm{energy damage}, and a \plus2 bonus to \glossterm{resistance} against \glossterm{physical damage}.

            As long as you have a free hand, the barrier also manifests as a shield that provides a \plus1 bonus to Armor defense.
            This bonus is considered to come from a shield, and does not stack with the benefits of using a physical shield.

            This ability lasts until you use it again or until you \glossterm{dismiss} it as a free action.
            In addition, it is automatically dismissed if you wear body armor.
        \end{durationability}

        \cf{Wiz}[1]{Spellcasting}
        You become a rank 1 arcane spellcaster.
        You learn two rank 1 \glossterm{spells} from arcane \glossterm{mystic spheres} you have access to.
        You can also spend \glossterm{insight points} to learn one additional rank 1 spell per \glossterm{insight point}.
        Unless otherwise noted in a spell's description, casting a spell requires a \glossterm{standard action}.

        When you gain access to a new \glossterm{mystic sphere} or spell \glossterm{rank},
            you can forget any number of spells you know to learn that many new spells in exchange,
            including spells of the higher rank.
        All of those spells must be from arcane mystic spheres you have access to.

        \cf{Wiz}[2]{Spell Rank} You become a rank 2 arcane spellcaster.
        This gives you access to spells that require a minimum rank of 2.

        \cf{Wiz}[2]{Spell Knowledge} You learn an additional arcane \glossterm{spell} from a \glossterm{mystic sphere} you have access to.

        \cf*{Wiz}[3]{Spell Rank} You become a rank 3 arcane spellcaster.
        This gives you access to spells that require a minimum rank of 3 and can improve the effectiveness of your existing spells.

        \cf{Wiz}[3]{Greater Mage Armor}
        The defense bonus from the body armor created by your \textit{mage armor} ability increases to \plus3.
        In addition, its bonus to \glossterm{resistance} against \glossterm{energy damage} increases to be equal to three times your rank in this archetype, and its bonus to resistance against \glossterm{physical damage} increases to be equal to your rank in this archetype.

        \cf*{Wiz}[4]{Spell Rank} You become a rank 4 arcane spellcaster.
        This gives you access to spells that require a minimum rank of 4 and can improve the effectiveness of your existing spells.

        \cf*{Wiz}[4]{Spell Knowledge} You learn an additional arcane \glossterm{spell} from a \glossterm{mystic sphere} you have access to.

        \cf*{Wiz}[5]{Spell Rank} You become a rank 5 arcane spellcaster.
        This gives you access to spells that require a minimum rank of 5 and can improve the effectiveness of your existing spells.

        \cf*{Wiz}[6]{Spell Rank} You become a rank 6 arcane spellcaster.
        This gives you access to spells that require a minimum rank of 6 and can improve the effectiveness of your existing spells.

        \cf{Wiz}[6]{Supreme Mage Armor}
        The defense bonus from the body armor created by your \textit{mage armor} ability increases to \plus4.
        In addition, its bonus to \glossterm{resistance} against \glossterm{energy damage} increases to be equal to four times your rank in this archetype, and its bonus to resistances against \glossterm{physical damage} increases to be equal to twice your rank in this archetype.

        \cf*{Wiz}[7]{Spell Rank} You become a rank 7 arcane spellcaster.
        This gives you access to spells that require a minimum rank of 7 and can improve the effectiveness of your existing spells.

        \cf*{Wiz}[7]{Spell Knowledge} You learn an additional arcane \glossterm{spell} from a \glossterm{mystic sphere} you have access to.

    \newpage
    \subsection{Arcane Scholar}
        This archetype deepens your study of arcane magic.
        You have the Arcane Magic archetype from the wizard class to gain the abilities from this archetype.

        \cf{Wiz}[0]{Deep Knowledge} You gain a \plus2 bonus to all Knowledge skills.
        In addition, using the \textit{desperate exertion} ability on a Knowledge check only causes you to increase your \glossterm{fatigue level} by one instead of two.

        \cf{Wiz}[1]{Ritualist}[Magical] You gain the ability to perform arcane rituals to create unique magical effects (see \pcref{Rituals}).
        The maximum \glossterm{rank} of arcane ritual you can learn or perform is equal to the maximum rank of arcane spell that you can cast.
        If you have the ability to cast rank 1 arcane spells, you may immediately scribe one rank 1 ritual without paying the normal costs.
        In addition, whenever you gain access to a new spell rank, you can scribe one ritual of that rank or lower without paying the normal costs.

        \cf{Wiz}[2]{Scholastic Insight}[Magical]
        You gain one of the following insights.
        Some insights can be chosen multiple times, as indicated in their descriptions.

        {
            \parhead{Esoteric Spell Knowledge} You learn a single spell from any arcane \glossterm{mystic sphere}.
            You do not not need to have access to that mystic sphere.
            This does not grant you access to that mystic sphere for any other purposes.
            Whenever you gain access to a new mystic sphere or spell rank, you may choose a different spell with this ability.
            \par You can choose this insight multiple times, learning an additional spell each time.

            \parhead{Expanded Sphere Access} You gain access to a new \glossterm{mystic sphere}.
            \par You cannot choose this insight multiple times.

            \parhead{Sphere Specialization} Choose a a \glossterm{mystic sphere} you have access to.
            You gain a \plus1 bonus to \glossterm{accuracy} with abilities from that \glossterm{mystic sphere}.
            In addition, you learn an additional \glossterm{spell} from that \glossterm{mystic sphere}.
            In exchange, you must lose access to another \glossterm{mystic sphere} you have.
            You must exchange all spells you know from that \glossterm{mystic sphere} with spells from other \glossterm{mystic spheres} you have access to.
            \par You cannot choose this insight multiple times.

            \parhead{Memorized Sphere} % TODO: clarify you need to be high enough rank?
            Choose a \glossterm{mystic sphere} you have access to.
            You can perform rituals from that \glossterm{mystic sphere} without having them written in your ritual book.
            If you lead a ritual from that \glossterm{mystic sphere}, it requires half the normal amount of time to perform and requires half the normal number of \glossterm{fatigue levels}, to a minimum of zero fatigue levels.
            \par You can choose this insight multiple times, choosing a different \glossterm{mystic sphere} each time.
        }

        \cf{Wiz}[3]{Contingency}[Magical] You gain the ability to prepare a spell so it takes effect automatically if specific circumstances arise.
        % If any spells take more than one standard action, they would need to be excluded from Contingency, but none exist
        % You can apply this ability to any arcane spell that can be cast as a \glossterm{standard action} or \glossterm{minor action}.
        Preparing a spell with this ability takes 5 minutes.
        When the preparation is complete, the spell has no immediate effect.
        Instead, it automatically takes effect when some specific circumstances arise.
        During the time required to cast the spell, you specify what circumstances cause the spell to take effect.

        The spell can be set to trigger in response to any circumstances that a typical human observing you and your situation could detect.
        For example, you could specify ``when I fall at least 50 feet'' or ``when I take a \glossterm{vital wound}'', but not ``when there is an invisible creature within 50 feet of me'' or ``when I have only one \glossterm{hit point} remaining.''
        The more specific the required circumstances, the better -- vague requirements, such as ``when I am in danger'', may cause the spell to trigger unexpectedly or fail to trigger at all.
        If you attempt to specify multiple separate triggering conditions, such as ``when I take damage or when an enemy is adjacent to me'', the spell will randomly ignore all but one of the conditions.

        If the spell needs to be targeted, the trigger condition can specify a simple rule for identifying how to target the spell, such as ``the closest enemy''.
        If the rule is poorly worded or imprecise, the spell may target incorrectly or fail to activate at all.
        Any spells which require decisions, such as the \spell{dimension door} spell, must have those decisions made at the time it is cast.
        You cannot alter those decisions when the contingency takes effect.

        You can have only one spell with this ability active at a time.
        If you use this ability again with a different spell, the old contingency is removed.

        \cf*{Wiz}[4]{Scholastic Insight}[Magical]
        You learn an additional \textit{scholastic insight}.

        \cf{Wiz}[5]{Greater Ritualist}[Magical] Whenever you lead a ritual, it requires half the normal number of \glossterm{fatigue levels} and half the normal time to complete, to a minimum of zero fatigue levels.

        \cf*{Wiz}[6]{Scholastic Insight}[Magical]
        You learn an additional \textit{scholastic insight}.

        \cf{Wiz}[7]{Multiple Contingency}[Magical] You may have two separate \textit{contingency} abilities active at the same time.
        Each contingency can have separate triggering conditions.
        Only one contigency can trigger each round.
        If multiple contingencies would activate simultaneously, choose one to activate randomly.

    \newpage
    \subsection{Arcane Spell Mastery}
        This archetype improves the arcane spells you cast.
        You must have the Arcane Magic archetype from the wizard class to gain the abilities from this archetype.
        All abilities from this archetype are \glossterm{magical}.

        \cf{Wiz}[0]{Mystic Sphere} You gain access to an additional arcane \glossterm{mystic sphere}, including all \glossterm{cantrips} from that sphere.

        \cf{Wiz}[1]{Mystic Insight}
        You gain your choice of one of the following abilities.
        Whenever you increase your rank in this archetype, you may change your choice.
        This can allow you to apply the benefits of insights like \textit{signature spell} to higher rank spells.
        {
            \parhead{Focused Caster} You reduce your \glossterm{focus penalty} by 1.
                You cannot choose this ability multiple times.
            \parhead{Insight Point} You gain an additional \glossterm{insight point}.
                You can choose this ability multiple times, gaining an additional insight point each time.
            \parhead{Rituals} You gain the ability to perform arcane rituals to create unique magical effects (see \pcref{Rituals}).
                The maximum \glossterm{rank} of arcane ritual you can learn or perform is equal to the maximum rank of arcane spell that you can cast.
                You cannot choose this ability multiple times.
            \parhead{Signature Spell} Choose an arcane \glossterm{spell} you know.
                The spell loses the \abilitytag{Focus} tag, allowing you to cast it without lowering your guard in combat.
                In adition, you gain a \plus1 bonus to \glossterm{accuracy} with that spell.
                You can choose this ability multiple times, choosing a different spell each time.
            \parhead{Spell Power} Choose an arcane \glossterm{spell} you know.
                You gain a bonus equal to your rank in this archetype to your \glossterm{power} with that spell.
                You can choose this ability multiple times, choosing a different spell each time.
        }

        \cf{Wiz}[2]{Arcane Attunement} You gain an additional \glossterm{attunement point}.

        \cf{Wiz}[3]{Wellspring of Power}
        You gain a \plus2 bonus to your \glossterm{magical} \glossterm{power}.

        \cf*{Wiz}[4]{Mystic Insight}
        You gain an additional \textit{mystic insight} ability.

        \cf{Wiz}[5]{Greater Arcane Attunement} The number of attunement points you gain from your \textit{arcane attunement} ability increases to two.

        \cf{Wiz}[6]{Greater Wellspring of Power}
        The bonus from your \textit{wellspring of power} ability increases to \plus6.

        \cf*{Wiz}[7]{Mystic Insight}
        You gain an additional \textit{mystic insight} ability.
