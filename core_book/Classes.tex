\chapter{Classes}\label{Classes}

Your character's class represents the things your character has chosen to train in.
This choice determines a great deal about your character's abilities.

\section{How Classes Work}
    When you first create a character, you choose a class.
    Each class grants some basic class features to all members of that class.
    In addition, each class has a number of \glossterm{archetypes} that grant more powerful and specific abilities.

    \subsection{Archetypes}\label{Archetypes}
        Each class has a number of \glossterm{archetypes}.
        An archetype is a collection of thematically related abilities.
        For examples, barbarians have the Battlerager archetype, which grants abilities related to being angry and flying into a rage in combat.
        You have an \glossterm{archetype rank} associated with each archetype you have.

        \subsubsection{Archetype Ranks}\label{Archetype Ranks}
            Each ability from an archetype has a minimum rank required to gain the ability.
            When you gain a rank in an archetype, you gain all abilities associated with that rank.
            In addition, some of your existing abilities may increase their power based on your rank in that archetype.

            At 1st level, you choose three of the archetypes associated with your class.
            You are Rank 1 in one of those archetypes, and rank 0 in your other two archetypes.
            You have no ranks at all in any other archetypes, and can never gain abilities from archetypes other than your chosen three.

            Every level after 1st level, you increase your rank in one archetype of your choice.
            This gives you the abilities associated with that rank.
            Each \glossterm{archetype rank} has a minimum level, as shown on \trefnp{Archetype Ranks by Level}.
            This minimum level is included in each class table as a reminder.

            \begin{dtable}
                \lcaption{Archetype Ranks by Level}
                \begin{dtabularx}{\columnwidth}{l >{\lcol}X}
                    \tb{Archetype Rank} & \tb{Minimum Level} \tableheaderrule
                    1 & 1  \\
                    2 & 4  \\
                    3 & 7  \\
                    4 & 10 \\
                    5 & 13 \\
                    6 & 16 \\
                    7 & 19 \\
                \end{dtabularx}
            \end{dtable}

        \subsubsection{Duplicate Archetypes}\label{Duplicate Archetypes}
            Some archetypes can be gained by multiple classes.
            For example, both clerics and paladins have the Divine Magic archetype.
            You cannot gain two archetypes with the same name, even if you can choose archetypes from multiple classes.

        \subsection{Multiclass Characters}\label{Multiclass Characters}
            You can spend two \glossterm{insight points} to become a \glossterm{multiclass} character (see \pcref{Insight Points}).
            If you do, choose a class other than your original class.
            You gain the following benefits relating to that class.
            \begin{itemize}
                \item You gain the \glossterm{class skills} of that class in addition to your existing \glossterm{class skills}.
                \item You can choose which of your classes you are considered for the purpose of determining which basic class abilities, such as defense bonuses and weapon proficiencies, you start with.
                    You must choose a single starting class for this purpose, so you cannot gain the defense bonuses from one class and the weapon proficiencies from another.
                \item If that class has any special class abilities which are not part of an archetype, such as a warlock's \textit{soul pact} ability, you gain those abilities.
                \item You may gain one \glossterm{archetype} from your that class in place of one \glossterm{archetype} from your base class.
            \end{itemize}

            You may gain access to multiple classes in this way, spending two \glossterm{insight points} for each class.

\section{Class Introductions}

    There are nine classes in Rise.
    \begin{itemize}
        \item Barbarians are mighty warriors who draw power from their physical prowess.
        \item Clerics are divine spellcasters who draw power from their veneration of a deity.
        \item Druids are nature spellcasters who draw power from their veneration of the natural world.
        \item Fighters are highly disciplined warriors who excel in physical combat of any variety.
        \item Monks are agile masters of ``ki'' who hone their personal abilities to strike down foes and perform supernatural feats.
        \item Paladins are divinely empowered warriors embody a particular alignment.
        \item Rangers are skilled hunters who bridge the divide between nature and civilization.
        \item Rogues are exceptionally skillful characters known for their ability to strike at their foe's weak points in combat.
        \item Sorcerers are arcane spellcasters who draw power from their inherently magical nature.
            % \item Spellwarped wield a unique blend of martial skill and narrowly focused magical abilities.
        \item Warlocks are pact spellcasters who draw their power from a dark pact made with infernal creatures.
        \item Wizards are arcane spellcasters who study magic to unlock its powerful secrets.
    \end{itemize}

    \subsection{Class Description Format}
        Each class is described from the perspective of a member of that class, using ``you'' in the description.

        \parhead{Class Table}
        The class's table describes the special abilities a member of that class gains at each level, assuming they have all of that class's \glossterm{archetypes}.

        \parhead{Alignment}
        Some classes require specific alignments (see \pcref{Alignment}).
        Most classes allow characters of any alignment.

        \parhead{Skills}
        Each class has specific \glossterm{skills} that members of that class are typically good at (see \pcref{Skills}).
        These skills are called \glossterm{class skills}.
        It is easier to become \glossterm{mastered} in class skills than in other skills.
        For details, see \pcref{Skill Training}.

        \parhead{Defenses}
        Each class grants bonuses to specific defenses.

        \parhead{Weapon Proficiencies}
        This indicates the types of weapons that members of this class are proficient with.

        \parhead{Armor Proficiencies}
        This indicates the types of armor that members of this class are proficient with.

        \parhead{Other Special Abilities}
        Some classes have abilities shared by all members of the class that are not part of an archetype, such as a druid's \textit{druidic language} ability.

        \parhead{Archetypes}
        The abilities associated with each of the three archetypes the class has.

\input{generated/classes.tex}
