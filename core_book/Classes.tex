\chapter{Classes}\label{Classes}

Your character's class represents the things your character has chosen to train in.
This choice determines a great deal about your character's abilities.

\section{How Classes Work}
    When you first create a character, you choose a class.
    Each class grants some basic class features to all members of that class.
    In addition, each class has a number of \glossterm{archetypes} that grant more powerful and specific abilities.

    \subsection{Archetypes}\label{Archetypes}
        Each class has five class archetypes.
        An archetype is a collection of thematically related abilities.
        For examples, barbarians have the Battlerager archetype, which grants abilities related to being angry and flying into a rage in combat.

        \subsubsection{Archetype Ranks}\label{Archetype Ranks}
            You have an \glossterm{archetype rank} associated with each of your archetypes.
            Each ability from an archetype has a minimum rank required to gain the ability.
            When you gain a rank in an archetype, you gain all abilities associated with that rank.
            In addition, some of your existing abilities may increase their power based on your rank in that archetype.

            At 1st level, you choose a single archetype from your class (or classes).
            You become rank 1 in that archetype, and you do not have any other archetypes.
            At 2nd and 3rd level, you choose an additional archetype from your class or classes.
            Each time, you become rank 1 in that archetype.

            After 3rd level, you never gain additional archetypes.
            Instead, at each level, you increase your rank in one of your existing archetypes.
            Each \glossterm{archetype rank} has a minimum level, as shown on \trefnp{Archetype Ranks by Level}.
            The minimum level is included in each class table as a reminder.
            In practice, this means that you have to increase all of your ranks evenly instead of specializing in a single archetype.

            \begin{dtable}
                \lcaption{Archetype Ranks by Level}
                \begin{dtabularx}{\columnwidth}{l >{\lcol}X}
                    \tb{Archetype Rank} & \tb{Minimum Level} \tableheaderrule
                    1 & 1  \\
                    2 & 4  \\
                    3 & 7  \\
                    4 & 10 \\
                    5 & 13 \\
                    6 & 16 \\
                    7 & 19 \\
                \end{dtabularx}
            \end{dtable}

        \subsubsection{Duplicate Archetypes}\label{Duplicate Archetypes}
            Some archetypes can be gained by multiple classes.
            For example, both clerics and paladins have the Divine Magic archetype.
            You cannot gain two archetypes with the same name, even if you can choose archetypes from multiple classes.

        \subsection{Multiclass Characters}\label{Multiclass Characters}
            You can spend two \glossterm{insight points} to become a \glossterm{multiclass} character (see \pcref{Insight Points}).
            If you do, choose a class other than your original class.
            You gain the following benefits relating to that class.
            \begin{itemize}
                \item You gain the \glossterm{class skills} of that class in addition to your existing \glossterm{class skills}.
                \item If that class has any special class abilities which are not part of an archetype, such as a warlock's \textit{soul pact} ability, you gain those abilities.
                \item Whenever you gain a new archetype, you may choose an archetype from either your original class or your new class.
                    If have multiple archetypes when you become a multiclass character, you may exchange any number of archetypes from your original class for that many archetypes from your new class.
                    However, you must always have at least one archetype from your original class.
            \end{itemize}

            You may gain access to multiple classes in this way, spending two \glossterm{insight points} for each class.
            Unless your GM says otherwise, you can only become a multiclass character as part of initial character creation.
            Changing a higher level character character into a multiclass character can be a major change that requires losing access to abilities that you have already had for some time.
            Generally, you should have an explicit narrative justification for a major character change like that.
            Of course, every game is different, so talk to your GM.

\section{Class Description Format}
    Each class is described from the perspective of a member of that class, using ``you'' in the description.

    \parhead{Class Table}
    Each class's table describes the special abilities a member of that class gains at each rank of each of that class's archetypes.

    \parhead{Alignment}
    Some classes require specific alignments (see \pcref{Alignment}).
    Most classes allow characters of any alignment.

    \parhead{Skills}
    Each class has specific \glossterm{skills} that members of that class are typically good at (see \pcref{Skills}).
    These skills are called \glossterm{class skills}.
    For details, see \pcref{Trained Skills}.

    \parhead{Defenses}
    Each class grants bonuses to specific defenses.

    \parhead{Weapon Proficiencies}
    This indicates the types of weapons that members of this class are proficient with.

    \parhead{Armor Proficiencies}
    This indicates the types of armor that members of this class are proficient with.

    \parhead{Other Special Abilities}
    Some classes have abilities shared by all members of the class that are not part of an archetype, such as a druid's \textit{druidic language} ability.

    \parhead{Archetypes}
    The abilities associated with each of the three archetypes the class has.

\input{generated/classes.tex}
