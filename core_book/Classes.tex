\chapter{Classes}\label{Classes}

Your character's class represents the things your character has chosen to train in.
This choice determines a great deal about your character's abilities.

\section{How Classes Work}
    When you first create a character, you choose a class.
    You gain all abilities granted by the \glossterm{archetypes} of your chosen class at the levels indicated in the archetype's description (see Archetypes, below).
    As you gain levels, you gain more abilities from your class.

    \subsection{Archetypes}\label{Archetypes}
        Each class has three \glossterm{archetypes}.
        An archetype is a collection of thematically related class abilities.
        For examples, barbarians have the Battlerager archetype, which grants abilities related to flying into a rage in combat.
        Normally, a member of a class has all three archetypes associated with that class.
        Characters with the Class Versatility feat can gain achetypes from two different classes (see \featpcref{Class Versatility}).

        \subsubsection{Archetype Ranks}\label{Archetype Ranks}
            You have an \glossterm{archetype rank} associated with each archetype you have.
            Each ability from an archetype has a minimum rank required to gain the ability.

            At 1st level, you are Rank 1 in each archetype.
            This gives you the Rank 1 abilities from each of your archetypes.
            Every level thereafter, you increase your rank in one archetype of your choice.
            This gives you the abilities associated with that rank.

            Each \glossterm{archetype rank} has a minimum level, as shown on \trefnp{Archetype Ranks by Level}.

            \begin{dtable}
                \lcaption{Archetype Ranks by Level}
                \begin{dtabularx}{\columnwidth}{>{\ccol}p{\levelcol} >{\lcol}X}
                    \tb{Archetype Rank} & \tb{Minimum Level} \\\bottomrule
                    1     & 1
                    \\ 2  & 2
                    \\ 3  & 5
                    \\ 4  & 8
                    \\ 5 & 11
                    \\ 6 & 14
                    \\ 7 & 17
                    \\ 8 & 20
                \end{dtabularx}
            \end{dtable}

        \subsubsection{Duplicate Archetypes}\label{Duplicate Archetypes}
            Some archetypes can be gained by multiple classes.
            For example, both clerics and paladins have the Divine Spellcasting archetype.
            You cannot gain two archetypes with the same name, even if you can choose archetypes from multiple classes.

\section{Class Introductions}

    There are nine classes in Rise.
    \begin{itemize}
        \item Barbarians are mighty warriors who can enter a deadly battlerage.
        \item Clerics are divine spellcasters who draw power from their veneration of a deity.
        \item Druids are nature spellcasters who draw power from their veneration of the natural world.
        \item Fighters are highly disciplined warriors who excel in physical combat of any variety.
        \item Mages are arcane spellcasters who wield the mystic forces of magic to create almost any effect.
        \item Monks are agile masters of ``ki'' who hone their personal abilities to strike down foes and perform supernatural feats.
        \item Paladins are divinely empowered warriors whose devotion to an alignment grants them the ability to discern and smite their foes.
        \item Rangers are skilled hunters who bridge the divide between nature and civilization.
        \item Rogues are exceptionally skillful characters known for their ability to strike at their foe's weak points in combat.
            % \item Spellwarped wield a unique blend of martial skill and narrowly focused magical abilities.
        \item Warlocks are pact spellcasters who draw their power from a dark pact made with infernal creatures.
    \end{itemize}

    \subsection{Class Description Format}
        Each class is described from the perspective of a member of that class, using ``you'' in the description.

        \parhead{Class Table}
        The class's table describes the special abilities a member of that class gains at each level, assuming they have all of that class's \glossterm{archetypes}.

        \parhead{Alignment}
        Some classes require specific alignments (see \pcref{Alignment}).
        Most classes allow characters of any alignment.

        \parhead{Skills}
        Each class has specific \glossterm{skills} that members of that class are typically good at (see \pcref{Skills}).
        These skills are called \glossterm{class skills}.
        It is easier to become \glossterm{mastered} in class skills than in other skills.
        For details, see \pcref{Skill Training}.

        \parhead{Defenses}
        Each class grants bonuses to specific defenses.

        \parhead{Weapon and Armor Proficiencies}
        These are the types of equipment that members of this class are trained in using.

        \parhead{Other Special Abilities}
        Some classes have abilities shared by all members of the class that are not part of an archetype, such as a druid's \textit{druidic language} ability.

        \parhead{Archetypes}
        The abilities associated with each of the three archetypes the class has.

\newpage
\section{Barbarian}\label{Barbarian}
    \begin{dtable}
        \lcaption{Barbarian Progression}
        \begin{dtabularx}{\columnwidth}{c >{\lcol}X >{\lcol}X >{\lcol}X}
            \tb{Rank} & \tb{Battleforged Resilience} & \tb{Battlerager}  & \tb{Primal Warrior} \\\bottomrule
               1 & Fury of the storm         & Rage                    & Primal smash
            \\ 2 & Battle-scarred            & Determined rage         & Primal exertions
            \\ 3 & Battleforged fortitude    & Focused rage            & Athletic prowess
            \\ 4 & Soulscarred               & Deathless rage           & Primal exertion
            \\ 5 & Greater fury of the storm & Deathless rage          & Greater primal smash
            \\ 6 & Agile defense             & Greater fearsome rage   & Primal exertion
            \\ 7 & Greater battle-scarred    & Greater deathless rage  & Primal prowess
            \\ 8 & Battleforged supremacy    & Greater determined rage & Primal dominance, primal exertion
        \end{dtabularx}
    \end{dtable}

    \classbasics{Alignment} Any nonlawful.

    \classbasics{Archetypes} Barbarians have the Battlerager, Primal Warrior, and Battleforged Resilience \glossterm{archetypes}.

    \subsection{Basic Class Abilities}
        If you are a barbarian, you gain the following abilities.

        \cf{Bbn}{Defenses}
        You gain the following bonuses to your \glossterm{defenses}: \plus5 Fortitude, \plus4 Reflex, \plus3 Mental.

        \cf{Bbn}{Skills}
        You have the following \glossterm{class skills}:
        \begin{itemize}
            \item \subparhead{Strength} Climb, Jump, Swim.
            \item \subparhead{Dexterity} Acrobatics, Ride.
            \item \subparhead{Intelligence} Craft.
            \item \subparhead{Perception} Awareness, Creature Handling, Survival.
            \item \subparhead{Other} Bluff, Intimidate, Persuasion, Profession.
        \end{itemize}

        \cf{Bbn}{Weapon and Armor Proficiencies} 
        You are proficient with simple weapons, any four other weapon groups, light armor, medium armor, and shields.

    \subsection{Battlerager}\label{Rage}
        This archetype grants you a devastating rage, improving your combat prowess.

        \cf{Bbn}[1]{Rage} You can use the \textit{rage} ability as a \glossterm{free action}.
        \begin{attuneability}{Rage}[\glossterm{Attune} (self), \glossterm{Swift}]
            You gain the following benefits and drawbacks:
            \begin{itemize}
                \item You gain a \plus2 bonus to \glossterm{power} with \glossterm{physical attacks}.
                \item You take a \minus1 penalty to \glossterm{accuracy} with \glossterm{physical attacks}.
                \item You are unable to take \glossterm{standard actions} that do not cause you to make physical attacks.
                \item At the end of each round, if you did not attack a creature or object that round, you take \glossterm{subdual damage} equal to your level.
                    This damage ignores damage reduction and any similar abilities.
            \end{itemize}
            When this ability ends, you become \fatigued and unable to use it again until you take a \glossterm{short rest}.
        \end{attuneability}

        \cf{Bbn}[2]{Determined Rage}
        You are immune to the first \glossterm{condition} that would affect you during a \textit{rage}.

        \cf{Bbn}[3]{Focused Rage}
        Your \textit{rage} ability no longer imposes a penalty to \glossterm{accuracy} with \glossterm{strikes}.

        \cf{Bbn}[4]{Deathless Rage}
        During a \textit{rage}, you reduce your \glossterm{vital damage penalties} by 3.

        \cf{Bbn}[5]{Tireless Rage} You are no longer \glossterm{fatigued} after raging, and you can use the ability again before taking a \glossterm{short rest}.

        \cf{Bbn}[6]{Greater Rage} The \glossterm{power} bonus from your \textit{rage} ability increases to \plus4.

        \cf{Bbn}[7]{Greater Deathless Rage}
        The penalty reduction from your \textit{deathless rage} ability increases to 6.

        \cf{Bbn}[8]{Greater Determined Rage}
        The number of conditions you can become immune to with your \textit{determined rage} ability increases to two.

    \subsection{Primal Warrior}
        This archetype grants you abilities to use in combat and improves your physical skills.

        \cf{Bbn}[1]{Primal Smash} You can use the \textit{primal smash} ability as a standard action.
        \begin{freeability}{Primal Smash}
            Make a melee \glossterm{strike}.
            If you deal damage with the strike, you regain hit points equal to your \glossterm{power}.
        \end{freeability}

        \cf{Bbn}[2]{Primal Exertions}
        You can channel your primal energy into ferocious attacks.
        Choose one \textit{primal exertion} from the list below.
        You can also spend \glossterm{insight points} to learn one additional \textit{primal exertion} per \glossterm{insight point}.
        You can use any \textit{primal exertion} ability you know as a \glossterm{standard action}.
        {
            \begin{apability}{Battle Cry}[\glossterm{Mind}]
                You and any number of willing creatures within a \arealarge radius burst from you heal hit points equal to \glossterm{standard damage}.

                \rankline
                \rank{4} You gain a \plus1d bonus to the amount healed.
                \rank{6} The healing bonus increases to \plus2d.
                \rank{8} The healing bonus increases to \plus3d.
            \end{apability}

            \begin{freeability}{Brace for Impact}[\glossterm{Swift}]
                You take half damage from all attacks.
                This halving is applied before \glossterm{damage reduction} and similar abilities.
                This ability lasts until the end of the round.

                \rankline
                \rank{4} You also gain a \plus1 bonus to all defenses.
                \rank{6} The defense bonus increases to \plus2.
                \rank{8} The defense bonus increases to \plus3.
            \end{freeability}

            \begin{apability}{Certain Strike}
                Make a \glossterm{strike} with a \plus3 bonus to accuracy.

                \rankline
                \rank{4} The accuracy bonus increases to \plus4.
                \rank{6} The accuracy bonus increases to \plus5.
                \rank{8} The accuracy bonus increases to \plus6.
            \end{apability}

            \begin{apability}{Counterattack}
                Make a \glossterm{strike}.
                If the target attacked you in the same phase, you gain a \plus2 bonus to accuracy and a \plus2d bonus to damage.
                Otherwise, you regain the action point spent to use this ability.

                \rankline
                \rank{4} The damage bonus increases to \plus3d.
                \rank{6} The damage bonus increases to \plus4d.
                \rank{8} The damage bonus increases to \plus5d.
            \end{apability}

            \begin{apability}{Daunting Blow}[\glossterm{Mind}]
                Make a melee \glossterm{strike}.
                If the target takes damage from the strike, it is \glossterm{shaken} by you as a \glossterm{condition}.
                Otherwise, you regain the \glossterm{action point} spent to use this ability.

                \rankline
                \rank{4} You gain a \plus1 bonus to accuracy on the strike.
                \rank{6} The accuracy bonus increases to \plus2.
                \rank{8} The accuracy bonus increases to \plus3.
            \end{apability}

            \begin{apability}{Demoralizing Shout}[\glossterm{Mind}]
                Make an attack vs. Mental against all enemies within a \arealarge radius burst from you.
                \hit Each target is \glossterm{shaken} by you as a \glossterm{condition}.

                \rankline
                \rank{4} You gain a \plus1 bonus to \glossterm{accuracy} with the attack.
                \rank{6} The accuracy bonus increases to \plus2.
                \rank{8} The accuracy bonus increases to \plus3.
            \end{apability}

            \begin{apability}{Ground Pound}
                You can only use this ability while standing on solid ground.
                Make an attack vs. Reflex against all enemies standing on solid ground adjacent to you.
                If you use this ability during the \glossterm{action phase}, you can also make a \glossterm{strike} during the \glossterm{delayed action phase}.
                \hit Each target is knocked \prone.

                \rankline
                \rank{4} The area increases to a \areamed radius burst.
                \rank{6} The area increases to a \arealarge radius burst.
                \rank{8} The area increases to a \areahuge radius burst.
            \end{apability}

            \begin{apability}{Leaping Strike}
                You make a Jump check to leap and move as normal for the leap, up to a maximum distance equal to your \glossterm{base speed} (see \pcref{Leap}).
                You can make a \glossterm{strike} with a \plus1d bonus to damage from any location you occupy during the leap.

                \rankline
                \rank{4} The damage bonus increases to \plus2d.
                \rank{6} The damage bonus increases to \plus3d.
                \rank{8} The damage bonus increases to \plus4d.
            \end{apability}

            \begin{apability}{Liver Shot}
                Make a \glossterm{strike} with a bludgeoning weapon.
                If the target takes damage from the strike, it is \sickened as a \glossterm{condition}.
                Otherwise, you regain the \glossterm{action point} spent to use this ability.

                \rankline
                \rank{4} You gain a \plus1 bonus to \glossterm{accuracy} with the strike.
                \rank{6} The accuracy bonus increases to \plus2.
                \rank{8} The accuracy bonus increases to \plus3.
            \end{apability}

            \begin{apability}{Power Attack}
                Make a \glossterm{strike} with a \plus2d bonus to damage.

                \rankline
                \rank{4} The damage bonus increases to \plus3d.
                \rank{6} The damage bonus increases to \plus4d.
                \rank{8} The damage bonus increases to \plus5d.
            \end{apability}

            \begin{apability}{Pulverizing Smash}
                Make a \glossterm{strike} with a piercing weapon.
                The attack is made against the target's Fortitude defense instead of its Armor defense.

                \rankline
                \rank{4} You gain a \plus1d bonus to damage with the strike.
                \rank{6} The damage bonus increases to \plus2d.
                \rank{8} The damage bonus increases to \plus3d.
            \end{apability}

            \begin{apability}{Rapid Assault}
                Make a \glossterm{strike} against a creature.
                If you use this ability during the \glossterm{action phase}, you can make another strike during the \glossterm{delayed action phase}.
                You take a \minus2 penalty to accuracy on both strikes.

                \rankline
                \rank{4} The accuracy penalty is reduced to \minus1.
                \rank{6} The accuracy penalty is removed.
                \rank{8} You gain a \plus1 bonus to accuracy with both strikes.
            \end{apability}

            \begin{apability}{Reaping Charge}
                Move up to your movement speed in a straight line.
                You can make a melee \glossterm{strike} with a slashing or bludgeoning weapon.
                The strike targets any number of creatures and objects that you \glossterm{threaten} at any point during your movement, except for the space you start in and the space you end in.
                You take a \minus2 penalty to \glossterm{accuracy} on the strike.

                \rankline
                \rank{4} You gain a \plus1d bonus to damage with the strike.
                \rank{6} The damage bonus increases to \plus2d.
                \rank{8} The damage bonus increases to \plus3d.
            \end{apability}

            \begin{apability}{Strip the Flesh}
                Make a \glossterm{strike} with a slashing weapon.
                At the end of the \glossterm{action phase} of the next round, if you hit with the strike and the target is not \glossterm{bloodied}, it takes additional damage equal to the damage you dealt with the strike.

                \rankline
                \rank{4} If you hit with the strike, the target continues taking the same damage at the end of each \glossterm{action phase} until it becomes \glossterm{bloodied}.
                This is a \glossterm{condition}, and can be removed by abilities that remove conditions.
                \rank{6} You gain a \plus1d bonus to damage with the strike.
                \rank{8} The damage bonus increases to \plus2d.
            \end{apability}

            \begin{apability}{Sweeping Strike}
                Make a melee \glossterm{strike} with a slashing or bludgeoning weapon.
                The strike targets each of up to three creatures or objects you \glossterm{threaten}.
                You take a \minus1 penalty to \glossterm{accuracy} with the strike.

                \rankline
                \rank{4} You gain a \plus1d bonus to damage with the strike.
                \rank{6} The damage bonus increases to \plus2d.
                \rank{8} The damage bonus increases to \plus3d.
            \end{apability}

            \begin{apability}{Thunderous Shout}[\glossterm{Sonic}]
                Make an attack vs. Fortitude against all creatures and objects in a \areamed cone-shaped burst from you.
                \hit Each target takes sonic \glossterm{standard damage} and is \glossterm{deafened} as a \glossterm{condition}.

                \rankline
                \rank{4} You gain a \plus1d bonus to damage with the attack.
                \rank{6} The area increases to \arealarge.
                \rank{8} The damage bonus increases to \plus2d.
            \end{apability}

            \begin{apability}{Whirlwind Spin}
                Make a melee \glossterm{strike} with a slashing weapon.
                The strike targets all creatures you \glossterm{threaten}.
                You take a \minus2 penalty to \glossterm{accuracy} with the strike.

                \rankline
                \rank{4} You gain a \plus1d bonus to damage with the strike.
                \rank{6} The damage bonus increases to \plus2d.
                \rank{8} The damage bonus increases to \plus3d.
            \end{apability}
        }

        \cf{Bbn}[3]{Athletic Prowess} You gain two additional \glossterm{skill points}.

        \cf*{Bbn}[4]{Primal Exertion}
        You learn an additional \textit{primal exertion}.

        \cf{Bbn}[5]{Greater Primal Smash} The healing from your \textit{primal smash} ability increases to be equal to twice your \glossterm{power}.

        \cf*{Bbn}[6]{Primal Exertion}
        You learn an additional \textit{primal exertion}.

        \cf{Bbn}[7]{Primal Prowess}
        You gain a \plus2 bonus to Strength-based and Dexterity-based checks.

        \cf{Bbn}[8]{Primal Dominance}
        You gain a \plus1d bonus to damage with \glossterm{strikes}.

        \cf*{Bbn}[8]{Primal Exertion}
        You learn an additional \textit{primal exertion}.

    \subsection{Battleforged Resilience}
        This archetype improves your defenses in combat.

        \cf{Bbn}[1]{Fury of the Storm} You gain a \plus1 bonus to \glossterm{overwhelm resistance}.

        \cf{Bbn}[2]{Battle-Scarred} You gain \glossterm{damage reduction} equal to your level against damage from \glossterm{physical attacks}.

        \cf{Bbn}[3]{Battleforged Fortitude} You gain a \plus2 bonus to Fortitude defense.

        \cf{Bbn}[4]{Soulscarred} The \glossterm{damage reduction} from your \textit{battle-scarred} ability applies against all damage, not just damage from physical attacks.

        \cf{Bbn}[5]{Greater Fury of the Storm}
        The bonus from your \textit{fury of the storm} ability increases to \plus2.

        \cf{Bbn}[6]{Agile Defense}
        You gain a \plus1 bonus to Armor defense.

        \cf{Bbn}[7]{Greater Battle-Scarred}
        The \glossterm{damage reduction} from your \textit{battle-scarred} ability increases to twice your level.

        \cf{Bbn}[8]{Battleforged Supremacy}
        You gain a \plus1 bonus to all defenses.

    \subsection{Ex-Barbarians}
        If you become lawful, you cannot use your \textit{rage} ability.
        You retain all of your other class abilities.
        If you stop being lawful, you can use your \textit{rage} ability once more.

\newpage
\section{Cleric}\label{Cleric}
    \begin{dtable}
        \lcaption{Cleric Progression}
        \begin{dtabularx}{\columnwidth}{c >{\lcol}X >{\lcol}X >{\lcol}X}
            \tb{Rank} & \tb{Divine Spellcasting} & \tb{Divine Spell Mastery}  & \tb{Domain Influence} \\\bottomrule
               1 & Mystic spheres         & Wellspring of magic         & Domains, domain gift
            \\ 2 & Spells                 & Mystic knowledge            & Domain gift
            \\ 3 & Spell level (2)        & Augments                    & Domain aspect
            \\ 4 & Spell level (3), spell & Mystic knowledge            & Domain aspect
            \\ 5 & Spell level (4)        & Augment                     & Domain essence
            \\ 6 & Spell level (5), spell & Greater wellspring of magic & Domain essence
            \\ 7 & Spell level (6)        & Mystic knowledge            & Domain mastery
            \\ 8 & Spell level (7), spell & Mystic supremacy            & Miracle
        \end{dtabularx}
    \end{dtable}

    \classbasics{Alignment} Your alignment must be within one step of your deity's (that is, it may be one step away on either the lawful-chaotic axis or the good-evil axis, but not both).

    \classbasics{Archetypes} Clerics have the Divine Spellcasting, Domain Influence, and Divine Spell Mastery \glossterm{archetypes}.

    \subsection{Basic Class Abilities}
        If you are a cleric, you gain the following abilities.

        \cf{Clr}{Defenses}
        You gain the following bonuses to your \glossterm{defenses}: \plus4 Fortitude, \plus3 Reflex, \plus5 Mental.

        \cf{Clr}{Skills}
        You have the following \glossterm{class skills}:
        \begin{itemize}
            \item \subparhead{Intelligence} Craft, Deduction, Heal, Knowledge (arcana, local, religion, the planes), Linguistics.
            \item \subparhead{Perception} Awareness, Sense Motive, Spellcraft.
            \item \subparhead{Other} Bluff, Intimidate, Persuasion, Profession.
        \end{itemize}

        \cf{Clr}{Weapon and Armor Proficiencies}
        You are proficient with simple weapons, any two other weapon groups, light and medium armor, and shields.

        \cf{Clr}{Deity}
        You must worship a specific deity to be a cleric.
        Deities and their associated domains are listed in \trefnp{Deities}.

        \begin{dtable!*}
            \lcaption{Deities}
            \begin{dtabularx}{\textwidth}{X l X}
                \tb{Deity} & \tb{Alignment} & \tb{Domains} \\
                \bottomrule
                Guftas, horse god of justice          & Lawful good     & Good, Law, Strength, Travel         \\
                Lucied, paladin god of justice        & Lawful good     & Destruction, Good, Protection, War  \\
                Simor, fighter god of protection      & Lawful good     & Good, Protection, Strength, War     \\
                %Pabst, dwarf god of drink             & Neutral good & Good, Life, Strength, Wild \\
                Rucks, monk god of pragmatism         & Neutral good    & Good, Law, Protection, Travel       \\
                Vanya, centaur god of nature          & Neutral good    & Good, Strength, Travel, Wild        \\
                Brushtwig, pixie god of creativity    & Chaotic good    & Chaos, Good, Trickery, Wild         \\
                Chavi, god of stories                 & Chaotic good    & Chaos, Knowledge, Trickery          \\
                Ivan Ivanovitch, bear god of strength & Chaotic good    & Chaos, Strength, War, Wild          \\
                Krunch, barbarian god of destruction  & Chaotic good    & Destruction, Good, Strength, War    \\
                Sir Cakes, dwarf god of freedom       & Chaotic good    & Chaos, Good, Strength               \\
                Raphael, monk god of retribution      & Lawful neutral  & Death, Law, Protection, Travel      \\
                Declan, god of fire                   & True neutral    & Destruction, Fire, Knowledge, Magic \\
                Kurai, shaman god of nature           & True neutral    & Air, Earth, Fire, Water             \\
                %Amanita, druid god of decay           & Chaotic neutral & Chaos, Destruction, Life, Wild \\
                %Antimony, elf god of necromancy       & Chaotic neutral & Death, Knowledge, Life, Magic \\
                Clockwork, elf god of time            & Chaotic neutral & Chaos, Magic, Trickery, Travel      \\
                %Lord Khallus, fighter god of pride    & Chaotic neutral & Chaos, Strength, War \\
                %Celeano, sorcerer god of deception    & Chaotic neutral & Chaos, Magic, Protection, Trickery \\
                Murdoc, god of mercenaries            & Chaotic neutral & Destruction, Knowledge, Travel, War \\
                Ribo, halfling god of trickery        & Chaotic neutral & Chaos, Trickery, Water              \\
                Tak, orc god of war                   & Lawful evil     & Law, Strength, Trickery, War        \\
                Theodolus, sorcerer god of ambition   & Neutral evil    & Evil, Knowledge, Magic, Trickery    \\
                Daeghul, demon god of slaughter       & Chaotic evil    & Destruction, Evil, Magic, War       \\
            \end{dtabularx}
        \end{dtable!*}

    \subsection{Divine Spellcasting}
        This archetype grants you the ability to cast divine spells.
        You must have a starting Willpower of at least 1 to gain this archetype.

        \cf{Clr}[1]{Mystic Spheres}[Magical]
        Your deity grants you the ability to use divine magic.
        You gain access to two divine \glossterm{mystic spheres} (see \pcref{Divine Mystic Spheres}).
        As a \glossterm{standard action}, you can cast the \glossterm{cantrip} spell from any mystic sphere you have access to.

        For details about mystic spheres and casting spells, see \pcref{Spell and Ritual Mechanics}.

        \cf{Clr}[2]{Spells} You learn two divine \glossterm{spells}.
        You can also spend \glossterm{insight points} to learn one additional divine spell per \glossterm{insight point}.
        You can learn any 1st level spells from the divine \glossterm{mystic spheres} you have access to.
        As a \glossterm{standard action}, you can spend an \glossterm{action point} to cast any \glossterm{spell} you know.
        When you gain access to a new \glossterm{mystic sphere} or spell level, you can exchange any number of spells you know for spells of other spells you know, including spells of the higher level.

        \cf{Clr}[3]{Spell Level}[Magical] You gain the ability to cast 2nd level divine spells.

        \cf*{Clr}[4]{Spell Level}[Magical] You gain the ability to cast 3rd level divine spells.

        \cf{Clr}[4]{Spell} You gain an additional \glossterm{spell} for any divine \glossterm{mystic sphere} you know.

        \cf*{Clr}[5]{Spell Level}[Magical] You gain the ability to cast 4th level divine spells.

        \cf*{Clr}[6]{Spell Level}[Magical] You gain the ability to cast 5th level divine spells.

        \cf*{Clr}[6]{Spell} You gain an additional \glossterm{spell} for any divine \glossterm{mystic sphere} you know.

        \cf*{Clr}[7]{Spell Level}[Magical] You gain the ability to cast 6th level divine spells.

        \cf*{Clr}[8]{Spell Level}[Magical] You gain the ability to cast 7th level divine spells.

        \cf*{Clr}[8]{Spell} You gain an additional \glossterm{spell} for any divine \glossterm{mystic sphere} you know.

    \subsection{Domain Influence}
        This archetype grants you divine influence over two domains of your choice.

        \cf{Clr}[1]{Domains}
        You choose two domains which represent your personal spiritual inclinations.
        You must choose your domains from among those your deity offers.
        The domains are listed below.

        \begin{itemize}
            \item{Air}
            \item{Chaos}
            \item{Death}
            \item{Destruction}
            \item{Earth}
            \item{Evil}
            \item{Fire}
            \item{Good}
            \item{Knowledge}
            \item{Law}
            \item{Life}
            \item{Magic}
            \item{Protection}
            \item{Strength}
            \item{Travel}
            \item{Trickery}
            \item{War}
            \item{Water}
            \item{Wild}
        \end{itemize}

        \cf{Clr}[1]{Domain Gift}[Magical]
        Each domain has a corresponding \textit{domain gift}.
        You gain the \textit{domain gift} for one of your domains (see \pcref{Cleric Domain Abilities}).

        \cf*{Clr}[2]{Domain Gift}[Magical]
        You gain the \textit{domain gift} for another one of your domains.

        \cf{Clr}[3]{Domain Aspect}[Magical]
        Each domain has a corresponding \textit{domain aspect}.
        You gain the \textit{domain aspect} for one of your domains (see \pcref{Cleric Domain Abilities}).

        \cf*{Clr}[4]{Domain Aspect}[Magical]
        You gain the \textit{domain aspect} for another one of your domains.

        \cf{Clr}[5]{Domain Essence}[Magical]
        Each domain has a corresponding \textit{domain essence}.
        You gain the \textit{domain essence} for one of your domains (see \pcref{Cleric Domain Abilities}).

        % This seems weak
        \cf*{Clr}[6]{Domain Essence}[Magical]
        You gain the \textit{domain essence} for another one of your domains.

        % This seems strong
        \cf{Clr}[7]{Domain Mastery}[Magical]
        Each domain has a corresponding \textit{domain mastery}.
        You gain the \textit{domain mastery} for both of your domains (see \pcref{Cleric Domain Abilities}).

        \cf{Clr}[8]{Miracle}[Magical]
        Once per week, you can request a miracle as a standard action.
        You mentally specify your request, and your deity fulfills that request in the manner it sees fit.
        This can emulate the effects of any spell or ritual, or have any other effect of a similar power level.
        If the deity has a direct interest in your situation, the miracle may be of even greater power.

        If you perform an extraordinary service for your deity, you can gain the ability to request an additional miracle that week.

    \subsection{Divine Spell Mastery}
        This archetype improves the divine spells you cast.
        You must be able to cast divine spells to gain the abilities from this archetype.

        \cf{Clr}[1]{Wellspring of Power}[Magical]
        You gain a \plus1 bonus to \glossterm{power} with divine spells.

        \cf{Clr}[2]{Mystic Knowledge}[Magical]
        You gain your choice of one of the following abilities.
        {
            \parhead{Insight} You gain an additional \glossterm{insight point} (see \pcref{Insight Points}).
            \parhead{Rituals} You gain the ability to perform divine rituals to create unique magical effects (see \pcref{Rituals}).
            The maximum level of divine ritual you can learn or perform is equal to the maximum divine spell level you can cast.
            \parhead{Mystic Sphere Access} You gain access to an additional divine \glossterm{mystic sphere}.
                You can choose this ability multiple times, gaining access to an additional \glossterm{mystic sphere} each time.
        }

        \cf{Clr}[3]{Augments}[Magical]
        Choose one \glossterm{augment} (see \pcref{Augments}).
        You can also spend \glossterm{insight points} to learn one additional augment per \glossterm{insight point}.
        You can apply those augments to divine spells you cast and divine rituals you perform.
        When you gain access to a new spell level, you may change which augments you know.

        \cf*{Clr}[4]{Mystic Knowledge}[Magical]
        You gain an additional \textit{mystic knowledge} ability.

        \cf{Clr}[5]{Augment}[Magical]
        You learn an additional \glossterm{augment} that you can apply to divine spells.

        \cf{Clr}[6]{Greater Wellspring of Power}[Magical]
        The bonus from your \textit{wellspring of power} ability increases to \plus2.

        \cf*{Clr}[7]{Mystic Knowledge}[Magical]
        You gain an additional \textit{mystic knowledge} ability.

        \cf{Clr}[8]{Mystic Supremacy}[Magical]
        The maximum spell level you can cast is increased to 8.
        In addition, you learn an additional \glossterm{augment} that you can apply to divine spells.

    \subsection{Cleric Domain Abilities}\label{Cleric Domain Abilities}
        These domain abilities can be granted by the \textit{domain influence} cleric archetype.
        All cleric domain abilities are \glossterm{magical} unless otherwise specified.

        \subsubsection{Air}
            \parhead{Gift} You add the Jump skill to your \glossterm{class skill} list and gain a \plus5 bonus to Jump attacks and checks (see \pcref{Jump}).
            \parhead{Aspect} You gain a \glossterm{glide speed} equal to your \glossterm{base speed} (see \pcref{Gliding}).
            \parhead{Essence} You can use the \textit{speak with air} ability as a standard action.
            \begin{attuneability}{Speak with Air}[\glossterm{Attune} (self)]
                You can speak with and command air within a \areahuge zone from your location.
                You can ask the air simple questions and understand its responses.
                If you command the air to perform a task, it will do so do the best of its ability until this effect ends.
                You cannot compel the air to move faster than 50 mph.

                After you use this ability on a particular area of air, you cannot use it again on that same area for 24 hours.
            \end{attuneability}
            \parhead{Mastery} You gain a \glossterm{fly speed} with good \glossterm{maneuverability} equal to your \glossterm{base speed} (see \pcref{Flying}).

        \subsubsection{Chaos}
            \parhead{Gift} When you roll a 10 on a \glossterm{check} on your first attempt, you gain a \plus5 bonus to the check.
            \parhead{Aspect} If you roll a 1 on an attack roll, it explodes (see \pcref{Exploding Attacks}).
            This does not affect additional dice rolled if the attack roll explodes.
            \parhead{Essence} You can use the \textit{twist of fate} ability as a standard action.
            \begin{apability}{Twist of Fate}
                An improbable event occurs within \rnglong range.
                You can specify in general terms what you want to happen, such as ``Make the bartender leave the bar''.
                You cannot control the exact nature of the event, though it always beneficial for you in some way.
                After using this ability, you cannot use it again for an hour.
            \end{apability}
            \parhead{Mastery} When you make an attack roll, if it misses, you can reroll.
            You must accept the second result.

        \subsubsection{Death}
            \parhead{Gift} When you deal damage to a creature that had vital damage before your attack, it immediately dies.
            This is a \glossterm{Death} effect.
            \parhead{Aspect} You radiate an aura of death within a \arealarge radius emanation.
            Once per round, if a creature in the aura dies, you heal hit points equal to your \glossterm{power}.
            \parhead{Essence} As a standard action, you can spend an \glossterm{action point} to use this ability.
            If you do, you make an attack vs. Fortitude against a creature within \rngmed range.
            \hit The target takes life \glossterm{standard damage} \plus2d.
            This damage is increased by \plus2d if the target is \glossterm{bloodied}.
            \parhead{Mastery} You constantly radiate an aura of death in a \areahuge radius emanation from you.
            If a living enemy in the area takes \glossterm{vital damage}, it immediately dies.

        \subsubsection{Destruction}
            \parhead{Gift} When you deal damage to a target, you first negate an amount of \glossterm{damage reduction} equal to your \glossterm{power}.
            In addition, you negate an amount of \glossterm{hardness} equal to half your \glossterm{power}.
            Hardness and damage reduction negated in this way does not apply against your attack or against any other attacks during the current phase.
            This effect lasts until the end of the round.
            % TODO: upgrade timing is a little weird
            \parhead{Aspect} You can use the \textit{destructive attack} ability as a standard action.
            \begin{apability}{Destructive Attack}
                Make a \glossterm{strike} with a \plus2d bonus to damage.

                \rankline
                \rank{4} The damage bonus increases to \plus3d.
                \rank{6} The damage bonus increases to \plus4d.
                \rank{8} The damage bonus increases to \plus5d.
            \end{apability}
            \parhead{Essence} You can use the \textit{lay waste} ability as a standard action.
            \begin{apability}{Lay Waste}
                You make an attack vs. Fortitude against all unattended objects in a \arealarge radius.
                You may freely exclude any number of 5-ft\. cubes from the area, as long as the resulting area is still contiguous.
                \hit For each target, if its \glossterm{hardness} is lower than your \glossterm{power}, it crumbles into a fine power and is irreparably \glossterm{broken}.
            \end{apability}
            \parhead{Mastery} You gain a \plus1d bonus to damage with \glossterm{strikes}.

        \subsubsection{Earth}
            \parhead{Gift} You gain a \plus2 bonus to Fortitude defense.
            \parhead{Aspect} You gain \glossterm{damage reduction} equal to your \glossterm{power} against damage from \glossterm{physical attacks}.
            \parhead{Essence} You can use the \textit{speak with earth} ability as a standard action.
            \begin{attuneability}{Speak with Earth}[\glossterm{Attune} (self)]
                You can speak with and command earth within a \areahuge zone from your location.
                You can ask the earth simple questions and understand its responses.
                If you command the earth to perform a task, it will do so do the best of its ability until this effect ends.
                You cannot compel the earth to move faster than 10 feet per round.

                After you use this ability on a particular area of earth, you cannot use it again on that same area for 24 hours.
            \end{attuneability}
            \parhead{Mastery} Your damage reduction from this domain's aspect protects you against all damage, rather than only against damage from physical attacks.
            In addition, you gain a \plus1 bonus to all defenses as long as you are on solid ground.

        \subsubsection{Evil}
            \parhead{Gift} At the start of each phase, you may choose an adjacent willing creature.
            If you do, that creature takes half of all damage you take that phase (rounded down) instead of you.
            Any abilities it has that would reduce or negate the effects of the attack have no effect.
            You take the remaining half of the damage, and suffer any non-damaging effects of all attacks normally.
            \parhead{Aspect} You can use this domain's domain gift to target any willing creature within \rngclose range.
            \parhead{Essence} You can use the \textit{compel evil} ability as a standard action.
            \begin{apability}{Compel Evil}[\glossterm{Compulsion}, \glossterm{Mind}]
                Make an attack vs. Mental against a creature within \rngmed range.
                Creatures who have strict codes prohibiting them from taking evil actions, such as paladins devoted to Good, are immune to this ability.
                \hit The target takes an evil action as soon as it can.
                You have no control over the act the creature takes, but circumstances can make the target more likely to take an action you desire.
            \end{apability}
            \parhead{Mastery} When you use your Evil domain gift, you can spend an \glossterm{action point}.
            If you do, you can redirect your damage to an unwilling creature that phase.
            You cannot use this ability to redirect damage to the same unwilling creature more than once between \glossterm{short rests}.

        \subsubsection{Fire}
            \parhead{Gift} You add the \spell{pyromancy} spell to your divine \glossterm{spell list}.
            \parhead{Aspect} When you would take fire damage, you heal half that many hit points instead.
            % TODO: define those terms?
            This applies before any damage reduction and damage immunity abilities you have.
            \parhead{Essence} You can use the \textit{speak with fire} ability as a standard action.
            \begin{attuneability}{Speak with Fire}[\glossterm{Attune} (self)]
                You can speak with and command fire within a \areahuge zone from your location.
                You can ask the fire simple questions and understand its responses.
                If you command the fire to perform a task, it will do so do the best of its ability until this effect ends.
                You cannot compel the fire to move farther than 30 feet in a single round.
                Fire that ends the round on non-combustable materials usually goes out, depending on the circumstances.

                This effect lasts as long as you \glossterm{sustain} it as a \glossterm{minor action}.
                After you use this ability on a particular area of fire, you cannot use it again on that same area for 24 hours.
                % TODO: What does an ``area of fire'' mean?
            \end{attuneability}
            \parhead{Mastery} When you deal fire damage to a creature, that creature becomes \ignited as a \glossterm{condition}.

        \subsubsection{Good}
            \parhead{Gift} At the start of each phase, you may choose an adjacent willing creature.
            If you do, you take half of all damage that creature would take that phase (rounded down) instead of the creature.
            Any abilities you have that would reduce or negate the effects of the attack have no effect.
            The protected creature takes the remaining half of the damage, and suffers any non-damaging effects of attacks normally.
            \parhead{Aspect} You can use this domain's domain gift to target any willing creature within \rngclose range.
            \parhead{Essence} You can use the \textit{compel good} ability as a standard action.
            \begin{apability}{Compel Good}[\glossterm{Compulsion}, \glossterm{Mind}]
                Make an attack vs. Mental against a creature within \rngmed range.
                Creatures who have strict codes prohibiting them from taking evil actions, such as paladins devoted to Good, are immune to this ability.
                \hit The target takes an good action as soon as it can.
                You have no control over the act the creature takes, but circumstances can make the target more likely to take an action you desire.
            \end{apability}
            \parhead{Mastery} When you use your Good domain gift, you can redirect all damage to you instead of only half.

        \subsubsection{Knowledge}
            \parhead{Gift} You add all Knowledge skills to your cleric \glossterm{class skill} list.
            In addition, you gain two skill points.
            \parhead{Aspect} Your extensive knowledge of all methods of attack and defense grants you a \plus1 bonus to all defenses.
            \parhead{Essence} You can use the \textit{share knowledge} ability as a standard action.
            \begin{apability}{Share Knowledge}
                You make a Knowledge check of any kind with a bonus equal to your \glossterm{power}.
                You and all willing creatures within a \arealarge radius learn the results of your check.
                Creatures believe the information gained in this way to be true as if they it had seen it with their own eyes.

                You cannot alter the knowledge you gain with this check in any way, such as by adding or withholding information.
            \end{apability}
            \parhead{Mastery} You gain a \plus1 bonus to accuracy with all attacks.

        \subsubsection{Law}
            \parhead{Gift} You gain a \plus2 bonus to Mental defense.
            % Clarify - does this apply to exploding dice?
            \parhead{Aspect} When you roll a 1 on an \glossterm{attack roll}, it is treated as if you had rolled a 6.
            \parhead{Essence} You can use the \textit{compel law} ability as a standard action.
            \begin{apability}{Compel Law}[\glossterm{Compulsion}, \glossterm{Mind}]
                Make an attack vs. Mental against all creatures within a \arealarge radius.
                \hit Each target is unable to break the laws that apply in the area, and any attempt to do so simply fails.
                The laws which are applied are those which are most appropriate for the area, regardless of whether you or any other creature know those laws.

                % Sufficiently clear that this isn't part of the hit effect?
                When you use this ability, you also gain the condition.
                If this condition is removed from you, it is also removed from all other affected creatures.
                In areas under ambiguous or nonexistent government, this ability may have unexpected effects, or it may have no effect at all.
            \end{apability}
            \parhead{Mastery} When you roll less than a 5 on an \glossterm{attack roll}, it is treated as if you had rolled a 5.

        \subsubsection{Life}
            \parhead{Gift} You gain additional hit points equal to your \glossterm{power}.
                In addition, you gain a \plus1 bonus to \glossterm{power} with all abilities you use that heal hit points.
            % TODO: wording?
            \parhead{Aspect} The bonus to \glossterm{power} from your Life domain gift increases to 2.
            \parhead{Essence} You can use the \textit{revivify} ability as a standard action.
            \begin{apability}{Revivify}
                Choose a dead creature adjacent to you.
                If it was dead for no more than 5 minutes, it is restored to life, as the \ritual{resurrection} ritual.
                When the target is returned to life in this way, you expend all of your remaining \glossterm{action points}.
            \end{apability}
            % Need special text to clarify that it is affected by the Aspect?
            \parhead{Mastery} At the end of each \glossterm{action phase}, you heal hit points equal to your \glossterm{power}.

        \subsubsection{Magic}
            \parhead{Gift} You learn an additional \glossterm{spell} for a divine \glossterm{mystic sphere} you have access to.
            \parhead{Aspect} You gain access to an additional divine \glossterm{mystic sphere}.
            \parhead{Essence} Whenever you cast a spell, you heal hit points equal to your \glossterm{power}.
            \parhead{Mastery} Whenever you are affected by a \glossterm{magical} ability, you heal hit points equal to your \glossterm{power}.
            This healing applies even if the spell has an attack that fails to beat your defenses.
            It stacks with the healing from this domain's essence if you target yourself with a spell.

        \subsubsection{Strength}
            \parhead{Gift} You add Climb, Jump, and Swim to your cleric \glossterm{class skill} list.
            In addition, you gain two skill points.
            \parhead{Aspect} You reduce the \glossterm{encumbrance} of body armor you wear by 1.
            \parhead{Essence} You gain a \plus5 bonus to Strength for the purpose of checks and determining your carrying capacity.
            \parhead{Mastery} You gain a \plus1 bonus to your starting Strength.

        \subsubsection{Travel}
            \parhead{Gift} You add Knowledge (geography), Survival, and Swim to your cleric \glossterm{class skill} list.
            In addition, you gain two skill points.
            \parhead{Aspect} You gain a \plus20 foot \glossterm{magic bonus} to your \glossterm{base speed}, up to a maximum of double your normal speed.
            \parhead{Essence} You can use the \textit{dimensional jaunt} ability as a standard action.
            \begin{apability}{Dimensional Jaunt}[\glossterm{Teleportation}]
                You teleport up to 1 mile in any direction.
                You do not need \glossterm{line of sight} or \glossterm{line of effect} to your destination, but you must be able to clearly visualize it.
            \end{apability}
            \parhead{Mastery} When you move, you can teleport the same distance instead.
            This does not change the total distance you can move, but you can teleport in any direction, including vertically.

            You can even attempt to move to locations outside of \glossterm{line of sight} and \glossterm{line of effect}, up to the limit of your remaining movement speed.
            If your intended destination is invalid, the distance you tried to teleport is taken from your remaining movement, but you suffer no other ill effects.

        \subsubsection{Trickery}
            \parhead{Gift} You add Bluff, Disguise, and Stealth to your cleric \glossterm{class skill} list.
            In addition, you gain two skill points.
            \parhead{Aspect} You gain a \plus2 bonus to Bluff, Disguise, and Stealth.
            \parhead{Essence} You can use the \textit{compel belief} ability as a standard action.
            \begin{apability}{Compel Belief}[\glossterm{Compulsion}, \glossterm{Mind}, \glossterm{Sustain} (minor)]
                Make an attack vs. Mental against a creature within \rngmed range.
                You must also choose a belief that the target has.
                The belief may be a lie that you told it, or even a simple misunderstanding (such as believing a hidden creature is not present in a room).
                If the creature does not already hold the chosen belief, this ability automatically fails.
                \hit The target continues to maintain the chosen belief, regardless of any evidence to the contrary.
                It will interpret any evidence that the falsehood is incorrect to be somehow wrong -- an illusion, a conspiracy to decieve it, or any other reason it can think of to continue believing the falsehood.
                At the end of the effect, the creature can decide whether it believes the falsehood or not, as normal.
            \end{apability}
            \parhead{Mastery} You are undetectable by Divination spells and effects.
            They cannot detect your presence, sounds you make, or any actions you take.

        \subsubsection{War}
            \parhead{Gift} You gain proficiency with heavy armor, tower shields, and an additional weapon group of your choice.
            \parhead{Aspect} You gain a \plus1 bonus to \glossterm{accuracy} with \glossterm{physical attacks}.
            \parhead{Essence} You can use the \textit{battlefield magic} ability as a \glossterm{minor action}.
            \begin{apability}{Battlefield Magic}[Swift]
                % TODO: wording
                One spell you cast this phase gains one of the following effects when it resolves:
                \begin{itemize}
                    \item Legion: If the spell would normally affect five or more specific targets, it instead affects five times that many targets.
                    \item Selective: If the spell has an area, you may freely exclude any areas from the spell's effect.
                    All squares in the final area of the spell must be contiguous.
                    You cannot create split a spell's area into multiple completely separate areas.
                    \item Widened: If the spell has an area, the size of the area is doubled.
                \end{itemize}
                You cannot use this ability on spells with the \glossterm{Attune} tag.
            \end{apability}
            \parhead{Mastery} You gain a \plus1 bonus to \glossterm{accuracy} with \glossterm{physical attacks}.

        \subsubsection{Water}
            \parhead{Gift} You add Swim to your cleric \glossterm{class skill} list and gain a \plus5 bonus to Swim attacks and checks.
            \parhead{Aspect} You can breathe water as easily as a human breathes air, preventing you from drowning or suffocating underwater.
            You also gain a \glossterm{swim speed} equal to your \glossterm{base speed}.
            \parhead{Essence} You can use the \textit{speak with water} ability as a standard action.
            \begin{attuneability}{Speak with Water}[\glossterm{Attune} (self)]
                You can speak with and command water within a \areahuge zone from your location.
                You can ask the water simple questions and understand its responses.
                If you command the water to perform a task, it will do so do the best of its ability until this effect ends.
                You cannot compel the water to move faster than 30 feet per round.

                After you use this ability on a particular area of water, you cannot use it again on that same area for 24 hours.
            \end{attuneability}
            \parhead{Mastery}
            When you move, you can transform yourself into a rushing flow of water with a volume roughly equal to your normal volume until your movement is complete.
            In this form, you may move wherever water could go, you cannot take other actions, such as jumping, attacking, or casting spells.
            You may move through squares occupied by enemies without penalty.
            \par Your speed is halved when moving uphill and doubled when moving downhill.
            Unusually steep inclines may cause greater movement differences while in this form.
            \par If the water is split, you may reform from anywhere the water has reached, to as little as a single ounce of water.
            If not even an ounce of water exists contiguously, your body reforms from all of the largest available sections of water, cut into pieces of appropriate size.
            This usually causes you to die.

        \subsubsection{Wild}
            \parhead{Gift} You add Creature Handling, Knowledge (nature), and Survival to your cleric \glossterm{class skill} list.
            In addition, you gain two skill points.
            % Does this even make sense?
            \parhead{Aspect} When you gain this ability, you choose one wild aspect ability, as if you were a druid of a level equal to your cleric level (see Wild Aspect, \pref{Drd:Wild Aspect}).
            As a standard action, you can spend an \glossterm{action point} to embody that wild aspect for 1 hour.
            % TODO: essence, mastery

        \subsubsection{Ex-Clerics}
            If you grossly violate the code of conduct required by your deity, you lose all spells and magical cleric class abilities.
            You cannot regain those abilities until you atone for your transgressions to your deity.

\newpage
\section{Druid}\label{Druid}
    \begin{dtable}
        \lcaption{Druid Progression}
        \begin{dtabularx}{\columnwidth}{c >{\lcol}X >{\lcol}X >{\lcol}X}
            \tb{Rank} & \tb{Nature Spellcasting} & \tb{Nature Spell Mastery}  & \tb{Natural Influence} \\\bottomrule
               1 & Mystic spheres         & Wellspring of magic         & Wild speech
            \\ 2 & Spells                 & Mystic knowledge            & Wild aspect
            \\ 3 & Spell level (2)        & Augments                    & Natural lore
            \\ 4 & Spell level (3), spell & Mystic knowledge            & Wild aspect
            \\ 5 & Spell level (4)        & Augment                     & Nature's champion
            \\ 6 & Spell level (5), spell & Greater wellspring of magic & Wild aspect
            \\ 7 & Spell level (6)        & Mystic knowledge            & Fluid aspect
            \\ 8 & Spell level (7), spell & Mystic supremacy            & Avatar of nature
        \end{dtabularx}
    \end{dtable}

    \classbasics{Alignment} Neutral good, lawful neutral, neutral, chaotic neutral, or neutral evil.

    \classbasics{Archetypes} Druids have the Natural Influence, Nature Spellcasting, and Nature Spell Mastery \glossterm{archetypes}.

    \subsection{Basic Class Abilities}
        If you are a druid, you gain the following abilities.

        \cf{Drd}{Defenses}
        You gain the following bonuses to your \glossterm{defenses}: \plus5 Fortitude, \plus3 Reflex, \plus4 Mental.

        \cf{Drd}{Skills}
        You have the following \glossterm{class skills}:
        \begin{itemize}
            \item \subparhead{Strength} Climb, Jump, Swim.
            \item \subparhead{Dexterity} Acrobatics, Ride, Stealth.
            \item \subparhead{Intelligence} Craft, Deduction, Heal, Knowledge (geography, nature).
            \item \subparhead{Perception} Awareness, Creature Handling, Survival.
            \item \subparhead{Other} Bluff, Intimidate, Persuasion, Profession.
        \end{itemize}

        \cf{Drd}{Weapon and Armor Proficiencies}
        Druids are proficient with simple weapons, any one other weapon group, scimitars, sickles, and slings.
        In addition, druids are proficient with light armor, medium armor, and shields.
        However, a druid cannot use metal armor; see the Metal Abhorrence ability, below.

        \cf{Drd}{Druidic Language}
        You know Druidic, a secret language known only to druids, in addition to your normal languages.
        Druids are forbidden to teach this language to nondruids.
        Druidic has its own alphabet.

        \cf{Drd}{Metal Abhorrence}
        The oaths that you swear as part of your druidic initiation prohibit you from wearing armor made of metal.
        If you wear prohibited armor or carry a prohibited shield, you are unable to cast druid spells or use any of your \glossterm{magical} druid abilities while doing so and for 24 hours thereafter.

        You can avoid this penalty by using armor made of wood altered with the \ritual{ironwood} ritual.
        Such wood is as strong as steel.

    \subsection{Natural Influence}
        This archetype grants you influence over the natural world, and the ability to embody aspects the natural world in your own form.

        \cf{Drd}[1]{Wild Speech}[Magical] You can use the \textit{wild speech} ability as a standard action.
        \begin{attuneability}{Wild Speech}[\glossterm{Attune} (self)]
            Choose an animal within \rnglong range.
            You can speak to and understand the speech of the target animal, and any other animals of the same species.

            This ability does not make the target any more friendly or cooperative than normal.
            Wary and cunning animals are likely to be terse and evasive, while stupid ones tend to make inane comments and are unlikely to say or understand anything of use.
        \end{attuneability}

        \cf{Drd}[2]{Wild Aspect}[Magical]
        You gain the ability to embody an aspect of an animal or of nature itself.
        Choose a single wild aspect from the list below.
        You can also spend \glossterm{insight points} to learn one additional \textit{wild aspect} per \glossterm{insight point}.

        As a \glossterm{standard action}, you can gain the effects of one wild aspect that you know.
        That effect lasts until you activate a different wild aspect you know.

        The abilities in the list below describe the effects of the aspect.
        Your appearance also changes to match the aspect's effects, but the nature of this change is not described.
        Different druids change in different ways.
        For example, one druid might grow brown fur when using the Form of the Bear, while another might instead change their face to become broader and more bear-shaped when embodying the same aspect.
        You choose how your appearance changes when you gain a wild aspect.
        This change cannot be used to gain an additional substantive benefit beyond the effects given in the description of the aspect.

        Many wild aspects grant natural weapons.
        See \pcref{Natural Weapons}, for details about natural weapons.

        {
            \begin{freeability}{Form of the Bear}
                You gain a \plus1 bonus to Fortitude defense.
                In addition, you mouth and hands transform, granting you bite and claw \glossterm{natural weapons}.
                The bite deals \plus0d damage, and the claws deal \minus1d damage.

                \rankline
                \rank{4} The Fortitude bonus increases to \plus2.
                \rank{6} You gain a \plus1d bonus to damage with the natural weapons.
                \rank{8} The Fortitude bonus increases to \plus4.
            \end{freeability}

            \begin{freeability}{Form of the Bull}
                You gain a \plus1 bonus to \glossterm{accuracy} with the \textit{shove} ability (see \pcref{Shove}).
                In addition, your head transforms, granting you a gore \glossterm{natural weapon}.
                The weapon deals \plus0d damage, and has the Forceful weapon tag (see \pcref{Weapon Tags}).

                \rankline
                \rank{4} The accuracy bonus increases to \plus2.
                \rank{6} You gain a \plus1d bonus to damage with the natural weapon.
                \rank{8} The accuracy bonus increases to \plus4.
            \end{freeability}

            \begin{freeability}{Form of the Constrictor}
                You gain a \plus1 bonus to \glossterm{accuracy} with the Grapple ability (see \pcref{Grapple}).
                In addition, you gain a constrict \glossterm{natural weapon}.
                This weapon deals \plus1d damage, and it has the Grappling weapon tag (see \pcref{Weapon Tags}).
                It can only be used against a foe you are grappling with.
                \rankline
                \rank{4} The accuracy bonus increases to \plus2.
                \rank{6} You gain a \plus1d bonus to damage with the natural weapon.
                \rank{8} The accuracy bonus increases to \plus4.
            \end{freeability}

            \begin{freeability}{Form of the Fish}
                You gain a \glossterm{swim speed} equal to your \glossterm{base speed}.
                In addition, you gain a bite \glossterm{natural weapon} that deals \plus0d damage.

                \rankline
                \rank{4} You can breathe water as easily as a human breathes air, preventing you from drowning or suffocating underwater.
                \rank{6} You suffer no penalties for acting underwater.
                \rank{8} You are immune to \glossterm{magical} effects that restrict your mobility.
                In addition, you gain a \plus5 bonus to defenses against the \textit{grapple} ability (see \pcref{Grapple}).
            \end{freeability}

            \begin{freeability}{Form of the Hawk}
                You gain \glossterm{low-light vision}.
                If you already have low-light vision, you double its benefit, allowing you to treat sources of light as if they had four times their normal illumination range.
                In addition, your feet transform, granting you a talon \glossterm{natural weapon}.
                The weapon deals \minus1d damage.

                \rankline
                \rank{4} You grow wings, granting your a glide speed equal to your \glossterm{base speed} (see \pcref{Gliding}).
                \rank{6} You gain a \plus1d bonus to damage with the natural weapon.
                \rank{8} The glide speed is replaced with a \glossterm{fly speed} equal to your \glossterm{base speed} (see \pcref{Flying}).
            \end{freeability}

            \begin{freeability}{Form of the Hound}
                You gain the \glossterm{scent} ability.
                In addition, you gain a bite \glossterm{natural weapon} that deals \plus0d damage.

                \rankline
                \rank{4} You gain the ability to move on all four limbs.
                When doing so, you gain a \plus20 foot bonus to your land speed, up to a maximum of double your original speed.
                When not using your hands to move, your ability to use your hands is unchanged.
                Descending to four legs and rising up to stand on two legs again does not take an action.
                \rank{6} You gain a \plus1d bonus to damage with the natural weapon.
                \rank{8} You gain an additional \plus10 bonus to scent-based Awareness checks (see \pcref{Awareness}).
            \end{freeability}

            % Seems boring? What abilities would make sense?
            \begin{freeability}{Form of the Monkey}
                You gain a \glossterm{climb speed} equal to your \glossterm{base speed}.
                In addition, you gain a bite \glossterm{natural weapon} that deals \plus0d damage.

                \rankline
                \rank{4} You gain a \plus1 bonus to your Reflex defense.
                \rank{6} You gain a \plus1d bonus to damage with the natural weapon.
                \rank{8} The defense bonus increases to \plus2.
            \end{freeability}

            \begin{freeability}{Form of the Mouse}
                You gain a \plus2 bonus to the Escape Artist and Stealth skills.
                In addition, you gain a bite \glossterm{natural weapon} that deals \plus0d damage.
                
                \rankline
                \rank{4} When you use this wild aspect, you can choose to shrink by one \glossterm{size category}.
                \rank{6} The skill bonuses increases to \plus3.
                \rank{8} The skill bonuses increases to \plus4. In addition, when you use this wild aspect, you can choose to shrink by up to two \glossterm{size categories} instead of only one.
            \end{freeability}

            \begin{freeability}{Form of the Oak}
                You move at half speed.
                In exchange, you gain a \plus1 bonus to Armor defense and Fortitude defense.
                \rankline
                \rank{4} You gain \glossterm{damage reduction} against \glossterm{physical damage} equal to half your \glossterm{power}.
                \rank{6} The damage reduction increases to be equal to your \glossterm{power}.
                \rank{8} The defense bonuses increase to \plus2.
            \end{freeability}

            \begin{freeability}{Form of the Viper}
                You gain a \glossterm{climb speed} equal to half your \glossterm{base speed}.
                You do not need to use your hands to climb in this way.
                In addition, you gain a bite \glossterm{natural weapon} that deals \plus0d damage.

                \rankline
                \rank{4} When a creatures takes damage from your bite \glossterm{natural weapon}, it is poisoned.
                At the end of each \glossterm{action phase} in subsequent rounds, you make an attack vs. Fortitude against the target.
                If you hit, the target is \glossterm{sickened} until it removes the poison.
                The poison is removed if you miss the target on this attack three times.
                \rank{6} You gain a \plus1d bonus to damage with the natural weapon.
                \rank{8} The poison makes the target \glossterm{nauseated} instead of \glossterm{sickened}.
            \end{freeability}

            \begin{freeability}{Form of the Wolf}
                You gain a \plus1 bonus to \glossterm{accuracy} with \glossterm{physical attacks} against \glossterm{overwhelmed} creatures.
                In addition, you gain a bite \glossterm{natural weapon} that deals \plus0d damage.

                \rankline
                \rank{4} You gain the ability to move on all four limbs.
                When doing so, you gain a \plus20 foot bonus to your land speed, up to a maximum of double your original speed.
                When not using your hands to move, your ability to use your hands is unchanged.
                Descending to four legs and rising up to stand on two legs again does not take an action.
                \rank{6} You gain a \plus1d bonus to damage with the natural weapon.
                \rank{8} You increase your \glossterm{overwhelm value} by 1.
            \end{freeability}

            \begin{freeability}{Myriad Form}
                You can use your \glossterm{power} in place of your Disguise skill when making Disguise checks to alter your own appearance.
                \rankline
                \rank{4} When you use this wild aspect, you can choose to grow or shrink by one \glossterm{size category}.
                \rank{6} You can use the \textit{Disguise Creature} ability to disguise yourself as a \glossterm{standard action} (see \pcref{Disguise Creature}).
                \rank{8} When you use this wild aspect, you can choose to grow or shrink by up to two \glossterm{size categories} instead of only one.
            \end{freeability}

            \begin{freeability}{Photosynthesis}
                As long as you are in natural sunlight, you gain a \plus10 foot bonus to your \glossterm{base speed}.
                \rankline
                \rank{4} As long as you are in natural sunlight, you heal hit points equal to half your \glossterm{power} at the end of each \glossterm{action phase}.
                In addition, being in sunlight for 8 hours gives you the same benefit as consuming a full day's worth of food and water, allowing you to survive without eating or drinking.
                \rank{6} The healing increases to be equal to your \glossterm{power}.
                \rank{8} The speed bonus increases to \plus30 feet.
            \end{freeability}
        }

        \cf{Drd}[3]{Natural Lore}
        You gain two extra skill points.

        \cf*{Drd}[4]{Wild Aspect}[Magical]
        You learn an additional \textit{wild aspect}.
        %TODO: wording
        In addition, you gain the ability to maintain the effects of up to two \textit{wild aspects} simultaneously.
        When you activate a new \textit{wild aspect}, you can choose which of your wild aspects remain active. 

        \cf{Drd}[5]{Nature's Champion}
        You gain a \plus2 bonus to the Creature Handling, Heal, Knowledge (geography), Knowledge (nature), Ride, and Survival skills.

        \cf*{Drd}[6]{Wild Aspect}[Magical]
        You learn an additional \textit{wild aspect}.
        In addition, you gain the ability to maintain the effects of up to three \textit{wild aspects} simultaneously.
        When you activate a new \textit{wild aspect}, you can choose which of your wild aspects remain active. 

        \cf{Drd}[7]{Fluid Aspect}
        % TODO: wording; this just replaces the action time, not gives you an extra one active
        You can activate a \textit{wild aspect} as a \glossterm{minor action} instead of as a \glossterm{standard action}.

        \cf{Drd}[8]{Avatar of Nature}[Magical]
        If you die, except if by old age, you may choose to have your body and soul become an instrument of nature's will.
        Your body immediately decomposes or otherwise disappears, and your soul does not travel to an afterlife.
        You has no physical form, and cannot use any of your normal abilities.
        Instead, you have a \glossterm{fly speed} of 100 feet, with special maneuverability.
        As a standard action, you can temporarily possess any living plants or animals within a 10 mile radius of the place of your death.

        While possessing a living plant or animal, you can see through its senses and control its actions completely.
        In addition, you may cast spells, and the spells take effect as if the plant or animal had cast them.
        You use the plant or animal's position to determine range, visible targets, and so on.
        You do not require verbal or somatic components to cast your spells in this form.
        % TODO: are there any spells or rituals that have material components or focus objects?
        % but are unable to cast spells or perform rituals that require material components or focus objects.

        While not possessing a plant or animal, you can rest, or you can focus on reincarnating your physical form.
        Creating a new body in this way takes 12 consecutive hours of concentration.
        At the end of that time, you are reincarnated in a new body in your location, as the effect of the \ritual{reincarnation} ritual, except that you can choose your species from among the species listed (not including the ``Other'' species).

        While you are an avatar of nature, you do not age and you cannot die of old age.
        You can continue to exist in this form indefinitely.

    \subsection{Nature Spellcasting}
        This archetype grants you the ability to cast nature spells.
        You must have a starting Perception of at least 1 to gain this archetype.

        \cf{Drd}[1]{Mystic Spheres}[Magical]
        Your connection to nature grants you the ability to use nature magic.
        You gain access to two nature \glossterm{mystic spheres} (see \pcref{Nature Mystic Spheres}).
        As a \glossterm{standard action}, you can cast the \glossterm{cantrip} spell from any mystic sphere you have access to.

        For details about mystic spheres and casting spells, see \pcref{Spell and Ritual Mechanics}.

        \cf{Drd}[2]{Spells} You learn two nature \glossterm{spells}.
        You can also spend \glossterm{insight points} to learn one additional nature spell per \glossterm{insight point}.
        You can learn any 1st level spells from the nature \glossterm{mystic spheres} you have access to.
        As a \glossterm{standard action}, you can spend an \glossterm{action point} to cast any \glossterm{spell} you know.
        When you gain access to a new \glossterm{mystic sphere} or spell level, you can exchange any number of spells you know for spells of other spells you know, including spells of the higher level.

        \cf{Drd}[3]{Spell Level}[Magical] You gain the ability to cast 2nd level nature spells.

        \cf*{Drd}[4]{Spell Level}[Magical] You gain the ability to cast 3rd level nature spells.

        \cf{Drd}[4]{Spell} You gain an additional \glossterm{spell} for any nature \glossterm{mystic sphere} you know.

        \cf*{Drd}[5]{Spell Level}[Magical] You gain the ability to cast 4th level nature spells.

        \cf*{Drd}[6]{Spell Level}[Magical] You gain the ability to cast 5th level nature spells.

        \cf*{Drd}[6]{Spell} You gain an additional \glossterm{spell} for any nature \glossterm{mystic sphere} you know.

        \cf*{Drd}[7]{Spell Level}[Magical] You gain the ability to cast 6th level nature spells.

        \cf*{Drd}[8]{Spell Level}[Magical] You gain the ability to cast 7th level nature spells.

        \cf*{Drd}[8]{Spell} You gain an additional \glossterm{spell} for any nature \glossterm{mystic sphere} you know.

    \subsection{Nature Spell Mastery}
        This archetype improves the nature spells you cast.
        You must be able to cast nature spells to gain the abilities from this archetype.

        \cf{Drd}[1]{Wellspring of Power}[Magical]
        You gain a \plus1 bonus to \glossterm{power} with nature spells.

        \cf{Drd}[2]{Mystic Knowledge}
        You gain your choice of one of the following abilities.
        {
            \parhead{Insight} You gain an additional \glossterm{insight point} (see \pcref{Insight Points}).
            \parhead{Rituals} You gain the ability to perform nature rituals to create unique magical effects (see \pcref{Rituals}).
            \parhead{Mystic Sphere Access} You gain access to an additional divine \glossterm{mystic sphere}.
                You can choose this ability multiple times, gaining access to an additional \glossterm{mystic sphere} each time.
        }

        \cf{Drd}[3]{Augments}[Magical]
        Choose one \glossterm{augments} (see \pcref{Augments}).
        You can also spend \glossterm{insight points} to learn one additional augment per \glossterm{insight point}.
        You can apply those augments to nature spells you cast and nature rituals you perform.
        When you gain access to a new spell level, you may change which augments you know.

        \cf*{Drd}[4]{Mystic Knowledge}[Magical]
        You gain an additional \textit{mystic knowledge} ability.

        \cf{Drd}[5]{Augment}[Magical]
        You learn an additional \glossterm{augment} that you can apply to nature spells.

        \cf{Drd}[6]{Greater Wellspring of Power}[Magical]
        The bonus from your \textit{wellspring of power} ability increases to \plus2.

        \cf*{Drd}[7]{Mystic Knowledge}[Magical]
        You gain an additional \textit{mystic knowledge} ability.

        \cf{Drd}[8]{Mystic Supremacy}[Magical]
        The maximum spell level you can cast is increased to 8.
        In addition, you learn an additional \glossterm{augment} that you can apply to nature spells.

        % \subcf{15th -- Air Mantle}
        % You are surrounded by a mantle of air.
        % Thrown and projectile weapons have a 50\% chance to miss your while this effect is active.
        % Unusually large weapons, such as a giant's boulders, may suffer a decreased miss chance as appropriate to their size.
        % \subcf{15th -- Aqueous Step}
        % Wherever the druid moves, you leave a path of animated water that can grab creatures.
        % When a creature crosses the path, the druid makes a Reflex attack to trip the creature, causing it to fall prone and waste the rest of its movement.
        % Your accuracy is equal to your druid level \add your Constitution.
        % \subcf{15th -- Flaming Step}
        % Wherever the druid moves, you leave a path of burning flame behind your that lasts for 1 round.
        % When a creature crosses the path, the druid makes a Reflex attack to deal damage to the creature.
        % The attack deals 1d8 points of fire damage per two druid levels.
        % Your accuracy is equal to your druid level \add your Constitution.
        % A failed attack deals half damage.
        % \subcf{15th -- Lifegiving Step}
        % Wherever the druid moves, you leave a path of small, living plants that entangle foes for 1 round.
        % When a creature crosses the path, the druid makes a Reflex attack to entangle the creature, causing it to waste the rest of its movement.
        % Your accuracy is equal to your druid level \add your Constitution.
        % The plants appear on any surface, and will continue to grow if they can survive, though they may die quickly if they appear on inhospitable terrain.
        % \subcf{17th -- Flaming Soul}
        % You gain the fire subtype, making your immune to fire but giving your a 50\% vulnerability to cold damage.
        % In addition, when you deal fire damage to a creature, the creature is \ignited for 5 rounds.
        % \subcf{17th -- Sunblessed Rejuvenation}
        % You gain fast healing equal to your druid level as long as you remain in sunlight or touches a plant of your size or larger.
        % \subcf{17th -- Sunscour}
        % This aspect functions like the heart of the sun natural aspect, except that it also suppresses shadow effects and the visual components of illusions within the area of bright light.

        % \subcf{17th -- Water's Flow}
        % As a minor action, the druid can transform herself into a rushing flow of water with a volume roughly equal to your normal volume until the end of your turn.
        % In this form, she may move wherever water could go, but she cannot take other actions, such as jumping, attacking, or casting spells.
        % Your speed is halved when moving uphill and doubled when moving downhill.
        % She may move through squares occupied by enemies without penalty.
        % She may return to your normal form as a free action.
        % \par If the water is split, she may reform from anywhere the water has reached, to as little as a single ounce of water.
        % If not even an ounce of water exists contiguously, your body reforms from the largest available parts of water, cut into pieces of appropriate size.
        % This usually causes the druid to die.

    \subsection{Ex-Druids}
        A druid who ceases to revere nature, changes to a prohibited alignment, or teaches the Druidic language to a nondruid loses all spells and magical druid class abilities.
        She cannot thereafter gain levels as a druid until you atone for your transgressions.

        % \subsection{Variant Druids}

        %     \subsubsection{Blighter}

        %         Blighters draw power from nature, as do other druids. However, while other druids revere nature and draw power from it gently, blighters steal power from nature forcefully. Wherever a blighter goes, destruction and death surely follows.

        %         \altcf{Blight} Instead of meditating to regain spell slots, a blighter draws power from your environment forcefully.
        %         This affects a \areahuge radius zone centered on your, and the process takes 1 minute of concentration.
        %         At the end of every round, every living thing in the area other than the blighter takes damage equal to your nature power.
        %         All inanimate plants of Huge size or smaller immediately wither and die.
        %         The earth becomes cracked and infertile, and any nutrients from the soil are destroyed.
        %         This ability has no effect on artificial environments or materials, such as metal or worked stone.
        %         At the end of the minute, the blighter regains your spent nature spell slots.

        %         A blighter can only blight your surroundings in this way once per hour.
        %         If your surroundings are already blighted or are not natural terrain, she cannot use this ability to regain your spells.
        %         Instead, she must meditate for 8 hours to slowly draw power from your surroundings, as a normal druid.

        %         \altcf{Spells} As normal, except that a blighter adds all Vivimancy arcane spells to your spell list.

        %         \altcf[2]{Wild Speech} As normal, except that a blighter gains a \plus5 bonus to Intimidate against your wild speech targets, and a \minus5 penalty to Persuasion.

        %         \altcf[10]{Blightcasting}

        %         \altcf[20]{Improved Blightcasting}

        % \subsubsection{Rotbringer}

        %     While most druids seek to emulate and interact with animals, rotbringers focus on the power of fungi, decay, and regeneration.

        %     \altcf{Invoke Rot} Instead of meditating to regain spell slots, a rotbringer accelerates the natural forces of decomposition and decay on your environment.
        %     This affects a \areahuge radius zone centered on your, and the process takes 1 minute of concentration.
        %     All organic objects of Huge size or smaller, such as plants and corpses, decompose.
        %     This decomposition kills inanimate, living plants.
        %     All organic objects, regardless of size, are covered with various fungi.
        %     This ability has no effect on artificial environments or materials, such as metal or worked stone.
        %     At the end of the minute, the rotbringer regains your spent nature spell slots.

        %     If the rotbringer decomposes a Huge object with this ability, or a combination of smaller objects equivalent in size to a Huge object, you gain an bonus nature spell slot of your highest available spell level.
        %     This extra spell slot lasts until it is used, or until you regain your spell slots again.

        %     A rotbringer can only invoke rot on your surroundings in this way once per hour.
        %     If your surroundings are already decomposed or are not natural terrain, she cannot use this ability to regain your spells.
        %     Instead, she must meditate for 8 hours to slowly draw power from your surroundings, as a normal druid.

        %     \altcf[2]{Wild Speech} The rotbringer gains the ability to speak with plants at 2nd level.
        %     You gain the ability to speak with animals at 6th level, instead of at 2nd level.

        %     \altcf[3rd]{Wild Aspect} The rotbringer does not gain this ability.

        %     \altcf[3rd]{Rot Spell} The druid learns an additional spell slot and spell known.
        %     The spell must be taken from the following list of spells.
        %     The spell's level cannot exceed half your druid level.
        %     If she already knows a spell from the list at every spell level you ha access to, she may instead learn any nature spell (see \pcref{Nature Spells}).

        %     At 5th level, and every odd level, the druid may learn a new spell.

        %     \begin{dtable}
        %         \begin{dtabularx}{\columnwidth}{l X}
        %             \tb{Spell level} & \tb{Rotbringer Spells} \\
        %             1st & \spell{excrete slime}, \spell{lesser regeneration} \\
        %             2nd & \spell{fungal growth} \\
        %             3rd & \spell{rotburst} \\
        %             4th & \spell{poison} \\
        %             6th & \spell{regeneration} \\
        %             7th & \spell{greater rotburst} \\
        %         \end{dtabularx}
        %     \end{dtable}

        %     \altcf[7]{Fungal Armor} The rotbringer becomes covered in fungus that protects your from attacks. You gain a \plus1 bonus to Armor and Fortitude defense.

        %     This bonus increases by 1 at your 7th druid level, and every 4 druid levels thereafter.

\newpage
\section{Fighter}\label{Fighter}
    \begin{dtable}
        \lcaption{Fighter Progression}
        \begin{dtabularx}{\columnwidth}{c >{\lcol}X >{\lcol}X >{\lcol}X}
            \tb{Rank} & \tb{Combat Discipline} & \tb{Equipment Training}  & \tb{Martial Mastery} \\\bottomrule
               1 & Disciplined Threat          & Armor expertise         & Martial exertions
            \\ 2 & Discipline                  & Weapon focus            & Martial lore
            \\ 3 & Disciplined defense         & Greater armor expertise & Martial exertion
            \\ 4 & Greater disciplined threat  & Weapon master           & Martial lore
            \\ 5 & Greater discipline          & Supreme armor expertise & Martial exertion
            \\ 6 & Greater disciplined defense & Greater weapon focus    & Exertion expertise
            \\ 7 & Legendary discipline        & Armored juggernaut      & Martial exertion
            \\ 8 &                             &
        \end{dtabularx}
    \end{dtable}

    \classbasics{Alignment} Any.

    \classbasics{Archetypes} Fighters have the Martial Mastery, Equipment Training, and Combat Discipline \glossterm{archetypes}.

    \subsection{Basic Class Abilities}
        If you are a fighter, you gain the following abilities.

        \cf{Ftr}{Defenses}
        You gain the following bonuses to your \glossterm{defenses}: \plus5 Fortitude, \plus3 Reflex, \plus4 Mental.

        \cf{Ftr}{Skills}
        You have the following \glossterm{class skills}:
        \begin{itemize}
            \item \subparhead{Strength} Climb, Jump, Swim.
            \item \subparhead{Dexterity} Acrobatics, Escape Artist, Ride.
            \item \subparhead{Intelligence} Craft.
            \item \subparhead{Perception} Awareness.
            \item \subparhead{Other} Bluff, Intimidate, Persuasion, Profession.
        \end{itemize}

        \cf{Ftr}{Weapon and Armor Proficiencies}
        You are proficient with simple weapons, any four other weapon groups, all armor (heavy, medium, and light), and shields.

    \subsection{Martial Mastery}
        This archetype grants you special abilities to use in combat.

        \cf{Ftr}[1]{Masterful Strike} You can use the \textit{masterful strike} ability as a standard action.
        \begin{freeability}{Masterful Strike}
            Make a \glossterm{strike} with a \plus1 bonus to \glossterm{accuracy}.
        \end{freeability}

        \cf{Ftr}[2]{Martial Exertions}
        You can channel your martial prowess into devastating attacks.
        Choose one \textit{martial exertions} from the list below.
        You can also spend \glossterm{insight points} to learn one additional \textit{martial exertion} per \glossterm{insight point}.
        You can use any \textit{martial exertion} ability you know as a \glossterm{standard action}.
        {
            \begin{freeability}{Brace for Impact}[\glossterm{Swift}]
                You take half damage from all attacks.
                This halving is applied before \glossterm{damage reduction} and similar abilities.
                This ability lasts until the end of the round.

                \rankline
                \rank{4} You also gain a \plus1 bonus to all defenses.
                \rank{6} The defense bonus increases to \plus2.
                \rank{8} The defense bonus increases to \plus3.
            \end{freeability}

            \begin{apability}{Certain Strike}
                Make a \glossterm{strike} with a \plus3 bonus to accuracy.

                \rankline
                \rank{4} The accuracy bonus increases to \plus4.
                \rank{6} The accuracy bonus increases to \plus5.
                \rank{8} The accuracy bonus increases to \plus6.
            \end{apability}

            \begin{freeability}{Challenge}[\glossterm{Attune} (self)]
                Make a melee \glossterm{strike}.
                If the strike hits, you gain a \plus4 bonus to \glossterm{threat} against the struck creature.
                This effect lasts until you take a \glossterm{short rest} or until you use this ability on a different creature.

                % TODO: weird scaling
                \rankline
                \rank{4} The threat bonus increases to \plus6.
                \rank{6} The threat bonus increases to \plus8.
                \rank{8} The threat bonus increases to \plus10.
            \end{freeability}

            \begin{apability}{Counterattack}
                Make a \glossterm{strike}.
                If the target attacked you in the same phase, you gain a \plus2 bonus to accuracy and a \plus2d bonus to damage.
                Otherwise, you regain the action point spent to use this ability.

                \rankline
                \rank{4} The damage bonus increases to \plus3d.
                \rank{6} The damage bonus increases to \plus4d.
                \rank{8} The damage bonus increases to \plus5d.
            \end{apability}

            \begin{apability}{Daunting Blow}[\glossterm{Mind}]
                Make a melee \glossterm{strike}.
                If the target takes damage from the strike, it is \glossterm{shaken} by you as a \glossterm{condition}.
                Otherwise, you regain the \glossterm{action point} spent to use this ability.

                \rankline
                \rank{4} You gain a \plus1 bonus to accuracy on the strike.
                \rank{6} The accuracy bonus increases to \plus2.
                \rank{8} The accuracy bonus increases to \plus3.
            \end{apability}

            \begin{apability}{Liver Shot}
                Make a \glossterm{strike} with a bludgeoning weapon.
                If the target takes damage from the strike, it is \sickened as a \glossterm{condition}.
                Otherwise, you regain the \glossterm{action point} spent to use this ability.

                \rankline
                \rank{4} You gain a \plus1 bonus to \glossterm{accuracy} with the strike.
                \rank{6} The accuracy bonus increases to \plus2.
                \rank{8} The accuracy bonus increases to \plus3.
            \end{apability}

            \begin{apability}{Penetrating Strike}
                Make a \glossterm{strike} with a piercing weapon.
                The attack is made against the target's Reflex defense instead of its Armor defense.

                \rankline
                \rank{4} You gain a \plus1d bonus to damage with the strike.
                \rank{6} The damage bonus increases to \plus2d.
                \rank{8} The damage bonus increases to \plus3d.
            \end{apability}

            \begin{apability}{Power Attack}
                Make a \glossterm{strike} with a \plus2d bonus to damage.

                \rankline
                \rank{4} The damage bonus increases to \plus3d.
                \rank{6} The damage bonus increases to \plus4d.
                \rank{8} The damage bonus increases to \plus5d.
            \end{apability}

            \begin{apability}{Pulverizing Smash}
                Make a \glossterm{strike} with a piercing weapon.
                The attack is made against the target's Fortitude defense instead of its Armor defense.

                \rankline
                \rank{4} You gain a \plus1d bonus to damage with the strike.
                \rank{6} The damage bonus increases to \plus2d.
                \rank{8} The damage bonus increases to \plus3d.
            \end{apability}

            \begin{apability}{Rally the Troops}
                You and any number of willing creatures within a \areamed radius from you can each remove one \glossterm{condition}.

                \rankline
                \rank{4} The area increases to \arealarge.
                \rank{6} Each target can instead remove two conditions.
                \rank{8} Each target can instead remove three conditions.
            \end{apability}

            \begin{apability}{Rapid Assault}
                Make a \glossterm{strike} against a creature.
                If you use this ability during the \glossterm{action phase}, you can make another strike during the \glossterm{delayed action phase}.
                You take a \minus2 penalty to accuracy on both strikes.

                \rankline
                \rank{4} The accuracy penalty is reduced to \minus1.
                \rank{6} The accuracy penalty is removed.
                \rank{8} You gain a \plus1 bonus to accuracy with both strikes.
            \end{apability}

            \begin{apability}{Reaping Charge}
                Move up to your movement speed in a straight line.
                You can make a melee \glossterm{strike} with a slashing or bludgeoning weapon.
                The strike targets any number of creatures and objects that you \glossterm{threaten} at any point during your movement, except for the space you start in and the space you end in.
                You take a \minus2 penalty to \glossterm{accuracy} on the strike.

                \rankline
                \rank{4} You gain a \plus1d bonus to damage with the strike.
                \rank{6} The damage bonus increases to \plus2d.
                \rank{8} The damage bonus increases to \plus3d.
            \end{apability}

            \begin{apability}{Strip the Flesh}
                Make a \glossterm{strike} with a slashing weapon.
                At the end of the \glossterm{action phase} of the next round, if you hit with the strike and the target is not \glossterm{bloodied}, it takes additional damage equal to the damage you dealt with the strike.

                \rankline
                \rank{4} If you hit with the strike, the target continues taking the same damage at the end of each \glossterm{action phase} until it becomes \glossterm{bloodied}.
                This is a \glossterm{condition}, and can be removed by abilities that remove conditions.
                \rank{6} You gain a \plus1d bonus to damage with the strike.
                \rank{8} The damage bonus increases to \plus2d.
            \end{apability}

            \begin{apability}{Sweeping Strike}
                Make a melee \glossterm{strike} with a slashing or bludgeoning weapon.
                The strike targets each of up to three creatures or objects you \glossterm{threaten}.
                You take a \minus1 penalty to \glossterm{accuracy} with the strike.

                \rankline
                \rank{4} You gain a \plus1d bonus to damage with the strike.
                \rank{6} The damage bonus increases to \plus2d.
                \rank{8} The damage bonus increases to \plus3d.
            \end{apability}

            \begin{apability}{Whirlwind Spin}
                Make a melee \glossterm{strike} with a slashing weapon.
                The strike targets all creatures you \glossterm{threaten}.
                You take a \minus2 penalty to \glossterm{accuracy} with the strike.

                \rankline
                \rank{4} You gain a \plus1d bonus to damage with the strike.
                \rank{6} The damage bonus increases to \plus2d.
                \rank{8} The damage bonus increases to \plus3d.
            \end{apability}
        }

        \cf{Ftr}[3]{Martial Lore} You gain two additional skill points.

        \cf*{Ftr}[4]{Martial Exertion}
        You learn an additional \textit{martial exertion}.

        \cf{Ftr}[5]{Greater Masterful Strike} The accuracy bonus from your \textit{masterful strike} ability increases to \plus2.

        \cf*{Ftr}[6]{Martial Exertion} 
        You learn an additional \textit{martial exertion}.

        \cf{Ftr}[7]{Master of Arms} You gain a \plus1d bonus to \glossterm{damage} with \glossterm{strikes}.

        \cf*{Ftr}[8]{Martial Exertion}
        You learn an additional \textit{martial exertion}.

    \subsection{Equipment Training}
        This archetype improves your combat prowess with weapons and armor.

        \cf{Ftr}[1]{Armor Expertise}
        You gain a \plus1 bonus to Armor defense while wearing body armor.
        In addition, you reduce the \glossterm{encumbrance} of body armor you wear by 1.

        \cf{Ftr}[2]{Weapon Training} You can use the \textit{weapon training} ability by spending an hour training with a weapon.
        \begin{freeability}{Weapon Training}
            Choose a \glossterm{weapon group} that the weapon you trained with belongs to.
            If you were already proficient with that weapon group, you gain proficiency with \glossterm{exotic weapons} from that weapon group.
            Otherwise, you gain proficiency with that weapon group.

            This ability's effect lasts until you use this ability again.
        \end{freeability}

        \cf{Ftr}[3]{Greater Armor Expertise}
        The \glossterm{encumbrance} reduction from your \textit{armor expertise} ability increases to 2.
        In addition, you treat body armor were one encumbrance category lighter than normal when doing so would be beneficial for you.

        \cf{Ftr}[4]{Weapon Expertise} You gain a \plus1 bonus to \glossterm{accuracy} with \glossterm{physical attacks}.

        \cf{Ftr}[5]{Supreme Armor Expertise}
        The \glossterm{encumbrance} reduction from your \textit{armor expertise} ability increases to 3.
        In addition, you treat body armor as if it were an additional encumbrance category lighter than normal when doing so would be beneficial for you.

        % Too weak?
        \cf{Ftr}[6]{Greater Weapon Training} You can use your \textit{weapon focus} ability as a \glossterm{standard action}.

        \cf{Ftr}[7]{Armored Juggernaut}
        As long as you are wearing body armor, you gain a \plus1 bonus to all defenses.

    \subsection{Combat Discipline}
        This archetype allows you to control your \glossterm{threat} and improves your defenses, especially against special abilities.

        \cf{Ftr}[1]{Disciplined Threat}
        You know how to control your actions to appear more or less threatening, as you choose.
        As a \glossterm{minor action}, you can choose to enter a threatening stance, enter a nonthreatening stance, or act normally.
        If you enter a threatening stance, you gain a \plus2 bonus to your \glossterm{threat}.
        If you enter a nonthreatening stance, you take a \minus2 penalty to your \glossterm{threat}.

        \cf{Ftr}[2]{Discipline} You can use the \textit{discipline} ability as a \glossterm{minor action}.
        \begin{apability}{Discipline}[Swift]
            Remove one \glossterm{condition} affecting you.
            Because this ability has the \glossterm{Swift} tag, the penalties from the condition do not affect you during the current phase.
        \end{apability}

        \cf{Ftr}[3]{Disciplined Defense}
        You gain a \plus1 bonus to Armor defense.

        \cf{Ftr}[4]{Greater Disciplined Threat}
        You gain more control over how threatening you appear.
        When you use your \textit{disciplined threat} ability, you can change your threat by any amount, up to a maximum bonus of \plus4 and a maximum penalty of \minus4.

        \cf{Ftr}[5]{Greater Discipline}
        You can remove an additional condition with your \textit{discipline} ability.

        \cf{Ftr}[6]{Greater Disciplined Defense}
        The defense bonus from your \textit{disciplined defense} ability increases to \plus2.

        \cf{Ftr}[7]{Supreme Disciplined Threat}
        When you use your \textit{disciplined threat} ability, the maximum bonus or penalty you can apply is increased to \plus6 or \minus6.

        \cf{Ftr}[8]{Legendary Discipline}
        You can use your \textit{discipline} ability without spending an \glossterm{action point}.

\newpage
\section{Mage}\label{Mage}
    \begin{dtable}
        \lcaption{Mage Progression}
        \begin{dtabularx}{\columnwidth}{c >{\lcol}X >{\lcol}X >{\lcol}X}
            \tb{Rank} & \tb{Arcane Spellcasting} & \tb{Arcane Spell Mastery}  & \tb{Arcane Lore} \\\bottomrule
            1 & Mystic spheres            & Wellspring of magic         & Mage armor
            \\ 2 & Spells                 & Mystic knowledge            & Arcane insights
            \\ 3 & Spell level (2)        & Augments                    & Essence lore
            \\ 4 & Spell level (3), spell & Mystic knowledge            & Arcane insight
            \\ 5 & Spell level (4)        & Augment                     & Greater essence lore
            \\ 6 & Spell level (5), spell & Greater wellspring of magic & Arcane insight
            \\ 7 & Spell level (6)        & Mystic knowledge            & Supreme essence lore
            \\ 8 & Spell level (7), spell & Mystic supremacy            & Archmage
        \end{dtabularx}
    \end{dtable}

    \classbasics{Alignment} Any.

    \classbasics{Archetypes} Mages have the Arcane Spellcasting, Arcane Lore, and Arcane Spell Mastery \glossterm{archetypes}.

    \subsection{Basic Class Abilities}
        If you are a mage, you gain the following abilities.

        \cf{Mge}{Defenses}
        You gain the following bonuses to your \glossterm{defenses}: \plus3 Fortitude, \plus4 Reflex, \plus5 Mental.

        \cf{Mge}{Skills}
        You have the following \glossterm{class skills}:
        \begin{itemize}
            \item \subparhead{Intelligence} Craft, Deduction, Knowledge (all kinds, taken individually), Linguistics.
            \item \subparhead{Perception} Awareness, Spellcraft.
            \item \subparhead{Other} Bluff, Intimidate, Persuasion, Profession.
        \end{itemize}

        \cf{Mge}{Weapon and Armor Proficiencies}
        You are proficient with simple weapons and one other weapon group.
        You are not proficient with any type of armor or shield.
        Armor of any type interferes with your arcane gestures, which can cause your spells with somatic components to fail.

        \cf{Mge}{Arcane Essence}
        All mages have access to great arcane power.
        However, not all mages acquired this power in the same way.
        You choose an arcane essence.
        Many mage abilities have special effects based on whether you are a sorcerer or a wizard.
        \parhead{Sorcerer} Sorcerers have an intuitive connection to magic that allows them to cast spells without preparation or training.
        They cast spells with their Willpower.
        \parhead{Wizard} Wizards study arcane mysteries for years to learn the secret ways of magic.
        They cast spells with their Intelligence.

    \subsection{Arcane Spellcasting}
        This archetype grants you the ability to cast arcane spells.
        % TODO: less boring attribute integration?
        If you are a sorcerer, you must have a starting Willpower of at least 1 to gain this archetype.
        If you are a wizard, you must have an starting Intelligence of at least 2 to gain this archetype.

        \cf{Mge}[1]{Arcane Spell Failure}
        When you cast an arcane spell or perform an arcane ritual while using \glossterm{armor}, you must roll 1d10.
        If your result is less than or equal to your \glossterm{encumbrance}, you \glossterm{miscast} the ability (see \pcref{Miscasting}).
        When you perform a ritual, this roll must be repeated at the end of each action phase during the ritual.

        \cf{Mge}[1]{Mystic Spheres}[Magical]
        You can use arcane magic.
        You gain access to two arcane \glossterm{mystic spheres} (see \pcref{Arcane Mystic Spheres}).
        As a \glossterm{standard action}, you can cast the \glossterm{cantrip} spell from any mystic sphere you have access to.

        For details about mystic spheres and casting spells, see \pcref{Spell and Ritual Mechanics}.

        \cf{Mge}[2]{Spells} You learn two arcane \glossterm{spells}.
        You can also spend \glossterm{insight points} to learn one additional arcane spell per \glossterm{insight point}.
        You can learn any 1st level spells from the arcane \glossterm{mystic spheres} you have access to.
        As a \glossterm{standard action}, you can spend an \glossterm{action point} to cast any \glossterm{spell} you know.
        When you gain access to a new \glossterm{mystic sphere} or spell level, you can exchange any number of spells you know for spells of other spells you know, including spells of the higher level.

        \cf{Mge}[3]{Spell Level}[Magical] You gain the ability to cast 2nd level arcane spells.

        \cf*{Mge}[4]{Spell Level}[Magical] You gain the ability to cast 3rd level arcane spells.

        \cf{Mge}[4]{Spell} You gain an additional \glossterm{spell} for any arcane \glossterm{mystic sphere} you know.

        \cf*{Mge}[5]{Spell Level}[Magical] You gain the ability to cast 4th level arcane spells.

        \cf*{Mge}[6]{Spell Level}[Magical] You gain the ability to cast 5th level arcane spells.

        \cf*{Mge}[6]{Spell} You gain an additional \glossterm{spell} for any arcane \glossterm{mystic sphere} you know.

        \cf*{Mge}[7]{Spell Level}[Magical] You gain the ability to cast 6th level arcane spells.

        \cf*{Mge}[8]{Spell Level}[Magical] You gain the ability to cast 7th level arcane spells.

        \cf*{Mge}[8]{Spell} You gain an additional \glossterm{spell} for any arcane \glossterm{mystic sphere} you know.

    \subsection{Arcane Lore}
        This archetype grants you esoteric abilities relating to arcane magic.
        You must have the Arcane Spellcasting archetype to gain the abilities from this archetype.

        \cf{Mge}[1]{Mage Armor}[Magical] You can use the \textit{mage armor} ability as a standard action.
        \begin{freeability}{Mage Armor}
            You create a translucent suit of magical armor on your body.
            This functions like body armor that provides a \plus2 bonus to Armor defense and has no \glossterm{encumbrance}.
            You cannot wear it alongside other armor.

            This ability lasts until you \glossterm{dismiss} it as a free action.
        \end{freeability}

        \cf{Mge}[2]{Arcane Insights}[Magical]
        You gain two of the following insights.
        Some insights can be chosen multiple times, as indicated in their descriptions.

        {
            \parhead{Esoteric Insight} You forget a \glossterm{spell} you know.
            In exchange, you gain access to a new \glossterm{mystic sphere}.
            \par If you choose the insight multiple times, you must forget a different spell each time.

            \parhead{Focused Insight} You lose access to a \glossterm{mystic sphere} you know.
            In exchange, you learn a \glossterm{spell} from another \glossterm{mystic sphere} you know.
            \par If you choose the insight multiple times, you must lose access to a different \glossterm{mystic sphere} each time.

            \parhead{Innate Sphere} Choose a \glossterm{mystic sphere} you know.
            You no longer need verbal or somatic components to cast spells from that \glossterm{mystic sphere}.
            \par If you choose this insight multiple times, you must choose a different \glossterm{mystic sphere} each time.

            \parhead{Signature Sphere} Choose a \glossterm{mystic sphere} you have access to.
            You gain a \plus1 bonus to \glossterm{accuracy} with abilities from that \glossterm{mystic sphere}.
            In addition, you gain a \plus10 bonus to \glossterm{concentration} checks you make to use abilities from that \glossterm{mystic sphere}.
            \par If you choose this insight multiple times, you must choose a different \glossterm{mystic sphere} each time.

            \parhead{Specialization} Choose a school of magic that you know a \glossterm{mystic sphere} from.
            You learn an additional \glossterm{spell} from that school of magic.
            You can exchange this spell for other spells as you gain access to new spell levels, but the spell must always be used for the chosen school.
            In exchange, you must ban two other schools of magic.
            You can never gain access to \glossterm{mystic spheres} from your banned schools.
            If you know \glossterm{mystic spheres} from a banned school, you must immediately learn different \glossterm{mystic spheres} from unbanned schools in their place.
            You cannot ban the Channeling school of magic.
            \par If you choose this insight multiple times, you must choose to specialize in the same school each time.

            \parhead{Arcane Reservoir} You can only gain this arcane insight if you are a sorcerer.
            You gain an additional \glossterm{reserve action point}.
            \par You cannot choose this insight multiple times.

            \parhead{Memorized Ritual} You can only gain this arcane insight if you are a wizard.
            % TODO: clarify you need to be high enough level?
            Choose an arcane ritual.
            You do not need your ritual book to cast that ritual or any of its subrituals.
            If you lead the ritual, it takes half the normal amount of time to perform and costs half the normal number of action points (minimum 1).
            \par You cannot choose this insight multiple times.
        }

        \cf{Mge}[3]{Essence Lore}[Magical]
        You gain an ability based on your choice of arcane essence.
        % TODO: wording
        \subparhead{Sorcerer} When a spell resolves, if you were a target of the spell, you heal hit points equal to that spell's \glossterm{power}.
        This healing applies even if the spell's attack fails.
        You can only gain this healing once per round.
        \subparhead{Wizard} You gain two additional skill points.
        In addition, you gain a \plus2 bonus to all Knowledge skills.

        \cf*{Mge}[4]{Arcane Insight}[Magical]
        You learn an additional \textit{arcane insight}.

        \cf{Mge}[5]{Greater Essence Lore}[Magical]
        You gain an ability based on your choice of arcane essence.
        \subparhead{Sorcerer} You gain a \plus2 bonus to \glossterm{defenses} against \glossterm{magical} abilities.
        \subparhead{Wizard} You learn the Contingency augment, allowing you to prepare a spell so it takes effect automatically if specific circumstances arise.
        The Contingency augment adds two levels to a spell's level.
        You can apply this augment to any arcane spell that can be cast as a \glossterm{standard action}.

        Casting a spell with the Contingency augment takes 5 minutes.
        When the casting is complete, the spell has no immediate effect.
        Instead, it automatically takes effect when some specific circumstances arise.
        During the time required to cast the spell, you specify what circumstances cause the spell to take effect.

        The spell can be set to trigger in response to any circumstances that a typical human observing you and your situation could detect.
        For example, you could specify ``when I fall at least 50 feet'' or ``when I become bloodied'', but not ``when there is an invisible creature within 50 feet of me'' or ``when I am at 17 hit points or fewer.''
        The more specific the required circumstances, the better -- vague requirements, such as ``when I am in danger'', may cause the spell to trigger unexpectedly or fail to trigger at all.
        If you attempt to specify multiple separate triggering conditions, such as ``when I take damage or when an enemy is adjacent to me'', the spell will randomly ignore all but one of the conditions.

        If the spell needs to be targeted, the trigger condition can specify a simple rule for identifying how to target the spell, such as ``the closest enemy''.
        If the rule is poorly worded or imprecise, the spell may target incorrectly or fail to activate at all.
        Any spells which require decisions, such as the \spell{dimension door} spell, must have those decisions made at the time it is cast.
        You cannot alter those decisions when the contingency takes effect.

        You can have only one spell with this augment active at a time.
        If you use the augment again with a different spell, the new spell.

        \cf*{Mge}[6]{Arcane Insight}[Magical]
        You learn an additional \textit{arcane insight}.

        \cf{Mge}[7]{Supreme Essence Lore}[Magical]
        You gain an ability based on your choice of arcane essence.

        \subparhead{Sorcerer} When a \glossterm{spell} affects you and fails to beat your defenses, you gain the ability to cast that spell once.
        The spell retains all augments, effects from feats and other abilities, and similar modifications from the original caster, and you cannot choose any other augments or apply effects from your own abilities.
        However, you make all other decisions required to cast the spell, and use your \glossterm{accuracy} and \glossterm{power} to determine the spell's effects.
        You must spend an \glossterm{action point} to cast the spell.
        Once you cast the spell, you expend the absorbed energy, and you cannot cast it again.

        If you resist multiple spells simultaneously, or if you resist another spell before casting the previous spell you resisted, you choose which spell you gain the ability to cast.

        \subparhead{Wizard} You may have two spells active with the Contingency augment, rather than only one.
        When you cast a new spell with the Contingency augment, you choose which existing contingency to replace.

        Only one contingency can trigger in a given round.
        If both would trigger simultaneously, only the first spell cast triggers.
        The second spell cast does not trigger that round.

        \cf{Mge}[8]{Archmage}[Magical]
        You gain a \plus2 bonus to \glossterm{accuracy} and \glossterm{power} with arcane spells.

    \subsection{Arcane Spell Mastery}
        This archetype improves the arcane spells you cast.
        You must have the Arcane Spellcasting archetype to gain the abilities from this archetype.
        % TODO: This wording is ugly
        Abilities from this archetype that affect arcane spells only affect arcane spells gained from the Arcane Spellcasting archetype.

        \cf{Mge}[1]{Wellspring of Power}[Magical]
        You gain a \plus1 bonus to \glossterm{power} with arcane spells.

        \cf{Mge}[2]{Mystic Knowledge}[Magical]
        You gain your choice of one of the following abilities.
        {
            \parhead{Insight} You gain an additional \glossterm{insight point} (see \pcref{Insight Points}).
            \parhead{Rituals} You can only choose this mystic knowledge if you are a wizard.
                You gain the ability to perform arcane rituals to create unique magical effects (see \pcref{Rituals}).
            \parhead{Mystic Sphere Access} You gain access to an additional divine \glossterm{mystic sphere}.
                You can choose this ability multiple times, gaining access to an additional \glossterm{mystic sphere} each time.
        }

        \cf{Mge}[3]{Augments}[Magical]
        Choose one \glossterm{augment} (see \pcref{Augments}).
        You can also spend \glossterm{insight points} to learn one additional augment per \glossterm{insight point}.
        You can apply those augments to arcane spells you cast and arcane rituals you perform.
        When you gain access to a new spell level, you may change which augments you know.

        \cf*{Mge}[4]{Mystic Knowledge}[Magical]
        You gain an additional \textit{mystic knowledge} ability.

        \cf{Mge}[5]{Augment}[Magical]
        You learn an additional \glossterm{augment} that you can apply to arcane spells.

        \cf{Mge}[6]{Greater Wellspring of Power}[Magical]
        The bonus from your \textit{wellspring of power} ability increases to \plus2.

        \cf*{Mge}[7]{Mystic Knowledge}[Magical]
        You gain an additional \textit{mystic knowledge} ability.

        \cf{Mge}[8]{Mystic Supremacy}[Magical]
        The maximum spell level you can cast is increased to 8.
        In addition, you learn an additional \glossterm{augment} that you can apply to arcane spells.

\newpage
\section{Monk}\label{Monk}
    \begin{dtable}
        \lcaption{Monk Progression}
        \begin{dtabularx}{\columnwidth}{c >{\lcol}X >{\lcol}X >{\lcol}X}
            \tb{Rank} & \tb{Ki} & \tb{Transcendent Sage} & \tb{Unfettered Warrior} \\\bottomrule
               1 & Ki barrier, ki manifestations    & Serene strike         & Unarmed warrior, unfettered defense
            \\ 2 & Ki vessel            & Sage lore             & Unfettered athletics
            \\ 3 & Ki manifestation     & Transcend frailty     & Intuitive reaction
            \\ 4 & Dual manifestation   & Greater serene strike & Greater unarmed warrior
            \\ 5 & Ki manifestation     & Transcend flesh       & Greater intuitive reaction
            \\ 6 & Ki wellspring        & Inner peace           & Greater unfettered defense
            \\ 7 & Ki manifestation     & Transcend mortality   & Supreme intuitive reaction
            \\ 8 & Myriad manifestation &                       &
        \end{dtabularx}
    \end{dtable}

    \classbasics{Alignment} Any nonchaotic.

    \classbasics{Archetypes} Monks have the Ki, Unfettered Warrior, and Transcendent Sage \glossterm{archetypes}.

    \subsection{Basic Class Abilities}
        If you are a monk, you gain the following abilities.

        \cf{Mnk}{Defenses}
        You gain the following bonuses to your \glossterm{defenses}: \plus3 Fortitude, \plus5 Reflex, \plus4 Mental.

        \cf{Mnk}{Skills}
        You have the following \glossterm{class skills}:
        \begin{itemize}
            \item \subparhead{Strength} Climb, Jump, Swim.
            \item \subparhead{Dexterity} Acrobatics, Escape Artist, Ride, Stealth.
            \item \subparhead{Intelligence} Craft, Deduction, Heal.
            \item \subparhead{Perception} Awareness, Spellcraft, Survival.
            \item \subparhead{Other} Bluff, Intimidate, Perform, Persuasion, Profession.
        \end{itemize}

        \cf{Mnk}{Weapon and Armor Proficiencies}
        Monks are proficient with simple weapons, monk weapons, and any one other weapon group.
        Monks are not proficient with any armor or shields.

    \subsection{Ki}
        This archtype grants you abilities you can use in combat.
        If you wear armor, use a shield, or have \glossterm{encumbrance}, you lose the benefit of all abilities from this archetype.

        \cf{Mnk}[1]{Ki Barrier}[Magical]
        If you are not wearing armor and have no \glossterm{encumbrance}, you gain a ki barrier around your body.
        This functions like body armor that provides a \plus2 bonus to Armor defense and has no \glossterm{encumbrance}.
        You lose this bonus when you are \helpless.

        \cf{Mnk}[1]{Ki Strike}[Magical] You can use the \textit{ki strike} ability as a standard action.
        \begin{freeability}{Ki Strike}
            Make a melee \glossterm{strike}.
            % TODO: wording
            Because this is a \glossterm{magical} ability, you use your \glossterm{power} with magical abilities to determine your damage with the strike.
        \end{freeability}

        \cf{Mnk}[2]{Ki Manifestations}[Magical]
        You can channel your ki to temporarily enhance your abilities.
        Choose one \textit{ki manifestation} from the list below.
        You can also spend \glossterm{insight points} to learn one additional \textit{ki manifestation} per \glossterm{insight point}.
        You can use any \textit{ki manifestation} ability you know using the type of action indicated in the ability's description.
        You cannot use more than one \textit{ki manifestation} per round.
        {
            \begin{apability}{Abandon the Fragile Self}[\glossterm{Swift}]
                You can use this ability as a \glossterm{minor action}.
                Remove one \glossterm{condition} affecting you.
                Because this ability has the \glossterm{Swift} tag, the penalties from the condition do not affect you during the current phase.

                \rankline
                \rank{4} You are also immune to all \glossterm{conditions} during the current phase, preventing new conditions from being applied to you.
                \rank{6} The immunity to conditions lasts until the end of the round.
                \rank{8} The immunity to conditions lasts until the end of the next round.
            \end{apability}

            \begin{apability}{Burst of Blinding Speed}
                You can use this ability at the start of the round.
                It lasts until the end of the round.
                You gain a \plus30 foot \glossterm{magic bonus} to your land speed, up to a maximum of double your original speed.

                \rankline
                \rank{4} You can also ignore \glossterm{difficult terrain}.
                \rank{6} You can also move or stand on liquids as if they were solid.
                \rank{8} The speed bonus increases to \plus60 feet, up to a maximum of triple your original speed.
            \end{apability}

            \begin{apability}{Distant Strike}
                You can use this ability as a standard action.
                When you do, make a \glossterm{strike}.
                Your \glossterm{reach} is increased by 5 feet for the strike.
                Because this is a \glossterm{magical} ability, you use your \glossterm{power} with magical abilities to determine your damage with the strike.

                \rankline
                \rank{4} You gain a \plus1d bonus to damage with the strike.
                \rank{6} The bonus to your reach increases to 10 feet. 
                \rank{8} The damage bonus increases to \plus2d.
            \end{apability}

            \begin{apability}{Elegant Whirl of Fluid Motion}
                You can use this ability at the start of the round.
                It lasts until the end of the round.
                You gain a \plus5 bonus to Acrobatics (see \pcref{Acrobatics}).

                \rankline
                \rank{4} The bonus increases to \plus10.
                \rank{6} The bonus lasts until the end of the next round.
                \rank{8} The bonus increases to \plus20.
            \end{apability}

            \begin{apability}{Flash Step}[\glossterm{Teleportation}]
                You can use this ability at the start of the round.
                It lasts until the end of the round.
                % TODO: is 'horizontally' the correct word?
                When you move, you can teleport horizontally up to half your movement speed instead.
                This replaces the entire distance you would have moved that phase.
                If your \glossterm{line of effect} to your destination is blocked, or if this teleportaiton would somehow place you inside a solid object, your movement is cancelled and you remain where you are.

                \rankline
                \rank{4} You can teleport in multiple steps within the same phase.
                Each step costs movement equal to twice the distance you teleport.
                \rank{6} The movement cost to teleport is reduced to be equal to the distance you teleport.
                \rank{8} You can attempt to teleport to locations outside of \glossterm{line of sight} and \glossterm{line of effect}.
                If your intended destination is invalid, the distance you spent teleporting is wasted, but you suffer no other ill effects.
            \end{apability}

            \begin{apability}{Leap of the Heavens}
                You can use this ability at the start of the round.
                It lasts until the end of the round.
                You gain a \plus5 bonus to Jump (see \pcref{Jump}).

                \rankline
                \rank{4} The bonus increases to \plus10.
                \rank{6} The bonus lasts until the end of the next round.
                \rank{8} The bonus increases to \plus20.
            \end{apability}

            \begin{apability}{Scale the Highest Tower}
                You can use this ability at the start of the round.
                It lasts until the end of the round.
                You gain a \plus5 bonus to Climb (see \pcref{Climb}).
                % TODO: is this wording correct?

                \rankline
                \rank{4} The Climb bonus increases to \plus10.
                \rank{6} The bonus lasts until the end of the next round.
                \rank{6} The bonus increases to \plus20.
            \end{apability}

            \begin{apability}{See the Flow of Life}
                You can use this ability at the start of the round.
                It lasts until the end of the round.
                You gain the ability to see the ki of living creatures.
                You can ``see'' any living creatures and their equipment within 50 feet perfectly, regardless of lighting conditions, blindness, invisibility, or any other means of concealment.
                This cannot detect living creatures through solid walls, however.

                \rankline
                \rank{4} The range of the ability is increased to 100 feet.
                \rank{6} This ability gains the \glossterm{Sustain} (minor) tag.
                \rank{8} This ability gains the \glossterm{Attune} tag in place of the \glossterm{Sustain} (minor) tag.
            \end{apability}

            \begin{apability}{Step Between the Mystic Worlds}[\glossterm{Swift}]
                You can use this ability as a \glossterm{minor action}.
                It lasts until the end of the round.
                You gain a \plus2 bonus to \glossterm{defenses} against \glossterm{magical} abilities that directly target you.
                This does not protect you from abilities that affect an area.

                \rankline
                \rank{4} The defense bonus is increased to \plus4.
                \rank{6} The defense bonus is increased to \plus6.
                \rank{8} You cannot be directly targeted by \glossterm{magical} abilities.
            \end{apability}

            \begin{apability}{Surpass the Mortal Limits}
                You can use this ability at the start of the round.
                It lasts until the end of the round.
                You can use your \glossterm{power} in place of your Strength, Dexterity, and Constitution when making checks.

                \rankline
                \rank{4} You also gain a \plus2 bonus to checks based on Strength, Dexterity, and Constitution.
                \rank{6} The effect lasts until the end of the next round.
                \rank{8} The bonus increases to \plus4.
            \end{apability}

            \begin{freeability}{Tranquil Ward of Unshakeable Peace}[\glossterm{Swift}]
                You take half damage from all attacks.
                This halving is applied before \glossterm{damage reduction} and similar abilities.
                This ability lasts until the end of the round.

                \rankline
                \rank{4} You also gain a \plus1 bonus to all defenses.
                \rank{6} The defense bonus increases to \plus2.
                \rank{8} The defense bonus increases to \plus3.
            \end{freeability}
        }

        \cf{Mnk}[3]{Greater Ki Barrier} The bonus from your \textit{ki barrier} ability increases to \plus3.

        \cf*{Mnk}[4]{Ki Manifestation}[Magical]
        You learn an additional \textit{ki manifestation}.

        \cf{Mnk}[5]{Greater Ki Strike} You gain a \plus1d bonus to damage with your \textit{ki strike} ability.

        \cf*{Mnk}[6]{Ki Manifestation}[Magical]
        You learn an additional \textit{ki manifestation}.

        \cf{Mnk}[7]{Supreme Ki Barrier} The bonus from your \textit{ki barrier} ability increases to \plus4.

        \cf{Mnk}[8]{Dual Manifestation}[Magical] You can use up to two \textit{ki manifestation} abilities per round.
        If you do, the second ability does not cost an action point.

        \cf*{Mnk}[8]{Ki Manifestation}[Magical]
        You learn an additional \textit{ki manifestation}.

    \subsection{Unfettered Warrior}\label{Unfettered Warrior}
        This archetype improves your combat prowess while unarmed and unencumbered.

        \cf{Mnk}[1]{Unarmed Warrior}
        You are \glossterm{proficient} with your \glossterm{unarmed attack}.
        In addition, you gain a \plus2d bonus to damage with your unarmed attack.
        For details about how to fight while unarmed, see \pcref{Unarmed Combat}.

        \cf{Mnk}[1]{Unfettered Defense}
        While you have a free hand, have no \glossterm{encumbrance}, and are not using a shield, you gain a \plus1 bonus to Armor defense.

        % TODO: wording
        \cf{Mnk}[2]{Evasion} When you are attacked by an ability that affects an area, you can use your Reflex defense in place of any other defenses against that attack.

        \cf{Mnk}[3]{Unfettered Athletics} You gain two additional skill points.

        \cf*{Mnk}[4]{Greater Unarmed Warrior} The damage bonus from your \textit{unarmed warrior} ability increases to \plus3d.

        \cf{Mnk}[5]{Greater Unfettered Defense}
        The bonus from your \textit{unfettered defense} ability increases to \plus2.

        \cf{Mnk}[6]{Intuitive Reaction}
        You gain a \plus2 bonus to Reflex defense and \glossterm{initiative}.

        \cf{Mnk}[7]{Supreme Unarmed Warrior} The damage bonus from your \textit{unarmed warrior} ability increases to \plus4d.

    \subsection{Transcendent Sage}
        This archetype grants you abilities to resist or remove conditions.

        \cf{Mnk}[1]{Clear the Mind} You can use the \textit{clear the mind} ability as a standard action.
        \begin{freeability}{Clear the Mind}
            You remove one \glossterm{condition} affecting you.
            This cannot remove a condition applied during the current round.
        \end{freeability}

        \cf{Mnk}[2]{Sage Lore} You gain two additional skill points.

        \cf{Mnk}[3]{Transcend Frailty}[Magical]
        You are immune to being \glossterm{fatigued} and \glossterm{sickened}.

        \cf{Mnk}[4]{Inner Peace} You are immune to hostile \glossterm{Mind} \glossterm{conditions}.

        \cf{Mnk}[5]{Transcend Flesh}[Magical]
        You are immune to being \glossterm{exhausted} and \glossterm{nauseated}.
        % TODO: do you still take aging penalties?
        In addition, you no longer take penalties to your attributes for aging, and cannot be magically aged.
        You still die of old age when your time is up.

        \cf{Mnk}[6]{Greater Inner Peace}
        You are immune to hostile \glossterm{Mind} abilities.

        \cf{Mnk}[7]{Transcend Senses}[Magical]
        You are imune to being \glossterm{blinded} and \glossterm{deafened}.
        % TODO: wording?
        In addition, you gain the ability to see living creatures perfectly, regardless of concealment or invisibility.

        % This feels like a rank 8 ability
        \cf{Mnk}[8]{Transcend Mortality}[Magical]
        If you die, you may choose to retain control of your body and soul through sheer force of will.
        Your body immediately disappears, and your soul does not travel to an afterlife.
        Instead, your body reforms with no trace of its injuries 8 hours later.
        The reformed body is in perfect health and can be any age you choose, to a minimum of the age of adulthood for your species.
        You can reform your body at the place where you died, or in any place on the same plane that is deeply familiar to you.

        After each time you reform yourself in this way, it takes an additional hour to reform the next time you ``die''.
        You can only be permanently killed by the direct intervention of a deity.

    \subsection{Ex-Monks}
        As long as you are chaotic, you lose all of your \glossterm{magical} monk abilities.

\newpage
\section{Paladin}\label{Paladin}
    \begin{dtable}
        \lcaption{Paladin Progression}
        \begin{dtabularx}{\columnwidth}{c >{\lcol}X >{\lcol}X >{\lcol}X}
            \tb{Rank} & \tb{Devoted Paragon} & \tb{Divine Spellcasting}  & \tb{Zealous Warrior} \\\bottomrule
               1 & Lay on hands               & Mystic spheres  & Smite
            \\ 2 & Aligned aura               & Spells                 & Unfaltering champion
            \\ 3 & Greater lay on hands       & Spell level (2)           & Unfaltering zeal
            \\ 4 & Unbending devotion         & Spell level (3), spell & Greater smite
            \\ 5 & Greater aligned aura       & Spell level (4)           & Greater unfaltering champion
            \\ 6 & Supreme lay on hands       & Spell level (5), spell & Supreme smite
            \\ 7 & Greater unbending devotion & Spell level (6)           & Supreme unfaltering champion
            \\ 8 & Aligned soul               & Spell level (7), spell &
        \end{dtabularx}
    \end{dtable}

    \classbasics{Alignment} Any other than true neutral.

    \classbasics{Archetypes} Paladins have the Devoted Paragon, Divine Spellcasting, and Zealous Warrior \glossterm{archetypes}.

    \subsection{Basic Class Abilities}
        If you are a paladin, you gain the following abilities.

        \cf{Pal}{Defenses}
        You gain the following bonuses to your \glossterm{defenses}: \plus5 Fortitude, \plus3 Reflex, \plus4 Mental.

        \cf{Pal}{Skills}
        You have the following \glossterm{class skills}:
        \begin{itemize}
            \item \subparhead{Dexterity} Ride.
            \item \subparhead{Intelligence} Craft, Deduction, Heal, Knowledge (local, religion).
            \item \subparhead{Perception} Awareness, Intimidate, Sense Motive.
            \item \subparhead{Other} Bluff, Intimidate, Persuasion, Profession.
        \end{itemize}

        \cf{Pal}{Weapon and Armor Proficiencies}
        Paladins are proficient with simple weapons, any three other weapon groups, all types of armor (heavy, medium, and light), and shields.

        \cf{Pal}{Devoted Alignment} 
        You are devoted to a specific alignment.
        You must choose one of your alignment components: good, evil, lawful, or chaotic.
        The alignment you choose is your devoted alignment.
        Your paladin abilities are affected by this choice.
        % seems unnecessary
        % You excel at slaying creatures with alignments opposed to your devoted alignment.
        Your alignment cannot be changed without extraordinary repurcussions.

    \subsection{Devoted Paragon}
        This archetype grants you healing abiliies and an aura reflecting your alignment.

        \cf{Pal}[1]{Lay on Hands}[Magical] You can use the \textit{lay on hands} ability as a standard action.
        \begin{freeability}{Lay on Hands}[Life]
            Choose yourself or a willing creature within your \glossterm{reach}.
            The target is healed for hit points equal to \glossterm{standard damage}.
        \end{freeability}

        \cf{Pal}[2]{Aligned Aura}[Magical]
        Your devotion to your alignment affects the world around you, bringing it closer to your ideals.
        You constantly radiate an aura in a \areamed radius \glossterm{emanation} from you.
        You can freely choose whether creatures in the area are affected by the aura, including yourself.
        The effect of the aura depends on your devoted alignment, as described below.
        You can suppress or resume the aura as a \glossterm{minor action}.

        \subparhead{Chaos} When a target rolls a 1 on an attack roll when making a \glossterm{strike}, the creature can reroll the die.
        This cannot affect the same roll twice, and it does not affect bonus dice rolled for exploding attacks (see \pcref{Exploding Attacks}).
        \subparhead{Evil} When a target takes damage, it increases that damage by an amount equal to your \glossterm{power}.
        This increase can only be applied to an individual creature once per round.
        % TODO: clarify what happens if multiple people try to Good aura the same target
        \subparhead{Good} When a target takes damage, you may take half that damage (rounded down) instead.
        Any abilities you have that would reduce or negate the effects of the attack have no effect.
        The protected creature takes the remaining half of the damage, and suffers any non-damaging effects of the attack normally.
        \subparhead{Law} When a target rolls a 1 on an attack roll when making a \glossterm{strike}, the attack roll is treated as a 6.
        This does not affect bonus dice rolled for exploding attacks (see \pcref{Exploding Attacks}).

        \cf{Pal}[3]{Greater Lay on Hands}[Magical]
        You gain a \plus1d bonus to the healing from your \textit{lay on hands} ability.
        In addition, you can spend an \glossterm{action point} when you use that ability.
        If you do, the target also removes one \glossterm{condition} affecting it.

        \cf{Pal}[4]{Unbending Devotion}[Magical]
        You are immune to \glossterm{Mind} \glossterm{conditions}.

        \cf{Pal}[5]{Greater Aligned Aura}[Magical]
        The area of your \textit{aligned aura} ability increases to \arealarge.
        In addition, the effect of your \textit{aligned aura} becomes stronger, as described below.

        \subparhead{Chaos} The effect applies to all attacks, not just \glossterm{strikes}.
        \subparhead{Evil} The damage increases to be equal to twice your \glossterm{power}.
        \subparhead{Good} When you redirect damage from a creature with this aura, you can redirect all damage from the attack to you instead of only half the damage.
        The protected creature still suffers any non-damaging effects of the attack normally.
        \subparhead{Law} The effect applies to all attacks, not just \glossterm{strikes}.

        \cf{Pal}[6]{Greater Unbending Devotion}
        You are immune to all hostile \glossterm{Mind} effects.

        \cf{Pal}[7]{Supreme Lay on Hands}[Magical]
        You gain a \plus1d bonus to the healing from your \textit{lay on hands} ability.
        In addition, if you use it to remove conditions, the target can remove an additional \glossterm{condition}.

        \cf{Pal}[8]{Aligned Soul}[Magical]
        While you are dead, you may approach the deity or governing figure of your afterlife and request to be returned to life to continue your mission.
        Travelling to the relevant figure and making the request takes 12 hours.
        Unless there are extenuating circumstances, this request is almost always granted, and you are resurrected in a new body at a location of the entity's choice.
        This functions like the \ritual{resurrection} ritual, except that no part of the body is required, and a new body is created by the entity.
        You can be resurrected in this way regardless of the condition of your body, but not if your soul has been trapped or otherwise prevented from going to the correct afterlife.

    \subsection{Divine Spellcasting}
        This archetype grants you the ability to cast divine spells.
        You must have a starting Willpower of at least 1 to gain this archetype.

        \cf{Pal}[1]{Mystic Spheres}[Magical]
        Your devotion to your alignment grants you the ability to use divine magic.
        You gain access to two divine \glossterm{mystic spheres} (see \pcref{Divine Mystic Spheres}).
        As a \glossterm{standard action}, you can cast the \glossterm{cantrip} spell from any mystic sphere you have access to.

        For details about mystic spheres and casting spells, see \pcref{Spell and Ritual Mechanics}.

        \cf{Pal}[2]{Spells} You learn two divine \glossterm{spells}.
        You can also spend \glossterm{insight points} to learn one additional divine spell per \glossterm{insight point}.
        You can learn any 1st level spells from the divine \glossterm{mystic spheres} you have access to.
        As a \glossterm{standard action}, you can spend an \glossterm{action point} to cast any \glossterm{spell} you know.
        When you gain access to a new \glossterm{mystic sphere} or spell level, you can exchange any number of spells you know for spells of other spells you know, including spells of the higher level.

        \cf{Pal}[3]{Spell Level}[Magical] You gain the ability to cast 2nd level divine spells.

        \cf*{Pal}[4]{Spell Level}[Magical] You gain the ability to cast 3rd level divine spells.

        \cf{Pal}[4]{Spell} You gain an additional \glossterm{spell} for any divine \glossterm{mystic sphere} you know.

        \cf*{Pal}[5]{Spell Level}[Magical] You gain the ability to cast 4th level divine spells.

        \cf*{Pal}[6]{Spell Level}[Magical] You gain the ability to cast 5th level divine spells.

        \cf*{Pal}[6]{Spell} You gain an additional \glossterm{spell} for any divine \glossterm{mystic sphere} you know.

        \cf*{Pal}[7]{Spell Level}[Magical] You gain the ability to cast 6th level divine spells.

        \cf*{Pal}[8]{Spell Level}[Magical] You gain the ability to cast 7th level divine spells.

        \cf*{Pal}[8]{Spell} You gain an additional \glossterm{spell} for any divine \glossterm{mystic sphere} you know.

    \subsection{Zealous Warrior}
        This archetype improves your combat prowess, especially against foes who do not share your devoted alignment.

        \cf{Pal}[1]{Unfaltering Champion}
        You gain a \plus2 bonus to \glossterm{threat} and a \plus1 bonus to Armor defense.

        \cf{Pal}[2]{Smite}[Magical] You can use the \textit{smite} ability as a standard action.
        \begin{apability}{Smite}
            Make a \glossterm{strike}.
            If your target shares your devoted alignment, the strike deals no damage.
            Otherwise, the strike gains a \plus2d bonus to damage.
        \end{apability}

        \cf{Pal}[3]{Unfaltering Zeal}
        You gain a \plus1 bonus to Fortitude and Mental defense.

        \cf{Pal}[4]{Greater Smite}[Magical] The damage bonus from your \textit{smite} ability increases to \plus3d.
        In addition, when you use your \textit{smite} ability, you can spend an additional \glossterm{action point}.
        If you do, the target of your strike is treated as if it had the alignment opposed to your devoted alignment for the purpose of all abilities, including for your initial \textit{smite}.
        This only affects its alignment along the alignment axis your devoted alignment is on.
        For example, if your devoted alignment was evil, a chaotic neutral target would be treated as chaotic good.

        You can use this ability to do battle against foes who share your alignment, but you should exercise caution in doing so.
        Persecution of allies can lead you to fall and become an ex-paladin.

        \cf{Pal}[5]{Greater Unfaltering Champion} The \glossterm{threat} bonus from your \textit{unfaltering champion} ability increases to \plus4.

        \cf{Pal}[6]{Supreme Smite}[Magical] The damage bonus from your \textit{smite} ability increases to \plus4d.

        \cf{Pal}[7]{Supreme Unfaltering Champion}
        The Armor defense bonus from your \textit{unfaltering champion} ability increases to \plus2.

        % TODO: Doesn't work with new spell system
        % \cf{Pal}[20]{Martyr's Retribution}[Mag]
        % If you die in the service of your devoted alignment, you may choose to have your fallen body erupt in an immense burst of divine energy.
        % If you do, your body is almost completely consumed, preventing you from being raised with \ritual{resurrection} and similar effects that require an intact body.
        % This burst has two effects.
        % First, a \spell{sunburst} spell immediately takes effect over the area where you died.
        % Second, a \spell{storm of vengeance} spell begins to take effect, centered on the same area.
        % The spell lasts for 10 rounds, and the lightning strikes target the paladin's enemies.
        % Both of these effects harm only the paladin's foes, and do not harm your allies.
        % However, your allies' vision is still impeded by the \spell{storm of vengeance}.

    \subsection{Ex-Paladins}
        If you cease to follow your devoted alignment, you lose all \glossterm{magical} paladin class abilities.
        If your atone for your misdeeds and resume the service of your devoted alignment, you can regain your abilities.

\newpage
\section{Ranger}\label{Ranger}
    \begin{dtable}
        \lcaption{Ranger Progression}
        \begin{dtabularx}{\columnwidth}{c >{\lcol}X >{\lcol}X >{\lcol}X}
            \tb{Rank} & \tb{Huntmaster}  & \tb{Keen Senses} & \tb{Wilderness Warrior} \\\bottomrule
               1 & Quarry         & Keen vision        & Wild exertions
            \\ 2 & Tracker        & Learned perception & Wilderness lore
            \\ 3 & Hunting styles & Blindsense         & Wild exertion
            \\ 4 & Hunting lore   & Farsight           & Wilderness lore
            \\ 5 & Hunting style  & Blindsight         & Wild exertion
            \\ 6 & Fluid style    & Greater farsight   & Wilderness mastery
            \\ 7 & Hunting style  & Truesight          & Wild exertion
            \\ 8 & Dual quarry    &                    &
        \end{dtabularx}
    \end{dtable}

    \classbasics{Alignment} Any.

    \classbasics{Archetypes} Rangers have the Keen Senses, Wilderness Warrior, and Huntmaster \glossterm{archetypes}.

    \subsection{Basic Class Abilities}
        If you are a ranger, you gain the following abilities.

        \cf{Rgr}{Defenses}
        You gain the following bonuses to your \glossterm{defenses}: \plus4 Fortitude, \plus5 Reflex, \plus3 Mental.

        \cf{Rgr}{Skills}
        You have the following \glossterm{class skills}:
        \begin{itemize}
            \item \subparhead{Strength} Climb, Jump, Swim.
            \item \subparhead{Dexterity} Acrobatics, Escape Artist, Ride, Stealth.
            \item \subparhead{Intelligence} Craft, Deduction, Heal, Knowledge (dungeoneering, geography, nature).
            \item \subparhead{Perception} Awareness, Creature Handling, Survival.
            \item \subparhead{Other} Bluff, Intimidate, Persuasion, Profession.
        \end{itemize}

        \cf{Rgr}{Weapon and Armor Proficiencies}
        A ranger is proficient with simple weapons, any two other weapon groups, light and medium armor, and shields.
        You are also proficient with your choice of bows, crossbows, or thrown weapons.

    \subsection{Keen Senses}
        This archetype improves your senses.

        \cf{Rgr}[1]{Keen Vision}
        Your sight improves, allowing you to see more easily.
        You gain a \plus2 bonus to \glossterm{Awareness}.
        In addition, you gain \glossterm{low-light vision}, allowing you to treat sources of light as if they had double their normal illumination range.
        If you already have low-light vision, you double its benefit, allowing you to treat sources of light as if they had four times their normal illumination range.

        \cf{Rgr}[2]{Learned Perception} You gain two skill points.

        \cf{Rgr}[3]{Blindsense}
        Your perceptions are so finely honed that you can sense your enemies without seeing them.
        You gain the \glossterm{blindsense} ability out to 50 feet.
        This ability allows you to sense the presence and location of objects and foes within 50 feet without seeing them.
        If you already have the blindsense ability, you increase its range by 50 feet.

        \cf{Rgr}[4]{Farsight}
        You reduce your \glossterm{range increment} penalties for attacking at long range by 2.

        \cf{Rgr}[5]{Blindsight}
        You gain the \glossterm{blindsight} ability, allowing you to ``see'' perfectly without your eyes in a 50 foot radius around you.
        With this ability, you can fight just as well with your eyes closed as with them open.
        In addition, the range of your \glossterm{blindsense} ability increases by 50 feet.

        \cf{Rgr}[6]{Greater Farsight}
        You increase the range of your your \glossterm{blindsense} by 100 feet, and your \glossterm{blindsight} by 50 feet.
        In addition, the penalty reduction of \glossterm{range increment} penalties from your \textit{farsight} ability increases to 4.

        % This may not actually be very strong? Also inconsistent with other ``truesight'' abilities
        \cf{Rgr}[7]{Truesight}[Magical]
        Your perceptions are accurate enough to defeat even powerful magic.
        You can see through normal and magical darkness, see the truth behind visual figments and glamers, and see the true form of creatures and objects affected by \glossterm{Shaping} abilities.
        This ability works at any range.

    \subsection{Wilderness Warrior}
        This archetype grants you abilities to use in combat and improves your wilderness skills.

        \cf{Rgr}[1]{Wanderer's Strike} You can use the \textit{wanderer's strike} ability as a standard action during the \glossterm{action phase}.
        \begin{freeability}{Wanderer's Strike}
            You can either move up to half your movement speed or make a \glossterm{strike}.
            %TODO: wording
            During the \glossterm{delayed action phase}, you can take the action you did not take during the \glossterm{action phase}.
        \end{freeability}

        \cf{Rgr}[2]{Wild Exertions} 
        You can channel your martial prowess into devastating attacks.
        Choose one \textit{wild exertion} from the list below.
        You can also spend \glossterm{insight points} to learn one additional \textit{wild exertion} per \glossterm{insight point}.
        As a standard action, you can spend an \glossterm{action point} to use a \textit{wild exertion} ability.
        {
            \begin{freeability}{Brace for Impact}[\glossterm{Swift}]
                You take half damage from all attacks.
                This halving is applied before \glossterm{damage reduction} and similar abilities.
                This ability lasts until the end of the round.

                \rankline
                \rank{4} You also gain a \plus1 bonus to all defenses.
                \rank{6} The defense bonus increases to \plus2.
                \rank{8} The defense bonus increases to \plus3.
            \end{freeability}

            \begin{apability}{Certain Strike}
                Make a \glossterm{strike} with a \plus3 bonus to accuracy.

                \rankline
                \rank{4} The accuracy bonus increases to \plus4.
                \rank{6} The accuracy bonus increases to \plus5.
                \rank{8} The accuracy bonus increases to \plus6.
            \end{apability}

            \begin{apability}{Hamstring}
                Make a \glossterm{strike} with a slashing or piercing weapon.
                If the target takes damage from the strike, it is \glossterm{slowed} as a \glossterm{condition}.
                Otherwise, you regain the \glossterm{action point} spent to use this ability.

                \rankline
                \rank{4} You gain a \plus1 bonus to accuracy on the strike.
                \rank{6} The accuracy bonus increases to \plus2.
                \rank{8} The accuracy bonus increases to \plus3.
            \end{apability}

            \begin{freeability}{Hunting Strike}
                Make a \glossterm{strike} against a creature.
                After making the strike, you gain a \plus1 bonus to \glossterm{accuracy} against the target with all attacks.
                This effect stacks with itself, up to a maximum of a \plus4 bonus.
                It lasts until you take a \glossterm{short rest} or use this ability on a different creature.

                \rankline
                \rank{4} The accuracy bonus increases by \plus2 per strike instead of \plus1.
                \rank{6} The maximum accuracy bonus is increased to \plus6.
                \rank{8} The maximum accuracy bonus is increased to \plus8.
            \end{freeability}

            \begin{apability}{Liver Shot}
                Make a \glossterm{strike} with a bludgeoning weapon.
                If the target takes damage from the strike, it is \sickened as a \glossterm{condition}.
                Otherwise, you regain the \glossterm{action point} spent to use this ability.

                \rankline
                \rank{4} You gain a \plus1 bonus to \glossterm{accuracy} with the strike.
                \rank{6} The accuracy bonus increases to \plus2.
                \rank{8} The accuracy bonus increases to \plus3.
            \end{apability}

            \begin{apability}{Leaping Strike}
                You make a Jump check to leap and move as normal for the leap, up to a maximum distance equal to your \glossterm{base speed} (see \pcref{Leap}).
                You can make a \glossterm{strike} with a \plus1d bonus to damage from any location you occupy during the leap.

                \rankline
                \rank{4} The damage bonus increases to \plus2d.
                \rank{6} The damage bonus increases to \plus3d.
                \rank{8} The damage bonus increases to \plus4d.
            \end{apability}

            \begin{apability}{Penetrating Strike}
                Make a \glossterm{strike} with a piercing weapon.
                The attack is made against the target's Reflex defense instead of its Armor defense.

                \rankline
                \rank{4} You gain a \plus1d bonus to damage with the strike.
                \rank{6} The damage bonus increases to \plus2d.
                \rank{8} The damage bonus increases to \plus3d.
            \end{apability}

            \begin{apability}{Power Attack}
                Make a \glossterm{strike} with a \plus2d bonus to damage.

                \rankline
                \rank{4} The damage bonus increases to \plus3d.
                \rank{6} The damage bonus increases to \plus4d.
                \rank{8} The damage bonus increases to \plus5d.
            \end{apability}

            \begin{apability}{Pulverizing Smash}
                Make a \glossterm{strike} with a piercing weapon.
                The attack is made against the target's Fortitude defense instead of its Armor defense.

                \rankline
                \rank{4} You gain a \plus1d bonus to damage with the strike.
                \rank{6} The damage bonus increases to \plus2d.
                \rank{8} The damage bonus increases to \plus3d.
            \end{apability}

            \begin{apability}{Rapid Assault}
                Make a \glossterm{strike} against a creature.
                If you use this ability during the \glossterm{action phase}, you can make another strike during the \glossterm{delayed action phase}.
                You take a \minus2 penalty to accuracy on both strikes.

                \rankline
                \rank{4} The accuracy penalty is reduced to \minus1.
                \rank{6} The accuracy penalty is removed.
                \rank{8} You gain a \plus1 bonus to accuracy with both strikes.
            \end{apability}

            \begin{apability}{Reaping Charge}
                Move up to your movement speed in a straight line.
                You can make a melee \glossterm{strike} with a slashing or bludgeoning weapon.
                The strike targets any number of creatures and objects that you \glossterm{threaten} at any point during your movement, except for the space you start in and the space you end in.
                You take a \minus2 penalty to \glossterm{accuracy} on the strike.

                \rankline
                \rank{4} You gain a \plus1d bonus to damage with the strike.
                \rank{6} The damage bonus increases to \plus2d.
                \rank{8} The damage bonus increases to \plus3d.
            \end{apability}

            \begin{apability}{Strip the Flesh}
                Make a \glossterm{strike} with a slashing weapon.
                At the end of the \glossterm{action phase} of the next round, if you hit with the strike and the target is not \glossterm{bloodied}, it takes additional damage equal to the damage you dealt with the strike.

                \rankline
                \rank{4} If you hit with the strike, the target continues taking the same damage at the end of each \glossterm{action phase} until it becomes \glossterm{bloodied}.
                This is a \glossterm{condition}, and can be removed by abilities that remove conditions.
                \rank{6} You gain a \plus1d bonus to damage with the strike.
                \rank{8} The damage bonus increases to \plus2d.
            \end{apability}

            \begin{apability}{Sweeping Strike}
                Make a melee \glossterm{strike} with a slashing or bludgeoning weapon.
                The strike targets each of up to three creatures or objects you \glossterm{threaten}.
                You take a \minus1 penalty to \glossterm{accuracy} with the strike.

                \rankline
                \rank{4} You gain a \plus1d bonus to damage with the strike.
                \rank{6} The damage bonus increases to \plus2d.
                \rank{8} The damage bonus increases to \plus3d.
            \end{apability}

            \begin{apability}{Whirlwind Spin}
                Make a melee \glossterm{strike} with a slashing weapon.
                The strike targets all creatures you \glossterm{threaten}.
                You take a \minus2 penalty to \glossterm{accuracy} with the strike.

                \rankline
                \rank{4} You gain a \plus1d bonus to damage with the strike.
                \rank{6} The damage bonus increases to \plus2d.
                \rank{8} The damage bonus increases to \plus3d.
            \end{apability}
        }

        \cf{Rgr}[3]{Wilderness Lore} You gain two extra skill points.

        \cf*{Rgr}[4]{Wild Exertion}
        You learn an additional \textit{wild exertion}.

        \cf{Rgr}[5]{Greater Wanderer's Strike} The distance you can move with your \textit{wanderer's strike} ability is increased to your full movement speed.

        \cf*{Rgr}[6]{Wild Exertion} 
        You learn an additional \textit{wild exertion}.

        \cf{Rgr}[7]{Wilderness Mastery} You gain a \plus2 bonus to the Creature Handling, Heal, Knowledge (geography), Knowledge (nature), Ride, and Survival skills.

        \cf*{Rgr}[8]{Wild Exertion} 
        You learn an additional \textit{wild exertion}.

    \subsection{Huntmaster}
        This archetype grants you and your allies abilities to hunt down specific foes.

        \cf{Rgr}[1]{Quarry}\label{Quarry} You can use the \textit{quarry} ability as a \glossterm{minor action}.
        \begin{attuneability}{Quarry}[\glossterm{Attune} (self)]
            Choose a creature within \rnglong range.
            The target becomes your quarry.
            In addition, choose any number of willing creatures within the same range.
            Each target gains a \plus1 bonus to \glossterm{accuracy} with \glossterm{physical attacks} against your quarry.
            You and the willing creatures affected by this ability are called your \glossterm{hunting party}.
            In addition, you gain a \plus5 bonus to checks made to follow the target's tracks.
        \end{attuneability}

        \cf{Rgr}[2]{Tracker}
        You gain a \plus2 bonus to checks made to follow tracks.
        In addition, you may use your \textit{quarry} ability on any creature whose tracks you are following, regardless of the creature's current location.

        \cf{Rgr}[3]{Hunting Styles}
        You learn specific hunting styles to defeat particular quarries.
        Choose one hunting style from the list below.
        You can also spend \glossterm{insight points} to learn one additional \textit{hunting style} per \glossterm{insight point}.
        When you use your \textit{quarry} ability, you may also use one of your \textit{hunting styles}.
        Each \textit{hunting style} ability lasts as long as the \textit{quarry} ability you used it with.
        {
            \begin{freeability}{Anchoring}[Magical]
                As long as your quarry is \glossterm{threatened} by at least two members of your \glossterm{hunting party}, it cannot travel extradimensionally.
                This prevents all \glossterm{Manifestation}, \glossterm{Planar}, and \glossterm{Teleportation} effects.

                \rankline
                \rank{5} This effect instead applies if your quarry is within \rngmed range of at two members of your hunting party.
                \rank{7} This effect instead applies if your quarry is within \rnglong range of any member of your hunting party.
            \end{freeability}

            \begin{freeability}{Brutal Assault}
                The accuracy bonus from your \textit{quarry} ability is replaced with a \minus1 penalty to accuracy with \glossterm{strikes} against your quarry.
                In exchange, your hunting party gains a \plus1d \glossterm{magic bonus} to damage with \glossterm{strikes} against your quarry.

                \rankline
                \rank{5} The accuracy penalty is removed.
                \rank{7} The damage bonus is increased to \plus2d.
            \end{freeability}

            \begin{freeability}{Coordinated Stealth}
                Your quarry takes a \minus5 penalty to Awareness checks to notice members of your \glossterm{hunting party}.

                \rankline
                \rank{5} Your hunting party gains a \plus1d bonus to damage against your quarry if it is \unaware of every member of the hunting party.
                \rank{7} The damage bonus increases to \plus2d.
            \end{freeability}

            \begin{freeability}{Cover Weaknesses}
                The accuracy bonus against your quarry is replaced with a \plus1 bonus to defenses against your quarry's attacks.

                \rankline
                \rank{5} Your hunting party gains \glossterm{damage reduction} equal to half your level against your quarry's attacks.
                \rank{7} The damage reduction increases to be equal to your level.
            \end{freeability}

            \begin{freeability}{Decoy}
                If you \glossterm{threaten} your quarry, it takes a \minus2 penalty to accuracy on attacks against members of your \glossterm{hunting party} other than you.

                \rankline
                \rank{5} This penalty increases to \minus3.
                \rank{7} This penalty increases to \minus4.
            \end{freeability}

            \begin{freeability}{Lifeseal}[Magical]
                As long as your quarry is \glossterm{threatened} by at least two members of your \glossterm{hunting party}, it cannot regain hit points.

                \rankline
                \rank{5} This effect instead applies if the target is within \rngmed range of at two members of your hunting party.
                \rank{7} This effect instead applies if your quarry is within \rnglong range of any member of your hunting party.
            \end{freeability}

            \begin{freeability}{Martial Suppression}
                As long as your quarry is \glossterm{threatened} by at least two members of your \glossterm{hunting party}, it takes a \minus1 penalty to accuracy with \glossterm{physical attacks}.

                \rankline
                \rank{5} The penalty increases to \minus2 if your quarry is threatened by at least two members of your hunting party.
                \rank{7} The penalty increases to \minus3 if your quarry is threatened by at least three members of your hunting party.
            \end{freeability}

            \begin{freeability}{Mystic Guidance}
                The accuracy bonus from your \textit{quarry} ability applies to all attacks your \glossterm{hunting party} makes against your quarry, instead of only to \glossterm{physical attacks}.
                This is a \glossterm{magical} ability.

                \rankline
                \rank{5} Your hunting party gains a \plus1 bonus to Fortitude, Reflex, and Mental defenses against attacks from your quarry.
                \rank{7} The accuracy bonus against your quarry is increased to \plus2.
            \end{freeability}

            \begin{freeability}{Mystic Suppression}
                As long as your quarry is \glossterm{threatened} by at least two members of your \glossterm{hunting party}, it takes a \minus1 penalty to \glossterm{accuracy} with \glossterm{magical} attacks.

                \rankline
                \rank{5} The penalty increases to \minus2 if your quarry is threatened by at least two members of your hunting party.
                \rank{7} The penalty increases to \minus3 if your quarry is threatened by at least three members of your hunting party.
            \end{freeability}

            \begin{freeability}{Solo Hunter}
                Your hunting party other than you gains no benefit from your \textit{quarry} ability.
                In exchange, you gain a \plus1 bonus to defenses against your quarry.

                \rankline
                \rank{5} The accuracy bonus against your quarry increases to \plus2.
                \rank{7} The defense bonus increases to \plus2.
            \end{freeability}

            \begin{freeability}{Swarm Hunter}
                When you use your \textit{quarry} ability, you can target any number of creatures.

                \rankline
                \rank{5} You gain a \plus1 bonus to \glossterm{overwhelm resistance}.
                \rank{7} The bonus to overwhelm resistance applies to all members of your \glossterm{hunting party}.
            \end{freeability}

            \begin{freeability}{Unerring}
                Your \glossterm{hunting party} rolls twice and takes the better result for miss chances on attacks against your quarry, such as from \glossterm{active cover}.

                \rankline
                \rank{5} Your hunting party also ignores the defense bonus from \glossterm{cover} on attacks against your quarry.
                \rank{7} Your hunting party ignores all miss chances on attacks against your quarry.
            \end{freeability}

            \begin{freeability}{Wolfpack}
                As long as your quarry is \glossterm{threatened} by at least two members of your \glossterm{hunting party}, it is \glossterm{slowed}.

                \rankline
                \rank{5} The quarry is \glossterm{decelerated} instead of being slowed.
                \rank{7} The accuracy bonus against your quarry increases to \plus2 if it is threatened by at least two members of your hunting party.
            \end{freeability}
        }

        \cf{Rgr}[4]{Hunting Lore} You gain two extra skill points.

        \cf*{Rgr}[5]{Hunting Style}
        You learn an additional \textit{hunting style}.

        \cf{Rgr}[6]{Fluid Style}
        At the start of each round, you can change which \textit{hunting style} you are using.
        This does not cost an \glossterm{action point} or change the target of your \textit{quarry} ability.

        \cf*{Rgr}[7]{Hunting Style}
        You learn an additional \textit{hunting style}.

        \cf{Rgr}[8]{Dual Quarry} You can attune to two separate uses of your \textit{quarry} ability simultaneously.
        You can use different \textit{hunting style} abilities with each \textit{quarry}.

\newpage
\section{Rogue}\label{Rogue}
    \begin{dtable}
        \lcaption{Rogue Progression}
        \begin{dtabularx}{\columnwidth}{c >{\lcol}X >{\lcol}X >{\lcol}X}
            \tb{Rank} & \tb{Assassin} & \tb{Jack of All Trades}  & \tb{Scoundrel} \\\bottomrule
               1 & Sneak attack         & Skill lore             & Combat tricks
            \\ 2 & Uncanny Dodge        & Skill exemplar         & Tricky lore
            \\ 3 & Darkstalker          & Skill lore             & Combat trick
            \\ 4 & Greater sneak attack & Greater skill exemplar & Tricky lore
            \\ 5 & Assassinate, evasion & Skillful defense       & Combat trick
            \\ 6 & Supreme sneak attack & Supreme skill exemplar & Skill trick
            \\ 7 & Greater darkstalker  & Legendary fortune      & Combat trick
            \\ 8 &                      &                        &
        \end{dtabularx}
    \end{dtable}

    \classbasics{Alignment} Any.

    \classbasics{Archetypes} Rogues have the Assassin, Jack of All Trades, and Scoundrel \glossterm{archetypes}.

    \subsection{Basic Class Abilities}
        If you are a rogue, you gain the following abilities.

        \cf{Rog}{Defenses}
        You gain the following bonuses to your \glossterm{defenses}: \plus3 Fortitude, \plus5 Reflex, \plus4 Mental.

        \cf{Rog}{Skills}
        You have the following \glossterm{class skills}:
        \begin{itemize}
            \item \subparhead{Strength} Climb, Jump, Swim.
            \item \subparhead{Dexterity} Acrobatics, Escape Artist, Sleight of Hand, Stealth.
            \item \subparhead{Intelligence} Craft, Deduction, Devices, Disguise, Knowledge (dungeoneering, local), Linguistics.
            \item \subparhead{Perception} Awareness, Sense Motive.
            \item \subparhead{Other} Bluff, Intimidate, Perform, Persuasion, Profession.
        \end{itemize}

        \cf{Rog}{Weapon and Armor Proficiencies}
        Rogues are proficient with simple weapons, any two other weapon groups, light armor, and bucklers.
        They are also proficient with saps.

    \subsection{Assassin}
        This archetype improves your agility, stealth, and combat prowess against unaware targets.

        \cf{Rog}[1]{Assassination} You can use the \textit{assassination} ability as a \glossterm{standard action}.
        \begin{freeability}{Assassination}
            You study a creature within \rngclose range, finding weak points you can take advantage of.
            Until the end of the next round, if you make a melee \glossterm{strike} against the target while it is \unaware, your attack deals maximum damage.
        \end{freeability}

        % TODO: wording
        \cf{Rog}[2]{Evasion} When you are attacked by an ability that affects an area, you can use your Reflex defense in place of any other defenses against that attack.

        \cf*{Rog}[3]{Darkstalker}[Magical] You can use the \textit{darkstalker} ability as a standard action.
        \begin{attuneability}{Darkstalker}[\glossterm{Attune} (self)]
            You become completely undetectable by your choice of one of the following senses:
            \begin{itemize}
                \item Blindsense and blindsight
                \item Darkvision
                \item Scent
                \item Tremorsense and tremorsight
            \end{itemize}
        \end{attuneability}

        \cf{Rog}[4]{Greater Assassination} You can use your \textit{assassination} ability as a \glossterm{minor action}.
        In addition, you can target a creature within \rnglong range.

        \cf{Rog}{5}{Greater Darkstalker} When you use your \textit{darkstalker} ability, you become undetectable by your choice of two of the senses instead of only one.

        % TODO: wording
        \cf{Rog}[6]{Hide in Plain Sight} You can use the Stealth skill to hide while observed if you take a \minus5 penalty.
        You must still have cover or concealment to hide successfully.

        \cf{Rog}[7]{Supreme Darkstalker}[Magical] When you use your \textit{darkstalker} ability, you become undetectable by all of the listed senses, not just one.

    \subsection{Jack of All Trades}
        This archetype improves your skills.

        \cf{Rog}{Skill Lore} You gain three extra skill points.

        \cf{Rog}[2]{Skill Exemplar} You gain a \plus1 bonus to all skills.

        \cf*{Rog}[3]{Skill Lore} You gain three extra skill points.

        \cf{Rog}[4]{Greater Skill Exemplar} The skill bonus from your \textit{skill exemplar} ability increases to \plus2.

        \cf{Rog}[5]{Skillful Defense} You gain a \plus1 bonus to all defenses.

        \cf{Rog}[6]{Supreme Skill Exemplar} The skill bonus from your \textit{skill exemplar} ability increases to \plus3.

        \cf{Rog}[8]{Legendary Fortune} When you make a skill attack or check, you roll twice and take either result.

    \subsection{Scoundrel}
        This archetype grants you abilities to use in combat and improves your combat prowess.

        \cf{Rog}[1]{Sneak Attack} You can use the \textit{sneak attack} ability as a standard action.
        \begin{freeability}{Sneak Attack}
            Make a \glossterm{strike} against a creature within \rngclose range.
            If the target is \unaware, \defenseless, or \glossterm{overwhelmed}, you gain a \plus2d bonus to damage with the strike.
            You do not gain this damage bonus against creatures who are immune to \glossterm{critical hits} or who lack a discernible body structure, such as oozes.
        \end{freeability}

        \cf{Rog}[2]{Combat Tricks}
        You can confuse and confound your foes in combat.
        Choose one \textit{combat trick} from the list below.
        You can also spend \glossterm{insight points} to learn one additional \textit{combat trick} per \glossterm{insight point}.
        As a standard action, you can spend an \glossterm{action point} to use a \textit{combat trick} ability.
        {
            \begin{apability}{Certain Strike}
                Make a \glossterm{strike} with a \plus3 bonus to accuracy.

                \rankline
                \rank{4} The accuracy bonus increases to \plus4.
                \rank{6} The accuracy bonus increases to \plus5.
                \rank{8} The accuracy bonus increases to \plus6.
            \end{apability}

            \begin{apability}{Counterattack}
                Make a \glossterm{strike}.
                If the target attacked you in the same phase, you gain a \plus2 bonus to accuracy and a \plus2d bonus to damage.
                Otherwise, you regain the action point spent to use this ability.

                \rankline
                \rank{4} The damage bonus increases to \plus3d.
                \rank{6} The damage bonus increases to \plus4d.
                \rank{8} The damage bonus increases to \plus5d.
            \end{apability}

            \begin{apability}{Daunting Blow}[\glossterm{Mind}]
                Make a melee \glossterm{strike}.
                If the target takes damage from the strike, it is \glossterm{shaken} by you as a \glossterm{condition}.
                Otherwise, you regain the \glossterm{action point} spent to use this ability.

                \rankline
                \rank{4} You gain a \plus1 bonus to accuracy on the strike.
                \rank{6} The accuracy bonus increases to \plus2.
                \rank{8} The accuracy bonus increases to \plus3.
            \end{apability}

            \begin{apability}{Hamstring}
                Make a \glossterm{strike} with a slashing or piercing weapon.
                If the target takes damage from the strike, it is \glossterm{slowed} as a \glossterm{condition}.
                Otherwise, you regain the \glossterm{action point} spent to use this ability.

                \rankline
                \rank{4} You gain a \plus1 bonus to accuracy on the strike.
                \rank{6} The accuracy bonus increases to \plus2.
                \rank{8} The accuracy bonus increases to \plus3.
            \end{apability}

            \begin{apability}{Head Shot}[\glossterm{Mind}]
                Make a \glossterm{strike} with a bludgeoning weapon.
                If the target takes damage from the strike, it is \glossterm{dazed} as a \glossterm{condition}.
                Otherwise, you regain the \glossterm{action point} spent to use this ability.

                \rankline
                \rank{4} You gain a \plus1 bonus to accuracy on the strike.
                \rank{6} The accuracy bonus increases to \plus2.
                \rank{8} The accuracy bonus increases to \plus3.
            \end{apability}

            \begin{apability}{Liver Shot}
                Make a \glossterm{strike} with a bludgeoning weapon.
                If the target takes damage from the strike, it is \sickened as a \glossterm{condition}.
                Otherwise, you regain the \glossterm{action point} spent to use this ability.

                \rankline
                \rank{4} You gain a \plus1 bonus to \glossterm{accuracy} with the strike.
                \rank{6} The accuracy bonus increases to \plus2.
                \rank{8} The accuracy bonus increases to \plus3.
            \end{apability}

            \begin{apability}{Penetrating Strike}
                Make a \glossterm{strike} with a piercing weapon.
                The attack is made against the target's Reflex defense instead of its Armor defense.

                \rankline
                \rank{4} You gain a \plus1d bonus to damage with the strike.
                \rank{6} The damage bonus increases to \plus2d.
                \rank{8} The damage bonus increases to \plus3d.
            \end{apability}

            \begin{apability}{Rapid Assault}
                Make a \glossterm{strike} against a creature.
                If you use this ability during the \glossterm{action phase}, you can make another strike during the \glossterm{delayed action phase}.
                You take a \minus2 penalty to accuracy on both strikes.

                \rankline
                \rank{4} The accuracy penalty is reduced to \minus1.
                \rank{6} The accuracy penalty is removed.
                \rank{8} You gain a \plus1 bonus to accuracy with both strikes.
            \end{apability}

            \begin{apability}{Strip the Flesh}
                Make a \glossterm{strike} with a slashing weapon.
                At the end of the \glossterm{action phase} of the next round, if you hit with the strike and the target is not \glossterm{bloodied}, it takes additional damage equal to the damage you dealt with the strike.

                \rankline
                \rank{4} If you hit with the strike, the target continues taking the same damage at the end of each \glossterm{action phase} until it becomes \glossterm{bloodied}.
                This is a \glossterm{condition}, and can be removed by abilities that remove conditions.
                \rank{6} You gain a \plus1d bonus to damage with the strike.
                \rank{8} The damage bonus increases to \plus2d.
            \end{apability}

            \begin{apability}{Sweeping Strike}
                Make a melee \glossterm{strike} with a slashing or bludgeoning weapon.
                The strike targets each of up to three creatures or objects you \glossterm{threaten}.
                You take a \minus1 penalty to \glossterm{accuracy} with the strike.

                \rankline
                \rank{4} You gain a \plus1d bonus to damage with the strike.
                \rank{6} The damage bonus increases to \plus2d.
                \rank{8} The damage bonus increases to \plus3d.
            \end{apability}
        }

        \cf{Rog}[3]{Tricky Lore} You gain two extra skill points.

        \cf*{Rog}[4]{Combat Trick}
        You learn an additional \textit{combat trick}.

        \cf{Rog}[5]{Greater Sneak Attack} The damage bonus from your \textit{sneak attack} ability increases to \plus3d.

        \cf*{Rog}[6]{Combat Trick}
        You learn an additional \textit{combat trick}.

        \cf{Rog}[7]{Skill Trick} % TODO

        \cf*{Rog}[8]{Combat Trick}
        You learn an additional \textit{combat trick}.

    % \subsection{Bard}
    %     This archetype grants you the ability to inspire your allies and impair your foes with magical performances.

    %     \cf{Rog}{1}{Bardic Performances}
    %     Choose two \textit{bardic performances} from the list below.
    %     As a standard action, you can spend an \glossterm{action point} to use a \textit{bardic performance}.
    %     All \textit{bardic performances} have the \glossterm{Sustain} (minor) tag.

    %     When you use a \textit{bardic performance} ability, you begin a performance using one of your Perform skills.
    %     Depending on the nature of your performance, the ability may have the \glossterm{Auditory} or \glossterm{Visual} tags.
    %     If a target can neither see nor hear your performance, the effect immediately ends for that target.
    %     {
    %         \begin{apability}{Mesmerizing Performance}[\glossterm{Delusion}, \glossterm{Mind}]
    %             Make an attack vs. Mental against up to five creatures within \rngmed range.
    %             \hit Each target is \fascinated by you.
    %             Any act by you or your apparent allies that damages a target or that causes it to feel that it is in danger breaks the effect for that creature.
    %             An observant target may interpret overt threats to its allies as a threat to itself.

    %             \rankline
    %             \rank{3} 

    %         \end{apability}

    %         \begin{apability}{Inspiring Performance}[\glossterm{Delusion}, \glossterm{Mind}]
    %             Choose a willing creature within \rngmed range.
    %             The target gains a \plus1 \glossterm{magic bonus} to \glossterm{accuracy} and a \plus2 \glossterm{magic bonus} to \glossterm{checks}.
    %         \end{apability}
    %     }

\newpage
\section{Warlock}\label{Warlock}
    \begin{dtable}
        \lcaption{Warlock Progression}
        \begin{dtabularx}{\columnwidth}{c >{\lcol}X >{\lcol}X >{\lcol}X}
            \tb{Rank} & \tb{Pact Spellcasting} & \tb{Blessings of the Abyss}  & \tb{Pact Magic} \\\bottomrule
               1 & Mystic spheres         & Whispers of the lost         & Wellspring of magic
            \\ 2 & Pact augments          & Armor tolerance              & Malevolent boon
            \\ 3 & Mystic sphere          & Abyssal lore                 & Augments
            \\ 4 & Pact augment           & Greater armor tolerance      & Greater malevolent boon
            \\ 5 & Infernal power         & Greater whispers of the lost & Augment
            \\ 6 & Pact augment           & Supreme armor tolerance      & Greater wellspring of magic
            \\ 7 & Greater infernal power & Abyssal defense              & Augment
            \\ 8 & Dual pact              &                              &
        \end{dtabularx}
    \end{dtable}

    \classbasics{Alignment} Any.

    \classbasics{Archetypes} Warlocks have the Pact Spellcasting, Blessings of the Abyss, and Pact Magic \glossterm{archetypes}.

    \subsection{Basic Class Abilities}
        If you are a warlock, you gain the following abilities.

        \cf{War}{Defenses}
        You gain the following bonuses to your \glossterm{defenses}: \plus4 Fortitude, \plus3 Reflex, \plus5 Mental.

        \cf{War}{Skills}
        You have the following \glossterm{class skills}:
        \begin{itemize}
            \item \subparhead{Dexterity} Ride.
            \item \subparhead{Intelligence} Craft, Deduction, Disguise, Knowledge (arcana, planes, religion), Linguistics.
            \item \subparhead{Perception} Awareness, Sense Motive, Spellcraft.
            \item \subparhead{Other} Bluff, Intimidate, Persuasion, Profession.
        \end{itemize}

        \cf{War}{Weapon and Armor Proficiencies}
        You are proficient with simple weapons, any two other weapon groups, light and medium armor, and shields.

        \cf{War}{Infernal Pact}[Magical]
        To become a warlock, you must make a pact with a powerful demon or devil.
        % TODO: wording; intended to prevent Null warlocks who have no pact
        If you somehow lose this ability, you lose all other warlock abilities.
        You must make a dark sacrifice, the details of which are subject to negotiation, and offer a part of your immortal soul.
        In exchange, you gain the powers of a warlock.
        The creature you make the pact with is called your soulkeeper.

        Offering your soul to an entity in this way grants it the ability to communicate with you in limited ways.
        This communication typically manifests as unnatural emotional urges or whispered voices audible only to you.

        Your pact specifies how much of your soul is granted to your soulkeeper, and the circumstances of the transfer.
        The most common arrangement is for a soulkeeper to gain possession of your soul immediately after you die.
        It will keep the soul for one decade per year of your life that you spend as a warlock.
        During that time, it will not prevent you from being resurrected.
        At the end of that time, if your soul remains intact, your soul will pass on to its intended afterlife.
        However, other arrangements are possible, and each warlock's pact can be unique.

        The longer you spend in an afterlife that is not your own, the more likely you are to lose your sense of self and become subsumed by the plane you are on.
        Only a soul of extraordinary strength can maintain its integrity after decades or centuries in the Abyss.
        Many warlocks seek power zealously while mortal to gain the mental fortitude necessary to keep their soul after death.

    \subsection{Pact Spellcasting}
        This archetype grants you the ability to cast pact spells through the power of your pact.
        You must have a starting Willpower of at least 1 to gain this archetype.

        \cf{War}[1]{Pact Spell Failure}
        When you cast a pact spell or perform a pact ritual while using \glossterm{armor}, you must roll 1d10.
        If your result is less than or equal to your \glossterm{encumbrance}, you \glossterm{miscast} the ability (see \pcref{Miscasting}).
        When you perform a ritual, this roll must be repeated at the end of each action phase during the ritual.

        \cf{War}[1]{Mystic Spheres}[Magical]
        You can use pact magic.
        You gain access to two pact \glossterm{mystic spheres} (see \pcref{Pact Mystic Spheres}).
        As a \glossterm{standard action}, you can cast the \glossterm{cantrip} spell from any mystic sphere you have access to.
        For details about spellcasting, see \pcref{Spell and Ritual Mechanics}.

        When you gain access to a pact \glossterm{mystic sphere}, you automatically learn one 1st level spell of your choice from that \glossterm{mystic sphere}.
        You cannot learn pact spells by any other means.
        As a \glossterm{standard action}, you can spend an \glossterm{action point} to cast any \glossterm{spell} you know.

        \cf{War}[2]{Pact Augments}[Magical] You learn one pact augment from the list below.
        You can also spend \glossterm{insight points} to learn one additional pact augment per \glossterm{insight point}.
        Pact augments function like other \glossterm{augments} (see \pcref{Augments}), with the following exceptions.
        \begin{itemize}
            \item Pact augments can only be applied to pact spells, including \glossterm{cantrips}.
            \item To apply a pact augment to a spell, you must spend an \glossterm{action point}.
            \item You can only apply a single pact augment to a spell.
                You can apply that pact augment in addition to any other non-pact augments the spell has.
        \end{itemize}
        When you gain access to a new pact spell level, you may change which pact augments you know.
        Your \glossterm{power} with pact augments is equal to your \glossterm{power} with the spell you apply them to.
        {
            \augment{0}{Infernal Focus} You gain a \plus10 bonus to \glossterm{concentration} checks to cast the spell (see \pcref{Concentration}).

            \augment{0}{Sickening} Make an attack vs. Fortitude against one creature targeted by the spell.
            On a hit, the target is \glossterm{sickened} as a \glossterm{condition}.
            \par This augment can be applied to any spell that targets at least one creature.

            \augment{0}{Soulrending} When a creature is dealt damage by the spell, if it has no hit points remaining, it dies.
            \par This augment can be applied to any spell that deals damage.

            \augment{0}{Terrifying} Make an attack vs. Mental against one creature targeted by the spell.
            On a hit, the target is \glossterm{shaken} by you as a \glossterm{condition}.
            \par This augment can be applied to any spell that targets at least one creature.

            \augment{1}{Dark Expansion} The spell's area is increased by one step, to a maximum of \areahuge.
            The steps are, in order: \areasmall, \areamed, \arealarge, and \areahuge.
            Normally, a Small or Medium line is 5 ft.\ wide, while a Large or Huge line is 10 ft.\ wide.
            A line used to define a wall does not have a width.
            \par This augment can be applied to any spell with an area that is one of the above areas.

            \augment{1}{Untraceable} You do not need \glossterm{verbal components} or \glossterm{somatic components} to cast the spell.

            \augment{2}{Abyssal Flame} Non-physical damage dealt by the spell becomes life damage in place of its other damage types.
            In addition, you gain a \plus1d bonus to life damage dealt by the spell.
            This does not change any other aspects of the spell's effects.
            \par This augment can be applied to any spell that deals non-physical damage.

            \augment{2}{Bloodreaving} When a creature takes damage from the spell, you heal hit points equal to the damage dealt.
            \par This augment can be applied to any spell that deals damage.

            % TODO: taking damage multiple times in the same round abuse?
            \augment{2}{Exsanguinating} When a creature takes damage from the spell, it begins bleeding as a \glossterm{condition}.
            At the end of each \glossterm{action phase}, a bleeding creature takes life \glossterm{standard damage} \minus1d.
            \par This augment can be applied to any spell that deals damage.

            \augment{3}{Inevitable Doom} You gain a \plus2 bonus to accuracy with the spell.
            \par This augment can be applied to any spell that makes an attack.

            \augment{3}{Infernal Potency} You gain a \plus2d bonus to damage with the spell.
            \par This augment can be applied to any spell that deals damage based on a dice pool.

            \augment{4}{Dark Miasma} When a creature takes damage from the spell, it becomes \glossterm{sickened} as a \glossterm{condition}.
            \par This augment can be applied to any spell that deals damage.

            \augment{4}{Unholy Terror} When a creature takes damage from the spell, it becomes \glossterm{shaken} by you as a \glossterm{condition}.
            \par This augment can be applied to any spell that deals damage.

            \augment{5}{Soulclaiming} When a creature takes damage from the spell, its soul becomes marked by your soulkeeper as a \glossterm{condition}.
            It takes a \minus4 penalty to Mental defense.
            If it dies, its soul is immediately claimed by your soulkeeper, preventing it from being resurrected by any means.

            \augment{6}{Greater Inevitable Doom} You gain a \plus4 bonus to accuracy with the spell.
            \par This augment can be applied to any spell that makes an attack.

            \augment{6}{Greater Infernal Potency} You gain a \plus4d bonus to damage with the spell.
            \par This augment can be applied to any spell that deals damage based on a dice pool.
        }

        \cf*{War}[3]{Mystic Sphere}[Magical]
        You gain access to an additional pact \glossterm{mystic sphere} (see \pcref{Pact Mystic Spheres}).

        \cf{War}[4]{Pact Augment}[Magical] You learn an additional \textit{pact augment}.

        \cf{War}[5]{Infernal Power}[Magical]
        You gain a \plus1 bonus to \glossterm{power} with pact spells.

        \cf*{War}[6]{Pact Augment}[Magical] You learn an additional \textit{pact augment}.

        \cf{War}[7]{Greater Infernal Power}[Magical]
        The bonus from your \textit{infernal power} ability increases to \plus2.

        \cf{War}[8]{Dual Pact}[Magical] You can apply two different \textit{pact augments} to the same spell, paying separate action points for each.

    \subsection{Blessings of the Abyss}
        This archetype grants you powers relating to the connection to the Abyss granted by your pact.

        \cf{War}[1]{Whispers of the Lost}[Magical] You hear the voices of souls lost to the Abyss, linked to you through your soulkeeper.
        Choose one of the following types of whispers that you hear.
        {
            \subcf{Mentoring Whispers} You hear the voice of a dead warlock whose soul is bound to the same soulkeeper.
            This provides no benefit immediately, but the warlock can sometimes provide good advice.

            \subcf{Sycophantic Whispers} You hear the voices of adoring souls who praise your talents and everything you do.
            You gain a \plus2 bonus to Mental defense.

            \subcf{Warning Whispers} You hear the voices of paranoid and fearful souls warning you of danger, both real and imagined.
            You gain a \plus2 bonus to Reflex defense.

            \subcf{Whispers of the Mighty} Your soulkeeper forges the connection to your soul into a boon granted to any soul in the Abyss strong enough to claim it in battle.
            You hear the voice of whatever soul currently possesses the boon - until that soul is defeated.
            You gain a \plus2 bonus to Fortitude defense.
        }

        \cf{War}[2]{Armor Tolerance} You reduce your \glossterm{encumbrance} by 2 when determining your \textit{pact spell failure}.

        \cf{War}[3]{Abyssal Lore} You gain two additional skill points.

        \cf{War}[4]{Greater Armor Tolerance} The penalty reduction from your \textit{armor tolerance} ability increases to 3.

        \cf{War}[5]{Greater Whispers of the Lost}[Magical] You gain an additional ability depending on the voices you chose with your \textit{whispers of the lost} ability.
        {
            \subcf{Mentoring Whispers} You gain a \plus1 bonus to \glossterm{accuracy} and \glossterm{power} with pact spells.

            \subcf{Sycophantic Whispers} The defense bonus increases to \plus4.
            % Does Compulsion make sense?
            In addition, you are immune to \glossterm{Compulsion} and \glossterm{Fear} abilities.

            \subcf{Warning Whispers} The defense bonus increases to \plus4.
            In addition, you are not \glossterm{unaware} when attacked from surprise.

            \subcf{Whispers of the Mighty} The defense bonus increases to \plus4.
            In addition, you gain additional hit points equal to your level.
        }

        \cf{War}[6]{Supreme Armor Tolerance}[Magical] The penalty reduction from your \textit{armor tolerance} ability increases to 4.

        \cf{War}[7]{Abyssal Defense}[Magical] You gain a \plus1 bonus to Fortitude, Reflex, and Mental defenses.

    \subsection{Pact Magic}
        This archetype improves your ability to cast spells with the power of your dark pact.
        You must have the Pact Spellcasting archetype to gain the abilities from this archetype.

        \cf{War}[1]{Wellspring of Power}[Magical]
        You gain a \plus1 bonus to \glossterm{power} with pact spells.

        \cf{War}[2]{Malevolent Boon}[Magical]
        When you cast a pact spell that makes an attack roll, if you miss, you may regain the \glossterm{action point} spent to cast the spell (if any).
        If you are \glossterm{attuned} to the spell, you cannot regain an action point in this way.
        You cannot use this ability on spells that do not normally have attacks, even if you apply an \glossterm{augment} that makes a separate attack.

        \cf{War}[3]{Augments}[Magical]
        Choose one \glossterm{augment} (see \pcref{Augments}).
        You can also spend \glossterm{insight points} to learn one additional augment per \glossterm{insight point}.
        You can apply those augments to pact spells you cast.
        When you gain access to a new spell level, you may change which augments you know.

        \cf{War}[4]{Greater Malevolent Boon}[Magical] You can cast pact spells that make attack rolls without spending \glossterm{action points}.
        You cannot \glossterm{attune} to a spell you cast in this way.
        You cannot use this ability on spells that do not normally have attacks, even if you apply an \glossterm{augment} that makes a separate attack.

        \cf*{War}[5]{Augment}[Magical]
        You learn an additional \glossterm{augment} that you can apply to pact spells.

        \cf{War}[6]{Greater Wellspring of Power}[Magical]
        The bonus from your \textit{wellspring of power} ability increases to \plus2.

        \cf{War}[7]{Augment}[Magical]
        You learn an additional \glossterm{augment} that you can apply to pact spells.
