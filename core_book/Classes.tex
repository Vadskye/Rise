\chapter{Classes}\label{Classes}

Your character's class represents the things your character has chosen to train in.
This choice determines a great deal about your character's abilities.

\section{How Classes Work}
    When you first create a character, you choose a class.
    Your character has one level in that class.
    This grants your character all of the abilities your chosen class grants at 1st level, as given in the class description.
    Each time your character gains a level, you can choose to increase your level in your original class or gain a level in a new class.
    This grants your character all of the abilities your chosen class grants at the level your character just gained in it.

    \subsection{Base Classes}
        Every character has one \glossterm{base class}.
        You may choose any class your character has at least one level in as a base class.
        Whenever your character gain a level, you can change its base class to a different class it has.
        Your choice of base class affects your character's \glossterm{defenses}, \glossterm{skill points}, and \glossterm{class skills}.
        In addition, every class grants additional abilities if it is chosen as a base class, as given in the class description.

\section{Class Introductions}

    There are eleven classes in Rise.
    \begin{itemize}
        \item Barbarians are mighty warriors who can enter a deadly battlerage.
        \item Clerics are divine spellcasters who draw power from their veneration of a deity or ideal.
        \item Druids are nature spellcasters who draw power from their veneration of the natural world.
        \item Fighters are highly disciplined warriors who excel in physical combat of any variety.
        \item Wizards are arcane spellcasters who wield the mystic forces of magic to create almost any effect.
        \item Monks are agile masters of ``\ki'' who hone their personal abilities to strike down foes and perform supernatural feats.
        \item Paladins are divinely empowered warriors whose devotion to an alignment grants them the ability to discern and smite their foes.
        \item Rangers are skilled hunters who bridge the divide between nature and civilization.
        \item Rogues are exceptionally skillful characters known for their ability to strike at their foe's weak points in combat.
        \item Spellwarped wield a unique blend of martial skill and narrowly focused magical abilities.
    \end{itemize}

    \subsection{Class Description Format}
        Each class is described from the perspective of a member of that class, using ``you'' in the description.

        \parhead{Class Table}
        The class's table describes the special abilities they get at each level.

        \parhead{Alignment}
        Some classes require specific alignments (see \pcref{Alignment}).
        Most classes allow characters of any alignment.

        \parhead{Class Skills}
        These are skills that members of this class are typically good at (see \pcref{Skills}).

        \subsubsection{Base Class Abilities}
            Abilities contained within this heading only apply to characters with the current class as a \glossterm{base class}.

            \parhead{Skill Points}
            This is the number of skill points that members of this class get.

            \parhead{Defenses}
            Each class grants bonuses to specific defenses.
            These bonuses do not stack with other defense bonuses granted by base classes.
            If a character has multiple base classes, use the highest bonuses that apply to each defense.

            \parhead{Weapon and Armor Proficiencies}
            These are the types of equipment that members of this class are trained in using.

        \subsubsection{Class Abilities}
            The class abilities that a character gets for being a member of the class.

\section{Barbarian}\label{Barbarian}
    \begin{dtable}
        \lcaption{Barbarian Progression}
        \begin{dtabularx}{\columnwidth}{>{\ccol}p{\levelcol} >{\lcol}X}
            \tb{Level} & \tb{Special}                      \\
            \bottomrule
            \nth{1}    & Rage                              \\
            \nth{2}    & Battle-scarred                    \\
            \nth{3}    & Athletic prowess                  \\
            \nth{4}    & Uncontrollable rage               \\
            \nth{5}    & Uncanny dodge                     \\
            \nth{6}    & Titanic might                     \\
            \nth{7}    & Unstoppable rage                  \\
            \nth{8}    & Deep scars                        \\
            \nth{9}    & Improvised combat                 \\
            \nth{10}   & Mindless rage                     \\
            \nth{11}   & Greater uncanny dodge             \\
            \nth{12}   & Greater athletic prowess          \\
            \nth{13}   & Mighty rage                       \\
            \nth{14}   & Rapid recovery                    \\
            \nth{15}   & Greater improvised combat         \\
            \nth{16}   & Endless rage                      \\
            \nth{17}   & Soulscarred                       \\
            \nth{18}   & Greater titanic might             \\
            \nth{19}   & Blood frenzy                      \\
            \nth{20}   & Deathless rage, fury of the sotrm \\
        \end{dtabularx}
    \end{dtable}

    \classbasics{Alignment} Any nonlawful.

    \classbasics{Class Skills}
    \begin{itemize}
        \item \subparhead{Strength} Climb, Jump, Sprint, Swim.
        \item \subparhead{Dexterity} Acrobatics, Ride.
        \item \subparhead{Perception} Awareness, Creature Handling, Survival.
        \item \subparhead{Other} Bluff, Intimidate, Persuasion.
    \end{itemize}

    \subsection{Base Class Abilities}
        If you choose barbarian as a base class, you gain the following abilities.

        \classbasics{Skill Points} 10.

        \classbasics{Defenses} \plus4 Fortitude, \plus2 Reflex.

        \classbasics{Action Points} \plus1.

        \cf{Bbn}{Weapon and Armor Proficiency}
        You are proficient with simple weapons, any four other weapon groups, light armor, medium armor, and shields.

    \subsection{Class Abilities}
        If you are a barbarian, you have the following abilities.

        \subsubsection{Battlerager}\label{Rage}

            \cf{Bbn}{Rage} 
            As a \glossterm{free action}, you can spend an \glossterm{action point} to use this ability.
            \begin{ability}
                \begin{spelleffects}
                    \spelleffect You have the following benefits and drawbacks:
                    \begin{itemize}
                        \item You gain a \plus1d bonus to damage with \glossterm{strikes}.
                        \item You are unable to take any action that requires patience or concentration, such as casting spells.
                        \item At the end of each round if you did not attack a creature or object, you take \glossterm{nonlethal damage} equal to your level.
                            This damage ignores your damage reduction from this ability.
                    \end{itemize}
                    \spellspecial When this ability ends, you become \fatigued and unable to use it again until you take a \glossterm{short rest}.
                    \spelldur Attunement
                \end{spelleffects}
            \end{ability}

            \cf{Bbn}[4]{Uncontrollable Rage}
            You are immune to \glossterm{Compulsion} effects while raging.

            \cf{Bbn}[7]{Unstoppable Rage}
            You are immune to being \glossterm{staggered} while raging.

            \cf{Bbn}[10]{Mindless Rage}
            You are immune to hostile \glossterm{Mind} effects while raging.

            \cf{Bbn}[13]{Mighty Rage}
            Your damage bonus while raging increases to \plus2d.

            \cf{Bbn}[16]{Endless Rage} 
            You do not have to spend an \glossterm{action point} to use your \textit{rage} ability.

            \cf{Bbn}[19]{Blood Frenzy} 
            You are immune to being \glossterm{bloodied} while raging.

            \cf{Bbn}[20]{Deathless Rage} 
            While raging, the you ignore all penalties from \glossterm{vital damage}.
            However, if your vital damage exceeds your maximum hit points, you immediately die.

        \subsubsection{Battleforged Resilience}
            \cf{Bbn}[2]{Battle-Scarred} You gain \glossterm{damage reduction} against physical damage equal to your level.

            \cf{Bbn}[5]{Uncanny Dodge} You can react to danger before your senses would normally allow you to do so.
            You reduce your \glossterm{overwhelm penalties} by 2.
            If your overwhelm penalty is reduced to 0, you are not considered to be overwhelmed.
            In addition, you is not \unaware when attacked by surprise.

            \cf{Bbn}[8]{Deep Scars} Your \glossterm{damage reduction} from your \textit{battle-scarred} ability applies against all damage, not just physical damage.

            \cf{Bbn}[11]{Greater Uncanny Dodge}
            Your reduction of \glossterm{overwhelm penalties} increases to 4.

            \cf{Bbn}[14]{Rapid Recovery}
            At the end of each round, you heal hit points equal to your level.

            \cf{Bbn}[17]{Soulscarred}
            Your \glossterm{damage reduction} from your \textit{battle-scarred} ability increases to twice your level.

            \cf{Bbn}[20]{Fury of the Storm}
            You gain a \plus1d bonus to damage with \glossterm{strikes} against all creatures who \glossterm{threaten} you.
            In addition, you are immune to being \glossterm{overwhelmed}, and never suffer \glossterm{overwhelm penalties}.

        \subsubsection{Primal Warrior}
            \cf{Bbn}[3]{Athletic Prowess} You gain two additional skill points that must be spent on Strength or Dexterity-based skills.

            \cf{Bbn}[6]{Titanic Might}
            You gain a \plus1d bonus to damage with \glossterm{strikes}.

            \cf{Bbn}[9]{Improvised Combat} 
            You gain a \plus2 bonus to accuracy with \glossterm{combat maneuvers}.

            \cf{Bbn}[12]{Greater Athletic Prowess} You gain two additional skill points that must be spent on Strength or Dexterity-based skills.

            \cf{Bbn}[15]{Greater Improvised Combat} 
            The accuracy bonus from your \textit{improvised combat} ability increases to \plus4.

            \cf{Bbn}[18]{Greater Titanic Might}
            The damage bonus from your \textit{titanic might} ability increases to \plus2d.

        \subsubsection{Ex-Barbarians}
            If you become lawful, you cannot use your \textit{rage} ability.
            You retain all of your other class abilities.
            If you stop being lawful, you can use your \textit{rage} ability once more.

\section{Cleric}\label{Cleric}
    \begin{dtable}
        \lcaption{Cleric Progression}
        \begin{dtabularx}{\columnwidth}{>{\ccol}p{2em} c c >{\lcol}X}
            \tb{Level} & \tb{Spells} & \tb{Augments}\fn{1} & \tb{Special} \\
            \bottomrule
            \nth{1}  & 2 & \tdash   & Domain gift, rituals, spells \\
            \nth{2}  & 2 & \tdash   & Spell point                  \\
            \nth{3}  & 2 & \tdash   & Domain gift                  \\
            \nth{4}  & 2 & 1        & \tdash                       \\
            \nth{5}  & 3 & 1        & Domain spell                 \\
            \nth{6}  & 3 & 2 \add 1 & \tdash                       \\
            \nth{7}  & 3 & 2 \add 1 & Domain aspect                \\
            \nth{8}  & 4 & 3 \add 1 & Expanded spell knowledge     \\
            \nth{9}  & 4 & 3 \add 1 & Domain aspect                \\
            \nth{10} & 4 & 4 \add 2 & \tdash                       \\
            \nth{11} & 4 & 4 \add 2 & Domain essence               \\
            \nth{12} & 4 & 5 \add 2 & \tdash                       \\
            \nth{13} & 4 & 5 \add 2 & Domain essence               \\
            \nth{14} & 4 & 6 \add 3 & \tdash                       \\
            \nth{15} & 4 & 6 \add 3 & Domain mastery               \\
            \nth{16} & 4 & 7 \add 3 & \tdash                       \\
            \nth{17} & 4 & 7 \add 3 & Domain mastery               \\
            \nth{18} & 4 & 8 \add 3 & Spell point                  \\
            \nth{19} & 4 & 8 \add 3 &                              \\
            \nth{20} & 4 & 9 \add 3 & Miracle                      \\
        \end{dtabularx}
        1. Custom augments \add standard augments; see \pcref{Augments}.
    \end{dtable}

    \classbasics{Alignment} Your alignment must be within one step of your deity's (that is, it may be one step away on either the lawful-chaotic axis or the good-evil axis, but not both).

    \classbasics{Class Skills}
    \subparhead{Intelligence} Heal, Knowledge (arcana, local, religion, the planes), Linguistics.
    \subparhead{Perception} Awareness, Sense Motive, Spellcraft.
    \subparhead{Other} Bluff, Intimidate, Persuasion.

    \subsection{Base Class Abilities}
        If you choose cleric as a base class, you gain the following abilities.

        \classbasics{Skill Points} 5.

        \classbasics{Defenses} \plus2 Fortitude, \plus4 Mental.

        \classbasics{Action Points} \plus3.

        \cf{Clr}{Weapon and Armor Proficiency}
        You are proficient with simple weapons, any two other weapon groups, light and medium armor, and shields.

    \subsection{Class Abilities}
        If you are a cleric, you have the following abilities.

        \cf{Clr}{Divine Power}
        The \glossterm{power} of your cleric spells and abilities is determined by your \textit{divine power}.
        Your \textit{divine power} is equal to your level or your Willpower, whichever is higher.

        \cf{Clr}{Deity}
        You must worship a specific deity to be a cleric.
        Deities and their associated \glossterm{domains} are listed in \trefnp{Deities}.

        \begin{dtable!*}
            \lcaption{Deities}
            \begin{dtabularx}{\textwidth}{X l X}
                \tb{Deity} & \tb{Alignment} & \tb{Domains} \\
                \bottomrule
                Guftas, horse god of justice          & Lawful good     & Good, Law, Strength, Travel         \\
                Lucied, paladin god of justice        & Lawful good     & Destruction, Good, Protection, War  \\
                Simor, fighter god of protection      & Lawful good     & Good, Protection, Strength, War     \\
                %Pabst, dwarf god of drink             & Neutral good & Good, Life, Strength, Wild \\
                Rucks, monk god of pragmatism         & Neutral good    & Good, Law, Protection, Travel       \\
                Vanya, centaur god of nature          & Neutral good    & Good, Strength, Travel, Wild        \\
                Brushtwig, pixie god of creativity    & Chaotic good    & Chaos, Good, Trickery, Wild         \\
                Chavi, god of stories                 & Chaotic good    & Chaos, Knowledge, Trickery          \\
                Ivan Ivanovitch, bear god of strength & Chaotic good    & Chaos, Strength, War, Wild          \\
                Krunch, barbarian god of destruction  & Chaotic good    & Destruction, Good, Strength, War    \\
                Sir Cakes, dwarf god of freedom       & Chaotic good    & Chaos, Good, Strength               \\
                Raphael, monk god of retribution      & Lawful neutral  & Death, Law, Protection, Travel      \\
                Declan, god of fire                   & True neutral    & Destruction, Fire, Knowledge, Magic \\
                Kurai, shaman god of nature           & True neutral    & Air, Earth, Fire, Water             \\
                %Amanita, druid god of decay           & Chaotic neutral & Chaos, Destruction, Life, Wild \\
                %Antimony, elf god of necromancy       & Chaotic neutral & Death, Knowledge, Life, Magic \\
                Clockwork, elf god of time            & Chaotic neutral & Chaos, Magic, Trickery, Travel      \\
                %Lord Khallus, fighter god of pride    & Chaotic neutral & Chaos, Strength, War \\
                %Celeano, sorcerer god of deception    & Chaotic neutral & Chaos, Magic, Protection, Trickery \\
                Murdoc, god of mercenaries            & Chaotic neutral & Destruction, Knowledge, Travel, War \\
                Ribo, halfling god of trickery        & Chaotic neutral & Chaos, Trickery, Water              \\
                Tak, orc god of war                   & Lawful evil     & Law, Strength, Trickery, War        \\
                Theodolus, sorcerer god of ambition   & Neutral evil    & Evil, Knowledge, Magic, Trickery    \\
                Daeghul, demon god of slaughter       & Chaotic evil    & Destruction, Evil, Magic, War       \\
            \end{dtabularx}
        \end{dtable!*}

        \subsubsection{Spellcasting}

            \cf{Clr}{Divine Spells} 
            You can cast divine spells using your devotion to your deity.
            You learn two divine spells from the divine \glossterm{spell list} (see \pcref{Divine Spells}).
            Your \glossterm{spellpower} with divine spells is equal to your \glossterm{divine power}.

            To cast a spell, you must normally spend an \glossterm{action point}.
            Every spell can also be cast as a cantrip.
            Cantrips are weaker, but do not require action points to cast.

            You can't cast spells of an alignment opposed to your own or your deity's.
            Spells associated with particular alignments are indicated by the \glossterm{Chaos}, \glossterm{Good}, \glossterm{Evil}, and \glossterm{Law} tags in their spell descriptions.

            \cf{Clr}{Rituals} 
            You can perform divine rituals to create unique magical effects (see \pcref{Rituals}).
            You have a ritual book containing two divine rituals of your choice (see \pcref{Divine Rituals}).

            \cf{Clr}[6]{Standard Augment}
            Choose one of the \glossterm{standard augments} (see \pcref{Standard Augments}).
            You can apply that augment to divine spells you cast.

            \cf{Clr}[10]{Standard Augment}
            Choose one of the \glossterm{standard augments} (see \pcref{Standard Augments}).
            You can apply that augment to divine spells you cast.

            \cf{Clr}[14]{Standard Augment}
            Choose one of the \glossterm{standard augments} (see \pcref{Standard Augments}).
            You can apply that augment to divine spells you cast.

            \cf{Clr}[18]{Spell Point} 
            You gain a spell point.
            A spell point can be spent to cast spells in place of an action point.
            You recover all spent spell points after a \glossterm{short rest}.

        \subsubsection{Domain Influence}
            All \textit{domain influence} abilities are \glossterm{Magical}.

            \cf{Clr}{Domains}
            You choose two domains which represent your personal spiritual inclinations.
            You must choose your domains from among those your deity offers.
            The domains are listed below.

            \begin{itemize}
                \item{Air}
                \item{Chaos}
                \item{Death}
                \item{Destruction}
                \item{Earth}
                \item{Evil}
                \item{Fire}
                \item{Good}
                \item{Knowledge}
                \item{Law}
                \item{Life}
                \item{Magic}
                \item{Protection}
                \item{Strength}
                \item{Travel}
                \item{Trickery}
                \item{War}
                \item{Water}
                \item{Wild}
            \end{itemize}

            \cf{Clr}{Domain Gift}
            Each domain has a corresponding \textit{domain gift}.
            You gain the \textit{domain gift} for one of your domains (see \pcref{Domain Gifts}).

            \cf{Clr}[3]{Domain Gift}
            You gain the \textit{domain gift} for another one of your domains.

            \cf{Clr}[5]{Domain Spell}
            You learn one of the two spells offered by your domains (see \pcref{Domain Spells}).

            \cf{Clr}[7]{Domain Aspect}
            Each domain has a corresponding \textit{domain aspect}.
            You gain the \textit{domain aspect} for one of your domains (see \pcref{Domain Gifts}).

            \cf{Clr}[9]{Domain Aspect} 
            You gain the \textit{domain aspect} for another one of your domains.

            \cf{Clr}[11]{Domain Essence}
            Each domain has a corresponding \textit{domain essence}.
            You gain the \textit{domain essence} for one of your domains (see \pcref{Domain Gifts}).

            \cf{Clr}[13]{Domain Essence} 
            You gain the \textit{domain essence} for another one of your domains.

            \cf{Clr}[15]{Domain Mastery}
            Each domain has a corresponding \textit{domain mastery}.
            You gain the \textit{domain mastery} for one of your domains (see \pcref{Domain Gifts}).

            \cf{Clr}[17]{Domain Mastery} 
            You gain the \textit{domain mastery} for another one of your domains.

            \cf{Clr}[20th]{Miracle}
            Once per week, you can request a miracle as a standard action.
            You mentally specify your request, and your deity fulfills that request in the manner it sees fit.
            This can emulate the effects of any spell or ritual, or have any other effect of a similar power level.
            If the deity has a direct interest in your situation, the miracle may be of even greater power.

            If you perform an extraordinary service for your deity, you can gain the ability to request an additional miracle that week.

        \subsubsection{Divine Spell Mastery}
            You must have the ability to cast divine spells to gain these abilities.

            \cf{Clr}[2]{Spell Point}
            You gain a spell point.
            A spell point can be spent to cast spells in place of an action point.
            You recover all spent spell points after a \glossterm{short rest}.

            \cf{Clr}[4]{Custom Augments}
            Choose a \glossterm{custom augment} for a divine spell you know.
            You can apply that augment to that spell (see \pcref{Augments}).
            At 6th level, and every two levels thereafter, you learn an additional custom augment for a divine spell you know.

            \cf{Clr}[8]{Expanded Spell Knowledge}
            You learn an additional spell from the divine \glossterm{spell list} (see \pcref{Divine Spells}).

    \subsection{Cleric Domain Abilities}
        All cleric domain abilities are \glossterm{magical} unless otherwise specified.

        \subsubsection{Air}
            \parhead{Gift} You add the Jump skill to your class skill list and gain a \plus5 bonus to Jump checks (see \pcref{Jump}).
            \parhead{Aspect} You gain a \glossterm{glide speed} equal to your land speed (see \pcref{Gliding}).
            \parhead{Essence} As a standard action, you can spend an \glossterm{action point} to use this ability.
            \begin{ability}
                \begin{spelltargetinginfo}
                    \spellrng{\rnglong}
                \end{spelltargetinginfo}
                \begin{spelleffects}
                    \spelleffect You can speak with and command air within range.
                    You can ask the air simple questions and understand its responses.
                    If you command the air to perform a task, it will do so do the best of its ability until this effect ends.
                    You cannot compel the air to move faster than 50 mph.
                    \spelldur Sustain (swift)
                    \spellspecial After you use this ability on a particular area of air, you cannot use it again on that same area for 24 hours.
                \end{spelleffects}
            \end{ability}
            \parhead{Mastery} You gain a \glossterm{fly speed} with \glossterm{good maneuverability} equal to your land speed (see \pcref{Flying}).

        \subsubsection{Chaos}
            \parhead{Gift} Whenever you roll a 10 on a check on your first attempt, you gain a \plus5 bonus to the check.
            \parhead{Aspect} Your attack rolls explode on a 9 or a 10 (see \pcref{Exploding Attacks}).
            \parhead{Essence} As a standard action, you can spend an \glossterm{action point} to use this ability.
            \begin{ability}
                \begin{spelltargetinginfo}
                    \spellrng{\rnglong}
                \end{spelltargetinginfo}
                \begin{spelleffects}
                    \spelleffect An improbable event occurs.
                    You can specify in general terms what you want to happen, such as ``Make the bartender leave the bar''.
                    You cannot control the exact nature of the event, though it always beneficial for you in some way.
                    \spellspecial After using this ability, you cannot use it again for an hour.
                \end{spelleffects}
            \end{ability}
            \parhead{Mastery} Your attack rolls explode on a 1 (see \pcref{Exploding Attacks}).

        \subsection{Death}
            \parhead{Gift} Whenever you deal damage to a creature with no hit points remaining, it immediately dies.
            This is a \glossterm{Death} effect.
            % Needs wording to clarify interaction with simultaneous damage
            \parhead{Aspect} Whenever you deal damage to a creature, any of your damage in excess of that creature's hit points is dealt as \glossterm{vital damage}.
            In addition, you are immune to \glossterm{Death} effects.
            \parhead{Essence} As a standard action, you can spend an \glossterm{action point} to use this ability.
            \begin{ability}
                \begin{spelltargetinginfo}
                    \spellquicktargeting{One living creature}{\rngmed}
                    \spellspecial When you use this ability, you choose whether to summon the essence of Death or banish it.
                \end{spelltargetinginfo}
                \begin{spelleffects}
                    \spelleffect If you summoned the essence of Death, the target draws closer to the brink.
                    If it takes \glossterm{vital damage}, it immediately dies.

                    If you banished the essence of Death, the target draws closer to life.
                    It is immune to \glossterm{Death} effects and gains a bonus to \glossterm{stabilization rolls} equal to your \textit{divine power}.
                    \spelldur Attunement
                    \spelltags{\glossterm{Death}}
                \end{spelleffects}
            \end{ability}
            \parhead{Mastery} You constantly radiate an emanation of death, described below.
            \begin{ability}
                \begin{spelltargetinginfo}
                    \spellarea{\areahuge radius emanation from you}
                    \spelltgts{All living enemies in the area}
                \end{spelltargetinginfo}
                \begin{spelleffects}
                    \spelleffect If the target takes \glossterm{vital damage}, it immediately dies.
                \end{spelleffects}
            \end{ability}

        \subsubsection{Destruction}
            \parhead{Gift} You gain a \plus1d bonus to damage with \glossterm{strikes}.
            \parhead{Aspect} Your attacks ignore an amount of \glossterm{hardness} and \glossterm{damage reduction} equal to your \textit{divine power}.
            \parhead{Essence} As a standard action, you can spend an \glossterm{action point} to use this ability.
            \begin{ability}
                \begin{spelltargetinginfo}
                    \spellarea{\arealarge radius burst}
                    \spelltgts{All unattended objects in the area}
                    \spellspecial You may freely exclude any number of 5-ft\. cubes from the area, as long as the resulting area is still contiguous.
                \end{spelltargetinginfo}
                \begin{spelleffects}
                    \begin{spellattack}{Divine power vs. Fortitude}
                        \spellspecial Nonmagical objects do not have a Fortitude defense, so this attack automatically succeeds against such objects.
                        \spellsuccess The target crumbles into a fine powder and is irreparably \glossterm{broken}.
                    \end{spellattack}
                \end{spelleffects}
            \end{ability}
            \parhead{Mastery} Whenever you deal damage to a creature or object, its damage reduction and hardness (if any) are reduced by an amount equal to your \textit{divine power}.
            In addition, Fortitude defense is reduced by 2.
            This is a \glossterm{condition}, and lasts until it is removed.
            This effect stacks with itself, but can only be applied to a target once per round.
            % how do you remove a condition from an object???

        \subsubsection{Earth}
            \parhead{Gift} You gain a \plus2 bonus to Fortitude defense.
            \parhead{Aspect} You gain \glossterm{damage reduction} against physical damage equal to your \textit{divine power}.
            \parhead{Essence} As a standard action, you can spend an \glossterm{action point} to use this ability.
            \begin{ability}
                \begin{spelltargetinginfo}
                    \spellrng{\rnglong}
                \end{spelltargetinginfo}
                \begin{spelleffects}
                    \spelleffect You can speak with and command earth within range.
                    You can ask the earth simple questions and understand its responses.
                    If you command the earth to perform a task, it will do so do the best of its ability until this effect ends.
                    You cannot compel the earth to move faster than 10 feet per round.
                    \spelldur Sustain (swift)
                    \spellspecial After you use this ability on a particular area of earth, you cannot use it again on that same area for 24 hours.
                \end{spelleffects}
            \end{ability}
            \parhead{Mastery} You gain the \glossterm{earth glide} ability, as an earth elemental.
            This allows you to glide through stone, dirt, or almost any other sort of earth as if it were air.
            You can walk or climb at any angle in the earth.
            However, you cannot breathe, speak, or hear while gliding in this way.
            While gliding, you can remain partially within the earth, granting you cover.

        \subsubsection{Evil}
            \parhead{Gift} Whenever you take damage, you may choose an adjacent willing creature.
            If you do, that creature takes half of that damage (rounded down) instead of you.
            Any abilities it has that would make the attack miss or fail have no effect, but its abilities that allow it to reduce or ignore the attack's effects work normally.
            You take the remaining half of the damage, and suffer any non-damaging effects of the attack normally.
            \par You may learn which attacks hit you in the current phase before deciding which attacks to redirect damage from, but not how much damage they would do (or any other effects they might have).
            \parhead{Aspect} You can use this domain's domain gift to redirect damage to any willing creature within \rngclose range.
            \parhead{Essence} As a standard action, you can spend an \glossterm{action point} to use this ability.
            \begin{ability}
                \begin{spelltargetinginfo}
                    \spellquicktargeting{One creature}{\rngmed}
                \end{spelltargetinginfo}
                \begin{spelleffects}
                    \begin{spellattack}{Divine power vs. Mental}
                        \spellspecial Creatures who have strict codes prohibiting them from taking evil actions, such as paladins devoted to Good, are immune to this attack.
                        \spellsuccess The target takes an evil action as soon as it can.
                        You have no control over the act the creature takes, but circumstances can make the target more likely to take an action you desire.
                    \end{spellattack}
                    \spelltags{\glossterm{Compulsion}, \glossterm{Mind}}
                \end{spelleffects}
            \end{ability}
            \parhead{Mastery} Whenever you take damage, you may use this ability as an \glossterm{immediate action}.
            If you do, you can use this domain's domain gift to redirect damage to an unwilling creature within \rngmed range.

        \subsubsection{Fire}
            \parhead{Gift} All of your \glossterm{Fire} spells and abilities do not deal damage to your allies.
            \parhead{Aspect} Whenever you would take fire damage, you heal that many hit points instead.
            This applies damage reduction, damage immunity, and similar effects.
            \parhead{Essence} As a standard action, you can spend an \glossterm{action point} to use this ability.
            \begin{ability}
                \begin{spelltargetinginfo}
                    \spellrng{\rnglong}
                \end{spelltargetinginfo}
                \begin{spelleffects}
                    \spelleffect You can speak with and command fire within range.
                    You can ask the fire simple questions and understand its responses.
                    If you command the fire to perform a task, it will do so do the best of its ability until this effect ends.
                    You cannot compel the fire to move farther than 30 feet in a single round.
                    Fire that ends the round on non-combustable materials usually goes out, depending on the circumstances.
                    \spelldur Sustain (swift)
                    \spellspecial After you use this ability on a particular area of fire, you cannot use it again on that same area for 24 hours.
                    % What does an ``area of fire'' mean?
                \end{spelleffects}
            \end{ability}
            \parhead{Mastery} Whenever you deal fire damage to a creature, that creature becomes \ignited.
            This is a \glossterm{condition}, and lasts until removed.

        \subsubsection{Good}
            \parhead{Gift} Whenever an adjacent creature takes damage, you may use this ability.
            If you do, you take half of that damage (rounded down) instead of the creature.
            Any abilities you have that would make the attack miss or fail have no effect, but your abilities that allow you to reduce or ignore its effects work normally.
            The creature takes the remaining half of the damage, and suffers any non-damaging effects of the attack normally.
            \par You may learn which attacks hit you and your allies in the current phase before deciding which attacks to redirect damage from, but not how much damage they would do (or any other effects they might have).
            \parhead{Aspect} You can use this domain's domain gift to redirect damage from any creature within \rngclose range.
            \parhead{Essence} As a standard action, you can spend an \glossterm{action point} to use this ability.
            \begin{ability}
                \begin{spelltargetinginfo}
                    \spellquicktargeting{One creature}{\rngmed}
                \end{spelltargetinginfo}
                \begin{spelleffects}
                    \begin{spellattack}{Divine power vs. Mental}
                        \spellspecial Creatures who have strict codes prohibiting them from taking good actions, such as paladins devoted to Evil, are immune to this attack.
                        \spellsuccess The target takes an good action as soon as it can.
                        You have no control over the act the creature takes, but circumstances can make the target more likely to take an action you desire.
                    \end{spellattack}
                    \spelltags{\glossterm{Compulsion}, \glossterm{Mind}}
                \end{spelleffects}
            \end{ability}
            \parhead{Mastery} Whenever you redirect damage with this domain's domain gift, you can redirect all effects of the attack instead of only half the damage.

        \subsubsection{Knowledge}
            \parhead{Gift} You add all Knowledge skills to your cleric class skill list.
            In addition, you gain two skill points which must be spent on Knowledge skills.
            \parhead{Aspect} Your extensive knowledge of all methods of attack and defense grants you a \plus1 bonus to all defenses.
            \parhead{Essence} As a standard action, you can spend an \glossterm{action point} to use this ability.
            \begin{ability}
                \begin{spelltargetinginfo}
                    \spellarea{\arealarge radius burst}
                    \spelltgts{All willing creatures in the area}
                \end{spelltargetinginfo}
                \begin{spelleffects}
                    \spelleffect You make a Knowledge check of any kind.
                    You gain a bonus to the Knowledge check equal to your \textit{divine power}.
                    The target also learns the results of your check.
                    It believes the information gained in this way to be true as if it had seen it with its own eyes.
                    \par You cannot alter the knowledge you gain with this check in any way, such as by adding or withholding information.
                \end{spelleffects}
            \end{ability}
            \parhead{Mastery} You gain a \plus1 bonus to accuracy with all attacks.

        \subsubsection{Law}
            \parhead{Gift} You gain a \plus2 bonus to Mental defense.
            % Clarify - does this apply to exploding dice?
            \parhead{Aspect} Whenever you roll a 1 on an \glossterm{attack roll}, it is treated as if you had rolled a 5.
            \parhead{Essence} As a standard action, you can spend an \glossterm{action point} to use this ability.
            \begin{ability}
                \begin{spelltargetinginfo}
                    \spellarea{\arealarge radius burst}
                    \spelltgts{All creatures in the area}
                \end{spelltargetinginfo}
                \begin{spelleffects}
                    \begin{spellattack}{Divine power vs. Mental}
                        \spellspecial This attack automatically succeeds against you.
                        \spellsuccess The target is unable to break the laws that apply in the area, and any attempt to do so simply fails.
                        The laws which are applied are those which are most appropriate for the area, regardless of whether the cleric or any other creature know those laws.
                        In areas under ambiguous or nonexistent government, this ability may have unexpected effects, or it may have no effect at all.
                    \end{spellattack}
                    \spelldur Condition
                    \spelltags{\glossterm{Compulsion}, \glossterm{Mind}}
                \end{spelleffects}
            \end{ability}
            \parhead{Mastery} Whenever you roll less than a 5 on an \glossterm{attack roll}, it is treated as if you had rolled a 5.

        \subsubsection{Life}
            \parhead{Gift} You gain additional hit points equal to your \glossterm{divine power}.
            \parhead{Aspect} All of your healing spells and abilities can cure \glossterm{vital damage} as easily as they cure hit points.
            \parhead{Essence} As a standard action, you can spend all of your remaining \glossterm{action points} (minimum 1) to use this ability.
            \begin{ability}
                \begin{spelltargetinginfo}
                    \spellquicktargeting{One dead creature}{Adjacent}
                \end{spelltargetinginfo}
                \begin{spelleffects}
                    \spelleffect If the target was dead for no more than 5 minutes, it is restored to life, as the \spell{resurrection} ritual.
                \end{spelleffects}
            \end{ability}
            % Need special text to clarify that it is affected by the Aspect?
            \parhead{Mastery} At the end of each round, you heal hit points equal to your \glossterm{divine power}.

        \subsubsection{Magic}
            \parhead{Gift} You gain a \plus1 bonus to spellpower with divine spells.
            \parhead{Aspect} The maximum spell level you can cast is increased by 1.
            \parhead{Essence} You gain \glossterm{magic resistance} equal to 5 \add your \textit{divine power}.
            If you already have \glossterm{magic resistance}, you can instead increase it by 2.
            \parhead{Mastery} You gain a \plus2 bonus to your \glossterm{magic resistance}.
            If you resist an ability with your magic resistance, you heal hit points equal to your \textit{divine power}.

        \subsubsection{Strength}
            \parhead{Gift} You add Climb, Jump, Sprint, and Swim to your cleric class skill list.
            In addition, you gain two skill points which must be spent on any combination of those skills.
            \parhead{Aspect} You may use your Strength to determine your \textit{divine power} in place of your Willpower.
            \parhead{Essence} As a standard action, you can spend an \glossterm{action point} to use this ability.
            \begin{ability}
                \begin{spelleffects}
                    \spelleffect You gain a \plus2d bonus to damage with \glossterm{strikes}.
                    In addition, you gain a \plus5 bonus to Strength for the purpose of checks and determining your carrying capacity.
                    \spelldur{Sustain (swift)}
                \end{spelleffects}
            \end{ability}
            \parhead{Mastery} You gain a \plus2 bonus to Strength.

        \subsubsection{Travel}
            \parhead{Gift} You add Knowledge (geography), Sprint, and Survival to your cleric class skill list.
            In addition, you gain two skill points which must be spent on any combination of those skills.
            \parhead{Aspect} You gain a \plus30 foot bonus to your speed in all movement modes, up to a maximum of double your normal speed.
            \parhead{Essence} As a standard action, you can spend an \glossterm{action point} to use this ability.
            \begin{ability}
                \begin{spelleffects}
                    \spelleffect You teleport yourself up to 1 mile in any direction.
                    You do not need \glossterm{line of sight} or \glossterm{line of effect} to your destination, but you must be able to clearly visualize it.
                \end{spelleffects}
            \end{ability}
            \parhead{Mastery} Whenever you move, you can teleport the same distance instead.
            You can teleport in any direction.

        \subsubsection{Trickery}
            \parhead{Gift} You add Bluff, Disguise, and Stealth to your cleric class skill list.
            In addition, you gain two skill points which must be spent on any combination of those skills.
            \parhead{Aspect} You gain a \plus2 bonus to Bluff, Disguise, and Stealth.
            \parhead{Essence} As a standard action, you can spend an \glossterm{action point} to use this ability.
            \begin{ability}
                \begin{spelltargetinginfo}
                    \spellquicktargeting{One creature}{\rngmed}
                    \spellspecial Choose a belief the target has.
                    The belief may be a lie that you told it, or even a simple misunderstanding (such as believing a hidden creature is not present in a room).
                    If the creature does not already hold the chosen belief, this ability automatically fails.
                \end{spelltargetinginfo}
                \begin{spelleffects}
                    \begin{spellattack}{Divine power vs. Mental}
                        \spellsuccess The target continues to maintain the chosen belief, regardless of any evidence to the contrary.
                        It will interpret any evidence that the falsehood is incorrect to be somehow wrong -- an illusion, a conspiracy to decieve it, or any other reason it can think of to continue believing the falsehood.
                        At the end of the effect, the creature can decide whether it believes the falsehood or not, as normal.
                    \end{spellattack}
                    \spelldur Sustain (Swift)
                    \spelltags{\glossterm{Delusion}, \glossterm{Mind}}
                \end{spelleffects}
            \end{ability}
            \parhead{Mastery} You are undetectable by Divination spells and effects.
            They cannot detect your presence, sounds you make, or any actions you take.

        \subsubsection{War}
            \parhead{Gift} You gain proficiency with heavy armor, tower shields, and an additional weapon group of your choice.
            \parhead{Aspect} You gain a \plus1d bonus to damage with \glossterm{strikes}.
            \parhead{Essence} Whenever you cast a spell, you can use this ability as an \glossterm{immediate action} by spending an \glossterm{action point}.
            If you do, the spell gains one of the following effects:
            \begin{itemize}
                \item Legion: If the spell would normally affect five or more specific targets, its range is doubled and it instead affects five times that many targets.
                \item Selective: If the spell has an area, it has no effect on your allies in the area.
                \item Widened: If the spell has an area, the size of the area is doubled.
            \end{itemize}
            \parhead{Mastery} You and all allies within a \arealarge radius emanation of you gain a \plus1d bonus to damage with \glossterm{strikes}.

        \subsubsection{Water}
            \parhead{Gift} You add Swim to your cleric class skill list and gain a \plus5 bonus to Swim checks.
            \parhead{Aspect} You can breathe water as easily as a human breathes air, preventing you from drowning or suffocating underwater.
            You also gain a \glossterm{swim speed} equal to your land speed.
            \parhead{Essence} As a standard action, you can spend an \glossterm{action point} to use this ability.
            \begin{ability}
                \begin{spelltargetinginfo}
                    \spellrng{\rnglong}
                \end{spelltargetinginfo}
                \begin{spelleffects}
                    \spelleffect You can speak with and command water within range.
                    You can ask the water simple questions and understand its responses.
                    If you command the water to perform a task, it will do so do the best of its ability until this effect ends.
                    You cannot compel the water to move faster than 30 feet per round.
                    \spelldur Sustain (swift)
                    \spellspecial After you use this ability on a particular area of water, you cannot use it again on that same area for 24 hours.
                \end{spelleffects}
            \end{ability}
            \parhead{Mastery}
            Whenever you move, you can transform yourself into a rushing flow of water with a volume roughly equal to your normal volume until your movement is complete
            In this form, you may move wherever water could go, you cannot take other actions, such as jumping, attacking, or casting spells.
            You may move through squares occupied by creatures or threatened by blocking enemies without penalty.
            \par Your speed is halved when moving uphill and doubled when moving downhill.
            Unusually steep inclines may cause greater movement differences while in this form.
            \par If the water is split, you may reform from anywhere the water has reached, to as little as a single ounce of water.
            If not even an ounce of water exists contiguously, your body reforms from all of the largest available sections of water, cut into pieces of appropriate size.
            This usually causes you to die.

        \subsubsection{Wild}
            \parhead{Gift} You add Creature Handling, Knowledge (nature), and Survival to your cleric class skill list.
            In addition, you gain two skill points which must be spent on any combination of those skills.
            % Does this even make sense?
            \parhead{Aspect} When you gain this ability, you choose one wild aspect ability, as if you were a druid of a level equal to your cleric level (see Wild Aspect, \pref{Drd:Wild Aspect}).
            As a standard action, you can spend an \glossterm{action point} to embody that wild aspect for 1 hour.

        \subsubsection{Ex-Clerics}
            If you grossly violate the code of conduct required by your deity, you lose all spells and magical cleric class abilities.
            You cannot regain those abilities until you atone (see the \spell{atonement} ritual).

\section{Druid}\label{Druid}
    \begin{dtable}
        \lcaption{Druid Progression}
        \begin{dtabularx}{\columnwidth}{>{\ccol}p{\levelcol} >{\ccol}p{3.5em} c >{\lcol}X}
            \tb{Level} & \tb{Active Aspects} & \tb{Spells} & \tb{Special} \\
            \bottomrule
            \nth{1}  & \tdash & 2  & Spells, rituals, wild speech \\
            \nth{2}  & 1      & 3  & Wild aspect                  \\
            \nth{3}  & 1      & 3  & Wild aspect                  \\
            \nth{4}  & 1      & 4  & \tdash                       \\
            \nth{5}  & 2      & 4  & Wild aspect                  \\
            \nth{6}  & 2      & 5  & \tdash                       \\
            \nth{7}  & 2      & 5  & Wild aspect                  \\
            \nth{8}  & 2      & 6  & \tdash                       \\
            \nth{9}  & 3      & 6  & Wild aspect                  \\
            \nth{10} & 3      & 7  & \tdash                       \\
            \nth{11} & 3      & 7  & Wild aspect                  \\
            \nth{12} & 3      & 8  & \tdash                       \\
            \nth{13} & 4      & 8  & Wild aspect                  \\
            \nth{14} & 4      & 9  & \tdash                       \\
            \nth{15} & 4      & 9  & Wild aspect                  \\
            \nth{16} & 4      & 10 & \tdash                       \\
            \nth{17} & 5      & 10 & Wild aspect                  \\
            \nth{18} & 5      & 11 & \tdash                       \\
            \nth{19} & 5      & 11 & Wild aspect                  \\
            \nth{20} & 5      & 12 & Avatar of nature             \\
        \end{dtabularx}
    \end{dtable}

    \classbasics{Alignment} Neutral good, lawful neutral, neutral, chaotic neutral, or neutral evil.

    \classbasics{Class Skills}
    \subparhead{Strength} Climb, Jump, Sprint, Swim.
    \subparhead{Dexterity} Acrobatics, Ride, Stealth.
    \subparhead{Intelligence} Heal, Knowledge (geography, nature).
    \subparhead{Perception} Awareness, Creature Handling, Survival.
    \subparhead{Other} Bluff, Intimidate, Persuasion.

    \subsection{Base Class Abilities}
        A character with druid as a base class gains the following abilities.

        \classbasics{Skill Points} 10.

        \classbasics{Defenses} \plus4 Fortitude, \plus2 Mental.

        \classbasics{Action Points} \plus3.

        \cf{Drd}{Weapon and Armor Proficiency}
        Druids are proficient with simple weapons, any one other weapon group, scimitars, sickles, and slings.
        In addition, druids are proficient with light armor, medium armor, and shields.
        However, a druid cannot use metal armor; see the Metal Abhorrence ability, below.

    \subsection{Class Abilities}
        If you are a druid, you have the following abilities.

        \cf{Drd}{Druidic Language}
        You know Druidic, a secret language known only to druids, in addition to your normal languages.
        Druids are forbidden to teach this language to nondruids.
        Druidic has its own alphabet.

        \cf{Drd}{Metal Abhorrence}
        The oaths that you swear as part of your druidic initiation prohibit you from wearing armor made of metal.
        If you wear prohibited armor or carry a prohibited shield, you are unable to cast druid spells or use any of your \glossterm{magical} druid abilities while doing so and for 24 hours thereafter.
        
        You can avoid this penalty by using armor made of wood altered with the \spell{ironwood} ritual.
        Such wood is as strong as steel.

        \cf{Drd}{Nature Power}
        The \glossterm{power} of your druid spells and abilities is determined by your \textit{nature power}.
        Your \textit{nature power} is equal to your level or your Perception, whichever is higher.

        \subsubsection{Spellcasting}

            \cf{Drd}{Nature Spells} 
            You can cast nature spells using your devotion to your deity.
            You learn two nature spells from the nature \glossterm{spell list} (see \pcref{Nature Spells}).
            Your \glossterm{spellpower} with nature spells is equal to your \glossterm{nature power}.

            To cast a spell, you must normally spend an \glossterm{action point}.
            Every spell can also be cast as a cantrip.
            Cantrips are weaker, but do not require action points to cast.

            You can't cast spells of an alignment opposed to your own or your deity's.
            Spells associated with particular alignments are indicated by the \glossterm{Chaos}, \glossterm{Good}, \glossterm{Evil}, and \glossterm{Law} tags in their spell descriptions.

            \cf{Drd}{Rituals} 
            You can perform nature rituals to create unique magical effects (see \pcref{Rituals}).
            You have a ritual book containing two nature rituals of your choice (see \pcref{Nature Rituals}).

            \cf{Drd}[6]{Standard Augment}
            Choose one of the \glossterm{standard augments} (see \pcref{Standard Augments}).
            You can apply that augment to nature spells you cast.

            \cf{Drd}[10]{Standard Augment}
            Choose one of the \glossterm{standard augments} (see \pcref{Standard Augments}).
            You can apply that augment to nature spells you cast.

            \cf{Drd}[14]{Standard Augment}
            Choose one of the \glossterm{standard augments} (see \pcref{Standard Augments}).
            You can apply that augment to nature spells you cast.

            \cf{Drd}[18]{Spell Point} 
            You gain a spell point.
            A spell point can be spent to cast spells in place of an action point.
            You recover all spent spell points after a \glossterm{short rest}.

        \subsubsection{Natural Influence}
            All of these abilities are \glossterm{Magical}.

            \cf{Drd}{Wild Speech} As a standard action, you can spend an \glossterm{action point} to use this ability.
            \begin{ability}
                \begin{spelltargetinginfo}
                    \spellquicktargeting{One animal}{\rngmed}
                \end{spelltargetinginfo}
                \begin{spelleffects}
                    \spelleffect You can speak to and understand the speech of the target animal, and any other animals of the same species.
                    This ability doesn't make the target any more friendly or cooperative than normal.
                    Wary and cunning animals are likely to be terse and evasive, while stupid ones tend to make inane comments and are unlikely to say or understand anything of use.
                    \spelldur Sustain (swift)
                \end{spelleffects}
            \end{ability}

            \cf{Drd}[3]{Wild Aspect}[Mag]
            You gain the ability to embody an aspect of an animal or of nature itself.
            Choose a single wild aspect from the list below.
            Many wild aspects have a minimum level prerequisite, as indicated in the title of the ability.
            That ability is normally active.
            You may suppress or resume the effects of any number of \textit{wild aspects} you have as a \glossterm{swift action}.

            The abilities in the list below describe the effects of the aspect.
            Your appearance also changes to match the aspect's effects, but the nature of this change is not described.
            Different druids change in different ways.
            For example, one druid might gain unusually large eyes when embodying the low-light vision aspect, while another might change their irises into slits, like a cat, when embodying the same aspect.
            You choose how your appearance changes when you gain a wild aspect.
            This change cannot be used to gain an additional substantive benefit beyond the effects given in the description of the aspect.

            Many wild aspects grant natural weapons.
            See \pcref{Natural Weapons}, for details about natural weapons.
            At 5th level, and every two levels thereafter, you gain an additional wild aspect.

            \subcf{Armaments of the Bear}
            Your mouth and hands transform, allowing you to perform bite and claw attacks.
            The bite attack deals 1d8 damage for a Medium creature, and the claws deal 1d6 damage.
            \subcf{Senses}
            You gain low-light vision.
            You treat sources of light as if they had double their normal illumination range.
            If you already have low-light vision, you double its benefit, allowing you to treat sources of light as if they had four times their normal illumination range.
            In addition, you gain \glossterm{darkvision} out to 50 feet, allowing you to see in complete darkness.
            If you already have darkvision, you increase its range by 50 feet.
            \subcf{Woodland Stride}
            You may move through any sort of undergrowth (such as natural thorns, briars, overgrown areas, and similar terrain) at your normal speed and without taking damage or suffering any other impairment.
            The plants bend of their own volition to allow the druid to pass.
            However, plants magically manipulated to impede motion still affect your.

            \subcf{5th -- Animal Affinity}
            You gain a \plus5 bonus to Creature Handling and Ride checks.
            \subcf{5th -- Climb}
            You gain a \glossterm{climb speed} equal to your land speed.
            \subcf{5th -- Constrict}
            Your body transforms, improving your grappling abilities.
            You gain a \plus2 bonus to accuracy with grapple attacks (see \pcref{Grapple}).
            In addition, you gain a constrict attack.
            This attack deals 1d10 damage for a Medium druid, but it can only be used against a foe you aregrappling with.
            \subcf{5th -- Gore}
            Your head transforms, allowing you to perform a gore attack.
            The attack deals 1d8 damage for a Medium druid.
            In addition, if you hit with a natural attack, you may attempt to shove your foe as an \glossterm{immediate action}.

            \subcf{7th -- A Thousand Faces}
            You may use your spellpower in place of your Disguise skill when making Disguise checks to alter your own appearance.
            \subcf{7th -- Hawk}
            You grow wings, granting your a glide speed equal to your land speed.
            See \pcref{Gliding}, for more details.
            In addition, your feet transform, allowing you to perform a talon attack.
            The attack deals 1d6 damage for a Medium creature.
            \subcf{7th -- Lope}
            You gain the ability to move on all four limbs.
            When doing so, you gain a \plus30 foot bonus to your land speed, up to a maximum of double your original speed.
            When not using your hands to move, your ability to use your hands is unchanged.
            Descending to four legs and rising up to stand on two legs again does not take an action.
            \subcf{7th -- Scent}
            You gain the \glossterm{scent} ability.
            \subcf{7th -- Shrink}
            You shrink by one size category (see \pcref{Size in Combat}).
            This is a \glossterm{Sizing} effect.
            \subcf{7th -- Slither}
            You gain a \glossterm{climb speed} equal to your land speed.
            You do not need to use your hands to climb in this way.
            In addition, you gain a bite attack that deals 1d8 damage for a Medium druid.
            % \subcf{7th -- Spikes}
            % Whenever a creature adjacent to the druid makes a physical attack against you, the attacking creature takes 1d4 piercing damage \plus1d per two nature power.
            % A creature can only be dealt damage by this effect once per round.

            \subcf{9th -- Barkskin} You gain \glossterm{damage reduction} against physical damage equal to your \textit{nature power}.
            Fire damage ignores this damage reduction and negates it for 1 round.
            \subcf{9th -- Natural Grab}
            If you hit with a natural attack, you may attempt to grapple your foe as an immediate action.
            \subcf{9th -- Natural Trip}
            If you hit with a natural attack, you may attempt to trip your foe as an immediate action.
            % Is this how poison actually works?
            \subcf{9th -- Venom}
            If you hit with a natural attack, you may inject poison into your foe as an immediate action.
            At the end of each round, you make a \textit{nature power} vs. Fortitude attack against all creatures you have poisoned.
            The effects of the poison are described below.
            \begin{itemize}
                \item First success: the target is \sickened.
                \item Second success: the target is \staggered.
                \item Third success: the target is \nauseated.
                \item Third failure: the target is no longer poisoned, and any lingering effects from the poison end.
            \end{itemize}
            % TODO poison stacking rules
            \par In addition, you gains a bite attack that deals 1d8 damage for a Medium druid.

            % \subcf{11th -- Elemental Retribution}
            % Whenever a creature within \rngmed range of you attacks you, the attacking creature takes 1d6 damage per two nature power of either cold, electricity, or fire damage.
            % You may choose the damage type independently for each attacking creature.
            % A creature can only be dealt damage by this effect once per round.
            \subcf{11th -- Fluid Motion} You are immune to effects that restrict your mobility, and you suffer no penalties for acting underwater.
            In addition, you gain a \plus10 bonus to Reflex defense against grapple attacks, as well as on grapple attacks or Escape Artist checks made to escape a grapple or a pin.
            \subcf{11th -- Grow}
            You increase in size by one size category (see \pcref{Size in Combat}).
            This is a \glossterm{Sizing} effect.
            \subcf{11th -- Wolfpack}
            Overwhelmed foes you threaten increase their overwhelm penalties by 1.

            \subcf{13th -- Wings}
            You grow wings, granting you a \glossterm{fly speed} equal to your land speed.
            While \unencumbered, you can fly (see \pcref{Flying}).
            % TODO: flight limitation

            \subcf{15th -- Earth Glide}
            You gain the earth glide ability, as an earth elemental.
            This allows you to glide through stone, dirt, or almost any other sort of earth as if it were air.
            You can walk or climb at any angle in the earth.
            However, you cannot breathe, speak, or hear while gliding in this way.
            While gliding, you can remain partially within the earth, granting you cover.
            \subcf{15th -- Natural Renewal}
            At the end of each round, you heal hit points equal to your \textit{nature power}.

            \subcf{19th -- Solar Radiance}
            The druid continuously radiates bright light out to a 500 foot radius (and shadowy illumination for an additional 500 feet).
            The illumination is so bright that you become hard to look at.
            Any creature attacking your from within the radius of bright light becomes \partiallyblinded for 2 rounds after the attack.

        \cf{Drd}[20th]{Avatar of Nature}[Mag]
        If you die, except if by old age, you may choose to have your body and soul become an instrument of nature's will.
        Your body immediately decomposes or otherwise disappears, and your soul does not travel to an afterlife.
        You has no physical form, and cannot use any of your normal abilities.
        Instead, you have a fly speed of 100 feet, with special maneuverability.
        As a standard action, you can temporarily possess any living plants or animals within a 10 mile radius of the place of your death.

        While possessing a living plant or animal, you can see through its senses and control its actions completely.
        In addition, you may cast spells, and the spells take effect as if the plant or animal had cast them.
        You use the plant or animal's position to determine range, visible targets, and so on.
        You do not require verbal or somatic components to cast your spells in this form, but are unable to cast spells or perform rituals that require material components or focus objects.

        While not possessing a plant or animal, you can rest, or you can focus on reincarnating your physical form.
        Creating a new body in this way takes 12 consecutive hours of concentration.
        At the end of that time, you are reincarnated in a new body in your location, as the effect of the \spell{reincarnate} ritual, except that you can choose your race from among the races listed (not including the ``Other'' race).

        While you are an avatar of nature, you do not age and you cannot die of old age.
        You can continue to exist in this form indefinitely.

        \subsubsection{Nature Spell Mastery}
            You must have the ability to cast nature spells to gain these abilities.

            \cf{Drd}[2]{Spell Point}
            You gain a spell point.
            A spell point can be spent to cast spells in place of an action point.
            You recover all spent spell points after a \glossterm{short rest}.

            \cf{Drd}[4]{Custom Augments}
            Choose a \glossterm{custom augment} for a nature spell you know.
            You can apply that augment to that spell (see \pcref{Augments}).
            At 6th level, and every two levels thereafter, you learn an additional custom augment for a nature spell you know.

            \cf{Drd}[8]{Expanded Spell Knowledge}
            You learn an additional spell from the nature \glossterm{spell list} (see \pcref{Nature Spells}).

    % \subcf{15th -- Air Mantle}
    % You are surrounded by a mantle of air.
    % Thrown and projectile weapons have a 50\% chance to miss your while this effect is active.
    % Unusually large weapons, such as a giant's boulders, may suffer a decreased miss chance as appropriate to their size.
    % \subcf{15th -- Aqueous Step}
    % Wherever the druid moves, you leave a path of animated water that can grab creatures.
    % Whenever a creature crosses the path, the druid makes a Reflex attack to trip the creature, causing it to fall prone and waste the rest of its movement.
    % Your accuracy is equal to your druid level \add your Constitution.
    % \subcf{15th -- Flaming Step}
    % Wherever the druid moves, you leave a path of burning flame behind your that lasts for 1 round.
    % Whenever a creature crosses the path, the druid makes a Reflex attack to deal damage to the creature.
    % The attack deals 1d8 points of fire damage per two druid levels.
    % Your accuracy is equal to your druid level \add your Constitution.
    % A failed attack deals half damage.
    % \subcf{15th -- Lifegiving Step}
    % Wherever the druid moves, you leave a path of small, living plants that entangle foes for 1 round.
    % Whenever a creature crosses the path, the druid makes a Reflex attack to entangle the creature, causing it to waste the rest of its movement.
    % Your accuracy is equal to your druid level \add your Constitution.
    % The plants appear on any surface, and will continue to grow if they can survive, though they may die quickly if they appear on inhospitable terrain.
    % \subcf{17th -- Flaming Soul}
    % You gain the fire subtype, making your immune to fire but giving your a 50\% vulnerability to cold damage.
    % In addition, whenever you deal fire damage to a creature, the creature is \ignited for 5 rounds.
    % \subcf{17th -- Sunblessed Rejuvenation}
    % You gain fast healing equal to your druid level as long as you remain in sunlight or touches a plant of your size or larger.
    % \subcf{17th -- Sunscour}
    % This aspect functions like the heart of the sun natural aspect, except that it also suppresses shadow effects and the visual components of illusions within the area of bright light.

    % \subcf{17th -- Water's Flow}
    % As a swift action, the druid can transform herself into a rushing flow of water with a volume roughly equal to your normal volume until the end of your turn.
    % In this form, she may move wherever water could go, but she cannot take other actions, such as jumping, attacking, or casting spells.
    % Your speed is halved when moving uphill and doubled when moving downhill.
    % She may move through squares occupied by creatures or threatened by blocking enemies without penalty.
    % She may return to your normal form as a free action.
    % \par If the water is split, she may reform from anywhere the water has reached, to as little as a single ounce of water.
    % If not even an ounce of water exists contiguously, your body reforms from the largest available parts of water, cut into pieces of appropriate size.
    % This usually causes the druid to die.

    \subsection{Ex-Druids}
        A druid who ceases to revere nature, changes to a prohibited alignment, or teaches the Druidic language to a nondruid loses all spells and magical druid class abilities.
        She cannot thereafter gain levels as a druid until you atone (see the \spell{atonement} ritual).

    \subsection{Variant Druids}

        \subsubsection{Blighter}

            Blighters draw power from nature, as do other druids. However, while other druids revere nature and draw power from it gently, blighters steal power from nature forcefully. Wherever a blighter goes, destruction and death surely follows.

            \altcf{Blight} Instead of meditating to regain spell slots, a blighter draws power from your environment forcefully.
            This affects a \areahuge radius zone centered on your, and the process takes 1 minute of concentration.
            At the end of every round, every living thing in the area other than the blighter takes damage equal to your nature power.
            All inanimate plants of Huge size or smaller immediately wither and die.
            The earth becomes cracked and infertile, and any nutrients from the soil are destroyed.
            This ability has no effect on artificial environments or materials, such as metal or worked stone.
            At the end of the minute, the blighter regains your spent nature spell slots.

            A blighter can only blight your surroundings in this way once per hour.
            If your surroundings are already blighted or are not natural terrain, she cannot use this ability to regain your spells.
            Instead, she must meditate for 8 hours to slowly draw power from your surroundings, as a normal druid.

            \altcf{Spells} As normal, except that a blighter adds all Vivimancy arcane spells to your spell list.

            \altcf[2nd]{Wild Speech} As normal, except that a blighter gains a \plus5 bonus to Intimidate against your wild speech targets, and a \minus5 penalty to Persuasion.

            \altcf[10th]{Blightcasting}

            \altcf[20th]{Improved Blightcasting}

        \subsubsection{Rotbringer}

            While most druids seek to emulate and interact with animals, rotbringers focus on the power of fungi, decay, and regeneration.

            \altcf{Invoke Rot} Instead of meditating to regain spell slots, a rotbringer accelerates the natural forces of decomposition and decay on your environment.
            This affects a \areahuge radius zone centered on your, and the process takes 1 minute of concentration.
            All organic objects of Huge size or smaller, such as plants and corpses, decompose.
            This decomposition kills inanimate, living plants.
            All organic objects, regardless of size, are covered with various fungi.
            This ability has no effect on artificial environments or materials, such as metal or worked stone.
            At the end of the minute, the rotbringer regains your spent nature spell slots.

            If the rotbringer decomposes a Huge object with this ability, or a combination of smaller objects equivalent in size to a Huge object, you gain an bonus nature spell slot of your highest available spell level.
            This extra spell slot lasts until it is used, or until you regain your spell slots again.

            A rotbringer can only invoke rot on your surroundings in this way once per hour.
            If your surroundings are already decomposed or are not natural terrain, she cannot use this ability to regain your spells.
            Instead, she must meditate for 8 hours to slowly draw power from your surroundings, as a normal druid.

            \altcf[2nd]{Wild Speech} The rotbringer gains the ability to speak with plants at 2nd level.
            You gain the ability to speak with animals at 6th level, instead of at 2nd level.

            \altcf[3rd]{Wild Aspect} The rotbringer does not gain this ability.

            \altcf[3rd]{Rot Spell} The druid learns an additional spell slot and spell known.
            The spell must be taken from the following list of spells.
            The spell's level cannot exceed half your druid level.
            If she already knows a spell from the list at every spell level you ha access to, she may instead learn any nature spell (see \pcref{Nature Spells}).

            At 5th level, and every odd level, the druid may learn a new spell.

            \begin{dtable}
                \begin{dtabularx}{\columnwidth}{l X}
                    \tb{Spell level} & \tb{Rotbringer Spells} \\
                    1st & \spell{excrete slime}, \spell{lesser regeneration} \\
                    2nd & \spell{fungal growth} \\
                    3rd & \spell{rotburst} \\
                    4th & \spell{poison} \\
                    6th & \spell{regeneration} \\
                    7th & \spell{greater rotburst} \\
                \end{dtabularx}
            \end{dtable}

            \altcf[7th]{Fungal Armor} The rotbringer becomes covered in fungus that protects your from attacks. You gain a \plus1 bonus to Armor and Fortitude defense.

            This bonus increases by 1 at your 7th druid level, and every 4 druid levels thereafter.

\section{Fighter}\label{Fighter}
    \begin{dtable}
        \lcaption{Fighter Progression}
        \begin{dtabularx}{\columnwidth}{>{\ccol}p{\levelcol} >{\lcol}X}
            \tb{Level} & \tb{Special} \\
            \bottomrule
            \nth{1}  & Combat supremacy, martial excellence \\
            \nth{2}  & Weapon discipline                    \\
            \nth{3}  & Combat feat                          \\
            \nth{4}  & Combat discipline                    \\
            \nth{5}  & Battlemaster                         \\
            \nth{6}  & Armor discipline                     \\
            \nth{7}  & Combat feat                          \\
            \nth{8}  & Improved weapon discipline           \\
            \nth{9}  & Combat mastery                       \\
            \nth{10} & Improved combat discipline           \\
            \nth{11} & Combat feat                          \\
            \nth{12} & Improved armor discipline            \\
            \nth{13} & Swift warrior                        \\
            \nth{14} & Greater weapon discipline            \\
            \nth{15} & Combat feat                          \\
            \nth{16} & Greater combat discipline            \\
            \nth{17} & Supreme battlemaster                 \\
            \nth{18} & Greater armor discipline             \\
            \nth{19} & Combat feat                          \\
            \nth{20} & Supreme discipline                   \\
        \end{dtabularx}
    \end{dtable}

    \classbasics{Alignment} Any.

    \classbasics{Class Skills}
    \subparhead{Strength} Climb, Jump, Sprint, Swim.
    \subparhead{Dexterity} Acrobatics, Escape Artist, Ride.
    \subparhead{Perception} Awareness.
    \subparhead{Other} Bluff, Intimidate, Persuasion.

    \subsection{Base Class Abilities}
        A character with fighter as a base class gains the following abilities.

        \classbasics{Skill Points} 10.

        \classbasics{Defenses} \plus4 Fortitude, \plus2 Mental.

        \classbasics{Action Points} \plus2.

        \cf{Ftr}{Weapon and Armor Proficiency}
        A fighter is proficient with simple weapons, any four other weapon groups,  all armor (heavy, medium, and light), and shields.

        \cf{Ftr}{Martial Excellence}
        A fighter gains a \plus1 bonus to accuracy and damage with physical attacks.
        In addition, you gain a \plus1 bonus to physical defenses.

    \subsection{Class Abilities}
        All fighters have the following abilities.

        \cf{Ftr}{Combat Supremacy}
        A fighter is a consummate warrior, and can stand toe to toe with even the toughest foes.
        Whenever the fighter deals damage to a creature with a physical attack, he may impede its fighting ability as an \glossterm{immediate action}.
        If you do, the struck creature is \impaired on attacks and checks against the fighter for 2 rounds, or until he uses this ability on that creature again.
        He may use this ability a number of times per day equal to your Intelligence or your fighter level, whichever is higher.

        \cf{Ftr}[2nd]{Weapon Discipline}
        The fighter's training grants you additional capability when using your weapons.
        He may choose a weapon group, or he may choose to train equally with all weapons.
        If he chooses a weapon group, you gain a \plus1 bonus to damage on attacks with weapons from that group.

        If he chooses not to focus on a specific group of weapons, you gain the ability to become proficient with any weapon group if he spends 1 hour training with a weapon from that group.
        He may only keep this proficiency with one weapon group at a time; if he trains with a new weapon group, you lose your proficiency in the previous group.

        \cf{Ftr}[3rd]{Combat Feat}
        The fighter gets a bonus Combat feat (see \pcref{Combat Feats}).
        He must use your fighter level in place of your character level to meet level prerequisites for the feat.
        You gain an additional bonus feat at your 7th fighter level and every four fighter levels thereafter.

        \cf{Ftr}[4th]{Combat Discipline}
        The fighter can use your superior training and focus to keep fighting in the face of debilitating effects.
        When a fighter is initially affected by one of the conditions listed on \trefnp{Combat Discipline Conditions}, he may use your combat discipline ability to instead suffer the mitigated condition one column to the right.
        He can suppress the condition up to 5 rounds.

        \par Using combat discipline takes no action, and can be done at any time, even when it isn't the fighter's turn.
        A fighter may use this ability a number of times per day equal to your Willpower or half your fighter level, whichever is higher.
        However, he cannot mitigate more than one condition at a time.
        If the fighter attempts to mitigate a new condition, the old condition resumes its normal effect immediately.

        \begin{dtable}
            \lcaption{Combat Discipline Conditions}
            \begin{dtabularx}{\columnwidth}{l *{3}{>{\lcol}X}}
                \tb{Original Condition} & \tb{Mitigated Condition} & \tb{Mitigated Condition} & \tb{Mitigated Condition} \\
                \bottomrule
                Panicked              & Frightened        & Shaken & None \\
                Petrified             & Paralyzed         & Slowed & None \\
                Blinded               & Visually impaired & None   & \tdash   \\
                Confused              & Disoriented       & None   & \tdash   \\
                Exhausted             & Fatigued          & None   & \tdash   \\
                Nauseated             & Sickened          & None   & \tdash   \\
                Severely impaired     & Impaired          & None   & \tdash   \\
                Stunned               & Dazed             & None   & \tdash   \\
                %Entangled             & None              & \tdash     & \tdash   \\
                Deafened              & None              & \tdash     & \tdash   \\
                Fascinated            & None              & \tdash     & \tdash   \\
                Ignited\fn{1}         & None              & \tdash     & \tdash   \\
                Immobilized           & None              & \tdash     & \tdash   \\
                Negative level\fn{2}  & None              & \tdash     & \tdash   \\
                Slowed                & None              & \tdash     & \tdash   \\
                Vulnerable            & None              & \tdash     & \tdash   \\
            \end{dtabularx}
            1.  Mitigates the impairment, but does not prevent the fighter from taking 1d6 fire damage per round until the fire is put out.  \\
            2.  Mitigate a single negative level. \\
        \end{dtable}

        \par A fighter can never use this ability more than once against a single source.
        For example, if a fighter is confused by a \spell{confusion} spell, he can use this ability to become disoriented instead of confused, but he can't then expend a second use to stop being disoriented.
        The lesser condition that this ability imposes may be cured or removed normally, but doing so does not affect the resurgence of the condition the fighter was originally afflicted with.
        If a fighter uses this ability to mitigate or negate a condition which he must suffer as a sacrifice or cost to gain some benefit, he automatically forfeits the benefit he would have gained.

        \cf{Ftr}[5th]{Battlemaster}
        The fighter becomes a master of controlling the ebb and flow of battle.
        When he use your combat supremacy ability, he may choose for the struck creature to be \goaded by you or \shaken by you instead of impaired.

        \cf{Ftr}[6th]{Armor Discipline}
        The fighter's training grants you additional capability in armor.
        He must choose to improve your agility or your resilience in armor.
        This applies to all armor discipline abilities the fighter has.

        If you improve your agility, he treats body armor he wears as less encumbering.
        He reduces its \glossterm{encumbrance penalty} by 2 its arcane spell failure by 10\%.
        In addition, he treats it were one encumbrance category lighter than it is whenever doing so would be beneficial for you.
        Heavy armor is treated as medium armor, medium armor is treated as light armor, and light armor is treated as being unarmored.
        This can remove the speed reduction and reduced Dexterity associated with the armor's encumbrance, as appropriate for the new encumbrance of the fighter's armor.

        If you improve your resilience, you gain damage reduction against physical damage equal to your character level. This allows you to ignore the first points of damage he would take each round.

        \cf{Ftr}[8th]{Improved Weapon Discipline}
        The fighter's training in your chosen weapons improves.
        He increases the \glossterm{critical multiplier} of your chosen weapons by 1.
        In addition, if he chose a specific weapon group, you increase your damage bonus to \plus2.
        If he did not, he becomes proficient with all weapon groups and all exotic weapons.
        He retains this benefit for one week after the training.

        \cf{Ftr}[9th]{Combat Mastery}
        The fighter may use your combat supremacy ability any number of times per day.

        \cf{Ftr}[10th]{Improved Combat Discipline}
        The fighter's ability to keep fighting despite negative influence improves.
        He can reduce conditions by two steps instead of one when using combat discipline.
        For example, a stunned fighter who used combat discipline would instead be staggered.
        \par In addition, a fighter may use combat discipline to reduce any penalties he suffers to your accuracy, damage, checks, or defenses by 2, even if the source of the penalty is not listed on the combat discipline chart.
        \par The fighter may also mitigate up to two conditions at once.

        \cf{Ftr}[12th]{Improved Armor Discipline}
        The fighter's training with your armor improves.
        If he chose agility, he treats body armor he wears as even less encumbering.
        He reduces its \glossterm{encumbrance penalty} by 4 and decreases its arcane spell failure by 20\%.
        In addition, he treats all it as if it were two encumbrance categories lighter than it is whenever doing so is beneficial for you.
        This does not stack with the benefit of the armor discipline ability.

        If he chose resilience, he may apply your damage reduction from the armor discipline ability against all damage, including from magical attacks.
        In addition, you gain a \plus1 bonus to Armor defense while wearing armor.

        \cf{Ftr}[13th]{Swift Warrior}
        The fighter can take an additional \glossterm{swift action} or \glossterm{immediate action} each round (see \pcref{Swift and Immediate Actions}).

        \cf{Ftr}[14th]{Greater Weapon Discipline}
        The fighter's training in your chosen weapons becomes still greater.
        He increases the \glossterm{critical range} of your chosen weapons by 1.
        This increase stacks with any other effects that affect critical range.
        For example, a fighter using heavy blades with the Weapon Focus feat (see \featpcref{Weapon Focus}) would have a critical range of 18-20.

        \cf{Ftr}[16th]{Greater Combat Discipline}
        The fighter's ability to keep fighting despite influence becomes still greater.
        When using combat discipline, he may ignore any condition listed on the combat discipline chart.
        In addition, he may use combat discipline to be \severelyimpaired with attacks and checks rather than suffer any non-damaging condition not listed on the chart.

        \par The fighter may also mitigate up to three conditions at once.

        \cf{Ftr}[17th]{Supreme Battlemaster}
        The fighter's ability to control the battle improves.
        Whenever you use your combat supremacy ability, he may choose for the struck creature to be \taunted or \frightened instead of impaired.

        \cf{Ftr}[18th]{Greater Armor Discipline}
        The fighter's training in your chosen armor becomes still greater.
        If he chose agility, he ignores all encumbrance penalties and arcane spell failure from body armor he wears.
        In addition, he treats body armor he wears as if it were three encumbrance categories lighter than it actually is whenever doing so would be beneficial to you.
        This does not stack with the benefits of armor discipline or improved armor discipline.
        In addition, you gain a \plus4 bonus to Reflex defense.

        If he chose resilience, you gain a \plus4 bonus to Fortitude defense.

        \cf{Ftr}[20th]{Supreme Discipline}
        The fighter's discipline in your chosen area is beyond equal.
        He must choose either weapon discipline, armor discipline, or combat discipline.
        Depending on which discipline he chooses, you gain a different bonus.
        \subcf{Supreme Weapon Discipline}
        Whenever the fighter deals damage to a creature with a physical attack, the struck creature is \severelyimpaired with attacks and checks for 2 rounds.
        \subcf{Supreme Armor Discipline}
        If the fighter chose agility, you gain a \plus2 bonus to physical defenses.

        If the fighter chose resilience, he doubles the damage reduction granted by your armor discipline ability.
        \subcf{Supreme Combat Discipline}
        The fighter can use combat discipline to be \impaired with attacks and checks instead of suffering any nondamaging negative effect with a duration.
        He may also mitigate up to four conditions at once.

\section{Mage}\label{Mage}
    \begin{dtable}
        \lcaption{Mage Progression}
        \begin{dtabularx}{\columnwidth}{>{\ccol}p{\levelcol} c >{\lcol}X}
            \tb{Level} & \tb{Spells} & \tb{Special} \\
            \bottomrule
            \nth{1}  & 3  & Arcane essence, rituals, spells \\
            \nth{2}  & 4  & Arcane defense \\
            \nth{3}  & 4  & Arcane insight \\
            \nth{4}  & 5  & Arcane essence \\
            \nth{5}  & 5  & Arcane insight \\
            \nth{6}  & 6  & Inherent magic \\
            \nth{7}  & 6  & Arcane insight \\
            \nth{8}  & 7  & Arcane essence \\
            \nth{9}  & 7  & Arcane insight \\
            \nth{10} & 8  & Inherent magic \\
            \nth{11} & 8  & Arcane insight \\
            \nth{12} & 9  & Arcane essence \\
            \nth{13} & 9  & Arcane insight \\
            \nth{14} & 10  & Inherent magic \\
            \nth{15} & 10  & Arcane insight \\
            \nth{16} & 11 & Arcane essence \\
            \nth{17} & 11 & Arcane insight \\
            \nth{18} & 12 & Inherent magic \\
            \nth{19} & 12 & Arcane insight \\
            \nth{20} & 13 & Archmage       \\
        \end{dtabularx}
    \end{dtable}

    \classbasics{Alignment} Any.

    \classbasics{Class Skills}
    \subparhead{Intelligence} Knowledge (all kinds, taken individually), Linguistics.
    \subparhead{Perception} Awareness, Spellcraft.
    \subparhead{Other} Bluff, Intimidate, Persuasion.

    \subsection{Base Class Abilities}
        A character with mage as a base class gains the following abilities.

        \classbasics{Skill Points} 5.

        \classbasics{Defenses} \plus4 Mental.

        \classbasics{Action Points} \plus3.

        \cf{Mge}{Weapon and Armor Proficiency}
        Mages are proficient with simple weapons and one other weapon group.
        They are not proficient with any type of armor or shield.
        Armor of any type interferes with a mage's arcane gestures, which can cause your spells with somatic components to fail.

        \cf{Mge}{Spell Point}
        The mage gains a spell point.
        A spell point can be spent to cast spells in place of an action point.
        She recovers all spent spell points after a \glossterm{short rest}.

    \subsection{Class Abilities}
        All mages have the following abilities.

        \cf{Mge}{Arcane Essence}
        All mages have access to great arcane power.
        However, not all mages acquired this power in the same way.
        A mage must choose an arcane essence.
        The mage's arcane essence affects several aspects of your class abilities.
        \parhead{Sorcerer} Sorcerers have an intuitive connection to magic that allows them to cast spells without preparation or training.
        \parhead{Wizard} Wizards studied arcane mysteries for years to learn the secret ways of magic.
        A wizard casts spells with your Intelligence.

        \cf{Mge}{Spells}
        A mage casts arcane spells.
        \textit{Sorcerer} --- A sorcerer's spellpower is equal to your character level or your Willpower, whichever is higher.
        The maximum spell level she can cast is equal to half your mage level (minimum 1).
        \textit{Wizard} --- A wizard's spellpower is equal to your character level or your Intelligence, whichever is higher.
        The maximum spell level she can cast is equal to half your mage level (minimum 1) or your Intelligence, whichever is lower.

        To cast a spell, a mage must spend an \glossterm{action point}.
        Every spell can also be cast as a cantrip.
        Cantrips are weaker, but do not require action points to cast.

        A 1st-level mage knows three spells.
        Every even level, you learn an additional spell.
        In addition, each time you gain a level, she may trade one of your existing spells for a different spell known.
        A mage's spells are drawn from the spells on the arcane spell list (see \pcref{Arcane Spells}).

        Almost all mages know at least one cantrip.
        Cantrips are special spells that do not require action points to cast.
        They are decribed first on the list of arcane spells (see \pcref{Cantrip List}).

        \cf{Mge}{Rituals}
        \subparhead{Sorcerer} Sorcerers cannot perform arcane rituals.
        \subparhead{Wizard} Wizards can perform arcane rituals to create unique magical effects (see \pcref{Rituals}).
        A wizard begins play with a ritual book containing two arcane rituals of your choice (see \pcref{Arcane Rituals}).

        \cf{Mge}[2nd]{Arcane Defense}[Mag]
        The mage gains a \plus2 bonus to defenses against spells.

        \cf{Mge}[3rd]{Arcane Insight}\label{Arcane Insight}
        The mage gains a greater understanding of magic.
        She chooses one of the following insights.
        Each insight can be chosen multiple times.
        Unless otherwise noted, their effects stack.
        \begin{itemize}
            \item Innate Spell: The mage chooses a spell you know as an innate spell.
                She no longer needs verbal or somatic components to cast the spell.
                \par If you choose this insight multiple times, she must choose a different spell each time.
            \item Personal Spell: The mage chooses a spell you know as a personal spell.
                She cannot miscast the spell (see \pcref{Miscasting}).
                If she would miscast it, the spell simply fails without effect.
                In addition, she automatically succeeds at all Concentration checks you make to cast the spell.
                \par If the mage chooses this insight multiple times, she must choose a different spell each time.
            \item Specialization: The mage chooses a school of magic.
                You gain an additional spell known that can only be used to learn a spell from that school.
                In exchange, she must ban a school of magic.
                She can never learn or cast spells or rituals from your banned schools.
                If you know spells from a banned school, she must immediately learn different spells from unbanned schools in their place.
                Divination cannot be chosen as a banned school.
                \par If the mage chooses this insight multiple times, she must choose to specialize in the same school each time.
        \end{itemize}

        \subparhead{Sorcerer} A sorcerer may also choose the Expanded Spell Knowledge insight.
        She chooses a single spell from the divine spell list or nature spell list and adds it to your arcane spell list.
        This does not grant your the spell as a spell known.
        \par If you choose this insight multiple times, she must choose a different spell each time.

        \subparhead{Wizard} A wizard may also choose the Ritual Spell insight.
        She scribes an arcane spell into your ritual book.
        You do not need to know the spell, and pays no cost to scribe it.
        The spell is treated as a ritual, and she can perform a one minute ritual to gain the spell's effect.
        Performing the ritual costs the normal amount of material components for a ritual of its level (see \pcref{Ritual Costs}).

        \cf{Mge}[4th]{Arcane Essence}[Mag]
        The mage gains an ability based on your choice of arcane essence.
        \subparhead{Sorcerer} The sorcerer gains bonus hit points equal to your sorcerer level or Willpower, whichever is higher.
        \subparhead{Wizard} The wizard gains a \plus2 bonus to all Knowledge skills.

        \cf{Mge}[6th]{Inherent Magic}[Mag]
        The mage becomes inherently magical.
        She may choose an arcane spell with a maximum level equal to half your mage level \sub 2.
        The spell must be a \glossterm{targeted spell} and have a duration of \durshort or longer.
        You do not need to know the spell.
        The mage constantly gains the benefits of that spell.
        If the spell has secondary effects, such as the \spell{avatar of suffering} spell, the duration of the secondary effects is not changed.

        When the mage chooses a spell in this way, the mage may also choose specific augments for that spell, as long as the spell's effective level does not exceed half your character level.
        If the spell's effect ends for any reason, such as if its effect is discharged, it takes effect on the mage again 5 minutes later.

        At your 10th mage level, and every 4 mage levels thereafter, the mage may choose an additional attuned spell.

        \cf{Mge}[8th]{Arcane Essence}[Mag]
        The mage gains an ability based on your choice of arcane essence.
        \subparhead{Sorcerer} The sorcerer gains \glossterm{magic resistance} equal to 10 \add your character level or Willpower, whichever is higher.
        This ability does not protect the sorcerer from your own miscast effects.
        If you already have magic resistance from another source, you increase that magic resistance by 2.
        \subparhead{Wizard} As a standard action, the wizard can spend an action point to begin casting two spells at once.
        She must designate one as the primary spell and the other as the secondary spell.
        The level of both spells, including any augments, must not exceed half your mage level.
        If the two spells are different levels, she must designate the higher level spell to be the primary spell.

        When you finishe casting, you choose which one of those spells takes effect.
        She spends an action point if necessary to cast the spell, and its effect takes place.
        The second spell dissipates without spending an action point or having any effect.

        If the wizard's concentration is disrupted, the primary spell is miscast.
        If she must spend an action point to cast that spell, you spend it.
        However, the secondary spell dissipates without spending an action point or having any effect.

        \cf{Mge}[12th]{Arcane Essence}[Mag]
        The mage gains an ability based on your choice of arcane essence.
        \subparhead{Sorcerer} The sorcerer gains a spell point.
        A spell point can be spent to cast spells in place of an action point.
        She recovers all spent spell points after a \glossterm{short rest}.
        \subparhead{Wizard} The wizard gains the ability to prepare a spell so it takes effect automatically if specific circumstances arise.
        To prepare a contingency, the wizard must spend 5 minutes preparing the spell.
        A contingency spell is treated as being two levels higher than its actual level for the purpose of determining whether the wizard can cast the spell.
        During this casting time, the wizard specifies what circumstances cause the spell to take effect.

        The contingency can be set to trigger in response to any circumstances that a typical human observing the wizard and your situation could detect.
        For example, a wizard could specify ``when I fall at least 50 feet'' or ``when I become bloodied'', but not ``when there is an invisible creature within 50 feet of me'' or ``when I am at 17 hit points or fewer.'' The more specific the required circumstances, the better -- vague requirements, such as ``when I am in danger'', may cause the contingency to trigger unexpectedly or fail to trigger at all.
        If a wizard attempts to specify multiple separate triggering conditions, such as ``when I take damage or when an enemy is adjacent to me'', the contingency will randomly ignore all but one of the conditions.

        The spell must have a casting time of 1 standard action or less.
        The contingency can specify a simple rule for identifying how to target the spell, such as ``the closest enemy''.
        If the rule is poorly worded or imprecise, the contingency may target incorrectly or fail to activate at all.
        Any spells which require decisions, such as the \spell{dimension door} spell, must have those decisions made at the time the contingency is created.
        The wizard cannot change those decisions when the contingency takes effect.

        A wizard can have only one contingency active at a time.
        If you create another contingency, it replaces your old contingency.

        \cf{Mge}[16th]{Arcane Essence}[Mag]
        The mage gains an ability based on your choice of arcane essence.
        \subparhead{Sorcerer} Whenever the sorcerer resists a spell with your magic resistance, you gain the ability to cast that spell once during the next 5 rounds.
        The spell retains all augments, effects from feats and other abilities, and similar modifications from the original caster, and the sorcerer cannot choose any other augments or apply effects from your own abilities.
        However, you make all other decisions required to cast the spell, and uses your spellpower to determine the spell's effects.
        Once you cast the spell, you expend the absorbed energy, can cannot cast it again.

        \subparhead{Wizard} The wizard may ready two spells in your contingency instead of a single spell.
        In addition, she may have two contingencies active at once instead of one.
        Only one contingency can trigger in a given round.
        If both would trigger, only the first contingency cast triggers, and the second does not.

        \cf{Mge}[20th]{Archmage}
        The mage no longer needs to spend action points to cast spells.
        If you ha any spell points, you lose those spell points and gains the same number of legend points.

\section{Monk}\label{Monk}
    \begin{dtable}
        \lcaption{Monk Progression}
        \begin{dtabularx}{\columnwidth}{>{\ccol}p{\levelcol} c >{\lcol}X}
            \tb{Level} & \tb{\Ki Strike} & \tb{Special} \\
            \bottomrule
            \nth{1}  & \plus1 & Enlightened defenses, unarmed warrior     \\
            \nth{2}  & \plus1 & Manifest \ki                              \\
            \nth{3}  & \plus1 & Wholeness of body, uncanny dodge \\
            \nth{4}  & \plus2 & Manifest \ki                              \\
            \nth{5}  & \plus2 & Flurry of blows                           \\
            \nth{6}  & \plus2 & Manifest \ki                              \\
            \nth{7}  & \plus3 & Perfect motion                            \\
            \nth{8}  & \plus3 & Manifest \ki                              \\
            \nth{9}  & \plus3 & Improved uncanny dodge, perfect soul      \\
            \nth{10} & \plus4 & Manifest \ki                              \\
            \nth{11} & \plus4 & Flow of life                              \\
            \nth{12} & \plus4 & Manifest \ki                              \\
            \nth{13} & \plus5 & Perfect mind                              \\
            \nth{14} & \plus5 & Manifest \ki                              \\
            \nth{15} & \plus5 & Perfect body                              \\
            \nth{16} & \plus6 & Manifest \ki                              \\
            \nth{17} & \plus6 & Perfect life                              \\
            \nth{18} & \plus6 & Manifest \ki                              \\
            \nth{19} & \plus7 & True perfection                           \\
            \nth{20} & \plus7 & Manifest \ki, transcend mortality         \\
        \end{dtabularx}
    \end{dtable}

    \classbasics{Alignment} Any nonchaotic.

    \classbasics{Class Skills}
    \subparhead{Strength} Climb, Jump, Sprint, Swim.
    \subparhead{Dexterity} Acrobatics, Escape Artist, Ride, Stealth.
    \subparhead{Intelligence} Heal.
    \subparhead{Perception} Awareness, Spellcraft, Survival.
    \subparhead{Other} Bluff, Intimidate, Perform, Persuasion.

    \subsection{Base Class Abilities}
        A character with monk as a base class gains the following abilities.

        \classbasics{Skill Points} 10.

        \classbasics{Defenses} \plus2 Fortitude, \plus4 Reflex, \plus4 Mental.

        \classbasics{Action Points} \plus3.

        \cf{Mnk}{Weapon and Armor Proficiency}
        Monks are proficient with simple weapons, monk weapons, and any one other weapon group.
        Monks are not proficient with any armor or shields.
        When wearing armor, using a shield, or carrying a medium or heavy load, a monk loses the benefit of your enlightened defense, fast movement, and \ki abilities.

        \cf{Mnk}{Ki Strike}[Mag] A monk's fists, and all weapons you use, gain a \plus1 \glossterm{enhancement bonus}.
        This functions like an enhancement bonus on a weapon, increasing your damage and offensive legend points per day (see \pcref{Weapon Enhancement Bonuses}).
        In addition, you are always treated as if you wa wearing \plus1 armor, granting your temporary hit points and defensive legend points (see \pcref{Armor Enhancement Bonuses}).
        This bonus increases by \plus1 at 4th level and every 3 levels thereafter.

    \subsection{Class Abilities}
        All monks have the following abilities.

        \cf{Mnk}{Enlightened Defense}[Mag]
        A monk's \ki shields your body from attacks.
        When \monkunencumbered, you gain a \plus2 bonus to Armor defense.
        She loses this bonus when you are helpless.

        \cf{Mnk}{Ki Power}
        Many monk abilities depend on your \ki power.
        A monk's \ki power is equal to your Willpower or your character level, whichever is higher.

        \cf{Mnk}{Unarmed Warrior}
        A monk's unarmed attacks are exceptionally deadly.
        You gain Unarmed Fighting as a bonus feat, making your proficient with your unarmed attack (see \featpcref{Unarmed Fighting}).
        In addition, you increase the damage of your unarmed attacks by two increments (see \pcref{Weapon Size}).
        For example, a Medium monk would deal 1d6 damage with your unarmed attack.
        For details about how to fight while unarmed, see \pcref{Unarmed Combat}.

        \cf{Mnk}[2nd]{Manifest Ki}[Mag]
        The monk gains the ability to channel your \ki energy to temporarily enhance your abilities.
        She chooses one \ki manifestation from the list below.

        Using a \ki manifestation costs a \ki point.
        The monk has a number of \ki points equal to your Willpower or your monk level, whichever is higher.
        \Ki points can be recovered by meditation.
        If the monk meditates for 1 hour, you recover all spent \ki points.

        Some \ki manifestations have minimum monk levels, as indicated in the title of the power.
        At your 4th monk level, and every even monk level thereafter, the monk learns an additional \ki manifestation.

        Some \ki manifestations have the effects of spells.
        Unless otherwise noted, the monk's effective spellpower with these abilities is equal to your \ki power.

        \subcf{Elegant Whirl of Fluid Motion}
        As a swift action, the monk can gain a \plus20 bonus to Acrobatics checks until the end of the round.
        \subcf{Leap of the Heavens}
        As a swift action, the monk can gain a \plus20 bonus to Jump checks until the end of the round.
        \subcf{Scale the Highest Tower}
        As a swift action, the monk can gain a \plus20 bonus to Climb checks until the end of the round.
        \subcf{4th -- Dance of Falling Feathers}
        As an immediate action, when the monk begins falling, she can gain the benefits of the \spell{feather fall} spell.
        \subcf{4th -- Fists of Distant Force}
        As a swift action, the monk can empower your unarmed attacks with \ki, allowing you to strike distant foes.
        Until the end of the round, you gain an additional ten feet of reach with your unarmed attacks, extending your threatened area.
        \subcf{6th -- Burst of Blinding Speed}
        As a swift action, the monk can gain a \plus30 foot bonus to your land speed, up to a maximum of double your original speed.
        In addition, she cannot be followed until the end of the round.
        \subcf{6th -- Dance of the Wayward Strike}
        As an immediate action, when a foe misses the monk with a melee strike, the monk can redirect the strike.
        Both the foe and the monk must threaten a third creature.
        If the monk redirects the strike, the foe rolls the same attack against the third creature.
        \subcf{6th -- Surpass the Mortal Limits}
        As a swift action, the monk can surpass the physical limitations of your body.
        Until the end of the round, she may use your \ki power in place of your Strength, Dexterity, and Constitution when making checks.
        \subcf{8th -- Flash Step}
        As a swift action during the movement phase, the monk can teleport to anywhere she can see within 30 feet.
        If your line of effect is blocked, even by an invisible barrier, or if this would somehow place your inside a solid object, the ability fails.
        \subcf{8th -- Focus the Wayward Mind}
        As a swift action, the monk can dispel all \glossterm{Mind} effects that are affecting your.
        This has no effect on Mind effects that cannot be dispelled.
        \subcf{8th -- Ki-Disrupting Strike}
        As an immediate action, when the monk hits with a melee strike, she can make the struck creature \impaired with attacks and checks for 2 rounds.
        \subcf{8th -- See the Flow of Life}
        As a swift action, the monk can gain the ability to see the \ki of living creatures until the end of the round.
        She can ``see'' any living creatures and their equipment within 50 feet perfectly, regardless of lighting conditions, invisibility, or any other means of concealment.
        This cannot detect living creatures through solid walls, however.
        \subcf{10th -- Dance of the Foolish Blow}
        This \ki manifestation functions as the \textit{dance of the wayward strike} manifestation, except that it can affect any melee attack, not just a single strike.
        The monk must have the \textit{dance of the wayward strike} \ki manifestation to learn this manifestation.
        \subcf{12th -- Diamond Fists}
        As a swift action, the monk can empower your unarmed attacks with incredible force.
        Until the end of the round, she may use your \ki power in place of your normal modifiers to physical damage with your unarmed attacks.
        In addition, you treat your unarmed strike as if it were an adamantine weapon for the purpose of overcoming damage reduction and hardness.
        \subcf{12th -- Stunning Fist}
        As an immediate action, when the monk deals damage with an unarmed melee strike, she can make a \Ki power vs. Fortitude attack against the struck creature.
        Success means the target is \staggered for 2 rounds.
        Critical success means the target is \stunned for 2 rounds.
        \norepeatnotes
        \subcf{14th -- Awaken the Pacifist Heart}
        As an immediate action, when the monk hits with a melee strike, she can make a \Ki power vs. Mental attack against the struck creature.
        Success means the target is unable to take violent actions, such as attacking, for 2 rounds.
        If the target takes damage after the current round, the effect is broken.
        \subcf{16th -- Flash Burst}
        As a swift action during the movement phase, the monk can teleport to anywhere within 1,000 feet.
        She must clearly visualize the destination, but you doe not need line of sight or line of effect.
        If the destination is occupied, or dramatically different from how she visualized it, the effect fails.
        \subcf{20th -- Ki-Shattering Strike}
        As an immediate action, when the monk hits with a melee strike, she can disrupt your foe's \ki.
        The target is \severelyimpaired with attacks and checks for 2 rounds.

        \cf{Mnk}[3rd]{Wholeness of Body}[Mag]
        With concentration and focus, the monk can correct the flow of energy within your body.
        She can use this ability as a standard action by spending a \ki point, or with one minute of meditation otherwise.
        If you doe, you heal 1d6 hit points per \ki power.

        \cf{Mnk}[3rd]{Uncanny Dodge}
        The monk can react to danger before your senses would normally allow your to do so.
        The monk reduces your overwhelm penalties by 1.
        If your overwhelm penalty is reduced to 0, you are not considered to be overwhelmed.
        In addition, you are not \unaware when attacked by surprise.

        \cf{Mnk}[5th]{Flurry of Blows}\label{Flurry of Blows}
        As a standard action, the monk can spend a \ki point to unleash a furious barrage of blows.
        She can make a \glossterm{standard attack} with an extra \glossterm{strike}.
        Alternately, she can choose to make one \glossterm{strike} against all foes you threaten.

        \cf{Mnk}[7th]{Perfect Motion}[Mag]
        The monk becomes immune to effects that restrict your mobility.
        She suffers no penalties for acting underwater.
        In addition, you gain a \plus20 bonus to Reflex defense against grapple attacks, as well as on grapple attacks or Escape Artist checks made to escape a grapple or a pin.

        \cf{Mnk}[9th]{Improved Uncanny Dodge}
        The monk reduces your overwhelm penalties by 2.
        This does not stack with the effects of uncanny dodge.
        If your overwhelm penalty is reduced to 0, you are not considered to be overwhelmed.

        \cf{Mnk}[9th]{Perfect Soul}[Mag]
        The monk gains \glossterm{magic resistance} equal to 10 \add your \ki power.

        \cf{Mnk}[11th]{Flow of Life}[Mag]
        At the end of each round, the monk heals hit points equal to your \ki power.

        \cf{Mnk}[13th]{Perfect Mind}[Mag]
        The monk becomes immune to hostile \glossterm{Mind} effects.

        \cf{Mnk}[15th]{Perfect Body}[Mag]
        The monk becomes immune to being blinded, deafened, fatigued, exhausted, nauseated, sickened, and staggered.
        In addition, she no longer takes penalties to your attribute scores for aging, and cannot be magically aged.
        The monk still dies of old age when your time is up.

        \cf{Mnk}[17th]{Perfect Life}[Mag]
        At the end of each round, the monk heals hit points equal to twice your \ki power.
        This replaces the benefit from your flow of life ability.
        If you ha taken vital damage, she instead heals vital damage equal to half your \ki power.
        In addition, you become immune to \glossterm{Death} effects.

        \cf{Mnk}[19th]{True Perfection}[Mag]
        The monk becomes inhumanly perfect.
        If you roll less than a 5 on any d20 roll, it is treated as a 5.

        \cf{Mnk}[20th]{Transcend Mortality}[Mag]
        If the monk dies, she may choose to retain control of your body and soul through sheer force of will.
        Your body immediately disappears, and your soul does not travel to an afterlife.
        Instead, your body reforms with no trace of its injuries 24 hours later.
        The reformed body is in perfect health and can be any age the monk chooses, to a minimum of the age of adulthood for your race.
        She can reform your body at the place where she died, or in any place on the same plane that is deeply familiar to your.

        After each time the monk reforms herself this way, it takes 24 additional hours to reform the next time she ``dies''.
        A monk with this ability can only be permanently killed by the direct intervention of a deity.

        \subsubsection{Ex-Monks}
            A monk who becomes chaotic loses your \ki powers, and cannot gain more levels as a monk.
            She retains all your other class abilities.
            If you stop being chaotic, you regain your \ki powers and ability to take monk levels.

\section{Paladin}\label{Paladin}
    \begin{dtable}
        \lcaption{Paladin Progression}
        \begin{dtabularx}{\columnwidth}{>{\ccol}p{\levelcol} >{\lcol}X}
            \tb{Level} & \tb{Special} \\
            \bottomrule
            \nth{1}  & Divine invocation (smite), divine protection \\
            \nth{2}  & Divine invocation                            \\
            \nth{3}  & Discernment (alignment)                      \\
            \nth{4}  & Divine presence                              \\
            \nth{5}  & Pass judgment                                \\
            \nth{6}  & Divine invocation                            \\
            \nth{7}  & Discernment (lies)                           \\
            \nth{8}  & Divine presence                              \\
            \nth{9}  & Expanded presence                            \\
            \nth{10} & Divine invocation                            \\
            \nth{11} & Discernment (invisibility)                   \\
            \nth{12} & Divine presence                              \\
            \nth{13} & Lingering presence                           \\
            \nth{14} & Divine invocation                            \\
            \nth{15} & Discernment (truth), martyr's retribution    \\
            \nth{16} & Divine presence                              \\
            \nth{17} & Mighty presence                              \\
            \nth{18} & Divine invocation                            \\
            \nth{19} & Discernment (thoughts)                       \\
            \nth{20} & Aligned soul, divine presence                 \\
        \end{dtabularx}
    \end{dtable}

    \classbasics{Alignment} Any other than true neutral.

    \classbasics{Class Skills}
    \subparhead{Dexterity} Ride.
    \subparhead{Intelligence} Heal, Knowledge (local, religion).
    \subparhead{Perception} Awareness, Intimidate, Sense Motive.
    \subparhead{Other} Bluff, Intimidate, Persuasion.

    \subsection{Base Class Abilities}
        A character with paladin as a base class gains the following abilities.

        \classbasics{Skill Points} 5.

        \classbasics{Defenses} \plus4 Fortitude, \plus4 Mental.

        \classbasics{Action Points} \plus2.

        \cf{Pal}{Weapon and Armor Proficiency}
        Paladins are proficient with simple weapons, any three other weapon groups, all types of armor (heavy, medium, and light), and shields.

        \cf{Pal}{Divine Protection}[Mag]
        The paladin's force of belief manifests a divine protection around your.
        She may add your Willpower to your Armor defense in place of Dexterity or Constitution (see \pcref{Defenses}).

    \subsection{Class Abilities}
        All paladins have the following abilities.

        \cf{Pal}{Devoted Alignment}
        A paladin is devoted to a specific alignment.
        She must choose one of your alignment components: good, evil, lawful, or chaotic.
        The alignment you choose is your devoted alignment.
        A paladin's class abilities are affected by this choice.
        She excels at slaying creatures with alignments opposed to your devoted alignment.
        A paladin's devoted alignment cannot be changed without extraordinary repurcussions to the paladin.

        \cf{Pal}{Divine Power}
        Many paladin abilities depend on your divine power.
        A paladin's divine power is equal to your Willpower or your character level, whichever is higher.

        \cf{Pal}{Divine Invocation}[Mag]
        A paladin can invoke the power of your alignment to achieve incredible effects.
        This ability can be used a number of times per day equal to your Willpower or your paladin level, whichever is higher.
        You gain the smite divine invocation.

        At 2nd level, and every even level thereafter, the paladin gains an additional divine invocation.
        Most divine invocations have minimum paladin levels, as indicated in the title of the ability.
        Some divine invocations are also restricted to paladins with specific devoted alignments.
        The paladin's accuracy with divine invocations is equal to your divine power.
        If a divine invocation emulates a spell, the paladin's effective spellpower is equal to your divine power.

        Divine powers marked with an asterisk are called smite powers.
        Smite powers function like the smite divine invocation, except that they also have additional effects.
        Unless otherwise noted, these additional effects only occur if the smited creature does not share the paladin's devoted alignment.

        \parhead{Any Alignment}

        \subcf{Smite}
        Once per round, when the paladin makes a strike, she may declare that strike to be a smite.
        She may use your divine power in place of your normal accuracy for that strike.
        If the struck creatture shares your devoted alignment, you deal no damage at all (not even normal weapon damage), but the use of the ability is still spent.
        Otherwise, your weapon deals maximum damage, and you deal bonus damage equal to your divine power.
        \subcf{2nd -- Bless} This invocation functions like the \spell{bless} spell.
        \subcf{2nd -- Lay on Hands}
        As a standard action, the paladin can lay hands on a creature.
        The target is healed for 1d6 damage per divine power.
        \subcf{2nd -- Protection from Alignment} This invocation functions like the \spell{protection from alignment} spell.
        The paladin must protect the target from the alignment opposed to your devoted alignment.
        \subcf{6th -- Exhausting Smite*}
        The paladin makes a Divine power vs. Fortitude attack against the struck creature.
        Success means it is \exhausted for 2 rounds.
        \subcf{6th -- Resounding Smite*}
        The struck creature is knocked prone.
        \subcf{6th -- Seeking Smite*}
        This smite attack ignores any miss chances, such as from active cover or visual impairment.
        The weapon must still be physically able to strike the target.
        \subcf{6th -- Goading Smite*}
        The struck creature is \goaded by the paladin for 2 rounds.
        \subcf{10th -- Dispelling Smite*}
        The struck creature is affected by \spell{dispel magic}.
        \subcf{10th -- Divine Might}
        This invocation functions like the \spell{divine might} spell.
        \subcf{10th -- Penetrating Smite*}
        The struck creature's armor is weakened.
        For the next 2 rounds, attacks against the target that would normally target Armor defense are instead made against the lower of the target's Armor and Reflex defenses.
        \subcf{14th -- Dazing Smite*}
        The struck creature is \dazed for 2 rounds.
        \subcf{14th -- Spellreaving Smite*}
        All spells and magical effects on the struck creature are dispelled.
        Spells and effects that cannot be removed by \spell{dispel magic} are unaffected.
        The paladin must have the dispelling smite invocation to choose this invocation.
        \subcf{14th -- Terrifying Smite*}
        The paladin makes a Divine power vs. Mental attack against the struck creature.
        Success means it is \frightened by your for 2 rounds.
        \subcf{18th -- Converting Smite*}
        The paladin's smite shows your foe the error of its ways.
        She makes a Divine power vs. Mental attack against the struck creature.
        Success means it is \confused for 2 rounds.
        Critical success means its alignment changes, and it gains the paladin's devoted alignment for 1 week.
        After that time, it can choose to return to its original alignment, or keep its new alignment permanently.
        Failure means it is \dazed for 2 rounds.
        \subcf{18th -- Immobilizing Smite*}
        The creature struck with this smite is \immobilized for 5 rounds.

        \parhead{Chaos Divine Invocations}
        \subcf{6th -- Confusion} This invocation functions like the \spell{confusion} spell, except that it affects targets within \rngmed range.
        \subcf{6th -- Freedom} This invocation functions like the \spell{freedom} spell.
        \subcf{10th -- Break the Chains}
        As a standard action, the paladin can break all shackles, bindings, and locks within a \arealarge radius burst of your.
        Nonmagical objects are automatically broken.
        To break magical objects, the paladin must make an attack with your divine power against a DR equal to 10 \add the object's spellpower.
        \subcf{10th -- Chaotic Redirection}
        As an immediate action, when the paladin or any of your allies within \rngclose range is struck by a physical attack, the paladin can redirect the attack to a random creature within \rngclose range of the paladin, including the paladin.
        The attack is made against that creature instead of its original target, using its original accuracy, and has its normal effects if it hits.
        After using this invocation, the paladin cannot use it for 5 rounds.
        \subcf{14th -- Discordant Song} This invocation functions like the \spell{discordant song} spell.

        \parhead{Good Divine Invocations}
        \subcf{Shield Other}
        This invocation functions like the \spell{share pain} spell.
        \subcf{6th -- Martyr's Shield}
        As an immediate action, when an ally within \rngmed range would take vital damage, the paladin can take that damage as regular damage instead.
        After using this invocation, the paladin cannot use it again for 5 rounds.

        \parhead{Evil Divine Invocations}
        \subcf{2nd -- Enfeeblement} This invocation functions like the \spell{enfeeblement} spell.
        \subcf{6th -- Agony} This invocation functions like the \spell{agony} spell.
        \subcf{10th -- Enervation} This invocation functions like the \spell{enervation} spell.
        \subcf{10th -- Executing Smite*}
        The paladin makes a Divine power vs. Fortitude attack against the struck creature.
        Success means the target dies if it has no hit points remaining after taking damage from the smite.

        \parhead{Law Divine Invocations}
        \subcf{2nd -- Command} This invocation functions like the \spell{command} spell.
        \subcf{2nd -- Hold Person} This invocation functions like the \spell{hold person} spell.
        \subcf{6th -- Read Mind} This invocation functions like the \spell{read mind} spell.
        \subcf{10th -- Prohibition} This invocation functions like the \spell{prohibition} spell.

        \cf{Pal}[3rd]{Discernment}[Mag]
        A paladin can discern truths about creatures you see as a swift action.
        When you use this ability, you learn which creatures within a \arealarge cone have your devoted alignment.

        The paladin may use your discernment ability a number of times per day equal to your Perception or half your paladin level (minimum 1), whichever is higher.

        At 7th level, the paladin can discern lies.
        Whenever a creature in the area intentionally lies, the paladin knows the statement was a lie.
        This does not reveal the truth, uncover unintentional inaccuracies, or necessarily reveal evasions.

        At 11th level, the paladin can also see invisible creatures and objects within the area as if they were normally visible.
        They are visible as translucent shapes, allowing you to easily distinguish between visible and invisible creatures and objects.

        At 15th level, the paladin can discern truth.
        When a lie is spoken, in addition to learning that the statement is a lie, the paladin learns what the creature believes the truth to be.
        This does not necessarily reveal the actual truth -- merely what the creature believes.

        At 19th level, the paladin can discern thoughts.
        She knows the surface thoughts of all creatures in the area.
        You gain a \plus4 bonus to Bluff, Persuasion, and Intimidate attacks and checks against a creature whose thoughts she can read.

        \cf{Pal}[4th]{Divine Presence}[Mag]
        The paladin's presence alters the world around your.
        She chooses a single divine presence from the list below.
        Each divine presence affects a \areamed radius emanation from the paladin, including herself.
        She may choose to suppress or resume your divine presence as a swift action.

        At your 8th paladin level, and every 4 levels thereafter, the paladin gains an additional divine presence.
        She may have multiple divine presences active simultaneously, and suppress or resume them individually.
        Most divine presences have minimum paladin levels, as indicated in the title of the ability.
        If a divine presence emulates a spell, the paladin's effective spellpower is equal to your paladin level or your Willpower, whichever is higher.

        \parhead{Any Alignment}
        \subcf{8th -- Worthy Foe}
        Enemies in the area are \goaded by the paladin.
        \subcf{16th -- Aura of Unbending Purpose}
        Allies in the area are immune to \glossterm{Mind} effects.

        \parhead{Chaotic Divine Presences}
        \subcf{Mobile Aura}
        Allies in the area can move at full speed through threatened squares.
        \subcf{8th -- Accelerating Aura}
        Allies in the area gain a \plus10 foot bonus to land speed.
        \subcf{12th -- Aura of Freedom}
        Allies in the area gain the benefits of the \spell{freedom} spell.
        \subcf{20th -- Maddening Aura}
        At the start of each round, enemies in the area are \disoriented that round.

        \parhead{Evil Divine Presences}
        \subcf{Overwhelming Aura}
        Enemies in the area that are \glossterm{overwhelmed} increase their overwhelm penalties by 1.
        \subcf{8th -- Lifefeeding Aura}
        At the end of each round, if another creature in the area took damage, the paladin regains hit points equal to the damage taken, up to a maximum of your divine power.
        \subcf{12th -- Baleful Aura}
        Enemies in the area are \impaired with attacks and checks.
        \subcf{12th -- Painful Aura}
        Whenever an enemy in the area takes damage, the damage is doubled, up to a maximum bonus equal to the paladin's divine power.
        Each enemy can only take this bonus damage each round.
        \subcf{16th -- Sacrificial Aura}
        Whenever you take damage, any other creature in the area may choose to take that damage instead.
        This damage ignores all forms of damage reduction and similar abilities.
        Any damage in excess of the creature's hit points is not redirected.

        \parhead{Good Divine Presences}
        \subcf{Defensive Aura}
        Allies in the area that are \glossterm{overwhelmed} reduce their overwhelm penalties by 1.
        If an ally's overwhelm penalty is reduced to 0, they are not considered to be overwhelmed.
        \subcf{8th -- Minor Healing Aura}
        Whenever an ally in the area regains hit points, the healing is doubled, up to a maximum bonus equal to the paladin's divine power.
        Each ally can only receive this bonus once per round.
        \subcf{12th -- Healing Aura}
        At the end of each round, allies in the area heal hit points equal to the paladin's divine power.
        \subcf{16th -- Martyr's Aura}
        Whenever an ally in the area takes damage, the paladin may choose to take that damage instead.
        This damage ignores all forms of damage reduction and similar abilities.
        If the paladin takes damage in excess of your hit points in this way, the excess damage is dealt directly as vital damage.

        \parhead{Lawful Divine Presences}
        \subcf{Aura of Fortification}
        Enemies in the area move at half speed through threatened squares.
        \subcf{8th -- Aura of Inhibition}
        Enemies in the area move at half speed.
        \subcf{12th -- Aura of Truth}
        All illusory figments and glamers are suppressed in the area.
        \subcf{20th -- Inescapable Aura}
        At the start of each round, enemies in the area are \immobilized that round.

        \cf{Pal}[5th]{Pass Judgment}[Mag]
        The paladin gains the ability to pass judgment on those you deem unworthy.
        Once per day as a swift action, she may pass judgment on a creature within 100 feet of your.
        For the purpose of spells and effects, the creature is treated as if it had the alignment opposed to the paladin's devoted alignment.
        This effect lasts for one week, or until the paladin changes your mind about the subject.
        This does not change the creature's actions or behavior, but the creature is subject to the paladin's smite attack, would detect as that alignment under the inspection of a \spell{detect alignment} spell, and so on.

        No attack is required for this effect, and it cannot be dispelled, but a \spell{remove curse}, \spell{miracle}, or \spell{wish} spell can remove it.
        The paladin can use this ability an additional time per day at your 9th paladin level and every four levels thereafter.
        A paladin should be careful when using this ability, as persecution of allies can lead overzealous paladins to fall.

        \cf{Pal}[9th]{Expanded Presence}[Mag]
        The paladin's divine presences affect an \arealarge radius.

        \cf{Pal}[13th]{Lingering Presence}[Mag]
        The effects of the paladin's divine presences continue for 1 round after targets leave the area.

        \cf{Pal}[15th]{Martyr's Retribution}[Mag]
        If the paladin dies in the devoted service of your devoted alignment, she may choose to have your fallen body erupt in an immense burst of divine energy.
        If you doe, your body is almost completely consumed, preventing your from being raised with \ritual{resurrection} and similar effects that require an intact body.
        This burst has two effects.
        First, a \spell{sunburst} spell immediately takes effect over the area where the paladin fell.
        Second, a \spell{storm of vengeance} spell begins to take effect, centered on the same area.
        The spell lasts for 10 rounds, and the lightning strikes target the paladin's enemies.
        Both of these effects harm only the paladin's foes, and do not harm your allies.
        However, your allies' vision is still impeded by the \spell{storm of vengeance}.

        \cf{Pal}[17th]{Mighty Presence}[Mag]
        The paladin's divine presences affect an \areahuge radius.
        In addition, their effects continue for 2 rounds after targets leave the area.

        \cf{Pal}[20th]{Aligned Soul}[Mag]
        While a paladin is dead, she may approach the deity or governing figure of your afterlife and request to be returned to life to continue your mission.
        Travelling to the relevant figure and making the request takes 12 hours.
        Unless there are extenuating circumstances, this request is almost always granted, and you are resurrected in a new body at a location of the entity's choice.
        This functions like the \spell{resurrection} ritual, except that no part of the body is required, and a new body is created by the entity.
        She can be resurrected in this way regardless of the condition of your body, but not if your soul has been trapped or otherwise prevented from going to the correct afterlife.

        \subsubsection{Ex-Paladins}
            A paladin who ceases to follow your devoted alignment loses all magical paladin class abilities.
            She may not gain any additional paladin levels.
            She regains your abilities and advancement potential if you atone for your violations (see the \spell{atonement} ritual), as appropriate.

\section{Ranger}\label{Ranger}
    \begin{dtable}
        \lcaption{Ranger Progression}
        \begin{dtabularx}{\columnwidth}{>{\ccol}p{\levelcol} c >{\lcol}X}
            \tb{Level} & \tb{Quarry} & \tb{Special} \\
            \bottomrule
            \nth{1}  & \plus2 & Quarry, tenacious hunter, wild speech \\
            \nth{2}  & \plus2 & Hunting skill, tracker                \\
            \nth{3}  & \plus2 & Free stride, keen vision              \\
            \nth{4}  & \plus2 & Hunting lore                          \\
            \nth{5}  & \plus3 & Rapid tracker                         \\
            \nth{6}  & \plus3 & Hunting skill                         \\
            \nth{7}  & \plus3 & Blindsense                            \\
            \nth{8}  & \plus3 & Hunting lore                          \\
            \nth{9}  & \plus3 & Perfect stride                        \\
            \nth{10} & \plus4 & Hunting skill                         \\
            \nth{11} & \plus4 & Blindsight                            \\
            \nth{12} & \plus4 & Hunting lore                          \\
            \nth{13} & \plus4 & Implacable hunter                     \\
            \nth{14} & \plus4 & Hunting skill                         \\
            \nth{15} & \plus5 & Farsight                              \\
            \nth{16} & \plus5 & Hunting lore                          \\
            \nth{17} & \plus5 & Eternal quarry                        \\
            \nth{18} & \plus5 & Hunting skill                         \\
            \nth{19} & \plus5 & Truesight                             \\
            \nth{20} & \plus6 & Hunting lore
        \end{dtabularx}
    \end{dtable}

    \classbasics{Alignment} Any.

    \classbasics{Class Skills}
    \subparhead{Strength} Climb, Jump, Sprint, Swim.
    \subparhead{Dexterity} Acrobatics, Escape Artist, Ride, Stealth.
    \subparhead{Intelligence} Heal, Knowledge (dungeoneering, geography, nature).
    \subparhead{Perception} Awareness, Creature Handling, Survival.
    \subparhead{Other} Bluff, Intimidate, Persuasion.

    \subsection{Base Class Abilities}
        A character with ranger as a base class gains the following abilities.

        \classbasics{Skill Points} 15.

        \classbasics{Defenses} \plus4 Fortitude, \plus4 Reflex, \plus2 Mental.

        \classbasics{Action Points} \plus1.

        \cf{Rgr}{Weapon and Armor Proficiency}
        A ranger is proficient with simple weapons, any two weapon groups, light and medium armor, and shields.
        You are also proficient with your choice of bows, crossbows, or thrown weapons.

        \cf{Rgr}{Tenacious Hunter}
        The ranger adds your quarry bonus to your defenses against attacks that your quarry makes.
        See the Quarry ability, below, for details.

        \cf{Rgr}{Wild Speech}[Mag]
        The ranger learns how to communicate with animals.
        This ability functions like the druid ability of the same name (see Wild Speech, \pref{Drd:Wild Speech}).
        A ranger can use this ability a number of times per day equal to your Perception or half your ranger level, whichever is higher.

    \subsection{Class Abilities}
        All rangers have the following abilities.

        \cf{Rgr}{Quarry}
        As a swift action, a ranger may designate any foe he sees as your quarry.
        A ranger gains a \plus2 bonus to damage with physical attacks and Awareness, Stealth, and Survival checks against your quarry.
        However, while a ranger has designated a quarry, he takes a \minus2 penalty on the same rolls against any target other than your quarry.

        A ranger may not normally have more than one quarry at once.
        He may not designate a new quarry until you defeat your old quarry, or until he gives up on the quarry.
        He may give up pursuing a quarry as a free action.
        If you do, you are unable to designate a new quarry until he rests for 5 minutes.

        Some abilities allow the ranger to designate multiple creatures as a quarry.
        Abilities which designate multiple creatures as a quarry always last for a specific amount of time.

        If the ranger does not see your quarry for more than a week, it is no longer considered your quarry.

        The amount by which a ranger's damage and skill checks increase against your quarry is called your quarry bonus.
        The ranger's quarry bonus improves by \plus1 at your 5th ranger level and every 5 ranger levels thereafter.
        Your penalties against targets other than your quarry remains the same.

        \cf{Rgr}[2nd]{Hunting Skill}
        The ranger gains any Skill feat with prerequisites that include one of the ranger class skills as a bonus feat.
        He must use your ranger level in place of your character level to meet level prerequisites for the feat.
        At your 6th ranger level, and every 4 ranger levels thereafter, you gain an additional feat.

        \cf{Rgr}[2nd]{Tracker}
        The ranger gains a \plus5 bonus to checks made to follow tracks.
        In addition, he may use your ranger level in place of the Survival skill to follow tracks (see \pcref{Survival}).

        \cf{Rgr}[4th]{Hunting Lore}
        The ranger gains an ability drawn from ancient hunting lore.
        He chooses a single hunting lore from the list below.
        Some hunting lores have minimum ranger levels, as indicated in the title of the ability.
        At your 8th ranger level, and every four ranger levels thereafter, the ranger gains an additional hunting lore.

        If a hunting lore grants the ranger a bonus feat, he must still meet any prerequisites for the feat to gain its effect.

        \subcf{Dual Quarry}
        The ranger may designate up to two targets whenever he chooses a quarry.
        You gain your quarry benefits against both targets.
        Whenever he designates new quarries, he may choose which previous quarry to give up (if any).

        \subcf{Endurance}
        The ranger gains the Endurance feat as a bonus feat (see \featpcref{Endurance}).

        \subcf{Goading Hunter}
        Whenever the ranger damages your quarry with a physical attack, it is \goaded by you for 2 rounds.

        \subcf{No Escape}
        Whenever the ranger damages your quarry with a physical attack, it moves at half speed for 2 rounds.

        \subcf{Swift Hunter}
        The ranger gains the Swift feat as a bonus feat (see \featpcref{Swift}).

        \subcf{8th -- Anchored Quarry}
        Whenever the ranger damages your quarry with a physical attack, it cannot travel extradimensionally for 2 rounds.
        This blocks teleportation and all planar travel abilities except planar rifts.

        \subcf{8th -- Flexible Quarry}
        After giving up on a quarry, the ranger does not need to wait before designating a new quarry.

        \subcf{8th -- Impaired Quarry}
        Whenever the ranger damages your quarry with a physical attack, it is \impaired with attacks and checks for 2 rounds.

        \subcf{12th -- Inescapable}
        Whenever the ranger damages your quarry with a physical attack, it moves at one-quarter speed as long as it remains your quarry.

        \subcf{12th -- Inevitable Hunter}
        The ranger ignores all miss chances and failure chances that would affect attacks and checks he makes against your quarry.

        \subcf{12th -- Punishing Presence}
        At the end of each round, if the ranger's quarry is within \rnglong range of you, it takes life damage equal to your ranger level.

        \subcf{16th -- Legendary Hunter}
        The ranger may always use legend points on attacks or checks he makes against your quarry, even when fighting monsters that normally prevent legend points from being used.

        \subcf{16th -- Master of the Hunt}
        When the ranger uses your quarry ability, he may also designate up to five willing creatures within \rnglong range of you.
        As long as they stay within \rnglong range of you, they also gain the damage bonus from the ranger's quarry ability against your quarry.
        Your allies do not suffer penalties against targets other than the quarry.

        \cf{Rgr}[3rd]{Free Stride}
        The ranger can move through any sort of natural terrain that slows or impedes movement at your normal speed without suffering any sort of impairment.
        If a skill check, such as Climb or Swim, would normally be required to move through the terrain, this ability does not help.

        \cf{Rgr}[3rd]{Keen Vision}
        The ranger's sight improves, allowing you to see more easily.
        You gain \glossterm{low-light vision}, allowing you to treat sources of light as if they had double their normal illumination range.
        If he already has low-light vision, he doubles its benefit, allowing you to treat sources of light as if they had four times their normal illumination range.
        In addition, you gain \glossterm{darkvision} out to 50 feet, allowing you to see in complete darkness.
        If he already has darkvision, he increases its range by 50 feet.

        \cf{Rgr}[5th]{Rapid Tracker}
        The ranger's ability to track your foes improves.
        He can move at your normal speed while following tracks without taking the normal \minus5 penalty.
        He takes only a \minus10 penalty (instead of the normal \minus20) when moving at up to twice normal speed while tracking.

        \cf{Rgr}[7th]{Blindsense}
        The ranger's perceptions are so finely honed that he can sense your enemies without seeing them.
        You gain the blindsense ability out to 50 feet.
        This ability allows you to sense the presence and location of objects and foes within 50 feet without seeing them.
        If he already has the blindsense ability, he increases its range by 50 feet.

        In addition, he increases the range of your darkvision by 150 feet.

        \cf{Rgr}[9th]{Perfect Stride}[Mag]
        The ranger's ability to surpass obstacles becomes unparalleled.
        He constantly acts as if he were under the effect of a \spell{freedom} spell, except that it does not allow you to act normally underwater.

        \cf{Rgr}[11th]{Blindsight}
        The ranger gains the \glossterm{blindsight} ability, allowing you to ``see'' perfectly without your eyes in a 50 foot radius around you.
        With this ability, he can fight just as well with your eyes closed as with them open.
        If he already has the blindsight ability, he increases its range by 50 feet.

        In addition, he increases the range of your darkvision by 300 feet, and your blindsense by 150 feet.

        \cf{Rgr}[13th]{Implacable Hunter}[Mag]
        The ranger cannot be blocked from pursuing your quarry.
        Your movement is not slowed by enemies threatening you (see \pcref{Moving Near Foes}).
        In addition, he may move at half speed through spaces occupied by foes.

        \cf{Rgr}[15th]{Farsight}
        The ranger increases the range of your darkvision by 500 feet, your blindsense by 300 feet, and your blindsight by 50 feet.
        In addition, you halve your \glossterm{range increment penalties} for attacking at long range.

        \cf{Rgr}[17th]{Eternal Quarry}
        The ranger's quarry never stops being your quarry until he chooses a different creature.
        In addition, once he has designated a creature as a quarry, he can always designate it as a quarry again, even if it is not in your sight.
        This only applies to creatures that he designates as a quarry after acquiring this ability.

        \cf{Rgr}[19th]{Truesight}[Mag]
        The ranger's perceptions are accurate enough to defeat even powerful magic.
        He can see through normal and magical darkness, see the truth behind visual figments and glamers, and see the true form of creatures and objects affected by \glossterm{Shaping} abilities.
        This ability works at any range.

\section{Rogue}\label{Rogue}
    \begin{dtable}
        \lcaption{Rogue Progression}
        \begin{dtabularx}{\columnwidth}{>{\ccol}p{\levelcol} c >{\lcol}X}
            \tb{Level} & \tb{Sneak Attack} & \tb{Special} \\
            \bottomrule
            \nth{1}  & \plus1d6  & Skill exemplar, sneak attack \\
            \nth{2}  & \plus1d6  & Skill talent                 \\
            \nth{3}  & \plus2d6  & Uncanny dodge                \\
            \nth{4}  & \plus2d6  & Combat trick                 \\
            \nth{5}  & \plus3d6  & Lucky slip                   \\
            \nth{6}  & \plus3d6  & Skill talent                 \\
            \nth{7}  & \plus4d6  & Improved uncanny dodge       \\
            \nth{8}  & \plus4d6  & Combat trick                 \\
            \nth{9}  & \plus5d6  & Lucky break                  \\
            \nth{10} & \plus5d6  & Skill talent                 \\
            \nth{11} & \plus6d6  & Slippery mind                \\
            \nth{12} & \plus6d6  & Combat trick                 \\
            \nth{13} & \plus7d6  & Lucky dodge                  \\
            \nth{14} & \plus7d6  & Skill talent                 \\
            \nth{15} & \plus8d6  & Persistent sneak attack      \\
            \nth{16} & \plus8d6  & Combat trick                 \\
            \nth{17} & \plus9d6  & Legendary luck               \\
            \nth{18} & \plus9d6  & Skill talent                 \\
            \nth{19} & \plus10d6 & Ambush master                \\
            \nth{20} & \plus10d6 & Combat trick                 \\
        \end{dtabularx}
    \end{dtable}

    \classbasics{Alignment} Any.

    \classbasics{Class Skills}
    \subparhead{Strength} Climb, Jump, Sprint, Swim.
    \subparhead{Dexterity} Acrobatics, Escape Artist, Sleight of Hand, Stealth.
    \subparhead{Intelligence} Devices, Disguise, Knowledge (dungeoneering, local), Linguistics.
    \subparhead{Perception} Awareness, Sense Motive.
    \subparhead{Other} Bluff, Intimidate, Perform, Persuasion.

    \subsection{Base Class Abilities}
        A character with rogue as a base class gains the following abilities.

        \classbasics{Skill Points} 15.

        \classbasics{Defenses} \plus4 Reflex, \plus2 Mental.

        \classbasics{Action Points} \plus2.

        \cf{Rog}{Weapon and Armor Proficiency}
        Rogues are proficient with simple weapons, any two weapon groups, light armor, and bucklers.
        They are also proficient with saps.

        \cf{Rog}{Skill Exemplar}
        A rogue gains a \plus2 bonus to all skills you are trained in.
        This bonus increases by 1 at your 5th rogue level and every 5 rogue levels thereafter.

    \subsection{Class Abilities}
        All rogues have the following abilities.

        \cf{Rog}{Sneak Attack}
        Once per round, when a rogue hits with a physical attack against a creature unable to defend itself effectively, she can make that attack a sneak attack.
        She can make a sneak attack if your target is \unaware, \defenseless, or suffering \glossterm{overwhelm penalties} from being surrounded by enemies (see \pcref{Overwhelm}).
        A sneak attack deals 1d6 points of bonus damage.

        The extra damage dealt by a sneak attack increases by 1d6 at your 3rd rogue level and every two rogue levels thereafter.

        Ranged attacks can count as sneak attacks only if the target is within 30 feet.
        A rogue can't strike with deadly accuracy from beyond that range.

        A rogue can only sneak attack creatures with a discernible body structure.
        Oozes, incorporeal creatures, and some plants lack vital areas to attack.
        Any creature that is immune to critical hits is not vulnerable to sneak attacks.
        The rogue must be able to see the target well enough to pick out a vital spot and must be able to reach such a spot.
        A rogue cannot sneak attack while striking the limbs of a creature whose vitals are beyond reach.
        Usually, this means she can only make sneak attacks against creatures no more than two size categories larger than your.

        \cf{Rog}[2nd]{Skill Talent}
        The rogue's skills improve.
        You gain an additional skill point, which she can place in any skill, and a bonus Skill feat for which you qualifie (see \pcref{Skill Feats}).
        She must use your rogue level in place of your character level to meet level prerequisites for the feat.
        At your 6th rogue level, and every four rogue levels thereafter, you gain an additional skill point and Skill feat.

        \cf{Rog}[3rd]{Uncanny Dodge}
        The rogue can react to danger before your senses would normally allow your to do so.
        The rogue reduces your overwhelm penalties by 1.
        If your overwhelm penalty is reduced to 0, you are not considered to be overwhelmed.
        In addition, you are not \unaware when attacked by surprise.

        \cf{Rog}[4th]{Combat Tricks}
        The rogue gains a combat trick to aid your and confound your foes.
        She chooses a single combat trick from the list below.
        Some combat tricks have minimum rogue levels, as indicated in the title of the ability.
        At your 8th rogue level, and every four rogue levels thereafter, the rogue gains an additional combat trick.

        Some combat tricks depend on a rogue's trick power.
        A rogue's trick power is equal to your Intelligence or your character level, whichever is higher.

        Tricks marked with an asterisk are called ambush attacks.
        Ambush attacks only function on the first sneak attack the rogue makes against a particular creature in an encounter.

        \subcf{Confusing Ambush*}
        The rogue makes a Trick power vs. Mental attack against the creature damaged by this ambush attack.
        Success means the struck creature is \dazed for 2 rounds.
        Critical success means it is instead \confused for 2 rounds.
        This is an ambush attack, and only works once per creature.

        \subcf{Distracting Ambush*}
        A creature damaged by this ambush attack automatically fails any Concentration checks it makes that round.
        This is an ambush attack, and only works once per creature.

        \subcf{Distant Precision}
        The rogue can make sneak attacks from up to 100 feet away.

        \subcf{Hamstring*}
        A creature damaged by this ambush attack has its land speed halved for 5 rounds.
        Despite the name, this can be used on creatures who do not have hamstrings.
        This is an ambush attack, and only works once per creature.

        \subcf{Merciful Blows}
        The rogue suffers no penalty to damage when attacking for \glossterm{nonlethal damage}, and can deal your full sneak attack damage when attacking nonlethally.

        %\subcf{Swift Poisoner}
        %The rogue can apply poison to a weapon you are holding as a swift action.

        \subcf{Tricky Maneuver}
        When performing a \glossterm{maneuver} against an \glossterm{overwhelmed} or \unaware creature, the rogue gains a bonus to accuracy equal to the number of sneak attack dice she would roll.
        The benefits of this trick apply even against creatures immune to critical hits.

        \subcf{8th -- Brutal Ambush*}
        The rogue rolls d8s instead of d6s for your sneak attack dice on this ambush attack.
        This is an ambush attack, and only works once per creature.

        \subcf{8th -- Immobilizing Ambush*}
        A creature damaged by this ambush is \immobilized for 2 rounds.
        This is an ambush attack, and only works once per creature.

        \subcf{12th -- Agonizing Ambush*}
        The rogue makes a Trick power vs. Mental attack against the struck creature.
        Success means that the target feels agonizing pain for 2 rounds.
        Whenever it takes physical damage, it takes additional physical damage equal to the rogue's trick power.
        This is an ambush attack, and only works once per creature.

        \subcf{12th -- Assassination}
        To use this ability, the rogue must spend a full round studying a creature within 100 feet of your who has not noticed your and who is not in combat.
        If she make a melee sneak attack against that target within 1 round, your attack deals maximum damage, including your sneak attack damage.
        If the target becomes aware of your presence before you attack, this ability has no benefit.

        \subcf{12th -- Perfect Precision}
        The rogue has no range limit on your sneak attacks.

        \subcf{16th -- Deadly Ambush*}
        The rogue makes a Trick power vs. Fortitude attack against the struck creature.
        Success means the target is \staggered, and dies if it loses all its hit points for 2 rounds.
        Critical success means the target immediately dies.
        This is an ambush attack, and only works once per creature.
        In addition, dying in this way is a \glossterm{Death} effect.

        \subcf{20th -- Dual Ambush}
        The rogue can apply the benefits of two ambush attacks to a single sneak attack.

        \subcf{20th -- Lingering Ambush}
        The effects of the rogue's ambush attacks last ten times longer than normal.
        This has no effect on ambush attacks that have no duration.

        \cf{Rog}[5th]{Lucky Slip}[Mag]
        The rogue gains an additional \glossterm{legend point}.
        Once per round, when you spend a legend point to force a foe to reroll a \glossterm{strike}, she may use this ability to treat the reroll as a 1.

        \cf{Rog}[7th]{Improved Uncanny Dodge}
        The rogue reduces your overwhelm penalties by 2.
        This does not stack with the effects of uncanny dodge.
        If your overwhelm penalty is reduced to 0, you are not considered to be overwhelmed.

        \cf{Rog}[9th]{Lucky Break}[Mag]
        The rogue gains an additional \glossterm{legend point}.
        Once per round, when you spend a legend point to reroll a skill check, she may use this ability to treat the reroll as a 20.

        \cf{Rog}[11th]{Slippery Mind}
        The rogue gains a \plus4 bonus to Mental defense.

        \cf{Rog}[13th]{Lucky Dodge}[Mag]
        The rogue gains an additional \glossterm{legend point}.
        In addition, she can use your lucky slip ability against all attacks, not just strikes.

        \cf{Rog}[15th]{Persistent Sneak Attack}
        The rogue can make two sneak attacks per round, rather than one.

        \cf{Rog}[17th]{Legendary Luck}[Mag]
        The rogue gains an additional \glossterm{legend point}.
        She can always use legend points, even when fighting monsters that normally prevent legend points from being used.

        \cf{Rog}[19th]{Ambush Master}
        The rogue's ambush attacks function on the first attack the rogue makes against a particular creature in a single round, rather than within a single encounter.

\section{Spellwarped}\label{Spellwarped}
    \begin{dtable}
        \lcaption{Spellwarped Progression}
        \begin{dtabularx}{\columnwidth}{>{\ccol}p{\levelcol} >{\lcol}X}
            \tb{Level} & \tb{Special} \\
            \bottomrule
            \nth{1}  & Innate magic, invocation, warp regeneration \\
            \nth{2}  & Spellwarped body                            \\
            \nth{3}  & Magical senses, spellwarped aspect          \\
            \nth{4}  & Invocation                                  \\
            \nth{5}  & Manipulate magic                            \\
            \nth{6}  & Invocation                                  \\
            \nth{7}  & Spellwarped aspect                          \\
            \nth{8}  & Invocation                                  \\
            \nth{9}  & Magic resistance                            \\
            \nth{10} & Invocation                                  \\
            \nth{11} & Spellwarped aspect                          \\
            \nth{12} & Invocation                                  \\
            \nth{13} & Improved manipulate magic                   \\
            \nth{14} & Invocation                                  \\
            \nth{15} & Spellwarped aspect                          \\
            \nth{16} & Invocation                                  \\
            \nth{17} & \tdash                                      \\
            \nth{18} & Invocation                                  \\
            \nth{19} & Spellwarped aspect                          \\
            \nth{20} & Invocation                                  \\
        \end{dtabularx}
    \end{dtable}

    \classbasics{Alignment} Any.

    \classbasics{Class Skills}
    \subparhead{Intelligence} Knowledge (arcana).
    \subparhead{Perception} Awareness, Spellcraft.
    \subparhead{Other} Bluff, Intimidate, Persuasion.

    \classbasics{Special Class Skills}
    A spellwarped gains additional class skills based on your choice of innate magic.
    \subparhead{Alteration} Disguise, Escape Artist, Jump.
    \subparhead{Pyromancy} Acrobatics, Jump, Sprint.
    \subparhead{Telekinesis} Devices, Escape Artist, Sleight of Hand.
    \subparhead{Temporal} Acrobatics, Sleight of Hand, Sprint.

    \subsection{Warp Damage}\label{Warp Damage}
        A spellwarped can use the innate magic within your body to generate powerful magical effects.
        However, doing so is physically taxing.
        Most spellwarped abilities cause the spellwarped to take some amount of \glossterm{warp damage}.
        Warp damage cannot be cured by effects that restore hit points, effectively reducing the spellwarped's maximum hit points.

        A spellwarped cannot voluntarily take warp damage that would exceed half your maximum hit points.
        An hour of rest cures warp damage equal to a character's level.

    \subsection{Base Class Abilities}
        A character with spellwarped as a base class gains the following abilities.

        \classbasics{Skill Points} 5.

        \classbasics{Defenses} \plus4 to one defense, \plus2 to other defenses (see the Innate Magic ability, below).

        \classbasics{Action Points} \plus3.

        \cf{Spl}{Weapon and Armor Proficiency}
        A spellwarped is proficient with simple weapons, any two weapon groups, light and medium armor, and shields.

        \cf{Spl}{Warp Regeneration}[Mag]
        The spellwarped does not need to rest to cure warp damage.
        He automatically heals warp damage equal to your level every hour, regardless of any activity he takes in the meantime.
        If he rests, he also recovers warp damage for resting, doubling the warp damage he heals.

    \subsection{Class Abilities}
        All spellwarped have the following abilities.

        \cf{Spl}{Spellpower}
        The strength of a spellwarped's spells and abilities are determined by your spellpower.
        Your spellpower is equal to your key attribute or your character level, whichever is higher.

        \cf{Spl}{Innate Magic}
        Each spellwarped draws your magical power from a particular kind of magic.
        This is a choice made when the first level of the class is taken, and it cannot thereafter be changed.
        The choices are listed below.
        Your choice of innate magic is not a \glossterm{magical} ability, but the active ability granted by that choice is a magical ability.

        \subcf{Alteration}
        The spellwarped can manipulate the physical forms of creatures.
        Your good defense is Fortitude, your key attribute is Intelligence, and he treats Disguise, Escape Artist, and Jump as class skills.
        An alteration spellwarped may be called an alterer, bodywarper, or shifter.

        As a standard action, he can change minor aspects of your appearance, such as removing a mole or lengthening your beard.
        This can grant you a \plus2 bonus to Disguise checks.
        Major changes are not possible.

        \subcf{Pyromancy}
        The spellwarped can manipulate fire and heat.
        Your good defense is Fortitude, your key attribute is Willpower, and he treats Acrobatics, Jump, and Sprint as class skills.
        A pyromancy spellwarped may be called a pyromancer.

        As a standard action, he can snap your fingers to create a small ember of flame in your hand for 5 minutes.
        This ember casts light as a torch, and can deal 1 point of fire damage with a successful touch attack.
        It can be dismissed as a swift action or extinguished as a move action.

        \subcf{Telekinesis}
        The spellwarped can manipulate objects and creatures with your mind.
        Your good defense is Mental, your key attribute is Willpower, and he treats Devices, Escape Artist, and Sleight of Hand as class skills.
        A telekinesis spellwarped may be called a telekine.

        As a standard action, he can concentrate to move objects within ten feet of you telekinetically.
        He can slowly lift or manipulate one object by up to one foot per round.
        The object can weigh up to five pounds.
        This level of control is insufficient to make skill checks or wield a weapon or shield effectively.

        \subcf{Temporal}
        The spellwarped can manipulate time.
        Your good defense is Reflex, your key attribute is Perception, and he treats Awareness, Sleight of Hand, and Sprint as class skills.
        A temporal spellwarped may be called a temporalist or timewarper.

        The spellwarped always knows exactly what time it is, and can track the passage of time precisely without effort.

        \cf{Spl}{Invocation}
        A spellwarped can invoke your innate magic to generate powerful effects.
        He chooses a single invocation at 1st level from those available based on your choice of innate magic.
        Using an invocation inflicts \glossterm{warp damage} to the spellwarped equal to half the minimum spellwarped level required to learn the invocation (minimum 1).

        At your 4th spellwarped level, and every two spellwarped levels thereafter, you gain an additional invocation.
        Some invocations have minimum spellwarped levels, as indicated in the title of the ability.
        The list of invocations is given at \pcref{Invocations}.

        The spellwarped's accuracy with invocations is equal to your spellpower.

        \cf{Spl}[2nd]{Spellwarped Body}[Mag]
        The spellwarped's body is fundamentally altered by exposure to magic.
        He shows signs of the magic coursing through your body: strangely or inconsistently colored hair, natural skin markings which often resemble runes, and so on.
        Anyone observing the spellwarped can make an Awareness or Spellcraft check with a DR equal to 20 \sub your spellwarped level to recognize that the character is a spellwarped.
        Critical success on the check allows the observer to determine the type of innate magic the spellwarped has.
        In addition, the spellwarped gains an ability based on your innate magic.
        \subcf{Alteration -- Sturdy Body}
        The spellwarped gains a \plus2 bonus to your Fortitude defense.
        This bonus increases by 1 at spellpower 5 and every 5 spellpower thereafter.
        \subcf{Pyromancy -- Energy Resistance}
        The spellwarped gains \glossterm{damage reduction} against cold and fire damage equal to twice your spellpower.
        \subcf{Telekinesis -- Tactile Telekinesis}
        The spellwarped gains a \plus2 bonus to Strength and Dexterity-based checks.
        This bonus increases by 1 at spellpower 5 and every 5 spellpower thereafter.
        \subcf{Temporal -- Accelerated Movement}
        The spellwarped gains a \plus2 bonus to your Reflex defense and a \plus10 foot bonus to your movement speed.
        At spellpower 8, 14, and 20, the defense bonus increases by 1 and the speed bonus increases by 10 feet.

        \cf{Spl}[3rd]{Magical Senses}[Mag]
        The spellwarped learns to recognize the telltale signs of your chosen magic.
        He must concentrate as a standard action to use this ability, and he may do so any number of times per day.
        \subcf{Alteration -- Perceive Alteration}
        The spellwarped can discern the true form of all creatures within 50 feet of you for 1 round, ignoring any effects which magically alter their shapes.
        This also grants you a \plus5 bonus to Awareness checks to see through disguises.
        \subcf{Pyromancy -- Flame of Life}
        The spellwarped can see the life-fire that lies within all living creatures, allowing you to clearly see all living creatures within 50 feet of you for 1 round.
        It also allows the spellwarped to see unusually warm objects, such as fires.
        This is a \glossterm{Detection} ability, and it can penetrate up to 1 foot of stone, 1 inch of common metal, a thin sheet of lead, or 3 feet of wood or dirt.
        \subcf{Telekinesis -- Spatial Awareness}
        The spellwarped can feel the forms of all objects and creatures around you, granting blindsense out to a 50 foot range for 1 round.
        \subcf{Temporal -- Rapid Search}
        The spellwarped can accelerate your mind to immediately search everything within a 10 foot radius of you with the Awareness skill as a standard action.
        Alternately, he may use this ability to read a book ten times as fast as normal.

        \cf{Spl}[3rd]{Spellwarped Aspect}[Mag]
        The spellwarped gains a new ability based on your continued exposure to magical energy.
        Most aspects are specific to particular kinds of innate magic, but some aspects can be taken by any spellwarped.
        These aspects are listed under the General heading.

        At your 7th spellwarped level, and every four spellwarped levels thereafter, the spellwarped gains an additional spellwarped aspect.
        Some aspects require a minimum spellwarped level, as indicated in the title of the ability.
        The full list of spellwarped aspects is given below.

        \parhead{General}

        \subcf{Resilient}
        The spellwarped increases the defense improved by your choice of innate magic by 2.
        This can increase your hit points, if appropriate.
        \subcf{Spellwarped Ritualist} The spellwarped can learn and perform rituals as if he were an arcane caster with a spellpower equal to your spellwarped spellpower.  The maximum level of ritual that he can learn or perform is equal to half your spellwarped level or half your spellwarped key attribute, whichever is lower.
        In addition, you gain an ability based on which type of spellwarped you are.
        \begin{itemize}
            \itemhead{Alteration} If a ritual requires a specific component with a value, the spellwarped can substitute its equivalent value in ritual components instead.
                This cannot be used to replace components without a value or components with special properties that alter the ritual's effect, such as the body for a \spell{resurrection} ritual.
            \itemhead{Pyromancy}
                The spellwarped can use any combustable item as a ritual component.
                It can replace an amount of normal ritual components equal to the value of the item.
                It cannot replace special ritual components.
            \itemhead{Telekinesis}
                The spellwarped can perform rituals from up to 30 feet away from the ritual components.
            \itemhead{Temporal}
                The spellwarped performs rituals twice as quickly.
        \end{itemize}
        \subcf{7th -- Warp Overload}
        The spellwarped can voluntarily take \glossterm{warp damage}, even if that would exceed half your maximum hit points.
        You are only unable to voluntarily take warp damage if the total warp damage would exceed your maximum hit points.
        \subcf{11th -- Persistent Senses}
        The spellwarped can constantly gain the benefit of your magical senses ability.
        He can toggle your enhanced senses on or off as a swift action.
        If the ability does not have a duration, such as the temporal magical senses ability, this aspect has no effect.

        \cf{Spl}[2nd]{Surge of Power}[Mag]
        \subcf{Telekinesis -- Kinetic Deflection}
        The spellwarped reflexively deflects attacks away with your mind.
        You gain a \plus2 bonus to your \glossterm{physical defenses}.
        \subcf{Temporal -- Accelerate Movement}
        The spellwarped accelerates your movement and reactions.
        You gain a \plus2 bonus to your Reflex defense and a \plus10 foot bonus to your movement speed.
        At spellpower 8, 14, and 20, the defense bonus increases by 1 and the speed bonus increases by 10 feet.

        \parhead{Alteration}
        \subcf{Alter Movement}
        The spellwarped gains your choice of the Acrobatics Mastery, Climb Mastery, Jump Mastery, or Swim Mastery feats.
        He may select this aspect multiple times, choosing a different bonus feat each time.
        \subcf{7th -- Damage Reduction}
        The spellwarped gains \glossterm{damage reduction} against physical damage equal to your spellpower.
        Adamantine weapons ignore this damage reduction and negate it for 1 round.
        \subcf{11th -- Alter Size}
        As a standard action, the spellwarped can grow or shrink by one size category, up to a maximum of one size category different from your normal size.
        This effect lasts until he uses this ability again.
        This is a \glossterm{Sizing} effect, and does not stack with other Sizing effects.
        \subcf{11th -- Regeneration}
        At the end of each round, the spellwarped heals hit points equal to your spellpower.

        \parhead{Pyromancy}
        \subcf{Flame Aura}
        The spellwarped emanates an aura of fire.
        At the end of each round, enemies adjacent to you take fire damage equal to your spellpower.
        He can suppress or resume this aura as a \glossterm{swift action}.
        \subcf{Intense Flames}
        The spellwarped can choose to have your spellwarped abilities ignore an amount of fire damage reduction equal to your spellpower.
        \subcf{7th -- Flame Eater}
        When the spellwarped resists fire damage with your spellwarped body ability, he heals hit points equal to the damage resisted.
        \subcf{7th -- Retributive Flames}
        When a creature makes a melee attack against the spellwarped, it takes \spelldamage{fire}[d6].
        Each creature can only take this damage once per round.
        \subcf{11th -- Intense Flame}
        The spellwarped gains a bonus to fire damage with all invocations equal to the number of dice he would roll for that invocation's fire damage.
        This does not affect invocations that do not deal fire damage.

        \parhead{Telekinesis}
        \subcf{Kinetic Deflection}
        The spellwarped may use your Intelligence in place of your Dexterity or Constitution to determine your \glossterm{physical defenses}.
        \subcf{Improved Object Manipulation}
        When the spellwarped uses your object manipulation ability, he can affect objects within 10 feet, with a weight limit of up to two pounds per spellpower.
        He has enough control to make checks with a DR of up to 10.
        \subcf{7th -- Force Barrier}
        The spellwarped gains a \plus1 bonus to \glossterm{physical defenses}.
        \subcf{7th -- Shieldbearer}
        The spellwarped may wield shields, except tower shields, telekinetically.
        The shield floats in your square, granting you its bonus to your physical defenses just as if he were wielding it.
        You do not need a free hand to wield the shield, and suffer no arcane spell failure for its use, though you are still affected by its \glossterm{encumbrance penalty}.
        The shield follows you as he moves.
        If it is forcibly removed from your square, he loses control over it and it falls to the ground.
        \subcf{11th -- Mind Armory}
        The spellwarped may control a number of weapons equal to half your Intelligence with your mind blade ability.
        This does not allow you to make additional attacks per round, but he may attack interchangeably with any weapon he controls.
        Each weapon threatens an area and contributes to overwhelm penalties, just as with your normal mind blade ability.

        \parhead{Temporal}
        \subcf{Accelerate Mind}
        The spellwarped gains a \plus2 bonus to Intelligence-based and Perception-based checks.
        \subcf{Sprint Mastery}
        The spellwarped gains the Sprint Mastery feat (see page \featpref{Sprint Mastery}).
        \subcf{Swift}
        The spellwarped gains the Swift feat (see page \featpref{Swift}).
        \subcf{Uncanny Dodge}
        The spellwarped can react to danger before your senses would normally allow you to do so.
        He reduces your overwhelm penalties by 1.
        If your overwhelm penalty is reduced to 0, you are not considered to be overwhelmed.
        In addition, you are not \unaware when attacked by surprise.
        \subcf{7th -- Improved Uncanny Dodge}
        The spellwarped reduces your overwhelm penalties by 2.
        If your overwhelm penalty is reduced to 0, you are not considered to be overwhelmed.
        \subcf{15th -- Accelerate Attack}
        Whenever the spellwarped makes a \glossterm{standard attack}, he can make an additional \glossterm{strike} at a \minus5 penalty.
        This does not stack with any other effects which grant extra strikes.

        \cf{Spl}[5th]{Manipulate Magic}[Su]
        The spellwarped can channel your innate magic to manipulate magical abilities.
        Using this ability causes the spellwarped to take \glossterm{warp damage} equal to half the \glossterm{power} of the target ability.
        \subcf{Alteration -- Absorption}
        As an immediate action, when the spellwarped makes successfully resists an attack against your Fortitude from a magical ability, he may absorb the magic harmlessly into your body.
        It has no effect on you, even if it would normally have an effect on a failed attack.
        \subcf{Pyromancy -- Fuel the Flame}
        As an immediate action, when you are affected by a magical ability, he may channel its energy into a burst of flame around you.
        Enemies within a \areamed radius of the spellwarped take fire damage equal to your spellpower.
        The spell still has its normal effect on the spellwarped.
        \subcf{Telekinesis -- Mind over Matter}
        As an immediate action, when you are subject to an attack against your Fortitude from a magical ability, he may use your Mental defense instead.
        \subcf{Temporal -- Accelerate Magic}
        As a swift action, the spellwarped can halve the duration of any magical ability affecting you.
        This can end the effect immediately if it has less than one round remaining.
        If this would reduce the duration by more than one day, the duration is instead reduced by one day.

        \cf{Spl}[9th]{Magic Resistance}[Mag]
        The magic within the spellwarped allows you to completely ignore other magic, granting you \glossterm{magic resistance} equal to 10 \add your Constitution or level, whichever is higher.

        \cf{Spl}[13th]{Improved Manipulate Magic}[Mag]
        The spellwarped can use your manipulate magic ability to affect any ally within \rngmed range of you.

    \subsection{Invocations}\label{Invocations}

        All invocations are \glossterm{magical} abilities unless otherwise noted.

        Some invocations mimic the effects of spells.
        The spellwarped's spellpower with these effects is equal to your spellpower with the spellwarped class.

        \subsubsection{Alteration Invocations}
            \subcf{1st -- Body Bludgeon}
            The spellwarped distorts a part of your body and strikes a foe with it.
            He makes a Spellpower vs. Armor defense physical attack against a foe within your \glossterm{reach}.
            This attack scores critical hits like other physical attacks.
            Success means the target takes 1d6 bludgeoning damage per two spellpower \add half your Strength.
            Failure means the target takes half damage.
            \par At your 4th spellwarped level, this damage increases to 1d6 bludgeoning damage per spellpower \add half your Strength.
            \subcf{1st -- Shrink}
            This invocation functions like the \spell{shrink} spell.

            \subcf{4th -- Enlarge}
            This invocation functions like the \spell{enlarge} spell.
            \subcf{4th -- Purge}
            The spellwarped can end any single effect or condition on you which requires a successful attack against your Fortitude.
            If he identify the effects on you, such as with a Spellcraft check, he may freely choose which effect to end.
            If he cannot differentiate the effects, choose the effect ended randomly.

            \subcf{6th -- Amorphous Body}
            The spellwarped transforms your body into an amorphous form for \durshort duration.
            In this form, you gain several benefits.
            You gain a \plus20 bonus to Reflex defense against grapple attacks, is immune to critical hits, and can move through spaces that are no more than two inches in width without \glossterm{squeezing}.
            While moving through spaces smaller than he could normally move through, he moves at half speed.
            \subcf{6th -- Healing Transformation}
            As a standard action, the spellwarped can heal a creature within \rngclose range by transforming it into a healthier version of its normal body.
            The target heals 1d6 points of damage per spellpower.
            This also removes any of the following conditions: blinded, deafened, diseased, exhausted, fatigued, nauseated, sickened, and poisoned.
            \subcf{6th -- Mighty Throw}
            This invocation functions like the \spell{mighty throw} spell.

            \subcf{8th -- Bludgeon the Horde}
            This attack functions like the body bludgeon attack, except that the spellwarped may attack all foes within your reach, as if he were wielding a reach weapon.
            Success deals 1d8 bludgeoning damage per two spellpower \add half your Strength.
            Failure deals half damage.
            \subcf{8th -- Flight}
            As a standard action, the spellwarped can grow wings that last for 5 rounds.
            The wings grant you a fly speed equal to your land speed.
            While \unencumbered, he can fly (see \pcref{Flying}).
            He can only fly for a number of rounds equal to half your spellpower, even if he uses this ability again to extend the duration of the wings.
            After that limit is reached, he must rest for 5 minutes before flying again.

            \subcf{12th -- Reinforced Flight}
            This functions like the flight invocation, except that the spellwarped can fly while encumbered.

        \subsubsection{Pyromancy Invocations}
            Unless otherwise noted, a pyromancer's invocations are \glossterm{Fire} effects, and shed light equivalent to a torch for their duration.

            \subcf{1st -- Burn}
            The spellwarped makes a Spellpower vs. Reflex defense attack against a foe within \rngmed range.
            Success means the target takes \spelldamage{fire}[d6].
            Critical success deals double damage.
            Failure means the target takes half damage.
            \par At your 4th spellwarped level, this damage increases to \spelldamage{fire}.
            \subcf{1st -- Flame Weapon}
            As a swift action, the spellwarped can create a weapon made of flame that lasts for 5 rounds.
            The weapon is sized appropriately for you, and may take the form of any weapon you are proficient with.
            He can attack with the weapon as if it were a normal weapon of its type, except that you use your spellpower to determine your damage in place of your level or Strength, and all damage dealt with the weapon is fire damage.
            All other damage modifiers apply normally.
            \par If the flame weapon leaves your hand, it is extinguished 1 round later.

            \subcf{4th -- Fiery Protection}
            As a standard action, the spellwarped can bestow fire and cold damage reduction equal to twice your spellpower on a creature within 30 feet of you.
            The protection lasts for \durshort duration.
            \subcf{4th -- Flame Shield}
            As a standard action, the spellwarped can wreath a willing creature within \rngclose range in flame for \durshort duration.
            Whenever a creature makes a melee attack against the target, the attacking creature takes \spelldamage{fire}[d6].
            A creature can only be dealt damage by this spell once per round.

            \subcf{6th -- Conflagration}
            As a standard action, the spellwarped can release a powerful explosion of flame.
            He makes a Spellpower vs. Reflex attack against everything within a \areamed radius burst of you.
            Success against a target means it takes \spelldamage{fire}[d8].
            Critical success deals double damage.
            Failure against a target means it takes half damage.
            \subcf{6th -- Fireball}
            This invocation functions like the \spell{fireball} spell.
            \subcf{6th -- Ignite}
            The spellwarped makes a Spellpower vs. Reflex attack against a foe within \rngclose range.
            Success means the target takes \spelldamage{fire}, and is \ignited for 2 rounds.
            Critical success deals double damage.
            Failure means the target takes half damage, but is still ignited.

            \subcf{8th -- Firestride}
            As a move action, the spellwarped can teleport to any active flame of at least Tiny size within \rngmed range.
            When you do so, he immolates himself and disappears into a pile of ash before stepping out from the flame unharmed.
            An ordinary torch is sufficient flame to teleport to, but not a candle.
            \subcf{8th -- Flameheart}
            As a standard action, the spellwarped can become a being of pure fire for \durshort duration.
            In this form, you are immune to physical damage and can pass through openings as small as one inch at no movement penalty.
            However, he cannot attack normally or use any of your items, as they meld into your body.
            He may use any of your invocations normally.
            In addition, as a standard action, he can make a Spellpower vs. Reflex attack to touch a creature.
            Success means the target takes \spelldamage{fire}.
            Critical success deals double damage.
            \subcf{8th -- Flight of the Phoenix}
            As a standard action, the spellwarped can create wings of flame that last for 5 rounds.
            The wings grant you a fly speed equal to your land speed.
            While \unencumbered, he can fly (see \pcref{Flying}).
            \subcf{10th -- Lifeseeking Flame}
            As a standard action, the spellwarped fires an orb of flame at a target within \rnglong range.
            If you do not target a creature, the flame automatically strikes a living creature within range.
            It is able to unerringly strike creatures you cannot see or are not aware of, including invisible or concealed creatures.
            You can direct the orb to avoid specific targets, allowing you to strike a hidden foe among your allies.
            If there is no valid target, the orb dissipates harmlessly.
            \par The struck target takes 1d10 fire damage per two spellpower.

            \subcf{14th -- Immolation}
            This invocation functions like the \spell{immolation} spell.

            \subcf{18th -- Phoenix Revival}
            When the spellwarped takes vital damage, he may ignore the damage as an immediate action, even if the vital damage would be sufficient to kill you.
            If you do, he dissolves into a pile of ash for 2 rounds.
            During this time, he can take no actions.
            If the pile of ash remains intact after 2 rounds, you are restored to your normal body.
            He has no hit points remaining, and warp damage equal to half your maximum hit points, but is healed of all vital damage.
            However, if the pile of ash is dispersed, the spellwarped dies.
            The ash cannot be harmed by fire damage, and if the pile of ash would take at least 50 points of fire damage during a round, the spellwarped returns one round sooner.
            You may take your normal actions in the round after you are restored.
            \par After using this ability, the spellwarped cannot use it again for 1 hour.

            \subcf{20th -- Immolate}
            As a standard action, the spellwarped makes a Spellpower vs. Fortitude against a foe within \rngclose range to consume it in flames from the inside out.
            Success deals \spelldamageemp{fire}.
            Critical success kills the target instantly.
            Failure deals half damage.

        \subsubsection{Telekinesis Invocations}
            \subcf{1st -- Mind Crush}
            As a standard action, the spellwarped can make a Spellpower vs. Fortitude attack against a creature within \rngclose range.
            Success means the target takes \spelldamage{bludgeoning}[d6] and is \sickened for 2 rounds.
            Critical success deals double damage.
            Failure means the target takes half damage, but is still sickened.
            \par At your 4th spellwarped level, this damage increases to \spelldamage{bludgeoning}.
            \subcf{1st -- Mind Blade}
            As a swift action, the spellwarped can telekinetically wield an unattended weapon within \rngclose range for 5 rounds.
            The weapon must be a light or medium weapon appropriate for your size.
            This allows you to attack with the weapon just as if he were holding it in one hand, except that you use your spellpower in place of your Strength or Dexterity.
            In all other respects, this functions as if he were wielding the weapon normally, including contributing to overwhelm penalties.
            The weapon floats in midair and threatens all squares adjacent to it.
            As a move action, he may move the weapon up to 30 feet in any direction, even vertically.
            If the weapon goes outside of \rngclose range, he loses control of it and it falls to the ground.

            \subcf{4th -- Dual Mind Blade}
            This invocation functions like your mind blade invocation, except that the spellwarped may wield two weapons at once.
            They must stay in the same space, and he may make two-weapon fighting attacks with the weapons, just as if he was wielding them with two hands.
            \subcf{4th -- Mighty Mind Blade}
            This invocation functions like the mind blade invocation, except that the spellwarped may also use a heavy weapon appropriate for your size, allowing you to attack with it just as if he were holding it in two hands.

            \subcf{6th -- Distant Manipulation}
            As a standard action, the spellwarped can mentally manipulate objects and creatures at up to \rngclose range for up to 5 rounds.
            This allows you to take any actions which he could normally take with your hands, using your spellpower in place of your Strength or Dexterity, as appropriate.

            \subcf{8th -- Immobilize}
            As a standard action, the spellwarped can make a creature within \rngmed range of you \immobilized for 2 rounds.

            \subcf{10th -- Telekinetic Blast}
            This invocation functions like the \spell{telekinetic blast} spell.

        \subsubsection{Temporal Invocations}
            Unless otherwise noted, all temporal invocations are \glossterm{Temporal} effects.

            \subcf{Decelerate}
            As a standard action, the spellwarped can attempt to slow a creature down.
            He makes a Spellpower vs. Mental attack against a creature within \rngmed range.
            Success means the target is \immobilized for 2 rounds.
            Failure means it moves at half speed for 2 rounds.
            \subcf{Timelock}
            As a standard action, the spellwarped can attempt to lock a creature in time.
            He makes a Spellpower vs. Mental attack against an adjacent creature.
            Success means the target slips out of time for 1 round.
            During that time, it can take no actions, but cannot be harmed, moved, or affected in any way.
            Critical success means it slips selectively out of time for 1 round.
            During that time, it can take no actions, but can be harmed or moved as normal.
            It is not considered \helpless.
            \par A creature affected by this ability is immune to the effect for 5 rounds, even if the attack failed.

            \subcf{4th -- Disjointed Time}
            As a standard action, the spellwarped chaotically disrupts the local flow of time of a creature within \rngclose range.
            The target is \impaired with attacks and checks for 5 rounds.
            \subcf{4th -- Haste}
            This invocation functions like the \spell{haste} spell.

            \subcf{6th -- Flash Step}
            As a move action, the spellwarped can accelerate a willing creature within \rngclose range so much that he can seem to pause time for everyone but the target.
            This allows the target to immediately take a single move action.
            During this move action, the target moves at double speed, cannot be followed or withdrawn from, and may move through squares occupied by creatures or threatened by blocking enemies without penalty.
            The target still suffers the effects of any environmental hazards.
            \subcf{6th -- Slow}
            This invocation functions like the \spell{slow} spell.

            \subcf{8th -- Inhuman Speed}
            As a move action, the spellwarped can accelerate himself to immense speed, allowing you to move up to five times your speed.
            During this time, he cannot be followed or withdrawn from, can move through squares occupied by enemies or threatened by blocking enemies without penalty, and can treat liquids as if they were solid ground.

            \subcf{10th -- Timestream}
            The spellwarped manipulates time in a \arealarge, 10 ft.\ wide line-shaped zone that extends out from you for 5 rounds.
            All creatures and objects that pass through the line are \slowed for 1 round.
            The spellwarped can exclude your allies from the effect.
            The timestream is virtually invisible, requiring a DR 30 Awareness check to notice in a clear environment, though objects passing through the effect can make it more obvious.
            \subcf{10th -- Mass Haste}
            This invocation functions like the \spell{haste} spell with the Mass augment applied.

            \subcf{12th -- Mass Flash Step}
            This invocation functions like the flashstep invocation, except that it affects up to five willing creatures.
            \subcf{12th -- Mass Slow}
            This invocation functions like the \spell{slow} spell with the Mass augment applied.

            \subcf{16th -- Time Reversal}
            As a swift action, the spellwarped can take one warp damage to create a ``time lock.'' The time lock persists for one round.
            As a standard action, he can take eight warp damage to make a Spellpower vs. Mental attack a creature within \rngmed range to reverse time for it.
            Success means the target is restored to its exact condition at the point immediately after the time lock was created.
            The effects of any actions that the creature took in the intervening time are undone, any damage it dealt or took is removed, it is returned to its original location, and the creature is restored in all other ways, just as if the intervening time had never occured.
            If its original location is occupied, the time reversal fails.
            The spellwarped cannot reverse time for himself in this way.
            \par After reversing time in this way, the spellwarped must wait 5 rounds before he can create a time lock or reverse time again.
            \subcf{16th -- Supreme Acceleration}
            As a standard action, the spellwarped can accelerate himself so much that he can take an additional round of actions immediately.
            During this round, all creatures he attacks are treated as \helpless, but he cannot perform a \glossterm{coup de grace} or similar ability; such an act requires more care and precision than is possible with such immense speed.
            After using this ability, he must wait 5 rounds before he can use it again.

            \subcf{18th -- Time Stop}
            As a standard action, the spellwarped can step into an alternate timestream, causing you to speed up so greatly that all other creatures seem frozen.
            He can act for 1d3\plus1 rounds of apparent time.
            During this time, all objects and creatures are frozen in place and are completely invulnerable to the spellwarped, though he may affect them with invocations normally.
            After using this ability, he must wait 5 rounds before he can use it again.

\section{Multiclass Characters}\label{Multiclass Characters}
    A character may add new classes as he or you progress in level, thus becoming a multiclass character.
    The class abilities from a character's different classes combine to determine a multiclass character's overall abilities.
    Multiclassing improves a character's versatility at the expense of focus.

    \subsection{Class And Level Features}
        As a general rule, the abilities of a multiclass character are the sum
        of the abilities of each of the character's classes.

        \parhead{Level}
        ``Character level'' is a character's total number of levels.
        It is used to determine when feats and attribute score boosts are gained, as noted on \tref{Character Advancement}.
        Whenever a creature's ``level'' is specified, without reference to a particular class, the character level is used.

        \par ``Class level'' is a character's level in a particular class.
        For a character whose levels are all in the same class, character level and class level are the same.

        \parhead{Hit Points}
        The normal rules for determining hit points apply to multiclass characters.

        \parhead{Defenses}
        The normal rules for determining defenses apply to multiclass characters.

        \parhead{Skills}
        A multiclass character gains all class skills from all of your classes.
        However, only the character's base class grants skill points.
        If a character has multiple base classes, he must choose which base class grants skill points.
        When a character gains new class skills, he may redistribute your skill points to gain training or mastery in your new class skills.

        \parhead{Class Abilities}
        A multiclass character gets all the class abilities of all your or your classes, but must also suffer the consequences of the special restrictions of all your or your classes.

        \par In some cases, two classes can have virtually identical abilities.
        Use the following guidelines to determine how abilities stack.
        \begin{itemize}
            \item If two identical class abilities are not based on level and are not gained following a specific pattern, they do not stack.
            \item If two identical class abilities are not explicitly based on level, but both classes gain them in a predictable pattern, the levels of the two classes stack for determining when the next improvement to the class ability will be gained.
            \item If two identical class abilities are explicitly based on level, the levels of the two classes stack for determining the power of the ability.
            \item If two identical class abilities say how they stack, those rules trump any other rules.
        \end{itemize}
        These are some examples of how to use these guidelines.
        \begin{itemize}
            \item Both a druid and a ranger gain wild speech.
                A druid/ranger who has wild speech from both classes has the same wild speech ability as a druid or ranger would.
            \item Both a barbarian and a rogue get uncanny dodge and improved uncanny dodge at the same level.
                A barbarian/rogue adds your barbarian and rogue levels together to determine when he acquires improved uncanny dodge.
        \end{itemize}

        \parhead{Weapon and Armor Proficiency}
        Only a character's base class grants weapon and armor proficiencies.
        If a character has multiple base classes, she must choose which base class grants weapon and armor profiencies.
