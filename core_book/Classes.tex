\chapter{Classes}\label{Classes}

Your character's class represents the things your character has chosen to train in.
This choice determines a great deal about your character's abilities.

\section{How Classes Work}
    When you first create a character, you choose a class.
    You gain all abilities granted by the \glossterm{archetypes} of your chosen class at the levels indicated in the archetype's description (see Archetypes, below).
    As you gain levels, you gain more abilities from your class.

    \subsection{Archetypes}\label{Archetypes}
        Each class has three \glossterm{archetypes}.
        An archetype is a collection of thematically related class abilities.
        For examples, barbarians have the Battlerager archetype, which grants abilities related to flying into a rage in combat.
        Normally, a member of a class has all three archetypes associated with that class.
        Characters with the Class Versatility feat can gain achetypes from two different classes (see \featpcref{Class Versatility}).

        \subsubsection{Duplicate Archetypes}\label{Duplicate Archetypes}
            Some archetypes can be gained by multiple classes.
            For example, both clerics and paladins have the Divine Spellcasting archetype.
            You cannot gain two archetypes with the same name, even if you can choose archetypes from multiple classes.

\section{Class Introductions}

    There are nine classes in Rise.
    \begin{itemize}
        \item Barbarians are mighty warriors who can enter a deadly battlerage.
        \item Clerics are divine spellcasters who draw power from their veneration of a deity or ideal.
        \item Druids are nature spellcasters who draw power from their veneration of the natural world.
        \item Fighters are highly disciplined warriors who excel in physical combat of any variety.
        \item Mages are arcane spellcasters who wield the mystic forces of magic to create almost any effect.
        \item Monks are agile masters of ``ki'' who hone their personal abilities to strike down foes and perform supernatural feats.
        \item Paladins are divinely empowered warriors whose devotion to an alignment grants them the ability to discern and smite their foes.
        \item Rangers are skilled hunters who bridge the divide between nature and civilization.
        \item Rogues are exceptionally skillful characters known for their ability to strike at their foe's weak points in combat.
            % \item Spellwarped wield a unique blend of martial skill and narrowly focused magical abilities.
        \item Warlocks are arcane spellcasters who draw their power from a dark pact made with infernal creatures.
    \end{itemize}

    \subsection{Class Description Format}
        Each class is described from the perspective of a member of that class, using ``you'' in the description.

        \parhead{Class Table}
        The class's table describes the special abilities a member of that class gains at each level, assuming they have all of that class's \glossterm{archetypes}.

        \parhead{Alignment}
        Some classes require specific alignments (see \pcref{Alignment}).
        Most classes allow characters of any alignment.

        \parhead{Skills}
        Each class has specific \glossterm{skills} that members of that class are typically good at (see \pcref{Skills}).
        These skills are called \glossterm{class skills}.
        It is easier to become \glossterm{mastered} in class skills than in other skills.
        For details, see \pcref{Skill Training}.

        \parhead{Defenses}
        Each class grants bonuses to specific defenses.

        \parhead{Weapon and Armor Proficiencies}
        These are the types of equipment that members of this class are trained in using.

        \parhead{Other Special Abilities}
        Some classes have abilities shared by all members of the class that are not part of an archetype, such as a cleric's \textit{divine power}.

        \parhead{Archetypes}
        The abilities associated with each of the three archetypes the class has.

\newpage
\section{Barbarian}\label{Barbarian}
    \begin{dtable}
        \lcaption{Barbarian Progression}
        \begin{dtabularx}{\columnwidth}{>{\ccol}p{\levelcol} >{\lcol}X}
            \tb{Level} & \tb{Abilities} \\\bottomrule
            \nth{1}     & Agile defense, primal exertion, rage
            \\ \nth{2}  & Battle-scarred
            \\ \nth{3}  & Athletic prowess
            \\ \nth{4}  & Blood frenzy
            \\ \nth{5}  & Battleforged awareness
            \\ \nth{6}  & Primal exertion
            \\ \nth{7}  & Frenzied assault
            \\ \nth{8}  & Soulscarred
            \\ \nth{9}  & Athletic prowess
            \\ \nth{10} & Mindless rage
            \\ \nth{11} & Brute resilience
            \\ \nth{12} & Primal exertion
            \\ \nth{13} & Greater blood frenzy
            \\ \nth{14} & Greater agile defense
            \\ \nth{15} & Primal prowess
            \\ \nth{16} & Greater mindless rage
            \\ \nth{17} & Greater battleforged awareness
            \\ \nth{18} & Primal exertion
            \\ \nth{19} & Deathless rage
            \\ \nth{20} & Greater soulscarred
        \end{dtabularx}
    \end{dtable}

    \classbasics{Alignment} Any nonlawful.

    \classbasics{Archetypes} Barbarians have the Battlerager, Primal Warrior, and Battleforged Resilience \glossterm{archetypes}.

    \subsection{Basic Class Abilities}
        If you are a barbarian, you gain the following abilities.

        \cf{Bbn}{Defenses}
        You gain the following bonuses to your \glossterm{defenses}: \plus5 Fortitude, \plus4 Reflex, \plus3 Mental.

        \cf{Bbn}{Skills}
        You gain 6 \glossterm{skill points}.
        In addition, you have the following \glossterm{class skills}:
        \begin{itemize}
            \item \subparhead{Strength} Climb, Jump, Swim.
            \item \subparhead{Dexterity} Acrobatics, Ride.
            \item \subparhead{Intelligence} Craft.
            \item \subparhead{Perception} Awareness, Creature Handling, Survival.
            \item \subparhead{Other} Bluff, Intimidate, Persuasion, Profession.
        \end{itemize}

        \cf{Bbn}{Weapon and Armor Proficiencies} 
        You are proficient with simple weapons, any four other weapon groups, light armor, medium armor, and shields.

    \subsection{Battlerager}\label{Rage}
        This archetype grants you a devastating rage, improving your combat prowess.

        \cf{Bbn}{Rage}
        \begin{ability}{Rage}[\glossterm{Attune}, \glossterm{Swift}]
            As a \glossterm{free action}, you can spend an \glossterm{action point} to use this ability.
            If you do, you gain the following benefits and drawbacks:
            \begin{itemize}
                \item You gain a \plus1d bonus to \glossterm{strike damage}.
                \item You gain a \plus2 bonus to \glossterm{threat}.
                \item You are unable to take any action that requires patience or concentration, such as casting spells.
                \item At the end of each round, if you did not attack a creature or object that round, you take \glossterm{subdual damage} equal to your level.
                    This damage ignores damage reduction and any similar abilities.
            \end{itemize}
            When this ability ends, you become \fatigued and unable to use it again until you take a \glossterm{short rest}.
        \end{ability}

        \cf{Bbn}[4]{Blood Frenzy}
        You reduce your penalties for being \glossterm{bloodied} by 2 while raging.

        \cf{Bbn}[7]{Frenzied Assault}
        You gain a \plus1 bonus to \glossterm{accuracy} with \glossterm{physical attacks}.

        \cf{Bbn}[10]{Mindless Rage}
        You are immune to \glossterm{Mind} \glossterm{conditions} while raging.

        \cf{Bbn}[13]{Greater Blood Frenzy} 
        You do not take penalties for being \glossterm{bloodied} while raging.

        \cf{Bbn}[16]{Greater Mindless Rage} 
        You are immune to all hostile \glossterm{Mind} abilities while raging.

        \cf{Bbn}[19]{Deathless Rage} 
        You reduce your \glossterm{vital damage penalties} by an amount equal to your level while raging.

    \subsection{Primal Warrior}
        This archetype grants you abilities to use in combat and improves your physical skills.

        \cf{Bbn}{Primal Exertion}
        You can channel your primal energy into ferocious attacks.
        Choose two \textit{primal exertions} from the list below.
        As a standard action, you can spend an \glossterm{action point} to use a \textit{primal exertion} ability.
        {
            \begin{ability}{Battle Cry}[\glossterm{Mind}, \glossterm{Sustain} (minor)]
                You and all allies within a \arealarge radius burst from you heal hit points equal to your Willpower.

                At 6th level, the healing increases to twice your Willpower.
                At 12th level, the area increases to a \areahuge radius.
                At 18th level, the healing increases to three times your Willpower.
            \end{ability}

            \begin{ability}{Brace for Impact}[\glossterm{Swift}]
                You take half damage from all attacks.
                This halving is applied before damage reduction and similar abilities.
                This ability lasts until the end of the round.

                At 6th level, you can \glossterm{sustain} this ability as a standard action.
                At 12th level, you also gain damage reduction equal to your level.
                At 18th level, the damage reduction increases to twice your level.
            \end{ability}

            \begin{ability}{Certain Strike}
                Make a \glossterm{strike} with a \plus2 bonus to accuracy.

                At 6th level, the accuracy bonus increases to \plus3.
                At 12th level, the accuracy bonus increases to \plus4.
                At 18th level, the accuracy bonus increases to \plus5.
            \end{ability}

            \begin{ability}{Demoralizing Shout}[\glossterm{Mind}]
                Make a Willpower vs. Mental attack against all enemies within a \arealarge radius burst from you.
                \hit Each target is \glossterm{shaken} by you as a \glossterm{condition}.
                \crit Each target is \glossterm{frightened} by you as a \glossterm{condition}.

                At 6th level, if you do not hit any targets with the attack, you regain the action point spent to use this ability.
                At 12th level, the area increases to a \areahuge radius.
                At 18th level, a critical hit makes each target \panicked instead of frightened.
                In addition, a hit makes each target frightened instead of shaken.
            \end{ability}

            \begin{ability}{Ground Pound}
                Make a Strength vs. Reflex attack against all enemies standing on solid ground adjacent to you.
                You can only use this ability while standing on solid ground.
                If you use this ability during the \glossterm{action phase}, you can also make a \glossterm{strike} during the \glossterm{delayed action phase}.
                \hit Each target is knocked \prone.

                At 6th level, the area increases to a \areamed radius burst.
                At 12th level, you gain a \plus1d bonus to damage with the strike.
                At 18th level, the area increases to a \arealarge radius burst.
            \end{ability}

            \begin{ability}{Gut Punch}
                Make a \glossterm{strike} with a bludgeoning weapon.
                The attack is made against Fortitude defense instead of Armor defense.
                If the target takes damage from the strike, it is \sickened as a \glossterm{condition}.

                At 6th level, you gain a \plus1 bonus to accuracy with the strike.
                At 12th level, you gain a \plus1d bonus to damage with the strike.
                At 18th level, the accuracy bonus increases to \plus2.
            \end{ability}

            \begin{ability}{Leaping Strike}
                You make a Jump check to leap and move as normal for the leap, up to a maximum distance equal to your \glossterm{base speed} (see \pcref{Leap}).
                If you use this ability during the \glossterm{action phase}, you can also make a \glossterm{strike} from your new location during the \glossterm{delayed action phase}.

                At 6th level, you gain a \plus1d bonus to damage with the strike per 10 feet of height you travelled downward towards your foe during the leap, up to a maximum of \plus2d.
                At 12th level, the maximum damage bonus increases to \plus4d.
                At 18th level, the maximum damage bonus increases to \plus6d.
            \end{ability}

            \begin{ability}{Potent Maneuver}
                You use a \glossterm{combat maneuver} with a \plus3 bonus to accuracy.

                At 6th level, the accuracy bonus increases to \plus4.
                At 12th level, the accuracy bonus increases to \plus5.
                At 18th level, the accuracy bonus increases to \plus6.
            \end{ability}

            \begin{ability}{Power Attack}
                Make a \glossterm{strike} with a \plus2d bonus to damage.

                At 6th level, if you miss with the strike, you regain the action point spent to use this ability.
                At 12th level, the damage bonus increases to \plus3d.
                At 18th level, the damage bonus increases to \plus4d.
            \end{ability}

            \begin{ability}{Rapid Assault}
                Make a \glossterm{strike} against a creature.
                If you use this ability during the \glossterm{action phase}, you can make another strike during the \glossterm{delayed action phase}.
                You take a \minus2 penalty to accuracy on both strikes.

                At 6th level, if you missed all of your targets, you regain the action point spent to use this ability.
                At 12th level, you gain a \plus1d bonus to damage with both strikes.
                At 18th level, the damage bonus increases to \plus2d.
            \end{ability}

            \begin{ability}{Reaping Charge}
                You can move up to your movement speed in a straight line.
                Choose either the right or left side of the line.
                You can make a melee \glossterm{strike} with a slashing or bludgeoning weapon against each creature and object on that side of the line that you \glossterm{threaten} at any point during your movement, except for the space you start in and the space you end in.
                You take a \minus2d penalty to damage on each strike.

                At 6th level, you do not have to choose a side of the line.
                Instead, you can attack creatures and objects that you threaten at any point during your movement.
                At 12th level, the damage penalty is reduced to \minus1d.
                At 18th level, the damage penalty is removed.
            \end{ability}

            \begin{ability}{Strip the Flesh}
                Make a \glossterm{strike} with a slashing weapon.
                At the end of the current phase, if you hit with the strike and the target is not \glossterm{bloodied}, it takes additional damage equal to the damage you dealt with the strike.

                At 6th level, if you hit with the strike, the target continues taking the same damage at the end of each \glossterm{action phase} until it becomes \glossterm{bloodied}.
                This is a \glossterm{condition}, and can be removed by abilities that remove conditions.
                At 12th level, you gain a \plus1d bonus to damage with the strike.
                At 18th level, the damage bonus increases to \plus2d.
            \end{ability}

            \begin{ability}{Sweeping Strike}
                Make a melee \glossterm{strike} with a slashing or bludgeoning weapon.
                The strike targets each of up to three creatures or objects you \glossterm{threaten}.
                You take a \minus1d penalty to \glossterm{strike damage} with the strike.

                At 6th level, if you missed all of your targets, you regain the action point spent to use this ability.
                At 12th level, the damage penalty is removed.
                At 18th level, you gain a \plus1d bonus to damage with the strike.
            \end{ability}

            \begin{ability}{Thunderous Shout}[\glossterm{Sonic}]
                Make a Constitution vs. Fortitude attack against all creatures and objects in a \areamed cone-shaped burst from you.
                Your \glossterm{power} with this ability is equal to your Constitution.
                \hit Each target takes sonic \glossterm{standard damage} -1d.

                At 6th level, a hit also makes each target \glossterm{deafened} as a \glossterm{condition}.
                At 12th level, the area increases to \arealarge.
                At 18th level, the damage increases by \plus1d.
            \end{ability}

            \begin{ability}{Whirlwind Spin}
                Make a melee \glossterm{strike} with a slashing weapon.
                The strike targets all creatures you \glossterm{threaten}.
                You take a \minus2d penalty to \glossterm{strike damage} with the strike.

                At 6th level, the damage penalty is reduced to \minus1d.
                At 12th level, the damage penalty is removed.
                At 18th level, you gain a \plus1d bonus to damage with the strike.
            \end{ability}
        }

        \cf{Bbn}[3]{Athletic Prowess} You gain two additional skill points.

        \cf*{Bbn}[6]{Primal Exertion}
        You learn an additional \textit{primal exertion}.

        \cf*{Bbn}[9]{Athletic Prowess} You gain two additional skill points.

        \cf*{Bbn}[12]{Primal Exertion}
        You learn an additional \textit{primal exertion}.

        \cf{Bbn}[15]{Primal Prowess}
        You gain a \plus2 bonus to Strength-based and Dexterity-based checks.

        \cf*{Bbn}[18]{Primal Exertion}
        You learn an additional \textit{primal exertion}.

    \subsection{Battleforged Resilience}
        This archetype improves your defenses in combat.

        \cf{Bbn}[1]{Fury of the Storm} You gain a \plus1 bonus to \glossterm{overwhelm resistance}.

        \cf{Bbn}[2]{Battle-Scarred} You gain \glossterm{damage reduction} equal to your level against damage from \glossterm{physical attacks}.

        \cf{Bbn}[5]{Battleforged Fortitude} You gain a \plus2 bonus to Fortitude defense.

        \cf{Bbn}[8]{Soulscarred} The \glossterm{damage reduction} from your \textit{battle-scarred} ability applies against all damage, not just damage from physical attacks.

        \cf{Bbn}[11]{Greater Fury of the Storm}
        The bonus from your \textit{fury of the storm} ability increases to \plus2.

        \cf{Bbn}[14]{Agile Defense}
        You gain a \plus1 bonus to Armor defense.

        \cf{Bbn}[17]{Greater Battle-Scarred}
        The \glossterm{damage reduction} from your \textit{battle-scarred} ability increases to twice your level.

    \subsection{Ex-Barbarians}
        If you become lawful, you cannot use your \textit{rage} ability.
        You retain all of your other class abilities.
        If you stop being lawful, you can use your \textit{rage} ability once more.

\newpage
\section{Cleric}\label{Cleric}
    \begin{dtable}
        \lcaption{Cleric Progression}
        \begin{dtabularx}{\columnwidth}{>{\ccol}p{2em} c c >{\lcol}X}
            \tb{Level} & \tb{Spells} & \tb{Subspells} & \tb{Abilities} \\\bottomrule
            \nth{1}     & 2 & \tdash   & Domain gift, rituals, spell point, spells
            \\ \nth{2}  & 2 & \tdash   & Spell knowledge
            \\ \nth{3}  & 3 & \tdash   & Domain gift
            \\ \nth{4}  & 3 & 2        & \tdash
            \\ \nth{5}  & 3 & 2        & Domain aspect
            \\ \nth{6}  & 3 & 3        & Augments
            \\ \nth{7}  & 3 & 3        & Domain aspect
            \\ \nth{8}  & 4 & 4        & \tdash
            \\ \nth{9}  & 4 & 4        & Cleansing prayer
            \\ \nth{10} & 4 & 5        & Augment
            \\ \nth{11} & 4 & 5        & Domain essence
            \\ \nth{12} & 4 & 6        & \tdash
            \\ \nth{13} & 4 & 6        & Domain essence
            \\ \nth{14} & 4 & 7        & Augment
            \\ \nth{15} & 4 & 7        & Domain mastery
            \\ \nth{16} & 4 & 8        & \tdash
            \\ \nth{17} & 4 & 8        & Domain mastery
            \\ \nth{18} & 4 & 9        & Augment
            \\ \nth{19} & 4 & 9        & Greater cleansing prayer
            \\ \nth{20} & 4 & 10       & Miracle
        \end{dtabularx}
    \end{dtable}

    \classbasics{Alignment} Your alignment must be within one step of your deity's (that is, it may be one step away on either the lawful-chaotic axis or the good-evil axis, but not both).

    \classbasics{Archetypes} Clerics have the Divine Spellcasting, Domain Influence, and Divine Spell Mastery \glossterm{archetypes}.

    \subsection{Basic Class Abilities}
        If you are a cleric, you gain the following abilities.

        \cf{Clr}{Defenses}
        You gain the following bonuses to your \glossterm{defenses}: \plus4 Fortitude, \plus3 Reflex, \plus5 Mental.

        \cf{Clr}{Skills}
        You gain 4 \glossterm{skill points}.
        In addition, have the following \glossterm{class skills}:
        \begin{itemize}
            \item \subparhead{Intelligence} Craft, Deduction, Heal, Knowledge (arcana, local, religion, the planes), Linguistics.
            \item \subparhead{Perception} Awareness, Sense Motive, Spellcraft.
            \item \subparhead{Other} Bluff, Intimidate, Persuasion, Profession.
        \end{itemize}

        \cf{Clr}{Weapon and Armor Proficiencies}
        You are proficient with simple weapons, any two other weapon groups, light and medium armor, and shields.

        \cf{Clr}{Divine Power}
        The \glossterm{power} of many cleric spells and abilities is determined by your \textit{divine power}.
        Your \textit{divine power} is equal to your level or your Willpower, whichever is higher.

        \cf{Clr}{Deity}
        You must worship a specific deity to be a cleric.
        Deities and their associated domains are listed in \trefnp{Deities}.

        \begin{dtable!*}
            \lcaption{Deities}
            \begin{dtabularx}{\textwidth}{X l X}
                \tb{Deity} & \tb{Alignment} & \tb{Domains} \\
                \bottomrule
                Guftas, horse god of justice          & Lawful good     & Good, Law, Strength, Travel         \\
                Lucied, paladin god of justice        & Lawful good     & Destruction, Good, Protection, War  \\
                Simor, fighter god of protection      & Lawful good     & Good, Protection, Strength, War     \\
                %Pabst, dwarf god of drink             & Neutral good & Good, Life, Strength, Wild \\
                Rucks, monk god of pragmatism         & Neutral good    & Good, Law, Protection, Travel       \\
                Vanya, centaur god of nature          & Neutral good    & Good, Strength, Travel, Wild        \\
                Brushtwig, pixie god of creativity    & Chaotic good    & Chaos, Good, Trickery, Wild         \\
                Chavi, god of stories                 & Chaotic good    & Chaos, Knowledge, Trickery          \\
                Ivan Ivanovitch, bear god of strength & Chaotic good    & Chaos, Strength, War, Wild          \\
                Krunch, barbarian god of destruction  & Chaotic good    & Destruction, Good, Strength, War    \\
                Sir Cakes, dwarf god of freedom       & Chaotic good    & Chaos, Good, Strength               \\
                Raphael, monk god of retribution      & Lawful neutral  & Death, Law, Protection, Travel      \\
                Declan, god of fire                   & True neutral    & Destruction, Fire, Knowledge, Magic \\
                Kurai, shaman god of nature           & True neutral    & Air, Earth, Fire, Water             \\
                %Amanita, druid god of decay           & Chaotic neutral & Chaos, Destruction, Life, Wild \\
                %Antimony, elf god of necromancy       & Chaotic neutral & Death, Knowledge, Life, Magic \\
                Clockwork, elf god of time            & Chaotic neutral & Chaos, Magic, Trickery, Travel      \\
                %Lord Khallus, fighter god of pride    & Chaotic neutral & Chaos, Strength, War \\
                %Celeano, sorcerer god of deception    & Chaotic neutral & Chaos, Magic, Protection, Trickery \\
                Murdoc, god of mercenaries            & Chaotic neutral & Destruction, Knowledge, Travel, War \\
                Ribo, halfling god of trickery        & Chaotic neutral & Chaos, Trickery, Water              \\
                Tak, orc god of war                   & Lawful evil     & Law, Strength, Trickery, War        \\
                Theodolus, sorcerer god of ambition   & Neutral evil    & Evil, Knowledge, Magic, Trickery    \\
                Daeghul, demon god of slaughter       & Chaotic evil    & Destruction, Evil, Magic, War       \\
            \end{dtabularx}
        \end{dtable!*}

    \subsection{Divine Spellcasting}
        This archetype grants you the ability to cast divine spells.

        \cf{Clr}{Divine Spells} 
        Your deity grants you the ability to cast divine spells.
        You learn two divine spells from the divine \glossterm{spell list} (see \pcref{Divine Spells}).
        Your \glossterm{spellpower} with divine spells is equal to your \textit{divine power}.

        To cast a spell, you must normally spend an \glossterm{action point}.
        Every spell can also be cast as a cantrip.
        Cantrips are weaker, but do not require action points to cast.

        \cf{Clr}{Rituals} 
        You can perform divine rituals to create unique magical effects (see \pcref{Rituals}).
        You have a ritual book containing one divine ritual of your choice (see \pcref{Divine Rituals}).

        \cf{Clr}[6]{Augments}
        Choose two \glossterm{augments} (see \pcref{Augments}).
        You can apply those augments to divine spells you cast and divine rituals you perform.
        At 10th, 14th level, and 18th level, you learn an additional augment.

        \cf*{Clr}[8]{Spell Knowledge}
        You learn an additional divine spell (see \pcref{Divine Spells}).

    \subsection{Domain Influence}
        This archetype grants you divine influence over two domains of your choice.
        All abilities from this archetype are \glossterm{magical}.

        \cf{Clr}{Domains}
        You choose two domains which represent your personal spiritual inclinations.
        You must choose your domains from among those your deity offers.
        The domains are listed below.

        \begin{itemize}
            \item{Air}
            \item{Chaos}
            \item{Death}
            \item{Destruction}
            \item{Earth}
            \item{Evil}
            \item{Fire}
            \item{Good}
            \item{Knowledge}
            \item{Law}
            \item{Life}
            \item{Magic}
            \item{Protection}
            \item{Strength}
            \item{Travel}
            \item{Trickery}
            \item{War}
            \item{Water}
            \item{Wild}
        \end{itemize}

        \cf{Clr}{Domain Gift}
        Each domain has a corresponding \textit{domain gift}.
        You gain the \textit{domain gift} for one of your domains (see \pcref{Cleric Domain Abilities}).

        \cf*{Clr}[3]{Domain Gift}
        You gain the \textit{domain gift} for another one of your domains.

        \cf{Clr}[5]{Domain Aspect}
        Each domain has a corresponding \textit{domain aspect}.
        You gain the \textit{domain aspect} for one of your domains (see \pcref{Cleric Domain Abilities}).

        \cf*{Clr}[7]{Domain Aspect} 
        You gain the \textit{domain aspect} for another one of your domains.

        \cf{Clr}[9]{Cleansing Prayer}
        When you use the \textit{recover} ability, you heal \plus1d hit points.
        In addition, you can remove an additional condition.

        \cf{Clr}[11]{Domain Essence}
        Each domain has a corresponding \textit{domain essence}.
        You gain the \textit{domain essence} for one of your domains (see \pcref{Cleric Domain Abilities}).

        \cf*{Clr}[13]{Domain Essence} 
        You gain the \textit{domain essence} for another one of your domains.

        \cf{Clr}[15]{Domain Mastery}
        Each domain has a corresponding \textit{domain mastery}.
        You gain the \textit{domain mastery} for one of your domains (see \pcref{Cleric Domain Abilities}).

        \cf*{Clr}[17]{Domain Mastery} 
        You gain the \textit{domain mastery} for another one of your domains.

        \cf{Clr}[19]{Greater Cleansing Prayer} 
        The bonus to healing from your \textit{cleansing prayer} ability increases to \plus2d.
        In addition, when you use the \textit{recover} ability, you can remove any number of \glossterm{conditions}.

        \cf{Clr}[20]{Miracle}
        Once per week, you can request a miracle as a standard action.
        You mentally specify your request, and your deity fulfills that request in the manner it sees fit.
        This can emulate the effects of any spell or ritual, or have any other effect of a similar power level.
        If the deity has a direct interest in your situation, the miracle may be of even greater power.

        If you perform an extraordinary service for your deity, you can gain the ability to request an additional miracle that week.

    \subsection{Divine Spell Mastery}
        This archetype improves the divine spells you cast.
        You must be able cast divine spells to gain the abilities from this archetype.

        \cf{Clr}{Spell Point}
        You gain a spell point.
        A spell point can be spent to cast spells in place of an action point.
        You recover all spent spell points after a \glossterm{short rest}.

        \cf{Clr}[2]{Spell Knowledge}
        You learn an additional divine spell (see \pcref{Divine Spells}).

        \cf{Clr}[4]{Subspells}
        Choose two \glossterm{subspells} for divine spells you know.
        You can use those subspell when you cast those spell (see \pcref{Subspells}).
        At 6th level, and every two levels thereafter, you learn an additional subspell for a divine spell you know.

    \subsection{Cleric Domain Abilities}\label{Cleric Domain Abilities}
        These domain abilities can be granted by the \textit{domain influence} cleric archetype.
        All cleric domain abilities are \glossterm{magical} unless otherwise specified.

        \subsubsection{Air}
            \parhead{Gift} You add the Jump skill to your \glossterm{class skill} list and gain a \plus5 bonus to Jump attacks and checks (see \pcref{Jump}).
            \parhead{Aspect} You gain a \glossterm{glide speed} equal to your \glossterm{base speed} (see \pcref{Gliding}).
            \parhead{Essence} As a standard action, you can spend an \glossterm{action point} to use the \textit{speak with air} ability.
            \begin{ability}{Speak with Air}[\glossterm{Sustain} (minor)]
                You can speak with and command air within \rnglong range.
                You can ask the air simple questions and understand its responses.
                If you command the air to perform a task, it will do so do the best of its ability until this effect ends.
                You cannot compel the air to move faster than 50 mph.

                After you use this ability on a particular area of air, you cannot use it again on that same area for 24 hours.
            \end{ability}
            \parhead{Mastery} You gain a \glossterm{fly speed} with good \glossterm{maneuverability} equal to your \glossterm{base speed} (see \pcref{Flying}).

        \subsubsection{Chaos}
            \parhead{Gift} Whenever you roll a 10 on a \glossterm{check} on your first attempt, you gain a \plus5 bonus to the check.
            \parhead{Aspect} If you roll a 1 on an attack roll, it explodes (see \pcref{Exploding Attacks}).
            This does not affect additional dice rolled if the attack roll explodes.
            \parhead{Essence} As a standard action, you can spend an \glossterm{action point} to use the \textit{twist of fate} ability.
            \begin{ability}{Twist of Fate}
                An improbable event occurs within \rnglong range.
                You can specify in general terms what you want to happen, such as ``Make the bartender leave the bar''.
                You cannot control the exact nature of the event, though it always beneficial for you in some way.
                After using this ability, you cannot use it again for an hour.
            \end{ability}
            \parhead{Mastery} Whenever you make an attack roll, if it misses, you can reroll.
            You must accept the second result.

        \subsubsection{Death}
            \parhead{Gift} Whenever you deal damage to a creature with no hit points remaining, it immediately dies.
            This is a \glossterm{Death} effect.
            % Needs wording to clarify interaction with simultaneous damage
            \parhead{Aspect} Whenever you deal damage to a creature, any of your damage in excess of that creature's hit points is dealt as \glossterm{vital damage}.
            In addition, you are immune to \glossterm{Death} effects.
            \parhead{Essence} As a standard action, you can spend an \glossterm{action point} to use this ability.
            If you do, you make a \textit{divine power} vs. Fortitude attack against a creature within \rngmed range.
            \hit The target takes life \glossterm{standard damage} \plus2d.
            This damage is increased by \plus2d if the target is \glossterm{bloodied}.
            \parhead{Mastery} You constantly radiate an aura of death in a \areahuge radius emanation from you.
            If a living enemy in the area takes \glossterm{vital damage}, it immediately dies.

        \subsubsection{Destruction}
            \parhead{Gift} As a standard action, you can spend an \glossterm{action point} to use the \textit{power attack} ability, as the fighter \textit{martial exertion} (see Martial Exertion, \pref{Ftr:Martial Exertion}).
            This includes all improvements to that ability appropriate for your level.
            \parhead{Aspect} Whenever you deal damage to a target, you first negate an amount of \glossterm{hardness} and \glossterm{damage reduction} equal to your \textit{divine power}.
            Hardness and damage reduction negated in this way does not apply against your attack or against any other attacks during the current phase.
            This effect lasts until the end of the round.
            \parhead{Essence} As a standard action, you can spend an \glossterm{action point} to use the \textit{lay waste} ability.
            \begin{ability}{Lay Waste}
                You make a \textit{divine power} vs. Fortitude attack against all unattended objects in a \arealarge radius.
                You may freely exclude any number of 5-ft\. cubes from the area, as long as the resulting area is still contiguous.
                \hit For each target, if its \glossterm{hardness} is lower than your divine power, it crumbles into a fine power and is irreparably \glossterm{broken}.
            \end{ability}
            \parhead{Mastery} Whenever you deal damage to a creature or object, its damage reduction and hardness (if any) are reduced by an amount equal to your \textit{divine power}.
            In addition, its Fortitude defense is reduced by 2.
            This is a \glossterm{condition}, and lasts until it is removed.
            This effect stacks with itself, but can only be applied to a target once per round.
            % how do you remove a condition from an object???

        \subsubsection{Earth}
            \parhead{Gift} You gain a \plus2 bonus to Fortitude defense.
            \parhead{Aspect} You gain \glossterm{damage reduction} equal to your \textit{divine power} against damage from \glossterm{physical attacks}.
            \parhead{Essence} As a standard action, you can spend an \glossterm{action point} to use the \textit{speak with earth} ability.
            \begin{ability}{Speak with Earth}[\glossterm{Sustain} (minor)]
                You can speak with and command earth within \rnglong range.
                You can ask the earth simple questions and understand its responses.
                If you command the earth to perform a task, it will do so do the best of its ability until this effect ends.
                You cannot compel the earth to move faster than 10 feet per round.

                After you use this ability on a particular area of earth, you cannot use it again on that same area for 24 hours.
            \end{ability}
            % TODO: clarify order
            \parhead{Mastery} As long as you are on solid ground, you gain a \plus1 bonus to all defenses.
            In addition, your damage reduction from this domain's aspect protects you against all damage, rather than only against damage from physical attacks.

        \subsubsection{Evil}
            \parhead{Gift} At the start of each phase, you may choose an adjacent willing creature.
            If you do, that creature takes half of all damage you take that phase (rounded down) instead of you.
            Any abilities it has that would make attacks miss or fail have no effect, but its abilities that allow it to reduce or ignore the effects of attacks work normally.
            You take the remaining half of the damage, and suffer any non-damaging effects of all attacks normally.
            \parhead{Aspect} You can use this domain's domain gift to redirect damage to any willing creature within \rngclose range.
            \parhead{Essence} As a standard action, you can spend an \glossterm{action point} to use the \textit{compel evil} ability.
            \begin{ability}{Compel Evil}[\glossterm{Compulsion}, \glossterm{Mind}]
                Make a \textit{divine power} vs. Mental attack against a creature within \rngmed range.
                Creatures who have strict codes prohibiting them from taking evil actions, such as paladins devoted to Good, are immune to this ability.
                \hit The target takes an evil action as soon as it can.
                You have no control over the act the creature takes, but circumstances can make the target more likely to take an action you desire.
            \end{ability}
            \parhead{Mastery} When you use your Evil domain gift, you can spend an \glossterm{action point}.
            If you do, you can redirect your damage to an unwilling creature that phase.
            You cannot use this ability to redirect damage to the same unwilling creature more than once between \glossterm{short rests}.

        \subsubsection{Fire}
            \parhead{Gift} All of your \glossterm{Fire} spells and abilities do not deal damage to your allies.
            \parhead{Aspect} Whenever you would take fire damage, you heal that many hit points instead.
            % TODO: define those terms?
            This applies before any damage reduction and damage immunity abilities you have.
            \parhead{Essence} As a standard action, you can spend an \glossterm{action point} to use the \textit{speak with fire} ability.
            \begin{ability}{Speak with Fire}
                You can speak with and command fire within \rnglong range.
                You can ask the fire simple questions and understand its responses.
                If you command the fire to perform a task, it will do so do the best of its ability until this effect ends.
                You cannot compel the fire to move farther than 30 feet in a single round.
                Fire that ends the round on non-combustable materials usually goes out, depending on the circumstances.

                This effect lasts as long as you \glossterm{sustain} it as a \glossterm{minor action}.
                After you use this ability on a particular area of fire, you cannot use it again on that same area for 24 hours.
                % TODO: What does an ``area of fire'' mean?
            \end{ability}
            \parhead{Mastery} Whenever you deal fire damage to a creature, that creature becomes \ignited as a \glossterm{condition}.

        \subsubsection{Good}
            \parhead{Gift} At the start of each phase, you may choose an adjacent willing creature.
            If you do, you take half of all damage that creature would take that phase (rounded down) instead of the creature.
            Any abilities you have that would make attacks miss or fail have no effect, but your abilities that allow you to reduce or ignore the effects of attacks work normally.
            The protected creature takes the remaining half of the damage, and suffers any non-damaging effects of attacks normally.
            \parhead{Aspect} You can use this domain's domain gift to redirect damage from any creature within \rngclose range.
            \parhead{Essence} As a standard action, you can spend an \glossterm{action point} to use the \textit{compel good} ability.
            \begin{ability}{Compel Good}[\glossterm{Compulsion}, \glossterm{Mind}]
                Make a \textit{divine power} vs. Mental attack against a creature within \rngmed range.
                Creatures who have strict codes prohibiting them from taking evil actions, such as paladins devoted to Good, are immune to this ability.
                \hit The target takes an good action as soon as it can.
                You have no control over the act the creature takes, but circumstances can make the target more likely to take an action you desire.
            \end{ability}
            \parhead{Mastery} Whenever you use your Good domain gift, you can redirect all damage to you instead of only half.

        \subsubsection{Knowledge}
            \parhead{Gift} You add all Knowledge skills to your cleric \glossterm{class skill} list.
            In addition, you gain two skill points.
            \parhead{Aspect} Your extensive knowledge of all methods of attack and defense grants you a \plus1 bonus to all defenses.
            \parhead{Essence} As a standard action, you can spend an \glossterm{action point} to use the \textit{share knowledge} ability.
            \begin{ability}{Share Knowledge}
                You make a Knowledge check of any kind with a bonus equal to your \textit{divine power}.
                You and all willing creatures within a \arealarge radius learn the results of your check.
                Creatures believe the information gained in this way to be true as if they it had seen it with their own eyes.

                You cannot alter the knowledge you gain with this check in any way, such as by adding or withholding information.
            \end{ability}
            \parhead{Mastery} You gain a \plus1 bonus to accuracy with all attacks.

        \subsubsection{Law}
            \parhead{Gift} You gain a \plus2 bonus to Mental defense.
            % Clarify - does this apply to exploding dice?
            \parhead{Aspect} Whenever you roll a 1 on an \glossterm{attack roll}, it is treated as if you had rolled a 6.
            \parhead{Essence} As a standard action, you can spend an \glossterm{action point} to use the \textit{compel law} ability.
            \begin{ability}{Compel Law}[\glossterm{Compulsion}, \glossterm{Mind}]
                Make a \textit{divine power} vs. Mental attack against all creatures within a \arealarge radius.
                \hit Each target is unable to break the laws that apply in the area, and any attempt to do so simply fails.
                The laws which are applied are those which are most appropriate for the area, regardless of whether you or any other creature know those laws.

                % Sufficiently clear that this isn't part of the hit effect?
                When you use this ability, you also gain the condition.
                If this condition is removed from you, it is also removed from all other affected creatures.
                In areas under ambiguous or nonexistent government, this ability may have unexpected effects, or it may have no effect at all.
            \end{ability}
            \parhead{Mastery} Whenever you roll less than a 5 on an \glossterm{attack roll}, it is treated as if you had rolled a 5.

        \subsubsection{Life}
            \parhead{Gift} You gain additional hit points equal to your \textit{divine power}.
            \parhead{Aspect} All of your \glossterm{magical} spells and abilities can heal \glossterm{vital damage} as easily as they heal hit points.
            \parhead{Essence} As a standard action, you can all of your remaining an \glossterm{action points} (minimum 1) to use the \textit{revivify} ability.
            \begin{ability}{Revivify}
                Choose a dead creature adjacent to you.
                If it was dead for no more than 5 minutes, it is restored to life, as the \ritual{resurrection} ritual.
            \end{ability}
            % Need special text to clarify that it is affected by the Aspect?
            \parhead{Mastery} At the end of each \glossterm{action phase}, you heal hit points equal to your \textit{divine power}.

        \subsubsection{Magic}
            \parhead{Gift} You learn an additional divine spell.
            \parhead{Aspect} You learn two additional subspells for divine spells you know.
            \parhead{Essence} You gain \glossterm{magic resistance} equal to 5 \add your \textit{divine power}.
            If you already have \glossterm{magic resistance}, you can instead increase it by 2.
            \parhead{Mastery} You gain a \plus2 bonus to your \glossterm{magic resistance}.
            If you resist an ability with your magic resistance, you heal hit points equal to your \textit{divine power}.

        \subsubsection{Strength}
            \parhead{Gift} You add Climb, Jump, and Swim to your cleric \glossterm{class skill} list.
            In addition, you gain two skill points.
            \parhead{Aspect} You reduce the \glossterm{encumbrance} of body armor you wear by 1.
            \parhead{Essence} You gain a \plus5 bonus to Strength for the purpose of checks and determining your carrying capacity.
            \parhead{Mastery} You gain a \plus1 bonus to your starting Strength.

        \subsubsection{Travel}
            \parhead{Gift} You add Knowledge (geography), Survival, and Swim to your cleric \glossterm{class skill} list.
            In addition, you gain two skill points.
            \parhead{Aspect} You gain a \plus30 foot bonus to your \glossterm{base speed}, up to a maximum of double your normal speed.
            \parhead{Essence} As a standard action, you can spend an \glossterm{action point} to use the \textit{dimensional jaunt} ability.
            \begin{ability}{Dimensional Jaunt}[\glossterm{Teleportation}]
                You teleport up to 1 mile in any direction.
                You do not need \glossterm{line of sight} or \glossterm{line of effect} to your destination, but you must be able to clearly visualize it.
            \end{ability}
            \parhead{Mastery} Whenever you move, you can teleport the same distance instead.
            This does not change the total distance you can move, but you can teleport in any direction, including vertically.

            You can even attempt to move to locations outside of \glossterm{line of sight} and \glossterm{line of effect}, up to the limit of your remaining movement speed.
            If your intended destination is invalid, the distance you tried to teleport is taken from your remaining movement, but you suffer no other ill effects.

        \subsubsection{Trickery}
            \parhead{Gift} You add Bluff, Disguise, and Stealth to your cleric \glossterm{class skill} list.
            In addition, you gain two skill points.
            \parhead{Aspect} You gain a \plus2 bonus to Bluff, Disguise, and Stealth.
            \parhead{Essence} As a standard action, you can spend an \glossterm{action point} to use the \textit{compel belief} ability.
            \begin{ability}{Compel Belief}[\glossterm{Compulsion}, \glossterm{Mind}, \glossterm{Sustain} (minor)]
                Make a \textit{divine power} vs. Mental attack against a creature within \rngmed range.
                You must also choose a belief that the target has.
                The belief may be a lie that you told it, or even a simple misunderstanding (such as believing a hidden creature is not present in a room).
                If the creature does not already hold the chosen belief, this ability automatically fails.
                \hit The target continues to maintain the chosen belief, regardless of any evidence to the contrary.
                It will interpret any evidence that the falsehood is incorrect to be somehow wrong -- an illusion, a conspiracy to decieve it, or any other reason it can think of to continue believing the falsehood.
                At the end of the effect, the creature can decide whether it believes the falsehood or not, as normal.
            \end{ability}
            \parhead{Mastery} You are undetectable by Divination spells and effects.
            They cannot detect your presence, sounds you make, or any actions you take.

        \subsubsection{War}
            \parhead{Gift} You gain proficiency with heavy armor, tower shields, and an additional weapon group of your choice.
            \parhead{Aspect} You gain a \plus1 bonus to \glossterm{accuracy} with \glossterm{physical attacks}.
            \parhead{Essence} As a \glossterm{minor action}, you can spend an \glossterm{action point} to use the \textit{battlefield magic} ability.
            \begin{ability}{Battlefield Magic}[Swift]
                % TODO: wording
                One spell you cast this phase gains one of the following effects when it resolves:
                \begin{itemize}
                    \item Legion: If the spell would normally affect five or more specific targets, it instead affects five times that many targets.
                    \item Selective: If the spell has an area, it has no effect on your allies in the area.
                    \item Widened: If the spell has an area, the size of the area is doubled.
                \end{itemize}
            \end{ability}
            \parhead{Mastery} You and all allies within a \arealarge radius emanation of you gain a \plus1 bonus to \glossterm{accuracy} with \glossterm{physical attacks}.

        \subsubsection{Water}
            \parhead{Gift} You add Swim to your cleric \glossterm{class skill} list and gain a \plus5 bonus to Swim attacks and checks.
            \parhead{Aspect} You can breathe water as easily as a human breathes air, preventing you from drowning or suffocating underwater.
            You also gain a \glossterm{swim speed} equal to your \glossterm{base speed}.
            \parhead{Essence} As a standard action, you can spend an \glossterm{action point} to use the \textit{speak with water} ability.
            \begin{ability}{Speak with Water}[\glossterm{Sustain} (minor)]
                You can speak with and command water within range.
                You can ask the water simple questions and understand its responses.
                If you command the water to perform a task, it will do so do the best of its ability until this effect ends.
                You cannot compel the water to move faster than 30 feet per round.

                After you use this ability on a particular area of water, you cannot use it again on that same area for 24 hours.
            \end{ability}
            \parhead{Mastery}
            Whenever you move, you can transform yourself into a rushing flow of water with a volume roughly equal to your normal volume until your movement is complete.
            In this form, you may move wherever water could go, you cannot take other actions, such as jumping, attacking, or casting spells.
            You may move through squares occupied by enemies without penalty.
            \par Your speed is halved when moving uphill and doubled when moving downhill.
            Unusually steep inclines may cause greater movement differences while in this form.
            \par If the water is split, you may reform from anywhere the water has reached, to as little as a single ounce of water.
            If not even an ounce of water exists contiguously, your body reforms from all of the largest available sections of water, cut into pieces of appropriate size.
            This usually causes you to die.

        \subsubsection{Wild}
            \parhead{Gift} You add Creature Handling, Knowledge (nature), and Survival to your cleric \glossterm{class skill} list.
            In addition, you gain two skill points.
            % Does this even make sense?
            \parhead{Aspect} When you gain this ability, you choose one wild aspect ability, as if you were a druid of a level equal to your cleric level (see Wild Aspect, \pref{Drd:Wild Aspect}).
            As a standard action, you can spend an \glossterm{action point} to embody that wild aspect for 1 hour.
            % TODO: essence, mastery

        \subsubsection{Ex-Clerics}
            If you grossly violate the code of conduct required by your deity, you lose all spells and magical cleric class abilities.
            You cannot regain those abilities until you atone for your transgressions to your deity.

\newpage
\section{Druid}\label{Druid}
    \begin{dtable}
        \lcaption{Druid Progression}
        \begin{dtabularx}{\columnwidth}{>{\ccol}p{\levelcol} >{\ccol}p{3.5em} c >{\lcol}X}
            \tb{Level} & \tb{Spells} & \tb{Subspells} & \tb{Abilities} \\\bottomrule
            \nth{1}     & 2 & \tdash   & Rituals, spell point, spells, wild speech
            \\ \nth{2}  & 2 & \tdash   & Natural lore, spell knowledge
            \\ \nth{3}  & 2 & \tdash   & Wild aspect
            \\ \nth{4}  & 2 & 2        & \tdash
            \\ \nth{5}  & 3 & 2        & Natural vigor
            \\ \nth{6}  & 3 & 3        & Augment
            \\ \nth{7}  & 3 & 3        & Wild aspect
            \\ \nth{8}  & 4 & 4        & \tdash
            \\ \nth{9}  & 4 & 4        & Natural lore
            \\ \nth{10} & 4 & 5        & Augment
            \\ \nth{11} & 4 & 5        & Wild aspect
            \\ \nth{12} & 4 & 6        & \tdash
            \\ \nth{13} & 4 & 6        & Natural vigor
            \\ \nth{14} & 4 & 7        & Augment
            \\ \nth{15} & 4 & 7        & Wild aspect
            \\ \nth{16} & 4 & 8        & \tdash
            \\ \nth{17} & 4 & 8        & Nature's champion
            \\ \nth{18} & 4 & 9        & Augment
            \\ \nth{19} & 4 & 9        & Wild aspect
            \\ \nth{20} & 4 & 10       & Avatar of nature
        \end{dtabularx}
    \end{dtable}

    \classbasics{Alignment} Neutral good, lawful neutral, neutral, chaotic neutral, or neutral evil.

    \classbasics{Archetypes} Druids have the Natural Influence, Nature Spellcasting, and Nature Spell Mastery \glossterm{archetypes}.

    \subsection{Basic Class Abilities}
        If you are a druid, you gain the following abilities.

        \cf{Drd}{Defenses}
        You gain the following bonuses to your \glossterm{defenses}: \plus5 Fortitude, \plus3 Reflex, \plus4 Mental.

        \cf{Drd}{Skills}
        You gain 6 \glossterm{skill points}.
        In addition, you have the following \glossterm{class skills}:
        \begin{itemize}
            \item \subparhead{Strength} Climb, Jump, Swim.
            \item \subparhead{Dexterity} Acrobatics, Ride, Stealth.
            \item \subparhead{Intelligence} Craft, Deduction, Heal, Knowledge (geography, nature).
            \item \subparhead{Perception} Awareness, Creature Handling, Survival.
            \item \subparhead{Other} Bluff, Intimidate, Persuasion, Profession.
        \end{itemize}

        \cf{Drd}{Weapon and Armor Proficiencies}
        Druids are proficient with simple weapons, any one other weapon group, scimitars, sickles, and slings.
        In addition, druids are proficient with light armor, medium armor, and shields.
        However, a druid cannot use metal armor; see the Metal Abhorrence ability, below.

        \cf{Drd}{Druidic Language}
        You know Druidic, a secret language known only to druids, in addition to your normal languages.
        Druids are forbidden to teach this language to nondruids.
        Druidic has its own alphabet.

        \cf{Drd}{Metal Abhorrence}
        The oaths that you swear as part of your druidic initiation prohibit you from wearing armor made of metal.
        If you wear prohibited armor or carry a prohibited shield, you are unable to cast druid spells or use any of your \glossterm{magical} druid abilities while doing so and for 24 hours thereafter.

        You can avoid this penalty by using armor made of wood altered with the \ritual{ironwood} ritual.
        Such wood is as strong as steel.

        \cf{Drd}{Nature Power}
        The \glossterm{power} of many druid spells and abilities is determined by your \textit{nature power}.
        Your \textit{nature power} is equal to your level or your Perception, whichever is higher.

    \subsection{Natural Influence}
        This archetype grants you influence over the natural world, and the ability to embody aspects the natural world in your own form.

        \cf{Drd}{Wild Speech}[Magical] As a standard action, you can spend an \glossterm{action point} to use this ability.
        If you do, choose an animal within \rnglong range.
        You can speak to and understand the speech of the target animal, and any other animals of the same species.
        This ability does not make the target any more friendly or cooperative than normal.
        Wary and cunning animals are likely to be terse and evasive, while stupid ones tend to make inane comments and are unlikely to say or understand anything of use.

        This effect lasts as long as you \glossterm{attune} to it.

        \cf{Drd}[2]{Natural Lore}
        You gain two extra skill points.

        \cf{Drd}[3]{Wild Aspect}[Magical]
        You gain the ability to embody an aspect of an animal or of nature itself.
        Choose a single wild aspect from the list below.
        Many wild aspects have a minimum level prerequisite, as indicated in the title of the ability.
        That ability is normally active.
        You may suppress or resume the effects of any number of \textit{wild aspects} you have as a \glossterm{minor action}.

        The abilities in the list below describe the effects of the aspect.
        Your appearance also changes to match the aspect's effects, but the nature of this change is not described.
        Different druids change in different ways.
        For example, one druid might gain unusually large eyes when embodying the low-light vision aspect, while another might change their irises into slits, like a cat, when embodying the same aspect.
        You choose how your appearance changes when you gain a wild aspect.
        This change cannot be used to gain an additional substantive benefit beyond the effects given in the description of the aspect.

        Many wild aspects grant natural weapons.
        See \pcref{Natural Weapons}, for details about natural weapons.
        At 7th level, and every four levels thereafter, you gain an additional wild aspect.

        \subcf{Animal Affinity}
        You gain a \plus2 bonus to Creature Handling and Ride checks.
        \subcf{Armaments of the Bear}
        Your mouth and hands transform, granting you bite and claw \glossterm{natural weapons}.
        The bite deals \plus0d damage for a Medium creature, and the claws deal \minus1d damage.
        \subcf{Gore}
        Your head transforms, granting you a gore \glossterm{natural weapon}.
        The weapon deals \plus0d damage for a Medium druid, and has the Forceful weapon tag (see \pcref{Weapon Tags}).
        \subcf{Monkey Climb}
        You gain a \glossterm{climb speed} equal to your \glossterm{base speed}.
        \subcf{Senses}
        You gain low-light vision.
        You treat sources of light as if they had double their normal illumination range.
        If you already have low-light vision, you double its benefit, allowing you to treat sources of light as if they had four times their normal illumination range.
        In addition, you gain \glossterm{darkvision} out to 50 feet, allowing you to see in complete darkness.
        If you already have darkvision, you increase its range by 50 feet.
        \subcf{Woodland Stride}
        You may move through any sort of undergrowth (such as natural thorns, briars, overgrown areas, and similar terrain) at your normal speed and without taking damage or suffering any other impairment.
        The plants bend of their own volition to allow you to pass.
        However, plants magically manipulated to impede motion still affect your movement.

        \subcf{7th -- A Thousand Faces}
        You may use your \textit{nature power} in place of your Disguise skill when making Disguise checks to alter your own appearance.
        \subcf{7th -- Constrict}
        Your body transforms, improving your grappling prowess.
        You gain a \plus2 bonus to accuracy with the \textit{grapple} ability and all grapple actions (see \pcref{Grapple}, and \pcref{Grapple Actions}).
        In addition, you gain a constrict \glossterm{natural weapon}.
        This weapon deals \plus1d damage for a Medium druid, and it has the Grappling weapon tag (see \pcref{Weapon Tags}).
        However, it can only be used against a foe you are grappling with.
        \subcf{7th -- Flight of the Hawk}
        You grow wings, granting your a glide speed equal to your \glossterm{base speed}.
        See \pcref{Gliding}, for more details.
        In addition, your feet transform, granting you a talon \glossterm{natural weapon}.
        The weapon deals \minus1d damage for a Medium creature.
        \subcf{7th -- Lope}
        You gain the ability to move on all four limbs.
        When doing so, you gain a \plus30 foot bonus to your land speed, up to a maximum of double your original speed.
        When not using your hands to move, your ability to use your hands is unchanged.
        Descending to four legs and rising up to stand on two legs again does not take an action.
        \subcf{7th -- Scent}
        You gain the \glossterm{scent} ability.
        \subcf{7th -- Shrink}
        You shrink by one size category (see \pcref{Size in Combat}).
        This is a \glossterm{Sizing} effect.
        \subcf{7th -- Slither}
        You gain a \glossterm{climb speed} equal to your \glossterm{base speed}.
        You do not need to use your hands to climb in this way.
        In addition, you gain a bite \glossterm{natural weapon} that deals \plus0d damage for a Medium druid.
        % \subcf{7th -- Spikes}
        % Whenever a creature adjacent to the druid makes a physical attack against you, the attacking creature takes 1d4 piercing damage \plus1d per two nature power.
        % A creature can only be dealt damage by this effect once per round.

        % Is this how poison actually works?

        % \subcf{11th -- Elemental Retribution}
        % Whenever a creature within \rngmed range of you attacks you, the attacking creature takes 1d6 damage per two nature power of either cold, electricity, or fire damage.
        % You may choose the damage type independently for each attacking creature.
        % A creature can only be dealt damage by this effect once per round.
        \subcf{11th -- Beetle's Carapace} You gain a \plus1 bonus to Armor defense.
        \subcf{11th -- Fluid Motion} You are immune to effects that restrict your mobility, and you suffer no penalties for acting underwater.
        In addition, you gain a \plus10 bonus to defenses against the \textit{grapple} ability (see \pcref{Grapple}).
        \subcf{11th -- Grow}
        You increase in size by one size category (see \pcref{Size in Combat}).
        However, you take a \minus1 penalty to \glossterm{strike damage}, which offsets the damage bonus you gain for increasing your size.
        This is a \glossterm{Sizing} effect.
        \subcf{11th -- Wolfpack}
        You increase your \glossterm{overwhelm value} by 1.
        \subcf{11th -- Venom}
        As a \glossterm{minor action}, you can secrete venom from your natural weapons.
        This is a \glossterm{swift ability}, and it lasts until the end of the round.
        When you deal damage with a \glossterm{strike} using a natural weapon, the target is poisoned.
        This is a \glossterm{condition}, and lasts until removed.
        When you poison a target, and at the end of each \glossterm{action phase} in subsequent rounds, you make a \textit{nature power} vs. Fortitude attack against all creatures you have poisoned in this way.
        The effects of the poison are described below.
        \begin{itemize}
            \item First success: the target is \sickened.
            \item Second success: the target is \nauseated.
            \item Third success: the target is \paralyzed.
            \item Third failure: the the condition is removed, causing any lingering effects from the poison to end.
        \end{itemize}
        % TODO poison stacking rules
        \par In addition, you gains a bite \glossterm{natural weapon} that deals \plus0d damage for a Medium druid.

        \subcf{15th -- Natural Renewal}
        At the end of each \glossterm{action phase}, you heal hit points equal to your \textit{nature power}.
        \subcf{15th -- Stable Foundation} As long as you are on solid ground, you gain a \plus1 bonus to all defenses.
        \subcf{15th -- Wings}
        You grow wings, granting you a \glossterm{fly speed} equal to your \glossterm{base speed}, allowing you to fly (see \pcref{Flying}).
        % TODO: standard flight limitation

        \subcf{19th -- Solar Radiance}
        You continuously radiate bright light out to a 500 foot radius (and shadowy illumination for an additional 500 feet).
        The illumination is so bright that you become hard to look at.
        Any creature making a \glossterm{strike} against you from within the radius of bright light becomes \dazzled as a \glossterm{condition} after the attack.

        \cf{Drd}[5]{Natural Vigor}
        At the end of each \glossterm{action phase}, you heal hit points equal to half your \textit{nature power}.

        \cf*{Drd}[9]{Natural Lore}
        You gain two extra skill points.

        \cf{Drd}[13]{Greater Natural Vigor}
        Your healing from the \textit{natural vigor} ability increases to be equal to your \textit{nature power}.

        \cf{Drd}[17]{Nature's Champion}
        You gain a \plus2 bonus to the Creature Handling, Heal, Knowledge (geography), Knowledge (nature), Ride, and Survival skills.

        \cf{Drd}[20]{Avatar of Nature}[Magical]
        If you die, except if by old age, you may choose to have your body and soul become an instrument of nature's will.
        Your body immediately decomposes or otherwise disappears, and your soul does not travel to an afterlife.
        You has no physical form, and cannot use any of your normal abilities.
        Instead, you have a fly speed of 100 feet, with special maneuverability.
        As a standard action, you can temporarily possess any living plants or animals within a 10 mile radius of the place of your death.

        While possessing a living plant or animal, you can see through its senses and control its actions completely.
        In addition, you may cast spells, and the spells take effect as if the plant or animal had cast them.
        You use the plant or animal's position to determine range, visible targets, and so on.
        You do not require verbal or somatic components to cast your spells in this form.
        % TODO: are there any spells or rituals that have material components or focus objects?
        % but are unable to cast spells or perform rituals that require material components or focus objects.

        While not possessing a plant or animal, you can rest, or you can focus on reincarnating your physical form.
        Creating a new body in this way takes 12 consecutive hours of concentration.
        At the end of that time, you are reincarnated in a new body in your location, as the effect of the \ritual{reincarnation} ritual, except that you can choose your race from among the races listed (not including the ``Other'' race).

        While you are an avatar of nature, you do not age and you cannot die of old age.
        You can continue to exist in this form indefinitely.

    \subsection{Nature Spellcasting}
        This archetype grants you the ability to cast nature spells.

        \cf{Drd}{Nature Spells} 
        Your worship of nature grants you the ability to cast nature spells.
        You learn two nature spells from the nature \glossterm{spell list} (see \pcref{Nature Spells}).
        Your \glossterm{spellpower} with nature spells is equal to your \textit{nature power}.

        To cast a spell, you must normally spend an \glossterm{action point}.
        Every spell can also be cast as a cantrip.
        Cantrips are weaker, but do not require action points to cast.

        \cf{Drd}{Rituals} 
        You can perform nature rituals to create unique magical effects (see \pcref{Rituals}).
        You have a ritual book containing one nature ritual of your choice (see \pcref{Nature Rituals}).

        \cf{Drd}[6]{Augments}
        Choose two \glossterm{augments} (see \pcref{Augments}).
        You can apply those augments to nature spells you cast and nature rituals you perform.
        At 10th, 14th level, and 18th level, you learn an additional augment.

        \cf*{Drd}[8]{Spell Knowledge}
        You learn an additional nature spell (see \pcref{Nature Spells}).

    \subsection{Nature Spell Mastery}
        This archetype improves the nature spells you cast.
        You must be able to cast nature spells to gain the abilities from this archetype.

        \cf{Drd}{Spell Point}
        You gain a spell point.
        A spell point can be spent to cast spells in place of an action point.
        You recover all spent spell points after a \glossterm{short rest}.

        \cf{Drd}[2]{Spell Knowledge} 
        You learn an additional nature spell (see \pcref{Nature Spells}).

        \cf{Drd}[4]{Subspells}
        Choose two \glossterm{subspells} for nature spells you know.
        You can use those subspells when you cast those spells (see \pcref{Subspells}).
        At 6th level, and every two levels thereafter, you learn an additional subspell for a nature spell you know.

        % \subcf{15th -- Air Mantle}
        % You are surrounded by a mantle of air.
        % Thrown and projectile weapons have a 50\% chance to miss your while this effect is active.
        % Unusually large weapons, such as a giant's boulders, may suffer a decreased miss chance as appropriate to their size.
        % \subcf{15th -- Aqueous Step}
        % Wherever the druid moves, you leave a path of animated water that can grab creatures.
        % Whenever a creature crosses the path, the druid makes a Reflex attack to trip the creature, causing it to fall prone and waste the rest of its movement.
        % Your accuracy is equal to your druid level \add your Constitution.
        % \subcf{15th -- Flaming Step}
        % Wherever the druid moves, you leave a path of burning flame behind your that lasts for 1 round.
        % Whenever a creature crosses the path, the druid makes a Reflex attack to deal damage to the creature.
        % The attack deals 1d8 points of fire damage per two druid levels.
        % Your accuracy is equal to your druid level \add your Constitution.
        % A failed attack deals half damage.
        % \subcf{15th -- Lifegiving Step}
        % Wherever the druid moves, you leave a path of small, living plants that entangle foes for 1 round.
        % Whenever a creature crosses the path, the druid makes a Reflex attack to entangle the creature, causing it to waste the rest of its movement.
        % Your accuracy is equal to your druid level \add your Constitution.
        % The plants appear on any surface, and will continue to grow if they can survive, though they may die quickly if they appear on inhospitable terrain.
        % \subcf{17th -- Flaming Soul}
        % You gain the fire subtype, making your immune to fire but giving your a 50\% vulnerability to cold damage.
        % In addition, whenever you deal fire damage to a creature, the creature is \ignited for 5 rounds.
        % \subcf{17th -- Sunblessed Rejuvenation}
        % You gain fast healing equal to your druid level as long as you remain in sunlight or touches a plant of your size or larger.
        % \subcf{17th -- Sunscour}
        % This aspect functions like the heart of the sun natural aspect, except that it also suppresses shadow effects and the visual components of illusions within the area of bright light.

        % \subcf{17th -- Water's Flow}
        % As a minor action, the druid can transform herself into a rushing flow of water with a volume roughly equal to your normal volume until the end of your turn.
        % In this form, she may move wherever water could go, but she cannot take other actions, such as jumping, attacking, or casting spells.
        % Your speed is halved when moving uphill and doubled when moving downhill.
        % She may move through squares occupied by enemies without penalty.
        % She may return to your normal form as a free action.
        % \par If the water is split, she may reform from anywhere the water has reached, to as little as a single ounce of water.
        % If not even an ounce of water exists contiguously, your body reforms from the largest available parts of water, cut into pieces of appropriate size.
        % This usually causes the druid to die.

    \subsection{Ex-Druids}
        A druid who ceases to revere nature, changes to a prohibited alignment, or teaches the Druidic language to a nondruid loses all spells and magical druid class abilities.
        She cannot thereafter gain levels as a druid until you atone for your transgressions.

        % \subsection{Variant Druids}

        %     \subsubsection{Blighter}

        %         Blighters draw power from nature, as do other druids. However, while other druids revere nature and draw power from it gently, blighters steal power from nature forcefully. Wherever a blighter goes, destruction and death surely follows.

        %         \altcf{Blight} Instead of meditating to regain spell slots, a blighter draws power from your environment forcefully.
        %         This affects a \areahuge radius zone centered on your, and the process takes 1 minute of concentration.
        %         At the end of every round, every living thing in the area other than the blighter takes damage equal to your nature power.
        %         All inanimate plants of Huge size or smaller immediately wither and die.
        %         The earth becomes cracked and infertile, and any nutrients from the soil are destroyed.
        %         This ability has no effect on artificial environments or materials, such as metal or worked stone.
        %         At the end of the minute, the blighter regains your spent nature spell slots.

        %         A blighter can only blight your surroundings in this way once per hour.
        %         If your surroundings are already blighted or are not natural terrain, she cannot use this ability to regain your spells.
        %         Instead, she must meditate for 8 hours to slowly draw power from your surroundings, as a normal druid.

        %         \altcf{Spells} As normal, except that a blighter adds all Vivimancy arcane spells to your spell list.

        %         \altcf[2]{Wild Speech} As normal, except that a blighter gains a \plus5 bonus to Intimidate against your wild speech targets, and a \minus5 penalty to Persuasion.

        %         \altcf[10]{Blightcasting}

        %         \altcf[20]{Improved Blightcasting}

        % \subsubsection{Rotbringer}

        %     While most druids seek to emulate and interact with animals, rotbringers focus on the power of fungi, decay, and regeneration.

        %     \altcf{Invoke Rot} Instead of meditating to regain spell slots, a rotbringer accelerates the natural forces of decomposition and decay on your environment.
        %     This affects a \areahuge radius zone centered on your, and the process takes 1 minute of concentration.
        %     All organic objects of Huge size or smaller, such as plants and corpses, decompose.
        %     This decomposition kills inanimate, living plants.
        %     All organic objects, regardless of size, are covered with various fungi.
        %     This ability has no effect on artificial environments or materials, such as metal or worked stone.
        %     At the end of the minute, the rotbringer regains your spent nature spell slots.

        %     If the rotbringer decomposes a Huge object with this ability, or a combination of smaller objects equivalent in size to a Huge object, you gain an bonus nature spell slot of your highest available spell level.
        %     This extra spell slot lasts until it is used, or until you regain your spell slots again.

        %     A rotbringer can only invoke rot on your surroundings in this way once per hour.
        %     If your surroundings are already decomposed or are not natural terrain, she cannot use this ability to regain your spells.
        %     Instead, she must meditate for 8 hours to slowly draw power from your surroundings, as a normal druid.

        %     \altcf[2]{Wild Speech} The rotbringer gains the ability to speak with plants at 2nd level.
        %     You gain the ability to speak with animals at 6th level, instead of at 2nd level.

        %     \altcf[3rd]{Wild Aspect} The rotbringer does not gain this ability.

        %     \altcf[3rd]{Rot Spell} The druid learns an additional spell slot and spell known.
        %     The spell must be taken from the following list of spells.
        %     The spell's level cannot exceed half your druid level.
        %     If she already knows a spell from the list at every spell level you ha access to, she may instead learn any nature spell (see \pcref{Nature Spells}).

        %     At 5th level, and every odd level, the druid may learn a new spell.

        %     \begin{dtable}
        %         \begin{dtabularx}{\columnwidth}{l X}
        %             \tb{Spell level} & \tb{Rotbringer Spells} \\
        %             1st & \spell{excrete slime}, \spell{lesser regeneration} \\
        %             2nd & \spell{fungal growth} \\
        %             3rd & \spell{rotburst} \\
        %             4th & \spell{poison} \\
        %             6th & \spell{regeneration} \\
        %             7th & \spell{greater rotburst} \\
        %         \end{dtabularx}
        %     \end{dtable}

        %     \altcf[7]{Fungal Armor} The rotbringer becomes covered in fungus that protects your from attacks. You gain a \plus1 bonus to Armor and Fortitude defense.

        %     This bonus increases by 1 at your 7th druid level, and every 4 druid levels thereafter.

\newpage
\section{Fighter}\label{Fighter}
    \begin{dtable}
        \lcaption{Fighter Progression}
        \begin{dtabularx}{\columnwidth}{>{\ccol}p{\levelcol} >{\lcol}X}
            \tb{Level} & \tb{Abilities} \\\bottomrule
            \nth{1}     & Armor expertise, disciplined threat, martial exertion
            \\ \nth{2}  & Weapon focus
            \\ \nth{3}  & Martial lore
            \\ \nth{4}  & Discipline
            \\ \nth{5}  & Greater armor expertise
            \\ \nth{6}  & Martial exertion
            \\ \nth{7}  & Disciplined defense
            \\ \nth{8}  & Weapon master
            \\ \nth{9}  & Martial lore
            \\ \nth{10} & Greater disciplined threat
            \\ \nth{11} & Supreme armor expertise
            \\ \nth{12} & Martial exertion
            \\ \nth{13} & Greater discipline
            \\ \nth{14} & Greater weapon focus
            \\ \nth{15} & Exertion expertise
            \\ \nth{16} & Greater disciplined defense
            \\ \nth{17} & Armored juggernaut
            \\ \nth{18} & Martial exertion
            \\ \nth{19} & Legendary discipline
            \\ \nth{20} & 
        \end{dtabularx}
    \end{dtable}

    \classbasics{Alignment} Any.

    \classbasics{Archetypes} Fighters have the Martial Mastery, Equipment Training, and Combat Discipline \glossterm{archetypes}.

    \subsection{Basic Class Abilities}
        If you are a fighter, you gain the following abilities.

        \cf{Ftr}{Defenses}
        You gain the following bonuses to your \glossterm{defenses}: \plus5 Fortitude, \plus3 Reflex, \plus4 Mental.

        \cf{Ftr}{Skills}
        You gain 6 \glossterm{skill points}.
        In addition, you have the following \glossterm{class skills}:
        \begin{itemize}
            \item \subparhead{Strength} Climb, Jump, Swim.
            \item \subparhead{Dexterity} Acrobatics, Escape Artist, Ride.
            \item \subparhead{Intelligence} Craft.
            \item \subparhead{Perception} Awareness.
            \item \subparhead{Other} Bluff, Intimidate, Persuasion, Profession.
        \end{itemize}

        \cf{Ftr}{Weapon and Armor Proficiencies}
        You are proficient with simple weapons, any four other weapon groups, all armor (heavy, medium, and light), and shields.

    \subsection{Martial Mastery}
        This archetype grants you abilities to use in combat and improves your resistance to combat maneuvers.

        \cf{Ftr}{Martial Exertion}
        You can channel your martial prowess into devastating attacks.
        Choose two \textit{martial exertions} from the list below.
        As a standard action, you can spend an \glossterm{action point} to use a \textit{martial exertion} ability.
        {
            \begin{ability}{Brace for Impact}[\glossterm{Swift}]
                You take half damage from all attacks.
                This halving is applied before damage reduction and similar abilities.
                This ability lasts until the end of the round.

                At 6th level, you can \glossterm{sustain} this ability as a standard action.
                At 12th level, you also gain damage reduction equal to your level.
                At 18th level, the damage reduction increases to twice your level.
            \end{ability}

            \begin{ability}{Certain Strike}
                Make a \glossterm{strike} with a \plus2 bonus to accuracy.

                At 6th level, the accuracy bonus increases to \plus3.
                At 12th level, the accuracy bonus increases to \plus4.
                At 18th level, the accuracy bonus increases to \plus5.
            \end{ability}

            \begin{ability}{Challenge}[\glossterm{Attune}]
                Make a \glossterm{strike}.
                % TODO: figure out how to deal with ``so this one time I hit a guy'' problem
                If the strike hits, you gain a \plus4 bonus to \glossterm{threat} against all creatures aware of the attack.
                If it misses, this bonus is decreased to \plus2.

                At 6th level, you gain a \plus1 bonus to accuracy with the strike.
                At 12th level, the accuracy bonus increases to \plus2.
                At 18th level, the accuracy bonus increases to \plus3.
            \end{ability}

            \begin{ability}{Counterattack}
                Make a \glossterm{strike}.
                If the target attacked you in the same phase, you gain a \plus2 bonus to accuracy and a \plus2d bonus to damage.

                At 6th level, if the target did not attack you in the same phase, you regain the action point spent to use this ability.
                At 12th level, the accuracy bonus increases to \plus3 and the damage bonus increases to \plus3d.
                At 18th level, the accuracy bonus increases to \plus4 and the damage bonus increases to \plus4d.
            \end{ability}

            \begin{ability}{Daunting Blow}[\glossterm{Mind}]
                Make a \glossterm{strike}.
                In addition to the strike's normal effects, you also compare the attack result against the target's Mental defense.
                \hit As a \glossterm{condition}, the target suffers a \minus3 penalty to defenses against your attacks.
                \crit As above, except that the penalty is increased to \minus6.

                At 6th level, you gain a \plus1 bonus to accuracy on the strike.
                At 12th level, the accuracy bonus increases to \plus2.
                At 18th level, the accuracy bonus increases to \plus3.
            \end{ability}

            \begin{ability}{Gut Punch}
                Make a \glossterm{strike} with a bludgeoning weapon.
                The attack is made against Fortitude defense instead of Armor defense.
                If the target takes damage from the strike, it is \sickened as a \glossterm{condition}.

                At 6th level, you gain a \plus1 bonus to accuracy with the strike.
                At 12th level, you gain a \plus1d bonus to damage with the strike.
                At 18th level, the accuracy bonus increases to \plus2.
            \end{ability}

            \begin{ability}{Penetrating Strike}
                Make a \glossterm{strike} with a piercing weapon.
                The attack is made against the target's Reflex defense instead of its Armor defense.

                At 6th level, if you miss with the strike, you regain the action point spent to use this ability.
                At 12th level, you gain a \plus1d bonus to damage with the strike.
                At 18th level, the damage bonus increases to \plus2d.
            \end{ability}

            \begin{ability}{Potent Maneuver}
                You use a \glossterm{combat maneuver} with a \plus3 bonus to accuracy.

                At 6th level, the accuracy bonus increases to \plus4.
                At 12th level, the accuracy bonus increases to \plus5.
                At 18th level, the accuracy bonus increases to \plus6.
            \end{ability}

            \begin{ability}{Power Attack}
                Make a \glossterm{strike} with a \plus2d bonus to damage.

                At 6th level, if you miss with the strike, you regain the action point spent to use this ability.
                At 12th level, the damage bonus increases to \plus3d.
                At 18th level, the damage bonus increases to \plus4d.
            \end{ability}

            \begin{ability}{Rally the Troops}
                You and all allies within a \areamed radius can each remove one \glossterm{condition}.

                At 6th level, the area increases to \arealarge.
                At 12th level, each target can instead remove two conditions.
                At 18th level, each target can instead remove any number of conditions.
            \end{ability}

            \begin{ability}{Rapid Assault}
                Make a \glossterm{strike} against a creature.
                If you use this ability during the \glossterm{action phase}, you can make another strike during the \glossterm{delayed action phase}.
                You take a \minus2 penalty to accuracy on both strikes.

                At 6th level, if you missed all of your targets, you regain the action point spent to use this ability.
                At 12th level, you gain a \plus1d bonus to damage with both strikes.
                At 18th level, the damage bonus increases to \plus2d.
            \end{ability}

            \begin{ability}{Reaping Charge}
                You can move up to your movement speed in a straight line.
                Choose either the right or left side of the line.
                You can make a melee \glossterm{strike} with a slashing or bludgeoning weapon against each creature and object on that side of the line that you \glossterm{threaten} at any point during your movement, except for the space you start in and the space you end in.
                You take a \minus2d penalty to damage on each strike.

                At 6th level, you do not have to choose a side of the line.
                Instead, you can attack creatures and objects that you threaten at any point during your movement.
                At 12th level, the damage penalty is reduced to \minus1d.
                At 18th level, the damage penalty is removed.
            \end{ability}

            \begin{ability}{Strip the Flesh}
                Make a \glossterm{strike} with a slashing weapon.
                At the end of the current phase, if you hit with the strike and the target is not \glossterm{bloodied}, it takes additional damage equal to the damage you dealt with the strike.

                At 6th level, if you hit with the strike, the target continues taking the same damage at the end of each \glossterm{action phase} in subsequent rounds until it becomes \glossterm{bloodied}.
                This is a \glossterm{condition}, and can be removed by abilities that remove conditions.
                At 12th level, you gain a \plus1d bonus to damage with the strike.
                At 18th level, the damage bonus increases to \plus2d.
            \end{ability}

            \begin{ability}{Sweeping Strike}
                Make a melee \glossterm{strike} with a slashing or bludgeoning weapon.
                The strike targets each of up to three creatures or objects you \glossterm{threaten}.
                You take a \minus1d penalty to \glossterm{strike damage} with the strike.

                At 6th level, if you missed all of your targets, you regain the action point spent to use this ability.
                At 12th level, the damage penalty is removed.
                At 18th level, you gain a \plus1d bonus to damage with the strike.
            \end{ability}

            \begin{ability}{Whirlwind Spin}
                Make a melee \glossterm{strike} with a slashing weapon.
                The strike targets all creatures you \glossterm{threaten}.
                You take a \minus2d penalty to \glossterm{strike damage} with the strike.

                At 6th level, the damage penalty is reduced to \minus1d.
                At 12th level, the damage penalty is removed.
                At 18th level, you gain a \plus1d bonus to damage with the strike.
            \end{ability}
        }

        \cf{Ftr}[3]{Martial Lore} You gain two additional skill points.

        \cf*{Ftr}[6]{Martial Exertion}
        You learn an additional \textit{martial exertion}.

        \cf*{Ftr}[9]{Martial Lore} You gain two additional skill points.

        \cf*{Ftr}[12]{Martial Exertion} 
        You learn an additional \textit{martial exertion}.

        \cf{Ftr}[15]{Exertion Expertise} You gain a \plus1d bonus to \glossterm{strike damage} when using any of your \textit{martial exertion} abilities.

        \cf*{Ftr}[18]{Martial Exertion}
        You learn an additional \textit{martial exertion}.

    \subsection{Equipment Training}
        This archetype improves your combat prowess with weapons and armor.

        \cf{Ftr}{Armor Expertise}
        You gain a \plus1 bonus to Armor defense while wearing body armor.
        In addition, you reduce the \glossterm{encumbrance} of body armor you wear by 1.

        \cf{Ftr}[2]{Weapon Focus} By spending an hour training with a weapon, you can use the \textit{weapon focus} ability.
        \begin{ability}{Weapon Focus}
            Choose a \glossterm{weapon group} that the weapon you trained with belongs to.
            You gain proficiency with that weapon group.
            In addition, you gain a \plus1 bonus to \glossterm{accuracy} with attacks using weapons from that weapon group.
            This ability does not grant you proficiency with \glossterm{exotic weapons} from that group.

            This ability's effect lasts until you use this ability again.
        \end{ability}

        \cf{Ftr}[5]{Greater Armor Expertise}
        The \glossterm{encumbrance} reduction from your \textit{armor expertise} ability increases to 2.
        In addition, you treat body armor were one encumbrance category lighter than normal whenever doing so would be beneficial for you.

        \cf{Ftr}[8]{Weapon Master} 
        You gain a \plus1d bonus to \glossterm{strike damage}.

        \cf{Ftr}[11]{Supreme Armor Expertise}
        The \glossterm{encumbrance} reduction from your \textit{armor expertise} ability increases to 3.
        In addition, you treat body armor as if it were an additional encumbrance category lighter than normal whenever doing so would be beneficical for you.

        \cf{Ftr}[14]{Greater Weapon Focus} You can use your \textit{weapon focus} ability as a \glossterm{minor action}.

        \cf{Ftr}[17]{Armored Juggernaut}
        As long as you are wearing body armor, you gain a \plus1 bonus to all defenses.

    \subsection{Combat Discipline}
        This archetype allows you to control your \glossterm{threat} and improves your defenses, especially against special abilities.

        \cf{Ftr}{Disciplined Threat}
        You know how to control your actions to appear more or less threatening, as you choose.
        As a \glossterm{minor action}, you can choose to enter a threatening stance, enter a nonthreatening stance, or act normally.
        If you enter a threatening stance, you gain a \plus2 bonus to your \glossterm{threat}.
        If you enter a nonthreatening stance, you take a \minus2 penalty to your \glossterm{threat}.

        \cf{Ftr}[4]{Discipline} As a \glossterm{minor action}, you can spend an \glossterm{action point} to use the \textit{discipline} ability.
        \begin{ability}{Discipline}[Swift]
            Remove one \glossterm{condition} affecting you.
            Because this ability has the \glossterm{Swift} tag, the penalties from the condition do not affect your attacks or defenses during the current phase.
        \end{ability}

        \cf{Ftr}[7]{Disciplined Defense}
        You gain a \plus1 bonus to Armor defense.

        \cf{Ftr}[10]{Greater Disciplined Threat}
        You gain more control over how threatening you appear.
        When you use your \textit{disciplined threat} ability, you can change your threat by any amount, up to a maximum bonus of \plus4 and a maximum penalty of \minus4.

        \cf{Ftr}[13]{Greater Discipline}
        You can remove an additional condition with your \textit{discipline} ability.

        \cf{Ftr}[16]{Greater Disciplined Defense}
        The defense bonus from your \textit{disciplined defense} ability increases to \plus2.

        \cf{Ftr}[19]{Legendary Discipline}
        You can use your \textit{discipline} ability without spending an \glossterm{action point}.

\newpage
\section{Mage}\label{Mage}
    \begin{dtable}
        \lcaption{Mage Progression}
        \begin{dtabularx}{\columnwidth}{>{\ccol}p{2em} c c >{\lcol}X}
            \tb{Level} & \tb{Spells} & \tb{Subspells} & \tb{Abilities} \\\bottomrule
            \nth{1}     & 2 & \tdash   & Mage armor, rituals, spell point, spells
            \\ \nth{2}  & 3 & \tdash   & Spell knowledge, spell point
            \\ \nth{3}  & 4 & \tdash   & Arcane insight, spell knowledge
            \\ \nth{4}  & 4 & 2        & \tdash
            \\ \nth{5}  & 4 & 2        & Lesser essence lore
            \\ \nth{6}  & 4 & 3        & Augments
            \\ \nth{7}  & 4 & 3        & Arcane insight
            \\ \nth{8}  & 4 & 4        & \tdash
            \\ \nth{9}  & 5 & 4        & Spell knowledge
            \\ \nth{10} & 5 & 5        & Augment
            \\ \nth{11} & 5 & 5        & Arcane insight
            \\ \nth{12} & 5 & 6        & \tdash
            \\ \nth{13} & 5 & 6        & Essence lore
            \\ \nth{14} & 5 & 7        & Augment
            \\ \nth{15} & 5 & 7        & Arcane insight
            \\ \nth{16} & 5 & 8        & \tdash
            \\ \nth{17} & 5 & 8        & Greater essence lore
            \\ \nth{18} & 5 & 9        & Augment
            \\ \nth{19} & 5 & 9        & Arcane insight
            \\ \nth{20} & 5 & 10       & Archmage
        \end{dtabularx}
    \end{dtable}

    \classbasics{Alignment} Any.

    \classbasics{Archetypes} Mages have the Arcane Spellcasting, Arcane Lore, and Arcane Spell Mastery \glossterm{archetypes}.

    \subsection{Basic Class Abilities}
        If you are a mage, you gain the following abilities.

        \cf{Mge}{Defenses}
        You gain the following bonuses to your \glossterm{defenses}: \plus3 Fortitude, \plus4 Reflex, \plus5 Mental.

        \cf{Mge}{Skills}
        You gain 4 \glossterm{skill points}.
        In addition, you have the following \glossterm{class skills}:
        \begin{itemize}
            \item \subparhead{Intelligence} Craft, Deduction, Knowledge (all kinds, taken individually), Linguistics.
            \item \subparhead{Perception} Awareness, Spellcraft.
            \item \subparhead{Other} Bluff, Intimidate, Persuasion, Profession.
        \end{itemize}

        \cf{Mge}{Weapon and Armor Proficiencies}
        You are proficient with simple weapons and one other weapon group.
        You are not proficient with any type of armor or shield.
        Armor of any type interferes with your arcane gestures, which can cause your spells with somatic components to fail.

        \cf{Mge}{Arcane Essence}
        All mages have access to great arcane power.
        However, not all mages acquired this power in the same way.
        You choose an arcane essence.
        Many mage abilities have special effects based on whether you are a sorcerer or a wizard.
        \parhead{Sorcerer} Sorcerers have an intuitive connection to magic that allows them to cast spells without preparation or training.
        \parhead{Wizard} Wizards study arcane mysteries for years to learn the secret ways of magic.
        They cast spells with their Intelligence.

    \subsection{Arcane Spellcasting}
        This archetype grants you the ability to cast arcane spells.

        \cf{Mge}{Arcane Spells} 
        You can cast arcane spells.
        You learn two arcane spells from the arcane \glossterm{spell list} (see \pcref{Arcane Spells}).
        \subparhead{Sorcerer} Your spellpower with arcane spells is equal to your level or your Willpower, whichever is higher.
        The maximum spell level you can cast is equal to half your level (minimum 1).
        \subparhead{Wizard} Your spellpower with arcane spells is equal to your character level or your Intelligence, whichever is higher.
        The maximum spell level you can cast is equal to half your level (minimum 1) or your Intelligence, whichever is lower.

        To cast a spell, you must normally spend an \glossterm{action point}.
        Every spell can also be cast as a cantrip.
        Cantrips are weaker, but do not require action points to cast.

        \cf{Mge}{Rituals}
        \subparhead{Sorcerer} You cannot perform arcane rituals.
        \subparhead{Wizard} You can perform arcane rituals to create unique magical effects (see \pcref{Rituals}).
        You have a ritual book containing one arcane ritual of your choice (see \pcref{Arcane Rituals}).

        \cf{Mge}{Arcane Spell Failure}
        Whenever you cast an arcane spell while using \glossterm{armor}, you must roll 1d10.
        If your result is less than or equal to your \glossterm{encumbrance}, you \glossterm{miscast} your spell.

        \cf{Mge}[6]{Augments}
        Choose two \glossterm{augments} (see \pcref{Augments}).
        You can apply those augments to arcane spells you cast and arcane rituals you perform.
        At 10th, 14th level, and 18th level, you learn an additional augment.

    \subsection{Arcane Lore}
        This archetype grants you esoteric abilities relating to arcane magic.
        You must be able to cast arcane spells to gain the abilities from this archetype.

        \cf{Mge}{Mage Armor}
        \begin{ability}{Mage Armor}
            As a standard action, you can use this ability.
            If you do, you create a translucent suit of magical armor on your body.
            This functions like body armor that provides a \plus2 bonus to Armor defense and has no \glossterm{encumbrance}.
            You cannot wear it alongside other armor.
            This ability lasts until you \glossterm{dismiss} it as a free action.
        \end{ability}

        \cf{Mge}[3]{Arcane Insight} 
        You gain a greater understanding of magic.
        You choose one of the following insights.
        Each insight can be chosen multiple times.
        At 7th level, and every four levels thereafter, you gain an additional arcane insight.
        \begin{itemize}
            \item Innate Spell: Choose a spell you know.
                You no longer need verbal or somatic components to cast that spell.
                \par If you choose this insight multiple times, you must choose a different spell each time.
            \item Personal Spell: Choose a spell you know.
                You cannot miscast that spell (see \pcref{Miscasting}).
                If you would miscast it, the spell simply fails without effect.
                In addition, you automatically succeeds at all Concentration checks you make to cast the spell.
                \par If you choose this insight multiple times, you must choose a different spell each time.
            \item Specialization: Choose a school of magic that you know a spell from.
                You learn an additional spell from that school of magic.
                In exchange, you must ban two other schools of magic.
                You can never learn or cast spells or rituals from your banned schools.
                If you know spells from a banned school, you must immediately learn different spells from unbanned schools in their place.
                \par If you choose this insight multiple times, you must choose to specialize in the same school each time.
        \end{itemize}

        \subparhead{Sorcerer} You may also choose the Expanded Spell Knowledge insight.
        You choose a single spell from the divine spell list or nature spell list and add it to your arcane spell list.
        This does not grant you the spell as a spell known, but you may exchange one of your spells known to learn that new spell.
        If do, you may learn subspells from this spell in the same way as you learn subspells from other spells you know.
        \par If you choose this insight multiple times, you must choose a different spell each time.

        \subparhead{Wizard} You may also choose the Ritual Spell insight.
        You scribe an arcane spell you know into your ritual book.
        The spell is treated as a ritual, and you can perform a one minute ritual to cause the spell's effect.
        The ritual costs one action point to perform, and other creatures may not participate in it.
        You can apply \glossterm{augments} or \glossterm{subspells} to the spell normally, increasing the ritual's level appropriately.

        \cf{Mge}[5]{Lesser Essence Lore}[Magical]
        You gain an ability based on your choice of arcane essence.
        % TODO: wording
        \subparhead{Sorcerer} Whenever a spell resolves, if you were a target of the spell, you heal hit points equal to that spell's spellpower.
        This healing applies even if the spell's attack fails or if you resist it with \glossterm{magic resistance}.
        You can only gain this healing once per round.
        \subparhead{Wizard} You gain two additional skill points.
        In addition, you gain a \plus2 bonus to all Knowledge skills.

        \cf*{Mge}[9]{Spell Knowledge}
        You learn an additional spell from the arcane \glossterm{spell list} (see \pcref{Arcane Spells}).

        \cf{Mge}[13]{Essence Lore}[Magical]
        You gain an ability based on your choice of arcane essence.
        \subparhead{Sorcerer} You gain \glossterm{magic resistance} equal to 5 \add your spellpower with arcane spells.
        \subparhead{Wizard} You learn the Contingency augment, allowing you to prepare a spell so it takes effect automatically if specific circumstances arise.
        The Contingency augment adds two levels to a spell's level.
        You can apply this augment to any arcane spell that can be cast as a \glossterm{standard action}.

        Casting a spell with the Contingency augment takes 5 minutes.
        When the casting is complete, the spell has no immediate effect.
        Instead, it automatically takes effect when some specific circumstances arise.
        During the time required to cast the spell, you specify what circumstances cause the spell to take effect.

        The spell can be set to trigger in response to any circumstances that a typical human observing you and your situation could detect.
        For example, you could specify ``when I fall at least 50 feet'' or ``when I become bloodied'', but not ``when there is an invisible creature within 50 feet of me'' or ``when I am at 17 hit points or fewer.''
        The more specific the required circumstances, the better -- vague requirements, such as ``when I am in danger'', may cause the spell to trigger unexpectedly or fail to trigger at all.
        If you attempt to specify multiple separate triggering conditions, such as ``when I take damage or when an enemy is adjacent to me'', the spell will randomly ignore all but one of the conditions.

        If the spell needs to be targeted, the trigger condition can specify a simple rule for identifying how to target the spell, such as ``the closest enemy''.
        If the rule is poorly worded or imprecise, the spell may target incorrectly or fail to activate at all.
        Any spells which require decisions, such as the \spell{dimension door} spell, must have those decisions made at the time it is cast.
        You cannot alter those decisions when the contingency takes effect.

        You can have only one spell with this augment active at a time.
        If you use the augment again with a different spell, the new spell.

        \cf{Mge}[17]{Greater Essence Lore}[Magical]
        You gain an ability based on your choice of arcane essence.

        \subparhead{Sorcerer} Whenever you resist a spell with your \glossterm{magic resistance}, you gain the ability to cast that spell once.
        The spell retains all augments, effects from feats and other abilities, and similar modifications from the original caster, and you cannot choose any other augments or apply effects from your own abilities.
        However, you make all other decisions required to cast the spell, and uses your spellpower to determine the spell's effects.
        Once you cast the spell, you expend the absorbed energy, and you cannot cast it again.

        If you resist multiple spells simultaneously, or if you resist another spell with your magic resistance before casting the previous spell you resisted, you choose which spell you gain the ability to cast.

        \subparhead{Wizard} You may have two spells active with the Contingency augment, rather than only one.
        Whenever you cast a new spell with the Contingency augment, you choose which existing contingency to replace.

        Only one contingency can trigger in a given round.
        If both would trigger simultaneously, only the first spell cast triggers.
        The second spell cast does not trigger that round.

        \cf{Mge}[20]{Archmage}
        You no longer need to spend action points to cast spells.
        If you have any spell points, you lose those spell points and gain the same number of legend points instead.

    \subsection{Arcane Spell Mastery}
        This archetype improves the arcane spells you cast.
        You must be able to cast arcane spells to gain the abilities from this archetype.

        \cf{Mge}{Spell Point}
        You gain a spell point.
        A spell point can be spent to cast spells in place of an action point.
        You recover all spent spell points after a \glossterm{short rest}.

        \cf{Mge}[2]{Spell Knowledge} 
        You learn an additional spell from the arcane \glossterm{spell list} (see \pcref{Arcane Spells}).

        \cf{Mge}[4]{Subspells}
        Choose two \glossterm{subspells} for arcane spells you know.
        You can use those subspells when you cast those spells (see \pcref{Subspells}).
        At 6th level, and every two levels thereafter, you learn an additional subspell for an arcane spell you know.

\newpage
\section{Monk}\label{Monk}
    \begin{dtable}
        \lcaption{Monk Progression}
        \begin{dtabularx}{\columnwidth}{>{\ccol}p{\levelcol} >{\lcol}X}
            \tb{Level} & \tb{Abilities} \\\bottomrule
            \nth{1}     & Ki manifestation, serene strike, unarmed warrior, unfettered defense
            \\ \nth{2}  & Unfettered athletics
            \\ \nth{3}  & Ki vessel
            \\ \nth{4}  & Sage lore
            \\ \nth{5}  & Intuitive reaction
            \\ \nth{6}  & Ki manifestation
            \\ \nth{7}  & Transcend frailty
            \\ \nth{8}  & Unfettered athletics
            \\ \nth{9}  & Dual manifestation
            \\ \nth{10} & Greater serene strike
            \\ \nth{11} & Greater intuitive reaction
            \\ \nth{12} & Ki manifestation
            \\ \nth{13} & Transcend flesh
            \\ \nth{14} & Greater unfettered defense
            \\ \nth{15} & Ki wellspring
            \\ \nth{16} & Inner peace
            \\ \nth{17} & Supreme intuitive reactoin
            \\ \nth{18} & Myriad manifestation
            \\ \nth{19} & Transcend mortality
            \\ \nth{20} & \tdash
        \end{dtabularx}
    \end{dtable}

    \classbasics{Alignment} Any nonchaotic.

    \classbasics{Archetypes} Monks have the Ki, Unfettered Warrior, and Transcendent Sage \glossterm{archetypes}.

    \subsection{Basic Class Abilities}
        If you are a monk, you gain the following abilities.

        \cf{Mnk}{Defenses}
        You gain the following bonuses to your \glossterm{defenses}: \plus3 Fortitude, \plus5 Reflex, \plus4 Mental.

        \cf{Mnk}{Skills}
        You gain 6 \glossterm{skill points}.
        In addition, you have the following \glossterm{class skills}:
        \begin{itemize}
            \item \subparhead{Strength} Climb, Jump, Swim.
            \item \subparhead{Dexterity} Acrobatics, Escape Artist, Ride, Stealth.
            \item \subparhead{Intelligence} Craft, Deduction, Heal.
            \item \subparhead{Perception} Awareness, Spellcraft, Survival.
            \item \subparhead{Other} Bluff, Intimidate, Perform, Persuasion, Profession.
        \end{itemize}

        \cf{Mnk}{Weapon and Armor Proficiencies}
        Monks are proficient with simple weapons, monk weapons, and any one other weapon group.
        Monks are not proficient with any armor or shields.
        When wearing armor, using a shield, or carrying a medium or heavy load, a monk loses the benefit of your \textit{enlightened defense} and \textit{fast movement} abilities, as well as all abilities from the Ki archetype.

    \subsection{Ki}
        This archtype grants you abilities you can use in combat.
        All abilities from this archetype are \glossterm{magical}.

        \cf{Mnk}{Ki Power}
        The \glossterm{power} of your ki abilities is determined by your \textit{ki power}.
        Your \textit{ki power} is equal to your level or your Willpower, whichever is higher.

        \cf{Mnk}{Ki Manifestation}
        You can channel your ki to temporarily enhance your abilities.
        Choose two \textit{ki manifestations} from the list below.
        You can spend an \glossterm{action point} to use a \textit{ki manifestation}, using the type of action indicated in the ability's description.
        You cannot use more than one \textit{ki manifestation} per round.
        {
            \begin{ability}{Abandon the Fragile Self}[\glossterm{Swift}]
                As a \glossterm{minor action}, you can use this ability.
                It lasts until the end of the round.
                You are immune to all \glossterm{conditions}.

                At 6th level, this ability lasts until the end of the next round.
                At 12th level, you can also remove one condition affecting you when you use this ability.
                At 18th level, you can remove two conditions instead of one.
            \end{ability}

            \begin{ability}{Burst of Blinding Speed}
                At the start of the round, you can use this ability.
                It lasts until the end of the round.
                You gain a \plus30 foot bonus to your land speed, up to a maximum of double your original speed.

                At 6th level, you can also ignore \glossterm{difficult terrain}.
                At 12th level, you can also move or stand on liquids as if they were solid.
                At 18th level, the speed bonus increases to \plus60 feet, up to a maximum of triple your original speed.
            \end{ability}

            \begin{ability}{Dance of Falling Feathers}
                At the start of the round, you can use this ability.
                It lasts until the end of the round.
                If you are in free-fall, your fall is dramatically slowed.
                You fall only 60 feet this round, and take no falling damage if you hit the ground.

                At 6th level, this ability's effect lasts as long as you \glossterm{sustain} it as a \glossterm{minor action}.
                At 12th level, you can control the speed of your fall, up to a maximum of 120 feet per round and a minimum of 30 feet per round.
                At 18th level, your maximum falling speed becomes 180 feet per round and your minimum falling speed becomes 10 feet per round.
            \end{ability}

            \begin{ability}{Elegant Whirl of Fluid Motion}
                At the start of the round, you can use this ability.
                It lasts until the end of the round.
                You gain a \plus5 bonus to Acrobatics (see \pcref{Acrobatics}).

                At 6th level, you also gain a \plus1 bonus to Armor and Reflex defenses.
                At 12th level, the Acrobatics bonus increases to \plus10.
                At 18th level, the defense bonus increases to \plus2.
            \end{ability}

            \begin{ability}{Flash Step}[\glossterm{Teleportation}]
                At the start of the round, you can use this ability.
                It lasts until the end of the round.
                % TODO: is 'horizontally' the correct word?
                Whenever you move, you can teleport horizontally up to ten feet instead.
                This replaces the entire distance you would have moved that phase.
                If your \glossterm{line of effect} to your destination is blocked, or if this teleportaiton would somehow place you inside a solid object, your movement is cancelled and you remain where you are.

                At 6th level, the distance you can teleport is increased to half your movement speed.
                At 12th level, you can teleport in multiple steps within the same phase.
                Each step costs movement equal to twice the distance you teleport.
                At 18th level, you can attempt to teleport to locations outside of \glossterm{line of sight} and \glossterm{line of effect}.
                If your intended destination is invalid, the distance you spent teleporting is wasted, but you suffer no other ill effects.
            \end{ability}

            \begin{ability}{Ki Strike}
                You can use this ability as a standard action.
                If you do, you make a \glossterm{strike}.
                You may use your \textit{ki power} in place of your Perception to determine your \glossterm{accuracy} with the strike, and in place of your Strength to determine your \glossterm{strike damage} with the strike.
                If the strike misses, you regain the action point spent to use this ability.

                At 6th level, your \glossterm{reach} is increased by 5 feet for the strike.
                At 12th level, the bonus to your reach increases to 10 feet. 
                At 18th level, this ability does not cost an action point to use.
            \end{ability}

            \begin{ability}{Leap of the Heavens}
                At the start of the round, you can use this ability.
                It lasts until the end of the round.
                You gain a \plus5 bonus to Jump (see \pcref{Jump}).

                At 6th level, the bonus increases to \plus10.
                At 12th level, the bonus increases to \plus15.
                At 18th level, the bonus increases to \plus20.
            \end{ability}

            \begin{ability}{Scale the Highest Tower}
                At the start of the round, you can use this ability.
                It lasts until the end of the round.
                You gain a \plus5 bonus to Climb (see \pcref{Climb}).
                % TODO: is this wording correct?

                At 6th level, you also take no penalty for climbing without using your hands.
                At 12th level, the Climb bonus increases to \plus10.
                At 18th level, you also take no penalty for climbing with only one limb, such as an arm or leg.
            \end{ability}

            \begin{ability}{See the Flow of Life}
                At the start of the round, you can use this ability.
                It lasts until the end of the round.
                You gain the ability to see the ki of living creatures.
                You can ``see'' any living creatures and their equipment within 50 feet perfectly, regardless of lighting conditions, blindness, invisibility, or any other means of concealment.
                This cannot detect living creatures through solid walls, however.

                At 6th level, range is increased to 100 feet.
                At 12th level, the effect lasts as long as you \glossterm{sustain} it as a \glossterm{minor action}.
                At 18th level, the effect lasts as long as you \glossterm{attune} to it.
            \end{ability}

            \begin{ability}{Step Between the Mystic Worlds}[\glossterm{Swift}]
                As a \glossterm{minor action}, you can use this ability.
                It lasts until the end of the round.
                You gain a \plus2 bonus to defenses against \glossterm{magical} abilities that directly target you.
                This does not protect you from abilities that affect an area.

                At 6th level, the defense bonus is increased to \plus4.
                At 12th level, the defense bonus is increased to \plus6.
                At 18th level, you cannot be directly targeted by \glossterm{magical} abilities.
            \end{ability}

            \begin{ability}{Surpass the Mortal Limits}
                At the start of the round, you can use this ability.
                It lasts until the end of the round.
                You can use your \textit{ki power} in place of your Strength, Dexterity, and Constitution when making checks.

                At 6th level, you also gain a \plus2 bonus to checks based on Strength, Dexterity, and Constitution.
                At 12th level, the bonus increases to \plus4.
                At 16th level, the bonus increases to \plus6.
            \end{ability}

            \begin{ability}{Tranquil Ward of Unshakeable Peace}[\glossterm{Swift}]
                As a standard action, you can use this ability.
                It lasts until the end of the round.
                You take half damage from all attacks.
                This halving is applied before damage reduction and similar abilties.

                At 6th level, you can \glossterm{sustain} this ability as a standard action.
                At 12th level, you also gain damage reduction equal to your level.
                At 18th level, the damage reduction increases to twice your level.
            \end{ability}
        }

        \cf*{Mnk}[3]{Ki Vessel} You gain an additional \glossterm{action point}.

        \cf*{Mnk}[6]{Ki Manifestation}
        You learn an additional \textit{ki manifestation}.

        \cf{Mnk}[9]{Dual Manifestation} You can use up to two \textit{ki manifestation} abilities per round.

        \cf*{Mnk}[12]{Ki Manifestation}
        You learn an additional \textit{ki manifestation}.

        \cf{Mnk}[15]{Ki Wellspring} You gain two additional \glossterm{action points}.

        % Should this be learning an additional ki manifestation instead?
        \cf{Mnk}[18]{Myriad Manifestation} You can use up to three \textit{ki manifestation} abilities per round.

    \subsection{Unfettered Warrior}
        This archetype improves your combat prowess while unarmed and unencumbered.

        \cf{Mnk}{Unarmed Warrior}
        You are \glossterm{proficient} with your \glossterm{unarmed attack}.
        In addition, you gain a \plus2d bonus to damage with your unarmed attack.
        For details about how to fight while unarmed, see \pcref{Unarmed Combat}.

        \cf{Mnk}{Unfettered Defense}[Magical]
        If you are not wearing armor and have no \glossterm{encumbrance}, you gain a \plus2 bonus to Armor defense.
        You lose this bonus when you are \helpless.

        \cf{Mnk}[2]{Unfettered Athletics} You gain two additional skill points.

        \cf{Mnk}[5]{Intuitive Reaction}
        You are not \unaware when attacked by surprise.
        In addition, you gain a \plus2 bonus to \glossterm{initiative}.

        \cf*{Mnk}[8]{Unfettered Athletics} You gain two additional skill points.

        \cf{Mnk}[11]{Greater Intuitive Reaction}
        The initiative bonus from your \textit{intuitive reaction} ability increases to \plus5.

        \cf{Mnk}[14]{Greater Unfettered Defense}
        Your bonus to Armor defense from your \textit{unfettered defense} ability improves to \plus3.

        \cf{Mnk}[17]{Supreme Intuitive Reaction}
        The initiative bonus from your \textit{intuitive reaction} ability increases to \plus10.

    \subsection{Transcendent Sage}
        This archetype grants you abilities to resist or remove conditions.

        \cf{Mnk}{Serene Strike}
        \begin{ability}{Serene Strike}
            As a standard action, you can spend an \glossterm{action point} to use this ability.
            If you do, you may remove one \glossterm{condition} affecting you.
            If you use this ability during the \glossterm{action phase}, you may also make a \glossterm{strike} during the \glossterm{delayed action phase}.
        \end{ability}

        \cf{Mnk}[4]{Sage Lore} You gain two additional skill points.

        \cf{Mnk}[7]{Transcend Frailty}
        You are immune to being \glossterm{deafened}, \glossterm{fatigued}, and \glossterm{sickened}.

        \cf{Mnk}[10]{Greater Serene Strike} You gain a \plus1d bonus to damage with the strike from your \textit{serene strike} ability.

        \cf{Mnk}[13]{Transcend Flesh}
        You are immune to being \glossterm{blinded}, \glossterm{exhausted}, and \glossterm{nauseated}.
        % TODO: do you still take aging penalties?
        In addition, you no longer take penalties to your attributes for aging, and cannot be magically aged.
        You still die of old age when your time is up.

        \cf{Mnk}[16]{Inner Peace}
        You are immune to hostile \glossterm{Mind} abilities.

        \cf{Mnk}[19]{Transcend Mortality}[Magical]
        If you die, you may choose to retain control of your body and soul through sheer force of will.
        Your body immediately disappears, and your soul does not travel to an afterlife.
        Instead, your body reforms with no trace of its injuries 8 hours later.
        The reformed body is in perfect health and can be any age you choose, to a minimum of the age of adulthood for your race.
        You can reform your body at the place where you died, or in any place on the same plane that is deeply familiar to you.

        After each time you reform yourself in this way, it takes an additional hour to reform the next time you ``die''.
        You can only be permanently killed by the direct intervention of a deity.

    \subsection{Ex-Monks}
        % TODO: this wording is repetitive
        If you become chaotic, you lose all of your \glossterm{magical} monk abilities.
        If you stop being chaotic, you regain your magical monk abilities.

\newpage
\section{Paladin}\label{Paladin}
    \begin{dtable}
        \lcaption{Paladin Progression}
        \begin{dtabularx}{\columnwidth}{>{\ccol}p{\levelcol} c >{\lcol}X}
            \tb{Level} & \tb{Spells} & \tb{Abilities} \\
            \bottomrule
            \nth{1}     & 2 & Lay on hands, smite, spells
            \\ \nth{2}  & 2 & Unfaltering champion
            \\ \nth{3}  & 2 & Cleansing touch
            \\ \nth{4}  & 2 & Zealous offense
            \\ \nth{5}  & 2 & Aligned aura
            \\ \nth{6}  & 2 & Augments, unfaltering zeal
            \\ \nth{7}  & 2 & Unbending devotion
            \\ \nth{8}  & 3 & Pass judgment
            \\ \nth{9}  & 3 & Expanded aura
            \\ \nth{10} & 3 & Augment, greater unfaltering champion
            \\ \nth{11} & 3 & Greater lay on hands
            \\ \nth{12} & 3 & Greater smite
            \\ \nth{13} & 3 & Greater aligned aura
            \\ \nth{14} & 3 & Augment, greater unfaltering zeal
            \\ \nth{15} & 3 & Greater cleansing touch
            \\ \nth{16} & 3 & Greater zealous offense
            \\ \nth{17} & 3 & Greater unbending devotion
            \\ \nth{18} & 3 & Augment, supreme unfaltering champion
            \\ \nth{19} & 3 & Greater expanded aura
            \\ \nth{20} & 3 & Aligned soul
        \end{dtabularx}
    \end{dtable}

    \classbasics{Alignment} Any other than true neutral.

    \classbasics{Archetypes} Paladins have the Devoted Paragon, Divine Spellcasting, and Zealous Warrior \glossterm{archetypes}.

    \subsection{Basic Class Abilities}
        If you are a paladin, you gain the following abilities.

        \cf{Pal}{Defenses}
        You gain the following bonuses to your \glossterm{defenses}: \plus5 Fortitude, \plus3 Reflex, \plus4 Mental.

        \cf{Pal}{Skills}
        You gain 4 \glossterm{skill points}.
        In addition, you have the following \glossterm{class skills}:
        \begin{itemize}
            \item \subparhead{Dexterity} Ride.
            \item \subparhead{Intelligence} Craft, Deduction, Heal, Knowledge (local, religion).
            \item \subparhead{Perception} Awareness, Intimidate, Sense Motive.
            \item \subparhead{Other} Bluff, Intimidate, Persuasion, Profession.
        \end{itemize}

        \cf{Pal}{Weapon and Armor Proficiencies}
        Paladins are proficient with simple weapons, any three other weapon groups, all types of armor (heavy, medium, and light), and shields.

        \cf{Pal}{Devoted Alignment} 
        You are devoted to a specific alignment.
        You must choose one of your alignment components: good, evil, lawful, or chaotic.
        The alignment you choose is your devoted alignment.
        Your paladin abilities are affected by this choice.
        % seems unnecessary
        % You excel at slaying creatures with alignments opposed to your devoted alignment.
        Your alignment cannot be changed without extraordinary repurcussions.

        \cf{Pal}{Devotion Power}
        The \glossterm{power} of many paladin spells and abilities is determined by your \textit{devotion power}.
        Your \textit{devotion power} is equal to your level or your Willpower, whichever is higher.

    \subsection{Devoted Paragon}
        This archetype grants you healing abiliies and an aura reflecting your alignment.

        \cf{Pal}{Lay on Hands}[Magical]
        \begin{ability}{Lay on Hands}
            As a standard action, you can spend an \glossterm{action point} to use this ability.
            If you do, choose an adjacent willing creature.
            The target is healed for hit points equal to \glossterm{standard damage} \plus1d.
            Your \glossterm{power} with this ability is equal to your \textit{devotion power}.
        \end{ability}

        \cf{Pal}[3]{Cleansing Touch} When you use your \textit{lay on hands} ability, the target can also remove one \glossterm{condition}.

        \cf{Pal}[5]{Aligned Aura}[Magical]
        Your devotion to your alignment affects the world around you, bringing it closer to your ideals.
        You constantly radiate an aura in a \areamed radius \glossterm{emanation} from you.
        The effect of the aura depends on your devoted alignment, as described below.
        You can suppress or resume the aura as a \glossterm{minor action}.

        \subparhead{Chaos} Whenever you or an ally in the area rolls a 1 on an attack roll when making a \glossterm{strike}, the attack roll explodes (see \pcref{Exploding Attacks}).
        This does not affect additional dice rolled if the attack roll explodes.
        \subparhead{Evil} All other creatures in the area suffer a \minus1 penalty to all defenses.
        \subparhead{Good} Whenever a creature in the area takes damage, you may take half that damage (rounded down) instead.
        Any abilities you have that would make the attack miss or fail have no effect, but your abilities that allow you to reduce or ignore its effects work normally.
        The protected creature takes the remaining half of the damage, and suffers any non-damaging effects of the attack normally.
        \subparhead{Law} Whenever you or an ally in the area rolls a 1 on an attack roll when making a \glossterm{strike}, the attack roll is treated as a 6.

        \cf{Pal}[7]{Unbending Devotion}[Magical]
        You are immune to \glossterm{Mind} \glossterm{conditions}.

        \cf{Pal}[9]{Expanded Aura}
        The area of your \textit{aligned aura} ability increases to \arealarge.

        \cf{Pal}[11]{Greater Lay on Hands} 
        You gain a \plus1d bonus to the healing from your \textit{lay on hands} ability.

        \cf{Pal}[13]{Greater Aligned Aura}[Magical]
        The effect of your \textit{aligned aura} becomes stronger based on your devoted alignment.

        \subparhead{Chaos} Whenever an enemy in the area rolls a 10 on an attack roll when making a \glossterm{strike}, it is forced to rereroll the attack roll and take the second result.
        \subparhead{Evil} The penalty imposed by the aura increases to \minus2.
        \subparhead{Good} When you redirect damage from an ally with this aura, you can redirect all effects of the attack to you instead of only half the damage.
        \subparhead{Law} Whenever an enemy in the area rolls a 10 on an attack roll when making a \glossterm{strike}, the attack roll is treated as a 6.

        \cf{Pal}[15]{Greater Cleansing Touch} When you use your \textit{lay on hands} ability, the target can remove any number of conditions, not just one.

        \cf{Pal}[17]{Greater Unbending Devotion}
        You are immune to all hostile \glossterm{Mind} effects.

        \cf{Pal}[19]{Greater Expanded Aura} The area of your \textit{aligned aura} ability increases to \areahuge.

        \cf{Pal}[20]{Aligned Soul}[Magical]
        While you are dead, you may approach the deity or governing figure of your afterlife and request to be returned to life to continue your mission.
        Travelling to the relevant figure and making the request takes 12 hours.
        Unless there are extenuating circumstances, this request is almost always granted, and you are resurrected in a new body at a location of the entity's choice.
        This functions like the \ritual{resurrection} ritual, except that no part of the body is required, and a new body is created by the entity.
        You can be resurrected in this way regardless of the condition of your body, but not if your soul has been trapped or otherwise prevented from going to the correct afterlife.

        % TODO: clarify interaction between this ability and cleric spellcasting;
        % you shouldn't have both, but they technically use different spellpowers
    \subsection{Divine Spellcasting}
        This archetype grants you the ability to cast divine spells.

        \cf{Pal}{Divine Spells}
        Your devotion to your alignment grants you the ability to cast divine spells.
        You learn two divine spells from the divine \glossterm{spell list} (see \pcref{Divine Spells}).
        Your \glossterm{spellpower} with divine spells is equal to your \textit{devotion power}.

        To cast a spell, you must normally spend an \glossterm{action point}.
        Every spell can also be cast as a cantrip.
        Cantrips are weaker, but do not require action points to cast.

        \cf{Pal}{Rituals}
        You can perform divine rituals to create unique magical effects (see \pcref{Rituals}).
        You have a ritual book containing one divine ritual of your choice (see \pcref{Divine Rituals}).

        \cf{Pal}[6]{Augments}
        Choose two \glossterm{augments} (see \pcref{Augments}).
        You can apply those augments to divine spells you cast and divine rituals you perform.
        At 10th, 14th level, and 18th level, you learn an additional augment.

        \cf{Pal}[8]{Spell Knowledge}
        You learn an additional divine spell (see \pcref{Divine Spells}).

    \subsection{Zealous Warrior}
        This archetype improves your combat prowess, especially against foes who do not share your devoted alignment.

        \cf{Pal}{Smite}[Magical]
        \begin{ability}{Smite}
            As a standard action, you can spend an \glossterm{action point} to use this ability.
            If you do, make a \glossterm{strike}.
            If your target shares your devoted alignment, the strike deals no damage.
            Otherwise, the strike gains a \plus2d bonus to \glossterm{strike damage}.
        \end{ability}

        \cf{Pal}[2]{Unfaltering Champion}
        You gain a \plus2 bonus to \glossterm{threat} and a a \plus1 bonus to Armor defense.

        \cf{Pal}[4]{Zealous Offense}[Magical]
        Whenever you deal damage with your \textit{smite} ability, you gain a \plus1 bonus to \glossterm{accuracy} until the end of the next round.

        \cf{Pal}[6]{Unfaltering Zeal}
        You gain a \plus1 bonus to Fortitude and Mental defense.

        \cf{Pal}[8]{Pass Judgment}[Magical]
        \begin{ability}{Pass Judgment}[\glossterm{Attuned}]
            Whenever you use your \textit{smite} ability, you can spend an additional \glossterm{action point} to use this ability.
            If you do, the target of your strike is treated as if it had the alignment opposed to your devoted alignment for all spells and effects, including for your initial \textit{smite}.
            This only affects its alignment along the alignment axis your devoted alignment is on.
            For example, if your devoted alignment was evil, a chaotic neutral target would be treated as chaotic good.

            You can use this ability to do battle against foes who share your alignment, but you should exercise caution in doing so.
            Persecution of allies can lead you to fall and become an ex-paladin.
        \end{ability}

        \cf{Pal}[10]{Greater Unfaltering Champion} The \glossterm{threat} bonus from your \textit{unfaltering champion} ability increases to \plus4.

        \cf{Pal}[12]{Greater Smite} The damage bonus from your \textit{smite} ability increases to \plus3d.

        \cf{Pal}[14]{Greater Unfaltering Zeal} The defense bonuses from your \textit{unfaltering zeal} ability increase to \plus2.

        \cf{Pal}[16]{Greater Zealous Offense}[Magical]
        The accuracy bonus from your \textit{zealous offense} ability increases to \plus2.

        \cf{Pal}[18]{Supreme Unfaltering Champion}
        The Armor defense bonus from your \textit{unfaltering champion} ability increases to \plus2.

        % TODO: Doesn't work with new spell system
        % \cf{Pal}[20]{Martyr's Retribution}[Mag]
        % If you die in the service of your devoted alignment, you may choose to have your fallen body erupt in an immense burst of divine energy.
        % If you do, your body is almost completely consumed, preventing you from being raised with \ritual{resurrection} and similar effects that require an intact body.
        % This burst has two effects.
        % First, a \spell{sunburst} spell immediately takes effect over the area where you died.
        % Second, a \spell{storm of vengeance} spell begins to take effect, centered on the same area.
        % The spell lasts for 10 rounds, and the lightning strikes target the paladin's enemies.
        % Both of these effects harm only the paladin's foes, and do not harm your allies.
        % However, your allies' vision is still impeded by the \spell{storm of vengeance}.

    \subsection{Ex-Paladins}
        If you cease to follow your devoted alignment, you lose all \glossterm{magical} paladin class abilities.
        If your atone for your misdeeds and resume the service of your devoted alignment, you can regain your abilities.

\newpage
\section{Ranger}\label{Ranger}
    \begin{dtable}
        \lcaption{Ranger Progression}
        \begin{dtabularx}{\columnwidth}{>{\ccol}p{\levelcol} >{\lcol}X}
            \tb{Level} & \tb{Abilities} \\\bottomrule
            \nth{1}     & Keen vision, quarry, wild exertion
            \\ \nth{2}  & Learned perception, tracker
            \\ \nth{3}  & Wilderness lore
            \\ \nth{4}  & Hunting style
            \\ \nth{5}  & Blindsense
            \\ \nth{6}  & Wild exertion
            \\ \nth{7}  & Hunting lore
            \\ \nth{8}  & Farsight
            \\ \nth{9}  & Wilderness lore
            \\ \nth{10} & Hunting style
            \\ \nth{11} & Blindsight
            \\ \nth{12} & Wild exertion
            \\ \nth{13} & Fluid style
            \\ \nth{14} & Greater farsight
            \\ \nth{15} & Wilderness mastery
            \\ \nth{16} & Hunting style
            \\ \nth{17} & Truesight
            \\ \nth{18} & Wild exertion
            \\ \nth{19} & Dual quarry
            \\ \nth{20} & 
        \end{dtabularx}
    \end{dtable}

    \classbasics{Alignment} Any.

    \classbasics{Archetypes} Rangers have the Keen Senses, Wilderness Warrior, and Master of the Wild \glossterm{archetypes}.

    \subsection{Basic Class Abilities}
        If you are a ranger, you gain the following abilities.

        \cf{Rgr}{Defenses}
        You gain the following bonuses to your \glossterm{defenses}: \plus4 Fortitude, \plus5 Reflex, \plus3 Mental.

        \cf{Rgr}{Skills}
        You gain 8 \glossterm{skill points}.
        In addition, you have the following \glossterm{class skills}:
        \begin{itemize}
            \item \subparhead{Strength} Climb, Jump, Swim.
            \item \subparhead{Dexterity} Acrobatics, Escape Artist, Ride, Stealth.
            \item \subparhead{Intelligence} Craft, Deduction, Heal, Knowledge (dungeoneering, geography, nature).
            \item \subparhead{Perception} Awareness, Creature Handling, Survival.
            \item \subparhead{Other} Bluff, Intimidate, Persuasion, Profession.
        \end{itemize}

        \cf{Rgr}{Weapon and Armor Proficiencies}
        A ranger is proficient with simple weapons, any two other weapon groups, light and medium armor, and shields.
        You are also proficient with your choice of bows, crossbows, or thrown weapons.

    \subsection{Keen Senses}
        This archetype improves your senses.

        \cf{Rgr}{Keen Vision}
        Your sight improves, allowing you to see more easily.
        You gain \glossterm{low-light vision}, allowing you to treat sources of light as if they had double their normal illumination range.
        If you already have low-light vision, you double its benefit, allowing you to treat sources of light as if they had four times their normal illumination range.

        In addition, you gain \glossterm{darkvision} out to 50 feet, allowing you to see in complete darkness.
        If you already have darkvision, you increase its range by 50 feet.

        \cf{Rgr}[2]{Learned Perception} You gain two skill points.

        \cf{Rgr}[5]{Blindsense}
        Your perceptions are so finely honed that you can sense your enemies without seeing them.
        You gain the \glossterm{blindsense} ability out to 50 feet.
        This ability allows you to sense the presence and location of objects and foes within 50 feet without seeing them.
        If you already have the blindsense ability, you increase its range by 50 feet.

        \cf{Rgr}[8]{Farsight}
        You increase the range of your \glossterm{darkvision} by 150 feet, and your \glossterm{blindsense} by 50 feet.
        In addition, you reduce your \glossterm{range increment} penalties for attacking at long range by 2.

        \cf{Rgr}[11]{Blindsight}
        You gain the \glossterm{blindsight} ability, allowing you to ``see'' perfectly without your eyes in a 50 foot radius around you.
        With this ability, you can fight just as well with your eyes closed as with them open.

        \cf{Rgr}[14]{Greater Farsight}
        You increase the range of your \glossterm{darkvision} by 500 feet, your \glossterm{blindsense} by 200 feet, and your \glossterm{blindsight} by 50 feet.
        In addition, the penalty reduction of \glossterm{range increment} penalties from your \textit{farsight} ability increases to 5.

        \cf{Rgr}[17]{Truesight} 
        Your perceptions are accurate enough to defeat even powerful magic.
        You can see through normal and magical darkness, see the truth behind visual figments and glamers, and see the true form of creatures and objects affected by \glossterm{Shaping} abilities.
        This ability works at any range.

    \subsection{Wilderness Warrior}
        This archetype grants you abilities to use in combat and improves your wilderness skills.

        \cf{Rgr}{Wild Exertion} 
        You can channel your martial prowess into devastating attacks.
        Choose two \textit{wild exertions} from the list below.
        As a standard action, you can spend an \glossterm{action point} to use a \textit{wild exertion} ability.
        {
            \begin{ability}{Brace for Impact}[\glossterm{Swift}]
                You take half damage from all attacks.
                This halving is applied before damage reduction and similar abilities.
                This ability lasts until the end of the round.

                At 6th level, you can \glossterm{sustain} this ability as a standard action.
                At 12th level, you also gain damage reduction equal to your level.
                At 18th level, the damage reduction increases to twice your level.
            \end{ability}

            \begin{ability}{Certain Strike}
                Make a \glossterm{strike} with a \plus2 bonus to accuracy.

                At 6th level, the accuracy bonus increases to \plus3.
                At 12th level, the accuracy bonus increases to \plus4.
                At 18th level, the accuracy bonus increases to \plus5.
            \end{ability}

            \begin{ability}{Gut Punch}
                Make a \glossterm{strike} with a bludgeoning weapon.
                The attack is made against Fortitude defense instead of Armor defense.
                If the target takes damage from the strike, it is \sickened as a \glossterm{condition}.

                At 6th level, you gain a \plus1 bonus to accuracy with the strike.
                At 12th level, you gain a \plus1d bonus to damage with the strike.
                At 18th level, the accuracy bonus increases to \plus2.
            \end{ability}

            \begin{ability}{Leaping Strike}
                You make a Jump check to leap and move as normal for the leap, up to a maximum distance equal to your \glossterm{base speed} (see \pcref{Leap}).
                If you use this ability during the \glossterm{action phase}, you can also make a \glossterm{strike} from your new location during the \glossterm{delayed action phase}.

                At 6th level, you gain a \plus1d bonus to damage with the strike per 10 feet of height you travelled downward towards your foe during the leap, up to a maximum of \plus2d.
                At 12th level, the maximum damage bonus increases to \plus4d.
                At 18th level, the maximum damage bonus increases to \plus6d.
            \end{ability}

            \begin{ability}{Penetrating Strike}
                Make a \glossterm{strike} with a piercing weapon.
                The attack is made against the target's Reflex defense instead of its Armor defense.

                At 6th level, if you miss with the strike, you regain the action point spent to use this ability.
                At 12th level, you gain a \plus1d bonus to damage with the strike.
                At 18th level, the damage bonus increases to \plus2d.
            \end{ability}

            \begin{ability}{Potent Maneuver}
                You use a \glossterm{combat maneuver} with a \plus3 bonus to accuracy.

                At 6th level, the accuracy bonus increases to \plus4.
                At 12th level, the accuracy bonus increases to \plus5.
                At 18th level, the accuracy bonus increases to \plus6.
            \end{ability}

            \begin{ability}{Power Attack}
                Make a \glossterm{strike} with a \plus2d bonus to damage.

                At 6th level, if you miss with the strike, you regain the action point spent to use this ability.
                At 12th level, the damage bonus increases to \plus3d.
                At 18th level, the damage bonus increases to \plus4d.
            \end{ability}

            \begin{ability}{Rapid Assault}
                Make a \glossterm{strike} against a creature.
                If you use this ability during the \glossterm{action phase}, you can make another strike during the \glossterm{delayed action phase}.
                You take a \minus2 penalty to accuracy on both strikes.

                At 6th level, if you missed all of your targets, you regain the action point spent to use this ability.
                At 12th level, you gain a \plus1d bonus to damage with both strikes.
                At 18th level, the damage bonus increases to \plus2d.
            \end{ability}

            \begin{ability}{Reaping Charge}
                You can move up to your movement speed in a straight line.
                Choose either the right or left side of the line.
                You can make a melee \glossterm{strike} with a slashing or bludgeoning weapon against each creature and object on that side of the line that you \glossterm{threaten} at any point during your movement, except for the space you start in and the space you end in.
                You take a \minus2d penalty to damage on each strike.

                At 6th level, you do not have to choose a side of the line.
                Instead, you can attack creatures and objects that you threaten at any point during your movement.
                At 12th level, the damage penalty is reduced to \minus1d.
                At 18th level, the damage penalty is removed.
            \end{ability}

            \begin{ability}{Strip the Flesh}
                Make a \glossterm{strike} with a slashing weapon.
                At the end of the current phase, if you hit with the strike and the target is not \glossterm{bloodied}, it takes additional damage equal to the damage you dealt with the strike.

                At 6th level, if you hit with the strike, the target continues taking the same damage at the end of each \glossterm{action phase} in subsequent rounds until it becomes \glossterm{bloodied}.
                This is a \glossterm{condition}, and can be removed by abilities that remove conditions.
                At 12th level, you gain a \plus1d bonus to damage with the strike.
                At 18th level, the damage bonus increases to \plus2d.
            \end{ability}

            \begin{ability}{Sweeping Strike}
                Make a melee \glossterm{strike} with a slashing or bludgeoning weapon.
                The strike targets each of up to three creatures or objects you \glossterm{threaten}.
                You take a \minus1d penalty to \glossterm{strike damage} with the strike.

                At 6th level, if you missed all of your targets, you regain the action point spent to use this ability.
                At 12th level, the damage penalty is removed.
                At 18th level, you gain a \plus1d bonus to damage with the strike.
            \end{ability}

            \begin{ability}{Whirlwind Spin}
                Make a melee \glossterm{strike} with a slashing weapon.
                The strike targets all creatures you \glossterm{threaten}.
                You take a \minus2d penalty to \glossterm{strike damage} with the strike.

                At 6th level, the damage penalty is reduced to \minus1d.
                At 12th level, the damage penalty is removed.
                At 18th level, you gain a \plus1d bonus to damage with the strike.
            \end{ability}
        }

        \cf{Rgr}[3]{Wilderness Lore} You gain two extra skill points.

        \cf*{Rgr}[6]{Wild Exertion}
        You learn an additional \textit{wild exertion}.

        \cf*{Rgr}[9]{Wilderness Lore} You gain two extra skill points.

        \cf*{Rgr}[12]{Wild Exertion} 
        You learn an additional \textit{wild exertion}.

        \cf{Rgr}[15]{Wilderness Mastery} You gain a \plus2 bonus to the Creature Handling, Heal, Knowledge (geography), Knowledge (nature), Ride, and Survival skills.

        \cf*{Rgr}[18]{Wild Exertion} 
        You learn an additional \textit{wild exertion}.

    \subsection{Master of the Hunt}
        This archetype grants you and your allies abilities to hunt down specific foes.

        \cf{Rgr}{Quarry}\label{Quarry} As a \glossterm{minor action}, you can spend an \glossterm{action point} to use the \textit{quarry} ability.
        \begin{ability}{Quarry}
            Choose a creature within \rnglong range.
            The target becomes your quarry.
            You and any allies within range gain a \plus1 bonus to \glossterm{accuracy} with \glossterm{physical attacks} against your quarry.
            You and your allies affected by this ability are called your \glossterm{hunting party}.
            In addition, you gain a \plus5 bonus to checks made to follow the target's tracks.

            This effect lasts as long as you \glossterm{attune} to it.
        \end{ability}

        \cf{Rgr}[2]{Tracker}
        You gain a \plus2 bonus to checks made to follow tracks.
        In addition, you may use your \textit{quarry} ability on any creature whose tracks you are following, regardless of the creature's current location.

        \cf{Rgr}[4]{Hunting Style}
        You learn specific hunting styles to defeat particular quarries.
        Choose two hunting styles from the list below.
        Whenever you use your \textit{quarry} ability, you may also use one of your \textit{hunting styles}.
        {
            \subcf{Anchoring}[Magical]
            As long as your quarry is \glossterm{threatened} by at least two members of your \glossterm{hunting party}, it cannot travel extradimensionally.
            This prevents all \glossterm{Manifestation}, \glossterm{Planar}, and \glossterm{Teleportation} effects.
            \par At 10th level, this effect instead applies if your quarry is within \rngmed range of at two members of your hunting party.
            At 16th level, this effect instead applies if your quarry is within \rnglong range of any member of your hunting party.

            \subcf{Brutal Assault}
            The accuracy bonus from your \textit{quarry} ability is replaced with a \minus1 penalty to accuracy with \glossterm{strikes} against your quarry.
            In exchange, your hunting party gains a \plus1d bonus to \glossterm{strike damage} against your quarry.
            \par At 10th level, the accuracy penalty is removed.
            At 16th level, the damage bonus is increased to \plus2d.

            \subcf{Coordinated Stealth}
            Your quarry takes a \minus5 penalty to Awareness checks to notice members of your \glossterm{hunting party}.
            \par At 10th level, your hunting party gains a \plus1d bonus to damage against your quarry if it is \unaware of every member of the hunting party.
            At 16th level, the damage bonus increases to \plus2d.

            \subcf{Cover Weaknesses}
            The accuracy bonus against your quarry is replaced with a \plus1 bonus to defenses against your quarry's attacks.
            \par At 10th level, your hunting party gains \glossterm{damage reduction} equal to half your level against your quarry's attacks.
            At 16th level, the damage reduction increases to be equal to your level.

            \subcf{Decoy}
            If you \glossterm{threaten} your quarry, it takes a \minus2 penalty to accuracy on attacks against members of your \glossterm{hunting party} other than you.
            \par At 10th level, this penalty increases to \minus3.
            At 16th level, this penalty increases to \minus4.

            \subcf{Lifeseal}[Magical]
            As long as your quarry is \glossterm{threatened} by at least two members of your \glossterm{hunting party}, it cannot regain hit points.
            \par At 10th level, this effect instead applies if the target is within \rngmed range of at two members of your hunting party.
            At 16th level, this effect instead applies if your quarry is within \rnglong range of any member of your hunting party.

            \subcf{Martial Suppression}
            As long as your quarry is \glossterm{threatened} by at least two members of your \glossterm{hunting party}, it takes a \minus1 penalty to accuracy with \glossterm{physical attacks}.
            \par At 10th level, the penalty increases to \minus2 if your quarry is threatened by at least two members of your hunting party.
            At 16th level, the penalty increases to \minus3 if your quarry is threatened by at least three members of your hunting party.

            \subcf{Merciless}
            Whenever your \glossterm{hunting party} deals damage to your quarry, any damage in excess of its remaining hit points is dealt as \glossterm{vital damage}.
            \par At 10th level, any vital damage dealt in this way is doubled.
            At 16th level, the accuracy bonus against your quarry is increased to \plus2 if it is \glossterm{bloodied}.

            \subcf{Mystic Guidance}
            The accuracy bonus from your \textit{quarry} ability applies to all attacks your \glossterm{hunting party} makes against your quarry, instead of only to \glossterm{physical attacks}.
            \par At 10th level, your hunting party gains a \plus1 bonus to Fortitude, Reflex, and Mental defenses against attacks from your quarry.
            At 16th level, the accuracy bonus against your quarry is increased to \plus2.

            \subcf{Mystic Suppression}
            As long as your quarry is \glossterm{threatened} by at least two members of your \glossterm{hunting party}, it takes a \minus1 penalty to accuracy with \glossterm{magical attacks}.
            \par At 10th level, the penalty increases to \minus2 if your quarry is threatened by at least two members of your hunting party.
            At 16th level, the penalty increases to \minus3 if your quarry is threatened by at least three members of your hunting party.

            \subcf{Solo Hunter}
            Your hunting party other than you gains no benefit from your \textit{quarry} ability.
            In exchange, you gain a \plus1 bonus to defenses against your quarry.
            \par At 10th level, the accuracy bonus against your quarry increases to \plus2.
            At 16th level, the defense bonus increases to \plus2.

            \subcf{Swarm Hunter}
            When you use your \textit{quarry} ability, you can target any number of creatures.
            \par At 10th level, you gain a \plus1 bonus to \glossterm{overwhelm resistance}.
            At 16th level, the bonus to overwhelm resistance applies to all members of your \glossterm{hunting party}.

            \subcf{Unerring}
            Your \glossterm{hunting party} rolls twice and takes the better result for miss chances on attacks against your quarry, such as from \glossterm{active cover}.
            \par At 10th level, your hunting party also ignores the defense bonus from \glossterm{cover} on attacks against your quarry.
            At 16th level, your hunting party ignores all miss chances on attacks against your quarry.

            \subcf{Wolfpack}
            As long as your quarry is \glossterm{threatened} by at least two members of your \glossterm{hunting party}, it moves at half speed.
            \par At 10th level, the quarry is \immobilized instead of moving at half speed.
            At 16th level, the accuracy bonus against your quarry increases to \plus2 if it is threatened by at least two members of your hunting party.
        }

        \cf{Rgr}[7]{Hunting Lore} You gain two extra skill points.

        \cf*{Rgr}[10]{Hunting Style}
        You learn an additional \textit{hunting style}.

        \cf{Rgr}[13]{Fluid Style}
        At the start of each round, you can change which \textit{hunting style} you are using.
        This does not cost an \glossterm{action point} or change the target of your \textit{quarry} ability.

        \cf*{Rgr}[16]{Hunting Style}
        You learn an additional \textit{hunting style}.

        \cf{Rgr}[19]{Dual Quarry} You can attune to two separate uses of your \textit{quarry} ability simultaneously.
        You can use different \textit{hunting style} abilities with each \textit{quarry}.

\newpage
\section{Rogue}\label{Rogue}
    \begin{dtable}
        \lcaption{Rogue Progression}
        \begin{dtabularx}{\columnwidth}{>{\ccol}p{\levelcol} >{\lcol}X}
            \tb{Level} & \tb{Abilities}
            \\\bottomrule
            \nth{1}  & Combat trick, skill lore, sneak attack
            \\ \nth{2}  & Assassinate
            \\ \nth{3}  & Uncanny dodge
            \\ \nth{4}  & Skill exemplar
            \\ \nth{5}  & Stealth lore
            \\ \nth{6}  & Combat trick
            \\ \nth{7}  & Skill lore
            \\ \nth{8}  & Darkstalker
            \\ \nth{9}  & Evasion
            \\ \nth{10} & Greater skill exemplar
            \\ \nth{11} & Greater sneak attack
            \\ \nth{12} & Combat trick
            \\ \nth{13} & Skillful defense
            \\ \nth{14} & Rapid assassination
            \\ \nth{15} & Greater uncanny dodge
            \\ \nth{16} & Supreme skill exemplar
            \\ \nth{17} & Greater darkstaler
            \\ \nth{18} & Combat trick
            \\ \nth{19} & Legendary fortune
            \\ \nth{20} &
        \end{dtabularx}
    \end{dtable}

    \classbasics{Alignment} Any.

    \classbasics{Archetypes} Rogues have the Assassin, Jack of All Trades, and Scoundrel \glossterm{archetypes}.

    \subsection{Basic Class Abilities}
        If you are a rogue, you gain the following abilities.

        \cf{Rog}{Defenses}
        You gain the following bonuses to your \glossterm{defenses}: \plus3 Fortitude, \plus5 Reflex, \plus4 Mental.

        \cf{Rog}{Skills}
        You gain 8 \glossterm{skill points}.
        In addition, you have the following \glossterm{class skills}:
        \begin{itemize}
            \item \subparhead{Strength} Climb, Jump, Swim.
            \item \subparhead{Dexterity} Acrobatics, Escape Artist, Sleight of Hand, Stealth.
            \item \subparhead{Intelligence} Craft, Deduction, Devices, Disguise, Knowledge (dungeoneering, local), Linguistics.
            \item \subparhead{Perception} Awareness, Sense Motive.
            \item \subparhead{Other} Bluff, Intimidate, Perform, Persuasion, Profession.
        \end{itemize}

        \cf{Rog}{Weapon and Armor Proficiencies}
        Rogues are proficient with simple weapons, any two other weapon groups, light armor, and bucklers.
        They are also proficient with saps.

    \subsection{Assassin}
        This archetype improves your stealth and combat prowess, especially against unaware or distracted foes.

        % Situational +d bonuses are the worst kind of effect - redesign?
        \cf{Rog}{Sneak Attack} You gain a \plus1d bonus to damage with attacks against creatures who are unable to defend themselves effectively.
        This applies against creatures who are \unaware, \defenseless, or \glossterm{overwhelmed}.

        You must be within \rngclose range of a creature to gain this damage bonus.
        In addition, you do not gain this damage bonus against creatures who are immune to \glossterm{critical hits} or who lack a discernible body structure, such as oozes.

        \cf{Rog}[2]{Assassinate} As a standard action, you can use the \textit{assassinate} ability.
        \begin{ability}{Assassinate}
            You study a creature within \rnglong range, finding weak points you can take advantage of.
            Until the end of the next round, if you make a melee \glossterm{strike} against the target while it is \unaware, your attack deals maximum damage.
        \end{ability}

        \cf{Rog}[5]{Stealth Lore} You gain two extra skill points.

        \cf*{Rog}[8]{Darkstalker} As a standard action, you can spend an \glossterm{action point} to use the \textit{darkstalker} ability.
        \begin{ability}{Darkstalker}[\glossterm{Attune}]
            You become completely undetectable by your choice of one of the following senses:
            \begin{itemize}
                \item Blindsense and blindsight
                \item Darkvision
                \item Scent
                \item Tremorsense and tremorsight
            \end{itemize}
        \end{ability}

        \cf{Rog}[11]{Greater Sneak Attack}
        The damage bonus from your \textit{sneak attack} ability increases to \plus2d.

        \cf{Rog}[14]{Rapid Assassination} You can use your \textit{assassinate} ability as a \glossterm{minor action}.

        \cf{Rog}[17]{Greater Darkstalker} When you use your \textit{darkstalker} ability, you become undetectable by all of the listed senses, not just one.

    \subsection{Jack of All Trades}
        This archetype improves your skills.

        \cf{Rog}{Skill Lore} You gain three extra skill points.

        \cf{Rog}[4]{Skill Exemplar} You gain a \plus1 bonus to all skills.

        \cf*{Rog}[7]{Skill Lore} You gain three extra skill points.

        \cf{Rog}[10]{Greater Skill Exemplar} The skill bonus from your \textit{skill exemplar} ability increases to \plus2.

        \cf{Rog}[13]{Skillful Defense} You gain a \plus1 bonus to all defenses.

        \cf{Rog}[16]{Supreme Skill Exemplar} The skill bonus from your \textit{skill exemplar} ability increases to \plus3.

        \cf{Rog}[19]{Legendary Fortune} Whenever you make a skill attack or check, you roll two dice and take either result.

    \subsection{Scoundrel}
        This archetype grants you abilities to use in combat and improves your combat prowess.

        \cf{Rog}{Combat Trick}
        You can confuse and confound your foes in combat.
        Choose two \textit{combat tricks} from the list below.
        As a standard action, you can spend an \glossterm{action point} to use a \textit{combat trick} ability.
        {
            \begin{ability}{Certain Strike}
                Make a \glossterm{strike} with a \plus2 bonus to accuracy.

                At 6th level, the accuracy bonus increases to \plus3.
                At 12th level, the accuracy bonus increases to \plus4.
                At 18th level, the accuracy bonus increases to \plus5.
            \end{ability}

            \begin{ability}{Counterattack}
                Make a \glossterm{strike}.
                If the target attacked you in the same phase, you gain a \plus2 bonus to accuracy and a \plus2d bonus to damage.

                At 6th level, if the target did not attack you in the same phase, you regain the action point spent to use this ability.
                At 12th level, the accuracy bonus increases to \plus3 and the damage bonus increases to \plus3d.
                At 18th level, the accuracy bonus increases to \plus4 and the damage bonus increases to \plus4d.
            \end{ability}

            \begin{ability}{Daunting Blow}[\glossterm{Mind}]
                Make a \glossterm{strike}.
                In addition to the strike's normal effects, you also compare the attack result against the target's Mental defense.
                \hit As a \glossterm{condition}, the target suffers a \minus3 penalty to defenses against your attacks.
                \crit As above, except that the penalty is increased to \minus6.

                At 6th level, you gain a \plus1 bonus to accuracy on the strike.
                At 12th level, the accuracy bonus increases to \plus2.
                At 18th level, the accuracy bonus increases to \plus3.
            \end{ability}

            \begin{ability}{Gut Punch}
                Make a \glossterm{strike} with a bludgeoning weapon.
                The attack is made against Fortitude defense instead of Armor defense.
                If the target takes damage from the strike, it is \sickened as a \glossterm{condition}.

                At 6th level, you gain a \plus1 bonus to accuracy with the strike.
                At 12th level, you gain a \plus1d bonus to damage with the strike.
                At 18th level, the accuracy bonus increases to \plus2.
            \end{ability}

            \begin{ability}{Hamstring}
                Make a \glossterm{strike} with a slashing or piercing weapon.
                In addition to the strike's normal effects, you also compare the attack result against the target's Reflex defense.
                \hit The target is \glossterm{slowed} as a \glossterm{condition}.
                \crit The target is \glossterm{immobilized} as a \glossterm{condition}.

                At 6th level, you gain a \plus1 bonus to accuracy on the strike.
                At 12th level, the accuracy bonus increases to \plus2.
                At 18th level, the accuracy bonus increases to \plus3.
            \end{ability}

            \begin{ability}{Head Shot}[\glossterm{Mind}]
                Make a \glossterm{strike} with a bludgeoning weapon.
                In addition to the strike's normal effects, you also compare the attack result against the target's Mental defense.
                \hit The target is \glossterm{dazed} as a \glossterm{condition}.
                \crit The target is \glossterm{stunned} as a \glossterm{condition}.

                At 6th level, you gain a \plus1 bonus to accuracy on the strike.
                At 12th level, the accuracy bonus increases to \plus2.
                At 18th level, the accuracy bonus increases to \plus3.
            \end{ability}

            \begin{ability}{Penetrating Strike}
                Make a \glossterm{strike} with a piercing weapon.
                The attack is made against the target's Reflex defense instead of its Armor defense.

                At 6th level, if you miss with the strike, you regain the action point spent to use this ability.
                At 12th level, you gain a \plus1d bonus to damage with the strike.
                At 18th level, the damage bonus increases to \plus2d.
            \end{ability}

            \begin{ability}{Potent Maneuver}
                You use a \glossterm{combat maneuver} with a \plus3 bonus to accuracy.

                At 6th level, the accuracy bonus increases to \plus4.
                At 12th level, the accuracy bonus increases to \plus5.
                At 18th level, the accuracy bonus increases to \plus6.
            \end{ability}

            \begin{ability}{Rapid Assault}
                Make a \glossterm{strike} against a creature.
                If you use this ability during the \glossterm{action phase}, you can make another strike during the \glossterm{delayed action phase}.
                You take a \minus2 penalty to accuracy on both strikes.

                At 6th level, if you missed all of your targets, you regain the action point spent to use this ability.
                At 12th level, you gain a \plus1d bonus to damage with both strikes.
                At 18th level, the damage bonus increases to \plus2d.
            \end{ability}

            \begin{ability}{Strip the Flesh}
                Make a \glossterm{strike} with a slashing weapon.
                At the end of the current phase, if you hit with the strike and the target is not \glossterm{bloodied}, it takes additional damage equal to the damage you dealt with the strike.

                At 6th level, if you hit with the strike, the target continues taking the same damage at the end of each \glossterm{action phase} until it becomes \glossterm{bloodied}.
                This is a \glossterm{condition}, and can be removed by abilities that remove conditions.
                At 12th level, you gain a \plus1d bonus to damage with the strike.
                At 18th level, the damage bonus increases to \plus2d.
            \end{ability}

            \begin{ability}{Sweeping Strike}
                Make a melee \glossterm{strike} with a slashing or bludgeoning weapon.
                The strike targets each of up to three creatures or objects you \glossterm{threaten}.
                You take a \minus1d penalty to \glossterm{strike damage} with the strike.

                At 6th level, if you missed all of your targets, you regain the action point spent to use this ability.
                At 12th level, the damage penalty is removed.
                At 18th level, you gain a \plus1d bonus to damage with the strike.
            \end{ability}
        }

        \cf{Rog}[3]{Uncanny Dodge} You gain a \plus2 bonus to Reflex defense and \glossterm{initiative} checks.
        In addition, you are not \unaware when attacked by surprise.

        \cf*{Rog}[6]{Combat Trick}
        You learn an additional \textit{combat trick}.

        \cf{Rog}[9]{Evasion}
        You may use your Reflex defense to resist any attack from an ability that affects an area, even if that attack would normally be made against a different defense.

        \cf*{Rog}[12]{Combat Trick}
        You learn an additional \textit{combat trick}.

        \cf{Rog}[15]{Greater Uncanny Dodge}
        The bonuses from your \textit{uncanny dodge} ability increase to \plus3.
        In addition, you gain a \plus1 bonus to \glossterm{overwhelm resistance}.

        \cf*{Rog}[18]{Combat Trick}
        You learn an additional \textit{combat trick}.

\newpage
\section{Warlock}\label{Warlock}
    \begin{dtable}
        \lcaption{Warlock Progression}
        \begin{dtabularx}{\columnwidth}{>{\ccol}p{\levelcol} c >{\lcol}X}
            \tb{Level} & \tb{Spells} & \tb{Abilities}
            \\\bottomrule
            \nth{1}  & 2 & Infernal pact, malevolent boon, spells, whispers of the lost
            \\ \nth{2}  & 3 & Spell knowledge
            \\ \nth{3}  & 3 & Armor tolerance
            \\ \nth{4}  & 3 & Pact augment
            \\ \nth{5}  & 3 & Abyssal lore
            \\ \nth{6}  & 3 & Augments, greater malevolent boon
            \\ \nth{7}  & 3 & Greater whispers of the lost
            \\ \nth{8}  & 3 & Pact augment
            \\ \nth{9}  & 3 & Abyssal lore
            \\ \nth{10} & 3 & Augment, supreme malevolent boon
            \\ \nth{11} & 3 & Greater armor tolerance
            \\ \nth{12} & 3 & Pact augment
            \\ \nth{13} & 3 & Supreme whispers of the lost
            \\ \nth{14} & 3 & Augment, infernal power
            \\ \nth{15} & 3 & Abyssal defense
            \\ \nth{16} & 3 & Pact augment
            \\ \nth{17} & 3 & Supreme armor tolerance
            \\ \nth{18} & 3 & Dual pact
            \\ \nth{19} & 3 &
            \\ \nth{20} & 3 &
        \end{dtabularx}
    \end{dtable}

    \classbasics{Alignment} Any.

    \classbasics{Archetypes} Warlocks have the Arcane Spellcasting, Blessings of the Abyss, and Pact Magic \glossterm{archetypes}.

    \subsection{Basic Class Abilities}
        If you are a warlock, you gain the following abilities.

        \cf{War}{Defenses}
        You gain the following bonuses to your \glossterm{defenses}: \plus4 Fortitude, \plus3 Reflex, \plus5 Mental.

        \cf{War}{Skills}
        You gain 4 \glossterm{skill points}.
        In addition, you have the following \glossterm{class skills}:
        \begin{itemize}
            \item \subparhead{Dexterity} Ride.
            \item \subparhead{Intelligence} Craft, Deduction, Disguise, Knowledge (arcana, planes, religion), Linguistics.
            \item \subparhead{Perception} Awareness, Sense Motive, Spellcraft.
            \item \subparhead{Other} Bluff, Intimidate, Persuasion, Profession.
        \end{itemize}

        \cf{War}{Weapon and Armor Proficiencies}
        You are proficient with simple weapons, any two other weapon groups, light and medium armor, and shields.

        \cf{War}{Infernal Pact}
        To become a warlock, you must make a pact with a powerful demon or devil.
        You must make a dark sacrifice, the details of which are subject to negotiation, and offer a part of your immortal soul.
        In exchange, you gain the powers of a warlock.
        The creature you make the pact with is called your soulkeeper.

        Offering your soul to an entity in this way grants it the ability to communicate with you in limited ways.
        This communication typically manifests as unnatural emotional urges or whispered voices audible only to you.

        Your pact specifies how much of your soul is granted to your soulkeeper, and the circumstances of the transfer.
        The most common arrangement is for a soulkeeper to gain possession of your soul immediately after you die.
        It will keep the soul for one decade per year of your life that you spend as a warlock.
        During that time, it will not prevent you from being resurrected.
        At the end of that time, if your soul remains intact, your soul will pass on to its intended afterlife.
        However, other arrangements are possible, and each warlock's pact can be unique.

        The longer you spend in an afterlife that is not your own, the more likely you are to lose your sense of self and become subsumed by the plane you are on.
        Only a soul of extraordinary strength can maintain its integrity after decades or centuries in the Abyss.
        Many warlocks seek power zealously while mortal to gain the mental fortitude necessary to keep their soul after death.

    \subsection{Arcane Spellcasting}
        This archetype grants you the ability to cast arcane spells.

        \cf{War}{Arcane Spells} 
        You can cast arcane spells.
        You learn two arcane spells from the arcane \glossterm{spell list} (see \pcref{Arcane Spells}).
        Your spellpower with arcane spells is equal to your level or your Willpower, whichever is higher.
        The maximum spell level you can cast is equal to half your level (minimum 1).

        To cast a spell, you must normally spend an \glossterm{action point}.
        Every spell can also be cast as a cantrip.
        Cantrips are weaker, but do not require action points to cast.

        \cf{War}{Arcane Spell Failure}
        Whenever you cast an arcane spell while using \glossterm{armor}, you must roll 1d10.
        If your result is less than or equal to your \glossterm{encumbrance}, you \glossterm{miscast} your spell.

        \cf{War}[6]{Augments}
        Choose two \glossterm{augments} (see \pcref{Augments}).
        You can apply those augments to arcane spells you cast and arcane rituals you perform.
        At 10th, 14th level, and 18th level, you learn an additional augment.

    \subsection{Blessings of the Abyss}
        This archetype grants you powers relating to the connection to the Abyss granted by your pact.

        \cf{War}{Whispers of the Lost} You hear the voices of souls lost to the Abyss, linked to you through your soulkeeper.
        Choose one of the following types of whispers that you hear.
        {
            \subcf{Mentoring Whispers} You hear the voice of a dead warlock whose soul is bound to the same soulkeeper.
            This provides no benefit immediately, but the warlock can sometimes provide good advice.

            \subcf{Sycophantic Whispers} You hear the voices of adoring souls who praise your talents and everything you do.
            You gain a \plus2 bonus to Mental defense, and you are immune to \glossterm{Fear} abilities.

            \subcf{Warning Whispers} You hear the voices of paranoid and fearful souls warning you of danger, both real and imagined.
            You gain a \plus2 bonus to Reflex defense, and you are not \unaware when attacked by surprise.

            \subcf{Whispers of the Mighty} Your soulkeeper forges the connection to your soul into a boon granted to any soul in the Abyss strong enough to claim it in battle.
            You hear the voice of whatever soul currently possesses the boon - until that soul is defeated.
            You gain a \plus2 bonus to Fortitude defense.
        }

        \cf{War}[3]{Armor Tolerance} You reduce your \glossterm{encumbrance} by 2 when determining your \glossterm{arcane spell failure}.

        \cf{War}[5]{Abyssal Lore} You gain two additional skill points.

        \cf{War}[7]{Greater Whispers of the Lost} You gain an additional ability depending on the voices you chose with your \textit{whispers of the lost} ability.
        {
            \subcf{Mentoring Whispers} If you can cast arcane spells, you learn an additional arcane spell.

            \subcf{Sycophantic Whispers} The defense bonus increases to \plus3.

            \subcf{Warning Whispers} You gain a \plus2 bonus to \glossterm{initiative} checks.

            \subcf{Whispers of the Mighty} You reduce your \glossterm{vital damage penalties} by an amount equal to half your level.
        }

        \cf*{War}[9]{Abyssal Lore} You gain two additional skill points.

        \cf{War}[11]{Greater Armor Tolerance} The penalty reduction from your \textit{armor tolerance} ability improves to 3.

        \cf{War}[13]{Supreme Whispers of the Lost} You gain an additional ability depending on the voices you chose with your \textit{whispers of the lost} ability.
        {
            \subcf{Mentoring Whispers} You gain a \plus1 bonus to \glossterm{accuracy} with arcane spells.

            \subcf{Sycophantic Whispers} The defense bonus increases to \plus4.

            \subcf{Warning Whispers} The initiative bonus increases to \plus4.

            \subcf{Whispers of the Mighty} The penalty reduction increases to be equal to your level.
        }

        \cf{War}[15]{Abyssal Defense} You gain a \plus1 bonus to Fortitude, Reflex, and Mental defenses.

        \cf{War}[17]{Supreme Armor Tolerance} The penalty reduction from your \textit{armor tolerance} ability improves to 4.

    \subsection{Pact Magic}
        This archetype improves your ability to cast spells with the power of your dark pact.
        You must be able to cast arcane spells to gain the abilities from this archetype.

        \cf{War}{Malevolent Boon}
        When you cast a spell that makes an attack roll, if you miss, you may regain the \glossterm{action point} spent to cast the spell (if any).
        If you are \glossterm{attuned} to the spell, you cannot regain an action point in this way.
        You can use this ability on spells that do not normally have attacks if you apply an \glossterm{augment} that makes an attack.
        You cannot use this ability when casting \glossterm{subspells}.

        \cf{War}[2]{Spell Knowledge}
        You learn an additional spell from the arcane \glossterm{spell list} (see \pcref{Arcane Spells}).

        \cf{War}[4]{Pact Augment} You learn one pact augment from the list below.
        Pact augments function like other \glossterm{augments} (see \pcref{Augments}), with the following exceptions.
        \begin{itemize}
            \item Pact augments cannot be applied to divine spells, nature spells, subspells, or rituals.
            \item To apply a pact augment to a spell, you must spend an \glossterm{action point}.
            \item You can only apply a single pact augment to a spell.
                You can apply that pact augment in addition to any other non-pact augments the spell has.
        \end{itemize}
        Your \glossterm{power} with pact augments is equal to your \glossterm{spellpower} with the spell you apply them to.
        {
            \augment{1}{Abyssal Flame} Non-physical damage dealt by the spell becomes life damage in place of its other damage types.
            In addition, you gain a \plus1d bonus to life damage dealt by the spell.
            This does not change any other aspects of the spell's effects.
            \par This augment can be applied to any spell that deals non-physical damage.

            \augment{1}{Dark Expansion} The spell's area is increased by one step, to a maximum of \areahuge.
            The steps are, in order: \areasmall, \areamed, \arealarge, and \areahuge.
            Normally, a Small or Medium line is 5 ft.\ wide, while a Large or Huge line is 10 ft.\ wide.
            A line used to define a wall does not have a width.
            \par This augment can be applied to any spell with an area that is one of the above areas.

            \augment{1}{Infernal Focus} You gain a \plus10 bonus to Concentration checks to cast the spell (see \pcref{Concentration}).
            In addition, you gain a \plus1 bonus to \glossterm{accuracy} with the spell.

            \augment{1}{Sickening} Make a \glossterm{power} vs. Fortitude attack against one creature targeted by the spell.
            On a hit, the target is \glossterm{sickened} as a \glossterm{condition}.
            \par This augment can be applied to any spell that targets at least one creature.

            \augment{1}{Soulrending} Whenever a creature is dealt damage by the spell, if it has no hit points remaining, it dies.
            \par This augment can be applied to any spell that deals damage.

            \augment{1}{Terrifying} Make a \glossterm{power} vs. Mental attack against one creature targeted by the spell.
            On a hit, the target is \glossterm{shaken} by you as a \glossterm{condition}.
            \par This augment can be applied to any spell that targets at least one creature.

            \augment{3}{Bloodreaving} Whenever a creature takes damage from the spell, you heal hit points equal to the damage dealt.
            \par This augment can be applied to any spell that deals damage.

            \augment{3}{Exsanguinating} Whenever a creature takes damage from the spell, it begins bleeding as a \glossterm{condition}.
            At the end of each \glossterm{action phase}, a bleeding creature takes life \glossterm{standard damage} \minus2d.
            \par This augment can be applied to any spell that deals damage.

            \augment{3}{Inevitable Doom} You gain a \plus2 bonus to accuracy with the spell.
            \par This augment can be applied to any spell that makes an attack.

            \augment{3}{Infernal Potency} You gain a \plus2d bonus to damage with the spell.
            \par This augment can be applied to any spell that deals damage based on a dice pool.

            \augment{5}{Dark Miasma} Whenever a creature takes damage from the spell, it becomes \glossterm{sickened} as a \glossterm{condition}.
            \par This augment can be applied to any spell that deals damage.

            \augment{5}{Unholy Terror} Whenever a creature takes damage from the spell, it becomes \glossterm{shaken} by you as a \glossterm{condition}.
            \par This augment can be applied to any spell that deals damage.

            \augment{7}{Greater Inevitable Doom} You gain a \plus4 bonus to accuracy with the spell.
            \par This augment can be applied to any spell that makes an attack.

            \augment{7}{Greater Infernal Potency} You gain a \plus4d bonus to damage with the spell.
            \par This augment can be applied to any spell that deals damage based on a dice pool.
        }

        \cf{War}[6]{Greater Malevolent Boon} You can cast spells that make attack rolls without spending \glossterm{action points}.
        You cannot \glossterm{attune} to a spell you cast in this way.
        In addition, you cannot use this ability when casting \glossterm{subspells}.

        \cf*{War}[8]{Pact Augment} You learn an additional \textit{pact augment}.

        \cf{War}[10]{Supreme Malevolent Boon} You can cast all spells without spending \glossterm{action points}.
        You cannot \glossterm{attune} to a spell you cast in this way.
        In addition, you cannot use this ability when casting \glossterm{subspells}.

        \cf*{War}[12]{Pact Augment} You learn an additional \textit{pact augment}.

        \cf{War}[14]{Infernal Power} You gain a \plus1 bonus to \glossterm{accuracy} with spells.

        \cf*{War}[16]{Pact Augment} You learn an additional \textit{pact augment}.

        \cf{War}[18]{Dual Pact} You can apply two different \textit{pact augments} to the same spell, paying separate action points for each.
