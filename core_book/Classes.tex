\chapter{Classes}\label{Classes}

Your character's class represents the things your character has chosen to train in.
This choice determines a great deal about your character's abilities.

\section{How Classes Work}
    When you first create a character, you choose a class.
    You gain all abilities granted by the \glossterm{archetypes} of your chosen class at the levels indicated in the archetype's description (see Archetypes, below).
    As you gain levels, you gain more abilities from your class.

    \subsection{Archetypes}\label{Archetypes}
        Each class has three or four \glossterm{archetypes}.
        An archetype is a collection of thematically related class abilities.
        For examples, barbarians have the Battlerager archetype, which grants abilities related to flying into a rage in combat.

        You choose three of the archetypes associated with your class at 1st level.
        You gain the abilities from those three archetypes, and you gain no abilities from any additional archetypes the class has.

        \subsubsection{Archetype Ranks}\label{Archetype Ranks}
            You have an \glossterm{archetype rank} associated with each archetype you have.
            Each ability from an archetype has a minimum rank required to gain the ability.

            At 1st level, you are Rank 1 in each archetype.
            This gives you the Rank 1 abilities from each of your archetypes.
            Every level thereafter, you increase your rank in one archetype of your choice.
            This gives you the abilities associated with that rank.

            Each \glossterm{archetype rank} has a minimum level, as shown on \trefnp{Archetype Ranks by Level}.
            This minimum level is included in each class table as a reminder.

            \begin{dtable}
                \lcaption{Archetype Ranks by Level}
                \begin{dtabularx}{\columnwidth}{l >{\lcol}X}
                    \tb{Archetype Rank} & \tb{Minimum Level} \tableheaderrule
                    1     & 1
                    \\ 2  & 2
                    \\ 3  & 5
                    \\ 4  & 8
                    \\ 5 & 11
                    \\ 6 & 14
                    \\ 7 & 17
                    \\ 8 & 20
                \end{dtabularx}
            \end{dtable}

        \subsubsection{Duplicate Archetypes}\label{Duplicate Archetypes}
            Some archetypes can be gained by multiple classes.
            For example, both clerics and paladins have the Divine Magic archetype.
            You cannot gain two archetypes with the same name, even if you can choose archetypes from multiple classes.

        \subsection{Multiclass Characters}\label{Multiclass Characters}
            You can spend two \glossterm{insight points} to become a \glossterm{multiclass} character (see \pcref{Insight Points}).
            If you do, choose a class other than your base class.
            You gain the following benefits relating to that class.
            \begin{itemize}
                \item You gain the \glossterm{class skills} of that class in addition to your existing \glossterm{class skills}.
                \item You can exchange any number of your basic class abilities, such as weapon proficiencies, for the corresponding abilities of that class.
                \item If that class has any class abilities which are not part of an archetype, such as a cleric's \textit{divine power}, you gain those abilities.
                \item You may gain one \glossterm{archetype} from your that class in place of one \glossterm{archetype} from your base class.
            \end{itemize}

            You may gain access to multiple classes in this way, spending two \glossterm{insight points} for each class.

\section{Class Introductions}

    There are nine classes in Rise.
    \begin{itemize}
        \item Barbarians are mighty warriors who draw power from their physical prowess.
        \item Clerics are divine spellcasters who draw power from their veneration of a deity.
        \item Druids are nature spellcasters who draw power from their veneration of the natural world.
        \item Fighters are highly disciplined warriors who excel in physical combat of any variety.
        \item Monks are agile masters of ``ki'' who hone their personal abilities to strike down foes and perform supernatural feats.
        \item Paladins are divinely empowered warriors embody a particular alignment.
        \item Rangers are skilled hunters who bridge the divide between nature and civilization.
        \item Rogues are exceptionally skillful characters known for their ability to strike at their foe's weak points in combat.
        \item Sorcerers are arcane spellcasters who draw power from their inherently magical nature.
            % \item Spellwarped wield a unique blend of martial skill and narrowly focused magical abilities.
        \item Warlocks are pact spellcasters who draw their power from a dark pact made with infernal creatures.
        \item Wizards are arcane spellcasters who study magic to unlock its powerful secrets.
    \end{itemize}

    \subsection{Class Description Format}
        Each class is described from the perspective of a member of that class, using ``you'' in the description.

        \parhead{Class Table}
        The class's table describes the special abilities a member of that class gains at each level, assuming they have all of that class's \glossterm{archetypes}.

        \parhead{Alignment}
        Some classes require specific alignments (see \pcref{Alignment}).
        Most classes allow characters of any alignment.

        \parhead{Skills}
        Each class has specific \glossterm{skills} that members of that class are typically good at (see \pcref{Skills}).
        These skills are called \glossterm{class skills}.
        It is easier to become \glossterm{mastered} in class skills than in other skills.
        For details, see \pcref{Skill Training}.

        \parhead{Defenses}
        Each class grants bonuses to specific defenses.

        \parhead{Weapon Proficiencies}
        This indicates the types of weapons that members of this class are proficient with.

        \parhead{Armor Proficiencies}
        This indicates the types of armor that members of this class are proficient with.

        \parhead{Other Special Abilities}
        Some classes have abilities shared by all members of the class that are not part of an archetype, such as a druid's \textit{druidic language} ability.

        \parhead{Archetypes}
        The abilities associated with each of the three archetypes the class has.

\newpage
\section{Barbarian}\label{Barbarian}
    \begin{dtable!*}
        \lcaption{Barbarian Progression}
        \begin{dtabularx}{\textwidth}{l l l >{\lcol}X >{\lcol}X >{\lcol}X >{\lcol}X}
            \tb{Rank} & \tb{Min Level} & \tb{Battleforged Resilience}   & \tb{Battlerager}       & \tb{Outland Savage}      & \tb{Primal Warrior}    \tableheaderrule
            1 & 1  & Battleforged endurance         & Rage                   & Exotic weaponry          & Primal maneuvers     \\
            2 & 2  & Toughened body                 & Seething anger         & Fast movement            & Primal lore          \\
            3 & 5  & Battle-scarred                 & Painless rage          & Savage precision         & Primal power         \\
            4 & 8  & Greater battleforged endurance & Enduring anger         & Savage instincts         & Primal maneuver      \\
            5 & 11 & Greater toughened body         & Powerful rage          & Greater fast movement    & Greater primal lore  \\
            6 & 14 & Greater battle-scarred         & Greater painless rage  & Greater savage precision & Greater primal power \\
            7 & 17 & Supreme battleforged endurance & Greater seething anger & Greater savage instincts & Primal maneuver      \\
            8 & 20 &                                & Titanic rage           &                          &                      \\
        \end{dtabularx}
    \end{dtable!*}

    Barbarians are primal warriors that draw power from the physical strength of their bodies.
    A typical barbarian is strong, fast, and durable - and a terrifying threat for their unfortunate enemies to behold.
    They are more inclined to chaos than law, and often have little respect for civilized society, though they may respect authorities with the power to back up their claims.
    They tend to forgo defense in favor of offense, trusting their physical durability to keep them in the fight.
    Barbarians have limited utility outside of combat, though their physical prowess often gives them high physical skills.

    \classbasics{Alignment} Any.

    \classbasics{Archetypes} Barbarians have the Battlerager, Battleforged Resilience, Outland Savage, and Primal Warrior \glossterm{archetypes}.

    \subsection{Basic Class Abilities}
        If you are a barbarian, you gain the following abilities.

        \cf{Bbn}{Defenses}
        You gain the following bonuses to your \glossterm{defenses}: \plus2 Armor, \plus6 Fortitude, \plus4 Reflex, \plus3 Mental.

        \cf{Bbn}{Skills}
        You have the following \glossterm{class skills}:
        \begin{itemize}
            \item \subparhead{Strength} Climb, Jump, Swim.
            \item \subparhead{Dexterity} Acrobatics, Escape Artist, Ride.
            \item \subparhead{Constitution} Endurance.
            \item \subparhead{Intelligence} Craft, Heal.
            \item \subparhead{Perception} Awareness, Creature Handling, Survival.
            \item \subparhead{Other} Deception, Intimidate, Persuasion, Profession.
        \end{itemize}

        \cf{Bbn}{Weapon Proficiencies} 
        You are proficient with simple weapons and any three other \glossterm{weapon groups}.

        \cf{Bbn}{Armor Proficiencies} 
        You are proficient with light and medium armor.

    \subsection{Battleforged Resilience}
        This archetype improves your durability in combat.

        \cf{Bbn}[1]{Battleforged Endurance} You gain an additional \glossterm{hit point}.

        \cf{Bbn}[2]{Toughened Body} You gain a \plus2 bonus to Fortitude defense.

        \cf{Bbn}[3]{Battle-Scarred} You gain a bonus equal to half your level to your \glossterm{resistances} against \glossterm{physical damage}.

        \cf{Bbn}[4]{Greater Battleforged Endurance} You gain an additional \glossterm{hit point}.

        \cf{Bbn}[5]{Greater Toughened Body} The bonus from your \textit{toughened body} ability increases to \plus4.

        \cf{Bbn}[6]{Greater Battle-Scarred}
        The bonus from your \textit{battle-scarred} ability increases to be equal to your level.

        \cf{Bbn}[7]{Supreme Battleforged Endurance} You gain an additional \glossterm{hit point}.

    \subsection{Battlerager}\label{Rage}
        This archetype grants you a devastating rage, improving your combat prowess.
        % Rank 7 summary:
        % +4 power with mundane abilities
        % +2x level to physical resistances
        % +4 intimidate
        % +4 Mental defense
        % -2 Armor and Reflex defenses

        \cf{Bbn}[1]{Rage} You can use the \textit{rage} ability as a \glossterm{free action}.
        For most barbarians, this represents entering a furious rage.
        Some barbarians instead enter a joyous battle trance or undergo a partial physical transformation into a more fearsome form.
        % TODO: prevent recovering this AP; may need to be Attunement again?
        \begin{attuneability}{Rage}[\glossterm{Emotion}, \glossterm{Attune} (self), \glossterm{Swift}]
            You gain the following benefits and drawbacks:
            \begin{itemize}
                \item You increase your maximum \glossterm{hit points} by two and regain two \glossterm{hit points}.
                    When this ability ends, you lose two \glossterm{hit points} and your maximum \glossterm{hit points} are restored to normal.
                \item You gain a \plus2 bonus to \glossterm{power} with \glossterm{mundane} abilities.
                \item You take a \minus2 penalty to Armor and Reflex defenses.
                \item You are unable to take \glossterm{standard actions} that do not cause you to make \glossterm{mundane} attacks.
                \item At the end of each round, if you did not make a \glossterm{mundane} attack that round, this ability ends.
                    If the ability ends this way, you regain the \glossterm{action point} you spent to attune to this ability the next time you take a \glossterm{short rest}.
            \end{itemize}
        \end{attuneability}

        \cf{Bbn}[2]{Seething Anger}
        You gain a \plus2 bonus to the \glossterm{Intimidate} skill and to Mental defense.

        \cf{Bbn}[3]{Painless Rage} Your \textit{rage} ability also grants you a bonus equal to half your \glossterm{power} to your \glossterm{resistances} against \glossterm{physical damage}.

        \cf{Bbn}[4]{Enduring Anger} You gain an additional \glossterm{hit point}.

        \cf{Bbn}[5]{Powerful Rage} The bonus to \glossterm{power} from your \textit{rage} ability increases to \plus4.

        \cf{Bbn}[6]{Greater Seething Anger}
        The bonuses from your \textit{seething anger} ability increase to \plus4.

        \cf{Bbn}[7]{Greater Painless Rage} The bonus to \glossterm{resistances} from your \textit{painless rage} ability increases to be equal to your \glossterm{power}.

        \cf{Bbn}[8]{Titanic Rage}
        When you use your \textit{rage} ability, you can grow by one \glossterm{size category}.

    \subsection{Outland Savage}
        This archetype improves your mobility and combat prowess with direct, brutal abilities.

        \cf{Bbn}[1]{Exotic Weaponry} You gain proficiency with \glossterm{exotic weapons} from all weapon groups you are proficient with.

        \cf{Bbn}[2]{Fast Movement} You gain a \plus10 foot bonus to your \glossterm{base speed}.

        \cf{Bbn}[3]{Savage Precision} You gain a \plus1 bonus to \glossterm{accuracy} with the \textit{dirty trick}, \textit{disarm}, \textit{grapple}, \textit{overrun}, \textit{shove}, and \textit{trip} abilities (see \pcref{Special Combat Abilities}).
        In addition, you gain a \plus1 bonus to \glossterm{accuracy} with grapple actions (see \pcref{Grapple Actions}).

        \cf{Bbn}[4]{Savage Instincts} You gain a \plus2 bonus to \glossterm{initiative} checks and Reflex defense.

        \cf{Bbn}[5]{Greater Fast Movement} The speed bonus from your \textit{fast movement} ability increases to \plus20 feet.

        \cf{Bbn}[6]{Greater Savage Precision} The bonuses from your \textit{savage precision} ability increase to \plus2.

        \cf{Bbn}[7]{Greater Savage Instints} The bonuses from your \textit{savage instincts} ability increase to \plus4.

    \subsection{Primal Warrior}
        This archetype grants you abilities to use in combat and improves your physical skills.
        % Rank 7 summary:
        % Standard active abilities
        % +2 skill points
        % +2 Str/Dex checks

        \cf{Bbn}[1]{Primal Maneuvers}
        You can channel your primal energy into ferocious attacks.
        You learn two \glossterm{maneuvers} from the primal maneuver list (see \pcref{Primal Maneuvers}).
        You can also spend \glossterm{insight points} to learn one additional \glossterm{maneuver} per \glossterm{insight point}.
        As a \glossterm{standard action}, you can use any \glossterm{maneuver} you know.

        \cf{Bbn}[2]{Primal Lore} You gain two additional \glossterm{skill points}.
        In addition, you gain a \plus1 bonus to Strength-based \glossterm{checks}.

        \cf{Bbn}[3]{Primal Power} You gain a \plus1 bonus to \glossterm{power} with \glossterm{mundane} abilities.

        \cf{Bbn}[4]{Primal Maneuver}
        You learn an additional primal \glossterm{maneuver} (see \pcref{Primal Maneuvers}).

        \cf{Bbn}[5]{Greater Primal Lore} The bonus from your \textit{primal prowess} ability increases to \plus3.

        \cf{Bbn}[6]{Greater Primal Power} The bonus from your \textit{primal poer} ability increases to \plus2.

        \cf*{Bbn}[7]{Primal Maneuver}
        You learn an additional primal \glossterm{maneuver} (see \pcref{Primal Maneuvers}).

    \subsection{Ex-Barbarians}
        If you become lawful, you cannot use your \textit{rage} ability.
        You retain all of your other class abilities.
        If you stop being lawful, you can use your \textit{rage} ability once more.

\newpage
\section{Cleric}\label{Cleric}
    \begin{dtable!*}
        \lcaption{Cleric Progression}
        \begin{dtabularx}{\textwidth}{l l >{\lcol}X >{\lcol}X >{\lcol}X >{\lcol}X}
            \tb{Rank} & \tb{Min Level} & \tb{Divine Magic} & \tb{Divine Spell Mastery}   & \tb{Domain Influence}  & \tb{Healer} \tableheaderrule
            1 & 1  & Spellcasting   & Mystic insight              & Domains, domain gift  & Divine healing         \\
            2 & 2  & Mystic sphere  & Battle priest               & Domain gift           & Vital healer           \\
            3 & 5  & Spell rank (3) & Wellspring of power         & Domain aspect         & Divine cleanse         \\
            4 & 8  & Spell rank (4) & Mystic insight              & Domain aspect         & Greater divine healing \\
            5 & 11 & Spell rank (5) & Greater battle priest       & Domain essence        & Greater vital healer   \\
            6 & 14 & Spell rank (6) & Greater wellspring of power & Domain essence        & Greater divine cleanse \\
            7 & 17 & Spell rank (7) & Mystic insight              & Miracle               & Supreme divine healing \\
            8 & 20 & Spell rank (8) &                             & Domain masteries     \\
        \end{dtabularx}
    \end{dtable!*}

    Clerics are divine spellcasters that draw power from their worship of a specific deity.
    Divine magic tends to be more subtle than other forms of magic, and focuses more on aiding yourself and allies than destroy foes with flashy bursts.
    A cleric's deity often grants them influence over specific aspects of the world that the deity has purview over, aligning the cleric's power more closely with that of the deity.
    Some clerics pursue a path of healing, allowing them to keep themselves and their allies healthy through dangerous trials.
    These clerics are commonly found in temples of their gods, tending to the needs of their fellow followers or the wider community.

    \classbasics{Alignment} Your alignment must be within one step of your deity's (that is, it may be one step away on either the lawful-chaotic axis or the good-evil axis, but not both).

    \classbasics{Archetypes} Clerics have the Divine Magic, Domain Influence, and Divine Spell Mastery \glossterm{archetypes}.

    \subsection{Basic Class Abilities}
        If you are a cleric, you gain the following abilities.

        \cf{Clr}{Defenses}
        You gain the following bonuses to your \glossterm{defenses}: \plus2 Armor, \plus4 Fortitude, \plus3 Reflex, \plus6 Mental.

        \cf{Clr}{Skills}
        You have the following \glossterm{class skills}:
        \begin{itemize}
            \item \subparhead{Intelligence} Craft, Deduction, Heal, Knowledge (arcana, local, religion, the planes), Linguistics.
            \item \subparhead{Perception} Awareness, Sense Motive, Spellcraft.
            \item \subparhead{Other} Deception, Intimidate, Persuasion, Profession.
        \end{itemize}

        \cf{Bbn}{Weapon Proficiencies} 
        You are proficient with simple weapons and any two other \glossterm{weapon groups}.

        \cf{Bbn}{Armor Proficiencies} 
        You are proficient with light and medium armor.

        \cf{Clr}{Deity}
        You must worship a specific deity to be a cleric.
        Deities and their associated domains are listed in \trefnp{Deities}.

        \begin{dtable!*}
            \lcaption{Deities}
            \begin{dtabularx}{\textwidth}{X l X}
                \tb{Deity} & \tb{Alignment} & \tb{Domains} \tableheaderrule
                Guftas, horse god of justice          & Lawful good     & Good, Law, Strength, Travel         \\
                Lucied, paladin god of justice        & Lawful good     & Destruction, Good, Protection, War  \\
                Simor, fighter god of protection      & Lawful good     & Good, Protection, Strength, War     \\
                %Pabst, dwarf god of drink             & Neutral good & Good, Life, Strength, Wild \\
                Rucks, monk god of pragmatism         & Neutral good    & Good, Law, Protection, Travel       \\
                Vanya, centaur god of nature          & Neutral good    & Good, Strength, Travel, Wild        \\
                Brushtwig, pixie god of creativity    & Chaotic good    & Chaos, Good, Trickery, Wild         \\
                Chavi, god of stories                 & Chaotic good    & Chaos, Knowledge, Trickery          \\
                Ivan Ivanovitch, bear god of strength & Chaotic good    & Chaos, Strength, War, Wild          \\
                Krunch, barbarian god of destruction  & Chaotic good    & Destruction, Good, Strength, War    \\
                Sir Cakes, dwarf god of freedom       & Chaotic good    & Chaos, Good, Strength               \\
                Raphael, monk god of retribution      & Lawful neutral  & Death, Law, Protection, Travel      \\
                Declan, god of fire                   & True neutral    & Destruction, Fire, Knowledge, Magic \\
                Kurai, shaman god of nature           & True neutral    & Air, Earth, Fire, Water             \\
                %Amanita, druid god of decay           & Chaotic neutral & Chaos, Destruction, Life, Wild \\
                %Antimony, elf god of necromancy       & Chaotic neutral & Death, Knowledge, Life, Magic \\
                Clockwork, elf god of time            & Chaotic neutral & Chaos, Magic, Trickery, Travel      \\
                %Lord Khallus, fighter god of pride    & Chaotic neutral & Chaos, Strength, War \\
                %Celeano, sorcerer god of deception    & Chaotic neutral & Chaos, Magic, Protection, Trickery \\
                Murdoc, god of mercenaries            & Chaotic neutral & Destruction, Knowledge, Travel, War \\
                Ribo, halfling god of trickery        & Chaotic neutral & Chaos, Trickery, Water              \\
                Tak, orc god of war                   & Lawful evil     & Law, Strength, Trickery, War        \\
                Theodolus, sorcerer god of ambition   & Neutral evil    & Evil, Knowledge, Magic, Trickery    \\
                Daeghul, demon god of slaughter       & Chaotic evil    & Destruction, Evil, Magic, War       \\
            \end{dtabularx}
        \end{dtable!*}

    \subsection{Divine Magic}
        This archetype grants you the ability to cast divine spells.

        \cf{Clr}[1]{Spellcasting}[Magical]
        Your deity grants you the ability to use divine magic.
        You gain access to one divine \glossterm{mystic sphere} (see \pcref{Divine Mystic Spheres}).
        You may spend \glossterm{insight points} to learn one additional \glossterm{mystic sphere} per two \glossterm{insight points}.
        Each \glossterm{mystic sphere} has a set of \glossterm{spells} associated with it.

        You automatically learn all \glossterm{cantrips} from any mystic sphere you have access to.
        In addition, you learn two rank 1 \glossterm{spells} from your chosen \glossterm{mystic sphere}.
        You can also spend \glossterm{insight points} to learn one additional divine spell per \glossterm{insight point}.
        Unless otherwise noted in a spell's description, casting a spell requires a \glossterm{standard action}.

        When you gain access to a new \glossterm{mystic sphere} or spell \glossterm{rank},
            you can exchange any number of spells you know for other spells,
            including spells of the higher rank.

        Divine spells require \glossterm{verbal components} to cast (see \pcref{Casting Components}).
        For details about mystic spheres and casting spells, see \pcref{Spell and Ritual Mechanics}.

        \cf{Clr}[2]{Divine Focus} You gain access to an additional divine \glossterm{mystic sphere}, including all \glossterm{cantrips} from that sphere.
        In addition, you learn an additional divine \glossterm{spell}.

        \cf{Clr}[3]{Spell Rank}[Magical] You become a rank 3 divine spellcaster.
        This can improve the effectiveness of your spells and give you access to spells that require a minimum rank of 3.

        \cf*{Clr}[4]{Spell Rank}[Magical] You become a rank 4 divine spellcaster.
        This can improve the effectiveness of your spells and give you access to spells that require a minimum rank of 4.

        \cf*{Clr}[5]{Spell Rank}[Magical] You become a rank 5 divine spellcaster.
        This can improve the effectiveness of your spells and give you access to spells that require a minimum rank of 5.

        \cf*{Clr}[6]{Spell Rank}[Magical] You become a rank 6 divine spellcaster.
        This can improve the effectiveness of your spells and give you access to spells that require a minimum rank of 6.

        \cf*{Clr}[7]{Spell Rank}[Magical] You become a rank 7 divine spellcaster.
        This can improve the effectiveness of your spells and give you access to spells that require a minimum rank of 7.

        \cf*{Clr}[8]{Spell Rank}[Magical] You become a rank 8 divine spellcaster.
        This can improve the effectiveness of your spells and give you access to spells that require a minimum rank of 8.

    \subsection{Divine Spell Mastery}
        This archetype improves the divine spells you cast.
        You must be able to cast divine spells to gain the abilities from this archetype.

        \cf{Clr}[1]{Mystic Insight}[Magical]
        You gain your choice of one of the following abilities.
        {
            \parhead{Focused Caster} You reduce your \glossterm{focus penalty} by 1.
                You cannot choose this ability multiple times.
            \parhead{Mystic Sphere Access} You gain access to an additional divine \glossterm{mystic sphere} (see \pcref{Divine Mystic Spheres}).
                You cannot choose this ability multiple times.
            \parhead{Rituals} You gain the ability to perform divine rituals to create unique magical effects (see \pcref{Rituals}).
                The maximum \glossterm{rank} of divine ritual you can learn or perform is equal to the maximum \glossterm{rank} of divine spell that you can cast.
                You cannot choose this ability multiple times.
            \parhead{Spell Known} You learn an additional divine \glossterm{spell} (see \pcref{Divine Mystic Spheres}).
                You can choose this ability multiple times, learning an additional spell each time.
            \parhead{Spell Power} Choose an divine \glossterm{spell} you know.
                You gain a \plus2 bonus to \glossterm{power} with that spell.
                You can choose this ability multiple times, choosing a different spell each time.
            \parhead{Spell Precision} Choose an divine \glossterm{spell} you know.
                You gain a \plus1 bonus to \glossterm{accuracy} with that spell.
                You can choose this ability multiple times, choosing a different spell each time.
        }

        \cf{Clr}[2]{Battle Priest} You reduce your \glossterm{focus penalty} by 1.

        \cf{Clr}[3]{Wellspring of Power}[Magical]
        You gain a \plus1 bonus to \glossterm{power} with \glossterm{magical} abilities.

        \cf*{Clr}[4]{Mystic Insight}[Magical]
        You gain an additional \textit{mystic insight} ability.

        \cf{Clr}[5]{Greater Battle Priest} The penalty reduction from your \textit{battle priest} ability increases to 2.

        \cf{Clr}[6]{Greater Wellspring of Power}[Magical]
        The bonus from your \textit{wellspring of power} ability increases to \plus2.

        \cf*{Clr}[7]{Mystic Insight}[Magical]
        You gain an additional \textit{mystic insight} ability.

    \subsection{Domain Influence}
        This archetype grants you divine influence over two domains of your choice.

        \cf{Clr}[1]{Domains}
        You choose two domains which represent your personal spiritual inclinations.
        You must choose your domains from among those your deity offers.
        The domains are listed below.

        \begin{itemize}
            \item{Air}
            \item{Chaos}
            \item{Death}
            \item{Destruction}
            \item{Earth}
            \item{Evil}
            \item{Fire}
            \item{Good}
            \item{Knowledge}
            \item{Law}
            \item{Life}
            \item{Magic}
            \item{Protection}
            \item{Strength}
            \item{Travel}
            \item{Trickery}
            \item{War}
            \item{Water}
            \item{Wild}
        \end{itemize}

        \cf{Clr}[1]{Domain Gift}[Magical]
        Each domain has a corresponding \textit{domain gift}.
        You gain the \textit{domain gift} for one of your domains (see \pcref{Cleric Domain Abilities}).

        \cf*{Clr}[2]{Domain Gift}[Magical]
        You gain the \textit{domain gift} for another one of your domains.

        \cf{Clr}[3]{Domain Aspect}[Magical]
        Each domain has a corresponding \textit{domain aspect}.
        You gain the \textit{domain aspect} for one of your domains (see \pcref{Cleric Domain Abilities}).

        \cf*{Clr}[4]{Domain Aspect}[Magical]
        You gain the \textit{domain aspect} for another one of your domains.

        \cf{Clr}[5]{Domain Essence}[Magical]
        Each domain has a corresponding \textit{domain essence}.
        You gain the \textit{domain essence} for one of your domains (see \pcref{Cleric Domain Abilities}).

        % This seems weak
        \cf*{Clr}[6]{Domain Essence}[Magical]
        You gain the \textit{domain essence} for another one of your domains.

        \cf{Clr}[7]{Miracle}[Magical]
        Once per week, you can request a miracle as a standard action.
        You mentally specify your request, and your deity fulfills that request in the manner it sees fit.
        This can emulate the effects of any spell or ritual, or have any other effect of a similar power level.
        If the deity has a direct interest in your situation, the miracle may be of even greater power.

        If you perform an extraordinary service for your deity, you can gain the ability to request an additional miracle that week.

        \cf{Clr}[8]{Domain Masteries}[Magical]
        Each domain has a corresponding \textit{domain mastery}.
        You gain the \textit{domain mastery} for both of your domains (see \pcref{Cleric Domain Abilities}).

    \subsection{Healer}
        This archetype grants you healing abilities.

        \cf{Clr}[1]{Mitigate Wound} You can use the \textit{mitigate wound} ability as a standard action.
        \begin{freeability}{Mitigate Wound}[\glossterm{Magical}]
            Choose yourself or a living \glossterm{ally} within \rngclose range.
            The target can roll again for its most recent \glossterm{wound roll} with a \plus2 bonus.
            It can decide whether to keep its original roll or use the new roll.
            In either case, the \glossterm{wound roll} for that \glossterm{vital wound} cannot be modified again.

            \rankline
            \rank{3} The bonus increases to \plus3.
            \rank{5} The range increases to \rnglong.
            \rank{7} The bonus increases to \plus4.
        \end{freeability}

        \cf{Clr}[2]{Healer's Grace} You gain a \plus1 bonus to all defenses.
        After you attack a living creature, you lose this bonus until the end of the next round.

        \cf{Clr}[3]{Divine Healing} You can use the \textit{divine healing} ability as a standard action.
        \begin{apability}{Divine Healing}[\glossterm{Magical}]
            You or a living \glossterm{ally} within \rngclose range removes a \glossterm{vital wound}.

            \rankline
            \rank{5} The target removes an additional \glossterm{vital wound}.
            \rank{7} The range increases to \rnglong.
        \end{apability}

        \cf{Clr}[4]{Reflexive Healing} When you suffer a \glossterm{vital wound} that makes you \glossterm{unconscious}, you automatically use your \textit{mitigate wound} ability on yourself at the end of the round.

        \cf{Clr}[5]{Greater Healer's Grace} The bonus from your \textit{healer's grace} ability increases to \plus2.

        \cf{Clr}[6]{Revivify} You can use the \textit{revivify} ability as a standard action.
        \begin{apability}{Revivify}
            Choose a dead creature within \rngclose range.
            If it was dead for no more than 1 minute, it is restored to life, as the \ritual{resurrection} ritual.
            After using this ability, you cannot use it for 24 hours.

            \rankline
            \rank{8} You can use this ability again after 1 hour instead of after 24 hours.
        \end{apability}

        % TODO: why 4? Ideally, this should correspond to some sort of ``half as strong'' benchmark.
        \cf{Clr}[7]{Wellspring of Health} When you use your \textit{divine healing} ability,
            if the target's level is at least four levels lower than your level,
            you regain the \glossterm{action point} spent to use the ability.

    \subsection{Cleric Domain Abilities}\label{Cleric Domain Abilities}
        These domain abilities can be granted by the \textit{domain influence} cleric archetype.
        All cleric domain abilities are \glossterm{magical} unless otherwise specified.

        \subsubsection{Air}
            If you choose this domain, you add the \sphere{aeromancy} \glossterm{mystic sphere} to your list of divine mystic spheres (see \pcref{Mystic Spheres}).
            In addition, you add the Jump skill to your \glossterm{class skill} list.

            \parhead{Gift} You gain a \plus4 \glossterm{magic bonus} to the Jump skill (see \pcref{Jump}).
            In addition, you gain a \plus1 bonus to Reflex defense.
            \parhead{Aspect} You gain a \glossterm{glide speed} equal to your \glossterm{base speed} (see \pcref{Gliding}).
            \parhead{Essence} You can use the \textit{speak with air} ability as a standard action.
            \begin{attuneability}{Speak with Air}[\glossterm{Attune} (self)]
                You can speak with and command air within a \areahuge \glossterm{zone} from your location.
                You can ask the air simple questions and understand its responses.
                If you command the air to perform a task, it will do so do the best of its ability until this effect ends.
                You cannot compel the air to move faster than 50 mph.

                After you use this ability on a particular area of air, you cannot use it again on that same area for 24 hours.
            \end{attuneability}
            \parhead{Mastery} As long as you are within 100 feet of the ground, you gain a \glossterm{fly speed} equal to your \glossterm{base speed} (see \pcref{Flying}).
            In addition, your \textit{aspect} from this domain has no time limit.

        \subsubsection{Chaos}
            If you choose this domain, you add the Deception skill to your \glossterm{class skill} list.

            \parhead{Gift} You are immune to \glossterm{Compulsion} effects.
            \parhead{Aspect} If you roll a 1 on an attack roll, it explodes (see \pcref{Exploding Attacks}).
            This does not affect bonus dice rolled for exploding attacks (see \pcref{Exploding Attacks}).
            \parhead{Essence} You can use the \textit{twist of fate} ability as a standard action.
            \begin{freeability}{Twist of Fate}
                An improbable event occurs within \rnglong range.
                You can specify in general terms what you want to happen, such as ``Make the bartender leave the bar''.
                You cannot control the exact nature of the event, though it always beneficial for you in some way.
                After using this ability, you cannot use it again for an hour.
            \end{freeability}
            \parhead{Mastery} You gain a \plus5 bonus to \glossterm{accuracy} with any attack roll that explodes (see \pcref{Exploding Attacks}).

        \subsubsection{Death}
            % Lame gift
            \parhead{Gift} You gain a \plus2 bonus to Fortitude defense.
            \parhead{Aspect} Whenever you kill a creature, you gain a \plus2 bonus to \glossterm{accuracy} until the end of the next round.
            \parhead{Essence} You can use the \textit{speak with dead} ability as a standard action.
            \begin{attuneability}{Speak with Dead}[\glossterm{Attune} (self)]
                Choose a corpse within \rngclose range.
                The corpse must have died no more than 24 hours ago.
                It regains a semblance of life, allowing you to speak with it as if it were the creature the corpse belonged to.
                The creature is able to refuse to speak with you, though you can attempt to persuade it to speak normally, and some creatures may be more willing to talk if they know they are already dead.
                The corpse must have an intact mouth to be able to speak.
                This ability ends if 24 hours have passed since the creature died.
            \end{attuneability}
            \parhead{Mastery} Whenever you cause a target to gain a \glossterm{vital wound}, it gains an additional \glossterm{vital wound}.

        \subsubsection{Destruction}
            % TODO: upgrade timing is a little weird
            \parhead{Gift} You can use the \textit{destructive attack} ability as a standard action.
            \begin{freeability}{Destructive Attack}
                Make a \glossterm{strike} with a \minus2 penalty to \glossterm{accuracy} and a \plus2d bonus to damage.

                \rankline
                \rank{3} The damage bonus increases to \plus3d.
                \rank{5} The damage bonus increases to \plus4d.
                \rank{7} The damage bonus increases to \plus5d.
            \end{freeability}
            \parhead{Aspect} Your abilities deal double damage to objects.
            \parhead{Essence} You can use the \textit{lay waste} ability as a standard action.
            \begin{freeability}{Lay Waste}
                You make an attack vs. Fortitude against all unattended objects in a \arealarge radius.
                You may freely exclude any number of 5-ft\. cubes from the area, as long as the resulting area is still contiguous.
                \hit For each target, if its \glossterm{damage resistance} against \glossterm{physical damage} is lower than your \glossterm{power}, it crumbles into a fine power and is irreparably \glossterm{broken}.
            \end{freeability}
            \parhead{Mastery} Your abilities deal triple damage to objects.

        \subsubsection{Earth}
            If you choose this domain, you add the \sphere{terramancy} \glossterm{mystic sphere} to your list of divine mystic spheres (see \pcref{Mystic Spheres}).

            \parhead{Gift} You gain a \plus2 bonus to Fortitude defense.
            \parhead{Aspect} You gain an additional \glossterm{hit point}.
            \parhead{Essence} You can use the \textit{speak with earth} ability as a standard action.
            \begin{attuneability}{Speak with Earth}[\glossterm{Attune} (self)]
                You can speak with earth within a \areahuge \glossterm{zone} from your location.
                You can ask the earth simple questions and understand its responses.

                After you use this ability on a particular area of earth, you cannot use it again on that same area for 24 hours.
            \end{attuneability}
            \parhead{Mastery} You gain an additional \glossterm{hit point}.

        \subsubsection{Evil}
            \parhead{Gift} At the start of each phase, you may choose an adjacent \glossterm{ally}.
            If you do, the first time you would lose a \glossterm{hit point} that phase, the target loses that hit point instead.
            If the target is unable to lose hit points, such as if it has no hit points remaining, you suffer the hit point loss normally.
            \parhead{Aspect} You can use this domain's domain gift to target any \glossterm{ally} within \rngclose range.
            \parhead{Essence} You can use the \textit{compel evil} ability as a standard action.
            \begin{freeability}{Compel Evil}[\glossterm{Compulsion}]
                Make an attack vs. Mental against a creature within \rngmed range.
                Creatures who have strict codes prohibiting them from taking evil actions, such as paladins devoted to Good, are immune to this ability.
                \hit The target takes an evil action as soon as it can.
                You have no control over the act the creature takes, but circumstances can make the target more likely to take an action you desire.
            \end{freeability}
            \parhead{Mastery} You can use your domain gift to redirect your hit point loss to an adjacent unwilling creature.
            You cannot target the same unwilling creature more than once with this ability between \glossterm{short rests}.

        \subsubsection{Fire}
            If you choose this domain, you add the \sphere{pyromancy} \glossterm{mystic sphere} to your list of divine mystic spheres (see \pcref{Mystic Spheres}).

            \parhead{Gift} You gain a bonus equal to your \glossterm{power} to \glossterm{resistances} against fire damage.
            \parhead{Aspect} Whenever \glossterm{injure} a creature with a \glossterm{critical hit} using fire damage, that creature takes fire \glossterm{standard damage} \minus3d at the end of the next round.
            This effect can be removed before the target takes damage if it makes a \glossterm{difficulty rating} 10 Dexterity check as a \glossterm{move action} to put out the flames.
            Dropping \glossterm{prone} as part of this action gives a \plus5 bonus to this check.
            \parhead{Essence} You can use the \textit{speak with fire} ability as a standard action.
            \begin{attuneability}{Speak with Fire}[\glossterm{Attune} (self)]
                You can speak with and command fire within a \areahuge radius \glossterm{zone} from your location.
                You can ask the fire simple questions and understand its responses.
                If you command the fire to perform a task, it will do so do the best of its ability until this effect ends.
                You cannot compel the fire to move farther than 30 feet in a single round.
                Fire that ends the round on non-combustable materials usually goes out, depending on the circumstances.

                After you use this ability on a particular area of fire, you cannot use it again on that same area for 24 hours.
                % TODO: What does an ``area of fire'' mean?
            \end{attuneability}
            \parhead{Mastery} This domain's domain aspect makes the target \glossterm{ignited} as a \glossterm{condition} instead of dealing damage directly.
            The condition can still be removed in the same way.

        \subsubsection{Good}
            \parhead{Gift} Whenever an adjacent \glossterm{ally} suffers a \glossterm{vital wound}, you may gain a \glossterm{vital wound} instead.
            You gain a \plus2 bonus to the \glossterm{wound roll} of each \glossterm{vital wound} you gain this way.
            The original target suffers any other effects of the attack normally.
            \parhead{Aspect} This domain's domain gift affects any \glossterm{ally} within a \areamed radius \glossterm{emanation} from you.
            \parhead{Essence} You can use the \textit{compel good} ability as a standard action.
            \begin{freeability}{Compel Good}[\glossterm{Compulsion}]
                Make an attack vs. Mental against a creature within \rngmed range.
                Creatures who have strict codes prohibiting them from taking evil actions, such as paladins devoted to Good, are immune to this ability.
                \hit The target takes an good action as soon as it can.
                You have no control over the act the creature takes, but circumstances can make the target more likely to take an action you desire.
            \end{freeability}
            \parhead{Mastery} Whenever an \glossterm{ally} within a \areamed radius \glossterm{emanation} from you loses one or more \glossterm{hit points}, you may lose a single hit point instead.
            The target suffers any other effects of the attack normally.

        \subsubsection{Knowledge}
            If you choose this domain, you add all Knowledge skills to your cleric \glossterm{class skill} list.

            \parhead{Gift} You gain two skill points.
            \parhead{Aspect} Your extensive knowledge of all methods of attack and defense grants you a \plus1 bonus to all defenses.
            \parhead{Essence} You can use the \textit{share knowledge} ability as a standard action.
            \begin{freeability}{Share Knowledge}
                You make a Knowledge check of any kind with a bonus equal to your \glossterm{power}.
                You and your \glossterm{allies} within a \arealarge radius learn the results of your check.
                Creatures believe the information gained in this way to be true as if they it had seen it with their own eyes.

                You cannot alter the knowledge you gain with this check in any way, such as by adding or withholding information.
            \end{freeability}
            \parhead{Mastery} You gain a \plus1 bonus to \glossterm{accuracy} with all attacks.

        \subsubsection{Law}
            \parhead{Gift} You gain a \plus2 bonus to Mental defense.
            % Clarify - does this apply to exploding dice?
            \parhead{Aspect} When you roll a 1 on an \glossterm{attack roll}, it is treated as if you had rolled a 6.
            \parhead{Essence} You can use the \textit{compel law} ability as a standard action.
            \begin{freeability}{Compel Law}[\glossterm{Compulsion}]
                Make an attack vs. Mental against all creatures within a \arealarge radius.
                \hit Each target is unable to break the laws that apply in the area, and any attempt to do so simply fails.
                The laws which are applied are those which are most appropriate for the area, regardless of whether you or any other creature know those laws.

                % Sufficiently clear that this isn't part of the hit effect?
                When you use this ability, you also gain the condition.
                If this condition is removed from you, it is also removed from all other affected creatures.
                In areas under ambiguous or nonexistent government, this ability may have unexpected effects, or it may have no effect at all.
            \end{freeability}
            \parhead{Mastery} When you roll less than a 5 on an \glossterm{attack roll}, it is treated as if you had rolled a 5.

        \subsubsection{Life}
            \parhead{Gift} You gain a \plus4 bonus to the Heal skill (see \pcref{Heal}).
            \parhead{Aspect} You gain a \plus1 bonus to \glossterm{wound rolls}.
            \parhead{Essence} At the end of each round, if you became \glossterm{unconscious} from a \glossterm{vital wound} that round, you can use one \glossterm{magical} ability you have that modifies \glossterm{wound rolls} or removes \glossterm{vital wounds} on yourself without taking an action.
            \parhead{Mastery} The bonus to \glossterm{wound rolls} from this domain's aspect increases to \plus3.

        \subsubsection{Magic}
            \parhead{Gift} You gain a \plus1 bonus to \glossterm{power} with \glossterm{magical} abilities.
            \parhead{Aspect} You learn an additional \glossterm{spell} for a divine \glossterm{mystic sphere} you have access to.
            \parhead{Essence} You learn an \glossterm{augment} (see \pcref{Augments}).
            You can apply augments to spells to change their effects.
            When you gain access to a new divine spell \glossterm{rank}, you may change which augments you know.
            \parhead{Mastery} You learn an additional \glossterm{augment} (see \pcref{Augments}).
            In addition, the power bonus from this domain's gift increases to \plus2.

        \subsubsection{Strength}
            If you choose this domain, you add the Climb, Jump, and Swim skills to your cleric \glossterm{class skill} list.

            \parhead{Gift} You gain two skill points.
            \parhead{Aspect} You gain a \plus4 bonus to Strength for the purpose of checks and determining your carrying capacity.
            \parhead{Essence} You can use the \textit{divine strength} ability as a \glossterm{minor action}.
            \begin{attuneability}{Divine Strength}[\glossterm{Attune} (self)]
                You gain a \plus4 \glossterm{magic bonus} to Strength.
            \end{attuneability}
            \parhead{Mastery} You gain a \plus1 bonus to your base Strength.

        \subsubsection{Travel}
            If you choose this domain, you add the \sphere{astromancy} \glossterm{mystic sphere} to your list of divine mystic spheres (see \pcref{Mystic Spheres}).
            In addition, you add the Knowledge (geography), Survival, and Swim skills to your cleric \glossterm{class skill} list.

            \parhead{Gift} You gain two skill points.
            \parhead{Aspect} You gain a \plus20 foot \glossterm{magic bonus} to your \glossterm{base speed}, up to a maximum of double your normal speed.
            \parhead{Essence} You can use the \textit{dimensional travel} ability as a standard action.
            \begin{freeability}{Dimensional Travel}[\glossterm{Teleportation}]
                You teleport up to 1 mile in any direction.
                You do not need \glossterm{line of sight} or \glossterm{line of effect} to your destination, but you must be able to clearly visualize it.
            \end{freeability}
            \parhead{Mastery} When you move, you can teleport the same distance instead.
            This does not change the total distance you can move, but you can teleport in any direction, including vertically.

            You can even attempt to move to locations outside of \glossterm{line of sight} and \glossterm{line of effect}, up to the limit of your remaining movement speed.
            If your intended destination is invalid, the distance you tried to teleport is taken from your remaining movement, but you suffer no other ill effects.

        \subsubsection{Trickery}
            If you choose this domain, you add the \sphere{glamer} \glossterm{mystic sphere} to your list of divine mystic spheres (see \pcref{Mystic Spheres}).
            In addition, you add the Deception, Disguise, and Stealth skills to your cleric \glossterm{class skill} list.

            \parhead{Gift} You gain two skill points.
            \parhead{Aspect} You gain a \plus2 \glossterm{magic bonus} to the Deception, Disguise, and Stealth skills.
            \parhead{Essence} You can use the \textit{compel belief} ability as a standard action.
            \begin{freeability}{Compel Belief}[\glossterm{Compulsion}, \glossterm{Sustain} (minor)]
                Make an attack vs. Mental against a creature within \rngmed range.
                You must also choose a belief that the target has.
                The belief may be a lie that you told it, or even a simple misunderstanding (such as believing a hidden creature is not present in a room).
                If the creature does not already hold the chosen belief, this ability automatically fails.
                \hit The target continues to maintain the chosen belief, regardless of any evidence to the contrary.
                It will interpret any evidence that the falsehood is incorrect to be somehow wrong -- an illusion, a conspiracy to decieve it, or any other reason it can think of to continue believing the falsehood.
                At the end of the effect, the creature can decide whether it believes the falsehood or not, as normal.
            \end{freeability}
            \parhead{Mastery} You are undetectable by Divination spells and effects.
            They cannot detect your presence, sounds you make, or any actions you take.

        \subsubsection{War}
            \parhead{Gift} You gain proficiency with heavy armor and an additional \glossterm{weapon group} of your choice.
            \parhead{Aspect} You gain a \plus1 bonus to \glossterm{accuracy} with \glossterm{mundane} abilities.
            \parhead{Essence} You learn one of the Legion, Selective, or Widened \glossterm{augments} (see \pcref{Augments}).
            You can apply augments to spells to change their effects.
            When you gain access to a new arcane spell \glossterm{rank}, you may change which augments you know, though this augment can only be one of the three listed.
            \parhead{Mastery} You gain a \plus2 bonus to \glossterm{power} with \glossterm{mundane} abilities.

        \subsubsection{Water}
            If you choose this domain, you add the \sphere{aquamancy} \glossterm{mystic sphere} to your list of divine mystic spheres (see \pcref{Mystic Spheres}).
            In addition, you add the Swim skill to your cleric \glossterm{class skill} list.

            \parhead{Gift} You gain a \plus4 \glossterm{magic bonus} to the Swim skill.
            In addition, you gain a \plus1 bonus to Reflex defense.
            \parhead{Aspect} You can breathe water as easily as a human breathes air, preventing you from drowning or suffocating underwater.
            \parhead{Essence} You can use the \textit{speak with water} ability as a standard action.
            \begin{attuneability}{Speak with Water}[\glossterm{Attune} (self)]
                You can speak with and command water within a \areahuge \glossterm{zone} from your location.
                You can ask the water simple questions and understand its responses.
                If you command the water to perform a task, it will do so do the best of its ability until this effect ends.
                You cannot compel the water to move faster than 30 feet per round.

                After you use this ability on a particular area of water, you cannot use it again on that same area for 24 hours.
            \end{attuneability}
            \parhead{Mastery}
            When you move, you can transform yourself into a rushing flow of water with a volume roughly equal to your normal volume until your movement is complete.
            In this form, you may move wherever water could go, you cannot take other actions, such as jumping, attacking, or casting spells.
            You may move through squares occupied by enemies without penalty.
            \par Your speed is halved when moving uphill and doubled when moving downhill.
            Unusually steep inclines may cause greater movement differences while in this form.
            \par If the water is split, you may reform from anywhere the water has reached, to as little as a single ounce of water.
            If not even an ounce of water exists contiguously, your body reforms from all of the largest available sections of water, cut into pieces of appropriate size.
            This usually causes you to die.

        \subsubsection{Wild}
            If you choose this domain, you add the \sphere{verdamancy} \glossterm{mystic sphere} to your list of divine mystic spheres (see \pcref{Mystic Spheres}).
            In addition, you add the Creature Handling, Knowledge (nature), and Survival skills to your cleric \glossterm{class skill} list.

            \parhead{Gift} You gain two skill points.
            % TODO: clarify whether you can have this and the druid ability
            \parhead{Aspect} This ability functions like the \textit{wild aspect} druid ability from the Shifter archetype (see \pcref{Shifter}), except that you cannot spend \glossterm{insight points} to learn additional wild aspects.
            \parhead{Essence} You learn an additional \textit{wild aspect}.
            \parhead{Mastery} You learn an additional \textit{wild aspect}.
            % TODO: essence, mastery

        \subsubsection{Ex-Clerics}
            If you grossly violate the code of conduct required by your deity, you lose all spells and magical cleric class abilities.
            You cannot regain those abilities until you atone for your transgressions to your deity.

\newpage
\section{Druid}\label{Druid}
    \begin{dtable!*}
        \lcaption{Druid Progression}
        \begin{dtabularx}{\textwidth}{l l >{\lcol}X >{\lcol}X l >{\lcol}X >{\lcol}X}
            \tb{Rank} & \tb{Min Level} & \tb{Elementalist}         & \tb{Nature Magic} & \tb{Nature Spell Mastery}   & \tb{Shifter}               & \tb{Wildspeaker}         \tableheaderrule
            1 & 1  & Elemental balance         & Spellcasting   & Mystic insight              & Wild aspects               & Natural servant         \\
            2 & 2  & Elemental sphere          & Natural focus  & Natural spells              & Shifting defense           & Animal speech           \\
            3 & 5  & Elemental influence       & Spell rank (3) & Wellspring of power         & Shift body                 & Wild guardian           \\
            4 & 8  & Greater elemental balance & Spell rank (4) & Mystic insight              & Wild aspect                & Greater natural servant \\
            5 & 11 & Elemental sphere          & Spell rank (5) & Greater natural spells      & Greater shifting defense   & Plant speech            \\
            6 & 14 & Elemental control         & Spell rank (6) & Greater wellspring of power & Greater shift body         & Greater wild guardian   \\
            7 & 17 & Supreme elemental balance & Spell rank (7) & Mystic insight              & Triple aspect, wild aspect & Supreme natural servant \\
            8 & 20 &                           & Spell rank (8) &                             & Avatar of nature           &                         \\
        \end{dtabularx}
    \end{dtable!*}

    Druids are nature spellcasters that draw power from their veneration of the natural world.
    Nature magic can be used to help allies or to crush foes with natural elements like air or fire.
    Many druids live outside of organized society, either on their own or in small communes with other druids and similarly minded people.
    Druids have strong ties to the natural world with a variety of abilities other than their spells.
    They can speak with and command animals or elements, or even change their body to take on on aspects of animals themselves.

    \classbasics{Alignment} Neutral good, lawful neutral, neutral, chaotic neutral, or neutral evil.

    \classbasics{Archetypes} Druids have the Elementalist, Nature Spellcasting, Nature Spell Mastery, Shifter, and Wildspeaker \glossterm{archetypes}.

    \subsection{Basic Class Abilities}
        If you are a druid, you gain the following abilities.

        \cf{Drd}{Defenses}
        You gain the following bonuses to your \glossterm{defenses}: \plus2 Armor, \plus5 Fortitude, \plus4 Reflex, \plus5 Mental.

        \cf{Drd}{Skills}
        You have the following \glossterm{class skills}:
        \begin{itemize}
            \item \subparhead{Strength} Climb, Jump, Swim.
            \item \subparhead{Dexterity} Acrobatics, Ride, Stealth.
            \item \subparhead{Constitution} Endurance.
            \item \subparhead{Intelligence} Craft, Deduction, Heal, Knowledge (geography, nature).
            \item \subparhead{Perception} Awareness, Creature Handling, Survival.
            \item \subparhead{Other} Deception, Intimidate, Persuasion, Profession.
        \end{itemize}

        \cf{Drd}{Weapon Proficiencies} 
        You are proficient with simple weapons, any one other \glossterm{weapon group}, scimitars, and sickles.

        \cf{Drd}{Armor Proficiencies} 
        You are proficient with light and medium armor.
        However, you cannot use metal armor; see the Metal Abhorrence ability, below.

        \cf{Drd}{Metal Abhorrence}
        The oaths that you swear as part of your druidic initiation prohibit you from using armor made of metal.
        If you wear prohibited armor or carry a prohibited shield, you are unable to cast druid spells or use any of your \glossterm{magical} druid abilities while doing so and for 24 hours thereafter.
        % TODO: if weapon materials were more clearly defined, this could be prohibited as well
        This does not prevent you from using metallic weapons.

        You can avoid this penalty by using armor made of wood altered with the \ritual{ironwood} ritual.
        Such wood is as strong as steel.

    \subsection{Elementalist}\label{Elementalist}
        This archetype grants you influence over four elements that define the natural world: air, earth, fire, and water.

        \cf{Drd}[1]{Elemental Balance} You gain a small benefit from each of the four elements.
        \begin{itemize}
            \item Air: You gain a \plus2 bonus to the Jump skill.
            \item Earth: You gain a \plus1 bonus to Fortitude defense.
            \item Fire: You gain a bonus to \glossterm{resistances} against fire damage equal to half your \glossterm{power}.
            \item Water: You gain a \plus2 bonus to the Swim skill.
        \end{itemize}

        \cf{Drd}[2]{Elemental Sphere} Choose one of the \glossterm{mystic spheres} associated with the four elements: \sphere{aeromancy}, \sphere{aquamancy}, \sphere{pyromancy}, or \sphere{terramancy}.
        You gain access to that mystic sphere, including all \glossterm{cantrips} from that sphere.
        In addition, you learn an additional nature \glossterm{spell}.

        \cf{Drd}[3]{Elemental Influence} You can use the \textit{elemental influence} ability as a standard action.
        \begin{freeability}{Elemental Influence}[\glossterm{Sustain} (standard)]
            You can speak with air, earth, fire, and water within a \areahuge \glossterm{zone} from your location.
            You can ask the elements simple questions and understand their responses.
            Each element is only able to give information based on what it touched, not any more subtle characteristics.
            For example, air and water could tell you the size category and general shape (such as bipedal or four-legged) of a creature that passed through them recently, but not any details about color or subjective appearance.
            Each element has different limitations on its memory and awareness, as described below.

            \begin{itemize}
                \item Air: Air can remember events up to an hour ago on a very calm day or only a few minutes ago on a windy day.
                    Moving air is aware of events near where it blew through, not necessarily in your current location.
                \item Earth: Earth can remember events up to a year ago, but its awareness is extremely limited.
                    It can only remember very large events, such as giant creatures tearing up the terrain, earthquakes, or major construction.
                    Earth can tell you whether there exist underground tunnels within the area, but any sort of detailed mapping is beyond its ability to communicate.
                \item Fire: Fire can remember everything it touched and consumed since it started burning.
                    Individual pieces of a very large fire, such as a particular burning tree in a forest fire, are not aware of the behavior of the entirety of the fire.
                    However, the fire on burning tree could tell you how it got to the tree and everything it burned along the way, including the event that started the forest fire.
                \item Water: Water can remember events up to a day ago in a very calm pool or only a few minutes ago in a rushing river.
                    Moving water is aware of events near where it moved through, not necessarily in your current location.
            \end{itemize}
        \end{freeability}

        \cf{Drd}[4]{Greater Elemental Balance} The bonuses from your \textit{elemental balance} ability improve.
        \begin{itemize}
            \item Air: You gain a \glossterm{glide speed} equal to your \glossterm{base speed}.
            \item Earth: The bonus to Fortitude defense increases to \plus2.
            \item Fire: The bonus to \glossterm{resistances} increases to be equal to your \glossterm{power}.
            \item Water: You gain a \glossterm{swim speed} equal to your \glossterm{base speed}.
        \end{itemize}

        \cf*{Drd}[5]{Elemental Sphere} Choose one of the \glossterm{mystic spheres} associated with the four elements: \sphere{aeromancy}, \sphere{aquamancy}, \sphere{pyromancy}, or \sphere{terramancy}.
        You gain access to that mystic sphere, including all \glossterm{cantrips} from that sphere.
        In addition, you learn an additional nature \glossterm{spell}.

        \cf{Drd}[6]{Elemental Control} When you use your \textit{elemental influence} ability, you can also command the elements to move as you desire.
        Each element has different limitations on its ability to move, as described below.
        \begin{itemize}
            \item Air: You can change the wind speed of air by up to 50 miles per hour.
                If you reduce the air's speed to 0 and then increase it again, you can change the direction the air blows.
            \item Earth: You can reshape earth or unworked stone at a rate of up to one foot per round.
            \item Fire: You can make fire leap up to 30 feet between combustable materials, suppress fire so it smolders without being extinguished, or snuff out fire entirely.
            \item Water: You can change the speed of water by up to 30 feet per round.
                If you reduce the water's speed to 0 and then increase it again, you can change the direction the water flows.
        \end{itemize}

        \cf{Drd}[7]{Supreme Elemental Balance} The bonuses from your \textit{elemental balance} ability improve.
        \begin{itemize}
            \item Air: You gain a \glossterm{fly speed} equal to your \glossterm{base speed} while you are within 100 feet of the ground.
            \item Earth: The bonus to Fortitude defense increases to \plus3.
            \item Fire: The bonus to \glossterm{resistances} increases to be equal to twice your \glossterm{power}.
            \item Water: You suffer no penalties for fighting underwater, and you can breathe water as if it was air.
        \end{itemize}

    \subsection{Nature Magic}
        This archetype grants you the ability to cast nature spells.

        \cf{Drd}[1]{Spellcasting}[Magical]
        Your connection to nature grants you the ability to use nature magic.
        You gain access to one nature \glossterm{mystic sphere} (see \pcref{Nature Mystic Spheres}).
        You may spend \glossterm{insight points} to learn one additional \glossterm{mystic sphere} per two \glossterm{insight points}.
        Each \glossterm{mystic sphere} has a set of \glossterm{spells} associated with it.

        You automatically learn all \glossterm{cantrips} from any mystic sphere you have access to.
        In addition, you learn two rank 1 \glossterm{spells} from your chosen \glossterm{mystic sphere}.
        You can also spend \glossterm{insight points} to learn one additional nature spell per \glossterm{insight point}.
        Unless otherwise noted in a spell's description, casting a spell requires a \glossterm{standard action}.

        When you gain access to a new \glossterm{mystic sphere} or spell \glossterm{rank},
            you can exchange any number of spells you know for other spells,
            including spells of the higher rank.

        Nature spells require \glossterm{verbal components} to cast (see \pcref{Casting Components}).
        For details about mystic spheres and casting spells, see \pcref{Spell and Ritual Mechanics}.

        \cf{Drd}[2]{Natural Focus} You reduce your \glossterm{focus penalties} by 2.

        \cf{Drd}[3]{Spell Rank}[Magical] You become a rank 3 nature spellcaster.
        This can improve the effectiveness of your spells and give you access to spells that require a minimum rank of 3.

        \cf*{Drd}[4]{Spell Rank}[Magical] You become a rank 4 nature spellcaster.
        This can improve the effectiveness of your spells and give you access to spells that require a minimum rank of 4.

        \cf*{Drd}[5]{Spell Rank}[Magical] You become a rank 5 nature spellcaster.
        This can improve the effectiveness of your spells and give you access to spells that require a minimum rank of 5.

        \cf*{Drd}[6]{Spell Rank}[Magical] You become a rank 6 nature spellcaster.
        This can improve the effectiveness of your spells and give you access to spells that require a minimum rank of 6.

        \cf*{Drd}[7]{Spell Rank}[Magical] You become a rank 7 nature spellcaster.
        This can improve the effectiveness of your spells and give you access to spells that require a minimum rank of 7.

        \cf*{Drd}[8]{Spell Rank}[Magical] You become a rank 8 nature spellcaster.
        This can improve the effectiveness of your spells and give you access to spells that require a minimum rank of 8.

    \subsection{Nature Spell Mastery}
        This archetype improves the nature spells you cast.
        You must be able to cast nature spells to gain the abilities from this archetype.

        \cf{Drd}[1]{Mystic Insight}
        You gain your choice of one of the following abilities.
        {
            \parhead{Focused Caster} You reduce your \glossterm{focus penalty} by 1.
                You cannot choose this ability multiple times.
            \parhead{Mystic Sphere Access} You gain access to an additional nature \glossterm{mystic sphere}.
                You cannot choose this ability multiple times.
            \parhead{Rituals} You gain the ability to perform nature rituals to create unique magical effects (see \pcref{Rituals}).
                The maximum \glossterm{rank} of nature ritual you can learn or perform is equal to the maximum rank of divine spell that you can cast.
                You cannot choose this ability multiple times.
            \parhead{Spell Known} You learn an additional nature \glossterm{spell} (see \pcref{Nature Mystic Spheres}).
                You can choose this ability multiple times, learning an additional spell each time.
            \parhead{Spell Power} Choose a nature \glossterm{spell} you know.
                You gain a \plus2 bonus to \glossterm{power} with that spell.
                You can choose this ability multiple times, choosing a different spell each time.
            \parhead{Spell Precision} Choose a nature \glossterm{spell} you know.
                You gain a \plus1 bonus to \glossterm{accuracy} with that spell.
                You can choose this ability multiple times, choosing a different spell each time.
        }

        \cf{Drd}[2]{Natural Spells} Your spells ignore \glossterm{cover} and \glossterm{concealment} provided by living plants (but not \glossterm{total cover}).

        \cf{Drd}[3]{Wellspring of Power}[Magical]
        You gain a \plus1 bonus to \glossterm{power} with \glossterm{magical} abilities.

        \cf*{Drd}[4]{Mystic Insight}[Magical]
        You gain an additional \textit{mystic insight} ability.

        \cf{Drd}[5]{Greater Natural Spells} Your spells ignore \glossterm{total cover} provided by living plants.
        This does not allow you to see targets through plants.

        \cf{Drd}[6]{Greater Wellspring of Power}[Magical]
        The bonus from your \textit{wellspring of power} ability increases to \plus2.

        \cf*{Drd}[7]{Mystic Insight}[Magical]
        You gain an additional \textit{mystic insight} ability.

    \subsection{Shifter}\label{Shifter}
        This archetype grants you the ability to embody aspects of the natural world in your own form.

        \cf{Drd}[1]{Wild Aspects}[Magical]
        You gain the ability to embody an aspect of an animal or of nature itself.
        Choose a single wild aspect from the list below.
        You can also spend \glossterm{insight points} to learn one additional \textit{wild aspect} per \glossterm{insight point}.

        As a \glossterm{standard action}, you can gain the effects of one wild aspect that you know.
        That effect lasts until you activate a different wild aspect you know.

        The abilities in the list below describe the effects of the aspect.
        Your appearance also changes to match the aspect's effects, but the nature of this change is not described.
        Different druids change in different ways.
        For example, one druid might grow brown fur when using the Form of the Bear, while another might instead change their face to become broader and more bear-shaped when embodying the same aspect.
        You choose how your appearance changes when you gain a wild aspect.
        This change cannot be used to gain an additional substantive benefit beyond the effects given in the description of the aspect.

        Many wild aspects grant natural weapons.
        See \pcref{Natural Weapons}, for details about natural weapons.

        {
            \begin{freeability}{Form of the Bear}
                You gain a \plus2 bonus to Fortitude defense.
                In addition, your mouth and hands transform, granting you bite and claw \glossterm{natural weapons}.
                The bite deals \plus0d damage, and the claws deal \minus1d damage.

                \rankline
                \rank{3} You gain a \plus2 bonus to \glossterm{power} with natural weapons.
                \rank{5} The Fortitude bonus increases to \plus4.
                \rank{7} The bonus to \glossterm{power} increases to \plus4.
            \end{freeability}

            \begin{freeability}{Form of the Bull}
                You gain a \plus2 bonus to \glossterm{accuracy} with the \textit{shove} ability (see \pcref{Shove}, and \pcref{Overrun}).
                In addition, your head transforms, granting you a ram \glossterm{natural weapon}.
                The weapon deals \plus0d damage, and has the Forceful weapon tag (see \pcref{Weapon Tags}).

                \rankline
                \rank{3} Your penalties to accuracy and defenses from the \textit{charge} ability are reduced by 1.
                \rank{5} Your \textit{shove} ability moves the target as a \glossterm{knockback} instead of as a \glossterm{push}.
                In addition, the accuracy bonus increases to \plus4.
                \rank{7} Your penalties to accuracy and defenses from the \textit{charge} ability are removed.
            \end{freeability}

            \begin{freeability}{Form of the Constrictor}
                You gain a \plus2 bonus to \glossterm{accuracy} with the Grapple ability and all grapple actions (see \pcref{Grapple}).
                In addition, you gain a constrict \glossterm{natural weapon}.
                This weapon deals \plus1d damage, and it has the Grappling weapon tag (see \pcref{Weapon Tags}).
                It can only be used against a foe you are grappling with.

                \rankline
                \rank{3} You can contort your body, allowing it to act as a free hand for the purpose of using the \textit{grapple} ability and grapple actions.
                \rank{5} The accuracy bonus increases to \plus4.
                \rank{7} When you grapple a creature the \textit{grapple} ability, you are not considered to be \glossterm{grappled} (see \pcref{Asymmetric Grappling}).
            \end{freeability}

            \begin{freeability}{Form of the Fish}
                You gain a \glossterm{swim speed} equal to your \glossterm{base speed}.
                In addition, you gain a bite \glossterm{natural weapon} that deals \plus0d damage.

                \rankline
                \rank{3} You can breathe water as easily as a human breathes air, preventing you from drowning or suffocating underwater.
                \rank{5} You suffer no penalties for acting underwater.
                \rank{7} You are immune to \glossterm{magical} effects that restrict your mobility.
                In addition, you gain a \plus4 bonus to defenses against the \textit{grapple} ability and grapple actions (see \pcref{Grapple}).
            \end{freeability}

            \begin{freeability}{Form of the Hawk}
                You gain \glossterm{low-light vision}.
                If you already have low-light vision, you double its benefit, allowing you to treat sources of light as if they had four times their normal illumination range.
                In addition, you gain a \plus3 bonus to Awareness.

                \rankline
                \rank{3} You grow wings, granting your a glide speed equal to your \glossterm{base speed} (see \pcref{Gliding}).
                \rank{5} The Awareness bonus increases to \plus6.
                \rank{7} The glide speed is replaced with a \glossterm{fly speed} equal to your \glossterm{base speed} (see \pcref{Flying}).
            \end{freeability}

            \begin{freeability}{Form of the Hound}
                You gain the \glossterm{scent} ability.
                In addition, you gain a bite \glossterm{natural weapon} that deals \plus0d damage.

                \rankline
                \rank{3} You gain the ability to move on all four limbs.
                When doing so, you gain a \plus20 foot bonus to your land speed, up to a maximum of double your original speed.
                When not using your hands to move, your ability to use your hands is unchanged.
                You can descend to four legs and rise up to stand on two legs again as part of movement.
                \rank{5} You gain an additional \plus10 bonus to scent-based Awareness checks (see \pcref{Awareness}).
                \rank{7} You gain a \plus10 foot bonus to your land speed.
            \end{freeability}

            % Seems boring? What abilities would make sense?
            \begin{freeability}{Form of the Monkey}
                You gain a \glossterm{climb speed} equal to your \glossterm{base speed}.
                In addition, you gain a bite \glossterm{natural weapon} that deals \plus0d damage.

                \rankline
                \rank{3} You grow a tail that you can use as a free hand for the purpose of climbing.
                \rank{5} You gain a \plus10 foot bonus to your climb speed.
                \rank{7} You can use the \textit{creature climb} ability against creatures only two or more size categories larger than you instead of three size categories.
            \end{freeability}

            \begin{freeability}{Form of the Mouse}
                You gain a \plus2 bonus to the Escape Artist and Stealth skills.
                In addition, you gain a bite \glossterm{natural weapon} that deals \plus0d damage.
                
                \rankline
                \rank{3} When you use this wild aspect, you can choose to shrink by one \glossterm{size category}.
                \rank{5} The skill bonuses increases to \plus4.
                \rank{7} When you use this wild aspect, you can choose to shrink by up to two \glossterm{size categories} instead of only one.
            \end{freeability}

            \begin{freeability}{Form of the Oak}
                Your movement speed is halved.
                In exchange, you gain a \plus1 bonus to Armor defense and Fortitude defense.
                \rankline
                \rank{3} You gain a \glossterm{magic bonus} equal to half your level to \glossterm{resistances} against \glossterm{physical damage}.
                \rank{5} The defense bonuses increase to \plus2.
                \rank{7} The resistance bonus increases to be equal to your level.
            \end{freeability}

            \begin{freeability}{Form of the Viper}
                You gain a \glossterm{climb speed} equal to half your \glossterm{base speed}.
                You do not need to use your hands to climb in this way.
                In addition, you gain a bite \glossterm{natural weapon} that deals \plus0d damage.

                \rankline
                \rank{3} When a creature takes damage from your bite \glossterm{natural weapon}, it is poisoned.
                At the end of each round, you make an attack vs. Fortitude against the target.
                If you hit, the target is \glossterm{sickened} until it removes the poison.
                The poison is removed if you miss the target on this attack three times.
                \rank{5} On a critical hit with the poison, the target is \glossterm{nauseated} instead of \glossterm{sickened}.
                \rank{7} The poison makes the target \glossterm{nauseated} instead of \glossterm{sickened}.
            \end{freeability}

            \begin{freeability}{Form of the Wolf}
                You gain a \plus1 bonus to \glossterm{accuracy} against \glossterm{overwhelmed} creatures.
                In addition, you gain a bite \glossterm{natural weapon} that deals \plus0d damage.

                \rankline
                \rank{3} You gain a \plus2 bonus to \glossterm{power} with \glossterm{natural weapons}.
                \rank{5} You are treated as one \glossterm{size category} larger for the purpose of determining \glossterm{overwhelm penalties} you inflict on other creatures.
                \rank{7} The bonus to \glossterm{power} increases to \plus4.
            \end{freeability}

            \begin{freeability}{Myriad Form}
                You can use your \glossterm{power} in place of your Disguise skill when making Disguise checks to alter your own appearance.
                \rankline
                \rank{3} When you use this wild aspect, you can choose to grow or shrink by one \glossterm{size category}.
                \rank{5} You can use the \textit{Disguise Creature} ability to disguise yourself as a \glossterm{standard action} (see \pcref{Disguise Creature}).
                \rank{7} When you use this wild aspect, you can choose to grow or shrink by up to two \glossterm{size categories} instead of only one.
            \end{freeability}

            \begin{freeability}{Photosynthesis}
                As long as you are in natural sunlight, you gain a \plus10 foot bonus to your \glossterm{base speed}.
                \rankline
                \rank{3} As long as you are in natural sunlight, you do not gain hunger or thirst.
                When you leave natural sunlight, you continue gaining hunger or thirst at your normal rate, ignoring any time you spent in natural sunlight.
                \rank{5} The speed bonus increases to \plus20 feet.
                \rank{7} When you take a \glossterm{short rest} while you are in natural sunlight, you remove a \glossterm{vital wound}.
            \end{freeability}

            \begin{freeability}{Plantspeaker}
                Your speed is not reduced when moving in light or heavy \glossterm{undergrowth}.
                In addition, you can ignore \glossterm{cover} and \glossterm{concealment} (but not \glossterm{total cover}) from plants whenever doing so would be beneficial to you, as the plants move out of the way to help you.

                \rankline
                \rank{3} You gain a \plus1 bonus to Armor and Reflex defenses while standing in \glossterm{undergrowth}.
                For example, creatures cannot use concealment from plants to hide from you.
                \rank{5} The movement penalties from \glossterm{undergrowth} are doubled for enemies within a \areahuge radius emanation from you.
                \rank{7} You can ignore \glossterm{total cover} from plants whenever doing so would be beneficial to you.
            \end{freeability}
        }

        \cf{Drd}[2]{Shifting Defense} You gain a \plus1 bonus to Armor, Fortitude, Reflex, or Mental defense.
        You can change the defense this bonus applies to as a \glossterm{standard action}.

        \cf{Drd}[3]{Shift Body} You gain a \plus1 bonus to the base value of one physical \glossterm{attribute} of your choice: Strength, Dexterity, or Constitution.
        You can change the attribute this bonus applies to whenever you take a \glossterm{long rest}.
        If you change the attribute during a rest, the bonus does not apply to any attribute.

        \cf{Drd}[4]{Wild Aspect}[Magical]
        You learn an additional \textit{wild aspect}.

        \cf{Drd}[5]{Greater Shifting Defense}
        The bonus from your \textit{shifting defense} ability increases to \plus2.

        \cf{Drd}[6]{Greater Shift Body}
        The bonus from your \textit{shift body} ability increases to \plus2.

        \cf*{Drd}[7]{Wild Aspect}[Magical]
        You learn an additional \textit{wild aspect}.

        \cf{Drd}[8]{Avatar of Nature}[Magical]
        If you die, except if by old age, you may choose to have your body and soul become an instrument of nature's will.
        Your body immediately decomposes or otherwise disappears, and your soul does not travel to an afterlife.
        You has no physical form, and cannot use any of your normal abilities.
        Instead, you have a \glossterm{fly speed} of 100 feet, with special maneuverability.
        As a standard action, you can temporarily possess any living plants or animals within a 10 mile radius of the place of your death.

        While possessing a living plant or animal, you can see through its senses and control its actions completely.
        In addition, you may cast spells, and the spells take effect as if the plant or animal had cast them.
        You use the plant or animal's position to determine range, visible targets, and so on.
        You do not require \glossterm{verbal components} to cast your spells in this form.
        % TODO: are there any spells or rituals that have material components or focus objects?
        % but are unable to cast spells or perform rituals that require material components or focus objects.

        While not possessing a plant or animal, you can rest, or you can focus on reincarnating your physical form.
        Creating a new body in this way takes 12 consecutive hours of concentration.
        At the end of that time, you are reincarnated in a new body in your location, as the effect of the \ritual{reincarnation} ritual, except that you can choose your species from among the species listed (not including the ``Other'' species).

        While you are an avatar of nature, you do not age and you cannot die of old age.
        You can continue to exist in this form indefinitely.

    \subsection{Wildspeaker}\label{Wildspeaker}

        \cf{Drd}[1]{Natural Servant}[Magical]
        You can use the \textit{natural servant} ability.
        This ability requires spending 1 hour performing rituals in a natural area.
        \begin{attuneability}{Natural Servant}[\glossterm{Attune} (self), \glossterm{Magical}]
            An animal native to the local environment appears to help you.
            It follows your directions to the best of its ability as long as you remain in its natural environment.
            Animals are unable to understand complex concepts, so their ability to obey convoluted instructions is limited.
            If you leave the animal's natural habitat, it remains behind and this effect ends.
            If the animal gains a \glossterm{vital wound} or has no hit points remaining at the end of a phase, this effect ends.

            Your magical connection to the animal improves its resilience and strength in combat.
            The animal's statistics use the values below, except that each animal also gains a special ability based on the environment you are in.
            \begin{itemize}
                \item Its \glossterm{damage resistance} and \glossterm{wound resistance} are equal to the base values for your level (see \pcref{Character Advancement}).
                \item Its \glossterm{hit points} are equal to 6.
                \item Each of its \glossterm{defenses} is equal to 4 \add your level.
                \item Its \glossterm{accuracy} is equal to the your level \add half your base Perception.
                \item Its \glossterm{power} with its attacks is equal to your \glossterm{power}.
                \item Its \glossterm{base speed} is the normal base speed for its size.
                \item It has no \glossterm{action points}.
            \end{itemize}
        \end{attuneability}

        The special ability of the animal that appears depends on your environment, as described below.
        You may choose a different animal native to that environment that is similar in size and type, but that does not change the animal's statistics.
        For example, your \textit{natural servant} in an aquatic environment may be a fish or seal instead of a shark.
        Unusual environments may have different animals than the standard animals listed below.
        \begin{itemize}
            \item Aquatic: A Small shark appears that has a 30 foot \glossterm{swim speed} and no land speed.
                It has a bite \glossterm{natural weapon}.
            \item Arctic: An Small arctic fox appears that has no penalties for being in cold environments.
                It has a bite \glossterm{natural weapon}.
            \item Desert: A Small hyena appears that has no penalties for being in hot environments.
                It has a bite \glossterm{natural weapon}.
            \item Mountain: A Small goat appears that can move up or down steep slopes without slowing its movement.
                It has a ram \glossterm{natural weapon}.
            \item Forest: A Small wolverine appears that has one additional \glossterm{hit point}.
                It has a bite \glossterm{natural weapon}.
            \item Plains: A Small wolf appears that has the \glossterm{scent} ability.
                It has a bite \glossterm{natural weapon}.
            \item Swamp: A Small crocodile appears that has a 15 foot \glossterm{land speed} and a 25 foot \glossterm{swim speed}.
                It has a bite \glossterm{natural weapon}.
            \item Underground: A Small dire rat appears that has \glossterm{low-light vision}.
                It has a bite \glossterm{natural weapon}.
        \end{itemize}

        \cf{Drd}[2]{Animal Speech}[Magical] You can use the \textit{animal speech} ability as a standard action.
        \begin{attuneability}{Animal Speech}[\glossterm{Sustain} (minor)]
            Choose an animal within \rnglong range.
            You can speak to and understand the speech of the target animal, and any other animals of the same species.

            This ability does not make the target any more friendly or cooperative than normal.
            Wary and cunning animals are likely to be terse and evasive, while stupid ones tend to make inane comments and are unlikely to say or understand anything of use.
        \end{attuneability}

        \cf{Drd}[3]{Wild Guardian}[Magical] Wild animals and plants will not willingly attack you.
        They can be compelled to attack you with a Creature Handling check with a \glossterm{difficulty rating} equal to 10 \add your \glossterm{power}.
        If you attack an animal or plant, this ability no longer provides protection against it and its \glossterm{allies}.
        This effect does not protect your allies from attack.

        \cf{Drd}[4]{Greater Natural Servant}[Magical] Your \textit{natural servant} gains an \glossterm{action point}.
        This action point is shared among any creatures you summon with your \textit{natural servant} ability, and is only recovered when you take a \glossterm{long rest}.
        In addition, you can summon an alternate \textit{natural servant} based on your local environment, as described below.
        \begin{itemize}
            \item Aquatic: A Medium shark appears that has a 40 foot \glossterm{swim speed} and no land speed.
                It has a bite \glossterm{natural weapon}.
            \item Arctic: An Medium polar bear appears that has a bonus to \glossterm{resistances} against cold damage equal to your \glossterm{power}.
                It has a bite \glossterm{natural weapon} and two claw \glossterm{natural weapons}.
            \item Desert: A Medium camel appears that has no penalties for being in hot environments.
                It has a bite \glossterm{natural weapon}.
            \item Mountain: A Medium goat appears that can move up or down steep slopes without slowing its movement.
                It has a ram \glossterm{natural weapon}.
            \item Forest: A Medium bear appears that has two additional \glossterm{hit points}.
                It has a bite \glossterm{natural weapon} and two claw \glossterm{natural weapons}.
            \item Plains: A Medium wolf appears that has the \glossterm{scent} ability.
                It has a bite \glossterm{natural weapon}.
            % TODO: define shallow water
            \item Swamp: A Medium crocodile appears that has a 20 foot \glossterm{land speed} and a 30 foot \glossterm{swim speed}.
                It has a bite \glossterm{natural weapon}.
            \item Underground: A Medium dire rat appears that has \glossterm{low-light vision} and \glossterm{darkvision}.
                It has a bite \glossterm{natural weapon}.
        \end{itemize}

        \cf{Drd}[5]{Plant Speech}[Magical] When you use your \textit{animal speech} ability, you can target a plant instead of an animal.
        When you do, you can speak to and understand the speech of the target plant, and any other plants of the same species.

        \cf{Drd}[6]{Greater Wild Guardian}[Magical] Your \glossterm{allies} within a \arealarge radius \glossterm{emanation} from you also gain the benefit of your \textit{wild guardian} ability.
        The protection ends if any target attacks an animal or plant instead of only if you attack.
        In addition, the \glossterm{difficulty rating} of the Creature Handling check increases to 20 \add your \glossterm{power}.

        \cf{Drd}[7]{Supreme Natural Servant}[Magical] Your \textit{natural servant} gains two additional \glossterm{action points}.
        These action points are shared among any creatures you summon with your \textit{natural servant} ability, and are only recovered when you take a \glossterm{long rest}.

        % \subcf{15th -- Air Mantle}
        % You are surrounded by a mantle of air.
        % Thrown and projectile weapons have a 50\% chance to miss your while this effect is active.
        % Unusually large weapons, such as a giant's boulders, may suffer a decreased miss chance as appropriate to their size.
        % \subcf{15th -- Aqueous Step}
        % Wherever the druid moves, you leave a path of animated water that can grab creatures.
        % When a creature crosses the path, the druid makes a Reflex attack to trip the creature, causing it to fall prone and waste the rest of its movement.
        % Your accuracy is equal to your druid level \add your Constitution.
        % \subcf{15th -- Flaming Step}
        % Wherever the druid moves, you leave a path of burning flame behind your that lasts for 1 round.
        % When a creature crosses the path, the druid makes a Reflex attack to deal damage to the creature.
        % The attack deals 1d8 points of fire damage per two druid levels.
        % Your accuracy is equal to your druid level \add your Constitution.
        % A failed attack deals half damage.
        % \subcf{15th -- Lifegiving Step}
        % Wherever the druid moves, you leave a path of small, living plants that entangle foes for 1 round.
        % When a creature crosses the path, the druid makes a Reflex attack to entangle the creature, causing it to waste the rest of its movement.
        % Your accuracy is equal to your druid level \add your Constitution.
        % The plants appear on any surface, and will continue to grow if they can survive, though they may die quickly if they appear on inhospitable terrain.
        % \subcf{17th -- Flaming Soul}
        % You gain the fire subtype, making your immune to fire but giving your a 50\% vulnerability to cold damage.
        % In addition, when you deal fire damage to a creature, the creature is \ignited for 5 rounds.
        % \subcf{17th -- Sunblessed Rejuvenation}
        % You gain fast healing equal to your druid level as long as you remain in sunlight or touches a plant of your size or larger.
        % \subcf{17th -- Sunscour}
        % This aspect functions like the heart of the sun natural aspect, except that it also suppresses shadow effects and the visual components of illusions within the area of bright light.

        % \subcf{17th -- Water's Flow}
        % As a minor action, the druid can transform herself into a rushing flow of water with a volume roughly equal to your normal volume until the end of your turn.
        % In this form, she may move wherever water could go, but she cannot take other actions, such as jumping, attacking, or casting spells.
        % Your speed is halved when moving uphill and doubled when moving downhill.
        % She may move through squares occupied by enemies without penalty.
        % She may return to your normal form as a free action.
        % \par If the water is split, she may reform from anywhere the water has reached, to as little as a single ounce of water.
        % If not even an ounce of water exists contiguously, your body reforms from the largest available parts of water, cut into pieces of appropriate size.
        % This usually causes the druid to die.

    \subsection{Ex-Druids}
        A druid who ceases to revere nature or who changes to a prohibited alignment loses all magical druid class abilities.
        They cannot thereafter gain levels as a druid until they atone for their transgressions.

        % \subsection{Variant Druids}

        %     \subsubsection{Blighter}

        %         Blighters draw power from nature, as do other druids. However, while other druids revere nature and draw power from it gently, blighters steal power from nature forcefully. Wherever a blighter goes, destruction and death surely follows.

        %         \altcf{Blight} Instead of meditating to regain spell slots, a blighter draws power from your environment forcefully.
        %         This affects a \areahuge radius zone centered on your, and the process takes 1 minute of concentration.
        %         At the end of every round, every living thing in the area other than the blighter takes damage equal to your nature power.
        %         All inanimate plants of Huge size or smaller immediately wither and die.
        %         The earth becomes cracked and infertile, and any nutrients from the soil are destroyed.
        %         This ability has no effect on artificial environments or materials, such as metal or worked stone.
        %         At the end of the minute, the blighter regains your spent nature spell slots.

        %         A blighter can only blight your surroundings in this way once per hour.
        %         If your surroundings are already blighted or are not natural terrain, she cannot use this ability to regain your spells.
        %         Instead, she must meditate for 8 hours to slowly draw power from your surroundings, as a normal druid.

        %         \altcf{Spells} As normal, except that a blighter adds all Vivimancy arcane spells to your spell list.

        %         \altcf[2]{Wild Speech} As normal, except that a blighter gains a \plus5 bonus to Intimidate against your wild speech targets, and a \minus5 penalty to Persuasion.

        %         \altcf[10]{Blightcasting}

        %         \altcf[20]{Improved Blightcasting}

        % \subsubsection{Rotbringer}

        %     While most druids seek to emulate and interact with animals, rotbringers focus on the power of fungi, decay, and regeneration.

        %     \altcf{Invoke Rot} Instead of meditating to regain spell slots, a rotbringer accelerates the natural forces of decomposition and decay on your environment.
        %     This affects a \areahuge radius zone centered on your, and the process takes 1 minute of concentration.
        %     All organic objects of Huge size or smaller, such as plants and corpses, decompose.
        %     This decomposition kills inanimate, living plants.
        %     All organic objects, regardless of size, are covered with various fungi.
        %     This ability has no effect on artificial environments or materials, such as metal or worked stone.
        %     At the end of the minute, the rotbringer regains your spent nature spell slots.

        %     If the rotbringer decomposes a Huge object with this ability, or a combination of smaller objects equivalent in size to a Huge object, you gain an bonus nature spell slot of your highest available spell level.
        %     This extra spell slot lasts until it is used, or until you regain your spell slots again.

        %     A rotbringer can only invoke rot on your surroundings in this way once per hour.
        %     If your surroundings are already decomposed or are not natural terrain, she cannot use this ability to regain your spells.
        %     Instead, she must meditate for 8 hours to slowly draw power from your surroundings, as a normal druid.

        %     \altcf[2]{Wild Speech} The rotbringer gains the ability to speak with plants at 2nd level.
        %     You gain the ability to speak with animals at 6th level, instead of at 2nd level.

        %     \altcf[3rd]{Wild Aspect} The rotbringer does not gain this ability.

        %     \altcf[3rd]{Rot Spell} The druid learns an additional spell slot and spell known.
        %     The spell must be taken from the following list of spells.
        %     The spell's level cannot exceed half your druid level.
        %     If she already knows a spell from the list at every spell level you ha access to, she may instead learn any nature spell (see \pcref{Nature Spells}).

        %     At 5th level, and every odd level, the druid may learn a new spell.

        %     \begin{dtable}
        %         \begin{dtabularx}{\columnwidth}{l X}
        %             \tb{Spell level} & \tb{Rotbringer Spells} \\
        %             1st & \spell{excrete slime}, \spell{lesser regeneration} \\
        %             2nd & \spell{fungal growth} \\
        %             3rd & \spell{rotburst} \\
        %             4th & \spell{poison} \\
        %             6th & \spell{regeneration} \\
        %             7th & \spell{greater rotburst} \\
        %         \end{dtabularx}
        %     \end{dtable}

        %     \altcf[7]{Fungal Armor} The rotbringer becomes covered in fungus that protects your from attacks. You gain a \plus1 bonus to Armor and Fortitude defense.

        %     This bonus increases by 1 at your 7th druid level, and every 4 druid levels thereafter.

\newpage
\section{Fighter}\label{Fighter}
    \begin{dtable!*}
        \lcaption{Fighter Progression}
        \begin{dtabularx}{\textwidth}{l l >{\lcol}X >{\lcol}X >{\lcol}X >{\lcol}X}
            \tb{Rank} & \tb{Min Level} & \tb{Combat Discipline}       & \tb{Equipment Training}      & \tb{Martial Mastery}   & \tb{Tactician} \tableheaderrule
            1 & 1  & Discipline                   & Armor expertise              & Martial maneuvers      & Battle tactics                \\
            2 & 2  & Enduring discipline          & Weapon training              & Martial lore           & Tactical advice               \\
            3 & 5  & Disciplined stance           & Equipment efficiency         & Martial power          & Tactical coordination         \\
            4 & 8  & Disciplined reaction         & Greater armor expertise      & Martial maneuver       & Battle tactic                 \\
            5 & 11 & Greater enduring discipline  & Weapon expertise             & Greater martial power  & Greater battle tactics        \\
            6 & 14 & Disciplined vitality         & Greater equipment efficiency & Steady expertise       & Greater tactical coordination \\
            7 & 17 & Supreme disciplined reaction & Supreme armor expertise      & Martial maneuver       & Battle tactic                 \\
            8 & 20 &                              &                              &                       \\
        \end{dtabularx}
    \end{dtable!*}

    Fighters are disciplined warriors that excel in all aspects of physical combat.
    What they lack in utility, they make up for with tactical talent, combat experience, and physical skill.
    Fighters can span all variety of alignments, social stations, and roles in life.

    \classbasics{Alignment} Any.

    \classbasics{Archetypes} Fighters have the Combat Discipline, Equipment Training, Martial Mastery, and Tactician \glossterm{archetypes}.

    \subsection{Basic Class Abilities}
        If you are a fighter, you gain the following abilities.

        \cf{Ftr}{Defenses}
        You gain the following bonuses to your \glossterm{defenses}: \plus3 Armor, \plus5 Fortitude, \plus3 Reflex, \plus4 Mental.

        \cf{Ftr}{Skills}
        You have the following \glossterm{class skills}:
        \begin{itemize}
            \item \subparhead{Strength} Climb, Jump, Swim.
            \item \subparhead{Dexterity} Acrobatics, Escape Artist, Ride.
            \item \subparhead{Constitution} Endurance.
            \item \subparhead{Intelligence} Craft.
            \item \subparhead{Perception} Awareness.
            \item \subparhead{Other} Deception, Intimidate, Persuasion, Profession.
        \end{itemize}

        \cf{Ftr}{Weapon Proficiencies} 
        You are proficient with simple weapons and any three other \glossterm{weapon groups}.

        \cf{Ftr}{Armor Proficiencies} 
        You are proficient with all armor.

    \subsection{Combat Discipline}
        This archetype allows you to improve your defenses and resist conditions.

        \cf{Ftr}[1]{Discipline} You can use the \textit{discipline} ability as a \glossterm{standard action}.
        \begin{freeability}{Discipline}
            Remove the most recent \glossterm{condition} affecting you.
            This cannot remove a condition applied during the current round.

            \rankline
            \rank{3} You can remove an additional \glossterm{condition}.
            \rank{5} This ability gains the \glossterm{Swift} tag.
            When you use it, the penalties from the removed conditions do not affect you during the current phase.
            \rank{7} You can remove any number of \glossterm{conditions}.
        \end{freeability}

        \cf{Ftr}[2]{Enduring Discipline}
        You gain an additional \glossterm{hit point}.

        \cf{Ftr}[3]{Disciplined Stance}
        You learn how to control your body to adapt to any threat you face.
        You can use the \textit{disciplined stance} ability as a \glossterm{minor action}.
        \begin{freeability}{Disciplined Stance}
            You gain a \plus2 bonus to one \glossterm{defense} of your choice and take a \minus2 penalty to a different defense of your choice.
            This ability lasts until you use it again.

            \rankline
            \rank{5} The defense penalty is reduced to \minus1.
            \rank{7} The defense penalty is removed.
        \end{freeability}

        \cf{Ftr}[4]{Disciplined Reaction}
        You do not suffer any effects from \glossterm{conditions} in the first round that they are applied.
        You suffer their normal effects in the following round.
        You gain a \plus2 bonus to the Deception and Intimidate skills.

        \cf{Ftr}[5]{Greater Enduring Discipline}
        You gain an additional \glossterm{hit point}.

        \cf{Ftr}[6]{Disciplined Vitality}
        You do not suffer any effects from \glossterm{vital wounds} until the end of the next round after they are applied.
        You suffer their normal effects after that time.

        \cf{Ftr}[7]{Supreme Disciplined Reaction}
        You do not suffer any effects from \glossterm{conditions} until the end of the next round after they are applied.
        You suffer their normal effects after that time.

    \subsection{Equipment Training}
        This archetype improves your combat prowess with weapons and armor.

        \cf{Ftr}[1]{Armor Expertise}
        You gain a \plus1 bonus to Armor defense while wearing body armor.
        In addition, you reduce the \glossterm{encumbrance} of body armor you wear by 1.

        \cf{Ftr}[2]{Weapon Training} You can use the \textit{weapon training} ability by spending an hour training with a weapon.
        You cannot use this ability with an \glossterm{exotic weapon} unless you are proficient with the \glossterm{weapon group} that weapon belongs to.
        \begin{freeability}{Weapon Training}
            You become proficient with the weapon you trained with.
            This ability's effect lasts until you use this ability again.

            \rankline
            \rank{4} You can use this ability with only five minutes of training.
            \rank{6} The effect is permanent.
            \rank{8} You can use this ability as a \glossterm{minor action}.
        \end{freeability}

        % TODO: wording
        \cf{Ftr}[3]{Equipment Efficiency} You can \glossterm{attune} to an additional magic weapon or magic armor without spending an \glossterm{action point}.
        After using this ability, you cannot use it again until you take a \glossterm{long rest}.

        \cf{Ftr}[4]{Greater Armor Expertise}
        The \glossterm{encumbrance} reduction from your \textit{armor expertise} ability increases to 2.
        In addition, you treat body armor were one encumbrance category lighter than normal when doing so would be beneficial for you.

        \cf{Ftr}[5]{Weapon Expertise} You gain a \plus1 bonus to \glossterm{accuracy} with \glossterm{mundane} abilities.

        \cf{Ftr}[6]{Greater Equipment Efficiency} The number of additional attunements you gain with your \textit{efficient equipment} ability increases to two.

        \cf{Ftr}[7]{Supreme Armor Expertise}
        The \glossterm{encumbrance} reduction from your \textit{armor expertise} ability increases to 3.
        In addition, you treat body armor as if it were an additional encumbrance category lighter than normal when doing so would be beneficial for you.

    \subsection{Martial Mastery}
        This archetype grants you special abilities to use in combat.

        \cf{Ftr}[1]{Martial Maneuvers}
        You can channel your martial prowess into devastating attacks.
        You learn two \glossterm{maneuvers} from the martial maneuver list (see \pcref{Martial Maneuvers}).
        You can also spend \glossterm{insight points} to learn one additional \glossterm{maneuver} per \glossterm{insight point}.
        As a \glossterm{standard action}, you can use any \glossterm{maneuver} you know.

        \cf{Ftr}[2]{Martial Expertise} You gain a \plus1 bonus to Armor defense.

        \cf{Ftr}[3]{Martial Power} You gain a \plus1 bonus to \glossterm{power} with \glossterm{mundane} abilities.

        \cf{Ftr}[4]{Martial Maneuver}
        You learn an additional \textit{martial maneuver}.

        \cf{Ftr}[5]{Greater Martial Expertise} The bonus from your \textit{martial expertise} ability increases to \plus2.

        \cf{Ftr}[6]{Greater Martial Power} The bonus from your \textit{martial power} ability increases to \plus2.

        \cf*{Ftr}[7]{Martial Maneuver}
        You learn an additional \textit{martial maneuver}.

    \subsection{Tactician}

        \cf{Ftr}[1]{Battle Tactics}
        You can lead your allies using tactics appropriate for the situation.
        Choose a single battle tactic from the list below.
        You can also spend \glossterm{insight points} to learn one additional \textit{battle tactic} per \glossterm{insight point}.

        You can initiate a \textit{battle tactic} as a standard action.
        Your \textit{battle tactics} affect your \glossterm{allies} within a \arealarge radius emanation from you who can either see or hear you.

        All \textit{battle tactics} have the \glossterm{Attune} (self) tag, so they last as long as you \glossterm{attune} to them (see \pcref{Attunement}).
        If you know multiple battle tactics, you can initiate multiple \textit{battle tactics} simultaneously.
        However, a creature can gain only the benefits of one \textit{battle tactic} at a time.
        If it has multiple options, it chooses at the start of each round which \textit{battle tactic} to follow.

        {
            \begin{attuneability}{Break Through}[\glossterm{Attune} (self)]
                Each target that is adjacent to at least one other target
                    gains a \plus2 bonus to \glossterm{accuracy} with the \textit{overrun} and \textit{shove} abilities (see \pcref{Special Combat Abilities}).

                \rankline
                \rank{3} The bonus increases to \plus3.
                \rank{5} The bonus increases to \plus4.
                \rank{7} The bonus increases to \plus4.
            \end{attuneability}

            \begin{attuneability}{Dogpile}[\glossterm{Attune} (self)]
                Each target that is adjacent to at least one other target
                gains a \plus2 bonus to \glossterm{accuracy} with the \textit{grapple} ability and with all grapple actions (see \pcref{Grapple}, and \pcref{Grapple Actions}).

                \rankline
                \rank{3} The bonus increases to \plus3.
                \rank{5} The bonus increases to \plus4.
                \rank{7} The bonus increases to \plus5.
            \end{attuneability}

            \begin{attuneability}{Duck and Cover}[\glossterm{Attune} (self)]
                Each target gains a \plus1 bonus to Armor defense against ranged \glossterm{physical attacks}.

                \rankline
                \rank{3} The bonus increases to \plus2.
                \rank{5} The bonus increases to \plus3.
                \rank{7} The bonus increases to \plus7.
            \end{attuneability}

            \begin{attuneability}{Group Up}[\glossterm{Attune} (self)]
                Each target that is adjacent to at least two other targets gains a \plus1 bonus to Armor defense.

                \rankline
                \rank{3} Each target affected by the Armor defense bonus also gains a \plus2 bonus to Mental defense.
                \rank{5} The Armor defense bonus increases to \plus2.
                \rank{7} The Mental defense bonus increases to \plus4.
            \end{attuneability}

            \begin{attuneability}{Hold Fast}[\glossterm{Attune} (self)]
                Each target that ends the \glossterm{movement phase} without moving gains a \plus1 bonus to Armor defense until it moves.

                \rankline
                \rank{3} Each target affected by the Armor defense bonus also gains a \plus2 bonus to Fortitude defense.
                \rank{5} The Armor defense bonus increases to \plus2.
                \rank{7} The Fortitude defense bonus increases to \plus4.
            \end{attuneability}

            \begin{attuneability}{Hustle}[\glossterm{Attune} (self)]
                Each target gains a \plus5 foot bonus to its \glossterm{base speed}.

                \rankline
                \rank{3} The speed bonus increases to \plus10 feet.
                \rank{5} The speed bonus increases to \plus15 feet.
                \rank{7} The speed bonus increases to \plus20 feet.
            \end{attuneability}

            \begin{attuneability}{Keep Moving}[\glossterm{Attune} (self)]
                Each target that ends the \glossterm{movement phase} at least twenty feet away from where it started the round gains a \plus1 bonus to Armor defense until the end of the round.

                \rankline
                \rank{3} Each target affected by the Armor defense bonus also gains a \plus2 bonus to Reflex defense.
                \rank{5} The Armor defense bonus increases to \plus2.
                \rank{7} The Reflex defense bonus increases to \plus4.
            \end{attuneability}
        }

        \cf{Ftr}[2]{Tactical Advice} You can use the \textit{tactical advice} ability as a standard action.
        \begin{freeability}{Tactical Advice}[\glossterm{Swift}]
            Choose an \glossterm{ally} within \rngmed range.
            During the current phase, the target gains a \plus2 bonus to \glossterm{accuracy} and rolls twice for any attacks it makes, keeping the better result.

            \rankline
            \rank{4} The accuracy bonus increases to \plus2.
            \rank{6} The accuracy bonus increases to \plus3.
            \rank{8} The accuracy bonus increases to \plus4.
        \end{freeability}

        \cf{Ftr}[3]{Tactical Coordination} You gain a \plus1 bonus to \glossterm{accuracy} against \glossterm{overwhelmed} creatures.

        \cf{Ftr}[4]{Battle Tactic} You learn an additional \textit{battle tactic}.

        \cf{Ftr}[5]{Greater Battle Tactics} The area affected by your \textit{battle tactics} increases to an \areahuge emanation.
        In addition, if are attuned to a \textit{battle tactics} ability, you can change which tactic you are attuned to as a \glossterm{minor action}.

        \cf{Ftr}[6]{Greater Tactical Coordination} The bonus from your \textit{tactical coordination} ability increases to \plus2.

        \cf*{Ftr}[7]{Battle Tactic} You learn an additional \textit{battle tactic}.

\newpage
\section{Monk}\label{Monk}
    \begin{dtable!*}
        \lcaption{Monk Progression}
        \begin{dtabularx}{\textwidth}{l l >{\lcol}X >{\lcol}X >{\lcol}X >{\lcol}X}
            \tb{Rank} & \tb{Min Level} & \tb{Esoteric Warrior}               & \tb{Ki}                                & \tb{Perfected Form}        & \tb{Transcendent Sage} \tableheaderrule
            1 & 1  & Esoteric maneuvers, unarmed warrior & Ki barrier, ki manifestations          & Fast movement              & Clear the mind        \\
            2 & 2  & Esoteric lore                       & Ki strike                              & Graceful athletics         & Transcend frailty     \\
            3 & 5  & Esoteric precision                  & Ki power                               & Perfect body               & Inner peace           \\
            4 & 8  & Esoteric maneuver                   & Ki manifestation                       & Greater fast movement      & Feel the flow of life \\
            5 & 11 & Greater esoteric lore               & Greater ki barrier                     & Greater graceful athletics & Transcend flesh       \\
            6 & 14 & Greater esoteric precisoin          & Greater ki power                       & Greater perfect body       & Greater inner peace   \\
            7 & 17 & Esoteric maneuver                   & Ki manifestation                       & Supreme fast movement      & See the flow of life  \\
            8 & 20 &                                     & Dual manifestation, supreme ki barrier &                            & Transcend mortality   \\
        \end{dtabularx}
    \end{dtable!*}

    Monks are agile warriors that use intensive training to move beyond the limits of their physical bodies.
    They are usually lightly armored and use light weapons - if they use weapons or armor at all.
    The intense dedication monks must have to their training means they tend not to be chaotic.

    \classbasics{Alignment} Any.

    \classbasics{Archetypes} Monks have the Ki, Esoteric Warrior, Transcendent Sage, and Perfected Form \glossterm{archetypes}.

    \subsection{Basic Class Abilities}
        If you are a monk, you gain the following abilities.

        \cf{Mnk}{Defenses}
        You gain the following bonuses to your \glossterm{defenses}: \plus3 Armor, \plus3 Fortitude, \plus5 Reflex, \plus4 Mental.

        \cf{Mnk}{Skills}
        You have the following \glossterm{class skills}:
        \begin{itemize}
            \item \subparhead{Strength} Climb, Jump, Swim.
            \item \subparhead{Dexterity} Acrobatics, Escape Artist, Ride, Stealth.
            \item \subparhead{Constitution} Endurance.
            \item \subparhead{Intelligence} Craft, Deduction, Heal.
            \item \subparhead{Perception} Awareness, Spellcraft, Survival.
            \item \subparhead{Other} Deception, Intimidate, Perform, Persuasion, Profession.
        \end{itemize}

        \cf{Mnk}{Weapon Proficiencies} 
        You are proficient with simple weapons, monk weapons, and any one other \glossterm{weapon group}.

        \cf{Mnk}{Armor Proficiencies} 
        You are not proficient with any armor.

    \subsection{Esoteric Warrior}\label{Esoteric Warrior}
        This archetype improves your combat prowess with unusual fighting styles like unarmed attacks and grappling.

        \cf{Mnk}[1]{Esoteric Maneuvers} 
        You learn two \glossterm{maneuvers} from the esoteric maneuver list (see \pcref{Esoteric Maneuvers}).
        You can also spend \glossterm{insight points} to learn one additional \glossterm{maneuver} per \glossterm{insight point}.
        As a \glossterm{standard action}, you can use any \glossterm{maneuver} you know.

        \cf{Mnk}[1]{Unarmed Warrior} You become \glossterm{proficient} with the unarmed weapons \glossterm{weapon group} (see \pcref{Unarmed Weapons}).
        In addition, you gain a \plus2d bonus to damage with \glossterm{Unarmed} weapons.
        For details about how to fight while unarmed, see \pcref{Unarmed Combat}.

        \cf{Mnk}[2]{Esoteric Lore} You gain two additional \glossterm{skill points}.
        In addition, you gain a \plus1 bonus to Dexterity-based checks, except \glossterm{initiative} checks.

        \cf{Mnk}[3]{Esoteric Precision} You gain a \plus1 bonus to \glossterm{accuracy} with \glossterm{physical attacks} using weapons from the monk weapons and unarmed weapons \glossterm{weapon groups}, and to any physical attack using a free hand.
        In addition, you gain a \plus1 bonus to \glossterm{accuracy} with grapple actions (see \pcref{Grapple Actions}).

        % TODO: wording?
        % \cf{Mnk}[3]{Evasion} When you are attacked by an ability that affects an area, you can use your Reflex defense in place of any other defenses against that attack.

        % \cf{Mnk}[3]{Unfettered Athletics} You gain two additional skill points.

        % \cf{Mnk}[6]{Intuitive Reaction}
        % You gain a \plus2 bonus to Reflex defense and \glossterm{initiative}.

        \cf{Mnk}[4]{Esoteric Maneuver} You learn an additional esoteric \glossterm{maneuver} (see \pcref{Esoteric Maneuvers}).

        \cf{Mnk}[5]{Greater Esoteric Lore} The bonus from your \textit{esoteric lore} ability increases to \plus3.

        \cf{Mnk}[6]{Greater Esoteric Precision} The bonus from your \textit{esoteric precision} ability increases to \plus2.

        \cf*{Mnk}[7]{Esoteric Maneuver} You learn an additional esoteric \glossterm{maneuver} (see \pcref{Esoteric Maneuvers}).

    \subsection{Ki}
        This archtype grants you abilities you can use in combat.
        If you wear armor, use a shield, or have \glossterm{encumbrance}, you lose the benefit of all abilities from this archetype.

        \cf{Mnk}[1]{Ki Barrier}[Magical]
        If you are not wearing armor and have no \glossterm{encumbrance}, you gain a ki barrier around your body.
        This functions like body armor that provides a \plus2 bonus to Armor defense and a bonus to \glossterm{resistances} against \glossterm{energy damage} equal to half your level (minimum 1).
        The armor has no \glossterm{encumbrance}.

        As long as you have a free hand, the barrier also manifests as a shield that provides a \plus1 bonus to Armor defense.
        This bonus is considered to come from a shield, and does not stack with the benefits of using a physical shield.

        \cf{Mnk}[1]{Ki Manifestations}[Magical]
        You can channel your ki to temporarily enhance your abilities.
        Choose one \textit{ki manifestation} from the list below.
        You can also spend \glossterm{insight points} to learn one additional \textit{ki manifestation} per \glossterm{insight point}.
        You can use any \textit{ki manifestation} ability you know using the type of action indicated in the ability's description.

        After you use a \textit{ki manifestation}, you cannot use a \textit{ki manifestation} until after the end of the next round.
        {
            \begin{freeability}{Abandon the Fragile Self}[\glossterm{Swift}]
                You can use this ability as a \glossterm{free action}.
                You can negate one \glossterm{condition} that would be applied to you this phase.
                In exchange, you take a \minus2 penalty to \glossterm{defenses} this phase.

                \rankline
                \rank{3} You can negate any number of conditions instead of only one condition.
                \rank{5} The effect lasts until the end of the round.
                \rank{7} The defense penalty is reduced to \minus1.
            \end{freeability}

            \begin{freeability}{Burst of Blinding Speed}[\glossterm{Swift}]
                You can use this ability as a \glossterm{free action}.
                You gain a \plus10 foot bonus to your land speed this phase.

                \rankline
                \rank{3} You can also ignore \glossterm{difficult terrain} this phase.
                \rank{5} The speed bonus increases to \plus20 feet.
                \rank{7} You can also move or stand on liquids as if they were solid this phase.
            \end{freeability}

            \begin{freeability}{Elegant Whirl of Fluid Motion}[\glossterm{Swift}]
                You can use this ability as a \glossterm{free action}.
                You gain a \plus5 bonus to the Acrobatics skill this round (see \pcref{Acrobatics}).

                \rankline
                \rank{3} The bonus increases to \plus10.
                \rank{5} The bonus lasts until the end of the next round.
                \rank{7} The bonus increases to \plus20.
            \end{freeability}

            \begin{freeability}{Extend the Flow of Ki}[\glossterm{Swift}]
                You can use this ability as a \glossterm{free action}.
                You gain a \plus5 foot \glossterm{magic bonus} to \glossterm{reach} this phase.

                \rankline
                \rank{3} 
                \rank{5} The bonus to \glossterm{reach} increases to 10 feet.
                \rank{7} 
            \end{freeability}

            \begin{freeability}{Flash Step}[\glossterm{Teleportation}]
                You can use this ability as part of movement.
                % TODO: is 'horizontally' the correct word?
                You teleport horizontally instead of moving normally.
                If your \glossterm{line of effect} to your destination is blocked, or if this teleportation would somehow place you inside a solid object, your teleportation is cancelled and you remain where you are.

                Teleporting a given distance costs movement equal to twice that distance.
                For example, if you have a 30 foot movement speed, you can move 10 feet, teleport 5 feet, and move an additional 10 feet before your movement ends.

                \rankline
                \rank{3} You can use this ability to move even if you are \glossterm{immobilized} or \glossterm{grappled}.
                \rank{5} The movement cost to teleport is reduced to be equal to the distance you teleport.
                \rank{7} You can attempt to teleport to locations outside of \glossterm{line of sight} and \glossterm{line of effect}.
                If your intended destination is invalid, the distance you spent teleporting is wasted, but you suffer no other ill effects.
            \end{freeability}

            \begin{freeability}{Leap of the Heavens}[\glossterm{Swift}]
                You can use this ability as a \glossterm{free action}.
                You gain a \plus5 bonus to the Jump skill this round (see \pcref{Jump}).

                \rankline
                \rank{3} The bonus increases to \plus10.
                \rank{5} The bonus lasts until the end of the next round.
                \rank{7} The bonus increases to \plus20.
            \end{freeability}

            \begin{freeability}{Scale the Highest Tower}[\glossterm{Swift}]
                You can use this ability as a \glossterm{free action}.
                You gain a \plus5 bonus to the Climb skill this round (see \pcref{Climb}).
                % TODO: is this wording correct?

                \rankline
                \rank{3} The Climb bonus increases to \plus10.
                \rank{5} The bonus lasts until the end of the next round.
                \rank{5} The bonus increases to \plus20.
            \end{freeability}

            \begin{freeability}{See the Flow of Life}[\glossterm{Swift}]
                You can use this ability as a \glossterm{free action}.
                You gain the \glossterm{lifesense} ability with a 50 foot range this phase.

                \rankline
                \rank{3} The range of the \glossterm{lifesense} ability is increased to 100 feet.
                \rank{5} You also gain the \glossterm{lifesight} ability with a 50 foot range.
                You can ``see'' any living creatures and their equipment within range perfectly, regardless of lighting conditions, blindness, invisibility, or any other means of concealment.
                \rank{7} The range of the \glossterm{lifesense} ability is increased to 200 feet, and the range of the \glossterm{lifesight} ability is increased to 100 feet.
            \end{freeability}

            \begin{freeability}{Sense the Mystic Truth}[\glossterm{Swift}]
                You can use this ability as a \glossterm{free action}.
                You gain a \plus5 bonus to the Spellcraft skill this round (see \pcref{Spellcraft}).

                \rankline
                \rank{3} The bonus increases to \plus10.
                \rank{5} The bonus lasts until the end of the next round.
                \rank{7} The bonus increases to \plus20.
            \end{freeability}

            \begin{freeability}{Step Between the Mystic Worlds}[\glossterm{Swift}]
                You can use this ability as a \glossterm{free action}.
                You gain a \plus2 bonus to \glossterm{defenses} against \glossterm{magical} abilities this phase.
                After the effect ends, you take a \minus2 penalty to \glossterm{defenses} against \glossterm{magical} attacks until the end of the next round.

                \rankline
                \rank{3} The defense bonus is increased to \plus3.
                \rank{5} The effect lasts until the end of the current round.
                \rank{7} The defense bonus is increased to \plus5.
            \end{freeability}

            \begin{freeability}{Surpass the Mortal Limits}[\glossterm{Swift}]
                You can use this ability as a \glossterm{free action}.
                You can use your \glossterm{power} in place of your Strength, Dexterity, and Constitution when making checks this phase.

                \rankline
                \rank{3} You also gain a \plus2 bonus to checks based on Strength, Dexterity, and Constitution.
                \rank{5} The effect lasts until the end of the current round.
                \rank{7} The bonus increases to \plus4.
            \end{freeability}

            % TODO: add more
        }

        \cf{Mnk}[2]{Ki Strike} You can treat all strikes you make as \glossterm{magical strikes}, allowing you to use your \glossterm{power} with magical abilities to determine your damage.

        \cf{Mnk}[3]{Ki Power} You gain a \plus1 bonus to \glossterm{power} with \glossterm{magical} abilities.

        \cf{Mnk}[4]{Ki Manifestation}[Magical]
        You learn an additional \textit{ki manifestation}.

        \cf{Mnk}[5]{Greater Ki Barrier}[Magical] The defense bonus from the body armor provided by your \textit{ki barrier} ability increases to \plus3.
        In addition, the bonus to \glossterm{resistances} increases to be equal to your level.

        \cf{Mnk}[6]{Greater Ki Power} The bonus from your \textit{ki power} ability increases to \plus2.

        \cf*{Mnk}[7]{Ki Manifestation}[Magical]
        You learn an additional \textit{ki manifestation}.

        \cf{Mnk}[8]{Rapid Manifestation}[Magical] After using a \textit{ki manifestation} ability, you can use another one after the end of the current round instead of the end of the next round. 

    \subsection{Perfected Form}
        This archetype improves the perfection of your physical body through rigorous training.

        \cf{Mnk}[1]{Fast Movement} You gain a \plus10 foot bonus to your \glossterm{base speed}.

        \cf{Mnk}[2]{Graceful Athletics} You gain a \plus2 bonus to Strength-based and Dexterity-based checks.

        \cf{Mnk}[3]{Perfect Body} You gain a \plus1 bonus to the base value of one physical \glossterm{attribute} of your choice: Strength, Dexterity, or Constitution.

        \cf{Mnk}[4]{Greater Fast Movement} The speed bonus from your \textit{fast movement} ability increases to \plus20 feet.

        \cf{Mnk}[5]{Greater Graceful Athletics} The bonus from your \textit{graceful athletics} ability increases to \plus4.

        \cf{Mnk}[6]{Greater Perfect Body} The bonus from your \textit{perfect body} ability applies to the base value of all physical attributes, not just the one you chose.

        \cf{Mnk}[7]{Supreme Fast Movement} The speed bonus from your \textit{fast movement} ability increases to \plus30 feet.

    \subsection{Transcendent Sage}
        This archetype grants you abilities to resist or remove conditions.

        \cf{Mnk}[1]{Clear the Mind} You can use the \textit{clear the mind} ability as a standard action.
        \begin{freeability}{Clear the Mind}
            Remove the most recent \glossterm{condition} affecting you.
            This cannot remove a condition applied during the current round.

            \rankline
            \rank{3} You can remove an additional \glossterm{condition}.
            \rank{5} This ability gains the \glossterm{Swift} tag.
            When you use it, the penalties from the removed conditions do not affect you during the current phase.
            \rank{7} You can remove any number of \glossterm{conditions}.
        \end{freeability}

        \cf{Mnk}[2]{Transcend Frailty}[Magical]
        You are immune to being \glossterm{sickened} and \glossterm{nauseated}.

        \cf{Mnk}[3]{Inner Peace} You are immune to hostile \glossterm{Emotion} abilities.

        \cf{Mnk}[4]{Feel the Flow of Life}[Magical] You gain the \glossterm{lifesense} ability with a 50 foot range.

        \cf{Mnk}[5]{Transcend Flesh}[Magical]
        You are immune to hostile \glossterm{Flesh} abilities.
        In addition, you no longer take penalties to your attributes for aging, and cannot be magically aged.
        You still die of old age when your time is up.

        \cf{Mnk}[6]{Greater Inner Peace}
        You are immune to hostile \glossterm{Compulsion} abilities.

        \cf{Mnk}[7]{See the Flow of Life}[Magical]
        You gain the \glossterm{lifesight} ability with 50 foot range.

        % This feels like a rank 8 ability
        \cf{Mnk}[8]{Transcend Mortality}[Magical]
        If you die, you may choose to retain control of your body and soul through sheer force of will.
        Your body immediately disappears, and your soul does not travel to an afterlife.
        Instead, your body reforms with no trace of its injuries 8 hours later.
        The reformed body is in perfect health and can be any age you choose, to a minimum of the age of adulthood for your species.
        You can reform your body at the place where you died, or in any place on the same plane that is deeply familiar to you.

        After each time you reform yourself in this way, it takes an additional hour to reform the next time you ``die''.
        You can only be permanently killed by the direct intervention of a deity.

    \subsection{Ex-Monks}
        As long as you are chaotic, you lose all of your \glossterm{magical} monk abilities.

\newpage
\section{Paladin}\label{Paladin}
    \begin{dtable!*}
        \lcaption{Paladin Progression}
        \begin{dtabularx}{\textwidth}{l l >{\lcol}X >{\lcol}X >{\lcol}X >{\lcol}X}
            \tb{Rank} & \tb{Min Level} & \tb{Devoted Paragon}  & \tb{Divine Magic} & \tb{Stalwart Guardian}      & \tb{Zealous Warrior}       \tableheaderrule
            1 & 1  & Aligned aura          & Spellcasting   & Lay on hands                & Smite                      \\
            2 & 2  & Aligned immunity      & Battle prayer  & Stalwart champion           & Zealous conviction         \\
            3 & 5  & Paragon power         & Spell rank (3) & Stalwart resilience         & Zealous power              \\
            4 & 8  & Greater aligned aura  & Spell rank (4) & Greater lay on hands        & Greater smite              \\
            5 & 11 & Devoted mind          & Spell rank (5) & Greater stalwart champion   & Greater zealous conviction \\
            6 & 14 & Greater paragon power & Spell rank (6) & Greater stalwart resilience & Greater zealous power      \\
            7 & 17 & Supreme aligned aura  & Spell rank (7) & Supreme lay on hands        & Supreme smite              \\
            8 & 20 & Aligned soul          & Spell rank (8) &                             &                            \\
        \end{dtabularx}
    \end{dtable!*}

    \classbasics{Alignment} Any other than true neutral.

    \classbasics{Archetypes} Paladins have the Devoted Paragon, Divine Magic, Zealous Warrior, and Stalwart Guardian \glossterm{archetypes}.

    \subsection{Basic Class Abilities}
        If you are a paladin, you gain the following abilities.

        \cf{Pal}{Defenses}
        You gain the following bonuses to your \glossterm{defenses}: \plus3 Armor, \plus5 Fortitude, \plus3 Reflex, \plus4 Mental.

        \cf{Pal}{Skills}
        You have the following \glossterm{class skills}:
        \begin{itemize}
            \item \subparhead{Dexterity} Ride.
            \item \subparhead{Intelligence} Craft, Deduction, Heal, Knowledge (local, religion).
            \item \subparhead{Constitution} Endurance.
            \item \subparhead{Perception} Awareness, Sense Motive.
            \item \subparhead{Other} Deception, Intimidate, Persuasion, Profession.
        \end{itemize}

        \cf{Pal}{Weapon Proficiencies} 
        You are proficient with simple weapons and any three other \glossterm{weapon groups}.

        \cf{Pal}{Armor Proficiencies} 
        You are proficient with all armor.

        \cf{Pal}{Devoted Alignment} 
        You are devoted to a specific alignment.
        You must choose one of your alignment components: good, evil, lawful, or chaotic.
        The alignment you choose is your devoted alignment.
        Your paladin abilities are affected by this choice.
        % seems unnecessary
        % You excel at slaying creatures with alignments opposed to your devoted alignment.
        Your alignment cannot be changed without extraordinary repurcussions.

    \subsection{Devoted Paragon}
        This archetype deepens your connection to your alignment, granting you an aura and improving your combat abilities.

        \cf{Pal}[1]{Aligned Aura}[Magical]
        Your devotion to your alignment affects the world around you, bringing it closer to your ideals.
        You constantly radiate an aura in a \areamed radius \glossterm{emanation} from you.
        You can freely choose whether yourself and your \glossterm{allies}, \glossterm{enemies}, and other creatures are affected by the aura.
        The effect of the aura depends on your devoted alignment, as described below.
        You can suppress or resume the aura as a \glossterm{minor action}.

        \subparhead{Chaos} When a target rolls a 1 on an attack roll with a \glossterm{strike}, it \glossterm{explodes} (see \pcref{Exploding Attacks}.
        This does not affect bonus dice rolled for exploding attacks (see \pcref{Exploding Attacks}).
        \subparhead{Evil} Each target suffers a \minus1 penalty to \glossterm{defenses} as long as it is affected by at least one \glossterm{condition}.
        % TODO: clarify what happens if multiple people try to Good aura the same target
        \subparhead{Good} When a target gains a \glossterm{vital wound}, you may gain a \glossterm{vital wound} instead.
        You gain a \plus2 bonus to the \glossterm{wound roll} of each \glossterm{vital wound} you gain this way.
        The target suffers any other effects of the attack normally.
        \subparhead{Law} When a target rolls a 1 on an attack roll with a \glossterm{strike}, the attack roll is treated as a 6.
        This does not affect bonus dice rolled for exploding attacks (see \pcref{Exploding Attacks}).

        \cf{Pal}[2]{Aligned Immunity}[Magical]
        Your ability to resist attacks based on your alignment improves.

        \subparhead{Chaos} You are immune to hostile \glossterm{Compulsion} abilities.
        \subparhead{Evil} You are immune to poisons and diseases.
        \subparhead{Good} You are immune to being \glossterm{shaken}, \glossterm{frightened}, and \glossterm{panicked}.
        \subparhead{Law} You are immune to hostile \glossterm{Emotion} abilities.

        \cf{Pal}[3]{Paragon Power}
        You gain a \plus1 bonus to \glossterm{power} with \glossterm{magical} abilities.

        \cf{Pal}[4]{Greater Aligned Aura}[Magical]
        The effect of your \textit{aligned aura} becomes stronger, as described below.

        \subparhead{Chaos} The effect applies to all attacks, not just \glossterm{strikes}.
        \subparhead{Evil} When a target would remove a \glossterm{condition}, the condition instead remains unless it spends an additional \glossterm{action point}.
        \subparhead{Good} When a target loses a \glossterm{hit point}, you may lose that hit point instead.
        The target suffers any other effects of the attack normally.
        \subparhead{Law} The effect applies to all attacks, not just \glossterm{strikes}.

        \cf{Pal}[5]{Devoted Mind} You gain a \plus2 bonus to Mental defense.

        \cf{Pal}[6]{Greater Paragon Power} The bonus from your \textit{paragon power} ability increases to \plus2.

        \cf{Pal}[7]{Supreme Aligned Aura}
        The area of your \textit{aligned aura} increases to a \arealarge radius \glossterm{emanation} from you.

        \cf{Pal}[8]{Aligned Soul}[Magical]
        While you are dead, you may approach the deity or governing figure of your afterlife and request to be returned to life to continue your mission.
        Travelling to the relevant entity and making the request takes 12 hours.
        This request is almost always granted unless there are extenuating circumstances.
        You are resurrected in a new body at a location of the entity's choice, which is usually a place of power for the entity.
        This functions like the \ritual{resurrection} ritual, except that no part of the body is required, and a new body is created by the entity.
        You can be resurrected in this way regardless of the condition of your body, but not if your soul has been trapped or otherwise prevented from going to the correct afterlife.

    \subsection{Divine Magic}
        This archetype grants you the ability to cast divine spells.

        \cf{Pal}[1]{Spellcasting}[Magical]
        Your devotion to your alignment grants you the ability to use divine magic.
        You gain access to one divine \glossterm{mystic sphere} (see \pcref{Divine Mystic Spheres}).
        You may spend \glossterm{insight points} to learn one additional \glossterm{mystic sphere} per two \glossterm{insight points}.
        Each \glossterm{mystic sphere} has a set of \glossterm{spells} associated with it.

        You automatically learn all \glossterm{cantrips} from any mystic sphere you have access to.
        In addition, you learn two rank 1 \glossterm{spells} from your chosen \glossterm{mystic sphere}.
        You can also spend \glossterm{insight points} to learn one additional divine spell per \glossterm{insight point}.
        Unless otherwise noted in a spell's description, casting a spell requires a \glossterm{standard action}.

        When you gain access to a new \glossterm{mystic sphere} or spell \glossterm{rank},
            you can exchange any number of spells you know for other spells,
            including spells of the higher rank.

        Divine spells require \glossterm{verbal components} to cast (see \pcref{Casting Components}).
        For details about mystic spheres and casting spells, see \pcref{Spell and Ritual Mechanics}.

        \cf{Pal}[2]{Battle Prayer} You reduce your \glossterm{focus penalties} by 2.

        \cf{Pal}[3]{Spell Rank}[Magical] You become a rank 3 divine spellcaster.
        This can improve the effectiveness of your spells and give you access to spells that require a minimum rank of 3.

        \cf*{Pal}[4]{Spell Rank}[Magical] You become a rank 4 divine spellcaster.
        This can improve the effectiveness of your spells and give you access to spells that require a minimum rank of 4.

        \cf*{Pal}[5]{Spell Rank}[Magical] You become a rank 5 divine spellcaster.
        This can improve the effectiveness of your spells and give you access to spells that require a minimum rank of 5.

        \cf*{Pal}[6]{Spell Rank}[Magical] You become a rank 6 divine spellcaster.
        This can improve the effectiveness of your spells and give you access to spells that require a minimum rank of 6.

        \cf*{Pal}[7]{Spell Rank}[Magical] You become a rank 7 divine spellcaster.
        This can improve the effectiveness of your spells and give you access to spells that require a minimum rank of 7.

        \cf*{Pal}[8]{Spell Rank}[Magical] You become a rank 8 divine spellcaster.
        This can improve the effectiveness of your spells and give you access to spells that require a minimum rank of 8.

    \subsection{Stalwart Guardian}
        This archetype grants you healing abilities and improves your defensive prowess.

        \cf{Pal}[1]{Lay on Hands}[Magical] You can use the \textit{lay on hands} ability as a standard action.
        \begin{freeability}{Lay on Hands}[Life]
            One \glossterm{ally} within your \glossterm{reach} gains a \plus2 bonus to its most recent \glossterm{wound roll}.
            The \glossterm{wound roll} for that \glossterm{vital wound} cannot be modified again.

            \rankline
            \rank{3} The bonus increases to \plus3.
            \rank{5} When you use this ability, you can spend an \glossterm{action point}.
            If you do, the target's most recent \glossterm{vital wound} is removed instead of having its \glossterm{wound roll} modified.
            \rank{7} The bonus increases to \plus5.
        \end{freeability}

        \cf{Pal}[2]{Enduring Defender}
        You gain an additional \glossterm{hit point}.

        \cf{Pal}[3]{Stalwart Resilience}
        You gain a \plus1 bonus to Fortitude and Mental defense.
        In addition, you gain a \plus1 bonus to \glossterm{wound rolls}.

        \cf{Pal}[4]{Defender's Boon}
        Whenever you are attacked, you gain a \plus1 bonus to \glossterm{accuracy} until the end of the next round.
        This bonus does not stack with itself.

        \cf{Pal}[5]{Greater Enduring Defender}
        You gain an additional \glossterm{hit point}.

        \cf{Pal}[6]{Greater Stalwart Resilience} The bonuses from your \textit{stalwart resilience} ability increase to \plus2.

        \cf{Pal}[7]{Greater Defender's Boon}[Magical]
        The bonus from your \textit{defender's boon} ability increase to \plus2.

    \subsection{Zealous Warrior}
        This archetype improves your combat prowess, especially against foes who do not share your devoted alignment.

        \cf{Pal}[1]{Smite}[Magical] You can use the \textit{smite} ability as a standard action.
        \begin{freeability}{Smite}
            Make a \glossterm{magical strike}.
            If your target shares your devoted alignment, the strike deals no damage.
            Otherwise, the strike gains a \plus1d bonus to damage.

            \rankline
            \rank{3} The damage bonus increases to \plus2d.
            \rank{5} The damage bonus increases to \plus3d.
            \rank{7} The damage bonus increases to \plus4d.
        \end{freeability}

        \cf{Pal}[2]{Zealous Conviction} You gain a \plus1 bonus to Fortitude and Mental defenses.

        \cf{Pal}[3]{Zealous Purge} You can use your \textit{zealous purge} ability as a standard action.
        \begin{freeability}{Zealous Purge}
            Make a \glossterm{magical strike}.
            If the target is \glossterm{injured} by your attack, the target stops being \glossterm{attuned} to one ability of its choice that it is currently attuned to.
            This ability does not affect attunement to magic items.

            \rankline
            \rank{5} If you deal damage with a \glossterm{critical hit}, the target stops being \glossterm{attuned} to all abilities that it is attuned to other than \glossterm{magic items}.
            \rank{7} You gain a \plus1 bonus to \glossterm{accuracy} with the strike.
        \end{freeability}

        \cf{Pal}[6]{Pass Judgment}[Magical] When you use your \textit{smite} ability, you can use the \textit{pass judgment} ability.
        \begin{attuneability}{Pass Judgment}[\glossterm{Attune} (self)]
            The target of your \textit{smite} is treated as if it had the alignment opposed to your devoted alignment for the purpose of all abilities, including for your initial \textit{smite}.
            This only affects its alignment along the alignment axis your devoted alignment is on.
            For example, if your devoted alignment was evil, a chaotic neutral target would be treated as chaotic good.

            You can use this ability to do battle against foes who share your alignment, but you should exercise caution in doing so.
            Persecution of those who share your ideals can lead you to fall and become an ex-paladin.
        \end{attuneability}

        \cf{Pal}[5]{Greater Zealous Conviction} The bonus from your \textit{zealous conviction} ability increases to \plus2.

        \cf{Pal}[6]{Zealous Offense} You gain a \plus1 bonus to \glossterm{accuracy}.

        \cf{Pal}[7]{Greater Pass Judgment} Your \textit{pass judgment} ability loses the \glossterm{Attune} tag.
        Instead, it lasts until you decide to lift your judgment on the target as a \glossterm{free action}.

        % TODO: Doesn't work with new spell system
        % \cf{Pal}[20]{Martyr's Retribution}[Mag]
        % If you die in the service of your devoted alignment, you may choose to have your fallen body erupt in an immense burst of divine energy.
        % If you do, your body is almost completely consumed, preventing you from being raised with \ritual{resurrection} and similar effects that require an intact body.
        % This burst has two effects.
        % First, a \spell{sunburst} spell immediately takes effect over the area where you died.
        % Second, a \spell{storm of vengeance} spell begins to take effect, centered on the same area.
        % The spell lasts for 10 rounds, and the lightning strikes target the paladin's enemies.
        % Both of these effects harm only the paladin's foes, and do not harm your allies.
        % However, your allies' vision is still impeded by the \spell{storm of vengeance}.

    \subsection{Ex-Paladins}
        If you cease to follow your devoted alignment, you lose all \glossterm{magical} paladin class abilities.
        If your atone for your misdeeds and resume the service of your devoted alignment, you can regain your abilities.

\newpage
\section{Ranger}\label{Ranger}
    \begin{dtable!*}
        \lcaption{Ranger Progression}
        \begin{dtabularx}{\textwidth}{l l >{\lcol}X >{\lcol}X >{\lcol}X > {\lcol}X}
            \tb{Rank} & \tb{Min Level} & \tb{Beastmaster}         & \tb{Huntmaster} & \tb{Keen Senses}    & \tb{Wilderness Warrior} \tableheaderrule
            1 & 1  & Animal companion         & Quarry                     & Keen vision         & Wild maneuvers     \\
            2 & 2  & Handling lore            & Hunting style              & Learned perception  & Wild lore          \\
            3 & 5  & Pack tactics             & Hunter's instincts         & Blindsense          & Wild force         \\
            4 & 8  & Greater animal companion & Tracker                    & Greater keen vision & Wild maneuver      \\
            5 & 11 & Greater handling lore    & Hunting style              & Perceive weakness   & Greater wild lore  \\
            6 & 14 & Greater pack tactics     & Greater hunter's instincts & Blindsight          & Greater wild force \\
            7 & 17 & Supreme animal companion & Greater tracker            & Supreme keen vision & Wild maneuver      \\
            8 & 20 &                          & Dual quarry                &                     &                    \\
        \end{dtabularx}
    \end{dtable!*}

    \classbasics{Alignment} Any.

    \classbasics{Archetypes} Rangers have the Beastmaster, Huntmaster, Keen Senses, and Wilderness Warrior \glossterm{archetypes}.

    \subsection{Basic Class Abilities}
        If you are a ranger, you gain the following abilities.

        \cf{Rgr}{Defenses}
        You gain the following bonuses to your \glossterm{defenses}: \plus3 Armor, \plus4 Fortitude, \plus5 Reflex, \plus3 Mental.

        \cf{Rgr}{Skills}
        You have the following \glossterm{class skills}:
        \begin{itemize}
            \item \subparhead{Strength} Climb, Jump, Swim.
            \item \subparhead{Dexterity} Acrobatics, Escape Artist, Ride, Stealth.
            \item \subparhead{Constitution} Endurance.
            \item \subparhead{Intelligence} Craft, Deduction, Heal, Knowledge (dungeoneering, geography, nature).
            \item \subparhead{Perception} Awareness, Creature Handling, Survival.
            \item \subparhead{Other} Deception, Intimidate, Persuasion, Profession.
        \end{itemize}

        \cf{Rgr}{Weapon Proficiencies} 
        You are proficient with simple weapons and any two other \glossterm{weapon groups}.
        In addition, you are also proficient with your choice of bows, crossbows, or thrown weapons.

        \cf{Rgr}{Armor Proficiencies} 
        You are proficient with light and medium armor.

    \subsection{Beastmaster}
        This archetype improves your connection to animals, allowing you to control and command them in battle.

        \cf{Rgr}[1]{Animal Companion}
        You can use the \textit{animal companion} ability.
        This ability requires 8 hours of training and attunement which the target must actively parcipate in.
        You can compel a wild animal to undergo this training by sustaining the \textit{command} ability from the Creature Handling skill (see \pcref{Command}).
        \begin{attuneability}{Animal Companion}[\glossterm{Attune} (self), \glossterm{Emotion}, \glossterm{Magical}]
            Choose a Medium or smaller animal \glossterm{ally} within your \glossterm{reach} with a level no higher than your level and a \glossterm{challenge rating} no higher than 1.
            The target serves as a loyal companion to you.
            It follows your directions to the best of its ability.
            Animals are unable to understand complex concepts, so their ability to obey convoluted instructions is limited.

            Your magical connection to the animal improves its resilience and strength in combat.
            If any of its statistics are higher than the normal values below, the animal uses its own statistics instead.
            All other aspects of the animal, such as its speed and natural weapons, are unchanged.
            \begin{itemize}
                \item Its \glossterm{damage resistance} and \glossterm{wound resistance} are equal to the base values for your level (see \pcref{Character Advancement}).
                \item Its \glossterm{hit points} are equal to 6.
                \item Each of its \glossterm{defenses} is normally equal to 4 \add your level.
                \item Its \glossterm{accuracy} is normally equal to your level \add half your base Perception.
                \item Its \glossterm{power} with its attacks is normally equal to your \glossterm{power} \sub 2.
            \end{itemize}
        \end{attuneability}

        \cf{Rgr}[2]{Handling Lore} You gain two additional \glossterm{skill points}.
        In addition, you gain a \plus2 bonus to the Creature Handling skill (see \pcref{Creature Handling}).

        \cf{Rgr}[3]{Pack Tactics} You gain a \plus1 bonus to \glossterm{defenses} against creatures that are \glossterm{overwhelmed}.

        \cf{Rgr}[4]{Greater Animal Companion} Your \textit{animal companion} gains an \glossterm{action point}.
        In addition, it gains a \plus1 bonus to \glossterm{accuracy}, \glossterm{defenses}, and \glossterm{wound rolls}.

        \cf{Rgr}[5]{Greater Handling Lore} The bonus from your \textit{handling lore} ability increases to \plus6.

        \cf{Rgr}[6]{Greater Pack Tactics} You gain a \plus1 bonus to \glossterm{accuracy} against \glossterm{overwhelmed} creatures.

        \cf{Rgr}[7]{Supreme Animal Companion} Your \textit{animal companion} gains an additional \glossterm{action point}.
        In addition, the bonuses from your \textit{greater animal companion} ability increase to \plus2.

    \subsection{Keen Senses}
        This archetype improves your senses.

        \cf{Rgr}[1]{Keen Vision}
        You reduce your \glossterm{range increment} penalties for attacking at long range by 1.
        In addition, you gain \glossterm{low-light vision}, allowing you to treat sources of light as if they had double their normal illumination range.
        If you already have low-light vision, you double its benefit, allowing you to treat sources of light as if they had four times their normal illumination range.

        \cf{Rgr}[2]{Learned Perception} You gain two additional skill points.
        In addition, you gain a \plus2 bonus to \glossterm{Awareness}.

        \cf{Rgr}[3]{Blindsense}
        Your perceptions are so finely honed that you can sense your enemies without seeing them.
        You gain the \glossterm{blindsense} ability out to 50 feet.
        This ability allows you to sense the presence and location of objects and foes within 50 feet without seeing them.
        If you already have the blindsense ability, you increase its range by 50 feet.

        \cf{Rgr}[4]{Greater Keen Vision} The penalty reduction from your \textit{keen vision} ability increases to 2.
        In addition, you gain \glossterm{darkvision} with a 50 foot range, allowing you to see in complete darkness clearly.

        \cf{Rgr}[5]{Perceive Weakness}
        You gain a \plus1 bonus to \glossterm{accuracy}.

        \cf{Rgr}[6]{Blindsight}
        You gain the \glossterm{blindsight} ability, allowing you to ``see'' perfectly without your eyes in a 50 foot radius around you.
        With this ability, you can fight just as well with your eyes closed as with them open.
        In addition, the range of your \glossterm{blindsense} ability increases by 50 feet.

        \cf{Rgr}[7]{Supreme Keen Vision}
        The penalty reduction from your \textit{keen vision} ability increases to 3.
        Your perceptions are accurate enough to defeat even powerful magic.
        You can see through normal and magical darkness, see the truth behind visual figments and glamers, and see the true form of creatures and objects affected by \glossterm{Shaping} abilities.
        This ability works at any range.

    \subsection{Huntmaster}
        This archetype grants you and your allies abilities to hunt down specific foes.

        \cf{Rgr}[1]{Quarry}\label{Quarry} You can use the \textit{quarry} ability as a \glossterm{minor action}.
        \begin{attuneability}{Quarry}[\glossterm{Attune} (self)]
            Choose a creature within \rnglong range.
            The target becomes your quarry.
            You and your \glossterm{allies} within the same range are called your hunting party.
            Your hunting party gains a \plus1 bonus to \glossterm{accuracy} with \glossterm{strikes} against your quarry.
        \end{attuneability}

        \cf{Rgr}[2]{Hunting Style}
        You learn specific hunting styles to defeat particular quarries.
        Choose one hunting style from the list below.
        You can also spend \glossterm{insight points} to learn one additional \textit{hunting style} per \glossterm{insight point}.
        When you use your \textit{quarry} ability, you may also use one of your \textit{hunting styles}.
        Each \textit{hunting style} ability lasts as long as the \textit{quarry} ability you used it with.
        {
            \begin{freeability}{Anchoring}[Magical]
                As long as your quarry is \glossterm{threatened} by at least two members of your hunting party, it cannot travel extradimensionally.
                This prevents all \glossterm{Manifestation}, \glossterm{Planar}, and \glossterm{Teleportation} effects.

                \rankline
                \rank{4} This effect instead applies if your quarry is within \rngmed range of at least two members of your hunting party.
                \rank{6} This effect instead applies if your quarry is within \rnglong range of at least two members of your hunting party.
                \rank{8} This effect instead applies if your quarry is within \rnglong range of any member of your hunting party.
            \end{freeability}

            \begin{freeability}{Brutal Assault}
                The accuracy bonus from your \textit{quarry} ability is replaced with a \minus1 penalty to accuracy with \glossterm{mundane} attacks against your quarry.
                In exchange, your hunting party gains a \plus2 bonus to \glossterm{power} with \glossterm{mundane} attacks against your quarry.

                \rankline
                \rank{4} The accuracy penalty is removed.
                \rank{6} The power bonus applies to both \glossterm{mundane} and \glossterm{magical} attacks against your quarry.
                \rank{8} The power bonus is increased to \plus4.
            \end{freeability}

            \begin{freeability}{Coordinated Stealth}
                Your quarry takes a \minus4 penalty to Awareness checks to notice members of your hunting party.

                \rankline
                % seems weak?
                \rank{4} Your hunting party gains a \plus2 bonus to \glossterm{power} against your quarry if it is \unaware of every member of the hunting party.
                \rank{6} The Awareness penalty increases to \minus8.
                \rank{8} The power bonus increases to \plus4.
            \end{freeability}

            \begin{freeability}{Cover Weaknesses}
                The accuracy bonus against your quarry is replaced with a \plus1 bonus to defenses against your quarry's attacks.

                \rankline
                \rank{4} Your party's \glossterm{overwhelm penalties} are reduced by 2 against your quarry's attacks.
                \rank{6} Your party ignores all \glossterm{overwhelm penalties} against your quarry's attacks.
                \rank{8} The defense bonus increases to \plus2.
            \end{freeability}

            \begin{freeability}{Decoy}
                If you \glossterm{threaten} your quarry, it takes a \minus2 penalty to accuracy on attacks against members of your hunting party other than you.

                \rankline
                \rank{4} The penalty increases to \minus3.
                \rank{6} The penalty increases to \minus4.
                \rank{8} The penalty increases to \minus5.
            \end{freeability}

            \begin{freeability}{Lifeseal}[Magical]
                As long as your quarry is \glossterm{threatened} by at least two members of your \glossterm{hunting party}, it cannot regain \glossterm{hit points}.

                \rankline
                \rank{4} This effect instead applies if the target is within \rngmed range of at least two members of your hunting party.
                \rank{6} This effect instead applies if your quarry is within \rnglong range of at least two member of your hunting party.
                \rank{8} This effect instead applies if your quarry is within \rnglong range of any member of your hunting party.
            \end{freeability}

            \begin{freeability}{Martial Suppression}
                As long as your quarry is \glossterm{threatened} by at least two members of your hunting party, it takes a \minus1 penalty to accuracy with \glossterm{physical attacks}.

                \rankline
                \rank{4} The penalty increases to \minus2.
                \rank{6} The penalty increases to \minus3.
                \rank{8} The penalty increases to \minus4.
            \end{freeability}

            \begin{freeability}{Mystic Guidance}[Magical]
                The accuracy bonus from your \textit{quarry} ability applies to all attacks your hunting party makes against your quarry, instead of only to \glossterm{physical attacks}.

                \rankline
                \rank{4} Your hunting party gains a \plus1 bonus to Fortitude, Reflex, and Mental defenses against attacks from your quarry.
                \rank{6} The accuracy bonus against your quarry is increased to \plus2.
                \rank{8} The defense bonuses increase to \plus2.
            \end{freeability}

            \begin{freeability}{Mystic Suppression}
                As long as your quarry is \glossterm{threatened} by at least two members of your hunting party, it takes a \minus1 penalty to \glossterm{accuracy} with \glossterm{magical} attacks.

                \rankline
                \rank{4} The penalty increases to \minus2. 
                \rank{6} The penalty increases to \minus3.
                \rank{8} The penalty increases to \minus4.
            \end{freeability}

            \begin{freeability}{Solo Hunter}
                Your hunting party other than you gains no benefit from your \textit{quarry} ability.
                In exchange, you gain a \plus1 bonus to defenses against your quarry.

                \rankline
                \rank{4} The accuracy bonus from your \textit{quarry} ability increases to \plus2.
                \rank{6} The defense bonus increases to \plus2.
                \rank{8} The accuracy bonus increases to \plus3.
            \end{freeability}

            \begin{freeability}{Swarm Hunter}
                When you use your \textit{quarry} ability, you can target any number of creatures to be your quarry.

                \rankline
                \rank{4} You reduce your \glossterm{overwhelm penalties} by 1.
                \rank{6} The penalty reduction applies to all members of your hunting party.
                \rank{8} The penalty reduction increases to 2.
            \end{freeability}

            \begin{freeability}{Wolfpack}
                At the start of each \glossterm{phase}, if your quarry is \glossterm{threatened} by at least two members of your hunting party, it is \glossterm{slowed} until the end of that phase.

                \rankline
                \rank{4} This effect instead applies if your quarry is within \rngmed range of at least two members of your hunting party.
                \rank{6} This effect instead applies if your quarry is within \rnglong range of at least two members of your hunting party.
                \rank{8} This effect instead applies if your quarry is within \rnglong range of any member of your hunting party.
            \end{freeability}
        }

        \cf{Rgr}[3]{Hunter's Instincts} You gain a \plus2 bonus to \glossterm{initiative} checks and Reflex defense.

        \cf{Rgr}[4]{Tracker} You gain two additional skill points.
        In addition, you gain a \plus2 bonus to the Survival skill.

        \cf*{Rgr}[5]{Hunting Style}
        You learn an additional \textit{hunting style}.

        \cf{Rgr}[6]{Greater Hunter's Instincts}
        The bonus from your \textit{hunter's instincts} ability increases to \plus4.

        \cf{Rgr}[7]{Greater Tracker} The bonus from your \textit{tracker} ability increases to \plus6.
        In addition, you can move at full speed while following tracks at no penalty.

        \cf{Rgr}[8]{Dual Quarry} You can attune to two separate uses of your \textit{quarry} ability simultaneously.
        You can use different \textit{hunting style} abilities with each \textit{quarry}.

    % Would be nice to call this ``Hunter'' and the maneuvers ``Hunting'' maneuvers,
    % but that conflicts with the naming for Huntmaster
    \subsection{Wilderness Warrior}
        This archetype grants you abilities to use in combat and improves your wilderness skills.

        \cf{Rgr}[1]{Wild Maneuvers} 
        You can channel your martial prowess into devastating attacks.
        You learn two \glossterm{maneuvers} from the wild maneuver list (see \pcref{Wild Maneuvers}).
        You can also spend \glossterm{insight points} to learn one additional \glossterm{maneuver} per \glossterm{insight point}.
        As a \glossterm{standard action}, you can use any \glossterm{maneuver} you know.

        \cf{Rgr}[2]{Wild Lore} You gain two additional \glossterm{skill points}.
        In addition, you gain a \plus1 bonus to Perception-based checks, except \glossterm{initiative} checks.

        \cf{Rgr}[3]{Wild Force} You gain a \plus1 bonus to \glossterm{power} with \glossterm{mundane} abilities.

        \cf{Rgr}[4]{Wild Maneuver}
        You learn an additional \textit{wild maneuver}.

        \cf{Rgr}[5]{Greater Wild Lore} The bonus from your \textit{wild lore} ability increases to \plus3.

        \cf{Rgr}[6]{Greater Wild Force} The bonus from your \textit{wild force} ability increases to \plus2.

        \cf*{Rgr}[7]{Wild Maneuver}
        You learn an additional \textit{wild maneuver}.


    % \subsection{Boundary Warden}
    %     This archetype grants you abilities improving your ability to guard the boundaries between civilization and nature.

    %     \cf{Rgr}[2]{Favored Terrain} Choose one of the following terrain types: 
    %     You can also spend \glossterm{insight points} to choose one additional creature type per \glossterm{insight point}.
    %     You gain a \plus1 bonus to all \glossterm{defenses} and a \plus10 foot bonus to land speed while in any terrain you chose.

\newpage
\section{Rogue}\label{Rogue}
    \begin{dtable!*}
        \lcaption{Rogue Progression}
        \begin{dtabularx}{\textwidth}{l l >{\lcol}X >{\lcol}X >{\lcol}X >{\lcol}X}
            \tb{Rank} & \tb{Min Level} & \tb{Assassin}               & \tb{Bard}              & \tb{Combat Trickster} & \tb{Jack of All Trades} \tableheaderrule
            1 & 1  & Sneak attack                & Bardic performances    & Tricky manuevers    & Skill lore             \\
            2 & 2  & Evasion                     & Bardic lore            & Trick lore          & Skill exemplar         \\
            3 & 5  & Darkstalker                 & Bardic combat training & Trick power         & Dabbler                \\
            4 & 8  & Hide in plain sight         & Bardic performance     & Trick manuever      & Skill lore             \\
            5 & 11 & Greater evasion             & Dual performance       & Greater trick lore  & Greater skill exemplar \\
            6 & 14 & Assassination               & Loremaster             & Greater trick power & Greater dabbler        \\
            7 & 17 & Greater hide in plain sight & Bardic performance     & Trick maneuver      & Master of masteries    \\
            8 & 20 & Greater darkstalker         &                        &                     &                        \\
        \end{dtabularx}
    \end{dtable!*}

    \classbasics{Alignment} Any.

    \classbasics{Archetypes} Rogues have the Assassin, Bard, Combat Trickster, and Jack of All Trades \glossterm{archetypes}.

    \subsection{Basic Class Abilities}
        If you are a rogue, you gain the following abilities.

        \cf{Rog}{Defenses}
        You gain the following bonuses to your \glossterm{defenses}: \plus2 Armor, \plus3 Fortitude, \plus6 Reflex, \plus4 Mental.

        \cf{Rog}{Skills}
        You have the following \glossterm{class skills}:
        \begin{itemize}
            \item \subparhead{Strength} Climb, Jump, Swim.
            \item \subparhead{Dexterity} Acrobatics, Escape Artist, Sleight of Hand, Stealth.
            \item \subparhead{Intelligence} Craft, Deduction, Devices, Disguise, Knowledge (dungeoneering, local), Linguistics.
            \item \subparhead{Perception} Awareness, Sense Motive.
            \item \subparhead{Other} Deception, Intimidate, Perform, Persuasion, Profession.
        \end{itemize}

        \cf{Rog}{Weapon Proficiencies} 
        You are proficient with simple weapons, any two other \glossterm{weapon groups}, and saps.

        \cf{Rog}{Armor Proficiencies} 
        You are proficient with light armor.

    \subsection{Assassin}
        This archetype improves your agility, stealth, and combat prowess against unaware targets.

        \cf{Rog}[1]{Sneak Attack} You can use the \textit{sneak attack} ability as a standard action.
        \begin{freeability}{Sneak Attack}
            Make a \glossterm{strike} with a \glossterm{light weapon} against a creature within \rngclose range.
            If the target is \unaware or \glossterm{overwhelmed}, you gain a \plus2d bonus to damage with the strike.
            You do not gain this damage bonus against creatures who are immune to \glossterm{critical hits} or who lack a discernible body structure, such as oozes.

            \rankline
            \rank{3} The damage bonus increases to \plus3d.
            \rank{5} The damage bonus increases to \plus4d.
            \rank{7} The damage bonus increases to \plus5d.
        \end{freeability}

        % TODO: wording
        \cf{Rog}[2]{Evasion} When you are attacked by an ability that affects an area, you can use your Reflex defense in place of your Armor defense against that attack.

        \cf*{Rog}[3]{Darkstalker}[Magical] You can use the \textit{darkstalker} ability as a standard action.
        \begin{attuneability}{Darkstalker}[\glossterm{Attune} (self)]
            You become completely undetectable by your choice of one of the following senses:
            \begin{itemize}
                \item \glossterm{Blindsense} and \glossterm{blindsight}
                \item \glossterm{Darkvision}
                \item \glossterm{Lifesense} and \glossterm{lifesight}
                \item \glossterm{Scent}
                \item \glossterm{Tremorsense} and \glossterm{tremorsight}
            \end{itemize}
        \end{attuneability}

        \cf{Rog}[4]{Hide in Plain Sight} You can use the Stealth skill to hide while observed.
        Creatures observing you while you try to hide gain a \plus10 bonus to checks to notice you.
        You must still have cover or concealment to hide successfully.

        \cf{Rog}[5]{Greater Evasion} You can use your Reflex defense in place of any other defense with your \textit{evasion} ability.

        \cf{Rog}[6]{Assassination} You can use the \textit{assassination} ability as a \glossterm{minor action}.
        \begin{freeability}{Assassination}[\glossterm{Swift}]
            You study a creature within \rngmed range, finding weak points you can take advantage of.
            Until the end of the next round, if you make a melee \glossterm{strike} against the target while it is \unaware, your attack deals maximum damage.
        \end{freeability}

        \cf{Rog}[7]{Greater Hide in Plain Sight} Creatures observing you while you try to hide gain a \plus5 bonus to checks to notice you instead of a \plus10 bonus.

        \cf{Rog}[8]{Greater Darkstalker}[Magical] When you use your \textit{darkstalker} ability, you become undetectable by all of the listed senses, not just one.

    \subsection{Bard}
        This archetype grants you the ability to inspire your allies and impair your foes with musical performances.

        \cf{Rog}[1]{Bardic Performances}
        You learn two \textit{bardic performances} from the list below.
        You can also spend \glossterm{insight points} to learn one additional bardic performance per \glossterm{insight point}.
        As a \glossterm{standard action}, you can use any bardic performance you know.

        All \textit{bardic performances} have the \glossterm{Auditory} and \glossterm{Sustain} (minor) tags.
        When you use a \textit{bardic performance} ability, you begin a performance using one of your Perform skills.
        You must be \glossterm{trained} with a Perform skill capable of making an auditory performance to use a bardic performance ability.
        If you are \glossterm{mastered} with an appropriate Perform skill, you gain a \plus2 bonus to \glossterm{accuracy} with the ability.

        The names of bardic performances do not have to precisely match your actual performance.
        For example, you can use the \textit{palliative poem} ability with a gentle song using Perform (wind instruments) or a distracting joke using Perform (comedy) instead of a poem.
        {
            \begin{freeability}{Ballad of Belligerence}[\glossterm{Auditory}, \glossterm{Emotion}, \glossterm{Sustain} (minor)]
                \target{All \glossterm{enemies} in a \areasmall radius from you}
                Make an attack vs. Mental against the target.
                \hit Each target is unable to take any \glossterm{standard actions} that do not cause it to make an attack.
                For example, a target could make a \glossterm{strike} or cast an offensive spell, but it could not heal itself or summon a creature.

                \rankline
                \rank{3} The area increases to a \areamed radius.
                \rank{5} The area increases to a \arealarge radius.
                \rank{7} The area increases to a \areahuge radius.
            \end{freeability}

            \begin{freeability}{Cacaphony}[\glossterm{Auditory}]
                \targets{All \glossterm{enemies} within a \areasmall radius from you}
                Make an attack vs. Fortitude against each target.
                \hit Each target takes energy \glossterm{standard damage} \minus1d.

                \rankline
                \rank{3} The area increases to a \areamed radius.
                \rank{5} The area increases to a \arealarge radius.
                \rank{7} The area increases to a \areahuge radius.
            \end{freeability}

            \begin{freeability}{Cadenza of Courage}[\glossterm{Auditory}, \glossterm{Emotion}, \glossterm{Sustain} (minor)]
                \target{One \glossterm{ally} within \rngmed range}
                The target gains a \plus1 \glossterm{magic bonus} to \glossterm{accuracy}.

                \rankline
                \rank{3} The target also gains a \plus2 \glossterm{magic bonus} to Mental defense.
                \rank{5} The accuracy bonus increases to \plus2.
                \rank{7} The bonus to Mental defense increases to \plus4.
            \end{freeability}

            \begin{freeability}{Cleansing Counterpoint}[\glossterm{Auditory}]
                \target{One \glossterm{ally} within \rngmed range}
                The target can remove its most recent \glossterm{condition}.
                This can only remove a condition applied during the previous round.

                \rankline
                \rank{3} The target can remove a condition applied during any previous round, but not during the current round.
                \rank{5} The target can remove its two most recent conditions.
                \rank{7} The target can remove its three most recent conditions.
            \end{freeability}

            \begin{freeability}{Dizzying Ditty}[\glossterm{Auditory}, \glossterm{Compulsion}, \glossterm{Sustain} (minor)]
                \target{One creature within \rngmed range}
                Make an attack vs. Mental against the target.
                \hit The target is \glossterm{dazed}.

                \rankline
                \rank{3} On a \glossterm{critical hit}, the target is \glossterm{stunned}.
                \rank{5} The target is stunned instead of dazed.
                \rank{7} On a critical hit, the target is \glossterm{confused}.
            \end{freeability}

            \begin{freeability}{Dirge of Doom}[\glossterm{Auditory}, \glossterm{Emotion}, \glossterm{Sustain} (minor)]
                \target{One creature within \rngmed range}
                Make an attack vs. Mental against the target.
                \hit The target is treated as having two fewer \glossterm{hit points} than normal for the purpose of determining whether it is \glossterm{bloodied}.

                \rankline
                \rank{3} On a \glossterm{critical hit}, the target is \glossterm{bloodied} regardless of its hit point total.
                \rank{5} The hit point reduction increases to four.
                \rank{7} The target is \glossterm{bloodied} regardless of its hit point total.
            \end{freeability}

            \begin{freeability}{Frightening Fugue}[\glossterm{Auditory}, \glossterm{Emotion}, \glossterm{Sustain} (minor)]
                \target{One creature within \rngmed range}
                Make an attack vs. Mental against the target.
                \hit The target is \glossterm{shaken}.

                \rankline
                \rank{3} On a \glossterm{critical hit}, the target is \glossterm{panicked}.
                \rank{5} The target is frightened instead of shaken.
                \rank{7} The target is panicked instead of shaken.
            \end{freeability}

            \begin{freeability}{Hypnotic Hymn}[\glossterm{Auditory}, \glossterm{Emotion}, \glossterm{Sustain} (minor)]
                \target{One creature within \rngmed range}
                \hit The target is \glossterm{charmed} by you.
                Any act by you or by creatures that appear to be your allies that threatens or damages the \glossterm{charmed} person breaks the effect.
                An observant target may interpret overt threats to its allies as a threat to itself.
                This ability does not have the \glossterm{Subtle} tag, so an observant target may notice it is being influenced.

                \rankline
                \rank{3} The ability targets all creatures in a \areasmall radius within \rngmed range.
                \rank{5} The area increases to a \areamed radius.
                \rank{7} The area increases to a \arealarge radius.
            \end{freeability}

            \begin{freeability}{Inspiring Intonation}[\glossterm{Auditory}, \glossterm{Emotion}, \glossterm{Sustain} (minor)]
                \target{One \glossterm{ally} within \rngmed range}
                The target gains a \plus2 \glossterm{magic bonus} to \glossterm{checks}.

                \rankline
                \rank{3} The bonus increases to \plus3.
                \rank{5} The bonus increases to \plus4.
                \rank{7} The bonus increases to \plus5.
            \end{freeability}

            \begin{freeability}{Mesmerizing Melody}[\glossterm{Auditory}, \glossterm{Emotion}, \glossterm{Sustain} (minor)]
                \targets{All creatures in a \areasmall radius within \rngmed range}
                Make an attack vs. Mental against each target.
                \hit Each target is \fascinated by you.
                Any act by you or your apparent allies that damages a target or that causes it to feel that it is in danger breaks the effect for that creature.
                An observant target may interpret overt threats to its allies as a threat to itself.

                \rankline
                \rank{3} The area increases to a \areamed radius.
                \rank{5} The range increases to \rnglong.
                \rank{7} The area increases to a \arealarge radius.
            \end{freeability}

            \begin{freeability}{Palliative Poem}[\glossterm{Auditory}, \glossterm{Emotion}, \glossterm{Sustain} (minor)]
                \targets{All \glossterm{allies} within a \areamed radius from you}
                Each target is immune to being \glossterm{bloodied}.

                \rankline
                \rank{3} The area increases to a \arealarge radius.
                \rank{5} The area increases to a \areahuge radius.
                \rank{7} The area increases to a \areaext radius.
            \end{freeability}

            \begin{freeability}{Serenade of Serenity}[\glossterm{Auditory}, \glossterm{Emotion}, \glossterm{Sustain} (minor)]
                \targets{All \glossterm{allies} within a \arealarge radius \glossterm{emanation} from you}
                Each target gains a \plus4 bonus to defenses against hostile \glossterm{Compulsion} and \glossterm{Emotion} effects.

                \rankline
                \rank{3} At the end of each round, each target removes all \glossterm{conditions} caused by Compulsion and Emotion effects that were not applied during that round.
                \rank{5} The area increases to a \areahuge radius.
                \rank{7} Each target is immune to hostile Compulsion and Emotion effects.
            \end{freeability}

            \begin{freeability}{Tranquil Tune}[\glossterm{Auditory}, \glossterm{Emotion}, \glossterm{Sustain} (minor)]
                \targets{All creatures within a \areamed radius from you}
                Make an attack vs. Mental against each target.
                \hit Each target has its emotions calmed.
                The effects of all other \glossterm{Emotion} abilities on that target are \glossterm{suppressed}.
                It cannot take violent actions (although it can defend itself) or do anything destructive.
                If the target takes damage or feels that it is in danger, this effect is \glossterm{dismissed}.

                \rankline
                \rank{3} The area increases to a \arealarge radius.
                \rank{5} The area increases to a \areahuge radius.
                \rank{7} The area increases to a \areaext radius.
            \end{freeability}

            \begin{freeability}{Vigorous Verse}[\glossterm{Auditory}, \glossterm{Emotion}, \glossterm{Sustain} (minor)]
                \target{One \glossterm{ally} within \rngmed range}
                The target increases its maximum \glossterm{hit points} by one and regains that many hit points.
                When this effect ends, the target loses hit points equal to the hit points it regained this way.

                \rankline
                \rank{3} The number of additional hit points increases to two.
                \rank{5} The number of additional hit points increases to three.
                \rank{7} The number of additional hit points increases to four.
            \end{freeability}
        }

        \cf{Rog}[2]{Bardic Lore} You gain two additional skill points.
        In addition, you gain a \plus2 bonus to all Knowledge skills and gain all Knowledge skills as \glossterm{class skills}.

        \cf{Rog}[3]{Bardic Combat Training} You gain a \plus1 bonus to Armor defense.

        \cf{Rog}[4]{Bardic Performance}
        You learn an additional bardic performance.

        \cf{Rog}[5]{Dual Performance}
        Once per round, you can \glossterm{sustain} two bardic performances as a single \glossterm{minor action}.

        \cf{Rog}[6]{Loremaster} The skill bonus from your \textit{bardic lore} ability increases to \plus5.

        \cf*{Rog}[7]{Bardic Performance}
        You learn an additional bardic performance.

    \subsection{Combat Trickster}
        This archetype grants you abilities to use in combat and improves your combat prowess.

        \cf{Rog}[1]{Trick Maneuvers}
        You can confuse and confound your foes in combat.
        You learn two \glossterm{maneuvers} from the trick maneuver list (see \pcref{Trick Maneuvers}).
        You can also spend \glossterm{insight points} to learn one additional trick \glossterm{maneuver} per \glossterm{insight point}.
        As a \glossterm{standard action}, you can use any \glossterm{maneuver} you know.

        \cf{Rog}[2]{Trick Lore} You gain two additional \glossterm{skill points}.
        In addition, you gain a \plus1 bonus to Dexterity-based checks, except \glossterm{initiative} checks.

        \cf{Rog}[3]{Trick Power} You gain a \plus1 bonus to \glossterm{power} with \glossterm{mundane} abilities.

        \cf{Rog}[4]{Trick Maneuver}
        You learn an additional \textit{trick maneuver}.

        \cf{Rog}[5]{Greater Trick Lore} The bonus from your \textit{trick lore} ability increases to \plus3.

        \cf{Rog}[6]{Greater Trick Power} The bonus from your \textit{trick power} ability increases to \plus2.

        \cf*{Rog}[7]{Trick Maneuver}
        You learn an additional \textit{trick maneuver}.

    \subsection{Jack of All Trades}
        This archetype improves your skills.

        \cf{Rog}[1]{Skill Lore} You gain three additional skill points.

        \cf{Rog}[2]{Skill Exemplar} You gain a \plus1 bonus to all skills.

        \cf{Rog}[3]{Dabbler} You gain an additional \glossterm{insight point}.

        \cf*{Rog}[4]{Skill Lore} You gain three additional skill points.

        \cf{Rog}[5]{Greater Skill Exemplar} The skill bonus from your \textit{skill exemplar} ability increases to \plus2.

        \cf{Rog}[6]{Greater Dabbler} You gain an additional \glossterm{insight point}.

        \cf{Rog}[7]{Master of Masteries} You gain a \plus2 bonus to skills you have \glossterm{mastered}.

        \cf{Rog}[8]{Supreme Skill Exemplar} The skill bonus from your \textit{skill exemplar} ability increases to \plus4.

    % \subsection{Suave Scoundrel}
    %     This archetype improves your social manipulation abilities.

    %     \cf{Rog}[1]{Confound} You can use the \textit{confound} ability as a standard action.
    %     \begin{freeability}{Confound}[\glossterm{Compulsion}, \glossterm{Speech}, \glossterm{Subtle}]
    %         Make a attack vs. Mental against a creature within \rngclose range.
    %         You can choose to use your Deception skill to attack in place of your normal \glossterm{accuracy}.
    %         \hit As a \glossterm{condition}, the target is \glossterm{dazed}.
    %         That target cannot be affected by this ability again until it takes a \glossterm{short rest}.

    %         \rankline
    %         \rank{3} On a \glossterm{critical hit}, the target is \glossterm{confused} in addition to being \glossterm{dazed} as part of the same \glossterm{condition}.
    %         \rank{5} The range increases to \rngmed.
    %         \rank{7} The target is \glossterm{confused} in addition to being \glossterm{dazed} as part of the same condition.
    %     \end{freeability}

    %     \cf{Rog}[2]{Glibness} You gain two additional \glossterm{skill points}.
    %     In addition, you gain a \plus2 bonus to the Deception, Intimidate, and Persuasion skills.

    %     \cf{Rog}[3]{Distract} You can use the \textit{distract} ability as a standard action.
    %     \begin{freeability}{Distract}[\glossterm{Compulsion}, \glossterm{Speech}, \glossterm{Subtle}]
    %         Make a attack vs. Mental against a creature within \rngclose range.
    %         You can choose to use your Deception skill to attack in place of your normal \glossterm{accuracy}.
    %         \hit Until the end of the next round, the target is \glossterm{fascinated} by a creature or object of your choice within range.
    %         That target cannot be affected by this ability again until it takes a \glossterm{short rest}.

    %         \rankline
    %         \rank{5} The range increases to \rngmed.
    %         \rank{7} The target is also distracted by the creature or object as a \glossterm{condition}.
    %     \end{freeability}

    %     \cf{Rog}[4]{Slippery Mind} You gain a \plus2 bonus to Mental defense.

    %     \cf{Rog}[5]{Greater Glibness} The bonus from your \textit{glibness} ability increases to \plus4.

    %     \cf{Rog}[6]{Compel Belief} You can use the \textit{compel belief} ability as a standard action.
    %     \begin{freeability}{Compel Belief}[\glossterm{Compulsion}, \glossterm{Magical}, \glossterm{Speech}, \glossterm{Subtle}]
    %         Make an attack vs. Mental against a creature within \rngmed range.
    %         You can use your Persuasion skill to attack in place of your normal \glossterm{accuracy}.
    %         You must also choose a belief.
    %         The belief may be a truth the target knows, a lie that you told it, or even a simple misunderstanding (such as believing a hidden creature is not present in a room).
    %         If the creature does not already hold the chosen belief, this ability automatically fails.
    %         \hit As a \glossterm{condition}, the target continues to maintain the chosen belief, regardless of any evidence to the contrary.
    %         It will interpret any evidence that the falsehood is incorrect to be somehow wrong -- an illusion, a conspiracy to decieve it, or any other reason it can think of to continue believing the falsehood.
    %         When the effect ends, the creature can decide whether it believes the falsehood or not, as normal.

    %         \rankline
    %         \rank{8} The first time the condition would be removed, it instead remains.
    %     \end{freeability}

    %     \cf{Rog}[7]{Slippery Mind} The bonus from your \textit{slippery mind} ability increases to \plus4.

\newpage
\section{Sorcerer}\label{Mage}
    \begin{dtable!*}
        \lcaption{Sorcerer Progression}
        \begin{dtabularx}{\textwidth}{l l >{\lcol}X >{\lcol}X >{\lcol}X >{\lcol}X}
            \tb{Rank} & \tb{Min Level} & \tb{Arcane Magic} & \tb{Arcane Spell Mastery} & \tb{Draconic Magic} & \tb{Innate Arcanist} & \tb{Wild Magic} \tableheaderrule
            1 & 1  & Spellcasting          & Mystic insight              & Draconic bloodline         & Innate magic             & Wildspell \\
            2 & 2  & Innate focus          & Mage armor                  & Draconic focus             & Mystic tolerance         & Wild power \\
            3 & 5  & Spell rank (3), spell & Wellspring of power         & Draconic precision         & Augment                  & Chaotic miscast \\
            4 & 8  & Spell rank (4)        & Mystic insight              & Draconic hide              & Spell absorption         & Controlled chaos \\
            5 & 11 & Spell rank (5)        & Greater mage armor          & Greater draconic focus     & Greater mystic tolerance & Greater wild power \\
            6 & 14 & Spell rank (6), spell & Greater wellspring of power & Greater draconic precision & Augment                  & Greater chaotic miscast \\
            7 & 17 & Spell rank (7)        & Mystic insight              & Greater draconic hide      & Greater spell absorption & Greater controlled chaos\\
            8 & 20 & Spell rank (8)        &                             &                            &                          & \\
        \end{dtabularx}
    \end{dtable!*}

    \classbasics{Alignment} Any.

    \classbasics{Archetypes} Sorcerers have the Arcane Magic, Arcane Spell Mastery, Draconic Magic, Innate Arcanist, and Wild Magic \glossterm{archetypes}.

    \subsection{Basic Class Abilities}
        If you are a sorcerer, you gain the following abilities.

        \cf{Sor}{Defenses}
        You gain the following bonuses to your \glossterm{defenses}: \plus1 Armor, \plus3 Fortitude, \plus4 Reflex, \plus7 Mental.

        \cf{Sor}{Skills}
        You have the following \glossterm{class skills}:
        \begin{itemize}
            \item \subparhead{Intelligence} Craft, Deduction, Knowledge (all kinds, taken individually), Linguistics.
            \item \subparhead{Perception} Awareness, Spellcraft.
            \item \subparhead{Other} Deception, Intimidate, Persuasion, Profession.
        \end{itemize}

        \cf{Sor}{Weapon Proficiencies} 
        You are proficient with simple weapons and any one other \glossterm{weapon group}.

        \cf{Sor}{Armor Proficiencies} 
        You are not proficient with any type of armor.
        Encumbrance from armor interferes with the gestures you make to cast spells, which can cause your spells with \glossterm{somatic components} to fail (see \pcref{Somatic Component Failure}).

    \subsection{Arcane Magic}
        This archetype grants you the ability to cast arcane spells.

        \cf{Sor}[1]{Spellcasting}[Magical]
        You have the ability to use arcane magic.
        You gain access to one arcane \glossterm{mystic sphere} (see \pcref{Arcane Mystic Spheres}).
        You may spend \glossterm{insight points} to learn one additional \glossterm{mystic sphere} per two \glossterm{insight points}.
        Each \glossterm{mystic sphere} has a set of \glossterm{spells} associated with it.

        You automatically learn all \glossterm{cantrips} from any mystic sphere you have access to.
        In addition, you learn two rank 1 \glossterm{spells} from your chosen \glossterm{mystic sphere}.
        You can also spend \glossterm{insight points} to learn one additional arcane spell per \glossterm{insight point}.
        Unless otherwise noted in a spell's description, casting a spell requires a \glossterm{standard action}.

        When you gain access to a new \glossterm{mystic sphere} or spell \glossterm{rank},
            you can exchange any number of spells you know for other spells,
            including spells of the higher rank.

        Arcane spells require both \glossterm{verbal components} and \glossterm{somatic components} to cast (see \pcref{Casting Components}).
        For details about mystic spheres and casting spells, see \pcref{Spell and Ritual Mechanics}.

        \cf{Sor}[2]{Innate Focus} You reduce your \glossterm{focus penalties} by 2.

        \cf{Sor}[3]{Spell Rank}[Magical] You become a rank 3 arcane spellcaster.
        This can improve the effectiveness of your spells and give you access to spells that require a minimum rank of 3.

        \cf{Sor}[3]{Spell} You learn an additional \glossterm{spell} for any arcane \glossterm{mystic sphere} you know.

        \cf*{Sor}[4]{Spell Rank}[Magical] You become a rank 4 arcane spellcaster.
        This can improve the effectiveness of your spells and give you access to spells that require a minimum rank of 4.

        \cf*{Sor}[5]{Spell Rank}[Magical] You become a rank 5 arcane spellcaster.
        This can improve the effectiveness of your spells and give you access to spells that require a minimum rank of 5.

        \cf*{Sor}[6]{Spell Rank}[Magical] You become a rank 6 arcane spellcaster.
        This can improve the effectiveness of your spells and give you access to spells that require a minimum rank of 6.

        \cf*{Sor}[6]{Spell} You learn an additional \glossterm{spell} for any arcane \glossterm{mystic sphere} you know.

        \cf*{Sor}[7]{Spell Rank}[Magical] You become a rank 7 arcane spellcaster.
        This can improve the effectiveness of your spells and give you access to spells that require a minimum rank of 7.

        \cf*{Sor}[8]{Spell Rank}[Magical] You become a rank 8 arcane spellcaster.
        This can improve the effectiveness of your spells and give you access to spells that require a minimum rank of 8.

    \subsection{Arcane Spell Mastery}
        This archetype improves the arcane spells you cast.
        You must have the Arcane Magic archetype from the sorcerer class to gain the abilities from this archetype.

        \cf{Sor}[1]{Mystic Insight}[Magical]
        You gain your choice of one of the following abilities.
        {
            \parhead{Focused Caster} You reduce your \glossterm{focus penalty} by 1.
                You cannot choose this ability multiple times.
            \parhead{Mystic Sphere Access} You gain access to an additional arcane \glossterm{mystic sphere}.
                You cannot choose this ability multiple times.
            \parhead{Spell Known} You learn an additional arcane \glossterm{spell} (see \pcref{Arcane Mystic Spheres}).
                You can choose this ability multiple times, learning an additional spell each time.
            \parhead{Spell Power} Choose an arcane \glossterm{spell} you know.
                You gain a \plus2 bonus to \glossterm{power} with that spell.
                You can choose this ability multiple times, choosing a different spell each time.
            \parhead{Spell Precision} Choose an arcane \glossterm{spell} you know.
                You gain a \plus1 bonus to \glossterm{accuracy} with that spell.
                You can choose this ability multiple times, choosing a different spell each time.
        }

        \cf{Sor}[2]{Mage Armor}[Magical] You can use the \textit{mage armor} ability as a standard action.
        \begin{freeability}{Mage Armor}
            You create a translucent suit of magical armor on your body and over your hands.
            This functions like body armor that provides a \plus2 bonus to Armor defense and has no \glossterm{encumbrance}.
            It also provides a bonus to \glossterm{resistances} against \glossterm{energy damage} equal to half your level.
            The body armor does not appear if you are wearing other body armor of any kind.

            As long as you have a free hand, the barrier also manifests as a shield that provides a \plus1 bonus to Armor defense.
            This bonus is considered to come from a shield, and does not stack with the benefits of using a physical shield.

            This ability lasts until you \glossterm{dismiss} it as a free action.
        \end{freeability}

        \cf{Sor}[3]{Wellspring of Power}[Magical]
        You gain a \plus1 bonus to \glossterm{power} with \glossterm{magical} abilities.

        \cf*{Sor}[4]{Mystic Insight}[Magical]
        You gain an additional \textit{mystic insight} ability.

        \cf{Sor}[5]{Greater Mage Armor}[Magical]
        The defense bonus from the body armor created by your \textit{mage armor} ability increases to \plus3.
        In addition, the bonus to \glossterm{resistances} increases to be equal to your level.

        \cf{Sor}[6]{Greater Wellspring of Power}[Magical]
        The bonus from your \textit{wellspring of power} ability increases to \plus2.

        \cf*{Sor}[7]{Mystic Insight}[Magical]
        You gain an additional \textit{mystic insight} ability.

    \subsection{Draconic Magic}
        Not all sorcerers know the reason for their innate connection to magic.
        Some discover that they have draconic blood in their veins, and some of those sorcerers learn how to tap into their heritage.
        This archetype deepens your magical connection to your draconic ancestor and enhances your spellcasting.
        You must have the Arcane Magic archetype from the sorcerer class to gain the abilities from this archetype.

        \cf{Sor}[1]{Draconic Bloodline} Choose a type of dragon from among the dragons on \trefnp{Dragon Types}.
        You have the blood of that type of dragon in your veins.
        You gain a bonus equal to your level to your \glossterm{resistances} against the damage type dealt by that dragon's breath weapon.
        In addition, you gain access to the \glossterm{mystic sphere} associated with your dragon.

        \begin{dtable}
            \lcaption{Dragon Types}
            \begin{dtabularx}{\columnwidth}{l >{\lcol}X >{\lcol}X}
                \tb{Dragon} & \tb{Damage Type} & \tb{Mystic Sphere} \tableheaderrule
                Black       & Acid             & Vivimancy  \\
                Blue        & Electricity      & Glamer      \\
                Brass       & Fire             & Delusion    \\
                Bronze      & Electricity      & Revelation  \\
                Copper      & Acid             & Terramancy  \\
                Gold        & Fire             & Photomancy  \\
                Green       & Acid             & Compulsion  \\
                Red         & Fire             & Pyromancy   \\
                Silver      & Cold             & Telekinesis \\
                White       & Cold             & Cryomancy   \\
            \end{dtabularx}
        \end{dtable}

        \cf{Sor}[2]{Draconic Focus} You reduce your \glossterm{focus penalties} by 1.

        \cf{Sor}[3]{Draconic Precision} You gain a \plus1 bonus to \glossterm{accuracy} with any spell that either deals damage of your dragon's damage type or is from your dragon's \glossterm{mystic sphere}.

        \cf{Sor}[4]{Draconic Hide} You gain a \plus1 bonus to Armor defense.
        In addition, the bonus to \glossterm{resistances} from your \textit{draconic bloodline} ability increases to be equal to twice your level.

        \cf{Sor}[5]{Greater Draconic Focus} The penalty reduction from your \textit{draconic focus} ability increases to 2.

        \cf{Sor}[6]{Greater Draconic Precision} The bonus from your \textit{draconic precision} ability increases to \plus2.

        \cf{Sor}[7]{Greater Draconic Hide} The bonus to Armor defense from your \textit{draconic hide} abiliity increases to \plus2.
        In addition, the bonus to resistances from your \textit{draconic bloodline} ability increases to be equal to three times your level.

    \subsection{Innate Arcanist}
        This archetype deepens your innate connection to arcane magic.
        You must have the Arcane Magic archetype from the sorcerer class to gain the abilities from this archetype.

        \cf{Sor}[1]{Innate Magic} None of your spells have \glossterm{somatic components} or \glossterm{verbal components}.

        \cf{Sor}[2]{Mystic Tolerance} You gain a bonus to \glossterm{resistances} against \glossterm{energy damage} equal to half your level.

        \cf{Sor}[3]{Augment} You learn one \glossterm{augment} (see \pcref{Augments}).
        You can apply augments to spells to change their effects.
        When you gain access to a new arcane spell \glossterm{rank}, you may change which augments you know.

        \cf{Sor}[4]{Spell Absorption} Whenever another creature uses a spell to attack you, you gain the ability to cast the spell once.
        The spell retains its original \glossterm{augments}, if any, and you cannot apply any \glossterm{augments} to it.
        When you cast the spell, you use your own \glossterm{accuracy}, \glossterm{power}, and abilities to determine the effects of the spell.
        Once you cast the spell, you expend the absorbed energy, and you cannot cast it again.

        Whenever you are attacked by a new spell, if you already have the ability to cast a spell with this ability, you choose which spell you gain the ability to cast.
        When you take a \glossterm{long rest}, you lose the ability to cast any spells you have stored with this ability.

        \cf{Sor}[5]{Greater Mystic Tolerance} The bonus from your \textit{mystic tolerance} ability increases to be equal to your level.

        \cf*{Sor}[6]{Augment} You learn an additional \glossterm{augment}.

        \cf{Sor}[7]{Greater Spell Absorption} You can store the ability to cast up to three spells with your \textit{spell absorption} ability.
        In addition, you can cast each stored spell twice before you lose the ability to cast it instead of only once.

    \subsection{Wild Magic}
        This archetype makes the magic you cast more chaotic, generally increasing its power at the cost of your control over your magic.

        \cf{Sor}[1]{Wildspell} Whenever you cast a spell, you may use this ability.
        When you do, roll 1d10 and apply the corresponding wild magic effect from \trefnp{Wild Magic Effects}.
        Some wild magic effects cannot be meaningfully applied to all spells.
        For example, doubling the number of targets does not meaningfully affect spells that only target yourself.
        Any wildspell effects that do not make sense for a particular spell should be ignored.

        \begin{dtable}
            \lcaption{Wild Magic Effects}
            \begin{dtabularx}{\textwidth}{l X}
                \tb{Roll} & \tb{Effect} \tableheaderrule
                1 or lower & You \glossterm{miscast} the spell \\
                2 & You take a \minus2 penalty to \glossterm{accuracy} and \glossterm{power} with the spell \\
                3 & The spell targets you in addition to any other targets \\
                4 & All damage dealt by the spell changes to physical damage \\
                5 & All damage dealt by the spell changes to energy damage \\
                6 & The spell's area is doubled \\
                7 & The spell affects twice as many targets \\
                8 & You roll damage twice for the spell and take the higher result \\
                9 & You roll the attack roll twice for the spell and take the higher result \\
                10 or higher & During the \glossterm{action phase} of the next round, the spell takes effect again with the same choices for all decisions, such as targets \\
            \end{dtabularx}
        \end{dtable}

        \cf{Sor}[2]{Wild Power} You gain a \plus1 bonus to \glossterm{power} with \glossterm{magical} abilities.

        \cf{Sor}[3]{Chaotic Miscast} Whenever you \glossterm{miscast} a spell for any reason other a wild magic effect, you may instead roll on the \trefnp{Wild Magic Effects} table to determine the spell's effect.
        You take a \minus5 penalty to the roll.
        If the spell's effect is impossible after this roll, such as if all of its targets are invalid, it fails with no effect and is not miscast.

        \cf{Sor}[4]{Controlled Chaos} You gain a \plus1 bonus to all rolls you make to determine wild magic effects.

        \cf{Sor}[5]{Greater Wild Power} The bonus from your \textit{wild power} ability increases to \plus2.

        \cf{Sor}[6]{Greater Chaotic Miscast} The penalty from your \textit{chaotic miscast} ability is reduced to \minus2.

        \cf{Sor}[7]{Greater Controlled Chaos} The bonus from your \textit{controlled chaos} ability increases to \plus2.

\newpage
\section{Warlock}\label{Warlock}
    \begin{dtable!*}
        \lcaption{Warlock Progression}
        \begin{dtabularx}{\textwidth}{l l >{\lcol}X >{\lcol}X >{\lcol}X >{\lcol}X}
            \tb{Rank} & \tb{Min Level} & \tb{Blessings of the Abyss}  & \tb{Pact Magic} & \tb{Pact Spell Mastery}     & \tb{Soulkeeper's Chosen} \tableheaderrule
            1 & 1  & Eldritch blast               & Spellcasting          & Mystic insight              & Possession                  \\
            2 & 2  & Fiendish resistance          & Soulbound focus       & Armor tolerance             & Empowering whispers         \\
            3 & 5  & Abyssal jaunt                & Spell rank (3), spell & Wellspring of power         & Influence possession        \\
            4 & 8  & Abyssal substitution         & Spell rank (4)        & Mystic insight              & Exchange soul fragment      \\
            5 & 11 & Greater fiendish resistance  & Spell rank (5)        & Greater armor tolerance     & Powerful possession         \\
            6 & 14 & Abyssal curse                & Spell rank (6), spell & Greater wellspring of power & Greater empowering whispers \\
            7 & 17 & Greater abyssal substitution & Spell rank (7)        & Mystic insight              & Control possession          \\
            8 & 20 &                              & Spell rank (8)        &                             &                             \\
        \end{dtabularx}
    \end{dtable!*}

    \classbasics{Alignment} Any.

    \classbasics{Archetypes} Warlocks have the Pact Magic, Pact Spell Mastery, and Blessings of the Abyss \glossterm{archetypes}.

    \subsection{Basic Class Abilities}
        If you are a warlock, you gain the following abilities.

        \cf{War}{Defenses}
        You gain the following bonuses to your \glossterm{defenses}: \plus2 Armor, \plus4 Fortitude, \plus3 Reflex, \plus6 Mental.

        \cf{War}{Skills}
        You have the following \glossterm{class skills}:
        \begin{itemize}
            \item \subparhead{Dexterity} Ride.
            \item \subparhead{Intelligence} Craft, Deduction, Disguise, Knowledge (arcana, planes, religion), Linguistics.
            \item \subparhead{Perception} Awareness, Sense Motive, Spellcraft.
            \item \subparhead{Other} Deception, Intimidate, Persuasion, Profession.
        \end{itemize}

        \cf{War}{Weapon Proficiencies} 
        You are proficient with simple weapons and any one other \glossterm{weapon group}.

        \cf{War}{Armor Proficiencies} 
        You are proficient with light armor.
        Encumbrance from armor interferes with the gestures you make to cast spells, which can cause your spells with \glossterm{somatic components} to fail (see \pcref{Somatic Component Failure}).

        \cf{War}{Infernal Pact}[Magical]
        To become a warlock, you must make a pact with a powerful demon or devil.
        % TODO: wording; intended to prevent Null warlocks who have no pact
        If you somehow lose this ability, you lose all other warlock abilities.
        You must make a dark sacrifice, the details of which are subject to negotiation, and offer a part of your immortal soul.
        In exchange, you gain the powers of a warlock.
        The creature you make the pact with is called your soulkeeper.

        Offering your soul to an entity in this way grants it the ability to communicate with you in limited ways.
        This communication typically manifests as unnatural emotional urges or whispered voices audible only to you.

        Your pact specifies how much of your soul is granted to your soulkeeper, and the circumstances of the transfer.
        The most common arrangement is for a soulkeeper to gain possession of your soul immediately after you die.
        It will keep the soul for one decade per year of your life that you spend as a warlock.
        During that time, it will not prevent you from being resurrected.
        At the end of that time, if your soul remains intact, your soul will pass on to its intended afterlife.
        However, other arrangements are possible, and each warlock's pact can be unique.

        The longer you spend in an afterlife that is not your own, the more likely you are to lose your sense of self and become subsumed by the plane you are on.
        Only a soul of extraordinary strength can maintain its integrity after decades or centuries in the Abyss.
        Many warlocks seek power zealously while mortal to gain the mental fortitude necessary to keep their soul after death.

        \cf{War}{Whispers of the Lost}[Magical]
        You hear the voices of souls lost to the Abyss, linked to you through your soulkeeper.
        Choose one of the following types of whispers that you hear.
        {
            \subcf{Mentoring Whispers} You hear the voice of a dead warlock whose soul is bound to the same soulkeeper as yours.

            \subcf{Spiteful Whispers} You hear the voices of cruel souls who berate you for your flaws and mistakes.

            \subcf{Sycophantic Whispers} You hear the voices of adoring souls who praise your talents and everything you do.

            \subcf{Warning Whispers} You hear the voices of paranoid and fearful souls warning you of danger, both real and imagined.

            \subcf{Whispers of the Mighty} Your soulkeeper forges the connection to your soul into a boon granted to any soul in the Abyss strong enough to claim it in battle.
            You hear the voice of whatever soul currently possesses the boon, which may change suddenly and unexpectedly.
        }

    \subsection{Blessings of the Abyss}
        This archetype enhances your connection to the Abyss and allows you to channel its sinister power more directly.

        \cf{War}[1]{Abyssal Blast} You can use the \textit{abyssal blast} ability as a standard action.
        \begin{freeability}{Abyssal Blast}[\glossterm{Magical}]
            Make an attack vs. Armor against one creature or object within \rngclose range.
            \hit The target takes fire \glossterm{standard damage} \plus1d.

            \rankline
            \rank{3} The damage increases to \glossterm{standard damage} \plus2d.
            \rank{5} The damage increases to \glossterm{standard damage} \plus3d.
            \rank{7} The damage increases to \glossterm{standard damage} \plus4d.
        \end{freeability}

        \cf{War}[2]{Fiendish Resistance} You gain a bonus equal to half your level to \glossterm{resistances} against \glossterm{energy damage}.

        \cf{War}[3]{Abyssal Jaunt} You can use the \textit{abyssal jaunt} ability as a standard action.
        \begin{freeability}{Abyssal Jaunt}[\glossterm{Magical}]
            Make an attack vs. Mental against one creature or object within \rngclose range.
            \hit The target is teleported into the Abyss and takes fire \glossterm{standard damage} \minus2d.
            At the end of the next round, it returns to its original location, or the closest open space if that location is occupied.

            \rankline
            \rank{5} The damage penalty is reduced to \minus1d.
            \rank{7} The damage penalty is removed.
        \end{freeability}

        \cf{War}[4]{Abyssal Substitution} Whenever you use an ability that deals damage, you can change the type of the damage to fire damage.
        Any other aspects of the ability remain unchanged.
        In addition, you gain a \plus2 bonus to \glossterm{power} with abilities that deal fire damage.

        \cf{War}[5]{Greater Fiendish Resistance}[Magical] The bonus from your \textit{fiendish resistance} ability increases to be equal to your level.

        \cf{War}[6]{Abyssal Curse} You can use the \textit{abyssal curse} ability as a standard action.
        \begin{freeability}{Abyssal Curse}[\glossterm{Curse}, \glossterm{Magical}]
            Make an attack vs. Fortitude against one creature or object within \rngclose range.
            \hit The target is \glossterm{nauseated}.

            \rankline
            \rank{8} On a \glossterm{critical hit}, the target is also \glossterm{paralyzed}.
        \end{freeability}

        \cf{War}[7]{Greater Abyssal Substitution} Whenever you use an ability that deals damage, you may change the type of damage it deals to be fire damage.
        Any other aspects of the ability remain unchanged.

    \subsection{Pact Magic}
        This archetype grants you the ability to cast pact spells.
        You must have a base Willpower of at least 1 to gain this archetype.

        \cf{War}[1]{Spellcasting}[Magical]
        Your soulkeeper grants you the ability to use pact magic.
        You gain access to one pact \glossterm{mystic sphere} (see \pcref{Pact Mystic Spheres}).
        Each \glossterm{mystic sphere} has a set of \glossterm{spells} associated with it.
        You may spend \glossterm{insight points} to learn one additional \glossterm{mystic sphere} per two \glossterm{insight points}.

        You automatically learn all \glossterm{cantrips} from any mystic sphere you have access to.
        In addition, you learn two rank 1 \glossterm{spells} from your chosen \glossterm{mystic sphere}.
        You can also spend \glossterm{insight points} to learn one additional pact spell per \glossterm{insight point}.
        Unless otherwise noted in a spell's description, casting a spell requires a \glossterm{standard action}.

        When you gain access to a new \glossterm{mystic sphere} or spell \glossterm{rank},
            you can exchange any number of spells you know for other spells,
            including spells of the higher rank.

        Pact spells require require both \glossterm{verbal components} and \glossterm{somatic components} to cast (see \pcref{Casting Components}).
        For details about mystic spheres and casting spells, see \pcref{Spell and Ritual Mechanics}.

        \cf{War}[2]{Soulbound Focus} You reduce your \glossterm{focus penalties} by 2.

        \cf{War}[3]{Spell Rank}[Magical] You become a rank 3 pact spellcaster.
        This can improve the effectiveness of your spells and give you access to spells that require a minimum rank of 3.

        \cf{War}[3]{Spell} You learn an additional \glossterm{spell} for any pact \glossterm{mystic sphere} you know.

        \cf*{War}[4]{Spell Rank}[Magical] You become a rank 4 pact spellcaster.
        This can improve the effectiveness of your spells and give you access to spells that require a minimum rank of 4.

        \cf*{War}[5]{Spell Rank}[Magical] You become a rank 5 pact spellcaster.
        This can improve the effectiveness of your spells and give you access to spells that require a minimum rank of 5.

        \cf*{War}[6]{Spell Rank}[Magical] You become a rank 6 pact spellcaster.
        This can improve the effectiveness of your spells and give you access to spells that require a minimum rank of 6.

        \cf*{War}[6]{Spell} You learn an additional \glossterm{spell} for any pact \glossterm{mystic sphere} you know.

        \cf*{War}[7]{Spell Rank}[Magical] You become a rank 7 pact spellcaster.
        This can improve the effectiveness of your spells and give you access to spells that require a minimum rank of 7.

        \cf*{War}[8]{Spell Rank}[Magical] You become a rank 8 pact spellcaster.
        This can improve the effectiveness of your spells and give you access to spells that require a minimum rank of 8.

        % {
        %     \augment{0}{Infernal Focus} You gain a \plus10 bonus to \glossterm{concentration} checks to cast the spell (see \pcref{Concentration}).

        %     \augment{0}{Sickening} Make an attack vs. Fortitude against one creature targeted by the spell.
        %     On a hit, the target is \glossterm{sickened} as a \glossterm{condition}.
        %     \par This augment can be applied to any spell that targets at least one creature.

        %     \augment{0}{Soulrending} When a creature is dealt damage by the spell, if it has no hit points remaining, it dies.
        %     \par This augment can be applied to any spell that deals damage.

        %     \augment{0}{Terrifying} Make an attack vs. Mental against one creature targeted by the spell.
        %     On a hit, the target is \glossterm{shaken} by you as a \glossterm{condition}.
        %     \par This augment can be applied to any spell that targets at least one creature.

        %     \augment{1}{Dark Expansion} The spell's area is increased by one step, to a maximum of \areahuge.
        %     The steps are, in order: \areasmall, \areamed, \arealarge, and \areahuge.
        %     Normally, a Small or Medium line is 5 ft.\ wide, while a Large or Huge line is 10 ft.\ wide.
        %     A line used to define a wall does not have a width.
        %     \par This augment can be applied to any spell with an area that is one of the above areas.

        %     \augment{1}{Untraceable} You do not need \glossterm{verbal components} or \glossterm{somatic components} to cast the spell.

        %     \augment{2}{Abyssal Flame} Non-physical damage dealt by the spell becomes life damage in place of its other damage types.
        %     In addition, you gain a \plus1d bonus to life damage dealt by the spell.
        %     This does not change any other aspects of the spell's effects.
        %     \par This augment can be applied to any spell that deals non-physical damage.

        %     \augment{2}{Bloodreaving} When a creature takes damage from the spell, you heal hit points equal to the damage dealt.
        %     \par This augment can be applied to any spell that deals damage.

        %     % TODO: taking damage multiple times in the same round abuse?
        %     \augment{2}{Exsanguinating} When a creature takes damage from the spell, it begins bleeding as a \glossterm{condition}.
        %     At the end of each \glossterm{action phase}, a bleeding creature takes \glossterm{standard damage} \minus1d.
        %     \par This augment can be applied to any spell that deals damage.

        %     \augment{3}{Inevitable Doom} You gain a \plus2 bonus to \glossterm{accuracy} with the spell.
        %     \par This augment can be applied to any spell that makes an attack.

        %     \augment{3}{Infernal Potency} You gain a \plus4 bonus to \glossterm{power} with the spell.
        %     \par This augment can be applied to any spell.

        %     \augment{4}{Dark Miasma} When a creature takes damage from the spell, it becomes \glossterm{sickened} as a \glossterm{condition}.
        %     \par This augment can be applied to any spell that deals damage.

        %     \augment{4}{Unholy Terror} When a creature takes damage from the spell, it becomes \glossterm{shaken} by you as a \glossterm{condition}.
        %     \par This augment can be applied to any spell that deals damage.

        %     \augment{5}{Soulclaiming} When a creature takes damage from the spell, its soul becomes marked by your soulkeeper as a \glossterm{condition}.
        %     It takes a \minus4 penalty to Mental defense.
        %     If it dies, its soul is immediately claimed by your soulkeeper, preventing it from being resurrected by any means.

        %     \augment{6}{Greater Inevitable Doom} You gain a \plus4 bonus to \glossterm{accuracy} with the spell.
        %     \par This augment can be applied to any spell that makes an attack.

        %     \augment{6}{Greater Infernal Potency} You gain a \plus8 bonus to \glossterm{power} with the spell.
        %     \par This augment can be applied to any spell.
        % }

    \subsection{Pact Spell Mastery}
        This archetype improves your ability to cast spells with the power of your dark pact.
        You must have the Pact Magic archetype to gain the abilities from this archetype.

        \cf{War}[1]{Mystic Insight}[Magical] You gain your choice of one of the following abilities.
        {
            \parhead{Focused Caster} You reduce your \glossterm{focus penalty} by 1.
                You cannot choose this ability multiple times.
            \parhead{Mystic Sphere Access} You gain access to an additional pact \glossterm{mystic sphere}.
                You cannot choose this ability multiple times.
            \parhead{Rituals} The maximum \glossterm{rank} of pact ritual you can learn or perform is equal to the maximum rank of pact spell that you can cast.
                You gain the ability to perform pact rituals to create unique magical effects (see \pcref{Rituals}).
                You cannot choose this ability multiple times.
            \parhead{Spell Known} You learn an additional pact \glossterm{spell} (see \pcref{Pact Mystic Spheres}).
                You can choose this ability multiple times, learning an additional spell each time.
            \parhead{Spell Power} Choose a pact \glossterm{spell} you know.
                You gain a \plus2 bonus to \glossterm{power} with that spell.
                You can choose this ability multiple times, choosing a different spell each time.
            \parhead{Spell Precision} Choose a pact \glossterm{spell} you know.
                You gain a \plus1 bonus to \glossterm{accuracy} with that spell.
                You can choose this ability multiple times, choosing a different spell each time.
        }

        \cf{War}[2]{Armor Tolerance} You reduce your \glossterm{encumbrance} by 2 when determining your \glossterm{somatic component failure}.

        \cf{War}[3]{Wellspring of Power}[Magical]
        You gain a \plus1 bonus to \glossterm{power} with \glossterm{magical} abilities.

        \cf*{War}[4]{Mystic Insight} You gain an additional \textit{mystic insight} ability.

        \cf{War}[5]{Greater Armor Tolerance}[Magical] The penalty reduction from your \textit{armor tolerance} ability increases to 3.

        \cf{War}[6]{Greater Wellspring of Power}[Magical]
        The bonus from your \textit{wellspring of power} ability increases to \plus2.

        \cf*{War}[7]{Mystic Insight} You gain an additional \textit{mystic insight} ability.

    \subsection{Soulkeeper's Chosen}
        This archetype enhances your connection to your soulkeeper, granting you abilities relating to your pact.

        \cf{War}[1]{Possession} You can use the \textit{possession} ability as a \glossterm{free action} to allow your soulkeeper to directly control your actions.
        Your soulkeeper's objectives may differ from your own, but except in very unusual circumstances, your soulkeeper is invested in continuing your life and ensuring your victory in difficult circumstances.
        \begin{attuneability}{Possession}[\glossterm{Attune} (self)]
            You gain the following benefits and drawbacks:
            \begin{itemize}
                \item You increase your maximum \glossterm{hit points} by two and regain two \glossterm{hit points}.
                    When this ability ends, you lose two \glossterm{hit points} and your maximum \glossterm{hit points} are restored to normal.
                \item You gain a \plus2 bonus to \glossterm{power} with \glossterm{magical} abilities.
                \item You take a \minus2 penalty to Mental defense.
                \item You are unable to fully control your actions. At the start of each round, your soulkeeper chooses one of three behavior patterns: attack your enemies, defend yourself or your allies, or flee to protect your own life.
                    You must follow the chosen behavior that round during the best of your ability.
                \item At the start of each round, if either you or your soulkeeper choose to end this ability, the ability ends.
                    If the ability ends this way, you regain the \glossterm{action point} you spent to attune to this ability the next time you take a \glossterm{short rest}.
            \end{itemize}
        \end{attuneability}

        \cf{War}[2]{Empowering Whispers}[Magical]
        You gain an ability based on the type of whispers you hear with your \textit{whispers of the lost} ability.
        {
            \subcf{Mentoring Whispers} You gain two additional \glossterm{skill points}.

            \subcf{Spiteful Whispers} Whenever you miss a creature with an attack, you gain a \plus1 bonus to \glossterm{accuracy} against the creature you missed until the end of the next round.

            \subcf{Sycophantic Whispers} You gain a \plus2 bonus to Mental defense.

            \subcf{Warning Whispers} You gain a \plus2 bonus to \glossterm{initiative} checks and Reflex defense.

            \subcf{Whispers of the Mighty} You gain a \plus2 bonus to Fortitude defense.
        }

        % does this need to be more explicit that you can act during the delayed action phase if you interrupt during the action phase?
        \cf{War}[3]{Influence Possession} During your \textit{possession} ability, you can choose one behavior category at the start of each round.
        Your soulkeeper cannot compel you to take the chosen behavior that round.

        \cf{War}[4]{Exchange Soul Fragment} Your connection to your soulkeeper deepens, allowing you to send a fragment of your experiences through the link.
        You can use the \textit{exchange soul fragment} ability as a \glossterm{minor action}.
        \begin{freeability}{Exchange Soul Fragment}[\glossterm{Magical}]
            You lose a \glossterm{hit point} and reduce your maximum \glossterm{hit points} by one.
            Remove the most recent \glossterm{condition} affecting you.
            This cannot remove a condition applied during the current round.
            When you take a \glossterm{short rest}, your maximum hit points are restored to their normal value.

            \rankline
            \rank{6} This ability does not reduce your maximum \glossterm{hit points}.
            \rank{8} You can remove any number of \glossterm{conditions}.
        \end{freeability}

        \cf{War}[5]{Powerful Possession} The bonus to \glossterm{power} from your \textit{possession} ability increases to \plus4.

        \cf{War}[6]{Greater Empowering Whispers}[Magical] You gain an additional ability depending on the voices you chose with your \textit{whispers of the lost} ability.
        {
            \subcf{Mentoring Whispers} You gain an additional \glossterm{insight point}.

            \subcf{Spiteful Whispers} The bonus from your \textit{empowering whispers} ability increases to \plus2.

            \subcf{Sycophantic Whispers} You are immune to being \glossterm{shaken}, \glossterm{frightened}, and \glossterm{panicked}.

            \subcf{Warning Whispers} You are never \glossterm{unaware}.

            \subcf{Whispers of the Mighty} You are immune to being \glossterm{sickened} and \glossterm{nauseated}.
        }

        \cf{War}[7]{Control Possession} During your \textit{possession} ability, you can choose your actions normally.
        Your soulkeeper may still choose to end the ability.

\newpage
\section{Wizard}\label{Wizard}
    \begin{dtable!*}
        \lcaption{Wizard Progression}
        \begin{dtabularx}{\textwidth}{l l >{\lcol}X >{\lcol}X >{\lcol}X >{\lcol}X}
            \tb{Rank} & \tb{Min Level} & \tb{Alchemist}               & \tb{Arcane Magic}     & \tb{Arcane Scholar}   & \tb{Arcane Spell Mastery}   \tableheaderrule
            1 & 1  & Portable workshop            & Spellcasting          & Scholastic lore       & Mystic insight              \\
            2 & 2  & Alchemical infusion          & Mystic sphere         & Ritualist             & Mage armor                  \\
            3 & 5  & Alchemical discovery         & Spell rank (3), spell & Scholastic insight    & Wellspring of power         \\
            4 & 8  & Alchemical expertise         & Spell rank (4)        & Ritual expertise      & Mystic insight              \\
            5 & 11 & Experienced quaffing         & Spell rank (5)        & Contingency           & Greater mage armor          \\
            6 & 14 & Alchemical discovery         & Spell rank (6), spell & Scholastic insight    & Greater wellspring of power \\
            7 & 17 & Greater alchemical expertise & Spell rank (7)        & Malleable contingency & Mystic insight              \\
            8 & 20 &                              & Spell rank (8)        &                       &                             \\
        \end{dtabularx}
    \end{dtable!*}

    \classbasics{Alignment} Any.

    \classbasics{Archetypes} Mages have the Alchemist, Arcane Magic, Arcane Spell Mastery, and Arcane Scholar \glossterm{archetypes}.

    \subsection{Basic Class Abilities}
        If you are a wizard, you gain the following abilities.

        \cf{Wiz}{Defenses}
        You gain the following bonuses to your \glossterm{defenses}: \plus1 Armor, \plus3 Fortitude, \plus4 Reflex, \plus7 Mental.

        \cf{Wiz}{Skills}
        You have the following \glossterm{class skills}:
        \begin{itemize}
            \item \subparhead{Intelligence} Craft, Deduction, Knowledge (all kinds, taken individually), Linguistics.
            \item \subparhead{Perception} Awareness, Spellcraft.
            \item \subparhead{Other} Deception, Intimidate, Persuasion, Profession.
        \end{itemize}

        \cf{Wiz}{Weapon Proficiencies} 
        You are proficient with simple weapons and any one other \glossterm{weapon group}.

        \cf{Wiz}{Armor Proficiencies}
        You are not proficient with any type of armor.
        Encumbrance from armor interferes with the gestures you make to cast spells, which can cause your spells with \glossterm{somatic components} to fail (see \pcref{Somatic Component Failure}).

    \subsection{Alchemist}
        This archetype improves your ability to use achemy to create unusual concoctions to aid your allies and harm your foes.

        \cf{Wiz}[1]{Portable Workshop} 
        You carry materials necessary to refine low-grade alchemical items wherever you are.
        Items created with this ability deteriorate and become useless after 24 hours or after you finish a long rest, whichever comes first.
        The items are just as effective when used as items created normally.
        You can use this ability create alchemical items with a item level up to your level.

        Creating an item in this way functions in the same way as crafting alchemical items normally, with the following changes.
        First, you do not require any raw materials.
        Seond, the difficulty rating goes up by 5 after each successful craft.
        Third, if you fail to craft an item in this way, you cannot try again until you destroy all items crafted with this ability and finish a short rest.

        If you use this ability extensively without visiting civilization or some other area where you can restock your personal supplies for weeks, you may run low on supplies and your ability to use this ability may be limited.

        \cf{Wiz}[2]{Alchemical Infusion} Whenever you create or use an alchemical item, you may use your \glossterm{power} with \glossterm{magical} abilities in place of the item's normal power to determine its effects.

        \cf{Wiz}[3]{Alchemical Discovery} You gain your choice of one of the following benefits.
        {
            \parhead{Complex Construction} You can use your portable workshop ability to create items with an item level up to two levels higher than your level.
            \parhead{Durable Construction} You double the range of alchemical items you create.
            \parhead{Efficient Crafting} It takes you half the normal time and material components to craft alchemical items.
            \parhead{Explosive Construction} You double the area affected by alchemical items you create.
            \parhead{Potent Construction} Alchemical items you create gain a \plus2 bonus to power.
            \parhead{Repetitive Construction} Whenever you use your portable workshop ability, you can create two copies of the same alchemical item.
            This only counts as one item for the purpose of determining the increase to your crafting difficulty.
            \parhead{Safe Construction} Creatures who are not trained in Craft (alchemy) can use alchemical items you create without a mishap chance.
        }

        \cf{Wiz}[4]{Alchemical Expertise} You gain a \plus3 bonus to the Craft (alchemy) skill and a \plus1 bonus to accuracy with alchemical items.

        \cf{Wiz}[5]{Experienced Quaffing} You can drink up to two doses of potions, elixirs, and other drinkable alchemical items simultaneously with no mishap risk.

        \cf*{Wiz}[6]{Alchemical Discovery} You gain an additional \textit{alchemical discovery} ability.

        \cf{Wiz}[7]{Greater Alchemical Expertise} The skill bonus from your \textit{alchemical expertise} ability increses to \plus6 and the accuracy bonus increases to \plus2.

    \subsection{Arcane Magic}
        This archetype grants you the ability to cast arcane spells.
        % TODO: less boring attribute integration?
        You must have an base Intelligence of at least 1 to gain this archetype.

        \cf{Wiz}[1]{Spellcasting}[Magical]
        You have the ability to use arcane magic.
        You gain access to one arcane \glossterm{mystic sphere} (see \pcref{Arcane Mystic Spheres}).
        You may spend \glossterm{insight points} to learn one additional \glossterm{mystic sphere} per two \glossterm{insight points}.
        Each \glossterm{mystic sphere} has a set of \glossterm{spells} associated with it.

        You automatically learn all \glossterm{cantrips} from any mystic sphere you have access to.
        In addition, you learn two rank 1 \glossterm{spells} from your chosen \glossterm{mystic sphere}.
        You can also spend \glossterm{insight points} to learn one additional arcane spell per \glossterm{insight point}.
        Unless otherwise noted in a spell's description, casting a spell requires a \glossterm{standard action}.

        When you gain access to a new \glossterm{mystic sphere} or spell \glossterm{rank},
            you can exchange any number of spells you know for other spells,
            including spells of the higher rank.

        Arcane spells require both \glossterm{verbal components} and \glossterm{somatic components} to cast (see \pcref{Casting Components}).
        For details about mystic spheres and casting spells, see \pcref{Spell and Ritual Mechanics}.

        \cf{Wiz}[2]{Mystic Sphere} You gain access to an additional arcane \glossterm{mystic sphere}, including all \glossterm{cantrips} from that sphere.
        In addition, you learn an additional arcane \glossterm{spell}.

        \cf{Wiz}[3]{Spell Rank}[Magical] You become a rank 3 arcane spellcaster.
        This can improve the effectiveness of your spells and give you access to spells that require a minimum rank of 3.

        \cf{Wiz}[4]{Spell} You learn an additional \glossterm{spell} for any arcane \glossterm{mystic sphere} you know.

        \cf*{Wiz}[4]{Spell Rank}[Magical] You become a rank 4 arcane spellcaster.
        This can improve the effectiveness of your spells and give you access to spells that require a minimum rank of 4.

        \cf*{Wiz}[5]{Spell Rank}[Magical] You become a rank 5 arcane spellcaster.
        This can improve the effectiveness of your spells and give you access to spells that require a minimum rank of 5.

        \cf*{Wiz}[6]{Spell Rank}[Magical] You become a rank 6 arcane spellcaster.
        This can improve the effectiveness of your spells and give you access to spells that require a minimum rank of 6.

        \cf*{Wiz}[6]{Spell} You learn an additional \glossterm{spell} for any arcane \glossterm{mystic sphere} you know.

        \cf*{Wiz}[7]{Spell Rank}[Magical] You become a rank 7 arcane spellcaster.
        This can improve the effectiveness of your spells and give you access to spells that require a minimum rank of 7.

        \cf*{Wiz}[8]{Spell Rank}[Magical] You become a rank 8 arcane spellcaster.
        This can improve the effectiveness of your spells and give you access to spells that require a minimum rank of 8.

    \subsection{Arcane Scholar}
        This archetype deepens your study of arcane magic.
        You have the Arcane Magic archetype from the wizard class to gain the abilities from this archetype.

        \cf{Wiz}[1]{Scholastic Lore} You gain a \plus2 bonus to the Spellcraft skill and all Knowledge skills.

        \cf{Wiz}[2]{Ritualist} You gain the ability to perform arcane rituals to create unique magical effects (see \pcref{Rituals}).
        The maximum \glossterm{rank} of arcane ritual you can learn or perform is equal to the maximum rank of arcane spell that you can cast.
        In addition, whenever you gain access to a new spell rank, you can scribe one ritual of that rank or lower without paying the normal costs.

        \cf{Wiz}[3]{Arcane Insight}[Magical]
        You gain one of the following insights.
        Some insights can be chosen multiple times, as indicated in their descriptions.

        {
            \parhead{Expanded Sphere Access} You gain access to a new \glossterm{mystic sphere}.
            \par You cannot choose this insight multiple times.

            \parhead{Expanded Spell Knowledge} You learn an additional \glossterm{spell}.
            \par You can choose this insight multiple times, learning an additional spell each time.

            \parhead{Signature Spell} Choose a \glossterm{spell} you know.
            The spell loses the \glossterm{Focus} tag, allowing you to cast it without lowering your guard in combat.
            \par If you choose this insight multiple times, you must choose a different \glossterm{spell} each time.

            \parhead{Sphere Specialization} Choose a a \glossterm{mystic sphere} you have access to.
            You gain a \plus1 bonus to \glossterm{accuracy} with abilities from that \glossterm{mystic sphere}.
            In addition, you learn an additional \glossterm{spell} from that \glossterm{mystic sphere}.
            In exchange, you must lose access to another \glossterm{mystic sphere} you have.
            You must exchange all spells you know from that \glossterm{mystic sphere} with spells from other \glossterm{mystic spheres} you have access to.
            \par You cannot choose this insight multiple times.

            \parhead{Memorized Sphere} % TODO: clarify you need to be high enough rank?
            Choose a \glossterm{mystic sphere} you have access to.
            You can perform rituals from that \glossterm{mystic sphere} without having them written in your ritual book.
            If you lead a ritual from that \glossterm{mystic sphere}, it requires half the normal amount of time to perform and costs half the normal number of \glossterm{action points} (minimum 1).
            \par You can choose this insight multiple times, choosing a different \glossterm{mystic sphere} each time.
        }

        \cf{Wiz}[4]{Greater Scholastic Lore} The bonuses from your \textit{scholastic lore} ability increase to \plus4.

        \cf{Wiz}[5]{Contingency}  You learn the Contingency \glossterm{augment}, allowing you to prepare a spell so it takes effect automatically if specific circumstances arise.
        Applying the Contingency \glossterm{augment} increases the \glossterm{rank} of the spell and all its upgrades by 2.
        % If any spells take more than one standard action, they would need to be excluded from Contingency, but none exist
        % You can apply this augment to any arcane spell that can be cast as a \glossterm{standard action} or \glossterm{minor action}.

        Casting a spell with the Contingency augment takes 5 minutes.
        When the casting is complete, the spell has no immediate effect.
        Instead, it automatically takes effect when some specific circumstances arise.
        During the time required to cast the spell, you specify what circumstances cause the spell to take effect.

        The spell can be set to trigger in response to any circumstances that a typical human observing you and your situation could detect.
        For example, you could specify ``when I fall at least 50 feet'' or ``when I take a \glossterm{vital wound}'', but not ``when there is an invisible creature within 50 feet of me'' or ``when I have only one \glossterm{hit point} remaining.''
        The more specific the required circumstances, the better -- vague requirements, such as ``when I am in danger'', may cause the spell to trigger unexpectedly or fail to trigger at all.
        If you attempt to specify multiple separate triggering conditions, such as ``when I take damage or when an enemy is adjacent to me'', the spell will randomly ignore all but one of the conditions.

        If the spell needs to be targeted, the trigger condition can specify a simple rule for identifying how to target the spell, such as ``the closest enemy''.
        If the rule is poorly worded or imprecise, the spell may target incorrectly or fail to activate at all.
        Any spells which require decisions, such as the \spell{dimension door} spell, must have those decisions made at the time it is cast.
        You cannot alter those decisions when the contingency takes effect.

        You can have only one spell with this augment active at a time.
        If you use the augment again with a different spell, the old contingency is removed.

        \cf*{Wiz}[6]{Arcane Insight}[Magical]
        You learn an additional \textit{arcane insight}.

        \cf{Wiz}[7]{Malleable Contingency} If you have a \textit{contingency} active, you can change the conditions to trigger your \textit{contingency} as a standard action.
        This does not allow you to change the spell stored in your \textit{contingency}.

    \subsection{Arcane Spell Mastery}
        This archetype improves the arcane spells you cast.
        You must have the Arcane Magic archetype from the wizard class to gain the abilities from this archetype.

        \cf{Wiz}[1]{Mystic Insight}[Magical]
        You gain your choice of one of the following abilities.
        {
            \parhead{Focused Caster} You reduce your \glossterm{focus penalty} by 1.
                You cannot choose this ability multiple times.
            \parhead{Mystic Sphere Access} You gain access to an additional arcane \glossterm{mystic sphere}.
                You cannot choose this ability multiple times.
            \parhead{Rituals} You gain the ability to perform arcane rituals to create unique magical effects (see \pcref{Rituals}).
                The maximum \glossterm{rank} of arcane ritual you can learn or perform is equal to the maximum rank of arcane spell that you can cast.
                You cannot choose this ability multiple times.
            \parhead{Spell Known} You learn an additional arcane \glossterm{spell} (see \pcref{Arcane Mystic Spheres}).
                You can choose this ability multiple times, learning an additional spell each time.
            \parhead{Spell Power} Choose an arcane \glossterm{spell} you know.
                You gain a \plus2 bonus to \glossterm{power} with that spell.
                You can choose this ability multiple times, choosing a different spell each time.
            \parhead{Spell Precision} Choose an arcane \glossterm{spell} you know.
                You gain a \plus1 bonus to \glossterm{accuracy} with that spell.
                You can choose this ability multiple times, choosing a different spell each time.
        }

        \cf{Wiz}[2]{Mage Armor}[Magical] You can use the \textit{mage armor} ability as a standard action.
        \begin{freeability}{Mage Armor}
            You create a translucent suit of magical armor on your body and over your hands.
            This functions like body armor that provides a \plus2 bonus to Armor defense and has no \glossterm{encumbrance}.
            It also provides a bonus to \glossterm{resistances} against \glossterm{energy damage} equal to half your level.
            The body armor does not appear if you are wearing other body armor of any kind.

            As long as you have a free hand, the barrier also manifests as a shield that provides a \plus1 bonus to Armor defense.
            This bonus is considered to come from a shield, and does not stack with the benefits of using a physical shield.

            This ability lasts until you \glossterm{dismiss} it as a free action.
        \end{freeability}

        \cf{Wiz}[3]{Wellspring of Power}[Magical]
        You gain a \plus1 bonus to \glossterm{power} with \glossterm{magical} abilities.

        \cf*{Wiz}[4]{Mystic Insight}[Magical]
        You gain an additional \textit{mystic insight} ability.

        \cf{Wiz}[5]{Greater Mage Armor}[Magical]
        The defense bonus from the body armor created by your \textit{mage armor} ability increases to \plus3.
        In addition, the bonus to \glossterm{resistances} increases to be equal to your level.

        \cf{Wiz}[6]{Greater Wellspring of Power}[Magical]
        The bonus from your \textit{wellspring of power} ability increases to \plus2.

        \cf*{Wiz}[7]{Mystic Insight}[Magical]
        You gain an additional \textit{mystic insight} ability.
