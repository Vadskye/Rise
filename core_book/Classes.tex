\chapter{Classes}\label{Classes}

Your character's class represents the things your character has chosen to train in.
This choice determines a great deal about your character's abilities.

\section{How Classes Work}
    When you first create a character, you choose a class.
    You gain all abilities granted by the \glossterm{archetypes} of your chosen class at the levels indicated in the archetype's description (see Archetypes, below).
    As you gain levels, you gain more abilities from your class.

    \subsection{Archetypes}\label{Archetypes}
        Each class has three or four \glossterm{archetypes}.
        An archetype is a collection of thematically related class abilities.
        For examples, barbarians have the Battlerager archetype, which grants abilities related to flying into a rage in combat.

        Normally, you choose three of the archetypes associated with your class at 1st level.
        You gain the abilities from those three archetypes, and you gain no abilities from any additional archetypes the class has.

        \subsubsection{Archetype Ranks}\label{Archetype Ranks}
            You have an \glossterm{archetype rank} associated with each archetype you have.
            Each ability from an archetype has a minimum rank required to gain the ability.

            At 1st level, you are Rank 1 in each archetype.
            This gives you the Rank 1 abilities from each of your archetypes.
            Every level thereafter, you increase your rank in one archetype of your choice.
            This gives you the abilities associated with that rank.

            Each \glossterm{archetype rank} has a minimum level, as shown on \trefnp{Archetype Ranks by Level}.

            \begin{dtable}
                \lcaption{Archetype Ranks by Level}
                \begin{dtabularx}{\columnwidth}{l >{\lcol}X}
                    \tb{Archetype Rank} & \tb{Minimum Level} \tableheaderrule
                    1     & 1
                    \\ 2  & 2
                    \\ 3  & 5
                    \\ 4  & 8
                    \\ 5 & 11
                    \\ 6 & 14
                    \\ 7 & 17
                    \\ 8 & 20
                \end{dtabularx}
            \end{dtable}

        \subsubsection{Duplicate Archetypes}\label{Duplicate Archetypes}
            Some archetypes can be gained by multiple classes.
            For example, both clerics and paladins have the Divine Spellcasting archetype.
            You cannot gain two archetypes with the same name, even if you can choose archetypes from multiple classes.

\section{Class Introductions}

    There are nine classes in Rise.
    \begin{itemize}
        \item Barbarians are mighty warriors who can enter a deadly battlerage.
        \item Clerics are divine spellcasters who draw power from their veneration of a deity.
        \item Druids are nature spellcasters who draw power from their veneration of the natural world.
        \item Fighters are highly disciplined warriors who excel in physical combat of any variety.
        \item Mages are arcane spellcasters who wield the mystic forces of magic to create almost any effect.
        \item Monks are agile masters of ``ki'' who hone their personal abilities to strike down foes and perform supernatural feats.
        \item Paladins are divinely empowered warriors whose devotion to an alignment grants them the ability to discern and smite their foes.
        \item Rangers are skilled hunters who bridge the divide between nature and civilization.
        \item Rogues are exceptionally skillful characters known for their ability to strike at their foe's weak points in combat.
            % \item Spellwarped wield a unique blend of martial skill and narrowly focused magical abilities.
        \item Warlocks are pact spellcasters who draw their power from a dark pact made with infernal creatures.
    \end{itemize}

    \subsection{Class Description Format}
        Each class is described from the perspective of a member of that class, using ``you'' in the description.

        \parhead{Class Table}
        The class's table describes the special abilities a member of that class gains at each level, assuming they have all of that class's \glossterm{archetypes}.

        \parhead{Alignment}
        Some classes require specific alignments (see \pcref{Alignment}).
        Most classes allow characters of any alignment.

        \parhead{Skills}
        Each class has specific \glossterm{skills} that members of that class are typically good at (see \pcref{Skills}).
        These skills are called \glossterm{class skills}.
        It is easier to become \glossterm{mastered} in class skills than in other skills.
        For details, see \pcref{Skill Training}.

        \parhead{Defenses}
        Each class grants bonuses to specific defenses.

        \parhead{Weapon and Armor Proficiencies}
        These are the types of equipment that members of this class are trained in using.

        \parhead{Other Special Abilities}
        Some classes have abilities shared by all members of the class that are not part of an archetype, such as a druid's \textit{druidic language} ability.

        \parhead{Archetypes}
        The abilities associated with each of the three archetypes the class has.

\newpage
\section{Barbarian}\label{Barbarian}
    \begin{dtable!*}
        \lcaption{Barbarian Progression}
        \begin{dtabularx}{\textwidth}{c >{\lcol}X >{\lcol}X >{\lcol}X >{\lcol}X}
            \tb{Rank} & \tb{Battleforged Resilience} & \tb{Battlerager}  & \tb{Primal Warrior} & \tb{Outland Savage} \tableheaderrule
            1    & Fury of the storm         & Rage                       & Primal maneuvers         & Fast movement
            \\ 2 & Battlefield lore          & Intimidating anger         & Athletic prowess     & Exotic weaponry
            \\ 3 & Battle-scarred            & Focused anger              & Primal power         & Savage precision, savage force
            \\ 4 & Battleforged fortitude    & Invigorating rage          & Primal maneuver      & Greater fast movement
            \\ 5 & Greater battle-scarred               & Greater rage               & Greater primal power & Savage resilience
            \\ 6 & Greater fury of the storm & Greater intimidating anger & Primal prowess       & Greater savage precision, greater savage force
            \\ 7 & Supreme battle-scarred    & Greater invigorating rage  & Primal maneuver      & Supreme fast movement
            \\ 8 & Battleforged supremacy    & Titanic rage               &
        \end{dtabularx}
    \end{dtable!*}

    \classbasics{Alignment} Any nonlawful.

    \classbasics{Archetypes} Barbarians have the Battlerager, Primal Warrior, Battleforged Resilience, and Outland Savage \glossterm{archetypes}.
    If you are a barbarian, you choose three of those four archetypes.

    \subsection{Basic Class Abilities}
        If you are a barbarian, you gain the following abilities.

        \cf{Bbn}{Defenses}
        You gain the following bonuses to your \glossterm{defenses}: \plus5 Fortitude, \plus4 Reflex, \plus3 Mental.

        \cf{Bbn}{Skills}
        You have the following \glossterm{class skills}:
        \begin{itemize}
            \item \subparhead{Strength} Climb, Jump, Swim.
            \item \subparhead{Dexterity} Acrobatics, Ride.
            \item \subparhead{Intelligence} Craft.
            \item \subparhead{Perception} Awareness, Creature Handling, Survival.
            \item \subparhead{Other} Bluff, Intimidate, Persuasion, Profession.
        \end{itemize}

        \cf{Bbn}{Weapon and Armor Proficiencies} 
        You are proficient with simple weapons, any four other weapon groups, light armor, medium armor, and shields.

    \subsection{Battlerager}\label{Rage}
        This archetype grants you a devastating rage, improving your combat prowess.
        % Rank 7 summary:
        % +4 power with mundane abilities
        % +2x level to physical resistances
        % +4 intimidate
        % +4 Mental defense
        % -2 Armor and Reflex defenses

        \cf{Bbn}[1]{Rage} You can use the \textit{rage} ability as a \glossterm{free action}.
        % TODO: prevent recovering this AP; may need to be Attunement again?
        \begin{attuneability}{Rage}[\glossterm{Emotion}, \glossterm{Attune} (self), \glossterm{Swift}]
            You gain the following benefits and drawbacks:
            \begin{itemize}
                \item You gain a \plus2 bonus to \glossterm{power} with \glossterm{mundane} abilities.
                \item You take a \minus2 penalty to Armor and Reflex defenses.
                \item You gain a bonus equal to your level to your \glossterm{resistances} against \glossterm{physical damage}.
                \item You are unable to take \glossterm{standard actions} that do not cause you to make \glossterm{mundane} attacks.
                \item At the end of each round, if you did not make a \glossterm{mundane} attack that round, this ability ends.
                    If the ability ends this way, you regain the \glossterm{action point} you spent to attune to this ability the next time you take a \glossterm{short rest}.
            \end{itemize}
        \end{attuneability}

        \cf{Bbn}[2]{Seething Anger}
        You gain a \plus2 bonus to the \glossterm{Intimidate} skill and to Mental defense.

        \cf{Bbn}[3]{Invigorating Rage} When you use your \textit{rage} ability, you may regain a \glossterm{hit point}.
        If you do, you lose that \glossterm{hit point} when the ability ends.

        \cf{Bbn}[4]{Enduring Anger} You gain an additional \glossterm{hit point}.

        \cf{Bbn}[5]{Painless Rage} The bonus to \glossterm{resistances} from your \textit{rage} ability increases to be equal to twice your level.

        \cf{Bbn}[6]{Greater Seething Anger}
        The bonuses from your \textit{seething anger} ability increase to \plus4.

        \cf{Bbn}[7]{Powerful Rage} The \glossterm{power} bonus from your \textit{rage} ability increases to \plus4.

        \cf{Bbn}[8]{Titanic Rage}
        When you use your \textit{rage} ability, you can grow by one \glossterm{size category}.

    \subsection{Primal Warrior}
        This archetype grants you abilities to use in combat and improves your physical skills.
        % Rank 7 summary:
        % Standard active abilities
        % +2 skill points
        % +2 Str/Dex checks

        \cf{Bbn}[1]{Primal Maneuvers}
        You can channel your primal energy into ferocious attacks.
        You learn one \glossterm{maneuver} from the primal maneuver list (see \pcref{Primal Maneuvers}).
        You can also spend \glossterm{insight points} to learn one additional \glossterm{maneuver} per \glossterm{insight point}.
        As a \glossterm{standard action}, you can use any \glossterm{maneuver} you know.

        \cf{Bbn}[2]{Athletic Prowess} You gain two additional \glossterm{skill points}.

        \cf{Bbn}[3]{Primal Power} You gain a \plus1 bonus to \glossterm{power} with \glossterm{mundane} abilities.

        \cf{Bbn}[4]{Primal Maneuver}
        You learn an additional primal \glossterm{maneuver} (see \pcref{Primal Maneuvers}).

        \cf{Bbn}[5]{Greater Primal Power} The bonus from your \textit{primal power} ability increases to \plus2.

        \cf{Bbn}[6]{Primal Prowess}
        You gain a \plus2 bonus to Strength-based and Dexterity-based checks.

        \cf*{Bbn}[7]{Primal Maneuver}
        You learn an additional primal \glossterm{maneuver} (see \pcref{Primal Maneuvers}).

    \subsection{Battleforged Resilience}
        This archetype improves your defenses in combat.
        % Rank 7 summary:
        % +2 overwhelm resistance
        % +2 skill points
        % +level to resistances
        % +2 Fortitude

        \cf{Bbn}[1]{Battleforged Endurance} You gain an additional \glossterm{hit point}.

        \cf{Bbn}[2]{Battle-Scarred} You gain a bonus equal to half your level to your \glossterm{resistances} against \glossterm{physical damage}.

        \cf{Bbn}[3]{Battlefield Lore} You gain two additional \glossterm{skill points}.

        \cf{Bbn}[4]{Greater Battle-Scarred} The bonus from your \textit{battle-scarred} ability increases to be equal to your level.

        \cf{Bbn}[5]{Greater Battleforged Endurance} You gain an additional \glossterm{hit point}.

        \cf{Bbn}[6]{Soulscarred}
        The bonus from your \textit{battle-scarred} ability applies against all damage.

        \cf{Bbn}[7]{Supreme Battle-Scarred}
        The bonus from your \textit{battle-scarred} ability increases to be equal to twice your level.

    \subsection{Outland Savage}
        This archetype improves your mobility and combat prowess with direct, brutal abilities.

        \cf{Bbn}[1]{Fast Movement} You gain a \plus10 foot bonus to your \glossterm{base speed}.

        \cf{Bbn}[2]{Exotic Weaponry} You gain proficiency with \glossterm{exotic weapons} from all weapon groups you are proficient with.

        \cf{Bbn}[3]{Savage Precision} You gain a \plus1 bonus to \glossterm{accuracy} with the \textit{dirty trick}, \textit{disarm}, \textit{grapple}, \textit{overrun}, \textit{shove}, and \textit{trip} abilities (see \pcref{Special Combat Abilities}).
        In addition, you gain a \plus1 bonus to \glossterm{accuracy} with grapple actions (see \pcref{Grapple Actions}).

        \cf{Bbn}[3]{Savage Force} You gain a \plus1 bonus to \glossterm{power} with \glossterm{mundane} abilities.

        \cf{Bbn}[4]{Greater Fast Movement} The speed bonus from your \textit{fast movement} ability increases to \plus20 feet.

        \cf{Bbn}[5]{Savage Resilience} You gain a \plus2 bonus to Fortitude defense.

        \cf{Bbn}[6]{Greater Savage Force} The bonus from your \textit{savage force} ability increases to \plus2.

        \cf{Bbn}[6]{Greater Savage Precision} The bonuses from your \textit{savage force} ability increase to \plus2.

        \cf{Bbn}[7]{Supreme Fast Movement} The speed bonus from your \textit{fast movement} ability increases to \plus30 feet.

    \subsection{Ex-Barbarians}
        If you become lawful, you cannot use your \textit{rage} ability.
        You retain all of your other class abilities.
        If you stop being lawful, you can use your \textit{rage} ability once more.

\newpage
\section{Cleric}\label{Cleric}
    \begin{dtable!*}
        \lcaption{Cleric Progression}
        \begin{dtabularx}{\textwidth}{c >{\lcol}X >{\lcol}X >{\lcol}X >{\lcol}X}
            \tb{Rank} & \tb{Divine Spellcasting} & \tb{Divine Spell Mastery}  & \tb{Domain Influence} & \tb{Healer} \tableheaderrule
            1    & Spellcasting  & Battle prayer               & Domains, domain gift & Divine healing
            \\ 2 & Mystic sphere   & Mystic knowledge            & Domain gift    & Vital healer
            \\ 3 & Spell level (2) & Wellspring of power         & Domain aspect  & Divine cleanse
            \\ 4 & Spell level (3) & Mystic knowledge            & Domain aspect  & Greater divine healing
            \\ 5 & Spell level (4) & Greater battle prayer       & Domain essence & Greater vital healer
            \\ 6 & Spell level (5) & Mystic knowledge            & Domain essence & Greater divine cleanse
            \\ 7 & Spell level (6) & Greater wellspring of power & Miracle        & Supreme divine healing
            \\ 8 & Spell level (7) &                             & Domain masteries
        \end{dtabularx}
    \end{dtable!*}

    \classbasics{Alignment} Your alignment must be within one step of your deity's (that is, it may be one step away on either the lawful-chaotic axis or the good-evil axis, but not both).

    \classbasics{Archetypes} Clerics have the Divine Spellcasting, Domain Influence, and Divine Spell Mastery \glossterm{archetypes}.

    \subsection{Basic Class Abilities}
        If you are a cleric, you gain the following abilities.

        \cf{Clr}{Defenses}
        You gain the following bonuses to your \glossterm{defenses}: \plus4 Fortitude, \plus3 Reflex, \plus5 Mental.

        \cf{Clr}{Skills}
        You have the following \glossterm{class skills}:
        \begin{itemize}
            \item \subparhead{Intelligence} Craft, Deduction, Heal, Knowledge (arcana, local, religion, the planes), Linguistics.
            \item \subparhead{Perception} Awareness, Sense Motive, Spellcraft.
            \item \subparhead{Other} Bluff, Intimidate, Persuasion, Profession.
        \end{itemize}

        \cf{Clr}{Weapon and Armor Proficiencies}
        You are proficient with simple weapons, any two other weapon groups, light and medium armor, and shields.

        \cf{Clr}{Deity}
        You must worship a specific deity to be a cleric.
        Deities and their associated domains are listed in \trefnp{Deities}.

        \begin{dtable!*}
            \lcaption{Deities}
            \begin{dtabularx}{\textwidth}{X l X}
                \tb{Deity} & \tb{Alignment} & \tb{Domains} \tableheaderrule
                Guftas, horse god of justice          & Lawful good     & Good, Law, Strength, Travel         \\
                Lucied, paladin god of justice        & Lawful good     & Destruction, Good, Protection, War  \\
                Simor, fighter god of protection      & Lawful good     & Good, Protection, Strength, War     \\
                %Pabst, dwarf god of drink             & Neutral good & Good, Life, Strength, Wild \\
                Rucks, monk god of pragmatism         & Neutral good    & Good, Law, Protection, Travel       \\
                Vanya, centaur god of nature          & Neutral good    & Good, Strength, Travel, Wild        \\
                Brushtwig, pixie god of creativity    & Chaotic good    & Chaos, Good, Trickery, Wild         \\
                Chavi, god of stories                 & Chaotic good    & Chaos, Knowledge, Trickery          \\
                Ivan Ivanovitch, bear god of strength & Chaotic good    & Chaos, Strength, War, Wild          \\
                Krunch, barbarian god of destruction  & Chaotic good    & Destruction, Good, Strength, War    \\
                Sir Cakes, dwarf god of freedom       & Chaotic good    & Chaos, Good, Strength               \\
                Raphael, monk god of retribution      & Lawful neutral  & Death, Law, Protection, Travel      \\
                Declan, god of fire                   & True neutral    & Destruction, Fire, Knowledge, Magic \\
                Kurai, shaman god of nature           & True neutral    & Air, Earth, Fire, Water             \\
                %Amanita, druid god of decay           & Chaotic neutral & Chaos, Destruction, Life, Wild \\
                %Antimony, elf god of necromancy       & Chaotic neutral & Death, Knowledge, Life, Magic \\
                Clockwork, elf god of time            & Chaotic neutral & Chaos, Magic, Trickery, Travel      \\
                %Lord Khallus, fighter god of pride    & Chaotic neutral & Chaos, Strength, War \\
                %Celeano, sorcerer god of deception    & Chaotic neutral & Chaos, Magic, Protection, Trickery \\
                Murdoc, god of mercenaries            & Chaotic neutral & Destruction, Knowledge, Travel, War \\
                Ribo, halfling god of trickery        & Chaotic neutral & Chaos, Trickery, Water              \\
                Tak, orc god of war                   & Lawful evil     & Law, Strength, Trickery, War        \\
                Theodolus, sorcerer god of ambition   & Neutral evil    & Evil, Knowledge, Magic, Trickery    \\
                Daeghul, demon god of slaughter       & Chaotic evil    & Destruction, Evil, Magic, War       \\
            \end{dtabularx}
        \end{dtable!*}

    \subsection{Divine Spellcasting}
        This archetype grants you the ability to cast divine spells.
        You must have a starting Willpower of at least 1 to gain this archetype.

        \cf{Clr}[1]{Spellcasting}[Magical]
        Your deity grants you the ability to use divine magic.
        You gain access to one divine \glossterm{mystic sphere} (see \pcref{Divine Mystic Spheres}).
        Each \glossterm{mystic sphere} has a set of \glossterm{spells} associated with it.

        You automatically learn all \glossterm{cantrips} from any mystic sphere you have access to.
        In addition, you learn one 1st level \glossterm{spell} from your chosen \glossterm{mystic sphere}.
        You can also spend \glossterm{insight points} to learn one additional divine spell per \glossterm{insight point}.
        Unless otherwise noted in a spell's description, casting a spell requires a \glossterm{standard action}.

        When you gain access to a new \glossterm{mystic sphere} or spell level,
            you can exchange any number of spells you know for other spells,
            including spells of the higher level.

        Divine spells require \glossterm{verbal components} to cast (see \pcref{Casting Components}).
        For details about mystic spheres and casting spells, see \pcref{Spell and Ritual Mechanics}.

        \cf{Clr}[2]{Mystic Sphere} You gain access to an additional divine \glossterm{mystic sphere}, including all \glossterm{cantrips} from that sphere.
        In addition, you learn an additional divine \glossterm{spell}.

        \cf{Clr}[3]{Spell Level}[Magical] You gain the ability to cast 2nd level divine spells.

        \cf*{Clr}[4]{Spell Level}[Magical] You gain the ability to cast 3rd level divine spells.

        \cf*{Clr}[5]{Spell Level}[Magical] You gain the ability to cast 4th level divine spells.

        \cf*{Clr}[6]{Spell Level}[Magical] You gain the ability to cast 5th level divine spells.

        \cf*{Clr}[7]{Spell Level}[Magical] You gain the ability to cast 6th level divine spells.

        \cf*{Clr}[8]{Spell Level}[Magical] You gain the ability to cast 7th level divine spells.

    \subsection{Domain Influence}
        This archetype grants you divine influence over two domains of your choice.

        \cf{Clr}[1]{Domains}
        You choose two domains which represent your personal spiritual inclinations.
        You must choose your domains from among those your deity offers.
        The domains are listed below.

        \begin{itemize}
            \item{Air}
            \item{Chaos}
            \item{Death}
            \item{Destruction}
            \item{Earth}
            \item{Evil}
            \item{Fire}
            \item{Good}
            \item{Knowledge}
            \item{Law}
            \item{Life}
            \item{Magic}
            \item{Protection}
            \item{Strength}
            \item{Travel}
            \item{Trickery}
            \item{War}
            \item{Water}
            \item{Wild}
        \end{itemize}

        \cf{Clr}[1]{Domain Gift}[Magical]
        Each domain has a corresponding \textit{domain gift}.
        You gain the \textit{domain gift} for one of your domains (see \pcref{Cleric Domain Abilities}).

        \cf*{Clr}[2]{Domain Gift}[Magical]
        You gain the \textit{domain gift} for another one of your domains.

        \cf{Clr}[3]{Domain Aspect}[Magical]
        Each domain has a corresponding \textit{domain aspect}.
        You gain the \textit{domain aspect} for one of your domains (see \pcref{Cleric Domain Abilities}).

        \cf*{Clr}[4]{Domain Aspect}[Magical]
        You gain the \textit{domain aspect} for another one of your domains.

        \cf{Clr}[5]{Domain Essence}[Magical]
        Each domain has a corresponding \textit{domain essence}.
        You gain the \textit{domain essence} for one of your domains (see \pcref{Cleric Domain Abilities}).

        % This seems weak
        \cf*{Clr}[6]{Domain Essence}[Magical]
        You gain the \textit{domain essence} for another one of your domains.

        \cf{Clr}[7]{Miracle}[Magical]
        Once per week, you can request a miracle as a standard action.
        You mentally specify your request, and your deity fulfills that request in the manner it sees fit.
        This can emulate the effects of any spell or ritual, or have any other effect of a similar power level.
        If the deity has a direct interest in your situation, the miracle may be of even greater power.

        If you perform an extraordinary service for your deity, you can gain the ability to request an additional miracle that week.

        \cf{Clr}[8]{Domain Masteries}[Magical]
        Each domain has a corresponding \textit{domain mastery}.
        You gain the \textit{domain mastery} for both of your domains (see \pcref{Cleric Domain Abilities}).

    \subsection{Divine Spell Mastery}
        This archetype improves the divine spells you cast.
        You must be able to cast divine spells to gain the abilities from this archetype.

        \cf{Clr}[1]{Battle Prayer} You gain a \plus2 bonus to \glossterm{concentration} checks to cast divine spells.

        \cf{Clr}[2]{Mystic Knowledge}[Magical]
        You gain your choice of one of the following abilities.
        {
            \parhead{Insight} You learn an additional divine \glossterm{spell} (see \pcref{Divine Mystic Spheres}).
                You can choose this ability multiple times, gaining access to an additional spell each time.
            \parhead{Rituals} You gain the ability to perform divine rituals to create unique magical effects (see \pcref{Rituals}).
                The maximum level of divine ritual you can learn or perform is equal to the maximum level of divine spell that you can cast.
                You cannot choose this ability multiple times.
            \parhead{Mystic Sphere Access} You gain access to an additional divine \glossterm{mystic sphere} (see \pcref{Divine Mystic Spheres}).
                You cannot choose this ability multiple times.
        }

        \cf{Clr}[3]{Wellspring of Power}[Magical]
        You gain a \plus1 bonus to \glossterm{power} with divine spells.

        \cf*{Clr}[4]{Mystic Knowledge}[Magical]
        You gain an additional \textit{mystic knowledge} ability.

        \cf{Clr}[5]{Greater Battle Prayer} The bonus from your \textit{battle prayer} ability increases to \plus4.

        \cf*{Clr}[6]{Mystic Knowledge}[Magical]
        You gain an additional \textit{mystic knowledge} ability.

        \cf{Clr}[7]{Greater Wellspring of Power}[Magical]
        The bonus from your \textit{wellspring of power} ability increases to \plus2.

    \subsection{Healer}
        This archetype grants you healing abilities.

        \cf{Clr}[1]{Mitigate Wound} You can use the \textit{mitigate wound} ability as a standard action.
        \begin{freeability}{Mitigate Wound}[\glossterm{Magical}]
            Choose yourself or a living \glossterm{ally} within \rngclose range.
            The target can roll again for its most recent \glossterm{wound roll} with a \plus2 bonus.
            It can decide whether to keep its original roll or use the new roll.
            In either case, the \glossterm{wound roll} for that \glossterm{vital wound} cannot be modified again.

            \rankline
            \rank{3} The bonus increases to \plus3.
            \rank{5} The range increases to \rnglong.
            \rank{7} The bonus increases to \plus4.
        \end{freeability}

        \cf{Clr}[2]{Divine Cleanse} You can use the \textit{divine cleanse} ability as a standard action.
        \begin{freeability}{Divine Cleanse}[\glossterm{Magical}]
            You or an \glossterm{ally} within \rngclose range removes one \glossterm{condition}.

            \rankline
            \rank{4} The ability gains the \glossterm{Swift} tag.
            This means that any effects of the removed condition do not affect the target during the current phase.
            \rank{6} The range increases to \rnglong.
            \rank{8} The target can remove an additional condition.
        \end{freeability}

        \cf{Clr}[3]{Divine Healing} You can use the \textit{divine healing} ability as a standard action.
        \begin{apability}{Divine Healing}[\glossterm{Magical}]
            You or a living \glossterm{ally} within \rngclose range removes a \glossterm{vital wound}.

            \rankline
            \rank{5} The target removes an additional \glossterm{vital wound}.
            \rank{7} The range increases to \rnglong.
        \end{apability}

        \cf{Clr}[4]{Healer's Grace} You gain a \plus1 bonus to all defenses.
        After you attack a living creature, you lose this bonus until the end of the next round.

        \cf{Clr}[5]{Revivify} You can use the \textit{revivify} ability as a standard action.
        \begin{apability}{Revivify}
            Choose a dead creature within \rngclose range.
            If it was dead for no more than 1 minute, it is restored to life, as the \ritual{resurrection} ritual.
            After using this ability, you cannot use it for 24 hours.

            \rankline
            \rank{7} You can use this ability again after 1 hour instead of after 24 hours.
        \end{apability}

        % \cf{Clr}[5]{Reflexive Healing} When you suffer a \glossterm{vital wound} that makes you \glossterm{unconscious}, you automatically use your \textit{mitigate wound} ability on yourself at the end of the round.

        % TODO: why 4? Ideally, this should correspond to some sort of ``half as strong'' benchmark.
        \cf{Clr}[6]{Wellspring of Health} When you use your \textit{divine healing} ability,
            if the target's level is at least four levels lower than your level,
            you regain the \glossterm{action point} spent to use the ability.

        \cf{Clr}[7]{Greater Healer's Grace} The bonus from your \textit{healer's grace} ability increases to \plus2.

    \subsection{Cleric Domain Abilities}\label{Cleric Domain Abilities}
        These domain abilities can be granted by the \textit{domain influence} cleric archetype.
        All cleric domain abilities are \glossterm{magical} unless otherwise specified.

        \subsubsection{Air}
            If you choose this domain, you add the \sphere{aeromancy} \glossterm{mystic sphere} to your list of divine mystic spheres (see \pcref{Mystic Spheres}).
            In addition, you add the Jump skill to your \glossterm{class skill} list.

            \parhead{Gift} You gain a \plus4 \glossterm{magic bonus} to the Jump skill (see \pcref{Jump}).
            In addition, you gain a \plus1 bonus to Reflex defense.
            \parhead{Aspect} You gain a \glossterm{glide speed} equal to your \glossterm{base speed} (see \pcref{Gliding}).
            \parhead{Essence} You can use the \textit{speak with air} ability as a standard action.
            \begin{attuneability}{Speak with Air}[\glossterm{Attune} (self)]
                You can speak with and command air within a \areahuge zone from your location.
                You can ask the air simple questions and understand its responses.
                If you command the air to perform a task, it will do so do the best of its ability until this effect ends.
                You cannot compel the air to move faster than 50 mph.

                After you use this ability on a particular area of air, you cannot use it again on that same area for 24 hours.
            \end{attuneability}
            \parhead{Mastery} You gain a \glossterm{fly speed} equal to your \glossterm{base speed} (see \pcref{Flying}).

        \subsubsection{Chaos}
            If you choose this domain, you add the Bluff skill to your \glossterm{class skill} list.

            \parhead{Gift} You gain a \plus5 bonus to \glossterm{defenses} against \glossterm{Compulsion} effects.
            \parhead{Aspect} If you roll a 1 on an attack roll, it explodes (see \pcref{Exploding Attacks}).
            This does not affect bonus dice rolled for exploding attacks (see \pcref{Exploding Attacks}).
            \parhead{Essence} You can use the \textit{twist of fate} ability as a standard action.
            \begin{freeability}{Twist of Fate}
                An improbable event occurs within \rnglong range.
                You can specify in general terms what you want to happen, such as ``Make the bartender leave the bar''.
                You cannot control the exact nature of the event, though it always beneficial for you in some way.
                After using this ability, you cannot use it again for an hour.
            \end{freeability}
            \parhead{Mastery} You gain a \plus5 bonus to \glossterm{accuracy} with any attack roll that explodes (see \pcref{Exploding Attacks}).

        \subsubsection{Death}
            % Lame gift
            \parhead{Gift} You gain a \plus2 bonus to Fortitude defense.
            \parhead{Aspect} You radiate an aura of death within a \areamed radius emanation.
            The \glossterm{wound resistance} of each enemy in the area is reduced by an amount equal to your \glossterm{power}.
            \parhead{Essence} You can use the \textit{speak with dead} ability as a standard action.
            \begin{attuneability}{Speak with Dead}
                Choose a corpse within \rngclose range.
                The corpse must have died no more than five minutes ago.
                It regains a semblance of life, allowing you to speak with it as if it were the creature the corpse belonged to.
                The corpse must have an intact mouth to be able to speak.
                This ability ends if five minutes have passed since the creature died.
            \end{attuneability}
            \parhead{Mastery} The bonus from this domain's aspect increases to be equal to twice your \glossterm{power}.
            In addition, the area increases to a \arealarge radius emanation.

        \subsubsection{Destruction}
            \parhead{Gift} Your attacks deal double damage to objects.
            % TODO: upgrade timing is a little weird
            \parhead{Aspect} You can use the \textit{destructive attack} ability as a standard action.
            \begin{freeability}{Destructive Attack}
                Make a \glossterm{strike} with a \plus1d bonus to damage.

                \rankline
                \rank{3} The damage bonus increases to \plus2d.
                \rank{5} The damage bonus increases to \plus3d.
                \rank{7} The damage bonus increases to \plus4d.
            \end{freeability}
            \parhead{Essence} You can use the \textit{lay waste} ability as a standard action.
            \begin{freeability}{Lay Waste}
                You make an attack vs. Fortitude against all unattended objects in a \arealarge radius.
                You may freely exclude any number of 5-ft\. cubes from the area, as long as the resulting area is still contiguous.
                \hit For each target, if its \glossterm{damage resistance} against \glossterm{physical damage} is lower than your \glossterm{power}, it crumbles into a fine power and is irreparably \glossterm{broken}.
            \end{freeability}
            \parhead{Mastery} You gain a \plus2 bonus to \glossterm{power} with \glossterm{mundane} abilities.

        \subsubsection{Earth}
            If you choose this domain, you add the \sphere{terramancy} \glossterm{mystic sphere} to your list of divine mystic spheres (see \pcref{Mystic Spheres}).

            \parhead{Gift} You gain a \plus2 bonus to Fortitude defense.
            \parhead{Aspect} You gain a \glossterm{magic bonus} equal to half your \glossterm{power} to \glossterm{resistances} against \glossterm{physical damage}.
            \parhead{Essence} You can use the \textit{speak with earth} ability as a standard action.
            \begin{attuneability}{Speak with Earth}[\glossterm{Attune} (self)]
                You can speak with earth within a \areahuge zone from your location.
                You can ask the earth simple questions and understand its responses.

                After you use this ability on a particular area of earth, you cannot use it again on that same area for 24 hours.
            \end{attuneability}
            \parhead{Mastery} The bonus from this domain's aspect increases to be equal to your \glossterm{power}.
            In addition, you gain a \plus1 bonus to all defenses as long as you are on solid ground.

        \subsubsection{Evil}
            \parhead{Gift} At the start of each phase, you may choose an adjacent \glossterm{ally}.
            If you do, that creature takes half of all damage you take that phase (rounded down) instead of you.
            Any abilities it has that would reduce or negate the effects of the attack have no effect.
            You take the remaining half of the damage, and suffer any non-damaging effects of all attacks normally.
            \parhead{Aspect} You can use this domain's domain gift to target any \glossterm{ally} within \rngclose range.
            \parhead{Essence} You can use the \textit{compel evil} ability as a standard action.
            \begin{freeability}{Compel Evil}[\glossterm{Compulsion}]
                Make an attack vs. Mental against a creature within \rngmed range.
                Creatures who have strict codes prohibiting them from taking evil actions, such as paladins devoted to Good, are immune to this ability.
                \hit The target takes an evil action as soon as it can.
                You have no control over the act the creature takes, but circumstances can make the target more likely to take an action you desire.
            \end{freeability}
            \parhead{Mastery} You can use your domain gift to redirect your damage to an adjacent unwilling creature.
            You cannot use this ability to redirect damage to the same unwilling creature more than once between \glossterm{short rests}.

        \subsubsection{Fire}
            If you choose this domain, you add the \sphere{pyromancy} \glossterm{mystic sphere} to your list of divine mystic spheres (see \pcref{Mystic Spheres}).

            \parhead{Gift} You gain a \glossterm{magic bonus} to equal to twice your \glossterm{power} to \glossterm{resistances} against fire damage.
            \parhead{Aspect} Whenever you deal fire damage to a creature with a \glossterm{critical hit}, that creature is \glossterm{ignited} as a \glossterm{condition}.
            This condition can be removed if the target makes a \glossterm{DR} 10 Dexterity check as a \glossterm{move action} to put out the flames.
            Dropping \glossterm{prone} as part of this action gives a \plus5 bonus to this check.
            \parhead{Essence} You can use the \textit{speak with fire} ability as a standard action.
            \begin{attuneability}{Speak with Fire}[\glossterm{Attune} (self)]
                You can speak with and command fire within a \areahuge radius zone from your location.
                You can ask the fire simple questions and understand its responses.
                If you command the fire to perform a task, it will do so do the best of its ability until this effect ends.
                You cannot compel the fire to move farther than 30 feet in a single round.
                Fire that ends the round on non-combustable materials usually goes out, depending on the circumstances.

                After you use this ability on a particular area of fire, you cannot use it again on that same area for 24 hours.
                % TODO: What does an ``area of fire'' mean?
            \end{attuneability}
            \parhead{Mastery} The \glossterm{ignited} condition from this domain's gift cannot be removed by putting out the flames as a move action.
            In addition, it deals fire \glossterm{standard damage} -1d instead of the normal 1d6 damage per round.

        \subsubsection{Good}
            \parhead{Gift} At the start of each phase, you may choose an adjacent \glossterm{ally}.
            If you do, you take half of all damage that creature would take that phase (rounded down) instead of the creature.
            Any abilities you have that would reduce or negate the effects of the attack have no effect.
            The protected creature takes the remaining half of the damage, and suffers any non-damaging effects of attacks normally.
            \parhead{Aspect} You can use this domain's domain gift to target any \glossterm{ally} within \rngclose range.
            \parhead{Essence} You can use the \textit{compel good} ability as a standard action.
            \begin{freeability}{Compel Good}[\glossterm{Compulsion}]
                Make an attack vs. Mental against a creature within \rngmed range.
                Creatures who have strict codes prohibiting them from taking evil actions, such as paladins devoted to Good, are immune to this ability.
                \hit The target takes an good action as soon as it can.
                You have no control over the act the creature takes, but circumstances can make the target more likely to take an action you desire.
            \end{freeability}
            \parhead{Mastery} When you use your Good domain gift, you can redirect all damage to you instead of only half.

        \subsubsection{Knowledge}
            If you choose this domain, you add all Knowledge skills to your cleric \glossterm{class skill} list.

            \parhead{Gift} You gain two skill points.
            \parhead{Aspect} Your extensive knowledge of all methods of attack and defense grants you a \plus1 bonus to all defenses.
            \parhead{Essence} You can use the \textit{share knowledge} ability as a standard action.
            \begin{freeability}{Share Knowledge}
                You make a Knowledge check of any kind with a bonus equal to your \glossterm{power}.
                You and your \glossterm{allies} within a \arealarge radius learn the results of your check.
                Creatures believe the information gained in this way to be true as if they it had seen it with their own eyes.

                You cannot alter the knowledge you gain with this check in any way, such as by adding or withholding information.
            \end{freeability}
            \parhead{Mastery} You gain a \plus1 bonus to \glossterm{accuracy} with all attacks.

        \subsubsection{Law}
            \parhead{Gift} You gain a \plus2 bonus to Mental defense.
            % Clarify - does this apply to exploding dice?
            \parhead{Aspect} When you roll a 1 on an \glossterm{attack roll}, it is treated as if you had rolled a 6.
            \parhead{Essence} You can use the \textit{compel law} ability as a standard action.
            \begin{freeability}{Compel Law}[\glossterm{Compulsion}]
                Make an attack vs. Mental against all creatures within a \arealarge radius.
                \hit Each target is unable to break the laws that apply in the area, and any attempt to do so simply fails.
                The laws which are applied are those which are most appropriate for the area, regardless of whether you or any other creature know those laws.

                % Sufficiently clear that this isn't part of the hit effect?
                When you use this ability, you also gain the condition.
                If this condition is removed from you, it is also removed from all other affected creatures.
                In areas under ambiguous or nonexistent government, this ability may have unexpected effects, or it may have no effect at all.
            \end{freeability}
            \parhead{Mastery} When you roll less than a 5 on an \glossterm{attack roll}, it is treated as if you had rolled a 5.

        \subsubsection{Life}
            \parhead{Gift} You gain a \plus4 bonus to the Heal skill (see \pcref{Heal}).
            \parhead{Aspect} You gain an additional \glossterm{hit point}.
            \parhead{Essence} At the end of each round, if you became \glossterm{unconscious} from a \glossterm{vital wound} that round, you can use one \glossterm{magical} ability you have that modifies \glossterm{wound rolls} or removes \glossterm{vital wounds} on yourself without taking an action.
            \parhead{Mastery} You gain an additional \glossterm{hit point}.
            In addition, you gain a \plus1 bonus to \glossterm{wound rolls}.

        \subsubsection{Magic}
            \parhead{Gift} You gain a \plus1 bonus to \glossterm{power} with divine spells.
            \parhead{Aspect} You learn an additional \glossterm{spell} for a divine \glossterm{mystic sphere} you have access to.
            \parhead{Essence} You learn an \glossterm{augment} (see \pcref{Augments}).
            You can apply augments to spells to change their effects.
            When you gain access to a new arcane spell level, you may change which augments you know.
            \parhead{Mastery} You learn an additional \glossterm{augment} (see \pcref{Augments}).
            In addition, the power bonus from this domain's gift increases to \plus2.

        \subsubsection{Strength}
            If you choose this domain, you add the Climb, Jump, and Swim skills to your cleric \glossterm{class skill} list.

            \parhead{Gift} You gain two skill points.
            \parhead{Aspect} You gain a \plus4 bonus to Strength for the purpose of checks and determining your carrying capacity.
            \parhead{Essence} You can use the \textit{divine strength} ability as a \glossterm{minor action}.
            \begin{attuneability}{Divine Strength}[\glossterm{Attune} (self)]
                You gain a \plus4 \glossterm{magic bonus} to Strength.
            \end{attuneability}
            \parhead{Mastery} You gain a \plus1 bonus to your starting Strength.

        \subsubsection{Travel}
            If you choose this domain, you add the \sphere{astromancy} \glossterm{mystic sphere} to your list of divine mystic spheres (see \pcref{Mystic Spheres}).
            In addition, you add the Knowledge (geography), Survival, and Swim skills to your cleric \glossterm{class skill} list.

            \parhead{Gift} You gain two skill points.
            \parhead{Aspect} You gain a \plus20 foot \glossterm{magic bonus} to your \glossterm{base speed}, up to a maximum of double your normal speed.
            \parhead{Essence} You can use the \textit{dimensional travel} ability as a standard action.
            \begin{freeability}{Dimensional Travel}[\glossterm{Teleportation}]
                You teleport up to 1 mile in any direction.
                You do not need \glossterm{line of sight} or \glossterm{line of effect} to your destination, but you must be able to clearly visualize it.
            \end{freeability}
            \parhead{Mastery} When you move, you can teleport the same distance instead.
            This does not change the total distance you can move, but you can teleport in any direction, including vertically.

            You can even attempt to move to locations outside of \glossterm{line of sight} and \glossterm{line of effect}, up to the limit of your remaining movement speed.
            If your intended destination is invalid, the distance you tried to teleport is taken from your remaining movement, but you suffer no other ill effects.

        \subsubsection{Trickery}
            If you choose this domain, you add the \sphere{glamer} \glossterm{mystic sphere} to your list of divine mystic spheres (see \pcref{Mystic Spheres}).
            In addition, you add the Bluff, Disguise, and Stealth skills to your cleric \glossterm{class skill} list.

            \parhead{Gift} You gain two skill points.
            \parhead{Aspect} You gain a \plus2 \glossterm{magic bonus} to the Bluff, Disguise, and Stealth skills.
            \parhead{Essence} You can use the \textit{compel belief} ability as a standard action.
            \begin{freeability}{Compel Belief}[\glossterm{Compulsion}, \glossterm{Sustain} (minor)]
                Make an attack vs. Mental against a creature within \rngmed range.
                You must also choose a belief that the target has.
                The belief may be a lie that you told it, or even a simple misunderstanding (such as believing a hidden creature is not present in a room).
                If the creature does not already hold the chosen belief, this ability automatically fails.
                \hit The target continues to maintain the chosen belief, regardless of any evidence to the contrary.
                It will interpret any evidence that the falsehood is incorrect to be somehow wrong -- an illusion, a conspiracy to decieve it, or any other reason it can think of to continue believing the falsehood.
                At the end of the effect, the creature can decide whether it believes the falsehood or not, as normal.
            \end{freeability}
            \parhead{Mastery} You are undetectable by Divination spells and effects.
            They cannot detect your presence, sounds you make, or any actions you take.

        \subsubsection{War}
            \parhead{Gift} You gain proficiency with heavy armor and an additional weapon group of your choice.
            \parhead{Aspect} You gain a \plus1 bonus to \glossterm{accuracy} with \glossterm{physical attacks}.
            \parhead{Essence} You learn one of the Legion, Selective, or Widened \glossterm{augments} (see \pcref{Augments}).
            You can apply augments to spells to change their effects.
            When you gain access to a new arcane spell level, you may change which augments you know, though this augment can only be one of the three listed.
            \parhead{Mastery} You gain a \plus2 bonus to \glossterm{power} with \glossterm{physical attacks}.

        \subsubsection{Water}
            If you choose this domain, you add the \sphere{aquamancy} \glossterm{mystic sphere} to your list of divine mystic spheres (see \pcref{Mystic Spheres}).
            In addition, you add the Swim skill to your cleric \glossterm{class skill} list.

            \parhead{Gift} You gain a \plus4 \glossterm{magic bonus} to the Swim skill.
            In addition, you gain a \plus1 bonus to Reflex defense.
            \parhead{Aspect} You can breathe water as easily as a human breathes air, preventing you from drowning or suffocating underwater.
            \parhead{Essence} You can use the \textit{speak with water} ability as a standard action.
            \begin{attuneability}{Speak with Water}[\glossterm{Attune} (self)]
                You can speak with and command water within a \areahuge zone from your location.
                You can ask the water simple questions and understand its responses.
                If you command the water to perform a task, it will do so do the best of its ability until this effect ends.
                You cannot compel the water to move faster than 30 feet per round.

                After you use this ability on a particular area of water, you cannot use it again on that same area for 24 hours.
            \end{attuneability}
            \parhead{Mastery}
            When you move, you can transform yourself into a rushing flow of water with a volume roughly equal to your normal volume until your movement is complete.
            In this form, you may move wherever water could go, you cannot take other actions, such as jumping, attacking, or casting spells.
            You may move through squares occupied by enemies without penalty.
            \par Your speed is halved when moving uphill and doubled when moving downhill.
            Unusually steep inclines may cause greater movement differences while in this form.
            \par If the water is split, you may reform from anywhere the water has reached, to as little as a single ounce of water.
            If not even an ounce of water exists contiguously, your body reforms from all of the largest available sections of water, cut into pieces of appropriate size.
            This usually causes you to die.

        \subsubsection{Wild}
            If you choose this domain, you add the \sphere{verdamancy} \glossterm{mystic sphere} to your list of divine mystic spheres (see \pcref{Mystic Spheres}).
            In addition, you add the Creature Handling, Knowledge (nature), and Survival skills to your cleric \glossterm{class skill} list.

            \parhead{Gift} You gain two skill points.
            % TODO: clarify whether you can have this and the druid ability
            \parhead{Aspect} This ability functions like the \textit{wild aspect} druid ability from the Shifter archetype (see \pcref{Shifter}), except that you cannot spend \glossterm{insight points} to learn additional wild aspects.
            \parhead{Essence} You learn an additional \textit{wild aspect}.
            In addition, you gain the ability to maintain two \textit{wild aspects} simultaneously.
            \parhead{Mastery} You learn an additional \textit{wild aspect}.
            In addition, you gain the ability to maintain three \textit{wild aspects} simultaneously.
            % TODO: essence, mastery

        \subsubsection{Ex-Clerics}
            If you grossly violate the code of conduct required by your deity, you lose all spells and magical cleric class abilities.
            You cannot regain those abilities until you atone for your transgressions to your deity.

\newpage
\section{Druid}\label{Druid}
    \begin{dtable!*}
        \lcaption{Druid Progression}
        \begin{dtabularx}{\textwidth}{c >{\lcol}X >{\lcol}X >{\lcol}X >{\lcol}X}
            \tb{Rank} & \tb{Nature Spellcasting} & \tb{Nature Spell Mastery} & \tb{Shifter} & \tb{Beastmaster} \tableheaderrule
            1    & Spellcasting  & Natural spells              & Wild aspects  & Animal speech
            \\ 2 & Mystic sphere   & Mystic knowledge            & Shifter lore               & Animal companion
            \\ 3 & Spell level (2) & Wellspring of power         & Shifting defense           & Beast lore
            \\ 4 & Spell level (3) & Mystic knowledge            & Dual aspect, wild aspect   & Greater wild speech
            \\ 5 & Spell level (4) & Greater natural spells      & Nature's champion          & Beast companion
            \\ 6 & Spell level (5) & Mystic knowledge            & Fluid aspect               & Beast tamer
            \\ 7 & Spell level (6) & Greater wellspring of power & Triple aspect, wild aspect & Supreme wild speech
            \\ 8 & Spell level (7) &                             & Avatar of nature           & True companion
        \end{dtabularx}
    \end{dtable!*}

    \classbasics{Alignment} Neutral good, lawful neutral, neutral, chaotic neutral, or neutral evil.

    \classbasics{Archetypes} Druids have the Beastmaster, Nature Spellcasting, Nature Spell Mastery, and Shifter \glossterm{archetypes}.

    \subsection{Basic Class Abilities}
        If you are a druid, you gain the following abilities.

        \cf{Drd}{Defenses}
        You gain the following bonuses to your \glossterm{defenses}: \plus5 Fortitude, \plus3 Reflex, \plus4 Mental.

        \cf{Drd}{Skills}
        You have the following \glossterm{class skills}:
        \begin{itemize}
            \item \subparhead{Strength} Climb, Jump, Swim.
            \item \subparhead{Dexterity} Acrobatics, Ride, Stealth.
            \item \subparhead{Intelligence} Craft, Deduction, Heal, Knowledge (geography, nature).
            \item \subparhead{Perception} Awareness, Creature Handling, Survival.
            \item \subparhead{Other} Bluff, Intimidate, Persuasion, Profession.
        \end{itemize}

        \cf{Drd}{Weapon and Armor Proficiencies}
        Druids are proficient with simple weapons, any one other weapon group, scimitars, sickles, and slings.
        In addition, druids are proficient with light armor, medium armor, and shields.
        However, a druid cannot use metal armor; see the Metal Abhorrence ability, below.

        \cf{Drd}{Druidic Language}
        You know Druidic, a secret language known only to druids, in addition to your normal languages.
        Druids are forbidden to teach this language to nondruids.
        Druidic has its own alphabet.

        \cf{Drd}{Metal Abhorrence}
        The oaths that you swear as part of your druidic initiation prohibit you from wearing armor made of metal.
        If you wear prohibited armor or carry a prohibited shield, you are unable to cast druid spells or use any of your \glossterm{magical} druid abilities while doing so and for 24 hours thereafter.

        You can avoid this penalty by using armor made of wood altered with the \ritual{ironwood} ritual.
        Such wood is as strong as steel.

    \subsection{Shifter}\label{Shifter}
        This archetype grants you the ability to embody aspects of the natural world in your own form.

        \cf{Drd}[1]{Wild Aspects}[Magical]
        You gain the ability to embody an aspect of an animal or of nature itself.
        Choose a single wild aspect from the list below.
        You can also spend \glossterm{insight points} to learn one additional \textit{wild aspect} per \glossterm{insight point}.

        As a \glossterm{standard action}, you can gain the effects of one wild aspect that you know.
        That effect lasts until you activate a different wild aspect you know.

        The abilities in the list below describe the effects of the aspect.
        Your appearance also changes to match the aspect's effects, but the nature of this change is not described.
        Different druids change in different ways.
        For example, one druid might grow brown fur when using the Form of the Bear, while another might instead change their face to become broader and more bear-shaped when embodying the same aspect.
        You choose how your appearance changes when you gain a wild aspect.
        This change cannot be used to gain an additional substantive benefit beyond the effects given in the description of the aspect.

        Many wild aspects grant natural weapons.
        See \pcref{Natural Weapons}, for details about natural weapons.

        {
            \begin{freeability}{Form of the Bear}
                You gain a \plus1 bonus to Fortitude defense.
                In addition, you mouth and hands transform, granting you bite and claw \glossterm{natural weapons}.
                The bite deals \plus0d damage, and the claws deal \minus1d damage.

                \rankline
                \rank{3} The Fortitude bonus increases to \plus2.
                \rank{5} You gain a \plus1d bonus to damage with the natural weapons.
                \rank{7} The Fortitude bonus increases to \plus4.
            \end{freeability}

            \begin{freeability}{Form of the Bull}
                You gain a \plus1 bonus to \glossterm{accuracy} with the \textit{shove} ability (see \pcref{Shove}).
                In addition, your head transforms, granting you a gore \glossterm{natural weapon}.
                The weapon deals \plus0d damage, and has the Forceful weapon tag (see \pcref{Weapon Tags}).

                \rankline
                \rank{3} The accuracy bonus increases to \plus2.
                \rank{5} You gain a \plus1d bonus to damage with the natural weapon.
                \rank{7} The accuracy bonus increases to \plus4.
            \end{freeability}

            \begin{freeability}{Form of the Constrictor}
                You gain a \plus1 bonus to \glossterm{accuracy} with the Grapple ability (see \pcref{Grapple}).
                In addition, you gain a constrict \glossterm{natural weapon}.
                This weapon deals \plus1d damage, and it has the Grappling weapon tag (see \pcref{Weapon Tags}).
                It can only be used against a foe you are grappling with.

                \rankline
                \rank{3} The accuracy bonus increases to \plus2.
                \rank{5} You gain a \plus1d bonus to damage with the natural weapon.
                \rank{7} The accuracy bonus increases to \plus4.
            \end{freeability}

            \begin{freeability}{Form of the Fish}
                You gain a \glossterm{swim speed} equal to your \glossterm{base speed}.
                In addition, you gain a bite \glossterm{natural weapon} that deals \plus0d damage.

                \rankline
                \rank{3} You can breathe water as easily as a human breathes air, preventing you from drowning or suffocating underwater.
                \rank{5} You suffer no penalties for acting underwater.
                \rank{7} You are immune to \glossterm{magical} effects that restrict your mobility.
                In addition, you gain a \plus5 bonus to defenses against the \textit{grapple} ability (see \pcref{Grapple}).
            \end{freeability}

            \begin{freeability}{Form of the Hawk}
                You gain \glossterm{low-light vision}.
                If you already have low-light vision, you double its benefit, allowing you to treat sources of light as if they had four times their normal illumination range.
                In addition, your feet transform, granting you a talon \glossterm{natural weapon}.
                The weapon deals \minus1d damage.

                \rankline
                \rank{3} You grow wings, granting your a glide speed equal to your \glossterm{base speed} (see \pcref{Gliding}).
                \rank{5} You gain a \plus1d bonus to damage with the natural weapon.
                \rank{7} The glide speed is replaced with a \glossterm{fly speed} equal to your \glossterm{base speed} (see \pcref{Flying}).
            \end{freeability}

            \begin{freeability}{Form of the Hound}
                You gain the \glossterm{scent} ability.
                In addition, you gain a bite \glossterm{natural weapon} that deals \plus0d damage.

                \rankline
                \rank{3} You gain the ability to move on all four limbs.
                When doing so, you gain a \plus20 foot bonus to your land speed, up to a maximum of double your original speed.
                When not using your hands to move, your ability to use your hands is unchanged.
                Descending to four legs and rising up to stand on two legs again does not take an action.
                \rank{5} You gain a \plus1d bonus to damage with the natural weapon.
                \rank{7} You gain an additional \plus10 bonus to scent-based Awareness checks (see \pcref{Awareness}).
            \end{freeability}

            % Seems boring? What abilities would make sense?
            \begin{freeability}{Form of the Monkey}
                You gain a \glossterm{climb speed} equal to your \glossterm{base speed}.
                In addition, you gain a bite \glossterm{natural weapon} that deals \plus0d damage.

                \rankline
                \rank{3} You gain a \plus1 bonus to your Reflex defense.
                \rank{5} You gain a \plus1d bonus to damage with the natural weapon.
                \rank{7} The defense bonus increases to \plus2.
            \end{freeability}

            \begin{freeability}{Form of the Mouse}
                You gain a \plus2 bonus to the Escape Artist and Stealth skills.
                In addition, you gain a bite \glossterm{natural weapon} that deals \plus0d damage.
                
                \rankline
                \rank{3} When you use this wild aspect, you can choose to shrink by one \glossterm{size category}.
                \rank{5} The skill bonuses increases to \plus3.
                \rank{7} The skill bonuses increases to \plus4. In addition, when you use this wild aspect, you can choose to shrink by up to two \glossterm{size categories} instead of only one.
            \end{freeability}

            \begin{freeability}{Form of the Oak}
                You move at half speed.
                In exchange, you gain a \plus1 bonus to Armor defense and Fortitude defense.
                \rankline
                \rank{3} You gain a \glossterm{magic bonus} equal to half your \glossterm{power} to \glossterm{resistances} against \glossterm{physical damage}.
                \rank{5} The resistance bonus increases to be equal to your \glossterm{power}.
                \rank{7} The defense bonuses increase to \plus2.
            \end{freeability}

            \begin{freeability}{Form of the Viper}
                You gain a \glossterm{climb speed} equal to half your \glossterm{base speed}.
                You do not need to use your hands to climb in this way.
                In addition, you gain a bite \glossterm{natural weapon} that deals \plus0d damage.

                \rankline
                \rank{3} When a creatures takes damage from your bite \glossterm{natural weapon}, it is poisoned.
                At the end of each \glossterm{action phase} in subsequent rounds, you make an attack vs. Fortitude against the target.
                If you hit, the target is \glossterm{sickened} until it removes the poison.
                The poison is removed if you miss the target on this attack three times.
                \rank{5} You gain a \plus1d bonus to damage with the natural weapon.
                \rank{7} The poison makes the target \glossterm{nauseated} instead of \glossterm{sickened}.
            \end{freeability}

            \begin{freeability}{Form of the Wolf}
                You gain a \plus1 bonus to \glossterm{accuracy} with \glossterm{physical attacks} against \glossterm{overwhelmed} creatures.
                In addition, you gain a bite \glossterm{natural weapon} that deals \plus0d damage.

                \rankline
                \rank{3} You gain the ability to move on all four limbs.
                When doing so, you gain a \plus20 foot bonus to your land speed, up to a maximum of double your original speed.
                When not using your hands to move, your ability to use your hands is unchanged.
                Descending to four legs and rising up to stand on two legs again does not take an action.
                \rank{5} You gain a \plus1d bonus to damage with the natural weapon.
                \rank{7} You increase your \glossterm{overwhelm value} by 1.
            \end{freeability}

            \begin{freeability}{Myriad Form}
                You can use your \glossterm{power} in place of your Disguise skill when making Disguise checks to alter your own appearance.
                \rankline
                \rank{3} When you use this wild aspect, you can choose to grow or shrink by one \glossterm{size category}.
                \rank{5} You can use the \textit{Disguise Creature} ability to disguise yourself as a \glossterm{standard action} (see \pcref{Disguise Creature}).
                \rank{7} When you use this wild aspect, you can choose to grow or shrink by up to two \glossterm{size categories} instead of only one.
            \end{freeability}

            \begin{freeability}{Photosynthesis}
                As long as you are in natural sunlight, you gain a \plus10 foot bonus to your \glossterm{base speed}.
                \rankline
                \rank{3} As long as you are in natural sunlight, you do not gain hunger or thirst.
                When you leave natural sunlight, you continue gaining hunger or thirst at your normal rate, ignoring any time you spent in natural sunlight.
                \rank{5} The speed bonus increases to \plus20 feet.
                \rank{7} When you take a \glossterm{short rest} while you are in natural sunlight, you remove a \glossterm{vital wound}.
            \end{freeability}

            \begin{freeability}{Plantspeaker}
                Your speed is not reduced when moving in light or heavy \glossterm{undergrowth}.
                In addition, you can ignore \glossterm{cover} (but not \glossterm{total cover}) from plants when attacking, as the plants move out of the way to help you.

                \rankline
                \rank{3} You can ignore \glossterm{cover} (but not \glossterm{total cover}) and \glossterm{concealment} from plants whenever doing so would be beneficial to you.
                For example, creatures cannot use concealment from plants to hide from you.
                \rank{5} The movement penalties from \glossterm{undergrowth} are doubled for enemies within a \areahuge radius emanation from you.
                \rank{7} You can ignore \glossterm{total cover} from plants whenever doing so would be beneficial to you.
            \end{freeability}
        }

        \cf{Drd}[2]{Shifter Lore}
        You gain two additional skill points.

        \cf{Drd}[3]{Shifting Defense} You gain a \plus1 bonus to Armor defense.

        \cf{Drd}[4]{Dual Aspect} 
        %TODO: wording
        You gain the ability to maintain the effects of up to two \textit{wild aspects} simultaneously.
        When you activate a new \textit{wild aspect}, you can choose which of your wild aspects remain active. 

        \cf{Drd}[4]{Wild Aspect}[Magical]
        You learn an additional \textit{wild aspect}.

        \cf{Drd}[5]{Nature's Champion}
        You gain a \plus2 bonus to the Creature Handling, Heal, Knowledge (geography), Knowledge (nature), Ride, and Survival skills.

        \cf{Drd}[6]{Fluid Aspect}
        % TODO: wording; this just replaces the action time, not gives you an extra one active
        You can activate a \textit{wild aspect} as a \glossterm{minor action} instead of as a \glossterm{standard action}.

        \cf*{Drd}[7]{Wild Aspect}[Magical]
        You learn an additional \textit{wild aspect}.

        \cf{Drd}[7]{Triple Aspect} 
        You gain the ability to maintain the effects of up to three \textit{wild aspects} simultaneously.
        When you activate a new \textit{wild aspect}, you can choose which of your wild aspects remain active. 

        \cf{Drd}[8]{Avatar of Nature}[Magical]
        If you die, except if by old age, you may choose to have your body and soul become an instrument of nature's will.
        Your body immediately decomposes or otherwise disappears, and your soul does not travel to an afterlife.
        You has no physical form, and cannot use any of your normal abilities.
        Instead, you have a \glossterm{fly speed} of 100 feet, with special maneuverability.
        As a standard action, you can temporarily possess any living plants or animals within a 10 mile radius of the place of your death.

        While possessing a living plant or animal, you can see through its senses and control its actions completely.
        In addition, you may cast spells, and the spells take effect as if the plant or animal had cast them.
        You use the plant or animal's position to determine range, visible targets, and so on.
        You do not require \glossterm{verbal components} to cast your spells in this form.
        % TODO: are there any spells or rituals that have material components or focus objects?
        % but are unable to cast spells or perform rituals that require material components or focus objects.

        While not possessing a plant or animal, you can rest, or you can focus on reincarnating your physical form.
        Creating a new body in this way takes 12 consecutive hours of concentration.
        At the end of that time, you are reincarnated in a new body in your location, as the effect of the \ritual{reincarnation} ritual, except that you can choose your species from among the species listed (not including the ``Other'' species).

        While you are an avatar of nature, you do not age and you cannot die of old age.
        You can continue to exist in this form indefinitely.

    \subsection{Nature Spellcasting}
        This archetype grants you the ability to cast nature spells.
        You must have a starting Perception of at least 1 to gain this archetype.

        \cf{Drd}[1]{Spellcasting}[Magical]
        Your deity grants you the ability to use nature magic.
        You gain access to one nature \glossterm{mystic sphere} (see \pcref{Nature Mystic Spheres}).
        Each \glossterm{mystic sphere} has a set of \glossterm{spells} associated with it.

        You automatically learn all \glossterm{cantrips} from any mystic sphere you have access to.
        In addition, you learn one 1st level \glossterm{spell} from your chosen \glossterm{mystic sphere}.
        You can also spend \glossterm{insight points} to learn one additional nature spell per \glossterm{insight point}.
        Unless otherwise noted in a spell's description, casting a spell requires a \glossterm{standard action}.

        When you gain access to a new \glossterm{mystic sphere} or spell level,
            you can exchange any number of spells you know for other spells,
            including spells of the higher level.

        Nature spells require \glossterm{verbal components} to cast (see \pcref{Casting Components}).
        For details about mystic spheres and casting spells, see \pcref{Spell and Ritual Mechanics}.

        \cf{Drd}[2]{Mystic Sphere} You gain access to an additional nature \glossterm{mystic sphere}, including all \glossterm{cantrips} from that sphere.
        In addition, you learn an additional nature \glossterm{spell}.

        \cf{Drd}[3]{Spell Level}[Magical] You gain the ability to cast 2nd level nature spells.

        \cf*{Drd}[4]{Spell Level}[Magical] You gain the ability to cast 3rd level nature spells.

        \cf*{Drd}[5]{Spell Level}[Magical] You gain the ability to cast 4th level nature spells.

        \cf*{Drd}[6]{Spell Level}[Magical] You gain the ability to cast 5th level nature spells.

        \cf*{Drd}[7]{Spell Level}[Magical] You gain the ability to cast 6th level nature spells.

        \cf*{Drd}[8]{Spell Level}[Magical] You gain the ability to cast 7th level nature spells.

    \subsection{Nature Spell Mastery}
        This archetype improves the nature spells you cast.
        You must be able to cast nature spells to gain the abilities from this archetype.

        \cf{Drd}[1]{Natural Spells} You gain a \plus1 bonus to \glossterm{accuracy} with spells against animals and plants.

        \cf{Drd}[2]{Mystic Knowledge}
        You gain your choice of one of the following abilities.
        {
            \parhead{Insight} You learn an additional nature \glossterm{spell} (see \pcref{Nature Mystic Spheres}).
                You can choose this ability multiple times, gaining access to an additional spell each time.
            \parhead{Rituals} You gain the ability to perform nature rituals to create unique magical effects (see \pcref{Rituals}).
                The maximum level of nature ritual you can learn or perform is equal to the maximum level of divine spell that you can cast.
                You cannot choose this ability multiple times.
            \parhead{Mystic Sphere Access} You gain access to an additional nature \glossterm{mystic sphere}.
                You cannot choose this ability multiple times.
        }

        \cf{Drd}[3]{Wellspring of Power}[Magical]
        You gain a \plus1 bonus to \glossterm{power} with nature spells.

        \cf*{Drd}[4]{Mystic Knowledge}[Magical]
        You gain an additional \textit{mystic knowledge} ability.

        \cf{Drd}[5]{Greater Natural Spells} The bonus from your \textit{natural spells} ability increases to \plus2.

        \cf*{Drd}[6]{Mystic Knowledge}[Magical]
        You gain an additional \textit{mystic knowledge} ability.

        \cf{Drd}[7]{Greater Wellspring of Power}[Magical]
        The bonus from your \textit{wellspring of power} ability increases to \plus2.

    \subsection{Beastmaster}
        This archetype improves your connection to animals, allowing you to control and command them in battle.

        \cf{Drd}[1]{Wild Speech}[Magical] You can use the \textit{wild speech} ability as a standard action.
        \begin{attuneability}{Wild Speech}[\glossterm{Attune} (self)]
            Choose an animal within \rnglong range.
            You can speak to and understand the speech of the target animal, and any other animals of the same species.

            This ability does not make the target any more friendly or cooperative than normal.
            Wary and cunning animals are likely to be terse and evasive, while stupid ones tend to make inane comments and are unlikely to say or understand anything of use.
        \end{attuneability}

        \cf{Drd}[2]{Animal Companion}
        You can use the \textit{animal companion} ability.
        This ability requires 8 hours of training and attunement which the target must actively parcipate in.
        You can compel a wild animal to undergo this training by sustaining the \textit{command} ability from the Creature Handling skill (see \pcref{Command}).
        \begin{attuneability}{Animal Companion}[\glossterm{Attune} (self), \glossterm{Emotion}, \glossterm{Magical}]
            Choose an animal \glossterm{ally} within your \glossterm{reach} with a level no higher than your level and a \glossterm{challenge rating} no higher than 1.
            The target serves as a loyal companion to you.
            It follows your directions to the best of its ability.
            Animals are unable to understand complex concepts, so their ability to obey convoluted instructions is limited.

            Your magical connection to the animal improves its resilience and strength in combat.
            If any of its statistics are higher than the normal values below, the animal uses its own statistics instead.
            Its \glossterm{damage resistance} is equal to your \glossterm{power}.
            Its \glossterm{wound resistance} is equal to three times your \glossterm{power}.
            Each of its defenses is normally equal to 4 \add your level.
            Its \glossterm{accuracy} is normally equal to the higher of your level and Perception.
            Its \glossterm{power} with its attacks is normally equal to your \glossterm{power}.
            It can heal \glossterm{vital damage} using your level in place of its level.
            All other aspects of the animal, such as its speed and natural weapons, are unchanged.
        \end{attuneability}

        \cf{Drd}[3]{Beast Lore} You gain two additional \glossterm{skill points}.

        \cf{Drd}[4]{Greater Wild Speech} When you use your \textit{wild speech} ability, you do not have to target an animal.
        Instead, you understand the speech of animals of any species.

        \cf{Drd}[5]{Beast Companion} When you use your \textit{animal companion} ability,
        you can target a \glossterm{magical beast} \glossterm{ally} with an Intelligence of \minus6 or lower instead of an animal.

        \cf{Drd}[6]{Beast Tamer} You gain a \plus4 bonus to the Creature Handling skill (see \pcref{Creature Handling}).

        \cf{Drd}[7]{Supreme Wild Speech} When you use your \textit{wild speech} ability, you also understand all spoken languages.

        \cf{Drd}[8]{True Companion} When you use your \textit{animal companion} ability,
        you can target a creature of any type with an Intelligence of \minus6 or lower.
        In addition, the maximum \glossterm{challenge rating} of your \textit{animal companion} increases to 2.

        % \subcf{15th -- Air Mantle}
        % You are surrounded by a mantle of air.
        % Thrown and projectile weapons have a 50\% chance to miss your while this effect is active.
        % Unusually large weapons, such as a giant's boulders, may suffer a decreased miss chance as appropriate to their size.
        % \subcf{15th -- Aqueous Step}
        % Wherever the druid moves, you leave a path of animated water that can grab creatures.
        % When a creature crosses the path, the druid makes a Reflex attack to trip the creature, causing it to fall prone and waste the rest of its movement.
        % Your accuracy is equal to your druid level \add your Constitution.
        % \subcf{15th -- Flaming Step}
        % Wherever the druid moves, you leave a path of burning flame behind your that lasts for 1 round.
        % When a creature crosses the path, the druid makes a Reflex attack to deal damage to the creature.
        % The attack deals 1d8 points of fire damage per two druid levels.
        % Your accuracy is equal to your druid level \add your Constitution.
        % A failed attack deals half damage.
        % \subcf{15th -- Lifegiving Step}
        % Wherever the druid moves, you leave a path of small, living plants that entangle foes for 1 round.
        % When a creature crosses the path, the druid makes a Reflex attack to entangle the creature, causing it to waste the rest of its movement.
        % Your accuracy is equal to your druid level \add your Constitution.
        % The plants appear on any surface, and will continue to grow if they can survive, though they may die quickly if they appear on inhospitable terrain.
        % \subcf{17th -- Flaming Soul}
        % You gain the fire subtype, making your immune to fire but giving your a 50\% vulnerability to cold damage.
        % In addition, when you deal fire damage to a creature, the creature is \ignited for 5 rounds.
        % \subcf{17th -- Sunblessed Rejuvenation}
        % You gain fast healing equal to your druid level as long as you remain in sunlight or touches a plant of your size or larger.
        % \subcf{17th -- Sunscour}
        % This aspect functions like the heart of the sun natural aspect, except that it also suppresses shadow effects and the visual components of illusions within the area of bright light.

        % \subcf{17th -- Water's Flow}
        % As a minor action, the druid can transform herself into a rushing flow of water with a volume roughly equal to your normal volume until the end of your turn.
        % In this form, she may move wherever water could go, but she cannot take other actions, such as jumping, attacking, or casting spells.
        % Your speed is halved when moving uphill and doubled when moving downhill.
        % She may move through squares occupied by enemies without penalty.
        % She may return to your normal form as a free action.
        % \par If the water is split, she may reform from anywhere the water has reached, to as little as a single ounce of water.
        % If not even an ounce of water exists contiguously, your body reforms from the largest available parts of water, cut into pieces of appropriate size.
        % This usually causes the druid to die.

    \subsection{Ex-Druids}
        A druid who ceases to revere nature, changes to a prohibited alignment, or teaches the Druidic language to a nondruid loses all spells and magical druid class abilities.
        She cannot thereafter gain levels as a druid until you atone for your transgressions.

        % \subsection{Variant Druids}

        %     \subsubsection{Blighter}

        %         Blighters draw power from nature, as do other druids. However, while other druids revere nature and draw power from it gently, blighters steal power from nature forcefully. Wherever a blighter goes, destruction and death surely follows.

        %         \altcf{Blight} Instead of meditating to regain spell slots, a blighter draws power from your environment forcefully.
        %         This affects a \areahuge radius zone centered on your, and the process takes 1 minute of concentration.
        %         At the end of every round, every living thing in the area other than the blighter takes damage equal to your nature power.
        %         All inanimate plants of Huge size or smaller immediately wither and die.
        %         The earth becomes cracked and infertile, and any nutrients from the soil are destroyed.
        %         This ability has no effect on artificial environments or materials, such as metal or worked stone.
        %         At the end of the minute, the blighter regains your spent nature spell slots.

        %         A blighter can only blight your surroundings in this way once per hour.
        %         If your surroundings are already blighted or are not natural terrain, she cannot use this ability to regain your spells.
        %         Instead, she must meditate for 8 hours to slowly draw power from your surroundings, as a normal druid.

        %         \altcf{Spells} As normal, except that a blighter adds all Vivimancy arcane spells to your spell list.

        %         \altcf[2]{Wild Speech} As normal, except that a blighter gains a \plus5 bonus to Intimidate against your wild speech targets, and a \minus5 penalty to Persuasion.

        %         \altcf[10]{Blightcasting}

        %         \altcf[20]{Improved Blightcasting}

        % \subsubsection{Rotbringer}

        %     While most druids seek to emulate and interact with animals, rotbringers focus on the power of fungi, decay, and regeneration.

        %     \altcf{Invoke Rot} Instead of meditating to regain spell slots, a rotbringer accelerates the natural forces of decomposition and decay on your environment.
        %     This affects a \areahuge radius zone centered on your, and the process takes 1 minute of concentration.
        %     All organic objects of Huge size or smaller, such as plants and corpses, decompose.
        %     This decomposition kills inanimate, living plants.
        %     All organic objects, regardless of size, are covered with various fungi.
        %     This ability has no effect on artificial environments or materials, such as metal or worked stone.
        %     At the end of the minute, the rotbringer regains your spent nature spell slots.

        %     If the rotbringer decomposes a Huge object with this ability, or a combination of smaller objects equivalent in size to a Huge object, you gain an bonus nature spell slot of your highest available spell level.
        %     This extra spell slot lasts until it is used, or until you regain your spell slots again.

        %     A rotbringer can only invoke rot on your surroundings in this way once per hour.
        %     If your surroundings are already decomposed or are not natural terrain, she cannot use this ability to regain your spells.
        %     Instead, she must meditate for 8 hours to slowly draw power from your surroundings, as a normal druid.

        %     \altcf[2]{Wild Speech} The rotbringer gains the ability to speak with plants at 2nd level.
        %     You gain the ability to speak with animals at 6th level, instead of at 2nd level.

        %     \altcf[3rd]{Wild Aspect} The rotbringer does not gain this ability.

        %     \altcf[3rd]{Rot Spell} The druid learns an additional spell slot and spell known.
        %     The spell must be taken from the following list of spells.
        %     The spell's level cannot exceed half your druid level.
        %     If she already knows a spell from the list at every spell level you ha access to, she may instead learn any nature spell (see \pcref{Nature Spells}).

        %     At 5th level, and every odd level, the druid may learn a new spell.

        %     \begin{dtable}
        %         \begin{dtabularx}{\columnwidth}{l X}
        %             \tb{Spell level} & \tb{Rotbringer Spells} \\
        %             1st & \spell{excrete slime}, \spell{lesser regeneration} \\
        %             2nd & \spell{fungal growth} \\
        %             3rd & \spell{rotburst} \\
        %             4th & \spell{poison} \\
        %             6th & \spell{regeneration} \\
        %             7th & \spell{greater rotburst} \\
        %         \end{dtabularx}
        %     \end{dtable}

        %     \altcf[7]{Fungal Armor} The rotbringer becomes covered in fungus that protects your from attacks. You gain a \plus1 bonus to Armor and Fortitude defense.

        %     This bonus increases by 1 at your 7th druid level, and every 4 druid levels thereafter.

\newpage
\section{Fighter}\label{Fighter}
    \begin{dtable!*}
        \lcaption{Fighter Progression}
        \begin{dtabularx}{\textwidth}{c >{\lcol}X >{\lcol}X >{\lcol}X >{\lcol}X}
            \tb{Rank} & \tb{Combat Discipline} & \tb{Equipment Training}  & \tb{Martial Mastery} & \tb{Battle Leader} \tableheaderrule
               1 & Discipline                   & Armor expertise         & Martial maneuvers         & Battle tactics
            \\ 2 & Enduring discipline          & Weapon training         & Martial lore          & Guided strike
            \\ 3 & Disciplined stance           & Greater armor expertise & Martial power         & Battle lore
            \\ 4 & Disciplined reaction         & Weapon expertise        & Martial maneuver      & Battle tactic
            \\ 5 & Greater enduring discipline  & Supreme armor expertise & Greater martial power & Brave leader
            \\ 6 & Disciplined vitality         & Weapon mastery          & Steady expertise      & Greater battle tactics
            \\ 7 & Supreme disciplined reaction & Armored juggernaut      & Martial maneuver      & Battle tactic
            \\ 8 &                              &                         &
        \end{dtabularx}
    \end{dtable!*}

    \classbasics{Alignment} Any.

    \classbasics{Archetypes} Fighters have the Martial Mastery, Equipment Training, Combat Discipline, and Battle Leader \glossterm{archetypes}.

    \subsection{Basic Class Abilities}
        If you are a fighter, you gain the following abilities.

        \cf{Ftr}{Defenses}
        You gain the following bonuses to your \glossterm{defenses}: \plus5 Fortitude, \plus3 Reflex, \plus4 Mental.

        \cf{Ftr}{Skills}
        You have the following \glossterm{class skills}:
        \begin{itemize}
            \item \subparhead{Strength} Climb, Jump, Swim.
            \item \subparhead{Dexterity} Acrobatics, Escape Artist, Ride.
            \item \subparhead{Intelligence} Craft.
            \item \subparhead{Perception} Awareness.
            \item \subparhead{Other} Bluff, Intimidate, Persuasion, Profession.
        \end{itemize}

        \cf{Ftr}{Weapon and Armor Proficiencies}
        You are proficient with simple weapons, any four other weapon groups, all body armor (light, medium, and heavy \glossterm{usage classes}), and shields.

    \subsection{Martial Mastery}
        This archetype grants you special abilities to use in combat.

        \cf{Ftr}[1]{Martial Maneuvers}
        You can channel your martial prowess into devastating attacks.
        You learn one \glossterm{maneuver} from the martial maneuver list (see \pcref{Martial Maneuvers}).
        You can also spend \glossterm{insight points} to learn one additional \glossterm{maneuver} per \glossterm{insight point}.
        As a \glossterm{standard action}, you can use any \glossterm{maneuver} you know.

        \cf{Ftr}[2]{Martial Lore} You gain two additional skill points.

        \cf{Ftr}[3]{Martial Power} You gain a \plus1 bonus to \glossterm{power} with \glossterm{mundane} abilities.

        \cf{Ftr}[4]{Martial Maneuver}
        You learn an additional \textit{martial maneuver}.

        \cf{Ftr}[5]{Greater Martial Power} The bonus from your \textit{martial power} ability increases to \plus2.

        \cf{Ftr}[6]{Steady Expertise} You gain a \plus1 bonus to \glossterm{accuracy} with \glossterm{mundane} abilities that do not have the \glossterm{AP} tag.

        \cf*{Ftr}[7]{Martial Maneuver}
        You learn an additional \textit{martial maneuver}.

    \subsection{Equipment Training}
        This archetype improves your combat prowess with weapons and armor.

        \cf{Ftr}[1]{Armor Expertise}
        You gain a \plus1 bonus to Armor defense while wearing body armor.
        In addition, you reduce the \glossterm{encumbrance} of body armor you wear by 1.

        \cf{Ftr}[2]{Weapon Training} You can use the \textit{weapon training} ability by spending an hour training with a weapon.
        You cannot use this ability with an \glossterm{exotic weapon} unless you are proficient with the \glossterm{weapon group} that weapon belongs to.
        \begin{freeability}{Weapon Training}
            You become proficient with the weapon you trained with.
            This ability's effect lasts until you use this ability again.

            \rankline
            \rank{4} You can use this ability with only five minutes of training.
            \rank{6} The effect is permanent.
            \rank{8} You can use this ability as a \glossterm{minor action}.
        \end{freeability}

        \cf{Ftr}[3]{Greater Armor Expertise}
        The \glossterm{encumbrance} reduction from your \textit{armor expertise} ability increases to 2.
        In addition, you treat body armor were one encumbrance category lighter than normal when doing so would be beneficial for you.

        \cf{Ftr}[4]{Weapon Expertise} You gain a \plus1 bonus to \glossterm{accuracy} with \glossterm{physical attacks}.

        \cf{Ftr}[5]{Supreme Armor Expertise}
        The \glossterm{encumbrance} reduction from your \textit{armor expertise} ability increases to 3.
        In addition, you treat body armor as if it were an additional encumbrance category lighter than normal when doing so would be beneficial for you.

        \cf{Ftr}[6]{Weapon Mastery} You gain a \plus2 bonus to \glossterm{power} with \glossterm{physical attacks}.

        \cf{Ftr}[7]{Armored Juggernaut}
        As long as you are wearing body armor, you gain a \plus1 bonus to all defenses.

    \subsection{Combat Discipline}
        This archetype allows you to improve your defenses and resist conditions.

        \cf{Ftr}[1]{Discipline} You can use the \textit{discipline} ability as a \glossterm{standard action}.
        \begin{freeability}{Discipline}
            Remove one \glossterm{condition} affecting you.

            \rankline
            \rank{3} You can remove an additional \glossterm{condition}.
            \rank{5} This ability gains the \glossterm{Swift} tag.
            When you use it, the penalties from the removed conditions do not affect you during the current phase.
            \rank{7} You can remove any number of \glossterm{conditions}.
        \end{freeability}

        \cf{Ftr}[2]{Enduring Discipline}
        You gain an additional \glossterm{hit point}.

        \cf{Ftr}[3]{Disciplined Stance}
        You learn how to control your body to adapt to any threat you face.
        You can use the \textit{disciplined stance} ability as a \glossterm{minor action}.
        \begin{freeability}{Disciplined Stance}
            You gain a \plus2 bonus to one \glossterm{defense} of your choice and take a \minus2 penalty to a different defense of your choice.
            This ability lasts until you use it again.

            \rankline
            \rank{5} The defense penalty is reduced to \minus1.
            \rank{7} The defense penalty is removed.
        \end{freeability}

        \cf{Ftr}[4]{Disciplined Reaction}
        You do not suffer any effects from \glossterm{conditions} in the first round that they are applied.
        You suffer their normal effects in the following round.
        You gain a \plus2 bonus to the Bluff and Intimidate skills.

        \cf{Ftr}[5]{Greater Enduring Discipline}
        You gain an additional \glossterm{hit point}.

        \cf{Ftr}[6]{Disciplined Vitality}
        You do not suffer any effects from \glossterm{vital wounds} until the end of the next round after they are applied.
        You suffer their normal effects after that time.

        \cf{Ftr}[7]{Supreme Disciplined Reaction}
        You do not suffer any effects from \glossterm{conditions} until the end of the next round after they are applied.
        You suffer their normal effects after that time.

    \subsection{Battle Leader}

        \cf{Ftr}[1]{Battle Tactics}
        You can lead your allies using tactics appropriate for the situation.
        Choose a single battle tactic from the list below.
        You can also spend \glossterm{insight points} to learn one additional \textit{battle tactic} per \glossterm{insight point}.

        You can initiate a \textit{battle tactic} as a standard action.
        Your \textit{battle tactics} affect your \glossterm{allies} within a \arealarge radius emanation from you who can either see or hear you.

        All \textit{battle tactics} have the \glossterm{Attune} (self) tag, so they last as long as you \glossterm{attune} to them (see \pcref{Attunement}).
        If you know multiple battle tactics, you can initiate multiple \textit{battle tactics} simultaneously.
        However, a creature can gain only the benefits of one \textit{battle tactic} at a time.
        If it has multiple options, it chooses at the start of each round which \textit{battle tactic} to follow.

        {
            \begin{attuneability}{Break Through}[\glossterm{Attune} (self)]
                Each target that is adjacent to at least one other target
                    gains a \plus2 bonus to \glossterm{accuracy} with the \textit{overrun} and \textit{shove} abilities (see \pcref{Special Combat Abilities}).

                \rankline
                \rank{3} The bonus increases to \plus3.
                \rank{5} The bonus increases to \plus4.
                \rank{7} The bonus increases to \plus4.
            \end{attuneability}

            \begin{attuneability}{Dogpile}[\glossterm{Attune} (self)]
                Each target that is adjacent to at least one other target
                    gains a \plus2 bonus to \glossterm{accuracy} with the \textit{grapple} ability (see \pcref{Grapple}).

                \rankline
                \rank{3} The bonus increases to \plus3.
                \rank{5} The bonus increases to \plus4.
                \rank{7} The bonus increases to \plus5.
            \end{attuneability}

            \begin{attuneability}{Duck and Cover}[\glossterm{Attune} (self)]
                Each target gains a \plus1 bonus to Armor defense against ranged \glossterm{physical attacks}.

                \rankline
                \rank{3} The bonus increases to \plus2.
                \rank{5} The bonus increases to \plus3.
                \rank{7} The bonus increases to \plus7.
            \end{attuneability}

            \begin{attuneability}{Group Up}[\glossterm{Attune} (self)]
                Each target that is adjacent to at least two other targets gains a \plus1 bonus to Armor defense.

                \rankline
                \rank{3} Each target affected by the Armor defense bonus also gains a \plus2 bonus to Mental defense.
                \rank{5} The Armor defense bonus increases to \plus2.
                \rank{7} The Mental defense bonus increases to \plus4.
            \end{attuneability}

            \begin{attuneability}{Hold Fast}[\glossterm{Attune} (self)]
                Each target that ends the \glossterm{movement phase} without moving gains a \plus1 bonus to Armor defense until it moves.

                \rankline
                \rank{3} Each target affected by the Armor defense bonus also gains a \plus2 bonus to Fortitude defense.
                \rank{5} The Armor defense bonus increases to \plus2.
                \rank{7} The Fortitude defense bonus increases to \plus4.
            \end{attuneability}

            \begin{attuneability}{Hustle}[\glossterm{Attune} (self)]
                Each target gains a \plus5 foot bonus to its \glossterm{base speed}.

                \rankline
                \rank{3} The speed bonus increases to \plus10 feet.
                \rank{5} The speed bonus increases to \plus15 feet.
                \rank{7} The speed bonus increases to \plus20 feet.
            \end{attuneability}

            \begin{attuneability}{Keep Moving}[\glossterm{Attune} (self)]
                Each target that ends the \glossterm{movement phase} at least twenty feet away from where it started the round gains a \plus1 bonus to Armor defense until the end of the round.

                \rankline
                \rank{3} Each target affected by the Armor defense bonus also gains a \plus2 bonus to Reflex defense.
                \rank{5} The Armor defense bonus increases to \plus2.
                \rank{7} The Reflex defense bonus increases to \plus4.
            \end{attuneability}
        }

        \cf{Ftr}[2]{Guided Strike} You can use the \textit{guided strike} ability as a standard action.
        \begin{freeability}{Guided Strike}[\glossterm{Swift}]
            Choose an \glossterm{ally} within \rngmed range.
            If the target makes a \glossterm{strike} during the current phase, it rolls twice and takes the higher result.

            \rankline
            \rank{4} The strike gains a \plus1 bonus to \glossterm{accuracy}.
            \rank{6} The bonus increases to \plus2.
            \rank{8} The bonus increases to \plus3.
        \end{freeability}

        \cf{Ftr}[3]{Battle Lore} You gain two additional \glossterm{skill points}. 

        \cf{Ftr}[4]{Battle Tactic} You learn an additional \textit{battle tactic}.

        \cf{Ftr}[5]{Brave Leader} You gain a \plus2 bonus to Mental defense.

        \cf{Ftr}[6]{Greater Battle Tactics} The area affected by your \textit{battle tactics} increases to an \areahuge emanation.

        \cf*{Ftr}[7]{Battle Tactic} You learn an additional \textit{battle tactic}.
\newpage
\section{Mage}\label{Mage}
    \begin{dtable!*}
        \lcaption{Mage Progression}
        \begin{dtabularx}{\textwidth}{c >{\lcol}X >{\lcol}X >{\lcol}X >{\lcol}X}
            \tb{Rank} & \tb{Arcane Spellcasting} & \tb{Arcane Spell Mastery} & \tb{Arcane Scholar} & \tb{Innate Arcanist} \tableheaderrule
            1    & Mystic spheres  & Mage armor                  & Scholastic lore       & Mystic healing
            \\ 2 & Spells          & Mystic knowledge            & Rituals               & Limited components
            \\ 3 & Spell level (2) & Wellspring of power         & Scholastic insight    & Spell knowledge
            \\ 4 & Spell level (3), spell & Mystic knowledge     & Ritual expertise      & Magic resistance
            \\ 5 & Spell level (4) & Greater mage armor          & Contingency           & Greater limited components
            \\ 6 & Spell level (5) & Mystic knowledge            & Scholastic insight    & Spell knowledge
            \\ 7 & Spell level (6), spell & Greater wellspring of power & Malleable contingency & Spell absorption
            \\ 8 & Spell level (7) &                             &                       &
        \end{dtabularx}
    \end{dtable!*}

    \classbasics{Alignment} Any.

    \classbasics{Archetypes} Mages have the Arcane Spellcasting, Arcane Spell Mastery, Arcane Scholar, and Innate Arcanist \glossterm{archetypes}.

    \subsection{Basic Class Abilities}
        If you are a mage, you gain the following abilities.

        \cf{Mge}{Defenses}
        You gain the following bonuses to your \glossterm{defenses}: \plus3 Fortitude, \plus4 Reflex, \plus5 Mental.

        \cf{Mge}{Skills}
        You have the following \glossterm{class skills}:
        \begin{itemize}
            \item \subparhead{Intelligence} Craft, Deduction, Knowledge (all kinds, taken individually), Linguistics.
            \item \subparhead{Perception} Awareness, Spellcraft.
            \item \subparhead{Other} Bluff, Intimidate, Persuasion, Profession.
        \end{itemize}

        \cf{Mge}{Weapon and Armor Proficiencies}
        You are proficient with simple weapons and one other weapon group.
        You are proficient with shields, but not with any type of armor.
        Encumbrance from armor interferes with the gestures you make to cast spells, which can cause your spells with somatic components to fail (see \pcref{Somatic Component Failure}).

        \cf{Mge}{Arcane Essence}
        All mages have access to great arcane power.
        However, not all mages acquired this power in the same way.
        You choose an arcane essence.
        Many mage abilities have special effects based on whether you are a sorcerer or a wizard.
        \parhead{Sorcerer} Sorcerers have an intuitive connection to magic that allows them to cast spells without preparation or training.
        They cast spells with their Willpower.
        \parhead{Wizard} Wizards study arcane mysteries for years to learn the secret ways of magic.
        They cast spells with their Intelligence.

    \subsection{Arcane Spellcasting}
        This archetype grants you the ability to cast arcane spells.
        % TODO: less boring attribute integration?
        If you are a sorcerer, you must have a starting Willpower of at least 1 to gain this archetype.
        If you are a wizard, you must have an starting Intelligence of at least 2 to gain this archetype.

        \cf{Mge}[1]{Spellcasting}[Magical]
        Your deity grants you the ability to use arcane magic.
        You gain access to one arcane \glossterm{mystic sphere} (see \pcref{Arcane Mystic Spheres}).
        Each \glossterm{mystic sphere} has a set of \glossterm{spells} associated with it.

        You automatically learn all \glossterm{cantrips} from any mystic sphere you have access to.
        In addition, you learn one 1st level \glossterm{spell} from your chosen \glossterm{mystic sphere}.
        You can also spend \glossterm{insight points} to learn one additional arcane spell per \glossterm{insight point}.
        Unless otherwise noted in a spell's description, casting a spell requires a \glossterm{standard action}.

        When you gain access to a new \glossterm{mystic sphere} or spell level,
            you can exchange any number of spells you know for other spells,
            including spells of the higher level.

        Arcane spells require both \glossterm{verbal components} and \glossterm{somatic components} to cast (see \pcref{Casting Components}).
        For details about mystic spheres and casting spells, see \pcref{Spell and Ritual Mechanics}.

        \cf{Mge}[2]{Mystic Sphere} You gain access to an additional arcane \glossterm{mystic sphere}, including all \glossterm{cantrips} from that sphere.
        In addition, you learn an additional arcane \glossterm{spell}.

        \cf{Mge}[3]{Spell Level}[Magical] You gain the ability to cast 2nd level arcane spells.

        \cf*{Mge}[4]{Spell Level}[Magical] You gain the ability to cast 3rd level arcane spells.

        \cf{Mge}[4]{Spell} You gain an additional \glossterm{spell} for any arcane \glossterm{mystic sphere} you know.

        \cf*{Mge}[5]{Spell Level}[Magical] You gain the ability to cast 4th level arcane spells.

        \cf*{Mge}[6]{Spell Level}[Magical] You gain the ability to cast 5th level arcane spells.

        \cf*{Mge}[7]{Spell Level}[Magical] You gain the ability to cast 6th level arcane spells.

        \cf*{Mge}[7]{Spell} You gain an additional \glossterm{spell} for any arcane \glossterm{mystic sphere} you know.

        \cf*{Mge}[8]{Spell Level}[Magical] You gain the ability to cast 7th level arcane spells.

    \subsection{Arcane Spell Mastery}
        This archetype improves the arcane spells you cast.
        You must have the Arcane Spellcasting archetype to gain the abilities from this archetype.

        \cf{Mge}[1]{Mage Armor}[Magical] You can use the \textit{mage armor} ability as a standard action.
        \begin{freeability}{Mage Armor}
            You create a translucent suit of magical armor on your body and over your hands.
            This functions like body armor that provides a \plus2 bonus to Armor defense and has no \glossterm{encumbrance}.
            The body armor does not appear if you are wearing other body armor of any kind.

            As long as you have a free hand, the barrier also manifests as a shield that provides a \plus1 bonus to Armor defense.
            This bonus is considered to come from a shield, and does not stack with the benefits of using a physical shield.

            This ability lasts until you \glossterm{dismiss} it as a free action.

            \rankline
            \rank{3} The body armor also gives you a \plus1 bonus to \glossterm{resistances} against \glossterm{physical damage}.
            \rank{5} The defense bonus from the body armor increases to \plus3.
            \rank{7} The bonus to \glossterm{resistances} from the body armor increases to \plus3.
        \end{freeability}

        \cf{Mge}[2]{Mystic Knowledge}[Magical]
        You gain your choice of one of the following abilities.
        {
            \parhead{Spell Knowledge} You learn an additional arcane \glossterm{spell} (see \pcref{Arcane Mystic Spheres}).
                You can choose this ability multiple times, gaining access to an additional spell each time.
            \parhead{Rituals} You can only choose this mystic knowledge if you are a wizard.
                The maximum level of arcane ritual you can learn or perform is equal to the maximum arcane level of arcane spell that you can cast.
                You gain the ability to perform arcane rituals to create unique magical effects (see \pcref{Rituals}).
                You cannot choose this ability multiple times.
            \parhead{Mystic Sphere Access} You gain access to an additional arcane \glossterm{mystic sphere}.
                You cannot choose this ability multiple times.
        }

        \cf{Mge}[3]{Wellspring of Power}[Magical]
        You gain a \plus1 bonus to \glossterm{power} with arcane spells.

        \cf*{Mge}[4]{Mystic Knowledge}[Magical]
        You gain an additional \textit{mystic knowledge} ability.

        \cf{Mge}[5]{Greater Mage Armor}[Magical]
        The defense bonus from the body armor from your \textit{mage armor} ability increases to \plus3.

        \cf*{Mge}[6]{Mystic Knowledge}[Magical]
        You gain an additional \textit{mystic knowledge} ability.

        \cf{Mge}[7]{Greater Wellspring of Power}[Magical]
        The bonus from your \textit{wellspring of power} ability increases to \plus2.

    \subsection{Arcane Scholar}
        This archetype deepens your study of arcane magic.
        You be a wizard and have the Arcane Spellcasting archetype to gain the abilities from this archetype.

        \cf{Mge}[1]{Scholastic Lore} You gain two additional skill points.

        \cf{Mge}[2]{Rituals} You gain the ability to perform arcane rituals to create unique magical effects (see \pcref{Rituals}).
        The maximum level of arcane ritual you can learn or perform is equal to the maximum level of arcane spell that you can cast.

        \cf{Mge}[3]{Scholastic Insight}[Magical]
        You gain one of the following insights.
        Some insights can be chosen multiple times, as indicated in their descriptions.

        {
            \parhead{Expanded Sphere Access} You gain access to a new \glossterm{mystic sphere}.
            \par You cannot choose this insight multiple times.

            \parhead{Expanded Spell Knowledge} You learn an additional \glossterm{spell}.
            \par You can choose this insight multiple times, learning an additional spell each time.

            \parhead{Signature Spell} Choose a \glossterm{spell} you know.
            You gain a \plus1 bonus to \glossterm{accuracy} with that spell.
            In addition, you gain a \plus5 bonus to \glossterm{concentration} checks you make to cast that spell.
            \par If you choose this insight multiple times, you must choose a different \glossterm{spell} each time.

            \parhead{Sphere Specialization} Choose a a \glossterm{mystic sphere} you have access to.
            You gain a \plus1 bonus to \glossterm{accuracy} with abilities from that \glossterm{mystic sphere}.
            In addition, you learn an additional \glossterm{spell} from that \glossterm{mystic sphere}.
            In exchange, you must lose access to another \glossterm{mystic sphere} you have.
            You must exchange all spells you know from that \glossterm{mystic sphere} with spells from other \glossterm{mystic spheres} you have access to.
            \par You cannot choose this insight multiple times.

            \parhead{Memorized Sphere} % TODO: clarify you need to be high enough level?
            Choose a \glossterm{mystic sphere} you have access to.
            You can perform rituals from that \glossterm{mystic sphere} without having them written in your ritual book.
            If you lead a ritual from that \glossterm{mystic sphere}, it requires half the normal amount of time to perform and costs half the normal number of \glossterm{action points} (minimum 1).
            \par You can choose this insight multiple times, choosing a different \glossterm{mystic sphere} each time.
        }

        \cf{Mge}[4]{Ritual Expertise} Each \glossterm{action point} you spend performing a \glossterm{ritual} contributes two action points towards the ritual's cost.

        % This may be a better fit?
        % \cf{Mge}[4]{Scholastic Knowledge} You gain a \plus2 bonus to all Knowledge skills.

        \cf{Mge}[5]{Contingency}  You learn the Contingency \glossterm{augment}, allowing you to prepare a spell so it takes effect automatically if specific circumstances arise.
        The Contingency \glossterm{augment} adds two levels to a spell's level.
        You can apply this augment to any arcane spell.
        % If any spells take more than one standard action, they would need to be excluded from Contingency, but none exist
        % You can apply this augment to any arcane spell that can be cast as a \glossterm{standard action} or \glossterm{minor action}.

        Casting a spell with the Contingency augment takes 5 minutes.
        When the casting is complete, the spell has no immediate effect.
        Instead, it automatically takes effect when some specific circumstances arise.
        During the time required to cast the spell, you specify what circumstances cause the spell to take effect.

        The spell can be set to trigger in response to any circumstances that a typical human observing you and your situation could detect.
        For example, you could specify ``when I fall at least 50 feet'' or ``when I take a \glossterm{vital wound}'', but not ``when there is an invisible creature within 50 feet of me'' or ``when I have only one \glossterm{hit point} remaining.''
        The more specific the required circumstances, the better -- vague requirements, such as ``when I am in danger'', may cause the spell to trigger unexpectedly or fail to trigger at all.
        If you attempt to specify multiple separate triggering conditions, such as ``when I take damage or when an enemy is adjacent to me'', the spell will randomly ignore all but one of the conditions.

        If the spell needs to be targeted, the trigger condition can specify a simple rule for identifying how to target the spell, such as ``the closest enemy''.
        If the rule is poorly worded or imprecise, the spell may target incorrectly or fail to activate at all.
        Any spells which require decisions, such as the \spell{dimension door} spell, must have those decisions made at the time it is cast.
        You cannot alter those decisions when the contingency takes effect.

        You can have only one spell with this augment active at a time.
        If you use the augment again with a different spell, the old contingency is removed.

        \cf*{Mge}[6]{Scholastic Insight}[Magical]
        You learn an additional \textit{scholastic insight}.

        \cf{Mge}[7]{Malleable Contingency} If you have a \textit{contingency} active, you can change the conditions of your \textit{contingency} as a standard action.

    \subsection{Innate Arcanist}
        This archetype deepens your innate connection to arcane magic.
        You be a sorcerer and have the Arcane Spellcasting archetype to gain the abilities from this archetype.

        \cf{Mge}[1]{Nonverbal Caster} When you cast a spell, you can omit that spell's \glossterm{verbal components}.
        This tests your focus when casting the spell, forcing you to make a \glossterm{concentration} check to cast the spell successfully (see \pcref{Concentration}).

        \cf{Mge}[2]{Innate Magic} You gain a \plus1 bonus to Fortitude and Mental defenses.

        \cf{Mge}[3]{Augment} You learn one \glossterm{augment} (see \pcref{Augments}).
        You can apply augments to spells to change their effects.
        When you gain access to a new arcane spell level, you may change which augments you know.

        \cf{Mge}[4]{Stilled Caster} When you cast a spell, you can omit that spell's \glossterm{somatic components}.
        This tests your focus when casting the spell, forcing you to make a \glossterm{concentration} check to cast the spell successfully (see \pcref{Concentration}).
        If you also omit the spell's \glossterm{somatic components} with your \textit{nonverbal caster} ability,
            you take a \minus2 penalty to the \glossterm{concentration} check to cast the spell,
            as normal when there are multiple effects testing your concentration.

        \cf{Mge}[5]{Magic Resistance} You gain a \plus2 bonus to \glossterm{defenses} against \glossterm{magical} abilities.

        \cf*{Mge}[6]{Augment} You learn an additional \glossterm{augment}.

        \cf{Mge}[7]{Spell Absorption} When a spell is used to attack you, if the attack fails to beat your defenses, you gain the ability to cast the spell once.
        The spell retains its original \glossterm{augments}, if any, and you cannot apply any \glossterm{augments} to it.
        When you cast the spell, you use your own \glossterm{accuracy}, \glossterm{power}, and abilities to determine the effects of the spell.
        Once you cast the spell, you expend the absorbed energy, and you cannot cast it again.

        If you resist multiple spells simultaneously, or if you resist another spell before casting the previous spell you resisted, you choose which spell you gain the ability to cast.

\newpage
\section{Monk}\label{Monk}
    \begin{dtable!*}
        \lcaption{Monk Progression}
        \begin{dtabularx}{\textwidth}{c >{\lcol}X >{\lcol}X >{\lcol}X >{\lcol}X}
            \tb{Rank} & \tb{Ki} & \tb{Transcendent Sage} & \tb{Esoteric Warrior} & \tb{Perfected Form} \tableheaderrule
            1    & Ki barrier, ki manifestations                 & Clear the mind      & Esoteric maneuvers, unarmed warrior            & Fast movement
            \\ 2 & Ki strike                              & Transcend frailty   & Unarmed expertise         & Graceful athletics
            \\ 3 & Ki power                               & Sage lore           & Greater unarmed warrior   & Perfect body
            \\ 4 & Ki manifestation                       & Inner peace         & Esoteric maneuver         & Greater fast movement
            \\ 5 & Greater ki barrier, greater ki strike  & Transcend flesh     & Greater unarmed expertise & Perfected defense
            \\ 6 & Greater ki power                       & Greater inner peace & Supreme unarmed warrior   & Greater perfect body
            \\ 7 & Ki manifestation                       & Transcend senses    & Esoteric maneuver         & Supreme fast movement
            \\ 8 & Dual manifestation, supreme ki barrier & Transcend mortality &
        \end{dtabularx}
    \end{dtable!*}

    \classbasics{Alignment} Any nonchaotic.

    \classbasics{Archetypes} Monks have the Ki, Esoteric Warrior, Transcendent Sage, and Perfected Form \glossterm{archetypes}.

    \subsection{Basic Class Abilities}
        If you are a monk, you gain the following abilities.

        \cf{Mnk}{Defenses}
        You gain the following bonuses to your \glossterm{defenses}: \plus3 Fortitude, \plus5 Reflex, \plus4 Mental.

        \cf{Mnk}{Skills}
        You have the following \glossterm{class skills}:
        \begin{itemize}
            \item \subparhead{Strength} Climb, Jump, Swim.
            \item \subparhead{Dexterity} Acrobatics, Escape Artist, Ride, Stealth.
            \item \subparhead{Intelligence} Craft, Deduction, Heal.
            \item \subparhead{Perception} Awareness, Spellcraft, Survival.
            \item \subparhead{Other} Bluff, Intimidate, Perform, Persuasion, Profession.
        \end{itemize}

        \cf{Mnk}{Weapon and Armor Proficiencies}
        Monks are proficient with simple weapons, monk weapons, and any one other weapon group.
        Monks are not proficient with any armor or shields.

    \subsection{Ki}
        This archtype grants you abilities you can use in combat.
        If you wear armor, use a shield, or have \glossterm{encumbrance}, you lose the benefit of all abilities from this archetype.

        \cf{Mnk}[1]{Ki Barrier}[Magical]
        If you are not wearing armor and have no \glossterm{encumbrance}, you gain a ki barrier around your body.
        This functions like body armor that provides a \plus2 bonus to Armor defense and has no \glossterm{encumbrance}.

        As long as you have a free hand, the barrier also manifests as a shield that provides a \plus1 bonus to Armor defense.
        This bonus is considered to come from a shield, and does not stack with the benefits of using a physical shield.

        \cf{Mnk}[1]{Ki Manifestations}[Magical]
        You can channel your ki to temporarily enhance your abilities.
        Choose one \textit{ki manifestation} from the list below.
        You can also spend \glossterm{insight points} to learn one additional \textit{ki manifestation} per \glossterm{insight point}.
        You can use any \textit{ki manifestation} ability you know using the type of action indicated in the ability's description.

        Whenever you use a \textit{ki manifestation}, you lose hit points equal to your level.
        You cannot use more than one \textit{ki manifestation} per round.
        {
            \begin{freeability}{Abandon the Fragile Self}
                You can use this ability at the start of the round.
                You can negate one \glossterm{condition} that would be applied to you this round.

                \rankline
                \rank{3} You can negate two conditions that would be applied to you this round.
                \rank{5} You can negate three conditions that would be applied to you this round.
                \rank{7} You are immune to conditions this round.
            \end{freeability}

            \begin{freeability}{Burst of Blinding Speed}
                You can use this ability at the start of the round.
                You gain a \plus10 foot bonus to your land speed this round.

                \rankline
                \rank{3} You can also ignore \glossterm{difficult terrain} this round.
                \rank{5} The speed bonus increases to \plus20 feet.
                \rank{7} You can also move or stand on liquids as if they were solid this round.
            \end{freeability}

            \begin{freeability}{Elegant Whirl of Fluid Motion}
                You can use this ability at the start of the round.
                You gain a \plus5 bonus to the Acrobatics skill this round (see \pcref{Acrobatics}).

                \rankline
                \rank{3} The bonus increases to \plus10.
                \rank{5} The bonus lasts until the end of the next round.
                \rank{7} The bonus increases to \plus20.
            \end{freeability}

            \begin{freeability}{Flash Step}[\glossterm{Teleportation}]
                You can use this ability as part of movement.
                % TODO: is 'horizontally' the correct word?
                You teleport horizontally instead of moving normally.
                If your \glossterm{line of effect} to your destination is blocked, or if this teleportation would somehow place you inside a solid object, your teleportation is cancelled and you remain where you are.

                Teleporting a given distance costs movement equal to twice that distance.
                For example, if you have a 30 foot movement speed, you can move 10 feet, teleport 5 feet, and move an additional 10 feet before your movement ends.

                \rankline
                \rank{3} You can use this ability to move even if you are \glossterm{immobilized} or \glossterm{grappled}.
                \rank{5} The movement cost to teleport is reduced to be equal to the distance you teleport.
                \rank{7} You can attempt to teleport to locations outside of \glossterm{line of sight} and \glossterm{line of effect}.
                If your intended destination is invalid, the distance you spent teleporting is wasted, but you suffer no other ill effects.
            \end{freeability}

            \begin{freeability}{Leap of the Heavens}
                You can use this ability at the start of the round.
                You gain a \plus5 bonus to the Jump skill this round (see \pcref{Jump}).

                \rankline
                \rank{3} The bonus increases to \plus10.
                \rank{5} The bonus lasts until the end of the next round.
                \rank{7} The bonus increases to \plus20.
            \end{freeability}

            \begin{freeability}{Scale the Highest Tower}
                You can use this ability at the start of the round.
                You gain a \plus5 bonus to the Climb skill this round (see \pcref{Climb}).
                % TODO: is this wording correct?

                \rankline
                \rank{3} The Climb bonus increases to \plus10.
                \rank{5} The bonus lasts until the end of the next round.
                \rank{5} The bonus increases to \plus20.
            \end{freeability}

            \begin{freeability}{See the Flow of Life}
                You can use this ability at the start of the round.
                You gain the \glossterm{lifesense} ability with a 50 foot range this round.
                You know the location of all living creatures within range.
                This can allow you to identify the presence of hidden living creatures or to identify that seemingly living creatures are not alive.
                You still take normal \glossterm{miss chances} for concealment, invisibility, and so on.

                \rankline
                \rank{3} The range of the \glossterm{lifesense} ability is increased to 100 feet.
                \rank{5} You also gain the \glossterm{lifesight} ability with a 50 foot range.
                You can ``see'' any living creatures and their equipment within range perfectly, regardless of lighting conditions, blindness, invisibility, or any other means of concealment.
                \rank{7} The range of the \glossterm{lifesense} ability is increased to 200 feet, and the range of the \glossterm{lifesight} ability is increased to 100 feet.
            \end{freeability}

            \begin{freeability}{Step Between the Mystic Worlds}
                You can use this ability at the start of the round.
                You gain a \plus2 bonus to \glossterm{defenses} against \glossterm{magical} abilities that directly target you this round.
                This does not protect you from abilities that affect an area.

                \rankline
                \rank{3} The defense bonus is increased to \plus3.
                \rank{5} The defense bonus is increased to \plus4.
                \rank{7} The defense bonus is increased to \plus5.
            \end{freeability}

            \begin{freeability}{Surpass the Mortal Limits}
                You can use this ability at the start of the round.
                You can use your \glossterm{power} in place of your Strength, Dexterity, and Constitution when making checks this round.

                \rankline
                \rank{3} You also gain a \plus2 bonus to checks based on Strength, Dexterity, and Constitution.
                \rank{5} The effect lasts until the end of the next round.
                \rank{7} The bonus increases to \plus4.
            \end{freeability}

            % TODO: add more
        }

        \cf{Mnk}[2]{Ki Strike} You can treat all strikes you make as \glossterm{magical strikes}, allowing you to use your \glossterm{power} with magical abilities to determine your damage.

        \cf{Mnk}[3]{Ki Power} You gain a \plus1 bonus to \glossterm{power} with \glossterm{magical} abilities.

        \cf{Mnk}[4]{Ki Manifestation}[Magical]
        You learn an additional \textit{ki manifestation}.

        \cf{Mnk}[5]{Greater Ki Barrier} The body armor bonus from your \textit{ki barrier} ability increases to \plus3.

        \cf{Mnk}[6]{Greater Ki Power} The bonus from your \textit{ki power} ability increases to \plus2.

        \cf*{Mnk}[7]{Ki Manifestation}[Magical]
        You learn an additional \textit{ki manifestation}.

        \cf{Mnk}[8]{Dual Manifestation}[Magical] You can use two \textit{ki manifestation} abilities per round.

        \cf{Mnk}[8]{Supreme Ki Barrier} The shield bonus from your \textit{ki barrier} ability increases to \plus2.

    \subsection{Esoteric Warrior}\label{Esoteric Warrior}
        This archetype improves your combat prowess with unusual fighting styles like unarmed attacks and grappling.

        \cf{Mnk}[1]{Esoteric Maneuvers} 
        You learn one \glossterm{maneuver} from the esoteric maneuver list (see \pcref{Esoteric Maneuvers}).
        You can also spend \glossterm{insight points} to learn one additional \glossterm{maneuver} per \glossterm{insight point}.
        As a \glossterm{standard action}, you can use any \glossterm{maneuver} you know.

        \cf{Mnk}[1]{Unarmed Warrior}
        You are \glossterm{proficient} with your \glossterm{unarmed attack}.
        In addition, you gain a \plus2d bonus to damage with your unarmed attack.
        For details about how to fight while unarmed, see \pcref{Unarmed Combat}.

        \cf{Mnk}[2]{Unarmed Expertise}
        When you attack with a free hand, you gain a \plus1 bonus to \glossterm{accuracy} with the \textit{dirty trick}, \textit{disarm}, \textit{grapple}, \textit{overrun}, \textit{shove}, and \textit{trip} abilities (see \pcref{Special Combat Abilities}).
        In addition, you gain a \plus1 bonus to \glossterm{accuracy} with grapple actions (see \pcref{Grapple Actions}).

        % TODO: wording?
        % \cf{Mnk}[3]{Evasion} When you are attacked by an ability that affects an area, you can use your Reflex defense in place of any other defenses against that attack.

        % \cf{Mnk}[3]{Unfettered Athletics} You gain two additional skill points.

        % \cf{Mnk}[6]{Intuitive Reaction}
        % You gain a \plus2 bonus to Reflex defense and \glossterm{initiative}.

        \cf*{Mnk}[3]{Greater Unarmed Warrior} The damage bonus from your \textit{unarmed warrior} ability increases to \plus3d.

        \cf{Mnk}[4]{Esoteric Maneuver} You learn an additional esoteric \glossterm{maneuver} (see \pcref{Esoteric Maneuvers}).

        \cf{Mnk}[5]{Greater Unarmed Expertise} The bonus from your \textit{unarmed expertise} ability increases to \plus2.

        \cf{Mnk}[6]{Supreme Unarmed Warrior} The damage bonus from your \textit{unarmed warrior} ability increases to \plus4d.

        \cf*{Mnk}[7]{Esoteric Maneuver} You learn an additional esoteric \glossterm{maneuver} (see \pcref{Esoteric Maneuvers}).

    \subsection{Transcendent Sage}
        This archetype grants you abilities to resist or remove conditions.

        \cf{Mnk}[1]{Clear the Mind} You can use the \textit{clear the mind} ability as a standard action.
        \begin{freeability}{Clear the Mind}
            You remove one \glossterm{condition} affecting you.
            This cannot remove a condition applied during the current round.
        \end{freeability}

        \cf{Mnk}[2]{Transcend Frailty}[Magical]
        You are immune to being \glossterm{sickened} and \glossterm{nauseated}.

        \cf{Mnk}[3]{Sage Lore} You gain two additional skill points.

        \cf{Mnk}[4]{Inner Peace} You are immune to hostile \glossterm{Emotion} abilities.

        \cf{Mnk}[5]{Transcend Flesh}[Magical]
        You are immune to hostile \glossterm{Flesh} abilities.
        In addition, you no longer take penalties to your attributes for aging, and cannot be magically aged.
        You still die of old age when your time is up.

        \cf{Mnk}[6]{Greater Inner Peace}
        You are immune to hostile \glossterm{Compulsion} abilities.

        \cf{Mnk}[7]{Transcend Senses}[Magical]
        You are imune to being \glossterm{blinded} and \glossterm{deafened}.
        % TODO: wording?
        In addition, you gain the ability to see living creatures perfectly, regardless of concealment or invisibility.

        % This feels like a rank 8 ability
        \cf{Mnk}[8]{Transcend Mortality}[Magical]
        If you die, you may choose to retain control of your body and soul through sheer force of will.
        Your body immediately disappears, and your soul does not travel to an afterlife.
        Instead, your body reforms with no trace of its injuries 8 hours later.
        The reformed body is in perfect health and can be any age you choose, to a minimum of the age of adulthood for your species.
        You can reform your body at the place where you died, or in any place on the same plane that is deeply familiar to you.

        After each time you reform yourself in this way, it takes an additional hour to reform the next time you ``die''.
        You can only be permanently killed by the direct intervention of a deity.

    \subsection{Perfected Form}
        This archetype improves the perfection of your physical body through rigorous training.

        \cf{Mnk}[1]{Fast Movement} You gain a \plus10 foot bonus to your \glossterm{base speed}.

        \cf{Mnk}[2]{Graceful Athletics} You gain two additional skill points.

        \cf{Mnk}[3]{Perfect Body} You gain a \plus1 bonus to one physical \glossterm{attribute} of your choice: Strength, Dexterity, or Constitution.

        \cf{Mnk}[4]{Greater Fast Movement} The speed bonus from your \textit{fast movement} ability increases to \plus20 feet.

        \cf{Mnk}[5]{Perfected Defense} You gain a \plus1 bonus to Armor defense and Fortitude defense.

        \cf{Mnk}[6]{Greater Perfect Body} You gain a \plus1 bonus to the physical \glossterm{attributes} you did not increase with your \textit{perfect body} ability.

        \cf{Mnk}[7]{Supreme Fast Movement} The speed bonus from your \textit{fast movement} ability increases to \plus30 feet.

    \subsection{Ex-Monks}
        As long as you are chaotic, you lose all of your \glossterm{magical} monk abilities.

\newpage
\section{Paladin}\label{Paladin}
    \begin{dtable!*}
        \lcaption{Paladin Progression}
        \begin{dtabularx}{\textwidth}{c >{\lcol}X >{\lcol}X >{\lcol}X >{\lcol}X}
            \tb{Rank} & \tb{Devoted Paragon} & \tb{Divine Spellcasting}  & \tb{Zealous Warrior} & \tb{Stalwart Guardian} \tableheaderrule
            1    & Aligned resistance         & Spellcasting         & Smite                                   & Lay on hands
            \\ 2 & Aligned aura               & Mystic spheres         & Recovering smite                                & Stalwart champion
            \\ 3 & Paragon power              & Spell level (2)        & Zealous power                                   & Stalwart resilience
            \\ 4 & Greater aligned resistance & Spell level (3)        & Greater smite                                   & Greater lay on hands
            \\ 5 & Greater aligned aura       & Spell level (4), spell & Greater recovering smite, greater zealous power & Greater stalwart champion
            \\ 6 & Greater paragon power      & Spell level (5)        & Supreme smite                                   & Greater stalwart resilience
            \\ 7 &                            & Spell level (6)        & Supreme recovering smite                        & Supreme lay on hands
            \\ 8 & Aligned soul               & Spell level (7), spell &                                                 &
        \end{dtabularx}
    \end{dtable!*}

    \classbasics{Alignment} Any other than true neutral.

    \classbasics{Archetypes} Paladins have the Devoted Paragon, Divine Spellcasting, Zealous Warrior, and Stalwart Guardian \glossterm{archetypes}.
    If you are a paladin, you choose three of those four archetypes.

    \subsection{Basic Class Abilities}
        If you are a paladin, you gain the following abilities.

        \cf{Pal}{Defenses}
        You gain the following bonuses to your \glossterm{defenses}: \plus5 Fortitude, \plus3 Reflex, \plus4 Mental.

        \cf{Pal}{Skills}
        You have the following \glossterm{class skills}:
        \begin{itemize}
            \item \subparhead{Dexterity} Ride.
            \item \subparhead{Intelligence} Craft, Deduction, Heal, Knowledge (local, religion).
            \item \subparhead{Perception} Awareness, Sense Motive.
            \item \subparhead{Other} Bluff, Intimidate, Persuasion, Profession.
        \end{itemize}

        \cf{Pal}{Weapon and Armor Proficiencies}
        Paladins are proficient with simple weapons, any three other weapon groups, all body armor (light, medium, and heavy \glossterm{usage classes}), and shields.

        \cf{Pal}{Devoted Alignment} 
        You are devoted to a specific alignment.
        You must choose one of your alignment components: good, evil, lawful, or chaotic.
        The alignment you choose is your devoted alignment.
        Your paladin abilities are affected by this choice.
        % seems unnecessary
        % You excel at slaying creatures with alignments opposed to your devoted alignment.
        Your alignment cannot be changed without extraordinary repurcussions.

    \subsection{Devoted Paragon}
        This archetype deepens your connection to your alignment, granting you an aura and improving your combat abilities.

        \cf{Pal}[1]{Aligned Resistance}[Magical]
        You gain the ability to resist attacks based on your devoted alignment.

        \subparhead{Chaos} You gain a \plus5 bonus to \glossterm{defenses} against \glossterm{Compulsion} effects.
        \subparhead{Evil} You gain a \plus5 bonus to \glossterm{defenses} against \glossterm{Disease} and \glossterm{Poison} effects.
        \subparhead{Good} You gain a \plus5 bonus to \glossterm{defenses} against \glossterm{Life} effects.
        \subparhead{Law} You gain a \plus5 bonus to \glossterm{defenses} against \glossterm{Emotion} effects.

        \cf{Pal}[2]{Aligned Aura}[Magical]
        Your devotion to your alignment affects the world around you, bringing it closer to your ideals.
        You constantly radiate an aura in a \areamed radius \glossterm{emanation} from you.
        You can freely choose whether creatures in the area are targeted by the aura, including yourself.
        The effect of the aura depends on your devoted alignment, as described below.
        You can suppress or resume the aura as a \glossterm{minor action}.

        \subparhead{Chaos} When a target rolls a 1 on an attack roll with a \glossterm{strike}, it \glossterm{explodes} (see \pcref{Exploding Attacks}.
        This does not affect bonus dice rolled for exploding attacks (see \pcref{Exploding Attacks}).
        \subparhead{Evil} Once per round, when a target takes damage, it increases that damage by an amount equal to your \glossterm{power}.
        If multiple targets take damage simultaneously, you can choose which target takes the increased damage after you know how much damage each target took,
        but before you know the results of the damage (such as whether a damaged creature takes a \glossterm{vital wound}).
        % TODO: clarify what happens if multiple people try to Good aura the same target
        \subparhead{Good} When a target takes damage, you may take half that damage (rounded down) instead.
        Any abilities you have that would reduce or negate the effects of the attack have no effect.
        The protected creature takes the remaining half of the damage, and suffers any non-damaging effects of the attack normally.
        \subparhead{Law} When a target rolls a 1 on an attack roll with a \glossterm{strike}, the attack roll is treated as a 6.
        This does not affect bonus dice rolled for exploding attacks (see \pcref{Exploding Attacks}).

        \cf{Pal}[3]{Paragon Power}
        You gain a \plus1 bonus to \glossterm{power} with \glossterm{magical} abilities.

        \cf{Pal}[4]{Greater Aligned Resistance}[Magical]
        Your ability to resist attacks based on your alignment improves.

        \subparhead{Chaos} You are immune to hostile \glossterm{Compulsion} abilities.
        \subparhead{Evil} You are immune to hostile \glossterm{Disease} and \glossterm{Poison} abilities.
        \subparhead{Good} You are immune to hostile \glossterm{Life} abilities.
        \subparhead{Law} You are immune to hostile \glossterm{Emotion} abilities.

        \cf{Pal}[5]{Greater Aligned Aura}[Magical]
        The effect of your \textit{aligned aura} becomes stronger, as described below.

        \subparhead{Chaos} The effect applies to all attacks, not just \glossterm{strikes}.
        \subparhead{Evil} The damage increases to be equal to twice your \glossterm{power}.
        \subparhead{Good} When you redirect damage from a creature with this aura, you can redirect all damage from the attack to you instead of only half the damage.
        The protected creature still suffers any non-damaging effects of the attack normally.
        \subparhead{Law} The effect applies to all attacks, not just \glossterm{strikes}.

        \cf{Pal}[6]{Greater Paragon Power} The bonus from your \textit{paragon power} ability increases to \plus2.

        \cf{Pal}[8]{Aligned Soul}[Magical]
        While you are dead, you may approach the deity or governing figure of your afterlife and request to be returned to life to continue your mission.
        Travelling to the relevant figure and making the request takes 12 hours.
        Unless there are extenuating circumstances, this request is almost always granted, and you are resurrected in a new body at a location of the entity's choice.
        This functions like the \ritual{resurrection} ritual, except that no part of the body is required, and a new body is created by the entity.
        You can be resurrected in this way regardless of the condition of your body, but not if your soul has been trapped or otherwise prevented from going to the correct afterlife.

    \subsection{Divine Spellcasting}
        This archetype grants you the ability to cast divine spells.
        You must have a starting Willpower of at least 1 to gain this archetype.

        \cf{Pal}[1]{Spellcasting}[Magical]
        Your deity grants you the ability to use divine magic.
        You gain access to one divine \glossterm{mystic sphere} (see \pcref{Divine Mystic Spheres}).
        Each \glossterm{mystic sphere} has a set of \glossterm{spells} associated with it.

        You automatically learn all \glossterm{cantrips} from any mystic sphere you have access to.
        In addition, you learn one 1st level \glossterm{spell} from your chosen \glossterm{mystic sphere}.
        You can also spend \glossterm{insight points} to learn one additional divine spell per \glossterm{insight point}.
        Unless otherwise noted in a spell's description, casting a spell requires a \glossterm{standard action}.

        When you gain access to a new \glossterm{mystic sphere} or spell level,
            you can exchange any number of spells you know for other spells,
            including spells of the higher level.

        Divine spells require \glossterm{verbal components} to cast (see \pcref{Casting Components}).
        For details about mystic spheres and casting spells, see \pcref{Spell and Ritual Mechanics}.

        \cf{Pal}[2]{Mystic Sphere} You gain access to an additional divine \glossterm{mystic sphere}, including all \glossterm{cantrips} from that sphere.
        In addition, you learn an additional divine \glossterm{spell}.

        \cf{Pal}[3]{Spell Level}[Magical] You gain the ability to cast 2nd level divine spells.

        \cf*{Pal}[4]{Spell Level}[Magical] You gain the ability to cast 3rd level divine spells.

        \cf{Pal}[4]{Spell} You gain an additional \glossterm{spell} for any divine \glossterm{mystic sphere} you know.

        \cf*{Pal}[5]{Spell Level}[Magical] You gain the ability to cast 4th level divine spells.

        \cf*{Pal}[6]{Spell Level}[Magical] You gain the ability to cast 5th level divine spells.

        \cf*{Pal}[6]{Spell} You gain an additional \glossterm{spell} for any divine \glossterm{mystic sphere} you know.

        \cf*{Pal}[7]{Spell Level}[Magical] You gain the ability to cast 6th level divine spells.

        \cf*{Pal}[8]{Spell Level}[Magical] You gain the ability to cast 7th level divine spells.

        \cf*{Pal}[8]{Spell} You gain an additional \glossterm{spell} for any divine \glossterm{mystic sphere} you know.

    \subsection{Zealous Warrior}
        This archetype improves your combat prowess, especially against foes who do not share your devoted alignment.

        \cf{Pal}[1]{Smite}[Magical] You can use the \textit{smite} ability as a standard action.
        \begin{freeability}{Smite}
            Make a \glossterm{magical strike}.
            If your target shares your devoted alignment, the strike deals no damage.
            Otherwise, the strike gains a \plus1d bonus to damage.
        \end{freeability}

        \cf{Pal}[2]{Zealous Conviction} You gain a \plus1 bonus to Fortitude and Mental defenses.

        \cf{Pal}[3]{Zealous Power} You gain a \plus1 bonus to \glossterm{power} with \glossterm{physical attacks}.

        \cf{Pal}[4]{Greater Smite}[Magical] The damage bonus from your \textit{smite} ability increases to \plus2d.
        In addition, when you use your \textit{smite} ability, you can use the \textit{pass judgment} ability.
        \begin{attuneability}{Pass Judgment}[\glossterm{Attune} (self)]
            The target of your \textit{smite} is treated as if it had the alignment opposed to your devoted alignment for the purpose of all abilities, including for your initial \textit{smite}.
            This only affects its alignment along the alignment axis your devoted alignment is on.
            For example, if your devoted alignment was evil, a chaotic neutral target would be treated as chaotic good.

            You can use this ability to do battle against foes who share your alignment, but you should exercise caution in doing so.
            Persecution of those who share your ideals can lead you to fall and become an ex-paladin.
        \end{attuneability}

        \cf{Pal}[5]{Greater Zealous Conviction} The bonus from your \textit{zealous conviction} ability increases to \plus2.

        \cf{Pal}[6]{Greater Zealous Power} The bonus from your \textit{zealous power} ability increases to \plus2.

        \cf{Pal}[7]{Supreme Smite}[Magical] The damage bonus from your \textit{smite} ability increases to \plus4d.

        % TODO: Doesn't work with new spell system
        % \cf{Pal}[20]{Martyr's Retribution}[Mag]
        % If you die in the service of your devoted alignment, you may choose to have your fallen body erupt in an immense burst of divine energy.
        % If you do, your body is almost completely consumed, preventing you from being raised with \ritual{resurrection} and similar effects that require an intact body.
        % This burst has two effects.
        % First, a \spell{sunburst} spell immediately takes effect over the area where you died.
        % Second, a \spell{storm of vengeance} spell begins to take effect, centered on the same area.
        % The spell lasts for 10 rounds, and the lightning strikes target the paladin's enemies.
        % Both of these effects harm only the paladin's foes, and do not harm your allies.
        % However, your allies' vision is still impeded by the \spell{storm of vengeance}.

    \subsection{Stalwart Guardian}
        This archetype grants you healing abilities and improves your defensive prowess.

        \cf{Pal}[1]{Lay on Hands}[Magical] You can use the \textit{lay on hands} ability as a standard action.
        \begin{freeability}{Lay on Hands}[Life]
            One \glossterm{ally} within your \glossterm{reach} gains a \plus2 bonus to its most recent \glossterm{wound roll}.
            The \glossterm{wound roll} for that \glossterm{vital wound} cannot be modified again.

            \rankline
            \rank{3} The bonus increases to \plus3.
            \rank{5} When you use this ability, you can spend an \glossterm{action point}.
            If you do, the target's most recent \glossterm{vital wound} is removed instead of having its \glossterm{wound roll} modified.
            \rank{7} The bonus increases to \plus5.
        \end{freeability}

        \cf{Pal}[2]{Enduring Defender}
        You gain an additional \glossterm{hit point}.

        \cf{Pal}[3]{Stalwart Resilience}
        You gain a \plus1 bonus to Fortitude and Mental defense.
        In addition, you gain a \plus1 bonus to \glossterm{wound rolls}.

        \cf{Pal}[4]{Defender's Boon}
        Whenever you are attacked, you gain a \plus1 bonus to \glossterm{accuracy} until the end of the next round.
        This bonus does not stack with itself.

        \cf{Pal}[5]{Greater Enduring Defender}
        You gain an additional \glossterm{hit point}.

        \cf{Pal}[6]{Greater Stalwart Resilience} The bonuses from your \textit{stalwart resilience} ability increase to \plus2.

        \cf{Pal}[7]{Greater Defender's Boon}[Magical]
        The bonus from your \textit{defender's boon} ability increase to \plus2.

    \subsection{Ex-Paladins}
        If you cease to follow your devoted alignment, you lose all \glossterm{magical} paladin class abilities.
        If your atone for your misdeeds and resume the service of your devoted alignment, you can regain your abilities.

\newpage
\section{Ranger}\label{Ranger}
    \begin{dtable!*}
        \lcaption{Ranger Progression}
        \begin{dtabularx}{\textwidth}{c >{\lcol}X >{\lcol}X >{\lcol}X > {\lcol}X}
            \tb{Rank} & \tb{Huntmaster}  & \tb{Keen Senses} & \tb{Wilderness Warrior} & \tb{Monster Slayer} \tableheaderrule
            1    & Quarry & Keen vision        & Wild maneuvers         & Favored enemy
            \\ 2 & Hunting style  & Learned perception & Wilderness lore    & Exotic weaponry
            \\ 3 & Tracking lore  & Blindsense         & Wild power         & Monstrous lore
            \\ 4 & Wary hunter    & Perceive weakness  & Wild maneuver      & Favored enemy
            \\ 5 & Hunting style  & Blindsight         & Wilderness mastery & Monster familiarity
            \\ 6 & Expert tracker & Farsight           & Greater wild power & Experienced defense
            \\ 7 & Fluid hunter   & Truesight          & Wild maneuver      & Greater favored enemy
            \\ 8 & Dual quarry    &                    &
        \end{dtabularx}
    \end{dtable!*}

    \classbasics{Alignment} Any.

    \classbasics{Archetypes} Rangers have the Keen Senses, Wilderness Warrior, Huntmaster, and Monster Slayer \glossterm{archetypes}.

    \subsection{Basic Class Abilities}
        If you are a ranger, you gain the following abilities.

        \cf{Rgr}{Defenses}
        You gain the following bonuses to your \glossterm{defenses}: \plus4 Fortitude, \plus5 Reflex, \plus3 Mental.

        \cf{Rgr}{Skills}
        You have the following \glossterm{class skills}:
        \begin{itemize}
            \item \subparhead{Strength} Climb, Jump, Swim.
            \item \subparhead{Dexterity} Acrobatics, Escape Artist, Ride, Stealth.
            \item \subparhead{Intelligence} Craft, Deduction, Heal, Knowledge (dungeoneering, geography, nature).
            \item \subparhead{Perception} Awareness, Creature Handling, Survival.
            \item \subparhead{Other} Bluff, Intimidate, Persuasion, Profession.
        \end{itemize}

        \cf{Rgr}{Weapon and Armor Proficiencies}
        A ranger is proficient with simple weapons, any two other weapon groups, light and medium armor, and shields.
        You are also proficient with your choice of bows, crossbows, or thrown weapons.

    \subsection{Keen Senses}
        This archetype improves your senses.

        \cf{Rgr}[1]{Keen Vision}
        You reduce your \glossterm{range increment} penalties for attacking at long range by 1.
        In addition, you gain \glossterm{low-light vision}, allowing you to treat sources of light as if they had double their normal illumination range.
        If you already have low-light vision, you double its benefit, allowing you to treat sources of light as if they had four times their normal illumination range.

        \cf{Rgr}[2]{Learned Perception} You gain two additional skill points.
        In addition, you gain a \plus2 bonus to \glossterm{Awareness}.

        \cf{Rgr}[3]{Blindsense}
        Your perceptions are so finely honed that you can sense your enemies without seeing them.
        You gain the \glossterm{blindsense} ability out to 50 feet.
        This ability allows you to sense the presence and location of objects and foes within 50 feet without seeing them.
        If you already have the blindsense ability, you increase its range by 50 feet.

        \cf{Rgr}[4]{Perceive Weakness}
        You gain a \plus1 bonus to \glossterm{accuracy}.

        \cf{Rgr}[5]{Blindsight}
        You gain the \glossterm{blindsight} ability, allowing you to ``see'' perfectly without your eyes in a 50 foot radius around you.
        With this ability, you can fight just as well with your eyes closed as with them open.
        In addition, the range of your \glossterm{blindsense} ability increases by 50 feet.

        \cf{Rgr}[6]{Greater Keen Vision}
        You increase the range of your your \glossterm{blindsense} by 100 feet, and your \glossterm{blindsight} by 50 feet.
        In addition, the penalty reduction from your \textit{keen vision} ability increaes to 2.

        % This may not actually be very strong? Also inconsistent with other ``truesight'' abilities
        \cf{Rgr}[7]{Truesight}[Magical]
        Your perceptions are accurate enough to defeat even powerful magic.
        You can see through normal and magical darkness, see the truth behind visual figments and glamers, and see the true form of creatures and objects affected by \glossterm{Shaping} abilities.
        This ability works at any range.

    % Would be nice to call this ``Hunter'' and the maneuvers ``Hunting'' maneuvers,
    % but that conflicts with the naming for Huntmaster
    \subsection{Wilderness Warrior}
        This archetype grants you abilities to use in combat and improves your wilderness skills.

        \cf{Rgr}[1]{Wild Maneuvers} 
        You can channel your martial prowess into devastating attacks.
        You learn one \glossterm{maneuver} from the wild maneuver list (see \pcref{Wild Maneuvers}).
        You can also spend \glossterm{insight points} to learn one additional \glossterm{maneuver} per \glossterm{insight point}.
        As a \glossterm{standard action}, you can use any \glossterm{maneuver} you know.

        \cf{Rgr}[2]{Wilderness Lore} You gain two additional skill points.

        \cf{Rgr}[3]{Wild Power} You gain a \plus1 bonus to \glossterm{power} with \glossterm{mundane} abilities.

        \cf{Rgr}[4]{Wild Maneuver}
        You learn an additional \textit{wild maneuver}.

        \cf{Rgr}[5]{Wilderness Mastery} You gain a \plus2 bonus to the Creature Handling, Heal, Knowledge (geography), Knowledge (nature), Ride, and Survival skills.

        \cf{Rgr}[6]{Greater Wild Power} The bonus from your \textit{wild power} ability increases to \plus2.

        \cf*{Rgr}[7]{Wild Maneuver}
        You learn an additional \textit{wild maneuver}.

    \subsection{Huntmaster}
        This archetype grants you and your allies abilities to hunt down specific foes.

        \cf{Rgr}[1]{Quarry}\label{Quarry} You can use the \textit{quarry} ability as a \glossterm{minor action}.
        \begin{attuneability}{Quarry}[\glossterm{Attune} (self)]
            Choose a creature within \rnglong range.
            The target becomes your quarry.
            You and your \glossterm{allies} within the same range are called your hunting party.
            Your hunting party gains a \plus1 bonus to \glossterm{accuracy} with \glossterm{physical attacks} against your quarry.
        \end{attuneability}

        \cf{Rgr}[2]{Hunting Style}
        You learn specific hunting styles to defeat particular quarries.
        Choose one hunting style from the list below.
        You can also spend \glossterm{insight points} to learn one additional \textit{hunting style} per \glossterm{insight point}.
        When you use your \textit{quarry} ability, you may also use one of your \textit{hunting styles}.
        Each \textit{hunting style} ability lasts as long as the \textit{quarry} ability you used it with.
        {
            \begin{freeability}{Anchoring}[Magical]
                As long as your quarry is \glossterm{threatened} by at least two members of your hunting party, it cannot travel extradimensionally.
                This prevents all \glossterm{Manifestation}, \glossterm{Planar}, and \glossterm{Teleportation} effects.

                \rankline
                \rank{4} This effect instead applies if your quarry is within \rngmed range of at least two members of your hunting party.
                \rank{6} This effect instead applies if your quarry is within \rnglong range of at least two members of your hunting party.
                \rank{8} This effect instead applies if your quarry is within \rnglong range of any member of your hunting party.
            \end{freeability}

            \begin{freeability}{Brutal Assault}
                The accuracy bonus from your \textit{quarry} ability is replaced with a \minus1 penalty to accuracy with \glossterm{mundane} attacks against your quarry.
                In exchange, your hunting party gains a \plus2 bonus to \glossterm{power} with \glossterm{mundane} attacks against your quarry.

                \rankline
                \rank{4} The accuracy penalty is removed.
                \rank{6} The power bonus applies to both \glossterm{mundane} and \glossterm{magical} attacks against your quarry.
                \rank{8} The power bonus is increased to \plus4.
            \end{freeability}

            \begin{freeability}{Coordinated Stealth}
                Your quarry takes a \minus4 penalty to Awareness checks to notice members of your hunting party.

                \rankline
                % seems weak?
                \rank{4} Your hunting party gains a \plus2 bonus to \glossterm{power} against your quarry if it is \unaware of every member of the hunting party.
                \rank{6} The Awareness penalty increases to \minus8.
                \rank{8} The power bonus increases to \plus4.
            \end{freeability}

            \begin{freeability}{Cover Weaknesses}
                The accuracy bonus against your quarry is replaced with a \plus1 bonus to defenses against your quarry's attacks.

                \rankline
                \rank{4} Your hunting party gains a bonus equal to half your level to \glossterm{resistances} against your quarry's attacks.
                \rank{6} The resistance bonus increases to be equal to your level.
                \rank{8} The defense bonus increases to \plus2.
            \end{freeability}

            \begin{freeability}{Decoy}
                If you \glossterm{threaten} your quarry, it takes a \minus2 penalty to accuracy on attacks against members of your hunting party other than you.

                \rankline
                \rank{4} The penalty increases to \minus3.
                \rank{6} The penalty increases to \minus4.
                \rank{8} The penalty increases to \minus5.
            \end{freeability}

            \begin{freeability}{Lifeseal}[Magical]
                As long as your quarry is \glossterm{threatened} by at least two members of your \glossterm{hunting party}, it cannot regain \glossterm{hit points}.

                \rankline
                \rank{4} This effect instead applies if the target is within \rngmed range of at least two members of your hunting party.
                \rank{6} This effect instead applies if your quarry is within \rnglong range of at least two member of your hunting party.
                \rank{8} This effect instead applies if your quarry is within \rnglong range of any member of your hunting party.
            \end{freeability}

            \begin{freeability}{Martial Suppression}
                As long as your quarry is \glossterm{threatened} by at least two members of your hunting party, it takes a \minus1 penalty to accuracy with \glossterm{physical attacks}.

                \rankline
                \rank{4} The penalty increases to \minus2.
                \rank{6} The penalty increases to \minus3.
                \rank{8} The penalty increases to \minus4.
            \end{freeability}

            \begin{freeability}{Mystic Guidance}[Magical]
                The accuracy bonus from your \textit{quarry} ability applies to all attacks your hunting party makes against your quarry, instead of only to \glossterm{physical attacks}.

                \rankline
                \rank{4} Your hunting party gains a \plus1 bonus to Fortitude, Reflex, and Mental defenses against attacks from your quarry.
                \rank{6} The accuracy bonus against your quarry is increased to \plus2.
                \rank{8} The defense bonuses increase to \plus2.
            \end{freeability}

            \begin{freeability}{Mystic Suppression}
                As long as your quarry is \glossterm{threatened} by at least two members of your hunting party, it takes a \minus1 penalty to \glossterm{accuracy} with \glossterm{magical} attacks.

                \rankline
                \rank{4} The penalty increases to \minus2. 
                \rank{6} The penalty increases to \minus3.
                \rank{8} The penalty increases to \minus4.
            \end{freeability}

            \begin{freeability}{Solo Hunter}
                Your hunting party other than you gains no benefit from your \textit{quarry} ability.
                In exchange, you gain a \plus1 bonus to defenses against your quarry.

                \rankline
                \rank{4} The accuracy bonus from your \textit{quarry} ability increases to \plus2.
                \rank{6} The defense bonus increases to \plus2.
                \rank{8} The accuracy bonus increases to \plus3.
            \end{freeability}

            \begin{freeability}{Swarm Hunter}
                When you use your \textit{quarry} ability, you can target any number of creatures to be your quarry.

                \rankline
                \rank{4} You gain a \plus1 bonus to \glossterm{overwhelm resistance}.
                \rank{6} The bonus to overwhelm resistance applies to all members of your hunting party.
                \rank{8} The bonus to \glossterm{overwhelm resistance} increases to \plus2.
            \end{freeability}

            \begin{freeability}{Unerring}
                Your hunting party rolls twice and takes the better result for miss chances on attacks against your quarry, such as from \glossterm{active cover}.

                \rankline
                \rank{4} Your hunting party also ignores the defense bonus from \glossterm{cover} on attacks against your quarry.
                \rank{6} Your hunting party ignores all miss chances on attacks against your quarry.
                \rank{8} The accuracy bonus from your \textit{quarry} ability increases to \plus2.
            \end{freeability}

            \begin{freeability}{Wolfpack}
                At the start of each \glossterm{phase}, if your quarry is \glossterm{threatened} by at least two members of your hunting party, it is \glossterm{slowed} until the end of that phase.

                \rankline
                \rank{4} This effect instead applies if your quarry is within \rngmed range of at least two members of your hunting party.
                \rank{6} This effect instead applies if your quarry is within \rnglong range of at least two members of your hunting party.
                \rank{8} This effect instead applies if your quarry is within \rnglong range of any member of your hunting party.
            \end{freeability}
        }

        \cf{Rgr}[3]{Tracking Lore}
        You gain two additional \glossterm{skill points}.

        \cf{Rgr}[4]{Wary Hunter} You gain a \plus1 bonus to all \glossterm{defenses} against your quarry.

        \cf*{Rgr}[5]{Hunting Style}
        You learn an additional \textit{hunting style}.

        \cf{Rgr}[6]{Expert Tracker} You gain a \plus4 bonus to the Survival skill.
        In addition, you may use your \textit{quarry} ability on any creature whose tracks you are following, regardless of the creature's current location.

        \cf{Rgr}[7]{Fluid Hunter}
        At the start of each round, you can change which \textit{hunting style} you are using.
        This does not change the target of your \textit{quarry} ability.

        \cf{Rgr}[8]{Dual Quarry} You can attune to two separate uses of your \textit{quarry} ability simultaneously.
        You can use different \textit{hunting style} abilities with each \textit{quarry}.

    \subsection{Monster Slayer}
        This archetype grants you abilities improving your ability to slay fearsome monsters.

        \cf{Rgr}[1]{Favored Enemy} Choose one of the following creature types: aberration, animate, animal, magical beast, monstrous humanoid, outsider, or undead.
        You can also spend \glossterm{insight points} to choose one additional creature type per \glossterm{insight point}.
        You gain a \plus1 bonus to \glossterm{accuracy} against creatures of any type you chose.

        \cf{Rgr}[2]{Exotic Weaponry} You gain proficiency with \glossterm{exotic weapons} from all weapon groups that you are proficient with.

        \cf{Rgr}[3]{Monstrous Lore} You gain two additional \glossterm{skill points}.

        \cf*{Rgr}[4]{Favored Enemy} You choose an additional creature type for your \textit{favored enemy} ability.

        \cf{Rgr}[5]{Monster Familiarity} You gain a \plus2 bonus to Knowledge checks to identify monsters.
        This bonus is doubled against creatures you chose for your \textit{favored enemy} ability.

        \cf{Rgr}[6]{Experienced Defense} You gain a \plus1 bonus to Armor defense.

        \cf{Rgr}[7]{Greater Favored Enemy} The bonus from your \textit{favored enemy} ability increases to \plus2.

    % \subsection{Boundary Warden}
    %     This archetype grants you abilities improving your ability to guard the boundaries between civilization and nature.

    %     \cf{Rgr}[2]{Favored Terrain} Choose one of the following terrain types: 
    %     You can also spend \glossterm{insight points} to choose one additional creature type per \glossterm{insight point}.
    %     You gain a \plus1 bonus to all \glossterm{defenses} and a \plus10 foot bonus to land speed while in any terrain you chose.

\newpage
\section{Rogue}\label{Rogue}
    \begin{dtable!*}
        \lcaption{Rogue Progression}
        \begin{dtabularx}{\textwidth}{c >{\lcol}X >{\lcol}X >{\lcol}X >{\lcol}X}
            \tb{Rank} & \tb{Assassin} & \tb{Jack of All Trades}  & \tb{Combat Trickster} & \tb{Suave Scoundrel} \tableheaderrule
               1 & Sneak attack         & Skill lore             & Tricky manuevers         & Glibness
            \\ 2 & Evasion              & Skill exemplar         & Trick lore          & Confound
            \\ 3 & Darkstalker          & Skill lore             & Trick power         & Suave lore
            \\ 4 & Greater sneak attack & Greater skill exemplar & Trick manuever      & Compel belief
            \\ 5 & Assassinate, evasion & Skillful defense       &                     & Greater confound
            \\ 6 & Hide in plain sight  & Supreme skill exemplar & Greater trick power & Greater glibness
            \\ 7 & Supreme sneak attack & Legendary fortune      & Trick maneuver      & Greater compel belief
            \\ 8 & Greater darkstalker  &                        &
        \end{dtabularx}
    \end{dtable!*}

    \classbasics{Alignment} Any.

    \classbasics{Archetypes} Rogues have the Assassin, Jack of All Trades, Combat Trickster, and Suave Scoundrel \glossterm{archetypes}.

    \subsection{Basic Class Abilities}
        If you are a rogue, you gain the following abilities.

        \cf{Rog}{Defenses}
        You gain the following bonuses to your \glossterm{defenses}: \plus3 Fortitude, \plus5 Reflex, \plus4 Mental.

        \cf{Rog}{Skills}
        You have the following \glossterm{class skills}:
        \begin{itemize}
            \item \subparhead{Strength} Climb, Jump, Swim.
            \item \subparhead{Dexterity} Acrobatics, Escape Artist, Sleight of Hand, Stealth.
            \item \subparhead{Intelligence} Craft, Deduction, Devices, Disguise, Knowledge (dungeoneering, local), Linguistics.
            \item \subparhead{Perception} Awareness, Sense Motive.
            \item \subparhead{Other} Bluff, Intimidate, Perform, Persuasion, Profession.
        \end{itemize}

        \cf{Rog}{Weapon and Armor Proficiencies}
        Rogues are proficient with simple weapons, any two other weapon groups, light armor, and bucklers.
        They are also proficient with saps.

    \subsection{Assassin}
        This archetype improves your agility, stealth, and combat prowess against unaware targets.

        \cf{Rog}[1]{Sneak Attack} You can use the \textit{sneak attack} ability as a standard action.
        \begin{freeability}{Sneak Attack}
            Make a \glossterm{strike} against a creature within \rngclose range.
            If the target is \unaware, \defenseless, or \glossterm{overwhelmed}, you gain a \plus2d bonus to damage with the strike.
            You do not gain this damage bonus against creatures who are immune to \glossterm{critical hits} or who lack a discernible body structure, such as oozes.
        \end{freeability}

        % TODO: wording
        \cf{Rog}[2]{Evasion} When you are attacked by an ability that affects an area, you can use your Reflex defense in place of any other defenses against that attack.

        \cf*{Rog}[3]{Darkstalker}[Magical] You can use the \textit{darkstalker} ability as a standard action.
        \begin{attuneability}{Darkstalker}[\glossterm{Attune} (self)]
            You become completely undetectable by your choice of one of the following senses:
            \begin{itemize}
                \item Blindsense and blindsight
                \item Darkvision
                \item Lifesense and lifesight
                \item Scent
                \item Tremorsense and tremorsight
            \end{itemize}
        \end{attuneability}

        \cf{Rog}[4]{Greater Sneak Attack} The damage bonus from your \textit{sneak attack} ability increases to \plus3d.

        \cf{Rog}[5]{Assassination} You can use the \textit{assassination} ability as a \glossterm{minor action}.
        \begin{freeability}{Assassination}[\glossterm{Swift}]
            You study a creature within \rngmed range, finding weak points you can take advantage of.
            Until the end of the next round, if you make a melee \glossterm{strike} against the target while it is \unaware, your attack deals maximum damage.
        \end{freeability}

        % TODO: wording?
        \cf{Rog}[6]{Hide in Plain Sight} You can use the Stealth skill to hide while observed.
        Creatures observing you while you try to hide gain a \plus5 bonus to checks to notice you.
        You must still have cover or concealment to hide successfully.

        \cf{Rog}[7]{Supreme Sneak Attack} The damage bonus from your \textit{sneak attack} ability increases to \plus4d.

        \cf{Rog}[8]{Greater Darkstalker}[Magical] When you use your \textit{darkstalker} ability, you become undetectable by all of the listed senses, not just one.

    \subsection{Jack of All Trades}
        This archetype improves your skills.

        \cf{Rog}[1]{Skill Lore} You gain three additional skill points.

        \cf{Rog}[2]{Skill Exemplar} You gain a \plus1 bonus to all skills.

        \cf*{Rog}[3]{Skill Lore} You gain three additional skill points.

        \cf{Rog}[4]{Greater Skill Exemplar} The skill bonus from your \textit{skill exemplar} ability increases to \plus2.

        \cf{Rog}[5]{Skillful Defense} You gain a \plus1 bonus to all defenses.

        \cf{Rog}[6]{Supreme Skill Exemplar} The skill bonus from your \textit{skill exemplar} ability increases to \plus3.

        \cf{Rog}[8]{Legendary Fortune} When you make a skill attack or check, you roll twice and take either result.

    \subsection{Combat Trickster}
        This archetype grants you abilities to use in combat and improves your combat prowess.

        \cf{Rog}[1]{Trick Maneuvers}
        You can confuse and confound your foes in combat.
        You learn one \glossterm{maneuver} from the trick maneuver list (see \pcref{Trick Maneuvers}).
        You can also spend \glossterm{insight points} to learn one additional trick \glossterm{maneuver} per \glossterm{insight point}.
        As a \glossterm{standard action}, you can use any \glossterm{maneuver} you know.

        \cf{Rog}[2]{Trick Lore} You gain two additional skill points.

        \cf{Rog}[3]{Trick Power} You gain a \plus1 bonus to \glossterm{power} with \glossterm{mundane} abilities.

        \cf{Rog}[4]{Trick Maneuver}
        You learn an additional \textit{trick maneuver}.

        % \cf{Rog}[5]{Skill Trick} % TODO

        \cf{Rog}[6]{Greater Trick Power} The bonus from your \textit{trick power} ability increases to \plus2.

        \cf*{Rog}[7]{Trick Maneuver}
        You learn an additional \textit{trick maneuver}.

    \subsection{Suave Scoundrel}
        This archetype improves your social manipulation abilities.

        \cf{Rog}[1]{Glibness} You gain a \plus2 bonus to the Bluff, Intimidate, and Persuasion skills.

        \cf{Rog}[2]{Confound} You can use the \textit{confound} ability as a standard action.
        \begin{freeability}{Confound}[\glossterm{Compulsion}, \glossterm{Magical}]
            Make a attack vs. Mental against a creature within \rngclose range.
            You can choose to use your Bluff skill to attack in place of your normal \glossterm{accuracy}.
            \hit As a \glossterm{condition}, the target is either \glossterm{dazed} or \glossterm{disoriented}, as you choose.
        \end{freeability}

        \cf{Rog}[3]{Suave Lore} You gain two additional \glossterm{skill points}.

        \cf{Rog}[4]{Compel Belief} You can use the \textit{compel belief} ability as a standard action.
        \begin{freeability}{Compel Belief}[\glossterm{Compulsion}, \glossterm{Magical}, \glossterm{Sustain} (minor)]
            Make an attack vs. Mental against a creature within \rngmed range.
            You can use your Persuasion skill to attack in place of your normal \glossterm{accuracy}.
            You must also choose a belief that the target has.
            The belief may be a lie that you told it, or even a simple misunderstanding (such as believing a hidden creature is not present in a room).
            If the creature does not already hold the chosen belief, this ability automatically fails.
            \hit The target continues to maintain the chosen belief, regardless of any evidence to the contrary.
            It will interpret any evidence that the falsehood is incorrect to be somehow wrong -- an illusion, a conspiracy to decieve it, or any other reason it can think of to continue believing the falsehood.
            At the end of the effect, the creature can decide whether it believes the falsehood or not, as normal.
        \end{freeability}

        \cf{Rog}[5]{Greater Confound} When you use the \textit{confound} ability, you can make the target \glossterm{stunned} instead of dazed or disoriented.

        \cf{Rog}[6]{Greater Glibness} The bonus from your \textit{glibness} ability increases to \plus4.

        \cf{Rog}[7]{Greater Compel Belief} You can use your \textit{compel belief} ability as a \glossterm{minor action}.

    % \subsection{Bard}
    %     This archetype grants you the ability to inspire your allies and impair your foes with magical performances.

    %     \cf{Rog}{1}{Bardic Performances}
    %     Choose two \textit{bardic performances} from the list below.
    %     As a standard action, you can use a \textit{bardic performance}.
    %     All \textit{bardic performances} have the \glossterm{Sustain} (minor) tag.

    %     When you use a \textit{bardic performance} ability, you begin a performance using one of your Perform skills.
    %     Depending on the nature of your performance, the ability may have the \glossterm{Auditory} or \glossterm{Visual} tags.
    %     If a target can neither see nor hear your performance, the effect immediately ends for that target.
    %     {
    %         \begin{freeability}{Mesmerizing Performance}[\glossterm{Delusion}, \glossterm{Mind}]
    %             Make an attack vs. Mental against up to five creatures within \rngmed range.
    %             \hit Each target is \fascinated by you.
    %             Any act by you or your apparent allies that damages a target or that causes it to feel that it is in danger breaks the effect for that creature.
    %             An observant target may interpret overt threats to its allies as a threat to itself.

    %             \rankline
    %             \rank{3} 

    %         \end{freeability}

    %         \begin{freeability}{Inspiring Performance}[\glossterm{Delusion}, \glossterm{Mind}]
    %             Choose a willing creature within \rngmed range.
    %             The target gains a \plus1 \glossterm{magic bonus} to \glossterm{accuracy} and a \plus2 \glossterm{magic bonus} to \glossterm{checks}.
    %         \end{freeability}
    %     }

\newpage
\section{Warlock}\label{Warlock}
    \begin{dtable}
        \lcaption{Warlock Progression}
        \begin{dtabularx}{\columnwidth}{c >{\lcol}X >{\lcol}X >{\lcol}X}
            \tb{Rank} & \tb{Pact Spellcasting} & \tb{Pact Spell Mastery}  & \tb{Blessings of the Abyss} \tableheaderrule
            1    & Spellcasting  & Armor tolerance             & Eldritch blast
            \\ 2 & Mystic sphere   & Spell knowledge             & Eldritch augment
            \\ 3 & Spell level (2) & Wellspring of power         & Empowering Whispers
            \\ 4 & Spell level (3), spell & Spell knowledge             & Fiendish resistance
            \\ 5 & Spell level (4) & Greater armor tolerance     & Eldritch Augment
            \\ 6 & Spell level (5) & Spell knowledge             & Greater Empowering Whispers
            \\ 7 & Spell level (6), spell & Greater wellspring of power & Greater eldritch augment
            \\ 8 & Spell level (7) &                             &
        \end{dtabularx}
    \end{dtable}

    \classbasics{Alignment} Any.

    \classbasics{Archetypes} Warlocks have the Pact Spellcasting, Pact Spell Mastery, and Blessings of the Abyss \glossterm{archetypes}.

    \subsection{Basic Class Abilities}
        If you are a warlock, you gain the following abilities.

        \cf{War}{Defenses}
        You gain the following bonuses to your \glossterm{defenses}: \plus4 Fortitude, \plus3 Reflex, \plus5 Mental.

        \cf{War}{Skills}
        You have the following \glossterm{class skills}:
        \begin{itemize}
            \item \subparhead{Dexterity} Ride.
            \item \subparhead{Intelligence} Craft, Deduction, Disguise, Knowledge (arcana, planes, religion), Linguistics.
            \item \subparhead{Perception} Awareness, Sense Motive, Spellcraft.
            \item \subparhead{Other} Bluff, Intimidate, Persuasion, Profession.
        \end{itemize}

        \cf{War}{Weapon and Armor Proficiencies}
        You are proficient with simple weapons, any two other weapon groups, light and medium armor, and shields.
        Encumbrance from armor interferes with the gestures you make to cast spells, which can cause your spells with somatic components to fail (see \pcref{Somatic Component Failure}).

        \cf{War}{Infernal Pact}[Magical]
        To become a warlock, you must make a pact with a powerful demon or devil.
        % TODO: wording; intended to prevent Null warlocks who have no pact
        If you somehow lose this ability, you lose all other warlock abilities.
        You must make a dark sacrifice, the details of which are subject to negotiation, and offer a part of your immortal soul.
        In exchange, you gain the powers of a warlock.
        The creature you make the pact with is called your soulkeeper.

        Offering your soul to an entity in this way grants it the ability to communicate with you in limited ways.
        This communication typically manifests as unnatural emotional urges or whispered voices audible only to you.

        Your pact specifies how much of your soul is granted to your soulkeeper, and the circumstances of the transfer.
        The most common arrangement is for a soulkeeper to gain possession of your soul immediately after you die.
        It will keep the soul for one decade per year of your life that you spend as a warlock.
        During that time, it will not prevent you from being resurrected.
        At the end of that time, if your soul remains intact, your soul will pass on to its intended afterlife.
        However, other arrangements are possible, and each warlock's pact can be unique.

        The longer you spend in an afterlife that is not your own, the more likely you are to lose your sense of self and become subsumed by the plane you are on.
        Only a soul of extraordinary strength can maintain its integrity after decades or centuries in the Abyss.
        Many warlocks seek power zealously while mortal to gain the mental fortitude necessary to keep their soul after death.

        \cf{War}{Whispers of the Lost}[Magical]
        You hear the voices of souls lost to the Abyss, linked to you through your soulkeeper.
        Choose one of the following types of whispers that you hear.
        {
            \subcf{Mentoring Whispers} You hear the voice of a dead warlock whose soul is bound to the same soulkeeper as yours.

            % \subcf{Spiteful Whispers} You hear the voices of cruel souls who berate you for your flaws and mistakes.

            \subcf{Sycophantic Whispers} You hear the voices of adoring souls who praise your talents and everything you do.

            \subcf{Warning Whispers} You hear the voices of paranoid and fearful souls warning you of danger, both real and imagined.

            \subcf{Whispers of the Mighty} Your soulkeeper forges the connection to your soul into a boon granted to any soul in the Abyss strong enough to claim it in battle.
            You hear the voice of whatever soul currently possesses the boon, which may change suddenly and unexpectedly.
        }

    \subsection{Pact Spellcasting}
        This archetype grants you the ability to cast pact spells.
        You must have a starting Willpower of at least 1 to gain this archetype.

        \cf{War}[1]{Spellcasting}[Magical]
        Your deity grants you the ability to use pact magic.
        You gain access to one pact \glossterm{mystic sphere} (see \pcref{Pact Mystic Spheres}).
        Each \glossterm{mystic sphere} has a set of \glossterm{spells} associated with it.

        You automatically learn all \glossterm{cantrips} from any mystic sphere you have access to.
        In addition, you learn one 1st level \glossterm{spell} from your chosen \glossterm{mystic sphere}.
        You can also spend \glossterm{insight points} to learn one additional pact spell per \glossterm{insight point}.
        Unless otherwise noted in a spell's description, casting a spell requires a \glossterm{standard action}.

        When you gain access to a new \glossterm{mystic sphere} or spell level,
            you can exchange any number of spells you know for other spells,
            including spells of the higher level.

        Pact spells require \glossterm{verbal components} to cast (see \pcref{Casting Components}).
        For details about mystic spheres and casting spells, see \pcref{Spell and Ritual Mechanics}.

        \cf{War}[2]{Mystic Sphere} You gain access to an additional pact \glossterm{mystic sphere}, including all \glossterm{cantrips} from that sphere.
        In addition, you learn an additional pact \glossterm{spell}.

        \cf{War}[3]{Spell Level}[Magical] You gain the ability to cast 2nd level pact spells.

        \cf*{War}[4]{Spell Level}[Magical] You gain the ability to cast 3rd level pact spells.

        \cf{War}[4]{Spell} You gain an additional \glossterm{spell} for any pact \glossterm{mystic sphere} you know.

        \cf*{War}[5]{Spell Level}[Magical] You gain the ability to cast 4th level pact spells.

        \cf*{War}[6]{Spell Level}[Magical] You gain the ability to cast 5th level pact spells.

        \cf*{War}[7]{Spell Level}[Magical] You gain the ability to cast 6th level pact spells.

        \cf*{War}[7]{Spell} You gain an additional \glossterm{spell} for any pact \glossterm{mystic sphere} you know.

        \cf*{War}[8]{Spell Level}[Magical] You gain the ability to cast 7th level pact spells.

        % {
        %     \augment{0}{Infernal Focus} You gain a \plus10 bonus to \glossterm{concentration} checks to cast the spell (see \pcref{Concentration}).

        %     \augment{0}{Sickening} Make an attack vs. Fortitude against one creature targeted by the spell.
        %     On a hit, the target is \glossterm{sickened} as a \glossterm{condition}.
        %     \par This augment can be applied to any spell that targets at least one creature.

        %     \augment{0}{Soulrending} When a creature is dealt damage by the spell, if it has no hit points remaining, it dies.
        %     \par This augment can be applied to any spell that deals damage.

        %     \augment{0}{Terrifying} Make an attack vs. Mental against one creature targeted by the spell.
        %     On a hit, the target is \glossterm{shaken} by you as a \glossterm{condition}.
        %     \par This augment can be applied to any spell that targets at least one creature.

        %     \augment{1}{Dark Expansion} The spell's area is increased by one step, to a maximum of \areahuge.
        %     The steps are, in order: \areasmall, \areamed, \arealarge, and \areahuge.
        %     Normally, a Small or Medium line is 5 ft.\ wide, while a Large or Huge line is 10 ft.\ wide.
        %     A line used to define a wall does not have a width.
        %     \par This augment can be applied to any spell with an area that is one of the above areas.

        %     \augment{1}{Untraceable} You do not need \glossterm{verbal components} or \glossterm{somatic components} to cast the spell.

        %     \augment{2}{Abyssal Flame} Non-physical damage dealt by the spell becomes life damage in place of its other damage types.
        %     In addition, you gain a \plus1d bonus to life damage dealt by the spell.
        %     This does not change any other aspects of the spell's effects.
        %     \par This augment can be applied to any spell that deals non-physical damage.

        %     \augment{2}{Bloodreaving} When a creature takes damage from the spell, you heal hit points equal to the damage dealt.
        %     \par This augment can be applied to any spell that deals damage.

        %     % TODO: taking damage multiple times in the same round abuse?
        %     \augment{2}{Exsanguinating} When a creature takes damage from the spell, it begins bleeding as a \glossterm{condition}.
        %     At the end of each \glossterm{action phase}, a bleeding creature takes \glossterm{standard damage} \minus1d.
        %     \par This augment can be applied to any spell that deals damage.

        %     \augment{3}{Inevitable Doom} You gain a \plus2 bonus to \glossterm{accuracy} with the spell.
        %     \par This augment can be applied to any spell that makes an attack.

        %     \augment{3}{Infernal Potency} You gain a \plus4 bonus to \glossterm{power} with the spell.
        %     \par This augment can be applied to any spell.

        %     \augment{4}{Dark Miasma} When a creature takes damage from the spell, it becomes \glossterm{sickened} as a \glossterm{condition}.
        %     \par This augment can be applied to any spell that deals damage.

        %     \augment{4}{Unholy Terror} When a creature takes damage from the spell, it becomes \glossterm{shaken} by you as a \glossterm{condition}.
        %     \par This augment can be applied to any spell that deals damage.

        %     \augment{5}{Soulclaiming} When a creature takes damage from the spell, its soul becomes marked by your soulkeeper as a \glossterm{condition}.
        %     It takes a \minus4 penalty to Mental defense.
        %     If it dies, its soul is immediately claimed by your soulkeeper, preventing it from being resurrected by any means.

        %     \augment{6}{Greater Inevitable Doom} You gain a \plus4 bonus to \glossterm{accuracy} with the spell.
        %     \par This augment can be applied to any spell that makes an attack.

        %     \augment{6}{Greater Infernal Potency} You gain a \plus8 bonus to \glossterm{power} with the spell.
        %     \par This augment can be applied to any spell.
        % }

    \subsection{Pact Spell Mastery}
        This archetype improves your ability to cast spells with the power of your dark pact.
        You must have the Pact Spellcasting archetype to gain the abilities from this archetype.

        \cf{War}[1]{Armor Tolerance} You reduce your \glossterm{encumbrance} by 2 when determining your \glossterm{somatic component failure}.

        \cf{War}[2]{Spell Knowledge} You learn an additional pact \glossterm{spell}.

        \cf{War}[3]{Wellspring of Power}[Magical]
        You gain a \plus1 bonus to \glossterm{power} with pact spells.

        \cf*{War}[4]{Spell Knowledge} You learn an additional pact \glossterm{spell}.

        \cf{War}[5]{Greater Armor Tolerance}[Magical] The penalty reduction from your \textit{armor tolerance} ability increases to 3.

        \cf*{War}[6]{Spell Knowledge} You learn an additional pact \glossterm{spell}.

        \cf{War}[7]{Greater Wellspring of Power}[Magical]
        The bonus from your \textit{wellspring of power} ability increases to \plus2.

    \subsection{Blessings of the Abyss}
        This archetype grants you powers relating to the connection to the Abyss granted by your pact.

        \cf{War}[1]{Eldritch Blast} You can use the \textit{eldritch blast} ability as a standard action.
        \begin{freeability}{Eldritch Blast}[\glossterm{Magical}]
            Make an attack vs. Armor against one creature or object within \rngclose range.
            \hit The target takes \glossterm{standard damage}.
        \end{freeability}

        \cf{War}[2]{Eldritch Augment} You gain the ability to increase the power of your \textit{eldritch blast}.
        You learn one eldritch augment from the list below.
        You can also spend \glossterm{insight points} to learn one additional eldritch augment per \glossterm{insight point}.
        Whenever you use your \textit{eldritch blast} ability, you may also use one eldritch augment to increase the power of the \textit{eldritch blast}.
        {
            \parhead{Bloodfeeding} If a target loses a \glossterm{hit point} from the spell, you gain a \plus2 bonus to your most recent \glossterm{wound roll}.
            The \glossterm{wound roll} for that \glossterm{vital wound} cannot be modified again.

            \parhead{Chaining} You can also target an additional creature or object within range.
            You take a \minus2 penalty to \glossterm{accuracy} against that target.

            \parhead{Empowered} You gain a \plus2 bonus to \glossterm{power}.

            \parhead{Exsanguinating} If your attack result beats the target's Fortitude defense, it begins bleeding as a \glossterm{condition}.
            At the end of each \glossterm{action phase} in subsequent rounds, it takes \glossterm{standard damage} \minus3d.

            \parhead{Extended} The range of the ability increases to \rngmed.

            \parhead{Precise} You gain a \plus1 bonus to \glossterm{accuracy}.

            \parhead{Sickening} If your attack result beats the target's Fortitude defense, it is \glossterm{sickened} as a \glossterm{condition}.

            \parhead{Fearful} If your attack result beats the target's Mental defense, it is \glossterm{shaken} by you as a \glossterm{condition}.
        }

        \cf{War}[3]{Empowering Whispers}[Magical]
        You gain an ability based on the type of whispers you hear with your \textit{whispers of the lost} ability.
        {
            \subcf{Mentoring Whispers} You gain two additional \glossterm{skill points}.

            % \subcf{Spiteful Whispers} 

            \subcf{Sycophantic Whispers} You are immune to \glossterm{Compulsion} abilities.

            \subcf{Warning Whispers} You are not \glossterm{unaware} when attacked from surprise.

            \subcf{Whispers of the Mighty} You are immune to \glossterm{Poison} and \glossterm{Disease} abilities.
        }

        \cf{War}[4]{Fiendish Resistance} You gain a bonus equal to your level to \glossterm{resistances} against \glossterm{energy damage}.

        \cf*{War}[5]{Eldritch Augment} You learn an additional \textit{eldritch augment}.
        In addition, you can also choose from the options on the list below.
        When you gain this ability, you can exchange any other eldritch augments you know for eldritch augments on this list.
        {
            \parhead{Greater Chaining} You can also target up to five additional creatures or objects within range.
            You take a \minus2 penalty to \glossterm{accuracy} against each of those targets.

            \parhead{Greater Bloodfeeding} If a target loses a \glossterm{hit point} from the spell, you gain a \plus4 bonus to your most recent \glossterm{wound roll}.
            The \glossterm{wound roll} for that \glossterm{vital wound} cannot be modified again.

            \parhead{Greater Empowered} You gain a \plus4 bonus to \glossterm{power}.

            \parhead{Greater Extended} The range of the ability increases to \rnglong.

            \parhead{Greater Precise} You gain a \plus2 bonus to \glossterm{accuracy}.

            \parhead{Nauseating} If your attack result beats the target's Fortitude defense, it is \glossterm{nauseated} as a \glossterm{condition}.

            \parhead{Terrifying} If your attack result beats the target's Mental defense, it is \glossterm{frightened} by you as a \glossterm{condition}.
        }

        \cf{War}[6]{Greater Empowering Whispers}[Magical] You gain an additional ability depending on the voices you chose with your \textit{whispers of the lost} ability.
        {
            \subcf{Mentoring Whispers} You gain a \plus1 bonus to \glossterm{power} with \glossterm{magical} abilities.

            \subcf{Sycophantic Whispers} You gain a \plus2 bonus to Mental defense.

            \subcf{Warning Whispers} You gain a \plus2 bonus to Reflex defense.

            \subcf{Whispers of the Mighty} You gain a \plus2 bonus to Fortitude defense.
        }

        \cf{War}[7]{Greater Eldritch Augment}[Magical] You can apply an additional \textit{eldritch augment} to your \textit{eldritch blast}.
