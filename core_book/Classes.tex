\chapter{Classes}\label{Classes}

Your character's class represents the things your character has chosen to train in.
This choice determines a great deal about your character's abilities.

\section{How Classes Work}
    When you first create a character, you choose a class.
    Your character has one level in that class.
    This grants your character all of the abilities your chosen class grants at 1st level, as given in the class description.
    Each time your character gains a level, you can choose to increase your level in your original class or gain a level in a new class.
    This grants your character all of the abilities your chosen class grants at the level your character just gained in it.

    \subsection{Base Classes}
        Every character has one \glossterm{base class}.
        You may choose any class your character has at least one level in as a base class.
        Whenever your character gain a level, you can change its base class to a different class it has.
        Your choice of base class affects your character's \glossterm{combat prowess}, \glossterm{defenses}, \glossterm{skill points}, and \glossterm{class skills}.
        In addition, every class grants special abilities if it is chosen as a base class, as given in the class description.

\section{Class Introductions}

    There are eleven classes in Rise.
    \begin{itemize}
        \item Barbarians are mighty warriors who can enter a deadly battlerage.
        \item Clerics are divine spellcasters who draw power from their veneration of a deity or ideal.
        \item Druids are nature spellcasters who draw power from their veneration of the natural world.
        \item Fighters are highly disciplined warriors who excel in physical combat of any variety.
        \item Monks are agile masters of ``\ki'' who hone their personal abilities to strike down foes and perform supernatural feats.
        \item Paladins are divinely empowered warriors whose devotion to an alignment grants them the ability to discern and smite their foes.
        \item Rangers are skilled hunters who bridge the divide between nature and civilization.
        \item Rogues are exceptionally skillful characters known for their ability to strike at their foe's weak points in combat.
        \item Sorcerers are arcane spellcasters with an intuitive and flexible understanding of magic.
        \item Spellwarped wield a unique blend of martial skill and narrowly focused magical abilities.
        \item Wizards are arcane spellcasters with a highly studied and deep understanding of magic.
    \end{itemize}

    \subsection{Class Description Format}

        \parhead{Class Table}
        The class's table describes its progression for \glossterm{combat prowess}.
        If a character has a high enough combat prowess to make multiple \glossterm{strikes}, the number of strikes it can make is in parentheses next to its combat prowess.
        For details about combat prowess progressions, see \pcref{Combat Prowess Progressions}.

        \parhead{Alignment}
        Some classes require specific alignments (see \pcref{Alignment}).
        Most classes allow characters of any alignment.

        \parhead{Class Skills}
        These are skills that members of this class are typically good at (see \pcref{Skills}).

        \subsubsection{Base Class Abilities}
            Abilities contained within this heading only apply to characters with the current class as a \glossterm{base class}.

            \parhead{Skill Points}
            This is the number of skill points that members of this class get.

            \parhead{Combat Prowess}
            Each class grants a bonus to combat prowess.
            This bonus is already included in the values given in the class table.
            Do not add the base class bonus for combat prowess to the value given for the class table.
            The bonus in the table must be removed when determining the combat prowess of a character with multiple classes (see \pcref{Multiclass Characters}).

            These bonuses do not stack with other combat prowess bonuses granted by base classes.
            If a character has multiple base classes, only use the highest bonus to combat prowess.

            \parhead{Defenses}
            Each class grants bonuses to specific defenses.
            These bonuses do not stack with other defense bonuses granted by base classes.
            If a character has multiple base classes, use the highest bonuses that apply to each defense.

            \parhead{Weapon and Armor Proficiencies}
            These are the types of equipment that members of this class are trained in using.

        \subsubsection{Class Abilities}
            The class abilities that a character gets for being a member of the class.

\section{Barbarian}\label{Barbarian}
    \begin{dtable}
        \lcaption{Barbarian Progression}
        \begin{dtabularx}{\columnwidth}{>{\ccol}p{\levelcol} >{\ccol}p{\babcolgood} c >{\lcol}X}
            \tb{Level} & \tb{Combat Prowess} & \tb{Rage} & \tb{Special} \\
            \hline
            \barbarianprogressionrow{1}  & \plus2 & Rage, damage reduction, grit \\
            \barbarianprogressionrow{2}  & \plus2 & Rage feat                    \\
            \barbarianprogressionrow{3}  & \plus2 & Fast movement, uncanny dodge \\
            \barbarianprogressionrow{4}  & \plus2 & Durable                      \\
            \barbarianprogressionrow{5}  & \plus3 & \tdash                       \\
            \barbarianprogressionrow{6}  & \plus3 & Rage feat                    \\
            \barbarianprogressionrow{7}  & \plus3 & Battle-scarred               \\
            \barbarianprogressionrow{8}  & \plus3 & Larger than life             \\
            \barbarianprogressionrow{9}  & \plus3 & Improved uncanny dodge       \\
            \barbarianprogressionrow{10} & \plus4 & Rage feat                    \\
            \barbarianprogressionrow{11} & \plus4 & Adaptive rage                \\
            \barbarianprogressionrow{12} & \plus4 & Fury of the storm            \\
            \barbarianprogressionrow{13} & \plus4 & Tireless rage                \\
            \barbarianprogressionrow{14} & \plus4 & Rage feat                    \\
            \barbarianprogressionrow{15} & \plus5 & \tdash                       \\
            \barbarianprogressionrow{16} & \plus5 & Larger than belief           \\
            \barbarianprogressionrow{17} & \plus5 & Deathless rage               \\
            \barbarianprogressionrow{18} & \plus5 & Rage feat                    \\
            \barbarianprogressionrow{19} & \plus5 & Supreme resilience           \\
            \barbarianprogressionrow{20} & \plus6 & Endless rage
        \end{dtabularx}
    \end{dtable}

    \classbasics{Alignment} Any nonlawful.

    \classbasics{Class Skills}
    \begin{itemize}
        \item \subparhead{Strength} Climb, Jump, Sprint, Swim.
        \item \subparhead{Dexterity} Acrobatics, Ride.
        \item \subparhead{Perception} Awareness, Creature Handling, Survival.
        \item \subparhead{Other} Bluff, Intimidate, Persuasion.
    \end{itemize}

    \subsection{Base Class Abilities}
        A character with barbarian as a base class gains the following abilities.

        \classbasics{Skill Points} 10.

        \classbasics{Combat Prowess} \plus2. This benefit is included in the class table.

        \classbasics{Defenses} \plus4 Fortitude, \plus2 Reflex.

        \cf{Bbn}{Weapon and Armor Proficiency}
        A barbarian is proficient with simple weapons, any four other weapon groups, light armor, medium armor, and shields.

        \cf{Bbn}{Damage Reduction}[Ex]
        A barbarian has the ability to shrug off some amount of injury from attacks.
        He has \glossterm{damage reduction} against physical damage equal to his character level.

        \cf{Bbn}{Grit}[Ex]
        The barbarian halves all \glossterm{critical damage} he takes (to a minimum of 1 damage).

    \subsection{Class Abilities}
        All barbarians have the following abilities.

        \cf{Bbn}{Rage}[Ex]
        A barbarian can fly into a rage as a free action.
        Raging has the following benefits and drawbacks:
        \begin{itemize}
            \item \plus2 bonus to damage with physical attacks.
            \item \plus2 bonus to Fortitude and Mental defense.
            \item 2 temporary hit points per Willpower.
                These extra hit points gained from raging are lost before any other hit points (see \pcref{Temporary Hit Points}).
            \item \minus2 to physical defenses (Armor, Maneuver, Reflex).
            \item Unable to take any action that requires patience or concentration, such as casting spells.
            \item If the barbarian does not spend a swift round to sustain the rage, it ends at the end of the round.
            \item At the end of each round, if the barbarian did not attack a creature or object, he takes nonlethal damage equal to his level.
        \end{itemize}

        A rage lasts for up to 5 rounds.
        At the end of the rage, the barbarian takes nonlethal damage equal to his level.
        If the barbarian has any temporary hit points remaining at the end of his rage, the nonlethal damage is dealt to those hit points before they go away.
        In addition, he becomes \fatigued and unable to rage until he rests for 5 minutes.

        The bonuses granted by a barbarian's rage increase with his barbarian level.
        This is called the barbarian's rage bonus.
        At his 5th barbarian level, and every 5 barbarian levels thereafter, the bonus to physical damage and the bonus to Fortitude and Mental defenses increases by \plus1.
        In addition, the number of hit points gained per Willpower increases by 1.
        His penalty to physical defenses while raging remains the same.

        \cf{Bbn}[2nd]{Rage Feat}
        The barbarian gains a bonus Rage feat.
        The bonus feat must be drawn from the list of Rage feats in \tref{Class Feats}.
        He must still meet all prerequisites for the bonus feat.
        He gains an additional bonus Rage feat at his 6th barbarian level and every four barbarian levels thereafter (6th, 10th, 14th, and 18th).

        \cf{Bbn}[3rd]{Fast Movement}[Ex]
        The barbarian increases his land speed by 10 feet while \unencumbered.

        \cf{Bbn}[3rd]{Uncanny Dodge}[Ex]
        A barbarian can react to danger before his senses would normally allow him to do so.
        He reduces his overwhelm penalties by 1.
        If his overwhelm penalty is reduced to 0, he is not considered to be overwhelmed.
        In addition, he is not \unaware when attacked by surprise.

        \cf{Bbn}[4th]{Durable}[Ex]
        The barbarian gains a \plus2 bonus to Fortitude defense.

        \cf{Bbn}[7th]{Battle-Scarred}[Ex]
        The barbarian doubles the benefit of any healing he receives.
        This affects both natural and magical healing.

        \cf{Bbn}[8th]{Larger than Life}[Ex]
        A barbarian holds the strength of a giant in the body of a man (or woman).
        The barbarian is treated as being one size category larger than he actually is for all purposes except his physical space and reach, and the weapons he wields.
        Although he uses weapons of the same size as normal, his weapons deal damage as if they were one size category larger, including natural weapons and unarmed strikes.
        The benefits of this ability stack with the effects of spells and abilities that increase the barbarian's size category.

        \cf{Bbn}[9th]{Improved Uncanny Dodge}[Ex]
        The barbarian reduces his overwhelm penalties by 2.
        This does not stack with the effects of uncanny dodge.
        If his overwhelm penalty is reduced to 0, he is not considered to be overwhelmed.

        \cf{Bbn}[11th]{Adaptive Rage}[Ex]
        While the barbarian is raging, at the end of each round, he may change which Rage feat he is using for that rage.
        Alternately, he may choose not to use any Rage feat for that rage.

        \cf{Bbn}[12th]{Fury of the Storm}[Ex]
        A barbarian cannot be overwhelmed.
        He does not suffer overwhelm penalties, regardless of the number of enemies threatening him.

        \cf{Bbn}[13th]{Tireless Rage}[Ex]
        The barbarian does not take damage when his rage ends, and is not \fatigued at the end of his rage.
        However, he must still rest for 5 minutes before he can rage again after finishing a rage.

        \cf{Bbn}[16th]{Larger than Belief}[Ex]
        The barbarian's larger than life ability improves.
        He is treated as being two size categories larger than he actually is.

        \cf{Bbn}[17th]{Deathless Rage}[Ex]
        While raging, the barbarian ignores all penalties from critical damage, and does not begin dying even he takes critical damage.
        However, if his critical damage exceeds his maximum hit points, the barbarian immediately dies.
        When his rage ends, if the barbarian has critical damage, he begins dying.

        \cf{Bbn}[19th]{Supreme Resilience}[Ex]
        The barbarian cannot take more than half his maximum hit points in damage during a single round.
        Any excess damage is ignored.

        \cf{Bbn}[20th]{Endless Rage}[Ex]
        A barbarian's rage no longer has a limited duration.
        He may rage indefinitely without stopping.
        In addition, he may rage immediately after finishing a rage, without needing to rest.

        \subsubsection{Ex-Barbarians}
            A barbarian who becomes lawful loses his ability to rage, and cannot gain more levels as a barbarian.
            He retains all his other class abilities.
            If he stops being lawful, he regains his ability to rage and take barbarian levels.

\section{Cleric}\label{Cleric}
    \begin{dtable}
        \lcaption{Cleric Progression}
        \begin{dtabularx}{\columnwidth}{>{\ccol}p{2em} >{\ccol}p{\babcolavg} c c >{\lcol}X}
            \tb{Level} & \tb{Combat Prowess} & \tb{Devotion} & \tb{Spells} & \tb{Special} \\
            \hline
            \clericprogressionrow{1}  & \plus2 & 2 & Devotion, domain gifts, spells \\
            \clericprogressionrow{2}  & \plus2 & 3 & Domain invocation, rituals            \\
            \clericprogressionrow{3}  & \plus2 & 3 & Domain spell                          \\
            \clericprogressionrow{4}  & \plus2 & 4 & Devotion feat, domain invocation      \\
            \clericprogressionrow{5}  & \plus3 & 4 & Domain spell         \\
            \clericprogressionrow{6}  & \plus3 & 5 & Domain aspect                         \\
            \clericprogressionrow{7}  & \plus3 & 5 & Domain spell                          \\
            \clericprogressionrow{8}  & \plus3 & 6 & Domain aspect                         \\
            \clericprogressionrow{9}  & \plus3 & 6 & Domain spell                          \\
            \clericprogressionrow{10} & \plus4 & 7 & Intercession         \\
            \clericprogressionrow{11} & \plus4 & 7 & Domain spell                          \\
            \clericprogressionrow{12} & \plus4 & 8 & Greater domain invocation             \\
            \clericprogressionrow{13} & \plus4 & 8 & Domain spell                          \\
            \clericprogressionrow{14} & \plus4 & 9 & Greater domain invocation             \\
            \clericprogressionrow{15} & \plus5 & 9 & Domain spell         \\
            \clericprogressionrow{16} & \plus5 & 10 & Domain mastery                        \\
            \clericprogressionrow{17} & \plus5 & 10 & Domain spell                          \\
            \clericprogressionrow{18} & \plus5 & 11 & Domain mastery                        \\
            \clericprogressionrow{19} & \plus5 & 11 & Endless devotion, domain spell        \\
            \clericprogressionrow{20} & \plus6 & 12 & Miracle              \\
        \end{dtabularx}
    \end{dtable}

    \classbasics{Alignment} The cleric's alignment must be within one step of his deity's (that is, it may be one step away on either the lawful-chaotic axis or the good-evil axis, but not both).

    \classbasics{Class Skills}
    \subparhead{Intelligence} Heal, Knowledge (arcana, local, religion, the planes), Linguistics.
    \subparhead{Perception} Awareness, Sense Motive, Spellcraft.
    \subparhead{Other} Bluff, Intimidate, Persuasion.

    \subsection{Base Class Abilities}
        A character with cleric as a base class gains the following abilities.

        \classbasics{Skill Points} 5.

        \classbasics{Combat Prowess} \plus2. This benefit is included in the class table.

        \classbasics{Defenses} \plus2 Fortitude, \plus4 Mental.

        \cf{Clr}{Weapon and Armor Proficiency}
        Clerics are proficient with simple weapons, any two other weapon groups, light and medium armor, and shields.

        \cf{Clr}{Domain Gifts}[Su]
        A cleric's abilities are shaped by his domains.
        He gains the domain gifts of both of his domains.
        Domain gifts are not activated.
        The gifts offered by each domain are listed at \pcref{Domain Gifts}.

        \cf{Clr}{Enhanced Divine Power}
        The cleric gains a \plus2 bonus to his divine power.

    \subsection{Class Abilities}
        All clerics have the following abilities.

        \cf{Clr}{Domains}
        A cleric chooses two domains, which represent his personal spiritual inclinations.
        He must choose his domains from among those his deity offers.
        A cleric's choice of domains has broad effects on the cleric's spellcasting and supernatural abilities.
        The domains are listed below.

        \begin{itemize}
            \item{Air}
            \item{Chaos}
            \item{Death}
            \item{Destruction}
            \item{Earth}
            \item{Evil}
            \item{Fire}
            \item{Good}
            \item{Knowledge}
            \item{Law}
            \item{Life}
            \item{Magic}
            \item{Protection}
            \item{Strength}
            \item{Travel}
            \item{Trickery}
            \item{War}
            \item{Water}
            \item{Wild}
        \end{itemize}

        \cf{Clr}{Divine Power}[Su]
        The strength of a cleric's spells and abilities are determined by his divine power.
        Normally, his divine power is equal to his character level.

        \cf{Clr}{Spells}
        A cleric casts divine spells using his devotion.
        The maximum spell level a cleric can learn or cast is equal to half his cleric level.
        A cleric's \glossterm{spellpower} with divine spells is normally equal to his divine power.

        A cleric begins play knowing two first-level spells.
        Every even level, he learns an additional spell of any level he has access to.
        In addition, each time he gains a level, he may trade one of his existing spells for a different spell known.
        However, he must always know at least one spell of every level he has access to.
        He may learn spells from both the divine spell list (see \pcref{Divine Spells}) and his domain spell lists (see \pcref{Cleric Domains}).
        Sometimes these domain spells are spells that are normally available on the divine spell list, but often they are only accessible by the domain.

        The number of spells a cleric can cast per day is given on \trefnp{Cleric Spell Slots}.
        In order to regain his spell slots for the day, the cleric must dismiss all his active spells and spend 1 hour performing a ritual, worshipping, or quietly contemplating.
        The cleric cannot regain spell slots in this way more than once per 24 hours.

        A cleric can't cast spells of an alignment opposed to his own or his deity's.
        Spells associated with particular alignments are indicated by the chaos, evil, good, and law descriptors in their spell descriptions.

        \begin{dtable}
            \lcaption{Cleric Spell Slots}
            \begin{dtabularx}{\columnwidth}{>{\ccol}X *{9}{>{\ccol}p{\spellcol}}}
                & \multicolumn{9}{c}{\tb{---{}---{}---{}---{}---{}---{}---{}---Spell Level---{}---{}---{}---{}---{}---{}---{}---}} \\
                \hline
                \tb{Level} & \tb{1st} & \tb{2nd} & \tb{3rd} & \tb{4th} & \tb{5th} & \tb{6th} & \tb{7th} & \tb{8th} & \tb{9th} \\
                1st  & 3 & \tdash & \tdash & \tdash & \tdash & \tdash & \tdash & \tdash & \tdash \\
                2nd  & 4 & \tdash & \tdash & \tdash & \tdash & \tdash & \tdash & \tdash & \tdash \\
                3rd  & 5 & \tdash & \tdash & \tdash & \tdash & \tdash & \tdash & \tdash & \tdash \\
                4th  & 6 & 3  & \tdash & \tdash & \tdash & \tdash & \tdash & \tdash & \tdash \\
                5th  & 6 & 4  & \tdash & \tdash & \tdash & \tdash & \tdash & \tdash & \tdash \\
                6th  & 6 & 5  & 3  & \tdash & \tdash & \tdash & \tdash & \tdash & \tdash \\
                7th  & 6 & 6  & 4  & \tdash & \tdash & \tdash & \tdash & \tdash & \tdash \\
                8th  & 6 & 6  & 5  & 3  & \tdash & \tdash & \tdash & \tdash & \tdash \\
                9th  & 6 & 6  & 6  & 4  & \tdash & \tdash & \tdash & \tdash & \tdash \\
                10th & 6 & 6  & 6  & 5  & 3  & \tdash & \tdash & \tdash & \tdash \\
                11th & 6 & 6  & 6  & 6  & 4  & \tdash & \tdash & \tdash & \tdash \\
                12th & 6 & 6  & 6  & 6  & 5  & 3  & \tdash & \tdash & \tdash \\
                13th & 6 & 6  & 6  & 6  & 6  & 4  & \tdash & \tdash & \tdash \\
                14th & 6 & 6  & 6  & 6  & 6  & 5  & 3  & \tdash & \tdash \\
                15th & 6 & 6  & 6  & 6  & 6  & 6  & 4  & \tdash & \tdash \\
                16th & 6 & 6  & 6  & 6  & 6  & 6  & 5  & 3  & \tdash \\
                17th & 6 & 6  & 6  & 6  & 6  & 6  & 6  & 4  & \tdash \\
                18th & 6 & 6  & 6  & 6  & 6  & 6  & 6  & 5  & 3  \\
                19th & 6 & 6  & 6  & 6  & 6  & 6  & 6  & 6  & 4  \\
                20th & 6 & 6  & 6  & 6  & 6  & 6  & 6  & 6  & 6  \\
            \end{dtabularx}
        \end{dtable}

        \begin{dtable!*}
            \lcaption{Deities}
            \begin{dtabularx}{\textwidth}{X l X}
                \tb{Deity} & \tb{Alignment} & \tb{Domains} \\
                \hline
                Guftas, horse god of justice          & Lawful good     & Good, Law, Strength, Travel         \\
                Lucied, paladin god of justice        & Lawful good     & Destruction, Good, Protection, War  \\
                Simor, fighter god of protection      & Lawful good     & Good, Protection, Strength, War     \\
                %Pabst, dwarf god of drink             & Neutral good & Good, Life, Strength, Wild \\
                Rucks, monk god of pragmatism         & Neutral good    & Good, Law, Protection, Travel       \\
                Vanya, centaur god of nature          & Neutral good    & Good, Strength, Travel, Wild        \\
                Brushtwig, pixie god of creativity    & Chaotic good    & Chaos, Good, Trickery, Wild         \\
                Chavi, god of stories                 & Chaotic good    & Chaos, Knowledge, Trickery          \\
                Ivan Ivanovitch, bear god of strength & Chaotic good    & Chaos, Strength, War, Wild          \\
                Krunch, barbarian god of destruction  & Chaotic good    & Destruction, Good, Strength, War    \\
                Sir Cakes, dwarf god of freedom       & Chaotic good    & Chaos, Good, Strength               \\
                Raphael, monk god of retribution      & Lawful neutral  & Death, Law, Protection, Travel      \\
                Declan, god of fire                   & True neutral    & Destruction, Fire, Knowledge, Magic \\
                Kurai, shaman god of nature           & True neutral    & Air, Earth, Fire, Water             \\
                %Amanita, druid god of decay           & Chaotic neutral & Chaos, Destruction, Life, Wild \\
                %Antimony, elf god of necromancy       & Chaotic neutral & Death, Knowledge, Life, Magic \\
                Clockwork, elf god of time            & Chaotic neutral & Chaos, Magic, Trickery, Travel      \\
                %Lord Khallus, fighter god of pride    & Chaotic neutral & Chaos, Strength, War \\
                %Celeano, sorcerer god of deception    & Chaotic neutral & Chaos, Magic, Protection, Trickery \\
                Murdoc, god of mercenaries            & Chaotic neutral & Destruction, Knowledge, Travel, War \\
                Ribo, halfling god of trickery        & Chaotic neutral & Chaos, Trickery, Water              \\
                Tak, orc god of war                   & Lawful evil     & Law, Strength, Trickery, War        \\
                Theodolus, sorcerer god of ambition   & Neutral evil    & Evil, Knowledge, Magic, Trickery    \\
                Daeghul, demon god of slaughter       & Chaotic evil    & Destruction, Evil, Magic, War       \\
            \end{dtabularx}
        \end{dtable!*}

        \cf{Clr}{Devotion}[Su]
        A cleric has a devotion pool with a number of devotion points in it equal to half his Willpower or half his cleric level, whichever is higher.
        Before making an attack or check, the cleric may spend a devotion point to gain a \plus2 bonus to the attack or check.
        The bonus granted by spending a devotion point increases by \plus1 at 5th level and every 5 levels thereafter.

        After regaining spells for the day, a cleric's devotion pool is full.
        He loses a point from his devotion pool if he acts against his deity.
        If the cleric's devotion pool has no points remaining, he takes a \minus2 penalty to his attacks, checks, and defenses.

        The cleric can refill his devotion pool by spending an hour in prayer, supplication, or contemplation.
        If he is in a location exclusively dedicated to his deity, such as a temple, the time required is reduced to five minutes.
        Exceptionally holy locations may allow even faster recovery of devotion points.

        \cf{Clr}[2nd]{Domain Invocations}[Su]
        As a standard action, a cleric can spend a devotion point to invoke divine power.
        He gains the domain invocations offered by two of his domains.

        All domain invocations affect a single creature within \rngmed range and require a special attack against a defense.
        The cleric's accuracy with domain invocations is equal to his divine power.
        If the attack succeeds, a domain invocation heals or inflicts 1d6 damage per divine power.
        If the attack fails, the invocation heals or inflicts half damage.

        \cf{Clr}[2nd]{Rituals}
        The cleric gains the ability to perform divine rituals.
        He does not automatically learn rituals, but must find a ritual book.
        For details, see \pcref{Rituals}.

        \cf{Clr}[3rd]{Domain Spell}
        The cleric chooses a spell from one of his domains.
        The spell's level cannot exceed half the cleric's level.
        If he knows that spell, he may spend a devotion point to cast it in place of a spell slot.

        At 5th level, and every odd level theareafter, the cleric gains an additional domain spell.

        \cf{Clr}[4th]{Devotion Feat}[Su]
        The cleric chooses a feat that he meets the prerequisites for.
        As long as his devotion pool is at least half full, he gains that feat as a bonus feat.
        At each level, the cleric may change this feat to a different feat he qualifies for.

        At 14th level, the cleric gains a second devotion feat.
        The cleric may not normally use these devotion feats as prerequisites for other feats or abilities.
        However, he he may use one devotion feat to meet a prerequiste for his second devotion feat.

        \cf{Clr}[6th]{Domain Aspect}[Su]
        The cleric gains a domain aspect from one of his domains.
        Domain aspects do not require an action to activate.
        Options for domain aspects are listed at \pcref{Domain Aspects}.

        At his 8th cleric level, the cleric gains an additional domain aspect from one of his domains.

        \cf{Clr}[10th]{Intercession}[Su]
        Once per day, the cleric can request a divine intercession as a standard action.
        He mentally specifies his request, and his deity fulfills that request in the manner it sees fit.
        This can emulate the effects of any spell of a level no greater than half of his level, or have any other effect of a similar power level.
        Divine intercessions tend to reflect the personality of the cleric's deity, not the cleric's personal preferences.

        If the cleric performs a significant service for his deity, he can gain the ability to request an additional intercession that day.

        \cf{Clr}[12th]{Greater Domain Invocations}[Su]
        The cleric gains the ability to invoke the power of one of his domains even more effectively.
        Greater domain invocations are described at \pcref{Greater Domain Invocations}.

        At his 14th cleric level, the cleric gains an additional greater domain invocation from one of his domains.

        \cf{Clr}[16th]{Domain Mastery}[Su]
        The cleric gains a domain mastery from one of his domains.
        Options for domain masteries are listed at \pcref{Domain Masteries}.

        At his 18th cleric level, the cleric gains an additional domain mastery from one of his domains.

        \cf{Clr}[19th]{Endless Devotion}[Su]
        Whenever the cleric casts a domain spell, if he did not spend any devotion points that round, he regains a spent devotion point.

        \cf{Clr}[20th]{Miracle}[Su]
        Once per week, the cleric can request a miracle as a standard action.
        He mentally specifies his request, and his deity fulfills that request in the manner it sees fit.
        This can emulate the effects of any spell or ritual, or have any other effect of a similar power level.
        If the deity has a direct interest in the cleric's situation, the miracle may be of even greater power.

        If the cleric performs an extraordinary service for his deity, he can gain the ability to request an additional miracle that week.

    \subsection{Cleric Domain Abilities}

        \subsubsection{Domain Gifts}\label{Domain Gifts}
            % If a domain gift involves an attack, the accuracy is equal to the cleric's divine power.

            \subcf{Air}
            The cleric adds the Jump skill (see \pcref{Jump}) to his cleric class skill list, and gains a \plus5 bonus to Jump checks.
            \subcf{Chaos}
            The cleric rolls twice for all \glossterm{random effects} and chooses his preferred result.
            \subcf{Death}
            The cleric halves all critical damage he takes (to a minimum of 1 damage).
            In addition, he is immune to \glossterm{Death} effects.
            \subcf{Destruction}
            When making physical attacks, the cleric ignores an amount of hardness and damage reduction equal to half his divine power.
            \subcf{Earth}
            The cleric gains the tremorsense ability with a range of 50 feet.
            If he is touching a surface, he can automatically pinpoint the location of anything within 50 feet that is in contact with the surface, including inanimate objects.
            \subcf{Evil}
            The cleric gains \glossterm{damage reduction} against physical damage from non-evil sources equal to half his divine power.
            \subcf{Fire}
            The cleric gains \glossterm{damage reduction} against fire and cold damage equal to his divine power.
            \subcf{Good}
            The cleric gains \glossterm{damage reduction} against physical damage from non-good sources equal to half his divine power.
            \subcf{Knowledge}
            The cleric adds all Knowledge skills to his cleric class skill list.
            In addition, he gains two skill points which must be spent on Knowledge skills.
            \subcf{Law}
            The cleric is immune to \glossterm{Delusion} effects.
            \subcf{Life}
            The cleric gains a \plus2 bonus to Fortitude defense.
            In addition, whenever he makes a Heal check, he rolls twice and takes the higher result.
            \subcf{Magic}
            The cleric gains a \plus1 bonus to spellpower with divine spells.
            \subcf{Protection}
            As an immediate action, when an ally adjacent to the cleric takes damage, the cleric can take half that damage instead of the ally.
            \subcf{Strength}
            The cleric adds Climb, Jump, Sprint, and Swim to his cleric class skill list.
            In addition, he gains two skill points which must be spent on Strength-based skills.
            \subcf{Travel}
            The cleric adds Knowledge (geography) and Survival to his cleric class skill list.
            In addition, he gains a \plus10 foot bonus to his land speed.
            \subcf{Trickery}
            The cleric adds Bluff, Disguise, and Stealth to his cleric class skill list.
            In addition, he gains two skill points which must be spent on any combination of those skills.
            \subcf{War}
            The cleric gains Weapon Focus with his deity's favored weapon group as a bonus feat.
            \subcf{Water}
            The cleric adds Swim to his cleric class skill list, gains a \plus5 bonus to Swim checks, and suffers no penalties for fighting underwater.
            \subcf{Wild}
            The cleric adds Creature Handling, Knowledge (nature), and Survival to his cleric class skill list.
            In addition, he gains two skill points which must be spent on any combination of those skills.

        \subsubsection{Domain Invocations}\label{Domain Invocations}

            \subcf{Air -- Reflex}
            The target takes elecricity damage.
            \subcf{Chaos -- Mental}
            This invocation randomly heals or inflicts damage.
            The cleric chooses the target after rolling to determine the effect.
            \subcf{Death -- Fortitude}
            The target takes divine damage.
            If this attack deals critical damage, the target is instantly killed.
            This is a death effect.
            \subcf{Destruction -- Fortitude}
            The target takes sonic damage.
            \subcf{Earth -- Reflex}
            The target takes bludgeoning damage if it is on the ground.
            \subcf{Evil -- Mental}
            This invocation does not heal or inflict damage.
            If the attack succeeds, and the target is good, it is \staggered for 5 rounds.
            \subcf{Fire -- Reflex}
            The target takes fire damage.
            \subcf{Good -- Mental}
            This invocation does not heal or inflict damage.
            If the attack succeeds, and the target is evil, it is \dazed for 5 rounds.
            \subcf{Knowledge -- Special}
            The target must make a Knowledge check.
            If its check result beats your attack result, it is healed.
            Otherwise, it takes damage.
            This invocation heals or inflicts 1d10 damage per two divine power instead of the normal value.
            \subcf{Law -- Mental}
            This invocation does not heal or inflict damage.
            If the attack succeeds, and the target is chaotic, it is \immobilized for 5 rounds.
            \subcf{Life -- Fortitude}
            The target is healed.
            This invocation heals 1d8 damage per divine power instead of the normal value.
            \subcf{Magic -- Mental}
            If the target can cast spells, it is healed.
            Otherwise, it takes divine damage.
            This invocation heals or inflicts 1d10 damage per two divine power instead of the normal value.
            \subcf{Protection -- Fortitude}
            This invocation does not heal or inflict damage.
            The target gains 1d10 temporary hit points per two divine power.
            \subcf{Strength -- Special}
            The target must make a Strength check.
            If its check result beats your attack result, it is healed.
            Otherwise, it takes divine damage.
            This invocation heals or inflicts 1d10 damage per two divine power instead of the normal value.
            \subcf{Travel -- Reflex}
            The target is healed.
            In addition, it gains a \plus10 foot bonus to its movement speed for 1 round.
            This invocation heals 1d10 damage per two divine power instead of the normal value.
            \subcf{Trickery -- Mental}
            If the attack succeeds, the target is \disoriented for 1 round.
            \subcf{War -- Fortitude}
            This invocation affects all enemies within a \areasmall radius of you instead of the normal target.
            The targets take divine damage.
            This invocation deals 1d8 damage per two divine power instead of the normal value.
            \subcf{Water -- Fortitude}
            The target takes nonlethal physical damage from water in its mouth and lungs.
            In addition, if the attack succeeds, the target is unable to speak for 1 round.
            \subcf{Wild -- Fortitude}
            The target takes divine damage.
            If the target is an animal or plant, the cleric may choose to heal it instead.

        \subsubsection{Domain Aspects}\label{Domain Aspects}

            \subcf{Air -- Limited Flight}
            The cleric gains a glide speed equal to his land speed.
            See \pcref{Gliding}, for more details.
            In addition, as a swift action, he can spend a devotion point to treat air as if it was solid ground until the end of the round.
            He can only do this once before touching solid ground again.
            \subcf{Chaos -- Chaotic Retribution}
            Whenever a lawful creature within 30 feet of you attacks you, means it takes 1d6 damage per two divine power.
            A creature can only be dealt damage by this effect once per round.
            \subcf{Death -- Lifedrinker}
            Whenever the cleric kills a creature, he gains temporary hit points equal to his divine power for a number of rounds equal to the creature's level.
            \subcf{Destruction -- Beacon of Destruction}
            All enemies within a \areamed radius emanation of the cleric have their damage reduction and hardness (if any) reduced by an amount equal to half the cleric's divine power.
            \subcf{Earth -- Hardened Skin}
            The cleric gains damage reduction against physical damage equal to his divine power.
            Adamantine weapons ignore this damage reduction and negate it for 1 round.
            \subcf{Evil -- Unholy Retribution}
            Whenever a good creature within 30 feet of you attacks you, it takes 1d6 damage per two divine power.
            A creature can only be dealt damage by this effect once per round.
            \subcf{Fire -- Friendly Fire}
            All of the cleric's fire spells and abilities do not deal fire damage to his allies.
            \subcf{Good -- Holy Retribution}
            Whenever an evil creature within 30 feet of you attacks you, it takes 1d6 damage per two divine power.
            A creature can only be dealt damage by this effect once per round.
            \subcf{Knowledge -- Knowledge Mastery}
            Whenever the cleric makes a Knowledge check, he may roll twice and take the higher result.
            In addition, he gains two skill points which must be spent on Knowledge skills.
            \subcf{Law -- Certain Retribution}
            Whenever an chaotic creature within 30 feet of you attacks you, it takes 1d6 damage per two divine power.
            A creature can only be dealt damage by this effect once per round.
            \subcf{Life -- Critical Healer}
            All of the cleric's healing spells and abilities cure critical damage as easily as they cure hit points.
            \subcf{Magic -- Magic Feat}
            The cleric gains a bonus magic feat or metamagic feat.
            \subcf{Protection -- Faithful Shield}
            The cleric may maintain concentration on \glossterm{Shielding} spells as a swift action.
            \subcf{Strength -- Strength of Will}
            The cleric may use his Strength in place of his divine power to determine the strength of his invocations.
            \subcf{Travel -- Rapid Traveller}
            The cleric gains a \plus30 foot bonus to his speed in all movement modes, up to a maximum of double his original speed.
            In addition, whenever he makes a Sprint check, he rolls twice and takes the higher result.
            \subcf{Trickery -- Legendary Trickster}
            The cleric gains his choice of Legendary Disguise, Legendary Liar, or Legendary Stealth as a bonus feat, even if he does not meet the prerequisites.
            \subcf{War -- Combat Feat}
            The cleric gains a combat feat of his choice.
            \subcf{Water -- Water Breathing} The cleric may breathe and speak normally while underwater, as the \spell{water breathing}
            ritual.
            He also gains a \glossterm{swim speed} equal to his base land speed.
            \subcf{Wild -- Wild Aspect}
            When the cleric gains this ability, he chooses one wild aspect ability, as if he were a were a druid of a level equal to his cleric level (see Wild Aspect, \pref{Drd:Wild Aspect}).
            He can spend a devotion point as a swift action to embody that wild aspect for 1 hour.

        \subsubsection{Greater Domain Invocations}\label{Greater Domain Invocations}

            \subcf{Air -- Command Air}
            As a standard action, the cleric can spend a devotion point to speak with and command air for 5 minutes.
            He can ask the air simple questions and understand its responses.
            If he commands the air to perform a task, it will do so do the best of its ability until the end of the effect's duration.
            The cleric cannot compel the air to go faster than 50 mph, and cannot affect air farther than 500 feet from him.

            \subcf{Chaos -- Sow Chaos}
            As a standard action, the cleric can spend a devotion point to cause an improbable event to occur.
            He can visualize in general terms what he wants to happen, such as ``Make the bartender leave the bar''.
            He cannot control exact nature of the event, though it always beneficial for him in some way.

            \subcf{Death -- Reaper's Boon}
            As a standard action, the cleric can spend a devotion point to summon or banish Death.
            In either case, it affects a living creature within 100 feet of him for 5 rounds.
            If he summons Death, the target immediately dies if it takes critical damage.
            If he banishes Death, the target is immune to Death effects and does not make \glossterm{stabilization rolls} if it takes critical damage.
            This does not prevent the target from taking critical damage, and it begins dying after the effect ends if it has took critical damage and has not been stabilized.

            \subcf{Destruction -- Dust to Dust}
            As a standard action, the cleric can spend a devotion point to destroy objects within a \arealarge radius burst centered on him.
            He makes a Divine power vs. Mental attack against all objects in the area.
            Success against an object means it crumbles into dust, and is irreparably broken.
            Unattended nonmagical objects do not have a Mental defense, and are automatically broken.
            The cleric may freely exclude any objects or squares from the effect.

            \subcf{Earth -- Command Earth}
            As a standard action, the cleric can spend a devotion point to speak with and command earth for 5 minutes.
            He can ask the earth simple questions and understand its responses.
            If he commands the earth to perform a task, it will do so do the best of its ability until the end of the effect's duration.
            The cleric cannot compel the earth to move faster than 10 feet per round, and cannot effect earth farther than 500 feet from him.

            \subcf{Evil -- Temptation}
            As a standard action, the cleric can spend a devotion point to compel a creature to commit an evil act.
            He makes a Divine power vs. Mental attack against the creature.
            Success means the target takes an evil action as soon as it can.
            The cleric has no control over the act the creature takes, but circumstances can make the target more likely to take an action the cleric desires.
            Creatures who have strict codes prohibiting them from taking evil actions, such as paladins devoted to Good, are immune to this effect.
            This is a [\glossterm{Compulsion}, \glossterm{Mind}] effect.

            \subcf{Fire -- Command Flames}
            As a standard action, the cleric can spend a devotion point to speak with and command fire for 5 minutes.
            He can ask the fire simple questions and understand its responses.
            If he commands the fire to perform a task, it will do so do the best of its ability until the end of the effect's duration.
            The fire can move up to 30 feet in a single round between combustible materials.
            The cleric cannot effect fire farther than 500 feet from him.

            \subcf{Good -- Salvation}
            As a standard action, the cleric can spend a devotion point to compel a creature to do a good deed.
            He makes a Divine power vs. Mental attack against the creature.
            Success means the target takes a good action as soon as it can.
            THe cleric has no control over the act the creature takes, but circumstances can make the target more likely to take an action the cleric desires.
            Creatures who have strict codes prohibiting them from taking good actions, such as paladins devoted to Evil, are immune to this effect.
            This is a [\glossterm{Compulsion}, \glossterm{Mind}] effect.

            \subcf{Knowledge -- Impart Truth}
            As a standard action, the cleric can spend a devotion point to grant knowledge.
            He may make a Knowledge check of any kind with a \plus20 bonus to the check.
            He may also cause any number of creatures within a \arealarge radius around him to learn the results of his check.
            Creatures granted knowledge in this way believe the information to be true as if they had seen it with their own eyes.
            Exceptionally stubborn or untrusting creatures may still not be convinced of its truth, however.

            \subcf{Law -- Infallible Enforcement}
            As a standard action, the cleric can spend a devotion point to enforce the law in a \arealarge radius \glossterm{zone} centered on him.
            He makes a Divine power vs. Mental attack against all creatures in the area.
            Success means the target is unable to break the law, and any attempt to do so simply fails.
            Failure means the target feels a compulsion not to break the law, but is able to overcome the compulsion if it desires.
            The laws which are applied are those which are most appropriate for the area, regardless of whether the cleric or any other creature know those laws.
            If the rightful laws are inconsistent or impossible to understand, those laws may not be enforced.
            This is a [\glossterm{Compulsion}, \glossterm{Mind}] effect.

            \subcf{Life -- Prayer of Resurrection}
            As a standard action, the cleric can resurrect a touched creature, as the \spell{resurrection} ritual.
            The target must have been dead for no more than 5 minutes.
            He must spend a number of devotion points equal to the target's level.
            He does not know the cost before resurrecting the target.
            If the cost exceeds his remaining devotion points, all his devotion points are spent, but the resurrection still succeeds as long as he spent at least one devotion point in this way.

            \subcf{Magic -- Manipulate Spell}
            As a standard action, the cleric can spend a devotion point to manipulate a currently active spell or spell-like ability within \rngmed range.
            He must make a Divine power check against a DR equal to 10 \add the spell's spellpower.
            Success means he identifies the spell perfectly, if he had not already done so, and can take one of the following four actions on the spell.
            \begin{itemize}
                \item Control: If the spell can be focused on to gain an effect or extend its duration, the cleric gains the ability to focus on the spell as if he was the one who originally cast it.
                    Its original caster loses the ability to focus on the spell.
                \item Dispel: The spell is dispelled, if it can be dispelled by \spell{dispel magic}.
                \item Persist: The spell's remaining duration increases by 5 minutes, up to a maximum of its starting duration (ignoring any duration increase from focusing on the spell).
                \item Suppress: The spell is suppressed for 5 rounds, if it can be dispelled by \spell{dispel magic}.
                    At the end of that time, the spell's effect resumes, if it still has duration remaining.
            \end{itemize}

            \subcf{Protection -- Divine Shield}
            As a standard action, the cleric can create a powerful protective shield around a creature or object within \rngclose range for 5 rounds.
            The target takes half damage from all attacks.
            In addition, whenever the target takes damage, the cleric can spend a devotion point as an immediate action to negate that damage.

            \subcf{Strength -- Might of the Gods}
            As a standard action, the cleric can spend a devotion point to gain titanic strength for 5 minutes.
            For the purpose of checks and determining carrying capacity, the cleric's Strength becomes equal to 10 \add his divine power.
            If he takes damage, the effect ends.

            \subcf{Travel -- Transcend Movement}
            As a standard action, the cleric can spend a devotion point to teleport himself, as the \spell{teleport} ritual.

            \subcf{Trickery -- Enduring Falsehood}
            As a standard action, the cleric can delude a creature within \rngmed range into believing a lie, regardless of evidence.
            He chooses a falsehood and makes a Divine power vs. Mental attack against the creature.
            The falsehood may be a lie, or a simple misunderstanding (such as believing a hidden creature is not present in a room).
            If the creature does not already believe the falsehood, the attack automatically fails.

            Success means that the target continues to believe the falsehood for 5 minutes, regardless of any evidence to the contrary.
            It will interpret any evidence that the falsehood is incorrect to be somehow wrong -- an illusion, a conspiracy to decieve it, or any other reason it can think of to continue believing the falsehood.
            At the end of the effect, the creature can decide whether it believes the falsehood or not, as normal.
            This is a [\glossterm{Delusion}, \glossterm{Mind}] effect.

            \subcf{War -- Mass Combat}
            As an immediate action, the cleric can spend a devotion point to augment a spell he casts with one of the following effects.
            \begin{itemize}
                \item Legion: If the spell would normally affect five or more specific targets, its range is doubled and it instead affects five times that many targets.
                \item Selective: If the spell has an area, it has no effect on his allies in the area.
                \item Widened: If the spell has an area, the size of the area is doubled.
            \end{itemize}

            \subcf{Water -- Command Water}
            As a standard action, the cleric can spend a devotion point to speak with and command water for 5 minutes.
            He can ask the water simple questions and understand its responses.
            If he commands the water to perform a task, it will do so do the best of its ability until the end of the effect's duration.
            The cleric cannot compel the earth to move faster than 30 feet per round, and cannot effect water farther than 500 feet from him.

        \subsubsection{Domain Masteries}\label{Domain Masteries}

            \subcf{Air -- Flight}
            The cleric gains a fly speed (good maneuverability) equal to his land speed.
            He may remain flying for up to 5 rounds at a time.
            After that, he must land for 1 round before he can fly again.
            See \pcref{Flying}, for more details.
            \subcf{Chaos -- Avatar of Luck}
            Once per round, the cleric can gain a \plus1d6 bonus to any check or physical attack.
            He may declare the use of the ability after failing the roll, but before any additional effects are resolved, potentially making it succeed where it would have failed.
            \subcf{Death -- Deathfeeder}
            The cleric constantly radiates a \areamed radius emanation of death.
            Whenever a creature dies within the area, the cleric gains the benefits of the \spell{death knell} spell as if it had been cast on the creature.
            \subcf{Destruction -- Ruinbringer}
            The cleric's attacks and spells ignore all damage reduction and hardness (but not damage immunity).
            \subcf{Earth -- Earth Glide}
            The cleric gains the earth glide ability, as an earth elemental.
            \subcf{Evil -- Avatar of Evil} The cleric continuously gains the benefits of the \spell{protection from good}
            spell, with a spellpower equal to his cleric level.
            If the effect is dispelled or suppressed, he can resume it as a swift action.
            \subcf{Fire -- Flaming Soul}
            The cleric gains the fire subtype, making him immune to fire but giving him a 50\% vulnerability to cold damage.
            In addition, whenever he deals fire damage to a creature, the creature is \ignited for 5 rounds.
            \subcf{Good -- Avatar of Good} The cleric continuously gains the benefits of the \spell{protection from evil}
            spell, with a spellpower equal to his cleric level.
            If the effect is dispelled or suppressed, he can resume it as a swift action.
            \subcf{Knowledge -- Combat Insight}
            The cleric gains a \plus2 bonus to accuracy, checks, and defenses against non-humanoid creatures he has identified with a successful Knowledge check.
            \subcf{Law -- Avatar of Order}
            Once per round, if the cleric rolls less than a 10 on a d20, he may treat the result as if it were a 10, potentially causing him to succeed where he would have failed.
            You must declare the use of this ability before any additional effects from the roll are resolved.
            \subcf{Life -- }
            \subcf{Magic -- Spellfeeder}
            The cleric gains spell resistance equal to 10 \add cleric level or Intelligence.
            To affect the cleric with a spell, a caster must make an attack with its spellpower.
            If the attack beats the cleric's spell resistance, the spell works normally.
            Otherwise, the spell has no effect on the cleric.

            In addition, whenever the cleric resists a spell with his spell resistance, he regains a spell slot of a level up to one lower than the level of the resisted spell.
            \subcf{Protection -- Martyr's Gift}
            The cleric constantly radiates a \areamed radius emanation of protective energy.
            Whenever a creature within the area takes damage, the cleric can choose to take half of that damage instead, as the \spell{share pain} spell.
            \subcf{Strength -- Might of the Gods}
            The cleric gains the larger than life ability, as the barbarian class ability (see Larger than Life, \pref{Bbn:Larger than Life}).
            \subcf{Travel -- Perfect Stride}
            The cleric gains perfect stride, as the ranger class ability.
            He constantly acts as if he were under the effect of a \spell{freedom} spell, except that it does not allow him to act normally underwater.
            \subcf{Trickery -- Exemplar of Deceit} The cleric continuously gains the benefits of the \spell{nondetection}
            spell, with a spellpower equal to his cleric level, except that it also protects him from any spells or effects which would prevent him from lying or reveal his lies.
            If the effect is dispelled or suppressed, he can resume it as a swift action.
            \subcf{War -- Warmaster's Favor} The cleric continuously gains the benefits of the \spell{divine favor}
            spell, with a spellpower equal to his cleric level.
            If the effect is dispelled or suppressed, he can resume it as a swift action.
            \subcf{Water -- Water's Flow}
            As a swift action, the cleric can transform himself into a rushing flow of water with a volume roughly equal to his normal volume until the end of his turn.
            In this form, he may move wherever water could go, but he cannot take other actions, such as jumping, attacking, or casting spells.
            His speed is halved when moving uphill and doubled when moving downhill.
            He may move through squares occupied by creatures or threatened by blocking enemies without penalty.
            He may return to his normal form as a free action.
            \par If the water is split, he may reform from anywhere the water has reached, to as little as a single ounce of water.
            If not even an ounce of water exists contiguously, his body reforms from the largest available parts of water, cut into pieces of appropriate size.
            This usually causes the cleric to die.
            \subcf{Wild -- Natural Casting}
            Whenever the cleric is in a natural environment, he gains the natural casting ability, as the druid class ability (see Natural Casting, \pref{Drd:Natural Casting}).
            He can use this ability at up to Close range.

        \subsubsection{Ex-Clerics}
            A cleric who grossly violates the code of conduct required by his god loses all spells and supernatural cleric class abilities.
            He cannot thereafter gain levels as a cleric of that god until he atones (see the \spell{atonement} spell description).

\section{Druid}\label{Druid}
    \begin{dtable}
        \lcaption{Druid Progression}
        \begin{dtabularx}{\columnwidth}{>{\ccol}p{\levelcol} >{\centering}p{\babcolavg} >{\ccol}p{3em} c >{\lcol}X}
            \tb{Level} & \tb{Combat Prowess} & \tb{Active Aspects} & \tb{Spells} & \tb{Special} \\
            \hline
            \druidprogressionrow{1}  & \tdash & 2 & Metal abhorrence, spells, wild speech \\
            \druidprogressionrow{2}  & 1      & 3 & Rituals, wild aspect                  \\
            \druidprogressionrow{3}  & 1      & 3 & Natural casting (10 ft.)              \\
            \druidprogressionrow{4}  & 1      & 4 & Wild aspect                           \\
            \druidprogressionrow{5}  & 2      & 4 & \tdash                                \\
            \druidprogressionrow{6}  & 2      & 5 & Wild aspect                           \\
            \druidprogressionrow{7}  & 2      & 5 & \tdash                                \\
            \druidprogressionrow{8}  & 2      & 6 & Wild aspect                           \\
            \druidprogressionrow{9}  & 3      & 6 & \tdash                                \\
            \druidprogressionrow{10} & 3      & 7 & Wild aspect                           \\
            \druidprogressionrow{11} & 3      & 7 & Natural casting (Close)               \\
            \druidprogressionrow{12} & 3      & 8 & Natural aspect                        \\
            \druidprogressionrow{13} & 4      & 8 & \tdash                                \\
            \druidprogressionrow{14} & 4      & 9 & Natural aspect                        \\
            \druidprogressionrow{15} & 4      & 9 & \tdash                                \\
            \druidprogressionrow{16} & 4      & 10 & Natural aspect                        \\
            \druidprogressionrow{17} & 5      & 10 & \tdash                                \\
            \druidprogressionrow{18} & 5      & 11 & Natural aspect                        \\
            \druidprogressionrow{19} & 5      & 11 & Natural casting (Medium)              \\
            \druidprogressionrow{20} & 5      & 12 & Avatar of nature, natural aspect      \\
        \end{dtabularx}
    \end{dtable}

    \classbasics{Alignment} Neutral good, lawful neutral, neutral, chaotic neutral, or neutral evil.

    \classbasics{Class Skills}
    \subparhead{Strength} Climb, Jump, Sprint, Swim.
    \subparhead{Dexterity} Acrobatics, Ride, Stealth.
    \subparhead{Intelligence} Heal, Knowledge (geography, nature).
    \subparhead{Perception} Awareness, Creature Handling, Survival.
    \subparhead{Other} Bluff, Intimidate, Persuasion.

    \subsection{Base Class Abilities}
        A character with druid as a base class gains the following abilities.

        \classbasics{Skill Points} 10.

        \classbasics{Combat Prowess} \plus2. This benefit is included in the class table.

        \classbasics{Defenses} \plus4 Fortitude, \plus2 Mental.

        \cf{Drd}{Weapon and Armor Proficiency}
        Druids are proficient with simple weapons, any one other weapon group, scimitars, sickles, and slings.
        In addition, druids are proficient with light armor, medium armor, and shields.
        However, a druid cannot use metal armor; see the Metal Abhorrence ability, below.

        \cf{Drd}{Druidic Language}
        Druids know Druidic, a secret language known only to druids, in addition to their normal languages.
        Druids are forbidden to teach this language to nondruids.
        Druidic has its own alphabet.

        \cf{Drd}{Wild Speech}[Su]
        Druids can communicate with animals.
        As a standard action, the druid can choose a type of animal, such as owl or wolf.
        She gains the ability to speak to and understand animals of that type for 5 minutes.
        A druid can use this ability a number of times per day equal to her Perception or half her druid level, whichever is higher.

        This ability doesn't make the animals any more friendly or cooperative than normal.
        Furthermore, wary and cunning animals are likely to be terse and evasive, while the more stupid ones make inane comments.
        If an animal is friendly toward the druid, she may be able to convince it to do some favor or service.

        \cf{Drd}{Enhanced Nature Power}
        The druid gains a \plus2 bonus to her nature power.

    \subsection{Class Abilities}
        All druids have the following abilities.

        \cf{Drd}{Metal Abhorrence}
        The oaths that druids swear as part of their initiation prohibit them from wearing armor made of metal.
        A druid who wears prohibited armor or carries a prohibited shield is unable to cast druid spells or use any of her supernatural class abilities while doing so and for 24 hours thereafter.
        (A druid may also wear wooden armor that has been altered by the \spell{ironwood} ritual so that it functions as though it were steel. See the ritual description.)

        \cf{Drd}{Nature Power}[Su]
        The strength of a druid's spells and abilities are determined by her connection to nature.
        Normally, her nature power is equal to her character level.

        \cf{Drd}{Spells}
        A druid casts nature spells using her connection to nature.
        The maximum spell level a druid can learn or cast is equal to half her druid level.
        A druid's \glossterm{spellpower} with nature spells is normally equal to her nature power.

        A druid begins play knowing two first-level spells.
        Every even level, she learns an additional spell of any level she has access to.
        In addition, each time she gains a level, she may trade one of her existing spells for a different spell known.
        However, she must always know at least one spell of every level she has access to.
        A druid's spells are drawn from the spells on the nature spell list (see \pcref{Nature Spells}).

        Druids have a limit on the number of spells they can cast, as given on \trefnp{Druid Spell Slots}.
        A druid can regain all spent spell slots by meditating and becoming in tune with her environment.
        This process takes one hour in a natural environment, or eight hours elsewhere.

        \begin{dtable}
            \lcaption{Druid Spell Slots}
            \centering
            \begin{dtabularx}{\columnwidth}{>{\ccol}X *{9}{>{\ccol}p{\spellcol}}}
                & \multicolumn{9}{c}{\tb{---{}---{}---{}---{}---{}---{}---{}---Spell Level---{}---{}---{}---{}---{}---{}---{}---}} \\
                \hline
                \tb{Level} & \tb{1st} & \tb{2nd} & \tb{3rd} & \tb{4th} & \tb{5th} & \tb{6th} & \tb{7th} & \tb{8th} & \tb{9th} \\
                1st  & 1 & \tdash & \tdash & \tdash & \tdash & \tdash & \tdash & \tdash & \tdash \\
                2nd  & 2 & \tdash & \tdash & \tdash & \tdash & \tdash & \tdash & \tdash & \tdash \\
                3rd  & 3 & \tdash & \tdash & \tdash & \tdash & \tdash & \tdash & \tdash & \tdash \\
                4th  & 3 & 1  & \tdash & \tdash & \tdash & \tdash & \tdash & \tdash & \tdash \\
                5th  & 3 & 2  & \tdash & \tdash & \tdash & \tdash & \tdash & \tdash & \tdash \\
                6th  & 3 & 3  & 1  & \tdash & \tdash & \tdash & \tdash & \tdash & \tdash \\
                7th  & 3 & 3  & 2  & \tdash & \tdash & \tdash & \tdash & \tdash & \tdash \\
                8th  & 3 & 3  & 3  & 1  & \tdash & \tdash & \tdash & \tdash & \tdash \\
                9th  & 3 & 3  & 3  & 2  & \tdash & \tdash & \tdash & \tdash & \tdash \\
                10th & 3 & 3  & 3  & 3  & 1  & \tdash & \tdash & \tdash & \tdash \\
                11th & 3 & 3  & 3  & 3  & 2  & \tdash & \tdash & \tdash & \tdash \\
                12th & 3 & 3  & 3  & 3  & 3  & 1  & \tdash & \tdash & \tdash \\
                13th & 3 & 3  & 3  & 3  & 3  & 2  & \tdash & \tdash & \tdash \\
                14th & 3 & 3  & 3  & 3  & 3  & 3  & 1  & \tdash & \tdash \\
                15th & 3 & 3  & 3  & 3  & 3  & 3  & 2  & \tdash & \tdash \\
                16th & 3 & 3  & 3  & 3  & 3  & 3  & 3  & 1  & \tdash \\
                17th & 3 & 3  & 3  & 3  & 3  & 3  & 3  & 2  & \tdash \\
                18th & 3 & 3  & 3  & 3  & 3  & 3  & 3  & 3  & 1  \\
                19th & 3 & 3  & 3  & 3  & 3  & 3  & 3  & 3  & 3  \\
                20th & 3 & 3  & 3  & 3  & 3  & 3  & 3  & 3  & 3  \\
            \end{dtabularx}
        \end{dtable}

        \cf{Drd}[2nd]{Rituals}
        The druid gains the ability to perform nature rituals.
        She does not automatically learn rituals, but must find a ritual book.
        For details, see \pcref{Rituals}.

        \cf{Drd}[2nd]{Wild Aspect}[Su]
        The druid gains the ability to embody an aspect of an animal.
        She chooses two wild aspects from the list below.
        Some wild aspects have minimum druid levels, as indicated in the title of the aspect.
        At her 4th druid level, and every even druid level thereafter, the druid learns an additional wild aspect.

        Unless otherwise noted, embodying a wild aspect is a standard action.
        Embodying a wild aspect costs a spell slot of any level.
        Once embodied, a wild aspect persists until the druid embodies a new aspect or dismisses the aspect.
        If the druid embodies a new wild aspect, the previous aspect ends immediately.
        All wild aspects can be dismissed as a swift action.

        All wild aspects are supernatural abilities unless otherwise noted.

        The descriptions below describe the effects of the aspect.
        With many aspects, the druid's appearance also changes to match the aspect, but this is not described.
        Different druids change in different ways.
        For example, one druid might gain unusually large eyes when embodying the low-light vision aspect, while another might change her irises into slits, like a cat, when embodying the same aspect.
        The changes made are up to the druid, but cannot be used to gain an additional substantive benefit beyond the effects given in the description of the aspect.

        Many wild aspects grant natural weapons.
        See \pcref{Natural Weapons}, for details about natural weapons.

        \subcf{Armaments of the Bear}
        The druid's mouth and hands transform, allowing her to perform bite and claw attacks.
        The bite attack deals 1d8 damage for a Medium druid, and the claws deal 1d6 damage.
        \subcf{Climb}
        The druid gains a \glossterm{climb speed} equal to her base land speed.
        \subcf{Senses}
        The druid, gains low-light vision.
        She treats sources of light as if they had double their normal illumination range.
        If she already has low-light vision, she doubles its benefit, allowing her to treat sources of light as if they had four times their normal illumination range.
        In addition, she gains \glossterm{darkvision} out to 50 feet, allowing her to see in complete darkness.
        If she already has darkvision, she increases its range by 50 feet.
        \subcf{Woodland Stride}
        The druid may move through any sort of undergrowth (such as natural thorns, briars, overgrown areas, and similar terrain) at her normal speed and without taking damage or suffering any other impairment.
        However, plants magically manipulated to impede motion still affect her.

        \subcf{4th -- Animal Affinity}
        The druid gains a \plus5 bonus to Creature Handling and Ride checks.
        \subcf{4th -- Constrict}
        The druid's body transforms, improving her grappling abilities.
        She gains a \plus5 bonus to accuracy with grapple attacks.
        In addition, she gains a constrict attack.
        This attack deals 1d10 damage for a Medium druid, but it can only be used against a foe she is grappling with.
        \subcf{4th -- Gore}
        The druid's head transforms, allowing her to perform a gore attack.
        The attack deals 1d8 damage for a Medium druid.
        In addition, if the druid hits with a natural attack, she may attempt to shove her foe as an immediate action.
        \subcf{4th -- Lope}
        The druid gains the ability to move on all four limbs.
        When doing so, she gains a \plus30 foot bonus to her land speed, up to a maximum of double her original speed.
        When not using her hands to move, her ability to use her hands is unchanged.
        \subcf{4th -- Slither}
        The druid gains a \glossterm{climb speed} equal to her base land speed.
        She does not need to use her hands to climb in this way.
        In addition, she gains a bite attack that deals 1d8 damage for a Medium druid.

        \subcf{6th -- A Thousand Faces}
        The druid's appearance changes, as if using the \spell{disguise self} spell.
        This affects the druid's body, but not her possessions.
        It is not an illusory effect, but a minor physical alteration of the druid's appearance, within the limits described for the spell.
        \subcf{6th -- Enhanced Natural Weapons}
        The druid's natural weapons gain a \glossterm{enhancement bonus} equal to one third of her nature power.
        This functions like an enhancement bonus on a weapon, increasing her damage and offensive legend points per day (see \pcref{Weapon Enhancement Bonuses}).
        \subcf{6th -- Hawk}
        The druid grows wings, granting her a glide speed equal to her land speed.
        See \pcref{Gliding}, for more details.
        In addition, her feet transform, allowing her to perform a talon attack.
        The attack deals 1d6 damage for a Medium druid.
        \subcf{6th -- Scent}
        The druid gains the scent ability, granting her a \plus10 bonus to scent-based Awareness checks (see \pcref{Awareness}).
        \subcf{6th -- Shrink}
        The druid shrinks by a size category.
        This functions like the \spell{reduce person} spell.
        This is a sizing effect.
        \subcf{6th -- Totemic Mind}
        The druid gains a \plus2 bonus to her choice of Intelligence, Perception, or Willpower.
        This cannot increase the chosen attribute to be higher than her highest physical attribute.
        \subcf{6th -- Totemic Power}
        The druid gains a \plus2 bonus to her choice of Strength, Dexterity, or Constitution.
        This cannot increase the chosen attribute to be higher than her highest mental attribute.

        \subcf{8th -- Grow}
        The druid increases in size by one size category.
        This functions as the \spell{enlarge} spell.
        This is a sizing effect, and does not stack with other sizing effects.
        \subcf{8th -- Natural Grab}
        If the druid hits with a natural attack, she may attempt to grapple her foe as an immediate action.
        \subcf{8th -- Natural Trip}
        If the druid hits with a natural attack, she may attempt to trip her foe as an immediate action.
        \subcf{8th -- Venom}
        If the druid hits with a natural attack, she may inject poison into her foe as an immediate action.
        At the end of every round, the druid makes a nature power vs. Fortitude attack against all creatures she has poisoned.
        The effects of the poison are described below.
        \begin{itemize}
            \item First success: the target is \sickened.
            \item Second success: the target is \staggered.
            \item Third success: the target is \nauseated.
            \item Third failure: the target is no longer poisoned, and any lingering effects from the poison end.
        \end{itemize}
        %TODO poison stacking rules
        \par In addition, the druid gains a bite attack that deals 1d8 damage for a Medium druid.

        \subcf{10th -- Natural Renewal}
        At the end of each round, the druid heals hit points equal to her nature power.
        \subcf{10th -- Swiftstrike}
        The druid's attack speed increases.
        When she makes a standard attack, she may make an additional strike.
        This strike must be made with a natural weapon.
        This effect does not stack with similar effects that grant extra strikes.
        \subcf{10th -- Wings}
        The druid grows wings, granting her a fly speed equal to her land speed with average maneuverability.
        See \pcref{Flying}, for details.
        She can only fly for a number of rounds equal to 3 \add half her Constitution.
        After that limit is reached, she must rest for 5 minutes before flying again.
        \subcf{10th -- Wolfpack}
        Overwhelmed foes the druid threatens increase their overwhelm penalties by 2.

        \cf{Drd}[3rd]{Natural Casting}[Ex]
        Whenever the druid casts a nature spell with an area that originate from her, such as most cone or line spells, she may cause the spell to originate from any location within 10 feet of her.
        All other aspects of the spell are unchanged.

        For example, a druid casting \spell{burning hands} could create a cone of fire originating from 10 feet to her right.
        The cone would extend 20 feet out from that point, as normal.
        If the druid directed the cone back towards her, she could potentially be affected by the spell.

        At 11th level, this ability's range improves to \rngclose.
        At 19th level, this ability's range improves to \rngmed.

        \cf{Drd}[5th]{Multiple Aspect}[Su]
        The druid gains the ability to embody two wild aspects at once.
        At 9th level, and every 4 levels thereafter, the druid gains the ability to embody an additional wild aspect at the same time.

        \cf{Drd}[12th]{Natural Aspect}[Su]
        The druid gains the ability to embody aspects of the natural world, including the elements, in addition to those of animals.
        She adds the options below to the list of abilities she can gain with her wild aspect ability.

        \subcf{Fire Shield} The druid gains the effects of the \spell{fire shield} spell.
        Its spellpower is equal to her nature power or her Willpower, whichever is higher.
        \subcf{Fluid Motion} The druid gains the effects of the \spell{freedom} spell.
        Its spellpower is equal to her nature power or her Dexterity, whichever is higher.
        In addition, she can move through her enemies, treating their space as difficult terrain.
        \subcf{Stoneskin} The druid gains the effects of the \spell{stoneskin} spell.
        Its spellpower is equal to her nature power or her Constitution, whichever is higher.
        \subcf{14th -- Flight} The druid gains a fly speed equal to her base land speed.
        See \pcref{Flying}, for details.
        She can only fly in this way for up to 5 rounds.
        After that limit is reached, she must rest for 5 minutes before flying again.
        \subcf{14th -- Earth Glide}
        The druid gains the earth glide ability, as the \spell{earth glide} spell.
        She can only glide through earth for up to 5 rounds.
        After that limit is reached, she must rest for 5 minutes before gliding through earth again.
        \subcf{18th -- Solar Radiance}
        The druid continuously radiates bright light out to a 500 foot radius (and shadowy illumination for an additional 500 feet).
        The illumination is so bright that she becomes hard to look at.
        Any creature attacking her from within the radius of bright light becomes \partiallyblinded for 2 rounds after the attack.

\begin{comment}

    \subcf{Heart of Air} The druid can breathe in any environment, and is immune to \spell{sickening cloud}
    and similar effects.
    In addition, she falls at half speed and takes no falling damage.
    \subcf{Heart of Earth} The druid gains the effects of the \spell{stoneskin}
    spell, with a spellpower equal to her druid level or Constitution.
    \subcf{Heart of Fire} The druid gains the effects of the warm version of the \spell{fire shield}
    spell, with a spellpower equal to her druid level or Constitution.
    \subcf{Heart of the Sun}
    The druid constantly radiates bright light out to a 100 foot radius (and shadowy illumination for an additional 100 feet).
    The illumination is so bright that she becomes hard to look at.
    Any creature attacking her from within the radius of bright light becomes dazzled for 5 rounds after the attack.
    \subcf{Heart of Oak} The druid gains the effects of the \spell{stoneskin}
    spell, with a spellpower equal to her druid level or Constitution, except that the damage reduction is overcome by fire instead of adamantine weapons.
    \subcf{Heart of Water} The druid gains the effects of the \spell{freedom}
    spell.
    \subcf{15th -- Air Mantle}
    The druid is surrounded by a mantle of air.
    Thrown and projectile weapons have a 50\% chance to miss her while this effect is active.
    Unusually large weapons, such as a giant's boulders, may suffer a decreased miss chance as appropriate to their size.
    \subcf{15th -- Aqueous Step}
    Wherever the druid moves, she leaves a path of animated water that can grab creatures.
    Whenever a creature crosses the path, the druid makes a Reflex attack to trip the creature, causing it to fall prone and waste the rest of its movement.
    Her accuracy is equal to her druid level \add her Constitution.
    \subcf{15th -- Flaming Step}
    Wherever the druid moves, she leaves a path of burning flame behind her that lasts for 1 round.
    Whenever a creature crosses the path, the druid makes a Reflex attack to deal damage to the creature.
    The attack deals 1d8 points of fire damage per two druid levels.
    Her accuracy is equal to her druid level \add her Constitution.
    A failed attack deals half damage.
    \subcf{15th -- Lifegiving Step}
    Wherever the druid moves, she leaves a path of small, living plants that entangle foes for 1 round.
    Whenever a creature crosses the path, the druid makes a Reflex attack to entangle the creature, causing it to waste the rest of its movement.
    Her accuracy is equal to her druid level \add her Constitution.
    The plants appear on any surface, and will continue to grow if they can survive, though they may die quickly if they appear on inhospitable terrain.
    \subcf{17th -- Flight}
    The druid gains a fly speed equal to her land speed, with good maneuverability.
    She may remain flying for up to 5 rounds at a time.
    After that, she must land for 1 round before she can fly again.
    See \pcref{Flying}, for more details.
    \subcf{17th -- Flaming Soul}
    The druid gains the fire subtype, making her immune to fire but giving her a 50\% vulnerability to cold damage.
    In addition, whenever she deals fire damage to a creature, the creature is \ignited for 5 rounds.
    \subcf{17th -- Sunblessed Rejuvenation}
    The druid gains fast healing equal to her druid level as long as she remains in sunlight or touches a plant of her size or larger.
    \subcf{17th -- Sunscour}
    This aspect functions like the heart of the sun natural aspect, except that it also suppresses shadow effects and the visual components of illusions within the area of bright light.

    \subcf{17th -- Water's Flow}
    As a swift action, the druid can transform herself into a rushing flow of water with a volume roughly equal to her normal volume until the end of her turn.
    In this form, she may move wherever water could go, but she cannot take other actions, such as jumping, attacking, or casting spells.
    Her speed is halved when moving uphill and doubled when moving downhill.
    She may move through squares occupied by creatures or threatened by blocking enemies without penalty.
    She may return to her normal form as a free action.
    \par If the water is split, she may reform from anywhere the water has reached, to as little as a single ounce of water.
    If not even an ounce of water exists contiguously, her body reforms from the largest available parts of water, cut into pieces of appropriate size.
    This usually causes the druid to die.

\end{comment}

    \cf{Drd}[20th]{Avatar of Nature}[Su]
    If the druid dies, except if by old age, she may choose to have her body and soul become an instrument of nature's will.
    Her body immediately decomposes or otherwise disappears, and her soul does not travel to an afterlife.
    She has no physical form, and cannot use any of her normal special abilities.
    Instead, she has a fly speed of 100 feet (perfect maneuverability).
    As a standard action, she can temporarily possess any living plants or animals within a 10 mile radius of the place of her death.

    While possessing a living plant or animal, she can see through its senses and control its actions completely.
    In addition, she may cast spells, and the spells take effect as if the plant or animal had cast them.
    She uses the plant or animal's position to determine range, visible targets, and so on.
    She does not require verbal or somatic components to cast her spells in this form, but is unable to cast spells or perform rituals that require material components or focus objects.

    While not possessing a plant or animal, the druid can meditate to regain spent spell slots, or she can focus on reincarnating her physical form.
    Creating a new body in this way takes 12 straight hours of concentration.
    At the end of that time, the druid is reincarnated in a new body in her location, as the \spell{reincarnate} ritual, except that she can choose her race from among the races listed (not including the ``Other'' race).

    While she is an avatar of nature, a druid does not age and does not die of old age.
    She can continue to exist in this form indefinitely.

    \subsection{Ex-Druids}
        A druid who ceases to revere nature, changes to a prohibited alignment, or teaches the Druidic language to a nondruid loses all spells and supernatural druid class abilities.
        She cannot thereafter gain levels as a druid until she atones (see the \spell{atonement} ritual).

    \subsection{Variant Druids}

        \subsubsection{Blighter}

            Blighters draw power from nature, as do other druids. However, while other druids revere nature and draw power from it gently, blighters steal power from nature forcefully. Wherever a blighter goes, destruction and death surely follows.

            \altcf{Blight} Instead of meditating to regain spell slots, a blighter draws power from her environment forcefully.
            This affects a \areahuge radius zone centered on her, and the process takes 1 minute of concentration.
            At the end of every round, every living thing in the area other than the blighter takes damage equal to her nature power.
            All inanimate plants of Huge size or smaller immediately wither and die.
            The earth becomes cracked and infertile, and any nutrients from the soil are destroyed.
            This ability has no effect on artificial environments or materials, such as metal or worked stone.
            At the end of the minute, the blighter regains her spent nature spell slots.

            A blighter can only blight her surroundings in this way once per hour.
            If her surroundings are already blighted or are not natural terrain, she cannot use this ability to regain her spells.
            Instead, she must meditate for 8 hours to slowly draw power from her surroundings, as a normal druid.

            \altcf{Spells} As normal, except that a blighter adds all Vivimancy arcane spells to her spell list.

            \altcf[2nd]{Wild Speech} As normal, except that a blighter gains a \plus5 bonus to Intimidate against her wild speech targets, and a \minus5 penalty to Persuasion.

            \altcf{Natural Casting} The blighter does not gain this ability.

            \altcf[10th]{Blightcasting}

            \altcf[20th]{Improved Blightcasting}

        \subsubsection{Rotbringer}

            While most druids seek to emulate and interact with animals, rotbringers focus on the power of fungi, decay, and regeneration.

            \altcf{Invoke Rot} Instead of meditating to regain spell slots, a rotbringer accelerates the natural forces of decomposition and decay on her environment.
            This affects a \areahuge radius zone centered on her, and the process takes 1 minute of concentration.
            All organic objects of Huge size or smaller, such as plants and corpses, decompose.
            This decomposition kills inanimate, living plants.
            All organic objects, regardless of size, are covered with various fungi.
            This ability has no effect on artificial environments or materials, such as metal or worked stone.
            At the end of the minute, the rotbringer regains her spent nature spell slots.

            If the rotbringer decomposes a Huge object with this ability, or a combination of smaller objects equivalent in size to a Huge object, she gains an bonus nature spell slot of her highest available spell level.
            This extra spell slot lasts until it is used, or until she regains her spell slots again.

            A rotbringer can only invoke rot on her surroundings in this way once per hour.
            If her surroundings are already decomposed or are not natural terrain, she cannot use this ability to regain her spells.
            Instead, she must meditate for 8 hours to slowly draw power from her surroundings, as a normal druid.

            \altcf[2nd]{Wild Speech} The rotbringer gains the ability to speak with plants at 2nd level.
            She gains the ability to speak with animals at 6th level, instead of at 2nd level.

            \altcf[3rd]{Wild Aspect} The rotbringer does not gain this ability.

            \altcf[3rd]{Rot Spell} The druid learns an additional spell slot and spell known.
            The spell must be taken from the following list of spells.
            The spell's level cannot exceed half her druid level.
            If she already knows a spell from the list at every spell level she has access to, she may instead learn any nature spell (see \pcref{Nature Spells}).

            At 5th level, and every odd level, the druid may learn a new spell.

            \begin{dtable}
                \begin{dtabularx}{\columnwidth}{l X}
                    \tb{Spell level} & \tb{Rotbringer Spells} \\
                    1st & \spell{excrete slime}, \spell{lesser regeneration} \\
                    2nd & \spell{fungal growth} \\
                    3rd & \spell{rotburst} \\
                    4th & \spell{poison} \\
                    6th & \spell{regeneration} \\
                    7th & \spell{greater rotburst} \\
                \end{dtabularx}
            \end{dtable}

            \altcf[7th]{Fungal Armor} The rotbringer becomes covered in fungus that protects her from attacks. She gains a \plus1 bonus to Armor and Fortitude defense.

            This bonus increases by 1 at her 7th druid level, and every 4 druid levels thereafter.

\section{Fighter}\label{Fighter}
    \begin{dtable}
        \lcaption{Fighter Progression}
        \begin{dtabularx}{\columnwidth}{>{\ccol}p{\levelcol} >{\ccol}p{\babcolgood} >{\lcol}X}
            \tb{Level} & \tb{Combat Prowess} & \tb{Special} \\
            \hline
            \fighterprogressionrow{1}   & Advanced training, weapon discipline \\
            \fighterprogressionrow{2}   & Combat feat                         \\
            \fighterprogressionrow{3}   & Armor discipline                   \\
            \fighterprogressionrow{4}   & Adaptive combat                     \\
            \fighterprogressionrow{5}   & Combat discipline                   \\
            \fighterprogressionrow{6}   & Combat feat                         \\
            \fighterprogressionrow{7}   & Improved weapon discipline           \\
            \fighterprogressionrow{8}   & Adaptive combat                     \\
            \fighterprogressionrow{9}   & Improved armor discipline          \\
            \fighterprogressionrow{10}  & Combat feat, swift warrior          \\
            \fighterprogressionrow{11}  & Improved combat discipline          \\
            \fighterprogressionrow{12}  & Adaptive combat                     \\
            \fighterprogressionrow{13}  & Greater weapon discipline            \\
            \fighterprogressionrow{14}  & Combat feat                         \\
            \fighterprogressionrow{15}  & Greater armor discipline           \\
            \fighterprogressionrow{16}  & Adaptive combat                     \\
            \fighterprogressionrow{17}  & Greater combat discipline           \\
            \fighterprogressionrow{18}  & Combat feat                         \\
            \fighterprogressionrow{19}  & True discipline                     \\
            \fighterprogressionrow{20}  & Adaptive combat, combat supremacy   \\
        \end{dtabularx}
    \end{dtable}

    \classbasics{Alignment} Any.

    \classbasics{Class Skills}
    \subparhead{Strength} Climb, Jump, Sprint, Swim.
    \subparhead{Dexterity} Acrobatics, Escape Artist, Ride.
    \subparhead{Perception} Awareness.
    \subparhead{Other} Bluff, Intimidate, Persuasion.

    \subsection{Base Class Abilities}
        A character with fighter as a base class gains the following abilities.

        \classbasics{Skill Points} 10.

        \classbasics{Combat Prowess} \plus2. This benefit is included in the class table.

        \classbasics{Defenses} \plus4 Fortitude, \plus2 Mental.

        \cf{Ftr}{Weapon and Armor Proficiency}
        A fighter is proficient with simple weapons, any four other weapon groups,  all armor (heavy, medium, and light), and shields.

        \cf{Ftr}{Advanced Training}[Ex]
        A fighter treats his combat prowess as if it were 2 points higher than it actually is for the purpose of meeting feat prerequisites.

    \subsection{Class Abilities}
        All fighters have the following abilities.

        \cf{Ftr}{Weapon Discipline}
        The fighter's training grants him additional capability when using his weapons.
        He may choose a weapon group, or he may choose to train equally with all weapons.
        If he chooses a weapon group, he gains a \plus1 bonus to accuracy with weapons from that group.

        If he chooses not to focus on a specific group of weapons, he gains the ability to become proficient with any weapon group if he spends 1 hour training with a weapon from that group.
        He may only keep this proficiency with one weapon group at a time; if he trains with a new weapon group, he loses his proficiency in the previous group.

        \cf{Ftr}[2nd]{Combat Feat}
        The fighter gets a bonus combat-oriented feat.
        This bonus feat must be drawn from the list of combat feats in \tref{Combat Feats}.
        He must still meet all prerequisites for the bonus feat.
        He gains an additional bonus feat at his 6th fighter level and every four fighter levels thereafter (6th, 10th, 14th, and 18th).

        \cf{Ftr}{Armor Discipline}
        A fighter's training grants him additional capability in armor.
        He must choose to improve his agility or his resilience in armor.
        This applies to all armor discipline abilities the fighter has.

        If he improves his agility, he treats body armor he wears as less encumbering.
        He reduces its \glossterm{encumbrance penalty} by 2 its arcane spell failure by 10\%.
        In addition, he treats it were one encumbrance category lighter than it is whenever doing so would be beneficial for him.
        Heavy armor is treated as medium armor, medium armor is treated as light armor, and light armor is treated as being unarmored.
        This can remove the speed reduction and reduced Dexterity associated with the armor's encumbrance, as appropriate for the new encumbrance of the fighter's armor.

        If he improves his resilience, he gains damage reduction against physical damage equal to his character level. This allows him to ignore the first points of damage he would take each round.

        \cf{Ftr}[4th]{Adaptive Combat}
        The fighter gains a flexible bonus feat which he can change periodically.
        The fighter chooses a number of combat feats equal to half his fighter level or half his Intelligence, whichever is higher.
        These feats comprise his adaptive feat pool.
        The fighter gains one of the feats from his adaptive feat pool as a bonus feat.
        By training for an hour, the fighter can change his current adapative combat feat to one of the other feats in his adaptive style feat pool.
        He must meet the prerequisites for the new feat.

        \par An adaptive style feat may be used normally as prerequisites for other feats or abilities.
        However, if an adaptive style feat is used as a prerequisite, it cannot be changed until the fighter no longer needs to use it as a prerequisite, such as might happen if the fighter takes the feat as a normal feat or bonus feat.

        \par In order to gain an adaptive style feat, it must be reasonably possible to do training related to the new feat.
        For example, a fighter could not gain Weapon Focus in axes without at least one axe available to train with.

        The fighter gains an additional adaptive combat feat at his 8th fighter level and every four fighter levels thereafter (8th, 12th, 16th, and 20th).
        He may change all of his adaptive combat feats at once when he trains.

        \cf{Ftr}[5th]{Combat Discipline}
        The fighter can use his superior training and focus to keep fighting in the face of debilitating effects.
        When a fighter is initially affected by one of the conditions listed on \trefnp{Combat Discipline Conditions}, he may use his combat discipline ability to instead suffer the mitigated condition one column to the right.
        He can suppress the condition up to 5 rounds.

        \par Using combat discipline takes no action, and can be done at any time, even when it isn't the fighter's turn.
        A fighter may use this ability a number of times per day equal to his Willpower or half his fighter level, whichever is higher.
        However, he cannot mitigate more than one condition at a time.
        If the fighter attempts to mitigate a new condition, the old condition resumes its normal effect immediately.

        \begin{dtable}
            \lcaption{Combat Discipline Conditions}
            \begin{dtabularx}{\columnwidth}{l *{3}{>{\lcol}X}}
                \tb{Original Condition} & \tb{Mitigated Condition} & \tb{Mitigated Condition} & \tb{Mitigated Condition} \\
                \hline
                Panicked              & Frightened        & Shaken & None \\
                Petrified             & Paralyzed         & Slowed & None \\
                Blinded               & Visually impaired & None   & \tdash   \\
                Confused              & Disoriented       & None   & \tdash   \\
                Exhausted             & Fatigued          & None   & \tdash   \\
                Nauseated             & Sickened          & None   & \tdash   \\
                Severely impaired     & Impaired          & None   & \tdash   \\
                Stunned               & Dazed             & None   & \tdash   \\
                %Entangled             & None              & \tdash     & \tdash   \\
                Deafened              & None              & \tdash     & \tdash   \\
                Fascinated            & None              & \tdash     & \tdash   \\
                Ignited\fn{2}         & None              & \tdash     & \tdash   \\
                Immobilized           & None              & \tdash     & \tdash   \\
                Negative level\fn{3}  & None              & \tdash     & \tdash   \\
                Slowed                & None              & \tdash     & \tdash   \\
                Vulnerable            & None              & \tdash     & \tdash   \\
            \end{dtabularx}
            1.  Mitigate up to half fighter level or half Constitution, whichever is greater. \\
            2.  Mitigates the impairment, but does not prevent the fighter from taking 1d6 fire damage per round until the fire is put out.  \\
            3.  Mitigate a single negative level. \\
        \end{dtable}

        \par A fighter can never use this ability more than once against a single source.
        For example, if a fighter is confused by a \spell{confusion} spell, he can use this ability to become disoriented instead of confused, but he can't then expend a second use to stop being disoriented.
        The lesser condition that this ability imposes may be cured or removed normally, but doing so does not affect the resurgence of the condition the fighter was originally afflicted with.
        If a fighter uses this ability to mitigate or negate a condition which he must suffer as a sacrifice or cost to gain some benefit, he automatically forfeits the benefit he would have gained.

        \cf{Ftr}[7th]{Improved Weapon Discipline}
        The fighter's training in his chosen weapons improves.
        He increases the \glossterm{critical multiplier} of his chosen weapons by 1.
        In addition, if he chose a specific weapon group, he gains a \plus10 bonus to resist disarm attempts against his chosen weapons.
        If he did not, he can apply all weapon group-specific feats he has to any weapon group that he trains with for 1 hour.
        He retains this benefit for one week after the training.

        \cf{Ftr}[9th]{Improved Armor Discipline}
        The fighter's training with his armor improves.
        If he chose agility, he treats body armor he wears as even less encumbering.
        He reduces its \glossterm{encumbrance penalty} by 4 and decreases its arcane spell failure by 20\%.
        In addition, he treats all it as if it were two encumbrance categories lighter than it is whenever doing so is beneficial for him.
        This does not stack with the benefit of the armor discipline ability.

        If he chose resilience, he may apply his damage reduction from the armor discipline ability against all damage, including from magical attacks.

        \cf{Ftr}[10th]{Swift Warrior}
        The fighter can take two swift or immediate actions each round (see \pcref{Swift and Immediate Actions}).

        \cf{Ftr}[11th]{Improved Combat Discipline}
        The fighter's ability to keep fighting despite negative influence improves.
        He can reduce conditions by two steps instead of one when using combat discipline.
        For example, a stunned fighter who used combat discipline would instead be staggered.
        \par In addition, a fighter may use combat discipline to reduce any penalties he suffers to his accuracy, damage, checks, or defenses by 2, even if the source of the penalty is not listed on the combat discipline chart.
        \par The fighter may also mitigate up to two conditions at once.

        \cf{Ftr}[13th]{Greater Weapon Discipline}
        The fighter's training in his chosen weapons becomes still greater.
        He increases the \glossterm{critical range} of his chosen weapons by 1.
        This increase stacks with any other effects that affect critical range.
        For example, a fighter using the Heartseeker combat style (see \featpcref{Heartseeker}) would have a critical range of 18-20.

        \cf{Ftr}[15th]{Greater Armor Discipline}
        The fighter's training in his chosen armor becomes still greater.
        If he chose agility, he ignores all encumbrance penalties and arcane spell failure from body armor he wears.
        In addition, he treats body armor he wears as if it were three encumbrance categories lighter than it actually is whenever doing so would be beneficial to him.
        Finally, he gains a \plus10 foot bonus to his speed.
        This does not stack with the benefits of armor discipline or improved armor discipline.

        If he chose resilience, he doubles the damage reduction granted by his armor discipline ability.

        \cf{Ftr}[17th]{Greater Combat Discipline}
        The fighter's ability to keep fighting despite influence becomes still greater.
        When using combat discipline, he may ignore any condition listed on the combat discipline chart.
        In addition, he may use combat discipline to be \severelyimpaired with attacks and checks rather than suffer any non-damaging condition not listed on the chart.

        \par The fighter may also mitigate up to three conditions at once.

        \cf{Ftr}[19th]{True Discipline}
        The fighter's discipline in his chosen area is beyond equal.
        He must choose either weapon discipline, armor discipline, or combat discipline.
        Depending on which discipline he chooses, he gains a different bonus.
        \subcf{True Weapon Discipline}
        When the fighter makes a \glossterm{standard attack}, he can make an additional \glossterm{strike}.
        This does not stack with any other effects which grant extra strikes.
        \subcf{True Armor Discipline}
        If the fighter chose agility, he gains a \plus10 bonus to Reflex defense.

        If the fighter chose resilience, he gains a \plus10 bonus to Fortitude defense.
        \subcf{True Combat Discipline}
        The fighter can use combat discipline to be \impaired with attacks and checks instead of suffering any nondamaging negative effect with a duration.
        He may also mitigate up to four conditions at once.

        \cf{Ftr}[20th]{Combat Supremacy}
        Whenever a fighter successfully deals damage to a creature with a physical attack, the struck creature is \severelyimpaired with attacks and checks for 2 rounds.

\section{Monk}\label{Monk}
    \begin{dtable}
        \lcaption{Monk Progression}
        \begin{dtabularx}{\columnwidth}{>{\ccol}p{\levelcol} >{\ccol}p{\babcolavg} c >{\lcol}X}
            \tb{Level} & \tb{Combat Prowess} & \tb{\Ki Strike} & \tb{Special} \\
            \hline
            \monkprogressionrow{1}  & \plus1 & Enlightened defenses, unarmed warrior     \\
            \monkprogressionrow{2}  & \plus1 & Manifest \ki                              \\
            \monkprogressionrow{3}  & \plus1 & Wholeness of body, uncanny dodge \\
            \monkprogressionrow{4}  & \plus2 & Manifest \ki                              \\
            \monkprogressionrow{5}  & \plus2 & Flurry of blows                           \\
            \monkprogressionrow{6}  & \plus2 & Manifest \ki                              \\
            \monkprogressionrow{7}  & \plus3 & Perfect motion                            \\
            \monkprogressionrow{8}  & \plus3 & Manifest \ki                              \\
            \monkprogressionrow{9}  & \plus3 & Improved uncanny dodge, perfect soul      \\
            \monkprogressionrow{10} & \plus4 & Manifest \ki                              \\
            \monkprogressionrow{11} & \plus4 & Flow of life                              \\
            \monkprogressionrow{12} & \plus4 & Manifest \ki                              \\
            \monkprogressionrow{13} & \plus5 & Perfect mind                              \\
            \monkprogressionrow{14} & \plus5 & Manifest \ki                              \\
            \monkprogressionrow{15} & \plus5 & Perfect body                              \\
            \monkprogressionrow{16} & \plus6 & Manifest \ki                              \\
            \monkprogressionrow{17} & \plus6 & Perfect life                              \\
            \monkprogressionrow{18} & \plus6 & Manifest \ki                              \\
            \monkprogressionrow{19} & \plus7 & True perfection                           \\
            \monkprogressionrow{20} & \plus7 & Manifest \ki, transcend mortality         \\
        \end{dtabularx}
    \end{dtable}

    \classbasics{Alignment} Any nonchaotic.

    \classbasics{Class Skills}
    \subparhead{Strength} Climb, Jump, Sprint, Swim.
    \subparhead{Dexterity} Acrobatics, Escape Artist, Ride, Stealth.
    \subparhead{Intelligence} Heal.
    \subparhead{Perception} Awareness, Spellcraft, Survival.
    \subparhead{Other} Bluff, Intimidate, Perform, Persuasion.

    \subsection{Base Class Abilities}
        A character with monk as a base class gains the following abilities.

        \classbasics{Skill Points} 10.

        \classbasics{Combat Prowess} \plus2. This benefit is included in the class table.

        \classbasics{Defenses} \plus2 Fortitude, \plus4 Reflex, \plus4 Mental.

        \cf{Mnk}{Weapon and Armor Proficiency}
        Monks are proficient with simple weapons, monk weapons, and any one other weapon group.
        Monks are not proficient with any armor or shields.
        When wearing armor, using a shield, or carrying a medium or heavy load, a monk loses the benefit of her enlightened defense, fast movement, and \ki abilities.

        \cf{Mnk}{Ki Strike} A monk's fists, and all weapons she uses, gain a \plus1 \glossterm{enhancement bonus}.
        This functions like an enhancement bonus on a weapon, increasing her damage and offensive legend points per day (see \pcref{Weapon Enhancement Bonuses}).
        In addition, she is always treated as if she was wearing \plus1 armor, granting her temporary hit points and defensive legend points (see \pcref{Armor Enhancement Bonuses}).
        This bonus increases by \plus1 at 4th level and every 3 levels thereafter.

    \subsection{Class Abilities}
        All monks have the following abilities.

        \cf{Mnk}{Enlightened Defense}[Ex]
        A monk's \ki shields her body from attacks.
        When \monkunencumbered, she gains a \plus2 bonus to Armor defense.
        She loses this bonus when she is helpless.

        \cf{Mnk}{Ki Power}[Su]
        Many monk abilities depend on her \ki power.
        A monk's \ki power is equal to her Willpower or her character level, whichever is higher.

        \cf{Mnk}{Unarmed Warrior}[Ex]
        A monk's unarmed attacks are exceptionally deadly while she is able to freely use her body.
        While \monkunencumbered, she gains the Unarmed Proficiency and Unarmed Might feats as bonus feats, even if she does not meet the prerequisites (see Unarmed Proficiency, page \featpref{Unarmed Proficiency}, and Unarmed Might, page \featpref{Unarmed Might}, for details).
        For details about how to fight while unarmed, see \pcref{Unarmed Combat}.

        \cf{Mnk}[2nd]{Manifest Ki}[Su]
        The monk gains the ability to channel her \ki energy to temporarily enhance her abilities.
        She chooses one \ki manifestation from the list below.

        Using a \ki manifestation costs a \ki point.
        The monk has a number of \ki points equal to her Willpower or her monk level, whichever is higher.
        \Ki points can be recovered by meditation.
        If the monk meditates for 1 hour, she recovers all spent \ki points.

        Some \ki manifestations have minimum monk levels, as indicated in the title of the power.
        At her 4th monk level, and every even monk level thereafter, the monk learns an additional \ki manifestation.

        Some \ki manifestations have the effects of spells.
        Unless otherwise noted, the monk's effective spellpower with these abilities is equal to her \ki power.
        All \ki manifestations are supernatural abilities unless otherwise noted.

        \subcf{Elegant Whirl of Fluid Motion}
        As a swift action, the monk can gain a \plus20 bonus to Acrobatics checks until the end of the round.
        \subcf{Leap of the Heavens}
        As a swift action, the monk can gain a \plus20 bonus to Jump checks until the end of the round.
        \subcf{Scale the Highest Tower}
        As a swift action, the monk can gain a \plus20 bonus to Climb checks until the end of the round.
        \subcf{4th -- Burst of Blinding Speed}
        As a swift action, the monk can gain a \plus30 foot bonus to her land speed, up to a maximum of double her original speed.
        In addition, she cannot be followed until the end of the round.
        \subcf{4th -- Dance of Falling Feathers}
        As an immediate action, when the monk begins falling, she can gain the benefits of the \spell{feather fall} spell.
        \subcf{4th -- Fists of Distant Force}
        As a swift action, the monk can empower her unarmed attacks with \ki, allowing her to strike distant foes.
        Until the end of the round, she gains an additional ten feet of reach with her unarmed attacks, extending her threatened area.
        \subcf{6th -- Dance of the Wayward Strike}
        As an immediate action, when a foe misses the monk with a melee strike, the monk can redirect the strike.
        Both the foe and the monk must threaten a third creature.
        If the monk redirects the strike, the foe rolls the same attack against the third creature.
        \subcf{6th -- Surpass the Mortal Limits}
        As a swift action, the monk can surpass the physical limitations of her body.
        Until the end of the round, she may use her \ki power in place of her Strength, Dexterity, and Constitution when making checks.
        \subcf{8th -- Flash Step}
        As a swift action during the movement phase, the monk can teleport to anywhere she can see within 30 feet.
        If her line of effect is blocked, even by an invisible barrier, or if this would somehow place her inside a solid object, the ability fails.
        \subcf{8th -- Focus the Wayward Mind}
        As a swift action, the monk can dispel all \glossterm{Mind} effects that are affecting her.
        This has no effect on Mind effects that cannot be dispelled.
        \subcf{8th -- Ki-Disrupting Strike}
        As an immediate action, when the monk hits with a melee strike, she can make the struck creature \impaired with all actions for 2 rounds.
        \subcf{8th -- See the Flow of Life}
        As a swift action, the monk can gain the ability to see the \ki of living creatures until the end of the round.
        She can ``see'' any living creatures and their equipment within 50 feet perfectly, regardless of lighting conditions, invisibility, or any other means of concealment.
        This cannot detect living creatures through solid walls, however.
        \subcf{10th -- Dance of the Foolish Blow}
        This \ki manifestation functions as the \textit{dance of the wayward strike} manifestation, except that it can affect any melee attack, not just a single strike.
        The monk must have the \textit{dance of the wayward strike} \ki manifestation to learn this manifestation.
        \subcf{12th -- Diamond Fists}
        As a swift action, the monk can empower her unarmed attacks with incredible force.
        Until the end of the round, she may use her \ki power in place of her normal modifiers to physical damage with her unarmed attacks.
        In addition, she treats her unarmed strike as if it were an adamantine weapon for the purpose of overcoming damage reduction and hardness.
        \subcf{12th -- Stunning Fist}
        As an immediate action, when the monk deals damage with an unarmed melee strike, she can make a \Ki power vs. Fortitude attack against the struck creature.
        Success means the target is \staggered for 2 rounds.
        Critical success means the target is \stunned for 2 rounds.
        \norepeatnotes
        \subcf{14th -- Awaken the Pacifist Heart}
        As an immediate action, when the monk hits with a melee strike, she can make a \Ki power vs. Mental attack against the struck creature.
        Success means the target is unable to take violent actions, such as attacking, for 2 rounds.
        If the target takes damage after the current round, the effect is broken.
        \subcf{16th -- Flash Burst}
        As a swift action during the movement phase, the monk can teleport to anywhere within 1,000 feet.
        She must clearly visualize the destination, but she does not need line of sight or line of effect.
        If the destination is occupied, or dramatically different from how she visualized it, the effect fails.
        \subcf{20th -- Ki-Shattering Strike}
        As an immediate action, when the monk hits with a melee strike, she can disrupt her foe's \ki.
        The target is \severelyimpaired with all actions for 2 rounds.

        \cf{Mnk}[3rd]{Wholeness of Body}
        With concentration and focus, the monk can correct the flow of energy within her body.
        She can use this ability as a standard action by spending a \ki point, or with one minute of meditation otherwise.
        If she does, she heals 1d6 hit points per \ki power.

        \cf{Mnk}[3rd]{Uncanny Dodge}[Ex]
        The monk can react to danger before her senses would normally allow her to do so.
        The monk reduces her overwhelm penalties by 1.
        If her overwhelm penalty is reduced to 0, she is not considered to be overwhelmed.
        In addition, she is not \unaware when attacked by surprise.

        \cf{Mnk}[5th]{Flurry of Blows}\label{Flurry of Blows}
        When attacking unarmed, the monk can attack with multiple parts of her body simultaneously.
        This allows her to make dual attacks with her unarmed strike (see \pcref{Dual Attacking}).

        \cf{Mnk}[7th]{Perfect Motion}[Su]
        The monk becomes immune to effects that restrict her mobility.
        She suffers no penalties for acting underwater.
        In addition, she gains a \plus20 bonus to Reflex defense against grapple attacks, as well as on grapple attacks or Escape Artist checks made to escape a grapple or a pin.

        \cf{Mnk}[9th]{Improved Uncanny Dodge}[Ex]
        The monk reduces her overwhelm penalties by 2.
        This does not stack with the effects of uncanny dodge.
        If her overwhelm penalty is reduced to 0, she is not considered to be overwhelmed.

        \cf{Mnk}[9th]{Perfect Soul}[Su]
        The monk gains spell resistance equal to 10 \add her \ki power.
        To affect the monk with a spell, a caster must make an attack with an accuracy equal to its spellpower.
        If the attack beats the monk's spell resistance, the spell works normally.
        Otherwise, the spell has no effect on the monk.

        \cf{Mnk}[11th]{Flow of Life}[Su]
        At the end of each round, the monk heals hit points equal to her \ki power.

        \cf{Mnk}[13th]{Perfect Mind}[Su]
        The monk becomes immune to hostile Mind effects.

        \cf{Mnk}[15th]{Perfect Body}[Su]
        The monk becomes immune to being blinded, deafened, fatigued, exhausted, nauseated, sickened, and staggered.
        In addition, she no longer takes penalties to her attribute scores for aging, and cannot be magically aged.
        The monk still dies of old age when her time is up.

        \cf{Mnk}[17th]{Perfect Life}[Su]
        At the end of each round, the monk heals hit points equal to twice her \ki power.
        This replaces the benefit from her flow of life ability.
        If she has taken critical damage, she instead heals critical damage equal to half her \ki power.
        In addition, she becomes immune to \glossterm{Death} effects.

        \cf{Mnk}[19th]{True Perfection}[Su]
        The monk becomes inhumanly perfect.
        If she rolls less than a 5 on any d20 roll, it is treated as a 5.

        \cf{Mnk}[20th]{Transcend Mortality}[Su]
        If the monk dies, she may choose to retain control of her body and soul through sheer force of will.
        Her body immediately disappears, and her soul does not travel to an afterlife.
        Instead, her body reforms with no trace of its injuries 24 hours later.
        The reformed body is in perfect health and can be any age the monk chooses, to a minimum of the age of adulthood for her race.
        She can reform her body at the place where she died, or in any place on the same plane that is deeply familiar to her.

        After each time the monk reforms herself this way, it takes 24 additional hours to reform the next time she ``dies''.
        A monk with this ability can only be permanently killed by the direct intervention of a deity.

        \subsubsection{Ex-Monks}
            A monk who becomes chaotic loses her \ki powers, and cannot gain more levels as a monk.
            She retains all her other class abilities.
            If she stops being chaotic, she regains her \ki powers and ability to take monk levels.

\section{Paladin}\label{Paladin}
    \begin{dtable}
        \lcaption{Paladin Progression}
        \begin{dtabularx}{\columnwidth}{>{\ccol}p{\levelcol} >{\ccol}p{\babcolgood} >{\lcol}X}
            \tb{Level} & \tb{Combat Prowess} & \tb{Special} \\
            \hline
            \paladinprogressionrow{1}  & Divine invocation (smite), divine protection \\
            \paladinprogressionrow{2}  & Divine invocation                            \\
            \paladinprogressionrow{3}  & Discernment (alignment)                      \\
            \paladinprogressionrow{4}  & Divine presence                              \\
            \paladinprogressionrow{5}  & Pass judgment                                \\
            \paladinprogressionrow{6}  & Divine invocation                            \\
            \paladinprogressionrow{7}  & Discernment (lies)                           \\
            \paladinprogressionrow{8}  & Divine presence                              \\
            \paladinprogressionrow{9}  & Expanded presence                            \\
            \paladinprogressionrow{10} & Divine invocation                            \\
            \paladinprogressionrow{11} & Discernment (invisibility)                   \\
            \paladinprogressionrow{12} & Divine presence                              \\
            \paladinprogressionrow{13} & Lingering presence                           \\
            \paladinprogressionrow{14} & Divine invocation                            \\
            \paladinprogressionrow{15} & Discernment (truth), martyr's retribution    \\
            \paladinprogressionrow{16} & Divine presence                              \\
            \paladinprogressionrow{17} & Mighty presence                              \\
            \paladinprogressionrow{18} & Divine invocation                            \\
            \paladinprogressionrow{19} & Discernment (thoughts)                       \\
            \paladinprogressionrow{20} & Aligned soul, divine presence                 \\
        \end{dtabularx}
    \end{dtable}

    \classbasics{Alignment} Any other than true neutral.

    \classbasics{Class Skills}
    \subparhead{Dexterity} Ride.
    \subparhead{Intelligence} Heal, Knowledge (local, religion).
    \subparhead{Perception} Awareness, Intimidate, Sense Motive.
    \subparhead{Other} Bluff, Intimidate, Persuasion.

    \subsection{Base Class Abilities}
        A character with paladin as a base class gains the following abilities.

        \classbasics{Skill Points} 5.

        \classbasics{Combat Prowess} \plus2. This benefit is included in the class table.

        \classbasics{Defenses} \plus4 Fortitude, \plus4 Mental.

        \cf{Pal}{Weapon and Armor Proficiency}
        Paladins are proficient with simple weapons, any three other weapon groups, all types of armor (heavy, medium, and light), and shields.

        \cf{Pal}{Divine Protection}[Su]
        The paladin's force of belief manifests a divine protection around her.
        She may add her Willpower to her physical defenses (Armor, Maneuver, and Reflex) in place of Dexterity or Constitution.

    \subsection{Class Abilities}
        All paladins have the following abilities.

        \cf{Pal}{Devoted Alignment}[Su]
        A paladin is devoted to a specific alignment.
        She must choose one of her alignment components: good, evil, lawful, or chaotic.
        The alignment she chooses is her devoted alignment.
        A paladin's class abilities are affected by this choice.
        She excels at slaying creatures with alignments opposed to her devoted alignment.
        A paladin's devoted alignment cannot be changed without extraordinary repurcussions to the paladin.

        \cf{Pal}{Divine Power}
        Many paladin abilities depend on her divine power.
        A paladin's divine power is equal to her Willpower or her character level, whichever is higher.

        \cf{Pal}{Divine Invocation}[Su]
        A paladin can invoke the power of her alignment to achieve incredible effects.
        This ability can be used a number of times per day equal to her Willpower or her paladin level, whichever is higher.
        She gains the smite divine invocation.

        At 2nd level, and every even level thereafter, the paladin gains an additional divine invocation.
        Most divine invocations have minimum paladin levels, as indicated in the title of the ability.
        Some divine invocations are also restricted to paladins with specific devoted alignments.
        All divine invocations are supernatural abilities unless otherwise noted.
        The paladin's accuracy with divine invocations is equal to her divine power.
        If a divine invocation emulates a spell, the paladin's effective spellpower is equal to her divine power.

        Divine powers marked with an asterisk are called smite powers.
        Smite powers function like the smite divine invocation, except that they also have additional effects.
        Unless otherwise noted, these additional effects only occur if the smited creature does not share the paladin's devoted alignment.

        \parhead{Any Alignment}

        \subcf{Smite}
        Once per round, when the paladin makes a strike, she may declare that strike to be a smite.
        She may use her divine power in place of her normal accuracy for that strike.
        If the struck creatture shares her devoted alignment, she deals no damage at all (not even normal weapon damage), but the use of the ability is still spent.
        Otherwise, her weapon deals maximum damage, and she deals bonus damage equal to her divine power.
        \subcf{2nd -- Bless} This invocation functions like the \spell{bless} spell.
        \subcf{2nd -- Lay on Hands}
        As a standard action, the paladin can lay hands on a creature.
        The target is healed for 1d6 damage per divine power.
        \subcf{2nd -- Protection from Alignment} This invocation functions like the \spell{protection from alignment} spell.
        The paladin must protect the target from the alignment opposed to her devoted alignment.
        \subcf{6th -- Exhausting Smite*}
        The paladin makes a Divine power vs. Fortitude attack against the struck creature.
        Success means it is \exhausted for 2 rounds.
        \subcf{6th -- Resounding Smite*}
        The struck creature is knocked prone.
        \subcf{6th -- Seeking Smite*}
        This smite attack ignores any miss chances, such as from active cover or visual impairment.
        The weapon must still be physically able to strike the target.
        \subcf{6th -- Taunting Smite*}
        The struck creature is \taunted by the paladin for 2 rounds.
        \subcf{10th -- Dispelling Smite*}
        The struck creature is affected by \spell{dispel magic}.
        \subcf{10th -- Divine Might}
        This invocation functions like the \spell{divine might} spell.
        \subcf{10th -- Penetrating Smite*}
        The struck creature's armor is weakened.
        For the next 2 rounds, attacks against the target that would normally target Armor defense are instead made against the lower of the target's Armor and Reflex defenses.
        \subcf{14th -- Dazing Smite*}
        The struck creature is \dazed for 2 rounds.
        \subcf{14th -- Spellreaving Smite*}
        All spells and magical effects on the struck creature are dispelled.
        Spells and effects that cannot be removed by \spell{dispel magic} are unaffected.
        The paladin must have the dispelling smite invocation to choose this invocation.
        \subcf{14th -- Terrifying Smite*}
        The paladin makes a Divine power vs. Mental attack against the struck creature.
        Success means it is \frightened by her for 2 rounds.
        \subcf{18th -- Converting Smite*}
        The paladin's smite shows her foe the error of its ways.
        She makes a Divine power vs. Mental attack against the struck creature.
        Success means it is \confused for 2 rounds.
        Critical success means its alignment changes, and it gains the paladin's devoted alignment for 1 week.
        After that time, it can choose to return to its original alignment, or keep its new alignment permanently.
        Failure means it is \dazed for 2 rounds.
        \subcf{18th -- Immobilizing Smite*}
        The creature struck with this smite is \immobilized for 5 rounds.

        \parhead{Chaos Divine Invocations}
        \subcf{6th -- Confusion} This invocation functions like the \spell{confusion} spell, except that it affects targets within \rngmed range.
        \subcf{6th -- Freedom} This invocation functions like the \spell{freedom} spell.
        \subcf{10th -- Break the Chains}
        As a standard action, the paladin can break all shackles, bindings, and locks within a \arealarge radius burst of her.
        Nonmagical objects are automatically broken.
        To break magical objects, the paladin must make an attack with her divine power against a DR equal to 10 \add the object's spellpower.
        \subcf{10th -- Chaotic Redirection}
        As an immediate action, when the paladin or any of her allies within \rngclose range is struck by a physical attack, the paladin can redirect the attack to a random creature within \rngclose range of the paladin, including the paladin.
        The attack is made against that creature instead of its original target, using its original accuracy, and has its normal effects if it hits.
        After using this invocation, the paladin cannot use it for 5 rounds.
        \subcf{14th -- Discordant Song} This invocation functions like the \spell{discordant song} spell.

        \parhead{Good Divine Invocations}
        \subcf{Shield Other}
        As a standard action, the paladin can redirect damage from an ally to herself, as the \spell{share pain} spell.
        \subcf{6th -- Martyr's Shield}
        As an immediate action, when an ally within \rngmed range would take critical damage, the paladin can take that damage as regular damage instead.
        After using this invocation, the paladin cannot use it again for 5 rounds.

        \parhead{Evil Divine Invocations}
        \subcf{2nd -- Enfeeblement} This invocation functions like the \spell{enfeeblement}
        spell.
        \subcf{6th -- Agony} This invocation functions like the \spell{agony} spell.
        \subcf{10th -- Enervation} This invocation functions like the \spell{enervation}
        spell.
        \subcf{10th -- Executing Smite*}
        The paladin makes a Divine power vs. Fortitude attack against the struck creature.
        Success means the target dies if it has no hit points remaining after taking damage from the smite.

        \parhead{Law Divine Invocations}
        \subcf{2nd -- Command} This invocation functions like the \spell{command} spell.
        \subcf{2nd -- Hold Person} This invocation functions like the \spell{hold person} spell.
        \subcf{6th -- Read Mind} This invocation functions like the \spell{read mind} spell.
        \subcf{10th -- Prohibition} This invocation functions like the \spell{prohibition} spell.

        \cf{Pal}[3rd]{Discernment}[Su]
        A paladin can discern truths about creatures she sees as a swift action.
        When she uses this ability, she learns which creatures within a \arealarge cone have her devoted alignment.

        The paladin may use her discernment ability a number of times per day equal to her Perception or half her paladin level (minimum 1), whichever is higher.

        At 7th level, the paladin can discern lies.
        Whenever a creature in the area intentionally lies, the paladin knows the statement was a lie.
        This does not reveal the truth, uncover unintentional inaccuracies, or necessarily reveal evasions.

        At 11th level, the paladin can also see invisible creatures and objects within the area, as the \spell{see invisibility} spell.

        At 15th level, the paladin can discern truth.
        When a lie is spoken, in addition to learning that the statement is a lie, the paladin learns what the creature believes the truth to be.
        This does not necessarily reveal the actual truth -- merely what the creature believes.

        At 19th level, the paladin can discern thoughts.
        She knows the surface thoughts of all creatures in the area, as the \spell{read mind} spell.

        \cf{Pal}[4th]{Divine Presence}[Su]
        The paladin's presence alters the world around her.
        She chooses a single divine presence from the list below.
        Each divine presence affects a \areamed radius emanation from the paladin, including herself.
        She may choose to suppress or resume her divine presence as a swift action.

        At her 8th paladin level, and every 4 levels thereafter, the paladin gains an additional divine presence.
        She may have multiple divine presences active simultaneously, and suppress or resume them individually.
        Most divine presences have minimum paladin levels, as indicated in the title of the ability.
        All divine presences are supernatural abilities unless otherwise noted.
        If a divine presence emulates a spell, the paladin's effective spellpower is equal to her paladin level or her Willpower.

        \parhead{Any Alignment}
        \subcf{8th -- Worthy Foe}
        Enemies in the area are \taunted by the paladin.
        \subcf{16th -- Aura of Unbending Purpose}
        Allies in the area are immune to \glossterm{Mind} effects.

        \parhead{Chaotic Divine Presences}
        \subcf{Mobile Aura}
        Allies in the area can move at full speed through threatened squares.
        \subcf{8th -- Accelerating Aura}
        Allies in the area gain a \plus30 foot bonus to land speed, up to maximum of double their original speed.
        \subcf{12th -- Aura of Freedom}
        Allies in the area gain the benefits of the \spell{freedom} spell.
        \subcf{20th -- Maddening Aura}
        At the start of each round, enemies in the area are \disoriented that round.

        \parhead{Evil Divine Presences}
        \subcf{Overwhelming Aura}
        Enemies in the area that are \glossterm{overwhelmed} increase their overwhelm penalties by 1.
        \subcf{8th -- Lifefeeding Aura}
        At the end of each round, if another creature in the area took damage, the paladin regains hit points equal to the damage taken, up to a maximum of her divine power.
        \subcf{12th -- Baleful Aura}
        Enemies in the area are \impaired with attacks and checks.
        \subcf{12th -- Painful Aura}
        Whenever an enemy in the area takes damage, the damage is doubled, up to a maximum bonus equal to the paladin's divine power.
        Each enemy can only take this bonus damage each round.
        \subcf{16th -- Sacrificial Aura}
        Whenever you take damage, any other creature in the area may choose to take that damage instead.
        This damage ignores all forms of damage reduction and similar abilities.
        Any damage in excess of the creature's hit points is not redirected.

        \parhead{Good Divine Presences}
        \subcf{Defensive Aura}
        Allies in the area that are \glossterm{overwhelmed} reduce their overwhelm penalties by 1.
        If an ally's overwhelm penalty is reduced to 0, they are not considered to be overwhelmed.
        \subcf{8th -- Minor Healing Aura}
        Whenever an ally in the area regains hit points, the healing is doubled, up to a maximum bonus equal to the paladin's divine power.
        Each ally can only receive this bonus once per round.
        \subcf{12th -- Healing Aura}
        At the end of each round, allies in the area heal hit points equal to the paladin's divine power.
        \subcf{16th -- Martyr's Aura}
        Whenever an ally in the area takes damage, the paladin may choose to take that damage instead.
        This damage ignores all forms of damage reduction and similar abilities.
        If the paladin takes damage in excess of her hit points in this way, the excess damage is dealt directly as critical damage.

        \parhead{Lawful Divine Presences}
        \subcf{Aura of Fortification}
        Enemies in the area move at half speed through threatened squares.
        \subcf{8th -- Aura of Inhibition}
        Enemies in the area move at half speed.
        \subcf{12th -- Aura of Truth}
        All illusory figments and glamers are suppressed in the area.
        \subcf{20th -- Inescapable Aura}
        At the start of each round, enemies in the area are \immobilized that round.

        \cf{Pal}[5th]{Pass Judgment}[Su]
        The paladin gains the ability to pass judgment on those she deems unworthy.
        Once per day as a swift action, she may pass judgment on a creature within 100 feet of her.
        For the purpose of spells and effects, the creature is treated as if it had the alignment opposed to the paladin's devoted alignment.
        This effect lasts for one week, or until the paladin changes her mind about the subject.
        This does not change the creature's actions or behavior, but the creature is subject to the paladin's smite attack, would detect as that alignment under the inspection of a \spell{detect alignment} spell, and so on.

        No attack is required for this effect, and it cannot be dispelled, but a \spell{remove curse}, \spell{miracle}, or \spell{wish} spell can remove it.
        The paladin can use this ability an additional time per day at her 9th paladin level and every four levels thereafter.
        A paladin should be careful when using this ability, as persecution of allies can lead overzealous paladins to fall.

        \cf{Pal}[9th]{Expanded Presence}[Su]
        The paladin's divine presences affect an \arealarge radius.

        \cf{Pal}[15th]{Martyr's Retribution}[Su]
        If the paladin dies in the devoted service of her devoted alignment, she may choose to have her fallen body erupt in an immense burst of divine energy.
        If she does, her body is almost completely consumed, preventing her from being raised with \ritual{resurrection} and similar effects that require an intact body.
        This burst has two effects.
        First, a \spell{sunburst} spell immediately takes effect over the area where the paladin fell.
        Second, a \spell{storm of vengeance} spell begins to take effect, centered on the same area.
        The spell lasts for 10 rounds, and the lightning strikes target the paladin's enemies.
        Both of these effects harm only the paladin's foes, and do not harm her allies.
        However, her allies' vision is still impeded by the \spell{storm of vengeance}.

        \cf{Pal}[13th]{Lingering Presence}[Su]
        The effects of the paladin's divine presences continue for 1 round after targets leave the area.

        \cf{Pal}[17th]{Mighty Presence}[Su]
        The paladin's divine presences affect an \areahuge radius.
        In addition, their effects continue for 2 rounds after targets leave the area.

        \cf{Pal}[20th]{Aligned Soul}[Su]
        While a paladin is dead, she may approach the deity or governing figure of her afterlife and request to be returned to life to continue her mission.
        Travelling to the relevant figure and making the request takes 12 hours.
        Unless there are extenuating circumstances, this request is almost always granted, and the paladin is resurrected in a new body at a location of the entity's choice.
        This functions like the \spell{resurrection} ritual, except that no part of the body is required, and a new body is created by the entity.
        She can be resurrected in this way regardless of the condition of her body, but not if her soul has been trapped or otherwise prevented from going to the correct afterlife.

        \subsubsection{Ex-Paladins}
            A paladin who ceases to follow her devoted alignment loses all supernatural paladin class abilities.
            She may not gain any additional paladin levels.
            She regains her abilities and advancement potential if she atones for her violations (see the \spell{atonement} ritual), as appropriate.

\section{Ranger}\label{Ranger}
    \begin{dtable}
        \lcaption{Ranger Progression}
        \begin{dtabularx}{\columnwidth}{>{\ccol}p{\levelcol} >{\ccol}p{\babcolgood} c >{\lcol}X}
            \tb{Level} & \tb{Combat Prowess} & \tb{Quarry} & \tb{Special} \\
            \hline
            \rangerprogressionrow{1}  & \plus2 & Quarry, tenacious hunter, wild speech \\
            \rangerprogressionrow{2}  & \plus2 & Hunting lore, tracker                 \\
            \rangerprogressionrow{3}  & \plus2 & Free stride, keen vision              \\
            \rangerprogressionrow{4}  & \plus2 & Terrain lore                          \\
            \rangerprogressionrow{5}  & \plus3 & Rapid tracker                         \\
            \rangerprogressionrow{6}  & \plus3 & Hunting lore                          \\
            \rangerprogressionrow{7}  & \plus3 & Blindsense                            \\
            \rangerprogressionrow{8}  & \plus3 & Terrain lore                          \\
            \rangerprogressionrow{9}  & \plus3 & Perfect stride                        \\
            \rangerprogressionrow{10} & \plus4 & Hunting lore                          \\
            \rangerprogressionrow{11} & \plus4 & Blindsight                            \\
            \rangerprogressionrow{12} & \plus4 & Terrain lore                          \\
            \rangerprogressionrow{13} & \plus4 & Unerring hunter                       \\
            \rangerprogressionrow{14} & \plus4 & Hunting lore                          \\
            \rangerprogressionrow{15} & \plus5 & Farsight                              \\
            \rangerprogressionrow{16} & \plus5 & Terrain lore                          \\
            \rangerprogressionrow{17} & \plus5 & Eternal quarry                        \\
            \rangerprogressionrow{18} & \plus5 & Hunting lore                          \\
            \rangerprogressionrow{19} & \plus5 & Truesight                             \\
            \rangerprogressionrow{20} & \plus6 & Terrain lore
        \end{dtabularx}
    \end{dtable}

    \classbasics{Alignment} Any.

    \classbasics{Class Skills}
    \subparhead{Strength} Climb, Jump, Sprint, Swim.
    \subparhead{Dexterity} Acrobatics, Escape Artist, Ride, Stealth.
    \subparhead{Intelligence} Heal, Knowledge (dungeoneering, geography, nature).
    \subparhead{Perception} Awareness, Creature Handling, Survival.
    \subparhead{Other} Bluff, Intimidate, Persuasion.

    \subsection{Base Class Abilities}
        A character with ranger as a base class gains the following abilities.

        \classbasics{Skill Points} 15.

        \classbasics{Combat Prowess} \plus2. This benefit is included in the class table.

        \classbasics{Defenses} \plus4 Fortitude, \plus4 Reflex, \plus2 Mental.

        \cf{Rgr}{Weapon and Armor Proficiency}
        A ranger is proficient with simple weapons, any two weapon groups, light and medium armor, and shields.
        He is also proficient with his choice of bows, crossbows, or thrown weapons.

        \cf{Rgr}{Tenacious Hunter}[Ex]
        The ranger adds his quarry bonus to his defenses against attacks that his quarry makes.
        See the Quarry ability, below, for details.

        \cf{Rgr}{Wild Speech}[Su]
        The ranger learns how to communicate with animals.
        This ability functions like the druid ability of the same name (see Wild Speech, \pref{Drd:Wild Speech}).
        A ranger can use this ability a number of times per day equal to his Perception or half his ranger level, whichever is higher.

    \subsection{Class Abilities}
        All rangers have the following abilities.

        \cf{Rgr}{Quarry}[Ex]
        As a swift action, a ranger may designate any foe he sees as his quarry.
        A ranger gains a \plus2 bonus to damage with physical attacks and Awareness, Stealth, and Survival checks against his quarry.
        However, while a ranger has designated a quarry, he takes a \minus2 penalty on the same rolls against any target other than his quarry.

        A ranger may not normally have more than one quarry at once.
        He may not designate a new quarry until he defeats his old quarry, or until he gives up on the quarry.
        He may give up pursuing a quarry as a free action.
        If he does, he is unable to designate a new quarry until he rests for 5 minutes.

        Some special abilities allow the ranger to designate multiple creatures as a quarry.
        Abilities which designate multiple creatures as a quarry always last for a specific amount of time.

        If the ranger does not see his quarry for more than a week, it is no longer considered his quarry.

        \par The amount by which a ranger's damage and skill checks increase against his quarry is called his quarry bonus.
        The ranger's quarry bonus improves by \plus1 at his 5th ranger level and every 5 ranger levels thereafter.
        His penalties against targets other than his quarry remains the same.

        \cf{Rgr}[2nd]{Fast Movement}[Ex]
        The ranger increases his movement speed by 10 feet when \unencumbered.

        \cf{Rgr}[2nd]{Hunting Power}[Ex]
        The strength of some ranger abilities are determined by his hunting power.
        Normally, his hunting power is equal to his Perception or his character level, whichever is higher.

        \cf{Rgr}[2nd]{Tracker}
        A ranger gains Track as a bonus feat, even if he does not meet its prerequisites (see \featref{Track}).
        In addition, he may his hunting power in place of his Survival skill to follow tracks.

        \cf{Rgr}[2nd]{Hunting Lore}
        The ranger gains an ability drawn from ancient hunting lore.
        He chooses a single hunting lore from the list below.
        Some hunting lores have minimum ranger levels, as indicated in the title of the ability.
        At his 6th ranger level, and every four ranger levels thereafter, the ranger gains an additional hunting lore.

        All hunting lore abilities are extraordinary abilities unless otherwise noted.

        \subcf{Combat Feat}
        The ranger gains a combat feat for which he qualifies (see Feats).
        This lore can be selected multiple times.

        \subcf{Energy Adaptation}
        The ranger continuously gains the benefits of the \spell{resist energy} spell, with a spellpower equal to his hunting power.

        \subcf{Survivalist}
        The ranger gains a feat with Survival skill prerequisites for which he qualifies as a bonus feat.
        This lore can be selected multiple times.

        \subcf{Undead Destroyer}
        Non-intelligent undead treat the ranger as if he were undead.
        In addition, he increases his quarry bonus by \plus2 against undead.

        \subcf{6th -- Aberrant Hunter}
        The ranger gains a \plus2 bonus to Mental defense.
        In addition, he increases his quarry bonus by \plus2 against aberrations.

        \subcf{6th -- Giantslayer}
        The ranger gains a \plus8 bonus to Fortitude defense against attacks that would move him.  % weird bonus
        In addition, he increases his quarry bonus by \plus2 against monstrous humanoids.

        \subcf{6th -- Golem Breaker}
        The ranger's attacks ignore an amount of hardness equal to half his hunting power.
        In addition, he increases his quarry bonus by \plus2 against constructs.

        \subcf{6th -- Hidden Hunter}
        The ranger becomes even more difficult for his quarry to detect.
        He adds his quarry bonus to his Stealth checks against his quarry.
        In addition, he continuously benefits from the effect of the \spell{nondetection} spell against all attempts that his quarry makes to detect him magically.
        The effect uses a spellpower equal to his hunting power.

        \subcf{6th -- Ooze Hunter}
        The ranger becomes immune to slime and engulf attacks.
        In addition, he increases his quarry bonus by \plus2 against oozes.

        \subcf{6th -- Scent}
        The ranger gains the scent ability (see \pcref{Scent}).

        \subcf{6th -- Slowed Quarry}
        If the ranger hits his quarry with a physical attack, that creature moves at half speed.
        This effect lasts as wrong as it remains the ranger's quarry.

        \subcf{10th -- Divine Avenger}
        The ranger gains \glossterm{spell resistance} against divine spells equal to 10 \add his hunting power.
        In addition, he increases his quarry bonus by \plus2 against divine spellcasters.

        \subcf{10th -- Dragonslayer}
        The ranger becomes immune to breath weapons, and his attacks ignore the damage reduction of dragons.
        In addition, he increases his quarry bonus by \plus2 against dragons.

        \subcf{10th -- Fey Stalker}
        The ranger becomes immune to hostile \glossterm{Mind} effects.
        In addition, he increases his quarry bonus by \plus2 against fey.

        \subcf{10th -- Mageslayer}
        The ranger gains \glossterm{spell resistance} against arcane spells equal to 10 \add his hunting power.
        In addition, he increases his quarry bonus by \plus2 against arcane spellcasters.

        \subcf{14th -- Master of the Hunt}
        As a standard action, the ranger may share the benefits of his quarry ability with all allies who can see and hear him for 5 rounds.
        This does not share the defensive benefits granted by the Tenacious Hunter ability.
        However, his allies do not suffer penalties against targets other than the quarry.

        \subcf{14th -- Flexible Quarry}
        The ranger does not need to wait to designate a new quarry after giving up on his previous quarry.

        \subcf{14th -- Impaired Quarry}
        If the ranger hits his quarry with a physical attack, that creature is \impaired with all attacks and checks.
        This effect lasts as long as it remains the ranger's quarry.

        \subcf{14th -- Multiquarry}
        The ranger may designate up to two targets whenever he chooses a quarry.
        He gains his quarry benefits against both targets.
        Whenever he designates new quarries, he may choose which previous quarry to give up (if any).

        \subcf{18th -- Anchored Quarry}
        If the ranger hits his quarry with a physical attack, that creature is dimensionally anchored, as the \spell{dimensional anchor} spell.
        This effect lasts as long as it remains the ranger's quarry.

        \cf{Rgr}[3rd]{Free Stride}[Ex]
        The ranger can move through any sort of natural terrain that slows or impedes movement at his normal speed without suffering any sort of impairment.
        If a skill check, such as Climb or Swim, would normally be required to move through the terrain, this ability does not help.

        \cf{Rgr}[3rd]{Keen Vision}[Ex]
        The ranger's sight improves, allowing him to see more easily.
        He gains \glossterm{low-light vision}, allowing him to treat sources of light as if they had double their normal illumination range.
        If he already has low-light vision, he doubles its benefit, allowing him to treat sources of light as if they had four times their normal illumination range.
        In addition, he gains \glossterm{darkvision} out to 50 feet, allowing him to see in complete darkness.
        If he already has darkvision, he increases its range by 50 feet.

        \cf{Rgr}[4th]{Terrain Lore}[Ex]
        The ranger becomes particularly attuned to certain kinds of terrain.
        He chooses two terrain-based lores from the list below.
        Usually, rangers favor lores relating to their home terrain, but a ranger may a lore from any kind of terrain that he has personally experienced at least once.
        At his 8th ranger level, and every four ranger levels thereafter, the ranger gains an additional terrain lore.

        Many terrain lores grant the ranger a particular terrain as a favored terrain.
        While in a favored terrain, a ranger gains a \plus2 bonus to Awareness, Stealth, and Survival checks.
        If he desires, he may leave no trace of his passage, causing attempts to track him to take a \minus20 penalty.
        The options for terrain lores are listed below.

        All terrain lore abilities are extraordinary abilities unless otherwise stated.

        \subcf{Aquatic}
        The ranger treats aquatic terrain as a favored terrain.
        In addition, he gains a \glossterm{swim speed} equal to his base land speed.
        If he already has a swim speed, he increases his swim speed by 10 feet.
        \subcf{Cold}
        The ranger treats cold terrain as a favored terrain.
        In addition, he gains cold damage reduction equal to his Constitution or his character level, whichever is higher.
        This allows him to ignore the first points of cold damage he would take each round.
        \subcf{Desert}
        The ranger treats deserts as a favored terrain.
        In addition, he becomes immune to fatigue.
        \subcf{Forest}
        The ranger treats forests as a favored terrain.
        In addition, he gains Skill Focus (Stealth) as a bonus feat.
        \subcf{Mountains}
        The ranger treats mountains as a favored terrain.
        In addition, he gains a \glossterm{climb speed} equal to his land speed.
        If he already has a climb speed, he increases his climb speed by 10 feet.
        \subcf{Plains}
        The ranger treats plains as a favored terrain.
        In addition, he increases his land speed by 10 feet.
        \subcf{Swamp}
        The ranger treats swamps as a favored terrain.
        In addition, he gains Perfect Health as a bonus feat.
        \subcf{Underground}
        The ranger treats underground terrain as a favored terrain.
        In addition, he gains Blind-Fight as a bonus feat.
        \subcf{Urban}
        The ranger treats urban terrain as a favored terrain.
        In addition, he gains Skill Focus (Persuasion) as a bonus feat.
        \subcf{8th -- Guide}
        Whenever the ranger is in a favored terrain, all allies that can see and hear the ranger gain his favored terrain bonuses in that terrain as well.
        \subcf{12th -- Camouflage}
        The ranger can use the Stealth skill to hide in any of his favored terrains, even if the terrain does not grant cover or concealment.
        \subcf{12th -- Planar Terrain}
        The ranger chooses a plane.
        He treats that plane as favored terrain.
        In addition, he is immune to any hostile planar effects from any plane he has chosen as favored terrain.
        \subcf{18th -- Hide in Plain Sight}
        While in any of his favored terrains, the ranger can use the Stealth skill to hide even while being observed, taking a \minus5 penalty to the Stealth check.
        He still needs cover or concealment to hide.

        \cf{Rgr}[5th]{Rapid Tracker}[Ex]
        The ranger's ability to track his foes improves.
        He can move at his normal speed while following tracks without taking the normal \minus5 penalty.
        He takes only a \minus10 penalty (instead of the normal \minus20) when moving at up to twice normal speed while tracking.

        \cf{Rgr}[7th]{Blindsense}[Ex]
        The ranger's perceptions are so finely honed that he can sense his enemies without seeing them.
        He gains the blindsense ability out to 50 feet.
        This ability allows him to sense the presence and location of objects and foes within 50 feet without seeing them.
        If he already has the blindsense ability, he increases its range by 50 feet.

        \cf{Rgr}[9th]{Perfect Stride}[Su]
        The ranger's ability to surpass obstacles becomes unparalleled.
        He constantly acts as if he were under the effect of a \spell{freedom} spell, except that it does not allow him to act normally underwater.

        \cf{Rgr}[11th]{Blindsight}[Ex]
        The ranger gains the ability to ``see'' perfectly without his eyes in a 50 foot radius around him.
        With this ability, he can fight just as well with his eyes closed as with them open.
        If he already has the blindsight ability, he increases its range by 50 feet.

        \cf{Rgr}[13th]{Unerring Hunter}[Su]
        The ranger's ability to hunt down his quarry improves to supernatural levels.
        Once per day, the ranger may concentrate for a full round to duplicate the effects of the \spell{discern location} ritual targeted at his quarry.

        \cf{Rgr}[15th]{Farsight}[Ex]
        The ranger increases the range of his darkvision and blindsight abilities by 100 feet.
        In addition, he halves his \glossterm{range increment penalties} for attacking at long range.

        \cf{Rgr}[17th]{Eternal Quarry}[Ex]
        The ranger's quarry never stops being his quarry until he chooses a different creature.
        In addition, once he has designated a creature as a quarry, he can always designate it as a quarry again, even if it is not in his sight.
        This only applies to creatures that he designates as a quarry after acquiring this ability.

        \cf{Rgr}[19th]{Truesight}[Su]
        The ranger's perceptions are accurate enough to defeat even powerful magic.
        He gains the ability to see all things as they actually are, as the \spell{true seeing} spell, out to a range of 50 feet.

\section{Rogue}\label{Rogue}
    \begin{dtable}
        \lcaption{Rogue Progression}
        \begin{dtabularx}{\columnwidth}{>{\ccol}p{\levelcol} >{\ccol}p{\babcolavg} c >{\lcol}X}
            \tb{Level} & \tb{Combat Prowess} & \tb{Sneak Attack} & \tb{Special} \\
            \hline
            \rogueprogressionrow{1}  & \plus1d6  & Advanced training      \\
            \rogueprogressionrow{2}  & \plus1d6  & Skill talent           \\
            \rogueprogressionrow{3}  & \plus2d6  & Uncanny dodge          \\
            \rogueprogressionrow{4}  & \plus2d6  & Combat trick           \\
            \rogueprogressionrow{5}  & \plus3d6  & Skill exemplar         \\
            \rogueprogressionrow{6}  & \plus3d6  & Skill talent           \\
            \rogueprogressionrow{7}  & \plus4d6  & Improved uncanny dodge \\
            \rogueprogressionrow{8}  & \plus4d6  & Combat trick           \\
            \rogueprogressionrow{9}  & \plus5d6  & Skill exemplar         \\
            \rogueprogressionrow{10} & \plus5d6  & Skill talent           \\
            \rogueprogressionrow{11} & \plus6d6  & Slippery mind          \\
            \rogueprogressionrow{12} & \plus6d6  & Combat trick           \\
            \rogueprogressionrow{13} & \plus7d6  & Skill exemplar         \\
            \rogueprogressionrow{14} & \plus7d6  & Skill talent           \\
            \rogueprogressionrow{15} & \plus8d6  & \tdash                 \\
            \rogueprogressionrow{16} & \plus8d6  & Combat trick           \\
            \rogueprogressionrow{17} & \plus9d6  & Skill exemplar         \\
            \rogueprogressionrow{18} & \plus9d6  & Skill talent           \\
            \rogueprogressionrow{19} & \plus10d6 & Ambush master          \\
            \rogueprogressionrow{20} & \plus10d6 & Combat trick           \\
        \end{dtabularx}
    \end{dtable}

    \classbasics{Alignment} Any.

    \classbasics{Class Skills}
    \subparhead{Strength} Climb, Jump, Sprint, Swim.
    \subparhead{Dexterity} Acrobatics, Escape Artist, Sleight of Hand, Stealth.
    \subparhead{Intelligence} Devices, Disguise, Knowledge (dungeoneering, local), Linguistics.
    \subparhead{Perception} Awareness, Sense Motive.
    \subparhead{Other} Bluff, Intimidate, Perform, Persuasion.

    \subsection{Base Class Abilities}
        A character with rogue as a base class gains the following abilities.

        \classbasics{Skill Points} 15.

        \classbasics{Combat Prowess} \plus2. This benefit is included in the class table.

        \classbasics{Defenses} \plus4 Reflex, \plus2 Mental.

        \cf{Rog}{Weapon and Armor Proficiency}
        Rogues are proficient with simple weapons, any two weapon groups, light armor, and bucklers.
        They are also proficient with saps.

        \cf{Rog}{Advanced Training}
        A rogue treats her skill ranks as if they were 2 points higher than they actually are for the purpose of meeting feat prerequisites.

    \subsection{Class Abilities}
        All rogues have the following abilities.

        \cf{Rog}{Sneak Attack}
        If a rogue can catch an opponent when it is unable to defend himself effectively from her attack, she can strike a vital spot for extra damage.
        She can choose to deal 1d6 points of extra damage if the target is unaware or is suffering overwhelm penalties from being surrounded by enemies (see \pcref{Overwhelm}).

        The extra damage increases by 1d6 at her 3rd rogue level and every two rogue levels thereafter.

        \par Ranged attacks can count as sneak attacks only if the target is within 30 feet.
        A rogue can't strike with deadly accuracy from beyond that range.

        With a sap (blackjack) or an unarmed strike, a rogue can make a sneak attack that deals nonlethal damage instead of lethal damage.
        She cannot use a weapon that deals lethal damage to deal nonlethal damage in a sneak attack, not even with the usual \minus4 penalty.

        A rogue can only sneak attack creatures with a discernible body structure -- oozes, incorporeal creatures, and some plants lack vital areas to attack.
        Any creature that is immune to critical hits is not vulnerable to sneak attacks.
        The rogue must be able to see the target well enough to pick out a vital spot and must be able to reach such a spot.
        A rogue cannot sneak attack while striking the limbs of a creature whose vitals are beyond reach.

        \cf{Rog}[2nd]{Skill Talent}[Ex]
        The rogue's skills improve.
        She gains an additional skill point, which she can place in any skill, and a bonus skill feat for which she qualifies.
        At her 6th rogue level, and every four rogue levels thereafter, she gains an additional skill point and skill feat.

        \cf{Rog}[3rd]{Uncanny Dodge}[Ex]
        The rogue can react to danger before her senses would normally allow her to do so.
        The rogue reduces her overwhelm penalties by 1.
        If her overwhelm penalty is reduced to 0, she is not considered to be overwhelmed.
        In addition, she is not \unaware when attacked by surprise.

        \cf{Rog}[4th]{Combat Tricks}
        The rogue gains a combat trick to aid her and confound her foes.
        She chooses a single combat trick from the list below.
        Some combat tricks have minimum rogue levels, as indicated in the title of the ability.
        At her 8th rogue level, and every four rogue levels thereafter, the rogue gains an additional combat trick.

        Some combat tricks depend on a rogue's trick power.
        A rogue's trick power is equal to her Intelligence or her character level, whichever is higher.

        Tricks marked with an asterisk are called ambush attacks.
        Ambush attacks only function on the first sneak attack the rogue makes against a particular creature in an encounter.

        All combat tricks are extraordinary abilities unless otherwise noted.

        \subcf{Combat Feat}
        The rogue gains a combat feat for which she qualifies (see Feats).
        This trick can be selected multiple times.

        \subcf{Confusing Ambush*}
        The rogue makes a Trick power vs. Mental attack against the struck creature.
        Success means the struck creature is \dazed for 2 rounds.
        Critical success means it is instead \confused for 2 rounds.
        This is an ambush attack, and only works once per creature.

        \subcf{Distracting Ambush*}
        A creature damaged by this ambush attack automatically fails any Concentration checks it makes that round.
        This is an ambush attack, and only works once per creature.

        \subcf{Distant Precision}
        The rogue can make sneak attacks from up to 100 feet away.

        \subcf{Hamstring*}
        A creature damaged by this ambush attack has its land speed halved for 5 rounds.
        Despite the name, this can be used on creatures who do not have hamstrings.
        This is an ambush attack, and only works once per creature.

        \subcf{Merciful Blows}
        The rogue suffers no penalty to physical attacks when attacking for nonlethal damage, and can deal her full sneak attack damage when attacking nonlethally.

        %\subcf{Swift Poisoner}
        %The rogue can apply poison to a weapon she is holding as a swift action.

        \subcf{Nauseating Ambush*}
        The rogue makes a Trick power vs. Fortitude attack against the struck creature.
        Success means the struck creature is \staggered for 2 rounds.
        This is an ambush attack, and only works once per creature.

        \subcf{Tricky Maneuver}
        When performing a maneuver against a creature she would be able to sneak attack, the rogue gains a bonus to accuracy equal to the number of sneak attack dice she would roll.
        The benefits of this trick apply even against creatures immune to critical hits.

        \subcf{8th -- Brutal Ambush*}
        The rogue rolls d8s instead of d6s for her sneak attack dice on this ambush attack.
        This is an ambush attack, and only works once per creature.

        \subcf{8th -- Dispelling Ambush (Su)*}
        A creature damaged by this ambush attack is affected by \spell{dispel magic}.
        The rogue's spellpower for this ability is equal to her trick power.
        This is an ambush attack, and only works once per creature.

        \subcf{8th -- Immobilizing Ambush*}
        A creature damaged by this ambush is \immobilized for 2 rounds.
        This is an ambush attack, and only works once per creature.

        \subcf{12th -- Assassination}
        To use this ability, the rogue must spend a full round studying a creature within 100 feet of her who has not noticed her and who is not in combat.
        If she make a melee sneak attack against that target within 1 round, her attack deals maximum damage, including her sneak attack damage.
        If the target becomes aware of her presence before she attacks, this ability has no benefit.

        \subcf{12th -- Confusing Ambush*}
        The rogue makes a special attack against the Mental defense of the creature struck by this ambush attack.
        Her accuracy is equal to her trick power.
        If the special attack succeeds, the struck creature is \disoriented for 5 rounds.
        If it critically succeeds, the struck creature is instead \confused for 5 rounds.
        This is an ambush attack, and only works once per creature.

        \subcf{12th -- Perfect Precision}
        The rogue has no range limit on her sneak attacks.

        \subcf{12th -- Spellreaving Ambush (Su)*}
        All spells and magical effects on the creature struck by this ambush attack are dispelled.
        Spells and effects that cannot be removed by \spell{dispel magic} are unaffected.
        This is an ambush attack, and only works once per creature.

        \subcf{12th -- Agonizing Ambush*}
        The rogue makes a Trick power vs. Mental attack against the struck creature.
        Success means the target takes double damage from all damage for 2 rounds.
        The damage from the ambush attack and other attacks during the same phase is not doubled.
        This is an ambush attack, and only works once per creature.

        \subcf{16th -- Deadly Ambush*}
        The rogue makes a Trick power vs. Fortitude attack against the struck creature.
        Success means the target is \staggered, and dies if it loses all its hit points for 2 rounds.
        Critical success means the target immediately dies.
        This is an ambush attack, and only works once per creature.
        In addition, dying in this way is a \glossterm{Death} effect.

        \subcf{20th -- Dual Ambush}
        The rogue can apply the benefits of two ambush attacks to a single sneak attack.

        \subcf{20th -- Lingering Ambush}
        The effects of the rogue's ambush attacks last ten times longer than normal.
        This has no effect on ambush attacks that have no duration.

        \cf{Rog}[5th]{Skill Exemplar}[Ex]
        The rogue gains a \plus5 bonus with a single skill of her choice.
        At her 9th rogue level, and every four rogue levels thereafter, she may gain this bonus with an additional skill.

        \cf{Rog}[7th]{Improved Uncanny Dodge}[Ex]
        The rogue reduces her overwhelm penalties by 2.
        This does not stack with the effects of uncanny dodge.
        If her overwhelm penalty is reduced to 0, she is not considered to be overwhelmed.

        \cf{Rog}[11th]{Slippery Mind}[Ex]
        Whenever the attack for a \glossterm{Mind} spell or effect beats the rogue's Mental defense by less than 10, she is affected normally at first.
        One round later, the rogue is instead affected as if the attack had failed.
        This does not help against instantaneous effects.

        \cf{Rog}[19th]{Ambush Master}[Ex]
        The rogue's ambush attacks function on the first attack the rogue makes against a particular creature in a single round, rather than within a single encounter.

\section{Sorcerer}\label{Sorcerer}
    \begin{dtable}
        \lcaption{Sorcerer Progression}
        \begin{dtabularx}{\columnwidth}{>{\ccol}p{\levelcol} >{\ccol}p{\babcolpoor} c >{\lcol}X}
            \tb{Level} & \tb{Combat Prowess} & \tb{Spells} & \tb{Special} \\
            \hline
            \sorcererprogressionrow{1}  & 2 & Cantrip, spells, wild magic, wild tolerance \\
            \sorcererprogressionrow{2}  & 3 & Cantrip, personal spell     \\
            \sorcererprogressionrow{3}  & 3 & Innate magic                \\
            \sorcererprogressionrow{4}  & 4 & Personal spell              \\
            \sorcererprogressionrow{5}  & 4 & Focused casting             \\
            \sorcererprogressionrow{6}  & 5 & Personal spell              \\
            \sorcererprogressionrow{7}  & 5 & Innate magic                \\
            \sorcererprogressionrow{8}  & 6 & Personal spell              \\
            \sorcererprogressionrow{9}  & 6 & Spell resistance            \\
            \sorcererprogressionrow{10} & 7 & Personal spell              \\
            \sorcererprogressionrow{11} & 7 & Innate magic                \\
            \sorcererprogressionrow{12} & 8 & Personal spell              \\
            \sorcererprogressionrow{13} & 8 & Spontaneous personalization \\
            \sorcererprogressionrow{14} & 9 & Personal spell              \\
            \sorcererprogressionrow{15} & 9 & Innate magic                \\
            \sorcererprogressionrow{16} & 10 & Personal spell              \\
            \sorcererprogressionrow{17} & 10 & Spell absorption            \\
            \sorcererprogressionrow{18} & 11 & Personal spell              \\
            \sorcererprogressionrow{19} & 11 & Innate magic                \\
            \sorcererprogressionrow{20} & 12 & Personal spell              \\
        \end{dtabularx}
    \end{dtable}

    \classbasics{Alignment} Any.

    \classbasics{Class Skills}
    \subparhead{Intelligence} Knowledge (arcana, the planes).
    \subparhead{Perception} Awareness, Spellcraft.
    \subparhead{Other} Bluff, Intimidate, Persuasion.

    \subsection{Base Class Abilities}
        A character with sorcerer as a base class gains the following abilities.

        \classbasics{Skill Points} 5.

        \classbasics{Combat Prowess} \plus1. This benefit is included in the class table.

        \classbasics{Defenses} \plus4 Mental.

        \cf{Sor}{Weapon and Armor Proficiency}
        Sorcerers are proficient with simple weapons  and one other weapon group.
        They are not proficient with any type of armor or shield.
        Armor of any type interferes with a sorcerer's arcane gestures, which can cause his spells with somatic components to fail.

        \cf{Sor}{Wild Tolerance}
        When the sorcerer fails a wild magic roll, the spell's normal effect happens in addition to its miscast effect.
        See Wild Magic, below, for details.

        \cf{Sor}{Enhanced Spellpower}
        The sorcerer gains a \plus2 bonus to spellpower with his sorcerer spells.

    \subsection{Class Abilities}
        All sorcerers have the following abilities.

        \cf{Sor}{Spells}
        A sorcerer casts arcane spells using his Willpower.
        The maximum spell level a sorcerer can learn or cast is equal to half his sorcerer level (minimum 1) or his Willpower, whichever is lower.
        A sorcerer's \glossterm{spellpower} is normally equal to his character level.

        A sorcerer begins play knowing two first-level spells.
        Every even level, he learns an additional spell of any level he has access to.
        In addition, each time he gains a level, he may trade one of his existing spells for a different spell known.
        However, he must always know at least one spell of every level he has access to.
        A sorcerer's spells are drawn from the spells on the arcane spell list (see \pcref{Arcane Spells}).

        Sorcerers do not have a limit on the number of spells they can cast each day.
        Their ability to cast spells is limited by their lack of control over their magic.
        See Wild Magic, below, for details.

        \cf{Sor}{Wild Magic}[Ex]\label{Wild Magic}
        Every time a sorcerer casts a spell, except cantrips, he must make a \glossterm{wild magic roll}.
        To make a wild magic roll, roll d20 \add half sorcerer level or half Willpower, whichever is higher.
        The DR of the roll is equal to 10 \add the spell's level.
        This roll is not an attack or check, and is not modified by abilities that affect attacks or checks.
        On a natural 1, this roll always fails, regardless of other modifiers.

        If the wild magic roll succeeds, the spell is cast normally.
        Failure means the spell is miscast instead (see \pcref{Miscasting}).
        In addition, if the sorcerer fails a wild magic roll, he cannot cast spells of the same level as the miscast spell for 10 minutes per spell level.
        Effects that prevent miscasts do not prevent the sorcerer from losing access to spells in this way.

        If a sorcerer has the ability to cast a spell without spending a spell slot, he may choose not to make a wild magic roll when casting the spell.

        \cf{Sor}{Cantrip}
        Cantrips are minor spells which do not require effort to use.
        A sorcerer chooses one cantrip from the list of cantrips on \pref{Cantrip List}.
        He may use the cantrip at will.
        Cantrips cannot be miscast, and are not affected by the sorcerer's wild magic ability.
        For all other purposes, cantrips are treated as 0th level spells.

        At his 2nd sorcerer level, the sorcerer learns a second cantrip of his choice.

        \cf{Sor}[2nd]{Personal Spell}
        The sorcerer's spells are unique to him.
        He chooses a single spell he knows with a maximum level equal to half his sorcerer level.
        It gains one of the following benefits, as he chooses.
        \begin{itemize}
            \item Augment: The sorcerer chooses one of the spell's augments.
                The cost to use that augment is reduced by one level (to a minimum of zero).
            \item Flawless: The sorcerer cannot miscast the spell.
                If he would miscast it, the spell simply fails without effect.
                In addition, he automatically succeeds at all Concentration checks he makes to cast the spell.
            \item Mastered: The sorcerer no longer needs verbal or somatic components to cast the spell.
            \item Metamagic Mastery: The sorcerer chooses a metamagic feat.
                The cost to use that metamagic feat on that spell is reduced by one level (to a minimum of zero).
        \end{itemize}

        If he trades the knowledge of his chosen spell for a different spell, he may choose a new personal spell (including the new spell he just gained).
        At his 4th sorcerer level, and every even sorcerer level thereafter, the sorcerer gains a new personal spell.
        He may choose a new spell, or a spell he has already chosen as a personal spell.
        However, the spell's level \add the number of personalizations on the spell cannot exceed half his sorcerer level \add 1.

        \cf{Sor}[3rd]{Innate Magic}[Su]
        The sorcerer becomes inherently magical.
        He may choose an arcane spell with a maximum level equal to half his sorcerer level.
        The spell must be a \glossterm{targeted spell} and have a duration of \durshort or longer.
        He does not need to know the spell.
        The sorcerer constantly gains the benefits of that spell.
        If the spell has secondary effects, such as the \spell{avatar of suffering} spell, the duration of the secondary effects is not changed.

        When the sorcerer chooses a spell in this way, the sorcerer may also choose specific augments for that spell, as long as the spell's effective level does not exceed half the sorcerer's level.
        If the spell's effect ends for any reason, such as if its effect is discharged, it takes effect on the sorcerer again 5 minutes later.

        At his 7th sorcerer level, and every 4 sorcerer levels thereafter, the sorcerer gains the benefit of an additional spell.

        \cf{Sor}[5th]{Focused Casting}[Ex]
        If the sorcerer spends one minute focusing on casting a spell, he does not need to make a wild magic roll.
        This prevents him from miscasting the spell due to wild magic.
        If his concentration is broken while casting a spell in this way, the sorcerer automatically miscasts the spell and suffers consequences as if he had failed a wild magic roll.

        \cf{Sor}[9th]{Spell Resistance}[Ex]
        The sorcerer gains \glossterm{spell resistance} equal to 10 \add his character level or Constitution, whichever is higher.
        To affect the sorcerer with a spell, a caster must make an attack with an accuracy equal to its spellpower.
        If the attack beats the sorcerer's spell resistance, the spell works normally.
        Otherwise, the spell has no effect on the sorcerer.

        This ability does not protect the sorcerer from his own miscast effects.

        \cf{Sor}[13th]{Spontaneous Augmentation}[Ex]
        Whenever he casts a spell, the sorcerer may reduce the level cost of one of the spell's augments by one level (to a minimum of zero).
        Alternately, he may reduce the level cost to apply one of his metamagic feats to the spell by one level (to a minimum of zero).

        \cf{Sor}[17th]{Spell Absorption}[Ex]
        Whenever the sorcerer resists a spell with his spell resistance, he gains the ability to cast that spell once during the next 5 rounds.
        The spell retains all augments, metamagic feats, and similar modifications from the original caster, and the sorcerer cannot choose any other augments or apply his own metamagic feats.
        However, he makes all other decisions required to cast the spell, and uses his spellpower to determine the spell's effects.
        Once he casts the spell, he expends the absorbed energy, can cannot cast it again.

        \cf{Sor}[20th]{Spontaneous Mastery}[Ex]
        The sorcerer may use his spontaneous personalization ability twice every time he casts a spell, reducing the total level costs by two levels.

    \subsection{Variant Sorcerer}

        Warlocks are sorcerers that draw power from pacts with otherworldly creatures.

        \subsubsection{Warlock}

            Warlocks cast spells without training, like sorcerers.
            However, while sorcerers have innate magical power, warlocks draw power from dark pacts they have made with demons, fae, or other otherworldly creatures.

            \altcf{Wild Magic} The warlock does not gain this ability.

            \altcf{Pact Magic} Every time the warlock casts a spell, he must make a pact magic roll.
            To make a pact magic roll, roll d20 \add half warlock level or half Willpower, whichever is higher.
            The DR of the roll is equal to 10 \add the spell's level.
            This roll is not an attack or check, and is not modified by abilities that affect attacks or checks.
            On a natural 1, this roll always fails, regardless of other modifiers.

            Success means the spell is cast normally.
            Failure means that the warlock's pact backfires, allowing the dark entities which gave him power to influence the world instead.

\section{Spellwarped}\label{Spellwarped}
    \begin{dtable}
        \lcaption{Spellwarped Progression}
        \begin{dtabularx}{\columnwidth}{>{\ccol}p{\levelcol} >{\ccol}p{\babcolgood} >{\lcol}X}
            \tb{Level} & \tb{Combat Prowess} & \tb{Special} \\
            \hline
            \spellwarpedprogressionrow{1}  & Innate magic, invocation, warp regeneration  \\
            \spellwarpedprogressionrow{2}  & Spellwarped body, surge of power             \\
            \spellwarpedprogressionrow{3}  & Magical senses, spellwarped aspect           \\
            \spellwarpedprogressionrow{4}  & Invocation                                   \\
            \spellwarpedprogressionrow{5}  & Manipulate magic                             \\
            \spellwarpedprogressionrow{6}  & Invocation                                   \\
            \spellwarpedprogressionrow{7}  & Spellwarped aspect                           \\
            \spellwarpedprogressionrow{8}  & Invocation                                   \\
            \spellwarpedprogressionrow{9}  & Spell resistance                             \\
            \spellwarpedprogressionrow{10} & Invocation                                   \\
            \spellwarpedprogressionrow{11} & Spellwarped aspect                           \\
            \spellwarpedprogressionrow{12} & Invocation                                   \\
            \spellwarpedprogressionrow{13} & Improved manipulate magic                    \\
            \spellwarpedprogressionrow{14} & Invocation                                   \\
            \spellwarpedprogressionrow{15} & Spellwarped aspect                           \\
            \spellwarpedprogressionrow{16} & Invocation                                   \\
            \spellwarpedprogressionrow{17} & Mass surge of power                          \\
            \spellwarpedprogressionrow{18} & Invocation                                   \\
            \spellwarpedprogressionrow{19} & Permanent surge of power, spellwarped aspect \\
            \spellwarpedprogressionrow{20} & Invocation                                   \\
        \end{dtabularx}
    \end{dtable}

    \classbasics{Alignment} Any.

    \classbasics{Class Skills}
    \subparhead{Intelligence} Knowledge (arcana).
    \subparhead{Perception} Awareness, Spellcraft.
    \subparhead{Other} Bluff, Intimidate, Persuasion.

    \classbasics{Special Class Skills}
    A spellwarped gains additional class skills based on his choice of innate magic.
    \subparhead{Alteration} Disguise, Escape Artist, Jump.
    \subparhead{Pyromancy} Acrobatics, Jump, Sprint.
    \subparhead{Telekinesis} Devices, Escape Artist, Sleight of Hand.
    \subparhead{Temporal} Acrobatics, Sleight of Hand, Sprint.

    \subsection{Warp Damage}\label{Warp Damage}
        A spellwarped can use the innate magic within his body to generate powerful magical effects.
        However, doing so is physically taxing.
        Most spellwarped abilities cause the spellwarped to take some amount of \glossterm{warp damage}.
        Warp damage cannot be cured by effects that restore hit points, effectively reducing the spellwarped's maximum hit points.

        A spellwarped cannot voluntarily take warp damage that would exceed half his maximum hit points.
        An hour of rest cures warp damage equal to a character's level.

    \subsection{Base Class Abilities}
        A character with spellwarped as a base class gains the following abilities.

        \classbasics{Skill Points} 5.

        \classbasics{Combat Prowess} \plus2. This benefit is included in the class table.

        \classbasics{Defenses} \plus4 to one defense, \plus2 to other defenses (see the Innate Magic ability, below).

        \cf{Spl}{Weapon and Armor Proficiency}
        A spellwarped is proficient with simple weapons, any two weapon groups, light and medium armor, and shields.

        \cf{Spl}{Warp Regeneration}
        The spellwarped does not need to rest to cure warp damage.
        He automatically heals warp damage equal to his level every hour, regardless of any activity he takes in the meantime.
        If he rests, he also recovers warp damage for resting, doubling the warp damage he heals.

    \subsection{Class Abilities}
        All spellwarped have the following abilities.

        \cf{Spl}{Spellpower}[Su]
        The strength of a spellwarped's spells and abilities are determined by his spellpower.
        His spellpower is equal to his key attribute or his character level, whichever is higher.

        \cf{Spl}{Innate Magic}[Su]
        Each spellwarped draws his magical power from a particular kind of magic.
        This is a choice made when the first level of the class is taken, and it cannot thereafter be changed.
        The choices are listed below.

        \subcf{Alteration}
        The spellwarped can manipulate the physical forms of creatures.
        His good defense is Fortitude, his key attribute is Intelligence, and he treats Disguise, Escape Artist, and Jump as class skills.
        An alteration spellwarped may be called an alterer, bodywarper, or shifter.

        As a standard action, he can change minor aspects of his appearance, such as removing a mole or lengthening his beard.
        This can grant him a \plus2 bonus to Disguise checks.
        Major changes are not possible.

        \subcf{Pyromancy}
        The spellwarped can manipulate fire and heat.
        His good defense is Fortitude, his key attribute is Willpower, and he treats Acrobatics, Jump, and Sprint as class skills.
        A pyromancy spellwarped may be called a pyromancer.

        As a standard action, he can snap his fingers to create a small ember of flame in his hand for 5 minutes.
        This ember casts light as a torch, and can deal 1 point of fire damage with a successful touch attack.
        It can be dismissed as a swift action or extinguished as a move action.

        \subcf{Telekinesis}
        The spellwarped can manipulate objects and creatures with his mind.
        His good defense is Mental, his key attribute is Willpower, and he treats Devices, Escape Artist, and Sleight of Hand as class skills.
        A telekinesis spellwarped may be called a telekine.

        As a standard action, he can concentrate to move objects within ten feet of him telekinetically.
        He can slowly lift or manipulate one object by up to one foot per round.
        The object can weigh up to five pounds.
        This level of control is insufficient to make skill checks or wield a weapon or shield effectively.

        \subcf{Temporal}
        The spellwarped can manipulate time.
        His good defense is Reflex, his key attribute is Perception, and he treats Awareness, Sleight of Hand, and Sprint as class skills.
        A temporal spellwarped may be called a temporalist or timewarper.

        The spellwarped always knows exactly what time it is, and can track the passage of time precisely without effort.

        \cf{Spl}{Invocation}
        A spellwarped can invoke his innate magic to generate powerful effects.
        He chooses a single invocation at 1st level from those available based on his choice of innate magic.
        Using an invocation inflicts \glossterm{warp damage} to the spellwarped equal to half the minimum spellwarped level required to learn the invocation (minimum 1).

        At his 4th spellwarped level, and every two spellwarped levels thereafter, he gains an additional invocation.
        Some invocations have minimum spellwarped levels, as indicated in the title of the ability.
        The list of invocations is given at \pcref{Invocations}.

        All invocations are supernatural abilities unless otherwise noted.
        The spellwarped's accuracy with invocations is equal to his spellpower.

        \cf{Spl}[2nd]{Surge of Power}[Su]
        As a swift action, the spellwarped can invoke a surge of magical power that allows him to embody his innate magic more fully for 5 rounds.
        Invoking a surge of power causes the spellwarped to take one warp damage.
        The effect of his surge depends on his choice of innate magic, as described below.
        \subcf{Alteration -- Fast Healing}
        The spellwarped heals his wounds rapidly.
        At the end of each round, he heals hit points equal to his spellpower.
        \subcf{Pyromancy -- Flame Aura}
        The spellwarped emanates an aura of fire.
        At the end of each round, enemies adjacent to him take fire damage equal to his spellpower.
        \subcf{Telekinesis -- Kinetic Deflection}
        The spellwarped reflexively deflects attacks away with his mind.
        He gains a \plus2 bonus to his physical defenses (Armor, Maneuver, Reflex).
        In addition, he may use his Intelligence to determine his physical defenses in place of his Dexterity or Constitution.
        \subcf{Temporal -- Accelerate Movement}
        The spellwarped accelerates his movement and reactions.
        He gains a \plus2 bonus to his Reflex defense and a \plus10 foot bonus to his movement speed.
        He also gains a bonus on his Sprint checks equal to his spellpower.
        At spellpower 8, 14, and 20, the defense bonus increases by 1 and the speed bonus increases by 10 feet.

        \cf{Spl}[2nd]{Spellwarped Body}[Ex]
        The spellwarped's body is fundamentally altered by exposure to magic.
        He shows signs of the magic coursing through his body: strangely or inconsistently colored hair, natural skin markings which often resemble runes, and so on.
        Anyone observing the spellwarped can make an Awareness or Spellcraft check with a DR equal to 20 \sub his spellwarped level to recognize that the character is a spellwarped.
        Critical success on the check allows the observer to determine the type of innate magic the spellwarped has.
        In addition, the spellwarped gains an ability based on his innate magic.
        \subcf{Alteration -- Sturdy Body}
        The spellwarped gains a \plus2 bonus to his Fortitude defense.
        This bonus increases by 1 at spellpower 5 and every 5 spellpower thereafter.
        \subcf{Pyromancy -- Energy Resistance}
        The spellwarped gains \glossterm{damage reduction} against cold and fire damage equal to twice his spellpower.
        \subcf{Telekinesis -- Tactile Telekinesis}
        The spellwarped gains a \plus1 bonus to Strength and Dexterity-based checks.
        This bonus increases by 1 at spellpower 5 and every 5 spellpower thereafter.
        \subcf{Temporal -- Accelerate Mind}
        The spellwarped gains a \plus2 bonus to Intelligence and Perception-based checks.
        This bonus increases by 1 at spellpower 5 and every 5 spellpower thereafter.

        \cf{Spl}[3rd]{Magical Senses}[Su]
        The spellwarped learns to recognize the telltale signs of his chosen magic.
        He must concentrate as a standard action to use this ability, and he may do so any number of times per day.
        \subcf{Alteration -- Perceive Alteration}
        The spellwarped can discern the true form of all creatures within 50 feet of him for 1 round, ignoring any effects which magically alter their shapes.
        This also grants him a \plus5 bonus to Awareness checks to see through disguises.
        \subcf{Pyromancy -- Flame of Life}
        The spellwarped can see the life-fire that lies within all living creatures, allowing him to clearly see all living creatures within 50 feet of him for 1 round.
        This ability can reveal creatures hiding in concealment and defeat figments and glamers such as \spell{invisibility}, but does not reveal creatures hiding behind cover.
        It also allows the spellwarped to see unusually warm objects, such as fires.
        \subcf{Telekinesis -- Spatial Awareness}
        The spellwarped can feel the forms of all objects and creatures around him, granting blindsense out to a 50 foot range for 1 round.
        \subcf{Temporal -- Rapid Search}
        The spellwarped can accelerate his mind to immediately search everything within a 10 foot radius of him with the Awareness skill as a standard action.
        Alternately, he may use this ability to read a book ten times as fast as normal.

        \cf{Spl}[3rd]{Spellwarped Aspect}[Su]
        The spellwarped gains a new ability based on his continued exposure to magical energy.
        Most aspects are specific to particular kinds of innate magic, but some aspects can be taken by any spellwarped.
        These aspects are listed under the General heading.

        At his 7th spellwarped level, and every four spellwarped levels thereafter, the spellwarped gains an additional spellwarped aspect.
        Some aspects require a minimum spellwarped level, as indicated in the title of the ability.
        The full list of spellwarped aspects is given below.

        \parhead{General}
        \subcf{Resilient}
        The spellwarped increases the defense improved by his choice of innate magic by 2.
        This can increase his hit points, if appropriate.
        \subcf{7th -- Warp Overload}
        The spellwarped can voluntarily take \glossterm{warp damage}, even if that would exceed half his maximum hit points.
        He is only unable to voluntarily take warp damage if the total warp damage would exceed his maximum hit points.
        \subcf{11th -- Persistent Senses}
        The spellwarped can constantly gain the benefit of his magical senses ability.
        He can toggle his enhanced senses on or off as a swift action.
        If the ability does not have a duration, such as the temporal magical senses ability, this aspect has no effect.

        \parhead{Alteration}
        \subcf{Damage Reduction}
        The spellwarped gains \glossterm{damage reduction} against physical damage equal to his spellpower.
        Adamantine weapons ignore this damage reduction and negate it for 1 round.
        \subcf{7th -- Alter Movement}
        The spellwarped gains his choice of the Legendary Balance, Legendary Climber, Legendary Leaper, Legendary Swimmer, or Legendary Tumbler feats, even if he does not meet the prerequisites.
        He may select this aspect multiple times, choosing a different bonus feat each time.
        \subcf{7th -- Alter Size}
        When the spellwarped uses his surge of power, he can increase or decrease his size by one size category, as he chooses.
        The size alteration lasts as long as his surge of power does.
        This is a sizing effect, and does not stack with other sizing effects.
        \subcf{11th -- Regeneration}
        While the spellwarped's surge of power is active, at the end of each round, he can heal critical damage equal to half his spellpower.
        If he does, that replaces his normal fast healing for that round.

        \parhead{Pyromancy}
        \subcf{Intense Flames}
        The spellwarped can choose to have his spellwarped abilities ignore an amount of fire damage reduction equal to his spellpower.
        \subcf{7th -- Flame Eater}
        When the spellwarped resists fire damage with his spellwarped body ability, he gains temporary hit points equal to the damage resisted for 5 minutes.
        \subcf{7th -- Retributive Flames}
        When a creature makes a melee attack against the spellwarped, it takes \spelldamage{fire}[d6].
        Each creature can only take this damage once per round.

        \parhead{Telekinesis}
        \subcf{Improved Object Manipulation}
        When the spellwarped uses his object manipulation ability, he can affect objects within 10 feet, with a weight limit of up to two pounds per spellpower.
        He has enough control to make checks with a DR of up to 10.
        \subcf{7th -- Shieldbearer}
        The spellwarped may wield shields, except tower shields, telekinetically.
        The shield floats in his square, granting him its bonus to his physical defenses just as if he were wielding it.
        He does not need a free hand to wield the shield and suffers no \glossterm{encumbrance penalty} or arcane spell failure from it.
        The shield follows him as he moves.
        If it is forcibly removed from his square, he loses control over it and it falls to the ground.
        \subcf{11th -- Mind Armory}
        The spellwarped may control a number of weapons equal to half his Intelligence with his mind blade ability.
        This does not allow him to make additional attacks per round, but he may attack interchangeably with any weapon he controls.
        Each weapon threatens an area and contributes to overwhelm penalties, just as with his normal mind blade ability.

        \parhead{Temporal}
        \subcf{Acceleration}
        The spellwarped gains a \plus10 foot bonus to speed in all his movement modes.
        \subcf{Uncanny Dodge}
        The spellwarped can react to danger before his senses would normally allow him to do so.
        He reduces his overwhelm penalties by 1.
        If his overwhelm penalty is reduced to 0, he is not considered to be overwhelmed.
        In addition, he is not \unaware when attacked by surprise.
        \subcf{7th -- Accelerate Attack}
        While his surge of power is active, the spellwarped can make an additional \glossterm{strike} at a \minus5 penalty when making a standard attack.
        This does not stack with any other effects which grant extra strikes.
        \subcf{7th -- Empowered Acceleration}
        The spellwarped gains a \plus30 foot bonus to speed in all his movement modes.
        \subcf{11th -- Improved Uncanny Dodge}
        The spellwarped reduces his overwhelm penalties by 2.
        If his overwhelm penalty is reduced to 0, he is not considered to be overwhelmed.

        \cf{Spl}[5th]{Manipulate Magic}[Su]
        The spellwarped can channel his innate magic to manipulate spells and spell-like abilities.
        Using this ability causes the spellwarped to take \glossterm{warp damage} equal to half the spellpower of the spell or ability affected.
        \subcf{Alteration -- Absorption}
        As an immediate action, when the spellwarped makes successfully resists an attack against his Fortitude from a spell or spell-like ability, he may absorb the magic harmlessly into his body.
        The spell has no effect on him, even if it would normally have an effect on a failed attack.
        \subcf{Pyromancy -- Fuel the Flame}
        As an immediate action, when the spellwarped is affected by a spell or spell-like ability, he may channel its energy into a burst of flame around him.
        Enemies within a \areamed radius of the spellwarped take fire damage equal to his spellpower.
        The spell still has its normal effect on the spellwarped.
        \subcf{Telekinesis -- Mind over Matter}
        As an immediate action, when the spellwarped is subject to an attack against his Fortitude from a spell or spell-like ability, he may use his Mental defense instead.
        \subcf{Temporal -- Accelerate Magic}
        As a swift action, the spellwarped can halve the duration of any spell or spell-like ability affecting him.
        This can end the effect immediately if it has less than one round remaining.
        If this would reduce the duration by more than one day, the duration is instead reduced by one day.

        \cf{Spl}[9th]{Spell Resistance}[Ex]
        The magic within the spellwarped allows him to completely ignore other magic, granting him spell resistance equal to 10 \add his Constitution or character level, whichever is higher.
        To affect the spellwarped with a spell, a caster must make an attack with its spellpower.
        If the attack beats the spellwarped's spell resistance, the spell works normally.
        Otherwise, the spell has no effect on the spellwarped.

        \cf{Spl}[13th]{Improved Manipulate Magic}[Su]
        The spellwarped can use his manipulate magic ability to affect any ally within \rngmed range of him.

        \cf{Spl}[17th]{Mass Surge of Power}[Su]
        The spellwarped can share the benefits of his surge of power with his allies.
        When he uses his surge of power, he can also affect up to five additional willing creatures within \rngmed range of him.

        \cf{Spl}[19th]{Permanent Surge of Power}[Su]
        The spellwarped can maintain the full power of his innate magic without limit.
        He can gain the effects of his surge of power indefinitely.
        He may toggle the ability on or off as a swift action at will, without taking warp damage.
        This does not allow him to activate his mass surge of power ability at will, and his allies only gain the benefits for 5 rounds.

    \subsection{Invocations}\label{Invocations}

        All invocations are spell-like abilities unless otherwise noted.
        An invocation's effective spell level is equal to half the minimum level required to learn it (minimum 1).

        Some invocations mimic the effects of spells.
        The spellwarped's spellpower with these effects is equal to his spellpower.

        \subsubsection{Alteration Invocations}
            \subcf{1st -- Body Bludgeon}
            The spellwarped distorts a part of his body and strikes a foe with it.
            He makes a Spellpower vs. Armor defense physical attack against a foe within his \glossterm{reach}.
            This attack scores critical hits like other physical attacks.
            Success means the target takes 1d6 bludgeoning damage per two spellpower \add half his Strength.
            Failure means the target takes half damage.
            \par At his 4th spellwarped level, this damage increases to 1d6 bludgeoning damage per spellpower \add half his Strength.
            \subcf{1st -- Shrink}
            This invocation functions like the \spell{shrink} spell.

            \subcf{4th -- Enlarge}
            This invocation functions like the \spell{enlarge} spell.
            \subcf{4th -- Purge}
            The spellwarped can end any single effect or condition on him which requires a successful attack against his Fortitude.
            If he identify the effects on him, such as with a Spellcraft check, he may freely choose which effect to end.
            If he cannot differentiate the effects, choose the effect ended randomly.

            \subcf{6th -- Amorphous Body}
            The spellwarped transforms his body into an amorphous form for \durshort duration.
            In this form, he gains several benefits.
            He gains a \plus20 bonus to Reflex defense against grapple attacks, is immune to critical hits, and can move through spaces that are no more than two inches in width without \glossterm{squeezing}.
            While moving through spaces smaller than he could normally move through, he moves at half speed.
            \subcf{6th -- Healing Transformation}
            As a standard action, the spellwarped can heal a creature within \rngclose range by transforming it into a healthier version of its normal body.
            The target heals 1d6 points of damage per spellpower.
            This also removes any of the following conditions: blinded, deafened, diseased, exhausted, fatigued, nauseated, sickened, and poisoned.
            \subcf{6th -- Mighty Throw}
            This invocation functions like the \spell{mighty throw} spell.

            \subcf{8th -- Bludgeon the Horde}
            This attack functions like the body bludgeon attack, except that the spellwarped may attack all foes within his reach, as if he were wielding a reach weapon.
            Success deals 1d8 bludgeoning damage per two spellpower \add half his Strength.
            Failure deals half damage.
            \subcf{8th -- Flight}
            The spellwarped grows physical wings that last for \durshort duration.
            The wings grant him a fly speed equal to his base land speed with good maneuverability (see \pcref{Flying}, for details).
            Unlike normal, he can fly with these wings even while encumbered.

        \subsubsection{Pyromancy Invocations}
            Unless otherwise noted, a pyromancer's invocations are \glossterm{Fire} effects, and shed light equivalent to a torch for their duration.

            \subcf{1st -- Burn}
            The spellwarped makes a Spellpower vs. Reflex defense attack against a foe within \rngmed range.
            Success means the target takes \spelldamage{fire}[d6].
            Critical success deals double damage.
            Failure means the target takes half damage.
            \par At his 4th spellwarped level, this damage increases to \spelldamage{fire}.
            \subcf{1st -- Flame Weapon}
            As a swift action, the spellwarped can create a weapon made of flame that lasts for 5 rounds.
            The weapon is sized appropriately for him, and may take the form of any weapon he is proficient with.
            He can attack with the weapon as if it were a normal weapon of its type, except that he uses his spellpower to determine his damage in place of his combat prowess or Strength, and all damage dealt with the weapon is fire damage.
            All other damage modifiers, such as from feats and abilities, apply normally.
            \par If the flame weapon leaves his hand, it is extinguished 1 round later.

            \subcf{4th -- Fiery Protection}
            As a standard action, the spellwarped can bestow fire and cold damage reduction equal to twice his spellpower on a creature within 30 feet of him.
            The protection lasts for \durshort duration.
            \subcf{4th -- Flame Shield}
            As a standard action, the spellwarped can wreath a willing creature within \rngclose range in flame for \durshort duration.
            Whenever a creature makes a melee attack against the target, the attacking creature takes \spelldamage{fire}[d6].
            A creature can only be dealt damage by this spell once per round.

            \subcf{6th -- Conflagration}
            As a standard action, the spellwarped can release a powerful explosion of flame.
            He makes a Spellpower vs. Reflex attack against everything within a \areamed radius burst of him.
            Success against a target means it takes \spelldamage{fire}[d8].
            Critical success deals double damage.
            Failure against a target means it takes half damage.
            \subcf{6th -- Fireball}
            This invocation functions like the \spell{fireball} spell.
            \subcf{6th -- Ignite}
            The spellwarped makes a Spellpower vs. Reflex attack against a foe within \rngclose range.
            Success means the target takes \spelldamage{fire}, and is \ignited for 2 rounds.
            Critical success deals double damage.
            Failure means the target takes half damage, but is still ignited.

            \subcf{8th -- Firestride}
            As a move action, the spellwarped can teleport to any active flame of at least Tiny size within \rngmed range.
            When he does so, he immolates himself and disappears into a pile of ash before stepping out from the flame unharmed.
            An ordinary torch is sufficient flame to teleport to, but not a candle.
            \subcf{8th -- Flameheart}
            As a standard action, the spellwarped can become a being of pure fire for \durshort duration.
            In this form, he is immune to physical damage and can pass through openings as small as one inch at no movement penalty.
            However, he cannot attack normally or use any of his items, as they meld into his body.
            He may use any of his invocations normally.
            In addition, as a standard action, he can make a Spellpower vs. Reflex attack to touch a creature.
            Success means the target takes \spelldamage{fire}.
            Critical success deals double damage.
            \subcf{8th -- Flight of the Phoenix}
            The spellwarped creates wings of flame that last for \durshort duration.
            The wings grant him a fly speed equal to his land speed with good maneuverability (see \pcref{Flying}, for details).
            Unlike normal, he can fly with these wings even while encumbered.

            \subcf{14th -- Immolation}
            This invocation functions like the \spell{immolation} spell.

            \subcf{18th -- Phoenix Revival}
            When the spellwarped takes critical damage, he may ignore the damage as an immediate action, even if the critical damage would be sufficient to kill him.
            If he does, he dissolves into a pile of ash for 2 rounds.
            During this time, he can take no actions.
            If the pile of ash remains intact after 2 rounds, the spellwarped is restored to his normal body.
            He has no hit points remaining, and warp damage equal to half his maximum hit points, but is healed of all critical damage.
            However, if the pile of ash is dispersed, the spellwarped dies.
            The ash cannot be harmed by fire damage, and if the pile of ash would take at least 50 points of fire damage during a round, the spellwarped returns one round sooner.
            The spellwarped may take his normal actions in the round after he is restored.
            \par After using this ability, the spellwarped cannot use it again for 1 hour.

            \subcf{20th -- Immolate}
            As a standard action, the spellwarped makes a special attack vs. Fortitude against a foe within \rngclose range to consume it in flames from the inside out.
            Success deals \spelldamage{fire}.
            Critical success kills the target instantly.
            Failure deals half damage.

        \subsubsection{Telekinesis Invocations}
            \subcf{1st -- Mind Crush}
            As a standard action, the spellwarped can make a Spellpower vs. Fortitude attack against a creature within \rngclose range.
            Success means the target takes \spelldamage{bludgeoning}[d6] and is \sickened for 2 rounds.
            Critical success deals double damage.
            Failure means the target takes half damage, but is still sickened.
            \par At his 4th spellwarped level, this damage increases to \spelldamage{bludgeoning}.
            \subcf{1st -- Mind Blade}
            As a swift action, the spellwarped can telekinetically wield an unattended weapon within \rngclose range for 5 rounds.
            The weapon must be a light or medium weapon appropriate for his size.
            This allows him to attack with the weapon just as if he were holding it in one hand, except that he uses his spellpower in place of his Strength or Dexterity.
            In all other respects, this functions as if he were wielding the weapon normally, including contributing to overwhelm penalties.
            The weapon floats in midair and threatens all squares adjacent to it.
            As a move action, he may move the weapon up to 30 feet in any direction, even vertically.
            If the weapon goes outside of \rngclose range, he loses control of it and it falls to the ground.

            \subcf{4th -- Dual Mind Blade}
            This invocation functions like his mind blade invocation, except that the spellwarped may wield two weapons at once.
            They must stay in the same space, and he may make two-weapon fighting attacks with the weapons, just as if he was wielding them with two hands.
            \subcf{4th -- Mighty Mind Blade}
            This invocation functions like the mind blade invocation, except that the spellwarped may also use a heavy weapon appropriate for his size, allowing him to attack with it just as if he were holding it in two hands.

            \subcf{6th -- Distant Manipulation}
            As a standard action, the spellwarped can mentally manipulate objects and creatures at up to \rngclose range for up to 5 rounds.
            This allows him to take any actions which he could normally take with his hands, using his spellpower in place of his Strength or Dexterity, as appropriate.

            \subcf{8th -- Immobilize}
            As a standard action, the spellwarped can make a creature within \rngmed range of him \immobilized for 2 rounds.

            \subcf{10th -- Telekinetic Blast}
            This invocation functions like the \spell{telekinetic blast} spell.

        \subsubsection{Temporal Invocations}
            Unless otherwise noted, all temporal invocations are \glossterm{Temporal} effects.

            \subcf{Haste}
            This invocation functions like the \spell{haste} spell.
            \subcf{Timelock}
            As a standard action, the spellwarped can attempt to lock a creature in time.
            He makes a Spellpower vs. Mental attack against an adjacent creature.
            Success means the target slips out of time for 1 round.
            During that time, it can take no actions, but cannot be harmed, moved, or affected in any way.
            Critical success means it slips selectively out of time for 1 round.
            During that time, it can take no actions, but can be harmed or moved as normal.
            It is not considered \helpless.
            \par A creature affected by this ability is immune to the effect for 5 rounds.

            \subcf{4th -- Disjointed Time}
            As a standard action, the spellwarped chaotically disrupts the local flow of time of a creature within \rngclose range.
            The target is \impaired with attacks and checks for 5 rounds.
            \subcf{4th -- Flash Step}
            As a move action, the spellwarped can accelerate a willing creature within \rngclose range so much that he can seem to pause time for everyone but the target.
            This allows the target to immediately take a single move action.
            During this move action, the target moves at double speed, cannot be followed or withdrawn from, and may move through squares occupied by creatures or threatened by blocking enemies without penalty.
            The target still suffers the effects of any environmental hazards.

            \subcf{6th -- Slow}
            This invocation functions like the \spell{slow} spell.

            \subcf{8th -- Inhuman Speed}
            As a move action, the spellwarped can accelerate himself to immense speed, allowing him to move up to five times his speed.
            During this time, he cannot be followed or withdrawn from, can move through squares occupied by enemies or threatened by blocking enemies without penalty, and can treat liquids as if they were solid ground.
            \subcf{8th -- Mass Haste}
            This invocation functions like the \spell{haste} spell with the Mass augment applied.

            \subcf{10th -- Mass Flash Step}
            This invocation functions like the flashstep invocation, except that it affects up to five willing creatures.
            \subcf{10th -- Timestream}
            The spellwarped manipulates time in a \arealarge, 10 ft.\ wide line-shaped zone that extends out from him for 5 rounds.
            All creatures and objects that pass through the line are \slowed for 1 round.
            The spellwarped can exclude his allies from the effect.
            The timestream is virtually invisible, requiring a DR 30 Awareness check to notice in a clear environment, though objects passing through the effect can make it more obvious.

            \subcf{14th -- Mass Slow}
            This invocation functions like the \spell{slow} spell with the Mass augment applied.

            \subcf{16th -- Time Reversal}
            As a swift action, the spellwarped can take one warp damage to create a ``time lock.'' The time lock persists for one round.
            As a standard action, he can take eight warp damage to make a Spellpower vs. Mental attack a creature within \rngmed range to reverse time for it.
            Success means the target is restored to its exact condition at the point immediately after the time lock was created.
            The effects of any actions that the creature took in the intervening time are undone, any damage it dealt or took is removed, it is returned to its original location, and the creature is restored in all other ways, just as if the intervening time had never occured.
            If its original location is occupied, the time reversal fails.
            The spellwarped cannot reverse time for himself in this way.
            \par After reversing time in this way, the spellwarped must wait 5 rounds before he can create a time lock or reverse time again.
            \subcf{16th -- Supreme Acceleration}
            As a standard action, the spellwarped can accelerate himself so much that he can take an additional round of actions immediately.
            During this round, all creatures he attacks are treated as \helpless, but he cannot perform a \glossterm{coup de grace} or similar ability; such an act requires more care and precision than is possible with such immense speed.
            After using this ability, he must wait 5 rounds before he can use it again.

            \subcf{18th -- Time Stop}
            As a standard action, the spellwarped can step into an alternate timestream, causing him to speed up so greatly that all other creatures seem frozen.
            He can act for 1d3\plus1 rounds of apparent time.
            During this time, all objects and creatures are frozen in place and are completely invulnerable to the spellwarped, though he may affect them with invocations normally.
            After using this ability, he must wait 5 rounds before he can use it again.

\section{Wizard}\label{Wizard}
    \begin{dtable}
        \lcaption{Wizard Progression}
        \begin{dtabularx}{\columnwidth}{>{\ccol}p{\levelcol} >{\ccol}p{\babcolpoor} c >{\lcol}X}
            \tb{Level} & \tb{Combat Prowess} & \tb{Spells} & \tb{Special} \\
            \hline
            \wizardprogressionrow{1}  & 2 & Cantrip, rituals, spells          \\
            \wizardprogressionrow{2}  & 3 & Arcane insight, cantrip           \\
            \wizardprogressionrow{3}  & 3 & Attuned spell                     \\
            \wizardprogressionrow{4}  & 4 & Arcane insight                    \\
            \wizardprogressionrow{5}  & 4 & Ritual master                     \\
            \wizardprogressionrow{6}  & 5 & Arcane insight                    \\
            \wizardprogressionrow{7}  & 5 & Attuned spell                     \\
            \wizardprogressionrow{8}  & 6 & Arcane insight                    \\
            \wizardprogressionrow{9}  & 6 & Persistent attunement             \\
            \wizardprogressionrow{10} & 7 & Arcane insight                    \\
            \wizardprogressionrow{11} & 7 & Attuned spell                     \\
            \wizardprogressionrow{12} & 8 & Arcane insight                    \\
            \wizardprogressionrow{13} & 8 & Contingency                       \\
            \wizardprogressionrow{14} & 9 & Arcane insight                    \\
            \wizardprogressionrow{15} & 9 & Attuned spell                     \\
            \wizardprogressionrow{16} & 10 & Arcane insight                    \\
            \wizardprogressionrow{17} & 10 & Augmented attunement              \\
            \wizardprogressionrow{18} & 11 & Arcane insight                    \\
            \wizardprogressionrow{19} & 11 & Attuned spell                     \\
            \wizardprogressionrow{20} & 12 & Arcane insight, chain contingency \\
        \end{dtabularx}
    \end{dtable}

    \classbasics{Alignment} Any.

    \classbasics{Class Skills}
    \subparhead{Intelligence} Knowledge (all kinds, taken individually), Linguistics.
    \subparhead{Perception} Awareness, Spellcraft.

    \subsection{Base Class Abilities}
        A character with wizard as a base class gains the following abilities.

        \classbasics{Skill Points} 5.

        \classbasics{Combat Prowess} \plus1. This benefit is included in the class table.

        \classbasics{Defenses} \plus4 Mental.

        \cf{Wiz}{Weapon and Armor Proficiency}
        Wizards are proficient with simple weapons, but not with any type of armor or shield.
        Armor of any type interferes with a wizard's movements, which can cause her spells with somatic components to fail.

        \cf{Wiz}{Advanced Training}
        A wizard treats her spells as if they were 1 level higher than they actually are for the purpose of meeting feat prerequisites.

        \cf{Wiz}{Enhanced Spellpower}
        The wizard gains a \plus2 bonus to spellpower with her wizard spells.

    \subsection{Class Abilities}
        All wizards have the following abilities.

        \cf{Wiz}{Spells}
        A wizard casts arcane spells using her Intelligence.
        The maximum spell level a wizard can learn or cast is equal to half her wizard level (minimum 1) or her Intelligence, whichever is lower.
        A wizard's spellpower is normally equal to her character level.

        A wizard begins play knowing two first-level spells.
        Every even level, she learns an additional spell of any level she has access to.
        In addition, each time she gains a level, she may trade one of her existing spells for a different spell known.
        However, she must always know at least one spell of every level she has access to.
        A wizard's spells are drawn from the spells on the arcane spell list (see \pcref{Arcane Spells}).

        The number of spells a wizard can cast per day is given on \trefnp{Wizard Spell Slots}.

        In order to regain her spells for the day, a wizard must dismiss all her active spells and rest for 8 hours.
        This rest does not have to involve sleep, but most wizards get this rest when they sleep for the night.

        \begin{dtable}
            \lcaption{Wizard Spell Slots}
            \centering
            \begin{dtabularx}{\columnwidth}{>{\ccol}X *{9}{>{\ccol}p{\spellcol}}}
                & \multicolumn{9}{c}{\tb{---{}---{}---{}---{}---{}---{}---{}---Spell Level---{}---{}---{}---{}---{}---{}---{}---}} \\
                \hline
                \tb{Level} & \tb{1st} & \tb{2nd} & \tb{3rd} & \tb{4th} & \tb{5th} & \tb{6th} & \tb{7th} & \tb{8th} & \tb{9th} \\
                1st  & 3 & \tdash & \tdash & \tdash & \tdash & \tdash & \tdash & \tdash & \tdash \\
                2nd  & 4 & \tdash & \tdash & \tdash & \tdash & \tdash & \tdash & \tdash & \tdash \\
                3rd  & 5 & \tdash & \tdash & \tdash & \tdash & \tdash & \tdash & \tdash & \tdash \\
                4th  & 6 & 3      & \tdash & \tdash & \tdash & \tdash & \tdash & \tdash & \tdash \\
                5th  & 6 & 4      & \tdash & \tdash & \tdash & \tdash & \tdash & \tdash & \tdash \\
                6th  & 6 & 5      & 3      & \tdash & \tdash & \tdash & \tdash & \tdash & \tdash \\
                7th  & 6 & 6      & 4      & \tdash & \tdash & \tdash & \tdash & \tdash & \tdash \\
                8th  & 6 & 6      & 5      & 3      & \tdash & \tdash & \tdash & \tdash & \tdash \\
                9th  & 6 & 6      & 6      & 4      & \tdash & \tdash & \tdash & \tdash & \tdash \\
                10th & 6 & 6      & 6      & 5      & 3      & \tdash & \tdash & \tdash & \tdash \\
                11th & 6 & 6      & 6      & 6      & 4      & \tdash & \tdash & \tdash & \tdash \\
                12th & 6 & 6      & 6      & 6      & 5      & 3      & \tdash & \tdash & \tdash \\
                13th & 6 & 6      & 6      & 6      & 6      & 4      & \tdash & \tdash & \tdash \\
                14th & 6 & 6      & 6      & 6      & 6      & 5      & 3      & \tdash & \tdash \\
                15th & 6 & 6      & 6      & 6      & 6      & 6      & 4      & \tdash & \tdash \\
                16th & 6 & 6      & 6      & 6      & 6      & 6      & 5      & 3      & \tdash \\
                17th & 6 & 6      & 6      & 6      & 6      & 6      & 6      & 4      & \tdash \\
                18th & 6 & 6      & 6      & 6      & 6      & 6      & 6      & 5      & 3      \\
                19th & 6 & 6      & 6      & 6      & 6      & 6      & 6      & 6      & 4      \\
                20th & 6 & 6      & 6      & 6      & 6      & 6      & 6      & 6      & 6      \\
            \end{dtabularx}
        \end{dtable}

        \cf{Wiz}{Cantrip}
        Cantrips are minor spells which do not require effort to use.
        A wizard chooses one cantrip from the list of cantrips on \pref{Cantrip List}.
        She may use the cantrip at will.
        Cantrips cannot be miscast.
        For all other purposes, cantrips are treated as 0th level spells.

        At her 2nd wizard level, the wizard gains a second cantrip, which can be chosen from any non-prohibited school.

        \cf{Wiz}{Rituals}
        Wizards can perform rituals to create unique magical effects (see \pcref{Rituals}).
        A wizard begins play with a ritual book containing two arcane rituals of her choice (see \pcref{Arcane Rituals}).

        \cf{Wiz}[2nd]{Arcane Insight}[Ex]\label{Arcane Insight}
        The wizard gains a greater understanding of magic.
        She chooses one of the following benefits.
        \begin{itemize}
            \item Bonus Spell: The wizard gains an additional spell slot at her highest level of spells known.
                As she gains new spell levels known, the spell slot increases in level accordingly.
            \item Ritual Spell: The wizard scribes an arcane spell with a maximum level equal to half her wizard level into her ritual book.
                She does not need to know the spell, and pays no cost to scribe it.
                The spell is treated as a ritual, and she can perform a one minute ritual to gain the spell's effect.
                Performing the ritual costs the normal amount of material components for a ritual of its level (see \pcref{Ritual Costs}).
            \item Specialization: The wizard gains an additional spell known.
                In exchange, she must ban a school of magic.
                She can never learn or cast spells or rituals from her banned schools.
                If she knows spells from a banned school, she must immediately learn different spells from unbanned schools in their place.
                Divination cannot be chosen as a banned school.
        \end{itemize}

        At her 4th wizard level, and every even wizard level thereafter, the wizard gains a new arcane insight.

        \cf{Wiz}[3rd]{Attuned Spell}[Ex]
        Whenever the wizard casts a spell or performs a ritual on herself, she may attune to it to increase its duration.
        The level of the spell or ritual, including any augments or metamagic, must not exceed half her wizard level.
        % Effects that reduce the level cost of augments or metamagic do not reduce the spell's total level for this purpose.
        The spell must be a \glossterm{targeted spell}, and have a duration of \durshort or longer.

        As long as the wizard is attuned to that spell, its duration becomes permanent.
        If it has secondary effects, such as the \spell{avatar of suffering} spell, the duration of the secondary effects is not changed.
        The wizard can only attune to one spell at a time.
        If she attunes to another spell, the other spell returns to its normal duration, and expires if appropriate.

        At her 7th wizard level, and every 4 wizard levels thereafter, the wizard gains the ability to attune to an additional spell simultaneously.
        Whenever she attunes to a new spell, she may choose which of her other attuned spells she stops being attuned to.

        \cf{Wiz}[5th]{Ritual Master}[Ex]
        When performing an arcane \glossterm{ritual}, the wizard may spend an arcane spell slot of the ritual's level or higher.
        If she does, she need only pay half the normal material component costs to perform the ritual.

        \cf{Wiz}[9th]{Persistent Attunement}[Ex]
        A wizard's attuned spells cannot be dispelled.

        \cf{Wiz}[13th]{Contingency}[Ex]
        The wizard gains the ability to prepare a spell so it takes effect automatically if specific circumstances arise.
        To prepare a contingency, the wizard must spend 1 minute preparing the spell, which consumes the a spell slot two levels higher than the spell's level.
        During this casting time, the wizard specifies what circumstances cause the spell to take effect.

        The contingency can be set to trigger in response to any circumstances that a typical human observing the wizard and her situation could detect.
        For example, a wizard could specify ``when I fall at least 50 feet'' or ``when I become bloodied'', but not ``when there is an invisible creature within 50 feet of me'' or ``when I am at 17 hit points or fewer.'' The more specific the required circumstances, the better -- vague requirements, such as ``when I am in danger'', may cause the contingency to trigger unexpectedly or fail to trigger at all.
        If a wizard attempts to specify multiple separate triggering conditions, such as ``when I take damage or when an enemy is adjacent to me'', the contingency will randomly ignore all but one of the conditions.

        The spell must have a casting time of 1 standard action or less.
        The contingency can specify a simple rule for identifying how to target the spell, such as ``the closest enemy''.
        If the rule is poorly worded or imprecise, the contingency may target incorrectly or fail to activate at all.
        Any spells which require decisions, such as \spell{dimension door}, must have those decisions made at the time the contingency is created.
        The wizard cannot change those decisions when the contingency takes effect.

        A wizard can have only one contingency active at a time.
        If she creates another contingency, it replaces her old contingency.

        \cf{Wiz}[17th]{Augmented Attunement}[Ex]
        When the wizard attunes to a spell, she may reduce the cost of one augment to the spell by one level.

        \cf{Wiz}[20th]{Chain Contingency}[Ex]
        The wizard may ready two spells in her contingency instead of a single spell.
        In addition, she may have two contingencies active at once instead of one.
        Only one contingency can trigger in a given round.
        If both would trigger, only the first contingency cast triggers, and the second does not.

\section{Multiclass Characters}\label{Multiclass Characters}
    A character may add new classes as he or she progresses in level, thus becoming a multiclass character.
    The class abilities from a character's different classes combine to determine a multiclass character's overall abilities.
    Multiclassing improves a character's versatility at the expense of focus.

    \subsection{Class And Level Features}
        As a general rule, the abilities of a multiclass character are the sum
        of the abilities of each of the character's classes.

        \parhead{Level}
        ``Character level'' is a character's total number of levels.
        It is used to determine when feats and attribute score boosts are gained, as noted on \tref{Character Advancement}.
        Whenever a creature's ``level'' is specified, without reference to a particular class, the character level is used.

        \par ``Class level'' is a character's level in a particular class.
        For a character whose levels are all in the same class, character level and class level are the same.

        \parhead{Hit Points}
        The normal rules for determining hit points apply to multiclass characters.

        \parhead{Combat Prowess}
        For each class, take the combat prowess given on the chart and subtract the combat prowess bonus granted to the class from its base class bonuses.
        Take the resulting combat prowess values for each class and add them together.
        Finally, add the base class bonus from the character's base class to that sum.
        The result is the character's total combat prowess.

        \parhead{Defenses}
        The normal rules for determining defenses apply to multiclass characters.

        \parhead{Skills}
        A multiclass character gains all class skills from all of his classes.
        However, only the character's base class grants skill points.
        If a character has multiple base classes, he must choose which base class grants skill points.
        When a character gains new class skills, he may redistribute his skill points to gain training or mastery in his new class skills.

        \parhead{Class Abilities}
        A multiclass character gets all the class abilities of all his or her classes, but must also suffer the consequences of the special restrictions of all his or her classes.

        \par In some cases, two classes can have virtually identical abilities.
        Use the following guidelines to determine how abilities stack.
        \begin{itemize}
            \item If two identical class abilities are not based on level and are not gained following a specific pattern, they do not stack.
            \item If two identical class abilities are not explicitly based on level, but both classes gain them in a predictable pattern, the levels of the two classes stack for determining when the next improvement to the class ability will be gained.
            \item If two identical class abilities are explicitly based on level, the levels of the two classes stack for determining the power of the ability.
            \item If two identical class abilities say how they stack, those rules trump any other rules.
        \end{itemize}
        These are some examples of how to use these guidelines.
        \begin{itemize}
            \item Both a druid and a ranger gain wild speech.
                A druid/ranger who has wild speech from both classes has the same wild speech ability as a druid or ranger would.
            \item Both a barbarian and a rogue get uncanny dodge and improved uncanny dodge at the same level.
                A barbarian/rogue adds his barbarian and rogue levels together to determine when he acquires improved uncanny dodge.
        \end{itemize}

        \parhead{Weapon and Armor Proficiency}
        Only a character's base class grants weapon and armor proficiencies.
        If a character has multiple base classes, she must choose which base class grants weapon and armor profiencies.

    \subsection{Spellcasters and Multiclassing}\label{Spellcasters and Multiclassing}
        The character gains spells from all of his or her spellcasting classes separately and tracks his spells per day, spells known, and spellpower separately with each class.

        A character with spellcasting abilities gains a special benefit when multiclassing.
        She must choose a specific spellcasting class he has.
        For every two levels that she in nonmagical classes, up to the number of levels she has in her chosen spellcasting class, she increases his spellcasting ability with that class.
        This increases her spells per day (if any) and spells known, as if she had gained a level in her chosen spellcasting class.
        No class abilities can be gained in this way.

        For example, Gish, a 2nd level fighter / 2th level wizard, would have the spells per day, spells known, and spellpower of a 3rd level wizard.
        If he gained two more fighter levels, his spellcasting ability would not increase.
