\chapter{Classes}\label{Classes}

Your character's class represents the things your character has chosen to train in.
This choice determines a great deal about your character's abilities.

\section{How Classes Work}
    When you first create a character, you choose a class.
    Your character has one level in that class.
    This grants your character all of the abilities your chosen class grants at 1st level, as given in the class description.
    Each time your character gains a level, you can choose to increase your level in your original class or gain a level in a new class.
    This grants your character all of the abilities your chosen class grants at the level your character just gained in it.

    \subsection{Archetypes}
        Each class has three \glossterm{archetypes}.
        Each archetype 

\section{Class Introductions}

    There are eleven classes in Rise.
    \begin{itemize}
        \item Barbarians are mighty warriors who can enter a deadly battlerage.
        \item Clerics are divine spellcasters who draw power from their veneration of a deity or ideal.
        \item Druids are nature spellcasters who draw power from their veneration of the natural world.
        \item Fighters are highly disciplined warriors who excel in physical combat of any variety.
        \item Mages are arcane spellcasters who wield the mystic forces of magic to create almost any effect.
        \item Monks are agile masters of ``\ki'' who hone their personal abilities to strike down foes and perform supernatural feats.
        \item Paladins are divinely empowered warriors whose devotion to an alignment grants them the ability to discern and smite their foes.
        \item Rangers are skilled hunters who bridge the divide between nature and civilization.
        \item Rogues are exceptionally skillful characters known for their ability to strike at their foe's weak points in combat.
        % \item Spellwarped wield a unique blend of martial skill and narrowly focused magical abilities.
    \end{itemize}

    \subsection{Class Description Format}
        Each class is described from the perspective of a member of that class, using ``you'' in the description.

        \parhead{Class Table}
        The class's table describes the special abilities they get at each level.

        \parhead{Alignment}
        Some classes require specific alignments (see \pcref{Alignment}).
        Most classes allow characters of any alignment.

        \parhead{Skill Points}
        This is the number of skill points that members of this class get.

        \parhead{Class Skills}
        These are skills that members of this class are typically good at (see \pcref{Skills}).

        \parhead{Defenses}
        Each class grants bonuses to specific defenses.

        \parhead{Weapon and Armor Proficiencies}
        These are the types of equipment that members of this class are trained in using.

        \parhead{Other Special Abilities}
        Some classes have abilities shared by all members of the class that are not part of an archetype, such as a cleric's \textit{divine power}.

        \parhead{Archetypes}
        The abilities associated with each of the three archetypes the class has.

\section{Barbarian}\label{Barbarian}
    \begin{dtable}
        \lcaption{Barbarian Progression}
        \begin{dtabularx}{\columnwidth}{>{\ccol}p{\levelcol} >{\lcol}X}
            \tb{Level} & \tb{Special} \\\bottomrule
            \nth{1}     & Athletic prowess, rage
            \\ \nth{2}  & Battle-scarred
            \\ \nth{3}  & Improvised combat
            \\ \nth{4}  & Uncontrollable rage
            \\ \nth{5}  & Uncanny dodge
            \\ \nth{6}  & Titanic might
            \\ \nth{7}  & Unstoppable rage
            \\ \nth{8}  & Deep scars
            \\ \nth{9}  & Athletic prowess
            \\ \nth{10} & Mindless rage
            \\ \nth{11} & Greater uncanny dodge
            \\ \nth{12} & Greater improvised combat
            \\ \nth{13} & Mighty rage
            \\ \nth{14} & Rapid recovery
            \\ \nth{15} & Athletic champion
            \\ \nth{16} & Endless rage
            \\ \nth{17} & Soulscarred
            \\ \nth{18} & Greater titanic might
            \\ \nth{19} & Blood frenzy
            \\ \nth{20} & Deathless rage, fury of the storm
        \end{dtabularx}
    \end{dtable}

    \classbasics{Alignment} Any nonlawful.

    \subsection{Class Abilities}
        If you are a barbarian, you gain the following abilities.

        \classbasics{Skill Points} 6.

        \classbasics{Class Skills}
        \begin{itemize}
            \item \subparhead{Strength} Climb, Jump, Sprint, Swim.
            \item \subparhead{Dexterity} Acrobatics, Ride.
            \item \subparhead{Perception} Awareness, Creature Handling, Survival.
            \item \subparhead{Other} Bluff, Intimidate, Persuasion.
        \end{itemize}

        \classbasics{Defenses} \plus3 Fortitude, \plus2 Reflex, \plus1 Mental.

        \classbasics{Weapon and Armor Proficiency}
        You are proficient with simple weapons, any four other weapon groups, light armor, medium armor, and shields.

        \subsubsection{Battlerager}\label{Rage}

            \cf{Bbn}{Rage} 
            As a \glossterm{free action}, you can spend an \glossterm{action point} to use this ability.
            \begin{ability}
                \begin{spelleffects}
                    \spelleffect You have the following benefits and drawbacks:
                    \begin{itemize}
                        \item You gain a \plus1d bonus to \glossterm{strike damage}.
                        \item You are unable to take any action that requires patience or concentration, such as casting spells.
                        \item At the end of each round if you did not attack a creature or object, you take \glossterm{nonlethal damage} equal to your level.
                            This damage ignores your damage reduction from this ability.
                    \end{itemize}
                    \spellspecial When this ability ends, you become \fatigued and unable to use it again until you take a \glossterm{short rest}.
                    \spelldur Attunement
                \end{spelleffects}
            \end{ability}

            \cf{Bbn}[4]{Uncontrollable Rage}
            You are immune to \glossterm{Compulsion} effects while raging.

            \cf{Bbn}[7]{Unstoppable Rage}
            You are immune to being \glossterm{staggered} while raging.

            \cf{Bbn}[10]{Senseless Rage}
            You are immune to \glossterm{Mind} \glossterm{conditions} while raging.

            \cf{Bbn}[13]{Mighty Rage}
            Your damage bonus from the \textit{rage} ability increases to \plus2d.

            \cf{Bbn}[16]{Mindless Rage} 
            You are immune to all hostile \glossterm{Mind} effects while raging.

            \cf{Bbn}[19]{Blood Frenzy} 
            You are immune to being \glossterm{bloodied} while raging.

            \cf{Bbn}[20]{Deathless Rage} 
            While raging, the you ignore all penalties from \glossterm{vital damage}.
            However, if your vital damage exceeds your maximum hit points, you immediately die.

        \subsubsection{Primal Warrior}
            \cf{Bbn}{Athletic Prowess} You gain two additional skill points that must be spent on Strength or Dexterity-based barbarian class skills.

            \cf{Bbn}[3]{Improvised Combat} You gain a \plus1 bonus to accuracy with \glossterm{combat maneuvers}.

            \cf{Bbn}[6]{Titanic Might}
            You gain a \plus1d bonus to \glossterm{strike damage}.

            \cf{Bbn}[9]{Athletic Prowess} You gain two additional skill points that must be spent on Strength or Dexterity-based barbarian class skills.

            \cf{Bbn}[12]{Greater Improvised Combat} 
            The accuracy bonus from your \textit{improvised combat} ability increases to \plus2.

            \cf{Bbn}[15]{Athletic Champion}
            You gain a \plus1 bonus to all Strength and Dexterity-based skills.

            \cf{Bbn}[18]{Greater Titanic Might}
            The damage bonus from your \textit{titanic might} ability increases to \plus2d.

        \subsubsection{Battleforged Resilience}
            \cf{Bbn}[2]{Battle-Scarred} You gain \glossterm{damage reduction} against physical damage equal to your level.

            \cf{Bbn}[5]{Uncanny Dodge} You can react to danger before your senses would normally allow you to do so.
            You reduce your \glossterm{overwhelm penalties} by 1.
            If your overwhelm penalty is reduced to 0, you are not considered to be overwhelmed.
            In addition, you are not \unaware when attacked by surprise.

            \cf{Bbn}[8]{Deep Scars} Your \glossterm{damage reduction} from your \textit{battle-scarred} ability applies against all damage, not just physical damage.

            \cf{Bbn}[11]{Greater Uncanny Dodge}
            Your reduction of \glossterm{overwhelm penalties} from the \textit{uncanny dodge} ability increases to 2.

            \cf{Bbn}[14]{Rapid Recovery}
            At the end of each round, you heal hit points equal to your level.

            \cf{Bbn}[17]{Soulscarred}
            Your \glossterm{damage reduction} from your \textit{battle-scarred} ability increases to twice your level.

            \cf{Bbn}[20]{Fury of the Storm}
            You gain a \plus1d bonus to \glossterm{strike damage} against all creatures who \glossterm{threaten} you.
            In addition, your reduction of overwhelm penalties from the \textit{uncanny dodge} ability increases to 4.

        \subsubsection{Ex-Barbarians}
            If you become lawful, you cannot use your \textit{rage} ability.
            You retain all of your other class abilities.
            If you stop being lawful, you can use your \textit{rage} ability once more.

\section{Cleric}\label{Ckleric}
    \begin{dtable}
        \lcaption{Cleric Progression}
        \begin{dtabularx}{\columnwidth}{>{\ccol}p{2em} c c >{\lcol}X}
            \tb{Level} & \tb{Spells} & \tb{Subspells} & \tb{Special} \\\bottomrule
            \nth{1}     & 2 & \tdash   & Domain gift, rituals, spells
            \\ \nth{2}  & 2 & \tdash   & Spell point
            \\ \nth{3}  & 3 & \tdash   & Domain gift, spell knowledge
            \\ \nth{4}  & 3 & 1        & \tdash
            \\ \nth{5}  & 3 & 1        & Domain aspect
            \\ \nth{6}  & 3 & 2        & \tdash
            \\ \nth{7}  & 3 & 2        & Domain aspect
            \\ \nth{8}  & 4 & 3        & \tdash
            \\ \nth{9}  & 4 & 3        & Cleansing prayer
            \\ \nth{10} & 4 & 4        & \tdash
            \\ \nth{11} & 4 & 4        & Domain essence
            \\ \nth{12} & 4 & 5        & \tdash
            \\ \nth{13} & 4 & 5        & Domain essence
            \\ \nth{14} & 4 & 6        & \tdash
            \\ \nth{15} & 4 & 6        & Domain mastery
            \\ \nth{16} & 4 & 7        & Spell point
            \\ \nth{17} & 4 & 7        & Domain mastery
            \\ \nth{18} & 4 & 8        & \tdash
            \\ \nth{19} & 4 & 8        & Greater cleansing prayer
            \\ \nth{20} & 4 & 9        & Miracle
        \end{dtabularx}
    \end{dtable}

    \classbasics{Alignment} Your alignment must be within one step of your deity's (that is, it may be one step away on either the lawful-chaotic axis or the good-evil axis, but not both).

    \subsection{Class Abilities}
        If you are a cleric, you gain the following abilities.

        \classbasics{Skill Points} 4.

        \classbasics{Class Skills}
        \subparhead{Intelligence} Heal, Knowledge (arcana, local, religion, the planes), Linguistics.
        \subparhead{Perception} Awareness, Sense Motive, Spellcraft.
        \subparhead{Other} Bluff, Intimidate, Persuasion.

        \classbasics{Defenses} \plus3 Mental, \plus2 Fortitude, \plus1 Reflex.

        \cf{Clr}{Weapon and Armor Proficiency}
        You are proficient with simple weapons, any two other weapon groups, light and medium armor, and shields.

        \cf{Clr}{Divine Power}
        The \glossterm{power} of many cleric spells and abilities is determined by your \textit{divine power}.
        Your \textit{divine power} is equal to your level or your Willpower, whichever is higher.

        \cf{Clr}{Deity}
        You must worship a specific deity to be a cleric.
        Deities and their associated \glossterm{domains} are listed in \trefnp{Deities}.

        \begin{dtable!*}
            \lcaption{Deities}
            \begin{dtabularx}{\textwidth}{X l X}
                \tb{Deity} & \tb{Alignment} & \tb{Domains} \\
                \bottomrule
                Guftas, horse god of justice          & Lawful good     & Good, Law, Strength, Travel         \\
                Lucied, paladin god of justice        & Lawful good     & Destruction, Good, Protection, War  \\
                Simor, fighter god of protection      & Lawful good     & Good, Protection, Strength, War     \\
                %Pabst, dwarf god of drink             & Neutral good & Good, Life, Strength, Wild \\
                Rucks, monk god of pragmatism         & Neutral good    & Good, Law, Protection, Travel       \\
                Vanya, centaur god of nature          & Neutral good    & Good, Strength, Travel, Wild        \\
                Brushtwig, pixie god of creativity    & Chaotic good    & Chaos, Good, Trickery, Wild         \\
                Chavi, god of stories                 & Chaotic good    & Chaos, Knowledge, Trickery          \\
                Ivan Ivanovitch, bear god of strength & Chaotic good    & Chaos, Strength, War, Wild          \\
                Krunch, barbarian god of destruction  & Chaotic good    & Destruction, Good, Strength, War    \\
                Sir Cakes, dwarf god of freedom       & Chaotic good    & Chaos, Good, Strength               \\
                Raphael, monk god of retribution      & Lawful neutral  & Death, Law, Protection, Travel      \\
                Declan, god of fire                   & True neutral    & Destruction, Fire, Knowledge, Magic \\
                Kurai, shaman god of nature           & True neutral    & Air, Earth, Fire, Water             \\
                %Amanita, druid god of decay           & Chaotic neutral & Chaos, Destruction, Life, Wild \\
                %Antimony, elf god of necromancy       & Chaotic neutral & Death, Knowledge, Life, Magic \\
                Clockwork, elf god of time            & Chaotic neutral & Chaos, Magic, Trickery, Travel      \\
                %Lord Khallus, fighter god of pride    & Chaotic neutral & Chaos, Strength, War \\
                %Celeano, sorcerer god of deception    & Chaotic neutral & Chaos, Magic, Protection, Trickery \\
                Murdoc, god of mercenaries            & Chaotic neutral & Destruction, Knowledge, Travel, War \\
                Ribo, halfling god of trickery        & Chaotic neutral & Chaos, Trickery, Water              \\
                Tak, orc god of war                   & Lawful evil     & Law, Strength, Trickery, War        \\
                Theodolus, sorcerer god of ambition   & Neutral evil    & Evil, Knowledge, Magic, Trickery    \\
                Daeghul, demon god of slaughter       & Chaotic evil    & Destruction, Evil, Magic, War       \\
            \end{dtabularx}
        \end{dtable!*}

        \subsubsection{Spellcasting}

            \cf{Clr}{Divine Spells} 
            Your deity grants you the ability to cast divine spells.
            You learn two divine spells from the divine \glossterm{spell list} (see \pcref{Divine Spells}).
            Your \glossterm{spellpower} with divine spells is equal to your \glossterm{divine power}.

            To cast a spell, you must normally spend an \glossterm{action point}.
            Every spell can also be cast as a cantrip.
            Cantrips are weaker, but do not require action points to cast.

            You can't cast spells of an alignment opposed to your own or your deity's.
            Spells associated with particular alignments are indicated by the \glossterm{Chaos}, \glossterm{Good}, \glossterm{Evil}, and \glossterm{Law} tags in their spell descriptions.

            \cf{Clr}{Rituals} 
            You can perform divine rituals to create unique magical effects (see \pcref{Rituals}).
            You have a ritual book containing two divine rituals of your choice (see \pcref{Divine Rituals}).

            \cf{Clr}[6]{Augment}
            Choose an \glossterm{augment} (see \pcref{Augments}).
            You can apply that augment to divine spells you cast.
            At 10th level, and every four levels thereafter, you learn an additional augment.

            \cf{Clr}[8]{Spell Knowledge}
            You learn an additional divine spell (see \pcref{Divine Spells}).

            \cf{Clr}[16]{Spell Point} 
            You gain a spell point.
            A spell point can be spent to cast spells in place of an action point.
            You recover all spent spell points after a \glossterm{short rest}.

        \subsubsection{Domain Influence}
            All \textit{domain influence} abilities are \glossterm{Magical}.

            \cf{Clr}{Domains}
            You choose two domains which represent your personal spiritual inclinations.
            You must choose your domains from among those your deity offers.
            The domains are listed below.

            \begin{itemize}
                \item{Air}
                \item{Chaos}
                \item{Death}
                \item{Destruction}
                \item{Earth}
                \item{Evil}
                \item{Fire}
                \item{Good}
                \item{Knowledge}
                \item{Law}
                \item{Life}
                \item{Magic}
                \item{Protection}
                \item{Strength}
                \item{Travel}
                \item{Trickery}
                \item{War}
                \item{Water}
                \item{Wild}
            \end{itemize}

            \cf{Clr}{Domain Gift}
            Each domain has a corresponding \textit{domain gift}.
            You gain the \textit{domain gift} for one of your domains (see \pcref{Domain Gifts}).

            \cf{Clr}[3]{Domain Gift}
            You gain the \textit{domain gift} for another one of your domains.

            \cf{Clr}[5]{Domain Aspect}
            Each domain has a corresponding \textit{domain aspect}.
            You gain the \textit{domain aspect} for one of your domains (see \pcref{Domain Gifts}).

            \cf{Clr}[7]{Domain Aspect} 
            You gain the \textit{domain aspect} for another one of your domains.

            \cf{Clr}[9]{Cleansing Prayer}
            When you use the \textit{recover} ability, you heal \plus1d hit points.
            In addition, instead of removing a condition, you can remove any \glossterm{sustained} effect on you.

            \cf{Clr}[11]{Domain Essence}
            Each domain has a corresponding \textit{domain essence}.
            You gain the \textit{domain essence} for one of your domains (see \pcref{Domain Gifts}).

            \cf{Clr}[13]{Domain Essence} 
            You gain the \textit{domain essence} for another one of your domains.

            \cf{Clr}[15]{Domain Mastery}
            Each domain has a corresponding \textit{domain mastery}.
            You gain the \textit{domain mastery} for one of your domains (see \pcref{Domain Gifts}).

            \cf{Clr}[17]{Domain Mastery} 
            You gain the \textit{domain mastery} for another one of your domains.

            \cf{Clr}[19]{Greater Cleansing Prayer} 
            The bonus to hit point recovery from the \textit{cleansing prayer} ability increases to \plus2d.
            In addition, when you use the \textit{recover} ability, you can remove any number of \glossterm{sustained} effects on you instead of removing a condition.

            \cf{Clr}[20]{Miracle}
            Once per week, you can request a miracle as a standard action.
            You mentally specify your request, and your deity fulfills that request in the manner it sees fit.
            This can emulate the effects of any spell or ritual, or have any other effect of a similar power level.
            If the deity has a direct interest in your situation, the miracle may be of even greater power.

            If you perform an extraordinary service for your deity, you can gain the ability to request an additional miracle that week.

        \subsubsection{Divine Spell Mastery}
            You must have the ability to cast divine spells to gain these abilities.

            \cf{Clr}[2]{Spell Point}
            You gain a spell point.
            A spell point can be spent to cast spells in place of an action point.
            You recover all spent spell points after a \glossterm{short rest}.

            \cf{Clr}[3]{Spell Knowledge}
            You learn an additional divine spell (see \pcref{Divine Spells}).

            \cf{Clr}[4]{Subspell}
            Choose a \glossterm{subspell} for a divine spell you know.
            You can use that subspell when you cast that spell (see \pcref{Subspells}).
            At 6th level, and every two levels thereafter, you learn an additional subspell for a divine spell you know.

    \subsection{Cleric Domain Abilities}
        All cleric domain abilities are \glossterm{magical} unless otherwise specified.

        \subsubsection{Air}
            \parhead{Gift} You add the Jump skill to your class skill list and gain a \plus5 bonus to Jump checks (see \pcref{Jump}).
            \parhead{Aspect} You gain a \glossterm{glide speed} equal to your land speed (see \pcref{Gliding}).
            \parhead{Essence} As a standard action, you can spend an \glossterm{action point} to use this ability.
            \begin{ability}
                \begin{spelltargetinginfo}
                    \spellrng{\rnglong}
                \end{spelltargetinginfo}
                \begin{spelleffects}
                    \spelleffect You can speak with and command air within range.
                    You can ask the air simple questions and understand its responses.
                    If you command the air to perform a task, it will do so do the best of its ability until this effect ends.
                    You cannot compel the air to move faster than 50 mph.
                    \spelldur Sustain (swift)
                    \spellspecial After you use this ability on a particular area of air, you cannot use it again on that same area for 24 hours.
                \end{spelleffects}
            \end{ability}
            \parhead{Mastery} You gain a \glossterm{fly speed} with \glossterm{good maneuverability} equal to your land speed (see \pcref{Flying}).

        \subsubsection{Chaos}
            \parhead{Gift} Whenever you roll a 10 on a check on your first attempt, you gain a \plus5 bonus to the check.
            \parhead{Aspect} If you roll a 1 on an attack roll, it explodes (see \pcref{Exploding Attacks}).
            This does not affect additional dice rolled if the attack roll explodes.
            \parhead{Essence} As a standard action, you can spend an \glossterm{action point} to use this ability.
            \begin{ability}
                \begin{spelltargetinginfo}
                    \spellrng{\rnglong}
                \end{spelltargetinginfo}
                \begin{spelleffects}
                    \spelleffect An improbable event occurs.
                    You can specify in general terms what you want to happen, such as ``Make the bartender leave the bar''.
                    You cannot control the exact nature of the event, though it always beneficial for you in some way.
                    \spellspecial After using this ability, you cannot use it again for an hour.
                \end{spelleffects}
            \end{ability}
            \parhead{Mastery} Whenever you make an attack roll, if it misses, you can reroll.
            You must accept the second result.

        \subsection{Death}
            \parhead{Gift} Whenever you deal damage to a creature with no hit points remaining, it immediately dies.
            This is a \glossterm{Death} effect.
            % Needs wording to clarify interaction with simultaneous damage
            \parhead{Aspect} Whenever you deal damage to a creature, any of your damage in excess of that creature's hit points is dealt as \glossterm{vital damage}.
            In addition, you are immune to \glossterm{Death} effects.
            \parhead{Essence} As a standard action, you can spend an \glossterm{action point} to use this ability.
            \begin{ability}
                \begin{spelltargetinginfo}
                    \spellquicktargeting{One living creature}{\rngmed}
                    \spellspecial When you use this ability, you choose whether to summon the essence of Death or banish it.
                \end{spelltargetinginfo}
                \begin{spelleffects}
                    \spelleffect If you summoned the essence of Death, the target draws closer to the brink.
                    If it takes \glossterm{vital damage}, it immediately dies.

                    If you banished the essence of Death, the target draws closer to life.
                    It is immune to \glossterm{Death} effects and gains a bonus to \glossterm{stabilization rolls} equal to your \textit{divine power}.
                    \spelldur Attunement
                    \spelltags{\glossterm{Death}}
                \end{spelleffects}
            \end{ability}
            \parhead{Mastery} You constantly radiate an emanation of death, described below.
            \begin{ability}
                \begin{spelltargetinginfo}
                    \spellarea{\areahuge radius emanation from you}
                    \spelltgts{All living enemies in the area}
                \end{spelltargetinginfo}
                \begin{spelleffects}
                    \spelleffect If the target takes \glossterm{vital damage}, it immediately dies.
                \end{spelleffects}
            \end{ability}

        \subsubsection{Destruction}
            \parhead{Gift} You gain a \plus1d bonus to \glossterm{strike damage}.
            \parhead{Aspect} Your attacks ignore an amount of \glossterm{hardness} and \glossterm{damage reduction} equal to your \textit{divine power}.
            \parhead{Essence} As a standard action, you can spend an \glossterm{action point} to use this ability.
            \begin{ability}
                \begin{spelltargetinginfo}
                    \spellarea{\arealarge radius burst}
                    \spelltgts{All unattended objects in the area}
                    \spellspecial You may freely exclude any number of 5-ft\. cubes from the area, as long as the resulting area is still contiguous.
                \end{spelltargetinginfo}
                \begin{spelleffects}
                    \begin{spellattack}{Divine power vs. Fortitude}
                        \spellspecial Nonmagical objects do not have a Fortitude defense, so this attack automatically succeeds against such objects.
                        \spellsuccess The target crumbles into a fine powder and is irreparably \glossterm{broken}.
                    \end{spellattack}
                \end{spelleffects}
            \end{ability}
            \parhead{Mastery} Whenever you deal damage to a creature or object, its damage reduction and hardness (if any) are reduced by an amount equal to your \textit{divine power}.
            In addition, Fortitude defense is reduced by 2.
            This is a \glossterm{condition}, and lasts until it is removed.
            This effect stacks with itself, but can only be applied to a target once per round.
            % how do you remove a condition from an object???

        \subsubsection{Earth}
            \parhead{Gift} You gain a \plus2 bonus to Fortitude defense.
            \parhead{Aspect} You gain \glossterm{damage reduction} against physical damage equal to your \textit{divine power}.
            \parhead{Essence} As a standard action, you can spend an \glossterm{action point} to use this ability.
            \begin{ability}
                \begin{spelltargetinginfo}
                    \spellrng{\rnglong}
                \end{spelltargetinginfo}
                \begin{spelleffects}
                    \spelleffect You can speak with and command earth within range.
                    You can ask the earth simple questions and understand its responses.
                    If you command the earth to perform a task, it will do so do the best of its ability until this effect ends.
                    You cannot compel the earth to move faster than 10 feet per round.
                    \spelldur Sustain (swift)
                    \spellspecial After you use this ability on a particular area of earth, you cannot use it again on that same area for 24 hours.
                \end{spelleffects}
            \end{ability}
            \parhead{Mastery} You gain the \glossterm{earth glide} ability, as an earth elemental.
            This allows you to glide through stone, dirt, or almost any other sort of earth as if it were air.
            You can walk or climb at any angle in the earth.
            However, you cannot breathe, speak, or hear while gliding in this way.
            While gliding, you can remain partially within the earth, granting you cover.

        \subsubsection{Evil}
            \parhead{Gift} Whenever you take damage, you may choose an adjacent willing creature.
            If you do, that creature takes half of that damage (rounded down) instead of you.
            Any abilities it has that would make the attack miss or fail have no effect, but its abilities that allow it to reduce or ignore the attack's effects work normally.
            You take the remaining half of the damage, and suffer any non-damaging effects of the attack normally.
            \par You may learn which attacks hit you in the current phase before deciding which attacks to redirect damage from, but not how much damage they would do (or any other effects they might have).
            \parhead{Aspect} You can use this domain's domain gift to redirect damage to any willing creature within \rngclose range.
            \parhead{Essence} As a standard action, you can spend an \glossterm{action point} to use this ability.
            \begin{ability}
                \begin{spelltargetinginfo}
                    \spellquicktargeting{One creature}{\rngmed}
                \end{spelltargetinginfo}
                \begin{spelleffects}
                    \begin{spellattack}{Divine power vs. Mental}
                        \spellspecial Creatures who have strict codes prohibiting them from taking evil actions, such as paladins devoted to Good, are immune to this attack.
                        \spellsuccess The target takes an evil action as soon as it can.
                        You have no control over the act the creature takes, but circumstances can make the target more likely to take an action you desire.
                    \end{spellattack}
                    \spelltags{\glossterm{Compulsion}, \glossterm{Mind}}
                \end{spelleffects}
            \end{ability}
            \parhead{Mastery} Whenever you take damage, you may use this ability as an \glossterm{immediate action}.
            If you do, you can use this domain's domain gift to redirect damage to an unwilling creature within \rngmed range.

        \subsubsection{Fire}
            \parhead{Gift} All of your \glossterm{Fire} spells and abilities do not deal damage to your allies.
            \parhead{Aspect} Whenever you would take fire damage, you heal that many hit points instead.
            This applies damage reduction, damage immunity, and similar effects.
            \parhead{Essence} As a standard action, you can spend an \glossterm{action point} to use this ability.
            \begin{ability}
                \begin{spelltargetinginfo}
                    \spellrng{\rnglong}
                \end{spelltargetinginfo}
                \begin{spelleffects}
                    \spelleffect You can speak with and command fire within range.
                    You can ask the fire simple questions and understand its responses.
                    If you command the fire to perform a task, it will do so do the best of its ability until this effect ends.
                    You cannot compel the fire to move farther than 30 feet in a single round.
                    Fire that ends the round on non-combustable materials usually goes out, depending on the circumstances.
                    \spelldur Sustain (swift)
                    \spellspecial After you use this ability on a particular area of fire, you cannot use it again on that same area for 24 hours.
                    % TODO: What does an ``area of fire'' mean?
                \end{spelleffects}
            \end{ability}
            \parhead{Mastery} Whenever you deal fire damage to a creature, that creature becomes \ignited.
            This is a \glossterm{condition}, and lasts until removed.

        \subsubsection{Good}
            \parhead{Gift} Whenever an adjacent creature takes damage, you may use this ability.
            If you do, you take half of that damage (rounded down) instead of the creature.
            Any abilities you have that would make the attack miss or fail have no effect, but your abilities that allow you to reduce or ignore its effects work normally.
            The protected creature takes the remaining half of the damage, and suffers any non-damaging effects of the attack normally.
            \par You may learn which attacks hit you and your allies in the current phase before deciding which attacks to redirect damage from, but not how much damage they would do (or any other effects they might have).
            \parhead{Aspect} You can use this domain's domain gift to redirect damage from any creature within \rngclose range.
            \parhead{Essence} As a standard action, you can spend an \glossterm{action point} to use this ability.
            \begin{ability}
                \begin{spelltargetinginfo}
                    \spellquicktargeting{One creature}{\rngmed}
                \end{spelltargetinginfo}
                \begin{spelleffects}
                    \begin{spellattack}{Divine power vs. Mental}
                        \spellspecial Creatures who have strict codes prohibiting them from taking good actions, such as paladins devoted to Evil, are immune to this attack.
                        \spellsuccess The target takes an good action as soon as it can.
                        You have no control over the act the creature takes, but circumstances can make the target more likely to take an action you desire.
                    \end{spellattack}
                    \spelltags{\glossterm{Compulsion}, \glossterm{Mind}}
                \end{spelleffects}
            \end{ability}
            \parhead{Mastery} Whenever you redirect damage with this domain's domain gift, you can redirect all effects of the attack to you instead of only half the damage.

        \subsubsection{Knowledge}
            \parhead{Gift} You add all Knowledge skills to your cleric class skill list.
            In addition, you gain two skill points which must be spent on Knowledge skills.
            \parhead{Aspect} Your extensive knowledge of all methods of attack and defense grants you a \plus1 bonus to all defenses.
            \parhead{Essence} As a standard action, you can spend an \glossterm{action point} to use this ability.
            \begin{ability}
                \begin{spelltargetinginfo}
                    \spellarea{\arealarge radius burst}
                    \spelltgts{All willing creatures in the area}
                \end{spelltargetinginfo}
                \begin{spelleffects}
                    \spelleffect You make a Knowledge check of any kind.
                    You gain a bonus to the Knowledge check equal to your \textit{divine power}.
                    The target also learns the results of your check.
                    It believes the information gained in this way to be true as if it had seen it with its own eyes.
                    \par You cannot alter the knowledge you gain with this check in any way, such as by adding or withholding information.
                \end{spelleffects}
            \end{ability}
            \parhead{Mastery} You gain a \plus1 bonus to accuracy with all attacks.

        \subsubsection{Law}
            \parhead{Gift} You gain a \plus2 bonus to Mental defense.
            % Clarify - does this apply to exploding dice?
            \parhead{Aspect} Whenever you roll a 1 on an \glossterm{attack roll}, it is treated as if you had rolled a 6.
            \parhead{Essence} As a standard action, you can spend an \glossterm{action point} to use this ability.
            \begin{ability}
                \begin{spelltargetinginfo}
                    \spellarea{\arealarge radius burst}
                    \spelltgts{All creatures in the area}
                \end{spelltargetinginfo}
                \begin{spelleffects}
                    \begin{spellattack}{Divine power vs. Mental}
                        \spellspecial This attack automatically succeeds against you.
                        \spellsuccess The target is unable to break the laws that apply in the area, and any attempt to do so simply fails.
                        The laws which are applied are those which are most appropriate for the area, regardless of whether the cleric or any other creature know those laws.
                        In areas under ambiguous or nonexistent government, this ability may have unexpected effects, or it may have no effect at all.
                    \end{spellattack}
                    \spelldur Condition
                    \spelltags{\glossterm{Compulsion}, \glossterm{Mind}}
                \end{spelleffects}
            \end{ability}
            \parhead{Mastery} Whenever you roll less than a 5 on an \glossterm{attack roll}, it is treated as if you had rolled a 5.

        \subsubsection{Life}
            \parhead{Gift} You gain additional hit points equal to your \glossterm{divine power}.
            \parhead{Aspect} All of your healing spells and abilities can cure \glossterm{vital damage} as easily as they cure hit points.
            \parhead{Essence} As a standard action, you can spend all of your remaining \glossterm{action points} (minimum 1) to use this ability.
            \begin{ability}
                \begin{spelltargetinginfo}
                    \spellquicktargeting{One dead creature}{Adjacent}
                \end{spelltargetinginfo}
                \begin{spelleffects}
                    \spelleffect If the target was dead for no more than 5 minutes, it is restored to life, as the \spell{resurrection} ritual.
                \end{spelleffects}
            \end{ability}
            % Need special text to clarify that it is affected by the Aspect?
            \parhead{Mastery} At the end of each round, you heal hit points equal to your \glossterm{divine power}.

        \subsubsection{Magic}
            \parhead{Gift} You gain a \plus1 bonus to spellpower with divine spells.
            \parhead{Aspect} The maximum spell level you can cast is increased by 1.
            \parhead{Essence} You gain \glossterm{magic resistance} equal to 5 \add your \textit{divine power}.
            If you already have \glossterm{magic resistance}, you can instead increase it by 2.
            \parhead{Mastery} You gain a \plus2 bonus to your \glossterm{magic resistance}.
            If you resist an ability with your magic resistance, you heal hit points equal to your \textit{divine power}.

        \subsubsection{Strength}
            \parhead{Gift} You add Climb, Jump, Sprint, and Swim to your cleric class skill list.
            In addition, you gain two skill points which must be spent on any combination of those skills.
            \parhead{Aspect} You may use your Strength to determine your \textit{divine power} in place of your Willpower.
            \parhead{Essence} As a standard action, you can spend an \glossterm{action point} to use this ability.
            \begin{ability}
                \begin{spelleffects}
                    \spelleffect You gain a \plus2d bonus to \glossterm{strike damage}.
                    In addition, you gain a \plus5 bonus to Strength for the purpose of checks and determining your carrying capacity.
                    \spelldur{Sustain (swift)}
                \end{spelleffects}
            \end{ability}
            \parhead{Mastery} You gain a \plus2 bonus to Strength.

        \subsubsection{Travel}
            \parhead{Gift} You add Knowledge (geography), Sprint, and Survival to your cleric class skill list.
            In addition, you gain two skill points which must be spent on any combination of those skills.
            \parhead{Aspect} You gain a \plus30 foot bonus to your speed in all movement modes, up to a maximum of double your normal speed.
            \parhead{Essence} As a standard action, you can spend an \glossterm{action point} to use this ability.
            \begin{ability}
                \begin{spelleffects}
                    \spelleffect You teleport yourself up to 1 mile in any direction.
                    You do not need \glossterm{line of sight} or \glossterm{line of effect} to your destination, but you must be able to clearly visualize it.
                \end{spelleffects}
            \end{ability}
            \parhead{Mastery} Whenever you move, you can teleport the same distance instead.
            This does not change the total distance you can move, but you can teleport in any direction, even vertically.

        \subsubsection{Trickery}
            \parhead{Gift} You add Bluff, Disguise, and Stealth to your cleric class skill list.
            In addition, you gain two skill points which must be spent on any combination of those skills.
            \parhead{Aspect} You gain a \plus2 bonus to Bluff, Disguise, and Stealth.
            \parhead{Essence} As a standard action, you can spend an \glossterm{action point} to use this ability.
            \begin{ability}
                \begin{spelltargetinginfo}
                    \spellquicktargeting{One creature}{\rngmed}
                    \spellspecial Choose a belief the target has.
                    The belief may be a lie that you told it, or even a simple misunderstanding (such as believing a hidden creature is not present in a room).
                    If the creature does not already hold the chosen belief, this ability automatically fails.
                \end{spelltargetinginfo}
                \begin{spelleffects}
                    \begin{spellattack}{Divine power vs. Mental}
                        \spellsuccess The target continues to maintain the chosen belief, regardless of any evidence to the contrary.
                        It will interpret any evidence that the falsehood is incorrect to be somehow wrong -- an illusion, a conspiracy to decieve it, or any other reason it can think of to continue believing the falsehood.
                        At the end of the effect, the creature can decide whether it believes the falsehood or not, as normal.
                    \end{spellattack}
                    \spelldur Sustain (Swift)
                    \spelltags{\glossterm{Delusion}, \glossterm{Mind}}
                \end{spelleffects}
            \end{ability}
            \parhead{Mastery} You are undetectable by Divination spells and effects.
            They cannot detect your presence, sounds you make, or any actions you take.

        \subsubsection{War}
            \parhead{Gift} You gain proficiency with heavy armor, tower shields, and an additional weapon group of your choice.
            \parhead{Aspect} You gain a \plus1d bonus to \glossterm{strike damage}.
            \parhead{Essence} Whenever you cast a spell, you can use this ability as an \glossterm{immediate action} by spending an \glossterm{action point}.
            If you do, the spell gains one of the following effects:
            \begin{itemize}
                \item Legion: If the spell would normally affect five or more specific targets, its range is doubled and it instead affects five times that many targets.
                \item Selective: If the spell has an area, it has no effect on your allies in the area.
                \item Widened: If the spell has an area, the size of the area is doubled.
            \end{itemize}
            \parhead{Mastery} You and all allies within a \arealarge radius emanation of you gain a \plus1d bonus to \glossterm{strike damage}.

        \subsubsection{Water}
            \parhead{Gift} You add Swim to your cleric class skill list and gain a \plus5 bonus to Swim checks.
            \parhead{Aspect} You can breathe water as easily as a human breathes air, preventing you from drowning or suffocating underwater.
            You also gain a \glossterm{swim speed} equal to your land speed.
            \parhead{Essence} As a standard action, you can spend an \glossterm{action point} to use this ability.
            \begin{ability}
                \begin{spelltargetinginfo}
                    \spellrng{\rnglong}
                \end{spelltargetinginfo}
                \begin{spelleffects}
                    \spelleffect You can speak with and command water within range.
                    You can ask the water simple questions and understand its responses.
                    If you command the water to perform a task, it will do so do the best of its ability until this effect ends.
                    You cannot compel the water to move faster than 30 feet per round.
                    \spelldur Sustain (swift)
                    \spellspecial After you use this ability on a particular area of water, you cannot use it again on that same area for 24 hours.
                \end{spelleffects}
            \end{ability}
            \parhead{Mastery}
            Whenever you move, you can transform yourself into a rushing flow of water with a volume roughly equal to your normal volume until your movement is complete
            In this form, you may move wherever water could go, you cannot take other actions, such as jumping, attacking, or casting spells.
            You may move through squares occupied by creatures or threatened by blocking enemies without penalty.
            \par Your speed is halved when moving uphill and doubled when moving downhill.
            Unusually steep inclines may cause greater movement differences while in this form.
            \par If the water is split, you may reform from anywhere the water has reached, to as little as a single ounce of water.
            If not even an ounce of water exists contiguously, your body reforms from all of the largest available sections of water, cut into pieces of appropriate size.
            This usually causes you to die.

        \subsubsection{Wild}
            \parhead{Gift} You add Creature Handling, Knowledge (nature), and Survival to your cleric class skill list.
            In addition, you gain two skill points which must be spent on any combination of those skills.
            % Does this even make sense?
            \parhead{Aspect} When you gain this ability, you choose one wild aspect ability, as if you were a druid of a level equal to your cleric level (see Wild Aspect, \pref{Drd:Wild Aspect}).
            As a standard action, you can spend an \glossterm{action point} to embody that wild aspect for 1 hour.

        \subsubsection{Ex-Clerics}
            If you grossly violate the code of conduct required by your deity, you lose all spells and magical cleric class abilities.
            You cannot regain those abilities until you atone (see the \spell{atonement} ritual).

\section{Druid}\label{Druid}
    \begin{dtable}
        \lcaption{Druid Progression}
        \begin{dtabularx}{\columnwidth}{>{\ccol}p{\levelcol} >{\ccol}p{3.5em} c >{\lcol}X}
            \tb{Level} & \tb{Spells} & \tb{Subspells} & \tb{Special} \\\bottomrule
            \nth{1}     & 2 & \tdash   & Rituals, spells, wild speech
            \\ \nth{2}  & 2 & \tdash   & Natural lore, spell point
            \\ \nth{3}  & 2 & \tdash   & Spell knowledge, wild aspect
            \\ \nth{4}  & 2 & 1        & \tdash
            \\ \nth{5}  & 3 & 1        & Natural vigor
            \\ \nth{6}  & 3 & 2        & \tdash
            \\ \nth{7}  & 3 & 2        & Wild aspect
            \\ \nth{8}  & 4 & 3        & \tdash
            \\ \nth{9}  & 4 & 3        & Natural lore
            \\ \nth{10} & 4 & 4        & \tdash
            \\ \nth{11} & 4 & 4        & Wild aspect
            \\ \nth{12} & 4 & 5        & \tdash
            \\ \nth{13} & 4 & 5        & Natural vigor
            \\ \nth{14} & 4 & 6        & \tdash
            \\ \nth{15} & 4 & 6        & Wild aspect
            \\ \nth{16} & 4 & 7        & Spell point
            \\ \nth{17} & 4 & 7        & Nature's champion
            \\ \nth{18} & 4 & 8        & \tdash
            \\ \nth{19} & 4 & 8        & Wild aspect
            \\ \nth{20} & 4 & 9        & Avatar of nature
        \end{dtabularx}
    \end{dtable}

    \classbasics{Alignment} Neutral good, lawful neutral, neutral, chaotic neutral, or neutral evil.

    \subsection{Class Abilities}
        If you are a druid, you gain the following abilities.

        \classbasics{Skill Points} 6.

        \classbasics{Class Skills}
        \subparhead{Strength} Climb, Jump, Sprint, Swim.
        \subparhead{Dexterity} Acrobatics, Ride, Stealth.
        \subparhead{Intelligence} Heal, Knowledge (geography, nature).
        \subparhead{Perception} Awareness, Creature Handling, Survival.
        \subparhead{Other} Bluff, Intimidate, Persuasion.

        \classbasics{Defenses} \plus3 Fortitude, \plus2 Mental, \plus1 Reflex.

        \cf{Drd}{Weapon and Armor Proficiency}
        Druids are proficient with simple weapons, any one other weapon group, scimitars, sickles, and slings.
        In addition, druids are proficient with light armor, medium armor, and shields.
        However, a druid cannot use metal armor; see the Metal Abhorrence ability, below.

        \cf{Drd}{Druidic Language}
        You know Druidic, a secret language known only to druids, in addition to your normal languages.
        Druids are forbidden to teach this language to nondruids.
        Druidic has its own alphabet.

        \cf{Drd}{Metal Abhorrence}
        The oaths that you swear as part of your druidic initiation prohibit you from wearing armor made of metal.
        If you wear prohibited armor or carry a prohibited shield, you are unable to cast druid spells or use any of your \glossterm{magical} druid abilities while doing so and for 24 hours thereafter.
        
        You can avoid this penalty by using armor made of wood altered with the \spell{ironwood} ritual.
        Such wood is as strong as steel.

        \cf{Drd}{Nature Power}
        The \glossterm{power} of many druid spells and abilities is determined by your \textit{nature power}.
        Your \textit{nature power} is equal to your level or your Perception, whichever is higher.

        \subsubsection{Spellcasting}

            \cf{Drd}{Nature Spells} 
            Your worship of nature grants you the ability to cast nature spells.
            You learn two nature spells from the nature \glossterm{spell list} (see \pcref{Nature Spells}).
            Your \glossterm{spellpower} with nature spells is equal to your \glossterm{nature power}.

            To cast a spell, you must normally spend an \glossterm{action point}.
            Every spell can also be cast as a cantrip.
            Cantrips are weaker, but do not require action points to cast.

            You can't cast spells of an alignment opposed to your own or your deity's.
            Spells associated with particular alignments are indicated by the \glossterm{Chaos}, \glossterm{Good}, \glossterm{Evil}, and \glossterm{Law} tags in their spell descriptions.

            \cf{Drd}{Rituals} 
            You can perform nature rituals to create unique magical effects (see \pcref{Rituals}).
            You have a ritual book containing two nature rituals of your choice (see \pcref{Nature Rituals}).

            \cf{Drd}[6]{Augment}
            Choose an \glossterm{augment} (see \pcref{Augments}).
            You can apply that augment to nature spells you cast.
            At 10th level, and every four levels thereafter, you learn an additional augment.

            \cf{Drd}[8]{Spell Knowledge}
            You learn an additional nature spell (see \pcref{Nature Spells}).

            \cf{Drd}[16]{Spell Point} 
            You gain a spell point.
            A spell point can be spent to cast spells in place of an action point.
            You recover all spent spell points after a \glossterm{short rest}.

        \subsubsection{Natural Influence}

            \cf{Drd}{Wild Speech}[Magical] As a standard action, you can spend an \glossterm{action point} to use this ability.
            \begin{ability}
                \begin{spelltargetinginfo}
                    \spellquicktargeting{One animal}{\rngmed}
                \end{spelltargetinginfo}
                \begin{spelleffects}
                    \spelleffect You can speak to and understand the speech of the target animal, and any other animals of the same species.
                    This ability doesn't make the target any more friendly or cooperative than normal.
                    Wary and cunning animals are likely to be terse and evasive, while stupid ones tend to make inane comments and are unlikely to say or understand anything of use.
                    \spelldur Sustain (swift)
                \end{spelleffects}
            \end{ability}

            \cf{Drd}[2]{Natural Lore}
            You gain two extra skill points which must be spent on the Creature Handling, Heal, Knowledge (geography), Knowledge (nature), Ride, or Survival skills.

            \cf{Drd}[3]{Wild Aspect}[Magical]
            You gain the ability to embody an aspect of an animal or of nature itself.
            Choose a single wild aspect from the list below.
            Many wild aspects have a minimum level prerequisite, as indicated in the title of the ability.
            That ability is normally active.
            You may suppress or resume the effects of any number of \textit{wild aspects} you have as a \glossterm{swift action}.

            The abilities in the list below describe the effects of the aspect.
            Your appearance also changes to match the aspect's effects, but the nature of this change is not described.
            Different druids change in different ways.
            For example, one druid might gain unusually large eyes when embodying the low-light vision aspect, while another might change their irises into slits, like a cat, when embodying the same aspect.
            You choose how your appearance changes when you gain a wild aspect.
            This change cannot be used to gain an additional substantive benefit beyond the effects given in the description of the aspect.

            Many wild aspects grant natural weapons.
            See \pcref{Natural Weapons}, for details about natural weapons.
            At 7th level, and every four levels thereafter, you gain an additional wild aspect.

            \subcf{Animal Affinity}
            You gain a \plus2 bonus to Creature Handling and Ride checks.
            \subcf{Armaments of the Bear}
            Your mouth and hands transform, allowing you to perform bite and claw attacks.
            The bite attack deals 1d8 damage for a Medium creature, and the claws deal 1d6 damage.
            \subcf{Gore}
            Your head transforms, allowing you to perform a gore attack.
            The attack deals 1d8 damage for a Medium druid.
            In addition, you gain a \plus2 bonus to accuracy with shove attacks (see \pcref{Shove}).
            \subcf{Monkey Climb}
            You gain a \glossterm{climb speed} equal to your land speed.
            \subcf{Senses}
            You gain low-light vision.
            You treat sources of light as if they had double their normal illumination range.
            If you already have low-light vision, you double its benefit, allowing you to treat sources of light as if they had four times their normal illumination range.
            In addition, you gain \glossterm{darkvision} out to 50 feet, allowing you to see in complete darkness.
            If you already have darkvision, you increase its range by 50 feet.
            \subcf{Woodland Stride}
            You may move through any sort of undergrowth (such as natural thorns, briars, overgrown areas, and similar terrain) at your normal speed and without taking damage or suffering any other impairment.
            The plants bend of their own volition to allow the druid to pass.
            However, plants magically manipulated to impede motion still affect your.

            \subcf{7th -- A Thousand Faces}
            You may use your spellpower in place of your Disguise skill when making Disguise checks to alter your own appearance.
            \subcf{7th -- Constrict}
            Your body transforms, improving your grappling abilities.
            You gain a \plus2 bonus to accuracy with grapple attacks (see \pcref{Grapple}).
            In addition, you gain a constrict attack.
            This attack deals 1d10 damage for a Medium druid, but it can only be used against a foe you aregrappling with.
            \subcf{7th -- Hawk}
            You grow wings, granting your a glide speed equal to your land speed.
            See \pcref{Gliding}, for more details.
            In addition, your feet transform, allowing you to perform a talon attack.
            The attack deals 1d6 damage for a Medium creature.
            \subcf{7th -- Lope}
            You gain the ability to move on all four limbs.
            When doing so, you gain a \plus30 foot bonus to your land speed, up to a maximum of double your original speed.
            When not using your hands to move, your ability to use your hands is unchanged.
            Descending to four legs and rising up to stand on two legs again does not take an action.
            \subcf{7th -- Scent}
            You gain the \glossterm{scent} ability.
            \subcf{7th -- Shrink}
            You shrink by one size category (see \pcref{Size in Combat}).
            This is a \glossterm{Sizing} effect.
            \subcf{7th -- Slither}
            You gain a \glossterm{climb speed} equal to your land speed.
            You do not need to use your hands to climb in this way.
            In addition, you gain a bite attack that deals 1d8 damage for a Medium druid.
            % \subcf{7th -- Spikes}
            % Whenever a creature adjacent to the druid makes a physical attack against you, the attacking creature takes 1d4 piercing damage \plus1d per two nature power.
            % A creature can only be dealt damage by this effect once per round.

            % Is this how poison actually works?

            % \subcf{11th -- Elemental Retribution}
            % Whenever a creature within \rngmed range of you attacks you, the attacking creature takes 1d6 damage per two nature power of either cold, electricity, or fire damage.
            % You may choose the damage type independently for each attacking creature.
            % A creature can only be dealt damage by this effect once per round.
            \subcf{11th -- Beetle's Carapace} You gain a \plus1 bonus to Armor defense.
            \subcf{11th -- Fluid Motion} You are immune to effects that restrict your mobility, and you suffer no penalties for acting underwater.
            In addition, you gain a \plus10 bonus to Reflex defense against grapple attacks, as well as on grapple attacks or Escape Artist checks made to escape a grapple or a pin.
            \subcf{11th -- Grow}
            You increase in size by one size category (see \pcref{Size in Combat}).
            This is a \glossterm{Sizing} effect.
            \subcf{11th -- Natural Grab}
            If you hit with a natural attack, you may attempt to grapple your foe as an immediate action.
            \subcf{11th -- Natural Trip}
            If you hit with a natural attack, you may attempt to trip your foe as an immediate action.
            \subcf{11th -- Wolfpack}
            Overwhelmed foes you threaten increase their \glossterm{overwhelm penalties} by 1.
            \subcf{11th -- Venom}
            If you hit with a natural attack, you may inject poison into your foe as an immediate action.
            This is a \glossterm{condition}, and lasts until removed.
            At the end of each round, you make a \textit{nature power} vs. Fortitude attack against all creatures you have poisoned.
            The effects of the poison are described below.
            \begin{itemize}
                \item First success: the target is \sickened until the effect ends.
                \item Second success: the target is \staggered until the end of the next round.
                \item Third success: the target is \nauseated until the effect ends.
                \item Third failure: the target is no longer poisoned, and any lingering effects from the poison end.
            \end{itemize}
            % TODO poison stacking rules
            \par In addition, you gains a bite attack that deals 1d8 damage for a Medium druid.

            \subcf{15th -- Earth Glide}
            You gain the earth glide ability, as an earth elemental.
            This allows you to glide through stone, dirt, or almost any other sort of earth as if it were air.
            You can walk or climb at any angle in the earth.
            However, you cannot breathe, speak, or hear while gliding in this way.
            While gliding, you can remain partially within the earth, granting you cover.
            \subcf{15th -- Natural Renewal}
            At the end of each round, you heal hit points equal to your \textit{nature power}.
            \subcf{15th -- Wings}
            You grow wings, granting you a \glossterm{fly speed} equal to your land speed.
            While \unencumbered, you can fly (see \pcref{Flying}).
            % TODO: standard flight limitation

            \subcf{19th -- Solar Radiance}
            The druid continuously radiates bright light out to a 500 foot radius (and shadowy illumination for an additional 500 feet).
            The illumination is so bright that you become hard to look at.
            Any creature attacking your from within the radius of bright light becomes \partiallyblinded for 2 rounds after the attack.

            \cf{Drd}[5]{Natural Vigor}
            At the end of each round, you heal hit points equal to half your \textit{nature power}.

            \cf{Drd}[9]{Natural Lore}
            You gain two extra skill points which must be spent on the Creature Handling, Heal, Knowledge (geography), Knowledge (nature), Ride, or Survival skills.

            \cf{Drd}[13]{Greater Natural Vigor}
            Your healing from the \textit{natural vigor} ability increases to be equal to your \textit{nature power}.

            \cf{Drd}[17]{Nature's Champion}
            You gain a \plus2 bonus to the Creature Handling, Heal, Knowledge (geography), Knowledge (nature), Ride, and Survival skills.

            \cf{Drd}[20]{Avatar of Nature}[Magical]
            If you die, except if by old age, you may choose to have your body and soul become an instrument of nature's will.
            Your body immediately decomposes or otherwise disappears, and your soul does not travel to an afterlife.
            You has no physical form, and cannot use any of your normal abilities.
            Instead, you have a fly speed of 100 feet, with special maneuverability.
            As a standard action, you can temporarily possess any living plants or animals within a 10 mile radius of the place of your death.

            While possessing a living plant or animal, you can see through its senses and control its actions completely.
            In addition, you may cast spells, and the spells take effect as if the plant or animal had cast them.
            You use the plant or animal's position to determine range, visible targets, and so on.
            You do not require verbal or somatic components to cast your spells in this form, but are unable to cast spells or perform rituals that require material components or focus objects.

            While not possessing a plant or animal, you can rest, or you can focus on reincarnating your physical form.
            Creating a new body in this way takes 12 consecutive hours of concentration.
            At the end of that time, you are reincarnated in a new body in your location, as the effect of the \spell{reincarnate} ritual, except that you can choose your race from among the races listed (not including the ``Other'' race).

            While you are an avatar of nature, you do not age and you cannot die of old age.
            You can continue to exist in this form indefinitely.

        \subsubsection{Nature Spell Mastery}
            You must have the ability to cast nature spells to gain these abilities.

            \cf{Drd}[2]{Spell Point}
            You gain a spell point.
            A spell point can be spent to cast spells in place of an action point.
            You recover all spent spell points after a \glossterm{short rest}.

            \cf{Drd}[3]{Spell Knowledge} 
            You learn an additional nature spell (see \pcref{Nature Spells}).

            \cf{Drd}[4]{Subspell}
            Choose a \glossterm{subspell} for a nature spell you know.
            You can use that subspell when you cast that spell (see \pcref{Subspells}).
            At 6th level, and every two levels thereafter, you learn an additional subspell for a nature spell you know.

    % \subcf{15th -- Air Mantle}
    % You are surrounded by a mantle of air.
    % Thrown and projectile weapons have a 50\% chance to miss your while this effect is active.
    % Unusually large weapons, such as a giant's boulders, may suffer a decreased miss chance as appropriate to their size.
    % \subcf{15th -- Aqueous Step}
    % Wherever the druid moves, you leave a path of animated water that can grab creatures.
    % Whenever a creature crosses the path, the druid makes a Reflex attack to trip the creature, causing it to fall prone and waste the rest of its movement.
    % Your accuracy is equal to your druid level \add your Constitution.
    % \subcf{15th -- Flaming Step}
    % Wherever the druid moves, you leave a path of burning flame behind your that lasts for 1 round.
    % Whenever a creature crosses the path, the druid makes a Reflex attack to deal damage to the creature.
    % The attack deals 1d8 points of fire damage per two druid levels.
    % Your accuracy is equal to your druid level \add your Constitution.
    % A failed attack deals half damage.
    % \subcf{15th -- Lifegiving Step}
    % Wherever the druid moves, you leave a path of small, living plants that entangle foes for 1 round.
    % Whenever a creature crosses the path, the druid makes a Reflex attack to entangle the creature, causing it to waste the rest of its movement.
    % Your accuracy is equal to your druid level \add your Constitution.
    % The plants appear on any surface, and will continue to grow if they can survive, though they may die quickly if they appear on inhospitable terrain.
    % \subcf{17th -- Flaming Soul}
    % You gain the fire subtype, making your immune to fire but giving your a 50\% vulnerability to cold damage.
    % In addition, whenever you deal fire damage to a creature, the creature is \ignited for 5 rounds.
    % \subcf{17th -- Sunblessed Rejuvenation}
    % You gain fast healing equal to your druid level as long as you remain in sunlight or touches a plant of your size or larger.
    % \subcf{17th -- Sunscour}
    % This aspect functions like the heart of the sun natural aspect, except that it also suppresses shadow effects and the visual components of illusions within the area of bright light.

    % \subcf{17th -- Water's Flow}
    % As a swift action, the druid can transform herself into a rushing flow of water with a volume roughly equal to your normal volume until the end of your turn.
    % In this form, she may move wherever water could go, but she cannot take other actions, such as jumping, attacking, or casting spells.
    % Your speed is halved when moving uphill and doubled when moving downhill.
    % She may move through squares occupied by creatures or threatened by blocking enemies without penalty.
    % She may return to your normal form as a free action.
    % \par If the water is split, she may reform from anywhere the water has reached, to as little as a single ounce of water.
    % If not even an ounce of water exists contiguously, your body reforms from the largest available parts of water, cut into pieces of appropriate size.
    % This usually causes the druid to die.

    \subsection{Ex-Druids}
        A druid who ceases to revere nature, changes to a prohibited alignment, or teaches the Druidic language to a nondruid loses all spells and magical druid class abilities.
        She cannot thereafter gain levels as a druid until you atone (see the \spell{atonement} ritual).

    % \subsection{Variant Druids}

    %     \subsubsection{Blighter}

    %         Blighters draw power from nature, as do other druids. However, while other druids revere nature and draw power from it gently, blighters steal power from nature forcefully. Wherever a blighter goes, destruction and death surely follows.

    %         \altcf{Blight} Instead of meditating to regain spell slots, a blighter draws power from your environment forcefully.
    %         This affects a \areahuge radius zone centered on your, and the process takes 1 minute of concentration.
    %         At the end of every round, every living thing in the area other than the blighter takes damage equal to your nature power.
    %         All inanimate plants of Huge size or smaller immediately wither and die.
    %         The earth becomes cracked and infertile, and any nutrients from the soil are destroyed.
    %         This ability has no effect on artificial environments or materials, such as metal or worked stone.
    %         At the end of the minute, the blighter regains your spent nature spell slots.

    %         A blighter can only blight your surroundings in this way once per hour.
    %         If your surroundings are already blighted or are not natural terrain, she cannot use this ability to regain your spells.
    %         Instead, she must meditate for 8 hours to slowly draw power from your surroundings, as a normal druid.

    %         \altcf{Spells} As normal, except that a blighter adds all Vivimancy arcane spells to your spell list.

    %         \altcf[2]{Wild Speech} As normal, except that a blighter gains a \plus5 bonus to Intimidate against your wild speech targets, and a \minus5 penalty to Persuasion.

    %         \altcf[10]{Blightcasting}

    %         \altcf[20]{Improved Blightcasting}

        % \subsubsection{Rotbringer}

        %     While most druids seek to emulate and interact with animals, rotbringers focus on the power of fungi, decay, and regeneration.

        %     \altcf{Invoke Rot} Instead of meditating to regain spell slots, a rotbringer accelerates the natural forces of decomposition and decay on your environment.
        %     This affects a \areahuge radius zone centered on your, and the process takes 1 minute of concentration.
        %     All organic objects of Huge size or smaller, such as plants and corpses, decompose.
        %     This decomposition kills inanimate, living plants.
        %     All organic objects, regardless of size, are covered with various fungi.
        %     This ability has no effect on artificial environments or materials, such as metal or worked stone.
        %     At the end of the minute, the rotbringer regains your spent nature spell slots.

        %     If the rotbringer decomposes a Huge object with this ability, or a combination of smaller objects equivalent in size to a Huge object, you gain an bonus nature spell slot of your highest available spell level.
        %     This extra spell slot lasts until it is used, or until you regain your spell slots again.

        %     A rotbringer can only invoke rot on your surroundings in this way once per hour.
        %     If your surroundings are already decomposed or are not natural terrain, she cannot use this ability to regain your spells.
        %     Instead, she must meditate for 8 hours to slowly draw power from your surroundings, as a normal druid.

        %     \altcf[2]{Wild Speech} The rotbringer gains the ability to speak with plants at 2nd level.
        %     You gain the ability to speak with animals at 6th level, instead of at 2nd level.

        %     \altcf[3rd]{Wild Aspect} The rotbringer does not gain this ability.

        %     \altcf[3rd]{Rot Spell} The druid learns an additional spell slot and spell known.
        %     The spell must be taken from the following list of spells.
        %     The spell's level cannot exceed half your druid level.
        %     If she already knows a spell from the list at every spell level you ha access to, she may instead learn any nature spell (see \pcref{Nature Spells}).

        %     At 5th level, and every odd level, the druid may learn a new spell.

        %     \begin{dtable}
        %         \begin{dtabularx}{\columnwidth}{l X}
        %             \tb{Spell level} & \tb{Rotbringer Spells} \\
        %             1st & \spell{excrete slime}, \spell{lesser regeneration} \\
        %             2nd & \spell{fungal growth} \\
        %             3rd & \spell{rotburst} \\
        %             4th & \spell{poison} \\
        %             6th & \spell{regeneration} \\
        %             7th & \spell{greater rotburst} \\
        %         \end{dtabularx}
        %     \end{dtable}

        %     \altcf[7]{Fungal Armor} The rotbringer becomes covered in fungus that protects your from attacks. You gain a \plus1 bonus to Armor and Fortitude defense.

        %     This bonus increases by 1 at your 7th druid level, and every 4 druid levels thereafter.

\section{Fighter}\label{Fighter}
    \begin{dtable}
        \lcaption{Fighter Progression}
        \begin{dtabularx}{\columnwidth}{>{\ccol}p{\levelcol} >{\lcol}X}
            \tb{Level} & \tb{Special} \\\bottomrule
            \nth{1}     & Consummate warrior, weapon adaptation
            \\ \nth{2}  & Armored agility
            \\ \nth{3}  & Discipline
            \\ \nth{4}  & Daunting strike
            \\ \nth{5}  & Weapon focus
            \\ \nth{6}  & Focused recovery
            \\ \nth{7}  & Twinstrike
            \\ \nth{8}  & Greater armored agility
            \\ \nth{9}  & Swift warrior
            \\ \nth{10} & Superior strike
            \\ \nth{11} & Weapon master
            \\ \nth{12} & Greater discipline
            \\ \nth{13} & Heartseeking strike
            \\ \nth{14} & Supreme armored agility
            \\ \nth{15} & Greater focused recovery
            \\ \nth{16} & Greater daunting strike
            \\ \nth{17} & Greater weapon focus
            \\ \nth{18} & Supreme discipline
            \\ \nth{19} & Warrior of legend
            \\ \nth{20} & Armored juggernaut, legendary discipline
        \end{dtabularx}
    \end{dtable}

    \classbasics{Alignment} Any.

    \subsection{Class Abilities}
        If you are a fighter, you gain the following abilities.

        \classbasics{Skill Points} 6.

        \classbasics{Class Skills}
        \subparhead{Strength} Climb, Jump, Sprint, Swim.
        \subparhead{Dexterity} Acrobatics, Escape Artist, Ride.
        \subparhead{Perception} Awareness.
        \subparhead{Other} Bluff, Intimidate, Persuasion.

        \classbasics{Defenses} \plus3 Fortitude, \plus2 Mental, \plus1 Reflex.

        \cf{Ftr}{Weapon and Armor Proficiency}
        A fighter is proficient with simple weapons, any four other weapon groups,  all armor (heavy, medium, and light), and shields.


        \subsubsection{Martial Supremacy}
            \cf{Ftr}{Consummate Warrior} You gain a \plus1d bonus to \glossterm{strike damage}.

            \cf{Ftr}[4]{Daunting Strike} As a standard action, you can use this ability.
            \begin{ability}
                \begin{spelleffects}
                    \spelleffect You make a \glossterm{strike} against one creature.
                    If that strike deals damage, the target suffers a \minus2 penalty to \glossterm{accuracy} on strikes against you.
                    This effect does not stack with itself.
                    \spelldur Condition
                \end{spelleffects}
            \end{ability}

            \cf{Ftr}[7]{Superior Strike} As a standard action, you can use this ability.
            \begin{ability}
                \begin{spelleffects}
                    \spelleffect You make a \glossterm{strike} against one creature.
                    You gain a \plus1d bonus to \glossterm{strike damage} with the strike.
                \end{spelleffects}
            \end{ability}

            \cf{Ftr}[10]{Twinstrike} As a standard action, you can use this ability.
            \begin{ability}
                \begin{spelleffects}
                    \spelleffect You make a \glossterm{strike} against up to two different creatures.
                    Both strikes must be made with the same weapon or weapons.
                \end{spelleffects}
            \end{ability}

            \cf{Ftr}[13]{Heartseeking Strike} As a standard action, you can use this ability.
            \begin{ability}
                \begin{spelleffects}
                    \spelleffect You make a \glossterm{strike} against one creature.
                    Your attack roll \glossterm{explodes} on an 8, 9, or 10 (see \pcref{Exploding Attacks}).
                    This does not affect additional dice rolled if the attack roll explodes.
                \end{spelleffects}
            \end{ability}

            \cf{Ftr}[16]{Greater Daunting Strike}
            When you deal damage to a creature with your \textit{daunting strike} ability, the target also suffers a \minus2 penalty to defenses against \glossterm{strikes} you make.

            \cf{Ftr}[19]{Warrior of Legend}
            The damage bonus from your \textit{consummate warrior} ability increases to \plus2d.

        \subsubsection{Equipment Training}
            \cf{Ftr}{Weapon Adaptation}
            If you spend an hour training with a weapon, you become proficient with that weapon's weapon group.
            You can only be proficient with one additional weapon group in this way at a time.

            \cf{Ftr}[2]{Armored Agility}
            You treat body armor you wear as less encumbering.
            You reduce its \glossterm{encumbrance penalty} by 2, and its arcane spell failure by 10\%.

            \cf{Ftr}[5]{Weapon Focus} 
            When you use your \textit{weapon adaptation} ability to gain proficiency with a weapon group, you gain a \plus1 bonus to \glossterm{accuracy} with attacks using weapons from that group.
            You can train with a weapon from a weapon group you are already proficient with to apply this ability to weapons from that group.

            \cf{Ftr}[8]{Greater Armored Agility}
            The reduction of body armor's encumbrance penalty from your \textit{armored agility} ability increases to 4, and your reduction of its arcane spell failure increases to 20\%.
            In addition, you treat it were one encumbrance category lighter than normal whenever doing so would be beneficial for you.

            \cf{Ftr}[11]{Weapon Master} 
            You gain a \plus1d bonus to \glossterm{strike damage}.

            \cf{Ftr}[14]{Supreme Armored Expertise}
            The reduction of body armor's encumbrance penalty from your \textit{armored agility} ability increases to 6, and your reduction of its arcane spell failure increases to 30\%.
            In addition, you treat it as if it were an additional encumbrance category lighter than normal whenever doing so would be beneficical for you.

            \cf{Ftr}[17]{Greater Weapon Focus} 
            Your bonus to accuracy from your \textit{weapon focus} ability increases to \plus2.

            \cf{Ftr}[20]{Armored Juggernaut}
            You gain a \plus1 bonus to all defenses while you are wearing body armor.

        \subsubsection{Combat Discipline}

            \cf{Ftr}[3]{Discipline} As a \glossterm{swift action}, you can spend an \glossterm{action point} to use this ability.
            \begin{ability}
                \begin{spelleffects}
                    \spelleffect Remove one \glossterm{condition} affecting you.
                \end{spelleffects}
            \end{ability}

            \cf{Ftr}[6]{Focused Recovery}
            When you use the \textit{recover} action, you heal \plus1d hit points.
            In addition, instead of removing a condition, you can remove any \glossterm{sustained} effect on you.

            \cf{Ftr}[9]{Swift Warrior}
            If you use only \glossterm{mundane} abilities in a given round, you can take an additional \glossterm{swift action} or \glossterm{immediate action} that round.
            After using this ability in a round, you are unable to use any \glossterm{magical} abilities until the next round.

            \cf{Ftr}[12]{Greater Discipline}
            When you use the \textit{discipline} ability, you can remove any number of conditions.

            \cf{Ftr}[15]{Greater Focused Recovery}
            The bonus to hit point recovery from the \textit{focused recovery} ability increases to \plus2d.
            In addition, when you use the \textit{recover} ability, you can remove any number of \glossterm{sustained} effects on you instead of removing a condition.

            \cf{Ftr}[18]{Endless Discipline}
            You do not need to spend an action point to use the \textit{discipline} ability.

            \cf{Ftr}[20]{Legendary Discipline} 
            You are immune to all \glossterm{conditions}.

\section{Mage}\label{Mage}
    \begin{dtable}
        \lcaption{Mage Progression}
        \begin{dtabularx}{\columnwidth}{>{\ccol}p{2em} c c >{\lcol}X}
            \tb{Level} & \tb{Spells} & \tb{Subspells} & \tb{Special} \\\bottomrule
            \nth{1}     & 2 & \tdash   & Arcane essence, rituals, spells
            \\ \nth{2}  & 3 & \tdash   & Spell knowledge, spell point
            \\ \nth{3}  & 4 & \tdash   & Arcane insight, spell knowledge
            \\ \nth{4}  & 4 & 1        & \tdash
            \\ \nth{5}  & 4 & 1        & Lesser essence lore
            \\ \nth{6}  & 4 & 2        & \tdash
            \\ \nth{7}  & 4 & 2        & Arcane insight
            \\ \nth{8}  & 4 & 3        & \tdash
            \\ \nth{9}  & 5 & 3        & Spell knowledge
            \\ \nth{10} & 5 & 4        & \tdash
            \\ \nth{11} & 5 & 4        & Arcane insight
            \\ \nth{12} & 5 & 5        & \tdash
            \\ \nth{13} & 5 & 5        & Essence lore
            \\ \nth{14} & 5 & 6        & \tdash
            \\ \nth{15} & 5 & 6        & Arcane insight
            \\ \nth{16} & 5 & 7        & Spell point
            \\ \nth{17} & 5 & 7        & Greater essence lore
            \\ \nth{18} & 5 & 8        & \tdash
            \\ \nth{19} & 5 & 8        & Arcane insight
            \\ \nth{20} & 5 & 9        & Archmage
        \end{dtabularx}
    \end{dtable}

    \classbasics{Alignment} Any.

    \subsection{Class Abilities}
        If you are a mage, you gain the following abilities.

        \classbasics{Skill Points} 4.

        \classbasics{Class Skills}
        \subparhead{Intelligence} Knowledge (all kinds, taken individually), Linguistics.
        \subparhead{Perception} Awareness, Spellcraft.
        \subparhead{Other} Bluff, Intimidate, Persuasion.

        \classbasics{Defenses} \plus3 Mental, \plus2 Reflex, \plus1 Fortitude.

        \cf{Mge}{Weapon and Armor Proficiency}
        Mages are proficient with simple weapons and one other weapon group.
        They are not proficient with any type of armor or shield.
        Armor of any type interferes with a mage's arcane gestures, which can cause your spells with somatic components to fail.

        \cf{Mge}{Arcane Essence}
        All mages have access to great arcane power.
        However, not all mages acquired this power in the same way.
        You choose an arcane essence.
        Many mage abilities have special effects based on whether you are a sorcerer or a mage.
        \parhead{Sorcerer} Sorcerers have an intuitive connection to magic that allows them to cast spells without preparation or training.
        \parhead{Wizard} Wizards studied arcane mysteries for years to learn the secret ways of magic.
        A wizard casts spells with your Intelligence.

        \subsubsection{Spellcasting}

            \cf{Mge}{Arcane Spells} 
            You can cast arcane spells.
            You learn two arcane spells from the arcane \glossterm{spell list} (see \pcref{Arcane Spells}).
            \subparhead{Sorcerer} Your spellpower with arcane spells is equal to your character level or your Willpower, whichever is higher.
            The maximum spell level you can cast is equal to half your mage level (minimum 1).
            \subparhead{Wizard} Your spellpower with arcane spells is equal to your character level or your Intelligence, whichever is higher.
            The maximum spell level you can cast is equal to half your mage level (minimum 1) or your Intelligence, whichever is lower.

            To cast a spell, you must normally spend an \glossterm{action point}.
            Every spell can also be cast as a cantrip.
            Cantrips are weaker, but do not require action points to cast.

            \cf{Mge}{Rituals}
            \subparhead{Sorcerer} You cannot perform arcane rituals.
            \subparhead{Wizard} You can perform arcane rituals to create unique magical effects (see \pcref{Rituals}).
            You have a ritual book containing two arcane rituals of your choice (see \pcref{Arcane Rituals}).

            \cf{Mge}[6]{Augment}
            Choose an \glossterm{augments} (see \pcref{Augments}).
            You can apply that augment to arcane spells you cast.
            At 10th level, and every four levels thereafter, you learn an additional augment.

            \cf{Mge}[16]{Spell Point}
            You gain a spell point.
            A spell point can be spent to cast spells in place of an action point.
            You recover all spent spell points after a \glossterm{short rest}.

        \subsubsection{Arcane Lore}
            You must have the ability to cast arcane spells to gain these abilities.

            \cf{Mge}[2]{Spell Knowledge} 
            You learn an additional spell from the arcane \glossterm{spell list} (see \pcref{Arcane Spells}).

            \cf{Mge}[3]{Arcane Insight} 
            You gain a greater understanding of magic.
            You choose one of the following insights.
            Each insight can be chosen multiple times.
            At 7th level, and every four levels thereafter, you gain an additional arcane insight.
            \begin{itemize}
                \item Innate Spell: Choose a spell you know.
                    You no longer need verbal or somatic components to cast that spell.
                    \par If you choose this insight multiple times, she must choose a different spell each time.
                \item Personal Spell: Choose a spell you know.
                    You cannot miscast that spell (see \pcref{Miscasting}).
                    If you would miscast it, the spell simply fails without effect.
                    In addition, you automatically succeeds at all Concentration checks you make to cast the spell.
                    \par If you choose this insight multiple times, you must choose a different spell each time.
                \item Specialization: Choose a school of magic.
                    You learn an additional spell from that school of magic.
                    In exchange, you must ban two other schools of magic.
                    You can never learn or cast spells or rituals from your banned schools.
                    If you know spells from a banned school, you must immediately learn different spells from unbanned schools in their place.
                    \par If you choose this insight multiple times, you must choose to specialize in the same school each time.
            \end{itemize}

            \subparhead{Sorcerer} You may also choose the Expanded Spell Knowledge insight.
            You choose a single spell from the divine spell list or nature spell list and add it to your arcane spell list.
            This does not grant you the spell as a spell known, but you may exchange one of your spells known to learn that new spell.
            \par If you choose this insight multiple times, you must choose a different spell each time.

            \subparhead{Wizard} You may also choose the Ritual Spell insight.
            You scribe an arcane spell you know into your ritual book.
            The spell is treated as a ritual, and you can perform a one minute ritual to gain the spell's effect.
            Performing the ritual costs the normal amount of material components for a ritual of its level (see \pcref{Ritual Costs}).
            You can apply standard augments or custom augments to the spell normally, increasing the ritual's level and material component cost appropriately.

            \cf{Mge}[5]{Lesser Essence Lore}[Magical]
            You gain an ability based on your choice of arcane essence.
            \subparhead{Sorcerer} You gain bonus hit points equal to your spellpower with arcane spells.
            \subparhead{Wizard} You gains a \plus2 bonus to all Knowledge skills.

            \cf{Mge}[9]{Spell Knowledge}
            You learn an additional spell from the arcane \glossterm{spell list} (see \pcref{Arcane Spells}).

            \cf{Mge}[13]{Essence Lore}[Magical]
            You gain an ability based on your choice of arcane essence.
            \subparhead{Sorcerer} You gain \glossterm{magic resistance} equal to 5 \add your spellpower with arcane spells.
            \subparhead{Wizard} You learn the Contingency augment, allowing you to prepare a spell so it takes effect automatically if specific circumstances arise.
            The Contingency augment adds two levels to a spell's level.
            You can apply this augment to any arcane spell with a casting time of a single standard action.

            Casting a spell with the Contingency augment takes 5 minutes.
            When the casting is complete, the spell has no immediate effect.
            Instead, it automatically takes effect when some specific circumstances arise.
            During this casting time, you specify what circumstances cause the spell to take effect.

            The spell can be set to trigger in response to any circumstances that a typical human observing you and your situation could detect.
            For example, you could specify ``when I fall at least 50 feet'' or ``when I become bloodied'', but not ``when there is an invisible creature within 50 feet of me'' or ``when I am at 17 hit points or fewer.''
            The more specific the required circumstances, the better -- vague requirements, such as ``when I am in danger'', may cause the spell to trigger unexpectedly or fail to trigger at all.
            If you attempt to specify multiple separate triggering conditions, such as ``when I take damage or when an enemy is adjacent to me'', the spell will randomly ignore all but one of the conditions.

            If the spell needs to be targeted, the trigger condition can specify a simple rule for identifying how to target the spell, such as ``the closest enemy''.
            If the rule is poorly worded or imprecise, the spell may target incorrectly or fail to activate at all.
            Any spells which require decisions, such as the \spell{dimension door} spell, must have those decisions made at the time it is cast.
            You cannot alter those decisions when the contingency takes effect.

            You can have only one spell with this augment active at a time.
            If you use the augment again with a different spell, the new spell.

            \cf{Mge}[17]{Greater Essence Lore}[Magical]
            You gain an ability based on your choice of arcane essence.

            \subparhead{Sorcerer} Whenever you resist a spell with your \glossterm{magic resistance}, you gain the ability to cast that spell once.
            The spell retains all augments, effects from feats and other abilities, and similar modifications from the original caster, and you cannot choose any other augments or apply effects from your own abilities.
            However, you make all other decisions required to cast the spell, and uses your spellpower to determine the spell's effects.
            Once you cast the spell, you expend the absorbed energy, and you cannot cast it again.

            If you resist multiple spells simultaneously, or if you resist another spell with your magic resistance before casting the previous spell you resisted, you choose which spell you gain the ability to cast.

            \subparhead{Wizard} You may have two spells active with the Contingency augment, rather than only one.
            Whenever you cast a new spell with the Contingency augment, you choose which existing contingency to replace.

            Only one contingency can trigger in a given round.
            If both would trigger simultaneously, only the first spell cast triggers.
            The second spell cast does not trigger that round.

            \cf{Mge}[20]{Archmage}
            You no longer need to spend action points to cast spells.
            If you have any spell points, you lose those spell points and gain the same number of legend points instead.

        \subsubsection{Arcane Spell Mastery}
            You must have the ability to cast arcane spells to gain these abilities.

            \cf{Mge}[2]{Spell Point}
            You gain a spell point.
            A spell point can be spent to cast spells in place of an action point.
            You recover all spent spell points after a \glossterm{short rest}.

            \cf{Mge}[3]{Spell Knowledge} 
            You learn an additional spell from the arcane \glossterm{spell list} (see \pcref{Arcane Spells}).

            \cf{Mge}[4]{Subspell}
            Choose a \glossterm{subspell} for an arcane spell you know.
            You can use that subspell when you cast that spell (see \pcref{Subspells}).
            At 6th level, and every two levels thereafter, you learn an additional subspell for an arcane spell you know.

\section{Monk}\label{Monk}
    \begin{dtable}
        \lcaption{Monk Progression}
        \begin{dtabularx}{\columnwidth}{>{\ccol}p{\levelcol} >{\lcol}X}
            \tb{Level} & \tb{Special} \\\bottomrule
            \nth{1}     & Manifest ki, Unarmed warrior, unfettered defense
            \\ \nth{2}  & Flurry of blows
            \\ \nth{3}  & Mental reserves
            \\ \nth{4}  & Manifest ki
            \\ \nth{5}  & Intuitive reaction
            \\ \nth{6}  & Transcend frailty
            \\ \nth{7}  & Manifest ki
            \\ \nth{8}  & Stunning fist
            \\ \nth{9}  & Serenity
            \\ \nth{10} & Manifest ki
            \\ \nth{11} & Greater intuitive reaction
            \\ \nth{12} & Transcend flesh
            \\ \nth{13} & Manifest ki
            \\ \nth{14} & Greater flurry of blows
            \\ \nth{15} & Inner peace
            \\ \nth{16} & Manifest ki
            \\ \nth{17} & Greater unfettered defense
            \\ \nth{18} & Transcend limits
            \\ \nth{19} & Manifest ki
            \\ \nth{20} & Transcend mortality
        \end{dtabularx}
    \end{dtable}

    \classbasics{Alignment} Any nonchaotic.

    \subsection{Class Abilities}
        If you are a monk, you gain the following abilities.

        \classbasics{Skill Points} 6.

        \classbasics{Class Skills}
        \subparhead{Strength} Climb, Jump, Sprint, Swim.
        \subparhead{Dexterity} Acrobatics, Escape Artist, Ride, Stealth.
        \subparhead{Intelligence} Heal.
        \subparhead{Perception} Awareness, Spellcraft, Survival.
        \subparhead{Other} Bluff, Intimidate, Perform, Persuasion.

        \classbasics{Defenses} \plus3 Reflex, \plus2 Mental, \plus1 Fortitude.

        \cf{Mnk}{Weapon and Armor Proficiency}
        Monks are proficient with simple weapons, monk weapons, and any one other weapon group.
        Monks are not proficient with any armor or shields.
        When wearing armor, using a shield, or carrying a medium or heavy load, a monk loses the benefit of your enlightened defense, fast movement, and \ki abilities.


        \subsubsection{Ki}
            \cf{Mnk}{Ki Power}
            The \glossterm{power} of your ki abilities is determined by your \textit{ki power}.
            Your \textit{ki power} is equal to your level or your Willpower, whichever is higher.

            % TODO: change this to choose two for the initial, have alternating non-choice abilities
            \cf{Mnk}{Manifest Ki} 
            You gain the ability to channel your ki to temporarily enhance your abilities.
            Choose a single ki manifestation from the list below.
            Many ki manifestations have a minimum level prerequisite, as indicated in the title of the ability.
            At the start of each round, you can spend an \glossterm{action point} to use a ki manifestation.
            This does not take an action, but you can only use one ki manifestation per round.

            At 4th level, and every three levels thereafter, you gain an additional ki manifestation.

            \subcf{Elegant Whirl of Fluid Motion}
            You gain a \plus10 bonus to Acrobatics checks until the end of the round.
            \subcf{Leap of the Heavens}
            You gain a \plus10 bonus to Jump checks until the end of the round.
            \subcf{Scale the Highest Tower}
            You gain a \plus10 bonus to Climb checks until the end of the round.

            \subcf{4th -- Dance of Falling Feathers}
            If you are in free-fall, your fall is dramatically slowed.
            You fall only 60 feet this round, and take no falling damage if you hit the ground.
            \subcf{4th -- Fists of Distant Force}
            You empower your unarmed attacks with \ki, allowing you to attack distant foes.
            Until the end of the round, you gain an additional ten feet of \glossterm{reach} with your unarmed attacks, extending your \textit{threatened area}.

            \subcf{7th -- Burst of Blinding Speed}
            You gain a \plus30 foot bonus to your land speed, up to a maximum of double your original speed, until the end of the round.
            In addition, you cannot be followed until the end of the round (see \pcref{Follow}).
            \subcf{7th -- See the Flow of Life}
            You gain the ability to see the \ki of living creatures until the end of the round.
            You can ``see'' any living creatures and their equipment within 50 feet perfectly, regardless of lighting conditions, invisibility, or any other means of concealment.
            This cannot detect living creatures through solid walls, however.
            \subcf{7th -- Surpass the Mortal Limits}
            Until the end of the round, you may use your \textit{ki power} in place of your Strength, Dexterity, and Constitution when making checks.

            \subcf{11th -- Diamond Fists}
            You gain a \plus2d bonus to damage with your unarmed attacks until the end of the round.
            In addition, you treat your unarmed attack as if it were an adamantine weapon for the purpose of overcoming damage reduction and hardness.
            \subcf{11th -- Flash Step}
            Until the end of the round, whenever you move, you teleport directly to your destination instead.
            This does not change the total distance you can move, but you can teleport in any direction, even vertically.
            If your \glossterm{line of effect} to your destination is blocked, or if this would somehow place you inside a solid object, your movement is cancelled and you remain where you are.
            You can teleport in multiple steps within the same movement to get around obstacles you see.

            % \subcf{14th -- Awaken the Pacifist Heart}
            % As an immediate action, when you hit with a melee strike, she can make a \Ki power vs. Mental attack against the struck creature.
            % Success means the target is unable to take violent actions, such as attacking, for 2 rounds.
            % If the target takes damage after the current round, the effect is broken.

            \subcf{17th -- Flash Burst}
            This ability functions like this \textit{flash step} ability, except that your movement speed is also increased tenfold.

        \subsubsection{Unfettered Warrior}
            \cf{Mnk}{Unarmed Warrior}
            You are \glossterm{proficient} with your \glossterm{unarmed attack}.
            In addition, you gain a \plus2d bonus to damage with your unarmed attack.
            For details about how to fight while unarmed, see \pcref{Unarmed Combat}.

            \cf{Mnk}{Unfettered Defense}[Magical]
            When \monkunencumbered, you gain a \plus2 bonus to Armor defense.
            You lose this bonus when you are \helpless.

            \cf{Mnk}[2]{Flurry of Blows} As a standard action, you can use this ability.
            \begin{ability}
                \begin{spelleffects}
                    \spelleffect Make an unarmed \glossterm{strike}.
                    You may roll the attack roll twice and take the result you prefer.
                \end{spelleffects}
            \end{ability}

            \cf{Mnk}[5]{Intuitive Reaction} You can react to danger before your senses would normally allow you to do so.
            You reduce your \glossterm{overwhelm penalties} by 1.
            If your overwhelm penalty is reduced to 0, you are not considered to be overwhelmed.
            In addition, you are not \unaware when attacked by surprise.

            \cf{Mnk}[8]{Stunning Fist} As a standard action, you can spend an \glossterm{action point} to use this ability.
            \begin{ability}
                \begin{spelleffects}
                    % TODO this wording is ugly
                    \spelleffect Make an unarmed \glossterm{strike} against a creature.
                    If you deal damage to the creature, it is \dazed.
                    If you deal damage with a \glossterm{critical hit}, the target is \stunned instead.
                    \spelldur Condition
                \end{spelleffects}
            \end{ability}

            \cf{Mnk}[11]{Greater Intuitive Reaction}
            Your reduction of \glossterm{overwhelm penalties} from the \textit{intuitive reaction} ability increases to 2.

            \cf{Mnk}[14]{Greater Flurry of Blows}
            When you use your \textit{flurry of blows} ability, you may make the \glossterm{strike} with any melee weapon you wield, rather than only your unarmed attack.

            \cf{Mnk}[17]{Greater Unfettered Defense}
            Your bonus to Armor defense from your \textit{unfettered defense} ability improves to \plus3.

        \subsubsection{Transcendent Sage}
            \cf{Mnk}[3]{Mental Reserves} You gain an additional \glossterm{action point}.
            As long as you have at least one action point remaining, you gain a \plus2 bonus to Mental defense.

            \cf{Mnk}[6]{Transcend Frailty}
            You are immune to being \glossterm{deafened}, \glossterm{fatigued}, and \glossterm{sickened}.

            \cf{Mnk}[9]{Serenity}[Magical]
            You gain a \plus2 bonus to Mental defense.

            \cf{Mnk}[12]{Transcend Flesh}
            You are immune to being \glossterm{blinded}, \glossterm{exhausted}, and \glossterm{nauseated}.
            % TODO: do you still take aging penalties?
            In addition, you no longer take penalties to your attributes for aging, and cannot be magically aged.
            You still die of old age when your time is up.

            \cf{Mnk}[15]{Inner Peace}
            The bonus to Mental defense from the \textit{serenity} ability increases to \plus4.

            \cf{Mnk}[18]{Transcend Limits}
            Whenever you spend your last \glossterm{action point}, you regain a spent action point at the end of the next round.

            \cf{Mnk}[20]{Transcend Mortality}[Magical]
            If you die, you may choose to retain control of your body and soul through sheer force of will.
            Your body immediately disappears, and your soul does not travel to an afterlife.
            Instead, your body reforms with no trace of its injuries 24 hours later.
            The reformed body is in perfect health and can be any age you choose, to a minimum of the age of adulthood for your race.
            You can reform your body at the place where you died, or in any place on the same plane that is deeply familiar to you.

            After each time you reform herself this way, it takes 24 additional hours to reform the next time she ``dies''.
            A monk with this ability can only be permanently killed by the direct intervention of a deity.

        \subsubsection{Ex-Monks}
            % TODO: this wording is repetitive
            If you become chaotic, you lose all of your \glossterm{magical} monk abilities.
            If you stop being chaotic, you regain your magical monk abilities.

\section{Paladin}\label{Paladin}
    \begin{dtable}
        \lcaption{Paladin Progression}
        \begin{dtabularx}{\columnwidth}{>{\ccol}p{\levelcol} c >{\lcol}X}
            \tb{Level} & \tb{Spells} & \tb{Special} \\
            \bottomrule
            \nth{1}     & 2 & Smite, spells
            \\ \nth{2}  & 2 & Spell point
            \\ \nth{3}  & 2 & Enduring smite
            \\ \nth{4}  & 2 & Lay on hands
            \\ \nth{5}  & 2 & Aligned aura
            \\ \nth{6}  & 2 & \tdash
            \\ \nth{7}  & 2 & Unfaltering zeal
            \\ \nth{8}  & 3 & Unbending devotion
            \\ \nth{9}  & 3 & Expanded aura
            \\ \nth{10} & 3 & \tdash
            \\ \nth{11} & 3 & Pass judgment
            \\ \nth{12} & 3 & Greater lay on hands
            \\ \nth{13} & 3 & Greater aligned aura
            \\ \nth{14} & 3 & \tdash
            \\ \nth{15} & 3 & Greater enduring smite
            \\ \nth{16} & 3 & Spell point
            \\ \nth{17} & 3 & Greater unbending devotion
            \\ \nth{18} & 3 & \tdash
            \\ \nth{19} & 3 & Greater unfaltering zeal
            \\ \nth{20} & 3 & Aligned soul
        \end{dtabularx}
    \end{dtable}

    \classbasics{Alignment} Any other than true neutral.

    \subsection{Class Abilities}
        If you are a paladin, you gain the following abilities.

        \classbasics{Skill Points} 4.

        \classbasics{Class Skills}
        \subparhead{Dexterity} Ride.
        \subparhead{Intelligence} Heal, Knowledge (local, religion).
        \subparhead{Perception} Awareness, Intimidate, Sense Motive.
        \subparhead{Other} Bluff, Intimidate, Persuasion.

        \classbasics{Defenses} \plus3 Fortitude, \plus2 Mental, \plus1 Reflex.

        \cf{Pal}{Weapon and Armor Proficiency}
        Paladins are proficient with simple weapons, any three other weapon groups, all types of armor (heavy, medium, and light), and shields.

        \cf{Pal}{Devoted Alignment} 
        You are devoted to a specific alignment.
        You must choose one of your alignment components: good, evil, lawful, or chaotic.
        The alignment you choose is your devoted alignment.
        Your paladin abilities are affected by this choice.
        % seems unnecessary
        % You excel at slaying creatures with alignments opposed to your devoted alignment.
        Your alignment cannot be changed without extraordinary repurcussions.

        \cf{Pal}{Devotion Power}
        The \glossterm{power} of many paladin spells and abilities is determined by your \textit{devotion power}.
        Your \textit{devotion power} is equal to your level or your Willpower, whichever is higher.

        \subsubsection{Devoted Paragon}

            \cf{Pal}[2]{Spell Point}
            You gain a spell point.
            A spell point can be spent to cast spells in place of an action point.
            You recover all spent spell points after a \glossterm{short rest}.

            \cf{Pal}[4]{Lay on Hands}[Magical] As a standard action, you can spend an \glossterm{action point} to use this ability.
            \begin{ability}
                \begin{spelltargetinginfo}
                    \spellquicktargeting{One willing creature}{Adjacent}
                \end{spelltargetinginfo}
                \begin{spelleffects}
                    \spelleffect The target is healed for 1d8 damage \add 1d per two \textit{devotion power}.
                    In addition, you may remove one \glossterm{condition} from the target.
                \end{spelleffects}
            \end{ability}

            \cf{Pal}[5]{Aligned Aura}[Magical]
            Your devotion to your alignment affects the world around you, bringing it closer to your ideals.
            You constantly radiate an aura in an \areamed radius \glossterm{emanation} from you.
            The effect of the aura depends on your devoted alignment, as described below.
            You can suppress or resume the aura as a \glossterm{swift action}.

            \subparhead{Chaos} Whenever you or an ally in the area rolls a 1 on an attack roll when making a \glossterm{strike}, the attack roll explodes (see \pcref{Exploding Attacks}).
            This does not affect additional dice rolled if the attack roll explodes.
            \subparhead{Evil} All other creatures in the area suffer a \minus1 penalty to all defenses.
            \subparhead{Good} Whenever a creature in the area takes damage, you may take half that damage (rounded down) instead.
            Any abilities you have that would make the attack miss or fail have no effect, but your abilities that allow you to reduce or ignore its effects work normally.
            The protected creature takes the remaining half of the damage, and suffers any non-damaging effects of the attack normally.
            \subparhead{Law} Whenever you or an ally in the area rolls a 1 on an attack roll when making a \glossterm{strike}, the attack roll is treated as a 6.

            \cf{Pal}[8]{Unbending Devotion}[Magical]
            You are immune to \glossterm{Mind} \glossterm{conditions}.

            \cf{Pal}[9]{Expanded Aura}
            The area of your \textit{aligned aura} becomes a \arealarge radius \glossterm{emanation} from you.

            \cf{Pal}[12]{Greater Lay on Hands} 
            You gain a \plus1d bonus to the healing from your \textit{lay on hands} ability.
            In addition, you can remove any number of conditions with that ability, rather than only one.

            \cf{Pal}[13]{Greater Aligned Aura}[Magical]
            The effect of your \textit{aligned aura} becomes stronger based on your devoted alignment.

            \subparhead{Chaos} Whenever an enemy in the area rolls a 10 on an attack roll when making a \glossterm{strike}, it is forced to rereroll the attack roll and take the second result.
            \subparhead{Evil} The penalty imposed by the aura increases to \minus2.
            \subparhead{Good} When you redirect damage from an ally with this aura, you can redirect all effects of the attack to you instead of only half the damage.
            \subparhead{Law} Whenever an enemy in the area rolls a 10 on an attack roll when making a \glossterm{strike}, the attack roll is treated as a 6.

            \cf{Pal}[17]{Greater Unbending Devotion}
            You are immune to all hostile \glossterm{Mind} effects.

            \cf{Pal}[20]{Aligned Soul}[Magical]
            While you are dead, you may approach the deity or governing figure of your afterlife and request to be returned to life to continue your mission.
            Travelling to the relevant figure and making the request takes 12 hours.
            Unless there are extenuating circumstances, this request is almost always granted, and you are resurrected in a new body at a location of the entity's choice.
            This functions like the \spell{resurrection} ritual, except that no part of the body is required, and a new body is created by the entity.
            You can be resurrected in this way regardless of the condition of your body, but not if your soul has been trapped or otherwise prevented from going to the correct afterlife.

        % TODO: clarify interaction between this ability and cleric spellcasting;
        % you shouldn't have both, but they technically use different spellpowers
        \subsubsection{Spellcasting}

            \cf{Pal}{Divine Spells}
            Your devotion to your alignment grants you the ability to cast divine spells.
            You learn two divine spells from the divine \glossterm{spell list} (see \pcref{Divine Spells}).
            Your \glossterm{spellpower} with divine spells is equal to your \glossterm{devotion power}.

            To cast a spell, you must normally spend an \glossterm{action point}.
            Every spell can also be cast as a cantrip.
            Cantrips are weaker, but do not require action points to cast.

            You can't cast spells of an alignment opposed to your own.
            Spells associated with particular alignments are indicated by the \glossterm{Chaos}, \glossterm{Good}, \glossterm{Evil}, and \glossterm{Law} tags in their spell descriptions.

            \cf{Pal}[6]{Augment}
            Choose an \glossterm{augment} (see \pcref{Standard Augments}).
            You can apply that augment to divine spells you cast.
            At 10th level, and every four levels thereafter, you learn an additional augment.

            \cf{Pal}[8]{Spell Knowledge}
            You learn an additional divine spell (see \pcref{Divine Spells}).

            \cf{Pal}[16]{Spell Point} 
            You gain a spell point.
            A spell point can be spent to cast spells in place of an action point.
            You recover all spent spell points after a \glossterm{short rest}.

        \subsubsection{Zealous Warrior}
            \cf{Pal}{Smite}[Magical] As a standard action, you can spend an \glossterm{action point} to use this ability.
            \begin{ability}
                \begin{spelleffects}
                    \spelleffect You make a \glossterm{strike}.
                    If your target shares your devoted alignment, the strike deals no damage.
                    Otherwise, the strike gains a \plus1d bonus to \glossterm{strike damage}, and you regain the action point spent to use this ability.
                \end{spelleffects}
            \end{ability}

            \cf{Pal}[3]{Zealous Offense}[Magical]
            If you deal damage to a creature with your \textit{smite} ability, you gain a \plus1d bonus to \glossterm{strike damage} against that creature.
            This effect lasts until you take a \glossterm{short rest}.

            \cf{Pal}[7]{Unfaltering Zeal}
            You gain a \plus1 bonus to Fortitude, Reflex, and Mental defense.

            \cf{Pal}[11]{Pass Judgment}[Magical] As a \glossterm{swift action}, you can spend an \glossterm{action point} to use this ability.
            \begin{ability}
                \begin{spelltargetinginfo}
                    \spellquicktargeting{One creature}{\rngmed}
                \end{spelltargetinginfo}
                \begin{spelleffects}
                    \spelleffect For the purpose of all spells and effects, the target is treated as if it had the alignment opposed to your devoted alignment.
                    This only affects its alignment along the alignment axis your devoted alignment is on.
                    For example, if your devoted alignment was evil, a chaotic neutral target would be treated as chaotic good.
                    In addition, the target is treated as if you had smited it for the purpose of the \textit{zealous offense} ability and similar effects.

                    You can use this ability to do battle against foes who share your alignment, but you should exercise caution in doing so.
                    Persecution of allies can lead you to fall and become an ex-paladin.
                    \spelldur{Attunement}
                \end{spelleffects}
            \end{ability}

            \cf{Pal}[15]{Greater Zealous Offense}[Magical]
            The damage bonus from the \textit{zealous offense} ability increases to \plus2d.

            \cf{Pal}[19]{Greater Unfaltering Zeal}
            The bonus to defenses from the \textit{unfaltering zeal} ability increases to \plus2.

            % TODO: Doesn't work with new spell system
            % \cf{Pal}[20]{Martyr's Retribution}[Mag]
            % If you die in the service of your devoted alignment, you may choose to have your fallen body erupt in an immense burst of divine energy.
            % If you do, your body is almost completely consumed, preventing you from being raised with \ritual{resurrection} and similar effects that require an intact body.
            % This burst has two effects.
            % First, a \spell{sunburst} spell immediately takes effect over the area where you died.
            % Second, a \spell{storm of vengeance} spell begins to take effect, centered on the same area.
            % The spell lasts for 10 rounds, and the lightning strikes target the paladin's enemies.
            % Both of these effects harm only the paladin's foes, and do not harm your allies.
            % However, your allies' vision is still impeded by the \spell{storm of vengeance}.

        \subsubsection{Ex-Paladins}
            If you cease to follow your devoted alignment, you lose all \glossterm{magical} paladin class abilities.
            If your atone for your misdeeds and resume the service of your devoted alignment, you can regain your abilities.

\section{Ranger}\label{Ranger}
    \begin{dtable}
        \lcaption{Ranger Progression}
        \begin{dtabularx}{\columnwidth}{>{\ccol}p{\levelcol} >{\lcol}X}
            \tb{Level} & \tb{Special} \\\bottomrule
            \nth{1}     & Keen vision, quarry
            \\ \nth{2}  & Learned perception, tracker
            \\ \nth{3}  & Wilderness lore
            \\ \nth{4}  & Hunting style
            \\ \nth{5}  & Blindsense
            \\ \nth{6}  & Survival of the fittest
            \\ \nth{7}  & Learned pursuit
            \\ \nth{8}  & Farsight
            \\ \nth{9}  & Wilderness lore
            \\ \nth{10} & Hunting style
            \\ \nth{11} & Blindsight
            \\ \nth{12} & 
            \\ \nth{13} & Lethal quarry
            \\ \nth{14} & Greater farsight
            \\ \nth{15} & 
            \\ \nth{16} & Hunting style
            \\ \nth{17} & Truesight
            \\ \nth{18} & 
            \\ \nth{19} & 
            \\ \nth{20} & 
        \end{dtabularx}
    \end{dtable}

    \classbasics{Alignment} Any.

    \subsection{Class Abilities}
        If you are a ranger, you gain the following abilities.

        \classbasics{Skill Points} 8.

        \classbasics{Class Skills}
        \subparhead{Strength} Climb, Jump, Sprint, Swim.
        \subparhead{Dexterity} Acrobatics, Escape Artist, Ride, Stealth.
        \subparhead{Intelligence} Heal, Knowledge (dungeoneering, geography, nature).
        \subparhead{Perception} Awareness, Creature Handling, Survival.
        \subparhead{Other} Bluff, Intimidate, Persuasion.

        \classbasics{Defenses} \plus3 Reflex, \plus2 Fortitude, \plus1 Mental.

        \cf{Rgr}{Weapon and Armor Proficiency}
        A ranger is proficient with simple weapons, any two weapon groups, light and medium armor, and shields.
        You are also proficient with your choice of bows, crossbows, or thrown weapons.

        \subsubsection{Keen Senses}
            \cf{Rgr}{Keen Vision}
            Your sight improves, allowing you to see more easily.
            You gain \glossterm{low-light vision}, allowing you to treat sources of light as if they had double their normal illumination range.
            If you already have low-light vision, he double its benefit, allowing you to treat sources of light as if they had four times their normal illumination range.

            In addition, you gain \glossterm{darkvision} out to 50 feet, allowing you to see in complete darkness.
            If you already have darkvision, you increase its range by 50 feet.

            \cf{Rgr}[2]{Learned Perception} You gain two skill points that must be spent on Perception-based ranger class skills.

            \cf{Rgr}[5]{Blindsense}
            Your perceptions are so finely honed that you can sense your enemies without seeing them.
            You gain the \glossterm{blindsense} ability out to 50 feet.
            This ability allows you to sense the presence and location of objects and foes within 50 feet without seeing them.
            If you already have the blindsense ability, you increase its range by 50 feet.

            \cf{Rgr}[8]{Farsight}
            You increase the range of your \glossterm{darkvision} by 150 feet, and your \glossterm{blindsense} by 50 feet.
            In addition, you reduce your \glossterm{range increment penalties} for attacking at long range by 2.

            \cf{Rgr}[11]{Blindsight}
            You gain the \glossterm{blindsight} ability, allowing you to ``see'' perfectly without your eyes in a 50 foot radius around you.
            With this ability, you can fight just as well with your eyes closed as with them open.

            \cf{Rgr}[14]{Greater Farsight}
            You increase the range of your \glossterm{darkvision} by 500 feet, your \glossterm{blindsense} by 200 feet, and your \glossterm{blindsight} by 50 feet.
            In addition, the penalty reduction for \glossterm{range increment penalties} from your \textit{farsight} ability increases to 5.

            \cf{Rgr}[17]{Truesight} 
            Your perceptions are accurate enough to defeat even powerful magic.
            You can see through normal and magical darkness, see the truth behind visual figments and glamers, and see the true form of creatures and objects affected by \glossterm{Shaping} abilities.
            This ability works at any range.

        \subsubsection{Wilderness Warrior}
            \cf{Rgr}[3]{Wilderness Lore} You gain two extra skill points which must be spent on the Creature Handling, Heal, Knowledge (geography), Knowledge (nature), Ride, or Survival skills.

            \cf{Rgr}[6]{Survival of the Fittest}
            You gain a \plus1d bonus to \glossterm{strike damage}.

            \cf{Rgr}[9]{Wilderness Lore} You gain two extra skill points which must be spent on the Creature Handling, Heal, Knowledge (geography), Knowledge (nature), Ride, or Survival skills.

            \cf{Rgr}[12]{} 

        \subsubsection{Focused Hunter}

            \cf{Rgr}{Quarry}
            As a \glossterm{swift action}, you can use this ability.
            \begin{ability}
                \begin{spelltargetinginfo}
                    \spellquicktargeting{One creature}{\rnglong}
                \end{spelltargetinginfo}
                \begin{spelleffects}
                    \spelleffect You gain a \plus2 bonus to checks made to follow the target's tracks, and a \plus1d bonus to damage on \glossterm{strike damage} against the target.
                    This ability lasts the target is \glossterm{defeated}, or until you use this ability again.
                \end{spelleffects}
            \end{ability}

            \cf{Rgr}[2]{Tracker}
            You gain a \plus5 bonus to checks made to follow tracks.
            In addition, you may use your level in place of the Survival skill to follow tracks (see \pcref{Survival}).

            \cf{Rgr}[4]{Hunting Style}
            You learn specific hunting styles to defeat particular quarries.
            Choose two hunting styles from the list below.
            Many hunting styles have a minimum level prerequisite, as indicated in the title of the ability.
            Whenever you use your \textit{quarry} ability, you can also gain the benefit of one hunting style you know.

            \subcf{Executioner}
            % TODO: clarify timing
            Whenever you deal damage to your quarry, any damage in excess of its remaining hit points is dealt as \glossterm{vital damage}.

            \subcf{Fearsome}
            Your quarry is \shaken by you as long as it remains your quarry.
            This is a \glossterm{Delusion}, \glossterm{Mind} effect.

            \subcf{Goading}
            Your quarry is \goaded by you as long as it remains your quarry.
            This is a \glossterm{Delusion}, \glossterm{Mind} effect.

            \subcf{Inescapable}
            Whenever you deal damage to your quarry, its movement speed is halved until the end of the next round.

            \subcf{Persistent}
            Your quarry remains your quarry even after you use the \textit{quarry} ability again.
            It remains your quarry until you use this ability again.

            \subcf{10th -- Anchored}[Magical]
            Whenever you deal damage to your quarry, it cannot travel extradimensionally until the end of the next round.
            This blocks teleportation and all planar travel abilities except planar rifts.

            \subcf{10th -- Leeching}
            Whenever you deal damage to your quarry, you heal hit points equal to your level.

            \subcf{10th -- Punishing}[Magical]
            At the end of each round, if your quarry is within \rnglong range of you, it takes life damage equal to your level.

            \subcf{10th -- Unerring}
            You ignore all miss chances and failure chances that would affect attacks and checks you make against your quarry.

            \subcf{10th -- Wolfpack}
            If your quarry is \glossterm{overwhelmed}, it increases its \glossterm{overwhelm penalties} by 1.

            \subcf{16th -- Greater Anchored}[Magical]
            Your quarry cannot travel extradimensionally.
            This blocks teleportation and all planar travel abilities except planar rifts.

            \subcf{16th -- Master of the Hunt}
            Your quarry takes a \minus2 penalty to defenses against attacks from creatures other than you.

            \subcf{16th -- Taunting}
            Your quarry is \taunted by you as long as it remains your quarry.
            This is a \glossterm{Delusion}, \glossterm{Mind} effect.

            \subcf{16th -- Terrifying}
            Your quarry is \frightened by you as long as it remains your quarry.
            This is a \glossterm{Delusion}, \glossterm{Mind} effect.

            \cf{Rgr}[7]{Learned Pursuit} You gain two extra skill points which must be spent on Strength or Dexterity-based ranger class skills.

            \cf{Rgr}[10]{Hunting Style}
            You learn an additional \textit{hunting style}.

            \cf{Rgr}[13]{Lethal Quarry}
            Your damage bonus from the \textit{quarry} ability increases to \plus2d.

            \cf{Rgr}[16]{Hunting Style}
            You learn an additional \textit{hunting style}.

\section{Rogue}\label{Rogue}
    \begin{dtable}
        \lcaption{Rogue Progression}
        \begin{dtabularx}{\columnwidth}{>{\ccol}p{\levelcol} >{\lcol}X}
            \tb{Level} & \tb{Special}
            \\\bottomrule
               \nth{1}  & Skill lore, sneak attack
            \\ \nth{2}  & Stealth lore
            \\ \nth{3}  & Combat trick
            \\ \nth{4}  & Skill exemplar
            \\ \nth{5}  & Uncanny dodge
            \\ \nth{6}  & Ambush attack
            \\ \nth{7}  & Skill lore
            \\ \nth{8}  & Assassinate
            \\ \nth{9}  & Lucky slip
            \\ \nth{10} & Greater skill exemplar
            \\ \nth{11} & Greater uncanny dodge
            \\ \nth{12} & Combat trick
            \\ \nth{13} & Lucky break
            \\ \nth{14} & Greater sneak attack
            \\ \nth{15} & Twist of fate
            \\ \nth{16} & Supreme skill exemplar
            \\ \nth{17} & 
            \\ \nth{18} & Greater ambush attack
            \\ \nth{19} &
            \\ \nth{20} &
        \end{dtabularx}
    \end{dtable}

    \classbasics{Alignment} Any.

    \subsection{Class Abilities}
        If you are a rogue, you gain the following abilities.

        \classbasics{Skill Points} 8.

        \classbasics{Class Skills}
        \subparhead{Strength} Climb, Jump, Sprint, Swim.
        \subparhead{Dexterity} Acrobatics, Escape Artist, Sleight of Hand, Stealth.
        \subparhead{Intelligence} Devices, Disguise, Knowledge (dungeoneering, local), Linguistics.
        \subparhead{Perception} Awareness, Sense Motive.
        \subparhead{Other} Bluff, Intimidate, Perform, Persuasion.

        \classbasics{Defenses} \plus3 Reflex, \plus2 Mental, \plus1 Fortitude.

        \cf{Rog}{Weapon and Armor Proficiency}
        Rogues are proficient with simple weapons, any two weapon groups, light armor, and bucklers.
        They are also proficient with saps.

        \subsubsection{Assassin}
            \cf{Rog}{Sneak Attack} You gain a \plus1d bonus to \glossterm{strike damage} against creatures who are unable to defend themselves effectively.
            This applies against creatures who are \unaware, \defenseless, or \glossterm{overwhelmed}.

            You must be within \rngclose range of a creature to gain this damage bonus.
            In addition, you do not gain this damage bonus against creatures who are immune to \glossterm{critical hits} or who lack a discernible body structure, such as oozes.

            \cf{Rog}[2]{Stealth Lore} You gain two extra skill points which must be spent on the Acrobatics, Awareness, Disguise, Sleight of Hand, or Stealth skills.

            \cf{Rog}[5]{Uncanny Dodge} You can react to danger before your senses would normally allow you to do so.
            You reduce your \glossterm{overwhelm penalties} by 1.
            If your overwhelm penalty is reduced to 0, you are not considered to be overwhelmed.
            In addition, you are not \unaware when attacked by surprise.

            \cf{Rog}[8]{Assassinate} As a standard action, you can use this ability.
            \begin{ability}
                \begin{spelltargetinginfo}
                    \spellquicktargeting{One creature}{\rngmed}
                \end{spelltargetinginfo}
                \begin{spelleffects}
                    \spelleffect You study the target, finding weak points you can take advantage of.
                    % TODO is ``make a sneak attack'' clear enough wording?
                    Until the end of the next round, if you make a melee \textit{sneak attack} against the target while it is \unaware, your attack deals maximum damage.
                \end{spelleffects}
            \end{ability}

            \cf{Rog}[11]{Greater Uncanny Dodge}
            Your reduction of \glossterm{overwhelm penalties} from the \textit{uncanny dodge} ability increases to 2.

            \cf{Rog}[14]{Greater Sneak Attack}
            The damage bonus from your \textit{sneak attack} ability increases to \plus2d.

            \cf{Rog}[17]{} 

        \subsubsection{Jack of All Trades}

            \cf{Rog}{Skill Lore} You gain three extra skill points which must be spent on rogue class skills.

            \cf{Rog}[4]{Skill Exemplar} You gain a \plus1 bonus to all skills.

            \cf{Rog}[7]{Skill Lore} You gain three extra skill points which must be spent on rogue class skills.

            \cf{Rog}[10]{Greater Skill Exemplar} The skill bonus from your \textit{skill exemplar} ability increases to \plus2.

            \cf{Rog}[13]{Lucky Break} Once per round, when you make a \glossterm{check}, you can spend an \glossterm{action point} to use this ability.
            If you do, you treat your roll as a 10.

            \cf{Rog}[16]{Supreme Skill Exemplar} The skill bonus from your \textit{skill exemplar} ability increases to \plus3.

        \subsubsection{Scoundrel}

            \cf{Rog}[3]{Combat Trick}
            You learn how to confuse and confound your foes in combat.
            Choose a two combat tricks from the list below.
            Many combat tricks have a minimum level prerequisite, as indicated in the title of the ability.

            \subparhead{Bewildering Blow} As a standard action, you can spend an \glossterm{action point} to use this ability.
            \begin{ability}
                \begin{spelleffects}
                    % TODO this wording is ugly
                    \spelleffect Make a \glossterm{strike} against a creature.
                    If you deal damage to the creature, it is \disoriented.
                    If you deal damage with a \glossterm{critical hit}, the creature is \stunned instead.
                    \spelldur Condition
                \end{spelleffects}
            \end{ability}

            \subparhead{Distant Precision} If you have the \textit{sneak attack} ability, the maximum range at which you can use that ability increases to \rnglong.

            \subparhead{Distracting Blow} As a standard action, you can use this ability.
            \begin{ability}
                \begin{spelleffects}
                    % TODO this wording is ugly
                    \spelleffect Make a \glossterm{strike} against a creature.
                    If you deal damage to the creature, it automatically fails any Concentration checks it makes until the end of the round (see \pcref{Concentration}).
                \end{spelleffects}
            \end{ability}

            \subparhead{Hamstring} As a standard action, you can use this ability.
            \begin{ability}
                \begin{spelleffects}
                    % TODO this wording is ugly
                    \spelleffect Make a \glossterm{strike} against a creature.
                    If you deal damage to the creature, it moves at half speed.
                    If you deal damage with a \glossterm{critical hit}, the creature is \immobilized instead.
                    \spelldur Condition
                \end{spelleffects}
            \end{ability}

            \subparhead{Merciful} You take no penalties when using weapons to deal \glossterm{nonlethal damage}.

            %TODO: make poison work
            %\subcf{Swift Poisoner}
            %The rogue can apply poison to a weapon you are holding as a swift action.

            \subparhead{Tricky Maneuvers} You gain a \plus1 bonus to \glossterm{accuracy} with \glossterm{combat maneuvers}.

            \cf{Rog}[6]{Ambush Attack}
            You gain a \plus1d bonus to \glossterm{strike damage} against creatures who did not \glossterm{threaten} you at the start of the round.
            The target does not have to be \unaware of your attack.

            \cf{Rog}[9]{Lucky Slip} Once per round, when you are hit by a \glossterm{strike}, you can spend an \glossterm{action point} to use this ability.
            If you do, the attacking creature rerolls the attack roll.

            \cf{Rog}[12]{Combat Trick}
            You learn an additional \textit{combat trick}.

            \cf{Rog}[15]{Twist of Fate} You can use your \textit{lucky slip} ability against any successful attack, not just a \glossterm{strike}.

            \cf{Rog}[18]{Greater Ambush Attack}
            The damage bonus from your \textit{ambush attack} ability increases to \plus2d.
