\chapter{Classes}\label{Classes}

A character's class represents their fundamental source of power.
Classes help establish a clear narrative for your character, including a justification for their special abilities and a sense of how they fit into the larger world.
They often provide unique actions with specific effects, such as spells and martial maneuvers.
They can also grant statistical benefits or even change how the characters interact with normal game rules.

Classes are flexible enough to allow significant customization.
Your class is intended to be a jumping off point for creativity that is less intimidating than a blank canvas, not a limitation on the possible character concepts you can be.
Two characters with the same class can have very different specialties, narratives, and play styles.

\section{How Classes Work}
  When you first create a character, you choose a class.
  Each class grants some basic class features to all members of that class.
  In addition, each class has a number of \glossterm{archetypes} that grant more powerful and specific abilities.

  \subsection{Base Class}\label{Base Class}
    Your class grants you a specific set of benefits.
    They are listed under the heading ``Base Class Effects''.
    If you have multiple classes, you only gain these benefits for one of your classes, which is called your \glossterm{base class}.

  \subsection{Archetypes}\label{Archetypes}
    Each class has five class archetypes.
    An archetype is a collection of thematically related abilities.
    For example, barbarians have the Battlerager archetype, which grants abilities related to being angry and flying into a rage in combat.

    \subsubsection{Archetype Ranks}\label{Archetype Ranks}
      You have an \glossterm{archetype rank} associated with each of your archetypes.
      Each ability from an archetype has a minimum rank required to gain the ability.
      When you gain a rank in an archetype, you gain all abilities associated with that rank.
      In addition, some of your existing abilities may increase their power based on your rank in that archetype.

      At 1st level, you choose a single archetype from your class (or classes).
      You become rank 1 in that archetype, and you do not have any other archetypes.
      At 2nd and 3rd level, you choose an additional archetype from your class or classes.
      Each time, you become rank 1 in that archetype.

      After 3rd level, you never gain additional archetypes.
      Instead, at each level, you increase your rank in one of your existing archetypes.
      Each \glossterm{archetype rank} has a minimum level, as shown on \trefnp{Archetype Ranks by Level}.
      The minimum level is included in each class table as a reminder.
      In practice, this means that you have to increase all of your ranks evenly instead of specializing in a single archetype.

      \begin{dtable}
        \lcaption{Archetype Ranks by Level}
        \begin{dtabularx}{\columnwidth}{l >{\lcol}X}
          \tb{Archetype Rank} & \tb{Minimum Level} \tableheaderrule
          1 & 1  \\
          2 & 4  \\
          3 & 7  \\
          4 & 10 \\
          5 & 13 \\
          6 & 16 \\
          7 & 19 \\
        \end{dtabularx}
      \end{dtable}

    \subsubsection{Duplicate Archetypes}\label{Duplicate Archetypes}
      Clerics and paladins have an identical Divine Magic archetype.
      You cannot gain two archetypes with the same name.

  \subsection{Multiclass Characters}\label{Multiclass Characters}
    You can spend two \glossterm{insight points} to become a \glossterm{multiclass} character (see \pcref{Insight Points}).
    If you do, choose a class you don't already have.
    You gain the following benefits relating to that class.
    \begin{itemize}
      \item You choose whether the class becomes your \glossterm{base class}.
        If your base class changes, you lose all benefits of your original base class, and instead gain the benefits of your new base class.
      \item If the class has any special class abilities which are not part of an archetype, such as a warlock's \textit{soul pact} ability, you gain those abilities.
      \item Whenever you gain a new archetype, you may choose an archetype from any class you belong to.
    \end{itemize}

    You may gain access to multiple classes in this way, spending two \glossterm{insight points} for each class.

  \subsection{Late Multiclassing}
    If have multiple archetypes when you become a multiclass character, you may exchange any number of your existing archetypes for that many archetypes from your new class.
    However, you must always have at least one archetype from your \glossterm{base class}.

    The GM can decide whether you are allowed to become a multiclass character after level 1.
    Multiclassing can significantly change your character's abilities.
    That may not be narratively appropriate for all campaigns.
    In general, the higher level you are before multiclassing, the stronger the narrative justification should be.

\section{Class Description Format}
  Each class is described from the perspective of a member of that class, using ``you'' in the description.

  \parhead{Class Table}
  Each class's table describes the special abilities a member of that class gains at each rank of each of that class's archetypes.

  \parhead{Alignment}
  Some classes require specific alignments (see \pcref{Alignment}).
  Most classes allow characters of any alignment.

  \parhead{Skills}
  Each class has specific \glossterm{skills} that members of that class are typically good at (see \pcref{Skills}).
  These skills are called \glossterm{class skills}.
  For details, see \pcref{Trained Skills}.

  \parhead{Defenses}
  Each class grants bonuses to specific defenses.

  \parhead{Weapon Proficiencies}
  This indicates the types of weapons that members of this class are proficient with.

  \parhead{Armor Proficiencies}
  This indicates the types of armor that members of this class are proficient with.

  \parhead{Other Special Abilities}
  Some classes have abilities shared by all members of the class that are not part of an archetype, such as a druid's \textit{druidic language} ability.

  \parhead{Archetypes}
  The abilities associated with each of the three archetypes the class has.

  \input{generated/classes.tex}
