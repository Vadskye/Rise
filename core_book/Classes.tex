\chapter{Classes}\label{Classes}

Your character's class represents the things your character has chosen to train in.
This choice determines a great deal about your character's abilities.

\section{How Classes Work}
    When you first create a character, you choose a class.
    Your character has one level in that class.
    This grants your character all of the abilities your chosen class grants at 1st level, as given in the class description.
    Each time your character gains a level, you can choose to increase your level in your original class or gain a level in a new class.
    This grants your character all of the abilities your chosen class grants at the level your character just gained in it.

    \subsection{Base Classes}
        Every character has one \glossterm{base class}.
        You may choose any class your character has at least one level in as a base class.
        Whenever your character gain a level, you can change its base class to a different class it has.
        Your choice of base class affects your character's \glossterm{defenses}, \glossterm{skill points}, and \glossterm{class skills}.
        In addition, every class grants additional abilities if it is chosen as a base class, as given in the class description.

\section{Class Introductions}

    There are eleven classes in Rise.
    \begin{itemize}
        \item Barbarians are mighty warriors who can enter a deadly battlerage.
        \item Clerics are divine spellcasters who draw power from their veneration of a deity or ideal.
        \item Druids are nature spellcasters who draw power from their veneration of the natural world.
        \item Fighters are highly disciplined warriors who excel in physical combat of any variety.
        \item Wizards are arcane spellcasters who wield the mystic forces of magic to create almost any effect.
        \item Monks are agile masters of ``\ki'' who hone their personal abilities to strike down foes and perform supernatural feats.
        \item Paladins are divinely empowered warriors whose devotion to an alignment grants them the ability to discern and smite their foes.
        \item Rangers are skilled hunters who bridge the divide between nature and civilization.
        \item Rogues are exceptionally skillful characters known for their ability to strike at their foe's weak points in combat.
        \item Spellwarped wield a unique blend of martial skill and narrowly focused magical abilities.
    \end{itemize}

    \subsection{Class Description Format}

        \parhead{Class Table}
        The class's table describes the special abilities they get at each level.

        \parhead{Alignment}
        Some classes require specific alignments (see \pcref{Alignment}).
        Most classes allow characters of any alignment.

        \parhead{Class Skills}
        These are skills that members of this class are typically good at (see \pcref{Skills}).

        \subsubsection{Base Class Abilities}
            Abilities contained within this heading only apply to characters with the current class as a \glossterm{base class}.

            \parhead{Skill Points}
            This is the number of skill points that members of this class get.

            \parhead{Defenses}
            Each class grants bonuses to specific defenses.
            These bonuses do not stack with other defense bonuses granted by base classes.
            If a character has multiple base classes, use the highest bonuses that apply to each defense.

            \parhead{Weapon and Armor Proficiencies}
            These are the types of equipment that members of this class are trained in using.

        \subsubsection{Class Abilities}
            The class abilities that a character gets for being a member of the class.

\section{Barbarian}\label{Barbarian}
    \begin{dtable}
        \lcaption{Barbarian Progression}
        \begin{dtabularx}{\columnwidth}{>{\ccol}p{\levelcol} c >{\lcol}X}
            \tb{Level} & \tb{Rage} & \tb{Special} \\
            \hline
            \nth{1}  & \plus2 & Rage, primal rage      \\
            \nth{2}  & \plus2 & Fast movement          \\
            \nth{3}  & \plus2 & Uncanny dodge          \\
            \nth{4}  & \plus2 & Durable                \\
            \nth{5}  & \plus3 & Larger than life       \\
            \nth{6}  & \plus3 & Rage feat              \\
            \nth{7}  & \plus3 & Improved uncanny dodge \\
            \nth{8}  & \plus3 & Battle-scarred         \\
            \nth{9}  & \plus3 & Primal resilience      \\
            \nth{10} & \plus4 & Instinctive rage       \\
            \nth{11} & \plus4 & Larger than belief     \\
            \nth{12} & \plus4 & Titanic resilience     \\
            \nth{13} & \plus4 & Fury of the storm      \\
            \nth{14} & \plus4 & Mindless rage          \\
            \nth{15} & \plus5 & \tdash                 \\
            \nth{16} & \plus5 & Supreme rage           \\
            \nth{17} & \plus5 & Titan of battle        \\
            \nth{18} & \plus5 & Deathless rage         \\
            \nth{19} & \plus5 & Endless rage           \\
            \nth{20} & \plus6 & \tdash                 \\
        \end{dtabularx}
    \end{dtable}

    \classbasics{Alignment} Any nonlawful.

    \classbasics{Class Skills}
    \begin{itemize}
        \item \subparhead{Strength} Climb, Jump, Sprint, Swim.
        \item \subparhead{Dexterity} Acrobatics, Ride.
        \item \subparhead{Perception} Awareness, Creature Handling, Survival.
        \item \subparhead{Other} Bluff, Intimidate, Persuasion.
    \end{itemize}

    \subsection{Base Class Abilities}
        A character with barbarian as a base class gains the following abilities.

        \classbasics{Skill Points} 10.

        \classbasics{Defenses} \plus4 Fortitude, \plus2 Reflex.

        \classbasics{Action Points} \plus1.

        \cf{Bbn}{Weapon and Armor Proficiency}
        A barbarian is proficient with simple weapons, any four other weapon groups, light armor, medium armor, and shields.

        \cf{Bbn}{Primal Rage}[Ex]
        The barbarian's damage reduction while raging is equal to his character level \add 2, rather than his barbarian level.
        In addition, he is not \fatigued after finishing a rage.
        He must still rest for 5 minutes before he can rage again after finishing a rage.

    \subsection{Class Abilities}
        All barbarians have the following abilities.

        \cf{Bbn}{Rage}[Ex]\label{Rage}
        A barbarian can fly into a rage by spending a \glossterm{action point} as a \glossterm{free action}.
        While raging, the barbarian has the following benefits and drawbacks:
        \begin{itemize}
            \item \plus1d bonus to damage with \glossterm{strikes}.
            \item \glossterm{Damage reduction} against physical damage equal to his barbarian level.
            \item Unable to take any action that requires patience or concentration, such as casting spells.
            \item At the end of each round, if the barbarian did not attack a creature or object, he takes \glossterm{nonlethal damage} equal to his level.
                This damage ignores his damage reduction from raging.
        \end{itemize}

        A rage lasts for up to 5 minutes, though he can end it early as a \glossterm{free action}.
        At the end of that time, his rage ends.
        When his rage ends, he takes nonlethal damage equal to his level.
        In addition, he becomes \fatigued and unable to rage until he rests for 5 minutes.

        \cf{Bbn}[2nd]{Fast Movement}[Ex]
        The barbarian gains a \plus10 foot bonus to speed in all movement modes.

        \cf{Bbn}[3rd]{Uncanny Dodge}[Ex]
        A barbarian can react to danger before his senses would normally allow him to do so.
        He reduces his \glossterm{overwhelm penalties} by 1.
        If his overwhelm penalty is reduced to 0, he is not considered to be overwhelmed.
        In addition, he is not \unaware when attacked by surprise.

        \cf{Bbn}[4th]{Durable}[Ex]
        The barbarian gains a \plus1 bonus to Fortitude defense.

        \cf{Bbn}[5th]{Larger than Life}[Ex]
        A barbarian holds the strength of a giant in the body of a man (or woman).
        He gains a \plus1d bonus to damage with \glossterm{strikes}.

        \cf{Bbn}[6th]{Rage Feat}
        The barbarian gains a bonus Combat feat (see \pcref{Combat Feats}).
        He only gains the benefits of the feat while he is raging, and he cannot use that feat as a prerequisite for any other feat or ability.

        \cf{Bbn}[7th]{Improved Uncanny Dodge}[Ex]
        The barbarian reduces his overwhelm penalties by 2.
        This does not stack with the effects of uncanny dodge.
        If his overwhelm penalty is reduced to 0, he is not considered to be overwhelmed.

        \cf{Bbn}[8th]{Battle-Scarred}[Ex]
        The barbarian doubles the benefit of any healing he receives.
        This affects both natural and magical healing.

        \cf{Bbn}[9th]{Primal Resilience}[Ex]
        The barbarian gains a \plus2 bonus to Fortitude defense.
        This replaces the effect of his durable ability.
        In addition, he gains a \plus1 bonus to Mental defense.

        \cf{Bbn}[10th]{Instinctive Rage}[Ex]
        The barbarian need only spend a \glossterm{free action} to maintain his rage, rather than a \glossterm{swift action}.

        \cf{Bbn}[11th]{Larger than Belief}[Ex]
        The barbarian's larger than life ability improves.
        He gains a \plus2d bonus to damage with \glossterm{strikes}.
        This does not stack with the effect of the larger than life ability.

        \cf{Bbn}[12th]{Titanic Resilience}[Ex]
        The barbarian cannot take more than half his maximum hit points in damage during a single round.
        Any excess damage is ignored.

        \cf{Bbn}[13th]{Fury of the Storm}[Ex]
        A barbarian cannot be overwhelmed.
        He does not suffer overwhelm penalties, regardless of the number of enemies threatening him.

        \cf{Bbn}[14th]{Mindless Rage}[Ex]
        The barbarian is immune to hostile \glossterm{Mind} effects while raging.

        \cf{Bbn}[16th]{Supreme Rage}[Ex]
        The barbarian no longer has any restrictions to his actions while raging.
        He may freely take actions that require concentration.
        In addition, failing to attack while raging no longer causes him to take damage at the end of the round.

        \cf{Bbn}[17th]{Titan of Battle}[Ex]
        The barbarian's larger than life ability improves.
        He gains a \plus3d bonus to damage with \glossterm{strikes}.
        This does not stack with the effect of the larger than life or larger than belief abilities.

        \cf{Bbn}[18th]{Deathless Rage}[Ex]
        While raging, the barbarian ignores all penalties from vital damage, and does not begin dying even if he takes vital damage.
        However, if his vital damage exceeds his maximum hit points, the barbarian immediately dies.
        When his rage ends, if the barbarian has vital damage, he begins dying.

        \cf{Bbn}[19th]{Endless Rage}[Ex]
        A barbarian's rage no longer has a limited duration.
        He may rage indefinitely without stopping.
        In addition, he may rage immediately after finishing a rage, without needing to rest.

        \subsubsection{Ex-Barbarians}
            A barbarian who becomes lawful loses his ability to rage, and cannot gain more levels as a barbarian.
            He retains all his other class abilities.
            If he stops being lawful, he regains his ability to rage and take barbarian levels.

\section{Cleric}\label{Cleric}
    \begin{dtable}
        \lcaption{Cleric Progression}
        \begin{dtabularx}{\columnwidth}{>{\ccol}p{2em} c >{\lcol}X}
            \tb{Level} & \tb{Spells} & \tb{Special} \\
            \hline
            \nth{1}  & 2  & Domain gifts, rituals, spells \\
            \nth{2}  & 3  & Devotion, domain invocation   \\
            \nth{3}  & 3  & Domain invocation                        \\
            \nth{4}  & 4  & \tdash             \\
            \nth{5}  & 4  & Domain aspect                        \\
            \nth{6}  & 5  & \tdash                 \\
            \nth{7}  & 5  & Domain aspect                        \\
            \nth{8}  & 6  & \tdash                 \\
            \nth{9}  & 6  & Intercession                        \\
            \nth{10} & 7  & \tdash                  \\
            \nth{11} & 7  & Greater domain invocation                        \\
            \nth{12} & 8  & \tdash     \\
            \nth{13} & 8  & Greater domain invocation                        \\
            \nth{14} & 9  & \tdash     \\
            \nth{15} & 9  & Domain mastery                        \\
            \nth{16} & 10 & \tdash                \\
            \nth{17} & 10 & Domain mastery                        \\
            \nth{18} & 11 & \tdash                \\
            \nth{19} & 11 & Endless devotion              \\
            \nth{20} & 12 & Miracle                       \\
        \end{dtabularx}
    \end{dtable}

    \classbasics{Alignment} The cleric's alignment must be within one step of his deity's (that is, it may be one step away on either the lawful-chaotic axis or the good-evil axis, but not both).

    \classbasics{Class Skills}
    \subparhead{Intelligence} Heal, Knowledge (arcana, local, religion, the planes), Linguistics.
    \subparhead{Perception} Awareness, Sense Motive, Spellcraft.
    \subparhead{Other} Bluff, Intimidate, Persuasion.

    \subsection{Base Class Abilities}
        A character with cleric as a base class gains the following abilities.

        \classbasics{Skill Points} 5.


        \classbasics{Defenses} \plus2 Fortitude, \plus4 Mental.

        \classbasics{Action Points} \plus3.

        \cf{Clr}{Weapon and Armor Proficiency}
        Clerics are proficient with simple weapons, any two other weapon groups, light and medium armor, and shields.

        \cf{Clr}{Domain Gifts}[Mag]
        A cleric's abilities are shaped by his domains.
        He gains the domain gifts of both of his domains.
        Domain gifts are not activated.
        The gifts offered by each domain are listed at \pcref{Domain Gifts}.

        \cf{Clr}{Enhanced Divine Power}[Mag]
        The cleric gains a \plus2 bonus to his divine power.

    \subsection{Class Abilities}
        All clerics have the following abilities.

        \cf{Clr}{Domains}
        A cleric chooses two domains which represent his personal spiritual inclinations.
        He must choose his domains from among those his deity offers.
        A cleric's choice of domains has broad effects on the cleric's spellcasting and abilities.
        The domains are listed below.

        \begin{itemize}
            \item{Air}
            \item{Chaos}
            \item{Death}
            \item{Destruction}
            \item{Earth}
            \item{Evil}
            \item{Fire}
            \item{Good}
            \item{Knowledge}
            \item{Law}
            \item{Life}
            \item{Magic}
            \item{Protection}
            \item{Strength}
            \item{Travel}
            \item{Trickery}
            \item{War}
            \item{Water}
            \item{Wild}
        \end{itemize}

        \cf{Clr}{Divine Power}
        The strength of a cleric's spells and abilities are determined by his divine power.
        His divine power is equal to his character level.

        \cf{Clr}{Spells}
        A cleric casts divine spells using his devotion.
        The maximum spell level a cleric can learn or cast is equal to half his cleric level.
        A cleric's \glossterm{spellpower} with divine spells is equal to his divine power.

        A cleric begins play knowing two first-level spells.
        Every even level, he learns an additional spell of any level he has access to.
        In addition, each time he gains a level, he may trade one of his existing spells for a different spell known.
        He may learn spells from the divine \glossterm{spell list} (see \pcref{Divine Spells}).
        In addition, he adds the spells from his domains to his divine spell list (see \pcref{Cleric Domains}).
        Sometimes these domain spells are spells that are normally available to divine spellcasters, but often they are only accessible with the domain.

        To cast a spell, a cleric must normally spend an \glossterm{action point}.
        Every spell can also be cast as a cantrip.
        Cantrips are weaker, but do not require action points to cast.

        A cleric can't cast spells of an alignment opposed to his own or his deity's.
        Spells associated with particular alignments are indicated by the chaos, evil, good, and law descriptors in their spell descriptions.

        \begin{dtable!*}
            \lcaption{Deities}
            \begin{dtabularx}{\textwidth}{X l X}
                \tb{Deity} & \tb{Alignment} & \tb{Domains} \\
                \hline
                Guftas, horse god of justice          & Lawful good     & Good, Law, Strength, Travel         \\
                Lucied, paladin god of justice        & Lawful good     & Destruction, Good, Protection, War  \\
                Simor, fighter god of protection      & Lawful good     & Good, Protection, Strength, War     \\
                %Pabst, dwarf god of drink             & Neutral good & Good, Life, Strength, Wild \\
                Rucks, monk god of pragmatism         & Neutral good    & Good, Law, Protection, Travel       \\
                Vanya, centaur god of nature          & Neutral good    & Good, Strength, Travel, Wild        \\
                Brushtwig, pixie god of creativity    & Chaotic good    & Chaos, Good, Trickery, Wild         \\
                Chavi, god of stories                 & Chaotic good    & Chaos, Knowledge, Trickery          \\
                Ivan Ivanovitch, bear god of strength & Chaotic good    & Chaos, Strength, War, Wild          \\
                Krunch, barbarian god of destruction  & Chaotic good    & Destruction, Good, Strength, War    \\
                Sir Cakes, dwarf god of freedom       & Chaotic good    & Chaos, Good, Strength               \\
                Raphael, monk god of retribution      & Lawful neutral  & Death, Law, Protection, Travel      \\
                Declan, god of fire                   & True neutral    & Destruction, Fire, Knowledge, Magic \\
                Kurai, shaman god of nature           & True neutral    & Air, Earth, Fire, Water             \\
                %Amanita, druid god of decay           & Chaotic neutral & Chaos, Destruction, Life, Wild \\
                %Antimony, elf god of necromancy       & Chaotic neutral & Death, Knowledge, Life, Magic \\
                Clockwork, elf god of time            & Chaotic neutral & Chaos, Magic, Trickery, Travel      \\
                %Lord Khallus, fighter god of pride    & Chaotic neutral & Chaos, Strength, War \\
                %Celeano, sorcerer god of deception    & Chaotic neutral & Chaos, Magic, Protection, Trickery \\
                Murdoc, god of mercenaries            & Chaotic neutral & Destruction, Knowledge, Travel, War \\
                Ribo, halfling god of trickery        & Chaotic neutral & Chaos, Trickery, Water              \\
                Tak, orc god of war                   & Lawful evil     & Law, Strength, Trickery, War        \\
                Theodolus, sorcerer god of ambition   & Neutral evil    & Evil, Knowledge, Magic, Trickery    \\
                Daeghul, demon god of slaughter       & Chaotic evil    & Destruction, Evil, Magic, War       \\
            \end{dtabularx}
        \end{dtable!*}

        \cf{Clr}{Rituals}
        Clerics can perform divine rituals to create unique magical effects (see \pcref{Rituals}).
        A cleric begins play with a ritual book containing two divine rituals of his choice (see \pcref{Divine Rituals}).

        \cf{Clr}[2nd]{Domain Invocation}[Mag]
        As a standard action, a cleric can spend an action point to invoke divine power.
        He gains the domain invocations offered by one of his domains.

        All domain invocations affect a single creature within \rngmed range and require a special attack against a defense.
        The cleric's accuracy with domain invocations is equal to his divine power.
        If the attack succeeds, a domain invocation heals or inflicts 1d6 damage per divine power.
        If the attack fails, the invocation heals or inflicts half damage.

        At 3rd level, the cleric gains another domain invocation from one of his domains.

        \cf{Clr}[5th]{Domain Aspect}[Mag]
        The cleric gains a domain aspect from one of his domains.
        Domain aspects do not require an action to activate.
        Options for domain aspects are listed at \pcref{Domain Aspects}.

        At his 7th cleric level, the cleric gains an additional domain aspect from one of his domains.

        \cf{Clr}[9th]{Intercession}[Mag]
        Once per day, the cleric can request a divine intercession as a standard action.
        He mentally specifies his request, and his deity fulfills that request in the manner it sees fit.
        This can emulate the effects of a spell from one of the cleric's domains of a level no greater than half of his level.
        Divine intercessions tend to reflect the personality of the cleric's deity, not the cleric's personal preferences.

        If the cleric performs a significant service for his deity, he can gain the ability to request an additional intercession that day.

        \cf{Clr}[11th]{Greater Domain Invocations}[Mag]
        The cleric gains the ability to invoke the power of one of his domains even more effectively.
        Greater domain invocations are described at \pcref{Greater Domain Invocations}.

        At his 13th cleric level, the cleric gains an additional greater domain invocation from one of his domains.

        \cf{Clr}[15th]{Domain Mastery}[Mag]
        The cleric gains a domain mastery from one of his domains.
        Options for domain masteries are listed at \pcref{Domain Masteries}.

        At his 17th cleric level, the cleric gains an additional domain mastery from one of his domains.

        \cf{Clr}[19th]{Endless Devotion}[Mag]
        Whenever the cleric casts a spell from one of his domains, if he did not spend any devotion points that round, he regains a spent devotion point.

        \cf{Clr}[20th]{Miracle}[Mag]
        Once per week, the cleric can request a miracle as a standard action.
        He mentally specifies his request, and his deity fulfills that request in the manner it sees fit.
        This can emulate the effects of any spell or ritual, or have any other effect of a similar power level.
        If the deity has a direct interest in the cleric's situation, the miracle may be of even greater power.

        If the cleric performs an extraordinary service for his deity, he can gain the ability to request an additional miracle that week.

    \subsection{Cleric Domain Abilities}
        All cleric domain abilities are \glossterm{magical} unless otherwise specified.

        \subsubsection{Domain Gifts}\label{Domain Gifts}
            % If a domain gift involves an attack, the accuracy is equal to the cleric's divine power.

            \subcf{Air}
            The cleric adds the Jump skill (see \pcref{Jump}) to his cleric class skill list, and gains a \plus5 bonus to Jump checks.
            \subcf{Chaos}
            The cleric rolls twice for all \glossterm{random effects} and chooses his preferred result.
            \subcf{Death}
            The cleric halves his penalties from \glossterm{vital damage}.
            In addition, he is immune to \glossterm{Death} effects.
            \subcf{Destruction}
            When making physical attacks, the cleric ignores an amount of \glossterm{hardness} and \glossterm{damage reduction} equal to half his divine power.
            \subcf{Earth}
            The cleric gains the \glossterm{tremorsense} ability with a range of 50 feet.
            If he is touching a surface, he can automatically pinpoint the location of anything within 50 feet that is in contact with the surface, including inanimate objects.
            \subcf{Evil}
            The cleric gains \glossterm{damage reduction} against physical damage from non-evil sources equal to half his divine power.
            \subcf{Fire}
            The cleric gains \glossterm{damage reduction} against fire and cold damage equal to his divine power.
            \subcf{Good}
            The cleric gains \glossterm{damage reduction} against physical damage from non-good sources equal to half his divine power.
            \subcf{Knowledge}
            The cleric adds all Knowledge skills to his cleric class skill list.
            In addition, he gains two skill points which must be spent on Knowledge skills.
            \subcf{Law}
            The cleric gains a \plus2 bonus to Mental defense.
            \subcf{Life}
            The cleric gains a \plus2 bonus to Fortitude defense.
            \subcf{Magic}
            The cleric gains \glossterm{damage reduction} against \glossterm{magical} damage equal to half his divine power.
            \subcf{Protection}
            As an \glossterm{immediate action}, when an ally adjacent to the cleric takes damage, the cleric can take half that damage instead of the ally.
            \subcf{Strength}
            The cleric adds Climb, Jump, Sprint, and Swim to his cleric class skill list.
            In addition, he gains two skill points which must be spent on Strength-based skills.
            \subcf{Travel}
            The cleric adds Knowledge (geography), Sprint, and Survival to his cleric class skill list.
            In addition, he gains two skill points which must be spent on any combination of those skills.
            \subcf{Trickery}
            The cleric adds Bluff, Disguise, and Stealth to his cleric class skill list.
            In addition, he gains two skill points which must be spent on any combination of those skills.
            \subcf{War}
            The cleric gains a \plus1 bonus to damage with physical attacks.
            \subcf{Water}
            The cleric adds Swim to his cleric class skill list, gains a \plus5 bonus to Swim checks, and suffers no penalties for fighting underwater.
            \subcf{Wild}
            The cleric adds Creature Handling, Knowledge (nature), and Survival to his cleric class skill list.
            In addition, he gains two skill points which must be spent on any combination of those skills.

        \subsubsection{Domain Invocations}\label{Domain Invocations}
            All domain invocations affect a single creature within \rngmed range and require a special attack against a defense.
            If the attack succeeds, a domain invocation heals or inflicts 1d6 damage per divine power.
            If the attack fails, the invocation heals or inflicts half damage.

            \subcf{Air -- Reflex}
            The target takes electricity damage.
            \subcf{Chaos -- Mental}
            This invocation randomly heals or inflicts divine damage.
            The cleric chooses the target after rolling to determine the effect.
            \subcf{Death -- Fortitude}
            The target takes divine damage.
            If this attack deals vital damage, the target is instantly killed.
            This is a death effect.
            \subcf{Destruction -- Fortitude}
            The target takes sonic damage.
            \subcf{Earth -- Reflex}
            The target takes bludgeoning damage if it is on the ground.
            \subcf{Evil -- Mental}
            This invocation does not heal or inflict damage.
            If the attack succeeds, and the target is non-evil, it is \staggered for 2 rounds.
            \subcf{Fire -- Reflex}
            The target takes fire damage.
            \subcf{Good -- Mental}
            This invocation does not heal or inflict damage.
            If the attack succeeds, and the target is non-good, it is \dazed for 2 rounds.
            \subcf{Knowledge -- Special}
            The target must make a Knowledge check.
            If its check result beats your attack result, you can choose to heal it.
            If it fails, you can choose to deal divine damage to it.
            This invocation heals or inflicts 1d10 damage per two divine power instead of the normal value.
            \subcf{Law -- Mental}
            This invocation does not heal or inflict damage.
            If the attack succeeds, and the target is non-lawful, it is \immobilized for 2 rounds.
            \subcf{Life -- Fortitude}
            The target is healed.
            This invocation heals 1d8 damage per divine power instead of the normal value.
            \subcf{Magic -- Mental}
            If the target can cast spells, it is healed.
            Otherwise, it takes divine damage.
            This invocation heals or inflicts 1d10 damage per two divine power instead of the normal value.
            \subcf{Protection -- None}
            This invocation does not heal or inflict damage, and no attack is required.
            The target gains 1d10 \glossterm{temporary hit points} per two divine power.
            \subcf{Strength -- Special}
            The target must make a Strength check.
            If its check result beats your attack result, you can choose to heal it.
            If it fails, you can choose to deal divine damage to it.
            This invocation heals or inflicts 1d10 damage per two divine power instead of the normal value.
            \subcf{Travel -- None}
            This invocation does not heal or inflict damage, and no attack is required.
            The target gains a \plus30 foot bonus to its speed in all its movement modes, up to a maximum of double its original speed.
            This effect lasts for 5 rounds.
            \subcf{Trickery -- Mental}
            The target takes divine damage.
            This is a \glossterm{Subtle} effect, making it hard to notice.
            This invocation heals or inflicts 1d10 damage per two divine power instead of the normal value.
            \subcf{War -- Fortitude}
            This invocation affects all enemies within a \areasmall radius of you instead of the normal target.
            The targets take divine damage.
            This invocation deals 1d8 damage per two divine power instead of the normal value.
            \subcf{Water -- Fortitude}
            The target takes physical \glossterm{nonlethal damage} from water in its mouth and lungs.
            In addition, if the attack succeeds, the target is unable to speak for 1 round.
            \subcf{Wild -- Fortitude}
            The target takes divine damage.
            If the target is an animal or plant, the cleric may choose to heal it instead.

        \subsubsection{Domain Aspects}\label{Domain Aspects}

            \subcf{Air -- Limited Flight}
            The cleric gains a glide speed equal to his land speed.
            See \pcref{Gliding}, for more details.
            In addition, as a swift action, he can spend a devotion point to treat air as if it was solid ground until the end of the round.
            He can only do this once before touching solid ground again.
            \subcf{Chaos -- Chaotic Retribution}
            Whenever a non-chaotic creature within 30 feet of you attacks you, it takes 1d6 damage per two divine power.
            A creature can only be dealt damage by this effect once per round.
            \subcf{Death -- Deathcaller}
            Whenever the cleric damages a creature, any of his damage in excess of that creature's hit points is dealt as \glossterm{vital damage}.
            \subcf{Destruction -- Ruinbringer}
            The cleric's attacks ignore an amount of \glossterm{damage reduction} and \glossterm{hardness} equal to his divine power.
            \subcf{Earth -- Hardened Skin}
            The cleric gains \glossterm{damage reduction} against physical damage equal to his divine power.
            Adamantine weapons ignore this damage reduction and negate it for 1 round.
            \subcf{Evil -- Unholy Retribution}
            Whenever a non-evil creature within 30 feet of you attacks you, it takes 1d6 damage per two divine power.
            A creature can only be dealt damage by this effect once per round.
            \subcf{Fire -- Friendly Fire}
            All of the cleric's fire spells and abilities do not deal fire damage to his allies.
            In addition, whenever the cleric would take fire damage, he may heal that many hit points instead.
            He may use this ability before applying damage reduction, damage immunity, and similar effects.
            \subcf{Good -- Holy Retribution}
            Whenever a non-good creature within 30 feet of you attacks you, it takes 1d6 damage per two divine power.
            A creature can only be dealt damage by this effect once per round.
            \subcf{Knowledge -- Knowledge Mastery}
            The cleric gains the Knowledge Mastery feat (see page \featpref{Knowledge Mastery}).
            \subcf{Law -- Certain Retribution}
            Whenever a non-lawful creature within 30 feet of you attacks you, it takes 1d6 damage per two divine power.
            A creature can only be dealt damage by this effect once per round.
            \subcf{Life -- Critical Healer}
            All of the cleric's healing spells and abilities cure vital damage as easily as they cure hit points.
            \subcf{Magic -- Improved Spellpower}
            The cleric gains a \plus1 bonus to spellpower with divine spells.
            \subcf{Protection -- Faithful Shield}
            The cleric may maintain concentration on \glossterm{Shielding} spells as a swift action.
            \subcf{Strength -- Mighty Devotion}
            The cleric may use his Strength in place of his divine power for the purpose of invocations he uses.
            In addition, he may use his Strength in place of his Willpower to determine the number of devotion points he has.
            \subcf{Travel -- Rapid Traveller}
            The cleric gains a \plus30 foot bonus to his speed in all movement modes, up to a maximum of double his normal speed.
            \subcf{Trickery -- Trick Master}
            The cleric gains his choice of Bluff Mastery, Disguise Mastery, or Stealth Mastery as a bonus feat, even if he does not meet the prerequisites.
            \subcf{War -- Warpriest}
            The cleric gains a \plus2 bonus to damage with physical attacks.
            \subcf{Water -- Water Breathing} The cleric can breathe water as easily as a human breathes air, preventing him from drowning or suffocating underwater.
            He also gains a \glossterm{swim speed} equal to his land speed.
            \subcf{Wild -- Wild Aspect}
            When the cleric gains this ability, he chooses one wild aspect ability, as if he were a were a druid of a level equal to his cleric level (see Wild Aspect, \pref{Drd:Wild Aspect}).
            He can spend a devotion point as a standard action to embody that wild aspect for 1 hour.

        \subsubsection{Greater Domain Invocations}\label{Greater Domain Invocations}

            \subcf{Air -- Command Air}
            As a standard action, the cleric can spend a devotion point to speak with and command air for 5 minutes.
            He can ask the air simple questions and understand its responses.
            If he commands the air to perform a task, it will do so do the best of its ability until the end of the effect's duration.
            The cleric cannot compel the air to go faster than 50 mph, and cannot affect air farther than 500 feet from him.

            \subcf{Chaos -- Sow Chaos}
            As a standard action, the cleric can spend a devotion point to cause an improbable event to occur.
            He can visualize in general terms what he wants to happen, such as ``Make the bartender leave the bar''.
            He cannot control exact nature of the event, though it always beneficial for him in some way.

            \subcf{Death -- Reaper's Boon}
            As a standard action, the cleric can spend a devotion point to summon or banish Death.
            In either case, it affects a living creature within 100 feet of him for 5 rounds.
            If he summons Death, the target immediately dies if it takes vital damage.
            If he banishes Death, the target is immune to Death effects and does not make \glossterm{stabilization rolls} if it takes vital damage.
            This does not prevent the target from taking vital damage, and it begins dying after the effect ends if it has took vital damage and has not been stabilized.

            \subcf{Destruction -- Dust to Dust}
            As a standard action, the cleric can spend a devotion point to destroy objects within a \arealarge radius burst centered on him.
            He makes a Divine power vs. Mental attack against all objects in the area.
            Success against an object means it crumbles into dust, and is irreparably broken.
            Unattended nonmagical objects do not have a Mental defense, and are automatically broken.
            The cleric may freely exclude any objects or squares from the effect.

            \subcf{Earth -- Command Earth}
            As a standard action, the cleric can spend a devotion point to speak with and command earth for 5 minutes.
            He can ask the earth simple questions and understand its responses.
            If he commands the earth to perform a task, it will do so do the best of its ability until the end of the effect's duration.
            The cleric cannot compel the earth to move faster than 10 feet per round, and cannot effect earth farther than 500 feet from him.

            \subcf{Evil -- Temptation}
            As a standard action, the cleric can spend a devotion point to compel a creature to commit an evil act.
            He makes a Divine power vs. Mental attack against the creature.
            Success means the target takes an evil action as soon as it can.
            The cleric has no control over the act the creature takes, but circumstances can make the target more likely to take an action the cleric desires.
            Creatures who have strict codes prohibiting them from taking evil actions, such as paladins devoted to Good, are immune to this effect.
            This is a [\glossterm{Compulsion}, \glossterm{Mind}] effect.

            \subcf{Fire -- Command Flames}
            As a standard action, the cleric can spend a devotion point to speak with and command fire for 5 minutes.
            He can ask the fire simple questions and understand its responses.
            If he commands the fire to perform a task, it will do so do the best of its ability until the end of the effect's duration.
            The fire can move up to 30 feet in a single round between combustible materials.
            The cleric cannot effect fire farther than 500 feet from him.

            \subcf{Good -- Salvation}
            As a standard action, the cleric can spend a devotion point to compel a creature to do a good deed.
            He makes a Divine power vs. Mental attack against the creature.
            Success means the target takes a good action as soon as it can.
            The cleric has no control over the act the creature takes, but circumstances can make the target more likely to take an action the cleric desires.
            Creatures who have strict codes prohibiting them from taking good actions, such as paladins devoted to Evil, are immune to this effect.
            This is a [\glossterm{Compulsion}, \glossterm{Mind}] effect.

            \subcf{Knowledge -- Impart Truth}
            As a standard action, the cleric can spend a devotion point to grant knowledge.
            He may make a Knowledge check of any kind with a \plus20 bonus to the check.
            He may also cause any number of creatures within a \arealarge radius around him to learn the results of his check.
            Creatures granted knowledge in this way believe the information to be true as if they had seen it with their own eyes.
            Exceptionally stubborn or untrusting creatures may still not be convinced of its truth, however.

            \subcf{Law -- Infallible Enforcement}
            As a standard action, the cleric can spend a devotion point to enforce the law in a \arealarge radius \glossterm{zone} centered on him.
            He makes a Divine power vs. Mental attack against all creatures in the area.
            Success means the target is unable to break the law, and any attempt to do so simply fails.
            Failure means the target feels a compulsion not to break the law, but is able to overcome the compulsion if it desires.
            The laws which are applied are those which are most appropriate for the area, regardless of whether the cleric or any other creature know those laws.
            If the rightful laws are inconsistent or impossible to understand, those laws may not be enforced.
            This is a [\glossterm{Compulsion}, \glossterm{Mind}] effect.

            \subcf{Life -- Prayer of Resurrection}
            As a standard action, the cleric can resurrect a touched creature, as the effect of the \spell{resurrection} ritual.
            The target must have been dead for no more than 5 minutes.
            He must spend a number of devotion points equal to the target's level.
            He does not know the cost before resurrecting the target.
            If the cost exceeds his remaining devotion points, all his devotion points are spent, but the resurrection still succeeds as long as he spent at least one devotion point in this way.

            \subcf{Magic -- Manipulate Spell}
            As a standard action, the cleric can spend a devotion point to manipulate a currently active spell within \rngmed range.
            He must make a Divine power check against a DR equal to 10 \add the spell's spellpower.
            Success means he identifies the spell perfectly, if he had not already done so, and can take one of the following four actions on the spell.
            \begin{itemize}
                \item Control: If the spell can be focused on to gain an effect or extend its duration, the cleric gains the ability to focus on the spell as if he was the one who originally cast it.
                    Its original caster loses the ability to focus on the spell.
                \item Dispel: The spell is dispelled, if it can be dispelled by \spell{dispel magic}.
                \item Persist: The spell's remaining duration increases by 5 minutes, up to a maximum of its starting duration (ignoring any duration increase from focusing on the spell).
                \item Suppress: The spell is suppressed for 5 rounds, if it can be dispelled by \spell{dispel magic}.
                    At the end of that time, the spell's effect resumes, if it still has duration remaining.
            \end{itemize}

            \subcf{Protection -- Divine Shield}
            As a standard action, the cleric can create a powerful protective shield around a creature or object within \rngclose range for 5 rounds.
            The target takes half damage from all attacks.
            In addition, whenever the target takes damage, the cleric can spend a devotion point as an immediate action to negate that damage.

            \subcf{Strength -- Might of the Gods}
            As a standard action, the cleric can spend a devotion point to gain titanic strength for 5 minutes.
            For the purpose of checks and determining carrying capacity, the cleric's Strength becomes equal to 10 \add his divine power.
            If he takes damage, the effect ends.

            \subcf{Travel -- Transcend Movement}
            As a standard action, the cleric can spend a devotion point to teleport himself, as the effect of the \spell{teleport} ritual.

            \subcf{Trickery -- Enduring Falsehood}
            As a standard action, the cleric can delude a creature within \rngmed range into believing a lie, regardless of evidence.
            He chooses a falsehood and makes a Divine power vs. Mental attack against the creature.
            The falsehood may be a lie, or a simple misunderstanding (such as believing a hidden creature is not present in a room).
            If the creature does not already believe the falsehood, the attack automatically fails.

            Success means that the target continues to believe the falsehood for 5 minutes, regardless of any evidence to the contrary.
            It will interpret any evidence that the falsehood is incorrect to be somehow wrong -- an illusion, a conspiracy to decieve it, or any other reason it can think of to continue believing the falsehood.
            At the end of the effect, the creature can decide whether it believes the falsehood or not, as normal.
            This is a [\glossterm{Delusion}, \glossterm{Mind}] effect.

            \subcf{War -- Mass Combat}
            As an immediate action, the cleric can spend a devotion point to augment a spell he casts with one of the following effects.
            \begin{itemize}
                \item Legion: If the spell would normally affect five or more specific targets, its range is doubled and it instead affects five times that many targets.
                \item Selective: If the spell has an area, it has no effect on his allies in the area.
                \item Widened: If the spell has an area, the size of the area is doubled.
            \end{itemize}

            \subcf{Water -- Command Water}
            As a standard action, the cleric can spend a devotion point to speak with and command water for 5 minutes.
            He can ask the water simple questions and understand its responses.
            If he commands the water to perform a task, it will do so do the best of its ability until the end of the effect's duration.
            The cleric cannot compel the earth to move faster than 30 feet per round, and cannot effect water farther than 500 feet from him.

        \subsubsection{Domain Masteries}\label{Domain Masteries}

            \subcf{Air -- Flight}
            The cleric gains a fly speed with good maneuverability equal to his land speed.
            While \unencumbered, he can fly.
            See \pcref{Flying}, for more details.
            \subcf{Chaos -- Avatar of Luck}
            Once per round, the cleric can gain a \plus1d6 bonus to any check or physical attack.
            He may declare the use of the ability after failing the roll, but before any additional effects are resolved, potentially making it succeed where it would have failed.
            \subcf{Death -- Deathfeeder}
            The cleric constantly radiates a \arealarge radius emanation of death.
            Whenever an enemy takes damage in the area, all damage in excess of its hit points is dealt as \glossterm{vital damage}.
            In addition, whenever a creature dies within the area, the cleric gains temporary hit points equal to twice his divine power.
            \subcf{Destruction -- Beacon of Destruction}
            The cleric constantly radiates a \arealarge radius emanation of destruction.
            All enemies and objects in the area have their \glossterm{damage reduction} and \glossterm{hardness} reduced by an amount equal to twice the cleric's divine power.
            \subcf{Earth -- Earth Glide}
            The cleric gains the earth glide ability, as an earth elemental.
            \subcf{Evil -- Avatar of Evil} The cleric continuously gains the benefits of the \spell{protection from good}
            spell, with a spellpower equal to his divine power.
            If the effect is dispelled or suppressed, he can resume it as a swift action.
            \subcf{Fire -- Flaming Soul}
            The cleric becomes immune to fire damage.
            In addition, whenever he deals fire damage to a creature, that creature is \ignited for 5 rounds.
            \subcf{Good -- Avatar of Good} The cleric continuously gains the benefits of the \spell{protection from evil}
            spell, with a spellpower equal to his divine power.
            If the effect is dispelled or suppressed, he can resume it as a swift action.
            \subcf{Knowledge -- Combat Insight}
            The cleric gains a \plus2 bonus to accuracy, checks, and defenses against non-humanoid creatures he has identified with a successful Knowledge check.
            \subcf{Law -- Avatar of Order}
            Once per round, if the cleric rolls less than a 10 on a d20, he may treat the result as if it were a 10, potentially causing him to succeed where he would have failed.
            You must declare the use of this ability before any additional effects from the roll are resolved.
            \subcf{Life -- }
            \subcf{Magic -- Spellfeeder}
            The cleric gains \glossterm{magic resistance} equal to 10 \add divine power.
            Whenever the cleric resists a spell with this magic resistance, he regains hit points equal to his divine power.
            \subcf{Protection -- Martyr's Boon}
            The cleric constantly radiates a \areamed radius emanation of protective energy.
            Whenever a creature within the area would lose hit points, the cleric can choose to protect it.
            If he does, the protected creature instead loses half that many hit points (rounded down), and he loses the other half (rounded up).
            This effect applies after damage reduction and all other similar effects, and hit point loss caused by this effect cannot be reduced in any way.
            If he takes damage in excess of his hit points in this way, the excess damage is dealt directly as vital damage.
            \subcf{Strength -- Might of the Gods}
            The cleric gains the larger than life ability, as the barbarian class ability (see Larger than Life, \pref{Bbn:Larger than Life}).
            \subcf{Travel -- Perfect Stride}
            The cleric is immune to effects that restrict its mobility. In addition, he gains a \plus20 bonus to Reflex defense against grapple attacks, as well as on grapple attacks or Escape Artist checks made to escape a grapple or a pin.
            \subcf{Trickery -- Exemplar of Deceit} The cleric continuously gains the benefits of the \spell{nondetection}
            spell, with a spellpower equal to his divine power, except that it also protects him from any spells or effects which would prevent him from lying or reveal his lies.
            If the effect is dispelled or suppressed, he can resume it as a swift action.
            \subcf{War -- Warmaster's Favor} The cleric continuously gains the benefits of the \spell{divine favor}
            spell, with a spellpower equal to his divine power.
            If the effect is dispelled or suppressed, he can resume it as a swift action.
            \subcf{Water -- Water's Flow}
            As a swift action, the cleric can transform himself into a rushing flow of water with a volume roughly equal to his normal volume until the end of his turn.
            In this form, he may move wherever water could go, but he cannot take other actions, such as jumping, attacking, or casting spells.
            His speed is halved when moving uphill and doubled when moving downhill.
            He may move through squares occupied by creatures or threatened by blocking enemies without penalty.
            He may return to his normal form as a free action.
            \par If the water is split, he may reform from anywhere the water has reached, to as little as a single ounce of water.
            If not even an ounce of water exists contiguously, his body reforms from the largest available parts of water, cut into pieces of appropriate size.
            This usually causes the cleric to die.
            \subcf{Wild -- \tdash}

        \subsubsection{Ex-Clerics}
            A cleric who grossly violates the code of conduct required by his deity loses all spells and magical cleric class abilities.
            He cannot thereafter gain levels as a cleric of that god until he atones (see the \spell{atonement} spell description).

\section{Druid}\label{Druid}
    \begin{dtable}
        \lcaption{Druid Progression}
        \begin{dtabularx}{\columnwidth}{>{\ccol}p{\levelcol} >{\ccol}p{3.5em} c >{\lcol}X}
            \tb{Level} & \tb{Active Aspects} & \tb{Spells} & \tb{Special} \\
            \hline
            \nth{1}  & \tdash & 2  & Spells, rituals, wild speech \\
            \nth{2}  & 1      & 3  & Wild aspect                  \\
            \nth{3}  & 1      & 3  & Wild aspect                  \\
            \nth{4}  & 1      & 4  & \tdash                       \\
            \nth{5}  & 2      & 4  & Wild aspect                  \\
            \nth{6}  & 2      & 5  & \tdash                       \\
            \nth{7}  & 2      & 5  & Wild aspect                  \\
            \nth{8}  & 2      & 6  & \tdash                       \\
            \nth{9}  & 3      & 6  & Wild aspect                  \\
            \nth{10} & 3      & 7  & \tdash                       \\
            \nth{11} & 3      & 7  & Wild aspect                  \\
            \nth{12} & 3      & 8  & \tdash                       \\
            \nth{13} & 4      & 8  & Wild aspect                  \\
            \nth{14} & 4      & 9  & \tdash                       \\
            \nth{15} & 4      & 9  & Wild aspect                  \\
            \nth{16} & 4      & 10 & \tdash                       \\
            \nth{17} & 5      & 10 & Wild aspect                  \\
            \nth{18} & 5      & 11 & \tdash                       \\
            \nth{19} & 5      & 11 & Wild aspect                  \\
            \nth{20} & 5      & 12 & Avatar of nature             \\
        \end{dtabularx}
    \end{dtable}

    \classbasics{Alignment} Neutral good, lawful neutral, neutral, chaotic neutral, or neutral evil.

    \classbasics{Class Skills}
    \subparhead{Strength} Climb, Jump, Sprint, Swim.
    \subparhead{Dexterity} Acrobatics, Ride, Stealth.
    \subparhead{Intelligence} Heal, Knowledge (geography, nature).
    \subparhead{Perception} Awareness, Creature Handling, Survival.
    \subparhead{Other} Bluff, Intimidate, Persuasion.

    \subsection{Base Class Abilities}
        A character with druid as a base class gains the following abilities.

        \classbasics{Skill Points} 10.

        \classbasics{Defenses} \plus4 Fortitude, \plus2 Mental.

        \classbasics{Action Points} \plus3.

        \cf{Drd}{Weapon and Armor Proficiency}
        Druids are proficient with simple weapons, any one other weapon group, scimitars, sickles, and slings.
        In addition, druids are proficient with light armor, medium armor, and shields.
        However, a druid cannot use metal armor; see the Metal Abhorrence ability, below.

        \cf{Drd}{Druidic Language}
        Druids know Druidic, a secret language known only to druids, in addition to their normal languages.
        Druids are forbidden to teach this language to nondruids.
        Druidic has its own alphabet.

        \cf{Drd}{Wild Speech}[Mag]
        Druids can communicate with animals.
        As a standard action, the druid can spend an action point and choose a type of animal, such as owl or wolf.
        She gains the ability to speak to and understand animals of that type for 5 minutes.

        This ability doesn't make the animals any more friendly or cooperative than normal.
        Furthermore, wary and cunning animals are likely to be terse and evasive, while the more stupid ones make inane comments.
        If an animal is friendly toward the druid, she may be able to convince it to do some favor or service.

        \cf{Drd}{Enhanced Nature Power}
        The druid gains a \plus2 bonus to her nature power.

    \subsection{Class Abilities}
        All druids have the following abilities.

        \cf{Drd}{Metal Abhorrence}[Mag]
        The oaths that druids swear as part of their initiation prohibit them from wearing armor made of metal.
        A druid who wears prohibited armor or carries a prohibited shield is unable to cast druid spells or use any of her \glossterm{magical} class abilities while doing so and for 24 hours thereafter (not including this ability).
        
        A druid can avoid this penalty by using armor made of wood altered with the \spell{ironwood} ritual.
        Such wood is as strong as steel.

        \cf{Drd}{Nature Power}
        The strength of a druid's spells and abilities are determined by her connection to nature.
        Her nature power is equal to her character level.

        \cf{Drd}{Spells}
        A druid casts nature spells using her connection to nature.
        The maximum spell level a druid can learn or cast is equal to half her druid level.
        A druid's \glossterm{spellpower} with nature spells is equal to her nature power.

        A druid begins play knowing two first-level spells.
        Every even level, she learns an additional spell of any level she has access to.
        In addition, each time she gains a level, she may trade one of her existing spells for a different spell known.
        A druid's spells are drawn from the spells on the nature spell list (see \pcref{Nature Spells}).

        To cast a spell, a druid must spend an \glossterm{action point}.
        Every spell can also be cast as a cantrip.
        Cantrips are weaker, but do not require action points to cast.

        \cf{Drd}{Rituals}
        Druids can perform nature rituals to create unique magical effects (see \pcref{Rituals}).
        A druid begins play with a ritual book containing two nature rituals of her choice (see \pcref{Nature Rituals}).

        \cf{Drd}[2nd]{Wild Aspect}[Mag]
        The druid gains the ability to embody an aspect of an animal or of nature itself.
        She chooses two wild aspects from the list below.
        Some wild aspects have minimum druid levels, as indicated in the title of the aspect.
        At her 3rd druid level, and every odd druid level thereafter, the druid learns an additional wild aspect.

        Unless otherwise noted, embodying a wild aspect is a standard action.
        Embodying a wild aspect costs an \glossterm{action point}.
        Once embodied, a wild aspect persists until the druid embodies a new aspect or dismisses the aspect.
        If the druid embodies a new wild aspect, the previous aspect ends immediately.
        All wild aspects can be dismissed as a swift action.

        The descriptions below describe the effects of the aspect.
        With many aspects, the druid's appearance also changes to match the aspect, but this is not described.
        Different druids change in different ways.
        For example, one druid might gain unusually large eyes when embodying the low-light vision aspect, while another might change her irises into slits, like a cat, when embodying the same aspect.
        The changes made are up to the druid, but cannot be used to gain an additional substantive benefit beyond the effects given in the description of the aspect.

        Many wild aspects grant natural weapons.
        See \pcref{Natural Weapons}, for details about natural weapons.

        \subcf{Armaments of the Bear}
        The druid's mouth and hands transform, allowing her to perform bite and claw attacks.
        The bite attack deals 1d8 damage for a Medium druid, and the claws deal 1d6 damage.
        \subcf{Senses}
        The druid, gains low-light vision.
        She treats sources of light as if they had double their normal illumination range.
        If she already has low-light vision, she doubles its benefit, allowing her to treat sources of light as if they had four times their normal illumination range.
        In addition, she gains \glossterm{darkvision} out to 50 feet, allowing her to see in complete darkness.
        If she already has darkvision, she increases its range by 50 feet.
        \subcf{Woodland Stride}
        The druid may move through any sort of undergrowth (such as natural thorns, briars, overgrown areas, and similar terrain) at her normal speed and without taking damage or suffering any other impairment.
        The plants bend of their own volition to allow the druid to pass.
        However, plants magically manipulated to impede motion still affect her.

        \subcf{5th -- Animal Affinity}
        The druid gains a \plus5 bonus to Creature Handling and Ride checks.
        \subcf{5th -- Climb}
        The druid gains a \glossterm{climb speed} equal to her land speed.
        \subcf{5th -- Constrict}
        The druid's body transforms, improving her grappling abilities.
        She gains a \plus5 bonus to accuracy with grapple attacks.
        In addition, she gains a constrict attack.
        This attack deals 1d10 damage for a Medium druid, but it can only be used against a foe she is grappling with.
        \subcf{5th -- Gore}
        The druid's head transforms, allowing her to perform a gore attack.
        The attack deals 1d8 damage for a Medium druid.
        In addition, if the druid hits with a natural attack, she may attempt to shove her foe as an immediate action.

        \subcf{7th -- A Thousand Faces}
        The druid's appearance changes, as if using the \spell{disguise self} spell.
        This affects the druid's body, but not her possessions.
        It is not an illusory effect, but a minor physical alteration of the druid's appearance, within the limits described for the spell.
        \subcf{7th -- Enhanced Natural Weapons}
        The druid's natural weapons gain a \glossterm{enhancement bonus} equal to one third of her nature power.
        This functions like an enhancement bonus on a weapon, increasing her damage and offensive legend points per day (see \pcref{Weapon Enhancement Bonuses}).
        \subcf{7th -- Hawk}
        The druid grows wings, granting her a glide speed equal to her land speed.
        See \pcref{Gliding}, for more details.
        In addition, her feet transform, allowing her to perform a talon attack.
        The attack deals 1d6 damage for a Medium druid.
        \subcf{7th -- Lope}
        The druid gains the ability to move on all four limbs.
        When doing so, she gains a \plus30 foot bonus to her land speed, up to a maximum of double her original speed.
        When not using her hands to move, her ability to use her hands is unchanged.
        \subcf{7th -- Scent}
        The druid gains the \glossterm{scent} ability.
        \subcf{7th -- Shrink}
        The druid shrinks by a size category.
        This functions like the \spell{reduce person} spell.
        This is a sizing effect.
        \subcf{7th -- Slither}
        The druid gains a \glossterm{climb speed} equal to her land speed.
        She does not need to use her hands to climb in this way.
        In addition, she gains a bite attack that deals 1d8 damage for a Medium druid.
        \subcf{7th -- Spikes}
        Whenever a creature adjacent to the druid makes a physical attack against her, the attacking creature takes 1d6 piercing damage per two nature power.
        A creature can only be dealt damage by this effect once per round.

        \subcf{9th -- Barkskin} The druid gains \glossterm{damage reduction} against physical damage equal to her nature power.
        Fire damage ignores this damage reduction and negates it for 1 round.
        \subcf{9th -- Natural Grab}
        If the druid hits with a natural attack, she may attempt to grapple her foe as an immediate action.
        \subcf{9th -- Natural Trip}
        If the druid hits with a natural attack, she may attempt to trip her foe as an immediate action.
        \subcf{9th -- Venom}
        If the druid hits with a natural attack, she may inject poison into her foe as an immediate action.
        At the end of every round, the druid makes a nature power vs. Fortitude attack against all creatures she has poisoned.
        The effects of the poison are described below.
        \begin{itemize}
            \item First success: the target is \sickened.
            \item Second success: the target is \staggered.
            \item Third success: the target is \nauseated.
            \item Third failure: the target is no longer poisoned, and any lingering effects from the poison end.
        \end{itemize}
        %TODO poison stacking rules
        \par In addition, the druid gains a bite attack that deals 1d8 damage for a Medium druid.

        \subcf{11th -- Elemental Retribution}
        Whenever a creature within \rngmed range of the druid attacks her, the attacking creature takes 1d6 damage per two nature power of either cold, electricity, or fire damage.
        The druid may choose the damage type independently for each attacking creature.
        A creature can only be dealt damage by this effect once per round.
        \subcf{11th -- Fluid Motion} The druid is immune to effects that restrict its mobility. She suffers no penalties for acting underwater. In addition, she gains a \plus20 bonus to Reflex defense against grapple attacks, as well as on grapple attacks or Escape Artist checks made to escape a grapple or a pin.
        \subcf{11th -- Grow}
        The druid increases in size by one size category, as the effect of the \spell{enlarge} spell.
        This is a \glossterm{Sizing} effect, and does not stack with other Sizing effects.
        \subcf{11th -- Natural Renewal}
        At the end of each round, the druid heals hit points equal to her nature power.
        \subcf{11th -- Wolfpack}
        Overwhelmed foes the druid threatens increase their overwhelm penalties by 2.

        \subcf{13th -- Swiftstrike}
        The druid's attack speed increases.
        When she makes a standard attack, she may make an additional strike.
        This strike must be made with a natural weapon.
        This effect does not stack with similar effects that grant extra strikes.
        \subcf{13th -- Wings}
        The druid grows wings, granting her a fly speed equal to her land speed.
        While \unencumbered, she can fly (see \pcref{Flying}).
        She can only fly for a number of rounds equal to half her nature power.
        After that limit is reached, she must rest for 5 minutes before flying again.

        \subcf{15th -- Earth Glide}
        The druid gains the earth glide ability, as an earth elemental.
        This allows her to glide through stone, dirt, or almost any other sort of earth as if it were air.
        She can walk or climb at any angle in the earth.
        However, she cannot breathe, speak, or hear while gliding in this way.
        While gliding, she can remain partially within the earth, granting it cover.
        She can only glide through earth for a number of rounds equal to half her nature power.
        After that limit is reached, she must rest for 5 minutes before gliding through earth again.

        \subcf{19th -- Solar Radiance}
        The druid continuously radiates bright light out to a 500 foot radius (and shadowy illumination for an additional 500 feet).
        The illumination is so bright that she becomes hard to look at.
        Any creature attacking her from within the radius of bright light becomes \partiallyblinded for 2 rounds after the attack.

\begin{comment}
    \subcf{15th -- Air Mantle}
    The druid is surrounded by a mantle of air.
    Thrown and projectile weapons have a 50\% chance to miss her while this effect is active.
    Unusually large weapons, such as a giant's boulders, may suffer a decreased miss chance as appropriate to their size.
    \subcf{15th -- Aqueous Step}
    Wherever the druid moves, she leaves a path of animated water that can grab creatures.
    Whenever a creature crosses the path, the druid makes a Reflex attack to trip the creature, causing it to fall prone and waste the rest of its movement.
    Her accuracy is equal to her druid level \add her Constitution.
    \subcf{15th -- Flaming Step}
    Wherever the druid moves, she leaves a path of burning flame behind her that lasts for 1 round.
    Whenever a creature crosses the path, the druid makes a Reflex attack to deal damage to the creature.
    The attack deals 1d8 points of fire damage per two druid levels.
    Her accuracy is equal to her druid level \add her Constitution.
    A failed attack deals half damage.
    \subcf{15th -- Lifegiving Step}
    Wherever the druid moves, she leaves a path of small, living plants that entangle foes for 1 round.
    Whenever a creature crosses the path, the druid makes a Reflex attack to entangle the creature, causing it to waste the rest of its movement.
    Her accuracy is equal to her druid level \add her Constitution.
    The plants appear on any surface, and will continue to grow if they can survive, though they may die quickly if they appear on inhospitable terrain.
    \subcf{17th -- Flaming Soul}
    The druid gains the fire subtype, making her immune to fire but giving her a 50\% vulnerability to cold damage.
    In addition, whenever she deals fire damage to a creature, the creature is \ignited for 5 rounds.
    \subcf{17th -- Sunblessed Rejuvenation}
    The druid gains fast healing equal to her druid level as long as she remains in sunlight or touches a plant of her size or larger.
    \subcf{17th -- Sunscour}
    This aspect functions like the heart of the sun natural aspect, except that it also suppresses shadow effects and the visual components of illusions within the area of bright light.

    \subcf{17th -- Water's Flow}
    As a swift action, the druid can transform herself into a rushing flow of water with a volume roughly equal to her normal volume until the end of her turn.
    In this form, she may move wherever water could go, but she cannot take other actions, such as jumping, attacking, or casting spells.
    Her speed is halved when moving uphill and doubled when moving downhill.
    She may move through squares occupied by creatures or threatened by blocking enemies without penalty.
    She may return to her normal form as a free action.
    \par If the water is split, she may reform from anywhere the water has reached, to as little as a single ounce of water.
    If not even an ounce of water exists contiguously, her body reforms from the largest available parts of water, cut into pieces of appropriate size.
    This usually causes the druid to die.

\end{comment}

        \cf{Drd}[5th]{Multiple Aspect}[Mag]
        The druid gains the ability to embody two wild aspects at once.
        At 9th level, and every 4 levels thereafter, the druid gains the ability to embody an additional wild aspect at the same time.

    \cf{Drd}[20th]{Avatar of Nature}[Mag]
    If the druid dies, except if by old age, she may choose to have her body and soul become an instrument of nature's will.
    Her body immediately decomposes or otherwise disappears, and her soul does not travel to an afterlife.
    She has no physical form, and cannot use any of her normal abilities.
    Instead, she has a fly speed of 100 feet, with special maneuverability.
    As a standard action, she can temporarily possess any living plants or animals within a 10 mile radius of the place of her death.

    While possessing a living plant or animal, she can see through its senses and control its actions completely.
    In addition, she may cast spells, and the spells take effect as if the plant or animal had cast them.
    She uses the plant or animal's position to determine range, visible targets, and so on.
    She does not require verbal or somatic components to cast her spells in this form, but is unable to cast spells or perform rituals that require material components or focus objects.

    While not possessing a plant or animal, the druid can rest, or she can focus on reincarnating her physical form.
    Creating a new body in this way takes 12 straight hours of concentration.
    At the end of that time, the druid is reincarnated in a new body in her location, as the effect of the \spell{reincarnate} ritual, except that she can choose her race from among the races listed (not including the ``Other'' race).

    While she is an avatar of nature, a druid does not age and does not die of old age.
    She can continue to exist in this form indefinitely.

    \subsection{Ex-Druids}
        A druid who ceases to revere nature, changes to a prohibited alignment, or teaches the Druidic language to a nondruid loses all spells and magical druid class abilities.
        She cannot thereafter gain levels as a druid until she atones (see the \spell{atonement} ritual).

    \subsection{Variant Druids}

        \subsubsection{Blighter}

            Blighters draw power from nature, as do other druids. However, while other druids revere nature and draw power from it gently, blighters steal power from nature forcefully. Wherever a blighter goes, destruction and death surely follows.

            \altcf{Blight} Instead of meditating to regain spell slots, a blighter draws power from her environment forcefully.
            This affects a \areahuge radius zone centered on her, and the process takes 1 minute of concentration.
            At the end of every round, every living thing in the area other than the blighter takes damage equal to her nature power.
            All inanimate plants of Huge size or smaller immediately wither and die.
            The earth becomes cracked and infertile, and any nutrients from the soil are destroyed.
            This ability has no effect on artificial environments or materials, such as metal or worked stone.
            At the end of the minute, the blighter regains her spent nature spell slots.

            A blighter can only blight her surroundings in this way once per hour.
            If her surroundings are already blighted or are not natural terrain, she cannot use this ability to regain her spells.
            Instead, she must meditate for 8 hours to slowly draw power from her surroundings, as a normal druid.

            \altcf{Spells} As normal, except that a blighter adds all Vivimancy arcane spells to her spell list.

            \altcf[2nd]{Wild Speech} As normal, except that a blighter gains a \plus5 bonus to Intimidate against her wild speech targets, and a \minus5 penalty to Persuasion.

            \altcf[10th]{Blightcasting}

            \altcf[20th]{Improved Blightcasting}

        \subsubsection{Rotbringer}

            While most druids seek to emulate and interact with animals, rotbringers focus on the power of fungi, decay, and regeneration.

            \altcf{Invoke Rot} Instead of meditating to regain spell slots, a rotbringer accelerates the natural forces of decomposition and decay on her environment.
            This affects a \areahuge radius zone centered on her, and the process takes 1 minute of concentration.
            All organic objects of Huge size or smaller, such as plants and corpses, decompose.
            This decomposition kills inanimate, living plants.
            All organic objects, regardless of size, are covered with various fungi.
            This ability has no effect on artificial environments or materials, such as metal or worked stone.
            At the end of the minute, the rotbringer regains her spent nature spell slots.

            If the rotbringer decomposes a Huge object with this ability, or a combination of smaller objects equivalent in size to a Huge object, she gains an bonus nature spell slot of her highest available spell level.
            This extra spell slot lasts until it is used, or until she regains her spell slots again.

            A rotbringer can only invoke rot on her surroundings in this way once per hour.
            If her surroundings are already decomposed or are not natural terrain, she cannot use this ability to regain her spells.
            Instead, she must meditate for 8 hours to slowly draw power from her surroundings, as a normal druid.

            \altcf[2nd]{Wild Speech} The rotbringer gains the ability to speak with plants at 2nd level.
            She gains the ability to speak with animals at 6th level, instead of at 2nd level.

            \altcf[3rd]{Wild Aspect} The rotbringer does not gain this ability.

            \altcf[3rd]{Rot Spell} The druid learns an additional spell slot and spell known.
            The spell must be taken from the following list of spells.
            The spell's level cannot exceed half her druid level.
            If she already knows a spell from the list at every spell level she has access to, she may instead learn any nature spell (see \pcref{Nature Spells}).

            At 5th level, and every odd level, the druid may learn a new spell.

            \begin{dtable}
                \begin{dtabularx}{\columnwidth}{l X}
                    \tb{Spell level} & \tb{Rotbringer Spells} \\
                    1st & \spell{excrete slime}, \spell{lesser regeneration} \\
                    2nd & \spell{fungal growth} \\
                    3rd & \spell{rotburst} \\
                    4th & \spell{poison} \\
                    6th & \spell{regeneration} \\
                    7th & \spell{greater rotburst} \\
                \end{dtabularx}
            \end{dtable}

            \altcf[7th]{Fungal Armor} The rotbringer becomes covered in fungus that protects her from attacks. She gains a \plus1 bonus to Armor and Fortitude defense.

            This bonus increases by 1 at her 7th druid level, and every 4 druid levels thereafter.

\section{Fighter}\label{Fighter}
    \begin{dtable}
        \lcaption{Fighter Progression}
        \begin{dtabularx}{\columnwidth}{>{\ccol}p{\levelcol} >{\lcol}X}
            \tb{Level} & \tb{Special} \\
            \hline
            \nth{1}  & Combat supremacy, martial excellence \\
            \nth{2}  & Weapon discipline                    \\
            \nth{3}  & Combat feat                          \\
            \nth{4}  & Combat discipline                    \\
            \nth{5}  & Battlemaster                         \\
            \nth{6}  & Armor discipline                     \\
            \nth{7}  & Combat feat                          \\
            \nth{8}  & Improved weapon discipline           \\
            \nth{9}  & Combat mastery                       \\
            \nth{10} & Improved combat discipline           \\
            \nth{11} & Combat feat                          \\
            \nth{12} & Improved armor discipline            \\
            \nth{13} & Swift warrior                        \\
            \nth{14} & Greater weapon discipline            \\
            \nth{15} & Combat feat                          \\
            \nth{16} & Greater combat discipline            \\
            \nth{17} & Supreme battlemaster                 \\
            \nth{18} & Greater armor discipline             \\
            \nth{19} & Combat feat                          \\
            \nth{20} & Supreme discipline                   \\
        \end{dtabularx}
    \end{dtable}

    \classbasics{Alignment} Any.

    \classbasics{Class Skills}
    \subparhead{Strength} Climb, Jump, Sprint, Swim.
    \subparhead{Dexterity} Acrobatics, Escape Artist, Ride.
    \subparhead{Perception} Awareness.
    \subparhead{Other} Bluff, Intimidate, Persuasion.

    \subsection{Base Class Abilities}
        A character with fighter as a base class gains the following abilities.

        \classbasics{Skill Points} 10.

        \classbasics{Defenses} \plus4 Fortitude, \plus2 Mental.

        \classbasics{Action Points} \plus2.

        \cf{Ftr}{Weapon and Armor Proficiency}
        A fighter is proficient with simple weapons, any four other weapon groups,  all armor (heavy, medium, and light), and shields.

        \cf{Ftr}{Martial Excellence}
        A fighter gains a \plus1 bonus to accuracy and damage with physical attacks.
        In addition, he gains a \plus1 bonus to physical defenses.

    \subsection{Class Abilities}
        All fighters have the following abilities.

        \cf{Ftr}{Combat Supremacy}
        A fighter is a consummate warrior, and can stand toe to toe with even the toughest foes.
        Whenever the fighter deals damage to a creature with a physical attack, he may impede its fighting ability as an \glossterm{immediate action}.
        If he does, the struck creature is \impaired on attacks and checks against the fighter for 2 rounds, or until he uses this ability on that creature again.
        He may use this ability a number of times per day equal to his Intelligence or his fighter level, whichever is higher.

        \cf{Ftr}[2nd]{Weapon Discipline}
        The fighter's training grants him additional capability when using his weapons.
        He may choose a weapon group, or he may choose to train equally with all weapons.
        If he chooses a weapon group, he gains a \plus1 bonus to damage on attacks with weapons from that group.

        If he chooses not to focus on a specific group of weapons, he gains the ability to become proficient with any weapon group if he spends 1 hour training with a weapon from that group.
        He may only keep this proficiency with one weapon group at a time; if he trains with a new weapon group, he loses his proficiency in the previous group.

        \cf{Ftr}[3rd]{Combat Feat}
        The fighter gets a bonus Combat feat (see \pcref{Combat Feats}).
        He must use his fighter level in place of his character level to meet level prerequisites for the feat.
        He gains an additional bonus feat at his 7th fighter level and every four fighter levels thereafter.

        \cf{Ftr}[4th]{Combat Discipline}
        The fighter can use his superior training and focus to keep fighting in the face of debilitating effects.
        When a fighter is initially affected by one of the conditions listed on \trefnp{Combat Discipline Conditions}, he may use his combat discipline ability to instead suffer the mitigated condition one column to the right.
        He can suppress the condition up to 5 rounds.

        \par Using combat discipline takes no action, and can be done at any time, even when it isn't the fighter's turn.
        A fighter may use this ability a number of times per day equal to his Willpower or half his fighter level, whichever is higher.
        However, he cannot mitigate more than one condition at a time.
        If the fighter attempts to mitigate a new condition, the old condition resumes its normal effect immediately.

        \begin{dtable}
            \lcaption{Combat Discipline Conditions}
            \begin{dtabularx}{\columnwidth}{l *{3}{>{\lcol}X}}
                \tb{Original Condition} & \tb{Mitigated Condition} & \tb{Mitigated Condition} & \tb{Mitigated Condition} \\
                \hline
                Panicked              & Frightened        & Shaken & None \\
                Petrified             & Paralyzed         & Slowed & None \\
                Blinded               & Visually impaired & None   & \tdash   \\
                Confused              & Disoriented       & None   & \tdash   \\
                Exhausted             & Fatigued          & None   & \tdash   \\
                Nauseated             & Sickened          & None   & \tdash   \\
                Severely impaired     & Impaired          & None   & \tdash   \\
                Stunned               & Dazed             & None   & \tdash   \\
                %Entangled             & None              & \tdash     & \tdash   \\
                Deafened              & None              & \tdash     & \tdash   \\
                Fascinated            & None              & \tdash     & \tdash   \\
                Ignited\fn{1}         & None              & \tdash     & \tdash   \\
                Immobilized           & None              & \tdash     & \tdash   \\
                Negative level\fn{2}  & None              & \tdash     & \tdash   \\
                Slowed                & None              & \tdash     & \tdash   \\
                Vulnerable            & None              & \tdash     & \tdash   \\
            \end{dtabularx}
            1.  Mitigates the impairment, but does not prevent the fighter from taking 1d6 fire damage per round until the fire is put out.  \\
            2.  Mitigate a single negative level. \\
        \end{dtable}

        \par A fighter can never use this ability more than once against a single source.
        For example, if a fighter is confused by a \spell{confusion} spell, he can use this ability to become disoriented instead of confused, but he can't then expend a second use to stop being disoriented.
        The lesser condition that this ability imposes may be cured or removed normally, but doing so does not affect the resurgence of the condition the fighter was originally afflicted with.
        If a fighter uses this ability to mitigate or negate a condition which he must suffer as a sacrifice or cost to gain some benefit, he automatically forfeits the benefit he would have gained.

        \cf{Ftr}[5th]{Battlemaster}
        The fighter becomes a master of controlling the ebb and flow of battle.
        When he use his combat supremacy ability, he may choose for the struck creature to be \goaded by him or \shaken by him instead of impaired.

        \cf{Ftr}[6th]{Armor Discipline}
        The fighter's training grants him additional capability in armor.
        He must choose to improve his agility or his resilience in armor.
        This applies to all armor discipline abilities the fighter has.

        If he improves his agility, he treats body armor he wears as less encumbering.
        He reduces its \glossterm{encumbrance penalty} by 2 its arcane spell failure by 10\%.
        In addition, he treats it were one encumbrance category lighter than it is whenever doing so would be beneficial for him.
        Heavy armor is treated as medium armor, medium armor is treated as light armor, and light armor is treated as being unarmored.
        This can remove the speed reduction and reduced Dexterity associated with the armor's encumbrance, as appropriate for the new encumbrance of the fighter's armor.

        If he improves his resilience, he gains damage reduction against physical damage equal to his character level. This allows him to ignore the first points of damage he would take each round.

        \cf{Ftr}[8th]{Improved Weapon Discipline}
        The fighter's training in his chosen weapons improves.
        He increases the \glossterm{critical multiplier} of his chosen weapons by 1.
        In addition, if he chose a specific weapon group, he increases his damage bonus to \plus2.
        If he did not, he becomes proficient with all weapon groups and all exotic weapons.
        He retains this benefit for one week after the training.

        \cf{Ftr}[9th]{Combat Mastery}
        The fighter may use his combat supremacy ability any number of times per day.

        \cf{Ftr}[10th]{Improved Combat Discipline}
        The fighter's ability to keep fighting despite negative influence improves.
        He can reduce conditions by two steps instead of one when using combat discipline.
        For example, a stunned fighter who used combat discipline would instead be staggered.
        \par In addition, a fighter may use combat discipline to reduce any penalties he suffers to his accuracy, damage, checks, or defenses by 2, even if the source of the penalty is not listed on the combat discipline chart.
        \par The fighter may also mitigate up to two conditions at once.

        \cf{Ftr}[12th]{Improved Armor Discipline}
        The fighter's training with his armor improves.
        If he chose agility, he treats body armor he wears as even less encumbering.
        He reduces its \glossterm{encumbrance penalty} by 4 and decreases its arcane spell failure by 20\%.
        In addition, he treats all it as if it were two encumbrance categories lighter than it is whenever doing so is beneficial for him.
        This does not stack with the benefit of the armor discipline ability.

        If he chose resilience, he may apply his damage reduction from the armor discipline ability against all damage, including from magical attacks.
        In addition, he gains a \plus1 bonus to Armor defense while wearing armor.

        \cf{Ftr}[13th]{Swift Warrior}
        The fighter can take an additional \glossterm{swift action} or \glossterm{immediate action} each round (see \pcref{Swift and Immediate Actions}).

        \cf{Ftr}[14th]{Greater Weapon Discipline}
        The fighter's training in his chosen weapons becomes still greater.
        He increases the \glossterm{critical range} of his chosen weapons by 1.
        This increase stacks with any other effects that affect critical range.
        For example, a fighter using heavy blades with the Weapon Focus feat (see \featpcref{Weapon Focus}) would have a critical range of 18-20.

        \cf{Ftr}[16th]{Greater Combat Discipline}
        The fighter's ability to keep fighting despite influence becomes still greater.
        When using combat discipline, he may ignore any condition listed on the combat discipline chart.
        In addition, he may use combat discipline to be \severelyimpaired with attacks and checks rather than suffer any non-damaging condition not listed on the chart.

        \par The fighter may also mitigate up to three conditions at once.

        \cf{Ftr}[17th]{Supreme Battlemaster}
        The fighter's ability to control the battle improves.
        Whenever he uses his combat supremacy ability, he may choose for the struck creature to be \taunted or \frightened instead of impaired.

        \cf{Ftr}[18th]{Greater Armor Discipline}
        The fighter's training in his chosen armor becomes still greater.
        If he chose agility, he ignores all encumbrance penalties and arcane spell failure from body armor he wears.
        In addition, he treats body armor he wears as if it were three encumbrance categories lighter than it actually is whenever doing so would be beneficial to him.
        This does not stack with the benefits of armor discipline or improved armor discipline.
        In addition, he gains a \plus4 bonus to Reflex defense.

        If he chose resilience, he gains a \plus4 bonus to Fortitude defense.

        \cf{Ftr}[20th]{Supreme Discipline}
        The fighter's discipline in his chosen area is beyond equal.
        He must choose either weapon discipline, armor discipline, or combat discipline.
        Depending on which discipline he chooses, he gains a different bonus.
        \subcf{Supreme Weapon Discipline}
        Whenever the fighter deals damage to a creature with a physical attack, the struck creature is \severelyimpaired with attacks and checks for 2 rounds.
        \subcf{Supreme Armor Discipline}
        If the fighter chose agility, he gains a \plus2 bonus to physical defenses.

        If the fighter chose resilience, he doubles the damage reduction granted by his armor discipline ability.
        \subcf{Supreme Combat Discipline}
        The fighter can use combat discipline to be \impaired with attacks and checks instead of suffering any nondamaging negative effect with a duration.
        He may also mitigate up to four conditions at once.

\section{Mage}\label{Mage}
    \begin{dtable}
        \lcaption{Mage Progression}
        \begin{dtabularx}{\columnwidth}{>{\ccol}p{\levelcol} c >{\lcol}X}
            \tb{Level} & \tb{Spells} & \tb{Special} \\
            \hline
            \nth{1}  & 3  & Arcane essence, rituals, spells \\
            \nth{2}  & 4  & Arcane defense \\
            \nth{3}  & 4  & Arcane insight \\
            \nth{4}  & 5  & Arcane essence \\
            \nth{5}  & 5  & Arcane insight \\
            \nth{6}  & 6  & Inherent magic \\
            \nth{7}  & 6  & Arcane insight \\
            \nth{8}  & 7  & Arcane essence \\
            \nth{9}  & 7  & Arcane insight \\
            \nth{10} & 8  & Inherent magic \\
            \nth{11} & 8  & Arcane insight \\
            \nth{12} & 9  & Arcane essence \\
            \nth{13} & 9  & Arcane insight \\
            \nth{14} & 10  & Inherent magic \\
            \nth{15} & 10  & Arcane insight \\
            \nth{16} & 11 & Arcane essence \\
            \nth{17} & 11 & Arcane insight \\
            \nth{18} & 12 & Inherent magic \\
            \nth{19} & 12 & Arcane insight \\
            \nth{20} & 13 & Archmage       \\
        \end{dtabularx}
    \end{dtable}

    \classbasics{Alignment} Any.

    \classbasics{Class Skills}
    \subparhead{Intelligence} Knowledge (all kinds, taken individually), Linguistics.
    \subparhead{Perception} Awareness, Spellcraft.
    \subparhead{Other} Bluff, Intimidate, Persuasion.

    \subsection{Base Class Abilities}
        A character with mage as a base class gains the following abilities.

        \classbasics{Skill Points} 5.

        \classbasics{Defenses} \plus4 Mental.

        \classbasics{Action Points} \plus3.

        \cf{Mge}{Weapon and Armor Proficiency}
        Mages are proficient with simple weapons and one other weapon group.
        They are not proficient with any type of armor or shield.
        Armor of any type interferes with a mage's arcane gestures, which can cause her spells with somatic components to fail.

        \cf{Mge}{Spell Point}
        The mage gains a spell point.
        A spell point can be spent to cast spells in place of an action point.
        She recovers all spent spell points after a \glossterm{short rest}.

        \cf{Mge}{Enhanced Spellpower}[Mag]
        The mage gains a \plus2 bonus to spellpower with her arcane spells.

    \subsection{Class Abilities}
        All mages have the following abilities.

        \cf{Mge}{Arcane Essence}
        All mages have access to great arcane power.
        However, not all mages acquired this power in the same way.
        A mage must choose an arcane essence.
        The mage's arcane essence affects several aspects of her class abilities.
        \parhead{Sorcerer} Sorcerers have an intuitive connection to magic that allows them to cast spells without preparation or training.
        \parhead{Wizard} Wizards studied arcane mysteries for years to learn the secret ways of magic.
        A wizard casts spells with her Intelligence.

        \cf{Mge}{Spells}
        A mage casts arcane spells.
        Her \glossterm{spellpower} with arcane spells is normally equal to her character level.
        \textit{Sorcerer} --- The maximum spell level a sorcerer can learn or cast is equal to half her mage level (minimum 1).
        \textit{Wizard} --- The maximum spell level a wizard can learn or cast is equal to half her mage level (minimum 1) or her Intelligence, whichever is lower.

        To cast a spell, a mage must spend an \glossterm{action point}.
        Every spell can also be cast as a cantrip.
        Cantrips are weaker, but do not require action points to cast.

        A mage begins play knowing three spells.
        Every even level, she learns an additional spell of any level she has access to.
        In addition, each time she gains a level, she may trade one of her existing spells for a different spell known.
        A mage's spells are drawn from the spells on the arcane spell list (see \pcref{Arcane Spells}).

        Almost all mages know at least one cantrip.
        Cantrips are special spells that do not require action points to cast.
        They are decribed first on the list of arcane spells (see \pcref{Cantrip List}).

        \cf{Mge}{Rituals}
        \subparhead{Sorcerer} Sorcerers cannot perform arcane rituals.
        \subparhead{Wizard} Wizards can perform arcane rituals to create unique magical effects (see \pcref{Rituals}).
        A wizard begins play with a ritual book containing two arcane rituals of her choice (see \pcref{Arcane Rituals}).

        \cf{Mge}[2nd]{Arcane Defense}[Mag]
        The mage gains a \plus2 bonus to defenses against spells.

        \cf{Mge}[3rd]{Arcane Insight}\label{Arcane Insight}
        The mage gains a greater understanding of magic.
        She chooses one of the following insights.
        Each insight can be chosen multiple times.
        Unless otherwise noted, their effects stack.
        \begin{itemize}
            \item Innate Spell: The mage chooses a spell she knows as an innate spell.
                She no longer needs verbal or somatic components to cast the spell.
                \par If she chooses this insight multiple times, she must choose a different spell each time.
            \item Personal Spell: The mage chooses a spell she knows as a personal spell.
                She cannot miscast the spell (see \pcref{Miscasting}).
                If she would miscast it, the spell simply fails without effect.
                In addition, she automatically succeeds at all Concentration checks she makes to cast the spell.
                \par If the mage chooses this insight multiple times, she must choose a different spell each time.
            \item Specialization: The mage chooses a school of magic.
                She gains an additional spell known that can only be used to learn a spell from that school.
                In exchange, she must ban a school of magic.
                She can never learn or cast spells or rituals from her banned schools.
                If she knows spells from a banned school, she must immediately learn different spells from unbanned schools in their place.
                Divination cannot be chosen as a banned school.
                \par If the mage chooses this insight multiple times, she must choose to specialize in the same school each time.
        \end{itemize}

        \subparhead{Sorcerer} A sorcerer may also choose the Expanded Spell Knowledge insight.
        She chooses a single spell from the divine spell list or nature spell list and adds it to her arcane spell list.
        The spell's level cannot exceed half her mage level.
        This does not grant her the spell as a spell known.
        \par If she chooses this insight multiple times, she must choose a different spell each time.

        \subparhead{Wizard} A wizard may also choose the Ritual Spell insight.
        She scribes an arcane spell with a maximum level equal to half her mage level into her ritual book.
        She does not need to know the spell, and pays no cost to scribe it.
        The spell is treated as a ritual, and she can perform a one minute ritual to gain the spell's effect.
        Performing the ritual costs the normal amount of material components for a ritual of its level (see \pcref{Ritual Costs}).

        \cf{Mge}[4th]{Arcane Essence}[Mag]
        The mage gains an ability based on her choice of arcane essence.
        \subparhead{Sorcerer} The sorcerer gains bonus hit points equal to her sorcerer level or Willpower, whichever is higher.
        \subparhead{Wizard} The wizard gains a \plus2 bonus to all Knowledge skills.

        \cf{Mge}[6th]{Inherent Magic}[Mag]
        The mage becomes inherently magical.
        She may choose an arcane spell with a maximum level equal to half her mage level \sub 2.
        The spell must be a \glossterm{targeted spell} and have a duration of \durshort or longer.
        She does not need to know the spell.
        The mage constantly gains the benefits of that spell.
        If the spell has secondary effects, such as the \spell{avatar of suffering} spell, the duration of the secondary effects is not changed.

        When the mage chooses a spell in this way, the mage may also choose specific augments for that spell, as long as the spell's effective level does not exceed half her character level.
        If the spell's effect ends for any reason, such as if its effect is discharged, it takes effect on the mage again 5 minutes later.

        At her 10th mage level, and every 4 mage levels thereafter, the mage may choose an additional attuned spell.

        \cf{Mge}[8th]{Arcane Essence}[Mag]
        The mage gains an ability based on her choice of arcane essence.
        \subparhead{Sorcerer} The sorcerer gains \glossterm{magic resistance} equal to 10 \add her character level or Willpower, whichever is higher.
        This ability does not protect the sorcerer from her own miscast effects.
        If she already has magic resistance from another source, she increases that magic resistance by 2.
        \subparhead{Wizard} As a standard action, the wizard can spend an action point to begin casting two spells at once.
        She must designate one as the primary spell and the other as the secondary spell.
        The level of both spells, including any augments, must not exceed half her mage level.
        If the two spells are different levels, she must designate the higher level spell to be the primary spell.

        When she finishes casting, she chooses which one of those spells takes effect.
        She spends an action point if necessary to cast the spell, and its effect takes place.
        The second spell dissipates without spending an action point or having any effect.

        If the wizard's concentration is disrupted, the primary spell is miscast.
        If she must spend an action point to cast that spell, she spends it.
        However, the secondary spell dissipates without spending an action point or having any effect.

        \cf{Mge}[12th]{Arcane Essence}[Mag]
        The mage gains an ability based on her choice of arcane essence.
        \subparhead{Sorcerer} The sorcerer gains a spell point.
        A spell point can be spent to cast spells in place of an action point.
        She recovers all spent spell points after a \glossterm{short rest}.
        \subparhead{Wizard} The wizard gains the ability to prepare a spell so it takes effect automatically if specific circumstances arise.
        To prepare a contingency, the wizard must spend 5 minutes preparing the spell.
        A contingency spell is treated as being two levels higher than its actual level for the purpose of spending action points and determining whether the wizard can cast the spell.
        During this casting time, the wizard specifies what circumstances cause the spell to take effect.

        The contingency can be set to trigger in response to any circumstances that a typical human observing the wizard and her situation could detect.
        For example, a wizard could specify ``when I fall at least 50 feet'' or ``when I become bloodied'', but not ``when there is an invisible creature within 50 feet of me'' or ``when I am at 17 hit points or fewer.'' The more specific the required circumstances, the better -- vague requirements, such as ``when I am in danger'', may cause the contingency to trigger unexpectedly or fail to trigger at all.
        If a wizard attempts to specify multiple separate triggering conditions, such as ``when I take damage or when an enemy is adjacent to me'', the contingency will randomly ignore all but one of the conditions.

        The spell must have a casting time of 1 standard action or less.
        The contingency can specify a simple rule for identifying how to target the spell, such as ``the closest enemy''.
        If the rule is poorly worded or imprecise, the contingency may target incorrectly or fail to activate at all.
        Any spells which require decisions, such as the \spell{dimension door} spell, must have those decisions made at the time the contingency is created.
        The wizard cannot change those decisions when the contingency takes effect.

        A wizard can have only one contingency active at a time.
        If she creates another contingency, it replaces her old contingency.

        \cf{Mge}[16th]{Arcane Essence}[Mag]
        The mage gains an ability based on her choice of arcane essence.
        \subparhead{Sorcerer} Whenever the sorcerer resists a spell with her magic resistance, she gains the ability to cast that spell once during the next 5 rounds.
        The spell retains all augments, effects from feats and other abilities, and similar modifications from the original caster, and the sorcerer cannot choose any other augments or apply effects from her own abilities.
        However, she makes all other decisions required to cast the spell, and uses her spellpower to determine the spell's effects.
        Once she casts the spell, she expends the absorbed energy, can cannot cast it again.

        \subparhead{Wizard} The wizard may ready two spells in her contingency instead of a single spell.
        In addition, she may have two contingencies active at once instead of one.
        Only one contingency can trigger in a given round.
        If both would trigger, only the first contingency cast triggers, and the second does not.

        \cf{Mge}[20th]{Archmage}
        The mage no longer needs to spend action points to cast spells of any level.
        If she has any spell points, she loses those spell points and gains the same number of legend points.

\section{Monk}\label{Monk}
    \begin{dtable}
        \lcaption{Monk Progression}
        \begin{dtabularx}{\columnwidth}{>{\ccol}p{\levelcol} c >{\lcol}X}
            \tb{Level} & \tb{\Ki Strike} & \tb{Special} \\
            \hline
            \nth{1}  & \plus1 & Enlightened defenses, unarmed warrior     \\
            \nth{2}  & \plus1 & Manifest \ki                              \\
            \nth{3}  & \plus1 & Wholeness of body, uncanny dodge \\
            \nth{4}  & \plus2 & Manifest \ki                              \\
            \nth{5}  & \plus2 & Flurry of blows                           \\
            \nth{6}  & \plus2 & Manifest \ki                              \\
            \nth{7}  & \plus3 & Perfect motion                            \\
            \nth{8}  & \plus3 & Manifest \ki                              \\
            \nth{9}  & \plus3 & Improved uncanny dodge, perfect soul      \\
            \nth{10} & \plus4 & Manifest \ki                              \\
            \nth{11} & \plus4 & Flow of life                              \\
            \nth{12} & \plus4 & Manifest \ki                              \\
            \nth{13} & \plus5 & Perfect mind                              \\
            \nth{14} & \plus5 & Manifest \ki                              \\
            \nth{15} & \plus5 & Perfect body                              \\
            \nth{16} & \plus6 & Manifest \ki                              \\
            \nth{17} & \plus6 & Perfect life                              \\
            \nth{18} & \plus6 & Manifest \ki                              \\
            \nth{19} & \plus7 & True perfection                           \\
            \nth{20} & \plus7 & Manifest \ki, transcend mortality         \\
        \end{dtabularx}
    \end{dtable}

    \classbasics{Alignment} Any nonchaotic.

    \classbasics{Class Skills}
    \subparhead{Strength} Climb, Jump, Sprint, Swim.
    \subparhead{Dexterity} Acrobatics, Escape Artist, Ride, Stealth.
    \subparhead{Intelligence} Heal.
    \subparhead{Perception} Awareness, Spellcraft, Survival.
    \subparhead{Other} Bluff, Intimidate, Perform, Persuasion.

    \subsection{Base Class Abilities}
        A character with monk as a base class gains the following abilities.

        \classbasics{Skill Points} 10.

        \classbasics{Defenses} \plus2 Fortitude, \plus4 Reflex, \plus4 Mental.

        \classbasics{Action Points} \plus3.

        \cf{Mnk}{Weapon and Armor Proficiency}
        Monks are proficient with simple weapons, monk weapons, and any one other weapon group.
        Monks are not proficient with any armor or shields.
        When wearing armor, using a shield, or carrying a medium or heavy load, a monk loses the benefit of her enlightened defense, fast movement, and \ki abilities.

        \cf{Mnk}{Ki Strike}[Mag] A monk's fists, and all weapons she uses, gain a \plus1 \glossterm{enhancement bonus}.
        This functions like an enhancement bonus on a weapon, increasing her damage and offensive legend points per day (see \pcref{Weapon Enhancement Bonuses}).
        In addition, she is always treated as if she was wearing \plus1 armor, granting her temporary hit points and defensive legend points (see \pcref{Armor Enhancement Bonuses}).
        This bonus increases by \plus1 at 4th level and every 3 levels thereafter.

    \subsection{Class Abilities}
        All monks have the following abilities.

        \cf{Mnk}{Enlightened Defense}[Mag]
        A monk's \ki shields her body from attacks.
        When \monkunencumbered, she gains a \plus2 bonus to Armor defense.
        She loses this bonus when she is helpless.

        \cf{Mnk}{Ki Power}
        Many monk abilities depend on her \ki power.
        A monk's \ki power is equal to her Willpower or her character level, whichever is higher.

        \cf{Mnk}{Unarmed Warrior}
        A monk's unarmed attacks are exceptionally deadly.
        She gains Unarmed Fighting as a bonus feat, making her proficient with her unarmed attack (see \featpcref{Unarmed Fighting}).
        In addition, she increases the damage of her unarmed attacks by two increments (see \pcref{Weapon Size}).
        For example, a Medium monk would deal 1d6 damage with her unarmed attack.
        For details about how to fight while unarmed, see \pcref{Unarmed Combat}.

        \cf{Mnk}[2nd]{Manifest Ki}[Mag]
        The monk gains the ability to channel her \ki energy to temporarily enhance her abilities.
        She chooses one \ki manifestation from the list below.

        Using a \ki manifestation costs a \ki point.
        The monk has a number of \ki points equal to her Willpower or her monk level, whichever is higher.
        \Ki points can be recovered by meditation.
        If the monk meditates for 1 hour, she recovers all spent \ki points.

        Some \ki manifestations have minimum monk levels, as indicated in the title of the power.
        At her 4th monk level, and every even monk level thereafter, the monk learns an additional \ki manifestation.

        Some \ki manifestations have the effects of spells.
        Unless otherwise noted, the monk's effective spellpower with these abilities is equal to her \ki power.

        \subcf{Elegant Whirl of Fluid Motion}
        As a swift action, the monk can gain a \plus20 bonus to Acrobatics checks until the end of the round.
        \subcf{Leap of the Heavens}
        As a swift action, the monk can gain a \plus20 bonus to Jump checks until the end of the round.
        \subcf{Scale the Highest Tower}
        As a swift action, the monk can gain a \plus20 bonus to Climb checks until the end of the round.
        \subcf{4th -- Dance of Falling Feathers}
        As an immediate action, when the monk begins falling, she can gain the benefits of the \spell{feather fall} spell.
        \subcf{4th -- Fists of Distant Force}
        As a swift action, the monk can empower her unarmed attacks with \ki, allowing her to strike distant foes.
        Until the end of the round, she gains an additional ten feet of reach with her unarmed attacks, extending her threatened area.
        \subcf{6th -- Burst of Blinding Speed}
        As a swift action, the monk can gain a \plus30 foot bonus to her land speed, up to a maximum of double her original speed.
        In addition, she cannot be followed until the end of the round.
        \subcf{6th -- Dance of the Wayward Strike}
        As an immediate action, when a foe misses the monk with a melee strike, the monk can redirect the strike.
        Both the foe and the monk must threaten a third creature.
        If the monk redirects the strike, the foe rolls the same attack against the third creature.
        \subcf{6th -- Surpass the Mortal Limits}
        As a swift action, the monk can surpass the physical limitations of her body.
        Until the end of the round, she may use her \ki power in place of her Strength, Dexterity, and Constitution when making checks.
        \subcf{8th -- Flash Step}
        As a swift action during the movement phase, the monk can teleport to anywhere she can see within 30 feet.
        If her line of effect is blocked, even by an invisible barrier, or if this would somehow place her inside a solid object, the ability fails.
        \subcf{8th -- Focus the Wayward Mind}
        As a swift action, the monk can dispel all \glossterm{Mind} effects that are affecting her.
        This has no effect on Mind effects that cannot be dispelled.
        \subcf{8th -- Ki-Disrupting Strike}
        As an immediate action, when the monk hits with a melee strike, she can make the struck creature \impaired with attacks and checks for 2 rounds.
        \subcf{8th -- See the Flow of Life}
        As a swift action, the monk can gain the ability to see the \ki of living creatures until the end of the round.
        She can ``see'' any living creatures and their equipment within 50 feet perfectly, regardless of lighting conditions, invisibility, or any other means of concealment.
        This cannot detect living creatures through solid walls, however.
        \subcf{10th -- Dance of the Foolish Blow}
        This \ki manifestation functions as the \textit{dance of the wayward strike} manifestation, except that it can affect any melee attack, not just a single strike.
        The monk must have the \textit{dance of the wayward strike} \ki manifestation to learn this manifestation.
        \subcf{12th -- Diamond Fists}
        As a swift action, the monk can empower her unarmed attacks with incredible force.
        Until the end of the round, she may use her \ki power in place of her normal modifiers to physical damage with her unarmed attacks.
        In addition, she treats her unarmed strike as if it were an adamantine weapon for the purpose of overcoming damage reduction and hardness.
        \subcf{12th -- Stunning Fist}
        As an immediate action, when the monk deals damage with an unarmed melee strike, she can make a \Ki power vs. Fortitude attack against the struck creature.
        Success means the target is \staggered for 2 rounds.
        Critical success means the target is \stunned for 2 rounds.
        \norepeatnotes
        \subcf{14th -- Awaken the Pacifist Heart}
        As an immediate action, when the monk hits with a melee strike, she can make a \Ki power vs. Mental attack against the struck creature.
        Success means the target is unable to take violent actions, such as attacking, for 2 rounds.
        If the target takes damage after the current round, the effect is broken.
        \subcf{16th -- Flash Burst}
        As a swift action during the movement phase, the monk can teleport to anywhere within 1,000 feet.
        She must clearly visualize the destination, but she does not need line of sight or line of effect.
        If the destination is occupied, or dramatically different from how she visualized it, the effect fails.
        \subcf{20th -- Ki-Shattering Strike}
        As an immediate action, when the monk hits with a melee strike, she can disrupt her foe's \ki.
        The target is \severelyimpaired with attacks and checks for 2 rounds.

        \cf{Mnk}[3rd]{Wholeness of Body}[Mag]
        With concentration and focus, the monk can correct the flow of energy within her body.
        She can use this ability as a standard action by spending a \ki point, or with one minute of meditation otherwise.
        If she does, she heals 1d6 hit points per \ki power.

        \cf{Mnk}[3rd]{Uncanny Dodge}
        The monk can react to danger before her senses would normally allow her to do so.
        The monk reduces her overwhelm penalties by 1.
        If her overwhelm penalty is reduced to 0, she is not considered to be overwhelmed.
        In addition, she is not \unaware when attacked by surprise.

        \cf{Mnk}[5th]{Flurry of Blows}\label{Flurry of Blows}
        As a standard action, the monk can spend a \ki point to unleash a furious barrage of blows.
        She can make a \glossterm{standard attack} with an extra \glossterm{strike}.
        Alternately, she can choose to make one \glossterm{strike} against all foes she threatens.

        \cf{Mnk}[7th]{Perfect Motion}[Mag]
        The monk becomes immune to effects that restrict her mobility.
        She suffers no penalties for acting underwater.
        In addition, she gains a \plus20 bonus to Reflex defense against grapple attacks, as well as on grapple attacks or Escape Artist checks made to escape a grapple or a pin.

        \cf{Mnk}[9th]{Improved Uncanny Dodge}
        The monk reduces her overwhelm penalties by 2.
        This does not stack with the effects of uncanny dodge.
        If her overwhelm penalty is reduced to 0, she is not considered to be overwhelmed.

        \cf{Mnk}[9th]{Perfect Soul}[Mag]
        The monk gains \glossterm{magic resistance} equal to 10 \add her \ki power.

        \cf{Mnk}[11th]{Flow of Life}[Mag]
        At the end of each round, the monk heals hit points equal to her \ki power.

        \cf{Mnk}[13th]{Perfect Mind}[Mag]
        The monk becomes immune to hostile \glossterm{Mind} effects.

        \cf{Mnk}[15th]{Perfect Body}[Mag]
        The monk becomes immune to being blinded, deafened, fatigued, exhausted, nauseated, sickened, and staggered.
        In addition, she no longer takes penalties to her attribute scores for aging, and cannot be magically aged.
        The monk still dies of old age when her time is up.

        \cf{Mnk}[17th]{Perfect Life}[Mag]
        At the end of each round, the monk heals hit points equal to twice her \ki power.
        This replaces the benefit from her flow of life ability.
        If she has taken vital damage, she instead heals vital damage equal to half her \ki power.
        In addition, she becomes immune to \glossterm{Death} effects.

        \cf{Mnk}[19th]{True Perfection}[Mag]
        The monk becomes inhumanly perfect.
        If she rolls less than a 5 on any d20 roll, it is treated as a 5.

        \cf{Mnk}[20th]{Transcend Mortality}[Mag]
        If the monk dies, she may choose to retain control of her body and soul through sheer force of will.
        Her body immediately disappears, and her soul does not travel to an afterlife.
        Instead, her body reforms with no trace of its injuries 24 hours later.
        The reformed body is in perfect health and can be any age the monk chooses, to a minimum of the age of adulthood for her race.
        She can reform her body at the place where she died, or in any place on the same plane that is deeply familiar to her.

        After each time the monk reforms herself this way, it takes 24 additional hours to reform the next time she ``dies''.
        A monk with this ability can only be permanently killed by the direct intervention of a deity.

        \subsubsection{Ex-Monks}
            A monk who becomes chaotic loses her \ki powers, and cannot gain more levels as a monk.
            She retains all her other class abilities.
            If she stops being chaotic, she regains her \ki powers and ability to take monk levels.

\section{Paladin}\label{Paladin}
    \begin{dtable}
        \lcaption{Paladin Progression}
        \begin{dtabularx}{\columnwidth}{>{\ccol}p{\levelcol} >{\lcol}X}
            \tb{Level} & \tb{Special} \\
            \hline
            \nth{1}  & Divine invocation (smite), divine protection \\
            \nth{2}  & Divine invocation                            \\
            \nth{3}  & Discernment (alignment)                      \\
            \nth{4}  & Divine presence                              \\
            \nth{5}  & Pass judgment                                \\
            \nth{6}  & Divine invocation                            \\
            \nth{7}  & Discernment (lies)                           \\
            \nth{8}  & Divine presence                              \\
            \nth{9}  & Expanded presence                            \\
            \nth{10} & Divine invocation                            \\
            \nth{11} & Discernment (invisibility)                   \\
            \nth{12} & Divine presence                              \\
            \nth{13} & Lingering presence                           \\
            \nth{14} & Divine invocation                            \\
            \nth{15} & Discernment (truth), martyr's retribution    \\
            \nth{16} & Divine presence                              \\
            \nth{17} & Mighty presence                              \\
            \nth{18} & Divine invocation                            \\
            \nth{19} & Discernment (thoughts)                       \\
            \nth{20} & Aligned soul, divine presence                 \\
        \end{dtabularx}
    \end{dtable}

    \classbasics{Alignment} Any other than true neutral.

    \classbasics{Class Skills}
    \subparhead{Dexterity} Ride.
    \subparhead{Intelligence} Heal, Knowledge (local, religion).
    \subparhead{Perception} Awareness, Intimidate, Sense Motive.
    \subparhead{Other} Bluff, Intimidate, Persuasion.

    \subsection{Base Class Abilities}
        A character with paladin as a base class gains the following abilities.

        \classbasics{Skill Points} 5.

        \classbasics{Defenses} \plus4 Fortitude, \plus4 Mental.

        \classbasics{Action Points} \plus2.

        \cf{Pal}{Weapon and Armor Proficiency}
        Paladins are proficient with simple weapons, any three other weapon groups, all types of armor (heavy, medium, and light), and shields.

        \cf{Pal}{Divine Protection}[Mag]
        The paladin's force of belief manifests a divine protection around her.
        She may add her Willpower to her Armor defense in place of Dexterity or Constitution (see \pcref{Defenses}).

    \subsection{Class Abilities}
        All paladins have the following abilities.

        \cf{Pal}{Devoted Alignment}
        A paladin is devoted to a specific alignment.
        She must choose one of her alignment components: good, evil, lawful, or chaotic.
        The alignment she chooses is her devoted alignment.
        A paladin's class abilities are affected by this choice.
        She excels at slaying creatures with alignments opposed to her devoted alignment.
        A paladin's devoted alignment cannot be changed without extraordinary repurcussions to the paladin.

        \cf{Pal}{Divine Power}
        Many paladin abilities depend on her divine power.
        A paladin's divine power is equal to her Willpower or her character level, whichever is higher.

        \cf{Pal}{Divine Invocation}[Mag]
        A paladin can invoke the power of her alignment to achieve incredible effects.
        This ability can be used a number of times per day equal to her Willpower or her paladin level, whichever is higher.
        She gains the smite divine invocation.

        At 2nd level, and every even level thereafter, the paladin gains an additional divine invocation.
        Most divine invocations have minimum paladin levels, as indicated in the title of the ability.
        Some divine invocations are also restricted to paladins with specific devoted alignments.
        The paladin's accuracy with divine invocations is equal to her divine power.
        If a divine invocation emulates a spell, the paladin's effective spellpower is equal to her divine power.

        Divine powers marked with an asterisk are called smite powers.
        Smite powers function like the smite divine invocation, except that they also have additional effects.
        Unless otherwise noted, these additional effects only occur if the smited creature does not share the paladin's devoted alignment.

        \parhead{Any Alignment}

        \subcf{Smite}
        Once per round, when the paladin makes a strike, she may declare that strike to be a smite.
        She may use her divine power in place of her normal accuracy for that strike.
        If the struck creatture shares her devoted alignment, she deals no damage at all (not even normal weapon damage), but the use of the ability is still spent.
        Otherwise, her weapon deals maximum damage, and she deals bonus damage equal to her divine power.
        \subcf{2nd -- Bless} This invocation functions like the \spell{bless} spell.
        \subcf{2nd -- Lay on Hands}
        As a standard action, the paladin can lay hands on a creature.
        The target is healed for 1d6 damage per divine power.
        \subcf{2nd -- Protection from Alignment} This invocation functions like the \spell{protection from alignment} spell.
        The paladin must protect the target from the alignment opposed to her devoted alignment.
        \subcf{6th -- Exhausting Smite*}
        The paladin makes a Divine power vs. Fortitude attack against the struck creature.
        Success means it is \exhausted for 2 rounds.
        \subcf{6th -- Resounding Smite*}
        The struck creature is knocked prone.
        \subcf{6th -- Seeking Smite*}
        This smite attack ignores any miss chances, such as from active cover or visual impairment.
        The weapon must still be physically able to strike the target.
        \subcf{6th -- Goading Smite*}
        The struck creature is \goaded by the paladin for 2 rounds.
        \subcf{10th -- Dispelling Smite*}
        The struck creature is affected by \spell{dispel magic}.
        \subcf{10th -- Divine Might}
        This invocation functions like the \spell{divine might} spell.
        \subcf{10th -- Penetrating Smite*}
        The struck creature's armor is weakened.
        For the next 2 rounds, attacks against the target that would normally target Armor defense are instead made against the lower of the target's Armor and Reflex defenses.
        \subcf{14th -- Dazing Smite*}
        The struck creature is \dazed for 2 rounds.
        \subcf{14th -- Spellreaving Smite*}
        All spells and magical effects on the struck creature are dispelled.
        Spells and effects that cannot be removed by \spell{dispel magic} are unaffected.
        The paladin must have the dispelling smite invocation to choose this invocation.
        \subcf{14th -- Terrifying Smite*}
        The paladin makes a Divine power vs. Mental attack against the struck creature.
        Success means it is \frightened by her for 2 rounds.
        \subcf{18th -- Converting Smite*}
        The paladin's smite shows her foe the error of its ways.
        She makes a Divine power vs. Mental attack against the struck creature.
        Success means it is \confused for 2 rounds.
        Critical success means its alignment changes, and it gains the paladin's devoted alignment for 1 week.
        After that time, it can choose to return to its original alignment, or keep its new alignment permanently.
        Failure means it is \dazed for 2 rounds.
        \subcf{18th -- Immobilizing Smite*}
        The creature struck with this smite is \immobilized for 5 rounds.

        \parhead{Chaos Divine Invocations}
        \subcf{6th -- Confusion} This invocation functions like the \spell{confusion} spell, except that it affects targets within \rngmed range.
        \subcf{6th -- Freedom} This invocation functions like the \spell{freedom} spell.
        \subcf{10th -- Break the Chains}
        As a standard action, the paladin can break all shackles, bindings, and locks within a \arealarge radius burst of her.
        Nonmagical objects are automatically broken.
        To break magical objects, the paladin must make an attack with her divine power against a DR equal to 10 \add the object's spellpower.
        \subcf{10th -- Chaotic Redirection}
        As an immediate action, when the paladin or any of her allies within \rngclose range is struck by a physical attack, the paladin can redirect the attack to a random creature within \rngclose range of the paladin, including the paladin.
        The attack is made against that creature instead of its original target, using its original accuracy, and has its normal effects if it hits.
        After using this invocation, the paladin cannot use it for 5 rounds.
        \subcf{14th -- Discordant Song} This invocation functions like the \spell{discordant song} spell.

        \parhead{Good Divine Invocations}
        \subcf{Shield Other}
        This invocation functions like the \spell{share pain} spell.
        \subcf{6th -- Martyr's Shield}
        As an immediate action, when an ally within \rngmed range would take vital damage, the paladin can take that damage as regular damage instead.
        After using this invocation, the paladin cannot use it again for 5 rounds.

        \parhead{Evil Divine Invocations}
        \subcf{2nd -- Enfeeblement} This invocation functions like the \spell{enfeeblement} spell.
        \subcf{6th -- Agony} This invocation functions like the \spell{agony} spell.
        \subcf{10th -- Enervation} This invocation functions like the \spell{enervation} spell.
        \subcf{10th -- Executing Smite*}
        The paladin makes a Divine power vs. Fortitude attack against the struck creature.
        Success means the target dies if it has no hit points remaining after taking damage from the smite.

        \parhead{Law Divine Invocations}
        \subcf{2nd -- Command} This invocation functions like the \spell{command} spell.
        \subcf{2nd -- Hold Person} This invocation functions like the \spell{hold person} spell.
        \subcf{6th -- Read Mind} This invocation functions like the \spell{read mind} spell.
        \subcf{10th -- Prohibition} This invocation functions like the \spell{prohibition} spell.

        \cf{Pal}[3rd]{Discernment}[Mag]
        A paladin can discern truths about creatures she sees as a swift action.
        When she uses this ability, she learns which creatures within a \arealarge cone have her devoted alignment.

        The paladin may use her discernment ability a number of times per day equal to her Perception or half her paladin level (minimum 1), whichever is higher.

        At 7th level, the paladin can discern lies.
        Whenever a creature in the area intentionally lies, the paladin knows the statement was a lie.
        This does not reveal the truth, uncover unintentional inaccuracies, or necessarily reveal evasions.

        At 11th level, the paladin can also see invisible creatures and objects within the area as if they were normally visible.
        They are visible as translucent shapes, allowing her to easily distinguish between visible and invisible creatures and objects.

        At 15th level, the paladin can discern truth.
        When a lie is spoken, in addition to learning that the statement is a lie, the paladin learns what the creature believes the truth to be.
        This does not necessarily reveal the actual truth -- merely what the creature believes.

        At 19th level, the paladin can discern thoughts.
        She knows the surface thoughts of all creatures in the area.
        She gains a \plus4 bonus to Bluff, Persuasion, and Intimidate attacks and checks against a creature whose thoughts she can read.

        \cf{Pal}[4th]{Divine Presence}[Mag]
        The paladin's presence alters the world around her.
        She chooses a single divine presence from the list below.
        Each divine presence affects a \areamed radius emanation from the paladin, including herself.
        She may choose to suppress or resume her divine presence as a swift action.

        At her 8th paladin level, and every 4 levels thereafter, the paladin gains an additional divine presence.
        She may have multiple divine presences active simultaneously, and suppress or resume them individually.
        Most divine presences have minimum paladin levels, as indicated in the title of the ability.
        If a divine presence emulates a spell, the paladin's effective spellpower is equal to her paladin level or her Willpower, whichever is higher.

        \parhead{Any Alignment}
        \subcf{8th -- Worthy Foe}
        Enemies in the area are \goaded by the paladin.
        \subcf{16th -- Aura of Unbending Purpose}
        Allies in the area are immune to \glossterm{Mind} effects.

        \parhead{Chaotic Divine Presences}
        \subcf{Mobile Aura}
        Allies in the area can move at full speed through threatened squares.
        \subcf{8th -- Accelerating Aura}
        Allies in the area gain a \plus10 foot bonus to land speed.
        \subcf{12th -- Aura of Freedom}
        Allies in the area gain the benefits of the \spell{freedom} spell.
        \subcf{20th -- Maddening Aura}
        At the start of each round, enemies in the area are \disoriented that round.

        \parhead{Evil Divine Presences}
        \subcf{Overwhelming Aura}
        Enemies in the area that are \glossterm{overwhelmed} increase their overwhelm penalties by 1.
        \subcf{8th -- Lifefeeding Aura}
        At the end of each round, if another creature in the area took damage, the paladin regains hit points equal to the damage taken, up to a maximum of her divine power.
        \subcf{12th -- Baleful Aura}
        Enemies in the area are \impaired with attacks and checks.
        \subcf{12th -- Painful Aura}
        Whenever an enemy in the area takes damage, the damage is doubled, up to a maximum bonus equal to the paladin's divine power.
        Each enemy can only take this bonus damage each round.
        \subcf{16th -- Sacrificial Aura}
        Whenever you take damage, any other creature in the area may choose to take that damage instead.
        This damage ignores all forms of damage reduction and similar abilities.
        Any damage in excess of the creature's hit points is not redirected.

        \parhead{Good Divine Presences}
        \subcf{Defensive Aura}
        Allies in the area that are \glossterm{overwhelmed} reduce their overwhelm penalties by 1.
        If an ally's overwhelm penalty is reduced to 0, they are not considered to be overwhelmed.
        \subcf{8th -- Minor Healing Aura}
        Whenever an ally in the area regains hit points, the healing is doubled, up to a maximum bonus equal to the paladin's divine power.
        Each ally can only receive this bonus once per round.
        \subcf{12th -- Healing Aura}
        At the end of each round, allies in the area heal hit points equal to the paladin's divine power.
        \subcf{16th -- Martyr's Aura}
        Whenever an ally in the area takes damage, the paladin may choose to take that damage instead.
        This damage ignores all forms of damage reduction and similar abilities.
        If the paladin takes damage in excess of her hit points in this way, the excess damage is dealt directly as vital damage.

        \parhead{Lawful Divine Presences}
        \subcf{Aura of Fortification}
        Enemies in the area move at half speed through threatened squares.
        \subcf{8th -- Aura of Inhibition}
        Enemies in the area move at half speed.
        \subcf{12th -- Aura of Truth}
        All illusory figments and glamers are suppressed in the area.
        \subcf{20th -- Inescapable Aura}
        At the start of each round, enemies in the area are \immobilized that round.

        \cf{Pal}[5th]{Pass Judgment}[Mag]
        The paladin gains the ability to pass judgment on those she deems unworthy.
        Once per day as a swift action, she may pass judgment on a creature within 100 feet of her.
        For the purpose of spells and effects, the creature is treated as if it had the alignment opposed to the paladin's devoted alignment.
        This effect lasts for one week, or until the paladin changes her mind about the subject.
        This does not change the creature's actions or behavior, but the creature is subject to the paladin's smite attack, would detect as that alignment under the inspection of a \spell{detect alignment} spell, and so on.

        No attack is required for this effect, and it cannot be dispelled, but a \spell{remove curse}, \spell{miracle}, or \spell{wish} spell can remove it.
        The paladin can use this ability an additional time per day at her 9th paladin level and every four levels thereafter.
        A paladin should be careful when using this ability, as persecution of allies can lead overzealous paladins to fall.

        \cf{Pal}[9th]{Expanded Presence}[Mag]
        The paladin's divine presences affect an \arealarge radius.

        \cf{Pal}[13th]{Lingering Presence}[Mag]
        The effects of the paladin's divine presences continue for 1 round after targets leave the area.

        \cf{Pal}[15th]{Martyr's Retribution}[Mag]
        If the paladin dies in the devoted service of her devoted alignment, she may choose to have her fallen body erupt in an immense burst of divine energy.
        If she does, her body is almost completely consumed, preventing her from being raised with \ritual{resurrection} and similar effects that require an intact body.
        This burst has two effects.
        First, a \spell{sunburst} spell immediately takes effect over the area where the paladin fell.
        Second, a \spell{storm of vengeance} spell begins to take effect, centered on the same area.
        The spell lasts for 10 rounds, and the lightning strikes target the paladin's enemies.
        Both of these effects harm only the paladin's foes, and do not harm her allies.
        However, her allies' vision is still impeded by the \spell{storm of vengeance}.

        \cf{Pal}[17th]{Mighty Presence}[Mag]
        The paladin's divine presences affect an \areahuge radius.
        In addition, their effects continue for 2 rounds after targets leave the area.

        \cf{Pal}[20th]{Aligned Soul}[Mag]
        While a paladin is dead, she may approach the deity or governing figure of her afterlife and request to be returned to life to continue her mission.
        Travelling to the relevant figure and making the request takes 12 hours.
        Unless there are extenuating circumstances, this request is almost always granted, and the paladin is resurrected in a new body at a location of the entity's choice.
        This functions like the \spell{resurrection} ritual, except that no part of the body is required, and a new body is created by the entity.
        She can be resurrected in this way regardless of the condition of her body, but not if her soul has been trapped or otherwise prevented from going to the correct afterlife.

        \subsubsection{Ex-Paladins}
            A paladin who ceases to follow her devoted alignment loses all magical paladin class abilities.
            She may not gain any additional paladin levels.
            She regains her abilities and advancement potential if she atones for her violations (see the \spell{atonement} ritual), as appropriate.

\section{Ranger}\label{Ranger}
    \begin{dtable}
        \lcaption{Ranger Progression}
        \begin{dtabularx}{\columnwidth}{>{\ccol}p{\levelcol} c >{\lcol}X}
            \tb{Level} & \tb{Quarry} & \tb{Special} \\
            \hline
            \nth{1}  & \plus2 & Quarry, tenacious hunter, wild speech \\
            \nth{2}  & \plus2 & Hunting skill, tracker                \\
            \nth{3}  & \plus2 & Free stride, keen vision              \\
            \nth{4}  & \plus2 & Hunting lore                          \\
            \nth{5}  & \plus3 & Rapid tracker                         \\
            \nth{6}  & \plus3 & Hunting skill                         \\
            \nth{7}  & \plus3 & Blindsense                            \\
            \nth{8}  & \plus3 & Hunting lore                          \\
            \nth{9}  & \plus3 & Perfect stride                        \\
            \nth{10} & \plus4 & Hunting skill                         \\
            \nth{11} & \plus4 & Blindsight                            \\
            \nth{12} & \plus4 & Hunting lore                          \\
            \nth{13} & \plus4 & Implacable hunter                     \\
            \nth{14} & \plus4 & Hunting skill                         \\
            \nth{15} & \plus5 & Farsight                              \\
            \nth{16} & \plus5 & Hunting lore                          \\
            \nth{17} & \plus5 & Eternal quarry                        \\
            \nth{18} & \plus5 & Hunting skill                         \\
            \nth{19} & \plus5 & Truesight                             \\
            \nth{20} & \plus6 & Hunting lore
        \end{dtabularx}
    \end{dtable}

    \classbasics{Alignment} Any.

    \classbasics{Class Skills}
    \subparhead{Strength} Climb, Jump, Sprint, Swim.
    \subparhead{Dexterity} Acrobatics, Escape Artist, Ride, Stealth.
    \subparhead{Intelligence} Heal, Knowledge (dungeoneering, geography, nature).
    \subparhead{Perception} Awareness, Creature Handling, Survival.
    \subparhead{Other} Bluff, Intimidate, Persuasion.

    \subsection{Base Class Abilities}
        A character with ranger as a base class gains the following abilities.

        \classbasics{Skill Points} 15.

        \classbasics{Defenses} \plus4 Fortitude, \plus4 Reflex, \plus2 Mental.

        \classbasics{Action Points} \plus1.

        \cf{Rgr}{Weapon and Armor Proficiency}
        A ranger is proficient with simple weapons, any two weapon groups, light and medium armor, and shields.
        He is also proficient with his choice of bows, crossbows, or thrown weapons.

        \cf{Rgr}{Tenacious Hunter}[Ex]
        The ranger adds his quarry bonus to his defenses against attacks that his quarry makes.
        See the Quarry ability, below, for details.

        \cf{Rgr}{Wild Speech}[Mag]
        The ranger learns how to communicate with animals.
        This ability functions like the druid ability of the same name (see Wild Speech, \pref{Drd:Wild Speech}).
        A ranger can use this ability a number of times per day equal to his Perception or half his ranger level, whichever is higher.

    \subsection{Class Abilities}
        All rangers have the following abilities.

        \cf{Rgr}{Quarry}
        As a swift action, a ranger may designate any foe he sees as his quarry.
        A ranger gains a \plus2 bonus to damage with physical attacks and Awareness, Stealth, and Survival checks against his quarry.
        However, while a ranger has designated a quarry, he takes a \minus2 penalty on the same rolls against any target other than his quarry.

        A ranger may not normally have more than one quarry at once.
        He may not designate a new quarry until he defeats his old quarry, or until he gives up on the quarry.
        He may give up pursuing a quarry as a free action.
        If he does, he is unable to designate a new quarry until he rests for 5 minutes.

        Some abilities allow the ranger to designate multiple creatures as a quarry.
        Abilities which designate multiple creatures as a quarry always last for a specific amount of time.

        If the ranger does not see his quarry for more than a week, it is no longer considered his quarry.

        The amount by which a ranger's damage and skill checks increase against his quarry is called his quarry bonus.
        The ranger's quarry bonus improves by \plus1 at his 5th ranger level and every 5 ranger levels thereafter.
        His penalties against targets other than his quarry remains the same.

        \cf{Rgr}[2nd]{Hunting Skill}
        The ranger gains any Skill feat with prerequisites that include one of the ranger class skills as a bonus feat.
        He must use his ranger level in place of his character level to meet level prerequisites for the feat.
        At his 6th ranger level, and every 4 ranger levels thereafter, he gains an additional feat.

        \cf{Rgr}[2nd]{Tracker}
        The ranger gains a \plus5 bonus to checks made to follow tracks.
        In addition, he may use his ranger level in place of the Survival skill to follow tracks (see \pcref{Survival}).

        \cf{Rgr}[4th]{Hunting Lore}
        The ranger gains an ability drawn from ancient hunting lore.
        He chooses a single hunting lore from the list below.
        Some hunting lores have minimum ranger levels, as indicated in the title of the ability.
        At his 8th ranger level, and every four ranger levels thereafter, the ranger gains an additional hunting lore.

        If a hunting lore grants the ranger a bonus feat, he must still meet any prerequisites for the feat to gain its effect.

        \subcf{Dual Quarry}
        The ranger may designate up to two targets whenever he chooses a quarry.
        He gains his quarry benefits against both targets.
        Whenever he designates new quarries, he may choose which previous quarry to give up (if any).

        \subcf{Endurance}
        The ranger gains the Endurance feat as a bonus feat (see \featpcref{Endurance}).

        \subcf{Goading Hunter}
        Whenever the ranger damages his quarry with a physical attack, it is \goaded by him for 2 rounds.

        \subcf{No Escape}
        Whenever the ranger damages his quarry with a physical attack, it moves at half speed for 2 rounds.

        \subcf{Swift Hunter}
        The ranger gains the Swift feat as a bonus feat (see \featpcref{Swift}).

        \subcf{8th -- Anchored Quarry}
        Whenever the ranger damages his quarry with a physical attack, it cannot travel extradimensionally for 2 rounds.
        This blocks teleportation and all planar travel abilities except planar rifts.

        \subcf{8th -- Flexible Quarry}
        After giving up on a quarry, the ranger does not need to wait before designating a new quarry.

        \subcf{8th -- Impaired Quarry}
        Whenever the ranger damages his quarry with a physical attack, it is \impaired with attacks and checks for 2 rounds.

        \subcf{12th -- Inescapable}
        Whenever the ranger damages his quarry with a physical attack, it moves at one-quarter speed as long as it remains his quarry.

        \subcf{12th -- Inevitable Hunter}
        The ranger ignores all miss chances and failure chances that would affect attacks and checks he makes against his quarry.

        \subcf{12th -- Punishing Presence}
        At the end of each round, if the ranger's quarry is within \rnglong range of him, it takes life damage equal to his ranger level.

        \subcf{16th -- Legendary Hunter}
        The ranger may always use legend points on attacks or checks he makes against his quarry, even when fighting monsters that normally prevent legend points from being used.

        \subcf{16th -- Master of the Hunt}
        When the ranger uses his quarry ability, he may also designate up to five willing creatures within \rnglong range of him.
        As long as they stay within \rnglong range of him, they also gain the damage bonus from the ranger's quarry ability against his quarry.
        His allies do not suffer penalties against targets other than the quarry.

        \cf{Rgr}[3rd]{Free Stride}
        The ranger can move through any sort of natural terrain that slows or impedes movement at his normal speed without suffering any sort of impairment.
        If a skill check, such as Climb or Swim, would normally be required to move through the terrain, this ability does not help.

        \cf{Rgr}[3rd]{Keen Vision}
        The ranger's sight improves, allowing him to see more easily.
        He gains \glossterm{low-light vision}, allowing him to treat sources of light as if they had double their normal illumination range.
        If he already has low-light vision, he doubles its benefit, allowing him to treat sources of light as if they had four times their normal illumination range.
        In addition, he gains \glossterm{darkvision} out to 50 feet, allowing him to see in complete darkness.
        If he already has darkvision, he increases its range by 50 feet.

        \cf{Rgr}[5th]{Rapid Tracker}
        The ranger's ability to track his foes improves.
        He can move at his normal speed while following tracks without taking the normal \minus5 penalty.
        He takes only a \minus10 penalty (instead of the normal \minus20) when moving at up to twice normal speed while tracking.

        \cf{Rgr}[7th]{Blindsense}
        The ranger's perceptions are so finely honed that he can sense his enemies without seeing them.
        He gains the blindsense ability out to 50 feet.
        This ability allows him to sense the presence and location of objects and foes within 50 feet without seeing them.
        If he already has the blindsense ability, he increases its range by 50 feet.

        In addition, he increases the range of his darkvision by 150 feet.

        \cf{Rgr}[9th]{Perfect Stride}[Mag]
        The ranger's ability to surpass obstacles becomes unparalleled.
        He constantly acts as if he were under the effect of a \spell{freedom} spell, except that it does not allow him to act normally underwater.

        \cf{Rgr}[11th]{Blindsight}
        The ranger gains the \glossterm{blindsight} ability, allowing him to ``see'' perfectly without his eyes in a 50 foot radius around him.
        With this ability, he can fight just as well with his eyes closed as with them open.
        If he already has the blindsight ability, he increases its range by 50 feet.

        In addition, he increases the range of his darkvision by 300 feet, and his blindsense by 150 feet.

        \cf{Rgr}[13th]{Implacable Hunter}[Mag]
        The ranger cannot be blocked from pursuing his quarry.
        His movement is not slowed by enemies threatening him (see \pcref{Moving Near Foes}).
        In addition, he may move at half speed through spaces occupied by foes.

        \cf{Rgr}[15th]{Farsight}
        The ranger increases the range of his darkvision by 500 feet, his blindsense by 300 feet, and his blindsight by 50 feet.
        In addition, he halves his \glossterm{range increment penalties} for attacking at long range.

        \cf{Rgr}[17th]{Eternal Quarry}
        The ranger's quarry never stops being his quarry until he chooses a different creature.
        In addition, once he has designated a creature as a quarry, he can always designate it as a quarry again, even if it is not in his sight.
        This only applies to creatures that he designates as a quarry after acquiring this ability.

        \cf{Rgr}[19th]{Truesight}[Mag]
        The ranger's perceptions are accurate enough to defeat even powerful magic.
        He can see through normal and magical darkness, see the truth behind visual figments and glamers, and see the true form of creatures and objects affected by \glossterm{Shaping} abilities.
        This ability works at any range.

\section{Rogue}\label{Rogue}
    \begin{dtable}
        \lcaption{Rogue Progression}
        \begin{dtabularx}{\columnwidth}{>{\ccol}p{\levelcol} c >{\lcol}X}
            \tb{Level} & \tb{Sneak Attack} & \tb{Special} \\
            \hline
            \nth{1}  & \plus1d6  & Skill exemplar, sneak attack \\
            \nth{2}  & \plus1d6  & Skill talent                 \\
            \nth{3}  & \plus2d6  & Uncanny dodge                \\
            \nth{4}  & \plus2d6  & Combat trick                 \\
            \nth{5}  & \plus3d6  & Lucky slip                   \\
            \nth{6}  & \plus3d6  & Skill talent                 \\
            \nth{7}  & \plus4d6  & Improved uncanny dodge       \\
            \nth{8}  & \plus4d6  & Combat trick                 \\
            \nth{9}  & \plus5d6  & Lucky break                  \\
            \nth{10} & \plus5d6  & Skill talent                 \\
            \nth{11} & \plus6d6  & Slippery mind                \\
            \nth{12} & \plus6d6  & Combat trick                 \\
            \nth{13} & \plus7d6  & Lucky dodge                  \\
            \nth{14} & \plus7d6  & Skill talent                 \\
            \nth{15} & \plus8d6  & Persistent sneak attack      \\
            \nth{16} & \plus8d6  & Combat trick                 \\
            \nth{17} & \plus9d6  & Legendary luck               \\
            \nth{18} & \plus9d6  & Skill talent                 \\
            \nth{19} & \plus10d6 & Ambush master                \\
            \nth{20} & \plus10d6 & Combat trick                 \\
        \end{dtabularx}
    \end{dtable}

    \classbasics{Alignment} Any.

    \classbasics{Class Skills}
    \subparhead{Strength} Climb, Jump, Sprint, Swim.
    \subparhead{Dexterity} Acrobatics, Escape Artist, Sleight of Hand, Stealth.
    \subparhead{Intelligence} Devices, Disguise, Knowledge (dungeoneering, local), Linguistics.
    \subparhead{Perception} Awareness, Sense Motive.
    \subparhead{Other} Bluff, Intimidate, Perform, Persuasion.

    \subsection{Base Class Abilities}
        A character with rogue as a base class gains the following abilities.

        \classbasics{Skill Points} 15.

        \classbasics{Defenses} \plus4 Reflex, \plus2 Mental.

        \classbasics{Action Points} \plus2.

        \cf{Rog}{Weapon and Armor Proficiency}
        Rogues are proficient with simple weapons, any two weapon groups, light armor, and bucklers.
        They are also proficient with saps.

        \cf{Rog}{Skill Exemplar}
        A rogue gains a \plus2 bonus to all skills she is trained in.
        This bonus increases by 1 at her 5th rogue level and every 5 rogue levels thereafter.

    \subsection{Class Abilities}
        All rogues have the following abilities.

        \cf{Rog}{Sneak Attack}
        Once per round, when a rogue hits with a physical attack against a creature unable to defend itself effectively, she can make that attack a sneak attack.
        She can make a sneak attack if her target is \unaware, \defenseless, or suffering \glossterm{overwhelm penalties} from being surrounded by enemies (see \pcref{Overwhelm}).
        A sneak attack deals 1d6 points of bonus damage.

        The extra damage dealt by a sneak attack increases by 1d6 at her 3rd rogue level and every two rogue levels thereafter.

        Ranged attacks can count as sneak attacks only if the target is within 30 feet.
        A rogue can't strike with deadly accuracy from beyond that range.

        A rogue can only sneak attack creatures with a discernible body structure.
        Oozes, incorporeal creatures, and some plants lack vital areas to attack.
        Any creature that is immune to critical hits is not vulnerable to sneak attacks.
        The rogue must be able to see the target well enough to pick out a vital spot and must be able to reach such a spot.
        A rogue cannot sneak attack while striking the limbs of a creature whose vitals are beyond reach.
        Usually, this means she can only make sneak attacks against creatures no more than two size categories larger than her.

        \cf{Rog}[2nd]{Skill Talent}
        The rogue's skills improve.
        She gains an additional skill point, which she can place in any skill, and a bonus Skill feat for which she qualifies (see \pcref{Skill Feats}).
        She must use her rogue level in place of her character level to meet level prerequisites for the feat.
        At her 6th rogue level, and every four rogue levels thereafter, she gains an additional skill point and Skill feat.

        \cf{Rog}[3rd]{Uncanny Dodge}
        The rogue can react to danger before her senses would normally allow her to do so.
        The rogue reduces her overwhelm penalties by 1.
        If her overwhelm penalty is reduced to 0, she is not considered to be overwhelmed.
        In addition, she is not \unaware when attacked by surprise.

        \cf{Rog}[4th]{Combat Tricks}
        The rogue gains a combat trick to aid her and confound her foes.
        She chooses a single combat trick from the list below.
        Some combat tricks have minimum rogue levels, as indicated in the title of the ability.
        At her 8th rogue level, and every four rogue levels thereafter, the rogue gains an additional combat trick.

        Some combat tricks depend on a rogue's trick power.
        A rogue's trick power is equal to her Intelligence or her character level, whichever is higher.

        Tricks marked with an asterisk are called ambush attacks.
        Ambush attacks only function on the first sneak attack the rogue makes against a particular creature in an encounter.

        \subcf{Confusing Ambush*}
        The rogue makes a Trick power vs. Mental attack against the creature damaged by this ambush attack.
        Success means the struck creature is \dazed for 2 rounds.
        Critical success means it is instead \confused for 2 rounds.
        This is an ambush attack, and only works once per creature.

        \subcf{Distracting Ambush*}
        A creature damaged by this ambush attack automatically fails any Concentration checks it makes that round.
        This is an ambush attack, and only works once per creature.

        \subcf{Distant Precision}
        The rogue can make sneak attacks from up to 100 feet away.

        \subcf{Hamstring*}
        A creature damaged by this ambush attack has its land speed halved for 5 rounds.
        Despite the name, this can be used on creatures who do not have hamstrings.
        This is an ambush attack, and only works once per creature.

        \subcf{Merciful Blows}
        The rogue suffers no penalty to damage when attacking for \glossterm{nonlethal damage}, and can deal her full sneak attack damage when attacking nonlethally.

        %\subcf{Swift Poisoner}
        %The rogue can apply poison to a weapon she is holding as a swift action.

        \subcf{Tricky Maneuver}
        When performing a \glossterm{maneuver} against an \glossterm{overwhelmed} or \unaware creature, the rogue gains a bonus to accuracy equal to the number of sneak attack dice she would roll.
        The benefits of this trick apply even against creatures immune to critical hits.

        \subcf{8th -- Brutal Ambush*}
        The rogue rolls d8s instead of d6s for her sneak attack dice on this ambush attack.
        This is an ambush attack, and only works once per creature.

        \subcf{8th -- Immobilizing Ambush*}
        A creature damaged by this ambush is \immobilized for 2 rounds.
        This is an ambush attack, and only works once per creature.

        \subcf{12th -- Agonizing Ambush*}
        The rogue makes a Trick power vs. Mental attack against the struck creature.
        Success means that the target feels agonizing pain for 2 rounds.
        Whenever it takes physical damage, it takes additional physical damage equal to the rogue's trick power.
        This is an ambush attack, and only works once per creature.

        \subcf{12th -- Assassination}
        To use this ability, the rogue must spend a full round studying a creature within 100 feet of her who has not noticed her and who is not in combat.
        If she make a melee sneak attack against that target within 1 round, her attack deals maximum damage, including her sneak attack damage.
        If the target becomes aware of her presence before she attacks, this ability has no benefit.

        \subcf{12th -- Perfect Precision}
        The rogue has no range limit on her sneak attacks.

        \subcf{16th -- Deadly Ambush*}
        The rogue makes a Trick power vs. Fortitude attack against the struck creature.
        Success means the target is \staggered, and dies if it loses all its hit points for 2 rounds.
        Critical success means the target immediately dies.
        This is an ambush attack, and only works once per creature.
        In addition, dying in this way is a \glossterm{Death} effect.

        \subcf{20th -- Dual Ambush}
        The rogue can apply the benefits of two ambush attacks to a single sneak attack.

        \subcf{20th -- Lingering Ambush}
        The effects of the rogue's ambush attacks last ten times longer than normal.
        This has no effect on ambush attacks that have no duration.

        \cf{Rog}[5th]{Lucky Slip}[Mag]
        The rogue gains an additional \glossterm{legend point}.
        Once per round, when she spends a legend point to force a foe to reroll a \glossterm{strike}, she may use this ability to treat the reroll as a 1.

        \cf{Rog}[7th]{Improved Uncanny Dodge}
        The rogue reduces her overwhelm penalties by 2.
        This does not stack with the effects of uncanny dodge.
        If her overwhelm penalty is reduced to 0, she is not considered to be overwhelmed.

        \cf{Rog}[9th]{Lucky Break}[Mag]
        The rogue gains an additional \glossterm{legend point}.
        Once per round, when she spends a legend point to reroll a skill check, she may use this ability to treat the reroll as a 20.

        \cf{Rog}[11th]{Slippery Mind}
        The rogue gains a \plus4 bonus to Mental defense.

        \cf{Rog}[13th]{Lucky Dodge}[Mag]
        The rogue gains an additional \glossterm{legend point}.
        In addition, she can use her lucky slip ability against all attacks, not just strikes.

        \cf{Rog}[15th]{Persistent Sneak Attack}
        The rogue can make two sneak attacks per round, rather than one.

        \cf{Rog}[17th]{Legendary Luck}[Mag]
        The rogue gains an additional \glossterm{legend point}.
        She can always use legend points, even when fighting monsters that normally prevent legend points from being used.

        \cf{Rog}[19th]{Ambush Master}
        The rogue's ambush attacks function on the first attack the rogue makes against a particular creature in a single round, rather than within a single encounter.

\section{Spellwarped}\label{Spellwarped}
    \begin{dtable}
        \lcaption{Spellwarped Progression}
        \begin{dtabularx}{\columnwidth}{>{\ccol}p{\levelcol} >{\lcol}X}
            \tb{Level} & \tb{Special} \\
            \hline
            \nth{1}  & Innate magic, invocation, warp regeneration \\
            \nth{2}  & Spellwarped body                            \\
            \nth{3}  & Magical senses, spellwarped aspect          \\
            \nth{4}  & Invocation                                  \\
            \nth{5}  & Manipulate magic                            \\
            \nth{6}  & Invocation                                  \\
            \nth{7}  & Spellwarped aspect                          \\
            \nth{8}  & Invocation                                  \\
            \nth{9}  & Magic resistance                            \\
            \nth{10} & Invocation                                  \\
            \nth{11} & Spellwarped aspect                          \\
            \nth{12} & Invocation                                  \\
            \nth{13} & Improved manipulate magic                   \\
            \nth{14} & Invocation                                  \\
            \nth{15} & Spellwarped aspect                          \\
            \nth{16} & Invocation                                  \\
            \nth{17} & \tdash                                      \\
            \nth{18} & Invocation                                  \\
            \nth{19} & Spellwarped aspect                          \\
            \nth{20} & Invocation                                  \\
        \end{dtabularx}
    \end{dtable}

    \classbasics{Alignment} Any.

    \classbasics{Class Skills}
    \subparhead{Intelligence} Knowledge (arcana).
    \subparhead{Perception} Awareness, Spellcraft.
    \subparhead{Other} Bluff, Intimidate, Persuasion.

    \classbasics{Special Class Skills}
    A spellwarped gains additional class skills based on his choice of innate magic.
    \subparhead{Alteration} Disguise, Escape Artist, Jump.
    \subparhead{Pyromancy} Acrobatics, Jump, Sprint.
    \subparhead{Telekinesis} Devices, Escape Artist, Sleight of Hand.
    \subparhead{Temporal} Acrobatics, Sleight of Hand, Sprint.

    \subsection{Warp Damage}\label{Warp Damage}
        A spellwarped can use the innate magic within his body to generate powerful magical effects.
        However, doing so is physically taxing.
        Most spellwarped abilities cause the spellwarped to take some amount of \glossterm{warp damage}.
        Warp damage cannot be cured by effects that restore hit points, effectively reducing the spellwarped's maximum hit points.

        A spellwarped cannot voluntarily take warp damage that would exceed half his maximum hit points.
        An hour of rest cures warp damage equal to a character's level.

    \subsection{Base Class Abilities}
        A character with spellwarped as a base class gains the following abilities.

        \classbasics{Skill Points} 5.

        \classbasics{Defenses} \plus4 to one defense, \plus2 to other defenses (see the Innate Magic ability, below).

        \classbasics{Action Points} \plus3.

        \cf{Spl}{Weapon and Armor Proficiency}
        A spellwarped is proficient with simple weapons, any two weapon groups, light and medium armor, and shields.

        \cf{Spl}{Warp Regeneration}[Mag]
        The spellwarped does not need to rest to cure warp damage.
        He automatically heals warp damage equal to his level every hour, regardless of any activity he takes in the meantime.
        If he rests, he also recovers warp damage for resting, doubling the warp damage he heals.

    \subsection{Class Abilities}
        All spellwarped have the following abilities.

        \cf{Spl}{Spellpower}
        The strength of a spellwarped's spells and abilities are determined by his spellpower.
        His spellpower is equal to his key attribute or his character level, whichever is higher.

        \cf{Spl}{Innate Magic}
        Each spellwarped draws his magical power from a particular kind of magic.
        This is a choice made when the first level of the class is taken, and it cannot thereafter be changed.
        The choices are listed below.
        His choice of innate magic is not a \glossterm{magical} ability, but the active ability granted by that choice is a magical ability.

        \subcf{Alteration}
        The spellwarped can manipulate the physical forms of creatures.
        His good defense is Fortitude, his key attribute is Intelligence, and he treats Disguise, Escape Artist, and Jump as class skills.
        An alteration spellwarped may be called an alterer, bodywarper, or shifter.

        As a standard action, he can change minor aspects of his appearance, such as removing a mole or lengthening his beard.
        This can grant him a \plus2 bonus to Disguise checks.
        Major changes are not possible.

        \subcf{Pyromancy}
        The spellwarped can manipulate fire and heat.
        His good defense is Fortitude, his key attribute is Willpower, and he treats Acrobatics, Jump, and Sprint as class skills.
        A pyromancy spellwarped may be called a pyromancer.

        As a standard action, he can snap his fingers to create a small ember of flame in his hand for 5 minutes.
        This ember casts light as a torch, and can deal 1 point of fire damage with a successful touch attack.
        It can be dismissed as a swift action or extinguished as a move action.

        \subcf{Telekinesis}
        The spellwarped can manipulate objects and creatures with his mind.
        His good defense is Mental, his key attribute is Willpower, and he treats Devices, Escape Artist, and Sleight of Hand as class skills.
        A telekinesis spellwarped may be called a telekine.

        As a standard action, he can concentrate to move objects within ten feet of him telekinetically.
        He can slowly lift or manipulate one object by up to one foot per round.
        The object can weigh up to five pounds.
        This level of control is insufficient to make skill checks or wield a weapon or shield effectively.

        \subcf{Temporal}
        The spellwarped can manipulate time.
        His good defense is Reflex, his key attribute is Perception, and he treats Awareness, Sleight of Hand, and Sprint as class skills.
        A temporal spellwarped may be called a temporalist or timewarper.

        The spellwarped always knows exactly what time it is, and can track the passage of time precisely without effort.

        \cf{Spl}{Invocation}
        A spellwarped can invoke his innate magic to generate powerful effects.
        He chooses a single invocation at 1st level from those available based on his choice of innate magic.
        Using an invocation inflicts \glossterm{warp damage} to the spellwarped equal to half the minimum spellwarped level required to learn the invocation (minimum 1).

        At his 4th spellwarped level, and every two spellwarped levels thereafter, he gains an additional invocation.
        Some invocations have minimum spellwarped levels, as indicated in the title of the ability.
        The list of invocations is given at \pcref{Invocations}.

        The spellwarped's accuracy with invocations is equal to his spellpower.

        \cf{Spl}[2nd]{Spellwarped Body}[Mag]
        The spellwarped's body is fundamentally altered by exposure to magic.
        He shows signs of the magic coursing through his body: strangely or inconsistently colored hair, natural skin markings which often resemble runes, and so on.
        Anyone observing the spellwarped can make an Awareness or Spellcraft check with a DR equal to 20 \sub his spellwarped level to recognize that the character is a spellwarped.
        Critical success on the check allows the observer to determine the type of innate magic the spellwarped has.
        In addition, the spellwarped gains an ability based on his innate magic.
        \subcf{Alteration -- Sturdy Body}
        The spellwarped gains a \plus2 bonus to his Fortitude defense.
        This bonus increases by 1 at spellpower 5 and every 5 spellpower thereafter.
        \subcf{Pyromancy -- Energy Resistance}
        The spellwarped gains \glossterm{damage reduction} against cold and fire damage equal to twice his spellpower.
        \subcf{Telekinesis -- Tactile Telekinesis}
        The spellwarped gains a \plus2 bonus to Strength and Dexterity-based checks.
        This bonus increases by 1 at spellpower 5 and every 5 spellpower thereafter.
        \subcf{Temporal -- Accelerated Movement}
        The spellwarped gains a \plus2 bonus to his Reflex defense and a \plus10 foot bonus to his movement speed.
        At spellpower 8, 14, and 20, the defense bonus increases by 1 and the speed bonus increases by 10 feet.

        \cf{Spl}[3rd]{Magical Senses}[Mag]
        The spellwarped learns to recognize the telltale signs of his chosen magic.
        He must concentrate as a standard action to use this ability, and he may do so any number of times per day.
        \subcf{Alteration -- Perceive Alteration}
        The spellwarped can discern the true form of all creatures within 50 feet of him for 1 round, ignoring any effects which magically alter their shapes.
        This also grants him a \plus5 bonus to Awareness checks to see through disguises.
        \subcf{Pyromancy -- Flame of Life}
        The spellwarped can see the life-fire that lies within all living creatures, allowing him to clearly see all living creatures within 50 feet of him for 1 round.
        It also allows the spellwarped to see unusually warm objects, such as fires.
        This is a \glossterm{Detection} ability, and it can penetrate up to 1 foot of stone, 1 inch of common metal, a thin sheet of lead, or 3 feet of wood or dirt.
        \subcf{Telekinesis -- Spatial Awareness}
        The spellwarped can feel the forms of all objects and creatures around him, granting blindsense out to a 50 foot range for 1 round.
        \subcf{Temporal -- Rapid Search}
        The spellwarped can accelerate his mind to immediately search everything within a 10 foot radius of him with the Awareness skill as a standard action.
        Alternately, he may use this ability to read a book ten times as fast as normal.

        \cf{Spl}[3rd]{Spellwarped Aspect}[Mag]
        The spellwarped gains a new ability based on his continued exposure to magical energy.
        Most aspects are specific to particular kinds of innate magic, but some aspects can be taken by any spellwarped.
        These aspects are listed under the General heading.

        At his 7th spellwarped level, and every four spellwarped levels thereafter, the spellwarped gains an additional spellwarped aspect.
        Some aspects require a minimum spellwarped level, as indicated in the title of the ability.
        The full list of spellwarped aspects is given below.

        \parhead{General}

        \subcf{Resilient}
        The spellwarped increases the defense improved by his choice of innate magic by 2.
        This can increase his hit points, if appropriate.
        \subcf{Spellwarped Ritualist} The spellwarped can learn and perform rituals as if he were an arcane caster with a spellpower equal to his spellwarped spellpower.  The maximum level of ritual that he can learn or perform is equal to half his spellwarped level or half his spellwarped key attribute, whichever is lower.
        In addition, he gains an ability based on which type of spellwarped he is.
        \begin{itemize}
            \itemhead{Alteration} If a ritual requires a specific component with a value, the spellwarped can substitute its equivalent value in ritual components instead.
                This cannot be used to replace components without a value or components with special properties that alter the ritual's effect, such as the body for a \spell{resurrection} ritual.
            \itemhead{Pyromancy}
                The spellwarped can use any combustable item as a ritual component.
                It can replace an amount of normal ritual components equal to the value of the item.
                It cannot replace special ritual components.
            \itemhead{Telekinesis}
                The spellwarped can perform rituals from up to 30 feet away from the ritual components.
            \itemhead{Temporal}
                The spellwarped performs rituals twice as quickly.
        \end{itemize}
        \subcf{7th -- Warp Overload}
        The spellwarped can voluntarily take \glossterm{warp damage}, even if that would exceed half his maximum hit points.
        He is only unable to voluntarily take warp damage if the total warp damage would exceed his maximum hit points.
        \subcf{11th -- Persistent Senses}
        The spellwarped can constantly gain the benefit of his magical senses ability.
        He can toggle his enhanced senses on or off as a swift action.
        If the ability does not have a duration, such as the temporal magical senses ability, this aspect has no effect.

        \cf{Spl}[2nd]{Surge of Power}[Mag]
        \subcf{Telekinesis -- Kinetic Deflection}
        The spellwarped reflexively deflects attacks away with his mind.
        He gains a \plus2 bonus to his \glossterm{physical defenses}.
        \subcf{Temporal -- Accelerate Movement}
        The spellwarped accelerates his movement and reactions.
        He gains a \plus2 bonus to his Reflex defense and a \plus10 foot bonus to his movement speed.
        At spellpower 8, 14, and 20, the defense bonus increases by 1 and the speed bonus increases by 10 feet.

        \parhead{Alteration}
        \subcf{Alter Movement}
        The spellwarped gains his choice of the Acrobatics Mastery, Climb Mastery, Jump Mastery, or Swim Mastery feats.
        He may select this aspect multiple times, choosing a different bonus feat each time.
        \subcf{7th -- Damage Reduction}
        The spellwarped gains \glossterm{damage reduction} against physical damage equal to his spellpower.
        Adamantine weapons ignore this damage reduction and negate it for 1 round.
        \subcf{11th -- Alter Size}
        As a standard action, the spellwarped can grow or shrink by one size category, up to a maximum of one size category different from his normal size.
        This effect lasts until he uses this ability again.
        This is a \glossterm{Sizing} effect, and does not stack with other Sizing effects.
        \subcf{11th -- Regeneration}
        At the end of each round, the spellwarped heals hit points equal to his spellpower.

        \parhead{Pyromancy}
        \subcf{Flame Aura}
        The spellwarped emanates an aura of fire.
        At the end of each round, enemies adjacent to him take fire damage equal to his spellpower.
        He can suppress or resume this aura as a \glossterm{swift action}.
        \subcf{Intense Flames}
        The spellwarped can choose to have his spellwarped abilities ignore an amount of fire damage reduction equal to his spellpower.
        \subcf{7th -- Flame Eater}
        When the spellwarped resists fire damage with his spellwarped body ability, he heals hit points equal to the damage resisted.
        \subcf{7th -- Retributive Flames}
        When a creature makes a melee attack against the spellwarped, it takes \spelldamage{fire}[d6].
        Each creature can only take this damage once per round.
        \subcf{11th -- Intense Flame}
        The spellwarped gains a bonus to fire damage with all invocations equal to the number of dice he would roll for that invocation's fire damage.
        This does not affect invocations that do not deal fire damage.

        \parhead{Telekinesis}
        \subcf{Kinetic Deflection}
        The spellwarped may use his Intelligence in place of his Dexterity or Constitution to determine his \glossterm{physical defenses}.
        \subcf{Improved Object Manipulation}
        When the spellwarped uses his object manipulation ability, he can affect objects within 10 feet, with a weight limit of up to two pounds per spellpower.
        He has enough control to make checks with a DR of up to 10.
        \subcf{7th -- Force Barrier}
        The spellwarped gains a \plus1 bonus to \glossterm{physical defenses}.
        \subcf{7th -- Shieldbearer}
        The spellwarped may wield shields, except tower shields, telekinetically.
        The shield floats in his square, granting him its bonus to his physical defenses just as if he were wielding it.
        He does not need a free hand to wield the shield, and suffers no arcane spell failure for its use, though he is still affected by its \glossterm{encumbrance penalty}.
        The shield follows him as he moves.
        If it is forcibly removed from his square, he loses control over it and it falls to the ground.
        \subcf{11th -- Mind Armory}
        The spellwarped may control a number of weapons equal to half his Intelligence with his mind blade ability.
        This does not allow him to make additional attacks per round, but he may attack interchangeably with any weapon he controls.
        Each weapon threatens an area and contributes to overwhelm penalties, just as with his normal mind blade ability.

        \parhead{Temporal}
        \subcf{Accelerate Mind}
        The spellwarped gains a \plus2 bonus to Intelligence-based and Perception-based checks.
        \subcf{Sprint Mastery}
        The spellwarped gains the Sprint Mastery feat (see page \featpref{Sprint Mastery}).
        \subcf{Swift}
        The spellwarped gains the Swift feat (see page \featpref{Swift}).
        \subcf{Uncanny Dodge}
        The spellwarped can react to danger before his senses would normally allow him to do so.
        He reduces his overwhelm penalties by 1.
        If his overwhelm penalty is reduced to 0, he is not considered to be overwhelmed.
        In addition, he is not \unaware when attacked by surprise.
        \subcf{7th -- Improved Uncanny Dodge}
        The spellwarped reduces his overwhelm penalties by 2.
        If his overwhelm penalty is reduced to 0, he is not considered to be overwhelmed.
        \subcf{15th -- Accelerate Attack}
        Whenever the spellwarped makes a \glossterm{standard attack}, he can make an additional \glossterm{strike} at a \minus5 penalty.
        This does not stack with any other effects which grant extra strikes.

        \cf{Spl}[5th]{Manipulate Magic}[Su]
        The spellwarped can channel his innate magic to manipulate magical abilities.
        Using this ability causes the spellwarped to take \glossterm{warp damage} equal to half the \glossterm{power} of the target ability.
        \subcf{Alteration -- Absorption}
        As an immediate action, when the spellwarped makes successfully resists an attack against his Fortitude from a magical ability, he may absorb the magic harmlessly into his body.
        It has no effect on him, even if it would normally have an effect on a failed attack.
        \subcf{Pyromancy -- Fuel the Flame}
        As an immediate action, when the spellwarped is affected by a magical ability, he may channel its energy into a burst of flame around him.
        Enemies within a \areamed radius of the spellwarped take fire damage equal to his spellpower.
        The spell still has its normal effect on the spellwarped.
        \subcf{Telekinesis -- Mind over Matter}
        As an immediate action, when the spellwarped is subject to an attack against his Fortitude from a magical ability, he may use his Mental defense instead.
        \subcf{Temporal -- Accelerate Magic}
        As a swift action, the spellwarped can halve the duration of any magical ability affecting him.
        This can end the effect immediately if it has less than one round remaining.
        If this would reduce the duration by more than one day, the duration is instead reduced by one day.

        \cf{Spl}[9th]{Magic Resistance}[Mag]
        The magic within the spellwarped allows him to completely ignore other magic, granting him \glossterm{magic resistance} equal to 10 \add his Constitution or level, whichever is higher.

        \cf{Spl}[13th]{Improved Manipulate Magic}[Mag]
        The spellwarped can use his manipulate magic ability to affect any ally within \rngmed range of him.

    \subsection{Invocations}\label{Invocations}

        All invocations are \glossterm{magical} abilities unless otherwise noted.

        Some invocations mimic the effects of spells.
        The spellwarped's spellpower with these effects is equal to his spellpower with the spellwarped class.

        \subsubsection{Alteration Invocations}
            \subcf{1st -- Body Bludgeon}
            The spellwarped distorts a part of his body and strikes a foe with it.
            He makes a Spellpower vs. Armor defense physical attack against a foe within his \glossterm{reach}.
            This attack scores critical hits like other physical attacks.
            Success means the target takes 1d6 bludgeoning damage per two spellpower \add half his Strength.
            Failure means the target takes half damage.
            \par At his 4th spellwarped level, this damage increases to 1d6 bludgeoning damage per spellpower \add half his Strength.
            \subcf{1st -- Shrink}
            This invocation functions like the \spell{shrink} spell.

            \subcf{4th -- Enlarge}
            This invocation functions like the \spell{enlarge} spell.
            \subcf{4th -- Purge}
            The spellwarped can end any single effect or condition on him which requires a successful attack against his Fortitude.
            If he identify the effects on him, such as with a Spellcraft check, he may freely choose which effect to end.
            If he cannot differentiate the effects, choose the effect ended randomly.

            \subcf{6th -- Amorphous Body}
            The spellwarped transforms his body into an amorphous form for \durshort duration.
            In this form, he gains several benefits.
            He gains a \plus20 bonus to Reflex defense against grapple attacks, is immune to critical hits, and can move through spaces that are no more than two inches in width without \glossterm{squeezing}.
            While moving through spaces smaller than he could normally move through, he moves at half speed.
            \subcf{6th -- Healing Transformation}
            As a standard action, the spellwarped can heal a creature within \rngclose range by transforming it into a healthier version of its normal body.
            The target heals 1d6 points of damage per spellpower.
            This also removes any of the following conditions: blinded, deafened, diseased, exhausted, fatigued, nauseated, sickened, and poisoned.
            \subcf{6th -- Mighty Throw}
            This invocation functions like the \spell{mighty throw} spell.

            \subcf{8th -- Bludgeon the Horde}
            This attack functions like the body bludgeon attack, except that the spellwarped may attack all foes within his reach, as if he were wielding a reach weapon.
            Success deals 1d8 bludgeoning damage per two spellpower \add half his Strength.
            Failure deals half damage.
            \subcf{8th -- Flight}
            As a standard action, the spellwarped can grow wings that last for 5 rounds.
            The wings grant him a fly speed equal to his land speed.
            While \unencumbered, he can fly (see \pcref{Flying}).
            He can only fly for a number of rounds equal to half his spellpower, even if he uses this ability again to extend the duration of the wings.
            After that limit is reached, he must rest for 5 minutes before flying again.

            \subcf{12th -- Reinforced Flight}
            This functions like the flight invocation, except that the spellwarped can fly while encumbered.

        \subsubsection{Pyromancy Invocations}
            Unless otherwise noted, a pyromancer's invocations are \glossterm{Fire} effects, and shed light equivalent to a torch for their duration.

            \subcf{1st -- Burn}
            The spellwarped makes a Spellpower vs. Reflex defense attack against a foe within \rngmed range.
            Success means the target takes \spelldamage{fire}[d6].
            Critical success deals double damage.
            Failure means the target takes half damage.
            \par At his 4th spellwarped level, this damage increases to \spelldamage{fire}.
            \subcf{1st -- Flame Weapon}
            As a swift action, the spellwarped can create a weapon made of flame that lasts for 5 rounds.
            The weapon is sized appropriately for him, and may take the form of any weapon he is proficient with.
            He can attack with the weapon as if it were a normal weapon of its type, except that he uses his spellpower to determine his damage in place of his level or Strength, and all damage dealt with the weapon is fire damage.
            All other damage modifiers apply normally.
            \par If the flame weapon leaves his hand, it is extinguished 1 round later.

            \subcf{4th -- Fiery Protection}
            As a standard action, the spellwarped can bestow fire and cold damage reduction equal to twice his spellpower on a creature within 30 feet of him.
            The protection lasts for \durshort duration.
            \subcf{4th -- Flame Shield}
            As a standard action, the spellwarped can wreath a willing creature within \rngclose range in flame for \durshort duration.
            Whenever a creature makes a melee attack against the target, the attacking creature takes \spelldamage{fire}[d6].
            A creature can only be dealt damage by this spell once per round.

            \subcf{6th -- Conflagration}
            As a standard action, the spellwarped can release a powerful explosion of flame.
            He makes a Spellpower vs. Reflex attack against everything within a \areamed radius burst of him.
            Success against a target means it takes \spelldamage{fire}[d8].
            Critical success deals double damage.
            Failure against a target means it takes half damage.
            \subcf{6th -- Fireball}
            This invocation functions like the \spell{fireball} spell.
            \subcf{6th -- Ignite}
            The spellwarped makes a Spellpower vs. Reflex attack against a foe within \rngclose range.
            Success means the target takes \spelldamage{fire}, and is \ignited for 2 rounds.
            Critical success deals double damage.
            Failure means the target takes half damage, but is still ignited.

            \subcf{8th -- Firestride}
            As a move action, the spellwarped can teleport to any active flame of at least Tiny size within \rngmed range.
            When he does so, he immolates himself and disappears into a pile of ash before stepping out from the flame unharmed.
            An ordinary torch is sufficient flame to teleport to, but not a candle.
            \subcf{8th -- Flameheart}
            As a standard action, the spellwarped can become a being of pure fire for \durshort duration.
            In this form, he is immune to physical damage and can pass through openings as small as one inch at no movement penalty.
            However, he cannot attack normally or use any of his items, as they meld into his body.
            He may use any of his invocations normally.
            In addition, as a standard action, he can make a Spellpower vs. Reflex attack to touch a creature.
            Success means the target takes \spelldamage{fire}.
            Critical success deals double damage.
            \subcf{8th -- Flight of the Phoenix}
            As a standard action, the spellwarped can create wings of flame that last for 5 rounds.
            The wings grant him a fly speed equal to his land speed.
            While \unencumbered, he can fly (see \pcref{Flying}).
            \subcf{10th -- Lifeseeking Flame}
            As a standard action, the spellwarped fires an orb of flame at a target within \rnglong range.
            If he does not target a creature, the flame automatically strikes a living creature within range.
            It is able to unerringly strike creatures you cannot see or are not aware of, including invisible or concealed creatures.
            You can direct the orb to avoid specific targets, allowing you to strike a hidden foe among your allies.
            If there is no valid target, the orb dissipates harmlessly.
            \par The struck target takes 1d10 fire damage per two spellpower.

            \subcf{14th -- Immolation}
            This invocation functions like the \spell{immolation} spell.

            \subcf{18th -- Phoenix Revival}
            When the spellwarped takes vital damage, he may ignore the damage as an immediate action, even if the vital damage would be sufficient to kill him.
            If he does, he dissolves into a pile of ash for 2 rounds.
            During this time, he can take no actions.
            If the pile of ash remains intact after 2 rounds, the spellwarped is restored to his normal body.
            He has no hit points remaining, and warp damage equal to half his maximum hit points, but is healed of all vital damage.
            However, if the pile of ash is dispersed, the spellwarped dies.
            The ash cannot be harmed by fire damage, and if the pile of ash would take at least 50 points of fire damage during a round, the spellwarped returns one round sooner.
            The spellwarped may take his normal actions in the round after he is restored.
            \par After using this ability, the spellwarped cannot use it again for 1 hour.

            \subcf{20th -- Immolate}
            As a standard action, the spellwarped makes a Spellpower vs. Fortitude against a foe within \rngclose range to consume it in flames from the inside out.
            Success deals \spelldamageemp{fire}.
            Critical success kills the target instantly.
            Failure deals half damage.

        \subsubsection{Telekinesis Invocations}
            \subcf{1st -- Mind Crush}
            As a standard action, the spellwarped can make a Spellpower vs. Fortitude attack against a creature within \rngclose range.
            Success means the target takes \spelldamage{bludgeoning}[d6] and is \sickened for 2 rounds.
            Critical success deals double damage.
            Failure means the target takes half damage, but is still sickened.
            \par At his 4th spellwarped level, this damage increases to \spelldamage{bludgeoning}.
            \subcf{1st -- Mind Blade}
            As a swift action, the spellwarped can telekinetically wield an unattended weapon within \rngclose range for 5 rounds.
            The weapon must be a light or medium weapon appropriate for his size.
            This allows him to attack with the weapon just as if he were holding it in one hand, except that he uses his spellpower in place of his Strength or Dexterity.
            In all other respects, this functions as if he were wielding the weapon normally, including contributing to overwhelm penalties.
            The weapon floats in midair and threatens all squares adjacent to it.
            As a move action, he may move the weapon up to 30 feet in any direction, even vertically.
            If the weapon goes outside of \rngclose range, he loses control of it and it falls to the ground.

            \subcf{4th -- Dual Mind Blade}
            This invocation functions like his mind blade invocation, except that the spellwarped may wield two weapons at once.
            They must stay in the same space, and he may make two-weapon fighting attacks with the weapons, just as if he was wielding them with two hands.
            \subcf{4th -- Mighty Mind Blade}
            This invocation functions like the mind blade invocation, except that the spellwarped may also use a heavy weapon appropriate for his size, allowing him to attack with it just as if he were holding it in two hands.

            \subcf{6th -- Distant Manipulation}
            As a standard action, the spellwarped can mentally manipulate objects and creatures at up to \rngclose range for up to 5 rounds.
            This allows him to take any actions which he could normally take with his hands, using his spellpower in place of his Strength or Dexterity, as appropriate.

            \subcf{8th -- Immobilize}
            As a standard action, the spellwarped can make a creature within \rngmed range of him \immobilized for 2 rounds.

            \subcf{10th -- Telekinetic Blast}
            This invocation functions like the \spell{telekinetic blast} spell.

        \subsubsection{Temporal Invocations}
            Unless otherwise noted, all temporal invocations are \glossterm{Temporal} effects.

            \subcf{Decelerate}
            As a standard action, the spellwarped can attempt to slow a creature down.
            He makes a Spellpower vs. Mental attack against a creature within \rngmed range.
            Success means the target is \immobilized for 2 rounds.
            Failure means it moves at half speed for 2 rounds.
            \subcf{Timelock}
            As a standard action, the spellwarped can attempt to lock a creature in time.
            He makes a Spellpower vs. Mental attack against an adjacent creature.
            Success means the target slips out of time for 1 round.
            During that time, it can take no actions, but cannot be harmed, moved, or affected in any way.
            Critical success means it slips selectively out of time for 1 round.
            During that time, it can take no actions, but can be harmed or moved as normal.
            It is not considered \helpless.
            \par A creature affected by this ability is immune to the effect for 5 rounds, even if the attack failed.

            \subcf{4th -- Disjointed Time}
            As a standard action, the spellwarped chaotically disrupts the local flow of time of a creature within \rngclose range.
            The target is \impaired with attacks and checks for 5 rounds.
            \subcf{4th -- Haste}
            This invocation functions like the \spell{haste} spell.

            \subcf{6th -- Flash Step}
            As a move action, the spellwarped can accelerate a willing creature within \rngclose range so much that he can seem to pause time for everyone but the target.
            This allows the target to immediately take a single move action.
            During this move action, the target moves at double speed, cannot be followed or withdrawn from, and may move through squares occupied by creatures or threatened by blocking enemies without penalty.
            The target still suffers the effects of any environmental hazards.
            \subcf{6th -- Slow}
            This invocation functions like the \spell{slow} spell.

            \subcf{8th -- Inhuman Speed}
            As a move action, the spellwarped can accelerate himself to immense speed, allowing him to move up to five times his speed.
            During this time, he cannot be followed or withdrawn from, can move through squares occupied by enemies or threatened by blocking enemies without penalty, and can treat liquids as if they were solid ground.

            \subcf{10th -- Timestream}
            The spellwarped manipulates time in a \arealarge, 10 ft.\ wide line-shaped zone that extends out from him for 5 rounds.
            All creatures and objects that pass through the line are \slowed for 1 round.
            The spellwarped can exclude his allies from the effect.
            The timestream is virtually invisible, requiring a DR 30 Awareness check to notice in a clear environment, though objects passing through the effect can make it more obvious.
            \subcf{10th -- Mass Haste}
            This invocation functions like the \spell{haste} spell with the Mass augment applied.

            \subcf{12th -- Mass Flash Step}
            This invocation functions like the flashstep invocation, except that it affects up to five willing creatures.
            \subcf{12th -- Mass Slow}
            This invocation functions like the \spell{slow} spell with the Mass augment applied.

            \subcf{16th -- Time Reversal}
            As a swift action, the spellwarped can take one warp damage to create a ``time lock.'' The time lock persists for one round.
            As a standard action, he can take eight warp damage to make a Spellpower vs. Mental attack a creature within \rngmed range to reverse time for it.
            Success means the target is restored to its exact condition at the point immediately after the time lock was created.
            The effects of any actions that the creature took in the intervening time are undone, any damage it dealt or took is removed, it is returned to its original location, and the creature is restored in all other ways, just as if the intervening time had never occured.
            If its original location is occupied, the time reversal fails.
            The spellwarped cannot reverse time for himself in this way.
            \par After reversing time in this way, the spellwarped must wait 5 rounds before he can create a time lock or reverse time again.
            \subcf{16th -- Supreme Acceleration}
            As a standard action, the spellwarped can accelerate himself so much that he can take an additional round of actions immediately.
            During this round, all creatures he attacks are treated as \helpless, but he cannot perform a \glossterm{coup de grace} or similar ability; such an act requires more care and precision than is possible with such immense speed.
            After using this ability, he must wait 5 rounds before he can use it again.

            \subcf{18th -- Time Stop}
            As a standard action, the spellwarped can step into an alternate timestream, causing him to speed up so greatly that all other creatures seem frozen.
            He can act for 1d3\plus1 rounds of apparent time.
            During this time, all objects and creatures are frozen in place and are completely invulnerable to the spellwarped, though he may affect them with invocations normally.
            After using this ability, he must wait 5 rounds before he can use it again.

\section{Multiclass Characters}\label{Multiclass Characters}
    A character may add new classes as he or she progresses in level, thus becoming a multiclass character.
    The class abilities from a character's different classes combine to determine a multiclass character's overall abilities.
    Multiclassing improves a character's versatility at the expense of focus.

    \subsection{Class And Level Features}
        As a general rule, the abilities of a multiclass character are the sum
        of the abilities of each of the character's classes.

        \parhead{Level}
        ``Character level'' is a character's total number of levels.
        It is used to determine when feats and attribute score boosts are gained, as noted on \tref{Character Advancement}.
        Whenever a creature's ``level'' is specified, without reference to a particular class, the character level is used.

        \par ``Class level'' is a character's level in a particular class.
        For a character whose levels are all in the same class, character level and class level are the same.

        \parhead{Hit Points}
        The normal rules for determining hit points apply to multiclass characters.

        \parhead{Defenses}
        The normal rules for determining defenses apply to multiclass characters.

        \parhead{Skills}
        A multiclass character gains all class skills from all of his classes.
        However, only the character's base class grants skill points.
        If a character has multiple base classes, he must choose which base class grants skill points.
        When a character gains new class skills, he may redistribute his skill points to gain training or mastery in his new class skills.

        \parhead{Class Abilities}
        A multiclass character gets all the class abilities of all his or her classes, but must also suffer the consequences of the special restrictions of all his or her classes.

        \par In some cases, two classes can have virtually identical abilities.
        Use the following guidelines to determine how abilities stack.
        \begin{itemize}
            \item If two identical class abilities are not based on level and are not gained following a specific pattern, they do not stack.
            \item If two identical class abilities are not explicitly based on level, but both classes gain them in a predictable pattern, the levels of the two classes stack for determining when the next improvement to the class ability will be gained.
            \item If two identical class abilities are explicitly based on level, the levels of the two classes stack for determining the power of the ability.
            \item If two identical class abilities say how they stack, those rules trump any other rules.
        \end{itemize}
        These are some examples of how to use these guidelines.
        \begin{itemize}
            \item Both a druid and a ranger gain wild speech.
                A druid/ranger who has wild speech from both classes has the same wild speech ability as a druid or ranger would.
            \item Both a barbarian and a rogue get uncanny dodge and improved uncanny dodge at the same level.
                A barbarian/rogue adds his barbarian and rogue levels together to determine when he acquires improved uncanny dodge.
        \end{itemize}

        \parhead{Weapon and Armor Proficiency}
        Only a character's base class grants weapon and armor proficiencies.
        If a character has multiple base classes, she must choose which base class grants weapon and armor profiencies.
