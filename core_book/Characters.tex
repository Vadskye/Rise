\chapter{Characters}

There are five major customization systems in Rise that all characters share.
In rough order of how much they affect your character's play style, they are your class archetypes, attributes, skills, insight points, and species.
At the GM's discretion, you may also have one or more feats, which can have a strong impact on your character's identity (see \pcref{Feats}).

This chapter explains those five fundamental elements.
If you plan on playing a premade character, the information in this chapter will still be useful so you can understand how your character works, but you can skip the Character Creation section at the end.
Some of the information in this chapter won't fully make sense until you've read future chapters.
You can either skim past terms you don't yet understand or look them up as you go along.

\section{Classes Overview}
    Each character has one of ten classes.
    Your class determines what your character's fundamental source of power is, and has a large impact on the play style of your character.
    Of course, any two members of the same class can be very different in both narrative style and mechanics based on the other choices they have made.
    Classes are intended as an aid to help give your character a cohesive identity, not a limitation on the possible character concepts you can fulfill.

    The ten classes are briefly summarized below.
    Each class has five archetypes, and any individual character chooses three of the five archetypes from their class.
    For full details about how each class works, see \pcref{Classes}.
    \begin{itemize}
        \item Barbarians are mighty warriors who draw power from their physical prowess.
        \item Clerics are divine spellcasters who draw power from their veneration of a deity.
        \item Druids are nature spellcasters who draw power from their veneration of the natural world.
        \item Fighters are highly disciplined warriors who excel in physical combat of any kind.
        \item Monks are agile masters of ``ki'' who hone their personal abilities to strike down foes and perform supernatural feats.
        \item Paladins are divinely empowered warriors embody a particular alignment.
        \item Rangers are skilled hunters who bridge the divide between nature and civilization.
        \item Rogues are exceptionally skillful characters known for their ability to strike at their foe's weak points in combat.
        \item Sorcerers are arcane spellcasters who draw power from their inherently magical nature.
        \item Warlocks are pact spellcasters who draw their power from a dark pact made with infernal creatures.
        \item Wizards are arcane spellcasters who study magic to unlock its powerful secrets.
    \end{itemize}

\section{Attributes}\label{Attributes}

    Each character has six \glossterm{attributes}: Strength (Str), Dexterity (Dex), Constitution (Con), Intelligence (Int), Perception (Per), and Willpower (Wil).
    Each attribute represents a character's raw talent in that area.
    A 0 in an attribute represents average human capacity.
    That doesn't mean that every commoner has a 0 in every attribute; not everyone is average, after all.

    \subsection{Strength (Str)}\label{Strength}
        {
            Strength measures your muscle and physical power.
            Characters with a high Strength tend to have strong offensive capabilities with nonmagical abilities, and prefer wearing heavier armor.
            It has the following effects:
            \begin{itemize}
                \item Strength determines how much you can carry (see \tref{Weight Limits by Strength}).
                    You generally need a Strength of at least 1 to wear heavy body armor.
                \item You gain a bonus (or penalty) to your \glossterm{power} with \glossterm{mundane} abilities, such as normal weapon attacks, equal to your Strength (see \pcref{Power}).
                \item You reduce your total \glossterm{encumbrance} from \glossterm{armor} by an amount equal to your base Strength (see \pcref{Encumbrance}).
                \item You gain a bonus (or penalty) to your \glossterm{fatigue tolerance} equal to your base Strength (see \pcref{Fatigue}).
                \item Your Strength affects Strength-based \glossterm{skills}: Climb, Jump, and Swim (see \pcref{Skills}).
            \end{itemize}
        }

    \subsection{Dexterity (Dex)}\label{Dexterity}
        {
            Dexterity measures your hand-eye coordination, agility, and reflexes.
            Characters with a high Dexterity tend to have strong defensive capabilities, and prefer wearing lighter armor.
            It has the following effects:
            \begin{itemize}
                \item You gain a bonus (or penalty) to your Reflex defense equal to your base Dexterity.
                \item You gain a bonus (or penalty) to your Armor defense equal to your base Dexterity.
                    This bonus can be reduced if you wear medium or heavy armor (see \tref{Armor and Shields}).
                \item You gain a bonus (or penalty) to your \glossterm{initiative} equal to your base Dexterity (see \pcref{Initiative}).
                \item Your Dexterity affects Dexterity-based \glossterm{skills}: Balance, Flexibility, Ride, Sleight of Hand, and Stealth (see \pcref{Skills}).
            \end{itemize}
        }

    \subsection{Constitution (Con)}\label{Constitution}
        {
            Constitution represents your health and stamina.
            Characters with a high Constitution tend to have strong defensive capabilities.
            It has the following effects:
            \begin{itemize}
                \item You gain a bonus (or penalty) to your \glossterm{hit points} equal to twice your Constitution.
                \item You gain a bonus (or penalty) to your \glossterm{damage resistance} equal to your Constitution (see \pcref{Damage Resistance}).
                \item You gain a bonus (or penalty) to your Fortitude defense equal to your base Constitution.
                \item You gain a bonus (or penalty) to your Armor defense equal to half your base Constitution.
                \item Your Constitution affects the Constitution-based \glossterm{skill}: Endurance (see \pcref{Skills}).
            \end{itemize}
        }

    \subsection{Intelligence (Int)}\label{Intelligence}
        {
            Intelligence represents how well you learn and reason.
            Characters with a high Intelligence tend to have more options and special abilities.
            It has the following effects:

            \begin{itemize}
                \item If your base Intelligence is positive, you become \glossterm{trained} in a number of skills equal to your Intelligence (see \pcref{Trained Skills}).
                \item You gain a bonus (or penalty) to \glossterm{insight points} equal to your base Intelligence (see \pcref{Insight Points}).
                \item Your Intelligence affects Intelligence-based \glossterm{skills}: Craft, Deduction, Disguise, Knowledge, Linguistics, and Medicine (see \pcref{Skills}).
            \end{itemize}

            \par An animal has an Intelligence score of \minus6 or lower.
            A creature of humanlike intelligence has a score of at least a \minus5 Intelligence.
        }

    \subsection{Perception (Per)}\label{Perception}
        {
            Perception describes your ability to observe and be aware of your surroundings.
            Characters with a high Perception tend to have strong offensive capabilities.
            It has the following effects:
            \begin{itemize}
                \item You gain a bonus (or penalty) to your \glossterm{accuracy} with all attacks equal to half your base Perception (see \pcref{Accuracy}).
                \item You gain a bonus (or penalty) to your \glossterm{power} with all abilities equal to half your Perception (see \pcref{Power}).
                \item You gain a bonus (or penalty) to your \glossterm{initiative} equal to your base Perception (see \pcref{Initiative}).
                \item Your Perception affects Perception-based \glossterm{skills}: Awareness, Creature Handling, Social Insight, Spellsense, and Survival (see \pcref{Skills}).
            \end{itemize}
        }

    \subsection{Willpower (Wil)}\label{Willpower}
        {
            Willpower represents your ability to endure mental hardships.
            Characters with a high Willpower tend to have strong offensive capabilities with magical abilities.
            It has the following effects:
            \begin{itemize}
                \item You gain a bonus (or penalty) to your \glossterm{power} with \glossterm{magical} abilities, such as spells, equal to your Willpower (see \pcref{Power}).
                \item You gain a bonus (or penalty) to your Mental defense equal to your base Willpower.
                \item You gain a bonus (or penalty) to your \glossterm{fatigue tolerance} equal to your base Willpower (see \pcref{Fatigue}).
            \end{itemize}
        }

    \subsection{Base Attributes}\label{Base Attributes}
        Most statistics depend on your \glossterm{base attributes}, rather than your total attributes.
        Your attributes automatically increase with level as defined in \tref{Increasing Attributes With Level}, but your base attributes do not automatically increase with level.

        A small number of abilities can increase your \glossterm{base attributes} after 1st level.
        If you change a base attribute, the total value for that attribute changes appropriately.

        \begin{dtable}
            \lcaption{Increasing Attributes With Level}
            \begin{dtabularx}{\columnwidth}{l X}
                \tb{Base Attribute} & \tb{Total Attribute}        \\
                1 or lower          & Same as base attribute      \\
                2                   & 2 \add one quarter level    \\
                3                   & 3 \add half level           \\
                4                   & 4 \add three-quarters level \\
                5                   & 5 \add level                \\
            \end{dtabularx} 
        \end{dtable}

    \subsection{Extraordinary Attributes}
        Some abilities can increase your base attributes above 5.
        For each point of base attribute beyond 5, you increase your total attribute by one quarter of your level.

        For example, a 20th level character might have a total base Strength of 6.
        Their total Strength would be equal to 4 \add one and a half times their level, for a total of 34.

\section{Skills Overview}
    Skills represent the myriad of talents that people can have, such as cooking or swimming.
    Each character is trained in a certain number of skills.
    If you are trained in a skill, you have a higher likelihood of succeeding when you try to use it.
    The number of skills you are trained in is mostly determined by your class and Intelligence.

    The twenty-six skills are summarized below.
    For full details about how each skill works, see \pcref{Skills}.

    \begin{itemize}
        \item The Awareness skill represents your ability to observe things which you might otherwise fail to notice.
        \item The Balance skill represents your ability to maintain your balance and poise in difficult circumstances.
        \item The Climb skill represents your ability to climb obstacles.
        \item The Craft skills represent your ability to construct objects from raw materials.
        \item The Creature Handling skill represents your ability to influence non-sapient creatures.
        \item The Deception skill represents your ability to lie or otherwise mislead people without being caught.
        \item The Deduction skill represents your ability to make logical deductions based on evidence.
        \item The Devices skill represents your ability to to manipulate mechanical devices such as locks, traps, and other contraptions.
        \item The Disguise skill represents your ability to create disguises to conceal the appearance of creatures or objects.
        \item The Endurance skill represents your ability to persevere through physical trials.
        \item The Flexibility skill represents your ability to escape bindings and move through small areas by contorting your body.
        \item The Intimidate skill represents your ability to intimidate and coerce people into doing what you want.
        \item The Jump skill represents your ability to jump.
        \item The Knowledge skills represent your understanding of particular aspects of the world.
        \item The Linguistics skill represents your mastery of spoken and written languages.
        \item The Medicine skill represents your practical understanding of how to tend to the wounds of living creatures.
        \item The Perform skills represent your ability to create particular forms of entertainment.
        \item The Persuasion skill represents your ability to convince people to think what you want them to.
        \item The Profession skills represent your practical understanding of a particular profession.
        \item The Ride skill represents your ability to ride and control horses and other mounts.
        \item The Sleight of Hand skill represents your ability to pick pockets, palm objects, and perform other feats of legerdemain.
        \item The Social Insight skill represents your ability to read body language and emotion.
        \item The Spellsense skill represents your ability to notice and understand spells and magical effects.
        \item The Stealth skill represents your ability to escape detection while moving or taking large-scale actions.
        \item The Survival skill represents your ability to take care of yourself and others in the wilderness, including the ability to follow tracks.
        \item The Swim skill represents your ability to swim.
    \end{itemize}

\section{Insight Points}\label{Insight Points}
    You can spend \glossterm{insight points} to gain new special abilities or to learn new proficiencies.
    Your \glossterm{class} gives you a certain number of insight points, and you gain a bonus (or penalty) to that number of insight points equal to your base Intelligence.
    Some abilities can also grant insight points.

    Any character can spend insight points in any of the following ways.
    \begin{itemize}
        \item You can spend two \glossterm{insight points} to become a \glossterm{multiclass} character (see \pcref{Multiclass Characters}).
        \item You can spend an \glossterm{insight point} to gain an additional \glossterm{trained skill}.
        \item You can spend an \glossterm{insight point} to gain proficiency in an additional \glossterm{usage class} of armor (light, medium, or heavy).
            You must be proficient with light armor to become proficient with medium armor, and you must be proficient with medium armor to become proficient with heavy armor.
        \item You can spend an \glossterm{insight point} to gain proficiency in an additional \glossterm{weapon group}.
        \item You can spend two \glossterm{insight points} to gain proficiency with \glossterm{exotic weapons} from a single \glossterm{weapon group} you are already proficient with.
        \item You can spend an \glossterm{insight point} to learn two \glossterm{common languages} or one \glossterm{rare language} (see \pcref{Communication and Languages}).
    \end{itemize}
    In addition, every class has at least one way to spend \glossterm{insight points} to learn additional abilities.

\section{Species}\label{Species}
    Each character has a species.
    There are seven common species described below.
    At the GM's discretion, you may be able to play a character with a more unusual species (see \pcref{Uncommon Species}).

    \subsection{Humans}
        \parhead{Size} Medium.
        \parhead{Attributes} No change.
        \parhead{Speed} 30 feet.
        \parhead{Special Abilities}
        \begin{itemize}
            \itemhead*{Flexible}: Humans gain an additional \glossterm{insight point}.
                Insight points can be spent to learn new special abilities (see \pcref{Insight Points}).
            \itemhead*{Skilled}: Humans gain two additional \glossterm{skill points}. They can spend those skill points on any skills (see \pcref{Skills}).
        \end{itemize}
        \parhead{Automatic Language} Common, any two \glossterm{common languages} or one \glossterm{rare language} (see \pcref{Communication and Languages}).

    \subsection{Dwarves}
        \parhead{Size} Medium.
        \parhead{Attributes} \plus1 base Constitution (max 4), \minus1 base Dexterity.
        \parhead{Speed} 25 feet.
        \parhead{Special Abilities}
        \begin{itemize}
            \itemhead*{Darkvision}: Dwarves can see in the dark clearly up to 60 feet. Darkvision does not function if a dwarf is in a brightly lit area, and does not resume functioning until the end of the next round after the dwarf leaves the brightly lit area.
            \itemhead*{Depth Sense}: Dwarves can intuitively sense their approximate depth underground as naturally as a human can sense which way is up.
            \itemhead*{Dwarven Endurance}: Wearing medium or heavy \glossterm{body armor} does not reduce a dwarf's movement speed (see \pcref{Armor Usage Classes}).
            \itemhead*{Earthen Crafting}: Dwarves gain a \plus2 bonus to the Craft (metal) and Craft (stone) skills.
            \itemhead*{Stable}: Dwarves reduce the distance they are moved by unwilling \glossterm{knockback} and \glossterm{push} effects by 10 feet.
        \end{itemize}
        \parhead{Automatic Languages} Common, Dwarven, any one \glossterm{common language} (see \tref{Common Languages}).

    \subsection{Elves}
        \parhead{Size} Medium.
        \parhead{Attributes} \plus1 base Dexterity (max 4), \minus1 base Constitution.
        \parhead{Speed} 30 feet.
        \parhead{Special Abilities}
        \begin{itemize}
            \itemhead*{Elven Serenity}: Elves gain a \plus1 bonus to Mental defense and reduce their \glossterm{focus penalty} by 1.
            \itemhead*{Keen Senses}: Elves gain a \plus2 bonus to the Awareness skill (see \pcref{Awareness}).
            \itemhead*{Low-light Vision}: Elves treat sources of light as if they had double their normal illumination range.
            \itemhead*{Sure-Footed}: Elves gain a \plus2 bonus to the Balance skill (see \pcref{Balance}).
            \itemhead*{Trance}: Elves do not sleep, and are immune to \glossterm{magical} effects that would cause them to sleep.
                Instead of sleeping, elves can trance for 4 hours.
                An elf in trance may make Perception-based checks at a \minus5 penalty.
                Elves must still avoid strenuous activity for 8 hours to heal and gain other benefits of taking a \glossterm{long rest}.
        \end{itemize}
        \parhead{Automatic Languages} Common, Elven, any one \glossterm{common language} (see \tref{Common Languages}).

    \subsection{Gnomes}
        \parhead{Size} Medium.
        \parhead{Attributes} \plus1 base Constitution (max 4), \minus1 base Strength.
        \parhead{Speed} 25 feet.
        \parhead{Special Abilities}
        \begin{itemize}
            \itemhead*{Fae Light} (Magical): A gnome can use the \textit{fae light} ability as a \glossterm{minor action}.
                \begin{freeability}{Fae Light}
                    A Tiny glowing orb appears at a location within \rngmed range.
                    It sheds pale, \glossterm{bright illumination} in a \areasmall radius, and \glossterm{shadowy illumination} in a \areamed radius.
                    The orb is intangible, and cannot be moved once placed.

                    This ability lasts until you use it again or until you \glossterm{dismiss} it as a free action.
                \end{freeability}
            \itemhead*{Low-light Vision}: Gnomes treat sources of light as if they had double their normal illumination range.
            \itemhead*{Magic Affinity}: Gnomes gain a bonus equal to a quarter of their level (minimum 1) to their \glossterm{power} with \glossterm{magical} abilities.
            \itemhead*{Short Stature}: Gnomes gain a \plus2 bonus to the Stealth skill.
            \itemhead*{Tinker}: Gnomes gain a \plus2 bonus to two Craft skills of their choice (see \pcref{Craft}).
        \end{itemize}
        \parhead{Automatic Languages} Common, Gnome, either Sylvan or any one \glossterm{common language} (see \tref{Common Languages}).

    \subsection{Half-Elves}\label{Half-Elves}
        \parhead{Size} Medium.
        \parhead{Attributes} No change.
        \parhead{Speed} 30 feet.
        \parhead{Special Abilities}
        \begin{itemize}
            \itemhead*{Diplomatic}: Half-elves gain a \plus2 bonus to the Persuasion skill.
            \itemhead*{Dual Heritage}: For all effects related to species, a half-elf is considered both a human and an elf.
            \itemhead*{Low-light Vision}: Half-elves treat sources of light as if they had double their normal illumination range.
            \itemhead*{Versatile}: Half-elves only need to spend one \glossterm{insight point} to gain access to an additional class (see \pcref{Multiclass Characters}).
        \end{itemize}
        \parhead{Automatic Language} Common, Elven, any two \glossterm{common languages} or one \glossterm{rare language} (see \pcref{Communication and Languages}).

    \subsection{Half-Orcs}
        \parhead{Size} Medium.
        \parhead{Attributes} \plus1 base Strength (max 4), \minus1 base Intelligence.
        \parhead{Speed} 30 feet.
        \parhead{Special Abilities}
        \begin{itemize}
            \itemhead*{Darkvision}: Half-orcs can see in the dark clearly up to 60 feet. Darkvision does not function if a half-orc is in a brightly lit area, and does not resume functioning until the end of the next round after the half-orc leaves the brightly lit area.
            \itemhead*{Dual Heritage}: For all effects related to species, a half-orc is considered both a human and an orc.
            \itemhead*{Flexible}: Half-orcs gain an additional \glossterm{insight point}.
                Insight points can be spent to learn new special abilities (see \pcref{Insight Points}).
            \itemhead*{Intimidating}: Half-orcs gain a \plus2 bonus to the Intimidate skill (see \pcref{Intimidate}).
        \end{itemize}
        \parhead{Automatic Languages} Common, Orc.

    \subsection{Halflings}
        \parhead{Size} Medium.
        \parhead{Attributes} \plus1 base Dexterity (max 4), \minus1 base Strength.
        \parhead{Speed} 25 feet.
        \parhead{Special Abilities}
        \begin{itemize}
            \itemhead*{Nimble Combatant}: Halflings gain a \plus1 bonus to Armor defense.
            \itemhead*{Short Stature}: Halflings gain a \plus2 bonus to the Stealth skill.
            \itemhead*{Stout-Hearted}: Halflings gain a \plus1 bonus to Mental defense.
            \itemhead*{Sure-Footed}: Halflings gain a \plus2 bonus to the Balance skill (see \pcref{Balance}).
        \end{itemize}
        \parhead{Automatic Languages} Common, Halfling, any one \glossterm{common language} (see \tref{Common Languages}).

\section{Alignment}\label{Alignment}
    A creature's general moral and personal attitudes are represented by its alignment: lawful good, neutral good, chaotic good, lawful neutral, neutral, chaotic neutral, lawful evil, neutral evil, or chaotic evil.

    Alignment is a tool for developing your identity.
    It is not a straitjacket for restricting your actions.
    Each alignment represents a broad range of personality types or personal philosophies, so two characters of the same alignment can still be quite different from each other.
    In addition, few people are completely consistent.

    \subsection{Good vs. Evil}
        The ancient battle between good and evil takes many forms, and distinguishing good from evil is a deeply complex task.
        For the purposes of Rise, good and evil are strictly defined according to selfishness vs\. altruism.
        The actions of good characters may at times be morally reprehensible, and the actions of evil characters may seem to be virtuous.
        However, this narrow definition of good and evil avoids the complexities of defining a more robust moral system while preserving the fundamental conflict between good and evil.

        \parhead{Good} Good characters are altruistic.
        They take other creatures into account when making decisions, and actively try to help or improve others around them.
        Good characters may have significant disagreements about what actions are best, but they consistently prioritize the good of others or the ``greater good'' over their own desires.
        Different good characters may also have different perspectives on who they should take into account when making decisions.
        For example, some good characters actively work to protect animals and plants, while others only care about sentient beings.

        Sometimes, altruistic characters can commit reprehensible actions out of necessity or because they believe that a greater good is being served.
        As long as their motivation is selfless, those characters are still considered to be ``good'' from the perspective of Rise's alignment system, which does not attempt to model all of the complexities of real-world morality.

        \parhead{Evil} Evil characters are selfish.
        They consistently prioritize their own desires and needs over the desires of others, even their allies or friends.
        Evil characters may perform good deeds, but their ultimate motivation is to help themselves or make themselves feel better, not to help others.

        \parhead{Neutral} Characters that are neutral between good and evil are neither consistently altruistic nor consistently selfish.
        Most neutral characters behave altruistically in some ways and selfishly in other ways -- either at different times, or about different aspects of life.
        They often have strong bonds to particular individuals who they care about selflessly, but are not altruistic in a general sense.
        Non-sentient beings such as animals are neutral rather than good or evil.

    \subsection{Law vs. Chaos}
        \parhead{Law} Lawful characters value consistency.
        They obey rules that guide their actions.
        Some lawful characters draw their rules from external forces, such as serving a particular master or following the legal laws of the land.
        Other lawful characters follow rules they make for themselves.

        \parhead{Chaos} Chaotic characters value flexibility and freedom.
        They make decisions based on what they think or feel at the time, even if it is inconsistent with their previous statements or actions.

        \parhead{Neutral} Characters that are neutral between law and chaos are neither exceptionally consistent nor exceptionally inconsistent.
        They tend to be generally consistent but may change their minds under the right circumstances.
        Non-sentient beings such as animals are neutral rather than lawful or chaotic.

\section{Personal Appearance}
    \begin{dtable!*}
        \lcaption{Typical Ages}
        \begin{dtabularx}{\textwidth}{l *{5}{>{\ccol}X}}
            \tb{Species} & \tb{Adulthood} & \tb{Middle Age} & \tb{Old} & \tb{Venerable} & \tb{Maximum Age} \tableheaderrule
            Human        & 15 years       & 35 years              & 53 years       & 70 years             & \plus4d10 years \\
            Dwarf        & 40 years       & 125 years             & 188 years      & 250 years            & \plus2d\% years \\
            Elf          & 110 years      & 175 years             & 263 years      & 350 years            & \plus4d\% years \\
            Gnome        & 40 years       & 100 years             & 150 years      & 200 years            & \plus3d\% years \\
            Half-elf     & 20 years       & 62 years              & 93 years       & 125 years            & \plus6d10 years \\
            Half-orc     & 14 years       & 30 years              & 45 years       & 60 years             & \plus2d10 years \\
            Halfling     & 20 years       & 50 years              & 75 years       & 100 years            & \plus1d\% years \\
        \end{dtabularx}
    \end{dtable!*}

    \begin{dtable}
        \lcaption{Typical Height and Weight}
        \begin{dtabularx}{\columnwidth}{l *{2}{>{\lcol}X}}
            \tb{Species} & \tb{Average Height} & \tb{Average Weight} \tableheaderrule
            Human        & 5' 5''              & 150 lb. \\
            Dwarf        & 4' 2''              & 160 lb. \\
            Elf          & 5' 0''              & 110 lb. \\
            Gnome        & 3' 4''              & 50 lb.  \\
            Half-elf     & 5' 2''              & 130 lb. \\
            Half-orc     & 5' 10''             & 200 lb. \\
        \end{dtabularx}
    \end{dtable}

    \subsection{Age}
        The typical age for each species is listed in \trefnp{Typical Ages}.
        If you are old, you take a \minus2 penalty to \glossterm{checks} based on Strenth, Dexterity, Constitution, and Perception.
        However, you gain a \plus2 bonus to \glossterm{checks} based on Intelligence and Willpower.
        If you are venerable, these modifiers change to \minus4 and \plus4 respectively.
        In general, player characters should not start as old or venerable age, but the GM can always allow it for specific campaigns if they want.

        When you reach venerable age, the GM secretly rolls your maximum age, which is the number from the Venerable column on \trefnp{Aging Effects} plus the result of the dice roll indicated on the Maximum Age column on that table.
        They record the result.
        If you reach your maximum age, you die of old age at some time during the following year.

        The maximum ages are for player characters. Most people in the world at large die from pestilence, accidents, infections, or violence before getting to venerable age.

    \subsection{Height and Weight}
        The typical height and weight for each species is listed in \tref{Typical Height and Weight}.
        The average man from each species is slightly taller and heavier than the average woman, but this is not a restriction for player characters.

\section{Character Creation}\label{Character Creation}

    Creating a charcter involves a mixture of thematic and mechanical decisions that will work together to create a fun character that is rewarding to play.
    As mentioned earlier in this chapter, there are four core systems for customizing your character's mechanics: class, attributes, skills, and species.
    In addition, there are five core thematic considerations when creating a character: concept, personality, motivation, background, and appearance.

    These decisions are described below in a order that makes sense for many characters, and full details for each decision are given after this initial list.
    It is essentially a sandwich, with narrative decisions wrapped around a central core of your character's mechanical components.
    However, you can make several of these decisions in any order, and you may find it easier to create a character in a different way.
    The only real limitation is that your skills must be the last mechanical choice you make, since they are strongly affected by all of your other choices.

    \begin{enumerate}
        \item Character concept: Describe your character with a short, simple phrase that captures their essence.
        \item Motivation and goal: Describe what your character wants.
        \item Alignment: Describe your character's moral compass.

        \item Species: Define your character's species.
        \item Attributes: Define your character's fundamental physical and mental potential.
        \item Class archetypes: Define your character's source of power.
        \item Insight points: Learn new abilities and gain additional proficiencies.
        \item Skills: Define your character's areas of non-combat expertise.

        \item Personality: Describe how your character acts and reacts to the world.
        \item Background: Describe what made your character become who they are now.
        \item Appearance: Describe what your character looks like.
        \item Name: Choose a name.
    \end{enumerate}

    \subsection{Step 1: Character Concept}

        Fundamentally, who is your character?
        You should think of a short phrase that describes the core concept behind the character you will create.
        It's best to think in broad strokes when creating a character concept.
        Your concept should be more than just a factual description of your species or what you do.
        It should be something that makes you memorable.
        Some sample character concepts are given below for inspiration.

        \begin{itemize}
            \item Pragmatic wanderer
            \item Artistic pixie
            \item Mushroom-obsessed hermit
            \item Bumbling do-gooder
            \item Dim-witted bodyguard
            \item Cowardly storyteller
            \item Bear-barian
            \item Parsimonious law enforcer
            \item Peaceful naturalist
            \item Trigger-happy pyromaniac
            \item Heroic, simple-minded warrior
            \item Friendly necromancer
            \item Chaotic speed demon
            \item Pompous ex-noble
            \item Sarcastic mercenary
            \item Battle-scarred priest
            \item Ambitious arcane prodigy
            \item Charismatic musician
            \item Aloof scholar
            \item Blunt-spoken warrior
            \item Crazed prophet
            \item Polite warrior
            \item World-weary pirate
            \item Devout cultist
            \item Con artist with a heart of gold
        \end{itemize}

    \subsection{Step 2: Motivation and Goal}
        Why does your character put in all of the effort that adventuring requires?
        They probably have a goal that they are trying to achieve, or an ideal that they are trying to embody.
        Writing down a specific goal or ideal can be helpful as an anchor point when defining the character.

    \subsection{Step 3: Alignment}
        Your character's alignment reflects their moral character: are they more inclined to good or evil, and to chaos or order?
        Alignments are described in more detail at \pcref{Alignment}.

    \subsection{Step 4: Species}
        It's often convenient to make your species your first mechanically relevant choice.
        Your species can have a strong effect on your personality and narrative, but it has a relatively small effect on your character's play style.
        It's also easier to know your species before you choose your attributes, since your species can slightly modify your attributes.

        Choose one of the seven common species options, or talk with your GM about choosing an uncommon species (see \pcref{Uncommon Species}).
        Record any specific abilities the species gives you on your character sheet, but if this is your first mechanical choice, you won't be able to finalize any of your statistics yet.
        You should also choose the languages that you can speak, since that is influenced by your species (see \pcref{Communication and Languages}).

    \subsection{Step 5: Attributes}
        Your attributes are a good option for your second mechanically relevant choice.
        They have a large impact on your character's strengths and weaknesses, so it's useful to know them as soon as possible.
        They're also much easier to understand and finalize than your class archetypes.

        There are two common ways for you to determine your attribute scores: using a predefined set of scores, or using a point buy system.
        Once you have chosen your attributes and applied your species modifer to attributes (if any), you should record in your character sheets the various effects that your attributes have on your statistics.

        \subsubsection{Predefined Attribute Scores}
            This is the simplest method.
            Simply take the following set of attribute scores and distribute them as you choose among your attributes:

            3, 2, 2, 1, 1, 0

            This set of attribute scores is called the ``elite array''.
            For more extreme characters, you may use the ``savant array'':

            4, 2, 1, 1, 0, 0.

            Finally, for more well-rounded characters, you may use the ``balanced array'':

            2, 2, 2, 2, 1, 1

        \subsubsection{Point Buy}
            With this method, you can fully control your attribute scores to precisely match what you want to be able to do.
            All your attribute scores start at 0.
            You get 10 points to distribute among your attributes.
            Attributes can be bought according to the costs on \trefnp{Attribute Score Point Costs}.
            The listed cost is the total cost required to gain the listed base attribute.
            You are 1st level when you start, which adds appropriately to your total attribute score.

            \begin{dtable}
                \lcaption{Attribute Score Point Costs}
                \begin{dtabularx}{\columnwidth}{X X l}
                    \tb{Cumulative Point Cost} & \tb{Base Attribute} & \tb{Total Attribute}        \tableheaderrule
                    0                          & 0                         & 0                           \\
                    1                          & 1                         & 1                           \\
                    2                          & 2                         & 2 \add one quarter level    \\
                    4                          & 3                         & 3 \add half level           \\
                    6                          & 4                         & 4 \add three-quarters level \\
                    \tdash                     & 5                         & 5 \add level                \\
                \end{dtabularx}
            \end{dtable}

        \subsubsection{Attribute Penalties}\label{Attribute Penalties}
            You can voluntarily take penalties to your attributes.
            If you reduce an attribute to a total of \minus1, you become \glossterm{trained} in an additional skill (see \pcref{Trained Skills}).
            If you reduce an attribute to a total of \minus2, you instead gain an additional \glossterm{insight point} (see \pcref{Insight Points}).
            You cannot gain these benefits from reducing more than two attributes below 0 in this way.

    \subsection{Step 6: Class and Class Archetypes}
        This is the most complicated choice you have to make for your character.
        It requires the most reading in the Classes chapter to understand what your options are and which classes and class archetypes are interesting to you.
        Class details can be found in \pcref{Classes}.

        You should choose one of the ten classes, and then any three of the five archetypes within that class.
        Once you have chosen your three archetypes, you'll need to choose which one of those three archetypes will be rank 1, giving you access to an extra ability from that archetype.
        The other two archetypes will be rank 0 until you gain a level.
        You should also choose the \glossterm{weapon groups} that you have access to, since that is influenced by your class (see \pcref{Weapon Groups}).

        If you are particularly adventurous, this is also when you should choose if you want to be a multiclass character.
        Multiclass characters can gain access to archetypes from multiple classes.
        This does not increase the number of archetypes you know, so it does not directly increase your power.
        However, multiclass characters can be more specialized or more versatile than single-class characters, and can represent unusual character concepts.

    \subsection{Step 7: Insight Points}
        Once you have chosen your class archetypes, attributes, and species, you know how many insight points you have, and can choose how to spend them.
        Don't forget to record on your character sheet how you spent each insight point.
        Otherwise, you might get confused later about why you have more spells known or skills trained than you normally would.

        In rare circumstances, you might want to delay spending your insight points until you are higher level.
        For example, a fighter/sorcerer multiclass character who wants to have both spells and maneuvers can't have access to both spells and maneuvers at level 1, so they wouldn't be able to spend insight points on both spells and maneuvers.
        You aren't forced to spend all of your insight points, so you can save them up for later.
        You can also talk to your GM about spending them on something minor at level 1 and then retraining those insight points once you are higher level.

    \subsection{Step 8: Skills}
        You should choose which skills you have \glossterm{trained} (see \pcref{Skills}).
        Your \glossterm{class} gives you a certain number of trained skills from among the \glossterm{class skills} for that class.
        The class skills for each class are summarized in \tref{Class Skills}.

        There are other ways to become trained in skills that are not part of your class.
        If your base Intelligence is positive, you gain additional trained skills equal to your base Intelligence.
        You can also spend \glossterm{insight points} to gain one trained skill per insight point (see \pcref{Insight Points}).
        Some abilities can grant additional trained skills.

        If you are untrained in a skill, your bonus with that skill is equal to half of its associated attribute (if any).
        If you are trained in a skill, your bonus with that skill is equal to 3 \add the higher of its associated attribute (if any) and half your level.
        Many abilities can increase or decrease your bonus with particular skills.

        The number of skills you can have trained, and which skills those are, depend on every preceding step, so it's a good place to finish.

        Sometimes, you might have more trained skills than you know what to do with, especially if you are still figuring out the details of your character concept.
        You aren't forced to decide all of your trained skills at level 1, so you can save them up and choose more trained skills when you level up.
        You can also talk to your GM about letting you decide your trained skills on the fly during the first game session or two based on what actions you take during the session.
        This can be a fun way to figure out what your character's personality is through the process of playing them.

    \subsection{Step 9: Starting Equipment}
        When you create a character, they can start with some basic items.
        Items have \glossterm{item levels} that indicate the approximate level that characters can reasonably get access to them.
        Typically, you can start with a single level 1 or lower item, up to three level 1/2 items, and a standard adventuring kit.
        Individual campaigns or character backstories may significantly change what starting equipment is available, so check with your GM.

    \subsection{Step 10: Personality}

        How does your character behave?
        You should decide, in broad terms, what your character's personality is.
        This will change over time, especially as you start playing the character in the game, so you don't need to define everything perfectly.
        However, having a general sense of how your character behaves is helpful.

        For most games, it's important to have a personality that can tolerate working with others in a group.
        A character that is excessively aloof, moody, or obnoxious can make the game more difficult to enjoy for everyone.
        Likewise, a character who tries to speak for everyone or who repeatedly steals the spotlight from others can be frustrating to work with.
        You should figure out the right balance with your fellow players and your GM.\@

    \subsection{Step 11: Background}
        What happened in their character's past to make them the way that they are?
        What were their parents like, and where are they now?
        You don't have to have all of the answers when you first create a character, but it's good to have some idea.
        The richer your backstory, the more the GM can weave that into the narrative of the current story.
        Sometimes, it's fun to take a break from saving the world to go visit someone's grandma.

    \subsection{Step 12: Appearance}
        What does your character look like?
        What would someone's first impression of them be?
        This can be helpful for understanding how other characters in the game world - or even monsters - would react to you.

    \subsection{Step 13: Name}
        What is your character's name?
        This seemingly minor choice can reveal a lot about the tone your character will set in the universe.
        If your name is Sir Patty Cakes or Shanky, the game is likely to be lighter and sillier in tone.
        Fancy fantasy-appropriate names ike Ayala or Theodolus tend to push the game in a slightly more serious direction, especially if you make the daring choice to include a canonical last name.
        As always, stay in tune with what the GM and the other players are expecting.

\section{Character Advancement}\label{Character Advancement}

    As you accomplish challenges and defeats foes, you gain experience.
    If you have enough experience, you gain a level.
    You gain some abilities at specific levels, as described in \trefnp{Character Advancement}.

    When you gain a level, the following things happen:
    \begin{itemize}
        \item Your \glossterm{hit points} and \glossterm{damage resistance} increase (see \tref{Character Advancement}).
        \item You gain an additional \glossterm{archetype rank} (see \pcref{Archetypes})
        \item Your \glossterm{attributes} may increase (see \tref{Increasing Attributes With Level})
        \item Your bonus with \glossterm{trained} skills may increase (see \pcref{Trained Skills})
        \item If the level is even, your \glossterm{accuracy} increases by 1 (see \pcref{Accuracy})
        \item If the level is even, all of your \glossterm{defenses} increase by 1 (see \pcref{Defenses})
        \item At 2nd level, and every 3 levels thereafter, you gain an additional \glossterm{attunement point} (see \pcref{Attunement Points}).
        \item At 3rd level, and every 6 levels thereafter, you gain a \glossterm{legacy item} upgrade (see \pcref{Legacy Items}).
        \item At 4th level, and every 3 levels thereafter, your maximum \glossterm{archetype rank} increases (see \pcref{Archetype Ranks}).
    \end{itemize}

    \begin{dtable}
        \lcaption{Character Advancement}
        \begin{dtabularx}{\columnwidth}{l >{\lcol}X l l >{\lcol}X l l}
            \tb{Level} & \tb{Max Rank}\fn{1} & \tb{HP} & \tb{DR}\fn{2} & \tb{Legacy Item}\fn{3} & \tb{AP}\fn{4} & \tb{XP} \tableheaderrule
            1st        & 1                   & 11      & 2             & \tdash                 & (Class)       & 0      \\
            2nd        & \tdash              & 12      & 3             & \tdash                 & \tdash        & 20     \\
            3rd        & \tdash              & 13      & 3             & 1                      & \tdash        & 50     \\
            4th        & 2                   & 15      & 3             & \tdash                 & \tdash        & 90     \\
            5th        & \tdash              & 17      & 4             & \tdash                 & \plus1        & 150    \\
            6th        & \tdash              & 19      & 4             & \tdash                 & \tdash        & 230    \\
            7th        & 3                   & 22      & 5             & \tdash                 & \tdash        & 350    \\
            8th        & \tdash              & 25      & 6             & \tdash                 & \tdash        & 510    \\
            9th        & \tdash              & 28      & 7             & 2                      & \tdash        & 750    \\
            10th       & 4                   & 31      & 8             & \tdash                 & \tdash        & 1,050  \\
            11th       & \tdash              & 35      & 9             & \tdash                 & \plus1        & 1,550  \\
            12th       & \tdash              & 39      & 10            & \tdash                 & \tdash        & 2,200  \\
            13th       & 5                   & 44      & 11            & \tdash                 & \tdash        & 3,150  \\
            14th       & \tdash              & 50      & 12            & \tdash                 & \tdash        & 4,450  \\
            15th       & \tdash              & 56      & 14            & 3                      & \tdash        & 6,350  \\
            16th       & 6                   & 63      & 15            & \tdash                 & \tdash        & 8,900  \\
            17th       & \tdash              & 70      & 17            & \tdash                 & \plus1        & 13,000 \\
            18th       & \tdash              & 78      & 19            & \tdash                 & \tdash        & 18,000 \\
            19th       & 7                   & 88      & 22            & \tdash                 & \tdash        & 25,500 \\
            20th       & \tdash              & 100     & 25            & \tdash                 & \tdash        & 36,000 \\
            21st       & \tdash              & 115     & 28            & 4                      & \tdash        & 60,000 \\
        \end{dtabularx}
        1. See \pcref{Archetype Ranks}. \\
        2. See \pcref{Damage Resistance}. \\
        3. See \pcref{Legacy Items}. \\
        4. See \pcref{Attunement Points}. \\
    \end{dtable}

    \subsection{Legacy Items}\label{Legacy Items}

        Over time, items associated with places and people of great power gain magical properties.
        This process takes place for you as you gain levels in addition to in the world as a whole.

        At 3rd level, you choose a nonmagical weapon, piece of armor, apparel item, or implement you own.
        That item becomes a \glossterm{legacy item}, and gains a magic item ability you choose.
        You do not have to \glossterm{attune} to your legacy item to gain its benefits.
        The ability's level must be no greater than 5th level, and it must be appropriate for the category of item you chose: weapon, armor, apparel, or implement.
        You do not have to precisely match the location of an apparel item.
        For example, you can choose an amulet as your legacy item and give it the effect of the \mitem{boots of translocation}.

        At 9th, 15th, and 21st level, your legacy item increases in power again.
        You choose an ability of the appropriate type with a level no greater than two levels higher than your level when you choose the ability.
        In addition, you can change all of your lower level legacy item abilities.
        Each ability must meet the same maximum level requirement that it had when you first chose it.
        You cannot create a legacy item with two versions of the same ability on it, such as armor with both the \textit{invulnerability} and \textit{greater invulnerability} properties.

        If you lose your legacy item, you must retrieve it to regain its power.
        There are rituals to facilitate this retrieval such as \ritual{seek legacy} and \ritual{retrieve legacy}.
        If your legacy item is \glossterm{destroyed}, you can designate a new item of the same type to be your legacy item, causing it to gain all of your legacy item abilities.
        Designating a new item in this way requires taking a \glossterm{long rest} while holding or wearing the replacement item.

        Your legacy item can be reforged by a skilled smith without losing its legacy item abilities.
        This can allow you to reforge legacy body armor with special materials, which is important at high levels.

        \parhead{Unique Legacy Items}
            Legacy items are fundamentally a reflection of the character who wields them.
            Their effects can be more unusual and complex than abilities on normal magic items, and they can have a larger effect on the way that character interacts with the world.
            As a player, you can work with your GM to create custom magical effects of an appropriate power that are a better reflection of your character's personality and powers than the magic item abilities that exist.

\section{Sample Characters}

    This section lists sample characters for each class archetype.
    You can simply pick up one of these characters and use it as your character.
    Alternately, you can use a sample character as a starting point and adjust it to match your own character concept.
    The sample characters are ordered by class first, and by archetype within each class second.

    \subsection{Barbarian}

        \subsubsection{Battleforged Resilience}
            \parhead{Species} Dwarf.
            \parhead{Attributes} 2 Str, 0 Dex, 4 Con, 0 Int, 2 Per, 1 Wil (after species modifiers).
            \parhead{Class} Barbarian.
            \parhead{Archetypes} Battleforged Resilience first, Primal Warrior second, Totemist (bear totem) third.
            \parhead{Insight Points} 1 point for heavy armor.
            \parhead{Skills} Awareness, Climb, Endurance, Medicine, Persuasion, Survival
            \parhead{Weapon Groups} Axes, thrown weapons.
            \parhead{Languages} Common, Dwarven, Giant.
            \parhead{Equipment} Battleaxe, standard shield, scale mail. As you gain levels, use the best heavy armor you can afford.
            \parhead{Legacy Item} Shield.
                At level 3, choose \mitem{covering shield}.
                At level 9, choose \mitem{greater shield of arrow catching} and \mitem{covering shield}.
                At level 15, choose \mitem{shield of mystic reflection}, \mitem{greater shield of arrow catching}, and \mitem{covering shield}.
            \parhead{Combat Styles} Herald of War, Unbreakable Defense.
            \parhead{Suggested Maneuvers}
            \begin{itemize}
                \item Rank 1: \maneuver{boastful battlecry}, \maneuver{guard the pass}, \maneuver{shield slam}
                \item Rank 2: \maneuver{cleanse}, \maneuver{defensive strike}, \maneuver{directed shout}
                \item Rank 3: \maneuver{challenging strike}, \maneuver{flamboyant parry}, \maneuver{revitalizing strike}
                \item Rank 4: \maneuver{fearsome roar}, \maneuver{rally the troops}
                \item Rank 5: \maneuver{greater goading roar}, \maneuver{redirecting parry}
                \item Rank 6: \maneuver{greater directed shout}, \maneuver{revitalizing battlecry}
                \item Rank 7: \maneuver{reflective parry}, \maneuver{stunning roar}
            \end{itemize}
            \parhead{Suggested Feats} Shieldbearer, Regenerator, Toughness.
            \parhead{Combat Tactics} You are extremely difficult to kill.
            Take advantage of that by wading into the front lines of combat and drawing attention away from your more vulnerable allies.
            If you find yourself in danger, use defensive maneuvers like \maneuver{defensive strike} and \maneuver{flamboyant parry} to keep yourself safe.
            On the other hand, if your foes try to ignore you after realizing how durable you are, force them to engage with you using maneuvers like \maneuver{challenging strike} and \maneuver{guard the pass}.

        \subsubsection{Battlerager}
            \parhead{Species} Half-orc.
            \parhead{Attributes} 4 Str, 2 Dex, 2 Con, -1 Int, 2 Per, 0 Wil (after species modifiers).
            \parhead{Class} Barbarian.
            \parhead{Archetypes} Battlerager first, Primal Warrior second, Totemist (lion totem) third.
            \parhead{Insight Points} 1 point for extra maneuver.
            \parhead{Skills} Awareness, Climb, Endurance, Intimidate, Jump, Swim.
            \parhead{Weapon Groups} Club-like weapons, crossbows.
            \parhead{Equipment} Greatmace, scale mail. As you gain levels, buy a heavy crossbow and use the best medium armor you can afford.
            \parhead{Legacy Item} Weapon.
                At level 3, choose \mitem{potency}.
                At level 9, choose \mitem{greater potency} and \mitem{onslaught}.
                At level 15, choose \mitem{supreme potency}, \mitem{honed}, and \mitem{onslaught}.
            \parhead{Combat Styles} Flurry of Blows, Unbreakable Defense.
            \parhead{Suggested Maneuvers}
            \begin{itemize}
                \item Rank 1: \maneuver{deathseeking flurry}, \maneuver{frenzied strike}, \maneuver{twinstrike}.
                \item Rank 2: \maneuver{cleanse}, \maneuver{rebounding flurry}, \maneuver{steadfast strike}
                \item Rank 3: \maneuver{revitalizing strike}, \maneuver{strike flurry}
                \item Rank 4: \maneuver{desperate flurry}, \maneuver{second wind}
                \item Rank 5: \maneuver{bracing strike}
                \item Rank 6: \maneuver{greater defensive strike}, \maneuver{triplestrike}
                \item Rank 7: \maneuver{greater desperate flurry}
            \end{itemize}
            \parhead{Suggested Feats} Greatweapon Warrior, Rapid Reaction, Swift.
            \parhead{Combat Tactics} You are a furious frenzy of devastating damage and lethal critical hits.
            When you roll a 10 on an attack roll, whatever you attacked will probably die.
            Staying close to your allies is generally a good plan, since you don't have the durability to run into the middle of a horde of enemies safely.
            Your maneuvers help you deal with high-Armor enemies and enemy swarms, and give you the ability to sacrifice most of your statistics other than damage in exchange for more damage.

        \subsubsection{Outland Savage}
            If you want to quickly create a character based on this archetype, make the following choices:
            \parhead{Species} Half-orc.
            \parhead{Attributes} 5 Str, 2 Dex, 1 Con, -1 Int, 0 Per, 1 Wil (after species modifiers).
            \parhead{Class} Barbarian.
            \parhead{Archetypes} Outland Savage first, Primal Warrior second, Totemist (wolf totem) third.
            \parhead{Insight Points} 1 point for proficiency with exotic armor weapons.
            \parhead{Skills} Awareness (M), Climb (M), Endurance (M), Jump (M), Swim (M).
            \parhead{Weapon Groups} Armor weapons, flexible weapons.
            \parhead{Languages} Common, Orc.
            \parhead{Equipment} Flail, scale mail. As you gain levels, use the best medium armor you can afford, and get spikes and a spiked knee crafted onto it.
            \parhead{Legacy Item} Apparel.
                At level 3, choose \mitem{phasestep boots}.
                At level 9, choose \mitem{greater phasestep boots} and \mitem{boots of speed}.
                At level 15, choose \mitem{enlarging belt}, \mitem{greater phasestep boots}, and \mitem{boots of speed}.
            \parhead{Combat Styles} Dirty Fighting, Mobile Assault.
            \parhead{Suggested Maneuvers}
            \begin{itemize}
                \item Rank 1: \maneuver{anklesprainer}, \maneuver{wanderer's strike}
                \item Rank 2: \maneuver{knockback shove}, \maneuver{strangle}
                \item Rank 3: \maneuver{battering ram}, \maneuver{revitalizing strike}
                \item Rank 4: \maneuver{greater anklesprainer}, \maneuver{steal weapon}
                \item Rank 5: \maneuver{eye-averting strike}, \maneuver{spellbreaker strike}
                \item Rank 6: \maneuver{greater reaping harvest}, \maneuver{greater revitalizing strike}
                \item Rank 7: \maneuver{greater steal weapon}, \maneuver{instant pin}
            \end{itemize}
            \parhead{Suggested Feats} Savage, Brawler, Swift.
            \parhead{Combat Tactics} You can move around the battlefield very quickly, and you are incredibly accurate with special combat actions like shoving and grappling enemies.
            Make the most of that by repositioning enemies, tripping them, or holding them in grapples so your allies can hit them.
            While you aren't in a grapple, use your flail in two hands to maximize your damage.
            When you enter a grapple, use your spiked knee to attack, since your flail is much less effective while grappling.
            If you don't have any allies who like being on the front lines, you won't be as effective at helping them deal damage to enemies, but you're still very skilled at preventing enemies from reaching your allies.
            In that case, consider choosing bear totem or shark totem instead of wolf totem.

        \subsubsection{Primal Warrior}
            \parhead{Species} Human.
            \parhead{Attributes} 2 Str, 2 Dex, 2 Con, 2 Int, 2 Per, 0 Wil.
            \parhead{Class} Barbarian.
            \parhead{Archetypes} Primal Warrior first, Battleforged Resilience second, Outland Savage third.
            \parhead{Insight Points} 4 points for additional maneuvers.
            \parhead{Skills} Awareness (M), Balance (M), Climb (M), Endurance (M), Medicine (T), Jump (M), Swim (M), Survival (M).
            \parhead{Weapon Groups} Axes, crossbows.
            \parhead{Languages} Common, Dwarven, Orc.
            \parhead{Equipment} Greataxe, scale mail. As you gain levels, buy a heavy crossbow and use the best medium armor you can afford.
            \parhead{Legacy Item} Weapon.
                At level 3, choose \mitem{surestrike}.
                At level 9, choose \mitem{greater surestrike} and \mitem{blessed}.
                At level 15, choose \mitem{supreme surestrike}, \mitem{greater shocking}, and \mitem{blessed}.
            \parhead{Combat Styles} Dirty Fighting, Herald of War, Unbreakable Defense.
            \parhead{Suggested Maneuvers} You can learn most of the maneuvers available at each rank from your combat styles, so it's not meaningful to list specific maneuvers here.
            Choose whatever is most interesting to you.
            \parhead{Suggested Feats} Greatweapon Warrior, Weapon Focus, Swift.
            \parhead{Combat Tactics} You have a great breadth of options available to you thanks to the number of maneuvers you know.
            You have the survivability to stand in close combat, especially if you use maneuvers from Unreakable Defense, but you can also shout at mobile enemies from range with maneuvers from Herald of War.
            Both Dirty Fighting and Herald of War give you maneuvers that work well against enemies with a high Armor defense, so you can adapt to whatever battle you find yourself in.
            You can make the most of your versatility by learning maneuvers like \maneuver{disarm weapon} that are sometimes useless, but which can be devastatingly effective in the right context.

        \subsubsection{Totemist}
            Characters from this archetype can be very different based on their chosen totem.
            A bear totem character might resemble the typical character for the Battleforged Resilience archetype.
            A lion totem or shark totem character might resemble the typical character for the Battlerager archetype.
            A wolf totem character might resemble the typical character for the Outland Savage archetype.

            If you want to quickly create a character based on the eagle totem from this archetype, make the following choices:
            \parhead{Species} Human.
            \parhead{Attributes} 2 Str, 1 Dex, 0 Con, 0 Int, 4 Per, 1 Wil.
            \parhead{Class} Barbarian.
            \parhead{Archetypes} Totemist (eagle totem) first, Primal Warrior second, Battlerager third.
            \parhead{Insight Points} 1 point for proficiency with exotic bows, 1 point for additional maneuvers.
            \parhead{Skills} Awareness (M), Balance (M), Climb (T), Creature Handling (M), Endurance (M), Jump (T), Swim (T), Survival (M).
            \parhead{Weapon Groups} Bows, thrown weapons.
            \parhead{Languages} Common, Elven, Giant.
            \parhead{Equipment} Longbow, leather body armor. As you gain levels, buy a flatbow and use the best light armor you can afford.
            \parhead{Legacy Item} Weapon.
                At level 3, choose \mitem{surestrike}.
                At level 9, choose \mitem{greater surestrike} and \mitem{longshot}.
                At level 15, choose \mitem{supreme surestrike}, \mitem{greater freezing}, and \mitem{longshot}.
            \parhead{Combat Styles} Flurry of Blows, Mobile Assault, Penetrating Precision.
            \parhead{Suggested Maneuvers}
            \begin{itemize}
                \item Rank 1: \maneuver{deathblow}, \maneuver{penetrating strike}, \maneuver{wanderer's strike}
                \item Rank 2: \maneuver{arrowguide}, \maneuver{quickshot}
                \item Rank 3: \maneuver{heartpiercing strike}, \maneuver{penetrating shot}
                \item Rank 4: \maneuver{barrage}, \maneuver{groundspike}
                \item Rank 5: \maneuver{greater deathblow}, \maneuver{volley fire}
                \item Rank 6: \maneuver{greater eye gouge}, \maneuver{greater retreating strike}
                \item Rank 7: \maneuver{greater desperate flurry}, \maneuver{greater groundspike}
            \end{itemize}
            \parhead{Suggested Feats} Sniper, Weapon Focus, Swift.
            \parhead{Combat Tactics} You have incredible accuracy from very long range.
            Your defenses are low, but as long as you stay far enough away from your foes, they can't take advantage of that weakness.
            You have the ability to prioritize any target on the battlefield, so make the most of your maneuvers that impose conditions or deal additional damage on weakened foes.

    \subsection{Cleric}

        \subsubsection{Divine Magic}
            \parhead{Species} Gnome.
            \parhead{Attributes} 0 Str, 0 Dex, 3 Con, 1 Int, 2 Per, 3 Wil (after species modifiers).
            \parhead{Class} Cleric.
            \parhead{Archetypes} Divine Magic first, Divine Spell Mastery second, Domain Influence third.
            \parhead{Insight Points} 2 points for an additional mystic sphere, 1 point for an additional spell known.
            \parhead{Skills} Knowledge (religion) (M), Medicine (M), Persuasion (T), Spellsense (M), Social Insight (T)
            \parhead{Weapon Group} Club-like weapons.
            \parhead{Languages} Common, Dwarven, Halfling.
            \parhead{Equipment} Mace, scale mail. As you gain levels, use the best medium armor you can afford.
            \parhead{Legacy Item} 1-handed implement.
                At level 3, choose \mitem{staff of precision}.
                At level 9, choose \mitem{greater staff of precision} and \mitem{staff of focus}.
                At level 15, choose \mitem{supreme staff of precision}, \mitem{fearsome staff}, and \mitem{staff of focus}.
            \parhead{Domains} Good, Magic
            \parhead{Mystic Spheres} Bless, Channel Divinity, and Vivimancy
            \parhead{Suggested Spells}
            \begin{itemize}
                \item Rank 1: \spell{blessing of endurance}, \spell{boon of precision}, \spell{divine judgment}, \spell{inflict wound}
                \item Rank 2: \spell{cure wound}, \spell{divine conduit}, \spell{word of faith}
                \item Rank 3: \spell{boon of cleansing}, \spell{lifesteal}, \spell{mantle of faith}
                \item Rank 4: \spell{greater divine judgment}, \spell{greater inflict wound}, \spell{greater word of faith}
                \item Rank 5: \spell{circle of life}, \spell{cure vital wound}, \spell{fear of the divine}
                \item Rank 6: \spell{boon of invulnerability}, \spell{cleansing benediction}, \spell{greater divine presence}
                \item Rank 7: \spell{avasculate}, \spell{supreme divine judgment}
            \end{itemize}
            \parhead{Suggested Feats} Leadership, Celestial Heritage, Sphere Focus: Channel Divinity
            \parhead{Combat Tactics} You can protect and heal your allies and invoke divine wrath on your foes.
            You have a mixture of attacks against both Fortitude and Mental defense, so use the best spells for the situation.
            If you are facing a foe that not particularly vulnerable to either, you can focus on keeping your allies healed and using "boon" spells to make their actions more effective.

        \subsubsection{Divine Spell Mastery}
            Use the typical character for the Divine Magic cleric archetype.
            Even if you focus on spells through this archetype, you should generally still rank up your spells before improving your rank in this archetype.

        \subsubsection{Domain Influence}
            Characters from this archetype can be very different based on their chosen domains.
            A character with spellcasting-focused domains might resemble the typical character for the Divine Magic cleric archetype.
            If you want to quickly create a more martial character based on the Strength and War domains from this archetype, make the following choices:

            \parhead{Species} Dwarf.
            \parhead{Attributes} 3 Str, 0 Dex, 3 Con, 0 Int, 2 Per, 1 Wil (after species modifiers).
            \parhead{Class} Cleric.
            \parhead{Archetypes} Domain Influence first, Divine Magic second, Preacher third.
            \parhead{Insight Points} 3 points for additional spells known.
            \parhead{Skills} Awareness (T), Climb (T), Knowledge (religion) (M), Jump (T), Medicine (M), Persuasion (M), Spellsense (T), Swim (T)
            \parhead{Weapon Group} Club-like weapons.
            \parhead{Languages} Common, Draconic, Dwarven.
            \parhead{Equipment} Morning star, standard shield, scale mail. As you gain levels, use the best heavy armor you can afford.
            \parhead{Legacy Item} Shield.
                At level 3, choose \mitem{protective shield}.
                At level 9, choose \mitem{greater protective shield} and \mitem{shield of arrow catching}.
                At level 15, choose \mitem{supreme protective shield}, \mitem{greater shield of arrow deflection}, and \mitem{shield of arrow catching}.
            \parhead{Domains} Strength, War
            \parhead{Mystic Spheres} Channel Divinity
            \parhead{Suggested Spells}
            \begin{itemize}
                \item Rank 1: \spell{divine power}, \spell{divine favor}, \spell{divine authority}, \spell{stunning judgment}
                \item Rank 2: \spell{astral refuge}
                \item Rank 3: \spell{banish anathema}, \spell{divine might}, \spell{divine presence}, \spell{mantle of faith}
                \item Rank 4: \spell{greater word of faith}
                \item Rank 5: \spell{agent of the divine}, \spell{fear of the divine}
                \item Rank 6: \spell{divine offering}, \spell{greater divine presence}
                % \item Rank 7: TODO
            \end{itemize}
            \parhead{Suggested Feats} Weapon Focus, Sphere Focus: Channel Divinity, Shieldbearer
            \parhead{Combat Tactics} You are a frontline fighter first and foremost.
            Your high defenses and magically enhanced resistances make you durable in combat, though you lack mobility. 
            When you need to distract foes or face down hordes, you can use your abilities from the Preacher archetype, which do not have the \abilitytag{Focus} tag.
            If you can't take the Weapon Focus feat, consider taking the Destruction domain instead of the Strength domain, since that gives you a standard action ability to help you deal damage with your weapon.

        \subsubsection{Healer}
            \parhead{Species} Gnome.
            \parhead{Attributes} -1 Str, 2 Dex, 3 Con, 0 Int, 0 Per, 4 Wil (after species modifiers).
            \parhead{Class} Cleric.
            \parhead{Archetypes} Healer first, Divine Magic second, Domain Influence third.
            \parhead{Insight Points} 3 points for additional spells known.
            \parhead{Skills} Awareness (T), Climb (T), Knowledge (religion) (M), Jump (T), Medicine (M), Persuasion (M), Spellsense (T), Swim (T)
            \parhead{Weapon Group} Club-like weapons.
            \parhead{Languages} Common, Draconic, Halfling.
            \parhead{Equipment} Morning star, standard shield, scale mail. As you gain levels, use the best medium armor you can afford.
            \parhead{Legacy Item} 1-handed implement.
                At level 3, choose \mitem{staff of potency}.
                At level 9, choose \mitem{reaching staff} and \mitem{staff of potency}.
                At level 15, choose \mitem{greater staff of the archmagi}, \mitem{reaching staff}, and \mitem{staff of focus}.
            \parhead{Domains} Life, Protection
            \parhead{Mystic Spheres} Vivimancy
            \parhead{Suggested Spells}
            \begin{itemize}
                \item Rank 1: \spell{drain life}, \spell{draining grasp}, \spell{lifegift}, \spell{retributive lifebond}
                \item Rank 2: \spell{cure wound}, \spell{sickening miasma}, \spell{wellspring of life}
                \item Rank 3: \spell{circle of death}, \spell{lifesteal}, \spell{vital persistance}
                \item Rank 4: \spell{greater inflict wound}, \spell{greater retributive lifebond}
                \item Rank 5: \spell{circle of death}, \spell{nauseating miasma}, \spell{steal vitality}
                % \item Rank 6: TODO
                \item Rank 7: \spell{avasculate}, \spell{supreme retributive lifebond}
            \end{itemize}
            \parhead{Suggested Feats} Sphere Focus: Vivimancy, Boongiver, Iron Will
            \parhead{Combat Tactics} You have an unmatched mastery of healing and protection.
            You have high defenses, so you can take to the front lines as necessary to make the most of \ability{restoration} and \ability{divine protection}, but it's generally better to let your allies take hits instead of you.
            Since \ability{restoration} is much less effective at healing yourself, you can use spells like \spell{cure wounds} or \spell{lifesteal} to heal yourself if you lose hit points.
            Although your \ability{healer's grace} ability is powerful, you shouldn't feel bad about attacking enemies.
            That's especially important early in a fight when your allies don't need healing yet and your enemies haven't realized that it's pointless to attack your allies while you are still standing.

    \subsection{Druid}

        \subsubsection{Elementalist}
            \parhead{Species} Human.
            \parhead{Attributes} 0 Str, 0 Dex, 0 Con, 2 Int, 4 Per, 2 Wil.
            \parhead{Class} Druid.
            \parhead{Archetypes} Nature Magic first, Elementalist second, Nature Spell Mastery third.
            \parhead{Insight Points} 2 points for mystic spheres, 4 points for spells
            % 18 total skill points, counting the 3 from auto-training
            \parhead{Skills} Awareness (M), Creature Handling (M), Endurance (T), Jump (M), Knowledge (geography, nature) (M), Spellsense (M), Survival (M), Swim (M)
            \parhead{Weapon Group} Headed weapons
            \parhead{Languages} Common, Sylvan
            \parhead{Equipment} Sickle, standard shield, scale mail. As you gain levels, use the best medium armor you can afford.
            You may want to keep leather armor around in case you need to do a lot of jumping or swimming.
            \parhead{Legacy Item} 1-handed implement.
                At level 3, choose \mitem{staff of potency}.
                At level 9, choose \mitem{greater staff of potency} and \mitem{staff of focus}.
                At level 15, choose \mitem{supreme staff of potency}, \mitem{reaching staff}, and \mitem{staff of focus}.
            \parhead{Mystic Spheres} Any three of the four elemental mystic spheres.
            Your \textit{elemental versatility} ability gives you access to spells from the fourth mystic sphere.
            That means that the specific three mystic spheres you choose mostly just affect which wands you can use and which feats you can take.
            \parhead{Suggested Spells}
            You have access to spells from all four elemental mystic spheres, so you have a massive pool of spells available to you.
            The list below is just one of the possible paths you could take.
            \begin{itemize}
                \RaggedRight
                % at every level: one single-target damage, one AOE damage, one debuff, one self-buff, each from a different sphere
                \item Rank 1: \spell{desiccation}, \spell{firebolt}, \spell{shrapnel blast}, \spell{wind screen}
                \item Rank 2: \spell{combustion}, \spell{downdraft}, \spell{rocky shell}, \spell{wave of dehydration}
                \item Rank 3: \spell{earthbind}, \spell{fireball}, \spell{geyser}, \spell{retributive winds}
                \item Rank 4: \spell{constraining bubble}, \spell{flight}, \spell{fissure}, \spell{immolate}
                \item Rank 5: \spell{earthglide}, \spell{greater firebolt}, \spell{greater gust of wind}, \spell{greater wave of dehydration}
                \item Rank 6: \spell{agile flight}, \spell{earthcage}, \spell{greater flame dash}, \spell{supreme fountain}
                \item Rank 7: \spell{blinding dust cloud}, \spell{earthquake}, \spell{soul of the phoenix}, \spell{supreme forceful aquajet}
            \end{itemize}
            \parhead{Suggested Feats} Sphere Focus: Aeromancy, Aquamancy, Pyromancy, or Terramancy
            \parhead{Combat Tactics} You are a master of all four elements, so you have an immense variety of options available to you - if you choose the right spells.
            You have a very high accuracy thanks to your Perception and a reasonably high magical power, so your primary role in combat will usually be to deploy the perfect damaging spell or debuff for the situation.
            Your skills and Elementalist abilities give you a lot of narrative power, so stay alert for opportunities to overcome challenges without needing to fight at all.

        \subsubsection{Nature Magic}
            If you want to quickly create a character based on this archetype, make the following choices:
            \parhead{Species} Elf.
            \parhead{Attributes} 1 Str, 3 Dex, 0 Con, 0 Int, 2 Per, 3 Wil.
            \parhead{Class} Druid.
            \parhead{Archetypes} Nature Magic first, Nature Spell Mastery second, Elementalist third.
            \parhead{Insight Points} 2 points for a mystic sphere, 1 point for a spell
            % 12 total skill points, counting the 3 from auto-training
            \parhead{Skills} Awareness (M), Balance (T), Creature Handling (M), Knowledge (nature) (M), Ride (T), Stealth (M), Survival (M)
            \parhead{Weapon Group} Headed weapons
            \parhead{Languages} Common, Sylvan
            \parhead{Equipment} Sickle, standard shield, leather armor. As you gain levels, use the best light armor you can afford.
            \parhead{Legacy Item} 1-handed implement.
                At level 3, choose \mitem{staff of precision}.
                At level 9, choose \mitem{greater staff of precision} and \mitem{staff of focus}.
                At level 15, choose \mitem{supreme staff of precision}, \mitem{extending staff}, and \mitem{staff of focus}.
            \parhead{Mystic Spheres} Aquamancy, Verdamancy
            \parhead{Suggested Spells} 
            \begin{itemize}
                \item Rank 1: \spell{aquajet blast}, \spell{barkskin}, \spell{crushing wave}, \spell{poison -- sassone leaf}
                \item Rank 2: \spell{entangle}, \spell{forceful aquajet}, \spell{obscuring mist}, \spell{poison -- nitharit}, \spell{wave of dehydration}
                \item Rank 3: \spell{desiccating curse}, \spell{fire seeds}, \spell{greater aquajet blast}, \spell{poison -- arsenic}, \spell{wall of thorns}
                \item Rank 4: \spell{aqueous form}, \spell{constraining bubble}, \spell{greater vine whip}, \spell{raging river}
                \item Rank 5: \spell{fluid motion}, \spell{greater entangle}, \spell{greater wave of dehydration}, \spell{poison -- black lotus}
                \item Rank 6: \spell{greater fire seeds}, \spell{greater geyser}, \spell{ring of mist}
                \item Rank 7: \spell{strangling vines}, \spell{supreme desiccation}, \spell{supreme vine whip}
            \end{itemize}
            \parhead{Suggested Feats} Sphere Focus: Aquamancy, Sphere Focus: Verdamancy, Herbalist
            \parhead{Combat Tactics} You are a master of plants and nature.
            Your spells excel at moving foes around the battlefield and constraining their movement while dealing reasonable damage.
            You also have access to dangerous poisons to weaken your foes while they remain safely kept at bay.

        \subsubsection{Nature Spell Mastery}
            Use the typical character for the Nature Magic druid archetype.
            Even if you focus on spells through this archetype, you should generally still rank up your spells before improving your rank in this archetype.

        \subsubsection{Shifter}
            If you want to quickly create a character based on this archetype, make the following choices:
            \parhead{Species} Half-orc.
            \parhead{Attributes} 3 Str, 3 Dex, 2 Con, 0 Int, 0 Per, 1 Wil.
            \parhead{Class} Druid.
            \parhead{Archetypes} Shifter first, Nature Magic second, Nature Spell Mastery third.
            \parhead{Insight Points} 2 points for wild aspects, 2 points for spells
            % 15 total skill points, counting the 6 from auto-training
            \parhead{Skills} Awareness (M), Balance (M), Climb (M), Jump (M), Ride (T), Stealth (M), Survival (M), Swim (M)
            \parhead{Weapon Group} Bows
            \parhead{Languages} Common, Sylvan
            \parhead{Equipment} Natural weapon, standard shield, chain shirt. As you gain levels, use the best light armor you can afford.
            Use your natural weapons instead of manufactured weapons unless you need to fight at long range.
            \parhead{Legacy Item} Apparel.
                At level 3, choose \mitem{amulet of mighty fists}.
                At level 9, choose \mitem{greater amulet of mighty fists} and \mitem{ring of blessed protection}.
                At level 15, choose \mitem{supreme amulet of mighty fists}, \mitem{enlarging belt}, and \mitem{ring of blessed protection}.
            \parhead{Mystic Sphere} Polymorph
            \parhead{Suggested Wild Aspects} Your choice of wild aspect has a significant effect on your capabilities, and they are less complicated to evaluate than spell, so choose wild aspects that match your goals.
            The Bear, Viper, and Wolf forms excel at dealing damage in combat.
            The Bull and Constrictor forms improve your ability to take unusual combat actions.
            Other forms can be useful in specific circumstances and out of combat.
            \parhead{Suggested Spells} 
            \begin{itemize}
                \item Rank 1: \spell{camouflage}, \spell{mighty claw}, \spell{organ failure}, \spell{stoneskin}, \spell{twisting claw}
                \item Rank 2: \spell{brief regeneration}, \spell{bleed}, \spell{distant claw}, \spell{shrink}
                \item Rank 3: \spell{enlarge}, \spell{scent}, \spell{spikeform}
                \item Rank 4: \spell{draconic senses}, \spell{eyebite}, \spell{malleable body}
                \item Rank 5: \spell{baleful polymorph}, \spell{greater bleed}, \spell{vital regeneration}
                \item Rank 6: \spell{extruding spikes}
                \item Rank 7: \spell{cripple}, \spell{sludgeform}
            \end{itemize}
            \parhead{Suggested Feats} Sphere Focus: Polymorph, Regenerator, Brawler, Savage
            \parhead{Combat Tactics} You are a lethal blend of claws and teeth.
            You can shift your form to gain the perfect abilities for your current circumstances, and your high physical attributes make you hard to kill and hard to ignore.
            Your flexibility between natural weapons, spells, and high physical skills give you a lot of options in and out of combat.

        \subsubsection{Wildspeaker}
            If you want to quickly create a character based on this archetype, make the following choices:
            \parhead{Species} Gnome.
            \parhead{Attributes} -1 Str, 0 Dex, 3 Con, 0 Int, 4 Per, 2 Wil.
            \parhead{Class} Druid.
            \parhead{Archetypes} Wildspeaker first, Nature Magic second, Nature Spell Mastery third.
            \parhead{Insight Points} 3 points for spells.
            % 12 total skill points, counting the 3 from auto-training
            \parhead{Skills} Awareness (M), Creature Handling (M), Knowledge (nature) (M), Ride (M), Stealth (M), Survival (M)
            \parhead{Weapon Group} Headed weapons
            \parhead{Languages} Common, Gnome, Sylvan
            \parhead{Equipment} Sickle, standard shield, scale mail. As you gain levels, use the best medium armor you can afford.
            \parhead{Legacy Item} Apparel.
                At level 3, choose \mitem{belt of healing}.
                At level 9, choose \mitem{shrinking belt} and \mitem{ring of blessed protection}.
                At level 15, choose \mitem{supreme belt of healing}, \mitem{shrinking belt}, and \mitem{ring of blessed protection}.
            \parhead{Mystic Sphere} Electromancy
            \parhead{Suggested Spells} 
            \begin{itemize}
                \item Rank 1: \spell{electric jolt}, \spell{electroshock}, \spell{energize}, \spell{shocking grasp}
                \item Rank 2: \spell{lightning storm}, \spell{stunning discharge}
                \item Rank 3: \spell{call lightning}, \spell{lightning bolt}, \spell{thunderdash}
                \item Rank 4: \spell{greater electric jolt}, \spell{greater electroshock}, \spell{shock and awe}
                \item Rank 5: \spell{chain lightning}, \spell{electrocute}, \spell{electromagnetic bolt}
                \item Rank 6: \spell{greater call lightning}, \spell{greater thunderdash}, \spell{supreme lightning storm}
                \item Rank 7: \spell{greater lightning breath}, \spell{greater shock and awe}, \spell{supreme electric jolt}
            \end{itemize}
            \parhead{Suggested Feats} Leadership, Sphere Focus: Electromancy, Skill Specialization: Creature Handling, Toughness
            \parhead{Combat Tactics} You lead your faithful natural servant in battle.
            It distracts your enemies while you blast them with lightning from afar.
            You can also use your leadership skills to inspire and command your allies in battle.
            Once you get a \textit{shrinking belt} or some other way to shrink yourself, you can ride your \textit{natural servant} into battle, which compensates for your short gnomish legs.
            If you are both lucky and persuasive, you be able to use your \textit{speak with animals} ability to convince an animal to aid you on your journey, at least for a short time, in addition to your \textit{natural servant}.

    \subsection{Fighter}

        \subsubsection{Combat Discipline}
            \parhead{Species} Dwarf.
            \parhead{Attributes} 3 Str, 0 Dex, 4 Con, 0 Int, 0 Per, 1 Wil.
            \parhead{Class} Fighter.
            \parhead{Archetypes} Combat Discipline first, Martial Mastery second, Equipment Training third.
            \parhead{Insight Points} 2 points for maneuvers.
            % 7 total skill points, counting the 1 from auto-training
            \parhead{Skills} Climb (T), Endurance (M), Jump (T), Perception (M), Swim (T)
            \parhead{Weapon Group} Axes, blades
            \parhead{Languages} Common, Dwarven, Orc
            \parhead{Equipment} Battleaxe, standard shield, scale mail. As you gain levels, use the best heavy armor you can afford.
            You can switch between a shepherd's axe for hard to hit enemies, a battleaxe for multi-enemy fights or fights where you need the extra damage from holding it in two hands, and throwing axes when you need a ranged weapon.
            \parhead{Legacy Item} Shield.
                At level 3, choose \mitem{protective shield}.
                At level 9, choose \mitem{greater protective shield} and \mitem{shield of arrow catching}.
                At level 15, choose \mitem{supreme protective shield}, \mitem{hardblock shield}, and \mitem{shield of arrow catching}.
            \parhead{Combat Styles} Flurry of Blows, Mobile Assaut, Rip and Tear
            \parhead{Suggested Maneuvers} 
            \begin{itemize}
                % each level: damage, HP debuff, maybe non-HP debuff, maybe AOE
                \item Rank 1: \maneuver{hamstring}, \maneuver{rend the hide}, \maneuver{quickdraw}, \maneuver{wanderer's strike}
                \item Rank 2: \maneuver{followup strike}, \maneuver{reckless charge}, \maneuver{strip the armor}, \maneuver{sweeping strike}
                \item Rank 3: \maneuver{desperate flurry}, \maneuver{spring attack}, \maneuver{strip the flesh}
                \item Rank 4: \maneuver{brow gash}, \maneuver{greater wanderer's strike}, \maneuver{spinning slash}
                \item Rank 5: \maneuver{bloodletting strike}, \maneuver{greater retreating strike}
                \item Rank 6: \maneuver{greater strip the flesh}, \maneuver{greater reaping harvest}
                \item Rank 7: \maneuver{greater brow gash}, \maneuver{greater desperate flurry}
            \end{itemize}
            \parhead{Suggested Feats} Shieldbearer, Toughness, Iron Will, Regenerator
            \parhead{Combat Tactics} You are extremely difficult to kill, and your ability to ignore and remove conditions makes it hard for your foes to whittle you down over time.
            You can charge confidently into the middle of battle, cutting down enemy ranged attackers regardless of their surrounding allies.
            Alternately, you can hold the line to protect your own allies.
