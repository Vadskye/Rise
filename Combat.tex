\chapter{Combat}\label{Combat}

\section{How Combat Works}
Combat takes place in a series of ``rounds'', which represents about six seconds of action. Each round, every character gets a turn to act. The order in which creatures take turns is determined by their initiative (see \pcref{Initiative}). 

When it's your turn, you can take one standard action, one move action, and one swift action. If you want, you can take a full-round action instead of your standard and move actions. You can always ``downgrade'' an action to a lesser action: turning a standard action into a move action, or a move action into a swift action.

\section{Attacks}

\subsection{Standard Attack}
As a standard action, you can try to strike a foe with a weapon you are wielding. To do so, make an attack roll with a weapon you are wielding, adding your attack modifier to the roll. If your result equals or exceeds your foe's Armor Class, your attack hits, and your foe takes damage.

You can also make an attack as an attack of opportunity. See \pcref{Attacks of Opportunity}. 

\subsubsection{Attack Modifier}
Your attack modifier is equal to the following:

\begin{figure}[h]
\centering Base attack bonus \add attack attribute \add size modifier \add competence bonuses \add enhancement bonus
\end{figure}

With medium or heavy weapons, your attack attribute is your Strength. With projectile weapons, your attack attribute is your Dexterity. With light weapons, you can choose between your Strength and Dexterity. 

Your size modifier is described in \tref{Size in Combat}.

\subsubsection{Damage}
If your attack succeeds, you deal damage equal to your weapon's damage die \add half your Strength.

\parhead{Dealing Nonlethal Damage} If you take a \minus4 penalty to your attack roll, you can deal nonlethal damage instead of lethal damage when you hit. See \pcref{Nonlethal Damage}.

\subsection{Combat Maneuver}\label{Combat Maneuver}
As a standard action, you can try to hinder your foe by performing a combat maneuver, such as by disarming or tripping him. To do so, make an attack roll, adding your maneuver modifier to the roll instead of your attack modifier. If your result equals or exceeds your foe's Maneuver Class, your attack hits, and your foe suffers the effects of the maneuver.

You can also perform some maneuvers as an attack of opportunity. See \pcref{Attacks of Opportunity}.

\subsubsection{Maneuver Modifier}
Your maneuver modifier is equal to the following:

\begin{figure}[h]
\centering Base attack bonus \add attack attribute \add special size modifier \add competence bonuses \add enhancement bonus
\end{figure}

Your attack attribute depends on the maneuver you are using, as described in \pcref{Combat Maneuvers}.

Your special size modifier is described in \tref{Size in Combat}.

\subsection{Reach}\label{Reach}
Normally, you can attack anyone within five feet of you. The range at which you can attack is called your ``reach'', and the area that you can attack into is called your ``threatened area''. Reach for larger and smaller creatures is determined by size, as shown on \trefnp{Size in Combat}.


\section{Defenses}

\subsection{Armor Class}
Your Armor Class (AC) represents how hard it is for opponents to land a solid, damaging blow on you. It's the attack roll result that an opponent needs to achieve to hit you. Unlike most combat statistics, your AC is made up of the sum of other modifiers, each of which can have competence and enhancement bonuses applied separately. You can also apply competence and enhancement bonuses directly. Your AC is equal to the following:

\begin{figure}[h]
    \centering 10 \add half base attack bonus \add Dexterity \add armor modifier \add shield modifier \add natural armor modifier \add dodge modifier \add size modifier \add competence bonuses \add enhancement bonus
\end{figure}

\parhead{Base Attack Bonus} Your experience and aptitude in combat affects your ability to defend yourself; experienced warriors know how to recognize and avoid or parry blows that would easily fell novices. As a result, you add half your base attack bonus to your dodge modifier to armor class. This is an inherent bonus, and stacks with all other bonuses to your dodge modifier.
\parhead{Natural Armor Modifier} Natural armor, such as from having unusually tough skin or thick hide, improves your AC. Armor and natural armor do not fully stack; add the higher modifier plus half the lower modifier to your AC. For example, if a warhorse (\plus4 natural armor modifier) wears chainmail barding (\plus6 armor modifier), it gets a total of a \plus8 bonus to AC: the chainmail provides \plus6, and its natural armor is halved to give a \plus2 bonus.
\parhead{Dodge Modifier} Your dodge modifier represents your ability to actively avoid blows. Any situation that denies you your Dexterity bonus also denies you your dodge modifier.
\parhead{Size Modifier} Your size modifier is described on \tref{Size in Combat}.

\subsubsection{Flat-Footed Armor Class}
Sometimes you can't use your agility to avoid an attack. Your flat-footed armor class is equal to your armor class, ignoring your Dexterity, dodge modifier, and shield modifier. If your Dexterity is negative, you always apply the penalty, even while flat-footed.

\subsubsection{Touch Armor Class}
Sometimes your armor doesn't help you avoid an attack. Your touch armor class is equal to your armor class, ignoring your armor modifier and natural armor modifier. Most touch attacks come from spells.

\subsection{Maneuver Class}
Your Maneuver Class (MC) represents how good you are at defending against combat maneuvers. It's the attack roll result that an opponent needs to achieve to affect you with a maneuver. Like your Armor Class, your MC is made up of the sum of other modifiers, each of which can have competence and enhancement bonuses applied separately. You can also apply competence and enhancement bonuses directly. Your MC is equal to the following:

\begin{figure}[h]
    \centering 10 \add base attack bonus \add Strength \add Dexterity \add shield modifier \add dodge modifier \add special size modifier \add competence bonuses \add enhancement bonus \add AC bonuses and penalties
\end{figure}

Any effect which gives a bonus or a penalty to your AC directly is also applied to your MC. For example, while prone, you take a \minus4 penalty to your MC, just like you do to your AC. 

Your special size modifier is described in \tref{Size in Combat}.

\subsection{Hit Points}
Your hit points is a defense that represents how much more punishment you can take. When you run out hit points, you can't act anymore, and you might die, as described in \pcref{Injury, Death, and Healing}.

\subsection{Saving Throws}
Generally, when you are subject to an unusual or magical attack, you get a saving throw to avoid or reduce the effect. You make a saving throw by rolling a d20 and adding your saving throw modifier. If it is greater than the save DC of the attack, you resist the attack (though it may still have partial effects on you). Your saving throw modifier is calculated as follows:

\begin{figure}[h]
\centering Base save bonus \add primary attribute \add 1/2 secondary attribute \add competence bonuses \add enhancement bonus
\end{figure}

\subsubsection{Saving Throw Types}
The three different kinds of saving throws are Fortitude, Reflex, and Will.
\parhead{Fortitude} These saves measure your ability to stand up to physical punishment or attacks against your vitality and health. Apply your Constitution and half your Strength to your Fortitude saving throws.
\parhead{Reflex} These saves test your ability to dodge area attacks. Apply your Dexterity and half your Wisdom to your Reflex saving throws.
\parhead{Will} These saves reflect your resistance to mental influence as well as many magical effects. Apply your Charisma and half your Intelligence to your Will saving throws.

\section{Attacks of Opportunity}\label{Attacks of Opportunity}
Sometimes a combatant in a melee lets her guard down. In this case, combatants near her can take advantage of her lapse in defense to attack her for free. These free attacks are called attacks of opportunity.

\subsection{Provoking Attacks of Opportunity}
You can provoke attacks of opportunity in two ways.

\parhead{Leaving an Opening} To defend yourself, you must be actively wielding a weapon or shield you are proficient with. If at any point during your turn, you can't use a weapon or shield to defend yourself, you provoke an attack of opportunity. This can happen for several reasons. Some examples are given below.

\begin{itemize*}
    \item If you are unarmed, you provoke.
    \item If you can't use your weapon or shield (such as if you are trying to wield a heavy weapon in one hand), you provoke.
    \item If you are busy doing something else (such as casting a spell), you provoke.
    \item If you can do nothing at all (such as if you are helpless), you provoke.
\end{itemize*}

\parhead{Leaving the Battle} You provoke an attack of opportunity if you move farther away from an opponent who threatens you. This can be mitigated with the withdraw action.

\subsubsection{Forced Movement} You never provoke attacks of opportunity for movement you didn't take intentionally, such as from a bull rush (see \pcref{Bull Rush}).

\subsection{Taking Attacks of Opportunity}
If a creature within your threatened area provokes an attack of opportunity, you can immediately make a single melee attack against that creature. The attack of opportunity ``interrupts'' anything else the creature was been doing. The creature can continue its turn after your attack of opportunity is resolved.

You can make a number of attacks of opportunity each round equal to 1 \add half your Dexterity, but never more than one per opportunity. You don't have to make an attack of opportunity if you don't want to.

\section{Movement and Positioning}

\subsection{Taking up Space}
A typical human takes up a 5-ft. by 5-ft. space in combat. For convenience, this is often called a ``square''. Differently sized creatures can take up more or less space, as indicated on \tref{Size in Combat}. Normally, other creatures can't be in any squares you occupy.

Sometimes, movement and distance are represented in squares. A 30-ft. movement is the same thing as moving six squares.

\subsection{Ways to Move}

\parhead{Move} As a move action, you can move up to your speed.
\parhead{Stand Up} As a move action, you can stand up from being prone. For most creatures, this requires using a hand to help get up.
\parhead{Withdraw} As a standard action, you can move up to your speed. Before you do so, you can designate one creature who threatens you. This movement does not provoke attacks of opportunity from that creature.
\parhead{Charge}\label{Charge} As a full-round action, you can move up to twice your speed and make a single attack with a \plus2 circumstance bonus to hit at the end of your charge. While charging, and until the start of your next turn, you take a \minus2 penalty to armor class.
\par You must move at least 30 feet to gain the benefit of a charge, and all movement must be in a single straight line. If there are any obstacles in your path which hinder your movement, you cannot charge. A charge that fails or becomes invalid partway through becomes a hustle action.
\parhead{Run} As a full-round action, you can run as fast as you can. This allows you to move up to quadruple your speed. You move at only triple speed if you are wearing medium or heavy armor. While running, and until the start of your next turn, you are flat-footed against all attacks.
\parhead{Struggle} As a full-round action, you can move five feet, regardless of movement penalties. You can use this to move even if your speed is decreased below five feet by penalties.

\subsection{Measuring Movement}
For simplicity, all movement is measured in five-foot increments. While it is possible to be more precise than that, it's generally not worth the complexity.

\parhead{Diagonals} When measuring distance, the first diagonal counts as five feet of movement, and the second counds as ten feet of movement. The third costs five feet, the fourth costs ten feet, and so on.

You can't move diagonally past a corner. You can move diagonally past a creature, even an opponent. You can also move diagonally past other impassable obstacles, such as pits.
\parhead{Closest Creature} When it's important to determine the closest square or creature to a location, if two squares or creatures are equally close, pick one randomly.

\subsection{Movement Impediments}

\subsubsection{Difficult Terrain}
Some terrain is hard to move through, like thick bushes or a swamp. If a square is difficult terrain, it doubles the movement cost required to move through the square. That generally means it takes ten feet of movement, or fifteen feet if you are moving diagonally. If a square is considered difficult terrain for multiple reasons, the cost increases stack. For example, a square in a swamp that also has thick bushes blocking your passage would take twenty feet of movement, or thirty feet to move diagonally!

\subsubsection{Obstacles}
An obstacle is anything that gets in your way. Enemies and large solid objects like walls are blocking obstacles: they completely block your movement. If you can get past an obstacle, like a low wall, that square is treated as difficult terrain. Some obstacles require a skill check to bypass.

Enemies three size categories larger than you, and all allies, are not considered obstacles, and do not hinder your movement.

\subsubsection{Squeezing}

In some cases, you may have to squeeze into or through an area that isn't as wide as the space you take up. You can squeeze through or into a space that is at least half as wide as your normal space. While squeezing, you move half as fast, and you take a \minus4 penalty on attack rolls and AC. You can squeeze into tighter spaces with the Escape Artist skill.

Creatures that take up multiple squares take up half their normal number of squares while squeezing. For example, a Large creature who normally takes up four spaces takes up two spaces while squeezing.

\parhead{Accidentally Squeezing} Sometimes a character ends its movement while moving through a space where it's not normally allowed to stop. When that happens, the character is squeezing in the space until it can move. If squeezing is impossible, it immediately moves to the closest available space where it can be. Try not to do this.

\section{Initiative}\label{Initiative}

At the start of a battle, each creature that is aware of the combat makes an initiative check. Your initiative check is calculated as follows:

\begin{figure}[h]
    \centering Dexterity \add half Wisdom \add competence bonuses \add enhancement bonus
\end{figure}

Creatures act in order of their initiative, highest to lowest. After the creature with the lowest initiative acts, a new round begins and the creature with the highest initiative acts again. You usually keep your initiative for the whole encounter, even if you can't act. However, some special actions can change your initiative count.

\subsubsection{Surprise Attacks}\label{Surprise Attacks}
Sometimes, some creatures are not aware of the combat when it starts. This most commonly happens with ambushes. Any creature that is not aware of the combat doesn't get to make an initiative check during the first round of combat. Until it takes its first action, it is flat-footed and can't take attacks of opportunity.

Sometimes, everyone is surprised, such as if a guard walks around a corner to unexpectedly find a group of muggers. In that case, initiative is rolled normally, but all creatures are flat-footed and can't take attacks of opportunity until they take their first action.

\section{Injury, Death, and Healing}\label{Injury, Death, and Healing}
Your hit points measure how hard you are to kill. No matter how many hit points you lose, your character isn't hindered in any way until your hit points drop to 0.

\subsection{Losing Hit Points}
When you take lethal damage, you subtract that damage from your hit points.
\parhead{What Hit Points Represent} Hit points represent a combination of durability, luck, divine providence, and sheer determination, depending on the nature of your character. When you take 10 damage from an orc with a greataxe, the axe did not literally carve into your skin without affecting your ability to fight. Instead, you avoided the worst of the blow, but it bruised you through your armor, the effort to dodge the blow fatigued your character, or it barely nicked you through sheer luck -- and everyone's luck runs out eventually.

\subsection{Stages of Injury}

\subsubsection{Healthy} 
When you are above half hit points, you suffer no significant effects from losing hit points. If you take damage, you can become bloodied.

\subsubsection{Bloodied}
When you drop to half your hit points or below, you are bloodied. This makes you more vulnerable to certain spells and effects, but you suffer no direct penalties. If you take additional damage, you can become staggered. Some special attacks can cause you to immediately begin dying.

\subsubsection{Staggered}
When you take damage that would reduce your hit points to 0, you become staggered.  While staggered, you may take a single move action or standard action each round, but not both. You cannot take full-round actions, but you may take swift actions. In addition, you are vulnerable, causing you to take a \minus2 penalty on attack rolls, saving throws, checks, DCs, and AC.

If you take additional damage while at 0 hit points, you begin dying.

\subsubsection{Dying}
When you take damage while you have no hit points remaining, that damage represents serious physical injury to your body. This is called critical damage. When you take critical damage, you immediately fall unconscious and begin dying. While you have critical damage, magical healing which would normally restore hit points cannot restore your hit points, though it can prevent you from dying.

While dying, you must make a DC 15 Fortitude save every round. The DC is equal to 15 \add the critical damage you have taken. If you fail three times, you die. If you succeed three times, or receive healing that would normally hit points, you become stable. Another character can give first aid to help you stabilize (see \pcref{Heal}).

If you take additional damage, you can die. 

\subsubsection{Dead}
If you critical damage exceeds 10 \add your Constitution, you die. You can also die from taking ability damage or suffering ability damage or drain that reduces your Constitution to 0.

\subsubsection{Stable}
If you have taken critical damage but managed to stave off death, you become stable. You remain unconscious until your hit points exceed your critical damage.

Unless your hit points go above 0, you remain unconscious. As long as you have critical damage, magical healing that restores hit points has no effect on your hit points.

\subsubsection{Overkill Damage}
Normally, when you take damage that would reduce your hit points to 0, any excess damage from the attack is wasted. However, some attacks deal such massive damage that you begin dying immediately, rather than just becoming staggered. If the damage dealt by an attack exceeds your maximum hit points (not current hit points), any damage past what would reduce your hit points to 0 is dealt as critical damage rather than being wasted.

\subsection{Healing}
After taking damage, you can recover hit points through natural healing or through magical healing. In any case, you can't regain hit points past your full normal hit point total.

\parhead{Natural Healing} With 8 hours of rest, you recover half your hit points. Any significant interruption (such as combat or the like) during your rest prevents you from healing. If you rest for an entire day (16 hours), you recover all your hit points.

\parhead{Magical Healing} Various abilities and spells can restore hit points. However, only certain spells can heal critical damage, as specified in the spell description. Unless a spell says it can cure critical damage, it cannot -- though it can still stabilize dying characters. Magical healing has no effect on the hit points of creatures with critical damage.

\parhead{Healing Ability Damage} Ability damage is temporary, just as hit point damage is. Ability damage returns at the rate of 1 point per 8 hours of rest for each affected attribute score.

\parhead{Healing Critical Damage} Critical damage takes much longer to heal than hit point damage. Resting for 1 week restores an amount of critical damage equal to 1 \add half the character's Constitution. A character can have both hit points and critical damage. As long as a character has critical damage, he is staggered, even if he is at full hit points.

\subsection{Nonlethal Damage}\label{Nonlethal Damage}
Some attacks and environmental effects deal nonlethal damage. Nonlethal damage is not subtracted from your hit points. Instead, it is tracked separately. If your nonlethal damage exceeds your hit points, you become staggered, just as if you were at 0 hit points. If you take additional damage while staggered, you fall unconscious. However, you do not begin dying unless your hit points are actually below 0. 

\subsubsection{Healing Nonlethal Damage}
You heal half your hit points in nonlethal damage with 1 hour of rest. When a spell or a magical ability cures hit point damage, it also removes an equal amount of nonlethal damage.

\subsection{Temporary Hit Points}\label{Temporary Hit Points}
Certain effects give a character temporary hit points which act as a protection against damage. Whenever a character takes damage, if he has temporary hit points, the damage is applied to his temporary hit points first. Any excess damage is then applied to his hit points as normal. Temporary hit points are not ``real'' hit points, and cannot be healed. If a character has temporary hit points from multiple effects, only the highest value is used.

\section{Circumstances, Bonuses, and Penalties}

\begin{dtable}
    \lcaption{Attack Roll Bonuses and Penalties}
    \begin{tabularx}{\columnwidth}{l X}
        \thead{Attacker's Condition} & \thead{Effect} \\
        Entangled & \minus2 \\
        Invisible & \x\fn{1} \\
        Prone & \minus4\fn{2} \\
        Squeezing through a space & \minus4 \\
        Vulnerable & \minus2 \\
    \end{tabularx}
    1 The defender is flat-footed. \\
    2 Most ranged weapons can't be used while the attacker is prone, but you can use a crossbow or shuriken while prone at no penalty.
\end{dtable}

\begin{dtable}
    \lcaption{Armor Class Bonuses and Penalties}
    \begin{tabularx}{\columnwidth}{>{\lcol}X >{\ccol}X}
        \thead{Defender's Condition} & \thead{Effect} \\
        Behind active cover & 20\% miss \\
        Behind passive cover & \plus4 \\
        Blinded & \x\fn{1} \\
        Concealed & \plus4 \\
        Cowering & \minus2\fn{1} \\
        Crouching or kneeling & \minus2\fn{2} \\
        Entangled & \minus2 \\
        Grappling (but attacker is not) & \plus0\fn{1} \\
        Helpless (such as paralyzed, sleeping, or bound) & \plus0\fn{3} \\
        Invisible & see Invisibility \\
        Overwhelmed & special\fn{4} \\
        Pinned & \minus4\fn{3} \\
        Prone & \minus4\fn{2} \\
        Squeezing through a space & \minus4 \\
        Stunned & \minus2\fn{1} \\
        Vulnerable & \minus2 \\
    \end{tabularx}
    1 The defender is flat-footed. \\
    2 Treat as a bonus against ranged attacks, instead of a penalty \\
    2 The defender is flat-footed, and treat the defender's Dexterity as \minus10. \\
    3 The creature suffers a penalty equal to the number of creatures threatening it.
\end{dtable}

\subsection{Cover}

Cover represents any obstacle that physically prevents you from striking your target, such as a tree or intervening creature. A creature with cover is more difficult to attack.

\parhead{Determining Cover} When making a melee attack against an adjacent target, your target has cover if any line from your square to the target's square goes through a wall (including a low wall) or other similar solid obstacle. If you occupy multiple squares, choose one square you occupy for this purpose.

When making an attack against a target that is not adjacent to you, choose a corner of any square you occupy. In addition, choose a square the target occupies. If any line from this corner to any corner of the target square passes through a square or border that blocks line of effect or provides cover, or through a square occupied by a creature, the target has cover.

There are two types of cover: active cover and passive cover.

\subsubsection{Active Cover}

If the obstacle is active and mobile, such as a creature or tree branches blowing in the wind, the defender has active cover. Any attacks against a creature with active cover relative to you have a 20\% miss chance. After rolling the attack, the attacker must make a miss chance percentile roll to see if the attack misses due to active cover. If an attack misses due to active cover, the attack is made against the intervening obstacle instead. If the attack is successful, the obstacle takes any damage from the attack normally. 

\subsubsection{Passive Cover}

If the obstacle is stationary, such as a tree trunk or wall, the defender has passive cover. A creature with passive cover relative to you has a \plus4 circumstance bonus to armor class.

\parhead{Reflex Saves} A creature with passive cover gains a \plus2 bonus on Reflex saves against attacks that originate or burst out from a point on the other side of the cover from it. Note that spread effects can extend around corners and thus negate this cover bonus.
\parhead{Low Obstacles} A low obstacle (such as a wall no higher than half your height) provides passive cover, but only to creatures within 30 feet (6 squares) of it. The attacker can ignore the cover if he's closer to the obstacle than his target. If two creatures are equally distant from the wall, it grants cover to both of them.
\parhead{Attacks of Opportunity} You can't execute an attack of opportunity against an opponent with passive cover relative to you.
\parhead{Hide Checks} You can use passive cover to make a Hide check, but not active cover. Without cover, you usually need concealment (see below) to make a Hide check.
\parhead{Total Cover} If you don't have line of effect to your target, he is considered to have total cover from you. You can't make an attack against a target that has total cover.

\subsubsection{Improved Cover}

A creature can benefit from both passive and active cover. However, cover of the same type generally doesn't stack; a creature behind two trees is not substantially more protected than a creature behind a single tree. In some cases, cover may stack. In that case, each additional obstacle increases the miss chance by 10\% or grants an additional \plus2 circumstance bonus to AC, as appropritae.

Exceptionally well covered opponents, such as a creature behind an arrow slit in a castle, may receive additional benefits. For example, it might gain improved evasion, and there may be limitations on what kind of attacks are possible.

\subsection{Concealment}
Concealment represents anything which makes it more difficult to see your target, such as dim lighting. A creature with concealment from you gains a \plus4 circumstance bonus to Armor Class. Concealment bonuses do not apply if you can't see your opponent (such as if you close your eyes).

\parhead{Determining Concealment} When making a melee attack against an adjacent target, your target has concealment if his space is entirely within an effect that grants concealment.

Deterining concealment for making an attack against a target that is not adjacent to you works exactly like determining cover for ranged attacks.

In addition, some magical effects provide concealment against all attacks, regardless of whether any intervening concealment exists.

\parhead{Concealment and Hide Checks} You can use concealment to make a Hide check. Without concealment, you usually need cover to make a Hide check.

\subsection{Invisibility}
If it is impossible to see your target, you can't attack him normally. However, you can attack a square that you think he occupies. An attack into a square occupied by an invisible enemy has a 50\% miss chance. If an adjacent invisible creature strikes you, you can automatically identify the square he occupied when he struck you.

You can't execute an attack of opportunity against an invisible opponent, even if you know what square or squares the opponent occupies.

\subsection{Overwhelm}
When a creature is being attacked by multiple foes at once, it is less able to defend itself. A creature is considered overwhelmed if it is being threatened by more than one creature. Multiple creatures occupying the same square count as a single creature when determining overwhelm penalties. If a creature is overwhelmed, it takes a penalty to armor class equal to the number of creatures threatening it.

\subsection{Helpless Defenders}
A helpless opponent is someone who is bound, sleeping, paralyzed, unconscious, or otherwise at your mercy.

\parhead{Regular Attack} A helpless character takes a \minus4 penalty to AC against melee attacks, but no penalty to AC against ranged attacks. A helpless defender is flat-footed. In fact, his Dexterity score is treated as if it were \minus10, giving him a \minus10 penalty to AC.

\parhead{Coup de Grace} As a full-round action, you can use a melee weapon to deliver a coup de grace to a helpless opponent. You can also use a bow or crossbow, provided you are adjacent to the target. You automatically hit and score a critical hit. If the defender survives the damage, he must make a Fortitude save (DC 10 \add damage dealt) or die. A rogue also gets her extra sneak attack damage against a helpless opponent when delivering a coup de grace.

Delivering a coup de grace provokes attacks of opportunity from
threatening opponents because it involves focused concentration
and methodical action on the part of the attacker. If any attacks of opportunity hit, you must make a Concentration check (DC 10 \add total damage dealt from all attacks of opportunity) or else the coup de grace fails.

You can't deliver a coup de grace against a creature that is immune to critical hits. You can deliver a coup de grace against a creature with total concealment, but doing this requires two consecutive full-round actions (one to ``find" the creature once you've determined what square it's in, and one to deliver the coup de grace).

\subsection{Range Increments}
When using a ranged weapon, you take a \minus2 penalty per range increment between you and your target. For example, when using a longbow with a range increment of 100 feet against a target 170 feet away, you take a \minus2 penalty to attack rolls.

\section{Combat Maneuvers}\label{Combat Maneuvers}
A maneuver is a special attack that imposes some effect on an opponent using your skill in combat. Possible maneuvers are described below.

\begin{dtable*}
    \lcaption{Combat Maneuvers}
    \begin{tabularx}{\textwidth}{l l l l X}
        \thead{Combat Maneuver}  & \thead{Action} & \thead{Key Attribute} & \thead{Weapon?} & \thead{Brief Description} \\
        Bull rush & Standard & Strength & No &  Push an opponent back 5 feet or more \\
        Dirty Trick & Attack\fn{1} & Strength or Dexterity & No & Impose penalty by using environment or fighting dirty \\
        Disarm & Attack\fn{1} & Strength or Dexterity & Any & Strike an object away from your foe \\
        Feint & Attack\fn{1} & Strength or Dexterity & Any & Trick your foe into being flat-footed \\
        Grapple & Standard & Strength & No & Wrestle with an opponent \\
        Overrun & Full-round & Strength & No & Plow past or over an opponent as you move \\
        Trip & Attack\fn{1} & Strength or Dexterity & Some & Trip an opponent \\
    \end{tabularx}
    1 Can be used in place of an attack, even an attack of opportunity. \\
\end{dtable*}

\subsection{Performing a Maneuver}
To perform a maneuver, make an attack roll and add your maneuver modifier, as described on \pref{Combat Maneuver}. If your result equals or exceeds your target's Maneuver Class, the maneuver succeeds.

Unless otherwise noted, you can only perform a maneuver on a creature no more than one size category larger than you, and you must have a free hand.

\subsection{Bull Rush}\label{Bull Rush}
A bull rush is an attempt to push your opponent away from you. You can bull rush as a standard action or as part of a charge, in place of the melee attack. Bull rushing is Strength-based. You can bull rush a creature of any size.

If your bull rush succeeds, your target is pushed back 5 feet. For every 5 points by which you succeed, you can push the target back an additional 5 feet. You must move with the target to push it back. If you don't have the movement remaining to move with the target, you must stop pushing it.

If a creature you are pushing back encounters a solid object, you can't push it any farther. If it encounters another creature, you can try to push them both. If you do, apply your original attack roll to the second creature's Maneuver Class, taking a \minus5 penalty. If you are successful, you can continue to push both of the creatures a distance equal to the lowest result.

For example, if a fighter bull rushes a goblin for a total of 15 feet, but there is another goblin 5 feet behind the first, he can push both goblins 15 feet away from his starting location. If, however, the second goblin was instead a mighty dragon, he could only push the first goblin 5 feet before it would stop.

\subsection{Dirty Trick}\label{Dirty Trick}
A dirty trick is any sort of situational attack that imposes a penalty on a foe for a short period of time. Examples include knocking an opponent's head to bewilder him, pulling down an enemy's pants to entangle him, or hitting a foe in a sensitive spot to make him sickened. You can perform a dirty trick in place of an attack. Performing a dirty trick can be Strength-based or Dexterity-based, as you choose.

If your dirty trick succeeds, your target becomes vulnerable for 1 round. \vulnerableexplanation For every 5 points by which you succeed, your target is vulnerable for an additional round. The penalty can be removed if the target spends a standard action.

\subsection{Disarm}\label{Disarm}
A disarm is an attempt to take an object from a foe. You can disarm in place of an attack. If you are wielding a weapon, you can disarm with the weapon, using your attack modifier with the weapon in place of your maneuver modifier. Otherwise, disarming can be Strength-based or Dexterity-based, as you choose.

When you disarm an opponent, you choose one object it is holding or wearing. If the item is well secured, such as a ring or suit of armor, the disarm attempt automatically fails. If your disarm succeeds, the object falls to the ground in your target's square. If you are not using a weapon to disarm, you can grab the item in your hands instead.

\subsection{Feint}\label{Feint}
A feint is an attempt to leave your foe off-balance. You can feint in place of an attack. If you are wielding a weapon, you can disarm with the weapon, using your attack modifier with the weapon in place of your maneuver modifier. Otherwise, a feinting can be Strength-based or Dexterity-based, as you choose. You can feint a creature of any size.

If your feint succeeds, your foe is flat-footed against the next melee attack you make against it. This attack must be made before the creature's next turn.

\subsection{Grapple}\label{Grapple}
A grapple is an attempt to physically grab and restrain your foe. You can grapple as a standard action. Grappling is Strength-based. 

If your grapple succeeds, you and your target become ``grappled'', which limits your options. If your target is not adjacent to you, it moves into the closest open space adjacent to you. If no space is available, your grapple fails.

\subsubsection{Being Grappled}
While grappled, you suffer certain penalties and restrictions, as described below.
\begin{itemize*}
    \item You must use one of your hands (or equivalent limbs) to grapple, preventing you from taking any actions which would require having two free hands. For example, you cannot attack with a two-handed weapon while grappling. You cannot use a hand holding a shield (except a buckler) to grapple.
    \item You are flat-footed against all opponents except the one you are grappling.
    \item You have active cover from the other participant in the grapple against attacks from creatures not adjacent to you. Any attacks that miss because of this miss chance are made against that creature. This does not apply if you are larger than the other participant.
    \item You do not threaten any opponents except for the creature you are grappling with.
    \item You take a \minus4 penalty to attack rolls made with weapons that are not small, since they are too large and cumbersome to be used effectively in a grapple.
    \item You cannot cast spells with somatic components.
    \item Casting a spell without somatic components requires a DC 20 \add double spell level Concentration check.
    \item You cannot move normally (but see Move the Grapple, below).
\end{itemize*}

Other than the restrictions listed above, you can act normally. You also gain the ability to use three special actions, each of which can be used as a standard action. Against these attacks, your foe uses 10 \add his maneuver modifier in place of his Maneuver Class.

\parhead{Escape the Grapple} You can make a grapple attack or Escape Artist check to attempt to escape the grapple. You apply your result against every creature grappling you. If you succeed against a creature, you are no longer grappling that creature.
\par You cannot make an Escape Artist check as an attack action, even if you have Improved Grapple.
\parhead{Move the Grapple} You can make a grapple attack to attempt to move the grapple. You apply your result against every creature grappling you. If you succeed against every creature, you can move both yourself and all of the other creatures up to half your speed. At the end of your movement, you can choose which spaces creatures grappling you are in, as long as they stay adjacent to you. This can be used to place creatures in dangerous positions, such as over a pit.
\par You can only move the grapple once per round, regardless of how many grapple attacks you can make.
\parhead{Pin} You can make a grapple attack to attempt to pin a single opponent. If you succeed, that opponent becomes pinned, while you remain grappled.

\subsubsection{Being Pinned}
A pinned creature is helpless, but it cannot be the target of a coup de grace. The only action it can take that requires movement is to free itself through a grapple attack or Escape Artist check (using its normal Dexterity, not treating its Dexterity as 0). A pinned creature can take mental actions, but cannot cast any spells that require a somatic or material component or focus. At the opponent's option, a pinned creature may be unable to speak, in which case it is incapable of casting spells which require a verbal component. A pinned character who attempts to cast a spell or use a spell-like ability must make a concentration check (DC 10 \add grappler's maneuver modifier \add double spell level) or lose the spell. Being pinned is a more severe version of being grappled, and their effects do not stack.

\subsubsection{Multiple Grapplers}
Multiple creatures can attempt to grapple one target. Each of them acts independently.

\subsubsection{Grappling a Mounted Opponent}
You may make a grapple attempt against a mounted opponent. The defender may make a Ride check and use that result instead of his Maneuver Class if it is higher. If you succeed, you pull the rider from his mount, and he lands prone in a square adjacent to both you and his horse. However, neither of you are considered grappled, and you must succeed on a second grapple attack to grapple him.

\subsubsection{Binding an Opponent}
Once you have pinned an opponent, you may attempt to bind them with ropes or other forms of restraint. You must have the restraint on hand. To do so, make a grapple attack against the pinned opponent. The opponent can resist with a grapple attack. If you succeed, your opponent is bound and helpless. To escape the bindings, the opponent must exceed your check result with an Escape Artist check.

Once your opponent is bound and helpless, you can take 20 on your grapple attack to bind him or her more securely. You can also use the Devices skill to bind an opponent who is not struggling.

\subsection{Overrun}\label{Overrun}
An overrun is an attempt to move directly through anyone in your way. You can overrun as a full-round action. Overrunning is Strength-based.

When you overrun, you move up to your speed. Any creature in your way can choose to avoid you. If it does not, you must make an overrun attack against it. Success means you move through its space, treating the area as difficult terrain. Success by 5 or more means the creature is also knocked prone, and the area is not considered difficult terrain for you.

Unusually stable creatures, such as creatures with multiple legs, gain a \plus4 inherent bonus to Maneuver Class against overrun attempts. Some creatures, such as oozes, cannot be overrun.

\subsection{Trip}\label{Trip}
A trip is an attempt to knock your foe off his feet. You can trip in place of an attack. If you are wielding a weapon that can be used to trip, you can trip with the weapon, using your attack modifier with the weapon in place of your maneuver modifier. Otherwise, tripping can be Strength-based or Dexterity-based, as you choose.

If your trip succeeds, the target is knocked prone.

Unusually stable creatures, such as creatures with multiple legs, gain a \plus4 inherent bonus to Maneuver Class against trip attempts. Some creatures, such as oozes, cannot be tripped.

\section{Special Actions}

\subsection{Cast a Spell}\label{Cast a Spell}
As a standard action, you can cast a spell. See \pcref{Casting Spells} for more information. 

\subsection{Partial Full-Round Action}\label{Partial Full-Round Action}
If you are restricted to only taking a move or standard action, but not both, you can spend a standard action to perform a partial version of that action. For most actions, you spend the first round starting the action, and use a second standard action to complete it. Some full-round actions have specific partial versions described below which you take instead. You can only take these partial actions when you cannot take full-round actions. 

\subsubsection{Partial Charge}
As a standard action, you can move up to your speed and make a single attack against a foe. You must move in a straight line, and your movement must not be impeded in any way. In addition, you take a \minus2 penalty to AC until the start of your next turn. An interrupted partial charge becomes a move action.

\subsubsection{Partial Run}
As a standard action, you can move up to double your speed. You are flat-footed until the start of your next turn. You cannot take this action if your running speed would be reduced by medium or heavy armor. 

\subsubsection{Partial Withdraw}
As a standard action, you can move up to half your speed. Before you do so, you can designate one creature who threatens you. This movement does not provoke attacks of opportunity from that creature.

\subsection{Total Defense}\label{Total Defense}
As a standard action, you can focus entirely on defense, granting you a \plus4 circumstance bonus to your dodge modifier for 1 round. While using the total defense action, you can't make attacks of opportunity, but you still threaten squares normally for the purpose of overwhelm penalties and similar effects.

\subsection{Delay}\label{Delay}
By choosing to delay, you take no action and then act normally on whatever initiative count you decide to act. When you delay, you voluntarily reduce your own initiative result for the rest of the combat. When your new, lower initiative count comes up later in the same round, you can act normally. You can specify this new initiative result or just wait until some time later in the round and act then, thus fixing your new initiative count at that point.

You never get back the time you spend waiting to see what's going to happen. Additionally, you can't interrupt anyone else's action (as you can with a readied action).

\parhead{Initiative Consequences of Delaying} Your initiative result becomes the count on which you took the delayed action. If you come to your next action and have not yet performed an action, you don't get to take a delayed action (though you can delay again).

If you take a delayed action in the next round, before your regular turn comes up, your initiative count rises to that new point in the order of battle, and you do not get your regular action that round.

\subsection{Ready}\label{Ready}
The ready action lets you prepare to take an action later, after your turn is over but before your next one has begun. Readying is a standard action. It does not provoke an attack of opportunity (though the action that you ready might do so).

\parhead{Readying an Action} You can ready a standard action or a move action. To do so, specify the action you will take and the conditions under which you will take it. You cannot ready in response to an action that you take -- the action must be outside of your control. Then, any time before your next action, you may take the readied action in response to that condition. The action occurs just before the action that triggers it. If the triggered action is part of another character's activities, you interrupt the other character. Assuming he is still capable of doing so, he continues his actions once you complete your readied action. Your initiative result changes. For the rest of the encounter, your initiative result is the count on which you took the readied action, and you act immediately ahead of the character whose action triggered your readied action.

You can take some free actions as part of your readied action, but in general you can take no actions other than the action you readied.

\parhead{Initiative Consequences of Readying} Your initiative result becomes the count on which you took the readied action. If you come to your next action and have not yet performed your readied action, you don't get to take the readied action (though you can ready the same action again). If you take your readied action in the next round, before your regular turn comes up, your initiative count rises to that new point in the order of battle, and you do not get your regular action that round.

\parhead{Distracting Spellcasters} You can ready a full attack against a spellcaster with the trigger ``if she starts casting a spell." If you damage
the spellcaster, she may lose the spell she was trying to cast (as determined by her Concentration check result).

\parhead{Readying to Counterspell} You may ready a counterspell against a spellcaster (often with the trigger ``if she starts casting a spell"). In this case, when the spellcaster starts a spell, you get a chance to identify it with a Spellcraft check (DC 15 \add spell level). If you do, and if you can cast that same spell (are able to cast it and have it prepared, if you prepare spells), you can cast the spell as a counterspell and automatically ruin the other spellcaster's spell. Counterspelling works even if one spell is divine and the other arcane.

A spellcaster can use \spell{dispel magic} to counterspell another spellcaster, but it doesn't always work.

\parhead{Readying a Weapon against a Charge} You can ready certain piercing weapons, setting them to receive charges (see \trefcp{Weapons}). A readied weapon of this type deals double damage on your first attack with it against a charging character.

\section{Special Rules}\label{Special Rules}

\subsection{Critical Success and Failure}\label{Critical Success and Failure}
A natural 1 (the d20 comes up 1) on an attack roll or saving throw is treated as rolling a \minus10. A natural 20 (the d20 comes up 20) is treated as rolling a 30. Under normal circumstances, a natural 1 automatically misses, and a natural 20 automatically hits.

\subsubsection{Critical Hits}\label{Critical Hits}
When you roll a natural 20 on an attack roll and hit, you have scored a critical threat. Roll another attack roll at the same attack bonus. If that attack also hits, you deal double damage.

Many weapons can also score critical threats on lower numbers, or deal additional damage on a critical hit.

\subsection{Doubling}\label{Doubling}
If you double any in-game value twice, it becomes three times as large. An additional doubling would make it four times as large, and so on. For example, if you make an attack that deals double damage and you get a critical hit, you deal triple damage.

\parhead{Real-World Values} Values that are ``real'', such as movement and distance, are an exception. If you double a real-world value twice, it becomes four times as large. 

\subsection{Size in Combat}\label{Size in Combat}
Size affects your space and reach in combat. In addition, your attack modifier and armor class is affected by your size modifier. These effects are shown on \trefnp{Size in Combat}. 

\begin{dtable*}
    \lcaption{Size in Combat}
    \begin{tabularx}{\textwidth}{l l l l l X}
        \thead{Size} & \thead{Space\fn{1}} & \thead{Reach\fn{1}} & \thead{Size Modifier\fn{2}} & \thead{Special Size Modifier\fn{3}} & \thead{Example Creature} \\
        Fine & 1/2 ft. & 0 & \plus8 & \minus16 & Fly\\
        Diminutive & 1 ft. & 0 & \plus4 & \minus12 & Toad \\
        Tiny & 2-1/2 ft. & 0 & \plus2 & \minus8 & Cat \\
        Small & 5 ft. & 5 ft. & \plus1 & \minus4 & Halfling \\
        Medium & 5 ft. & 5 ft. & \plus0 & \plus0 & Human \\
        Large (tall) & 10 ft. & 10 ft. & \minus1 & \plus4 & Ogre \\
        Large (tall) & 10 ft. & 5 ft. & \minus1 & \plus4 & Horse \\
        Huge (tall) & 15 ft. & 15 ft. & \minus2 & \plus8 & Cloud giant \\
        Huge (long) & 15 ft. & 10 ft. & \minus2 & \plus8 & Bulette \\
        Gargantuan (tall) & 20 ft. & 20 ft. & \minus4 & \plus12 & 50-ft. animated statue \\
        Gargantuan (long) & 20 ft. & 15 ft. & \minus4 & \plus12 & Kraken \\
        Colossal (tall) & 30\add ft. & 30\add ft. & \minus8 & \plus16 & Colossal animated object \\
        Colossal (long) & 30\add ft. & 20\add ft. & \minus8 & \plus16 & Great wyrm red dragon \\
    \end{tabularx}
    1 Creatures can vary in space and reach. These are simply typical values. \\
    2 Added to your attack modifier and Armor Class \\
    3 Added to your maneuver modifier and Maneuver Class \\
\end{dtable*}

Unusually large or small creatures also have other special rules apply to them. 

\subsubsection{Very Small Creatures}
\parhead{Space} If a creature takes up less than a single square of space, you can fit multiple creatures in that square. You can fit four Tiny creatures in a square, twenty-five Diminuitive creatures, or 100 Fine creatures.

\parhead{Reach} Creatures that take up less than 1 square of space typically have a natural reach of 0 feet, meaning they can't reach into adjacent squares. They must enter an opponent's square to attack in melee. This provokes an attack of opportunity from the opponent. You can attack into your own square if you need to, so you can attack such creatures normally. Since they have no natural reach, they do not threaten the squares around them, allowing you to move past them without provoking attacks of opportunity.

If a creature without a natural reach uses a reach weapon, it gains no benefits or penalties.

\subsubsection{Very Large Creatures}
\parhead{Space} Very large creatures take up multiple squares. Anything which affects a single square the creature occupies affects the creature. 

\parhead{Reach} Creatures that take up more than 1 square typically have a natural reach of 10 feet or more, meaning that they can reach targets even if they aren't in adjacent squares. Creatures with a large natural reach can attack anyone within their reach, including adjacent foes.

Creatures with a large natural reach using reach weapons can strike at up to double their natural reach but can't strike at their natural reach or less, just like Medium sized creatures.

\subsection{Mounted Combat}\label{Mounted Combat}
\parhead{Horses in Combat} Warhorses and warponies can serve readily as combat steeds. Light horses, ponies, and heavy horses, however, are frightened by combat. If you don't dismount, you must make a DC 20 Ride check each round as a move action to control such a horse. If you succeed, you can perform a standard action after the move action. If you fail, you can't do anything else until your next turn.

Your mount acts on your initiative count as you direct it. You move at its speed, but the mount uses its action to move.

\parhead{Space} A horse (not a pony) is a Large creature, and thus takes up a space 10 feet (2 squares) across. While mounted, you share your mount's space completely. Anyone threatening your mount can attack either you or your mount. However, your reach is still that of a creature of your normal size. Thus, a Medium paladin would threaten all squares adjacent to his Large horse with a longsword, and all squares 10 feet away from his mount with a lance.

In the case of abnormally large mounts (two or more size categories larger than you), you may not completely share space. Such situations should be handled on a case-by-case basis, depending on the nature of the mount.

\parhead{Combat while Mounted} With a DC 5 Ride check, you can guide your mount with your knees so as to use both hands to attack or defend yourself. This is a free action.

If your mount moves more than its normal speed when charging, you also take the AC penalty associated with a charge. If you make an attack at the end of such a charge, you receive the bonus gained from the charge. When charging on horse-back, you deal double damage with a lance (see \pcref{Charge}).

You can use ranged weapons while your mount is taking a double move, but at a \minus4 penalty on the attack roll. You can use ranged weapons while your mount is running (quadruple speed), at a \minus8 penalty. In either case, you make the attack roll when your mount has completed half its movement. You can make a full attack with a ranged weapon while your mount is moving. Likewise, you can take move actions normally.

\parhead{Casting Spells while Mounted} You can cast a spell normally if your mount moves up to a normal move (its speed) either before or after you cast. If you have your mount move both before and after you cast a spell, then you're casting the spell while the mount is moving, and you have to make a Concentration check due to the vigorous motion (DC 10 \add double spell level) or lose the spell. If the mount is running (quadruple speed), you can cast a spell when your mount has moved up to twice its speed, but your Concentration check is more difficult due to the violent motion (DC 15 \add double spell level).

\parhead{If Your Mount Falls in Battle} If your mount falls, you have to succeed on a DC 15 Ride check to make a soft fall and take no damage. If the check fails, you take 1d6 points of damage.

\parhead{If You Are Dropped} If you are knocked unconscious, you have a 50\% chance to stay in the saddle (or 75\% if you're in a military saddle). Otherwise you fall and take 1d6 points of damage. Without you to guide it, most mounts avoid combat.

\subsection{Two-Weapon Fighting}\label{Two-Weapon Fighting}
If you wield a second weapon in your off hand, you can attack with both weapons at once whenever you attack. Roll a single attack roll for both weapons. Apply your attack bonuses with each of your weapons separately, taking a \minus5 penalty with your off-hand attack. If you hit with your main hand, you deal damage with your main weapon. If you hit with your off-hand (after taking into account the \minus5 penalty), you also deal damage with your off-hand weapon. You only apply your Strength with the weapon in your main hand. 

\par Precision-based damage, such as sneak attack damage, is only dealt once. It is possible to critical with both weapons. Use each weapon's critical threat range separately, but roll only once to confirm a critical threat, using the same attack bonus as with the original attack. Damage reduction only applies once against the damage dealt by both weapons.

\par Fighting in this way is difficult, and you suffer a \minus2 penalty to your attack roll. You can mitigate this penalty if your off-hand weapon is light. (An unarmed strike is always considered light.) The Two-Weapon Fighting feat grants a \plus2 competence bonus to attack rolls when fighting with two weapons at once.

You take no penalties for alternating attacks between two (or more) weapons, as long as you do not attack with both weapons at once.

\par For example, Felix the 1st-level fighter is wielding a longsword and a short sword against an evil goblin. The goblin has an AC of 15. Felix has a Strength of 3 and a base attack bonus of \plus1. This means his attack bonus with either weapon individually is \plus4. If he attacks with both weapons at once, he takes no penalty to his attack roll (because his off-hand weapon is light), but his attack with his off-hand weapon takes a \minus5 penalty. So his attack bonus would be \plus4 (with his longsword) and \minus1 (with his short sword). If he rolls a 15, he will hit the goblin with his longsword, but not with his short sword.

\par If Felix had the Two-Weapon Fighting feat, his attack bonus would be \plus6 with his longsword and \plus1 with his short sword. Assuming he rolls a 15 again, he would hit the goblin with both weapons, dealing damage with both of them.

