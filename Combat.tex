\chapter{Combat}\label{Combat}

\section{How Combat Works}
Combat takes place in a series of ``rounds'', which represent about six seconds of action. Each round, every character gets a turn to act. The order in which creatures take turns is determined by their initiative (see \pcref{Initiative}). 

Once in combat, creatures attack each other (see \pcref{Attacks}) and defend themselves (see \pcref{Defenses}), while moving around the battlefield (see \pcref{Movement and Positioning}). When your defenses fail, you can get hurt (see \pcref{Injury, Death, and Healing}). In unusual situations, you might become more or less likely to succeed at your actions (see \pcref{Circumstances, Bonuses, and Penalties}).

When it's your turn, you can take one standard action, one move action, and one swift action. If you want, you can take a full-round action instead of your standard and move actions. You can always ``downgrade'' an action to a lesser action: turning a standard action into a move action, or a move action into a swift action.

\section{Attacks}\label{Attacks}
An ``attack'' is anything that affects another creatures in a potentially harmful way. There are two kinds of attacks: physical attacks, which are made with weapons or fists, and special attacks, which are made with magic or supernatural power. All physical attacks, and most special attacks, require making an attack roll against a defense. If the result of the attack roll meets or exceeds the defense, the attack succeeds.

\subsection{Standard Attack}
As a standard action, you can try to strike a foe with a weapon you are wielding. To do so, make an attack roll with a weapon you are wielding, adding your attack modifier to the roll. If your result equals or exceeds your foe's Armor defense, your attack hits, and your foe takes damage.

You can also make a single attack as an attack of opportunity. See \pcref{Attacks of Opportunity}. 

\subsubsection{Attack Modifier}
Your attack modifier is equal to the following:

\begin{figure}[h]
\centering Base attack bonus \add attack attribute \add enhancement bonus \add size modifier \add other bonuses and penalties
\end{figure}

With medium or heavy weapons, your attack attribute is your Strength. With projectile weapons, your attack attribute is your Dexterity. With light weapons, you can choose between your Strength and Dexterity. 

Your size modifier is described in \tref{Size in Combat}.

\subsubsection{Damage}
If your attack succeeds, you deal damage equal to your weapon's damage die \add half your Strength.

\parhead{Dealing Nonlethal Damage} If you take a \minus4 penalty to your attack roll, you can deal nonlethal damage instead of lethal damage when you hit. See \pcref{Nonlethal Damage}.

\subsubsection{Multiple Attacks}
If your base attack bonus is \plus6 or higher, you can make multiple attacks as part of a standard attack. Each attack after the first takes a cumulative \minus5 penalty to hit. This progression is shown on \trefnp{Attacks per Round}.
\begin{dtable}
    \lcaption{Attacks per Round}
    \begin{tabularx}{\columnwidth}{*{3}{>{\lcol}X}}
        \thead{Base Attack Bonus} & \thead{Attacks per Round} & \thead{Attack Penalties} \\
        1-5 & 1 & \plus0\\
        6-10 & 2 & \plus0, \minus5 \\
        11-15 & 3 & \plus0, \minus5, \minus10 \\
        16-20 & 4 & \plus0, \minus5, \minus10, \minus15\\
    \end{tabularx}
\end{dtable}

Some special abilities, such as the \spell{haste} spell, also grant you the ability to make multiple attacks. In all cases, making multiple attacks requires using a standard action to make a standard attack. You cannot take multiple attacks on an attack of opportunity. Additionally, any penalties imposed by taking multiple attacks do not affect any other attacks you make, such as attacks of opportunity.

\subsection{Reach}\label{Reach}
Normally, you can attack anyone within five feet of you. The range at which you can attack is called your ``reach'', and the area that you can attack into is called your ``threatened area''. Reach for larger and smaller creatures is determined by size, as shown on \trefnp{Size in Combat}.

\subsection{Attacks of Opportunity}\label{Attacks of Opportunity}
Sometimes a combatant in a melee lets her guard down. In this case, combatants near her can take advantage of her lapse in defense to attack her for free. These free attacks are called attacks of opportunity.

\subsubsection{Provoking Attacks of Opportunity}
You can provoke attacks of opportunity in two ways.

\parhead{Being Defenseless}\label{Being Defenseless} Whenever you take an action, if you are unable to defend yourself at any point during the action, you provoke an attack of opportunity. You can be defenseless for several reasons. Some examples are given below.

\begin{itemize*}
    \item If you can't use a weapon or shield to defend yourself, you are defenseless. This can happen if you are trying to wield a heavy weapon in one hand, if you aren't proficient with your only weapon, or if you have no weapons at all. 
    \item If you are busy doing something else that requires your full attention (such as casting a spell), you are defenseless.
\end{itemize*}

\parhead{Leaving the Battle} You provoke an attack of opportunity if you move farther away from an opponent who threatens you. This can be mitigated with the withdraw action, which allows you to avoid provoking from specific opponents (see \pcref{Withdraw}).

\parhead{Forced Movement} You never provoke attacks of opportunity for movement you didn't take intentionally, such as from a shove (see \pcref{Shove}).

\subsubsection{Taking Attacks of Opportunity}
If a creature within your threatened area provokes an attack of opportunity, you can immediately make a single melee attack against that creature. The attack of opportunity ``interrupts'' anything else the creature was been doing. The creature can continue its turn after your attack of opportunity is resolved.

You can make a number of attacks of opportunity each round equal to 1 \add half your Dexterity, but never more than one per round against a particular creature. You don't have to make an attack of opportunity if you don't want to.

\subsection{Special Attacks}

\subsubsection{Feint}\label{Feint}
Instead of striking to deal damage, you can feint to leave your foe off-balance. You make a melee attack with your weapon as normal, except that you target the creature's Reflex defense instead of its Armor defense. Success means you deal no damage, but the creature takes a \minus4 penalty to physical defenses for 1 round.

\subsubsection{Touch Attacks}\label{Touch Attacks}
Instead of striking to deal damage, you can try to just touch your opponent. This is usually done to channel spells that require touching a creature. Touch attacks target Reflex defense instead of Armor defense.

\subsection{Combat Maneuvers}\label{Combat Maneuvers}
A combat maneuver is an attempt to physically hinder your foe with your body, such as by shoving or tripping it. Unless otherwise noted, using a combat maneuver requires a free hand and takes a standard action. You make an attack against a creature within your reach, adding your maneuver modifier to the roll instead of your attack modifier. If your result equals or exceeds your foe's Maneuver defense, your attack hits, and your foe is affected by the maneuver.

\begin{dtable}
    \lcaption{Combat Maneuvers}
    \begin{tabularx}{\columnwidth}{l l l X}
        \thead{Maneuver}  & \thead{Action} & \thead{Attribute} & \thead{Brief Description} \\
        Dirty Trick & Standard & Str or Dex & Impose penalty on an foe \\
        Disarm & Standard & Str or Dex & Force foe to drop item \\
        Grapple & Standard & Str & Wrestle with an foe \\
        Overrun & Full-round & Str & Move through an foe \\
        Shove & Standard & Str &  Move an foe \\
        Trip & Standard & Str or Dex & Trip an foe \\
    \end{tabularx}
\end{dtable}

\subsubsection{Maneuver Modifier}
Your maneuver modifier is equal to the following:

\begin{figure}[h]
\centering Base attack bonus \add attack attribute \add enhancement bonus \add special size modifier \add other bonuses and penalties
\end{figure}

Your attack attribute depends on the maneuver you are using, as described in \pcref{Combat Maneuvers}. Your special size modifier is described in \tref{Size in Combat}.

\subsubsection{Maneuver Descriptions}

\parhead{Dirty Trick}\label{Dirty Trick} You strike your foe in a sensitive spot, pull its pants down, or creatively use your environment to attack. Success means the creature is vulnerable for 1 round, causing giving it a \minus2 penalty to attacks, defenses, and checks. For every 5 points by which you succeed, the creature is vulnerable for an additional round. You can perform a dirty trick with either Str or Dexterity.

\parhead{Disarm}\label{Disarm} You can strike an item your foe is wearing or holding. Success means you hold the item instead of your foe. Well-secured items, such as rings and body armor, cannot be disarmed. You can perform a disarm with either Strength or Dexterity. You can also with any weapon, rather than using a free hand. If you disarm with a weapon, the item falls to the ground in the creature's square.

\parhead{Grapple}\label{Grapple} You physically grab and restrain your foe. Success means you and the creature become grappled, which limits your ability to act. See \pcref{Grappling} for more details.

\parhead{Overrun}\label{Overrun} You move directly through your foe. This maneuver requires a full-round action. When you overrun, you move up to your speed. Each creature in your way can choose to avoid you, allowing you to pass through its square unhindered. Each time you encounter a creature in your way that didn't avoid you, you make an overrun attack. Success means you can move through the creature's space, though it is considered difficult terrain. Success by 10 or more means the creature is knocked prone, and the area is not considered difficult terrain. You can only overrun a creature with Strength.

\parhead{Shove}\label{Shove} You shove your foe where you want it to go. Success means you move the creature 5 feet in a direction of your choice. For every 5 points by which you succeed, you can move it an additional 5 feet. You cannot move the creature further after moving it outside of your reach, but you can take a move action to move with the creature as you shove it to move it farther. If the creature encounters a solid object or creature, you must stop shoving it. You can only shove a creature with Strength.

\parhead{Trip}\label{Trip} You try to trip your foe. Success means the creature falls prone, causing it to take a \minus4 penalty to physical attacks and defenses. You can trip a creature with Strength or Dexterity.

\section{Defenses}\label{Defenses}
Usually, when you are attacked, the attacker has to make an attack roll against a specific defense. If the attack roll is at least as high as that defense, the attack succeeds. There are three physical defenses and two special defenses.
\begin{itemize*}
    \item Armor defense: Your Armor defense protects you from normal physical attacks, such as attempts to stab you with a sword. It is the most commonly used defense. Armor defense is a physical defense.
    \item Maneuver defense: Your Maneuver defense protects you from unusual physical attacks, such as attempts to trip or disarm you. Maneuver defense is a physical defense.
    \item Reflex defense: Your Reflex protects you from attacks you have to avoid, such as explosions or falling rocks. Reflex defense is a physical defense.
    \item Fortitude defense: Your Fortitude defense protects you from attacks you have to physically endure or resist, such as poisons and deadly spells. Fortitude defense is a special defense. 
    \item Will defense: Your Will defense protects you from attacks you have to mentally endure or resist, such as terrifying creatures and magical manipulation. Will defense is a special defense.
\end{itemize*}

\subsection{Physical Defenses}

Each of your physical defenses is the sum of other bonuses and modifiers, as shown on \trefnp{Physical Defense Calculations}. 

\begin{dtable!*}
    \lcaption{Physical Defense Calculations}
    \begin{tabularx}{\textwidth}{l c c c c c c c >{\ccol}X}
        \thead{Defense Name} & \thead{Base Bonus} & \thead{Str} & \thead{Dex} & \thead{Wis} & \thead{Armor Modifier} & \thead{Shield Modifier} & \thead{Dodge Modifier} & \thead{Size Modifier} \\
        Armor defense & Half base attack bonus & \x & Yes & \x & Yes & Yes & Yes & Yes \\
        Maneuver defense & Base attack bonus & Yes & Yes & \x & \x & Yes & Yes & Special \\
        Reflex defense & Base Reflex bonus & \x & Yes & Half & \x & \x & Yes & Yes \\
    \end{tabularx}
\end{dtable!*}

\parhead{Base Attack Bonus} Your experience and aptitude in combat affects your ability to defend yourself; experienced warriors know how to recognize and avoid or parry blows that would easily fell novices. As a result, you add half your base attack bonus to your Armor defense. Your fighting experience is even more important when defending against combat maneuvers, so you add your full base attack bonus to your Maneuver defense.
\parhead{Enhancement Bonuses} You can have separate enhancement bonuses to your armor modifier, shield modifier, and dodge modifier. All of these enhancement bonuses stack to improve your physical defenses.
\parhead{Natural Armor} Creatures with unusually tough skin or thick hide, including most monsters, gain bonuses to their armor modifier. These bonuses stack with any armor such creatures might wear.
\parhead{Size Modifiers} Your size modifier and special size modifier are described on \tref{Size in Combat}.

\subsection{Special Defenses}

Your special defenses are simpler to calculate than your physical defenses. Your Fortitude defense is calculated as follows:

\begin{figure}[h]
\centering Base Fortitude bonus \add Constitution \add half Strength \add enhancement bonus \add other bonuses and penalties
\end{figure}

Your Will defense is calculated as follows:

\begin{figure}[h]
\centering Base Will bonus \add Charisma \add half Intelligence \add enhancement bonus \add other bonuses and penalties
\end{figure}
\parhead{Base Defense Bonus} Your base Fortitude bonus and base Will bonus come from your classes. See \pcref{Base Defense Progressions}, for more details.

\subsection{Hit Points}
Your hit points represent how much punishment you can take. When you run out of hit points, your actions are limited and you might die, as described in \pcref{Injury, Death, and Healing}.

\section{Movement and Positioning}\label{Movement and Positioning}

\subsection{Taking up Space}
A typical human takes up a 5-ft. by 5-ft. space in combat. For convenience, this is often called a ``square''. Differently sized creatures can take up more or less space, as indicated on \tref{Size in Combat}. Normally, other creatures can't be in any squares you occupy.

Sometimes, movement and distance are represented in squares. A 30-ft. movement is the same thing as moving six squares.

\subsection{Ways to Move}

\parhead{Move} As a move action, you can move up to your speed.
\parhead{Stand Up} As a move action, you can stand up from being prone. For most creatures, this requires using a hand to help get up.
\parhead{Withdraw}\label{Withdraw} As a standard action, you can move up to your speed. Before you do so, you can designate one creature who threatens you. This movement does not provoke attacks of opportunity from that creature. At base attack bonus \plus6, \plus11, and \plus16, you may avoid provoking attacks of opportunity from an additional opponent of your choice.
\parhead{Charge}\label{Charge} As a full-round action, you can move up to twice your speed and make a single attack with a \plus2 bonus to hit at the end of your charge. While charging, and until the start of your next turn, you take a \minus2 penalty to physical defenses.
\par You must move at least 30 feet to gain the benefit of a charge, and all movement must be in a single straight line. If there are any obstacles in your path which hinder your movement, you cannot charge. If your charge fails or becomes invalid partway through, you move as far as you can and stop.
\par  At base attack bonus \plus6, the attack you make at the end of a charge deals double damage if you hit by 5 or more. At base attack bonus \plus 11, the attack deals triple damage if you hit by 10 or more. At base attack bonus \plus16, the attack deals quadruple damage if you hit by 15 or more.
\parhead{Struggle} As a full-round action, you can move five feet, regardless of movement penalties. You can use this to move even if your speed is decreased below five feet by penalties.

\subsection{Measuring Movement}
For simplicity, all movement is measured in five-foot increments. While it is possible to be more precise than that, it's generally not worth the complexity.

\parhead{Diagonals} When measuring distance, the first diagonal counts as five feet of movement, and the second counds as ten feet of movement. The third costs five feet, the fourth costs ten feet, and so on.

You can't move diagonally past a corner. You can move diagonally past a creature, even an opponent. You can also move diagonally past other impassable obstacles, such as pits.
\parhead{Closest Creature} When it's important to determine the closest square or creature to a location, if two squares or creatures are equally close, pick one randomly.

\subsection{Movement Impediments}

\subsubsection{Difficult Terrain}
Some terrain is hard to move through, like thick bushes or a swamp. If a square is difficult terrain, it doubles the movement cost required to move through the square. That generally means it takes ten feet of movement, or fifteen feet if you are moving diagonally. If a square is considered difficult terrain for multiple reasons, the cost increases stack. For example, a square in a swamp that also has thick bushes blocking your passage would take twenty feet of movement, or thirty feet to move diagonally!

\subsubsection{Obstacles}
An obstacle is anything that gets in your way. Enemies and large solid objects like walls are blocking obstacles: they completely block your movement. If you can get past an obstacle, like a low wall, that square is treated as difficult terrain. Some obstacles require a skill check to bypass.

\subsubsection{Squeezing}

In some cases, you may have to squeeze into or through an area that isn't as wide as the space you take up. You can squeeze through or into a space that is at least half as wide as your normal space. While squeezing, you move half as fast, and you take a \minus4 penalty on physical attacks and defenses. You can squeeze into tighter spaces with the Escape Artist skill.

Creatures that take up multiple squares take up half their normal number of squares while squeezing. For example, a Large creature who normally takes up four spaces takes up two spaces while squeezing.

\parhead{Accidentally Squeezing} Sometimes a character ends its movement while moving through a space where it's not normally allowed to stop. When that happens, the character is squeezing in the space until it can move. If squeezing is impossible, the creature immediately moves to the closest available space. Try not to do this.

\subsection{Special Movement Modes}\label{Special Movement Modes}
Some spells and abilities grant creatures the ability to move in unusual ways. These forms of movement are described here.

\subsubsection{Burrowing}\label{Burrowing}
A creature with a burrow speed can move through the ground at the indicated speed in any direction, even vertically. Unless otherwise noted, the creature can only burrow through dirt and loose earth, not rock or harder substances. It does not leave behind a usable tunnel for other creatures. 

\subsubsection{Climbing}\label{Climbing}
A creature with a climb speed can move at the indicated speed while climbing. In addition, it has a \plus5 bonus to Climb checks (see \pcref{Climb}). It can always choose to take 10 on Climb checks, even if rushed or threatened. It cannot make an accelerated climb. 

\subsubsection{Flying}\label{Flying}
A creature with a fly speed can fly through the air at the indicated speed. It must be unencumbered (see \pcref{Carrying Capacity}). If a creature with a fly speed is encumbered, it is treated as having a glide speed instead (see \pcref{Gliding}), which it can use to glide even though it is encumbered.

Each creature with a fly speed also has a maneuverability: good, average, poor, or special. Normally, a flying creature must move forward by at least half its fly speed each round. If it does not, it falls. Turning by 90 degrees costs 5 feet of movement, and it can't turn in the same place by more than 90 degrees. It can move up by only one square vertically per square traveled horizontally, but it can fly directly down if it chooses. The creature can fly up at half speed, but can fly down twice as fast.

\parhead{Good Maneuverability} A flying creature with good maneuverability need not move forward to maintain its flight, allowing it to hover or fly directly up if it chooses. It must spend a move action each round to move, even if it simply hovers in place. In addition, turning does not cost movement, and it can freely turn in place.

\parhead{Poor Maneuverability} A flying creature with poor maneuverability must spend five feet of movement to turn by 45 degrees, and it can't turn in the same place by more than 45 degrees. In addition, it can only descend by up to one square vertically per square traveled horizontally without falling.

\parhead{Special Maneuverability} A flying creature with special maneuverability does not experience gravity like other creatures. In addition to the effects of good maneuverability, it moves up and down at the same speed as it moves horizontally. It can also hover in place without spending a move action.

\parhead{Falling} If a flying creature loses control, usually by failing to maintain its minimum forward speed, or loses the ability to fly, it falls. While falling, a flying creature can attempt to recover by making a DC 15 Dexterity check as a move action. If it succeeds, it can begin flying as normal. Otherwise, it continues falling for another round.

\subsection{Gliding}\label{Gliding}
A creature with a glide speed can glide through the air at the indicated speed. It must be unencumbered (see \pcref{Carrying Capacity}).

While in the air, a creature with a glide speed can control its fall as a move action. This allows it to move up to its speed horizontally in a direction of its choice while moving only five feet down. If it desires, it can move half as far horizontally and fall down twice as fast. It takes no falling damage if it touches the ground while gliding.

\section{Initiative}\label{Initiative}

At the start of a battle, each creature that is aware of the combat makes an initiative check. Your initiative check is calculated as follows:

\begin{figure}[h]
    \centering Dexterity \add half Wisdom \add enhancement bonus \add other bonuses and penalties
\end{figure}

Creatures act in order of their initiative, highest to lowest. After the creature with the lowest initiative acts, a new round begins and the creature with the highest initiative acts again. You usually keep your initiative for the whole encounter, even if you can't act. However, some special actions can change your initiative count.

\subsubsection{Surprise Attacks}\label{Surprise Attacks}
Sometimes, some creatures are not aware of the combat when it starts. This most commonly happens with ambushes. Any creature that is not aware of the combat doesn't get to make an initiative check during the first round of combat. Until it takes its first action, it is vulnerable (\minus2 penalty to attacks, defenses, and checks) and can't take attacks of opportunity.

Sometimes, everyone is surprised, such as if a guard walks around a corner to unexpectedly find a group of muggers. In that case, initiative is rolled normally, but all creatures are vulnerable and can't take attacks of opportunity until they take their first action.

\section{Injury, Death, and Healing}\label{Injury, Death, and Healing}
Your hit points measure how hard you are to kill. No matter how many hit points you lose, your character isn't hindered in any way until your hit points drop to 0.

\subsection{Losing Hit Points}
When you take lethal damage, you subtract that damage from your hit points.
\parhead{What Hit Points Represent} Hit points represent a combination of durability, luck, divine providence, and sheer determination, depending on the nature of your character. When you take 10 damage from an orc with a greataxe, the axe did not literally carve into your skin without affecting your ability to fight. Instead, you avoided the worst of the blow, but it bruised you through your armor, the effort to dodge the blow fatigued your character, or it barely nicked you through sheer luck -- and everyone's luck runs out eventually.

\subsection{Stages of Injury}

\subsubsection{Healthy} 
When you are above half hit points, you suffer no significant effects from losing hit points. If you take damage, you can become bloodied (see Bloodied, below).

\subsubsection{Bloodied}
When you drop to half your hit points or below, you are bloodied. This makes you more vulnerable to certain spells and effects, but you suffer no direct penalties. If you take additional damage, you can become staggered.

\subsubsection{Staggered}
When you take damage that would reduce your hit points to 0, you become staggered. While staggered, you may take a single move action or standard action each round, but not both. You cannot take full-round actions, but you may take swift actions. In addition, you are vulnerable, causing you to take a \minus2 penalty to attacks, defenses, and checks.

If you take additional damage while at 0 hit points, you begin dying (see Dying, below).

\subsubsection{Dying}\label{Dying}
When you take damage while you have no hit points remaining, that damage represents serious physical injury to your body. This is called critical damage. When you take critical damage, you begin dying. While you have critical damage, magical healing which would normally restore hit points cannot restore your hit points, though it can stabilize you, preventing you from dying.

If you have no hit points, and your critical damage exceeds your Constitution, you fall unconscious. Otherwise, you are staggered as long as you are dying.

While dying, you must make a Constitution check every round. The DC is equal to 10 \add the critical damage you have taken. If you fail three times, you die. If you succeed three times, or receive healing that would normally restore hit points, you become stable (see Stable, below). Another character can give first aid to help you stabilize (see \pcref{Heal}).

If you take additional damage, you can die (see Dead, below).

\subsubsection{Dead}
If your critical damage exceeds 10 \add your Constitution, you die. You can also die from taking ability damage or suffering ability damage or drain that reduces your Constitution to \minus 10.

\subsubsection{Stable}\label{Stable}
If you have taken critical damage but managed to stave off death, you become stable. As long as you have critical damage, magical healing that restores hit points has no effect on your hit points. If you became unconscious from critical damage, you regain consciousness as soon as you have hit points.

\subsubsection{Overkill Damage}
Normally, when you take damage that would reduce your hit points to 0, any excess damage from the attack is wasted. However, some attacks deal such massive damage that you begin dying immediately, rather than just becoming staggered. If the damage dealt by an attack exceeds your maximum hit points (not current hit points), any damage past what would reduce your hit points to 0 is dealt as critical damage rather than being wasted.

\subsection{Healing}
After taking damage, you can recover hit points through natural healing or through magical healing. In any case, you can't regain hit points past your full normal hit point total.

\parhead{Natural Healing} With 8 hours of rest, you recover half your hit points. Any significant interruption (such as combat or the like) during your rest prevents you from healing. If you rest for an entire day (16 hours), you recover all your hit points.

\parhead{Magical Healing} Various abilities and spells can restore hit points. However, only certain spells can heal critical damage, as specified in the spell description. Unless a spell says it can cure critical damage, it cannot -- though it can still stabilize dying characters. Magical healing has no effect on the hit points of creatures with critical damage.

\parhead{Healing Ability Damage} Ability damage is temporary, just as hit point damage is. Ability damage returns at the rate of 1 point per 8 hours of rest for each affected attribute score.

\parhead{Healing Critical Damage} Critical damage takes much longer to heal than hit point damage. Resting for 1 week restores an amount of critical damage equal to 1 \add half the character's Constitution (minimum 1). A character can have both hit points and critical damage. As long as a character has critical damage, he is staggered, even if he is at full hit points.

\subsection{Nonlethal Damage}\label{Nonlethal Damage}
Some attacks and environmental effects deal nonlethal damage. Nonlethal damage is not subtracted from your hit points. Instead, it is tracked separately. If your nonlethal damage exceeds your hit points, you become staggered, just as if you were at 0 hit points. If you take additional damage while staggered, you fall unconscious. However, you do not begin dying unless your hit points are actually below 0. 

\subsubsection{Healing Nonlethal Damage}
You heal half your hit points in nonlethal damage with 1 hour of rest. When a spell or a magical ability cures hit point damage, it also removes an equal amount of nonlethal damage.

\subsection{Temporary Hit Points}\label{Temporary Hit Points}
Certain effects give a character temporary hit points which act as a protection against damage. Whenever a character takes damage, if he has temporary hit points, the damage is applied to his temporary hit points first. Any excess damage is then applied to his hit points as normal. Temporary hit points are not ``real'' hit points, and cannot be healed. If a character has temporary hit points from multiple effects, only the highest value is used.

\section{Circumstances, Bonuses, and Penalties}

\begin{dtable}
    \lcaption{Attack Roll Bonuses and Penalties}
    \begin{tabularx}{\columnwidth}{l X}
        \thead{Attacker's Condition} & \thead{Effect} \\
        Entangled & \minus2 \\
        Invisible & \x\fn{1} \\
        Prone & \minus4\fn{2} \\
        Squeezing through a space & \minus4 \\
        Vulnerable & \minus2 \\
    \end{tabularx}
    1 The defender is defenseless, causing it to provoke attacks of opportunity for all its actions. \\
    2 Most ranged weapons can't be used while the attacker is prone, but you can use a crossbow or shuriken while prone at no penalty.
\end{dtable}

\begin{dtable}
    \lcaption{Physical Defense Bonuses and Penalties}
    \begin{tabularx}{\columnwidth}{>{\lcol}X >{\ccol}X}
        \thead{Defender's Condition} & \thead{Effect} \\
        Behind active cover & 20\% miss \\
        Behind passive cover & \plus4 \\
        Blinded & \x\fn{1} \\
        Concealed & \plus4 \\
        Cowering & \minus2 \\
        Crouching or kneeling & \minus2\fn{2} \\
        Entangled & \minus2 \\
        Grappling (but attacker is not) & \plus0\fn{1} \\
        Helpless (such as paralyzed, sleeping, or bound) & \x\fn{3} \\
        Invisible & see Invisibility \\
        Overwhelmed & special\fn{4} \\
        Pinned & \x\fn{3} \\
        Prone & \minus4\fn{2} \\
        Squeezing through a space & \minus4 \\
        Stunned & \minus2\fn{1} \\
        Unaware of attacker & \fn{3} \\
        Total defense & \plus4 \\
        Vulnerable & \minus2 \\
    \end{tabularx}
    1 The defender is defenseless, causing it to provoke attacks of opportunity for all its actions. \\
    2 Treat as a bonus against ranged attacks, instead of a penalty \\
    3 The defender's physical defenses are equal to 10 \add size modifier. \\
    4 The creature suffers a penalty equal to the number of creatures threatening it.
\end{dtable}

Bonuses and penalties are the most basic way that a roll or numerical statistic can be modified. A bonus increases the roll or statistic, and a penalty decreases it. Bonuses and penalties are also called modifiers.

\subsection{Stacking Rules}\label{Stacking Rules}
Usually, modifiers stack with each other, meaning that you add or subtract all of the modifiers to get the final result. However, enhancement bonuses and certain specific kinds of modifiers do not stack with each other. When bonuses don't stack with each other, you only apply the highest bonus. Likewise, when penalties don't stack with each ather, you only apply the highest penalty.

\parhead{Enhancement Bonuses} Enhancement bonuses are always granted by magic or supernatural abilities. Unlike other bonuses, enhancement bonuses do not stack with each other. Each enhancement bonus will specify what it applies to, such as ``physical attacks'' or ``Fortitude defense''. Use only the highest enhancement bonus that applies. If you have an enhancement bonus that affects a broad category of rolls, such as ``all special defenses'', it doesn't stack with more narrow enhancement bonuses, such as ``Fortitude defense''.

\parhead{Special Exceptions}

\begin{itemize*}
    \item Effects from the same source do not stack. Any spell or ability with the same name has the same source. 
    \item Magical effects that increase size do not stack.
    \item Damage reduction does not stack. Only the best value applicable to the attack applies.
    \item Effects that grant extra attacks (such as the \spell{haste} spell) do not stack.
    \item Temporary hit points do not stack.
    \item If a character has two separate abilities which let him add the same attribute to a given roll or numerical attribute, the attribute is still only added once.
    \item Effects that reduce the effective spell level of a spell affected by metamagic can never reduce a spell below its original level.
\end{itemize*}

\subsection{Cover}\label{Cover}

Cover represents any obstacle that physically prevents you from striking your target, such as a tree or intervening creature. A creature with cover is more difficult to attack.

\parhead{Determining Cover} When making a melee attack against an adjacent target, your target has cover if any line from your square to the target's square goes through a wall (including a low wall) or other similar solid obstacle. If you occupy multiple squares, choose one square you occupy for this purpose.

When making an attack against a target that is not adjacent to you, choose a corner of any square you occupy. In addition, choose a square the target occupies. If any line from this corner to any corner of the target square passes through a square or border that blocks line of effect or provides cover, or through a square occupied by a creature, the target has cover.

There are two types of cover: active cover and passive cover.

\subsubsection{Active Cover}

If the obstacle is active and mobile, such as a creature or tree branches blowing in the wind, the defender has active cover. Any attacks against a creature with active cover relative to you have a 20\% miss chance. After rolling the attack, the attacker must make a miss chance percentile roll to see if the attack misses due to active cover. If an attack misses due to active cover, the attack is made against the intervening obstacle instead. If the attack is successful, the obstacle takes any damage from the attack normally. 

\parhead{Small Obstacles} Generally, an obstacle smaller than you are does not provide active cover (so a halfling does not provide active cover to a human). 

\subsubsection{Passive Cover}

If the obstacle is stationary, such as a tree trunk or wall, the defender has passive cover. A creature with passive cover relative to you has a \plus4 bonus to physical defenses.

\parhead{Small Obstacles} A low obstacle (such as a wall no higher than half your height) provides passive cover, but only to creatures within 30 feet (6 squares) of it. The attacker can ignore the cover if he's closer to the obstacle than his target. If two creatures are equally distant from the wall, it grants cover to both of them.
\parhead{Attacks of Opportunity} You can't execute an attack of opportunity against an opponent with passive cover relative to you.
\parhead{Stealth Checks} You can use passive cover to make a Stealth check to hide, but not active cover.
\parhead{Total Cover} If you don't have line of effect to your target, he is considered to have total cover from you. You can't make an attack against a target that has total cover.

\subsubsection{Improved Cover}

A creature can benefit from both passive and active cover. However, cover of the same type generally doesn't stack; a creature behind two trees is not substantially more protected than a creature behind a single tree. In some cases, cover may stack. In that case, each additional obstacle increases the miss chance by 10\% or grants an additional \plus2 bonus to defenses, as appropritae.

Exceptionally well covered opponents, such as a creature behind an arrow slit in a castle, may receive additional benefits. For example, it might gain improved evasion, and there may be limitations on what kind of attacks are possible.

\subsection{Concealment}\label{Concealment}
Concealment represents anything which makes it more difficult to see your target, such as dim lighting. A creature with concealment from you gains a \plus4 bonus to physical defenses. Concealment bonuses do not apply if you can't see your opponent (such as if you close your eyes).

\parhead{Determining Concealment} When making a melee attack against an adjacent target, your target has concealment if his space is entirely within an effect that grants concealment.

Deterining concealment for making an attack against a target that is not adjacent to you works exactly like determining cover for ranged attacks.

In addition, some magical effects provide concealment against all attacks, regardless of whether any intervening concealment exists.

\parhead{Concealment and Stealth Checks} You can use concealment to make a Stealth check to hide.

\subsection{Grappling}
Grappling creatures are physically struggling with each other. While grappled, you suffer certain penalties and restrictions, as described below. 

\subsubsection{Being In A Grapple}
While grappling, you suffer certain penalties and restrictions, as described below. Other than these restrictions, you can act normally. You can also take certain actions in a grapple, as described in \pcref{Grapple Actions} 
\begin{itemize*}
    \item You must use a free hand (or equivalent limbs) to grapple, preventing you from taking any actions which would require having two free hands. If you cannot free a hand, you suffer a \minus10 penalty to all physical attacks, including grapple attacks, until you have a free hand.
    \item You take a \minus4 penalty to physical defenses against creatures you are not grappling with.
    \item You take a \minus4 penalty to attack rolls made with weapons that are not light, since they are too large and cumbersome to be used effectively in a grapple.
    \item Spellcasting is extremely difficult. You cannot cast spells with somatic components. Using a spell-like ability or casting a spell without somatic components requires a Concentration check with a DC equal to 20 \add double spell level.
    \item You cannot move normally (but see Move the Grapple, below).
\end{itemize*}

\subsubsection{Grapple Actions}
While grappled, you can take four special actions to try to affect the grapple.

\parhead{Bind Foe} As a standard action, you can make a grapple attack to bind your foe with ropes or other restraints. You must have the restraints in hand (in addition to the free hand required to grapple). Apply the result against the Maneuver defense of a creature grappling you. Success means the creature is bound, rendering it helpless and effectively paralyzed. The only physical action a bound creature can take is to escape the bindings, which requires a grapple attack or Escape Artist check which beats the grapple attack made to bind it. If you have the time, you can take 20 on your grapple attack to secure bindings on a helpless foe, including a foe rendered helpless through this attack.
\parhead{Escape the Grapple} As a standard action, you can make a grapple attack or Escape Artist check to escape from the grapple. Apply the result against the Maneuver defense of each creature grappling you. Success against an individual creature means you are no longer grappling that creature.
\parhead{Move the Grapple} As a move action, you can make a grapple attack to move the grapple. Apply the result against the MC of each creature grappling you. If you beat every creature's MC, you can move yourself and all other creatures in the grapple up to half your speed. At the end of your movement, you choose which spaces creatures grappling you are in, as long as they stay adjacent to you.
\parhead{Pin} As a standard action, you can make a grapple attack to pin a foe. If you succeed, that creature becomes pinned (in addition to being grappled), while you remain only grappled. The only physical action a pinned creature can take is to escape the grapple, though it can take mental actions. At your discretion, you can cover the mouth of a pinned creature to prevent it from speaking. 

\subsection{Invisibility}
If it is impossible to see your target, you can't attack him normally. However, you can attack a square that you think he occupies. An attack into a square occupied by an invisible enemy has a 50\% miss chance. If an adjacent invisible creature strikes you, you can automatically identify the square he occupied when he struck you.

You can't execute an attack of opportunity against an invisible opponent, even if you know what square or squares the opponent occupies.

\subsection{Overwhelm}\label{Overwhelm}
When a creature is being attacked by multiple foes at once, it is less able to defend itself. A creature is considered overwhelmed if it is being threatened by more than one creature. Multiple creatures occupying the same square count as a single creature when determining overwhelm penalties. If a creature is overwhelmed, it takes a penalty to physical defenses equal to the number of creatures threatening it.

\subsection{Total Defense}\label{Total Defense}
As a standard action, you can focus entirely on defense, granting you a \plus4 bonus to your dodge modifier for 1 round. While using the total defense action, you can't make attacks of opportunity, but you still threaten squares normally for the purpose of overwhelm penalties and similar effects.

\subsection{Helpless Defenders}
A helpless creature is completely at an opponent's mercy. Its physical defenses are equal to 10 \add its size modifier. Paralyzed, bound, and unconscious creatures are helpless, as well as creatures completely unaware of an attack.

\parhead{Coup de Grace}\label{Coup de Grace} As a full-round action, you can deliver a coup de grace to a helpless opponent adjacent to you. You automatically hit with your weapon and score a critical hit. If the damage exceeds the struck creature's Fortitude defense, it immediately dies.

Delivering a coup de grace requires focused concentration and methodical action. This leaves you defenseless, causing you to provoke attacks of opportunity. If you take damage in excess of your Fortitude defense while delivering a coup de grace, your attempt fails.

You can't deliver a coup de grace against a creature that is immune to critical hits. You can deliver a coup de grace against a creature with total concealment, but doing this requires two consecutive full-round actions (one to ``find'' the creature once you've determined what square it's in, and one to deliver the coup de grace).

\subsection{Range Increments}
When using a ranged weapon, you take a \minus2 penalty per range increment between you and your target. For example, when using a longbow with a range increment of 100 feet against a target 170 feet away, you take a \minus2 penalty to attack rolls.

\section{Special Actions}

\subsection{Partial Actions}

If you are restricted to only taking a move or standard action, but not both, you can spend a standard action to perform a partial version of a full-round action. For most actions, you spend the first round starting the action, and use a second standard action to complete it. Some full-round actions have specific partial versions described below which you take instead. You can only take these partial actions when you cannot take full-round actions. 

\subsubsection{Partial Charge}
As a standard action, you can move up to your speed and make a single attack against a foe. You must move in a straight line, and your movement must not be impeded in any way. In addition, you take a \minus2 penalty to physical defenses until the start of your next turn. An interrupted partial charge becomes a move action.

\subsection{Delay}\label{Delay}
By choosing to delay, you take no action and then act normally on whatever initiative count you decide to act. When you delay, you voluntarily reduce your own initiative result for the rest of the combat. When your new, lower initiative count comes up later in the same round, you can act normally. You can specify this new initiative result or just wait until some time later in the round and act then, thus fixing your new initiative count at that point.

You never get back the time you spend waiting to see what's going to happen. Additionally, you can't interrupt anyone else's action (as you can with a readied action).

\parhead{Initiative Consequences of Delaying} Your initiative result becomes the count on which you took the delayed action. If you come to your next action and have not yet performed an action, you don't get to take a delayed action (though you can delay again).

If you take a delayed action in the next round, before your regular turn comes up, your initiative count rises to that new point in the order of battle, and you do not get your regular action that round.

\subsection{Ready}\label{Ready}
The ready action lets you prepare to take an action later, after your turn is over but before your next one has begun. Readying is a standard action. It does not provoke an attack of opportunity (though the action that you ready might do so).

\parhead{Readying an Action} You can ready a standard action or a move action. To do so, specify the action you will take and the conditions under which you will take it. You cannot ready in response to an action that you take -- the action must be outside of your control. Then, any time before your next action, you may take the readied action in response to that condition. The action occurs just before the action that triggers it. If the triggered action is part of another character's activities, you interrupt the other character. Assuming he is still capable of doing so, he continues his actions once you complete your readied action. Your initiative result changes. For the rest of the encounter, your initiative result is the count on which you took the readied action, and you act immediately ahead of the character whose action triggered your readied action.

You can take some free actions as part of your readied action, but in general you can take no actions other than the action you readied.

\parhead{Initiative Consequences of Readying} Your initiative result becomes the count on which you took the readied action. If you come to your next action and have not yet performed your readied action, you don't get to take the readied action (though you can ready the same action again). If you take your readied action in the next round, before your regular turn comes up, your initiative count rises to that new point in the order of battle, and you do not get your regular action that round.

\parhead{Distracting Spellcasters} You can ready a full attack against a spellcaster with the trigger ``if she starts casting a spell." If you damage
the spellcaster, she may lose the spell she was trying to cast (as determined by her Concentration check result).

\parhead{Readying to Counterspell} You may ready a counterspell against a spellcaster (often with the trigger ``if she starts casting a spell"). In this case, when the spellcaster starts a spell, you get a chance to identify it with a Spellcraft check (DC 15 \add spell level). If you do, and if you can cast that same spell (are able to cast it and have it prepared, if you prepare spells), you can cast the spell as a counterspell and automatically ruin the other spellcaster's spell. Counterspelling works even if one spell is divine and the other arcane.

A spellcaster can use \spell{dispel magic} to counterspell another spellcaster, but it doesn't always work.

\parhead{Readying a Weapon against a Charge} You can ready certain piercing weapons, setting them to receive charges (see \trefcp{Weapons}). A readied weapon of this type deals double damage on your first attack with it against a charging character.

\section{Special Rules}\label{Special Rules}

\subsection{Critical Success and Failure}\label{Critical Success and Failure}
A natural 1 (the d20 comes up 1) on an attack roll is treated as rolling a \minus10. A natural 20 (the d20 comes up 20) is treated as rolling a 30. Under normal circumstances, a natural 1 automatically misses, and a natural 20 automatically hits.

\subsubsection{Critical Hits}\label{Critical Hits}
When you roll a natural 20 on an attack roll and hit, you have scored a critical threat. Roll another attack roll at the same attack bonus. If that attack also hits, you deal double damage.

Keen weapons threaten a critical hit on a 19 or 20, while impact weapons deal triple damage when they score a critical hit.

\subsection{Doubling}\label{Doubling}
If you double any in-game value twice, it becomes three times as large. An additional doubling would make it four times as large, and so on. For example, if you make an attack that deals double damage and you get a critical hit, you deal triple damage.

\parhead{Real-World Values} Values that are ``real'', such as movement and distance, are an exception. If you double a real-world value twice, it becomes four times as large. 

\subsection{Size in Combat}\label{Size in Combat}
Size affects your space and reach in combat. In addition, your physical attacks and defefenses are affected by your size modifier. These effects are shown on \trefnp{Size in Combat}. 

\begin{dtable*}
    \lcaption{Size in Combat}
    \begin{tabularx}{\textwidth}{l l l l l X}
        \thead{Size} & \thead{Space\fn{1}} & \thead{Reach\fn{1}} & \thead{Size Modifier\fn{2}} & \thead{Special Size Modifier\fn{3}} & \thead{Example Creature} \\
        Fine & 1/2 ft. & 0 & \plus8 & \minus16 & Fly\\
        Diminutive & 1 ft. & 0 & \plus4 & \minus12 & Toad \\
        Tiny & 2-1/2 ft. & 0 & \plus2 & \minus8 & Cat \\
        Small & 5 ft. & 5 ft. & \plus1 & \minus4 & Halfling \\
        Medium & 5 ft. & 5 ft. & \plus0 & \plus0 & Human \\
        Large (tall) & 10 ft. & 10 ft. & \minus1 & \plus4 & Ogre \\
        Large (long) & 10 ft. & 5 ft. & \minus1 & \plus4 & Horse \\
        Huge (tall) & 15 ft. & 15 ft. & \minus2 & \plus8 & Cloud giant \\
        Huge (long) & 15 ft. & 10 ft. & \minus2 & \plus8 & Bulette \\
        Gargantuan (tall) & 20 ft. & 20 ft. & \minus4 & \plus12 & 50-ft. animated statue \\
        Gargantuan (long) & 20 ft. & 15 ft. & \minus4 & \plus12 & Kraken \\
        Colossal (tall) & 30\add ft. & 30\add ft. & \minus8 & \plus16 & Colossal animated object \\
        Colossal (long) & 30\add ft. & 20\add ft. & \minus8 & \plus16 & Great wyrm red dragon \\
    \end{tabularx}
    1 Creatures can vary in space and reach. These are simply typical values. \\
    2 Bonus to physical attacks and defenses other than for maneuvers \\
    3 Penalty to maneuver attack and defense, bonus to Stealth \\
\end{dtable*}

Unusually large or small creatures also have other special rules apply to them.

\subsubsection{Very Small Creatures}
\parhead{Space} If a creature takes up less than a single square of space, you can fit multiple creatures in that square. You can fit four Tiny creatures in a square, twenty-five Diminuitive creatures, or 100 Fine creatures.

\parhead{Reach} Creatures that take up less than 1 square of space typically have a natural reach of 0 feet, meaning they can't reach into adjacent squares. They must enter an opponent's square to attack in melee. This provokes an attack of opportunity from the opponent. You can attack into your own square if you need to, so you can attack such creatures normally. Since they have no natural reach, they do not threaten the squares around them, allowing you to move past them without provoking attacks of opportunity.

If a creature without a natural reach uses a reach weapon, it gains no benefits or penalties.

\parhead{Movement} Creatures three size categories smaller than you are not considered obstacles and do not hinder your movement.

\parhead{Stealth} Small creatures gain a bonus to Stealth checks equal to their special size modifier.

\subsubsection{Very Large Creatures}
\parhead{Space} Very large creatures take up multiple squares. Anything which affects a single square the creature occupies affects the creature. 

\parhead{Reach} Creatures that take up more than 1 square typically have a natural reach of 10 feet or more, meaning that they can reach targets even if they aren't in adjacent squares. Creatures with a large natural reach can attack anyone within their reach, including adjacent foes.

Creatures with a large natural reach using reach weapons can strike at up to double their natural reach but can't strike at their natural reach or less, just like Medium sized creatures.

\parhead{Movement} Creatures three size categories larger than you are not considered obstacles and do not hinder your movement.

\parhead{Stealth} Large creatures take a penalty to Stealth checks equal to their special size modifier.

\parhead{Immunities} Creatures at least three size categories larger than you are difficult to fight. You cannot score critical hits or contribute to overwhelm penalties against such creatures. If you can reach a vulnerable point on the creature, such as by flying or by using ranged weapons, you can score critical hits, but you still do not contribute to overwhelm penalties.

\subsection{Mounted Combat}\label{Mounted Combat}
\parhead{Horses in Combat} Warhorses and warponies can serve readily as combat steeds. Light horses, ponies, and heavy horses, however, are frightened by combat. If you don't dismount, you must make a DC 20 Ride check each round as a move action to control such a horse. If you succeed, you can perform a standard action after the move action. If you fail, you can't do anything else until your next turn.

Your mount acts on your initiative count as you direct it. You move at its speed, but the mount uses its action to move.

\parhead{Space} A horse (not a pony) is a Large creature, and thus takes up a space 10 feet (2 squares) across. While mounted, you share your mount's space completely. Anyone threatening your mount can attack either you or your mount. However, your reach is still that of a creature of your normal size. Thus, a Medium paladin would threaten all squares adjacent to his Large horse with a longsword, and all squares 10 feet away from his mount with a lance.

In the case of abnormally large mounts (two or more size categories larger than you), you may not completely share space. Such situations should be handled on a case-by-case basis, depending on the nature of the mount.

\parhead{Combat while Mounted} With a DC 5 Ride check, you can guide your mount with your knees so as to use both hands to attack or defend yourself. This is a free action.

If your mount moves more than its normal speed when charging, you also take the physical defense penalty associated with a charge. If you make an attack at the end of such a charge, you receive the bonus gained from the charge. When charging on horse-back, you deal double damage with a lance (see \pcref{Charge}).

You can use ranged weapons while your mount is taking a double move, but at a \minus4 penalty on the attack roll. You can use ranged weapons while your mount is sprinting, but at a \minus8 penalty (see \pcref{Sprint}). In either case, you make the attack roll when your mount has completed half its movement. You can make a full attack with a ranged weapon while your mount is moving. Likewise, you can take move actions normally.

\parhead{Casting Spells while Mounted} You can cast a spell normally if your mount moves up to a normal move (its speed) either before or after you cast. If you have your mount move both before and after you cast a spell, then you're casting the spell while the mount is moving, and you have to make a Concentration check due to the vigorous motion (DC 10 \add double spell level) or lose the spell. If the mount is sprinting, your Concentration check is more difficult due to the violent motion (DC 15 \add double spell level).

\parhead{If Your Mount Falls in Battle} If your mount falls, you have to succeed on a DC 15 Ride check to make a soft fall and take no damage. If the check fails, you take 1d6 points of damage.

\parhead{If You Are Dropped} If you are knocked unconscious, you have a 50\% chance to stay in the saddle (or 75\% if you're in a military saddle). Otherwise you fall and take 1d6 points of damage. Without you to guide it, most mounts avoid combat.

\subsection{Two-Weapon Fighting}\label{Two-Weapon Fighting}
If you can wield two weapons at once, you can attack with both weapons at once whenever you attack. Roll a single attack roll for both weapons, designating one weapon as your primary weapon and the other as your secondary weapon. Apply your attack bonuses with each of your weapons separately, taking a \minus5 penalty with your secondary attack. If you hit with your primary weapon, you deal damage with it. If you hit with your secondary weapon (after taking into account the \minus5 penalty), you also deal damage with it. You only apply your Strength to your primary weapon.

\par Precision-based damage, such as sneak attack damage, is only dealt once. It is possible to critical with both weapons. Use each weapon's critical threat range separately, but roll only once to confirm a critical threat, using the same attack bonus as with the original attack. Damage reduction only applies once against the damage dealt by both weapons.

\par Fighting in this way is difficult, and you suffer a \minus2 penalty to your attack roll. You can mitigate this penalty if both your weapons are light. The Two-Weapon Fighting feat grants a \plus2 bonus to attack rolls when fighting with two weapons at once (see Two-Weapon Fighting, \pref{feat:Two-Weapon Fighting}).

You take no penalties for alternating attacks between two (or more) weapons, as long as you do not attack with both weapons at once. Normally, you can't make unarmed attacks as if fighting with two weapons. However, monks and other characters who have the special ability to treat multiple parts of their body as weapons can use two-weapon fighting

\par For example, Felix the 1st-level fighter is wielding a longsword and a short sword against an evil goblin. The goblin has a physical defense of 15. Felix has a Strength of 3 and a base attack bonus of \plus1. This means his attack bonus with either weapon individually is \plus4. If he attacks with both weapons at once, he takes no penalty to his attack roll (because his off-hand weapon is light), but his attack with his off-hand weapon takes a \minus5 penalty. So his attack bonus would be \plus4 (with his longsword) and \minus1 (with his short sword). If he rolls a 15, he will hit the goblin with his longsword, but not with his short sword.

\par If Felix had the Two-Weapon Fighting feat, his attack bonus would be \plus6 with his longsword and \plus1 with his short sword. Assuming he rolls a 15 again, he would hit the goblin with both weapons, dealing damage with both of them.

\subsection{Unarmed Combat}
Every creature can attack with its body using an unarmed strike. You cannot defend yourself with an unarmed strike, so you may provoke attacks of opportunity (see \pcref{Attacks of Opportunity}). You are only considered to have one unarmed strike, so you cannot two-weapon fight with only your unarmed strike (but see the unarmed warrior monk ability, \pref{Mnk:Unarmed Warrior}).

If you have the Improved Unarmed Strike feat, you treat your unarmed strike as if it were a lethal weapon. This allows you to defend yourself and takes attacks of opportunity while unarmed, just as if you were using another melee weapon.
