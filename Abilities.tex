\chapter{Attributes}
Each character has six attributes: Strength (Str), Dexterity (Dex), Constitution (Con), Intelligence (Int), Wisdom (Wis), and Charisma (Cha). These attributes represent a character's raw talent in that area. A 0 in an attribute represents average human capacity. That doesn't mean that every commoner has a 0 in every attribute; not everyone is average, after all.

\section{Using Attributes}
When you do something related to an attribute, you usually add the attribute score to your roll. Attributes are also used for other things, such as determining how much you can carry or how difficult you are to kill. 

\subsection{Attributes and Spellcasters}
Using magic requires a strong mind. The attribute that governs spellcasting depends on what type of spellcaster your character is: Intelligence for wizards; Wisdom for druids; or Charisma for paladins, sorcerers, bards, and most clerics. That attribute is called your casting attribute. In order to cast spells of a given level, your casting attribute must be at least equal to the spell's level. For example, a sorcerer with a 4 Charisma can cast up to 4th level spells. In addition to having a high attribute score, a spellcaster must be of high enough class level to be able to cast spells of a given spell level. (See the class descriptions for details.)

\subsection{Attribute Limits}
Your attribute scores can never exceed 10. Magical and extraordinary creatures can exceed this limitation, but ordinary mortals are limited by their physical form.

\section{The Attributes}
Each attribute partially describes your character and affects some of his or her actions.

\subsection{Strength (Str)}
Strength measures your character's muscle and physical power. This attribute is especially important for fighters, barbarians, paladins, rangers, and monks because it helps them prevail in combat. Strength also limits the amount of equipment your character can carry.

You apply your character's Strength to:
\begin{itemize*}
\item Attack rolls with medium and heavy weapons, including medium and heavy thrown weapons.
\item Climb, Jump, and Swim checks. These are the skills that have Strength as their key attribute.
\item Strength checks (for breaking down doors and the like).
\end{itemize*}
You apply half your character's Strength to:
\begin{itemize*}
\item Fortitude saving throws to resist poisons and similar effects.
\item Damage rolls when using weapons. (\emph{Note:} Some weapons, such as crossbows, do not apply Strength to damage.)
\item Skill points for skills which have Strength as their key attribute.
\end{itemize*}
\par A character needs a 3 Strength to wield medium weapons in one hand, or to wield heavy weapons effectively at all.

\subsection{Dexterity (Dex)}
Dexterity measures hand-eye coordination, agility, and reflexes. This attribute is the most important attribute for rogues, but it's also high on the list for characters who typically wear light or medium armor (rangers and barbarians) or no armor at all (monks, wizards, and sorcerers), and for anyone who wants to be a skilled archer.

You apply your character's Dexterity to:
\begin{itemize*}
\item Ranged attack rolls, except for those made with medium or heavy thrown weapons.
\item Attack rolls with light weapons
\item Armor Class (AC), provided that the character can react to the attack.
\item Reflex saving throws, for avoiding fireballs and other attacks that you can escape by moving quickly.
\item Initiative checks to react first in combat (along with Wisdom).
\item Acrobatics, Escape Artist, Ride, Sleight of Hand, and Stealth.  These are the skills that have Dexterity as their key attribute.
\end{itemize*}
You apply your half character's Dexterity to:
\begin{itemize*}
\item Skill points for skills which have Dexterity as their key attribute.
\end{itemize*}

\subsection{Constitution (Con)}
Constitution represents your character's health and stamina. A Constitution bonus increases a character's hit points, so the attribute is important for all classes.

You apply your character's Constitution to:
\begin{itemize*}
\item Fortitude saving throws, for resisting poison and similar threats.
\item Concentration checks. Concentration is important to spellcasters.
\end{itemize*}
You apply half your character's Constitution to:
\begin{itemize*}
\item The hit points gained when a character gains a level. If a character's Constitution score changes enough to alter his or her Constitution value, the character's hit points also increase or decrease accordingly.
\item Skill points for skills which have Constitution as their key attribute.
\end{itemize*}

\subsection{Intelligence (Int)}
Intelligence determines how well your character learns and reasons. This attribute is important for wizards because it affects how many spells they can cast, how hard their spells are to resist, and how powerful their spells can be.

You apply your character's Intelligence to:
\begin{itemize*}
\item Craft, Disable Device, Forgery, Knowledge, and Linguistics checks. These are the skills that have Intelligence as their key attribute.
\end{itemize*}

\par You apply half your character's Intelligence to:
\begin{itemize*}
\item Will saving throws to resist mental influence.
\item The number of languages your character knows at the start of the game.
\item Skill points for skills which have Intelligence as their key attribute.
\end{itemize*}

\par An animal has an Intelligence score of \minus5 or lower. A creature of humanlike intelligence has a score of at least \minus4.

\subsection{Wisdom (Wis)}
Wisdom describes a character's common sense, perception, and intuition. While Intelligence represents one's attribute to analyze information, Wisdom represents being in tune with and aware of one's surroundings. Wisdom is the most important attribute druids, and it is also important for rangers. If you want your character to be perceptive, put a high score in Wisdom. Every creature has a Wisdom score.

You apply your character's Wisdom to:
\begin{itemize*}
\item Heal, Listen, Perception, Profession, Spellcraft, and Survival checks. These are the skills that have Wisdom as their key attribute.
\end{itemize*}
You apply half your character's Wisdom to:
\begin{itemize*}
\item Reflex saving throws to recognize and avoid dangerous effects.
\item Initiative checks to react first in combat (along with Dexterity).
\item Skill points for skills which have Wisdom as their key attribute.
\end{itemize*}

\subsection{Charisma (Cha)}
Charisma measures a character's force of personality, willpower, persuasiveness, personal magnetism, and attribute to lead. This attribute represents actual strength of personality, not merely how one is perceived by others in a social setting. Charisma is most important for bards, most clerics, paladins, and sorcerers. Every creature has a Charisma score.

You apply your character's Charisma to:
\begin{itemize*}
\item Will saving throws to resist mental influence.
\item Bluff, Disguise, Handle Animal, Intimidate, Perform, and Persuasion checks. These are the skills that have Charisma as their key attribute.
\end{itemize*}
You apply your half character's Charisma to:
\begin{itemize*}
\item Skill points for skills which have Charisma as their key attribute.
\end{itemize*}

When an attribute score changes, all attributes associated with that score change accordingly. However, enhancement bonuses, circumstance bonuses, and penalties of any kind do not affect a character's skill points.

For information on how to determine your character's attribute scores, see Determining Attribute Scores, below.

%Better placed elsewhere. But where?
\section{Determining Attribute Scores}
There are several options for how to determine attribute scores.

\subsubsection{Predefined Attribute Scores}
This is the simplest method. Simply take the following set of attribute scores and distribute them as you choose among your character's abilities:

4, 3, 2, 1, 0, \minus1

This set of attribute scores is called the ``elite array''. For more extreme characters, you may use the ``savant array'':

5, 2, 1, 0, 0, \minus2.

Finally, for more well-balanced characters, you may use the ``balanced array'':

3, 3, 2, 1, 1, 0

\subsection{Point Buy}
With this method, you can fully control your character's attribute scores to match what you want your character to be. All your character's attribute scores start at 0. You get 10 points to distribute among your character's attribute scores. Attribute scores can be bought according to the costs on \trefnp{Attribute Score Point Costs}.

\begin{dtable}
\lcaption{Attribute Score Point Costs}
\begin{tabularx}{\columnwidth}{X X X X}
\thead{Attribute Score} & \thead{Point Cost} & \thead{Attribute Score} & \thead{Point Cost} \\
\minus2 & \minus2\fn{1} & 2 & 2 \\
\minus2 & \minus1\fn{1} & 3 & 3 \\
0 & 0 & 4 & 5 \\
1 & 1 & 5 & 8 \\
\end{tabularx}
1 No more than two attribute scores can be reduced below 0 in this way.
\end{dtable}

\subsection{Other Methods}
Point buy offers the fairest and most customizable system for determining attribute scores, ensuring that players can be almost any character they want to be. However, some groups may wish to determine attribute scores differently. Other options are provided below.

\subsubsection{Semi-Randomized Point Buy}
With this method, you have only a small degree of control over your character's attribute scores, but all characters generated in this way are equally powerful. As with the point buy method, all your character's attribute scores start at 0, and you get 10 points to distribute among your character's attribute scores. However, you do not have full control over how to distribute those points.

Roll 4d6 for each attribute score, dropping a die of your choice with each roll. First roll for the attribute scores that you care about most, and save the least important attribute scores for last. After rolling for an attribute score, sum results on the three highest dice and consult \trefnp{Semi-Randomized Point Buy Results} and spend the appropriate number of points to yield an attribute score, as indicated by \trefnp{Attribute Score Point Costs}. If you do not have enough points remaining to spend the amount indicated by the die roll, spend as many as you can and move on to the next ability.

If you have points remaining after rolling all of your attribute scores, you may distribute the points freely among your abilities, using the normal point buy rules. You cannot increase any attribute above 3 during this stage.

After all of your points have been spent, you may swap any two of your attribute scores.

\begin{dtable}
\lcaption{Semi-Randomized Point Buy Results}
\begin{tabularx}{\columnwidth}{X X X}
\thead{Roll} & \thead{Attribute Score} & \thead{Point Cost} \\
3-7   & \minus2 & \minus2\fn{1} \\
8-9   & \minus1 & \minus1\fn{1} \\
10-11 & 0  & 0 \\
12-13 & 1  & 1 \\
13-14 & 2  & 2 \\
15-16 & 3  & 3 \\
17    & 4  & 5 \\
18    & 5  & 8 \\
\end{tabularx}
1 You gain extra points for having low stats. You can gain these points any number of times per character. \\
\end{dtable}

For characters with more extreme attribute scores, use the following approach for each attribute score, starting with 10 points as normal:
\begin{itemize*}
  \item Roll 2d8
  \item Take the average, rounding down
  \item Subtract 3
  \item Spend the points as indicated on \trefnp{Attribute Score Point Costs} until you have no points left.
\end{itemize*}

\subsubsection{Random Point Buy}
This method gives you no control over the character whatsoever, while still ensuring that all characters generated are equally powerful. It functions as the semi-randomized point buy method, except that you also randomize the order in which the attribute scores are rolled.

\begin{comment}
\subsubsection{Weighted Semi-Randomized Point Buy}
This method gives you more control over your attribute scores, while still requiring you to deal with attribute score advantages or disadvantages that you might not have chosen. It functions like the semi-randomized point buy method, except that you do not roll 3d6 for every attribute score. You have a pool of 18 dice. You can distribute those dice as you choose between each attribute score, except the last one, which is automatically assigned your remaining points. If you roll more than three dice for an attribute score, keep only the highest 3 dice. If you assign fewer than three dice to roll for an attribute score, add 1 to the result for every die less than 3 you are rolling. You can roll as many as six dice for a single attribute score, and you cannot roll less than one die for an attribute score.

For example, player hoping to play a barbarian might assign six dice to Strength, five dice to Constitution, three dice to Dexterity, and two dice to each of Intelligence and Wisdom, leaving the rest of the points to Charisma.
\end{comment}

\subsubsection{Classic Hardcore}

This method is completely random and can generate very overpowered or underpowered chracters. It represents the unfairness of the world, where some people are just better or worse than others. Roll 1d8 for each attribute score and subtract 3 from each result. The result is the attribute score.
