\chapter{Feats}
Feats are special abilities that every character has. Feats can be used to specialize your character particular area, to grant your character new abilities, or to change the way your character does certain things.

\section{Gaining Feats}
Your character gains a feat every odd level: 1st, 3rd, 5th, and so on. Classes can sometimes grant bonus feats as well, which are in addition to these feats which every character gets.

\subsection{Bonus Feats}
Some class features and abilities grant a character bonus feats. Unless otherwise specified, the character must still meet any prerequisites for the feat. If the character does not meet the prerequisites at the time the bonus feat is granted, the character does not gain the feat. If the character later meets the prerequisites, the character immediately gains the benefit of the bonus feat.

If a character gains a feat as a bonus feat that he or she has already acquired through other means, the character may select instead any other feat for which she qualifies.

\section{Retraining Feats}
At every even level, your character can choose to retrain an old feat in exchange for a new feat. You can only retrain feats for other feats you could have acquired at the time you took the original feat. Thus, you cannot retrain feats gained through class features which give you a specific feat, since there were no other feats you could have taken. However, a 6th level fighter can retrain his 2nd level fighter bonus feat for any other combat feat that he qualified for at 2nd level.

\section{Prerequisites}
Some feats have prerequisites. Your character must have the indicated attribute score, class feature, feat, skill, base attack bonus, or other quality designated in order to select or use that feat. A character can gain a feat at the same level at which he or she gains the prerequisite.

A character can't use a feat if he or she has lost a prerequisite.

\section{Types Of Feats}
Some feats are general, meaning that no special rules govern
them as a group. Others belong to particular categories. The categories are given below:
\begin{itemize*}
\item Skill feats affect a character's mastery of his or her skills.
\item Combat feats affect a character's prowess in combat.
\item Combat maneuver feats are combat feats which grant a character particular talent with specific maneuvers.
\item Combat style feats grant a character the ability fight in particular styles. A character can only fight in one style at once.
\item Bloodline feats allow a character to tap into latent magical abilities in their blood.
\item Metamagic feats let spellcasters cast spells with greater effect, albeit as if the spells were a higher level than they actually are.
\item Item creation feats allow spellcasters to create magic items of various kinds.
\end{itemize*}

\subsection{Class Feats}
Class feats improve a character's class features.

\subsection{Racial feats}
Racial feats improve a character's racial abilities or grant new abilities unique to the character's race.

\subsection{Skill Feats}
Skill feats always affect a character's ability to use skills. Rogues can gain skill feats with their skill trick class feature.

\subsection{Combat Feats}
Combat feats always affect a character's combat abilities. Fighters gain bonus feats selected from a subset of the feat list presented in \tref{Combat Feats}. Any feat designated as a combat feat can be selected as a fighter's bonus feat.

\subsection{Combat Maneuver Feats}
Combat maneuver feats are combat feats which specifically deal with combat maneuvers -- pushing opponents around, grappling them, tripping them, or other unusual attacks. Any feat designated as a combat maneuver feat is also considered a combat feat and can be selected as a fighter's bonus feat.

\subsection{Combat Style Feats}
Combat style feats grant a character the ability to fight in a particular style, granting them bonuses while fighting in that style. A character can only fight in one style at once. Initiating a style or changing to a different style is a swift action, but a style can be stopped as a free action. Any feat designated as a combat maneuver feat is also considered a combat feat and can be selected as a fighter's bonus feat.

Some style feats have style requirements. To gain the benefits of a combat style with a style requirement, a character must take some action or fulfill some condition during his turn. If the condition is not met, the character automatically stops fighting in that style at the end of his turn. For example, a character using the Power Attack style must charge or full attack with a melee weapon each round. If the character takes the total defense action instead, he stops using the Power Attack style at the end of his turn.

\subsection{Bloodline Feats}

Some characters have traces of monstrous blood running in their veins. Most of those will never understand the full potential of their unusual heritage. Bloodline feats allow characters to explore those posibilities by gaining abilites related to their ancestry. Each bloodline feat belongs to a specific type of monster, such as ``dragon''. Some bloodline feats have stronger effects if you have more feats from that heritage.

\subsection{Metamagic Feats}
As a spellcaster's knowledge of magic grows, she can learn to cast spells in ways slightly different from the ways in which the spells were originally designed or learned. Preparing and casting a spell in such a way is harder than normal but, thanks to metamagic feats, at least it is possible. Spells modified by a metamagic feat use a spell slot higher than normal. All effects dependent on spell level (such as the ability to penetrate a \spell{lesser globe of invulnerability}) are calculated according to the spell's modified level.

\parhead{Applying Metamagic Feats} Spellcasters apply metamagic feats on the spot. Therefore, most spellcasters must also take more time to cast a metamagic spell (one enhanced by a metamagic feat) than they do to cast a regular spell. If the spell's normal casting time is a standard action, casting a metamagic version is a full-round action. (This isn't the same as a 1-round casting time.) For a spell with a longer casting time, it takes an extra full-round action to cast the spell. For a spell with a shorter casting time, it takes a standard action to cast the spell.

\par Sorcerers have such an intuitive grasp of magic that they do not need to take extra time to cast spells affected by metamagic feats.

\parhead{Spontaneous Casting and Metamagic Feats} A cleric spontaneously casting a \spell{cure} or \spell{inflict} spell can cast a metamagic version of it instead.

\parhead{Effects of Metamagic Feats on a Spell} In all ways, a metamagic spell operates at its original spell level, even though it is prepared and cast as a higher-level spell.

The modifications made by these feats only apply to spells cast directly by the feat user. A spellcaster can't use a metamagic feat to alter a spell being cast from a wand, scroll, or other device.

Metamagic feats that eliminate components of a spell don't eliminate the attack of opportunity provoked by casting a spell while threatened. However, casting a spell modified by Quicken Spell does not provoke an attack of opportunity.

Metamagic feats cannot be used with all spells. See the specific feat descriptions for the spells that a particular feat can't modify.

\parhead{Multiple Metamagic Feats on a Spell} A spellcaster can apply multiple metamagic feats to a single spell. Changes to its level are cumulative. You can't apply the same metamagic feat more than once to a single spell.

\parhead{Magic Items and Metamagic Spells} With the right item creation feat, you can store a metamagic version of a spell in a scroll, potion, or wand. Level limits for potions and wands apply to the spell's higher spell level (after the application of the metamagic feat). A character doesn't need the metamagic feat to activate an item storing a metamagic version of a spell.

\parhead{Counterspelling Metamagic Spells} Whether or not a spell has been enhanced by a metamagic feat does not affect its vulnerability to counterspelling or its ability to counterspell another spell.

\subsection{Item Creation Feats}
An item creation feat lets a spellcaster create a magic item of a certain type. Regardless of the type of items they involve, the various item creation feats all have certain features in common.

\parhead{Raw Materials Cost} The cost of creating a magic item equals one-half the sale cost of the item.

Using an item creation feat also requires access to a laboratory or magical workshop, special tools, and so on. A character generally has access to what he or she needs unless unusual circumstances apply.

\parhead{Negative Levels} Power and energy that a spellcaster would normally have is expended when making a magic item. While crafting a magic item, a spellcaster gains a negative level that cannot be removed. After the magic item is complete, the spellcaster still suffers the negative level for two days per day required to make the item. Scrolls and potions are less draining, and only bestow negative levels for one day per day required to make the item. These negative levels cannot be removed by any means.

\parhead{Time} The time to create a magic item depends on the feat and the cost of the item. The minimum time is one day.

\parhead{Item Cost} Brew Potion, Craft Wand, and Scribe Scroll create items that directly reproduce spell effects, and the power of these items depends on their caster level -- that is, a spell from such an item has the power it would have if cast by a spellcaster of that level. The price of these items (and thus the cost of the raw materials) also depends on the caster level. The caster level must be high enough that the spellcaster creating the item can cast the spell at that level. To find the final price in each case, multiply the caster level by the spell level, then multiply the result by a constant, as shown below:

\subparhead{Scrolls} Base price = spell level \mtimes caster level \mtimes 25 gp.
\subparhead{Potions} Base price = spell level \mtimes  caster level \mtimes 50 gp.
\subparhead{Wands} Base price = spell level \mtimes  caster level \mtimes 750 gp.

\par A 0-level spell is considered to have a spell level of 1/2 for the
purpose of this calculation. The minimum caster level to craft an item is always 2, regardless of the level of the spell used.

\parhead{Extra Costs} Any potion, scroll, or wand that stores a spell with a costly material component also carries a commensurate cost. For potions and scrolls, the creator must expend the material component when creating the item. For a wand, the creator must expend fifty copies of the material component.

\par Some magic items similarly incur extra costs in material components, as noted in their descriptions.

\begin{dtable!*}
\lcaption{General Feats}
\begin{tabularx}{\textwidth}{>{\lcol}p{15em} >{\lcol}p{15em} >{\lcol}X}
\thead{General Feats} & \thead{Prerequisites} & \thead{Benefit} \\
%Brute Fortitude & \x & Use Str and half Con for Fortitute saves \\
Combat Casting  & \x &  \plus2 bonus on Concentration checks to cast spells, reroll 1/day \\
Endurance & Con 3 & Convert damage to nonlethal damage \\
Deathless & Con 5 or base Fortitude save \plus10 & Immune to death effects \\
Diehard & Con 3 & Remain conscious after taking critical damage \\
%Eschew Materials  & \x &  Cast spells without material components \\
%Force of Intellect & \x & Use Int and half Cha for Will saves \\
Fearless & Cha 5 or base Will save \plus10 & Immune to fear and hostile morale effects \\
Great Fortitude  & \x &  \plus2 bonus on Fortitude saves, reroll 1/day \\
Improved Counterspell  & \x &  Counterspell with spell of same school \\
%Intuitive Reflexes & \x & Use Wis and half Dex for Reflex saves \\
Iron Will  & \x &  \plus2 bonus on Will saves, reroll 1/day \\
Lightning Reflexes  & \x &  \plus2 bonus on Reflex saves, reroll 1/day \\
Magical Synthesis & Levels in two magical classes & Gain magic levels in two magical classes at once \\
Mental Fortress & Cha 9 or base Will save \plus18 & Immune to hostile mind-affecting effects \\
Perfect Health & Con 3 or base Fortitude save \plus6 & Immune to disease, later poison \\
Rapid Metamagic & Spellcraft 8 ranks, ability to cast spells, one metamagic feat & Apply metamagic effects more quickly \\
Retributive Counterspell & Ability to cast 4th level spells & Countered spells rebound on original caster \\
Ritual Caster & Int 3 & Gain ability to perform rituals \\
Ritual Master & Spellcraft 8 ranks, ability to cast spells & Perform rituals more quickly, \plus3 to ritual checks \\
Spell Focus\fn{1} & Magic level 4th &  \plus2 caster level with specific type of magic \\
\tind Augment Summoning & Spell Focus (conjuration) & Summoned creatures gain \plus2 Str, \plus2 Con \\
\tind Spell Specialization & Magic level 8th, Spell Focus &  \plus4 caster level with specific type of magic, \minus2 penalty with other types. \\
%Spell Penetration  & Caster level 4th &  \plus2 bonus on caster level checks to defeat spell resistance \\
%\tind Greater Spell Penetration & Caster level 10th, Spell Penetration & \plus4 bonus to caster level checks to defeat spell resistance \\
Swift & \x & Increase speed by 5 feet \\
Toughness & \x &  \plus3 hit points \plus1 per level above 3 \\
\end{tabularx}
1 You can gain this feat multiple times. Its effects do not stack. Each time you take the feat, it has a different effect. \\
\end{dtable!*}

\begin{dtable!*}
\lcaption{Class Feats}
\begin{tabularx}{\textwidth}{>{\lcol}p{15em} >{\lcol}p{15em} >{\lcol}X}
\thead{Class Feats} & \thead{Prerequisites} & \thead{Benefit} \\
Extend Rage & Ability to rage & Can rage for 5 rounds longer \\
Extra Channeling & Ability to channel energy & Can turn or rebuke 3 more times per day \\
Extra Invocation & Ability to use an arcane invocation & Learn a new arcane invocation \\
Extra Music & Bardic music & Can use bardic music 3 more times per day \\
Extra Rage & Ability to rage & Can rage 1 more time per day\\
Extra Smiting\fn{1} & Ability to smite & Can smite 3 more times per day \\
Extra Stunning & Stunning Fist & Can use stunning fist 3 more times per day \\
Extra Wild Aspect & Wild aspect & Can use wild aspects 3 more times per day \\
Improved Channeling & Channel energy 3d6 & \plus2 level when channeling energy \\
Selective Channeling & Ability to channel energy & Can exclude two additional creatures \\
\end{tabularx}
\end{dtable!*}

\begin{dtable!*}
\lcaption{Racial Feats}
\begin{tabularx}{\textwidth}{>{\lcol}p{15em} >{\lcol}p{15em} >{\lcol}X}
\thead{Racial Feats} & \thead{Prerequisites} & \thead{Benefit} \\
Focused Mind & Elf & Use Intelligence to concentrate instead of Constitution \\
Giantfighter & Dwarf, gnome, or halfling & \plus2 to dodge against Large or larger creatures \\
Gnomish Tricks & Gnome, Cha 0 & Gain minor spell-like abilities \\
Keen Senses & Elf & \plus4 to Perception, automatically notice secret doors \\
Light-Footed & Elf or halfling & \plus4 to Stealth and become harder to track \\
Stonecunning & Dwarf & Gain a sixth sense about stonework \\
\end{tabularx}
\end{dtable!*}

\begin{dtable!*}
\lcaption{Skill Feats}
\begin{tabularx}{\textwidth}{>{\lcol}p{15em} >{\lcol}p{15em} >{\lcol}X}
\thead{Skill Feats} & \thead{Prerequisites} & \thead{Benefit} \\
Dilettante & Int 3 & Use some Knowledge skills despite being untrained \\
Legendary Balance & Acrobatics 13 ranks & Balance on impossible surfaces \\
Legendary Climber & Climb 13 ranks & Gain climb speed, climb on impossible surfaces \\
Legendary Craftsman & Craft 13 ranks & Craft items with fewer material components \\
Legendary Devicesmith & Devices 13 ranks & Disable active spell effects \\
Legendary Disguise & Disguise 13 ranks & Alter magical auras with disguise \\
Legendary Escapist & Escape Artist 13 ranks & Escape from magical effects \\
Legendary Liar & Bluff 13 ranks & Lies become undetectable by magic \\
Legendary Tumbler & Acrobatics 13 ranks & Tumble through enemies without provoking \\
Open Minded & \x & Gain two skill points. \\
Ranged Legerdemain & Ability to cast 2nd level spells & Use Disable Device or Sleight of Hand at range \\
Skill Focus\fn{1} & \x &  \plus3 bonus on checks with selected skill, reroll 1/day \\
Skill Training\fn{1} & \x & Gain two skills as class skills \\
Track  & \x &  Use Survival skill to track \\
Veteran's Knowledge & Base attack bonus \plus8 & Identify monsters without Knowledge \\
\end{tabularx}
1 You can gain this feat multiple times. Its effects do not stack. Each time you take the feat, it has a different effect. \\
\end{dtable!*}

\begin{dtable!*}
\lcaption{Combat Feats}
\begin{tabularx}{\textwidth}{>{\lcol}p{10em} >{\lcol}p{10em} >{\lcol}X >{\lcol}p{10em}}
    \thead{Combat Feats} & \thead{Prerequisites} & \thead{Benefit} & \thead{Feat Type} \\
Acrobatic Charge & Acrobatics 8 ranks & Charge on difficult terrain & Mobility \\
Active Defense & Base attack bonus \plus4 & Take attacks of opportunity normally while fighting defensively & Reaction \\
Armor Familiarity & Proficiency with armor & Reduce penalties from wearing armor & Equipment \\
Armor Proficiency (light) & \x &  No armor check penalty on attack rolls & Equipment \\
\tind Armor Proficiency (medium) & Armor Proficiency (light) & No armor check penalty on attack rolls & Equipment \\
\tind \tind Armor Proficiency (heavy) & Armor Proficiency (medium) & No armor check penalty on attack rolls & Equipment \\
Blind-Fight & \x &  Reroll miss chance for concealment in melee & Awareness, Style \\
Bulwark of Defense & Base attack bonus \plus4 & Your threatened area is difficult terrain & Style \\
Cautious Attack & \x & Trade damage for CMD & Defense, Style \\
Chargebreaker & Base attack bonus \plus4 & Gain \plus2 to attack and damage when readying against approaching foe & Reaction \\
Cleave & Str 3, base attack bonus \plus4 & Extra melee attack after dropping target & Power \\
\tind Cleaving Stride & Str 3, base attack bonus \plus8, Cleave & Continue movement after dropping target & Power \\
Combat Expertise & Int 3 & Trade attack bonus for AC & Defense, Style \\
Contingent Attack & Int 5, base attack bonus \plus12 & Prepare to attack as immediate action & Reaction \\
Contingent Counter & Int 3, base attack bonus \plus8 & Prepare to attack when attack misses you & Reaction \\
Counterstorm & Base attack bonus \plus16 & Foes that miss you provoke & Reaction, Style\\
Covering Fire & \x & Targets struck by ranged attacks suffer attack penalties & Precision, Style \\
Deadly Aim & Dex 3 & Trade ranged attack bonus for damage & Precision, Style \\
Defensive Stance & Base attack bonus \plus4 & Trade ability to move for CMD & Defense, Style \\
\tind Improved Defensive Stance & Base attack bonus \plus12 & Trade ability to move for AC & Defense, Style \\
Distracting Foe & Base attack bonus \plus4 & Threatened foes suffer Concentration penalties & Style \\
Dodge & Dex 3 & \plus4 AC against some attacks of opportunity from selected target & Defense, Mobility \\
\tind Mobility & Dodge & Avoid some attacks of opportunity from selected target & Defense, Mobility \\
\tind \tind Shot on the Run & Dex 3, Dodge, Mobility, base attack bonus \plus4 & Move before and after ranged attacks & Mobility, Style \\
\tind \tind Spring Attack & Dex 3, Dodge, Mobility, base attack bonus \plus4 & Move before and after melee attacks & Mobility, Style \\
\end{tabularx}
\end{dtable!*}

\begin{dtable!*}
\lcaption{Combat Feats (cont.)}
\begin{tabularx}{\textwidth}{>{\lcol}p{10em} >{\lcol}p{10em} >{\lcol}X >{\lcol}p{10em}}
    \thead{Combat Feats} & \thead{Prerequisites} & \thead{Benefit} & \thead{Feat Type} \\
Executioner & Base attack bonus \plus12 & Gain free attacks against foes near death & Style \\
Exotic Weapon Proficiency\footnotetemp{1} & Base attack bonus \plus1 & Don't provoke when attacking with exotic weapons & Equipment \\
Eye of the Storm & Base attack bonus \plus4 & You are more difficult to overwhelm & Awareness, Defense, Style \\
Far Shot & \x & Increase range increment by 50\% or 100\% & Precision \\
Feign Weakness & Base attack bonus \plus4 & Provoke attack to render foe flat-footed \\
Guardian & \x & Adjacent allies suffer reduced overwhelm penalties. & Defense, Style \\
Heartseeker & Base attack bonus \plus8 & Double critical threat range & Style \\
Improved Bull Rush & Base attack bonus \plus4 & \plus2 bonus on bull rush attacks; push target back without moving & Maneuver, Power \\
Improved Dirty Trick & Base attack bonus \plus4 & \plus2 bonus on dirty trick attacks; dirty tricks last longer & Finesse, Maneuver \\
Improved Disarm & Base attack bonus \plus4 & \plus2 bonus on disarm attacks; knock foe's weapon away after disarming & Finesse, Maneuver \\
Improved Feint & Base attack bonus \plus4 & \plus2 bonus on feint attacks; feint as move action & Finesse, Maneuver \\
Improved Grapple & Base attack bonus \plus4 & \plus2 bonus on grapple attacks; attack faster in grapple & Maneuver, Power \\
Improved Initiative & \x &  \plus4 bonus on initiative checks & Reaction \\
Improved Overrun & Base attack bonus \plus4 & \plus2 bonus on overrun attacks; target can't avoid & Maneuver, Power \\
Improved Trip & Base attack bonus \plus4 & \plus2 bonus on trip attacks; tripped foe provokes attacks of opportunity & Finesse, Maneuver \\
Improved Unarmed Strike & \x &  Considered armed even when unarmed & \x \\
\tind Deflect Arrows & Dex 3, Improved Unarmed Strike & Deflect one ranged attack per round & Defense, Reaction \\
\tind \tind Snatch Arrows & Dex 5, Deflect Arrows, Improved Unarmed Strike & Catch a deflected ranged attack & Defense, Reaction \\
\tind Stunning Fist & Dex 3, Wis 3, Improved Unarmed Strike, base attack bonus \plus4 & Stun opponent with unarmed strike & Power \\
Inescapable Bulwark of Defense & Base attack bonus \plus8, Bulwark of Defense & Enemies cannot avoid provoking attacks of opportunity & Reaction, Style \\
Intimidating Stance & Intimidate 8 ranks & Swiftly intimidate struck foe & Skill, Style\\
Legendary Awareness & Base attack bonus \plus12, any three Awareness feats & Immune to overwhelm & Awareness \\
Legendary Finesse & Base attack bonus \plus12, any three Finesse feats & Add half Dexterity to damage & Finesse \\
Legendary Maneuver Master & Base attack bonus \plus12, any three Maneuver feats & Never provoke, deal damage with very successful maneuvers & Maneuver \\ 
Legendary Mobility & Base attack bonus \plus12, any three Mobility feats & Movement does not provoke & Mobility\\
Legendary Mounted Warrior & Base attack bonus \plus12, any three Mounted feats & Share damage with mount & Mounted\\
Legendary Power & Base attack bonus \plus12, any three Power feats & Wield weapons as if they were less encumbering & Power\\ 
Legendary Precision & Base attack bonus \plus12, any three Precision feats & Very accurate hits deal maximum damage & Precision\\ 
Legendary Style & Base attack bonus \plus12, any three Style feats & Use two styles at once & \x \\
\end{tabularx}
\end{dtable!*}

\begin{dtable!*}
\lcaption{Combat Feats (cont.)}
\begin{tabularx}{\textwidth}{>{\lcol}p{10em} >{\lcol}p{10em} >{\lcol}X >{\lcol}p{10em}}
    \thead{Combat Feats} & \thead{Prerequisites} & \thead{Benefit} & \thead{Feat Type} \\
Manyshot & Dex 7, base attack bonus \plus11 & Shoot two projectiles simultaneously & Precision, Style \\
Master Tactician & Int 3, base attack bonus \plus12 & Ready full-round actions with allies & Reaction \\
Mounted Archery & Ride 1 rank & Reduced penalty for ranged attacks while mounted by 4 & Mounted, Precision \\
Mounted Combat & Ride 1 rank & Negate hits on mount with Ride check & Defense, Mounted \\
Opportunist & Dex 3 & \plus2 to attack and damage on attacks of opportunity & Finesse, Style \\
Overpowering Assault & Str 3 & Trade AC for combat maneuver bonus & Power, Style \\
Overwhelming Force & Str 5, base attack bonus \plus8 & Apply full Strength to damage when using two hands & Power \\
Parry & Dex 3 & Ready yourself to parry incoming blows & Defense, Reaction \\
\tind Shielded Parry & Dex 3, shield proficiency, Parry & Add shield modifier to parry attempts & Defense, Reaction \\
\tind Riposte & Dex 3, base attack bonus \plus4, Parry & Foes provoke if you parry very well & Reaction \\
Perfect Shot & Dex 5, base attack bonus \plus8 & Ignore less than total cover/concealment on ranged attacks & Precision, Style \\
Pierce Wings & Base attack bonus \plus8 & Temporarily remove foe's ability to fly \\ 
Point Blank Shot & \x &  \plus2 bonus on ranged damage within 30 ft. & Precision, Style \\
Precise Shot & \x & Add half Wis to damage & Precision, Style \\
Power Attack & Str 3 & Trade melee attack bonus for damage & Power, Style \\
Quick Draw & \x & Draw weapon as swift action & Reaction \\
Reveal the Weak Point & Base attack bonus \plus4 & Sacrifice attack and damage to penalize foe's AC & Style \\
Ride-By Attack & Ride 8 ranks & Move before and after a mounted charge & Mobility, Mounted \\
\tind Spirited Charge & Ride-By Attack & Double damage with mounted charge & Mounted, Power \\
Shield Proficiency  & \x &  No armor check penalty on attack rolls & Equipment \\
\tind Tower Shield Proficiency & Shield Proficiency & No armor check penalty on attack rolls & Equipment \\
Tactical Analysis & Int 3, Base attack bonus \plus4 & Identify foe's strengths and weaknesses & Awareness \\
Tactical Prediction & Int 3, Base attack bonus \plus8 & Predict foe's next action & Awareness \\
Tactical Readiness & Int 3 & Ready two actions at once & Reaction \\
Threatening Fire & Base attack bonus \plus4 & Help overwhelm foes with ranged weapons & Precision, Style \\
Trample & Ride 8 ranks & Target cannot avoid mounted overrun & Mobility, Mounted \\
Two-Weapon Fighting & Dex 3 & Gain \plus2 attack bonus when fighting with two weapons & Finesse \\
%\tind Ambidexterity & Add full Strength value to damage with both hands & Str 3, Two-Weapon Fighting, base attack bonus \plus12 \\
\tind Two-Weapon Defense & Two-Weapon Fighting & Off-hand weapon grants \plus1 shield bonus to AC, later \plus3 & Defense, Finesse \\
Wall Slam & Str 5, base attack bonus \plus8 & Slam opponent into wall to deal bonus damage & Maneuver, Power \\
Weapon Focus\fn{1} & Proficiency with weapon group, base attack bonus \plus1 & Special ability with weapon group & Equipment \\
\tind Weapon Specialization & Base attack bonus \plus8, proficiency with weapon group, Weapon Focus with weapon group & Special ability with weapon group & Equipment \\
Weapon Proficiency\fn{1} & \x &  Don't provoke when attacking with weapon group & Equipment \\
Whirlwind Attack & Dex 5, base attack bonus \plus12 & One melee attack against each opponent within reach & Style \\
\end{tabularx}
1 You can gain this feat multiple times. Its effects do not stack. Each time you take the feat, it has a different effect. \\
\end{dtable!*}

\begin{dtable!*}
  \lcaption{Bloodline Feats}
\begin{tabularx}{\textwidth}{>{\lcol}p{15em} >{\lcol}p{15em} >{\lcol}X}
\thead{Bloodline Feats\fn{1}} & \thead{Prerequisites} & \thead{Benefit} \\
Celestial Heritage & Nonevil alignment & Smite evil 1/day \\
\tind Celestial Body & Nonevil alignment, Celestial Heritage & Gain physical damage reduction \\
\tind Celestial Smiting & Nonevil alignment, Celestial Heritage & Smite evil more often, more accurately \\
\tind Celestial Soul & Nonevil alignment, any three celestial feats & Gain spell resistance against evil \\
Draconic Heritage & \x & Resist damage from chosen dragon's energy type \\
\tind Draconic Breath & Con 3, any three dragon bloodline feats & Gain dragon's breath weapon \\
\tind Draconic Might & Any three dragon bloodline feats & Gain \plus1 to physical attribute \\
\tind Draconic Mind & Any three dragon bloodline feats & Gain \plus1 to mental attribute \\
\tind Draconic Scales & Draconic Heritage & Gain natural armor \\
\tind Draconic Senses & Draconic Heritage & Gain low-light vision, possibly darkvision \\
\tind Draconic Spellpower & Draconic Heritage & Gain bonus to caster level \\
\tind Draconic Voice & Draconic Heritage & Gain bonus to Intimidate and Persuasion checks \\
\tind Draconic Weapons & Draconic Heritage & Gain bite attack, possibly claws \\
\tind Draconic Wings & Any three dragon bloodline feats & Gain wings to slow falls, glide, eventually fly \\
Elemental Heritage & \x & Gain saving throw bonus based on elemental ancestor \\
\tind Elemental Body & Any three elemental bloodline feats & 25\% chance to ignore critical hits \\
\tind Elemental Force & Elemental heritage & Unleash element to attack \\
\tind Elemental Mastery & Elemental Heritage & Gain attack bonus in circumstances based on elemental ancestor \\
\tind Elemental Movement & Elemental Heritage & Gain movement ability based on elemental ancestor \\
\end{tabularx}
\end{dtable!*}

\begin{dtable!*}
\lcaption{Metamagic Feats}
\begin{tabularx}{\textwidth}{>{\lcol}p{15em} >{\lcol}p{7.5em} >{\lcol}p{7.5em} >{\lcol}X}
\thead{Metamagic Feats} & \thead{Prerequisites} & \thead{Spell Level Increase} & \thead{Benefit} \\
Empower Spell  & Caster level 4th & Variable & Cast spell as higher level \\
Energetic Substitution & Caster level 4th & \plus1 & Change spell's energy type \\
Enlarge Spell  & Caster level 4th & \plus1 & Double spell's range \\
%Extend Spell  & Caster level 1st & \plus1 & Double spell's duration \\
Imbued Spellstrike & Caster level 4th & \plus1 & Combine spell with weapon attack \\
\tind Improved Imbued Spellstrike & Caster level 6th & \plus2 & Imbue weapon with spell's power up to 5 minutes \\
%Maximize Spell  & Caster level 6th & \plus3 & Maximize spell's variable, numeric effects \\
Quicken Spell  & Caster level 6th & \plus2 & Cast spell as swift action, but lose next action \\
Reach Spell & Caster level 6th & \plus2 & Cast touch spell at Close range \\
Shape Spell & Caster level 6th & \plus2 & Exclude areas within spell's area \\
Silent Spell  & Caster level 4th & \plus1 & Cast spell without verbal components \\
Still Spell  & Caster level 4th & \plus1 & Cast spell without somatic components \\
Sustained Spell & Caster level 4th & \plus1 & Maintain concentration as swift action \\
Widen Spell  & Caster level 8th & \plus3 & Double spell's area
\end{tabularx}
\end{dtable!*}

\begin{dtable!*}
\lcaption{Item Creation Feats}
\begin{tabularx}{\textwidth}{>{\lcol}p{15em} >{\lcol}p{15em} >{\lcol}X}
\thead{Item Creation Feats} & \thead{Prerequisites} & \thead{Benefit} \\
Imbue Magic	          & Caster level 2nd or Craft (any) 5 ranks & Create magic items \\
Imbuement Admixture & Caster level 6th & Combine multiple spells to create items \\
Versatile Crafter & Craft (any) 8 ranks & Learn to craft more varieties of magic \\
\end{tabularx}
\end{dtable!*}

\section{Feat Descriptions}
Here is the format for feat descriptions.

\ssecfake{Feat Name [Type of Feat]}
\parhead{Prerequisite} A minimum attribute score, another feat or feats, a minimum base attack bonus, a minimum number of ranks in one or more skills, or a class level that a character must have in order to acquire this feat. This entry is absent if a feat has no prerequisite. A feat may have more than one prerequisite.
\parhead{Benefit} What the feat enables the character (``you'' in the feat description) to do. If a character has the same feat more than once, its benefits do not stack unless indicated otherwise in the description.
\par In general, characters cannot gain the same feat twice.
\parhead{Normal} What a character who does not have this feat is limited to or restricted from doing. If not having the feat causes no particular drawback, this entry is absent.
\parhead{Special} Additional facts about the feat that may be helpful when you decide whether to acquire the feat.

\subsection{General Feats}

\subsubsection{Agile Movement [General]}
\parhead{Prerequisite} Acrobatics 8 ranks.
\parhead{Benefit} While running or charging, you can make a single turn of up to 90 degrees in the middle of the movement.
\parhead{Normal} You cannot change direction while running or charging.

\subsubsection{Augment Summoning [General]}
\parhead{Prerequisite} Spell Focus (conjuration).
\parhead{Benefit} Each creature you conjure with a summoning spell gains a \plus2 enhancement bonus to Strength and Constitution for the duration of the spell that summoned it.

\subsubsection{Combat Casting [General]}
\parhead{Benefit} You get a \plus2 competence bonus on Concentration checks made to cast a spell or use a spell-like ability. In addition, once per day you may reroll a Concentration check made to cast a spell.

\subsubsection{Deathless [General]}
\featpre Constitution 5 or base Fortitude save \plus10.
\featben You become immune to death effects.

\subsubsection{Diehard [General]}
\parhead{Prerequisite} Constitution 3.
\parhead{Benefit} When you take critical damage, you automatically become stable. You may choose to act as if you were disabled, rather than dying. You must make this decision as soon as you take critical damage (even if it isn't your turn). If you do not choose to act as if you were disabled, you immediately fall unconscious.
\par When using this feat, you can take either a single move or standard action each turn, but not both, and you cannot take a full round action. You can take a move action without further injuring yourself, but if you perform any standard action (or any other action deemed as strenuous, including some swift actions, such as casting a quickened spell) you take 1 point of critical damage. You die normally if your critical damage exceeds your Constitution score \add your level.
\parhead{Normal} A character without this feat who takes critical damage is unconscious and dying.

\subsubsection{Extend Rage [General]}
\parhead{Prerequisite} Ability to rage.
\parhead{Benefit} Your rage lasts for five extra rounds.

\subsubsection{Extra Channeling [General]}
\parhead{Prerequisite} Ability to channel energy
\parhead{Benefit} You can channel energy three more times per day. 
\parhead{Normal} Without this feat, a character can channel energy a number of times per day equal to 3 \add half his or her Charisma.

\subsubsection{Extra Invocation [General]}
\parhead{Prerequisite} Ability to use an arcane invocation.
\parhead{Benefit} You learn a new arcane invocation. You cannot learn invocations from your prohibited schools, if any, with this feat.

\subsubsection{Extra Music [General]}
\parhead{Prerequisite} Bardic music ability.
\parhead{Benefit} You gain three extra uses of your bardic music ability.

\subsubsection{Extra Rage [General]}
\parhead{Prerequisite} Ability to rage.
\parhead{Benefit} You can rage one more time per day.

\subsubsection{Extra Smiting [General]}
\parhead{Prerequisite} Smite ability.
\parhead{Benefit} You can smite three more times per day. If you have more than one smite ability, you choose which ability this applies to.
\parhead{Special} You can take more this once if you have more than one smite ability. Its effects do not stack. Each time you take the feat, you choose a different smite ability you have and apply the effects to that ability.

\subsubsection{Extra Stunning [General]}
\parhead{Prerequisite} Stunning Fist
\parhead{Benefit} You gain three extra uses of your Stunning Fist ability.

\subsubsection{Extra Wild Aspect [General]}
\parhead{Prerequisite} Wild aspect ability.
\parhead{Benefit} You gain three extra uses of your wild aspect ability.

\subsubsection{Fearless [General]}
\featpre Charisma 5 or base Will save \plus10.
\featben You are immune to fear effects and hostile morale effects.

\subsubsection{Great Fortitude [General]}
\parhead{Benefit} You get a \plus2 competence bonus on all Fortitude saving throws. Once per day, you may reroll a Fortitude save. You must decide to use this ability before the results are revealed. You must take the second roll, even if it is worse.

%\subsubsection{Greater Spell Penetration [General]}
%\parhead{Prerequisites} Caster level 10th, Spell Penetration.
%\parhead{Benefit} You get a \plus4 competence bonus on caster level checks (1d20 \add caster level) made to overcome a creature's spell resistance.

\subsubsection{Improved Counterspell [General]}
\parhead{Benefit} When counterspelling, you may use any spell of the same school with a spell level at least as high as the target spell.
\parhead{Normal} Without this feat, you may counter a spell only with the same spell or with a spell specifically designated as countering the target spell.

\subsubsection{Improved Channeling [General]}
\parhead{Prerequisite} Channel energy 3d6.
\parhead{Benefit} You gain a \plus2 competence bonus to your effective cleric level when channeling energy.

\subsubsection{Iron Will [General]}
\parhead{Benefit} You get a \plus2 competence bonus on all Will saving throws. Once per day, you may reroll a Will save. You must decide to use this ability before the results are revealed. You must take the second roll, even if it is worse.

\subsubsection{Lightning Reflexes [General]}
\parhead{Benefit} You get a \plus2 competence bonus on all Reflex saving throws. Once per day, you may reroll a Reflex save. You must decide to use this ability before the results are revealed. You must take the second roll, even if it is worse.

\subsubsection{Magical Synthesis [General]}
Choose two magical classes you possess.
\parhead{Prerequisites} Levels in two magical classes.
\parhead{Benefit} When gaining levels in either of your chosen classes, you gain magic levels in the other class as if the class you are gaining levels in did not progress your magic level. See \pcref{Magic Level and Multiclassing}.

\subsubsection{Mental Fortress [General]}
\featpre Charisma 9 or base Will save \plus18.
\featben You become immune to hostile mind-affecting effects.

\subsubsection{Perfect Health [General]}
\featpre Constitution 3 or base Fortitude save \plus6.
\featben You become immune to disease, except supernatural diseases such as mummy rot. If you have a Constitution of 7 or a base Fortitude save of \plus14, you also become immune to poison and supernatural diseases.

\subsubsection{Rapid Metamagic [General]}
\parhead{Prerequisites} Spellcraft 8 ranks, ability to cast spells, one metamagic feat
\parhead{Benefit} When you apply a metamagic feat to a spell, the spell only takes its normal casting time.
\parhead{Normal} Without this feat, applying metamagic takes a standard action (if the spell normally requires less than a standard action), a full-round action (if the spell normally requires a standard action), or an additional full-round action (if the spell takes 1 full round or longer to cast).
\parhead{Special} At 1st level, a sorcerer gets Rapid Metamagic as a bonus feat, even if he does not have the prerequisites.

\subsubsection{Retributive Counterspell [General]}
\parhead{Prerequisites} Ability to cast 4th level spells.
\parhead{Benefit} As part of the action to counter a spell, you may expend an Abjuration (Negation) spell of 4th level or higher. If you do, the counterspelled spell is turned back on the caster as if it were affected by the \spell{spell turning} spell. If it cannot be affected by \spell{spell turning}, such as if it is a spell that only affects the caster, it is simply countered as normal.

\subsubsection{Ritual Caster [General]}
\featpre Intelligence 3.
\featben You can learn and perform rituals as if you were an arcane caster with a magic level was equal to your character level.

\subsubsection{Ritual Master [General]}
\parhead{Prerequisites} Spellcraft 8 ranks, ability to cast spells
\parhead{Benefit} You can perform rituals in half the normal time. In addition, you gain a \plus3 competence bonus to checks made to perform rituals.
\parhead{Special} At 1st level, a wizard gets Ritual Master as a bonus feat, even if she does not have the prerequisites.

\subsubsection{Selective Channeling [General]}
\parhead{Prerequisites} Ability to channel energy.
\parhead{Benefit} You can exclude up to two additional creatures from the effect when you channel energy.
\parhead{Normal} Without this feat, you can exclude a number of creatures from the effect equal to 1 \add half your Wisdom.

\subsubsection{Spell Focus [General]}
Choose a school of magic or a spell descriptor.
\parhead{Prerequisite} Magic level 4th.
\parhead{Benefit} You get a \plus2 competence bonus to your caster level when casting spells and using spell-like abilities from the school of magic or with the spell descriptor you select.
\parhead{Special} You can gain this feat multiple times. Its effects do not stack. Each time you take the feat, it applies to a new school of magic.

\subsubsection{Spell Specialization [General]}
Choose a school of magic or a spell descriptor to which you already have applied the Spell Focus feat.
\featpre Magic level 8th, Spell Focus.
\featben You gain a \plus4 competence bonus to your caster level when casting spells and using spell-like abilities from the school of magic or with the spell descriptor you select. In exchange, you take a \minus2 penalty to caster level with all other spells and spell-like abilities.

%\subsubsection{Spell Penetration [General]}
%\parhead{Prerequisite} Caster level 4th
%\parhead{Benefit} You get a \plus2 competence bonus on caster level checks (1d20 \add caster level) made to overcome a creature's spell resistance.

\subsubsection{Swift [General]}
\parhead{Benefit} You increase your base land speed by 5 feet. This bonus is considered a competence bonus.

\subsubsection{Toughness [General]}
\parhead{Benefit} You gain \plus3 hit points, \plus1 per level above 3.

\subsection{Racial Feats}

\subsubsection{Dwarven Resilience [Racial]}
\parhead{Prerequisite} Dwarf
\parhead{Benefit} You gain a \plus2 competence bonus on saving throws against poisons, spells, and spell-like effects.

\subsubsection{Focused Mind [Racial]}
\parhead{Prerequisite} Elf
\parhead{Benefit} You gain a \plus2 competence bonus to Will saves. In addition, you can use your Intelligence instead of your Constitution on Concentration checks, such as when you cast spells.

\subsubsection{Giantfighter [Racial]}
\parhead{Prerequisite} Dwarf, gnome, or halfling
\parhead{Benefit} You gain a \plus2 competence bonus to dodge modifier against creatures of size Large or larger.

\subsubsection{Keen Senses [Racial]}
\parhead{Prerequisite} Elf
\parhead{Benefit} You gain a \plus4 competence bonus to Perception checks. In addition, if you come within 10 feet of a secret or concealed door, you can make a Perception check to notice it as if you were actively searching.

\subsubsection{Light-Footed [Racial]}
\parhead{Prerequisite} Elf or halfling
\parhead{Benefit} You gain a \plus4 competence bonus to Stealth checks. In addition, the DC to follow tracks you leave increases by 5.

\subsubsection{Stonecunning [Racial]}
\parhead{Prerequisite} Dwarf
\parhead{Benefit} You gain a \plus2 competence bonus to Craft and Perception checks related to stone or metal. In addition, if you come within 10 feet of unusual stonework, you can make a Perception check to notice it as if you were actively searching. Finally, you can also intuit depth, sensing your approximate depth underground as naturally as a human can sense which way is up.

\subsection{Skill Feats}

\subsubsection{Dilettante [Skill]}
\parhead{Prerequisite} Int 3
\parhead{Benefit} Choose a number of Knowledge skills equal to your Intelligence. You are treated as trained in those skills, even if you possess no ranks, allowing you to make Knowledge checks in those areas. If your Intelligence increases after taking this feat, you may choose additional Knowledge skills.

\subsubsection{Legendary Balance [Skill]}
\featpre Acrobatics 13 ranks
\featben You can balance on surfaces that cannot support your weight. The DC is 30 for liquids such as water, 35 for dense gases and raw energy, and 40 for ordinary air. While balancing in this way, you must take a move action each round to continue moving; you cannot remain in the same place in consecutive rounds, or you will fall. The DC increases by 2 for each consecutive round that you spend balancing in this way.

\subsubsection{Legendary Climber [Skill]}
\featpre Climb 13 ranks.
\featben You gain a climb speed equal to your land speed. This grants several benefits. 
\begin{itemize*}
  \item You can always take 10 on Climb checks, even when rushed, endangered, or distracted. 
  \item You gain a \plus5 inherent bonus on Climb checks.
  \item A successful Climb check allows you to move a distance equal to your climb speed.
\end{itemize*}

In addition, you can now climb surfaces that are perfectly smooth. The DC is 30 for perfectly smooth vertical surfaces, and 40 to climb on a perfectly smooth ceiling.

\subsubsection{Legendary Craftsman [Skill]}
\featpre Craft (any) 13 ranks.
\featben You can craft items with whatever materials you have on hand. When making an item, if your Craft check exceeds 30, you use half the costly material components to make the item, replacing them with other, less expensive components. For every 10 points by which the check exceeds 30, you halve the required material components again. This only applies once per item; use the highest result achieved while making that item.

This feat does not affect the material components required to craft magical items.

\subsubsection{Legendary Devicesmith [Skill]}
\featpre Devices 13 ranks.
\featben You can disable spell effects on objects or areas as if they were merely complex devices. You can make a Devices check against a spell effect within 5 feet of you. If your check result exceeds 30, it may dispel the effect. The DC to dispel the effect is equal to 20 \add the spell's caster level. Failure indicates that the spell is not dispelled. 

You must be aware of a spell to disable it, either through the Spellcraft skill or because the effect is noticeable. You cannot disable spell effects on creatures.

\subsubsection{Legendary Disguise [Skill]}
\featpre Disguise 13 ranks.
\featben Your disguises can change the magic on a creature. When you make a Disguise check, if the result exceeds 30, you can decide how that creature and any items on the creature appear when examined by divination spells. For example, you could cause all of its equipment to appear nonmagical, or you could cause it to have a strong aura of good when examined with \spell{detect good}. You cannot create an aura of overwhelming strength with this skill.

Anyone using divination magic on the creature must make a caster level check with a DC equal to your Disguise check in order to perceive the truth. If that check fails, the caster is unaware that any deception exists. Whether or not the check succeeds, you are not aware whether the disguise was penetrated in this way.

\subsubsection{Legendary Escapist [Skill]}
\featpre Escape Artist 13 ranks.
\featben You can attempt to escape from magic itself, slipping hostile spells off of your body so they dissipate harmlessly. As a full-round action, you can make an Escape Artist check to throw off magical effects on you. If the result exceeds 30, it may dispel the effect. The DC to dispel the effect is equal to 11 \add the effect's caster level. Failure indicates that the spell is not dispelled.

You must be aware of a spell to disable it, either through the Spellcraft skill or because the effect is noticeable. You can only dispel spell effects which target you directly, not area effects which include you. If a spell targets multiple creatures, you can only remove its effects on you.

\subsubsection{Legendary Leaper [Skill]}
\featpre Jump 13 ranks.
\featben When jumping, if your check result exceeds 30, your jump defies gravity. Your vertical distance can be any amount up to your jump check result. 

\subsubsection{Legendary Liar [Skill]}
\featpre Bluff 13 ranks.
\featben Your lies can fool even magic. When you make a Bluff check, if the result exceeds 30, anyone using divination magic must make a caster level check with a DC equal to your Bluff check in order to perceive anything contrary to your lie. If that check fails, the caster is unaware that any deception exists. Whether or not the check succeeds, you are not aware whether your lie was revealed in this way.

\subsubsection{Legendary Tumbler [Skill]}
\featpre Acrobatics 13 ranks.
\featben When tumbling, if your check result exceeds 30, you gain two benefits. First, you do not provoke attacks of opportunity for that movement. Second, you can tumble through areas occupied by enemies as if they were unoccupied.

\subsubsection{Open Minded [Skill]}
\parhead{Benefit} You gain two skill points. You may spend these skill points immediately.

\subsubsection{Ranged Legerdemain [Skill]}
\parhead{Prerequisite} Ability to cast 2nd level spells
\parhead{Benefit} By expending an Evocation (Control) spell of 2nd level or higher, you can use the Disable Device or Sleight of Hand skills at \rngclose range for a number of rounds equal to half the level of the spell slot.

\subsubsection{Skill Focus [Skill]}
Choose a skill.
\parhead{Benefit} You get a \plus3 competence bonus on all checks involving that skill. In addition, you once per day you may reroll a skill check with this skill.
\parhead{Special} You can gain this feat multiple times. Its effects do not stack. Each time you take the feat, it applies to a new skill.

\subsubsection{Skill Mastery [Skill]}
Choose a skill.
\featpre 10 ranks in the chosen skill.
\featben When making a skill check with your chosen skill, you may take 10 even if stress and distraction would normally prevent you from doing so. When you take 10, you treat any roll of less than 10 as if it were a 10.
\parhead{Special} You can gain this feat multiple times. Its effects do not stack. Each time you take the feat, it applies to a new skill.

\subsubsection{Skill Training [Skill]}
\featben Choose any two skills. You treat those skills as class skills, and you gain a \plus1 competence bonus with those skills.

\subsubsection{Track [Skill]}
\parhead{Benefit} To find tracks or to follow them for 1 mile requires a successful Survival check. You must make another Survival check every time the tracks become difficult to follow.
You move at half your normal speed (or at your normal speed with a \minus5 penalty on the check, or at up to twice your normal speed with a \minus20 penalty on the check). The DC depends on the surface and the prevailing conditions, as given on the table below:

\begin{dtable}
\begin{tabularx}{\columnwidth}{>{\lcol}X c >{\lcol}X c}
\thead{Surface} & \thead{Survival DC}  & \thead{Surface} & \thead{Survival DC} \\
Very soft ground  & 5  & Firm ground  & 15 \\
Soft ground  & 10  & Hard ground  & 20
\end{tabularx}
\end{dtable}
\par \emph{Very Soft Ground:} Any surface (fresh snow, thick dust, wet mud) that holds deep, clear impressions of footprints.
\par \emph{Soft Ground:} Any surface soft enough to yield to pressure, but firmer than wet mud or fresh snow, in which a creature leaves frequent but shallow footprints.
\par \emph{Firm Ground:} Most normal outdoor surfaces (such as lawns, fields, woods, and the like) or exceptionally soft or dirty indoor surfaces (thick rugs and very dirty or dusty floors). The creature might leave some traces (broken branches or tufts of hair), but it leaves only occasional or partial footprints.
\par \emph{Hard Ground:} Any surface that doesn't hold footprints at all, such as bare rock or an indoor floor. Most streambeds fall into this category, since any footprints left behind are obscured or washed away. The creature leaves only traces (scuff marks or displaced pebbles).
\par Several modifiers may apply to the Survival check, as given on the table below.

\begin{dtable}
\begin{tabularx}{\columnwidth}{>{\lcol}X >{\rcol}p{6em}}
\thead{Condition}  & \thead{Survival DC Modifier} \\
Every three creatures in the group being tracked  & \minus1 \\
Size of creature or creatures being tracked:\footnotetemp{1} &  \\
Fine  & \plus8 \\
Diminutive  & \plus4 \\
Tiny  & \plus2 \\
Small  & \plus1 \\
Medium  & \plus0 \\
Large  & \minus1 \\
Huge  & \minus2 \\
Gargantuan  & \minus4 \\
Colossal  & \minus8 \\
Every 24 hours since the trail was made  & \plus1 \\
Every hour of rain since the trail was made  & \plus1 \\
Fresh snow cover since the trail was made  & \plus10 \\
\thead{Poor visibility:\footnotetemp{2}} &  \\
Overcast or moonless night  & \plus6 \\
Moonlight  & \plus3 \\
Fog or precipitation  & \plus3 \\
Tracked party hides trail (and moves at half speed)  & \plus5
\end{tabularx}
1 For a group of mixed sizes, apply only the modifier for the largest size category. \\
2 Apply only the largest modifier from this category.
\end{dtable}

If you fail a Survival check, you can retry after 1 hour (outdoors) or 10 minutes (indoors) of searching.
\parhead{Normal} Without this feat, you can use the Survival skill to find tracks, but you can follow them only if the DC for the task is 10 or lower. Alternatively, you can use the Search skill to find a footprint or similar sign of a creature's passage using the DCs given above, but you can't use Search to follow tracks, even if someone else has already found them.
\parhead{Special} A ranger automatically has Track as a bonus feat. He need not select it.

\subsubsection{Veteran's Experience [Skill]}
\featpre Base attack bonus \plus8
\featben You can use your battlefield experience in place of learned knowledge to identify monsters. When attempting to identify a monster, you may roll your base attack bonus \add your Intelligence. A successful check gives you the same information as a Knowledge check would.

\subsection{Combat Feats}

\subsubsection{Active Defense [Combat]}
\parhead{Prerequisite} Base attack bonus \plus4
\parhead{Benefit} While using fighting defensively or using the Combat Expertise combat style, you take no penalty on attacks of opportunity, and can make attacks of opportunity at a \minus4 penalty while taking the total defense action.

\subsubsection{Armor Familiarity [Combat]}
Choose one category of armor: light, medium, heavy, or shields.
\parhead{Prerequisite} Proficiency with the chosen armor category.
\parhead{Benefit} You reduce your armor check penalty by 2 and your arcane spell failure by 5\% when using your chosen armor. This effect cannot reduce those penalties below 0.

\subsubsection{Armor Proficiency (Heavy) [Combat]}
\parhead{Prerequisites} Armor Proficiency (light), Armor Proficiency (medium).
\parhead{Benefit} See Armor Proficiency (light).
\parhead{Normal} See Armor Proficiency (light).
\parhead{Special} Fighters, paladins, and clerics automatically have Armor Proficiency (heavy) as a bonus feat. They need not select it.

\subsubsection{Armor Proficiency (Light) [Combat]}
\parhead{Benefit} When you wear a type of armor with which you are proficient, the armor check penalty for that armor applies only to Acrobatics, Climb, Escape Artist, Jump, and Sleight of Hand, and Swim checks. You suffer the armor's normal arcane spell failure.
\parhead{Normal} A character who is wearing armor with which she is not proficient applies its armor check penalty to attack rolls and to all skill checks that involve moving, including Ride. The character also suffers double the normal arcane spell failure chance for wearing the armor.
\parhead{Special} All characters except wizards, sorcerers, and monks automatically have Armor Proficiency (light) as a bonus feat. They need not select it.

\subsubsection{Armor Proficiency (Medium) [Combat]}
\parhead{Prerequisite} Armor Proficiency (light).
\parhead{Benefit} See Armor Proficiency (light).
\parhead{Normal} See Armor Proficiency (light).
\parhead{Special} Fighters, barbarians, paladins, clerics, druids, and bards automatically have Armor Proficiency (medium) as a bonus feat. They need not select it.

\subsubsection{Chargebreaker [Combat]}
\parhead{Prerequisite} Base attack bonus \plus4
\parhead{Benefit} When you ready an action to attack a foe that approaches you, you gain a \plus2 circumstance bonus to attack and damage on your attacks for the readied action.

\subsubsection{Cleave [Combat]}
\parhead{Prerequisites} Str 3, base attack bonus \plus4
\parhead{Benefit} Whenever you deal a creature enough damage to make it drop (typically by dropping it to below 0 hit points or killing it), you get an immediate extra melee attack against another creature within reach. You cannot move before making this extra attack. The extra attack is with the same weapon and at the same bonus as the attack that dropped the previous creature. There is no limit to the number of times you can use this feat per round.

\subsubsection{Cleaving Stride [Combat]}
\parhead{Prerequisites} Str 3, base attack bonus \plus8, Cleave.
\parhead{Benefit} If you move to attack a foe, including by charging, and deal enough damage to drop it during your attack, you can continue your movement (if you have any movement remaining) to attack another foe. You may take your extra attack from the Cleave feat before or after continuing your movement.

\subsubsection{Combat Prowess [Combat]}
\featpre Base attack bonus \plus8.
\featben You gain a \plus1 competence bonus to attack rolls.

\subsubsection{Contingent Attack [Combat, Reaction]}
\featpre Int 5, base attack bonus \plus12.
\featben As a full-round action, you may concentrate to prepare a single attack as a contingent action. You may choose any trigger for the contingent action.

Contingent actions depend on a trigger condition. If the trigger condition occurs and you have a contingent action prepared, you may take the contingent action as an immediate action. A contingent action remains prepared until you rest.

\subsubseection{Contingent Counter [Combat, Reaction]}
\featpre Int 3, base attack bonus \plus8.
\featben As a full-round action, you may concentrate to prepare a single attack as a contingent action. The action triggers when an opponent misses you with a melee attack.

Contingent actions depend on a trigger condition. If the trigger condition occurs and you have a contingent action prepared, you may take the contingent action as an immediate action. A contingent action remains prepared until you rest.

\subsubsection{Counterstorm [Combat, Reaction, Style]}
\featpre Base attack bonus \plus16.
\featben Whenever an opponent misses you with a melee attack, they provoke an attack of opportunity from you.

\subsubsection{Deflect Arrows [Combat]}
\parhead{Prerequisites} Dex 3, Improved Unarmed Strike.
\parhead{Benefit} You must have at least one hand free (holding nothing) to use this feat. Once per round when you would normally be hit with a ranged weapon, you may deflect it so that you take no damage from it. You must be aware of the attack and not flatfooted. You can deflect one additional attack at base attack bonus \plus5, \plus10, \plus15, and \plus20, to a maximum number of arrows equal to your Dexterity.
\par Attempting to deflect a ranged weapon doesn't count as an action. Unusually massive ranged weapons and ranged attacks generated by spell effects can't be deflected.
\parhead{Special} A monk may select Deflect Arrows as a bonus feat at 2nd level, even if she does not meet the prerequisites.

\subsubsection{Dodge [Combat]}
\parhead{Prerequisite} Dex 3.
\parhead{Benefit} You may designate an opponent as a free action. You receive a \plus4 circumstance bonus to your dodge modifier against attacks of opportunity from that opponent that were provoked by movement. In addition, you gain a \plus4 circumstance bonus to Acrobatics checks made to tumble against that opponent.

\subsubsection{Endurance [General]}
\featpre Con 3
\parhead{Benefit} You convert the first two points of damage you take each round into nonlethal damage. You resist 1 additional point of damage per two levels. In addition, you may sleep in light or medium armor without becoming fatigued.

This resistance is considered damage reduction for the purpose of abilities which overcome damage reduction. However, it is a separate ability, so it stacks with damage reduction.
\parhead{Normal} A character without this feat who sleeps in medium or heavier armor is automatically fatigued the next day.
\parhead{Special} A barbarian automatically gains Endurance as a bonus feat at 3rd level. He need not select it.

\subsubsection{Exotic Weapon Proficiency [Combat]}
You understand how to use exotic weapons in combat.
\parhead{Prerequisite} Base attack bonus \plus1.
\parhead{Benefit} You don't provoke attacks of opportunity when attacking with exotic melee weapons from weapon groups that you are proficient with, and you can use exotic ranged weapons from those groups without penalty.
\parhead{Normal} When using a melee weapon with which you are not proficient, you provoke an attack of opportunity when you miss. When using a ranged weapon with which you are not proficient, you take a \minus4 penalty to attack rolls.

\subsubsection{Far Shot [Combat]}
\parhead{Benefit} When you use a projectile weapon, such as a bow, its range increment increases by one-half (multiply by 1-1/2). When you use a thrown weapon, its range increment is doubled.

%\subsubsection{Improved Fighting Retreat [Combat]}
%\parhead{Prerequisites} Combat Reflexes, base attack bonus \plus6
%\parhead{Benefit} You can take a full attack when taking the fighting retreat action. You do not provoke attacks of opportunity for your movement from any foe you attack.
%\parhead{Normal} You can only make a single attack when taking the fighting retreat action.

\subsubsection{Improved Initiative [Combat]}
\parhead{Benefit} You get a \plus4 competence bonus on initiative checks.

\subsubsection{Improved Unarmed Strike [Combat]}
\parhead{Benefit} You are considered to be armed even when unarmed -- that is, you do not provoke attacks or opportunity from armed opponents when you attack them while unarmed. However, you still get an attack of opportunity against any opponent who makes an unarmed attack on you.
In addition, your unarmed strikes can deal lethal or nonlethal damage as you choose.
\parhead{Normal} Without this feat, you are considered unarmed when attacking with an unarmed strike, and you can deal only nonlethal damage with such an attack.
\parhead{Special} A monk automatically gains Improved Unarmed Strike as a bonus feat at 1st level. She need not select it.

\subsubsection{Legendary Awareness [Awareness, Combat]}
\featpre Base attack bonus \plus12, any three Awareness feats.
\featben You cannot be overwhelmed, and never suffer overwhelm penalties. This ability can prevent rogues from sneak attacking you.

\subsubsection{Legendary Finesse [Combat, Finesse]}
\featpre Base attack bonus \plus12, any three Finesse feats.
\featben You can add half your Dexterity to damage with ranged and melee attacks. This is added in addition to half your Strength (if applicable).

\subsubsection{Legendary Maneuver Master [Combat]}
\featpre Base attack bonus \plus12, any three Maneuver feats.
\featben You never provoke an attack of opportunity for failing a combat maneuver. In addition, if you succeed at a combat maneuver attack by 10 or more, you deal normal damage with the weapon used to perform the maneuver in addition to gaining the successful effects of the maneuver. If the maneuver was performed without a weapon, you deal damage equivalent to an unarmed attack.

\subsubsection{Legendary Mobility [Combat, Mobility]}
\featpre Base attack bonus \plus12, any three Mobility feats.
\featben You do not provoke attacks of opportunity when you move.

\subsubsection{Legendary Mounted Warrior [Combat, Mounted]}
\featpre Base attack bonus \plus12, any three Mounted feats, Ride 10 ranks.
\featben When you take damage, you may choose to have your mount suffer half the damage instead of you (rounded down). Likewise, when your mount takes damage, you may choose to suffer half of that damage instead of your mount (rounded down).

\subsubsection{Legendary Power [Combat, Power]}
\featpres Base attack bonus \plus12, any three Power feats.
\featben You can use weapons as if they were one category less encumbering than they actually are. The weapon encumbrance categories are light, medium, and heavy. For example, you can use a greatsword as a medium weapon in one hand without suffering any penalties.

\subsubsection{Legendary Precision [Combat, Precison]}
\featpre Base attack bonus \plus12, any three Precision feats.
\featben When attacking, if you hit your opponent by 10 or more, you deal maximum damage with your weapon. If the attack is a critical threat, you automatically confirm the threat.

\subsubsection{Legendary Reaction [Combat, Reaction]}
\featpre Base attack bonus \plus12, any three Reaction feats.
\featben You cannot caught be flat-footed, and always retain your Dexterity, dodge, and shield modifiers to armor class. This ability can prevent rogues from sneak attacking you.

\subsubsection{Legendary Style [Combat]}
\featpre Base attack bonus \plus12, any three Style feats.
\featben You may have two styles active at once. Both styles can be changed as part of the same swift action.

\subsubsection{Master Tactician [Combat]}
\featpre Intelligence 3, base attack bonus \plus12.
\featben As a full-round action, you can ready a full-round action or a move action and a standard action. In addition, when you ready a full-round action, allies within 30 feet of you can also ready the same full-round action on their next turn, provided that you have not taken the action yet.

\subsubsection{Mobility [Combat]}
\parhead{Prerequisites} Dex 3, Dodge.
\parhead{Benefit} When you move, it does not provoke attacks of opportunity from your Dodge target.

\subsubsection{Mounted Archery [Combat]}
\featpre Ride 1 rank.
\parhead{Benefit} The penalty you take when using a ranged weapon while mounted is decreased by 4: \minus0 instead of \minus4 if your mount is taking a double move, and \minus4 instead of \minus8 if your mount is running.

\subsubsection{Mounted Combat [Combat]}
\parhead{Prerequisite} Ride 1 rank.
\parhead{Benefit} Once per round when your mount is hit in combat, you may attempt a Ride check (as a reaction) to negate the hit. The hit is negated if your Ride check result is greater than the opponent's attack roll. (Essentially, the Ride check result becomes the mount's Armor Class if it's higher than the mount's regular AC.)%Mounted Combat

\subsubsection{Overwhelming Force [Combat]}
\parhead{Prerequisite} Str 5, base attack bonus \plus8
\parhead{Benefit} You add your full Strength to damage when wielding a medium or large melee weapon in two hands.
\parhead{Normal} Without this feat, you add half your Strength to damage.

\subsubsection{Parry [Combat, Defense, Reaction]}
\featpre Dexterity 3.
\featben As a standard action, you may ready yourself to parry incoming blows. Until the start of your next turn, if you are you are attacked and are not flat-footed against the attack, you may make an attack roll. You may treat the result of your attack roll as your armor class against that attack if it would be higher. If your base attack bonus is high enough to grant you multiple attacks, you may also make multiple parry attempts. Each parry attempt after the first takes a cumulative \minus5 penalty, just like your attack rolls do.

\subsubsection{Pierce Wings [Combat, Maneuver]}
\featpre Base attack bonus \plus8.
\featben As a standard action, you may make a special combat maneuver attack with a ranged weapon against any foe within range. If you beat your foe's combat maneuver defense, it takes normal damage from the weapon and loses its ability to fly for 1 round. This only affects creatures who use wings or other physical means to fly, and has no effect on creatures with magical or supernatural flight.

\subsubsection{Quick Draw [Combat]}
\parhead{Benefit} You can draw light weapons as a free action and medium weapons as a swift action. You can draw heavy and hidden weapons (see the Sleight of Hand skill) as a move action that provokes attacks of opportunity.
\par A character who has selected this feat may throw light weapons at his full normal rate of attacks (much like a character with a bow).

If you have three or more Reaction feats, you can draw light and medium weapons as an immediate action, and heavy and hidden weapons as a swift action.
\parhead{Normal} Without this feat, you may draw light weapons as a swift action, medium and heavy weapons as a move action, and hidden weapons as a standard action.

%\subsubsection{Rapid Shot [Combat]}
%\parhead{Prerequisites} Dex 3, Precise Shot, base attack bonus \plus4.
%\parhead{Benefit} You can get one extra attack per round with a projectile weapon. The attack is at your highest base attack bonus, but each attack you make in that round (the extra one and the normal ones) takes a \minus2 penalty. You must use the full attack action to use this feat. This does not stack with any other abilities that grant extra attacks.

\subsubsection{Ride-by Attack [Combat]}
\featpre Ride 8 ranks.
\parhead{Benefit} When you are mounted and use the charge action, you may move and attack as if with a standard charge and then move again (continuing the straight line of the charge). You do not need to attack from the closest possible space when making a ride-by attack. Your total movement for the round can't exceed the distance you could normally move on a mounted charge. You and your mount do not provoke an attack of opportunity from the opponent that you attack.

\subsubsection{Riposte [Combat, Reaction]}
\featpre Dexterity 3, base attack bonus \plus4, Parry
\featben When readying yourself to parry incoming blows, if your parry attempt exceeds your opponent's attack roll by 10 or more, your foe provokes an attack of opportunity from you.

\subsubsection{Shield Proficiency [Combat]}
\parhead{Benefit} You can use a shield and take only the standard penalties.
\parhead{Normal} When you are using a shield with which you are not proficient, you take the shield's armor check penalty on attack rolls and on all skill checks that involve moving, including Ride checks.
\parhead{Special} Barbarians, bards, clerics, druids, fighters, paladins, and rangers automatically have Shield Proficiency as a bonus feat. They need not select it.

\subsubsection{Shielded Parry [Combat, Defense, Reaction]}
\featpres Dexterity 3, shield proficiency, Parry
\featben When readying yourself to parry incoming blows and using a shield, you may add your shield modifier to your attack roll made to parry.
\subsubsection{Snatch Arrows [Combat]}
\parhead{Prerequisites} Dex 5, Deflect Arrows, Improved Unarmed Strike.
\parhead{Benefit} When using the Deflect Arrows feat, you may catch the weapon instead of just deflecting it. Thrown weapons can immediately be thrown back at the original attacker (even though it isn't your turn) or kept for later use.
\par You must have at least one hand free (holding nothing) to use this feat.%Snatch Arrows

\subsubsection{Spirited Charge [Combat]}
\parhead{Prerequisites} Ride 1 rank, Ride-By Attack.
\parhead{Benefit} When mounted and using the charge action, you deal double damage with a melee weapon (or triple damage with a lance) on your attack at the end of the charge.%Spirited Charge

\subsubsection{Stunning Fist [Combat]}
\parhead{Prerequisites} Dex 3, Wis 3, Improved Unarmed Strike, base attack bonus \plus4.
\parhead{Benefit} You must declare that you are using this feat before you make your attack roll (thus, a failed attack roll ruins the attempt). Stunning Fist forces a foe damaged by your unarmed attack to make a Fortitude saving throw (DC 10 \add 1/2 your character level \add your Constitution), in addition to dealing damage normally. A bloodied defender who fails this saving throw is stunned for 1 round (until just before your next action). A stunned character can't act, is flat-footed, and takes a \minus2 penalty to AC. A healthy defender is staggered instead, and can only take a standard action each round. You may attempt a stunning attack a number of times per day equal to 1 \add half your Constitution, but no more than once per round. Constructs, oozes, plants, undead, incorporeal creatures, and creatures immune to critical hits cannot be stunned.
\parhead{Special} A monk may select Stunning Fist as a bonus feat at 2nd level, even if she does not meet the prerequisites. A monk who selects this feat may choose for the uses per day and saving throw DC to be based on Wisdom instead of Constitution.

\subsubsection{Tactical Analysis [Combat]}
\parhead{Prerequisite} Intelligence 3, Base attack bonus \plus4.
\parhead{Benefit} You can attempt to identify the strengths and weaknesses of creatures based on your combat experience. As a swift action, you can make a special check with a bonus equal to your base attack bonus \add your Intelligence \add 2 per round you have seen the creature fight. The DC is equal to 10 \add the creature's CR. If you succeed, you learn about the monster's combat abilities as if you had made a successful Knowledge check.

\subsubsection{Tactical Prediction [Combat]}
\parhead{Prerequisites} Intelligence 3, Base attack bonus \plus8
\parhead{Benefit} You can attempt to predict what your opponent will do. As a swift action, you can make a special check with a bonus equal to your base attack bonus \add your Intelligence \add 2 per round you have seen the creature fight. The DC is equal to 15 \add the creature's CR. If you succeed, you learn what the creature is planning to do during its next turn. Of course, it can change its plans, particularly if it hears you tell your allies what it will do.

\subsubsection{Tactical Readiness [Combat]}
\parhead{Prerequisites} Intelligence 3
\parhead{Benefit} When you take the Ready action, you can ready two actions at once, each with separate triggers.

\subsubsection{Tower Shield Proficiency [Combat]}
\parhead{Prerequisite} Shield Proficiency.
\parhead{Benefit} You can use a tower shield and suffer only the standard penalties.
\parhead{Normal} A character who is using a shield with which he or she is not proficient takes the shield's armor check penalty on attack rolls and on all skill checks that involve moving, including Ride.
\parhead{Special} Fighters and paladins are automatically proficient with tower shields.

\subsubsection{Trample [Combat]}
\parhead{Prerequisites} Ride 8 ranks.
\parhead{Benefit} When you attempt to overrun an opponent while mounted, your target may not choose to avoid you. Your mount may make one attack with an appropriate natural weapon (hoof, claw, or other leg-based attack) against any target you knock down, gaining the standard \plus4 circumstance bonus on attack rolls against prone targets.

\subsubsection{Two-Weapon Defense [Combat]}
\parhead{Prerequisites} Dex 3, Two-Weapon Fighting.
\parhead{Benefit} When wielding a double weapon or two weapons (not including natural weapons or unarmed strikes), you gain a \plus1 shield bonus to your AC. This bonus increases to \plus2 once your base attack bonus reaches \plus8, and to \plus3 at base attack bonus \plus16.
%\par Additionally, if you are proficient with bucklers, you may keep your buckler's shield bonus to AC while you attack with a weapon in your off-hand.
%\parhead{Normal} Without this feat, a character wielding a buckler loses the buckler's shield bonus to AC when using an off-hand weapon to attack.

\subsubsection{Two-Weapon Fighting [Combat]}
You can fight with a weapon in each hand more effectively.
\parhead{Prerequisite} Dex 3.
\parhead{Benefit} You gain a \plus2 competence bonus to attack rolls when attacking with two weapons at once.

%\subsubsection{Two-Weapon Rend [Combat]}
%\parhead{Prerequisites} Str 3, Two-Weapon Fighting, base attack bonus \plus11.
%\parhead{Benefit} When fighting with two weapons at once, you may add half your Strength to damage with your offhand.
%\parhead{Normal} Without this feat, you do not apply Strength to damage with your offhand.

\subsubsection{Wall Slam [Combat, Maneuver, Power]}
\featpre Str 5, base attack bonus \plus8
\featben If you bull rush an opponent into a wall or other solid object, he takes d6 damage \add half your Strength and provokes attacks of opportunity from all threatening creatures, including you.

\subsubsection{Weapon Focus [Combat]}
Choose one weapon group.
\parhead{Prerequisites} Proficiency with selected weapon group, base attack bonus \plus1.
\parhead{Benefit} You gain a special ability based on the weapon group chosen. These special abilities only function when you are making attacks with weapons from your chosen weapon group.
\begin{itemize*}
    \item Armor weapons: When you perform a shield bash, you still benefit from the shield's AC bonus.
    \item Axes: You increase your critical threat range by 1 when attacking foes without an armor bonus to AC. This is applied after any effects that multiply your threat range.
    \item Blades, heavy: You increase your critical multiplier by 1 when attacking foes without an armor bonus to AC.
    \item Blades, light: You increase your critical multiplier by 1 when attacking flat-footed or overwhelmed foes.
    \item Blunt weapons: When you deal damage to a creature, it takes a \minus2 penalty to Will saves for 1 round. This penalty is not cumulative.
    \item Bows: You suffer half the normal range increment penalty for firing at long range.
    \item Crossbows: The time required for you to reload crossbows is reduced to a free action (for a hand or light crossbow) or a move action (for a heavy crossbow). Reloading a heavy crossbow still provokes an attack of opportunity, but reloading hand and light crossbows does not.
    \item Flexible weapons: You gain a \plus2 circumstance bonus to attack foes with a shield bonus to AC.
    \item Headed weapons: When you get a critical hit, your foe is staggered for 1 round.
    \item Monk weapons: ??
    \item Polearms: You can switch grips to short haft or stop short hafting a polearm as a swift action, and you take no penalty while short hafting it.
    \item Spears: You gain a \plus2 circumstance bonus to attack on attacks of opportunity.
    \item Thrown weapons: You can throw weapons at targets adjacent to you without provoking attacks of opportunity.
\end{itemize*}
\parhead{Special} You can gain this feat multiple times. Its effects do not stack. Each time you take the feat, it applies to a new type of weapon.
\par You cannot choose simple weapons when you take this feat.

\subsubsection{Weapon Proficiency [Combat]}
Choose one weapon group.
\parhead{Benefit} You don't provoke attacks of opportunity when attacking with melee weapons from that group, and you can use ranged weapons from that group without penalty.
\parhead{Normal} When using a melee weapon with which you are not proficient, you provoke an attack of opportunity when you miss. When using a ranged weapon with which you are not proficient, you take a \minus4 penalty to attack rolls.
\parhead{Special} You can gain Weapon Proficiency multiple times. Each time you take the feat, it applies to a new weapon group. You cannot choose simple weapons.
\par Clerics who choose the War domain and paladins automatically gain the Weapon Proficiency feat related to their deity's favored weapon group as a bonus feat. They need not select it.

\subsubsection{Weapon Specialization [Combat]}
Choose one weapon group for which you have already selected the Weapon Focus feat.
\parhead{Prerequisites} Base attack bonus \plus8, proficiency with selected weapon group, Weapon Focus with selected weapon group.
\featben You gain a special ability based on the weapon group chosen. These special abilities only function when you are making attacks with weapons from your chosen weapon group.

\begin{itemize*}
    \item Armor weapons: If you attack with armor weapons during your turn, you gain a \plus2 circumstance bonus to AC against melee attacks for 1 round.
    \item Axes: You gain a \plus2 circumstance bonus to attack against foes with an armor bonus to AC.
    \item Blades, heavy: You gain a \plus2 circumstance bonus to attack foes without an armor bonus to AC.
    \item Blades, light: You gain a \plus2 circumstance bonus to attack flat-footed or overwhelmed foes.
    \item Blunt weapons: When you deal damage to a creature, it takes a \minus2 penalty to all saving throws for 1 round. This penalty replaces the penalty from Weapon Focus, and it is not cumulative with itself.
    \item Bows: You can ignore cover (but not total cover) provided by creatures and objects that are at least ten feet away from both you and your target.
    \item Crossbows: You can fire crossbows at targets adjacent to you without provoking attacks of opportunity.
    \item Flexible weapons: The first time you perform a combat maneuver with a flexible weapon in an encounter, your opponent is flat-footed against your attack.
    \item Headed weapons: You increase your critical threat range by 1. This is applied after any effects that multiply your threat range.
    \item Monk weapons: ??
    \item Polearms: When making melee attacks, you can ignore cover provided by creatures.
    \item Spears: If an opponent charges you, he provokes an attack of opportunity from you.
    \item Thrown weapons: When you attack, you can take a \minus4 penalty to attack in order to strike two adjacent targets with the same thrown weapon. You make one attack roll and apply the result to the AC of both targets. If you get a critical hit, only the primary target is suffers the critical hit. 
\end{itemize*}

\subsection{Combat Maneuver Feats}

\subsubsection{Improved Bull Rush [Combat Maneuver]}
\parhead{Prerequisites} Base attack bonus \plus4
\parhead{Benefit} When you perform a bull rush, you do not need to move with the target to move them back. You also gain a \plus2 competence bonus on bull rush attacks.

\subsubsection{Improved Dirty Trick [Combat Maneuver]}
\parhead{Prerequisites} Base attack bonus \plus4
\parhead{Benefit} The conditions imposed by your dirty tricks last for 1d4 rounds. You also gain a \plus2 competence bonus on dirty trick attacks.

\subsubsection{Improved Disarm [Combat Maneuver]}
\parhead{Prerequisites} Base attack bonus \plus4
\parhead{Benefit} When you disarm an opponent, the weapon can land up to 15 feet away in a random direction. You also gain a \plus2 competence bonus on disarm attacks.

\subsubsection{Improved Feint [Combat Maneuver]}
\parhead{Prerequisites} Base attack bonus \plus4
\parhead{Benefit} You can feint in combat as a move action, and you gain a \plus2 competence bonus to feint attacks.
\parhead{Normal} Feinting in combat is an attack action.

\subsubsection{Improved Grapple [Combat Maneuver]}
\parhead{Prerequisites} Base attack bonus \plus4
\parhead{Benefit} While in a grapple, you may make grapple checks as an attack action. You also gain a \plus2 competence bonus on grapple attacks and to your CMD against grapple attacks.
\parhead{Normal} While in a grapple, you make grapple attacks as a standard action.
\par A monk may select Improved Grapple as a bonus feat at 1st level, even if she does not meet the prerequisites.

\subsubsection{Improved Overrun [Combat Maneuver]}
\parhead{Prerequisites} Base attack bonus \plus4
\parhead{Benefit} When you attempt to overrun an opponent, the target may not choose to avoid you unless you let them. You also gain a \plus2 competence bonus on overrun attacks.
\parhead{Normal} Without this feat, the target of an overrun can choose to avoid you or to block you.

\begin{comment}
\subsubsection{Improved Sunder [Combat Maneuver]}
\parhead{Prerequisites} Base attack bonus \plus4
\parhead{Benefit} When you strike at an object held or carried by an opponent (such as a weapon or shield), you ignore half the hardness of the sundered item. You also gain a \plus2 competence bonus on sunder attacks.
\end{comment}

\subsubsection{Improved Trip [Combat Maneuver]}
\parhead{Prerequisites} Base attack bonus \plus4
\parhead{Benefit} When you successfully trip a foe, it immediately provokes an attack of opportunity from everyone threatening it, including you. These attacks are made as the creature is being tripped, so it does not have penalties for being prone. You also gain a \plus2 competence bonus on trip attacks.

\subsection{Combat Style Feats}

\subsubsection{Blind-Fight [Combat Style]}
\parhead{Benefit} In melee, every time you miss because of being unable to see your opponent, you can reroll your miss chance one time to see if you actually hit. In addition, you are not flat-footed against invisible attackers adjacent to you.
\par If you have 10 ranks in Perception, you can automatically pinpoint the location of any invisible creature adjacent to you. 
\parhead{Normal} You have a 50\% chance to miss opponents you can't see, and you are flat-footed against them.

\subsubsection{Bulwark of Defense [Combat Style]}
\parhead{Prerequisite} Base attack bonus \plus4
\parhead{Benefit} Foes that you threaten at the start of your turn treat all squares you threaten as difficult terrain. Creatures must pay double movement cost to move through squares of difficult terrain, and cannot run or charge while in difficult terrain.

\subsubsection{Cautious Attack [Combat Style]}
\parhead{Benefit} You take a \minus2 penalty to damage rolls to gain a \plus2 circumstance bonus to your Combat Maneuver Defense. \babscalingdescription
\parhead{Style Requirement} You must attack each round.

\subsubsection{Combat Expertise [Combat Style]}
\parhead{Prerequisite} Int 3.
\parhead{Benefit} You take a \minus2 penalty on attack rolls to gain a \plus2 circumstance bonus to your dodge modifier. At base attack bonus \plus4, and every 4 base attack bonus thereafter, you increase the penalty and bonus by 1. In addition, you increase the bonus gained from using the total defense action (see \pref{Total Defense}) by the same amount. You cannot fight defensively while using this combat style.
\parhead{Style Requirement} Must full attack or take the total defense action each round.
\parhead{Normal} A character without the Combat Expertise feat can fight defensively while using the full attack action to take a \minus4 penalty on attack rolls and gain a \plus2 circumstance bonus to its dodge modifier.

\subsubsection{Covering Fire [Combat Style]}
\parhead{Benefit} If you hit a creature with a ranged attack, it takes a \minus2 penalty to attack rolls for 1 round.

\subsubsection{Deadly Aim [Combat Style]}
\parhead{Prerequisites} Dex 3.
\parhead{Benefit} You take a \minus2 penalty on all ranged attack rolls and gain a \plus2 circumstance bonus on all ranged damage rolls. At base attack bonus \plus4, and every 4 base attack bonus thereafter, the penalty increases by 1 and the bonus increases by 2.
\parhead{Style Requirement} Must full attack with a ranged weapon each round.

\subsubsection{Defensive Stance [Combat Style]}
\parhead{Prerequisites} Base attack bonus \plus6
\parhead{Benefit} You gain a \plus2 circumstance bonus to Combat Maneuver Defense. \babscalingdescription However, you are unable to move until you change styles (a swift action). You can only use this style if you did not move in the previous round.

\subsubsection{Distracting Foe [Combat Style]}
\parhead{Prerequisites} Base attack bonus \plus4.
\parhead{Benefit} Foes you threaten take a \minus2 penalty to Concentration checks. This penalty increases to \minus3 at base attack bonus \plus8, to \minus4 at base attack bonus \plus12, and finally to \minus5 at base attack bonus \plus16.
\parhead{Style Requirement} Must wield a melee weapon.

\subsubsection{Feign Weakness[Combat Style]}
\featpre Base attack bonus \plus4.
\featben As part of a full attack, you may provoke an attack of opportunity from one opponent threatening you. You gain a \plus4 circumstance bonus to armor class against this attack. If the opponent takes the attack of opportunity, they are flat-footed against your next attack.

\subsubsection{Guardian [Combat Style]}
\parhead{Benefit} Allies adjacent to you are considered to be threatened by one fewer creature than they actually are for the purpose of determining overwhelm penalties.
\parhead{Style Requirement} Must wield a melee weapon.

\subsubsection{Executioner [Combat Style]}
\parhead{Prerequisites} Base attack bonus \plus12.
\parhead{Benefit} Whenever a foe you threaten becomes staggered by dropping to 0 hit points, you can immediately make an attack of opportunity against it.

\subsubsection{Eye of the Storm [Combat Style]}
\parhead{Prerequisite} Base attack bonus \plus4.
\parhead{Benefit} You are considered to be threatened by one fewer creature than you actually are for the purpose of determining overwhelm penalties.

\subsubsection{Heartseeker [Combat Style]}
\parhead{Prerequisite} Base attack bonus \plus8.
\parhead{Benefit} You double your critical threat range with any weapon you wield. This does not stack with any other effects which increase threat range.

\subsubsection{Inescapable Bulwark of Defense [Combat Style]}
\featpre Base attack bonus \plus8, Bulwark of Defense
\featben This style functions like Bulwark of Defense, except that foes who attack you or take a withdraw action away from you still provoke attacks of opportunity when they move away from you.

\subsubsection{Intimidating Stance [Combat Style]}
\parhead{Prerequisite} Intimidate 8 ranks.
\parhead{Benefit} As a swift action each round, you can make an intimidate check against a single foe you threaten and dealt damage to during the round.

\subsubsection{Manyshot [Combat Style]}
\parhead{Prerequisites} Dex 7, base attack bonus \plus12.
\parhead{Benefit} When you attack with a light thrown weapon or projectile weapon (except crossbows), you may make a flurry attack with two projectiles or weapons at once. If the attack hits, the first projectile hits. If the attack hits by 5 or more, both projectiles hit. As normal for flurry attacks, apply precision-based damage (such as sneak attack) and critical hit damage only once for this attack. Damage reduction and resistances apply once to the total damage dealt.
\parhead{Style Requirement} Must full attack with a non-crossbow projectile weapon each round.

\subsubsection{Opportunist [Combat Style]}
\parhead{Prerequisite} Dex 3.
\parhead{Benefit} You gain a \plus2 circumstance bonus to attack and damage on attacks of opportunity. \babscalingdescription
\parhead{Style Requirement} Must wield a melee weapon.

\subsubsection{Overpowering Assault [Combat Style]}
\parhead{Prerequisite} Str 3.
\parhead{Benefit} You take a \minus2 penalty to AC and gain a \plus2 circumstance bonus to combat maneuver attacks. \babscalingdescription The penalty to AC lasts until the start of the next turn after you end the style.
\parhead{Style Requirement} Must perform a combat maneuver each round.

\subsubsection{Perfect Shot [Combat Style]}
\parhead{Prerequisites} Dex 5, Precise Shot, base attack bonus \plus8.
\parhead{Benefit} Your ranged attacks ignore cover and concealment, except total cover and total concealment. In addition, when you shoot or throw ranged weapons at a grappling opponent, you automatically strike at the opponent you have chosen.

You must spend a full-round action to make a full attack while in this style. You can make a single attack as a standard action.

\subsubsection{Point Blank Shot [Combat Style]}
\parhead{Benefit} You get a \plus2 circumstance bonus on damage rolls with ranged weapons when attacking targets within half of your range increment. \babscalingdescription
\parhead{Style Requirement} None.

\subsubsection{Power Attack [Combat Style]}
\parhead{Prerequisite} Str 3.
\parhead{Benefit} You take a \minus2 penalty on melee attack rolls to gain a \plus2 bonus on all melee damage rolls. At base attack bonus \plus4, and every 4 base attack bonus thereafter, the penalty increases by 1 and the bonus increases by 2.
\par The penalty and bonus damage do not apply to combat maneuvers or touch attacks. The bonus damage is halved if you are making an attack with an off-hand weapon or light weapon.
\parhead{Style Requirement} Must charge or full attack with a melee weapon each round.

%Can't replace Strength; archers need to be strong
\subsubsection{Precise Shot [Combat Style]}
\parhead{Benefit} You can add half your Wisdom to damage with ranged attacks in addition to half your strength.

\subsubsection{Shot On The Run [Combat Style]}
\parhead{Prerequisites} Dex 3, Dodge, Mobility, base attack bonus \plus4.
\parhead{Benefit} When attacking with a ranged weapon, you can move both before and after each attack, provided that your total distance moved is not greater than your speed.
\parhead{Style Requirement} Must move each round, and must not be wearing heavy armor.

\subsubsection{Spring Attack [Combat Style]}
\parhead{Prerequisites} Dex 3, Dodge, Mobility, base attack bonus \plus4.
\parhead{Benefit} When attacking with a melee weapon, you can move both before and after each attack, provided that your total distance moved is not greater than your speed. Moving in this way does not provoke an attack of opportunity from your Dodge target, though it might provoke attacks of opportunity from other creatures, if appropriate.
\parhead{Style Requirement} Must move each round, and must not be wearing heavy armor.

\subsubsection{Reveal the Weak Point [Combat Style]}
\featpre Base attack bonus \plus4.
\parhead{Benefit} You take a \minus2 penalty to attack and damage. Any foe you strike with a weapon take a \minus2 penalty to AC for 1 round. This penalty does not stack with itself. The penalties you suffer and the penalty your foe takes increase to \minus3 at base attack bonus \plus8, to \minus4 at base attack bonus \plus12, and finally to \minus5 at base attack bonus \plus16.

\subsubsection{Threatening Fire [Combat Style]}
\parhead{Prerequisites} Base attack bonus \plus4.
\parhead{Benefit} When attacking with a ranged weapon, you are considered to be threatening your target for the purpose of determining overwhelm penalties.
\parhead{Style Requirement} Must direct all attacks against the same target when making a full attack.

\subsubsection{Whirlwind Attack [Combat Style]}
\parhead{Prerequisites} Dex 5, base attack bonus \plus12
\parhead{Benefit} As a full-round action, you can make one melee attack at your full base attack bonus against each opponent you threaten. This is considered a full attack, and replaces any other attacks you would normally make, whether from base attack bonus or spells such as \spell{haste}.

\subsection{Bloodline Feats}

\subsubsection{Celestial Body [Bloodline, Celestial]}
\parhead{Prerequisites} Nonevil alignment, Celestial Heritage.
\parhead{Benefit} You gain physical damage reduction 2/evil. This damage reduction allows you to ignore the first two points of damage you take each round. If you are hit by an evil-aligned attack, you cannot use your damage reduction for 1 round.

If you have four or more celestial bloodline feats, your damage reduction increases to be equal to half of your hit value.

\subsubsection{Celestial Heritage [Bloodline, Celestial]}
\parhead{Prerequisite} Nonevil alignment.
\parhead{Benefit} You have the blood of a celestial creature in your veins. Once per day, you can smite evil as part of an attack. If you smite an evil creature, you gain a circumstance bonus to attack equal to the number of celestial bloodline feats you have and a circumstance bonus to damage equal to your character level.

\subsubsection{Celestial Smiting [Bloodline, Celestial]}
\parhead{Prerequisites} Nonevil alignment, Celestial Heritage.
\parhead{Benefit} You can smite evil with your Celestial Heritage ability a number of times per day equal to the number of celestial bloodline feats you have \add your Charisma. You cannot smite more than once per round.

\subsubsection{Celestial Spell Conduit [Bloodline, Celestial]}
\parhead{Prerequisites} Nonevil alignment, Celestial Heritage.
\parhead{Benefit} You gain a \plus2 competence bonus to caster level with Evocation (Channeling) spells and spells from the Good domain. If you have four or more celestial bloodline feats, this bonus increases to \plus4.

\subsubsection{Celestial Soul [Bloodline, Celestial]}
\parhead{Prerequisites} Nonevil alignment, any three celestial feats.
\parhead{Benefit} You gain spell resistance against evil spells and spells cast by evil creatures.

\subsubsection{Draconic Breath [Bloodline, Dragon]}
\parhead{Prerequisites} Con 3, any three dragon bloodline feats.
\parhead{Benefit} You gain a breath weapon based on the type of dragon you chose for the Draconic Heritage feat. The shape of the breath weapon is given on the \trefnp{Dragon Types} chart: either a \arealarge, 5 ft. wide line or a \areamed cone. At 11th level, the size increases to a 100 ft. long, 10 ft. wide line or a \arealarge cone.

A breath weapon deals 1d6 damage per two Hit Values you possess and allows a Reflex save for half damage. The save DC is equal to 10 \add half your Hit Dice \add your Constitution. After using your breath weapon, you must wait 1d4 rounds before you can use it again.

\subsubsection{Draconic Heritage [Bloodline, Dragon]}
\parhead{Benefit} You have the blood of a dragon in your veins. When you take this feat, choose a type of dragon. You gain damage reduction against the damage type that that dragon's breath weapon deals. The value of the damage reduction is equal to 5 \mtimes the number of dragon bloodline feats that you have. A list of dragons and their associated damage type is given below.

\begin{dtable}
  \lcaption{Dragon Types}
  \begin{tabularx}{\columnwidth}{>{\lcol}X >{\lcol}X >{\lcol}X}
    \thead{Dragon} & \thead{Energy Type} & \thead{Breath Weapon} \\
    Black & Acid & Line \\
    Blue & Electricity & Line \\
    Brass & Fire & Line \\
    Bronze & Electricity & Line \\
    Copper & Acid & Line \\
    Gold & Fire & Cone \\
    Green & Acid & Cone \\
    Red & Fire & Cone \\
    Silver & Cold & Cone \\
    White & Cold & Cone \\
  \end{tabularx}
\end{dtable}

\subsubsection{Draconic Might [Bloodline, Dragon]}
\featpre Any three dragon bloodline feats.
\featben Choose a physical attribute: Strength, Dexterity, or Constitution. You gain a \plus1 competence bonus to that attribute.

\subsubsection{Draconic Mind [Bloodline, Dragon]}
\featpre Any three dragon bloodline feats.
\featben Choose a mental attribute: Intelligence, Wisdom, or Charisma. You gain a \plus1 competence bonus to that attribute.

\subsubsection{Draconic Scales [Bloodline, Dragon]}
\parhead{Prerequisite} Draconic Heritage.
\parhead{Benefit} You gain a \plus1 competence bonus to your natural armor modifier. If you have four or more dragon bloodline feats, this bonus increases to \plus2.

\subsubsection{Draconic Senses [Bloodline, Dragon]}
\parhead{Prerequisite} Draconic Heritage.
\parhead{Benefit} You gain low-light vision. If you already have low-light vision, you can now see four times as well in darkness. If you have four or more dragon bloodline feats, you gain darkvision with a 60 foot range, or the range of your darkvision increases by 60 feet.

\subsubsection{Draconic Spellpower [Bloodline, Dragon]}
\parhead{Prerequisite} Draconic Heritage.
\parhead{Benefit} You gain a \plus1 competence bonus to caster level with all spells and spell-like abilities. If you have four or more dragon bloodline feats, this bonus increases to \plus2.

\subsubsection{Draconic Voice [Bloodline, Dragon]}
\parhead{Prerequisite} Draconic Heritage.
\parhead{Benefit} You gain a \plus2 competence bonus to Intimidate and Persuasion checks. If you have four or more dragon bloodline feats, this bonus increases to \plus4.

\subsubsection{Draconic Weapons [Bloodline, Dragon]}
\parhead{Prerequisite} Draconic Heritage
\parhead{Benefit} You gain a bite natural attack that deals d8 damage for a Medium creature. If you have four or more dragon bloodline feats, you also gain a claw natural attack for each hand that deals d6 damage for a Medium creatures.

\subsubsection{Draconic Wings [Bloodline, Dragon]}
\parhead{Prerequisite} Any three dragon bloodline feats.
\parhead{Benefit} You gain wings that sprout from your back. These allow you to slow your fall so you fall no more than 60 feet per round, preventing you from taking falling damage. If you have 8 Hit Values, you can use the wings to glide at a rate equal to your base land speed. While gliding, you move forward each round at a rate equal to your base land speed, and you descend at a rate of 10 feet per round.

If you have 14 Hit Values, you gain a fly speed equal to your base land speed, though you can only fly for a number of rounds equal to 3 \add half your Constitution. After that limit is reached, you must rest for 5 minutes to recuperate. If you have 20 Hit Values, you can fly for any length of time without needing to rest.

\subsubsection{Elemental Body [Bloodline, Elemental]}
\featpre Any three elemental bloodline feats
\featben You have a 50\% chance to ignore critical hits on you, treating them as regular hits instead.

\subsubsection{Elemental Force [Bloodline, Elemental]}
\featpre Elemental Heritage
\featben Once per day per elemental bloodline feat you possess, you may unleash the power of your element on your foe as an attack action. Air allows you to make a bull rush attack with a \plus4 circumstance bonus, and you use your Constitution in place of your Strength to attack. Earth allows you to make a trip attack with a \plus4 circumstance bonus, and you use your Constitution in place of your Strength to attack. Fire allows you to make a touch attack that ignites your foe for 5 rounds if you hit. Water allows you to make a touch attack that dehydrates your foe for 5 rounds if you hit, making it vulnerable.

\subsubsection{Elemental Heritage [Bloodline, Elemental]}
\featben You have the essence of an elemental in your body. When you take this feat, choose a type of elemental to be your elemental ancestor: air, earth, fire, or water. Air and fire elemental heritage grants a \plus2 bonus to Reflex saves, while earth and water elemental heritage grant a \plus2 bonus to Fortitude saves.

\subsubsection{Elemental Mastery [Bloodline, Elemental]}
\featpre Elemental Heritage
\featben In circumstances that depend on your elemental ancestor, you gain a \plus1 circumstance bonus to attack rolls. Air grants a bonus when you are airborne or fighting airborne creatures. Earth grants a bonus when both you and your foe are standing on unworked earth or stone. Fire grants a bonus when either you or or foe is ignited, or when you are making attacks that deal fire damage. Water grants a bonus when both you and your foe are touching water. If you have four or more elemental bloodline feats, this bonus increases to \plus2.

\subsubsection{Elemental Movement [Bloodline, Elemental]}
\featpre Elemental Heritage
\featben You gain a movement ability based on you choice of elemental ancestor. Air halves the damage you take from falling and improves the maneuverability of any flight abilities you possess by one category. Earth gives you a \plus2 competence bonus to CMD against attacks that would force you to move, such as bull rush and trip attacks. Fire gives you a \plus5 foot competence bonus to your movement speed. Water gives you a swim speed equal to your base land speed.

\subsection{Metamagic Feats}

\subsubsection{Enlarge Spell [Metamagic]}
\parhead{Prerequisite} Caster level 4th.
\parhead{Benefit} An enlarged spell has its range doubled. This metamagic can only be applied to spells with a range of \rngclose, \rngmed, or \rnglong. An enlarged spell uses up a spell slot one level higher than the spell's actual level.

\begin{comment}
\subsubsection{Extend Spell [Metamagic]}
\parhead{Prerequisite} Caster level 8th.
\parhead{Benefit} An extended spell has its duration increased by one duration category: from Short, to Medium, to Long, to Extreme. This metamagic can only be applied to spells with a duration of \durshort, \durmed, or \durlong. An extended spell uses up a spell slot three levels higher than the spell's actual level.
\end{comment}

\subsubsection{Empower Spell [Metamagic]}
\parhead{Prerequisite} Caster level 4th.
\parhead{Benefit} When casting a heightened spell, you gain a \plus2 circumstance bonus to caster level. A heightened spell uses up a spell slot one level higher than the spell's actual level. Unlike other metamagic feats, you can apply this metamagic feat any number of times, increasing your caster level by 2 each time.

\subsubsection{Energetic Substitution [Metamagic]}
\parhead{Prerequisite} Caster level 4th.
\parhead{Benefit} When casting a substituted spell, you can choose what kind of energy damage it deals: cold, fire, or electricity. This can only be applied to spells that originally dealt cold, fire, or electricity damage. A substituted spell uses up a spell slot one level higher than the spell's actual level.

\subsubsection{Imbued Spellstrike [Metamagic]}
\parhead{Prerequisite} Caster level 4th.
\parhead{Benefit} As part of casting an imbued spellstrike spell, you can make a single attack with a weapon in your hand. If the attack hits, the struck creature is affected by the spell, as if it had been the target, in addition to taking damage from the weapon. The imbuement fades away without effect after 1 round (at the end of your next turn) if you have not struck a foe.

Only spells which affect a single target can be channeled in this way. An imbued spellstrike uses up a spell slot one level higher than the spell's actual level.

\subsubsection{Improved Imbued Spellstrike [Metamagic]}
\parhead{Prerequisite} Caster level 6th, Imbued Strike.
\parhead{Benefit} This metamagic functions like Imbued Strike, except that the imbuement lasts for 5 minutes if you have not struck a foe. If the weapon leaves your hands or if you cast another spell, the imbuement fades away without effect.

An improved imbued spellstrike uses up a spell slot two levels higher than the spell's actual level.

\subsubsection{Quicken Spell [Metamagic]}
\parhead{Prerequisite} Caster level 6th.
\parhead{Benefit} Casting a quickened spell is a swift action. You can perform another action, even casting another spell, in the same round as you cast a quickened spell. However, casting a quickened spell is mentally exhausting. In the turn after you cast it, you lose your standard action. You may cast only one quickened spell per round. A spell whose casting time is more than 1 standard action cannot be quickened. A quickened spell uses up a spell slot two levels higher than the spell's actual level. Casting a quickened spell doesn't provoke an attack of opportunity.
\parhead{Special} All spellcasters cast a quickened spell as a swift action, even if they would normally increase the casting time of spells with metamagic applied. This is an exception to the general rule that applying metamagic increases the casting time of a spell.

\subsubsection{Reach Spell [Metamatic]}
\parhead{Prerequisite} Caster level 6th.
\parhead{Benefit} When casting a reach spell, you can use a spell with a range of touch on a target within \rngclose range. You must succeed on a ranged touch attack. A reach spell uses up a spell slot two levels higher than the spell's actual level.

\subsubsection{Shape Spell [Metamagic]}
\parhead{Prerequisite} Caster level 6th.
\parhead{Benefit} When casting a shaped spell, you can exclude any number of 5-foot cubes within the spell's area. This allows you to prevent the spell from affecting your allies, while still allowing it to affect your enemies. The area affected by the spell must be contiguous.

Only area spells can be shaped. A shaped spell uses a spell slot two levels higher than the spell's original level.

\subsubsection{Silent Spell [Metamagic]}
\parhead{Prerequisite} Caster level 4th.
\parhead{Benefit} A silent spell can be cast with no verbal components. Spells without verbal components are not affected. A silent spell uses up a spell slot one level higher than the spell's actual level.
\parhead{Special} Bard spells cannot be enhanced by this metamagic feat.

\subsubsection{Still Spell [Metamagic]}
\parhead{Prerequisite} Caster level 4th.
\parhead{Benefit} A stilled spell can be cast with no somatic components. Spells without somatic components are not affected. A stilled spell uses up a spell slot one level higher than the spell's actual level.

\subsubsection{Sustained Spell [Metamagic]}
\featpre Caster level 4th.
\featben You can maintain concentration on a sustained spell as a swift action instead of as a standard action. If you cast any other spell, you lose the ability to sustain the spell. This only affects spell duration, and has no effect on spells with special effects based on concentration, such as \spell{call lightning}. A sustained spell uses up a spell slot one level higher than the spell's actual level.

\subsubsection{Widen Spell [Metamagic]}
\parhead{Prerequisite} Caster level 8th.
\parhead{Benefit} You can alter a burst, emanation, line, or spread shaped spell to increase its area. Any numeric measurements of the spell's area increase by 100\%. A widened spell uses up a spell slot three levels higher than the spell's actual level.
\par Spells that do not have an area of one of these four sorts are not affected by this feat.

\subsection{Item Creation Feats}

\subsubsection{Imbue Magic [Item Creation]}
\parhead{Prerequisite} Caster level 2nd or Craft (any) 6 ranks.
\parhead{Benefit} You can imbue items with magic using your spells or crafting ability. Imbuing an item with magic takes time and material components, as described in \pcref{Magic Item Creation}.

When you take this feat, you choose one subschool of magic for every 5 ranks that you have in each Craft skill. You can craft items from those subschools. If you later gain additional Craft ranks, you gain new subschools appropriately.

You can also mend a broken magic item if it is one that you could make. Doing so costs a tenth of the raw materials and a quarter of the time it would take to craft that item in the first place. You cannot mend a destroyed magic item.

\subsubsection{Imbuement Admixture [Item Creation]}
\parhead{Prerequisite}
\parhead{Benefit} You can blend two spells together to create magic items. 

\subsubsection{Versatile Crafter [Item Creation]}
\parhead{Prerequisite} Craft (any) 10 ranks.
\parhead{Benefit} You learn how to make items from one subschool for every two ranks you have in each Craft skill. See the Craft skill description for details.
\parhead{Normal} You learn how to make items from one subschool for every five ranks you have in each Craft skill.
