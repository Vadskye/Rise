\chapter{Feats}
Feats are special abilities that every character has. Feats can be used to specialize your character particular area, to grant your character new abilities, or to change the way your character does certain things.

\section{Gaining Feats}
Your character gains a feat every odd level: 1st, 3rd, 5th, and so on. Classes can sometimes grant bonus feats as well, which are in addition to these feats which every character gets.

\subsection{Bonus Feats}
Some class features and abilities grant a character bonus feats. Unless otherwise specified, the character must still meet any prerequisites for the feat. If the character does not meet the prerequisites at the time the bonus feat is granted, the character does not gain the feat. If the character later meets the prerequisites, the character immediately gains the benefit of the bonus feat.

If a character gains a feat as a bonus feat that he or she has already acquired through other means, the character may select instead any other feat for which she qualifies.

\section{Retraining Feats}
At every even level, your character can choose to retrain an old feat in exchange for a new feat. You can only retrain feats for other feats you could have acquired at the time you took the original feat. Thus, you cannot retrain feats gained through class features which give you a specific feat, since there were no other feats you could have taken. However, a 6th level fighter can retrain his 2nd level fighter bonus feat for any other combat feat that he qualified for at 2nd level.

\section{Prerequisites}
Some feats have prerequisites. Your character must have the indicated attribute score, class feature, feat, skill, base attack bonus, or other quality designated in order to select or use that feat. A character can gain a feat at the same level at which he or she gains the prerequisite.

A character can't use a feat if he or she has lost a prerequisite.

\section{Types Of Feats}
Some feats are general, meaning that no special rules govern
them as a group. Others belong to particular categories. The categories are given below:
\begin{itemize*}
\item Skill feats affect a character's mastery of his or her skills.
\item Combat feats affect a character's prowess in combat.
\item Combat maneuver feats are combat feats which grant a character particular talent with specific maneuvers.
\item Combat style feats grant a character the ability fight in particular styles. A character can only fight in one style at once.
\item Metamagic feats let spellcasters cast spells with greater effect, albeit as if the spells were a higher level than they actually are.
\item Item creation feats allow spellcasters to create magic items of various kinds.
\end{itemize*}

\subsection{Class Feats}
Class feats improve a character's class features.

\subsection{Racial feats}
Racial feats improve a character's racial abilities or grant new abilities unique to the character's race.

\subsection{Skill Feats}
Skill feats always affect a character's ability to use skills. Rogues can gain skill feats with their skill trick class feature.

\subsection{Combat Feats}
Combat feats always affect a character's combat abilities. Fighters gain bonus feats selected from a subset of the feat list presented in \tref{Combat Feats}. Any feat designated as a combat feat can be selected as a fighter's bonus feat.

\subsection{Combat Maneuver Feats}
Combat maneuver feats are combat feats which specifically deal with combat maneuvers -- pushing opponents around, grappling them, tripping them, or other unusual attacks. Any feat designated as a combat maneuver feat is also considered a combat feat and can be selected as a fighter's bonus feat.

\subsection{Combat Style Feats}
Combat style feats grant a character the ability to fight in a particular style, granting them bonuses while fighting in that style. A character can only fight in one style at once. Initiating a style or changing to a different style is a swift action, but a style can be stopped as a free action.

Some style feats have style requirements. To gain the benefits of a combat style with a style requirement, a character must take some action or fulfill some condition during his turn. If the condition is not met, the character automatically stops fighting in that style at the end of his turn. For example, a character using the Power Attack style must charge or full attack with a melee weapon each round. If the character takes the total defense action instead, he stops using the Power Attack style at the end of his turn.

\subsection{Metamagic Feats}
As a spellcaster's knowledge of magic grows, she can learn to cast spells in ways slightly different from the ways in which the spells were originally designed or learned. Preparing and casting a spell in such a way is harder than normal but, thanks to metamagic feats, at least it is possible. Spells modified by a metamagic feat use a spell slot higher than normal. All effects dependent on spell level (such as the ability to penetrate a \spell{lesser globe of invulnerability}) are calculated according to the spell's modified level.

\parhead{Applying Metamagic Feats} Spellcasters apply metamagic feats on the spot. Therefore, most spellcasters must also take more time to cast a metamagic spell (one enhanced by a metamagic feat) than they do to cast a regular spell. If the spell's normal casting time is a standard action, casting a metamagic version is a full-round action. (This isn't the same as a 1-round casting time.) For a spell with a longer casting time, it takes an extra full-round action to cast the spell. For a spell with a shorter casting time, it takes a standard action to cast the spell.

\par Sorcerers have such an intuitive grasp of magic that they do not need to take extra time to cast spells affected by metamagic feats.

\parhead{Spontaneous Casting and Metamagic Feats} A cleric spontaneously casting a \spell{cure} or \spell{inflict} spell can cast a metamagic version of it instead.

\parhead{Effects of Metamagic Feats on a Spell} In all ways, a metamagic spell operates at its original spell level, even though it is prepared and cast as a higher-level spell.

The modifications made by these feats only apply to spells cast directly by the feat user. A spellcaster can't use a metamagic feat to alter a spell being cast from a wand, scroll, or other device.

Metamagic feats that eliminate components of a spell don't eliminate the attack of opportunity provoked by casting a spell while threatened. However, casting a spell modified by Quicken Spell does not provoke an attack of opportunity.

Metamagic feats cannot be used with all spells. See the specific feat descriptions for the spells that a particular feat can't modify.

\parhead{Multiple Metamagic Feats on a Spell} A spellcaster can apply multiple metamagic feats to a single spell. Changes to its level are cumulative. You can't apply the same metamagic feat more than once to a single spell.

\parhead{Magic Items and Metamagic Spells} With the right item creation feat, you can store a metamagic version of a spell in a scroll, potion, or wand. Level limits for potions and wands apply to the spell's higher spell level (after the application of the metamagic feat). A character doesn't need the metamagic feat to activate an item storing a metamagic version of a spell.

\parhead{Counterspelling Metamagic Spells} Whether or not a spell has been enhanced by a metamagic feat does not affect its vulnerability to counterspelling or its ability to counterspell another spell.

\subsection{Item Creation Feats}
An item creation feat lets a spellcaster create a magic item of a certain type. Regardless of the type of items they involve, the various item creation feats all have certain features in common.

\parhead{Raw Materials Cost} The cost of creating a magic item equals one-half the sale cost of the item.

Using an item creation feat also requires access to a laboratory or magical workshop, special tools, and so on. A character generally has access to what he or she needs unless unusual circumstances apply.

\parhead{Negative Levels} Power and energy that a spellcaster would normally have is expended when making a magic item. While crafting a magic item, a spellcaster gains a negative level that cannot be removed. After the magic item is complete, the spellcaster still suffers the negative level for two days per day required to make the item. Scrolls and potions are less draining, and only bestow negative levels for one day per day required to make the item. These negative levels cannot be removed by any means.

\parhead{Time} The time to create a magic item depends on the feat and the cost of the item. The minimum time is one day.

\parhead{Item Cost} Brew Potion, Craft Wand, and Scribe Scroll create items that directly reproduce spell effects, and the power of these items depends on their caster level -- that is, a spell from such an item has the power it would have if cast by a spellcaster of that level. The price of these items (and thus the cost of the raw materials) also depends on the caster level. The caster level must be high enough that the spellcaster creating the item can cast the spell at that level. To find the final price in each case, multiply the caster level by the spell level, then multiply the result by a constant, as shown below:

\subparhead{Scrolls} Base price = spell level \mtimes caster level \mtimes 25 gp.
\subparhead{Potions} Base price = spell level \mtimes  caster level \mtimes 50 gp.
\subparhead{Wands} Base price = spell level \mtimes  caster level \mtimes 750 gp.

\par A 0-level spell is considered to have a spell level of 1/2 for the
purpose of this calculation. The minimum caster level to craft an item is always 2, regardless of the level of the spell used.

\parhead{Extra Costs} Any potion, scroll, or wand that stores a spell with a costly material component also carries a commensurate cost. For potions and scrolls, the creator must expend the material component when creating the item. For a wand, the creator must expend fifty copies of the material component.

\par Some magic items similarly incur extra costs in material components, as noted in their descriptions.

\begin{dtable!*}
\lcaption{General Feats}
\begin{tabularx}{\textwidth}{>{\lcol}p{15em} >{\lcol}p{15em} >{\lcol}X}
\thead{General Feats} & \thead{Prerequisites} & \thead{Benefit} \\
%Brute Fortitude & \x & Use Str and half Con for Fortitute saves \\
Combat Casting  & \x &  \plus2 bonus on Concentration checks to cast spells, reroll 1/day \\
Endurance & \x &  Stay conscious longer against nonlethal damage \\
\tind Diehard & Endurance & Remain conscious after taking critical damage \\
%Eschew Materials  & \x &  Cast spells without material components \\
%Force of Intellect & \x & Use Int and half Cha for Will saves \\
Great Fortitude  & \x &  \plus2 bonus on Fortitude saves, reroll 1/day \\
Improved Counterspell  & \x &  Counterspell with spell of same school \\
\tind Retributive Counterspell & Improved Counterspell, ability to cast 4th level spells & Countered spells rebound on original caster \\
%Intuitive Reflexes & \x & Use Wis and half Dex for Reflex saves \\
Iron Will  & \x &  \plus2 bonus on Will saves, reroll 1/day \\
Lightning Reflexes  & \x &  \plus2 bonus on Reflex saves, reroll 1/day \\
Magical Synthesis & Levels in two magical classes. & Gain magic levels in two magical classes at once. \\
Rapid Metamagic & Spellcraft 8 ranks, ability to cast spells, one metamagic feat & Apply metamagic effects more quickly \\
Ritual Master & Spellcraft 8 ranks, ability to cast spells & Perform rituals more quickly, \plus4 to ritual checks \\
Spell Focus\fn{1} & Caster level 4th &  \plus2 caster level with specific type of magic \\
\tind Augment Summoning & Spell Focus (conjuration) & Summoned creatures gain \plus2 Str, \plus2 Con \\
\tind Greater Spell Focus\fn{1} & Caster level 10th, Spell Focus &  \plus4 caster level with specific type of magic \\
%Spell Penetration  & Caster level 4th &  \plus2 bonus on caster level checks to defeat spell resistance \\
%\tind Greater Spell Penetration & Caster level 10th, Spell Penetration & \plus4 bonus to caster level checks to defeat spell resistance \\
Swift & \x &  Run and charge faster, related bonuses \\
Toughness & \x &  \plus3 hit points \plus1 per level above 3 \\
\end{tabularx}
1 You can gain this feat multiple times. Its effects do not stack. Each time you take the feat, it has a different effect. \\
\end{dtable!*}

\begin{dtable!*}
\lcaption{Class Feats}
\begin{tabularx}{\textwidth}{>{\lcol}p{15em} >{\lcol}p{15em} >{\lcol}X}
\thead{Class Feats} & \thead{Prerequisites} & \thead{Benefit} \\
Extend Rage & Ability to rage & Can rage for 5 rounds longer \\
Extra Channeling & Ability to channel energy & Can turn or rebuke 3 more times per day \\
Extra Invocation & Ability to use an arcane invocation & Learn a new arcane invocation \\
Extra Music & Bardic music & Can use bardic music 3 more times per day \\
Extra Rage & Ability to rage & Can rage 1 more time per day\\
Extra Smiting\fn{1} & Ability to smite & Can smite 3 more times per day \\
Extra Stunning & Stunning Fist & Can use stunning fist 3 more times per day \\
Extra Wild Aspect & Wild aspect & Can use wild aspects 3 more times per day \\
Improved Turning & Ability to turn or rebuke creatures & \plus2 turning level \\
Selective Channeling & Ability to channel energy & Can exclude two additional creatures \\
\end{tabularx}
\end{dtable!*}

\begin{dtable!*}
\lcaption{Racial Feats}
\begin{tabularx}{\textwidth}{>{\lcol}p{15em} >{\lcol}p{15em} >{\lcol}X}
\thead{Racial Feats} & \thead{Prerequisites} & \thead{Benefit} \\
Focused Mind & Elf & Use Intelligence to concentrate instead of Constitution \\
Giantfighter & Dwarf, gnome, or halfling & \plus2 to dodge against Large or larger creatures \\
Gnomish Tricks & Gnome, Cha 0 & Gain minor spell-like abilities \\
Keen Senses & Elf & Automatically notice secret doors, reroll Listen or Spot 1/day \\
Light-Footed & Elf or halfling & \plus3 to Move Silently and become harder to track \\
Stonecunning & Dwarf & Gain a sixth sense about stonework \\
\end{tabularx}
\end{dtable!*}

\begin{dtable!*}
\lcaption{Skill Feats}
\begin{tabularx}{\textwidth}{>{\lcol}p{15em} >{\lcol}p{15em} >{\lcol}X}
\thead{Skill Feats} & \thead{Prerequisites} & \thead{Benefit} \\
Acrobatic  & \x &  \plus2 bonus on Acrobatics and Jump checks, reroll 1/day \\
Agile  & \x &  \plus2 bonus on Escape Artist and Stealth checks, reroll 1/day \\
Animal Affinity  & \x &  \plus2 bonus on Handle Animal and Ride checks, reroll 1/day \\
Athletic  & \x &  \plus2 bonus on Climb and Swim checks, reroll 1/day \\
Deceitful  & \x &  \plus2 bonus on Bluff and Disguise checks, reroll 1/day \\
Diligent  & \x &  \plus2 bonus on Forgery and Linguistics checks, reroll 1/day \\
Dilettante & Int 3 & Use some Knowledge skills despite being untrained \\
Magical Aptitude  & \x &  \plus2 bonus on Spellcraft and Use Magic Device checks, reroll 1/day \\
Nimble Fingers  & \x &  \plus2 bonus on Disable Device and Sleight of Hand checks, reroll 1/day \\
Open Minded & \x & Gain two skill points. \\
Perceptive & \x &  \plus2 bonus on Perception and Sense Motive checks, reroll 1/day \\
Persuasive  & \x &  \plus2 bonus on Intimidate and Persuasion checks, reroll 1/day \\
Ranged Legerdemain & Ability to cast 2nd level spells & Use Disable Device or Sleight of Hand at range \\
Skill Devotion\fn{1} & 13 ranks in a skill & Gain special ability based on the skill \\
Skill Focus\fn{1} & \x &  \plus3 bonus on checks with selected skill, reroll 1/day \\
Survivalist & \x & \plus2 bonus on Heal and Survival checks, reroll 1/day \\
Track  & \x &  Use Survival skill to track \\
\end{tabularx}
1 You can gain this feat multiple times. Its effects do not stack. Each time you take the feat, it has a different effect. \\
\end{dtable!*}
\begin{dtable!*}
\lcaption{Combat Feats}
\begin{tabularx}{\textwidth}{>{\lcol}p{15em} >{\lcol}p{15em} >{\lcol}X}
\thead{Combat Feats} & \thead{Prerequisites} & \thead{Benefit} \\
Armor Proficiency (light)  & \x &  No armor check penalty on attack rolls \\
\tind Armor Proficiency (medium) & Armor Proficiency (light) & No armor check penalty on attack rolls \\
\tind \tind Armor Proficiency (heavy) & Armor Proficiency (medium) & No armor check penalty on attack rolls \\
\tind Whirlwind Attack & Dex 3, Combat Expertise, Dodge, Mobility, Spring Attack, base attack bonus \plus4 & One melee attack against each opponent within reach \\
Cleave & Str 3, base attack bonus \plus4 & Extra melee attack after dropping target \\
%Combat Finesse & \x & Add half Dexterity to damage with light, possibly medium weapons \\
%Combat Reflexes & Dex 3 & \plus4 bonus on attacks of opportunity \\
Dodge & Dex 3 & \plus4 AC against some attacks of opportunity from selected target \\
\tind Mobility & Dodge & Avoid some attacks of opportunity from selected target \\
\tind \tind Spring Attack & Mobility, base attack bonus \plus4 & Move before and after melee attacks \\
Exotic Weapon Proficiency\footnotetemp{1} & Base attack bonus \plus1 & Don't provoke when attacking with exotic weapons \\
Far Shot & \x & Increase range increment by 50\% or 100\% \\
%Improved Critical\footnotetemp{1} & Proficient with weapon, base attack bonus \plus8 & Double threat range of weapon \\
%Improved Fighting Retreat & Base attack bonus \plus6 & Take full attack when making a fighting retreat \\
Improved Initiative & \x &  \plus4 bonus on initiative checks \\
Improved Unarmed Strike & \x &  Considered armed even when unarmed \\
\tind Deflect Arrows & Dex 3, Improved Unarmed Strike & Deflect one ranged attack per round \\
\tind \tind Improved Deflect Arrows & Dex 7, Deflect Arrows, Improved Unarmed Strike, base attack bonus \plus4 & Deflect ranged attacks per round equal to Dexterity \\
\tind \tind Snatch Arrows & Dex 5, Deflect Arrows, Improved Unarmed Strike & Catch a deflected ranged attack \\
\tind Stunning Fist & Dex 3, Wis 3, Improved Unarmed Strike, base attack bonus \plus4 & Stun opponent with unarmed strike \\
\end{tabularx}
\end{dtable!*}

\begin{dtable!*}
\lcaption{Combat Feats (cont.)}
\begin{tabularx}{\textwidth}{>{\lcol}p{15em} >{\lcol}p{15em} >{\lcol}X}
\thead{Combat Feats} & \thead{Prerequisites} & \thead{Benefit} \\
Mounted Combat & Ride 1 rank & Negate hits on mount with Ride check \\
\tind Mounted Archery & Mounted Combat & Reduced penalty for ranged attacks while mounted by 4\\
\tind Ride-By Attack & Mounted Combat & Move before and after a mounted charge \\
\tind \tind Spirited Charge & Mounted Combat, Ride-By Attack & Double damage with mounted charge \\
\tind Trample & Mounted Combat & Target cannot avoid mounted overrun \\
Overwhelming Force & Str 5, base attack bonus \plus8 & Apply full Strength to damage when using two hands \\
Quick Draw & Base attack bonus \plus1 & Draw weapon as swift action \\
Rapid Reload & Proficiency with crossbows & Reload crossbows more quickly \\
Shield Proficiency  & \x &  No armor check penalty on attack rolls \\
\tind Improved Shield Bash & Shield Proficiency & Retain shield bonus to AC when shield bashing \\
\tind Tower Shield Proficiency & Shield Proficiency & No armor check penalty on attack rolls \\
Short Haft & Proficiency with a reach weapon, base attack bonus \plus4 & Short haft reach weapons faster, without penalty \\
Two-Weapon Fighting & Dex 5 & Gain \plus2 attack bonus when fighting with two weapons \\
%\tind Ambidexterity & Str 3, Two-Weapon Fighting, base attack bonus \plus12 & Add full Strength value to damage with both hands \\
\tind Two-Weapon Defense & Two-Weapon Fighting & Off-hand weapon grants \plus1 shield bonus to AC, later \plus3 \\
Weapon Focus\fn{1} & Proficiency with weapon group, base attack bonus \plus1 & \plus1 to attack with selected weapon group \\
\tind Weapon Specialization\footnotetemp{1} & Base attack bonus \plus8, proficiency with weapon group, Weapon Focus with weapon group & \plus2 to attack, double threat range with selected weapon group\\
Weapon Proficiency\fn{1} & \x &  Don't provoke when attacking with weapon group \\
\end{tabularx}
1 You can gain this feat multiple times. Its effects do not stack. Each time you take the feat, it has a different effect. \\
\end{dtable!*}

\begin{dtable!*}
\lcaption{Combat Maneuver Feats}
\begin{tabularx}{\textwidth}{>{\lcol}p{15em} >{\lcol}p{15em} >{\lcol}X}
\thead{Combat Maneuver Feats\fn{1}} & \thead{Prerequisites} & \thead{Benefit} \\
Improved Bull Rush & Base attack bonus \plus4 & \plus2 bonus on bull rush attacks; push target back without moving \\
Improved Dirty Trick & Base attack bonus \plus4 & \plus2 bonus on dirty trick attacks; dirty tricks last longer \\
Improved Disarm & Base attack bonus \plus4 & \plus2 bonus on disarm attacks; knock foe's weapon away after disarming \\
Improved Feint & Base attack bonus \plus4 & \plus2 bonus on feint attacks; feint as move action \\
Improved Grapple & Base attack bonus \plus4 & \plus2 bonus on grapple attacks; attack faster in grapple \\
Improved Overrun & Base attack bonus \plus4 & \plus2 bonus on overrun attacks; target can't avoid \\
%Improved Sunder & Base attack bonus \plus4 & \plus2 bonus on sunder attacks; ignore half hardness of sundered item \\
Improved Trip & Base attack bonus \plus4 & \plus2 bonus on trip attacks; tripped foe provokes attacks of opportunity \\
\end{tabularx}
1 Combat maneuver feats are considered combat feats. Abilities and class features that grant combat feats may be used to select combat maneuver feats.
\end{dtable!*}

\begin{dtable!*}
\lcaption{Combat Style Feats}
\begin{tabularx}{\textwidth}{>{\lcol}p{15em} >{\lcol}p{15em} >{\lcol}X}
\thead{Combat Style Feats\fn{1}} & \thead{Prerequisites} & \thead{Benefit} \\
Blind-Fight & \x &  Reroll miss chance for concealment in melee \\
Cautious Attack & \x & Trade damage for CMD \\
Combat Expertise & Int 3 & Trade attack bonus for AC \\
\tind Improved Combat Expertise & Int 5, Combat Expertise, base attack bonus \plus8 & Trade more attack bonus for AC \\
Covering Fire & Wis 3 & Targets struck by ranged attacks suffer attack penalties \\
Deadly Aim & Dex 3 & Trade ranged attack bonus for damage \\
\tind Improved Deadly Aim & Dex 5, Deadly Aim, base attack bonus \plus8 & Trade more ranged attack bonus for damage \\
Defensive Stance & Base attack bonus \plus6 & Trade ability to move for CMD \\
\tind Improved Defensive Stance & Base attack bonus \plus12 & Trade ability to move for AC \\
Distracting Foe & Base attack bonus \plus4 & Threatened foes suffer Concentration penalties \\
Executioner & Base attack bonus \plus12 & Gain free attacks against foes near death \\
Eye of the Storm & Base attack bonus \plus4 & Reduce overwhelm penalties by 2 \\
Heartseeker & Base attack bonus \plus8 & Double critical threat range \\
Intimidating Stance & Intimidate 8 ranks & Swiftly intimidate struck foes \\
Opportunist & Dex 3 & Bonus to attack and damage on attacks of opportunity \\
Overpowering Assault & Str 3 & Trade AC for combat maneuver bonus \\
Precise Shot & \x & Add half Wis to damage, no penalty for shooting into melee \\
\tind Improved Precise Shot & Dex 7, Precise Shot, base attack bonus \plus8 & Ignore less than total cover/concealment on ranged attack \\
\tind Point Blank Shot & Precise Shot &  \plus2 bonus on ranged damage within 30 ft. \\
%\tind Rapid Shot & Dex 3, Precise Shot, base attack bonus \plus4 & One extra ranged attack each round \\
\tind \tind Manyshot & Dex 9, Precise Shot, base attack bonus \plus11 & Shoot two projectiles simultaneously \\
Power Attack & Str 3 & Trade melee attack bonus for damage \\
\tind Improved Power Attack & Str 5, Power Attack, base attack bonus \plus8 & Trade more melee attack bonus for damage \\
Powerful Grip & Str 9 & Use larger weapons without penalty \\
Shot on the Run & Dex 5, Dodge, Mobility, base attack bonus \plus4 & Move before and after ranged attacks \\
\end{tabularx}
1 Combat style feats are considered combat feats. Abilities and class features that grant combat feats may be used to select combat style feats.
\end{dtable!*}

\begin{dtable!*}
\lcaption{Metamagic Feats}
\begin{tabularx}{\textwidth}{>{\lcol}p{15em} >{\lcol}p{7.5em} >{\lcol}p{7.5em} >{\lcol}X}
\thead{Metamagic Feats} & \thead{Prerequisites} & \thead{Spell Level Increase} & \thead{Benefit} \\
%Empower Spell  & Caster level 4th & \plus2 & Increase spell's variable, numeric effects by 50\% \\
Energetic Substitution & Caster level 4th & \plus1 & Change spell's energy type \\
Enlarge Spell  & Caster level 4th & \plus1 & Double spell's range \\
%Extend Spell  & Caster level 1st & \plus1 & Double spell's duration \\
Heighten Spell  & Caster level 6th & Variable & Cast spell as higher level \\
Imbued Spellstrike & Caster level 4th & \plus1 & Combine spell with weapon attack \\
\tind Improved Imbued Spellstrike & Caster level 6th & \plus2 & Imbue weapon with spell's power up to 5 minutes \\
%Maximize Spell  & Caster level 6th & \plus3 & Maximize spell's variable, numeric effects \\
Quicken Spell  & Caster level 6th & \plus2 & Cast spell as swift action, but lose next action \\
Reach Spell & Caster level 6th & \plus2 & Cast touch spell at Close range \\
Shape Spell & Caster level 6th & \plus2 & Exclude areas within spell's area \\
Silent Spell  & Caster level 4th & \plus1 & Cast spell without verbal components \\
Still Spell  & Caster level 4th & \plus1 & Cast spell without somatic components \\
Widen Spell  & Caster level 8th & \plus3 & Double spell's area
\end{tabularx}
\end{dtable!*}

\begin{dtable!*}
\lcaption{Item Creation Feats}
\begin{tabularx}{\textwidth}{>{\lcol}p{15em} >{\lcol}p{15em} >{\lcol}X}
\thead{Item Creation Feats} & \thead{Prerequisites} & \thead{Benefit} \\
Imbue Magic	          & Caster level 2nd or Craft (any) 5 ranks & Create magic items \\
\tind Imbuement Admixture & Caster level 6th & Combine multiple spells to create items \\
\tind Versatile Crafter & Craft (any) 8 ranks & Learn to craft more varieties of magic \\
\end{tabularx}
\end{dtable!*}

\section{Feat Descriptions}
Here is the format for feat descriptions.

\ssecfake{Feat Name [Type of Feat]}
\parhead{Prerequisite} A minimum attribute score, another feat or feats, a minimum base attack bonus, a minimum number of ranks in one or more skills, or a class level that a character must have in order to acquire this feat. This entry is absent if a feat has no prerequisite. A feat may have more than one prerequisite.
\parhead{Benefit} What the feat enables the character (``you'' in the feat description) to do. If a character has the same feat more than once, its benefits do not stack unless indicated otherwise in the description.
\par In general, characters cannot gain the same feat twice.
\parhead{Normal} What a character who does not have this feat is limited to or restricted from doing. If not having the feat causes no particular drawback, this entry is absent.
\parhead{Special} Additional facts about the feat that may be helpful when you decide whether to acquire the feat.

\subsection{General Feats}

\subsubsection{Augment Summoning [General]}
\parhead{Prerequisite} Spell Focus (conjuration).
\parhead{Benefit} Each creature you conjure with any summon spell gains a \plus2 enhancement bonus to Strength and Constitution for the duration of the spell that summoned it.

\subsubsection{Combat Casting [General]}
\parhead{Benefit} You get a \plus2 competence bonus on Concentration checks made to cast a spell or use a spell-like ability. In addition, once per day you may reroll a Concentration check made to cast a spell.

\subsubsection{Diehard [General]}
\parhead{Prerequisite} Endurance.
\parhead{Benefit} When you take critical damage, you automatically become stable. You may choose to act as if you were disabled, rather than dying. You must make this decision as soon as you take critical damage (even if it isn't your turn). If you do not choose to act as if you were disabled, you immediately fall unconscious.
\par When using this feat, you can take either a single move or standard action each turn, but not both, and you cannot take a full round action. You can take a move action without further injuring yourself, but if you perform any standard action (or any other action deemed as strenuous, including some swift actions, such as casting a quickened spell) you take 1 point of critical damage. You die normally if your critical damage exceeds your Constitution score \add your level.
\parhead{Normal} A character without this feat who takes critical damage is unconscious and dying.

\subsubsection{Extend Rage [General]}
\parhead{Prerequisite} Ability to rage.
\parhead{Benefit} Your rage lasts for five extra rounds.

\subsubsection{Extra Channeling [General]}
\parhead{Prerequisite} Ability to channel energy
\parhead{Benefit} You can channel energy three more times per day. 
\parhead{Normal} Without this feat, a character can channel energy a number of times per day equal to 3 \add half his or her Charisma.

\subsubsection{Extra Invocation [General]}
\parhead{Prerequisite} Ability to use an arcane invocation.
\parhead{Benefit} You learn a new arcane invocation. You cannot learn invocations from your prohibited schools, if any, with this feat.

\subsubsection{Extra Music [General]}
\parhead{Prerequisite} Bardic music ability.
\parhead{Benefit} You gain three extra uses of your bardic music ability.

\subsubsection{Extra Rage [General]}
\parhead{Prerequisite} Ability to rage.
\parhead{Benefit} You can rage one more time per day.

\subsubsection{Extra Smiting [General]}
\parhead{Prerequisite} Smite ability.
\parhead{Benefit} You can smite three more times per day. If you have more than one smite ability, you choose which ability this applies to.
\parhead{Special} You can take more this once if you have more than one smite ability. Its effects do not stack. Each time you take the feat, you choose a different smite ability you have and apply the effects to that ability.

\subsubsection{Extra Stunning [General]}
\parhead{Prerequisite} Stunning Fist
\parhead{Benefit} You gain three extra uses of your Stunning Fist ability.

\subsubsection{Extra Wild Aspect [General]}
\parhead{Prerequisite} Wild aspect ability.
\parhead{Benefit} You gain three extra uses of your wild aspect ability.

\subsubsection{Great Fortitude [General]}
\parhead{Benefit} You get a \plus2 competence bonus on all Fortitude saving throws. Once per day, you may reroll a Fortitude save. You must decide to use this ability before the results are revealed. You must take the second roll, even if it is worse.

\subsubsection{Greater Spell Focus [General]}
Choose a school of magic or a spell descriptor to which you already have applied the Spell Focus feat.
\parhead{Prerequisites} Caster level 10th with the school or descriptor, Spell Focus.
\parhead{Benefit} You get a \plus4 competence bonus to your caster level when casting spells and using spell-like abilities from the school of magic or with the spell descriptor you select.
\parhead{Special} You can gain this feat multiple times. Each time you take the feat, it applies to a new school of magic or spell descriptor to which you already have applied the Spell Focus feat.

%\subsubsection{Greater Spell Penetration [General]}
%\parhead{Prerequisites} Caster level 10th, Spell Penetration.
%\parhead{Benefit} You get a \plus4 competence bonus on caster level checks (1d20 \add caster level) made to overcome a creature's spell resistance.

\subsubsection{Improved Counterspell [General]}
\parhead{Benefit} When counterspelling, you may use a spell of the same school that is one or more spell levels higher than the target spell.
\parhead{Normal} Without this feat, you may counter a spell only with the same spell or with a spell specifically designated as countering the target spell.

%\subsubsection{Improved Turning [General]}
%\parhead{Prerequisite} Ability to turn or rebuke creatures.
%\parhead{Benefit} You gain a \plus2 competence bonus to your turning level when turning or rebuke creatures. If you have multiple abilities that let you turn or rebuke creatures, this applies to all of them.

\subsubsection{Iron Will [General]}
\parhead{Benefit} You get a \plus2 competence bonus on all Will saving throws. Once per day, you may reroll a Will save. You must decide to use this ability before the results are revealed. You must take the second roll, even if it is worse.

\subsubsection{Lightning Reflexes [General]}
\parhead{Benefit} You get a \plus2 competence bonus on all Reflex saving throws. Once per day, you may reroll a Reflex save. You must decide to use this ability before the results are revealed. You must take the second roll, even if it is worse.

\subsubsection{Magical Synthesis [General]}
Choose two magical classes you possess.
\parhead{Prerequisites} Levels in two magical classes.
\parhead{Benefit} When gaining levels in either of your chosen classes, you gain magic levels in the other class as if the class you are gaining levels in did not progress your magic level. See \pcref{Magic Level and Multiclassing}.

\subsubsection{Rapid Metamagic [General]}
\parhead{Prerequisites} Spellcraft 8 ranks, ability to cast spells, one metamagic feat
\parhead{Benefit} When you apply a metamagic feat to a spell, the spell only takes its normal casting time.
\parhead{Normal} Applying metamagic takes a standard action (if the spell normally requires less than a standard action), a full-round action (if the spell normally requires a standard action), or an additional full-round action (if the spell takes 1 full round or longer to cast).
\parhead{Special} At 1st level, a sorcerer gets Rapid Metamagic as a bonus feat, even if he does not have the prerequisites.

\subsubsection{Retributive Counterspell [General]}
\parhead{Prerequisites} Improved Counterspell, ability to cast 4th level spells.
\parhead{Benefit} As part of the action to counter a spell, you may expend an Abjuration (Negation) spell of 4th level or higher. If you do, the counterspelled spell is turned back on the caster as if it were affected by the \spell{spell turning} spell. If it cannot be affected by \spell{spell turning}, such as if it is a spell that only affects the caster, it is simply countered as normal.

\subsubsection{Ritual Master [General]}
\parhead{Prerequisites} Spellcraft 8 ranks, ability to cast spells
\parhead{Benefit} You can perform rituals in half the normal time. In addition, you gain a \plus4 competence bonus to checks made to perform rituals.
\parhead{Special} At 1st level, a wizard gets Ritual Master as a bonus feat, even if she does not have the prerequisites.

\subsubsection{Selective Channeling [General]}
\parhead{Prerequisites} Ability to channel energy
\parhead{Benefit} You can exclude up to two additional creatures from the effect when you channel energy.
\parhead{Normal} You can exclude a number of creatures from the effect equal to 1 \add half your Wisdom.

\subsubsection{Spell Focus [General]}
Choose a school of magic or a spell descriptor.
\parhead{Prerequisite} Caster level 4th
\parhead{Benefit} You get a \plus2 competence bonus to your caster level when casting spells and using spell-like abilities from the school of magic or with the spell descriptor you select.
\parhead{Special} You can gain this feat multiple times. Its effects do not stack. Each time you take the feat, it applies to a new school of magic.

%\subsubsection{Spell Penetration [General]}
%\parhead{Prerequisite} Caster level 4th
%\parhead{Benefit} You get a \plus2 competence bonus on caster level checks (1d20 \add caster level) made to overcome a creature's spell resistance.

\subsubsection{Swift [General]}
\parhead{Benefit} When charging, you move at three times your normal speed. When running, you move five times your normal speed (if wearing medium, light, or no armor and carrying no more than a medium load) or four times your speed (if wearing heavy armor or carrying a heavy load).  If you make a jump after a running start (see the Jump skill description), you gain a \plus4 circumstance bonus on your Jump check. You only need 10 feet of movement to get a running start for a Jump check. While running, you are not flat-footed.
\parhead{Normal} You move four times your speed while running (if wearing medium, light, or no armor and carrying no more than a medium load) or three times your speed (if wearing heavy armor or carrying a heavy load), you move at twice your speed when charging, and you become flat-footed while running.

\subsubsection{Toughness [General]}
\parhead{Benefit} You gain \plus3 hit points, \plus1 per level above 3.

\subsection{Racial Feats}

\subsubsection{Focused Mind [Racial]}
\parhead{Prerequisite} Elf
\parhead{Benefit} You can use your Intelligence instead of your Constitution on Concentration checks, such as when you cast spells.

\subsubsection{Giantfighter [Racial]}
\parhead{Prerequisite} Dwarf, gnome, or halfling
\parhead{Benefit} You gain a \plus2 competence bonus to dodge modifier against creatures of size Large or larger.

\subsubsection{Keen Senses [Racial]}
\parhead{Prerequisite} Elf
\parhead{Benefit} If you come within 10 feet of a secret or concealed door, you can make a Spot check to notice it as if you were actively searching. In addition, once per day you may reroll a skill check with either Listen or Spot.

\subsubsection{Light-Footed [Racial]}
\parhead{Prerequisite} Elf or halfling
\parhead{Benefit} You gain a \plus3 competence bonus to Move Silently checks. In addition, the DC to follow tracks you leave increases by 5.

\subsubsection{Stonecunning [Racial]}
\parhead{Prerequisite} Dwarf
\parhead{Benefit} You gain a \plus2 competence bonus to Craft and Spot checks related to stone or metal. In addition, if you come within 10 feet of unusual stonework, you can make a Spot check to notice it as if you were actively searching. Finally, you can also intuit depth, sensing your approximate depth underground as naturally as a human can sense which way is up.

\subsection{Skill Feats}

\subsubsection{Acrobatic [Skill]}
\parhead{Benefit} You get a \plus2 competence bonus on all Acrobatics checks and Jump checks. In addition, once per day you may reroll a skill check with either skill.

\subsubsection{Agile [Skill]}
\parhead{Benefit} You get a \plus2 competence bonus on all Escape Artist checks and Stealth checks. In addition, once per day you may reroll a skill check with either skill.

\subsubsection{Animal Affinity [Skill]}
\parhead{Benefit} You get a \plus2 competence bonus on all Handle Animal checks and Ride checks. In addition, once per day you may reroll a skill check with either skill.

\subsubsection{Athletic [Skill]}
\parhead{Benefit} You get a \plus2 competence bonus on all Climb checks and Swim checks. In addition, once per day you may reroll a skill check with either skill.

\subsubsection{Deceitful [Skill]}
\parhead{Benefit} You get a \plus2 competence bonus on all Bluff and Disguise checks. In addition, onec per day you may reroll a skill check with either skill.

\subsubsection{Dilettante [Skill]}
\parhead{Prerequisite} Int 3
\parhead{Benefit} Choose a number of Knowledge skills equal to your Intelligence. You are treated as trained in those skills, even if you possess no ranks, allowing you to make Knowledge checks in those areas. If your Intelligence increases after taking this feat, you may choose additional Knowledge skills.

\subsubsection{Diligent [Skill]}
\parhead{Benefit} You get a \plus2 competence bonus on all Forgery and Linguistics checks. In addition, once per day you may reroll a skill check with either skill.

\subsubsection{Magical Aptitude [Skill]}
\parhead{Benefit} You get a \plus2 competence bonus on all Spellcraft checks and Use Magic Device checks. In addition, once per day you may reroll a skill check with either skill.

\subsubsection{Nimble Fingers [Skill]}
\parhead{Benefit} You get a \plus2 competence bonus on all Disable Device checks and Sleight of Hand checks. In addition, once per day you may reroll a skill check with either skill.

\subsubsection{Open Minded [Skill]}
\parhead{Benefit} You gain two skill points. You may spend these skill points immediately.

\subsubsection{Perceptive [Skill]}
\parhead{Benefit} You get a \plus2 competence bonus on all Perception checks and Sense Motive checks. In addition, once per day you may reroll a skill check with either skill.

\subsubsection{Persuasive [Skill]}
\parhead{Benefit} You get a \plus2 competence bonus on all Intimidate checks and Persuasion checks. In addition, once per day you may reroll a skill check with either skill.

\subsubsection{Ranged Legerdemain [Skill]}
\parhead{Prerequisite} Ability to cast 2nd level spells
\parhead{Benefit} By expending an Evocation (Control) spell of 2nd level or higher, you can use the Disable Device or Sleight of Hand skills at \rngclose range for a number of rounds equal to half the level of the spell slot.

\subsubsection{Skill Devotion [Skill]}
Choose a skill.
\parhead{Prerequisite} 13 ranks in the skill
\parhead{Benefit} You gain a special ability based on the skill. Consult the list below.
\begin{itemize*}
\item Climb: You gain a climb speed equal to your land speed.
\item Jump: 
\item Swim: You gain a swim speed equal to your land speed.
\item Acrobatics: You can make Acrobatics checks while moving at full speed without penalty.
\item Escape Artist: You can escape from bindings or a grapple as a move action without penalty.
\item Ride: 
\item Stealth: You can make Stealth checks while moving at full speed without penalty.
\item Sleight of Hand: You can make Sleight of Hand checks as a swift action without penalty.
\item Craft: You do not need to choose a DC before crafting. Your progress is always equal to your result \mtimes your result, provided you beat the base DC required to craft the item. This only applies to one Craft skill of your choice.
\item Disable Device: You can make Disable Device checks as a standard action, regardless of what you are trying to disable.
\item Knowledge: You can take 10 on Knowledge checks. This only applies to one Knowledge skill of your choice.
\item Linguistics: 
\item Heal: 
\item Perception: You can make Perception checks as a swift action, except checks made to search. You can search as a move action.
\item Survival: 
\item Bluff: 
\item Intimidate:
\item Handle Animal: 
\item Perform:
\item Persuasion: You can make Persuasion checks as a full-round action without penalty.
\item Use Magic Device: You can take 10 on Use Magic Device checks if you would normally be able to take 10.
\end{itemize*}
\parhead{Special} You can gain this feat multiple times. Its effects do not stack. Each time you take the feat, it applies to a new skill.

\subsubsection{Skill Focus [Skill]}
Choose a skill.
\parhead{Benefit} You get a \plus3 competence bonus on all checks involving that skill. In addition, you once per day you may reroll a skill check with this skill.
\parhead{Special} You can gain this feat multiple times. Its effects do not stack. Each time you take the feat, it applies to a new skill.

\subsubsection{Survivalist [Skill]}
\parhead{Benefit} You get a \plus2 competence bonus on all Heal checks and Survival checks. In addition, once per day you may reroll a skill check with either skill.

\subsubsection{Track [Skill]}
\parhead{Benefit} To find tracks or to follow them for 1 mile requires a successful Survival check. You must make another Survival check every time the tracks become difficult to follow.
You move at half your normal speed (or at your normal speed with a \minus5 penalty on the check, or at up to twice your normal speed with a \minus20 penalty on the check). The DC depends on the surface and the prevailing conditions, as given on the table below:

\begin{dtable}
\begin{tabularx}{\columnwidth}{>{\lcol}X c >{\lcol}X c}
\thead{Surface} & \thead{Survival DC}  & \thead{Surface} & \thead{Survival DC} \\
Very soft ground  & 5  & Firm ground  & 15 \\
Soft ground  & 10  & Hard ground  & 20
\end{tabularx}
\end{dtable}
\par \emph{Very Soft Ground:} Any surface (fresh snow, thick dust, wet mud) that holds deep, clear impressions of footprints.
\par \emph{Soft Ground:} Any surface soft enough to yield to pressure, but firmer than wet mud or fresh snow, in which a creature leaves frequent but shallow footprints.
\par \emph{Firm Ground:} Most normal outdoor surfaces (such as lawns, fields, woods, and the like) or exceptionally soft or dirty indoor surfaces (thick rugs and very dirty or dusty floors). The creature might leave some traces (broken branches or tufts of hair), but it leaves only occasional or partial footprints.
\par \emph{Hard Ground:} Any surface that doesn't hold footprints at all, such as bare rock or an indoor floor. Most streambeds fall into this category, since any footprints left behind are obscured or washed away. The creature leaves only traces (scuff marks or displaced pebbles).
\par Several modifiers may apply to the Survival check, as given on the table below.

\begin{dtable}
\begin{tabularx}{\columnwidth}{>{\lcol}X >{\rcol}p{6em}}
\thead{Condition}  & \thead{Survival DC Modifier} \\
Every three creatures in the group being tracked  & \minus1 \\
Size of creature or creatures being tracked:\footnotetemp{1} &  \\
Fine  & \plus8 \\
Diminutive  & \plus4 \\
Tiny  & \plus2 \\
Small  & \plus1 \\
Medium  & \plus0 \\
Large  & \minus1 \\
Huge  & \minus2 \\
Gargantuan  & \minus4 \\
Colossal  & \minus8 \\
Every 24 hours since the trail was made  & \plus1 \\
Every hour of rain since the trail was made  & \plus1 \\
Fresh snow cover since the trail was made  & \plus10 \\
\thead{Poor visibility:\footnotetemp{2}} &  \\
Overcast or moonless night  & \plus6 \\
Moonlight  & \plus3 \\
Fog or precipitation  & \plus3 \\
Tracked party hides trail (and moves at half speed)  & \plus5
\end{tabularx}
1 For a group of mixed sizes, apply only the modifier for the largest size category. \\
2 Apply only the largest modifier from this category.
\end{dtable}

If you fail a Survival check, you can retry after 1 hour (outdoors) or 10 minutes (indoors) of searching.
\parhead{Normal} Without this feat, you can use the Survival skill to find tracks, but you can follow them only if the DC for the task is 10 or lower. Alternatively, you can use the Search skill to find a footprint or similar sign of a creature's passage using the DCs given above, but you can't use Search to follow tracks, even if someone else has already found them.
\parhead{Special} A ranger automatically has Track as a bonus feat. He need not select it.

\subsection{Combat Feats}

\subsubsection{Armor Proficiency (Heavy) [Combat]}
\parhead{Prerequisites} Armor Proficiency (light), Armor Proficiency (medium).
\parhead{Benefit} See Armor Proficiency (light).
\parhead{Normal} See Armor Proficiency (light).
\parhead{Special} Fighters, paladins, and clerics automatically have Armor Proficiency (heavy) as a bonus feat. They need not select it.

\subsubsection{Armor Proficiency (Light) [Combat]}
\parhead{Benefit} When you wear a type of armor with which you are proficient, the armor check penalty for that armor applies only to Balance, Climb, Escape Artist, Hide, Jump, Move Silently, Sleight of Hand, and Tumble checks. You suffer the armor's normal arcane spell failure.
\parhead{Normal} A character who is wearing armor with which she is not proficient applies its armor check penalty to attack rolls and to all skill checks that involve moving, including Ride. The character also suffers double the normal arcane spell failure chance for wearing the armor.
\parhead{Special} All characters except wizards, sorcerers, and monks automatically have Armor Proficiency (light) as a bonus feat. They need not select it.

\subsubsection{Armor Proficiency (Medium) [Combat]}
\parhead{Prerequisite} Armor Proficiency (light).
\parhead{Benefit} See Armor Proficiency (light).
\parhead{Normal} See Armor Proficiency (light).
\parhead{Special} Fighters, barbarians, paladins, clerics, druids, and bards automatically have Armor Proficiency (medium) as a bonus feat. They need not select it.

\subsubsection{Cleave [Combat]}
\parhead{Prerequisites} Str 3, Power Attack.
\parhead{Benefit} Whenever you deal a creature enough damage to make it drop (typically by dropping it to below 0 hit points or killing it), you get an immediate extra melee attack against another creature within reach. You cannot move before making this extra attack. The extra attack is with the same weapon and at the same bonus as the attack that dropped the previous creature. There is no limit to the number of times you can use this feat per round.

%\subsubsection{Combat Finesse [Combat]}
%\parhead{Benefit} You add half your Dexterity value to your damage rolls with light weapons. This is in addition to the damage provided by your Strength value. If you have a 5 Strength, you may also apply half your Dexterity value to your damage rolls with medium weapons.

%\subsubsection{Combat Reflexes [Combat]}
%\parhead{Benefit} When making attacks of opportunity, you gain a \plus4 circumstance bonus to attack rolls.

With this feat, you may also make attacks of opportunity while flat-footed.
\parhead{Normal} A character without this feat can't make attacks of opportunity while flat-footed.
\parhead{Special} A monk may select Combat Reflexes as a bonus feat at 2nd level.


\subsubsection{Deflect Arrows [Combat]}
\parhead{Prerequisites} Dex 3, Improved Unarmed Strike.
\parhead{Benefit} You must have at least one hand free (holding nothing) to use this feat. Once per round when you would normally be hit with a ranged weapon, you may deflect it so that you take no damage from it. You must be aware of the attack and not flatfooted.
\par Attempting to deflect a ranged weapon doesn't count as an action. Unusually massive ranged weapons and ranged attacks generated by spell effects can't be deflected.
\parhead{Special} A monk may select Deflect Arrows as a bonus feat at 2nd level, even if she does not meet the prerequisites.

\subsubsection{Dodge [Combat]}
\parhead{Prerequisite} Dex 3.
\parhead{Benefit} You may designate an opponent as a free action. You receive a \plus4 circumstance bonus to your dodge modifier against attacks of opportunity from that opponent that were provoked by movement. In addition, you gain a \plus4 circumstance bonus to Acrobatics checks made to tumble against that opponent.

\subsubsection{Endurance [General]}
\parhead{Benefit} You do not become unconscious when your nonlethal damage exceeds your hit points. Your nonlethal damage must exceed double your current hit points before you suffer any ill effect. Also, you may sleep in light or medium armor without becoming fatigued.
\parhead{Normal} A character without this feat who sleeps in medium or heavier armor is automatically fatigued the next day.
\parhead{Special} A barbarian automatically gains Endurance as a bonus feat at 3rd level. He need not select it.

\subsubsection{Exotic Weapon Proficiency [Combat]}
You understand how to use exotic weapons in combat.
\parhead{Prerequisite} Base attack bonus \plus1
\parhead{Benefit} You don't provoke attacks of opportunity when attacking with exotic melee weapons from weapon groups that you are proficient with, and you can use exotic ranged weapons from those groups without penalty.
\parhead{Normal} When using a melee weapon with which you are not proficient, you provoke an attack of opportunity when you miss. When using a ranged weapon with which you are not proficient, you take a \minus4 penalty to attack rolls.

\subsubsection{Far Shot [Combat]}
\parhead{Benefit} When you use a projectile weapon, such as a bow, its range increment increases by one-half (multiply by 1-1/2). When you use a thrown weapon, its range increment is doubled.

\subsubsection{Improved Deflect Arrows [Combat]}
\parhead{Prerequisites} Dex 7, Deflect Arrows, Improved Unarmed Strike, base attack bonus \plus4
\parhead{Benefit} Each round, you can deflect a number of arrows equal to your Dexterity.%Improved Deflect Arrows

%\subsubsection{Improved Fighting Retreat [Combat]}
%\parhead{Prerequisites} Combat Reflexes, base attack bonus \plus6
%\parhead{Benefit} You can take a full attack when taking the fighting retreat action. You do not provoke attacks of opportunity for your movement from any foe you attack.
%\parhead{Normal} You can only make a single attack when taking the fighting retreat action.

\subsubsection{Improved Initiative [Combat]}
\parhead{Benefit} You get a \plus4 competence bonus on initiative checks.

\subsubsection{Improved Shield Bash [Combat]}
\parhead{Prerequisite} Shield Proficiency.
\parhead{Benefit} When you perform a shield bash, you may still apply the shield's shield bonus to your AC.
\parhead{Normal} Without this feat, a character who performs a shield bash loses the shield's shield bonus to AC until his or her next turn.%Improved Shield Bash

\subsubsection{Improved Unarmed Strike [Combat]}
\parhead{Benefit} You are considered to be armed even when unarmed -- that is, you do not provoke attacks or opportunity from armed opponents when you attack them while unarmed. However, you still get an attack of opportunity against any opponent who makes an unarmed attack on you.
In addition, your unarmed strikes can deal lethal or nonlethal damage, and can be considered light or medium weapons, at your option.
\parhead{Normal} Without this feat, you are considered unarmed when attacking with an unarmed strike, you can deal only nonlethal damage with such an attack, and unarmed strikes are always considered light weapons.
\parhead{Special} A monk automatically gains Improved Unarmed Strike as a bonus feat at 1st level. She need not select it.

\subsubsection{Mobility [Combat]}
\parhead{Prerequisites} Dex 3, Dodge.
\parhead{Benefit} When you move, it does not provoke attacks of opportunity from your Dodge target.

\subsubsection{Mounted Archery [Combat]}
\parhead{Prerequisites} Ride 1 rank, Mounted Combat.
\parhead{Benefit} The penalty you take when using a ranged weapon while mounted is decreased by 4: \minus0 instead of \minus4 if your mount is taking a double move, and \minus4 instead of \minus8 if your mount is running.

\subsubsection{Mounted Combat [Combat]}
\parhead{Prerequisite} Ride 1 rank.
\parhead{Benefit} Once per round when your mount is hit in combat, you may attempt a Ride check (as a reaction) to negate the hit. The hit is negated if your Ride check result is greater than the opponent's attack roll. (Essentially, the Ride check result becomes the mount's Armor Class if it's higher than the mount's regular AC.)%Mounted Combat

\subsubsection{Overwhelming Force [Combat]}
\parhead{Prerequisite} Str 5, base attack bonus \plus8
\parhead{Benefit} You add your full Strength to damage when wielding a medium or large melee weapon in two hands.
\parhead{Normal} Without this feat, you add half your Strength to damage.

\subsubsection{Quick Draw [Combat]}
\parhead{Prerequisite} Base attack bonus \plus1.
\parhead{Benefit} You can draw a weapon as a swift action instead of as a move action. You can draw a hidden weapon (see the Sleight of Hand skill) as a move action.
\par A character who has selected this feat may throw weapons at his full normal rate of attacks (much like a character with a bow).
\parhead{Normal} Without this feat, you may draw a weapon as a move action, or (if your base attack bonus is \plus1 or higher) as a swift action as part of movement. Without this feat, you can draw a hidden weapon as a standard action.%Quick Draw

\subsubsection{Rapid Reload [Combat]}
\parhead{Prerequisite} Proficiency with crossbows.
\parhead{Benefit} The time required for you to reload crossbows is reduced to a free action (for a hand or light crossbow) or a move action (for a heavy crossbow). Reloading a heavy crossbow still provokes an attack of opportunity, but reloading hand and light crossbows does not.
\par With thiis feat, you may fire light and hand crossbows as many times in a full attack action as you could attack if you were using a bow.
\parhead{Normal} A character without this feat needs a move action to reload a hand or light crossbow, or a full-round action to reload a heavy crossbow.

%\subsubsection{Rapid Shot [Combat]}
%\parhead{Prerequisites} Dex 3, Precise Shot, base attack bonus \plus4.
%\parhead{Benefit} You can get one extra attack per round with a projectile weapon. The attack is at your highest base attack bonus, but each attack you make in that round (the extra one and the normal ones) takes a \minus2 penalty. You must use the full attack action to use this feat. This does not stack with any other abilities that grant extra attacks.

\subsubsection{Ride-by Attack [Combat]}
\parhead{Prerequisites} Ride 1 rank, Mounted Combat.
\parhead{Benefit} When you are mounted and use the charge action, you may move and attack as if with a standard charge and then move again (continuing the straight line of the charge). You do not need to attack from the closest possible space when making a ride-by attack. Your total movement for the round can't exceed double your mounted speed. You and your mount do not provoke an attack of opportunity from the opponent that you attack.%Ride-By Attack

\subsubsection{Shield Proficiency [Combat]}
\parhead{Benefit} You can use a shield and take only the standard penalties.
\parhead{Normal} When you are using a shield with which you are not proficient, you take the shield's armor check penalty on attack rolls and on all skill checks that involve moving, including Ride checks.
\parhead{Special} Barbarians, bards, clerics, druids, fighters, paladins, and rangers automatically have Shield Proficiency as a bonus feat. They need not select it.

\subsubsection{Short Haft [Combat]}
\parhead{Prerequisites} Proficiency with a reach weapon, base attack bonus \plus4.
\parhead{Benefit} You can switch grips to short haft a reach weapon (see \pref{Reach Weapons}) as a swift action, and you take no penalty while short hafting it.
\parhead{Normal} Switching grips with a reach weapon takes a move action, and you take a \minus4 penalty while short hafting the weapon.

\subsubsection{Snatch Arrows [Combat]}
\parhead{Prerequisites} Dex 5, Deflect Arrows, Improved Unarmed Strike.
\parhead{Benefit} When using the Deflect Arrows feat, you may catch the weapon instead of just deflecting it. Thrown weapons can immediately be thrown back at the original attacker (even though it isn't your turn) or kept for later use.
\par You must have at least one hand free (holding nothing) to use this feat.%Snatch Arrows

\subsubsection{Spirited Charge [Combat]}
\parhead{Prerequisites} Ride 1 rank, Mounted Combat, Ride-By Attack.
\parhead{Benefit} When mounted and using the charge action, you deal double damage with a melee weapon (or triple damage with a lance) on your attack at the end of the charge.%Spirited Charge

\subsubsection{Spring Attack [Combat]}
\parhead{Prerequisites} Dex 3, Dodge, Mobility, base attack bonus \plus4.
\parhead{Benefit} When attacking with a melee weapon, you can move both before and after each attack, provided that your total distance moved is not greater than your speed. Moving in this way does not provoke an attack of opportunity from your Dodge target, though it might provoke attacks of opportunity from other creatures, if appropriate. You can't use this feat if you are wearing heavy armor.
\par You must move at least 5 feet both before and after you make your attacks in order to utilize the benefits of Spring Attack.%Spring Attack

\subsubsection{Stunning Fist [Combat]}
\parhead{Prerequisites} Dex 3, Wis 3, Improved Unarmed Strike, base attack bonus \plus4.
\parhead{Benefit} You must declare that you are using this feat before you make your attack roll (thus, a failed attack roll ruins the attempt). Stunning Fist forces a foe damaged by your unarmed attack to make a Fortitude saving throw (DC 10 \add 1/2 your character level \add your Wis), in addition to dealing damage normally. A bloodied defender who fails this saving throw is stunned for 1 round (until just before your next action). A stunned character can't act, is flat-footed, and takes a \minus2 penalty to AC. A healthy defender is staggered instead, and can only take a standard action each round. You may attempt a stunning attack once per day for every two levels you have attained (but see Special), and no more than once per round. Constructs, oozes, plants, undead, incorporeal creatures, and creatures immune to critical hits cannot be stunned.
\parhead{Special} A monk may select Stunning Fist as a bonus feat at 1st level, even if she does not meet the prerequisites. A monk who selects this feat may attempt a stunning attack a number of times per day equal to her monk level, plus one more time per day for every four levels she has in classes other than monk.

\subsubsection{Tower Shield Proficiency [Combat]}
\parhead{Prerequisite} Shield Proficiency.
\parhead{Benefit} You can use a tower shield and suffer only the standard penalties.
\parhead{Normal} A character who is using a shield with which he or she is not proficient takes the shield's armor check penalty on attack rolls and on all skill checks that involve moving, including Ride.
\parhead{Special} Fighters and paladins are automatically proficient with tower shields.

\subsubsection{Trample [Combat]}
\parhead{Prerequisites} Ride 1 rank, Mounted Combat.
\parhead{Benefit} When you attempt to overrun an opponent while mounted, your target may not choose to avoid you. Your mount may make one attack with an appropriate natural weapon (hoof, claw, or other leg-based attack) against any target you knock down, gaining the standard \plus4 circumstance bonus on attack rolls against prone targets.

\subsubsection{Two-Weapon Defense [Combat]}
\parhead{Prerequisites} Dex 5, Two-Weapon Fighting.
\parhead{Benefit} When wielding a double weapon or two weapons (not including natural weapons or unarmed strikes), you gain a \plus1 shield bonus to your AC. This bonus increases by 1 once your base attack bonus reaches \plus8, and increases by 1 again once your base attack bonus reaches \plus16.
%\par Additionally, if you are proficient with bucklers, you may keep your buckler's shield bonus to AC while you attack with a weapon in your off-hand.
%\parhead{Normal} Without this feat, a character wielding a buckler loses the buckler's shield bonus to AC when using an off-hand weapon to attack.

\subsubsection{Two-Weapon Fighting [Combat]}
You can fight with a weapon in each hand more effectively.
\parhead{Prerequisite} Dex 5.
\parhead{Benefit} You gain a \plus2 circumstance bonus to attack rolls when attacking with two weapons at once.
\parhead{Special} A 2nd-level ranger is treated as having Two-Weapon Fighting, even if he does not have the prerequisites for it, but only when he is wearing light or no armor.

%\subsubsection{Two-Weapon Rend [Combat]}
%\parhead{Prerequisites} Str 3, Two-Weapon Fighting, base attack bonus \plus11.
%\parhead{Benefit} When fighting with two weapons at once, you may add half your Strength to damage with your offhand.
%\parhead{Normal} Without this feat, you do not apply Strength to damage with your offhand.

\subsubsection{Weapon Focus [Combat]}
Choose one weapon group.
\parhead{Prerequisites} Proficiency with selected weapon group, base attack bonus \plus1.
\parhead{Benefit} You gain a \plus1 competence bonus on all attack rolls you make using weapons from the selected weapon group.
\parhead{Special} You can gain this feat multiple times. Its effects do not stack. Each time you take the feat, it applies to a new type of weapon.
\par You cannot choose simple weapons when you take this feat.

\subsubsection{Weapon Proficiency [Combat]}
Choose a weapon group. You understand how to use weapons from that weapon group in combat.
\parhead{Benefit} You don't provoke attacks of opportunity when attacking with melee weapons from that group, and you can use ranged weapons from that group without penalty.
\parhead{Normal} When using a melee weapon with which you are not proficient, you provoke an attack of opportunity when you miss. When using a ranged weapon with which you are not proficient, you take a \minus4 penalty to attack rolls.
\parhead{Special} You can gain Weapon Proficiency multiple times. Each time you take the feat, it applies to a new weapon group. You cannot choose simple weapons.
\par Clerics who choose the War domain and paladins automatically gain the Weapon Proficiency feat related to their deity's favored weapon group as a bonus feat. They need not select it.

\subsubsection{Weapon Specialization [Combat]}
Choose one weapon group for which you have already selected the Weapon Focus feat. You deal extra damage when using weapons from this weapon group.
\parhead{Prerequisites} Base attack bonus \plus8, proficiency with selected weapon group, Weapon Focus with selected weapon group.
\parhead{Benefit} You gain a \plus2 competence bonus on all attack rolls you make using weapons from the selected weapon group. In addition, when using weapons from the selected weapon group, your threat range is doubled. This effect doesn't stack with any other effect that expands the threat range of a weapon.
\parhead{Special} You can gain this feat multiple times. Its effects do not stack. Each time you take the feat, it applies to a new type of weapon group. You cannot choose simple weapons.

\subsubsection{Whirlwind Attack [Combat]}
\parhead{Prerequisites} Dex 3, Int 3, Combat Expertise, Dodge, Mobility, Spring Attack, base attack bonus \plus4.
\parhead{Benefit} When you use the full attack action, you can give up your regular attacks and instead make one melee attack at your full base attack bonus against each opponent within reach.
\par When you use the Whirlwind Attack feat, you also forfeit any bonus or extra attacks granted by other feats, spells, or abilities.

\subsection{Combat Maneuver Feats}

\subsubsection{Improved Bull Rush [Combat Maneuver]}
\parhead{Prerequisites} Base attack bonus \plus4
\parhead{Benefit} When you perform a bull rush, you do not need to move with the target to move them back. You also gain a \plus2 competence bonus on bull rush attacks.

\subsubsection{Improved Dirty Trick [Combat Maneuver]}
\parhead{Prerequisites} Base attack bonus \plus4
\parhead{Benefit} The conditions imposed by your dirty tricks last for 1d4 rounds. You also gain a \plus2 competence bonus on dirty trick attacks.

\subsubsection{Improved Disarm [Combat Maneuver]}
\parhead{Prerequisites} Base attack bonus \plus4
\parhead{Benefit} When you disarm an opponent, the weapon can land up to 15 feet away in a random direction. You also gain a \plus2 competence bonus on disarm attacks.

\subsubsection{Improved Feint [Combat Maneuver]}
\parhead{Prerequisites} Base attack bonus \plus4
\parhead{Benefit} You can feint in combat as a move action, and you gain a \plus2 competence bonus to feint attacks.
\parhead{Normal} Feinting in combat is an attack action.

\subsubsection{Improved Grapple [Combat Maneuver]}
\parhead{Prerequisites} Base attack bonus \plus4
\parhead{Benefit} While in a grapple, you may make grapple checks as an attack action. You also gain a \plus2 competence bonus on grapple attacks and to your CMD against grapple attacks.
\parhead{Normal} While in a grapple, you make grapple attacks as a standard action.
\par A monk may select Improved Grapple as a bonus feat at 1st level, even if she does not meet the prerequisites.

\subsubsection{Improved Overrun [Combat Maneuver]}
\parhead{Prerequisites} Base attack bonus \plus4
\parhead{Benefit} When you attempt to overrun an opponent, the target may not choose to avoid you unless you let them. You also gain a \plus2 competence bonus on overrun attacks.
\parhead{Normal} Without this feat, the target of an overrun can choose to avoid you or to block you.

\begin{comment}
\subsubsection{Improved Sunder [Combat Maneuver]}
\parhead{Prerequisites} Base attack bonus \plus4
\parhead{Benefit} When you strike at an object held or carried by an opponent (such as a weapon or shield), you ignore half the hardness of the sundered item. You also gain a \plus2 competence bonus on sunder attacks.
\end{comment}

\subsubsection{Improved Trip [Combat Maneuver]}
\parhead{Prerequisites} Base attack bonus \plus4
\parhead{Benefit} When you successfully trip a foe, it immediately provokes an attack of opportunity from everyone threatening it, including you. These attacks are made as the creature is being tripped, so it does not have penalties for being prone. You also gain a \plus2 competence bonus on trip attacks.

\subsection{Combat Style Feats}

\subsubsection{Blind-Fight [Combat Style]}
\parhead{Benefit} In melee, every time you miss because of concealment, you can reroll your miss chance percentile roll one time to see if you actually hit.
\par An invisible attacker gets no advantages related to hitting you in melee. That is, you are not flat-footed, and the attacker doesn't get the usual \plus2 bonus for being invisible. The invisible attacker's bonuses do still apply for ranged attacks, however.
\parhead{Normal} Regular attack roll modifiers for invisible attackers trying to hit you apply, and you are flat-footed against invisible attackers. Your speed is halved in darkness and poor visibility.

\subsubsection{Bulwark of Defense [Combat Style]}
\parhead{Prerequisite} Base attack bonus \plus4
\parhead{Benefit} Foes that you threaten at the start of your turn treat all squares you threaten as difficult terrain. Creatures must pay double movement cost to move through squares of difficult terrain, and cannot run or charge while in difficult terrain.

\subsubsection{Cautious Attack [Combat Style]}
\parhead{Benefit} You can choose to take a \minus2 penalty to damage rolls to gain a \plus2 circumstance bonus to your Combat Maneuver Defense. \bonusscalingdescription
\parhead{Style Requirement} Must full attack each round.

\subsubsection{Combat Expertise [Combat Style]}
\parhead{Prerequisite} Int 3.
\parhead{Benefit} You can choose to take a \minus2 penalty on attack rolls, including combat maneuvers, to gain a \plus2 circumstance bonus to your dodge modifier. \bonusscalingdescription In addition, you increase the bonus gained from using the total defense action (see \pref{Total Defense}) by the same amount.
\parhead{Style Requirement} Must full attack or take the total defense action each round.
\parhead{Normal} A character without the Combat Expertise feat can fight defensively while using the full attack action to take a \minus4 penalty on attack rolls and gain a \plus2 circumstance bonus to its dodge modifier.

\subsubsection{Covering Fire [Combat Style]}
\parhead{Prerequisite} Wis 3.
\parhead{Benefit} If you hit a creature with a ranged attack, it takes a \minus2 penalty to attack rolls for 1 round.
\parhead{Style Requirement} Must full attack with a ranged weapon.

\subsubsection{Deadly Aim [Combat Style]}
\parhead{Prerequisites} Dex 3.
\parhead{Benefit} You take a \minus2 penalty on all ranged attack rolls and gain a \plus2 circumstance bonus on all ranged damage rolls. \bonusscalingdescription The bonus damage does not apply to touch attacks or effects that do not deal hit point damage.
\parhead{Style Requirement} Must full attack with a ranged weapon each round.

\subsubsection{Defensive Stance [Combat Style]}
\parhead{Prerequisites} Base attack bonus \plus6
\parhead{Benefit} You gain a \plus2 circumstance bonus to Combat Maneuver Defense. \bonusscalingdescription However, you are unable to move until you change styles (a swift action). You can only use this style if you did not move in the previous round.

\subsubsection{Distracting Foe [Combat Style]}
\parhead{Prerequisites} Base attack bonus \plus4.
\parhead{Benefit} Foes you threaten take a \minus2 penalty to Concentration checks. This penalty increases to \minus3 at base attack bonus \plus8, to \minus4 at base attack bonus \plus12, and finally to \minus5 at base attack bonus \plus16.
\parhead{Style Requirement} Must wield a melee weapon.

\subsubsection{Executioner [Combat Style]}
\parhead{Prerequisites} Base attack bonus \plus12.
\parhead{Benefit} Whenever a foe you threaten becomes staggered by dropping to 0 hit points, you can immediately make an attack of opportunity against it.
\parhead{Style Requirement} Attack a bloodied foe with a melee weapon each round.

\subsubsection{Eye of the Storm [Combat Style]}
\parhead{Prerequisite} Base attack bonus \plus4.
\parhead{Benefit} You reduce all overwhelm penalties you take by 2. If this reduces your penalties to 0, you are not considered overwhelmed.

\subsubsection{Heartseeker [Combat Style]}
\parhead{Prerequisite} Base attack bonus \plus8.
\parhead{Benefit} You double your critical threat range with any weapon you wield. This does not stack with any other effects which increase threat range.

\subsubsection{Improved Bulwark of Defense [Combat Style]}
\parhead{Prerequisites} Base attack bonus \plus10
\parhead{Benefit} This style functions like the Bulwark of Defense style, except that you can also take attacks of opportunity on foes that 

\subsubsection{Improved Combat Expertise [Combat Style]}
\parhead{Prerequisites} Int 5, Combat Expertise, base attack bonus \plus8.
\parhead{Benefit} This style functions like the Combat Expertise style, except that the penalty to attack rolls and AC bonus is doubled.
\parhead{Style Requirement} Must full attack with a melee weapon or take the total defense action each round.

\subsubsection{Improved Deadly Aim [Combat Style]}
\parhead{Prerequisites} Dex 5, Deadly Aim, base attack bonus \plus8.
\parhead{Benefit} This style functions like the Deadly Aim style, except that the penalty to attack rolls and damage bonus is doubled.
\parhead{Style Requirement} Must charge or full attack with a melee weapon each round.

\subsubsection{Improved Defensive Stance [Combat Style]}
\parhead{Prerequisites} Base attack bonus \plus12, Defensive Stance
\parhead{Benefits} This style functions like Defensive Stance, except that it grants a bonus to Armor Class instead of Combat Maneuver Defense.

\subsubsection{Improved Precise Shot [Combat Style]}
\parhead{Prerequisites} Dex 7, Precise Shot, base attack bonus \plus8.
\parhead{Benefit} This style functions like the Precise Shot style, except that your ranged attacks ignore the AC bonus from cover granted to targets by anything less than total cover, and the miss chance from concealment granted to targets by anything less than total concealment. Total cover and total concealment provide their normal benefits against your ranged attacks. In addition, when you shoot or throw ranged weapons at a grappling opponent, you automatically strike at the opponent you have chosen.

You must spend a full-round action to make a full attack while in this style. You can make a single attack as a standard action.
\parhead{Normal} See the normal rules on the effects of cover and concealment. Without this feat, a character who shoots or throws a ranged weapon at a target involved in a grapple must roll randomly to see which grappling combatant the attack strikes.%Improved Precise Shot

\subsubsection{Improved Power Attack [Combat Style]}
\parhead{Prerequisites} Str 5, Power Attack, base attack bonus \plus8.
\parhead{Benefit} This style functions like the Power Attack style, except that the penalty to attack rolls and damage bonus is doubled.
\parhead{Style Requirement} Must charge or full attack with a melee weapon each round.

\subsubsection{Intimidating Stance [Combat Style]}
\parhead{Prerequisite} Intimidate 8 ranks.
\parhead{Benefit} As a swift action each round, you can make an intimidate check against a single foe you threaten and dealt damage to during the round.
\parhead{Style Requirement} Must deal damage to an opponent each round.

\subsubsection{Manyshot [Combat Style]}
\parhead{Prerequisites} Dex 9, Precise Shot, base attack bonus \plus12.
\parhead{Benefit} When you attack with a projectile weapon (except crossbows), you may make a flurry attack with two projectiles or weapons at once. If the attack hits, the first projectile hits. If the attack hits by 5 or more, both projectiles hit. As normal for flurry attacks, apply precision-based damage (such as sneak attack) and critical hit damage only once for this attack. Damage reduction and resistances apply once to the total damage dealt.
\parhead{Style Requirement} Must full attack with a non-crossbow projectile weapon each round.

\subsubsection{Opportunist [Combat Style]}
\parhead{Prerequisite} Dex 3.
\parhead{Benefit} You gain a \plus2 circumstance bonus to attack and damage on attacks of opportunity. \babscalingdescription
\parhead{Style Requirement} Must wield a melee weapon.

\subsubsection{Overpowering Assault [Combat Style]}
\parhead{Prerequisite} Str 3.
\parhead{Benefit} You take a \minus2 penalty to AC and gain a \plus2 circumstance bonus to combat maneuver attacks. \babscalingdescription The penalty to AC lasts until the start of the next turn after you end the style.
\parhead{Style Requirement} Must perform a combat maneuver each round.

\subsubsection{Point Blank Shot [Combat Style]}
\parhead{Prerequisite} Precise Shot.
\parhead{Benefit} You get a \plus2 circumstance bonus on damage rolls with ranged weapons at ranges of up to 30 feet. \bonusscalingdescription
\parhead{Style Requirement} None.

\subsubsection{Power Attack [Combat Style]}
\parhead{Prerequisite} Str 3.
\parhead{Benefit} You can choose to take a \minus2 penalty on all melee attack rolls and combat maneuver checks to gain a \plus2 bonus on all melee damage rolls. \bonusscalingdescription You must choose to use this feat before making an attack roll, and its effects last until your next turn. The bonus damage does not apply to touch attacks or effects that do not deal hit point damage.
\par This bonus to damage is halved if you are making an attack with an off-hand weapon or light weapon.
\parhead{Style Requirement} Must charge or full attack with a melee weapon each round.

\subsubsection{Powerful Grip [Combat Style]}
\parhead{Prerequisite} Str 9.
\parhead{Benefit} You can use weapons as if they were one category less encumbering than they actually are. The weapon encumbrance categories are light, medium, and heavy. For example, you can use a greatsword as a medium weapon in one hand without suffering any penalties.

\subsubsection{Shot On The Run [Combat Style]}
\parhead{Prerequisites} Dex 5, Dodge, Mobility, base attack bonus \plus4.
\parhead{Benefit} When attacking with a ranged weapon, you can move both before and after each attack, provided that your total distance moved is not greater than your speed.
\parhead{Style Requirement} Must move each round.

%Can't replace Strength; archers need to be strong
\subsubsection{Precise Shot [Combat Style]}
\parhead{Benefit} You can choose to add half your Wisdom to damage with ranged weapons in addition to half your Strength. Additionally, you can shoot or throw ranged weapons at an opponent engaged in melee without taking the standard \minus4 penalty on your attack roll.

You must spend a full-round action to make a full attack while in this style. You can make a single attack as a standard action. You cannot use this style with medium or large thrown weapons.

\subsection{Metamagic Feats}


\subsubsection{Enlarge Spell [Metamagic]}
\parhead{Prerequisite} Caster level 4th.
\parhead{Benefit} An enlarged spell has its range doubled. This metamagic can only be applied to spells with a range of \rngclose, \rngmed, or \rnglong. An enlarged spell uses up a spell slot one level higher than the spell's actual level.

\begin{comment}
\subsubsection{Extend Spell [Metamagic]}
\parhead{Prerequisite} Caster level 8th.
\parhead{Benefit} An extended spell has its duration increased by one duration category: from Short, to Medium, to Long, to Extreme. This metamagic can only be applied to spells with a duration of \durshort, \durmed, or \durlong. An extended spell uses up a spell slot three levels higher than the spell's actual level.
\end{comment}

\subsubsection{Energetic Substitution [Metamagic]}
\parhead{Prerequisite} Caster level 4th.
\parhead{Benefit} When casting a substituted spell, you can choose what kind of energy damage it deals: acid, cold, fire, or electricity. This can only be applied to spells that originally dealt energy damage. A substituted spell uses up a spell slot one level higher than the spell's actual level.

\subsubsection{Heighten Spell [Metamagic]}
\parhead{Prerequisite} Caster level 4th.
\parhead{Benefit} When casting a heightened spell, you gain a \plus2 circumstance bonus to caster level. A heightened spell uses up a spell slot one level higher than the spell's actual level. Unlike other metamagic feats, you can apply this metamagic feat any number of times, increasing your caster level by 2 each time.

\subsubsection{Imbued Spellstrike [Metamagic]}
\parhead{Prerequisite} Caster level 4th.
\parhead{Benefit} When casting a channeled spell, you can make a single attack with a weapon in your hand. If the attack hits, the struck creature is affected by the spell, as if it had been the target, in addition to taking damage from the weapon. The imbuement fades away without effect after 1 round (at the end of your next turn) if you have not struck a foe.

Only spells which affect a single target can be channeled in this way. An imbued spellstrike uses up a spell slot one level higher than the spell's actual level.

\subsubsection{Improved Imbued Spellstrike [Metamagic]}
\parhead{Prerequisite} Caster level 6th, Imbued Strike.
\parhead{Benefit} This metamagic functions like Imbued Strike, except that the imbuement lasts for 5 minutes if you have not struck a foe. If the weapon leaves your hands or you cast another spell, the imbuement fades away without effect.

An improved imbued spellstrike uses up a spell slot two levels higher than the spell's actual level.

\subsubsection{Quicken Spell [Metamagic]}
\parhead{Prerequisite} Caster level 6th.
\parhead{Benefit} Casting a quickened spell is a swift action. You can perform another action, even casting another spell, in the same round as you cast a quickened spell. However, casting a quickened spell is mentally exhausting. In the turn after you cast it, you lose your standard action. You may cast only one quickened spell per round. A spell whose casting time is more than 1 standard action cannot be quickened. A quickened spell uses up a spell slot two levels higher than the spell's actual level. Casting a quickened spell doesn't provoke an attack of opportunity.
\parhead{Special} All spellcasters cast a quickened spell as a swift action, even if they would normally increase the casting time of spells with metamagic applied. This is an exception to the general rule that applying metamagic increases the casting time of a spell.

\subsubsection{Reach Spell [Metamatic]}
\parhead{Prerequisite} Caster level 6th.
\parhead{Benefit} When casting a reach spell, you can use a spell with a range of touch on a target within \rngclose range. You must succeed on a ranged touch attack. A reach spell uses up a spell slot two levels higher than the spell's actual level.

\subsubsection{Shape Spell [Metamagic]}
\parhead{Prerequisite} Caster level 6th.
\parhead{Benefit} When casting a shaped spell, you can exclude any number of 5-foot cubes within the spell's area. This allows you to prevent the spell from affecting your allies, while still allowing it to affect your enemies. The area affected by the spell must be contiguous.

Only area spells can be shaped. A shaped spell uses a spell slot two levels higher than the spell's original level.

\subsubsection{Silent Spell [Metamagic]}
\parhead{Prerequisite} Caster level 4th.
\parhead{Benefit} A silent spell can be cast with no verbal components. Spells without verbal components are not affected. A silent spell uses up a spell slot one level higher than the spell's actual level.
\parhead{Special} Bard spells cannot be enhanced by this metamagic feat.

\subsubsection{Still Spell [Metamagic]}
\parhead{Prerequisite} Caster level 4th.
\parhead{Benefit} A stilled spell can be cast with no somatic components. Spells without somatic components are not affected. A stilled spell uses up a spell slot one level higher than the spell's actual level.

\subsubsection{Widen Spell [Metamagic]}
\parhead{Prerequisite} Caster level 8th.
\parhead{Benefit} You can alter a burst, emanation, line, or spread shaped spell to increase its area. Any numeric measurements of the spell's area increase by 100\%. A widened spell uses up a spell slot three levels higher than the spell's actual level.
\par Spells that do not have an area of one of these four sorts are not affected by this feat.

\subsection{Item Creation Feats}

\subsubsection{Imbue Magic [Item Creation]}
\parhead{Prerequisite} Caster level 2nd or Craft (any) 6 ranks.
\parhead{Benefit} You can imbue items with magic using your spells or crafting ability. Imbuing an item with magic takes time and material components, as described in \pcref{Magic Item Creation}.

You can also mend a broken magic item if it is one that you could make. Doing so costs a tenth of the raw materials and a quarter of the time it would take to craft that item in the first place. You cannot mend a destroyed magic item.

\subsubsection{Imbuement Admixture [Item Creation]}
\parhead{Prerequisite}
\parhead{Benefit} You can blend two spells together to create magic items. 

\subsubsection{Versatile Crafter [Item Creation]}
\parhead{Prerequisite} Craft (any) 10 ranks.
\parhead{Benefit} You learn how to make items from one subschool for every two ranks you have in each Craft skill. See the Craft skill description for details.
\parhead{Normal} You learn how to make items from one subschool for every five ranks you have in each Craft skill.
