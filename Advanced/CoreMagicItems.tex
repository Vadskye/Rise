\chapter{Magic Item Basics}
Magic items are objects that have been imbued with magical energy. They can take almost any form, and their potential uses are only as limited as the magic that created them.

\section{Magic Item Types}
Magic items are divided into two broad categories:
\begin{itemize*}
  \item Apparel items provide access to their powers while worn. A \mitem{flaming burst full plate} and a \mitem{ring of protection} are apparel items.
  \item Implements provide access to their powers when wielded. A \mitem{flaming longsword} and a \mitem{wand of fire} are implements.
  \item Tools provide access to their powers when used in some way. A \mitem{bag of holding} is a tool.
\end{itemize*}

\parhead{Apparel} There are a wide variety of magic items, but there is a limit to how many apparel items a character can wear at any given time. For humanoid-shaped creatures, there are four core areas on the body that can hold magic items: the arms, the head, the torso, and the legs. A creature can wear only one item in the hands, arms, and legs, but it can use two torso items, provided that they are not identical (such as two belts). In addition, it can wear a suit of armor and two magic rings (one on each hand), for a total of eight item slots. Any additional items worn do not function.

Creatures with non-humanoid body structures may have different arrangement of items. For example, a wolf would be able to wear items on its head, its torso, and on two different sets of legs. Regardless of body shape or size, most creatures may not have more than eight item slots, and many have fewer. A list of common items and their placement on the body is given below.

\begin{itemize*}
  \item Arm items:
    \begin{itemize*}
      \item Bracers, bracelets, gauntlets, and gloves.
    \end{itemize*}
  \item Armor
  \item Head items:
    \begin{itemize*}
      \item Hats, headbands, and helmets
    \end{itemize*}
  \item Leg items:
    \begin{itemize*}
      \item Boots and shoes.
    \end{itemize*}
  \item Rings
  \item Torso items:
    \begin{itemize*}
      \item Amulets, belts, cloaks, mantles, necklaces, robes, shirts, and vests.
    \end{itemize*}
\end{itemize*}

Of course, a character may carry or possess as many items of the same type as he wishes. However, additional equipped items beyond those listed above have no effect.

A rare few apparel items can be ``worn'' without taking up space on a character's body. The description of an item indicates when it has this property.

\parhead{Implements} The most common implements are weapons and shields. Spellcasters also often use wands and staves to enhance their power.

\parhead{Tools} Tools can come in many varieties.

\section{Using Magic Items}

To use a magic item, it must be activated, although sometimes activation simply means wearing or holding the item.

\subsection{Activation Methods}
Magic items can be activated in one of four ways:

\begin{itemize*}
  \item Command word
  \item Specific action
  \item Spell completion
  \item Triggered
\end{itemize*}

These methods are described below.

\parhead{Command Word} A character must speak a special word defined by the item to activate it. Unless otherwise stated, activating a command word magic item is a standard action and does not provoke attacks of opportunity.

\parhead{Specific Action} A character must perform a specific action defined by the item to activate it. For example, a creature might need to drink the item, wrap its cloak around itself, or perform some other task. The activation time for such items can be varied. Unless otherwise stated, activating a specific action magic item is a standard action and does not provoke attacks of opportunity.

\parhead{Spell Completion} A character must complete a spell defined by the item to activate it. Unless otherwise stated, activating a spell completion magic item is a standard action and provokes attacks of opportunity as normal for casting. Both verbal and somatic components may be required, as appropriate to the spell to be completed.

In order to activate a spell completion item, a character must be able to cast spells of that level and he must have that spell on his spell list.

\parhead{Triggered} A creature must fulfill some triggering condition defined by the item to activate it. For example, a triggered magic item might activate when a character strikes a foe, is damaged, or is affected by a particular kind of magic. Unless otherwise stated, activating a triggered item is an immediate action that does not provoke attacks of opportunity. It is done completely mentally, requiring no physical action, so it can be done even while paralyzed. Some triggered items activate automatically, without requiring an action of any kind.

\section{Magic Item Effects}
Most magic items either provide numerical bonuses or emulate the effect of a spell in some way.

\subsection{Scaling Bonuses}
A number of magic items provide direct numerical bonuses to a particular aspect of a character. For some items, the bonus inherently provided by the item is the minimum bonus it grants. For example, a \mitem{\plus2 longsword} grants a \plus2 bonus to attack and damage to any character, even a 1st level character. However, a character of legendary might can draw more power from the same item. Any scaling item provides at \plus1 bonus when worn by a character of at least 4th level. This bonus increases to \plus2 at 8th level, to \plus3 at 12th level, to \plus4 at 16th level, and finally to \plus5 at 20th level. If an item scales, it is noted in its description.

Only class levels are considered when determining the scaling bonus of an item. Hit Values of any other type are not included.

\subsection{Saving Throws}

If a magic item allows a saving throw against its effects, the DC is listed in the item's description. Typically, the DC is equal to 10 \add half the character level of the item's wielder \add the minimum attribute required to cast a spell of the same level as the item's effect.

\begin{comment}
\subsection{Charges, Doses, and Multiple Uses}

Many items, particularly wands and staffs, are limited in power by the number of charges they hold. Normally, charged items have 50 charges at most. If such an item is found as a random part of a treasure, roll d\% and divide by 2 to determine the number of charges left (round down, minimum 1). If the item has a maximum number of charges other than 50, roll randomly to determine how many charges are left.

Prices listed are always for fully charged items. (When an item is created, it is fully charged.) An item with no charges left is worth half the price of a fully charged item. For an item that's worthless when its charges run out (which is the case for almost all charged items), the value of the partially used item is proportional to the number of charges left. For an item that has usefulness in addition to its charges, only part of the item's value is based on the number of charges left.
\end{comment}


\section{Magic Item Description Format}

Each general type of magic item gets an overall description, followed by descriptions of specific items.

General descriptions include notes on activation, random generation, and other material. The AC, hardness, hit points, and break DC are given for typical examples of some magic items. The AC assumes that the item is unattended and includes a \minus5 penalty for the item's effective Dexterity of \minus5. If a creature holds the item, use the creature's Dexterity in place of the \minus5 penalty.

Some individual items, notably those that simply store spells and nothing else, don't get full-blown descriptions. Reference the spell's description for details, modified by the form of the item (potion, scroll, wand, and so on). Assume that the spell is cast at the minimum level required to cast it.

Items with full descriptions have their powers detailed, and each of the following topics is covered in notational form at the end of the description.

\begin{itemize*}
\itemhead{Aura:} Most of the time, a Spellcraft check will reveal the school of magic associated with a magic item and the strength of the aura an item emits. This information (when applicable) is given at the beginning of the item's notational entry. See the Spellcraft skill for details.

\itemhead{Caster Level:} The next item in a notational entry gives the caster level of the item, indicating its relative power. The caster level determines the item's saving throw bonus, as well as other level-dependent aspects of the powers of the item (if variable). It also determines the level that must be contended with should the item come under the effect of a dispel magic spell or similar situation. This information is given in the form ``CL x,'' where ``CL'' is an abbreviation for caster level and ``x'' is a number representing the caster level itself.

For potions, scrolls, and wands, the creator can set the caster level of an item at any number high enough to cast the stored spell and not higher than her own caster level. For other magic items, the caster level is determined by the item itself. In this case, the creator's caster level must be as high as the item's caster level (and prerequisites may effectively put a higher minimum on the creator's level).

\itemhead{Requirements:} The qualifications that must be met to create the item, described in \pcref{Creating Magic Items}.

\itemhead{Market Price:} This gold piece value, given following the word ``Price,'' represents the price someone should expect to pay to buy the item. The market price for an item that can be constructed with an item creation feat is usually equal to the base price plus the price for any components.

\itemhead{Cost to Create:} The next part of a notational entry is the cost in gp to create the item, given following the word ``Cost.'' This information appears only for items with components which make their market prices higher than their base prices. The cost to create includes the costs derived from the base cost plus the costs of the components.

Items without components do not have a ``Cost'' entry. For them, the market price and the base price are the same. The cost in gp is 1/2 the market price.

\itemhead{Weight:} The notational entry for many wondrous items ends with a value for the item's weight. When a weight figure is not given, the item has no weight worth noting (for purposes of determining how much of a load a character can carry).
\end{itemize*}

\section{Arms and Armor}

\subsection{Armor and Shields}

Magic body armor and shields protect the wearer to a greater extent than their nonmagical equivalents. They always provide a scaling enhancement bonus to a character's armor modifier or shield modifier to AC; see \pcref{Scaling Bonuses} for details. In addition to an enhancement bonus, armor may have special abilities or be made of an unusual material.

Body armor is always created so that even if the type of armor comes with boots or gauntlets, these pieces can be switched for other magic boots or gauntlets.

\parhead{Armor Prices}
The prices of enhancement bonuses to armor are listed in \trefnp{Magic Armor and Shields}. If armor has a special ability, the price of the special ability is added to the price of the armor. The number of special abilities on the armor cannot exceed the enhancement bonus of the armor. Additionally, the price of all special abilities cannot exceed twice the price of the enhancement bonus on the armor.

\begin{dtable}
\caption{Magic Armor and Shields}
\begin{tabularx}{\columnwidth} {>{\ccol}X c c}
  \thead{Minimum Enhancement Bonus} & \thead{Base Price} & \thead{Item Level}\\
\plus1 armor/shield & 1,000 gp & 4th \\
\plus2 armor/shield & 4,000 gp & 6th \\
\plus3 armor/shield & 9,000 gp & 9th \\
\plus4 armor/shield & 16,000 gp & 12th \\
\plus5 armor/shield & 25,000 gp & 14th\\
\end{tabularx}
\end{dtable}

\parhead{Caster Level for Armor and Shields} The caster level of a magic shield or magic armor with a special ability is given in the item description. For an item with only an enhancement bonus, the caster level is three times the enhancement bonus. If an item has both an enhancement bonus and a special ability, the higher of the two caster level requirements must be met.

\parhead{Shields} Shield enhancement bonuses do not act as attack or damage bonuses when the shield is used in a bash. However, a shield can be enhanced as a weapon.

\parhead{Hardness and Hit Points} Each \plus1 of enhancement bonus adds 2 to an armor or shield's hardness and \plus10 to its hit points.

\parhead{Activation} A character benefits from magic armor and shields in exactly the way a character benefits from nonmagical armor and shields - by wearing them. Special abilities on body armor are usually activated if the character is struck or damaged, while special abilities on shields are usually activated if the character avoids an attack.

\begin{dtable}
\lcaption{Armor Special Abilities}
\begin{tabularx}{\columnwidth}{>{\lcol}X l l l}
  \thead{Special Ability} & \thead{Cost} & \thead{Item Level} & \thead{Location}\\
  Energy Resistance, Lesser & 1,000 gp & 4th & Armor\\
  Flaming Burst & 2,000 gp & 5th & Armor, Shield \\
  Freezing Burst & 2,000 gp & 5th & Armor, Shield \\
  Energy Resistance, Major & 4,000 gp & 6th & Armor\\
  Shocking Burst & 4,500 gp & 7th & Armor, Shield \\
  Spell Resistance & 25,000 gp & 14th & Armor \\
\end{tabularx}
\end{dtable}

\appdescription{Energy Resistance, Lesser}{1,000}{4nd}{Armor}{Faint Abjuration (Shielding)}{2nd}{Immediate (triggered) 3/day} When you take energy damage, you can activate this armor to reduce the damage by 5 \add half your character level.

\mitemreq{Abjuration}{Shielding}{1}{2nd}{(as armor) 6}

\appdescription{Energy Resistance}{4,500}{7th}{Armor}{Moderate Abjuration (Shielding)}{6th}{Immediate (triggered) 3/day} When you take energy damage, you can activate this armor to gain damage reduction equal to 20 \add half your character level against that type of energy damage. The damage reduction persists for 5 rounds.

\mitemreq{Abjuration}{Shielding}{3}{6th}{(as armor) 12}

\appdescription{Flaming Burst}{2,000}{5th}{Armor, Shield}{Faint Evocation (Energy) [Fire]}{4th}{Immediate (triggered) 1/day} When you are struck or missed by a melee attack, you can trigger a burst of flames which sear the foe that attacked you. Body armor triggers if the attack hits, and shields trigger if the attack misses.

If you activate the item, your foe takes 1d10 fire damage per two character levels (minimum 2d10). In addition, it is ignited for 5 rounds or until it puts out the flames, which requires a move action and a DC 15 Reflex save. While ignited, it takes d6 fire damage each round and is vulnerable, causing it to take a \minus2 penalty to attack rolls, saving throws, checks, DCs, and AC.

\mitemreq{Evocation}{Energy}{2}{4th}{(as armor) 8}

\appdescription{Freezing Burst}{2,000}{5th}{Armor, Shield}{Faint Evocation (Energy) [Cold]}{4th}{Immediate (triggered) 1/day} When you are struck or missed by a melee attack, you can trigger a frigid burst against the foe that attacked you. Body armor triggers if the attack hits, and shields trigger if the attack misses.

If you activate the item, your foe takes 1d10 cold damage per two character levels (minimum 2d10). In addition, it is fatigued for 5 rounds. While fatigued, it can neither run nor charge and is vulnerable, giving it a \minus2 penalty to attack rolls, saving throws, checks, DCs, and AC.

\mitemreq{Evocation}{Energy}{2}{4th}{(as armor) 8}

\appdescriptiondc{Shocking Burst}{4,500}{7th}{Armor, Shield}{Faint Evocation (Energy) [Electricity]}{6th}{Immediate (triggered) 1/day}{Fortitude DC 13} When you are struck or missed by a melee attack, you can trigger a powerful jolt of electricity that zaps the foe that attacked you. Body armor triggers if the attack hits, and shields trigger if the attack misses.

If you activate the item, your foe takes 1d10 electricity damage per two character levels (minimum 3d10). In addition, it is staggered for 5 rounds if it fails a Fortitude save. While staggered, it may take a single move action or standard action each round, but not both. It cannot take full-round actions, but it may take swift actions. In addition, it is vulnerable, causing it to take a \minus2 penalty on attack rolls, saving throws, checks, DCs, and AC.

\mitemreq{Evocation}{Energy}{3}{6th}{(as armor) 10}

\appdescription{Spell Resistance}{8,000}{9th}{Armor}{Moderate Abjuration (Shielding)}{8th}{Standard (specific action) 1/day} By crouching low and striking the ground with your fist, you command your armor to grant you spell resistance. The spell resistance lasts for as long as you remain crouching, and for 5 rounds theareafter. You can move at half speed while crouching.

During that time, you may always make a saving throw when a spell is cast on you. If you succeed, the spell has no effect on you. The type of saving throw made is indicated by the spell. If the spell also allows a saving throw of the same type, only one roll is made.

\mitemreq{Abjuration}{Shielding}{4}{8th}{(as armor) 12}

\subsection{Weapons}

Magic weapons improve a character's combat abilities. They always provide a scaling enhancement bonus to a character's attack and damage; see \pcref{Scaling Bonuses} for details. In addition to an enhancement bonus, weapons may have special abilities or be made of an unusual material.

\parhead{Weapon Prices} The prices of enhancement bonuses to weapons are listed in \trefnp{Magic Weapons}. If a weapon has a special ability, the price of the special ability is added to the price of the weapon. The number of special abilities on the weapon cannot exceed the enhancement bonus of the weapon. Additionally, the price of all special abilities cannot exceed twice the price of the enhancement bonus on the weapon.

\begin{dtable}
\caption{Magic Weapons}
\begin{tabularx}{\columnwidth} {>{\ccol}X c c}
  \thead{Minimum Enhancement Bonus} & \thead{Base Price} & \thead{Item Level}\\
\plus1 weapon & 1,000 gp & 4th \\
\plus2 weapon & 4,000 gp & 6th \\
\plus3 weapon & 9,000 gp & 9th \\
\plus4 weapon & 16,000 gp & 12th \\
\plus5 weapon & 25,000 gp & 14th \\
\end{tabularx}
\end{dtable}

\parhead{Caster Level for Weapons} The caster level of a magic weapon with a special ability is given in the item description. For an item with only an enhancement bonus, the caster level is three times the enhancement bonus. If an item has both an enhancement bonus and a special ability, the higher of the two caster level requirements must be met.

\parhead{Hardness and Hit Points} Each \plus1 of enhancement bonus adds 2 to an armor or shield's hardness and \plus10 to its hit points.

\parhead{Ranged Weapons and Ammunition} The enhancement bonus from a ranged weapon does not stack with the enhancement bonus from ammunition. Only the higher of the two enhancement bonuses applies. Special abilities are applied from both sources, as long as they are not identical. If conflicting special abilities exist, the special ability on the ammunition takes precedence.

Magic ammunition loses its magic after being fired, whether it hits or misses.

\parhead{Light Generation} Some magic weapons shed light equivalent to a light spell (bright light in a 20-foot radius, shadowy light in a 40-foot radius). These glowing weapons are quite obviously magical. Such a weapon can't be concealed when drawn, nor can its light be shut off. Some of the specific weapons detailed below always or never glow, as defined in their descriptions.

\parhead{Activation} Usually, a character benefits from a magic weapon in the same way a character benefits from a mundane weapon - by attacking with it. Special abilities on weapons are usually activated if the character strikes a foe with the weapon.

\parhead{Magic Weapons and Critical Hits} Some weapon qualities and some specific weapons have an extra effect on a critical hit. These special effects function against creatures not subject to critical hits. When fighting against such creatures, roll for critical hits as you would against any other creature subject to critical hits. On a successful critical roll, apply the special effect, but do not multiply the weapon's regular damage.

\begin{dtable}
\lcaption{Weapon Special Abilities}
\begin{tabularx}{\columnwidth}{>{\lcol}X l l}
  \thead{Special Ability} & \thead{Cost} & \thead{Item Level} \\
  Executioner & 2,000 gp & 5th \\
  Flaming & 2,000 gp & 5th \\
  Freezing & 2,000 gp & 5th \\
  Shocking & 4,500 gp & 7th \\
  Lifebonder & 8,000 gp & 9th \\
  Lifedrinker & 9,000 gp & 9th \\
  Heartseeker & 12,500 gp & 11th \\
  Soulreaver & 32,000 gp & 15th \\
  Vorpal & 40,500 gp & 18th \\
\end{tabularx}
\end{dtable}

\impdescription{Surestrike, Lesser}{500}{3rd}{Faint Divination (Knowledge)}{1st}{Immediate (triggered) 1/day} When you threaten a critical hit with this weapon, you can activate it to receive a brief glimpse of the future, showing you how to wound your foe deeply. If you do, you may roll the threat confirmation twice and take whichever roll you prefer.

\mitemreq{Divination}{Knowledge}{1}{2nd}{(as weapon) 8}

\impdescription{Surestrike}{9,000}{9th}{Faint Divination (Knowledge)}{6th}{Immediate (triggered) 1/day} When you miss an attack with this weapon, you can activate it 

\impdescription{Flaming}{1,000}{2nd}{Faint Evocation (Energy) [Fire]}{4th}{Immediate (triggered) 1/day} When you strike a foe with this weapon, you can engulf the weapon in flames. If you do, your foe is ignited for 5 rounds or until it puts out the flames, which requires a move action and a DC 15 Reflex save. While ignited, it takes d6 fire damage each round and is vulnerable, causing it to take a \minus2 penalty to attack rolls, saving throws, checks, DCs, and AC.

\mitemreq{Evocation}{Energy}{1}{2nd}{(as weapon) 8}

\impdescription{Freezing}{2,000}{5th}{Faint Evocation (Energy) [Cold]}{4th}{Immediate (triggered) 1/day} When you strike a foe with this weapon, you can unleash an icy blast from the weapon. If you do, your foe is fatigued for 5 rounds. While fatigued, it can neither run nor charge and is vulnerable, giving it a \minus2 penalty to attack rolls, saving throws, checks, DCs, and AC.

\mitemreq{Evocation}{Energy}{2}{4th}{(as weapon) 8}

%As 5th level spell because it feels right. Based losely on Discern
%Vulnerability, although this feels like Awareness, so True Seeing?
\impdescription{Heartseeker}{12,500}{11th}{Moderate Divination (Awareness)}{10th}{Immediate (triggered) 1/day} When you strike the same foe with this weapon for multiple rounds in a row, you can suddenly perceive a critical weakness in your foe's defenses. You must strike the foe for a number of consecutive rounds equal to the critical multiplier of the weapon you are using. If you activate the item, the final hit automatically becomes a confirmed critical hit. This has no effect on creatures immune to critical hits.

\mitemreq{Divination}{Awareness}{5}{10th}{(as weapon) 14}

%Link vitality triggered is 6th, *.8 for significant limitation of forcing it to
%affect you
\impdescriptiondc{Lifebonder}{8,000}{9th}{Moderate Necromancy (Life)}{8th}{Immediate (triggered) 1/day}{Will DC 18} When you damage a foe with this weapon, you can forcibly bond your life force with your foe's. If you do, and the struck creature fails a Will save, you and the creature share damage for 5 rounds, as the \spell{link vitality} spell. Whenever one creature takes damage or receives healing, the other also receives the same amount of damage or healing. This takes effect after the damage dealt by your initial attack.

\impdescription{Lifedrinker}{9,000}{9th}{Moderate Necromancy (Life) [Healing]}{6th}{Immediate (triggered) 3/day} When you damage a foe with this weapon, you can absorb your foe's life energy. If you do, you gain life equal to the damage dealt by the blow. You cannot gain more hit points than your foe has.

\mitemreq{Necromancy}{Life}{3}{6th}{(as weapon) 10}

%As 8th level spell? I have no idea. There aren't actually any spells this can
%be based on.
\impdescription{Soulreaver}{32,000}{15th}{Strong Necromancy (Soul)}{16th}{\x and standard (specific action)} This ghostly, translucent weapon strikes directly at the target's soul. It ignores all damage reduction, but it does not deal hit point damage. In fact, a creature struck by the weapon only feels the weapon pass through it harmlessly. Damage that would be dealt by the weapon is delayed for up to 24 hours. While the damage is delayed, it cannot be cured.

In order to convert the delayed damage into real damage, the wielder must stab himself through the heart with the weapon as a standard action. This deals no damage to the wielder, but any creatures that have been dealt damage by the weapon immediately take lethal damage equal to the delayed damage the weapon has stored up for them. Any such damage dealt in excess of the creature's hit points is converted directly into critical damage.

A soulreaver weapon has no effect on objects. While wielded, it has physical form only for its wielder, making it impossible to sunder or disarm. While not in use, it can be picked up and touched normally.

\mitemreq{Necromancy}{Soul}{8}{16th}{(as weapon) 20}

%3rd level effect (no range, so -1 modifier)
\impdescription{Shocking}{4,500}{7th}{Faint Evocation (Energy) [Electricity]}{4th}{Immediate (triggered) 1/day} When you strike a foe with this weapon, you can unleash an powerful electrical jolt from the weapon. If you do, your foe is staggered for 5 rounds if it fails a Fortitude save. While staggered, it may take a single move action or standard action each round, but not both. It cannot take full-round actions, but it may take swift actions. In addition, it is vulnerable, causing it to take a \minus2 penalty on attack rolls, saving throws, checks, DCs, and AC.

\mitemreq{Evocation}{Energy}{2}{4th}{(as weapon) 8}

%9th level like Power Word Kill
\impdescription{Vorpal}{40,500}{18th}{Strong Transmutation (Augment)}{18th}{Immediate (triggered) 1/day} When you roll a 20 with this weapon and confirm the critical hit, you can instantly decapitate your foe. If you do, it dies immediately, with no saving throw allowed. This has no effect on creatures without a discernable head, creatures unaffected by the loss of a single head, or creatures whose head you cannot reach.

\mitemreq{Transmutation}{Augment}{9}{18th}{(as weapon) 22}

\section{Apparel}

\subsection{Arms}

\subsection{Head}

\subsection{Legs}

\subsection{Rings}

\parhead{Physical Description} Rings have no appreciable weight. Although exceptions exist that are crafted from glass or bone, the vast majority of rings are forged from metal - usually precious metals such as gold, silver, and platinum. A ring has AC 13, 2 hit points, hardness 10, and a break DC of 25.

\parhead{Activation} Rings have highly varied activation methods. Some are active as long as they are worn, some are triggered by specific circumstances, and some respond to a command word.

\begin{comment}
\begin{dtable}
\lcaption{Rings}
\begin{tabularx}{\columnwidth}{>{\lcol}X l}
Ring & Market Price \\
Protection \plus1 & 2,000 gp \\
Feather falling & 2,200 gp \\
Climbing & 2,500 gp \\
Jumping & 2,500 gp \\
Sustenance & 2,500 gp \\
Swimming & 2,500 gp \\
Mind shielding & 8,000 gp \\
Protection \plus2 & 8,000 gp \\
Climbing, improved & 10,000 gp \\
Jumping, improved & 10,000 gp \\
Swimming, improved & 10,000 gp \\
Energy resistance, minor & 12,000 gp \\
Protection \plus3 & 18,000 gp \\
Energy resistance, major & 28,000 gp \\
Protection \plus4 & 32,000 gp \\
Energy resistance, greater & 44,000 gp \\
Protection \plus5 & 50,000 gp \\
\end{tabularx}
\end{dtable}
\end{comment}



\appdescription{Protection}{Varies}{see text}{Varied Abjuration (Shielding)}{varies}{\x}
A ring of protection grants a scaling enhancement bonus to your saving throws while worn. See \pref{Scaling Bonuses} for details on scaling bonuses. The properties of the ring depend on its minimum enhancement bonus, as shown in the table below.

\begin{dtable}
\caption{Ring of Protection}
\begin{tabularx}{\columnwidth} {>{\ccol}X c c}
  \thead{Minimum Enhancement Bonus} & \thead{Base Price} & \thead{Item Level} \\
\plus1  & 1,000 gp & 4th\\
\plus2 & 4,000 gp & 6th\\
\plus3 & 9,000 gp & 9th\\
\plus4 & 16,000 gp & 12th \\
\plus5 & 32,000 gp & 15th \\
\end{tabularx}
\end{dtable}

The caster level is equal to three times the item's minimum enhancement bonus. To craft the item, you must have a number of ranks in Craft (jewelry) equal to the item's caster level \add 4.

\mitemreq{Abjuration}{Shielding}{1}{varies}{jewelry varies}

\subsection{Torso}

\section{Implements}

\begin{comment}
\subsection{Rods}

Rods are scepterlike devices that have unique magical powers and do not usually have charges. Anyone can use a rod.

\parhead{Physical Description} Rods weigh approximately 5 pounds.

They range from 2 feet to 3 feet long and are usually made of iron or some other metal. (Many, as noted in their descriptions, can function as light maces or clubs due to their sturdy construction.)

These sturdy items have AC 9, 10 hit points, hardness 10, and a break DC of 27.

\parhead{Activation} Details relating to rod use vary from item to item. See the individual descriptions for specifics.
\end{comment}

\subsection{Scrolls}
A scroll is a spell (or collection of spells) that has been stored in written form. A spell on a scroll can be used only once. The writing vanishes from the scroll when the spell is activated. Using a scroll is basically like casting a spell.

\parhead{Physical Description} A scroll is a heavy sheet of fine vellum or high-quality paper. An area about 8 1/2 inches wide and 11 inches long is sufficient to hold one spell. The sheet is reinforced at the top and bottom with strips of leather slightly longer than the sheet is wide. A scroll holding more than one spell has the same width (about 8 1/2 inches) but is an extra foot or so long for each extra spell. Scrolls that hold three or more spells are usually fitted with reinforcing rods at each end rather than simple strips of leather. A scroll has AC 9, 1 hit point, hardness 0, and a break DC of 8.

To protect it from wrinkling or tearing, a scroll is rolled up from both ends to form a double cylinder. (This also helps the user unroll the scroll quickly.) The scroll is placed in a tube of ivory, jade, leather, metal, or wood. Most scroll cases are inscribed with magic symbols which often identify the owner or the spells stored on the scrolls inside. The symbols often hide magic traps.

\parhead{Activation} To activate a scroll, a spellcaster must read the spell written on it. Doing so involves several steps and conditions.

\parhead{Decipher the Writing} The writing on a scroll must be deciphered before a character can use it or know exactly what spell it contains. This requires a read magic spell or a successful Spellcraft check (DC 20 \add spell level).

Deciphering a scroll to determine its contents does not activate its magic unless it is a specially prepared cursed scroll. A character can decipher the writing on a scroll in advance so that he or she can proceed directly to the next step when the time comes to use the scroll.

\parhead{Activate the Spell} Activating a scroll requires reading the spell from the scroll. The character must be able to see and read the writing on the scroll. Activating a scroll spell requires no material components or focus. (The creator of the scroll provided these when scribing the scroll.) Note that some spells are effective only when cast on an item or items. In such a case, the scroll user must provide the item when activating the spell. Activating a scroll spell is subject to disruption just as casting a normally prepared spell would be. Using a scroll is like casting a spell for purposes of arcane spell failure chance.

To have any chance of activating a scroll spell, the scroll user must meet the following requirements.
\begin{itemize*}
\item The spell must be of the correct type (arcane or divine). Arcane spellcasters (wizards and sorcerers) can only use scrolls containing arcane spells, and divine spellcasters (clerics, druids, paladins, and rangers) can only use scrolls containing divine spells. (The type of scroll a character creates is also determined by his or her class.)
\item The user must have the spell on his or her spell list.
\item The user must have the requisite attribute score.
\end{itemize*}

If the user meets all the requirements noted above, and her caster level is at least equal to the spell's caster level, she can automatically activate the spell without a check. If she meets all three requirements but her own caster level is lower than the scroll spell's caster level, then she has to make a caster level check (DC = scroll's caster level \add 1) to cast the spell successfully. If she fails, she must make a DC 5 Wisdom check to avoid a mishap (see Scroll Mishaps, below). A natural roll of 1 always fails, whatever the modifiers.

\parhead{Determine Effect} A spell successfully activated from a scroll works exactly like a spell cast the normal way. Assume the scroll spell's caster level is always the minimum level required to cast the spell for the character who scribed the scroll (usually twice the spell's level, minus 1), unless the caster specifically desires otherwise.

The writing for an activated spell disappears from the scroll.

\parhead{Scroll Levels} Some spells are acquired by multiple classes at different levels. Use the entry on the table appropriate to the scribing of each individual scroll.

\begin{dtable}
\lcaption{Spell Scrolls}
\begin{tabularx}{\columnwidth}{>{\lcol}X l}
\thead{Common Scrolls} & \thead{Market Price} \\
0-Level Spells  & 12 gp 5 sp \\
1st-Level Spells & 50 gp \\
2nd-Level Spells & 200 gp \\
3rd-Level Spells & 450 gp \\
4th-Level Spells & 800 gp \\
5th-Level Spells & 1250 gp \\
6th-Level Spells & 1800 gp \\
7th-Level Spells & 2450 gp \\
8th-Level Spells & 3200 gp \\
9th-Level Spells & 4050 gp \\
\thead{Paladin/Ranger Scrolls} & \thead{Market Price\fn{2}} \\
1st-Level Paladin/Ranger Spells & 50 gp \\
2nd-Level Paladin/Ranger Spells & 500 gp \\
3rd-Level Paladin/Ranger Spells & 1200 gp \\
4th-Level Paladin/Ranger Spells & 2200 gp \\
\end{tabularx}
1 Includes cleric, druid, sorcerer, and wizard spells \\
2 Scrolls of paladin and ranger spells cost twice as much to buy because of their rarity. The cost to scribe them is no different than normal, and players attempting to sell such scrolls will find it difficult to find a buyer, so such items sell for a quarter of their market price.
\end{dtable}

\subsection{Staffs}

A staff is a long shaft, usually made of wood, that enhances a spellcaster's power. Staffs function exactly like wands (see below), except that they enhance all schools of magic at once.

\parhead{Staff Prices} Enhancement bonuses on staffs are three times as expensive as wands, but staffs otherwise use the same pricing rules as wands.

\begin{dtable}
\caption{Staff Prices}
\begin{tabularx}{\columnwidth} {>{\ccol}X c c}
  \thead{Minimum Enhancement Bonus} & \thead{Base Price} & \thead{Item Level}\\
\plus1 wand & 1,500 gp & 4th \\
\plus2 wand & 6,000 gp & 8th \\
\plus3 wand & 13,500 gp & 11th \\
\plus4 wand & 24,000 gp & 14th \\
\plus5 wand & 37,500 gp & 16th \\
\end{tabularx}
\end{dtable}

\parhead{Physical Description} A typical staff is 4 feet to 7 feet long and 2 inches to 3 inches thick, weighing about 5 pounds. Most staffs are wood, but a rare few are bone, metal, or even glass. (These are extremely exotic.) Staffs often have a gem or some device at their tip or are shod in metal at one or both ends. Staffs are often decorated with carvings or runes. A typical staff is like a walking stick, quarterstaff, or cudgel. It has AC 7, 10 hit points, hardness 5, and a break DC of 24.

\parhead{Activation} Staffs use the same activation method as wands.

\subsection{Wands}

A wand is a thin baton that enhances a spellcaster's power. Wands always provide a scaling enhancement bonus to caster level with a particular school of magic; see \pcref{Scaling Bonuses} for details. In addition to an enhancement bonus, wands may have special abilities or be made of an unusual material.

\parhead{Wand Prices} The prices of enhancement bonuses on wands are listed on \trefnp{Wands}. If a wand has a special ability, the price of the special ability is added to the price of the base enhancement bonus. The number of special abilities on the wand cannot exceed the base enhancement bonus of the wand. Additionally, the price of all special abilities cannot exceed twice the price of the enhancement bonus on the wand.

\subparhead{Multiple Schools} Some rare wands provide bonuses to two schools. The enhancement bonus on a wand costs twice as much if it provides bonuses to two schools. A wand cannot provide an enhancement bonus to more than two schools.

\begin{dtable}
\caption{Wand Prices}
\begin{tabularx}{\columnwidth} {>{\ccol}X c c}
  \thead{Minimum Enhancement Bonus} & \thead{Base Price} & \thead{Item Level}\\
\plus1 wand & 500 gp & 3rd \\
\plus2 wand & 2,000 gp & 5th \\
\plus3 wand & 4,500 gp & 7th \\
\plus4 wand & 8,000 gp & 9th \\
\plus5 wand & 12,500 gp & 11th \\
\end{tabularx}
\end{dtable}

\parhead{Physical Description} A typical wand is 6 inches to 12 inches long and about 1/4 inch thick, and often weighs no more than 1 ounce. Most wands are wood, but some are bone. A rare few are metal, glass, or even ceramic, but these are quite exotic. Occasionally, a wand has a gem or some device at its tip, and most are decorated with carvings or runes. A typical wand has AC 7, 5 hit points, hardness 5, and a break DC of 16.

\parhead{Activation} All wands provide a constant increase to caster level that requires no activation. Some wands also have special abilities. Unless otherwise noted, these special abilities are activated as an immediate action while casting a spell.

\parhead{School Restrictions} Most wand special abilities have an associated school. Special abiliites from a particular school can only be used with wands that provide bonuses to that school.

\begin{dtable}
\lcaption{Wand Special Abilities}
\begin{tabularx}{\columnwidth}{>{\lcol}X l l}
  \thead{Special Ability} & \thead{Cost} & \thead{Item Level} \\
  Flaming & 1,000 gp & 5th \\
  Freezing & 1,000 gp & 5th \\
  Shocking & 4,000 gp & 7th \\
\end{tabularx}
\end{dtable}
%Wand special abilities: priced as close range, immediate action trigger.
%Trigger needs to be immediate; otherwise, a caster casting a quickened spell
%could potentially unleash four separate effects simultaneously.
\impdescription{Enlarging}{500}{3rd}{Faint Universal}{2nd}{Immediate (specific action) 1/day} As you cast a spell, you can activate this wand to double the range of the spell.

\mitemreq{No school}{}{1}{2nd}{(as wand) 6}

\impdescription{Flaming}{1,000}{7th}{Faint Evocation (Energy) [Fire]}{6th}{Immediate (specific action) 1/day} As you cast a spell, you can activate this wand to ignite a single creature affected by the spell for 5 rounds.

An ignited creature is vulnerable, causing it to take a \minus2 penalty to attack rolls, saving throws, checks, DCs, and AC. In addition, it takes d6 damage per round from the fire. If the creature takes a move action, it can attempt a DC 15 Reflex save to put out the flames. This action provokes attacks of opportunity. Dropping prone as part of the action gives a \plus4 circumstance bonus on this save.

\mitemreq{Evocation}{Energy}{3}{6th}{(as wand) 10}

\impdescription{Freezing}{1,000}{7th}{Faint Evocation (Energy) [Cold]}{6th}{Immediate (specific action) 1/day} As you cast a spell, you can activate this wand to fatigue a single creature affected by the spell for 5 rounds. A fatigued creature cannot run or charge and is vulnerable, causing it to take a \minus2 penalty to attack rolls, saving throws, checks, DCs, and AC.

\mitemreq{Evocation}{Energy}{3}{6th}{(as wand) 10}

\impdescriptiondc{Shocking}{4,000}{9th}{Faint Evocation (Energy) [Cold]}{8th}{Immediate (specific action) 1/day}{16} As you cast a spell, you can activate this wand to stagger a single creature affected by the spell for 5 rounds. The target receives a Fortitude save to avoid being staggered.

A staggered character may take a single move action or standard action each round, but not both. She cannot take full-round actions, but she may take swift actions. In addition, she is vulnerable, causing her to take a \minus2 penalty on attack rolls, saving throws, checks, DCs, and AC.

\mitemreq{Evocation}{Energy}{4}{8th}{(as wand) 12}

\section{Tools}

\subsection{Potions and Oils}

A potion is a magic liquid that produces its effect when imbibed. Magic oils are similar to potions, except that oils are applied externally rather than imbibed. A potion or oil can be used only once. It can duplicate the effect of a spell of up to 3rd level that has a casting time of less than 1 minute.

Potions are like spells cast upon the imbiber. The character taking the potion doesn't get to make any decisions about the effect  - the caster who brewed the potion has already done so. The drinker of a potion is both the effective target and the effective caster of the effect.

The person applying an oil is the effective caster, but the object is the target.

\parhead{Physical Description} A typical potion or oil consists of 1 ounce of liquid held in a ceramic or glass vial fitted with a tight stopper. The stoppered container is usually no more than 1 inch wide and 2 inches high. The vial has AC 13, 1 hit point, hardness 1, and a break DC of 12. Vials hold 1 ounce of liquid.

\parhead{Identifying Potions} In addition to the standard methods of identification, PCs can sample from each container they find to attempt to determine the nature of the liquid inside. An experienced character learns to identify potions by memory -- for example, the last time she tasted a liquid that reminded her of almonds, it turned out to be a potion of cure moderate wounds.

\parhead{Activation} Drinking a potion or applying an oil requires no special skill. The user merely removes the stopper and swallows the potion or smears on the oil. The following rules govern potion and oil use.

Drinking a potion or using an oil on an item of gear is a standard action. The potion or oil takes effect immediately. Using a potion or oil provokes attacks of opportunity. A successful attack (including grapple attacks) against the character forces a Concentration check (as for casting a spell). If the character fails this check, she cannot drink the potion. An enemy may direct an attack of opportunity against the potion or oil container rather than against the character. A successful attack of this sort can destroy the container.

A creature must be able to swallow a potion or smear on an oil. Because of this, incorporeal creatures cannot use potions or oils.

Any corporeal creature can imbibe a potion. The potion must be swallowed, or in some other way ingested. Any corporeal creature can use an oil.

A character can carefully administer a potion to an unconscious creature as a full-round action, trickling the liquid down the creature's throat. Likewise, it takes a full-round action to apply an oil to an unconscious creature. Exceptionally large objects or creatures require a greater time expenditure.

\parhead{Potion Descriptions} The caster level for a standard potion is the minimum caster level needed to cast the spell (unless otherwise specified). Common potions refer to potions of spells on the cleric, druid, or unrestricted sorcerer/wizard spell lists. Any other spells, such as restricted sorcerer/wizard spells, are considered ``uncommon''.

\begin{dtable}
\lcaption{Potions and Oils}
\begin{tabularx}{\columnwidth}{X c c c}
  \thead{Potion or Oil} & \thead{Market Price} & \thead{Item Level} & \thead{Extra Price Modifier}\\
1st-level spell (common) & 50 gp & 1st & \plus50 gp per caster level \\
1st-level spell (uncommon) & 75 gp & 1st & \plus75 gp per caster level \\
2nd-level spell (common) & 400 gp & 3rd & \plus100 gp per caster level \\
2nd-level spell (uncommon) & 600 gp & 3rd & \plus200 gp per caster level \\
3rd-level spell (common) & 900 gp & 4th & \plus150 gp per caster level \\
3rd-level spell (uncommon) & 1350 gp & 4th & \plus225 gp per caster level
\end{tabularx}
\end{dtable}
