\chapter{Advanced Classes}
Characters above 6th level can increase substantially in power and versatility. Rules for these advanced characters are described below.

\section{Class Descriptions}

\subsection{Barbarian}
\begin{dtable*}
\lcaption{The Barbarian}
\begin{tabularx}{\textwidth}{>{\ccol}p{\levelcol} >{\ccol}p{\babcolgood} *{3}{>{\ccol}p{\babcolgood}} X}
\thead{Level} & \thead{Base Attack Bonus} & \thead{Fort Save} & \thead{Ref Save} & \thead{Will Save} & \thead{Special} \\
1st & \plus1         & \plus3 & \plus1 & \plus0 & Damage reduction, rage \plus2 \\
2nd & \plus2         & \plus4 & \plus2 & \plus1 & Fast movement, uncanny dodge \\
3rd & \plus3         & \plus5 & \plus3 & \plus1 & Endurance, channeled rage \\
4th & \plus4         & \plus6 & \plus4 & \plus2 & Grit \\
5th & \plus5         & \plus7 & \plus4 & \plus2 & \\
6th & \plus6/\plus1  & \plus8 & \plus5 & \plus3 & Improved uncanny dodge, channeled rage \\
7th & \plus7/\plus2  & \plus9 & \plus6 & \plus3 & Larger than life \\
8th & \plus8/\plus3  & \plus10& \plus7 & \plus4 & Rage \plus3 \\
9th & \plus9/\plus4  & \plus11& \plus8 & \plus4 & Channeled rage \\
10th& \plus10/\plus5 & \plus12& \plus8 & \plus5 & Greater uncanny dodge \\
11th& \plus11/\plus6/\plus1  & \plus13 & \plus9 & \plus5 & Tireless rage \\
12th& \plus12/\plus7/\plus2  & \plus14 & \plus10& \plus6 & Channeled rage, chaotic rage \\
13th& \plus13/\plus8/\plus3  & \plus15 & \plus10& \plus6 & Indomitable will \\
14th& \plus14/\plus9/\plus4  & \plus16 & \plus11& \plus7 & Rage \plus4 \\
15th& \plus15/\plus10/\plus5 & \plus17 & \plus12& \plus7 & Channeled rage \\
16th& \plus16/\plus11/\plus6/\plus1 & \plus18 & \plus13& \plus8 & Improved grit \\
17th& \plus17/\plus12/\plus7/\plus2 & \plus19 & \plus13& \plus8 & Larger than belief \\
18th& \plus18/\plus13/\plus8/\plus3 & \plus20 & \plus14& \plus9 & Channeled rage \\
19th& \plus19/\plus14/\plus9/\plus4 & \plus21 & \plus15& \plus9 & Deathless rage \\
20th& \plus20/\plus15/\plus10/\plus5& \plus22 & \plus16 & \plus10 & Limitless rage, rage \plus5
\end{tabularx}
\end{dtable*}

\cd{Alignment} Any nonlawful.

\cd{Hit Value} 7.

\sssecfake{Class Skills}
The barbarian's class skills (and the key attribute for each skill) are
Athletics (Str), Climb (Str), Swim (Str), Acrobatics (Dex), Ride (Dex), Perception (Wis), Survival (Wis), Creature Handling (Cha), and Intimidate (Cha).
\cf{Bbn}{Skill Points at 1st Level} 4

\sssecfake{Class Features}

All of the following are class features of the barbarian.
\cf{Bbn}{Weapon and Armor Proficiency}  A barbarian is proficient with simple weapons, any four other weapon groups, light armor, medium armor, and shields (except tower shields).

\cf{Bbn}{Damage Reduction (Ex)} A barbarian has the ability to shrug off some amount of injury from attacks. He has physical damage reduction equal to his barbarian level. This damage reduction allows the barbarian to ignore the first points of physical damage he takes each round.

\cf{Bbn}{Rage (Ex)} A barbarian can fly into a rage a certain number of times per day. While in a rage, he temporarily gains a \plus2 competence bonus to weapon damage rolls and Fortitude and Will saves. In addition, he gains 2 temporary hit points per barbarian level, but he takes a \minus2 penalty to Armor Class.
\par The extra hit points gained from raging are lost before any other hit points. (For more information, see \pref{Temporary Hit Points}.)

While raging, a barbarian cannot use any Charisma-, Dexterity-, or Intelligence-based skills other than Acrobatics, Escape Artist, Intimidate, and Ride, or any abilities that require patience or concentration, nor can he cast spells or activate magic items that require a command word, a spell trigger (such as a wand), or spell completion (such as a scroll) to function. He can use any feat he has except Combat Expertise, item creation feats, and metamagic feats.

\par A barbarian's ability to maintain his rage depends on his willpower. A fit of rage lasts for a number of rounds equal to 5 \add the barbarian's Charisma. He may prematurely end his rage, and it ends automatically if he becomes unconscious. At the end of the rage, the barbarian takes nonlethal damage equal to the number of temporary hit points he gained by raging, loses his rage bonuses and restrictions, and becomes fatigued (\minus2 penalty to Strength, \minus2 penalty to Dexterity, can't charge or run) for the duration of the current encounter.  If the barbarian has any temporary hit points remaining at the end of his rage, the nonlethal damage is dealt to those hit points before they go away.

\par A barbarian can fly into a rage once per day, plus an additional number of times per day equal to half his Constitution (minimum 0), to a maximum number of rages per day equal to his barbarian class level. He may not rage more than once per encounter. Entering a rage takes no time itself, but a barbarian can do it only during his action, not in response to someone else's action.

\cf{Bbn}{Fast Movement (Ex)} At 2nd level, a barbarian's land speed becomes faster than the norm for his race by \plus10 feet. This benefit applies only when he is wearing no armor, light armor, or medium  armor and not carrying a heavy load. Apply this bonus before modifying the barbarian's speed because of any load carried or armor worn. This bonus is a competence bonus.

\cf{Bbn}{Uncanny Dodge (Ex)} Starting at 2nd level, a barbarian can react to danger before his senses would normally allow him to do so. He may apply his Dexterity and dodge modifier to his armor class while flat-footed.

If a barbarian already has uncanny dodge from a different class, he stacks those levels to determine whether he gains improved uncanny dodge (see below) instead.

\cf{Bbn}{Endurance} A barbarian gains Endurance (see \hyperlink{feat:Endurance}{Endurance}) as a bonus feat at 3rd level. If he already has Endurance, he may gain any other feat for which he qualifies as a bonus feat.

\cf{Bbn}{Channeled Rage} At 3rd level, a barbarian gains the ability to enter a channeled rage whenever he rages. Each channeled rage grants the barbarian additional abilities while in that rage or changes the nature of his rage.
\par A barbarian can only be in one channeled rage at a time. By spending an additional use of his rage ability, he can change which channeled rage he is in without exiting the rage, but this does not extend the duration of the rage. A barbarian chooses one channeled rage that he may enter at 3rd level, plus one every three levels thereafter.
\par Channeled rage abilities are (Ex) abilities unless otherwise noted. All bonuses granted by channeled rages apply only while the barbarian is in that channeled rage.

\subcf{Athletic Rage} The barbarian adds his rage bonus as a competence bonus to his Athletics, Climb, and Swim checks. Additionally, he is always treated as having a running start when jumping.

\subcf{Agile Rage} The barbarian adds his rage bonus as a competence bonus to his Reflex saves and Dexterity-based skill checks.

\subcf{Endless Rage} The barbarian's rage lasts for an additional 5 rounds.

%Needs to be stronger
\subcf{Fearless Rage} The barbarian becomes immune to fear and harmful morale effects.

\subcf{Intimidating Rage} The barbarian adds his rage bonus as a competence bonus to his Intimidate checks. Any foe he intimidates remains shaken until the barbarian ends his rage.

\subcf{Mighty Rage} The barbarian adds his rage bonus as a competence bonus to his Strength. This replaces his competence bonuses to weapon damage rolls and Fortitude saves.

\subcf{Mindless Rage} The barbarian becomes immune to mind-affecting spells and effects for the duration of his rage.

\textit{Prerequisites:} Barbarian level 15th.

\subcf{Overpowering Rage} The barbarian adds his rage bonus as a competence bonus to his maneuver modifier. This replaces his competence bonus to weapon damage rolls.

\subcf{Savage Rage} The barbarian gains the unarmed warrior ability (see Unarmed Warrior, page \pref{Mnk:Unarmed Warrior}), increasing his power with unarmed attacks (1d6 damage for a Medium barbarian).

\subcf{Spellbreaker Rage (Su)} The barbarian gains spell resistance while raging. A creature with spell resistance may always make a saving throw when a spell is cast on it. The saving throw type is indicated by the spell. If it succeeds, the spell has no effect on it.

\textit{Prerequisites:} Barbarian level 9th.

\subcf{Terrifying Rage (Su)} Any enemy beginning its turn within the barbarian's threatened area must make a Will save or be shaken for 5 rounds. The save DC is equal to 10 \add the barbarian's level \add his Charisma modifier. This can only affect any individual creature once per 24 hours.

\textit{Prerequisites:} Barbarian level 6th.

\subcf{Unstoppable Rage} Each round, the barbarian can bull rush an opponent as a swift action that does not provoke attacks of opportunity.

\textit{Prerequisites:} Barbarian level 6th.

\subcf{Wary Rage} The barbarian does not suffer the normal \minus2 penalty to AC for raging.

\cf{Bbn}{Grit (Ex)} At 4th level, a barbarian's resilience allows him to shrug off magical effects. If he makes a successful Fortitude save against an attack that normally deals half damage on a successful save, he instead takes no damage.

\cf{Bbn}{Improved Uncanny Dodge (Ex)} At 6th level and higher, a barbarian is always treated as being threatened by two fewer creatures than he actually is for the purpose of determining overwhelm penalties. This defense can deny a rogue the ability to sneak attack the barbarian.
\par If a character already has improved uncanny dodge from a second class and gains improved uncanny dodge, the character stacks those levels to determine if he should gain greater uncanny dodge.

\cf{Bbn}{Larger than Life (Ex)} A barbarian of 7th level or higher holds the strength of a giant in the body of a man (or woman). The barbarian is treated as being one size category larger than he actually is for the purpose of maneuvers he performs or is the target of, checks that are affected by size (such as Strength checks to break down doors), and whether a creature's special attacks based on size can affect him if doing so is advantageous to him. In addition, though he uses weapons of the same size, his weapons deal damage as if they were one size category larger, including natural weapons and unarmed strikes. The barbarian's space and reach remain those of a creature of his actual size. The benefits of this class feature stack with the effects of spells and abilities that increase the barbarian's size category.

\cf{Bbn}{Rage} At 8th level, the barbarian's rage bonus increases to \plus3 (and 3 temporary hit points per barbarian level). It increases to \plus4 at 14th level, and to \plus5 at 20th level. The penalty to Armor Class remains the same.

\cf{Bbn}{Greater Uncanny Dodge (Ex)} At 10th level and higher, a barbarian no longer suffers overwhelm penalties, regardless of the number of foes surrounding him.

\cf{Bbn}{Tireless Rage (Ex)} At 11th level and higher, a barbarian no longer becomes fatigued at the end of his rage.

\cf{Bbn}{Chaotic Rage (Ex)} At 12th level, the barbarian gains the ability to change channeled rage abilities at will, without consuming an additional use of his rage ability. He may not change channeled rages in this way more than once per round.

\cf{Bbn}{Indomitable Will (Ex)} At 13th level, a barbarian becomes immune to compulsion and domination spells and effects.

\cf{Bbn}{Improved Grit (Ex)} At 16th level, a barbarian's fortitude knows no bounds. If he fails a Fortitude save against an effect that deals half damage on a successful save, he takes only half damage.

\cf{Bbn}{Larger than Belief (Ex)} At 17th level, the barbarian's larger than life ability improves. He is treated as being two size categories larger than he actually is.

\cf{Bbn}{Deathless Rage (Ex)} At 19th level and higher, a raging barbarian can scorn death and unconsciousness. As long as his rage continues, he is not staggered at 0 hit points, and cannot take critical damage. However, every 50 points of damage he takes in excess of his hit points reduces the duration of his rage by one round, and the Endless Rage channeled rage ability does not extend the duration of his rage if he is at 0 hit points. Once his rage ends, the effects of the barbarian's wounds apply normally if they have not been healed. This ability does not prevent death from spell effects such as \spell{finger of death} or \spell{disintegrate}.

\cf{Bbn}{Limitless Rage (Ex)} At 20th level, the barbarian may rage at will. He no longer has any limitation on the number of times he can rage each day. He may still rage no more than once per encounter.

\subsection{Cleric}
\begin{dtable*}
\lcaption{The Cleric}
\begin{tabularx}{\textwidth}{>{\ccol}p{2em} >{\ccol}p{7em} *{3}{>{\ccol}p{\savecol}} >{\lcol}X}
\thead{Level} & \thead{Base Attack Bonus} & \thead{Fort Save} & \thead{Ref Save} & \thead{Will Save} & \thead{Special} \\
1st & \plus0 & \plus1 & \plus0 & \plus3 & Matters of faith, lesser domain aspect, spontaneous casting \\
2nd & \plus1 & \plus2 & \plus1 & \plus4         & Channel energy, lesser domain aspect \\
3rd & \plus2 & \plus3 & \plus1 & \plus5         & Domain power \\
4th & \plus3 & \plus4 & \plus2 & \plus6         & Domain power \\
5th & \plus3 & \plus4 & \plus2 & \plus7         & Channelled domain power \\
6th & \plus4 & \plus5 & \plus3 & \plus8         & \x  \\
7th & \plus5 & \plus6 & \plus3 & \plus9         & Domain aspect  \\
8th & \plus6/\plus1 & \plus7 & \plus4 & \plus10    & \x  \\
9th & \plus6/\plus1 & \plus7 & \plus4 & \plus11    & Channelled domain power  \\
10th & \plus7/\plus2 & \plus8 & \plus5 & \plus12    & \x  \\
11th & \plus8/\plus3 & \plus9 & \plus5 & \plus13   & Domain aspect  \\
12th & \plus9/\plus4 & \plus10& \plus6 & \plus14    & \x  \\
13th & \plus9/\plus4 & \plus10& \plus6 & \plus15    & Greater channeled domain power  \\
14th & \plus10/\plus5 & \plus11& \plus7 & \plus16    & \x  \\
15th & \plus11/\plus6/\plus1 & \plus12& \plus7 & \plus17 & Greater channeled domain power  \\
16th & \plus12/\plus7/\plus2 & \plus13& \plus8 & \plus18 & \x  \\
17th & \plus12/\plus7/\plus2 & \plus13& \plus8 & \plus19 & Domain mastery  \\
18th & \plus13/\plus8/\plus3 & \plus14& \plus9 & \plus20 & \x  \\
19th & \plus14/\plus9/\plus4 & \plus15& \plus9 & \plus21 & Domain mastery  \\
20th & \plus15/\plus10/\plus5 & \plus16 & \plus10 & \plus22 & \x  \\
\end{tabularx}
\end{dtable*}

\cd{Alignment} A cleric's alignment must be within one step of his deity's (that is, it may be one step away on either the lawful-chaotic axis or the good-evil axis, but not both). A cleric may not be neutral unless his deity's alignment is also neutral.

\cd{Hit Value} 5.

\sssecfake{Class Skills}
The cleric's class skills (and the key attribute for each skill) are Knowledge (arcana) (Int), Knowledge (local) (Int), Knowledge (religion) (Int), Knowledge (the planes) (Int), Heal (Wis), Sense Motive (Wis), Spellcraft (Wis), Persuasion (Cha), and Intimidate (Cha).

\cf{Clr}{Skill Points at 1st Level} 2.

 \sssecfake{Class Features}
All of the following are class features of the cleric.

 \cf{Clr}{Weapon and Armor Proficiency}  Clerics are proficient with simple weapons, any two other weapon groups, light and medium armor, and shields (except tower shields).

\cf{Clr}{Bonus Languages} A cleric's bonus language options include Celestial, Abyssal, and Infernal (the languages of good, chaotic evil, and lawful evil outsiders, respectively). These choices are in addition to the bonus languages available to the character because of his race.

\cf{Clr}{Domains} A cleric chooses two domains, which represent his personal spiritual inclinations. If he has a deity, he must choose his domains from among those his deity offers. A cleric's choice of domains has broad effects on the cleric's spellcasting and supernatural abilities. Each domain has an associated domain attribute which is used for the domain's abilities. The domains and their attributes are listed below. If a domain offers a choice of multiple attributes, the cleric must choose which attribute to use when he gains the domain, and that choice cannot normally be changed.

\subcf{Air} Dexterity or Wisdom
\subcf{Chaos} Charisma
\subcf{Death} Constitution
\subcf{Destruction} Strength or Charisma
\subcf{Earth} Constitution or Wisdom
\subcf{Evil} Charisma
\subcf{Fire} Dexterity or Wisdom
\subcf{Good} Wisdom or Charisma
\subcf{Knowledge} Intelligence
\subcf{Law} Wisdom
\subcf{Magic} Intelligence
\subcf{Protection} Constitution or Wisdom
\subcf{Strength} Strength
\subcf{Travel} Any physical attribute
\subcf{Trickery} Dexterity or Charisma
\subcf{Vitality} Constitution
\subcf{War} Any 
\subcf{Water} Dexterity or Wisdom

\cf{Clr}{Spells} A cleric casts divine spells using his Charisma. To learn or cast a spell, a cleric must have a Charisma at least equal to half the spell's level. The Difficulty Class for a saving throw against a cleric's spell is 10 \add half the cleric's caster level \add the cleric's Charisma.

Like other spellcasters, the number of spells a cleric knows and can cast each day is limited. These limitations are given below on \trefnp{Cleric Spells per Day} and \trefnp{Cleric Spells Known}. A cleric's spells are drawn from the cleric spell list (\pcref{Cleric Spells}), as well as from his domains (\pcref{Cleric Domains}). Two spells at every spell level must be drawn from the cleric's domains; sometimes these spells are normal spells on the cleric's spell list, but often they are only accessible by the domain. A cleric may also choose spells from his domain lists with his normal spells known.

Clerics meditate or pray for their spells. Each cleric must choose a time at which he must spend 1 hour each day performing a ritual, worshipping, or quietly contemplating to regain his daily allotment of spells. They do not need to rest to regain spells.

A cleric can't cast spells of an alignment opposed to his own or his deity's (if he has one). Spells associated with particular alignments are indicated by the chaos, evil, good, and law descriptors in their spell descriptions.

A cleric's magic level is equal to his cleric level.

\begin{dtable}
    \lcaption{Cleric Spells per Day} 
    \centering
    \begin{tabularx}{\columnwidth}{>{\ccol}X *{9}{>{\ccol}p{\spellcol}}}
        & \multicolumn{9}{c}{\thead{---{}---{}---{}---{}---{}---{}---{}---Spell Level---{}---{}---{}---{}---{}---{}---{}---}} \\
        \thead{Level} & \thead{1st} & \thead{2nd} & \thead{3rd} & \thead{4th} & \thead{5th} & \thead{6th} & \thead{7th} & \thead{8th} & \thead{9th} \\
        1st & 3 & \x & \x & \x & \x & \x & \x & \x & \x \\
        2nd & 4 & \x & \x & \x & \x & \x & \x & \x & \x \\
        3rd & 5 & \x & \x & \x & \x & \x & \x & \x & \x \\
        4th & 6 & 3 & \x & \x & \x & \x & \x & \x & \x \\
        5th & 6 & 4 & \x & \x & \x & \x & \x & \x & \x \\
        6th & 6 & 5 & 3 & \x & \x & \x & \x & \x & \x \\
        7th & 6 & 6 & 4 & \x & \x & \x & \x & \x & \x \\
        8th & 6 & 6 & 5 & 3 & \x & \x & \x & \x & \x \\
        9th & 6 & 6 & 6 & 4 & \x & \x & \x & \x & \x \\
        10th & 6 & 6 & 6 & 5 & 3 & \x & \x & \x & \x \\
        11th & 6 & 6 & 6 & 6 & 4 & \x & \x & \x & \x \\
        12th & 6 & 6 & 6 & 6 & 5 & 3 & \x & \x & \x \\
        13th & 6 & 6 & 6 & 6 & 6 & 4 & \x & \x & \x \\
        14th & 6 & 6 & 6 & 6 & 6 & 5 & 3 & \x & \x \\
        15th & 6 & 6 & 6 & 6 & 6 & 6 & 4 & \x & \x \\
        16th & 6 & 6 & 6 & 6 & 6 & 6 & 5 & 3 & \x \\
        17th & 6 & 6 & 6 & 6 & 6 & 6 & 6 & 4 & \x \\
        18th & 6 & 6 & 6 & 6 & 6 & 6 & 6 & 5 & 3 \\
        19th & 6 & 6 & 6 & 6 & 6 & 6 & 6 & 6 & 4 \\
        20th & 6 & 6 & 6 & 6 & 6 & 6 & 6 & 6 & 6 \\
    \end{tabularx}
\end{dtable}

\begin{dtable}
\lcaption{Cleric Spells Known}
\centering
\begin{tabularx}{\columnwidth}{>{\ccol}X *{9}{>{\ccol}p{\spellcol}}}
& \multicolumn{9}{c}{\thead{---{}---{}---{}---{}---{}---{}---{}---Spell Level---{}---{}---{}---{}---{}---{}---{}---}} \\
\thead{Level} & \thead{1st} & \thead{2nd} & \thead{3rd} & \thead{4th} & \thead{5th} & \thead{6th} & \thead{7th} & \thead{8th} & \thead{9th} \\
1st  & 0\plus 2 & \x & \x & \x & \x & \x & \x & \x & \x \\
2nd  & 1\plus 2 & \x & \x & \x & \x & \x & \x & \x & \x \\
3rd  & 2\plus 2 & \x & \x & \x & \x & \x & \x & \x & \x \\
4th  & 2\plus 2 & 0\plus 2 & \x & \x & \x & \x & \x & \x & \x \\
5th  & 3\plus 2 & 1\plus 2 & \x & \x & \x & \x & \x & \x & \x \\
6th  & 3\plus 2 & 1\plus 2 & 0\plus 2 & \x & \x & \x & \x & \x & \x \\
7th  & 3\plus 2 & 2\plus 2 & 1\plus 2 & \x & \x & \x & \x & \x & \x \\
8th  & 3\plus 2 & 2\plus 2 & 1\plus 2 & 0\plus 2 & \x & \x & \x & \x & \x \\
9th  & 3\plus 2 & 2\plus 2 & 2\plus 2 & 1\plus 2 & \x & \x & \x & \x & \x \\
10th & 3\plus 2 & 2\plus 2 & 2\plus 2 & 1\plus 2 & 0\plus 2 & \x & \x & \x & \x \\
11th & 3\plus 2 & 2\plus 2 & 2\plus 2 & 2\plus 2 & 1\plus 2 & \x & \x & \x & \x \\
12th & 3\plus 2 & 2\plus 2 & 2\plus 2 & 2\plus 2 & 1\plus 2 & 0\plus 2 & \x & \x & \x \\
13th & 3\plus 2 & 2\plus 2 & 2\plus 2 & 2\plus 2 & 2\plus 2 & 1\plus 2 & \x & \x & \x \\
14th & 3\plus 2 & 2\plus 2 & 2\plus 2 & 2\plus 2 & 2\plus 2 & 1\plus 2 & 0\plus 2 & \x & \x \\
15th & 3\plus 2 & 2\plus 2 & 2\plus 2 & 2\plus 2 & 2\plus 2 & 2\plus 2 & 1\plus 2 & \x & \x \\
16th & 3\plus 2 & 2\plus 2 & 2\plus 2 & 2\plus 2 & 2\plus 2 & 2\plus 2 & 1\plus 2 & 0\plus 2 & \x \\
17th & 3\plus 2 & 2\plus 2 & 2\plus 2 & 2\plus 2 & 2\plus 2 & 2\plus 2 & 1\plus 2 & 1\plus 2 & \x \\
18th & 3\plus 2 & 2\plus 2 & 2\plus 2 & 2\plus 2 & 2\plus 2 & 2\plus 2 & 1\plus 2 & 1\plus 2 & 0\plus 2 \\
19th & 3\plus 2 & 2\plus 2 & 2\plus 2 & 2\plus 2 & 2\plus 2 & 2\plus 2 & 1\plus 2 & 1\plus 2 & 1\plus 2 \\
20th & 3\plus 2 & 2\plus 2 & 2\plus 2 & 2\plus 2 & 2\plus 2 & 2\plus 2 & 1\plus 2 & 1\plus 2 & 1\plus 2
\end{tabularx}
\end{dtable}

\begin{dtable!*}
\lcaption{Deities}
\begin{tabularx}{\textwidth}{X l X}
\thead{Deity} & \thead{Alignment} & \thead{Domains} \\
Guftas, horse god of justice & Lawful good & Animal, Law, Strength, Travel \\
Lucied, paladin god of justice & Lawful good & Destruction, Good, Leadership, War \\
Simor, fighter god of protection & Lawful good & Good, Law, Life, Protection \\
Vanya, centaur god of nature & Neutral good & Good, Plant, Strength, War \\
Brushtwig, pixie god of creativity & Chaotic good & Chaos, Good, Trickery \\
Chavi, bard god of stories & Chaotic good & Chaos, Knowledge, Leadership, Trickery \\
Ivan Ivanovitch, bear god of strength & Chaotic good & Animal, Chaos, Strength, War \\
Raphael, monk god of retribution & Lawful neutral & Death, Law, Protection, Travel \\
Declan, god of fire & True neutral & Destruction, Fire, Knowledge, Magic \\
%Kalten, half-orc god of smiths
Kurai, shaman god of nature & True neutral & Air, Earth, Fire, Water \\
Murdoc, bard god of mercenaries & Chaotic neutral & Destruction, Knowledge, Leadership, War\\
Daeghul, god of slaughter & Chaotic evil & Destruction, Evil, Magic, War \\
%Ribo, halfling god of trickery & Chaotic neutral & Chaos, Trickery, Water
\end{tabularx}
\end{dtable!*}

\cf{Clr}{Lesser Domain Aspect (Su)} A cleric's abilities are shaped by his domains. Each domain grants a lesser domain aspect. Lesser domain aspects are not activated. Options for domain aspects are listed at \pcref{Lesser Domain Aspects}. A cleric gains an additional lesser domain aspect at 2nd level.

\cf{Clr}{Matters of Faith (Ex)} A cleric gains a \plus10 competence bonus to Knowledge (religion) checks made concerning his faith, such as questions about his deity or philosophy, religious rites, holy sites, and so on. Further, he is treated as being trained in Knowledge (religion) when making such checks, whether or not he actually is.

\cf{Clr}{Channel Energy (Su)} At 2nd level, by channeling the power of his faith through his holy (or unholy) symbol, a cleric can act as a powerful conduit of divine energy. He must choose whether to channel positive or negative energy. Once this choice is made, it cannot be reversed. This decision also determines whether the cleric casts spontaneous \spellindirect{cure light wounds}{cure} or \spellindirect{inflict light wounds}{inflict} spells (see below).

When a cleric channels energy, it affects all creatures in a \areamed radius burst centered on him, including himself. The cleric may choose to exclude a number of creatures from the effect equal to 1 \add half his Wisdom. The amount of damage dealt (if negative energy is channeled) or healed (if positive energy is channeled) is equal to 1d6 damage per two cleric levels. Each affected creature can make a Fortitude save to halve the damage. The DC of this save is equal to 10 \add the cleric's level \add the cleric's Charisma.

Channeling energy is a standard action that does not provoke attacks of opportunity. A cleric can channel energy a number of times per day equal to 3 \add half his Charisma. A cleric must be able to present his holy symbol to use this ability. The abilities used 

\cf{Clr}{Domain Power (Su)} At 3rd level, a cleric gains a domain power from one of his domains. Using a domain power requires a standard action that does not provoke attacks of opportunity unless otherwise noted. All domain powers can be used at will unless otherwise noted. If a domain power allows a saving throw, the DC is equal to 10 \add the cleric's level \add the cleric's domain attribute.

The domain powers for each domain are described at \pcref{Domain Powers}. The cleric gains an additional domain power from one of his domains at 4th level.

\cf{Clr}{Channeled Domain Power (Su)} At 5th level, a cleric gains a channeled domain power from one of his domains. Unless otherwise stated, using a channelled domain power is identical to using channel energy and consumes a use of the cleric's channel energy ability. Instead of channeling positive or negative energy, the cleric instead gains the effect of the channelled domain power. If a channeled domain power deals damage, it functions like channeling negative energy unless otherwise noted. If a channeled domain power heals damage, it functions like channeling postive energy unless otherwise noted. The save DC of a cleric's channeled domain power is equal to 10 \add cleric level \add domain attribute. The channeled domain powers are described at \pcref{Channeled Domain Powers}. The cleric gains an additional channeled domain power at 9th level.

\cf{Clr}{Domain Aspect (Su)} At 7th level, a cleric gains a domain aspect from one of his domains. Domain aspects do not require an action to activate. Options for domain aspects are listed at \pcref{Domain Aspects}.  The cleric gains an additional domain aspect from one of his domains at 11th level.

\cf{Clr}{Greater Channeled Domain Power (Su)} At 13th level, a cleric gains a greater channeled domain power from one of his domains. Using a greater channeled domain power consumes two uses of the cleric's channel energy ability. Instead of channeling positive or negative energy, the cleric instead gains the effect of the greater channeled domain power. Options for greater channeled domain powers are listed at \pcref{Greater Channeled Domain Powers}.  The cleric gains an additional greater channeled domain power at 15th level.

\cf{Clr}{Domain Mastery (Su)} At 17th level, a cleric gains a domain mastery from one of his domains. Options for domain masteries are listed at \pcref{Domain Masteries}.  The cleric gains an additional domain mastery at 19th level.

\ssecfake{Cleric Domain Abilities}

\subsubsection{Lesser Domain Aspects}\label{Lesser Domain Aspects}

\subcf{Air} The cleric adds Athletics to his cleric class skill list and may treat it as a Dexterity-based skill whenever doing so would be beneficial to him.
\subcf{Chaos} Whenever the cleric rolls a 2 on a d20 roll, he immediately rerolls.
\subcf{Death} The cleric adds half his cleric level to the maximum amount of critical damage he can take without dying.
\subcf{Destruction} The cleric can ignore half of the hardness of any object he damages, whether with spells or weapons.
\subcf{Earth} The cleric gains Endurance as a bonus feat.
\subcf{Evil} The cleric gains Skill Focus (Intimidate) as a bonus feat.
\subcf{Fire} The cleric gains fire damage reduction equal to twice his cleric level. This damage reduction allows him to ignore the first points of fire damage he would take each round.
\subcf{Good} The cleric gains Skill Focus (Persuasion) as a bonus feat.
\subcf{Knowledge} The cleric adds all Knowledge skills to his cleric class skill list.
\subcf{Law} Whenever the cleric rolls a 2 on a d20 roll, he treats it as if he had rolled a 10.
\subcf{Magic} The cleric gains an additional spell slot at his highest level of spells.
\subcf{Protection} The cleric gains his choice of Covering Fire or Guardian as a bonus feat.
\subcf{Strength} The cleric adds Athletics, Climb, and Swim to his cleric class skill list.
\subcf{Travel} The cleric adds Knowledge (geography) and Survival to his cleric class skill list.
\subcf{Trickery} The cleric adds Bluff and Disguise to his cleric class skill list.
\subcf{Vitality} The cleric gains a \plus2 competence bonus to attack rolls with Vitalism spells.
\subcf{War} The cleric gains Weapon Focus with his deity's favored weapon group as a bonus feat. If he does not have a deity, he gains it in a weapon group related to his alignment or ideals.
\subcf{Water} The cleric adds Swim to his cleric class skill list and halves the penalties he takes for fighting underwater.

\subsubsection{Domain Powers}\label{Domain Powers}

\subcf{Air -- Lightning Arc} The cleric fires an arc of lightning at a creature within \rngmed range. If he hits on a ranged touch attack, the creature takes d8 electricity damage \add d8 per four cleric levels after 1st.
\subcf{Chaos -- Touch of Chaos} A creature within \rngclose range is imbued with the power of chaos for 5 rounds. Any attack roll, saving throw, or check it makes where it rolls an odd number gets a \minus2 penalty. Creatures affected by this power are treated as lawful for the purpose of the cleric's chaotic spells and effects.
\subcf{Death -- Death's Door} If the cleric succeeds on a melee touch attack, he rolls d6 per cleric level. If the total at least equals the touched creature's current hit points, it loses all of its hit points and is disabled. If the total at least equals the creature's critical damage, it is instantly slain. This is a death effect.
\subcf{Destruction -- Destructive Resonance} A touched creature or object takes d10 sonic damage \add d10 per four cleric levels after 1st.
\subcf{Earth -- Raise Earth} As a swift action, the cleric can command the earth to raise up any creature within \rngclose range, allowing it to stand from prone without taking an action. This ability only works on natural earth or stone.
\subcf{Evil -- Touch of Evil} A touched creature is sickened and is treated as good for the purpose of the cleric's evil spells and effects. This effect lasts for 5 rounds.
\subcf{Fire -- Firebolt} The cleric fires a bolt of fire at a creature within \rngclose range. If he hits on a ranged touch attack, the creature takes d8 fire damage \add d8 per four cleric levels after 1st.
\subcf{Good -- Holy Touch} A touched creature is dazzled and treated as evil for the purpose of good spells and effects that the cleric creates. The effect lasts for 5 rounds.
\subcf{Knowledge -- Minor Vision} A touched creature is granted a brief vision of the future, giving it a \plus2 enhancement bonus to attack rolls, weapon damage rolls, and checks for 1 round.
\subcf{Law -- Touch of Order} A creature within \rngclose range is imbued with the leveling power of order for 5 rounds. Any attack roll, saving throw, or check it makes with a roll of 11 or higher gets a \minus2 penalty. In addition, it is treated as being chaotic for the purpose of the cleric's lawful spells and effects.
\subcf{Magic -- Breach Defenses} The cleric fires a magical ray at a creature within \rngclose range. If he hits, the target takes d6 damage \add 1 per cleric level. In addition, it takes a \minus2 penalty to saving throws against  the cleric's spells for 1 round.
\subcf{Nature -- Wild Speech} The cleric gains wild speech, as the druid ability, with a druid level equal to half his cleric level (minimum 1). This power can be used a number of times per day equal to half the cleric's level.
\subcf{Protection -- Martyr's Touch} For the next 5 rounds, the cleric takes half of the damage the touched creature would take, as the \spell{shield other} spell. If the cleric get farther than \rngmed range from the touched creature during this time, the effect is broken.
\subcf{Strength -- Surge of Strength} The cleric gains a \plus2 enchancement bonus to Strength for 1 round. \bonusscalingdescription This power can be used a number of times per day equal to 1 \add half the cleric's Strength.
\subcf{Travel -- Free Stride} As a swift action, the cleric can gain the ability to move through difficult terrain at full speed for 1 round. This power can be used a number of times per day equal to 1 \add half the cleric's Dexterity.
\subcf{Trickery -- Liar's Boon} As a swift action, the cleric can gain a \plus3 enhancement bonus to Bluff and Disguise checks for 5 minutes. This power can be used a number of times per day equal to 1 \add half the cleric's Charisma.
\subcf{Vitality -- Vital Reach} As a swift action, the cleric can cause a dying creature within \rngmed range to stabilize or take 1 critical damage, as he desires.
\subcf{War -- Warrior's Boon} A creature touched creature gains a \plus2 enhancement bonus to attack rolls for 1 round. \bonusscalingdescription This power can be used a number of times per day equal to 1 \add half the cleric's Charisma.
\subcf{Water -- Drowning Orb} The cleric fires an orb of water at a creature within \rngclose range. If he hits on a ranged touch attack, the orb attempts to force itself into the creature's mouth and nose to drown it, dealing d8 nonlethal damage \add 1 per cleric level. In addition, the creature must make a Fortitude save or be forced to hold its breath for 5 rounds.

\subsubsection{Channeled Domain Powers}\label{Channeled Domain Powers}

\subcf{Air} The cleric channels electrical energy. This deals electricity damage and heals creatures with the air subtype. A Reflex save halves the damage.
\subcf{Chaos} The cleric channels anarchic energy. Roll randomly each time this power is used to determine whether it functions as channeling negative energy or channeling positive energy, except that it always heals chaotic creatures and harms lawful creatures.
\subcf{Death} The cleric channels negative energy, except that any creatures dealt critical damage by this power are instantly killed, with no saving throw allowed. This is a death effect.
\subcf{Destruction} The cleric channels destructive energy. This deals untyped damage and allows a Fortitude save for half damage.
\subcf{Earth} The cleric channels seismic energy. This deals physical bludgeoning damage to all creatures on the ground. A Reflex save halves the damage.
\subcf{Evil} The cleric channels negative energy, except that it has no effect on evil creatures.
\subcf{Fire} The cleric channels fiery energy. This deals fire damage and heals creatures with the fire subtype. A Reflex save halves the damage.
\subcf{Good} The cleric channels positive energy, except that it has no effect on evil creatures.
\subcf{Knowledge} The cleric channels insight from knowledge. This functions like channeling positive energy, except that each affected creature gains an enhancement bonus equal to half the cleric's level on the next attack roll, damage roll, saving throw, or check that it makes instead of being healed. If this bonus is not used within 5 rounds, it is wasted.
\subcf{Law} The cleric channels axiomatic energy. This deals 4 damage per two cleric levels to all creatures within a 40 ft. cube centered on the cleric. A Will save halves the damage, and it has no effect on lawful creatures.
\subcf{Magic} The cleric channels magical energy. This heals creatures who can cast spells and deals damage to creatures who cannot.
\subcf{Nature} The cleric channels positive energy, except that the cleric can decide whether it acts as positive or negative energy to animals and plants.
\subcf{Protection} The cleric grants each affected creature temporary hit points equal to half the amount that channelling positive energy would have healed. Undead gain temporary hit points as well. The temporary hit points last for 5 rounds.
\subcf{Strength} The cleric channels positive energy. Creatures at their maximum hit points after being healed gain a \plus1 enhancement bonus to Strength. This bonus increases to \plus2 at 11th level.
\subcf{Travel} The cleric channels positive energy. All healed creatures can ignore difficult terrain for 1 round.
\subcf{Trickery} The cleric channels negative energy. It deals half damage, but all damaged creatures who fail their Will saves are bewildered for 5 rounds. This is a mind-affecting effect.
\subcf{Vitality} The cleric channels energy as normal, except that he gains a \plus2 circumstance bonus to his cleric level.
\subcf{War} The cleric channels energy as normal, except that he can exclude two additional creatures from the effect.

\subsubsection{Domain Aspects}\label{Domain Aspects}

\subcf{Air -- Stormwalker} The cleric suffers no penalties for inclement weather or severe winds and takes half damage from falling damage.
\subcf{Chaos -- Fortune's Friend} Whenever the cleric rolls randomly for an effect, such as when he is affected by the \spell{confusion} spell, he may roll twice and take whichever result he prefers.
\subcf{Death -- Lifedrinker} Whenever the cleric kills a creature with a death effect other than \spell{death knell}, he automatically gains the benefits of a \spell{death knell} spell as if it was cast on the creature he killed.
\subcf{Destruction -- Swordcleaver} Whenever the cleric breaks or destroys an object with a melee attack, he may take a free melee attack on a creature adjacent to him at the same attack bonus.
\subcf{Earth -- Anchored} The cleric gains a \plus4 enhancement bonus to Maneuver Class against bull rush, overrun, and trip attempts while standing on solid ground.
\subcf{Evil -- Malevolent Magic} Good creatures take a \minus2 penalty to saving throws against the cleric's spells.
\subcf{Fire -- Flamebearer} The cleric gains Spell Focus (Fire) as a bonus feat.
\subcf{Good -- Purifying Magic} Evil creatures take a \minus2 penalty to saving throws against the cleric's spells.
\subcf{Knowledge -- Knowledge Mastery} The cleric may choose a number of Knowledge skills equal half to his Intelligence (minimum 1). He may take 10 with those skills if he is not in danger or rushed.
\subcf{Law -- Certain Triumph} Whenever the cleric would take 10, he may instead take 12, treating any roll lower than a 12 as if it had been a 12.
\subcf{Magic -- Metamagic Feat} The cleric gains a bonus metamagic feat.
\subcf{Nature -- Favored Terrain} The cleric gains a favored terrain, as the ranger class feature.
\subcf{Protection -- Faithful Shield} The cleric may maintain concentration on Abjuration (Shielding) effects as a swift action.
\subcf{Strength -- Mighty Magic} The cleric can add half his Strength to his casting attribute to meet the minimum attribute requirements to cast general cleric spells.
\subcf{Travel -- Rapid Traveller} The cleric gains a \plus10 foot competence bonus to his base land speed.
\subcf{Trickery -- }
\subcf{Vitality -- }
\subcf{War -- Weapon Specialization} The cleric gains Weapon Specialization in his deity's favored weapon group as a bonus feat. If he does not have a deity, he gains it in a weapon group related to his alignment or ideals.
\subcf{Water -- Water Breathing} The cleric may breathe and speak normally while underwater, as the \spell{water breathing} ritual. He may also pass through boggy or wet areas with no penalty to his movement speed.

\subsubsection{Greater Channeled Domain Powers}

\subcf{Air -- Mantle of Air} As a swift action, the cleric can surround himself in a mantle of air for 5 rounds. Thrown and projectile weapons have a 50\% chance to miss him while this effect is active. Unusually large weapons, such as a giant's boulders, may suffer a decreased miss chance as appropriate to their size.
\subcf{Chaos -- Anarchic Weapon} As a swift action, the cleric can imbue a touched weapon with the anarchic weapon property for 5 rounds.
\subcf{Death -- Channel Death} The cleric channels negative energy as the Death channeled domain power, except that any creature brought to 0 hit points by this effect immediately dies. This is a death effect.
\subcf{Destruction -- Tide of Destruction} The cleric channels destructive energy as the Destruction channeled domain power, except that any creature damaged by the effect is also filled with a destructive resonance for 5 rounds. the first time each round that each affected creature takes damage, that damage is increased by half the cleric's level. This is considered a circumstance bonus to damage.
\subcf{Earth -- Mantle of Earth} As a swift action, the cleric can surround himself in a mantle of earth for 5 rounds. He gains physical damage reduction equal to half his cleric level that is only overcome by adamantine weapons.
\subcf{Evil -- Unholy Weapon} As a swift action, the cleric can imbue a touched weapon with the unholy weapon property for 5 rounds.
\subcf{Fire -- Mantle of Fire} As a swift action, the cleric can surround himself in a mantle of fire for 5 rounds. He gains the effect of a \spell{fire shield} spell, with a caster level equal to his cleric level.
\subcf{Good -- Holy Weapon} As a swift action, the cleric can imbue a touched weapon with the holy weapon property for 5 rounds.
\subcf{Knowledge -- See the Truth} As a swift action, the cleric can gain the benefit of the \spell{true seeing} spell for 1 round.
\subcf{Law -- Axiomatic Weapon} As a swift action, the cleric can imbue a touched weapon with the axiomatic weapon property for 5 rounds.
\subcf{Life -- Persistent Life} The cleric can restore life, as the \spell{raise dead} spell, to a touched corpse that died no more than 5 rounds previously.
\subcf{Magic -- Enhance Metamagic} The cleric can use this power as part of casting a spell affected by a metamagic feat. If he does, the spell costs a spell slot of one level lower than normal, and applying the metamagic does not increase the casting time of the spell.
\subcf{Nature -- Wild Aspect} When the cleric gains this ability, he chooses one wild aspect ability, as if he were a were a druid of a level equal to his cleric level. When he uses this ability, he may assume that wild aspect. This effect lasts as long as that wild aspect would normally last.
\subcf{Protection -- Mass Sanctuary} The cleric channels protective energy as the Protection channeled domain power, except that instead of being healed, each affected creature separately gains the benefit of a \spell{sanctuary} spell for 5 rounds.  If a member of the group attacks, the effect is broken for that creature, but not for the whole group.
\subcf{Strength -- Surge of Strength} As a swift action, the cleric can add his cleric level as an enhancement bonus to his Strength for a single round.
\subcf{Travel -- Uninhibited Movement} As a swift action, the cleric can gain the ability to move without provoking any attacks of opportunity for a single round.
\subcf{Trickery -- Swift Invisibility} As a swift action, the cleric can gain the benefit of the \spell{invisibility} spell for a single round.
\subcf{War -- Warmaster's Boon} The cleric can use this power as part of casting a spell that targets himself with a duration of \durshort or longer and a range greater than personal. If he uses the power, the spell affects all of his allies within an \areamed radius. However, the spell lasts for no longer than half his Charisma score in rounds.
\subcf{Water -- Aquatic Globe} The cleric creates water out of thin air in an immobile \areamed radius emanation from his location for 5 rounds. Everything within the area is treated as if it were underwater. At the end of the duration, the water evaporates, leaving no trace that it was ever there.

\subsubsection{Domain Masteries}\label{Domain Masteries}

\subcf{Air -- Flight} The cleric gains a fly speed (good maneuverability) equal to his land speed. He may remain flying for up to 5 rounds at a time. After that, he must land for 1 round before he can fly again.
\subcf{Chaos -- Avatar of Luck} Once per round, the cleric can add d6 as a circumstance bonus to any attack roll or check. He may declare the use of the ability after failing the roll, but before any additional effects are resolved, potentially making it succeed where it would have failed.
\subcf{Death -- Deathfeeder} The cleric constantly radiates a \areamed radius emanation of death. Whenever a creature dies within the area, he gains the benefits of the \spell{death knell} spell as if it had been cast on the creature.
\subcf{Destruction -- Ruinbringer} The cleric's attacks and spells ignore all damage reduction and hardness (but not damage immunity).
\subcf{Earth -- Earth Glide} The cleric gains the earth glide ability, as an earth elemental.
\subcf{Evil -- Avatar of Evil} The cleric continuously gains the benefits of the \spell{protection from good} spell, with a caster level equal to his cleric level. If the effect is dispelled or suppressed, he can resume it as a swift action.
\subcf{Fire -- Flame Incarnate} The cleric gains the fire subtype, making him immune to fire but giving him a 50\% vulnerability to cold damage. When ever he uses fire spells or effects, he may freely exclude areas or creatures within the area of effect.
\subcf{Good -- Avatar of Good} The cleric continuously gains the benefits of the \spell{protection from evil} spell, with a caster level equal to his cleric level. If the effect is dispelled or suppressed, he can resume it as a swift action.
\subcf{Knowledge -- Combat Insight} The cleric gains a \plus2 circumstance bonus to attack rolls, checks, and saving throws against creatures he has identified with a successful Knowledge check.
\subcf{Law -- Avatar of Order} Once per round, the cleric can take 5 on an attack roll or check, even while stressed or distracted. He may declare the use of this ability after rolling below a 5, but before any additional effects are resolved, potentially causing it to succeed where it would have failed.
\subcf{Life -- Fountain of Life} The cleric gains fast healing 1. All of the cleric's healing spells and abilities restore critical damage as easily as if it were hit point damage.
\subcf{Magic -- Spellbreaker} The cleric gains spell resistance. A creature with spell resistance may always make a saving throw when a spell is cast on it. The saving throw type is indicated by the spell. If it succeeds, the spell has no effect on it.
\subcf{Nature -- Natural Power} Whenever the cleric is in natural terrain, he gains a \plus2 enhancement bonus to caster level and the improved natural casting ability, as the druid class feature.
\subcf{Protection -- Martyr's Gift} The cleric constantly radiates a \areamed radius emanation of protective energy. Whenever a creature within the area takes damage, the cleric can choose to take half of that damage instead, as the \spell{shield other} spell.
\subcf{Strength -- Might of the Gods} The cleric gains the larger than life ability, as the barbarian class feature.
\subcf{Travel -- Perfect Stride} The cleric gains perfect stride, as the ranger class feature. He constantly acts as if he were under the effect of a \spell{freedom} spell, except that it does not allow him to act normally underwater.
\subcf{Trickery -- Exemplar of Deceit} The cleric continuously gains the benefits of the \spell{nondetection} spell, with a caster level equal to his cleric level, except that it also protects him from any spells or effects which would prevent him from lying or reveal his lies. If the effect is dispelled or suppressed, he can resume it as a swift action.
\subcf{War -- Warmaster's Favor} The cleric continuously gains the benefits of the \spell{divine favor} spell, with a caster level equal to his cleric level. If the effect is dispelled or suppressed, he can resume it as a swift action.
\subcf{Water -- Water's Flow} At any time, the cleric can transform into a rushing flow of water as a move action that does not provoke attacks of opportunity. As part of the action, he may move up to his movement speed in any direction that water could go. His speed is halved when moving uphill and doubled when moving downhill. He does not provoke attacks of opportunity during this movement, and has physical damage reduction 10 while in this form. At the end of his movement, he regains his normal form.

\subsubsection{Ex-Clerics}
A cleric who grossly violates the code of conduct required by his god loses all spells and class features, except for armor and shield proficiencies and proficiency with simple weapons. He cannot thereafter gain levels as a cleric of that god until he atones (see the \spell{atonement} spell description).

\subsection{Druid}
\begin{dtable*}
\lcaption{The Druid}
\begin{tabularx}{\textwidth}{>{\ccol}p{\levelcol} >{\centering}p{\babcolavg} *{3}{>{\ccol}p{\savecol}} >{\ccol}X}
\thead{Level} & \thead{Base Attack Bonus} & \thead{Fort Save} & \thead{Ref Save} & \thead{Will Save} & \thead{Special} \\
1st & \plus0 & \plus3 & \plus0 & \plus1 & Nature sense, natural casting, wild speech \\
2nd & \plus1 & \plus4 & \plus1 & \plus2 & Woodland stride \\
3rd & \plus2 & \plus5 & \plus1 & \plus3 & Wild aspect \\
4th & \plus3 & \plus6 & \plus2 & \plus4 & Venom immunity \\
5th & \plus3 & \plus7 & \plus2 & \plus4 & Wild aspect \\
6th & \plus4 & \plus8 & \plus3 & \plus5 & Wild speech (plants) \\
7th & \plus5 & \plus9 & \plus3 & \plus6 & Wild aspect \\
8th & \plus6/\plus1 & \plus10& \plus4 & \plus7 & Improved wild speech \\
9th & \plus6/\plus1 & \plus11& \plus4 & \plus7 & Wild aspect \\
10th & \plus7/\plus2 & \plus12& \plus5 & \plus8 & Improved natural casting \\
11th & \plus8/\plus3 & \plus13 & \plus5 & \plus9  & Greater wild aspect \\
12th & \plus9/\plus4 & \plus14 & \plus6 & \plus10 & A thousand faces \\
13th & \plus9/\plus4 & \plus15 & \plus6 & \plus10 & Greater wild aspect \\
14th & \plus10/\plus5 & \plus16 & \plus7 & \plus11 & Timeless body \\
15th & \plus11/\plus6/\plus1 & \plus17 & \plus7 & \plus12 & Greater wild aspect \\
16th & \plus12/\plus7/\plus2 & \plus18 & \plus8 & \plus13 & Greater wild speech \\
17th & \plus12/\plus7/\plus2 & \plus19 & \plus8 & \plus13 & Greater wild aspect \\
18th & \plus13/\plus8/\plus3 & \plus20 & \plus9 & \plus14 & Totemic aspect \\
19th & \plus14/\plus9/\plus4 & \plus21 & \plus9 & \plus15 & Greater wild aspect \\
20th & \plus15/\plus10/\plus5 & \plus22 & \plus10 & \plus16 & Greater natural casting \\
\end{tabularx}
\end{dtable*}

\cd{Alignment} Neutral good, lawful neutral, neutral, chaotic
neutral, or neutral evil.
\cd{Hit Value} 5.
 \sssecfake{Class Skills}
The druid's class skills (and the key attribute for each skill) are Athletics (Str), Climb (Str), Swim (Str), Ride (Dex), Stealth (Dex), Knowledge (geography), Knowledge (nature) (Int), Heal (Wis), Perception (Wis), Survival (Wis), and Creature Handling (Cha).
\cf{Drd}{Skill Points at 1st Level} 4

\sssecfake{Class Features}
All of the following are class features of the druid.

\cf{Drd}{Weapon and Armor Proficiency} Druids are proficient with simple weapons, any one other weapon group, scimitars, sickles, and slings.
\par Druids are proficient with light and medium armor, but are prohibited from wearing metal armor; thus, they may wear only padded, leather, or hide armor. (A druid may also wear wooden armor that has been altered by the \spell{ironwood} spell so that it functions as though it were steel. See the \spell{ironwood} spell description.) Druids are proficient with shields (except tower shields) but must use only wooden ones.
\par A druid who wears prohibited armor or carries a prohibited shield is unable to cast druid spells or use any of her supernatural class abilities while doing so and for 24 hours thereafter.

\cf{Drd}{Bonus Languages} A druid's bonus language options include Sylvan, the language of magical woodland creatures. This choice is in addition to the bonus languages available to the character because of her race.

A druid also knows Druidic, a secret language known only to druids, which she learns upon becoming a 1st-level druid. Druidic is a free language for a druid; that is, she knows it in addition to her regular allotment of languages and it doesn't take up a language slot. Druids are forbidden to teach this language to nondruids. Druidic has its own alphabet.

\cf{Drd}{Spells} A druid casts divine spells using her Wisdom. To learn or cast a spell, a druid must have a Wisdom at least equal to half the spell's level. The Difficulty Class for a saving throw against a druid's spell is 10 \add half the druid's caster level \add the druid's Wisdom .

Like other spellcasters, the number of spells a druid knows and can cast each day is limited. These limitations are given below on \trefnp{Druid Spells per Day} and \trefnp{Druid Spells Known}. A druid's spells are drawn from the druid spell list (see \pcref{Druid Spells}).

Druids meditate or pray for their spells. Each druid must choose a time at which she must spend 1 hour each day performing a ritual, worshipping, or quietly contemplating to regain her daily allotment of spells. They do not need to rest to regain spells.

A druid's magic level is equal to her druid level.

\begin{dtable}
    \lcaption{Druid Spells per Day} 
    \centering
    \begin{tabularx}{\columnwidth}{>{\ccol}X *{9}{>{\ccol}p{\spellcol}}}
        & \multicolumn{9}{c}{\thead{---{}---{}---{}---{}---{}---{}---{}---Spell Level---{}---{}---{}---{}---{}---{}---{}---}} \\
        \thead{Level} & \thead{1st} & \thead{2nd} & \thead{3rd} & \thead{4th} & \thead{5th} & \thead{6th} & \thead{7th} & \thead{8th} & \thead{9th} \\
        1st & 3 & \x & \x & \x & \x & \x & \x & \x & \x \\
        2nd & 4 & \x & \x & \x & \x & \x & \x & \x & \x \\
        3rd & 5 & \x & \x & \x & \x & \x & \x & \x & \x \\
        4th & 6 & 3 & \x & \x & \x & \x & \x & \x & \x \\
        5th & 6 & 4 & \x & \x & \x & \x & \x & \x & \x \\
        6th & 6 & 5 & 3 & \x & \x & \x & \x & \x & \x \\
        7th & 6 & 6 & 4 & \x & \x & \x & \x & \x & \x \\
        8th & 6 & 6 & 5 & 3 & \x & \x & \x & \x & \x \\
        9th & 6 & 6 & 6 & 4 & \x & \x & \x & \x & \x \\
        10th & 6 & 6 & 6 & 5 & 3 & \x & \x & \x & \x \\
        11th & 6 & 6 & 6 & 6 & 4 & \x & \x & \x & \x \\
        12th & 6 & 6 & 6 & 6 & 5 & 3 & \x & \x & \x \\
        13th & 6 & 6 & 6 & 6 & 6 & 4 & \x & \x & \x \\
        14th & 6 & 6 & 6 & 6 & 6 & 5 & 3 & \x & \x \\
        15th & 6 & 6 & 6 & 6 & 6 & 6 & 4 & \x & \x \\
        16th & 6 & 6 & 6 & 6 & 6 & 6 & 5 & 3 & \x \\
        17th & 6 & 6 & 6 & 6 & 6 & 6 & 6 & 4 & \x \\
        18th & 6 & 6 & 6 & 6 & 6 & 6 & 6 & 5 & 3 \\
        19th & 6 & 6 & 6 & 6 & 6 & 6 & 6 & 6 & 4 \\
        20th & 6 & 6 & 6 & 6 & 6 & 6 & 6 & 6 & 6 \\
    \end{tabularx}
\end{dtable}

\begin{dtable}
    \lcaption{Druid Spells Known}
    \centering
    \begin{tabularx}{\columnwidth}{X *{9}{p{1.1em}}}
        & \multicolumn{9}{c}{\thead{---{}---{}---{}---{}---{}---{}---Spell Level---{}---{}---{}---{}---{}---{}---}} \\
        \thead{Level} & \thead{1st} & \thead{2nd} & \thead{3rd} & \thead{4th} & \thead{5th} & \thead{6th} & \thead{7th} & \thead{8th} & \thead{9th} \\
        1st  & 1 & \x & \x & \x & \x & \x & \x & \x & \x \\
        2nd  & 2 & \x & \x & \x & \x & \x & \x & \x & \x \\
        3rd  & 3 & \x & \x & \x & \x & \x & \x & \x & \x \\
        4th  & 3 & 1 & \x & \x & \x & \x & \x & \x & \x \\
        5th  & 4 & 2 & \x & \x & \x & \x & \x & \x & \x \\
        6th  & 4 & 2 & 1 & \x & \x & \x & \x & \x & \x \\
        7th  & 4 & 3 & 2 & \x & \x & \x & \x & \x & \x \\
        8th  & 4 & 3 & 2 & 1 & \x & \x & \x & \x & \x \\
        9th  & 4 & 3 & 3 & 2 & \x & \x & \x & \x & \x \\
        10th & 4 & 3 & 3 & 2 & 1 & \x & \x & \x & \x \\
        11th & 4 & 3 & 3 & 3 & 2 & \x & \x & \x & \x \\
        12th & 4 & 3 & 3 & 3 & 2 & 1 & \x & \x & \x \\
        13th & 4 & 3 & 3 & 3 & 3 & 2 & \x & \x & \x \\
        14th & 4 & 3 & 3 & 3 & 3 & 2 & 1 & \x & \x \\
        15th & 4 & 3 & 3 & 3 & 3 & 3 & 2 & \x & \x \\
        16th & 4 & 3 & 3 & 3 & 3 & 3 & 2 & 1 & \x \\
        17th & 4 & 3 & 3 & 3 & 3 & 3 & 2 & 2 & \x \\
        18th & 4 & 3 & 3 & 3 & 3 & 3 & 2 & 2 & 1 \\
        19th & 4 & 3 & 3 & 3 & 3 & 3 & 2 & 2 & 2 \\
        20th & 4 & 3 & 3 & 3 & 3 & 3 & 2 & 2 & 2
    \end{tabularx}
\end{dtable}

\cf{Drd}{Natural Casting (Ex)} A druid's spells channel of nature itself. Though her energy is a necessary component to bring natural power to bear, she need not be the focus of its might. Whenever she casts a druid area spell that would emanate from her, such as a cone or line spell, she may cause the spell to originate from any position within 10 feet of her. All other aspects of the spell are unchanged.

For example, a druid casting \spell{gust of wind} could create a line of wind originating from 10 feet to her right. The line would extend 50 feet out from that point, as normal. If the druid cause the line of wind to blow to the left, she could potentially be affected by the wind.

\cf{Drd}{Nature Sense (Ex)} A druid gains a \plus2 competence bonus on Knowledge (nature) and Survival checks. In addition, she can make those checks as if she were trained.

\cf{Drd}{Wild Speech (Su)} One of the first lessons a druid learns is how to commune with natural creatures. A druid can speak with animals a number of times per day equal to half her druid level \add her Charisma (minimum 1). Each time she uses this ability, she chooses a kind of animal, such as owl or wolf. She can then speak to and understand animals of that type for a number of minutes equal to her druid level.

This ability doesn't make the animals any more friendly or cooperative than normal. Furthermore, wary and cunning animals are likely to be terse and evasive, while the more stupid ones make inane comments. If an animal is friendly toward the druid, she may be able to convince it to do some favor or service.

\cf{Drd}{Woodland Stride (Ex)} Starting at 2nd level, a druid may move through any sort of undergrowth (such as natural thorns, briars, overgrown areas, and similar terrain) at her normal speed and without taking damage or suffering any other impairment. However, thorns, briars, and overgrown areas that have been magically manipulated to impede motion still affect her.

 \cf{Drd}{Wild Aspect (Su)} At 3rd level, a druid gains the ability to embody an aspect of an animal. Embodying a wild aspect is a standard action and doesn't provoke an attack of opportunity. Wild aspects last for 10 minutes per druid level, or until the druid dismisses the effect, unless otherwise stated. The druid can use this ability a number of times per day equal to half her druid class level \add her Constitution.
\par When a druid embodies a wild aspect, she gains all abilities from an animal that she knows. For example, a druid devoted to bulls, who chose the Lope and Gore aspects, would gain both abilities with a single use of wild aspect. A druid devoted do both bulls and eagles, who chose the Lesser Flight aspect from the eagle and the Gore aspect from the bull, would have to use two uses of wild aspect to use both abilities.
\par The descriptions below describe the effects of the aspect. With many aspects, the druid's appearance also changes to match the aspect, but this is not described. Different druids change in different ways. For example, one druid might gain unusually large eyes when embodying the low-light vision aspect, while another might change her irises into slits, like a cat, when embodying the same aspect. The changes made are up to the druid, but cannot be used to gain an additional substantive benefit beyond the effects given in the description of the aspect.
\par The druid may choose any wild aspect, provided that she has the minimum druid level indicated by the aspect. At 5th level, she learns how to take on a new wild aspect. Each wild aspect also grants an ability based on the number of abilities the druid has from that aspect. These bonuses are only granted when the druid has taken on that wild aspect.

\subcf{Animal}
While embodying the animal aspect, the druid gains an enhancement bonus to any attribute score of her choice equal to the number of animal abilities she possesses.
\begin{wildaspect}
\wilditem Low-Light Vision: The druid gains low-light vision.
\wilditem Scent: The druid gains the scent ability.
\wilditem Natural Attunement: The druid gains a \plus4 enhancement bonus to Creature Handling, Ride, and Survival checks.
\wilditem Animal Affinity: When the druid first acquires this ability, she may choose any wild aspect for which she qualifies. She may learn it and treat it as if it were an animal aspect.
\end{wildaspect}

\subcf{Ape}
While embodying the ape aspect, the druid gains an enhancement bonus to Strength equal to the number of ape abilities she possesses.
\begin{wildaspect}
\wilditem Climb: The druid gains a climb speed equal to her base land speed.
\wilditem Claws: The druid gains two claw attacks which can be used as light natural weapons. A Medium druid deals d6 damage with each claw.
\wilditem Rend: If the druid hits with both claw attacks, she latches on to her opponent's body and tears the flesh. This attack deals damage appropriate to a scimitar appropriate for the druid's size.
\wilditem Improved Grab: When the druid hits a foe with an unarmed strike or natural attack, she may attempt to grapple her foe as an immediate action without provoking an attack of opportunity.
\end{wildaspect}

\subcf{Bear}
While embodying the bear aspect, the druid gains an enhancement bonus to Constitution equal to the number of bear abilities she possesses.
\begin{wildaspect}
\wilditem Bite: The druid gains a bite attack which can be used as a heavy natural weapon. A Medium druid deals d8 damage with the bite.
\wilditem Claws: The druid gains two claw attacks which can be used as light natural weapons. A Medium druid deals d6 damage with each claw.
\wilditem
\wilditem Improved Grab: When the druid hits a foe with an unarmed strike or natural attack, she may attempt to grapple her foe as an immediate action without provoking an attack of opportunity.
\end{wildaspect}

\subcf{Bull}
While embodying the bull aspect, the druid gains an enhancement bonus to Strength equal to the number of bull abilities she possesses.
\begin{wildaspect}
\wilditem Lope: The druid gains the ability to move on all four legs. When doing so, she gains a \plus20 foot competence bonus to her speed, but she cannot use her hands for anything except moving. When not moving on four legs, her ability to use her hands is unchanged.
\wilditem Gore: The druid gains a gore attack which can be used as a heavy natural weapon. A Medium druid deals d8 damage with a gore.
\wilditem Rush: When the druid hits a foe with an unarmed strike or natural attack on a charge, she may attempt to bull rush her foe as an immediate action without provoking an attack of opportunity.
\wilditem
\end{wildaspect}

\subcf{Cat}
While embodying the cat aspect, the druid gains an enhancement bonus to Dexterity equal to the number of cat abilities she possesses.
\begin{wildaspect}
\wilditem Lope: The druid gains the ability to move on all four legs. When doing so, she gains a \plus20 foot competence bonus to her speed, but she cannot use her hands for anything except moving. When not moving on four legs, her ability to use her hands is unchanged.
\wilditem Bite: The druid gains a bite attack which can be used as a heavy natural weapon. A Medium druid deals d8 damage with a bite.
\wilditem Stealth: The druid gains a \plus4 enhancement bonus to Stealth checks.
\wilditem Pounce: During the first round of combat, the druid can make a full attack after charging. She gains no bonus to attack rolls from charging, but still takes the normal penalty to AC if applicable.
\end{wildaspect}

\subcf{Eagle}
While embodying the eagle aspect, the druid gains an enhancement bonus to Charisma equal to the number of eagle abilities she possesses.
\begin{wildaspect}
\wilditem Wings, partial: The druid gains a glide speed equal to her base land speed. While gliding, she cannot use her hands for anything except moving.
\wilditem Talons: The druid gains talons which can be used as a heavy natural weapon. A Medium druid deals d8 damage with her talons.
\wilditem Dive: When the druid hits a foe with an unarmed strike or natural attack on a charge while gliding or flying down, she deals double damage. If she can make multiple attacks on the charge, this effect only applies to the first attack.
\wilditem Wings, full: The druid gains a fly speed equal to her base land speed with average maneuverability. While flying she cannot use her hands for anything except moving. She can only fly for a number of rounds equal 3 \add half her Constitution. After that limit is reached, she must rest for 5 minutes to recuperate.
\end{wildaspect}

\subcf{Fox}
While embodying the fox aspect, the druid gains an enhancement bonus to Intelligence equal to the number of fox abilities she possesses.
\begin{wildaspect}
\wilditem Lope: The druid gains the ability to move on all four legs. When doing so, she gains a \plus20 foot competence bonus to her speed, but she cannot use her hands for anything except moving. When not moving on four legs, her ability to use her hands is unchanged.
\wilditem Bite: The druid gains a bite attack which can be used as a primary natural weapon. A Medium druid deals d8 damage with a bite.
\wilditem Stealth: The druid gains a \plus4 enhancement bonus on Stealth checks.
\wilditem
\end{wildaspect}

\subcf{Owl}
For each owl ability she possesses, the druid gains a \plus1 enhancement bonus to Wisdom  while embodying the owl aspect.
\begin{wildaspect}
\wilditem Wings, partial: The druid gains a glide speed equal to her base land speed. While gliding, she cannot use her hands for anything except moving.
\wilditem Talons: The druid gains talons which can be used as a heavy natural weapon. A Medium druid deals d8 damage with her talons.
\wilditem Senses: The druid gains a \plus4 enhancement bonus on Perception checks.
\wilditem Wings, full: The druid gains a fly speed equal to her base land speed with average maneuverability. While flying, she cannot use her hands for anything except moving. She can fly for a number of rounds equal to 3 \add half her Constitution. After that, she must rest for 5 minutes to recuperate.
\end{wildaspect}

\subcf{Serpent}
While embodying the serpent aspect, the druid gains an enhancement bonus to grapple attacks and saving throw DCs with any poison-based ability she uses equal to the number of serpent abilities she possesses.
\begin{wildaspect}
\wilditem Slither: The druid gains a climb speed equal to half her base land speed. She does not need to use her hands to climb in this way.
\wilditem Bite: The druid gains a bite attack which can be used as a heavy natural weapon. A Medium druid deals d8 damage with a bite.
\wilditem Constrict: After making a successful grapple attack to grapple or damage her foe, the druid may constrict her foe as an immediate action. A Medium druid deals d8 \add 1-1/2 her Strength when constricting.
\wilditem Venom: The druid's natural attacks, and any weapons she wields, become coated in poison. The poison deals initial and secondary damage of 1d4 Constitution damage. A Fortitude save (DC 10 \add 1/2 the druid's level \add the druid's Con) negates the damage as normal for poison. This ability lasts for one round per druid level.
\end{wildaspect}

\subcf{Wolf}
While embodying the wolf aspect, the druid gains an circumstance bonus to weapon damage against overwhelmed foes and an enhancement bonus to trip attacks equal to the number of wolf abilities she possesses.
\begin{wildaspect}
\wilditem Lope: The druid gains the ability to move on all four legs. When doing so, she gains a \plus20 foot competence bonus to her speed, but she cannot use her hands for anything except moving. When not moving on four legs, her ability to use her hands is unchanged.
\wilditem Bite: The druid gains a bite attack which can be used as a heavy natural weapon. A Medium druid deals d8 damage with a bite.
\wilditem Trip: When the druid hits a foe with an unarmed strike or natural attack, she may attempt to trip her foe as an immediate action without provoking an attack of opportunity.
\wilditem Wolfpack: Foes overwhelmed by the druid increase their overwhelm penalties by 1.
\end{wildaspect}

\cf{Drd}{Venom Immunity (Ex)} At 4th level, a druid gains immunity to all poisons.

\cd{Drd}{Wild Speech (Plants) (Ex)} At 6th level, a druid can also converse with plants and plant creatures using her wild speech ability. A regular plant's sense of its surroundings is limited, so it won't be able to give (or recognize) detailed descriptions of creatures or answer questions about events outside its immediate vicinity.

\cf{Drd}{Improved Wild Speech (Su)} At 8th level, anything the druid speaks with with using her wild speech ability must make a Will save to avoid being charmed, as the \spell{charm person} spell, by the druid. The effect lasts for the duration of the conversation, and for 1 hour thereafter. This ability is not mind-affecting, and can affect creatures or even objects of any kind that the druid can converse with. Objects are always considered to fail their Will save.  The druid can choose not to exert this influence.

\cf{Drd}{Improved Natural Casting (Ex)} At 10th level, a druid can cause area spells to originate from up to \rngclose range away from her, as the natural casting ability.

\cf{Drd}{Greater Wild Aspect (Su)} At 11th level, a druid gains the ability to assume aspects of the natural world, including the elements, in addition to those of animals. This ability functions as wild aspect, and assuming a greater wild aspect consumes a use of wild aspect, but the druid may choose from a different list of abilities. Unlike with wild aspects, greater wild aspects must be learned in order within an aspect; a druid cannot gain the air mantle aspect unless she has the endless air aspect. A druid can suppress or resume any greater wild aspect ability as a swift action.

Whenever the druid learns a greater wild aspect, she may choose to learn a wild aspect instead.
\subcf{Air}
\begin{greaterwildaspect}
\wilditem Profusion of Air: The druid constantly exudes good, clean air. She can breathe in any environment, and is immune to \spell{sickening cloud} and similar effects. In addition, she may use her wild speech ability to speak with any natural air.
\wilditem Air Mantle: The druid is surrounded by a mantle of air. Thrown and projectile weapons have a 50\% chance to miss her while this effect is active. Unusually large weapons, such as a giant's boulders, may suffer a decreased miss chance as appropriate to their size.
\wilditemplus Flight: The druid gains a fly speed equal to her land speed, with good maneuverability. She may remain flying for up to 5 rounds at a time. After that, she must land for 1 round before she can fly again.
\end{greaterwildaspect}

\subcf{Earth}
\begin{greaterwildaspect}
\wilditem Earthen Profusion: The druid constantly exudes fresh, solid earth wherever she steps. She gains a \plus4 enhancement bonus to Maneuver Class and Fortitude saves against effects that would move her. In addition, she may use her wild speech ability to speak with any natural earth or stone.
\wilditem Earth Mantle: The druid is surrounded by a mantle of earth. She gains a \plus1 enhancement bonus to natural armor class per four druid levels and damage reduction 5/adamantine.
\wilditemplus Earth Glide: The druid gains the earth glide ability, as an earth elemental. She may remain partly within the earth while fighting, granting her cover at no penalty to her own actions.
\end{greaterwildaspect}

\subcf{Fire}
\begin{greaterwildaspect}
\wilditem Flaming Profusion: Wherever the druid moves, she leaves a path of burning flame behind her that lasts until the end of her next turn. A creature who crosses the path takes 1d6 points of fire damage per two druid levels. It can make a Reflex save to halve the damage, with a DC equal to 10 \add the druid's level \add the druid's Charisma. In addition, the druid can her her wild speech ability to speak with any natural fire.
\wilditem Fire Mantle: The druid is surrounded by a mantle of fire. This functions as a warm \spell{fire shield} spell, with a caster level equal to the druid's level.
\wilditemplus Immolation: The druid gains the fire subtype, making her immune to fire but giving her a 50\% vulnerability to cold damage. Whenever she deals fire damage to a creature, she ignites the creature for 5 rounds. An ignited creature takes a \minus2 penalty to attack rolls, saving throws, checks, and AC, and takes d6 damage per round from the fire. If the creature takes a full-round action, it can attempt a Reflex save to put out the fire with a DC of 10 \add the druid's level \add the druid's Charisma. A creature hit by the druid's fire multiple times is not ignited multiple times; only the most recent effect is used.
\end{greaterwildaspect}

\subcf{Plant}
\begin{greaterwildaspect}
\wilditem Profusion of Life: Wherever the druid moves, she leaves a path of small, living plants that entangle foes until the end of her next turn. A creature crossing the path must make a Reflex save with a DC equal to 10 \add the druid's level \add the druid's Charisma or be entangled by the plants and unable to complete the movement. The plants appear on any surface, and will continue to grow if they can survive, though they may die quickly if they appear on inhospitable terrain.
\wilditem Plant Body: The druid's body takes on plantlike characteristics. She gains a \plus1 enhancement bonus to natural armor class per four druid levels and has a 50\% chance to ignore critical hits and sneak attacks.
\wilditemplus Rejuvenation: The druid gains fast healing 5 as long as she remains in sunlight or touches a plant of her size or larger.
\end{greaterwildaspect}

\subcf{Sun}
\begin{greaterwildaspect}
\wilditem Profusion of Light: The druid constantly radiates light in a \arealarge radius. This is treated as true sunlight, not ordinary magical light. Creatures and objects vulnerable to sunlight must make a Fortitude save every round to resist the effect on themselves, with a DC equal to 10 \add the druid's level \add the druid's Charisma.
\wilditem Mantle of Light: The druid glows so brightly that she becomes hard to look at. She gains concealment against all attacks, and any creature attacking her from within the radius of her profusion of light is dazzled for 5 rounds after the attack (no save). She cannot use this ability while suppressing her profusion of light ability, and she cannot use this concealment to hide.
\wilditemplus Piercing Radiance: The druid's illumination radius with her profusion of light ability increases to 100 feet. All visual illusions and shadow effects within the radius are suppressed except those that the druid chooses to allow. Any creature within the radius who attacks the druid is blinded for 1 round after the attack. A successful Fortitude save with a DC of 10 \add the druid's level \add the druid's Charisma prevents the creature from being blinded for the next round.
\end{greaterwildaspect}

\subcf{Water}
\begin{greaterwildaspect}
\wilditem Aqueous Profusion: Wherever the druid moves, she leaves a path of animated water that can grab creatures and cause them to trip. A creature crossing the path must make a Reflex save with a DC equal to 10 \add the druid's level \add the druid's Charisma or fall prone and be unable to complete the movement. In addition, the druid can speak with natural water using her wild speech ability.
\wilditem Watery Mantle: The druid becomes surrounded by an animate mantle of water that reaches out to deflect incoming blows. She gains a \plus10 enhancement bonus to resist grapple attacks and a \plus5 shield bonus.
\wilditemplus Water's Flow: At any time during the duration of this aspect, the druid can transform into a rushing flow of water as a move action that does not provoke attacks of opportunity. As part of the action, she may move up to her movement speed in any direction that water could go. Her speed is halved when moving uphill and doubled when moving downhill. She does not provoke attacks of opportunity during this movement, and has physical damage reduction 10 while in this form. At the end of her movement, she regains her normal form.
\end{greaterwildaspect}

\cf{Drd}{A Thousand Faces (Su)} At 12th level, a druid gains the ability to change
her appearance at will, as if using the \spell{disguise self} spell. This affects the druid's body but not her possessions. It is not an illusory effect, but a minor physical alteration of the druid's appearance, within the limits described for the spell.

\cf{Drd}{Timeless Body (Ex)} After attaining 15th level, a druid no longer takes attribute score penalties for aging and cannot be magically aged. Any penalties she may have already incurred, however, remain in place. Bonuses still accrue, and the druid still dies of old age when her time is up.

\cf{Drd}{Greater Wild Speech (Ex)} At 16th level, the druid can use her wild speech ability to control creatures' actions. If the druid converses with a creature using her wild speech ability, she may spend a standard action and an additional wild speech use to dominate one creature she is speaking with, as the \spell{dominate person} spell. A successful Will save negates this effect. This ability is not mind-affecting, and can affect creatures of any kind that the druid can converse with.

\cf{Drd}{Totemic Aspect (Su)} At 18th level, the druid can choose any one wild aspect (but not greater aspect). She permanently gains the abilities of that aspect, as if she was constantly manifesting it. She may suppress or resume this effect as a swift action. If the druid has multiple abilites from that aspect, she may suppress or resume them each individually.

\cf{Drd}{Greater Natural Casting (Ex)} At 20th level, a druid may cause area spells to originate from any point within \rngmed range of her, as the natural casting ability.

\subsubsection{Ex-Druids}
A druid who ceases to revere nature, changes to a prohibited alignment, or teaches the Druidic language to a nondruid loses all spells and druid abilities (not including weapon, armor, and shield proficiencies). She cannot thereafter gain levels as a druid until she atones (see the \spell{atonement} spell description).

\subsection{Fighter}
\begin{dtable}
\lcaption{The Fighter}
\begin{tabularx}{\columnwidth}{>{\ccol}p{\levelcol} >{\ccol}p{\babcolgood} *{3}{>{\ccol}p{\savecol}} >{\lcol}X}
\thead{Level} & \thead{Base Attack Bonus} & \thead{Fort Save} & \thead{Ref Save} & \thead{Will Save} & \thead{Special} \\
1st & \plus1                         & \plus3 & \plus0 & \plus1 & Armor discipline \\
2nd & \plus2                         & \plus4 & \plus1 & \plus2 & Bonus feat \\
3rd & \plus3                         & \plus5 & \plus1 & \plus3 & Weapon discipline \\
4th & \plus4                         & \plus6 & \plus2 & \plus4 & Adaptive style feat \\
5th & \plus5                         & \plus7 & \plus2 & \plus4 & Combat discipline \\
6th & \plus6/\plus1                  & \plus8 & \plus3 & \plus5 & Bonus feat \\
7th & \plus7/\plus2                  & \plus9 & \plus3 & \plus6 & Improved armor discipline \\
8th & \plus8/\plus3                  & \plus10& \plus4 & \plus7 & Adaptive style feat \\
9th & \plus9/\plus4                  & \plus11& \plus4 & \plus7 & Improved weapon discipline\\
10th & \plus10/\plus5                & \plus12& \plus5 & \plus8 & Battlemaster, bonus feat\\
11th & \plus11/\plus6/\plus1         & \plus13 & \plus5 & \plus9 & Improved combat discipline\\
12th & \plus12/\plus7/\plus2         & \plus14 & \plus6 & \plus10& Adaptive style feat \\
13th & \plus13/\plus8/\plus3         & \plus15 & \plus6 & \plus10& Greater armor discipline \\
14th & \plus14/\plus9/\plus4         & \plus16 & \plus7 & \plus11& Bonus feat, improved adaptive style \\
15th & \plus15/\plus10/\plus5        & \plus17 & \plus7 & \plus12& Greater weapon discipline\\
16th & \plus16/\plus11/\plus6/\plus1 & \plus18 & \plus8 & \plus13& Adaptive style feat \\
17th & \plus17/\plus12/\plus7/\plus2 & \plus19 & \plus8 & \plus13& Greater combat discipline\\
18th & \plus18/\plus13/\plus8/\plus3 & \plus20 & \plus9 & \plus14& Bonus feat, improved battlemaster\\
19th & \plus19/\plus14/\plus9/\plus4 & \plus21 & \plus9 & \plus15& True discipline \\
20th & \plus20/\plus15/\plus10/\plus5& \plus22 & \plus10 & \plus16 & Adaptive style feat, greater adaptive style
\end{tabularx}
\end{dtable}

\sssecfake{Class Skills}
The fighter's class skills (and the key attribute for each skill) are Athletics (Str), Climb (Str), Swim (Str), Ride (Dex), and Intimidate (Cha).
\cf{Ftr}{Skill Points at 1st Level} 2.

\sssecfake{Class Features}
All of the following are class features of the fighter.
 \cf{Ftr}{Weapon and Armor Proficiency}  A fighter is proficient with simple weapons,  any four other weapon groups,  all armor (heavy, medium, and light) and shields (including tower shields).

\cf{Ftr}{Armor Discipline} At 1st level, a fighter's training grants him additional capability when using his armor. He may choose an armor category (light, medium, heavy, or shields), or he may choose to train equally with all kinds of armor and shields. Whether or not he chooses a specific armor category, he reduces his armor check penalty by 2 and reduces his arcane spell failure by 5\% when using his chosen armor. These benefits apply separately to armor and shields, if the fighter uses both and chose not to focus in a particular armor category. This effect cannot reduce those penalties below 0.
\par If the fighter chose a particular armor category, he gains a \plus1 competence bonus to his dodge modifier while using armor of that category.

\cf{Ftr}{Bonus Feat} At 2nd level, a fighter gets a bonus combat-oriented feat. This bonus feat must be drawn from the feats noted as combat feats on \tref{Combat Feats}. A fighter must still meet all prerequisites for a bonus feat, including attribute score and base attack bonus minimums. The fighter gains an additional bonus feat at 6th level and every four fighter levels thereafter (6th, 10th, 14th, and 18th).

These bonus feats are in addition to the feats that a character of any class gets from advancing levels. A fighter is not limited to the list of combat feats when choosing those feats.

\cf{Ftr}{Weapon Discipline} At 3rd level, a fighter's training grants him additional capability when using his weapons. He may choose a weapon group, or he may choose to train equally with all weapons. If he chooses a weapon group, he gains a \plus1 competence bonus to attack rolls with weapons from that group.
\par If he chooses not to focus on a specific group of weapons, he gains the ability to become proficient with any weapon group if he spends 8 hours training with a weapon from that group. He may only keep this proficiency with one weapon group at a time; if he trains with a new weapon group, he loses his proficiency in the previous group.

\cf{Ftr}{Adaptive Style Feats} At 4th level, a fighter gets a flexible bonus feat
which must be drawn from the list of combat feats. A fighter must still
meet all prerequisites for any bonus feat chosen. At the start of each day, the fighter may train for an hour. If he does so, he may choose to change his adaptive style feats to any other feats for which he meets the prerequisites.  The fighter gains an additional adaptive style feat at 8th level and every four fighter levels thereafter (8th, 12th, 16th, and 20th).

\par An adaptive style feat may be used normally as prerequisites for other feats or abilities.
However, if an adaptive style feat is used as a prerequisite, it cannot be changed until the fighter no longer needs to use it as a prerequisite, such as might happen if the fighter takes the feat as a normal feat or bonus feat.
\par In order to gain a new adaptive style feat, it must be reasonably possible to do training related to the new feat. For example, a fighter could not gain Weapon Focus in axes without at least one axe available to train with.

\cf{Ftr}{Combat Discipline} At 5th level, a fighter can use his superior training and focus to keep fighting in the face of debilitating effects. When a fighter is initially affected by one of the conditions listed on \trefnp{Combat Discipline Conditions}, he may use his combat discipline ability to instead suffer the mitigated condition one column to the right. This lasts until the end of the original condition or for one round per two fighter levels, at which point the he suffers the effects of the original condition unless he uses his combat discipline ability again.
\par Using combat discipline takes no action, and can be done at any time, even when it isn't the fighter's turn. A fighter may use this ability a number of times per day equal to 3 \add his Constitution.

\begin{dtable}
\lcaption{Combat Discipline Conditions}
\begin{tabularx}{\columnwidth}{*{4}{>{\lcol}X}}
\thead{Original Condition} & \thead{Mitigated Condition} & \thead{Mitigated Condition} & \thead{Mitigated Condition} \\
Panicked & Frightened & Shaken & None  \\
Petrified  & Paralyzed & Slowed & None \\
Stunned       & Dazed & Staggered & None \\
Blinded & Dazzled & None  & \x \\
Confused & Bewildered & None & \x \\
Exhausted & Fatigued & None & \x \\
Nauseated & Sickened & None & \x \\
Ability damage\fn{1} & None & \x & \x \\
Ability penalty\fn{1} & None & \x & \x \\
Entangled & None & \x & \x \\
Deafened & None & \x & \x \\
Fascinated & None & \x & \x \\
Ignited\fn{2} & None & \x & \x \\
Immobilized & None & \x & \x \\
Negative level\fn{3} & None & \x & \x \\
\end{tabularx}
1. Allows the fighter to mitigate up to half his fighter level in ability damage or penalties per use of combat discipline. \\
2. Mitigates the penalties, but does not prevent the fighter from taking d6 fire damage per round until the fire is put out. \\
3. Allows the fighter to ignore a single negative level per use of combat discipline.
\end{dtable}

\par A fighter cannot use this ability more than once against a single source. For example, if a fighter is exhausted by a \spell{ray of exhaustion} spell, he can use this ability to downgrade the exhaustion to fatigue, but he can't then expend a second use to negate the fatigue. The lesser condition that this ability imposes may be cured or removed normally, but doing so does not affect the resurgence of the condition the fighter was originally afflicted with. If a fighter uses this ability to mitigate or negate a condition which he must suffer as a sacrifice or cost to gain some benefit, he automatically forfeits the benefit he would have gained.

\cf{Ftr}{Bonus Feats} 

\cf{Ftr}{Improved Armor Discipline} At 7th level, a fighter's training in his chosen armor category (or with all armor categories) improves. He reduces the armor check penalty by 4 and decreases the arcane spell failure by 15\%. In addition, he treats his chosen armor (or armors) as if it were one encumbrance category lighter than it is. This does not stack with the effects of armor discipline.
\par This ability means heavy armor is treated as medium, medium armor is treated as light armor, and light armor is treated as being unarmored. Likewise, tower shields are treated as heavy shields (and no longer impose a \minus2 penalty to attack rolls), heavy shields are treated as light shields, and both light shields and bucklers are treated as being unarmored. This can remove the halving of the fighter's Dexterity bonus, if appropriate for the new encumbrance of the fighter's armor.
\par This allows the fighter to qualify for class features using the reduced armor encumbrance category. For example, a fighter 9 / wizard 2 who reduces his encumbrance in light armor could cast without any arcane spell failure in light armor.
\par A fighter who chose a specific armor category gains a \plus2 competence bonus to his dodge modifier while using armor of that category.

\cf{Ftr}{Improved Weapon Discipline} At 9th level, a fighter's training in his chosen weapons improves. He gains a \plus4 competence bonus to resist disarm and sunder attempts when using his chosen weapons. If he chose a specific weapon group, he gains a \plus2 competence bonus to attack rolls with weapons from that group. If he did not, he can apply all weapon group-specific feats he has to any weapon group that he trains with for 8 hours. He retains this benefit for one week after the training.

\cf{Ftr}{Battlemaster} At 10th level, a fighter can improve his allies' combat abilities. As a standard action, he may grant the use one of his combat feats to allies within 30 feet of him who can see and hear him. He can affect a number of allies equal to 1 \add his Intelligence (minimum 1). Affected allies must meet all prerequisites for the granted feat, except that they can ignore any feat prerequisites. The effect lasts for 5 rounds. The fighter can use this ability a number of times per day equal to 3 \add his Charisma.

\cf{Ftr}{Improved Combat Discipline} At 11th level, a fighter's ability to keep fighting despite negative influence improves. He can reduce conditions by two steps instead of one when using combat discipline. For example, a stunned fighter who used combat discipline would instead be staggered.
\par In addition, a fighter may use combat discipline to reduce any penalties he suffers to attribute scores, attack rolls, weapon damage rolls, skill checks, or ability checks that come from negative effects not listed on the combat discipline chart by 2. This cannot be used to reduce the effects of ability damage or drain.

\cf{Ftr}{Greater Armor Discipline} At 13th level, a fighter's training in his chosen armor becomes still greater. He reduces his armor check penalty by 6 and decreases his arcane spell failure by 30\% when using his chosen armor. In addition, he treats his chosen armor as if it were two encumbrance categories lighter than it actually is whenever doing so would be beneficial to him. This does not stack with the benefits of armor discipline or improved armor discipline.
\par A fighter who chose a specific armor category gains a \plus3 competence bonus to his dodge modifier while using armor of that category.

\cf{Ftr}{Improved Adaptive Style} At 14th level, a fighter's ability to adapt to situations improves. He need only spend 1 minute training to change a single adaptive style feat. He may continue training as he wishes, changing one adaptive style feat per minute.

\cf{Ftr}{Greater Weapon Discipline} At 15th level, a fighter's training in his chosen weapons becomes still greater. He increases the critical threat range and critical multiplier of his chosen weapons by 1. This increase applies after and stacks with any other effects that affect critical threat range or critical multiplier. Thus, a fighter using the Heartseeker combat style and wielding a longsword would have a critical threat range of 16-20 (x3), while a similar fighter would have a critical threat range of 18-20 (x5) with a heavy pick.

\cf{Ftr}{Greater Combat Discipline} At 17th level, a fighter's ability to keep fighting despite influence becomes still greater. When using combat discipline, he may ignore any condition listed on the combat discipline chart. In addition, he may use combat discipline to be dazed rather than suffer any non-damaging condition not listed on the chart.

\cf{Ftr}{Improved Battlemaster} At 18th level, the fighter can improve his allies' combat abilities more effectively. When using his battlemaster ability, he can grant two feats at once. In addition, he can use his battlemaster ability as a swift action.

\cf{Ftr}{True Discipline} At 19th level, a fighter's discipline in his chosen area is beyond equal. He must choose either weapon discipline, armor discipline, or combat discipline. Depending on which discipline he chooses, he gains a different bonus.
\subcf{True Weapon Discipline} The fighter can take 10 on the first attack he makes each round and automatically confirms all critical threats while using his chosen weapons.
\subcf{True Armor Discipline} The fighter no longer suffers armor check penalties or arcane spell failure with his chosen armor. In addition, he treats his chosen armor as if it were three encumbrance categories lighter than it actually is.
\par A fighter who chose a specific armor category gains a \plus4 competence bonus to his dodge modifier while using armor of that category.
\subcf{True Combat Discipline} The fighter can use combat discipline to be staggered instead of suffering any nondamaging negative effect with a duration.

\cf{Ftr}{Greater Adaptive Style} At 20th level, a fighter's ability to react to situations is unparalleled. He need only spend a full-round action training to exchange an adaptive style feat. He may continue training as he wishes, changing one adaptive style feat per round.

\subsection{Monk}
\begin{dtable}
\lcaption{The Monk}
\begin{tabularx}{\columnwidth}{>{\ccol}p{\levelcol} >{\ccol}p{\babcolavg} *{3}{>{\ccol}p{\savecol}} >{\lcol}X}
\thead{Level} & \thead{Base Attack Bonus} & \thead{Fort Save} & \thead{Ref Save} & \thead{Will Save} & \thead{Special} \\
1st & \plus1                    & \plus1 & \plus3 & \plus3    & Unarmed warrior, unfettered defense \\
2nd & \plus2                    & \plus2 & \plus4 & \plus4    & \Ki power, uncanny dodge \\
3rd & \plus3                    & \plus3 & \plus5 & \plus5    & Fast movement, evasion \\
4th & \plus4                    & \plus4 & \plus6 & \plus6    & \Ki strike, Perfect Health\\
5th & \plus5                    & \plus4 & \plus7 & \plus7    & \Ki power, still mind \\
6th & \plus6/\plus1                    & \plus5 & \plus8 & \plus8    & Improved uncanny dodge \\
7th & \plus7/\plus2                    & \plus6 & \plus9 & \plus9    & Bodily perfection \\
8th & \plus8/\plus3             & \plus7 & \plus10& \plus10   & \Ki power, perfect speech \\
9th & \plus9/\plus4             & \plus7 & \plus11& \plus11   & Improved evasion \\
10th & \plus10/\plus5            & \plus8 & \plus12& \plus12   & Greater uncanny dodge \\
11th & \plus11/\plus6/\plus1            & \plus9  & \plus13 & \plus13 & \Ki power \\
12th & \plus12/\plus7/\plus2            & \plus10 & \plus14& \plus14 & Perfect soul \\
13th & \plus13/\plus8/\plus3            & \plus10 & \plus15 & \plus15 & Improved bodily perfection \\
14th & \plus14/\plus9/\plus4            & \plus11 & \plus16& \plus16 & \Ki power \\
15th & \plus15/\plus10/\plus5    & \plus12 & \plus17 & \plus17 & Timeless \\
16th & \plus16/\plus11/\plus6/\plus1    & \plus13 & \plus18 & \plus18 & Perfect mind \\
17th & \plus17/\plus12/\plus7/\plus2    & \plus13 & \plus19 & \plus19 & \Ki power\\
18th & \plus18/\plus13/\plus8/\plus3    & \plus14 & \plus20 & \plus20 & Uncanny foresight \\
19th & \plus19/\plus14/\plus9/\plus4    & \plus15 & \plus21 & \plus21 & Greater bodily perfection \\
20th & \plus20/\plus15/\plus10/\plus5   & \plus16 & \plus22 & \plus22 & True perfection \\
\end{tabularx}
\end{dtable}

\cd{Alignment} Any lawful.

\cd{Hit Value} 6

\sssecfake{Class Skills}
The monk's class skills (and the key attribute for each skill) are Athletics (Str), Climb (Str), Swim (Str), Acrobatics (Dex), Escape Artist (Dex), Stealth (Dex), Heal (Wis), Perception (Wis), Survival (Wis), and Perform (Cha).
\cf{Mnk}{Skill Points at 1st Level} 4

\sssecfake{Class Features}
All of the following are class features of the monk.
\cf{Mnk}{Weapon and Armor Proficiency}
Monks are proficient with simple weapons, monk weapons, and any one other weapon group. Monks are not proficient with any armor or shields. When wearing armor, using a shield, or carrying a medium or heavy load, a monk loses the benefit of her unfettered defense, fast movement, and \ki abilities.

\cf{Mnk}{Unfettered Defense (Ex)} A monk knows how to react intuitively to avoid blows. When unarmored and unencumbered, the monk may add her Wisdom to her AC.

\par This bonus to AC applies even against touch attacks or when the monk is flat-footed. She loses this bonus when she is helpless, when she wears any armor, when she carries a shield, or when she carries a medium or heavy load.

\begin{comment}  %Made AC too high
 \cfnl{\Ki Ward (Ex)}\label{Mnk:Ki Ward (Ex)} When unarmored and unencumbered, a monk gains a \plus1 armor bonus to AC at 2nd level. This bonus increases by 1 for every two monk levels thereafter (\plus2 at 4th, \plus3 at 6th, etc.).

\par The monk loses this bonus when she is
immobilized or helpless, when she wears any armor, when she carries a shield, or when she carries a medium or heavy load.
\end{comment}

\cf{Mnk}{Improved Unarmed Strike} The monk gains Improved Unarmed Strike as a bonus feat. She is treated as armed even while unarmed, and her unarmed strikes can deal lethal damage if she desires.

\cf{Mnk}{Unarmed Warrior (Ex)} The monk's unarmed attacks are exceptionally deadly. She deals damage with her unarmed strikes as if she were two size categories larger (1d6 for a Medium creature, or 1d4 for a small creature).

A monk's attacks may be with either fist interchangeably or even from elbows, knees, and feet. This means that a monk may even make unarmed strikes with her hands full. A monk's unarmed strike can be treated as a manufactured weapon, natural weapon, or both for the purpose of spells and effects, whichever is more beneficial for the monk. This allows monks to make unarmed strikes as if they were fighting with two weapons at once (see \pcref{Two-Weapon Fighting}). Monks can also use gauntlets, including enchanted gauntlets. The damage dealt by gauntlets is the same as the damage dealt by the monk's normal unarmed strike.

\cf{Mnk}{Ki Power (Su)} At 2nd level, the monk gains the ability to use channel her \ki energy to enhance her abilities temporarily. She may use this ability a number of times per day equal to half her monk level \add her Wisdom, but no more than once per round. If a \ki power allows a saving throw, the DC is equal to 10 \add the monk's level \add her Wisdom. She may choose one of the abilities listed below.

\subcf{Flurry of Blows} When the monk makes a standard attack, she can make an additional attack at a \minus5 penalty.

\subcf{Stunning Fist} As part of an unarmed attack, the monk can strike a weak point to interfere with her foe's \ki. If the attack deals damage, the creature must make a Fortitude save to avoid being staggered for 1 round. If the foe is bloodied, it is stunned for 1 round instead.

\subcf{Surpass Limits} As a swift action, the monk can surpass the physical limitations of her body. Until the start of her next turn, she adds her Wisdom to checks based on Strength and Dexterity. This does not affect checks based on multiple attributes, such as initiative checks.

\cf{Mnk}{Uncanny Dodge (Ex)} Starting at 2nd level, a monk can react to danger before her senses would normally allow her to do so. She may apply her Dexterity and dodge modifier to her armor class while flat-footed.

If a monk already has uncanny dodge from a different class, she stacks those levels to determine whether she gains improved uncanny dodge (see below) instead.

\cf{Mnk}{Evasion (Ex)} At 3rd level or higher, if a monk makes a successful Reflex saving throw against an attack that normally deals half damage on a successful save, she instead takes no damage. Evasion can be used only if a monk is wearing light armor or no armor. A helpless monk does not gain the benefit of evasion.

\cf{Mnk}{Fast Movement (Ex)} At 3rd level, a monk gains a \plus10 foot competence bonus to her land speed while unencumbered.

\cf{Mnk}{Ki Strike (Su)} At 4th level, a monk's attacks are empowered with \ki. While unarmored and unencumbered, she treats all weapons she wields, including her unarmed strike, as if they were \plus1 weapons, gaining a \plus1 enhancement bonus to attack and damage. This also grants her a \plus1 enhancement bonus with maneuvers. At 7th level, and every three levels thereafter, the bonus granted by the monk's \ki strike ability improves by 1. 

\cf{Mnk}{Perfect Health} At 4th level, the monk gains the Perfect Health feat as a bonus feat, making her immune to diseases of all kinds. If she already has the feat, she gains another feat of her choice for which she qualifies.

\cfnl{Mnk}{Ki Power (Su)} At 5th level, the monk gains an additional \ki power ability. She adds the abilities below to the list of abilities she can choose.

\subcf{Speed Boost} As a swift action, the monk can force her muscles to strain beyond their normal limits. For 1 round, she gains a \plus20 foot competence bonus to her land speed, and a \plus2 competence bonus to her dodge modifier.

\subcf{Wholeness of Body} As a standard action, the monk can correct the flow of energy within her body. She heals 1d6 points of damage per two monk levels.

\cf{Mnk}{Still Mind (Ex)} A monk of 5th level or higher may add half her Wisdom to Will saves in place of half her Intelligence.

\cf{Mnk}{Improved Uncanny Dodge (Ex)} At 6th level and higher, a monk can no longer be overwhelmed as easily; she can react to multiple opponents as easily as she can react to a single attacker. The monk is always treated as being threatened by two fewer creatures than she actually is for the purpose of determining overwhelm penalties.
\par If a character already has uncanny dodge (see above) from a second class and gains improved uncanny dodge, the character stacks those levels to determine if she should gain greater uncanny dodge.

\cf{Mnk}{Bodily Perfection (Ex)} At 7th level, the monk gains a \plus1 competence bonus to her Strength, Dexterity, and Constitution.

\cf{Mnk}{Perfect Speech (Ex)} At 8th level, the monk gains the ability to speak with and understand the speech of any living creature. This grants her no special ability to speak to or understand creatures that do not speak, such as animals.

\cf{Mnk}{Ki Power (Su)} At 8th level, the monk gains an additional \ki power ability. She adds the abilities below to the list of abilities she can choose.

\subcf{Rapid Step} As a move action, the monk can move up to her speed without provoking attacks of opportunity.

\subcf{Slow Fall} As an immediate action, the monk can slow the rate at which she falls to 60 feet per round, which is too slow to hurt her if she hits the ground. She must be touching a solid object to use this ability. The effect lasts for 1 round.

\cf{Mnk}{Improved Evasion (Ex)} At 9th level, a monk's evasion ability improves. She still takes no damage on a successful Reflex saving throw against attacks, but henceforth she takes only half damage on a failed save. A helpless monk does not gain the benefit of improved evasion.

\cf{Mnk}{Greater Uncanny Dodge (Ex)} At 10th level and higher, a monk can no longer be overwhelmed, regardless of the number of foes surrounding her.

\cf{Mnk}{Ki Power (Su)} At 11th level, the monk gains an additional \ki power ability. She adds the abilities below to the list of abilities she can choose.

\subcf{Diamond Fists} As a swift action, the monk can empower her unarmed strike with incredible force. Until the start of her next turn, she treats her unarmed strike as if it was an adamantine weapons for the purpose of overcoming damage reduction.

\subcf{Flash Step} As a move action, the monk can slip between spaces, allowing her to teleport to anywhere she can see within 30 feet. If her line of effect is blocked, even by an invisible barrier, or if this would somehow place her inside a solid object, the ability fails. 

\cf{Mnk}{Perfect Soul (Ex)} At 12th level, a monk gains spell resistance. A creature with spell resistance may always make a saving throw when a spell is cast on it. The saving throw type is indicated by the spell. If it succeeds, the spell has no effect on it.

\cf{Mnk}{Improved Bodily Perfection} At 13th level, the bonus granted by the monk's bodily perfection ability increases to \plus2.

\cf{Mnk}{Ki Power (Su)} At 14th level, the monk gains an additional \ki power ability. She adds the abilities below to the list of abilities she can choose.

\subcf{Empty Step} As a swift action, the monk can step into the Ethereal Plane for 1 round, as the \spell{ethereal jaunt} spell.

\subcf{Quivering Palm} As a standard action, the monk can make a single unarmed strike. If the attack deals damage, the struck creature is sickened by the disruption of the \ki within in its body for 5 rounds. At any point during that time, the monk can will the struck target to die (a free action). If she does, and the creature is bloodied, it must make a Fortitude save. If it fails, it loses all its hit points and takes 9 critical damage.

\cf{Mnk}{Timeless (Ex)} Upon attaining 15th level, a monk no longer takes penalties to her attribute scores for aging, and cannot be magically aged. She also gains the benefits of being middle-aged if she did not already possess them, granting her a \plus1 inherent bonus to her Intelligence, Wisdom, and Charisma. Any aging penalties she has are removed. The monk still dies of old age when her time is up.

\cf{Mnk}{Perfect Mind (Ex)} At 16th level, the monk becomes immune to hostile mind-affecting effects.

\cf{Mnk}{Ki Power (Su)} At 17th level, the monk gains an additional \ki power ability. She adds the abilities below to the list of abilities she can choose.

\subcf{Moment of Perfection} As an immediate action, the monk can align herself with the universe to achieve a single moment of perfection. She can add her monk level as an enhancement bonus to any one attack roll, opposed skill or ability check, or saving throw, or to her AC against any one attack, as if calling upon the effect of a \spell{moment of prescience} spell with a caster level equal to her monk level. After using this ability, she must wait five minutes before she can use it again.

\subcf{Empty Body} As a move action, the monk can step into the Ethereal Plane for 5 rounds, as the \spell{ethereal jaunt} spell.

\cf{Mnk}{Uncanny Foresight (Su)} At 18th level, the monk gains the ability to react to situations without premeditation or thought. She is never surprised or flat-footed, and always acts in the first round of combat.

\cf{Mnk}{Greater Bodily Perfection} At 19th level, the bonus granted by the monk's bodily perfection ability increases to \plus3.

\cf{Mnk}{True Perfection} At 20th level, a monk becomes inhumanly perfect. If she rolls less than a 5 on any d20 roll, it is treated as a 5. In addition, she is treated as an outsider rather than as a humanoid for the purpose of spells and magical effects whenever doing so is advantageous to her.

\subsubsection{Ex-Monks}
A monk who becomes nonlawful cannot gain new levels as a monk, but retains all monk abilities.

\subsection{Paladin}
\begin{dtable*}
\lcaption{The Paladin}
\begin{tabularx}{\textwidth}{>{\ccol}p{\levelcol} >{\ccol}p{\babcolgood} *{3}{>{\ccol}p{\savecolpoof}} X *{4}{>{\ccol}p{\spellcolpoof}}}
& & & & & & \multicolumn{4}{c}{\thead{---{}---Spells per Day---{}---}} \\
\thead{Level} & \thead{Base Attack Bonus} & \thead{Fort Save} & \thead{Ref Save} & \thead{Will Save} & \thead{Special} & \thead{1st} & \thead{2nd} & \thead{3rd} & \thead{4th} \\
1st  & \plus1                        & \plus3  & \plus0 & \plus3 & Aura of good, discernment (evil), smite evil & \x & \x & \x & \x \\
2nd  & \plus2                        & \plus4  & \plus1 & \plus4 & Improved smite, lay on hands & \x & \x & \x & \x \\
3rd  & \plus3                        & \plus5  & \plus1 & \plus5 & Aura of courage, bulwark of defense, divine health & \x & \x & \x & \x \\ 
4th  & \plus4                        & \plus6  & \plus2 & \plus6 & Divine grace & 1 & \x & \x & \x \\
5th  & \plus5                        & \plus7  & \plus2 & \plus7 & Discernment (chaos), improved smite & 2 & \x & \x & \x \\
6th  & \plus6/\plus1                 & \plus8  & \plus3 & \plus8 & Aura of resolve, holy ward & 3 & \x & \x & \x \\
7th  & \plus7/\plus2                 & \plus9  & \plus3 & \plus9 & Smite chaos & 3 & \x & \x & \x \\
8th  & \plus8/\plus3                 & \plus10 & \plus4 & \plus10& Improved smite & 3 & 1 & \x & \x \\
9th  & \plus9/\plus4                 & \plus11 & \plus4 & \plus11& Aura of determination, improved bulwark of defense & 3 & 2 & \x & \x \\
10th & \plus10/\plus5                & \plus12 & \plus5 & \plus12& Discernment (lies), pass judgment & 3 & 3 & \x & \x \\
11th & \plus11/\plus6/\plus1         & \plus13 & \plus5 & \plus13& Improved smite & 4 & 3 & \x & \x \\
12th & \plus12/\plus7/\plus2         & \plus14 & \plus6 & \plus14& Aura of protection & 4 & 3 & 1 & \x \\
13th & \plus13/\plus8/\plus3         & \plus15 & \plus6 & \plus15& Forgiving smite & 4 & 3 & 2 & \x \\
14th & \plus14/\plus9/\plus4         & \plus16 & \plus7 & \plus16& Improved smite & 4 & 3 & 3 & \x \\
15th & \plus15/\plus10/\plus5        & \plus17 & \plus7 & \plus17& Aura of warding & 4 & 4 & 3 & \x \\
16th & \plus16/\plus11/\plus6/\plus1 & \plus18 & \plus8 & \plus18& Glory of the martyr & 4 & 4 & 3 & 1 \\
17th & \plus17/\plus12/\plus7/\plus2 & \plus19 & \plus8 & \plus19& Improved smite & 4 & 4 & 3 & 2 \\
18th & \plus18/\plus13/\plus8/\plus3 & \plus20 & \plus9 & \plus20& Greater aura of warding & 4 & 4 & 3 & 3 \\
19th & \plus19/\plus14/\plus9/\plus4 & \plus21 & \plus9 & \plus21& Martyr's retribution & 4 & 4 & 4 & 3 \\
20th & \plus20/\plus15/\plus10/\plus5& \plus22 &\plus10 &\plus22 & Greater smite, improved smite & 4 & 4 & 4 & 4 \\
\end{tabularx}
\end{dtable*}

\cd{Alignment} Lawful good.
\cd{Hit Value} 6

\sssecfake{Class Skills}
The paladin's class skills (and the key attribute for each skill) are Ride (Dex), Knowledge (local) (Int), Knowledge (religion) (Int), Heal (Wis), Sense Motive (Wis), Intimidate (Cha), and Persuasion (Cha).
\cf{Pal}{Skill Points at 1st Level} 2.

\sssecfake{Class Features}
All of the following are class features of the paladin.
 \cf{Pal}{Weapon and Armor Proficiency}  Paladins are proficient with
simple weapons,  any two other weapon groups,  all types of armor (heavy, medium, and light), and with  shields (including tower shields). A paladin is also proficient with the favored weapon group of her deity. If she does not follow a deity, she is proficient with any other weapon group of her choice.
\cf{Pal}{Aura of Good (Ex)} The power of a paladin's aura of good (see the \spell{detect good} spell) is equal to her paladin level.

\cf{Pal}{Discernment (Su)} A paladin can discern truths about creatures he sees as a standard action. This functions as the \spell{detect evil} spell. The paladin may use this ability a number of times per day equal to half the paladin's level \add the paladin's Wisdom (minimum 1).

At 5th level, she may simultaneously detect chaotic alignments, as the \spell{detect chaos} spell.

\cf{Pal}{Smite (Su)} As part of an attack action, a paladin may attempt to smite evil with one normal melee attack. She adds her Charisma (if positive) as a circumstance bonus to attack. If she hits, the paladin deals gains a circumstance bonus to damage equal to her paladin level and gains a special effect.  A paladin can smite evil a number of times per day equal to half the paladin's class level \add the paladin's Charisma, but may only do so once per round.

If the paladin smites a creature that is not evil, the smite attack deals no damage at all (not even normal weapon damage), but the use of the ability is still spent.

\cf{Pal}{Improved Smite (Su)} At 2nd level, and every three levels thereafter, the paladin can select one improved smiting ability. Each improved smiting ability adds an effect to the paladin's smite ability. Whenever the paladin smites, she chooses one improved smiting ability she has and adds its effect to her smite. The save DC for a paladin's improved smiting abilities is 10 \add the paladin's level \add the paladin's Charisma. She may select one of the following improved smiting effects.
\subcf{Blinding} The paladin's smite manifests as a bright light. A creature struck by the smite must make a Will save to avoid being blinded for one round per four paladin levels (minimum 1). Creatures vulnerable to light (such as vampires) take extra damage equal to twice the paladin's level.
\subcf{Resounding} The paladin's smite knocks his foes off their feet. A creature struck by the smite must make a Reflex save to avoid being pushed back five feet per four paladin levels and being knocked prone.
\subcf{Staggering} The paladin's smite hits with incredible force. A creature struck by the smite must make a Fortitude save to avoid being staggered for one round per four paladin levels (minimum 1).

\cf{Pal}{Lay On Hands (Su)} Beginning at 2nd level, a paladin can heal wounds (her own or those of others) with a touch. Each use heals 1d8 hit points per paladin level as a standard action. The paladin can lay on hands a number of times per day equal to 1 \add half her Charisma (minimum 1). Against undead creatures, this ability instead deals positive energy damage. A touch attack is required to hit unwilling targets, and a successful Will save halves the damage or healing received.

 \cf{Pal}{Aura of Courage (Su)} Beginning at 3rd level, a paladin is immune to fear (magical or otherwise). Each ally within 10 feet of her gains a \plus4 enhancement bonus on saving throws against fear effects. This ability functions while the paladin is conscious, but not if she is unconscious or dead.

\cf{Pal}{Divine Health (Ex)} At 3rd level, a paladin gains immunity to all diseases, including supernatural and magical diseases.

\cf{Pal}{Divine Grace (Su)} At 4th level, whenever a paladin makes a successful save against an attack that would normally have a partial effect or deal half damage on a successful save, she instead ignores that aspect of the attack.

\cf{Pal}{Spells} Beginning at 4th level, a paladin gains the ability to cast a small number of divine spells, which are drawn from the paladin spell list.

Paladins do not require somatic components to cast spells, even if the spell would normally require a somatic component. A paladin need only request the favor of her deity to invoke divine magic.

\par To learn or cast a spell, a paladin must have a Charisma score at least equal to half the spell's level. The Difficulty Class for a saving throw against a paladin's spell is 10 \add half the paladin's caster level \add the paladin's Charisma.

\par Like other spellcasters, a paladin can cast only a certain number of spells of each spell level per day. Her base daily spell allotment is given on \trefnp{The Paladin}.

A paladin learns and casts spells the way a cleric does, though she does not have access to any domain spells or granted powers, as a cleric does. A paladin may learn and cast any spell on the paladin spell list, provided that she can cast spells of that level.

A paladin's selection of spells is limited. A paladin begins play knowing no spells, but gains one or more new spells at certain levels, as indicated on \trefnum{Paladin Spells Known}.

\par At 5th level, a paladin can choose to learn one new spell in place of one she already knows. In effect, the paladin ``loses'' the old spell in exchange for the new one. The new spell's level must be the same as that of the spell being exchanged. A paladin may swap only a single spell at any given level, and must choose whether or not to swap the spell at the same time that she gains new spells known for the level.

\par Through 3rd level, a paladin has no caster level or magic level. At 4th level and higher, her caster level is equal to her paladin level / 2, and her magic level is equal to her paladin level.

\begin{dtable}
\lcaption{Paladin Spells Known}
\begin{tabularx}{\columnwidth}{X *{4}{>{\ccol}X}}
& \multicolumn{4}{c}{\thead{---{}---{}---{}---{}---Spell Level---{}---{}---{}---{}---}} \\
\thead{Level} & \thead{1st} & \thead{2nd} & \thead{3rd} & \thead{4th} \\
1st  & \x & \x & \x & \x \\
2nd  & \x & \x & \x & \x \\
3rd  & \x & \x & \x & \x \\
4th  & 1 & \x & \x & \x \\
5th  & 2 & \x & \x & \x \\
6th  & 2 & \x & \x & \x \\
7th  & 3 & \x & \x & \x \\
8th  & 3 & 1 & \x & \x \\
9th  & 3 & 2 & \x & \x \\
10th & 4 & 2 & \x & \x \\
11th & 4 & 2 & \x & \x \\
12th & 4 & 3 & 1 & \x \\
13th & 4 & 3 & 2 & \x \\
14th & 4 & 3 & 2 & \x \\
15th & 4 & 3 & 2 & \x \\
16th & 4 & 3 & 3 & 1 \\
17th & 4 & 3 & 3 & 2 \\
18th & 4 & 3 & 3 & 2 \\
19th & 4 & 3 & 3 & 2 \\
20th & 4 & 3 & 3 & 3 \\
\end{tabularx}
\end{dtable}

\cfnl{Pal}{Improved Smiting (Su)} At 5th level, the paladin gains an additional improved smiting ability, and adds the following improved smiting abilities to the list of those that can be selected.
\subcf{Holy} The paladin's smite is filled with exceptional divine energy. The attack ignores all damage reduction of evil creatures. Undead and evil outsiders take extra damage equal to twice the paladin's level.
\subcf{Penetrating} The paladin's smite punches through her enemies' defenses. The attack ignores a number of points of damage reduction equal to the paladin's level, regardless of the type of damage reduction.
\subcf{Seeking} The paladin's smite is uncannily guided to its target. The attack ignores any miss chance, though the weapon must still be able to strike the target.

\cf{Pal}{Aura of Resolve (Su)} At 6th level, the paladin becomes immune to charm effects. Each ally within 10 feet of her gains a \plus4 enhancement bonus on saving throws against charm effects.

\cf{Pal}{Holy Ward (Sp)} A paladin is first and foremost a defender of her allies, and relies upon \spell{shield other} to defend her allies against harm from threats she cannot block with her skill at arms. A paladin of 6th level or higher who spends at least 10 points of healing from her lay on hands ability in a single round can bestow a \spell{shield other} effect on the healed ally, using her paladin level as her caster level.

\cf{Pal}{Smite Chaos (Su)} At 7th level, the paladin gains the ability to smite chaotic creatures as well as evil creatures with her smite ability, but she must choose which to smite before making the attack. If the paladin attempts to smite a chaotic creature, and that creature is not chaotic, the smite attack deals no damage at all (not even normal weapon damage), but the use of the ability is still spent.

\cfnl{Pal}{Improved Smiting (Su)} At 8th level, the paladin gains an additional improved smiting ability, and adds the following improved smiting abilities to the list of those that can be selected.
\subcf{Dazing} The paladin's smite shatters her foe's ability to concentrate. A creature struck by the smite must make a Fortitude save to avoid being dazed for one round.
\subcf{Impeding} The paladin's smite traps her foe in place, unable to escape her wrath. A creature struck by the smite must make a Reflex save to avoid having all its movement speeds reduced to 5 feet for one round.
\subcf{Coercing} The paladin's smite forces her foe to join the cause of righteousness, if only for a moment. A creature struck by the smite must make a Will save or else be affected by a \spell{suggestion}, as the spell, of the paladin's choice. The effect lasts for one round.

\cf{Pal}{Aura of Determination (Su)} At 9th level, the paladin becomes immune to compulsion effects. Each ally within 10 feet of her gains a \plus4 enhancement bonus on saving throws against compulsion effects.

\cf{Pal}{Discernment (Su)} At 10th level, the paladin may can also identify lies in the range of his discernment ability, as the \spell{discern lies} spell.

\cf{Pal}{Pass Judgment (Su)} At 10th level, the paladin gains the ability to freely pass judgment on those she deems unworthy. As a swift action, she may pass judgment on a creature within 100 feet of her once per day. The creature is treated as being evil, chaotic, or both, at the paladin's discretion, in place of their original alignment. This effect lasts for one day per paladin level, or until the paladin changes her mind about the subject (a free action). This does not change the creature's actions or behavior, but the creature is subject to smite evil or smite chaos, would register as evil under the inspection of a \spell{detect evil} spell, and so on.

No saving throw is allowed against this effect, and it cannot be dispelled, but a \spell{remove curse}, \spell{miracle}, or \spell{wish} spell can remove it. The paladin can use this ability an additional time per day at 13th level and every odd level thereafter. A paladin should be careful when using this ability, as persecution of the innocent can lead overzealous paladins to fall.

\cfnl{Pal}{Improved Smiting (Su)} At 11th level, the paladin gains an additional improved smiting ability, and adds the following improved smiting ability to the list of those that can be selected.
\subcf{Axiomatic} The paladin's smite is filled with exceptionally lawful divine energy. The attack ignores all damage reduction of chaotic creatures. Aberrations and chaotic outsiders take extra damage equal to twice the paladin's level.

\cf{Pal}{Aura of Protection (Su)} At 12th level, the paladin continuously radiates a \spell{magic circle against evil}, as the spell.

\cf{Pal}{Forgiving Smite (Su)} At 13th level, if a paladin smites a creature who is not evil or chaotic, the smite attempt is not wasted.

\cf{Pal}{Glory of the Martyr (Su)} At 14th level, if the paladin dies while fighting evil or protecting her allies, her fallen body erupts in a burst of positive energy, granting all her allies within 100 feet the benefit of a \spell{heal} spell.

\cfnl{Pal}{Improved Smiting (Su)} At 14th level, the paladin gains an additional improved smiting ability, and adds the following improved smiting ability to the list of those that can be selected.
\subcf{Dispelling} The paladin's smite strips away her foe's magical protections. A creature struck by the smite is subject to a targeted \spell{dispel magic} with a bonus equal to the paladin's level.

\cf{Pal}{Aura of Warding (Su)} At 15th level, the paladin continuously radiates a \spell{lesser globe of invulnerability}, as the spell. The effect travels with the paladin.

\cf{Pal}{Mighty Aura (Su)} At 16th level, the radius of a paladin's auras expands to 20 feet.

\cfnl{Pal}{Improved Smiting (Su)} At 17th level, the paladin gains an additional improved smiting ability, and adds the following improved smiting ability to the list of those that can be selected.
\subcf{Brilliant} The paladin's smite cannot be turned aside by mortal defenses. The smite is made against the enemy's touch armor class.

\cf{Pal}{Greater Aura of Warding (Su)} At 18th level, the paladin radiates a \spell{globe of invulnerability}, as the spell, instead of a \spell{lesser globe of invulnerability}. The effect continues to travel with the paladin.

\cf{Pal}{Martyr's Retribution (Su)} At 19th level, if the paladin dies while fighting evil or protecting her allies, she can choose to make the explosion of positive energy from her glory of the martyr ability painful to her foes. If she does, her body is almost completely consumed by holy power, preventing her from being raised with \spell{raise dead} and similar effects that require a body. This has two effects. First, a \spell{sunburst} spell immediately takes effect over the area where the paladin fell. Second, a \spell{storm of vengeance} spell begins to take effect, centered on the same area. The spell lasts for 10 rounds, and the lightning strikes focus on the paladin's enemies. Both of these effects harm only the paladin's foes, and do no damage to her allies. However, her allies' vision is still impeded by the \spell{storm of vengeance}.

\cf{Pal}{Greater Smite (Su)} At 20th level, the paladin can apply two improved smiting abilities to every smite attack she makes.

\subsection{Ranger}

\begin{dtable}
\lcaption{The Ranger}
\begin{tabularx}{\columnwidth}{>{\ccol}p{\levelcol} >{\ccol}p{\babcolgood} *{3}{>{\ccol}p{\savecol}} >{\lcol}X}
\thead{Level} & \thead{Base Attack Bonus} & \thead{Fort Save} & \thead{Ref Save} & \thead{Will Save} & \thead{Special} \\
1st  & \plus1                        & \plus3  & \plus1  & \plus1 & Quarry \plus2, Track, wild speech \\
2nd  & \plus2                        & \plus4  & \plus2  & \plus2 & Danger sense, favored terrain \\
3rd  & \plus3                        & \plus5  & \plus3  & \plus3 & Ranger lore \\
4th  & \plus4                        & \plus6  & \plus4  & \plus4 & Low-light vision, tracking expert \\
5th  & \plus5                        & \plus7  & \plus4  & \plus4 & Free stride, tenacious hunter \\
6th  & \plus6/\plus1                 & \plus8  & \plus5  & \plus5 & Favored terrain, ranger lore \\
7th  & \plus7/\plus2                 & \plus9  & \plus6  & \plus6 & Guide \\
8th  & \plus8/\plus3                 & \plus10 & \plus7  & \plus7 & Darkvision, quarry \plus3  \\
9th  & \plus9/\plus4                 & \plus11 & \plus7  & \plus7 & Ranger lore \\
10th & \plus10/\plus5                & \plus12 & \plus8  & \plus8 & Favored terrain (planar) \\
11th & \plus11/\plus6/\plus1         & \plus13 & \plus9  & \plus9 & Hidden hunter\\
12th & \plus12/\plus7/\plus2         & \plus14 & \plus10 & \plus10& Blindsense, advanced lore  \\
13th & \plus13/\plus8/\plus3         & \plus15 & \plus10 & \plus10& Terrain mastery  \\
14th & \plus14/\plus9/\plus4         & \plus16 & \plus11 & \plus11& Favored terrain (planar), quarry \plus4 \\
15th & \plus15/\plus10/\plus5        & \plus17 & \plus12 & \plus12& Advanced ranger lore \\
16th & \plus16/\plus11/\plus6/\plus1 & \plus18 & \plus13 & \plus13& Blindsight  \\
17th & \plus17/\plus12/\plus7/\plus2 & \plus19 & \plus13 & \plus14& Terrain mastery, unerring hunter \\
18th & \plus18/\plus13/\plus8/\plus3 & \plus20 & \plus14 & \plus14& Advanced ranger lore, favored terrain (planar)  \\
19th & \plus19/\plus14/\plus9/\plus4 & \plus21 & \plus15 & \plus15& Perfect stride  \\
20th & \plus20/\plus15/\plus10/\plus5& \plus22 & \plus16 & \plus16& Quarry \plus5, truesight
\end{tabularx}
\end{dtable}

\cd{Alignment} Any.

\cd{Hit Value} 6.

\sssecfake{Class Skills}
The ranger's class skills (and the key attribute for each skill) are Athletics (Str), Climb (Str), Swim (Str), Acrobatics (Dex), Escape Artist (Dex), Ride (Dex), Stealth (Dex), Knowledge (dungeoneering) (Int), Knowledge (geography) (Int), Knowledge (nature) (Int), Heal (Wis), Perception (Wis), Survival (Wis), and Creature Handling (Cha).
\cf{Rgr}{Skill Points at 1st Level} 8.

\sssecfake{Class Features}

All of the following are class features of the ranger.

\cf{Rgr}{Weapon and Armor Proficiency}  A ranger is proficient with simple weapons, any two weapon groups, light and medium armor, and shields (except tower shields). He is also proficient with his choice of bows, crossbows, or thrown weapons.

\cf{Rgr}{Quarry (Ex)} A ranger is a deadly hunter. As a swift action, a ranger may designate any foe he sees as his quarry. A ranger gains a \plus2 competence bonus to attack rolls, Perception checks, and Survival checks against his quarry. However, while a ranger is pursuing a quarry, he takes a \minus2 penalty on the same rolls against any target other than his quarry. A ranger may give up pursuing a quarry at any time. He may not have more than one quarry at once; if he designates a new quarry, the old target is no longer considered his quarry. If the ranger does not see his quarry for more than a week, it is no longer considered his quarry.

\par A ranger can designate a quarry a number of times per day equal to 1 \add half his Wisdom, to a maximum number of uses per day equal to his ranger class level. The ranger's quarry bonus improves to \plus3 at 8th level, to \plus4 at 14th level, and finally to \plus5 at 20th level.

\cf{Rgr}{Track} A ranger gains Track as a bonus feat.

\cf{Rgr}{Wild Speech (Su)} A ranger has the ability to communicate with animals. This ability functions like the druid ability of the same name. A ranger can use this ability a number of times per day equal to half his ranger level \add his Charisma.

\cf{Rgr}{Danger Sense (Ex)} Starting at 2nd level, a ranger has an intuitive sense that alerts him to danger, giving him a \plus2 competence bonus to initiative checks. This bonus increases by 1 at 5th level and every 3 levels thereafter.
\par If a character has danger sense from a multiple classes, the character stacks those levels to determine his bonus from danger sense.

\cf{Rgr}{Favored Terrain (Ex)} At 2nd level, a ranger becomes particularly attuned to certain kinds of terrain. He chooses one kind of terrain to select as a favored terrain from the list below. Usually, rangers favor their home terrain, but a ranger may choose any kind of terrain that he has personally experienced at least once. At 6th level, and every four levels thereafter, the ranger gains an additional favored terrain.
\par While in a favored terrain, a ranger gains a \plus2 competence bonus to Perception, Stealth, and Survival checks. If he desires, he may leave no trace of his passage, causing attempts to track him to take a \minus20 penalty. In addition, his experience with his favored terrain grants the ranger a single ability, regardless of whether he is currently in that terrain or not. The options for favored terrains are listed below.
\subcf{Aquatic} The ranger gains a swim speed equal to his base land speed. If he already has a swim speed, he gains a \plus10 competence bonus to his swim speed.
\subcf{Cold} The ranger gains cold damage reduction 5.
\subcf{Desert} The ranger becomes immune to heat effects and exhaustion. Anything that would make him exhausted makes him fatigued instead.
\subcf{Forest} The ranger gains Skill Focus (Stealth) as a bonus feat.
\subcf{Mountains} The ranger gains a climb speed equal to his base land speed. If he already has a climb speed, he gains a \plus10 competence bonus to his climb speed.
\subcf{Plains} The ranger gains Skill Focus (Perception) as a bonus feat.
\subcf{Swamp} 
\subcf{Underground} The ranger gains Blind-Fight as a bonus feat.
\subcf{Urban} The ranger gains Skill Focus (Persuasion) as a bonus feat.

\cf{Rgr}{Ranger Lore} At 3rd level, a ranger can choose an additional ability drawn from ancient ranger lore. All ranger lore abilities are extraordinary abilities unless specified otherwise. He may choose from any of the following options. At 6th level, and every 3 levels thereafter, the ranger gains an additional ranger lore ability.

\subcf{Combat Style} The ranger is skilled with the traditional ranger combat styles. He gains the Precise Shot and Two-Weapon Fighting feats if he meets the prerequisites. However, the benefits of this lore apply only when the ranger uses light or no armor.

\subcf{Evasion} If the ranger makes a successful Reflex saving throw against an attack that normally deals half damage on a successful save, he instead takes no damage. A helpless ranger does not gain the benefit of evasion.

\subcf{Fast Movement} The ranger gains a \plus10 foot competence bonus to movement speed.

\subcf{Favored Enemy} The ranger increases his quarry bonus by \plus2 against creatures of a particular kind. The possible creature options are listed below.

\begin{dtable}
\begin{tabularx}{\columnwidth}{X X}
Animals and vermin & Humanoids (uncivilized) \\
Dragons & Oozes and plants \\
Fey & Outsiders (inner planes) \\
Giants and monstrous humanoids & Outsiders (outer planes) \\
Humanoids (civilized)  & Undead and constructs \\
\end{tabularx}
\end{dtable}

\subcf{Master of the Hunt} The ranger may use a standard action to share the benefits of his quarry ability with all allies who can see and hear him. The bonus his allies get is considered an enhancement bonus.

\subcf{Improved Combat Style} The ranger increases his skill in the traditional ranger combat styles. He adds half his Wisdom to damage when using ranged attacks or when attacking with two weapons at once. Natural weapons qualify for this purpose if the ranger attacks with two natural weapons at once.

The ranger must have the combat style lore to select this lore. The benefits of this lore apply only when the ranger uses light or no armor.

\subcf{Scent} The ranger gains scent, as the monster ability.

\subcf{Trapfinding} The ranger gains trapfinding, as the rogue skill trick.

\cf{Rgr}{Low-light Vision (Ex)} At 4th level, a ranger's sight improves, allowing him to see in conditions of dim light more easily. He gains low-light vision, as the elf racial ability. If he already has low-light vision, he doubles its benefit, allowing him to see four times as far as a human in poor illumination.

\cf{Rgr}{Tracking Expert (Ex)} At 4th level, a ranger's ability to track his foes improves. He may always take 10 on Survival checks made to track, even if conditions would otherwise prevent this. Additionally, he can move at his normal speed while following tracks without taking the normal \minus5 penalty. He takes only a \minus10 penalty (instead of the normal \minus20) when moving at up to twice normal speed while tracking.

\cf{Rgr}{Free Stride (Ex)} At 5th level, a ranger can move through any sort of natural terrain that slows or impedes movement at his normal speed without suffering any sort of impairment. If a skill check, such as Climb or Swim, would normally be required to move through the terrain, this ability does not help.

\cf{Rgr}{Tenacious Hunter (Ex)} At 5th level, a ranger's ability to pursue his quarry improves. He adds his quarry bonus as a competence bonus to his dodge modifier and saving throws against attacks that his quarry makes.

\cf{Rgr}{Favored Terrain (Ex)} 

\cf{Rgr}{Guide (Ex)} At 7th level, whenever the ranger is are in his favored terrain, all allies that can see and hear the ranger gain his favored terrain bonuses in that terrain as well.

\cf{Rgr}{Darkvision (Ex)} At 8th level, a ranger's sight improves again, and he gains the ability to see even when there is no light at all. He gains darkvision out to 60 feet, as the dwarf ability. If he already has darkvision, he increases its range by 60 feet.

\cf{Rgr}{Greater Combat Style (Ex)} At 10th level, a ranger's aptitude in combat improves again. He is treated as having the Ambidexterity feat (\pref{Ambidexterity [Combat]}) and the Manyshot feat (\pref{Manyshot [Combat]}), even if he does not have the normal prerequisites for those feats.
\par As before, the benefits of the ranger's chosen style apply only when he wears light or no armor. He loses all benefits of his combat style when wearing medium or heavy armor.

\cf{Rgr}{Favored Terrain (Planar) (Ex)} After 10th level, a ranger may choose any plane as a favored terrain in addition to his normal options whenever he gains a new favored terrain. He is immune to any hostile planar effects from any plane he has chosen as favored terrain. In addition, he gains a \plus2 competence bonus to Knowledge checks relating to the plane and is always treated as trained in Knowledge (planar) for the purpose of such checks.

\cf{Rgr}{Hidden Hunter (Su)} At 11th level, the ranger becomes even more difficult for his quarry to detect. He adds his quarry bonus to his Stealth checks against his quarry. In addition, he continuously benefits from the effect of the \spell{nondectection} spell against all attempts that his quarry makes to detect him magically. The effect uses a caster level equal to his ranger level.

\cf{Rgr}{Advanced Ranger Lore} After 12th level, the ranger can choose an advanced ranger lore in place of a regular ranger lore. All advanced lore abilities are extraordinary abilities unless otherwise indicated. His options for advanced ranger lores are listed below.

\subcf{Camouflage} The ranger can use the Hide skill in any of his favored terrains, even if the terrain does not grant cover or concealment.

\subcf{Combat Style Mastery} The ranger's abilities with traditional ranger combat styles reach their peak. When using a ranged weapon, he can take a move action to study the weak points of a foe within one range increment. If he does, the next attack he makes against that foe, if it is made in the same turn, is made as a ranged touch attack. When wielding two weapons at once, he gains the pounce ability, allowing him to take a full attack action at the end of a charge.

The ranger must have the greater combat style lore to choose this lore. The benefits apply only if the ranger is wearing light or no armor.

\subcf{Greater Combat Style} The ranger's abilities with traditional ranger combat styles improves again. He gains the Two-Weapon Rend and Manyshot feats if he meets the prerequisites. He must have the improved combat style lore to choose this lore. The benefits of this lore only apply if the ranger is wearing light or no armor.

\subcf{Hail of Arrows} A number of times per day equal to 1 \add half the ranger's Constitution, he may take a full-round action to fire a single arrow at every enemy within a \areamed radius. All enemies must be within one range increment of the ranger. This lore can be used with any ranged weapon that the ranger is capable of making a full attack with.

\subcf{Improved Evasion} The ranger's ability to avoid damage improves. He still takes no damage on a successful Reflex saving throw against attacks, but henceforth he takes only half damage on a failed save. A helpless ranger does not gain the benefit of improved evasion.

\subcf{Storm of Blades} A number of times per day equal to 1 \add half the ranger's Constitution, he may take a standard action to make a single melee attack against every enemy he threatens.

\cf{Rgr}{Blindsense (Ex)} At 12th level, a ranger's perceptions are so finely honed that he can sense his enemies without seeing them. He gains the blindsense ability out to 60 feet. This ability allows him to sense the presence and location of objects and foes within 60 feet without seeing them. If he already has the blindsense ability, he increases its range by 60 feet.

\cf{Rgr}{Terrain Mastery (Ex)} At 13th level, a ranger gains a greater degree of mastery over some of his favored terrains. He chooses a single kind of terrain that he has already chosen as a favored terrain. At 17th level, he chooses an additional kind of terrain to master.
\par While in that terrain, his competence bonuses on Perception, Stealth, and Survival checks increase to \plus4. In addition, he gains another ability based on that terrain that is constantly active, whether or not he is currently in the terrain. The options for terrain masteries are given below.
\subcf{Aquatic} The ranger does not suffer penalties for fighting and moving underwater.
\subcf{Cold} The ranger gains cold damage reduction 30.
\subcf{Desert} The ranger becomes immune to fatigue.
\subcf{Forest} The ranger may use his wild speech ability to communicate with plants, as the druid ability.
\subcf{Mountains} The ranger is always treated as if he had a running start when jumping (see \pcref{Athletics}). In addition, he takes only half damage from falling damage.
\subcf{Plains} The ranger gains a \plus10 competence bonus to his land speed.
\subcf{Swamp} The ranger becomes immune to nausea.
\subcf{Underground} The ranger increases the range of his darkvision and blindsense by 60 feet.
\subcf{Urban} The ranger can treat cities as being natural terrain for the purpose of his camouflage and hide in plain sight abilities.

\cf{Rgr}{Blindsight (Ex)} At 16th level, a ranger gains the ability to ``see'' perfectly without his eyes in a 60 foot radius around him. With this ability, he can fight just as well with his eyes closed as with them open. If he already has the blindsight ability, he increases its range by 60 feet.

\cf{Rgr}{Unerring Hunter (Su)} At 17th level, a ranger's ability to hunt down his quarry improves to supernatural levels. Once per day, the ranger may concentrate for a full round to duplicate the effects of the \spell{discern location} spell targeted at his quarry.

\cf{Rgr}{Hide in Plain Sight (Ex)} While in any sort of natural terrain, a ranger of 17th level or higher can use the Hide skill even while being observed. He still needs cover or concealment to hide.

\cf{Rgr}{Perfect Stride (Su)} At 19th level, a ranger's ability to surpass obstacles becomes unparalleled. He constantly acts as if he were under the effect of a \spell{freedom} spell, except that it does not allow him to act normally underwater.

\cf{Rgr}{Truesight (Su)} At 20th level, a ranger's perceptions are accurate enough to defeat even powerful magic. He gains the ability to see all things as they actually are, as the \spell{true seeing} spell, out to a range of 60 feet.

\subsection{Rogue}
\begin{dtable*}
\lcaption{The Rogue}
\begin{tabularx}{\textwidth}{>{\ccol}p{\levelcol} >{\ccol}p{\babcolgood} *{3}{>{\ccol}p{\babcolgood}} X}
\thead{Level} & \thead{Base Attack Bonus} & \thead{Fort Save} & \thead{Ref Save} & \thead{Will Save} & \thead{Special} \\
1st  & \plus0                & \plus0 & \plus3  & \plus0 & Sneak attack \plus1d6 \\
2nd  & \plus1                & \plus1 & \plus4  & \plus1 & Uncanny dodge, skill trick, danger sense \\
3rd  & \plus2                & \plus1 & \plus5  & \plus1 & Ambush attack \plus1d6 \\
4th  & \plus3                & \plus2 & \plus6  & \plus2 & Evasion, combat trick \\
5th  & \plus3                & \plus2 & \plus7  & \plus2 & Skill trick, sneak attack \plus2d6 \\
6th  & \plus4                & \plus3 & \plus8  & \plus3 & Improved uncanny dodge \\
7th  & \plus5                & \plus3 & \plus9  & \plus3 & Ambush attack \plus2d6, combat trick\\
8th  & \plus6/\plus1         & \plus4 & \plus10 & \plus4 & Skill trick \\
9th  & \plus6/\plus1         & \plus4 & \plus11 & \plus4 & Sneak attack \plus3d6 \\
10th & \plus7/\plus2         & \plus5 & \plus12 & \plus5 & Greater uncanny dodge, combat trick \\
11th & \plus8/\plus3         & \plus5 & \plus13 & \plus5 & Advanced skill trick, ambush attack \plus3d6 \\
12th & \plus9/\plus4         & \plus6 & \plus14 & \plus6 & Jack of all trades \\
13th & \plus9/\plus4         & \plus6 & \plus15 & \plus6 & Sneak attack \plus4d6, advanced combat trick \\
14th & \plus10/\plus5        & \plus7 & \plus16 & \plus7 & Advanced skill trick \\
15th & \plus11/\plus6/\plus1 & \plus7 & \plus17 & \plus7 & Ambush attack \plus4d6 \\
16th & \plus12/\plus7/\plus2 & \plus8 & \plus18 & \plus8 & Advanced combat trick \\
17th & \plus12/\plus7/\plus2 & \plus8 & \plus19 & \plus8 & Advanced skill trick, sneak attack \plus5d6 \\
18th & \plus13/\plus8/\plus3 & \plus9 & \plus20 & \plus9 & Master of all trades \\
19th & \plus14/\plus9/\plus4 & \plus9 & \plus21 & \plus9 & Advanced combat trick, ambush attack \plus5d6 \\
20th & \plus15/\plus10/\plus5& \plus10& \plus22 & \plus10& Ambush master, advanced skill trick
\end{tabularx}
\end{dtable*}

\cd{Alignment} Any.

\cd{Hit Value} 5.

\sssecfake{Class Skills}
The rogue's class skills (and the key attribute for each skill) are
Athletics (Str), Climb (Str), Swim (Str), Acrobatics (Dex), Escape Artist (Dex),  Sleight of Hand (Dex), Stealth (Dex), Craft (Int), Devices (Int), Forgery (Int), Knowledge (dungeoneering), Knowledge (local) (Int), Linguistics (Int), Perception (Wis), Sense Motive (Wis), Bluff (Cha), Persuasion (Cha), Disguise (Cha), Intimidate (Cha), and Perform (Cha).
\cf{Rog}{Skill Points at 1st Level} 12.

\sssecfake{Class Features}
All of the following are class features of the rogue.

\cf{Rog}{Weapon and Armor Proficiency} Rogues are proficient with simple weapons,  any two weapon groups,  light armor,  and bucklers. A rogue is also proficient with saps.

\cf{Rog}{Sneak Attack}  If a rogue can catch an opponent when he is unable to defend himself effectively from her attack, she can strike a vital spot for extra damage. The rogue's attack deals extra damage if the target is flat-footed or is suffering overwhelm penalties.  This extra damage is 1d6 at 1st level, and it increases by 1d6 every four rogue levels thereafter. Should the rogue score a critical hit with a sneak attack, this extra damage is not multiplied. This damage bonus is treated as a circumstance bonus.

\par Ranged attacks can count as sneak attacks only if the target is within 30 feet. A rogue can't strike with deadly accuracy from beyond that range.

With a sap (blackjack) or an unarmed strike, a rogue can make a sneak attack that deals nonlethal damage instead of lethal damage. She cannot use a weapon that deals lethal damage to deal nonlethal damage in a sneak attack, not even with the usual \minus4 penalty.

A rogue can sneak attack only living creatures with discernible anatomies -- oozes, plants, and incorporeal creatures lack vital areas to attack. Any creature that is immune to critical hits is not vulnerable to sneak attacks. The rogue must be able to see the target well enough to pick out a vital spot and must be able to reach such a spot. A rogue cannot sneak attack while striking a creature with concealment or striking the limbs of a creature whose vitals are beyond reach.

\cf{Rog}{Danger Sense (Ex)} Starting at 2nd level, a rogue has an intuitive sense that alerts her to danger, giving her a \plus2 competence bonus to initiative checks. This bonus increases by 1 at 5th level and every 3 levels thereafter.
\par If a character has danger sense from a multiple classes, the character stacks those levels to determine her bonus from danger sense.

\cf{Rog}{Skill Tricks} As a rogue gains experience, she gains additional insight into how to perfect her skills. At 2nd level, a rogue gains one skill trick. She gains an additional skill trick at 5th level and every three levels thereafter. A rogue cannot select an individual skill trick more than once unless otherwise stated. All skill tricks are (Ex) abilities unless otherwise noted.

\subcf{Fast Acrobatics} The rogue reduces her penalties for moving quickly with the Acrobatics skill by 5.

\subcf{Fast Stealth} The rogue reduces her penalties for moving quickly with the Stealth skill by 5.

\subcf{Kip Up} The rogue can stand up from a prone position as a swift action. The rogue must be in light or no armor to perform this trick.

\subcf{Knowledgeable Strike} The rogue may sneak attack almost any foe she identifies with a successful Knowledge check (see the Knowledge skill, \pref{Knowledge (Int; Trained Only)}). She still may not sneak attack incorporeal foes, even if she successfully identifies them.

\subcf{Ledge Walker} The rogue may move along narrow surfaces at full speed using the Acrobatics skill without penalty. In addition, she is not flat-footed when using Acrobatics to move along narrow surfaces.

\subcf{Lingering Poison} When the rogue applies poison to a weapon, it lasts for twice as many doses as it normally would.

\subcf{Quick Disable} It takes the rogue half the normal amount of time to disable a trap or open a lock using the Devices skill (minimum 1 round).

\subcf{Quick Search} The rogue can search an area (with the Perception skill) as a move action rather than as a full-round action.

\subcf{Rogue Crawl} While prone, the rogue can move at half speed. This movement provokes attacks of opportunity as normal.

\subcf{Skill Feat} The rogue gains a bonus skill feat (see Feats). A rogue can select this trick multiple times.

\subcf{Standing Leap} The rogue is always treated as if she had a running start when making Athletics checks.

\subcf{Swift Poisoner} The rogue can apply poison to a weapon as a move action instead of a standard action.

\subcf{Trapfinding} As a full-round action, a rogue may move up 10 feet while searching every square within 10 feet of her for traps. If a rogue detects a trap partway through her movement, she may immediately stop moving.

\subcf{Trap Sense} Whenever the rogue comes within 10 feet of a trap, she receives an immediate Perception check to notice the trap. This check should be made secretly, so the rogue does not know whether she failed to notice a trap.

\cf{Rog}{Uncanny Dodge (Ex)} Starting at 2nd level, a rogue can react to danger before her senses would normally allow her to do so. She may apply her Dexterity and dodge modifier to her armor class while flat-footed.

If a rogue already has uncanny dodge from a different class, she stacks those levels to determine whether she gains improved uncanny dodge (see below) instead.

\cf{Rog}{Ambush Attack (Ex)} At 3rd level, a rogue learns how to deal extra damage when she ambushes her foe. The first time that a rogue successfully sneak attacks a particular foe in an encounter, the attack is considered an ambush attack. The sneak attack damage on an ambush attack increases by 1d6. After she has delivered an ambush attack against that foe, she cannot make any more ambush attacks against that same foe for the rest of the encounter. A rogue can deliver no more than one ambush attack per round. The extra damage dealt by an ambush attack increases by 1d6 at 7th level and four rogue levels thereafter.

\cf{Rog}{Combat Tricks} As a rogue gains experience, she learns a small number of talents that aid her and confound her foes. At 4th level, a rogue gains one combat trick. A rogue cannot select an individual combat trick more than once unless otherwise stated. All combat tricks are (Ex) abilities unless otherwise noted. The rogue gains an additional combat trick at 7th level and every three levels thereafter.

\par Tricks marked with an asterisk are called ambush tricks. Ambush tricks add additional effects to the rogue's ambush attacks. If an ambush trick allows a saving throw, the DC is 10 \add the rogue's level \add the rogue's Intelligence.The rogue chooses from the list of combat tricks below.

\subcf{Brutal Ambush*} The rogue rolls d8s instead of d6s for her sneak attack dice on this attack.

\subcf{Combat Feat} The rogue gains a combat feat (see Feats).

\subcf{Dispelling Ambush (Su)*} A foe damaged by this attack is affected by a targeted \spell{dispel magic} which affects only the highest-level spell effect active on the target. The caster level for this ability is equal to the rogue's level.

\subcf{Distracting Attack} A foe damaged by the rogue's sneak attack takes a penalty on its Concentration checks equal to the rogue's Intelligence for 5 rounds.

\subcf{Extended Precision} The rogue can make sneak attacks from up to 60 feet away.

\subcf{Hamstring*} A foe damaged by this attack has its speed with a single movement mode halved for 5 rounds. Despite the name, this can be used on foes who do not have hamstrings, though it only affects ground movement. Some forms of movement, such as magical flight, cannot be impeded by this attack.

\subcf{Merciful Blows} The rogue suffers no penalty to attack rolls when making attacking for nonlethal damage, and can deal her full sneak attack damage.

\subcf{Poison Use} The rogue cannot accidentally poison herself when applying poison to an object.

\subcf{Swift Poisoner} The rogue can apply poison to a weapon she is holding as a swift action.

\subcf{Tricky Maneuver} When performing maneuvers against foes she would be able to sneak attack, the rogue gains a circumstance bonus to attack equal to the number of sneak attack dice she would normally roll. The benefits of this trick apply even against foes immune to critical hits.

\cf{Rog}{Improved Uncanny Dodge (Ex)} At 6th level and higher, a rogue can no longer be overwhelmed as easily; she can react to multiple opponents as easily as she can react to a single attacker. The rogue is always treated as being threatened by two fewer creatures than she actually is for the purpose of determining overwhelm penalties. 

\par If a character already has improved uncanny dodge from a second class and gains improved uncanny dodge, the character stacks those levels to determine if she should gain greater uncanny dodge.

\cf{Rog}{Greater Uncanny Dodge (Ex)} At 10th level and higher, a rogue can no longer be overwhelmed, regardless of the number of foes surrounding her.

\cf{Rog}{Advanced Skill Trick} At 11th level, and every three levels thereafter, the rogue may choose any of the following advanced skill tricks in addition to her other options for skill tricks. All advanced skill tricks are (Ex) abilities unless otherwise noted.

\subcf{Disable Spell (Su)} The rogue can use Devices to dispel any currently active spell as if it were a magical trap. Doing so requires a full-round action that provokes attacks of opportunity. The DC to disable the spell is equal to 10 \add the caster level of the spell \add the level of the spell. If the spell is not subject to \spell{dispel magic}, it cannot be dispelled using this ability.

\subcf{Exemplar} The rogue must choose one skill that she already has Skill Focus in. The skill must be on the rogue class skill list. She gains a \plus6 competence bonus with the skill.

\subcf{Hide in Plain Sight} The rogue can use the Hide even while being observed. She still needs cover or concealment to hide.

\subcf{Rogue's Luck} Three times per day, the rogue can reroll any skill check. A single roll can never be rerolled more than once.

\subcf{Skill Mastery} The rogue gains the Skill Mastery feat in a number of skills equal to 1 \add half her Intelligence (minimum 1). Each skill must be on the rogue class skill list.

\cf{Rog}{Advanced Combat Trick} At 12th level, and every three levels thereafter, the rogue may choose any of the following advanced combat tricks in addition to her other options for combat tricks. All advanced combat tricks are (Ex) abilities unless otherwise noted.

\subcf{Ambush Strike} When the rogue uses a Strike feat (see \pcref{Strike Feats}) on an ambush attack, she may add her Intelligence to the saving throw DC against the strike.

\subcf{Assassination} To use this ability, the rogue must spend a full round studying a foe has not noticed her and who is not in combat. If she make a melee ambush attack on her next turn against that target, her attack deals maximum damage. If the target becomes aware of her presence before she attacks, this ability has no benefit.

\subcf{Crippling Ambush} A rogue with this trick can ambush attack opponents with such precision that her blows weaken and hamper them. An opponent damaged by one of her ambush attacks takes 2 points of Strength damage. Ability points lost to damage return on their own at the rate of 1 point per day for each damaged ability. If the rogue has no ambush attacks, she may use crippling strike as an ambush attack.

\subcf{Defensive Roll}  The rogue can roll with a potentially lethal blow to take less damage from it than she otherwise would.  A number of times per day equal to half the rogue's Wisdom, when she would be reduced to 0 or fewer hit points by damage in combat (from a weapon or other blow, not a spell or special ability), the rogue can attempt to roll with the damage. To use this ability, the rogue must attempt a Reflex saving throw (DC = damage dealt). If the save succeeds, she takes only half damage from the blow, and the damage is nonlethal; if it fails, she takes full damage. She must be aware of the attack and able to react to it in order to execute her defensive roll -- if she is flat-footed, she can't use this ability. Since this effect would not normally allow a character to make a Reflex save for half damage, the rogue's evasion ability does not apply to the defensive roll.

\subcf{Distant Precision} The rogue has no range limit on her sneak attacks. A rogue must have selected the extended precision combat trick before choosing this trick.

\subcf{Improved Evasion} This talent works like evasion, except that the rogue also takes only half damage on a failed save. A helpless rogue does not gain the benefit of improved evasion.

\subcf{Opportunist}  Once per round, the rogue can make an attack of opportunity against an opponent who has just been struck for damage in melee by another character. This attack counts as one of the rogue's attacks of opportunity for that round.

\subcf{Slippery Mind}  This ability represents the rogue's ability to wriggle free from magical effects that would otherwise control or compel her. If a rogue with slippery mind is affected by a mind-affecting spell or effect and fails her saving throw, she can attempt it again 1 round later at the same DC. She gets only this one extra chance to succeed on her saving throw.

\cf{Rog}{Jack of All Trades (Ex)} At 16th level, a rogue treats all skills as class skills.

\cf{Rog}{Master of All Trades (Ex)} At 18th level, a rogue is treated as having as having at least one skill point in all skills, except for trained-only skills. These ``phantom'' skill points give her ranks in the skills normally, but do not otherwise count as skill points.

\cf{Rog}{Ambush Master (Ex)} A 20th level rogue has achieved such a mastery of tricky combat that she can combine the effects of two different ambush tricks into a single ambush attack.

\subsection{Sorcerer}
\begin{dtable*}
\lcaption{The Sorcerer}
\begin{tabularx}{\textwidth}{>{\ccol}p{\levelcol} >{\ccol}p{7em} *{3}{>{\ccol}p{\savecol}} >{\lcol}X}
\thead{Level} & \thead{Base Attack Bonus} & \thead{Fort Save} & \thead{Ref Save} & \thead{Will Save} & \thead{Special} \\
1st & \plus0 & \plus0 & \plus0 & \plus3 & Arcane invocation, Rapid Metamagic, versatile spellcaster \\
2nd & \plus1 & \plus1 & \plus1 & \plus4     & Arcane invocation \\
3rd & \plus1 & \plus1 & \plus1 & \plus5     & Expanded spell knowledge \\
4th & \plus2 & \plus2 & \plus2 & \plus6     & Spellblend \\
5th & \plus2 & \plus2 & \plus2 & \plus7     & Expanded spell knowledge \\
6th & \plus3 & \plus3 & \plus3 & \plus8     & \x \\
7th & \plus3 & \plus3 & \plus3 & \plus9     & Expanded spell knowledge \\
8th & \plus4 & \plus4 & \plus4 & \plus10    & Improved spellblend \\
9th & \plus4 & \plus4 & \plus4 & \plus11    & Expanded spell knowledge \\
10th & \plus5 & \plus5 & \plus5 & \plus12    & \x  \\
11th & \plus5 & \plus5 & \plus5 & \plus13    & Expanded spell knowledge \\
12th & \plus6/\plus1 & \plus6 & \plus6 & \plus14& Versatile spellblend \\
13th & \plus6/\plus1 & \plus6 & \plus6 & \plus15& Expanded spell knowledge \\
14th & \plus7/\plus2 & \plus7 & \plus7 & \plus16& \x \\
15th & \plus7/\plus2 & \plus7 & \plus7 & \plus17& Expanded spell knowledge \\
16th & \plus8/\plus3 & \plus8 & \plus8 & \plus18 & Spellsurge \\
17th & \plus8/\plus3 & \plus8 & \plus8 & \plus19 & Expanded spell knowledge \\
18th & \plus9/\plus4 & \plus9 & \plus9 & \plus20 & \x \\
19th & \plus9/\plus4 & \plus9 & \plus9 & \plus21 & Expanded spell knowledge \\
20th & \plus10/\plus5 & \plus10& \plus10& \plus22 & Improved spellsurge \\
\end{tabularx}
\end{dtable*}

\cd{Alignment} Any.

\cd{Hit Value} 4

\sssecfake{Class Skills}
The sorcerer's class skills (and the key attribute for each skill) are Knowledge (arcana) (Int), Knowledge (the planes), Spellcraft (Wis), and Intimidate (Cha).
\cf{Sor}{Skill Points at 1st Level} 2.

\sssecfake{Class Features}
All of the following are class features of the sorcerer.

 \cf{Sor}{Weapon and Armor Proficiency}  Sorcerers are proficient with simple weapons  and one other weapon group.  They are not proficient with any type of armor or shield. Armor of any type interferes with a sorcerer's arcane gestures, which can cause his spells with somatic components to fail.

\cf{Sor}{Spells} A sorcerer casts arcane spells using his Charisma.  To learn or cast a spell, a sorcerer must have a Charisma at least equal to the spell's level. The Difficulty Class for a saving throw against a sorcerer's spell is 10 \add half the sorcerer's caster level \add the sorcerer's Charisma.

Like other spellcasters, the number of spells a sorcerer knows and can cast each day is limited. These limitations are given below on \trefnp{Sorcerer Spells per Day} and \trefnp{Sorcerer Spells Known}. A sorcerer's spells are drawn from the common spells on the sorcerer/wizard spell list (see \pcref{Sorcerer/Wizard Spells}). Sorcerers need to rest for eight hours in order to regain spells.

A sorcerer's magic level is equal to his sorcerer level.

\begin{dtable}
    \lcaption{Sorcerer Spells per Day}
    \centering
    \begin{tabularx}{\columnwidth}{>{\ccol}X *{9}{>{\ccol}p{\spellcol}}}
        & \multicolumn{9}{c}{\thead{---{}---{}---{}---{}---{}---{}---{}---Spell Level---{}---{}---{}---{}---{}---{}---{}---}} \\
        \thead{Level} & \thead{1st} & \thead{2nd} & \thead{3rd} & \thead{4th} & \thead{5th} & \thead{6th} & \thead{7th} & \thead{8th} & \thead{9th} \\
        1st & 3 & \x & \x & \x & \x & \x & \x & \x & \x \\
        2nd & 4 & \x & \x & \x & \x & \x & \x & \x & \x \\
        3rd & 5 & \x & \x & \x & \x & \x & \x & \x & \x \\
        4th & 6 & 3 & \x & \x & \x & \x & \x & \x & \x \\
        5th & 6 & 4 & \x & \x & \x & \x & \x & \x & \x \\
        6th & 6 & 5 & 3 & \x & \x & \x & \x & \x & \x \\
        7th & 6 & 6 & 4 & \x & \x & \x & \x & \x & \x \\
        8th & 6 & 6 & 5 & 3 & \x & \x & \x & \x & \x \\
        9th & 6 & 6 & 6 & 4 & \x & \x & \x & \x & \x \\
        10th & 6 & 6 & 6 & 5 & 3 & \x & \x & \x & \x \\
        11th & 6 & 6 & 6 & 6 & 4 & \x & \x & \x & \x \\
        12th & 6 & 6 & 6 & 6 & 5 & 3 & \x & \x & \x \\
        13th & 6 & 6 & 6 & 6 & 6 & 4 & \x & \x & \x \\
        14th & 6 & 6 & 6 & 6 & 6 & 5 & 3 & \x & \x \\
        15th & 6 & 6 & 6 & 6 & 6 & 6 & 4 & \x & \x \\
        16th & 6 & 6 & 6 & 6 & 6 & 6 & 5 & 3 & \x \\
        17th & 6 & 6 & 6 & 6 & 6 & 6 & 6 & 4 & \x \\
        18th & 6 & 6 & 6 & 6 & 6 & 6 & 6 & 5 & 3 \\
        19th & 6 & 6 & 6 & 6 & 6 & 6 & 6 & 6 & 4 \\
        20th & 6 & 6 & 6 & 6 & 6 & 6 & 6 & 6 & 6 \\
    \end{tabularx}
\end{dtable}

\begin{dtable}
    \lcaption{Sorcerer Spells Known}
    \begin{tabularx}{\columnwidth}{>{\ccol}X *{9}{>{\ccol}p{\spellcol}}}
        & \multicolumn{9}{c}{\thead{---{}---{}---{}---{}---{}---{}---{}---Spell Level---{}---{}---{}---{}---{}---{}---{}---}} \\
        \thead{Level} & \thead{1st} & \thead{2nd} & \thead{3rd} & \thead{4th} & \thead{5th} & \thead{6th} & \thead{7th} & \thead{8th} & \thead{9th} \\
        1st  & 1 & \x & \x & \x & \x & \x & \x & \x & \x \\
        2nd  & 2 & \x & \x & \x & \x & \x & \x & \x & \x \\
        3rd  & 3 & \x & \x & \x & \x & \x & \x & \x & \x \\
        4th  & 3 & 1 & \x & \x & \x & \x & \x & \x & \x \\
        5th  & 4 & 2 & \x & \x & \x & \x & \x & \x & \x \\
        6th  & 4 & 2 & 1 & \x & \x & \x & \x & \x & \x \\
        7th  & 4 & 3 & 2 & \x & \x & \x & \x & \x & \x \\
        8th  & 4 & 3 & 2 & 1 & \x & \x & \x & \x & \x \\
        9th  & 4 & 3 & 3 & 2 & \x & \x & \x & \x & \x \\
        10th & 4 & 3 & 3 & 2 & 1 & \x & \x & \x & \x \\
        11th & 4 & 3 & 3 & 3 & 2 & \x & \x & \x & \x \\
        12th & 4 & 3 & 3 & 3 & 2 & 1 & \x & \x & \x \\
        13th & 4 & 3 & 3 & 3 & 3 & 2 & \x & \x & \x \\
        14th & 4 & 3 & 3 & 3 & 3 & 2 & 1 & \x & \x \\
        15th & 4 & 3 & 3 & 3 & 3 & 3 & 2 & \x & \x \\
        16th & 4 & 3 & 3 & 3 & 3 & 3 & 2 & 1 & \x \\
        17th & 4 & 3 & 3 & 3 & 3 & 3 & 2 & 2 & \x \\
        18th & 4 & 3 & 3 & 3 & 3 & 3 & 2 & 2 & 1 \\
        19th & 4 & 3 & 3 & 3 & 3 & 3 & 2 & 2 & 2 \\
        20th & 4 & 3 & 3 & 3 & 3 & 3 & 2 & 2 & 2
    \end{tabularx}
\end{dtable}

\cf{Sor}{Arcane Invocation} All sorcerers master at least one arcane invocation. An arcane invocation allows a sorcerer to exert magical influence without expending the effort required to cast a spell. The sorcerer may choose to learn one invocation of her choice from the list of arcane invocations described in Chapter 11: Spells. At 2nd level, the sorcerer learns a second arcane invocation of his choice.

\cf{Sor}{Rapid Metamagic} A sorcerer gains Rapid Metamagic as a bonus feat at 1st level, even if he does not meet the prerequisites.

\cf{Sorc}{Versatile Spellcaster (Ex)} A sorcerer's intuitive grasp of magic allows him to be flexible in his use of arcane energy. A sorcerer can use two sorcerer spell slots of the same level to use an ability requiring a sorcerer spell slot of one level higher. For example, Serric the sorcerer can use two 0th-level spell slots to cast a single 1st-level spell he knows.

\cf{Sor}{Expanded Spell Knowledge (Ex)} At 3rd level, and every odd level afterwards, a sorcerer can learn how to cast a particularly esoteric spell. He may choose a restricted spell from the sorcerer/wizard spell list with a level of no more than half his sorcerer level (normally, the highest level spell he can cast) and add it to his spell list. He must still use a spell known to learn it, as normal.

\cf{Sor}{Spellblend (Ex)} At 4th level, a sorcerer may combine his arcane invocations with his spells. As a full-round action, the sorcerer may cast a spell that affects only himself using a spell slot one level higher than what the spell would normally require. If he does, he may also use an arcane invocation as part of the same action. The arcane invocation need not target the sorcerer.

\cf{Sor}{Improved Spellblend (Ex)} At 8th level, a sorcerer may combine two spells together. As a full-round action, the sorcerer may cast two spells at once, resolving each spell's effects separately. The spells cast in this way must be at least three spell levels apart, such as a 1st-level spell and a 4th-level spell. In addition, one of the two spells must affect only the sorcerer. Using improved spellblend costs a spell slot of one level higher than the highest level spell being cast.

\cf{Sor}{Versatile Spellblend (Ex)} At 12th level, a sorcerer may combine any two spells together. When using spellblend or improved spellblend, the sorcerer may cast any spells, regardless of whether they affect only the sorcerer. However, using versatile spellblend costs a spell slot of two levels higher than the highest level spell being cast.

\cf{Sor}{Spellsurge (Ex)} At 16th level, a sorcerer may enter a trance-like state once per day in which he can surpass his normal limits. A spellsurge trance lasts for one minute. During the trance, the sorcerer may use his versatile spellblend ability by expending a spell slot of one level lower than the highest level spell being cast. However, the sorcerer is forced to expel the arcane energy welling up inside him, and is forced to use his versatile spellblend ability with all of his actions. The sorcerer can suppress this effect for a round with a DC 30 Will save, but he takes nonlethal damage equal to his caster level if he does so. %what happens if the sorcerer can't act? Does he take damage?

\par At 19th level, a sorcerer can enter a spellsurge trance an additional time per day.

\cf{Sor}{Improved Spellsurge (Ex)} At 20th level, a sorcerer in a spellsurge trance can use his spellblend ability as a standard action instead of as a full-round action.

\subsection{Spellwarped}
\begin{dtable}
    \lcaption{The Spellwarped}
    \begin{tabularx}{\columnwidth}{>{\ccol}p{\levelcol} >{\ccol}p{\babcolavg} *{2}{>{\ccol}p{\savecolpoof}} >{\lcol}X}
        \thead{Level} & \thead{Base Attack Bonus} & \thead{Good Save}\fn{1} & \thead{Normal Saves}\fn{1} & \thead{Special} \\
        1st & \plus0                    & \plus2  & \plus1  & Innate magic, invoke power, spellwarp pool\\
        2nd & \plus1                    & \plus3  & \plus2  & Spellwarped body, surge of power \\
        3rd & \plus2                    & \plus4  & \plus3  & Attuned senses, spellwarped aspect \\
        4th & \plus3                    & \plus5  & \plus4  & Invoke power, resist magic \\
        5th & \plus3                    & \plus6  & \plus4  & Manipulate magic \\
        6th & \plus4                    & \plus7  & \plus5  & Invoke power \\
        7th & \plus5                    & \plus8  & \plus6  & Spellwarped aspect \\
        8th & \plus6/\plus1             & \plus9  & \plus7  & Invoke power \\
        9th & \plus6/\plus1             & \plus10 & \plus7  & Spell resistance\\
        10th & \plus7/\plus2            & \plus11 & \plus8  & Invoke power \\
        11th & \plus8/\plus3            & \plus12 & \plus9  & Spellwarped aspect \\
        12th & \plus9/\plus4            & \plus13 & \plus10 & Invoke power \\
        13th & \plus9/\plus4            & \plus14 & \plus10 & Improved manipulate magic \\
        14th & \plus10/\plus5           & \plus15 & \plus11 & Invoke power \\
        15th & \plus11/\plus6/\plus1    & \plus16 & \plus12 & Spellwarped aspect \\
        16th & \plus12/\plus7/\plus2    & \plus17 & \plus13 & Invoke power \\
        17th & \plus12/\plus7/\plus2    & \plus19 & \plus13 & Mass surge of power \\
        18th & \plus13/\plus8/\plus3    & \plus20 & \plus14 & Invoke power \\
        19th & \plus14/\plus9/\plus4    & \plus21 & \plus15 & Permanent surge of power, spellwarped aspect \\
        20th & \plus15/\plus10/\plus5   & \plus22 & \plus16 & Invoke power \\
    \end{tabularx}
    1 Each spellwarped has a good save determined by his choice of innate magic.
\end{dtable}

\cd{Alignment} Any.

\cd{Hit Value} 5.

\ssecfake{Class Skills}
The spellwarped's class skills (and the key attribute for each skill) are Swim (Str), Ride (Dex), Knowledge (arcana) (Int), Spellcraft (Wis), and Intimidate (Cha). He gains additional class skills based on his choice of innate magic.
\cf{Spl}{Skill Points at 1st Level} 4.

\sssecfake{Class Features}
All of the following are class features of the spellwarped.

\cf{Spl}{Weapon and Armor Proficiency}
A spellwarped is proficient with simple weapons, any two weapon groups, light and medium armor, and shields (except tower shields).

\cf{Spl}{Innate Magic (Ex)} Each spellwarped draws his power from a particular kind of magic. This is a choice made when the first level of the class is taken, and it cannot thereafter be changed. The choices are listed below.
\subcf{Alteration} The spellwarped can manipulate the physical forms of creatures. His good save is Fortitude, his key attribute is Intelligence, and he treats Athletics, Escape Artist, and Disguise as class skills. An alteration spellwarped may be called an alterer, bodywarper, or shifter.
\subcf{Pyromancy} The spellwarped can manipulate fire and heat. His good save is Will, his key attribute is Charisma, and he treats Acrobatics, Athletics, and Perform as class skills. A pyromancy spellwarped may be called a pyromancer.
\subcf{Telekinesis} The spellwarped can manipulate objects and creatures with his mind. His good save is Will, his key attribute is Intelligence, and he treats Craft, Devices, and Sleight of Hand as class skills. A telekinesis spellwarped may be called a telekine.
\subcf{Temporal} The spellwarped can manipulate time. His good save is Reflex, his key attribute is Wisdom, and he treats Acrobatics, Perception, and Sleight of Hand as class skills. A temporal spellwarped may be called a temporalist or timewarper.

\cf{Spl}{Spellwarp Pool (Su)} A spellwarped has the ability to tap into the latent magic within his body to generate magical effects. He has a maximum number of spellwarp points equal to half his spellwarped level \add his Constitution (minimum 1 point). Each hour, he regains a number of spellwarp points equal to his key attribute. As long as he has at least one spellwarp point remaining, he gains a minor ability based on his choice of magic.
\subcf{Alteration -- Alter Appearance} The spellwarped can change minor aspects of his appearance at will -- removing a mole or lengthening his beard slightly. This can grant him a \plus2 competence bonus to Disguise checks. Major changes are not possible.
\subcf{Pyromancy -- Ember} The spellwarped can snap his fingers as a swift action to create a small ember of flame in his hand for 5 minutes. This ember casts light as a torch, and can deal 1 point of fire damage with a successful touch attack. The ember can be dismissed as a swift action or extinguished as a move action.
\subcf{Telekinesis -- Object Manipulation} The spellwarped can concentrate as a standard action to move objects within five feet of him telekinetically. He can slowly lift or manipulate one object by up to one foot per round. The object can weigh up to five pounds. This level of control is insufficient to make skill checks or wield a weapon or shield effectively.
\subcf{Temporal -- Time Awareness} The spellwarped always knows exactly what time it is, and can track the passage of time precisely without effort.

\cf{Spl}{Invoke Power (Su)} A spellwarped can invoke his innate magic to generate powerful effects by spending a spellwarp point. If a saving throw is allowed, the DC is equal to 10 \add the spellwarped's class level \add his key attribute. The spellwarped gains an additional power at 4th level and every 2 levels thereafter. The powers a spellwarped gains depend on his choice of innate magic, as described in \pcref{Spellwarped Powers}.

\cf{Spl}{Surge of Power (Su)} At 2nd level, a spellwarped can invoke a surge of magical power that allows him to embody his innate magic more fully for 5 rounds. To invoke a surge of power, he must spend a spellwarp point as a a swift action. The effect of his surge depends on his choice of innate magic, as described below.
\subcf{Alteration -- Alter Body} The spellwarped enhances his physical ability. He gains a \plus2 enhancement bonus to a physical attribute of his choice. This bonus increases by 1 at 8th, 14th, and 20th level.
\subcf{Pyromancy -- Flame Aura} The spellwarped emanates an aura of fire for 5 rounds. At the start of each of his turns, creatures adjacent to him take one point of fire damage per spellwarped level.
\subcf{Telekinesis -- Kinetic Deflection} The spellwarped reflexibly deflects attacks away with his mind. He gains a \plus2 competence bonus to his shield modifier to AC. This bonus stacks with the bonus from using a shield. At 8th, 14th, and 20th level, the shield bonus increases by 1.
\subcf{Temporal -- Accelerate Movement} The spellwarped accelerates his movement and reactions. He gains a \plus2 enhancement bonus to his dodge modifier and a \plus10 foot enhancement bonus to his movement speed. At 8th, 14th, and 20th level, the dodge bonus increases by 1 and the speed bonus increases by 10 feet.

\cf{Spl}{Spellwarped Body (Ex)} At 2nd level, a spellwarped's body is fundamentally altered by exposure to magic. He shows signs of the magic coursing through his body: strangely or inconsistently colored hair, natural skin markings which often resemble runes, and so on. Anyone observing the spellwarped can make a Perception or Spellcraft check with a DC equal to 20 \sub the spellwarped's class level to recognize that the character is a spellwarped. In addition, the spellwarped gains an ability based on his innate magic.
\subcf{Alteration -- Augment Skin} The spellwarped gains a \plus1 competence bonus to his natural armor modifier. This bonus increases by 1 at 5th level and every 5 levels thereafter.
\subcf{Pyromancy -- Energy Resistance} The spellwarped gains cold and fire damage reduction equal to twice his spellwarped level, allowing him to ignore the first points of cold or fire damage he takes each round.
\subcf{Telekinesis -- Tactile Telekinesis} The spellwarped gains a \plus1 competence bonus to Strength and Dexterity-based skill checks. This bonus increases by 1 at 5th level and every 5 levels thereafter.
\subcf{Temporal -- Accelerate Reaction} The spellwarped gains a \plus2 competence bonus to initiative checks. This bonus increases by 1 at 5th level and every 3 levels thereafter.

\cf{Spl}{Attuned Senses (Su)} At 3rd level, the spellwarped learns to recognize the telltale signs of his chosen magic. He must concentrate as a standard action to use this ability, and he may do so any number of times per day.
\subcf{Alteration -- Perceive Alteration} The spellwarped can discern the true form of all creatures within 50 feet of him for 1 round, ignoring any effects which magically alter their shapes. This also grants him a \plus5 competence bonus to Perception checks to see through mundane disguises.
\subcf{Pyromancy -- Flame of Life} The spellwarped can see the life-fire that lies within all living creatures, allowing him to clearly see all living creatures within 50 feet of him for 1 round. This ability can reveal creatures hiding in concealment and defeat figments and glamers such as \spell{invisibility}, but does not reveal creatures hiding behind cover. It also allows the spellwarped to see unusually warm objects, such as fires.
\subcf{Telekinesis -- Spatial Awareness} The spellwarped can feel the forms of all objects and creatures around him, granting blindsense out to a 50 foot range for 1 round.
\subcf{Temporal -- Accelerated Search} The spellwarped can accelerate his mind to immediately search everything within a 10 foot radius of him with the Perception skill as a standard action. Alternately, he may use this ability to read a book ten times as fast as normal.

\cf{Spl}{Spellwarped Aspect (Su)} At 3rd level, the spellwarped gains a new ability based on his continued exposure to magical energy. Most aspects are specific to particular kinds of innate magic, but some aspects can be taken by any spellwarped. These aspects are listed under the General heading.  At 7th level, and every four levels thereafter, the spellwarped gains an additional spellwarped aspect. Some aspects require a minimum spellwarped level, as indicated below. The full list of spellwarped aspects is given below.

\parhead{General}
\subcf{Spellwarped Soul} The spellwarped may use his character level in place of his spellwarped level to determine the effects of his spellwarped abilities, including damage dealt and saving throw DCs. This does not affect the number of spellwarp points he has available.
\subcf{7th -- Expanded Senses} The range of the spellwarped's attuned senses ability doubles.
\subcf{11th -- Accelerated Recovery} The spellwarped regains spellwarp points once per 10 minutes, rather than once per hour.
\subcf{11th -- Rapid Senses} The spellwarped can constantly gain the benefit of his attuned senses ability. He can toggle his enhanced senses on or off as a swift action. If the ability does not have a duration, such as the temporal attuned senses ability, this aspect has no effect.

\parhead{Alteration}
\subcf{Damage Reduction} The spellwarped gains physical damage reduction against his choice of piercing, slashing, or bludgeoning damage. The amount of damage resisted is equal to half his class level, allowing him to ignore the first points of damage he takes each round. If he is hit by an adamantine weapon, he cannot use his damage reduction for 1 round.
\subcf{7th -- Improved Damage Reduction} The spellwarped's damage reduction applies against all forms of physical damage. The spellwarped must have the damage reduction aspect to gain this aspect.
\subcf{7th -- Alter Movement} The spellwarped gains his choice of the Legendary Balance, Legendary Climber, Legendary Leaper, or Legendary Swimmer feats, even if he does not meet the prerequisites. He may select this aspect multiple times, choosing a different bonus feat each time.
\subcf{11th -- Alter Size} When the spellwarped uses his surge of power, he can increase or decrease by a size category, as he chooses. The size alteration lasts as long as his surge of power does. This is a size-affecting effect, and does not stack with other size-affecting effects.
\subcf{15th -- Fast Healing} While his surge of power is active, the spellwarped gains fast healing equal to half his spellwarped level, allowing him to heal damage each round. This does not affect critical damage.

\parhead{Pyromancy}
\subcf{Friendly Fire} The spellwarped's attacks deal half damage to any creature he designates as an ally.
\subcf{Improved Ember} When the spellwarped uses his ember ability, he can strengthen the fire so that it illuminates up to a 40 foot radius with bright illumination. He can also throw the ember up to 100 feet. It burns for up to 5 rounds on its own before becoming extinguished.
\subcf{Intense Flames} The spellwarped's attacks can ignore an amount of fire damage reduction equal to his spellwarped level \add his Charisma.
\subcf{7th -- Flame Eater} When the spellwarped resists fire damage with his spellwarped body ability, he gains temporary hit points equal to the damage resisted for 5 minutes.

\parhead{Telekinesis}
\subcf{Improved Object Manipulation} When the spellwarped uses his object manipulation ability, he can affect objects within 10 feet, with a weight limit of up to two pounds per spellwarped level. He has enough control to make skill checks with a DC of up to 10.
\subcf{7th -- Shieldbearer} The spellwarped may wield shields, except tower shields, telekinetically. The shield floats in his square, granting him its AC bonus. He does not need a free hand to wield the shield and suffers no armor check penalty or arcane spell failure from it. The shield follows him as he moves. If it is forcibly removed from his square, he loses control over it and it falls to the ground.
\subcf{11th -- Mind Armory} The spellwarped may control a number of weapons equal to half his Intelligence with his mind blade ability. This does not allow him to make additional attacks per round, but he may attack interchangeably with any weapon he controls. Each weapon threatens an area and contributes to overwhelm penalties, just as with his normal mind blade ability.

\parhead{Temporal}
\subcf{Evasion} If the spellwarped makes a successful Reflex save against an attack that normally deals half damage on a successful save, he instead takes no damage. Evasion can only be used if the spellwarped is wearing light armor or no armor. A helpless spellwarped does not gain the benefit of evasion.
\subcf{Fast Movement} The spellwarped gains a \plus10 foot competence bonus to movement speed.
\subcf{Uncanny Dodge} The spellwarped may apply his Dexterity and dodge modifier to his armor class while flat-footed.
\subcf{7th -- Accelerate Attack} While his surge of power is active, the spellwarped can make an additional attack at a \minus5 penalty when making a full attack.

\cf{Spl}{Resist Magic (Ex)} At 4th level, the power of the magic with the spellwarped offers him some measure of protection against hostile magical effects. He gains a \plus1 competence bonus to saving throws against spells and spell-like abilities. This bonus increases by \plus1 at 8th level and every 4 levels thereafter.

\cf{Spl}{Manipulate Magic (Su)} At 5th level, the spellwarped can channel his innate magic to manipulate other forms of magic. Using this ability costs a spellwarp point. 
\subcf{Alteration -- Absorption} As an immediate action, when the spellwarped makes a successful Fortitude save against a spell or spell-like ability, he may absorb its energy harmlessly into his body. The spell has no effect on him, even if the spell would normally have an effect on a successful saving throw.
\subcf{Pyromancy -- Fuel the Flame} As an immediate action, when the spellwarped is affected by a spell or spell-like ability, he may channel its energy into a burst of flame around him. Creatures within a \areasmall radius of the spellwarped take fire damage equal to your spellwarped level. The spell still has its normal effect on the spellwarped.
\subcf{Telekinesis -- Mind over Matter} As an immediate action, when the spellwarped is affected by a spell or spell-like ability which allows a Fortitude save, he may make a Will save instead.
\subcf{Temporal -- Accelerate Magic} As a swift action, the spellwarped can increase or decrease the duration of any spell or spell-like ability affecting him by two rounds. This can end the effect immediately if it has no time remaining. The spellwarped can't increase the duration beyond twice the spell's original duration.

\cf{Spl}{Spell Resistance (Ex)} At 9th level, the magic within the spellwarped allows him to completely ignore other magic, granting him spell resistance. A creature with spell resistance may always make a saving throw when a spell is cast on it. The saving throw type is indicated by the spell. If it succeeds, the spell has no effect on it.

\cf{Spl}{Improved Manipulate Magic (Su)} At 13th level, the spellwarped can use his manipulate magic ability to affect any ally within \rngmed range of him.

\cf{Spl}{Mass Surge of Power (Su)} At 17th level, the spellwarped can share the benefits of his surge of power with his allies. When he uses his surge of power, he can also affect up to five additional creatures within \rngmed range of him.

\cf{Spl}{Permanent Surge of Power (Su)} At 19th level, the spellwarped can maintain the full power of his innate magic without limit. He can gain the effects of his surge of power indefinitely. He may toggle the ability on or off as a swift action at will, without expending spellwarp points. This does not allow him to activate his mass surge of power ability at will, and his allies only gain the benefits for 5 rounds.

\ssecfake{Spellwarped Powers}
As the spellwarped increases in level, he may choose more advanced spellwarped powers, listed below.

\subsubsection{Alteration Powers}
\parhead{1st -- Lesser Reduction} A creature within \rngclose range becomes one size category smaller for 2 rounds unless it makes a Fortitude save. It takes a \minus2 penalty to Strength, decreases its weapon damage dice by one size, and takes a \minus4 penalty to maneuvers and Maneuver Class. However, it gains a \plus2 bonus to Stealth checks and a \plus1 bonus to attack rolls and armor class. This is a size-affecting effect.
\parhead{4th -- Reduction} This power functions like the lesser reduction power, except that the foe is reduced for 5 rounds.
\parhead{6th -- Purge} The spellwarped can make a Fortitude save to end any effects or conditions on him which allow a Fortitude save to resist. If he can make a Spellcraft check to identify the effects on him, he may freely choose which effects not to end, allowing him to retain beneficial effects.
\parhead{8th -- Body Bludgeon} The spellwarped elongates and distorts a part of his body and strikes a foe with it. The foe must be within his reach, as if he were wielding a reach weapon. He must make an attack roll against the foe's AC. If he hits, he deals 1d6 bludgeoning damage per spellwarped level \add his Strength. In addition, whether he hits or misses, he may make a bull rush attack on the creature that does not provoke attacks of opportunity. He need not move with the creature to push it back.
\parhead{8th -- Enlargement} This power functions like the \spell{enlarge person} spell, except that it can affect creatures of any type.
\parhead{10th -- Amorphous Body} The spellwarped transforms his body into an amorphous form for 1 round, until the end of his next turn. In this form, he gains several benefits. He gains a \plus20 circumstance bonus against grapple attacks, is immune to critical hits, takes no penalties for squeezing, and can move through spaces that are no more than two inches in width, though doing so forces him to move at half speed.
\parhead{10th -- Heal Wounds} As a standard action, the spellwarped can spend two spellwarp points to remove his own injuries by transforming himself into a healthier version of his body. He heals 1d6 points of damage per spellwarped level. This also removes any of the following conditions: blinded, diseased, exhausted, fatigued, nauseated, sickened, and poisoned.
\parhead{12th -- Baleful Polymorph} This attack functions like the \spell{baleful polymorph} spell.
\parhead{14th -- Flight} As a swift action, the spellwarped can spend two spellwarp points to grow wings to fly for 5 rounds. His fly speed is equal to his base land speed, and his maneuverability is good. At the end of the duration, the wings are subsumed back into his body. This ability is draining to use, and the spellwarped must wait for five minutes after using it before he can use it again.
\parhead{14th -- Greater Amorphous Body} This power functions like the amorphous form power, except that it lasts for 5 rounds. The spellwarped must have the amorphous body power to select this power.
\parhead{16th -- Bludgeon the Horde} This attack functions like the body bludgeon attack, except that he may attack all foes within his reach, as if he were wielding a reach weapon. He deals 1d6 bludgeoning damage per two spellwarped levels \add his Strength to each foe.
\parhead{18th -- }
\parhead{20th -- }

\subsubsection{Pyromancy Powers}
\parhead{1st -- Lesser Ignite} As a standard action, the spellwarped can ignite a foe within \rngclose range, dealing 1d6 points of fire damage \add 1 per spellwarped level. A successful Reflex save halves the damage.
\parhead{1st -- Weapon of Flame} As a swift action, the spellwarped can create a weapon made of flame that lasts for 5 rounds. The weapon is sized appropriately for him, and may take the form of any weapon he is proficient with. He can attack with the weapon as if it were a normal weapon of its type, except that he adds half his Charisma to damage in place of half his Strength, and all damage dealt with the weapon is fire damage.
\par The flame weapon gains a \plus1 enhancement bonus to attack and damage at 4th level. At 7th level, and every 3 levels thereafter, the bonus increases by 1. If it leaves his hand, it is extinguished 1 round later.
\parhead{4th -- Ignite} This attack functions like the lesser ignite attack, except that it deals 1d6 points of fire damage per spellwarped level, and a target that fails its Reflex save is also ignited for 5 rounds. An ignited creature has been set on fire. It is vulnerable, causing it to take a \minus2 penalty to attack rolls, saving throws, checks, DCs, and AC. In addition, it takes d6 damage per round from the fire. If the creature takes a move action, it can attempt a DC 15 Reflex save to put out the flames. This action provokes attacks of opportunity. Dropping prone as part of the action gives a \plus4 circumstance bonus on this save.
\parhead{6th -- Ignite Weapon} As a swift action, the spellwarped can set one of his weapon on fire for 5 rounds. During this time, the spellwarped adds half his Charisma to damage with the weapon he wields in addition to half his Strength. This bonus damage is fire damage. If ignites a weapon created using his weapon of flame power, he adds his full Charisma to damage with the weapon instead of half his Charisma.
\parhead{6th -- Fiery Protection} As a standard action, the spellwarped can bestow fire and cold damage reduction equal to twice his spellwarped level on a creature within 30 feet of him. The protection lasts for 1 hour.
\parhead{8th -- Conflagration} As a standard action, the spellwarped can release a powerful explosion of flame. All creatures within a \areamed radius spread of him take 1d8 fire damage per two spellwarped levels. A successful Reflex save halves the damage.
\parhead{10th -- Fire Shield} As a standard action, the spellwarped can wreath himself in flame for 5 rounds. Any creature that hits him with its body or a melee weapon takes 1d6 fire damage per two spellwarped levels. Each individual creature can take this damage only once per round.
\parhead{10th -- Flameheart} As a standard action, the spellwarped can become a being of pure fire for 1 round. In this form, he is immune to physical damage and can pass through openings as small as one inch at no movement penalty. However, he cannot attack normally or use any of his items, as they meld into his body. He may invoke any of his spellwarped powers normally. In this form, he can make a touch attack as a standard action to deal 1d6 points of fire damage per spellwarped level.
\parhead{12th -- Firestride} As a move action, the spellwarped can may teleport to any active flame of at least Tiny size within \rngmed range. When he does so, he immolates himself and disappears into a pile of ash before stepping out from the flame unharmed. An ordinary torch is sufficient flame to teleport to, but not a candle.
\parhead{14th -- Flight of the Phoenix} As a swift action, the spellwarped can spend two spellwarp points to fly on wings of flame for 5 rounds. His fly speed is equal to his base land speed, and his maneuverability is good. At the end of the duration, the wings are extinguished. This ability is draining to use, and the spellwarped must wait for five minutes after using it before he can use it again.
\parhead{14th -- Greater Flameheart} This power functions like the flameheart power, except that it lasts for 5 rounds.
\parhead{16th -- } 
\parhead{18th -- Phoenix Revival} When the spellwarped takes critical damage, he may spend five spellwarp points as an immediate action, even if the critical damage would be sufficient to kill him. If he does, he ignores the critical damage he just took and dissolves into a pile of ash for 5 rounds. During this time, he can take no actions. If the pile of ash remains intact after 5 rounds, the spellwarped is restored to his normal body, with zero hit points but with all critical damage healed. However, if the pile of ash is dispersed, the spellwarped dies. The ash cannot be harmed by fire damage, and if the pile of ash would take at least 10 points of fire damage during a round, the spellwarped returns one round sooner. The spellwarped may take his normal actions immediately after being restored.
\parhead{20th -- Immolate} The spellwarped consumes the body of a foe within \rngclose range in flames from the inside out. It takes 1d6 points of fire damage per spellwarped level, and if it is bloodied after it takes this damage, it immediately dies. A successful Fortitude save halves the damage and leaves a bloodied creature with 0 hit points.

\subsubsection{Telekinesis Powers}
\parhead{1st -- Lesser Crush} As a standard action, the spellwarped can crush a creature or object within \rngclose range with telekinetic force, dealing 1d6 points of physical damage \add 1 per spellwarped level. A successful Fortitude save halves the damage.
\parhead{1st -- Mind Blade} As a swift action, the spellwarped can telekinetically wield an unattended weapon within \rngclose range. The weapon must be a light or medium weapon appropriate for his size. This allows him to attack with the weapon just as if he were holding it in his hand, except that he uses his Intelligence in place of his Strength. In all other respects, this functions as if he were wielding the weapon normally, including contributing to overwhelm penalties and taking attacks of opportunity. The weapon floats in midair and threatens all squares adjacent to it, and he may make attacks of opportunity with the weapon or with a weapon he wields in his hands, but not both.
As a move action, he may move the weapon up to 30 feet in any direction, even vertically. If the weapon goes outside of \rngclose range, he loses control of it and it falls to the ground.
\parhead{4th -- Crush} This power functions like the lesser crush attack, except that it deals 1d6 points of physical damage per spellwarped level, and a creature that fails its Fortitude save is also sickened. A sickened creature is vulnerable, causing it to take a \minus2 penalty to attack rolls, saving throws, checks, DCs, and AC.
\parhead{6th -- Distant Manipulation} As a standard action, the spellwarped can mentally exert influence at up to \rngclose range. This allows him to take any standard action which he could normally take with his hands, using his Intelligence in place of his Strength or Dexterity, as appropriate. He may take actions that require more than a standard action to complete by spending the same amount of time concentrating, spending one spellwarp point per two rounds that he spends concentrating.
\subcf{6th -- Dual Mind Blade} This power functions like his mind blade power, except that the spellwarped may wield two weapons at once. They must stay in the same space, and he may make two-weapon fighting attacks with the weapons, just as if he was wielding them with his hands. The spellwarped must have the mind blade power to select this power.
\parhead{8th -- }
\parhead{10th -- Telekinetic Force} This power functions like the \spell{telekinetic force} spell, using his Intelligence as his casting attribute.
\parhead{12th -- Strangle} The spellwarped crushes the windpipe of a foe within \rngclose range using his mind, dealing 1d6 damage per spellwarped level. A creature bloodied after the damage is dealt is nauseated and immobilized for 1 round. A successful Fortitude save halves the damage and negates the nausea. The spellwarped can maintain concentration on the ability as a standard action to deal additional damage and extend the duration of the nausea by 1 round, spending one spellwarp point per round.
\parhead{14th -- }
\parhead{16th -- } 
\parhead{18th -- }
\parhead{20th -- Mass Strangle} This attack functions like the strangle attack, except that the spellwarped can affect any creatures within a \areasmall radius.

\subsubsection{Temporal Powers}
\parhead{1st -- Lesser Slow} The spellwarped slows a foe within \rngclose range for 2 rounds unless it makes a Will save. A slowed creature can take a standard action or a move action each round, but not both. It cannot take full-round actions, but it may take swift actions. Additionally, it takes a \minus2 penalty to attack rolls, Strength and Dexterity-based checks, and armor class.
\parhead{4th -- Slow} This attack functions like the lesser slow attack, except that the foe is slowed for 5 rounds.
\parhead{6th -- Brief Acceleration} As a move action, the spellwarped can accelerate himself so much that he can seem to pause time for everyone but himself. This allows him to take a single move action. He does not provoke attacks of opportunity for any movement he makes during this time.
\parhead{8th -- Pause Time} The spellwarped completely stops time for a single creature within \rngclose range for 5 rounds. A successful Will save negates the effect. The affected creature can take no actions and cannot be moved, damaged, or even affected in any way until the effect ends.
\parhead{8th -- Brief Acceleration} As a move action, the spellwarped accelerates himself faster than his foes can react, allowing him to move up to his speed without provoking attacks of opportunity.
\parhead{10th -- }
\parhead{12th -- Swift Acceleration} This power functions like the brief acceleration power, except that it costs two spellwarp points and can be used as a swift action.
\parhead{12th -- Timestream} The spellwarped manipulates time in a \arealarge line that extends out from him for 5 rounds. All creatures and objects that pass through the line are slowed for 1 round, with no saving throw allowed. The spellwarped can exclude his allies from the effect. The timestream is virtually invisible, requiring a DC 30 Perception check to notice in a clear environment, though objects passing through the effect can make it obvious.
\parhead{14th -- }
\parhead{16th -- Inhuman Speed} As a move action, the spellwarped can accelerate himself to immense speed, allowing him to take a run action. He does not provoke attacks of opportunity for the movement, and is not flat-footed after running. After using this ability, the spellwarped must wait 5 rounds before he can use it again.
\parhead{16th -- Time Reversal} As a swift action, the spellwarped can spend a spellwarp point to create a ``time lock.'' The time lock persists for one round. As a standard action, he can bring a creature backwards through time to the point at which the time lock was created. A Will save negates this effect. An affected creature is perfectly restored to the point immediately after the time lock was created. The effects of any actions that the creature took in the intervening time are undone, any damage it dealt or took is removed, it is restored to its original location, and the creature is restored in all other ways, just as if the intervening time had never occured. The spellwarped cannot reverse time for himself in this way.
\parhead{18th -- Time Stop} As a standard action, the spellwarped can spend three spellwarp points to step into an alternate timestream, causing him to speed up so greatly that all other creatures seem frozen. He can take a single round of actions in this alternate timestream. All creatures he attacks are treated as helpless, but he cannot perform a coup de grace or similar ability; such an act requires more care and precision than is possible with such immense speed. After using this ability, he must wait 5 rounds before he can use it again.
\parhead{20th -- Sever Time} The spellwarped completely stops time for a single creature for 5 rounds. This attack functions like the pause time power, except that no saving throw is allowed.

\subsection{Wizard}
\begin{dtable*}
\lcaption{The Wizard}
\begin{tabularx}{\textwidth}{>{\ccol}p{\levelcol} >{\ccol}p{7em} *{3}{>{\ccol}p{\savecol}} >{\lcol}X}
\thead{Level} & \thead{Base Attack Bonus} & \thead{Fort Save} & \thead{Ref Save} & \thead{Will Save} & \thead{Special} \\
1st & \plus0 & \plus0 & \plus0 & \plus3 & Arcane invocation, ritual master \\
2nd & \plus1 & \plus1 & \plus1 & \plus4     & Arcane invocation, Scribe Scroll \\
3rd & \plus1 & \plus1 & \plus1 & \plus5     & Arcane insight \\
4th & \plus2 & \plus2 & \plus2 & \plus6     & Spell sequencer \\
5th & \plus2 & \plus2 & \plus2 & \plus7     & Arcane attunement (1 item), arcane insight \\
6th & \plus3 & \plus3 & \plus3 & \plus8     & \x \\
7th & \plus3 & \plus3 & \plus3 & \plus9     & Arcane insight \\
8th & \plus4 & \plus4 & \plus4 & \plus10    & Improved spell sequencer \\
9th & \plus4 & \plus4 & \plus4 & \plus11    & Arcane insight \\
10th & \plus5 & \plus5 & \plus5 & \plus12    & Arcane attunement (2 items) \\
11th & \plus5 & \plus5 & \plus5 & \plus13    & Arcane insight \\
12th & \plus6/\plus1 & \plus6 & \plus6 & \plus14& Contingency \\
13th & \plus6/\plus1 & \plus6 & \plus6 & \plus15& Arcane insight \\
14th & \plus7/\plus2 & \plus7 & \plus7 & \plus16& \x \\
15th & \plus7/\plus2 & \plus7 & \plus7 & \plus17& Arcane attunement (3 items), arcane insight \\
16th & \plus8/\plus3 & \plus8 & \plus8 & \plus18 & Versatile spell sequencer \\
17th & \plus8/\plus3 & \plus8 & \plus8 & \plus19 & Arcane insight \\
18th & \plus9/\plus4 & \plus9 & \plus9 & \plus20& \x \\
19th & \plus9/\plus4 & \plus9 & \plus9 & \plus21 & Arcane insight \\
20th & \plus10/\plus5 & \plus10& \plus10& \plus22 & Arcane attunement (4 items), chain contingency \\
\end{tabularx}
\end{dtable*}

\cd{Alignment} Any.
\cd{Hit Value} 4.

\sssecfake{Class Skills}
The wizard's class skills (and the key attribute for each skill) are
Knowledge (all skills, taken individually) (Int), Linguistics (Int), and Spellcraft (Wis).
\cf{Wiz}{Skill Points at 1st Level} 4

\sssecfake{Class Features}

All of the following are class features of the wizard.

\cf{Wiz}{Weapon and Armor Proficiency} Wizards are proficient with simple weapons, but not with any type of armor or shield. Armor of any type interferes with a wizard's movements, which can cause her spells with somatic components to fail.

\cf{Wiz}{Bonus Languages} A wizard may learn Draconic in addition to the bonus languages available to the character because of her race (see Chapter 2: Races). Many ancient tomes of magic are written in Draconic, and apprentice wizards often learn it as part of their studies.

\cf{Wiz}{Ritual Master} Wizards are thoroughly trained in study and memorization, and can perform difficult rituals with more ease than others. A wizard gains the Ritual Master feat as a bonus feat at 1st level, even if she does not have the prerequisites.

\cf{Wiz}{Spells} A wizard casts arcane spells using her Intelligence. To learn or cast a spell, a wizard must have an Intelligence at least equal to the spell's level. The Difficulty Class for a saving throw against a wizard's spell is 10 \add half the wizard's caster level \add the wizard's Intelligence.

Like other spellcasters, the number of spells a wizard knows and can cast each day is limited. These limitations are given below on \trefnp{Wizard Spells per Day} and \trefnp{Wizard Spells Known}. A wizard's spells are drawn from the common spells on the sorcerer/wizard spell list (see \pcref{Sorcerer/Wizard Spells}). Wizards need to rest for eight hours in order to regain spells. 

A wizard's magic level is equal to her wizard level.

\begin{dtable}
    \lcaption{Wizard Spells per Day} 
    \centering
    \begin{tabularx}{\columnwidth}{>{\ccol}X *{9}{>{\ccol}p{\spellcol}}}
        & \multicolumn{9}{c}{\thead{---{}---{}---{}---{}---{}---{}---{}---Spell Level---{}---{}---{}---{}---{}---{}---{}---}} \\
        \thead{Level} & \thead{1st} & \thead{2nd} & \thead{3rd} & \thead{4th} & \thead{5th} & \thead{6th} & \thead{7th} & \thead{8th} & \thead{9th} \\
        1st & 3 & \x & \x & \x & \x & \x & \x & \x & \x \\
        2nd & 4 & \x & \x & \x & \x & \x & \x & \x & \x \\
        3rd & 5 & \x & \x & \x & \x & \x & \x & \x & \x \\
        4th & 6 & 3 & \x & \x & \x & \x & \x & \x & \x \\
        5th & 6 & 4 & \x & \x & \x & \x & \x & \x & \x \\
        6th & 6 & 5 & 3 & \x & \x & \x & \x & \x & \x \\
        7th & 6 & 6 & 4 & \x & \x & \x & \x & \x & \x \\
        8th & 6 & 6 & 5 & 3 & \x & \x & \x & \x & \x \\
        9th & 6 & 6 & 6 & 4 & \x & \x & \x & \x & \x \\
        10th & 6 & 6 & 6 & 5 & 3 & \x & \x & \x & \x \\
        11th & 6 & 6 & 6 & 6 & 4 & \x & \x & \x & \x \\
        12th & 6 & 6 & 6 & 6 & 5 & 3 & \x & \x & \x \\
        13th & 6 & 6 & 6 & 6 & 6 & 4 & \x & \x & \x \\
        14th & 6 & 6 & 6 & 6 & 6 & 5 & 3 & \x & \x \\
        15th & 6 & 6 & 6 & 6 & 6 & 6 & 4 & \x & \x \\
        16th & 6 & 6 & 6 & 6 & 6 & 6 & 5 & 3 & \x \\
        17th & 6 & 6 & 6 & 6 & 6 & 6 & 6 & 4 & \x \\
        18th & 6 & 6 & 6 & 6 & 6 & 6 & 6 & 5 & 3 \\
        19th & 6 & 6 & 6 & 6 & 6 & 6 & 6 & 6 & 4 \\
        20th & 6 & 6 & 6 & 6 & 6 & 6 & 6 & 6 & 6 \\
    \end{tabularx}
\end{dtable}

\begin{dtable}
\lcaption{Wizard Spells Known}
\begin{tabularx}{\columnwidth}{>{\ccol}X *{9}{>{\ccol}p{\spellcol}}}
& \multicolumn{9}{c}{\thead{---{}---{}---{}---{}---{}---{}---{}---Spell Level---{}---{}---{}---{}---{}---{}---{}---}} \\
\thead{Level} & \thead{1st} & \thead{2nd} & \thead{3rd} & \thead{4th} & \thead{5th} & \thead{6th} & \thead{7th} & \thead{8th} & \thead{9th} \\
1st  & 1 & \x & \x & \x & \x & \x & \x & \x & \x \\
2nd  & 2 & \x & \x & \x & \x & \x & \x & \x & \x \\
3rd  & 3 & \x & \x & \x & \x & \x & \x & \x & \x \\
4th  & 3 & 1 & \x & \x & \x & \x & \x & \x & \x \\
5th  & 4 & 2 & \x & \x & \x & \x & \x & \x & \x \\
6th  & 4 & 2 & 1 & \x & \x & \x & \x & \x & \x \\
7th  & 4 & 3 & 2 & \x & \x & \x & \x & \x & \x \\
8th  & 4 & 3 & 2 & 1 & \x & \x & \x & \x & \x \\
9th  & 4 & 3 & 3 & 2 & \x & \x & \x & \x & \x \\
10th & 4 & 3 & 3 & 2 & 1 & \x & \x & \x & \x \\
11th & 4 & 3 & 3 & 3 & 2 & \x & \x & \x & \x \\
12th & 4 & 3 & 3 & 3 & 2 & 1 & \x & \x & \x \\
13th & 4 & 3 & 3 & 3 & 3 & 2 & \x & \x & \x \\
14th & 4 & 3 & 3 & 3 & 3 & 2 & 1 & \x & \x \\
15th & 4 & 3 & 3 & 3 & 3 & 3 & 2 & \x & \x \\
16th & 4 & 3 & 3 & 3 & 3 & 3 & 2 & 1 & \x \\
17th & 4 & 3 & 3 & 3 & 3 & 3 & 2 & 2 & \x \\
18th & 4 & 3 & 3 & 3 & 3 & 3 & 2 & 2 & 1 \\
19th & 4 & 3 & 3 & 3 & 3 & 3 & 2 & 2 & 2 \\
20th & 4 & 3 & 3 & 3 & 3 & 3 & 2 & 2 & 2
\end{tabularx}
\end{dtable}

\cf{Wiz}{Arcane Invocation} All wizards master at least one arcane invocation. An arcane invocation allows the wizard to exert magical influence without expending the effort required to cast a spell. The wizard may choose to learn one invocation of her choice from the list of arcane invocations described in Chapter 11: Spells. Specialist wizards must choose one of the invocations granted by their specialist school. At 2nd level, the wizard gains a second arcane invocation, which can be chosen from any non-prohibited school.

\cf{Wiz}{Arcane Insight (Ex)} At 3rd level, and every odd level afterwards, a wizard gains a greater understanding of magic. Generalist wizards gain expanded spell knowledge, as the sorcerer class feature. Specialist wizards may choose a spell of their chosen school from the sorcerer/wizard spell list, including restricted spells, and add it to their spells known. The spell's level must not be higher than half the wizard's class level -- normally, the highest level of spells that the wizard can cast.

\cf{Wiz}{Spell Sequencer (Ex)} At 4th level, a wizard gains the ability to create a sequence of a spell and invocation which she can cast rapidly later. To create a spell sequencer, the wizard must cast a spell which affects only herself and an arcane invocation, which may affect any target. Neither has any effect immediately. The wizard may later use a full-round action to cast both the spell and the invocation at once, choosing the target of the invocation at that time.
\par The wizard may have only one spell sequencer active at any time. If she creates a new spell sequencer, it replaces her existing spell sequencer.

\cf{Wiz}{Arcane Attunement (Su)} At 5th level, a wizard gains the ability to use arcane items like scrolls and wands with particular skill. At the beginning of each day, the wizard may attune himself to a single item with a spell trigger or spell completion activation method. If he does, he gains one of several benefits, depending on the type of magic item he attunes himself to. At 10th level, and every five levels thereafter, the wizard gains the ability to attune herself to an additional item.
\subparhead{Scroll} The wizard casts the spell from the scroll at his full caster level if it is higher than the scroll's caster level, and using his full attribute.
\subparhead{Wand} The wizard casts the spell from the wand at his full caster level if it is higher than the wand's caster level. In addition, once per day, the wizard may cast a spell from the wand using a spell slot of the appropriate level instead of a charge.
\subparhead{Staff} The wizard may cast spells from the staff using spell slots of the appropriate level instead charges. If the spell would require multiple charges when cast from the staff, the spell slot consumed must be one level higher than the spell's normal level for each charge beyond the first.
\par If other items exist that can produce similar effects, the wizard may gain appropriate bonuses for attuning himself to them, using the above examples as a guide.

\cf{Wiz}{Improved Spell Sequencer (Ex)} At 8th level, a wizard gains the ability to create a sequence of two spells which she can cast rapidly later. To create an improved spell sequencer, the wizard must cast two spells, one of which affects only herself. The spells must be at least three levels apart. Neither has any effect immediately. the wizard may later use a full-round action to cast both spells at once.
\par The wizard may have only one spell sequencer or improved spell sequencer active at any time.

\section{Character Advancement}

\begin{dtable}
\lcaption{Character Advancement}
\begin{tabularx}{\columnwidth}{*{5}{>{\ccol}X}}
  \thead{Character level} & \thead{XP} & \thead{Feats} & \thead{Attribute Increases\fn{1}} \\
1st & 0 & 1st & \x & \x \\
2nd & 2,000 & \x & 1st \\
3rd & 5,000 & 2nd & \x \\
4th & 9,000 & \x & 2nd \\
5th & 15,000 & 3rd & \x \\
6th & 23,000 & \x & 3rd \\
7th & 35,000 & 4th & \x \\
8th & 51,000 & \x & 4th \\
9th & 75,000 & 5th & \x \\
10th & 105,000 & \x & 5th \\
11th & 155,000 & 6th & \x \\
12th & 220,000 & \x & 6th \\
13th & 315,000 & 7th & \x \\
14th & 445,000 & \x & 7th \\
15th & 635,000 & 8th & \x \\
16th & 890,000 & \x & 8th \\
17th & 1,300,000 & 9th & \x \\
18th & 1,800,000 & \x & 9th \\
19th & 2,550,000 & 10th & \x \\
20th & 3,600,000 & \x & 10th
\end{tabularx}
1. The same attribute cannot be increased twice in a row.
\end{dtable}
