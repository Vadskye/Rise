\chapter{Monsters}

\section{Monster Attributes}

\subsection{Archetypes}

Monsters come in many shapes and sizes. However, their roles in combat can be described through the use of archetypes -- descriptive keywords which provide guidance as to a monster's abilities. An individual monster can have any number of these descriptive keywords.

\subsubsection{Ambush}
This kind of creature typically makes surprise attacks from a hidden position. Ambush creatures usually have a form of stealth.

\subsubsection{Artillery}
This kind of creature can attack readily from afar.

\subsubsection{Brute}
This kind of creature can take a lot of physical punishment, such as by having a high armor class, special abilities, or a lot of hit points

\subsubsection{Conditional}
Not an archetype, but a type of archetype. Always used in combination with another archetype, this means that the creature's role in combat can be countered by some specific strategy or item. For example, incorporeal creatures are conditional brutes; they are invulnerable to nonmagical weapons, but generally very weak against ghost touch weapons.

\subsubsection{Disabler}
This kind of creature can hinder opponents, such as by grappling or through the use of debilitating special abilities.

\subsubsection{Leader}
This kind of creature benefits from working in concert with other creatures.

\subsubsection{Mobile}
This kind of creature can move around the battlefield easily, such as by having a high movement speed or through special abilities related to movement.

\subsubsection{Nondamaging}
This kind of creature does not usually deal damage to creatures' hit points. Such creatures often affect their opponents in other ways, such as with ability damage or drain.

\subsubsection{Preparation}
This kind of creature has special abilities which it is likely to use before combat. Most preparation monsters have spells or spell-like abilities which they cast on themselves or their local environment.

\subsubsection{Striker}
This kind of creature can deal a lot of damage, whether through physical combat or special abilities.

\subsubsection{Utility}
This kind of creature has significant noncombat special abilities which it is likely to use as appropriate for the situation.

\subsection{Modifiers}

Sometimes, it is thematically appropriate for monsters to be a little different, depending on the situation. Any monster can have one of these modifiers applied. Each modifier changes something about a monster's statistics.

\subsubsection{Minion}
A minion has half the hit points or a normal monster of its type. In addition, a minion always deals average damage when it hits.

\subsubsection{Challenge Rating}

A typical encounter involves the same number of monsters as PCs. Each monster in a typical encounter has a CR equal to the level of the PCs.

\section{Monsters}

\begin{comment}

\subsection{Example}
\montypes{Align}{Size}{type}{CR}{Archetypes}
%\montypessubtypes{Align}{Size}{type}{subtypes}{CR}{Archetypes}
\monsenses{Init}{\plus Perception}
%\monsensesfull{Init}{\plus Perception}{\plus Spellcraft}{Special senses}
\monspace{Space}{Reach}; \monspeed{Speed ft.}%;\monaura{Aura}
\monlanguages{Languages}

\monlinerule 

\monac{Normal}{Touch}{Flat}{CMD}{modifiers}
\monhp{HP}{HV}
%\monimmune{immunities}
%\monresist{resistances}
\monsaves{Fort}{Ref}{Will}

\monlinerule

\monmelee{Melee}
\moncmb{BAB}{CMB}
%\monsa{}

\monlinerule 

\monattributes{}{}{}{}{}{}
\monfeats{Feats}
\monskills{Skills}
%\monitems{Items}

\monlinerule

\monability{Name}{Effect}

\mondescription{Monster Name}

\monbehavior{Monster Name}

\end{comment}

\subsection{Human Bandit}
\montypes{Usually evil}{Medium}{humanoid}{2}{Archetypes}
%\montypessubtypes{Align}{Size}{type}{subtypes}{CR}{Archetypes}
\monsenses{\plus1}{\plus1}
%\monsensesfull{Init}{\plus Perception}{\plus Spellcraft}{Special senses}
\monspace{5 ft.}{5 ft.}; \monspeed{30 ft.}%;\monaura{Aura}
\monlanguages{Common}

\monlinerule 

\monac{17}{14}{13}{17}{\plus3 studded leather, \plus1 Dexterity, \plus1 dodge, \plus2 heavy wooden shield}
\monhp{10}{2}
%\monimmune{immunities}
%\monresist{resistances}
\monsaves{6}{1}{1}

\monlinerule

\monmelee{Longsword \plus4 (d8+1)}
\moncmb{\plus2}{\plus4}
%\monsa{}

\monlinerule 

\monattributes{2}{1}{2}{0}{-1}{0}
%\monfeats{Feats}
\monskills{Perception \plus1}
%\monitems{Items}

\monlinerule

\mondescription{Human Bandit}

\monbehavior{Human Bandit}


\subsection{Human Militia}
\montypes{Varies}{Medium}{humanoid}{1}{}
\monsenses{0}{\plus0}
\monspace{5}{5}; \monspeed{30 ft.}%;\monaura{Aura}
\monlanguages{Common}

\monlinerule 

\monac{15}{13}{12}{14}{\plus2 armor, \plus2 shield, \plus1 Dexterity}
\monhp{5}{1}
\monsaves{3}{1}{0}

\monlinerule

\monmelee{Longsword \plus2 (d8)}
\moncmb{1}{1}
%\monsa{}

\monlinerule 

\monattributes{1}{1}{1}{0}{0}{0}
%\monfeats{}
%\monskills{}
\monitems{Leather armor, longsword, heavy wooden shield}

\monlinerule

\mondescription{Human Warrior}

\monbehavior{Human Warrior}

\subsection{Kobold Warrior}
\montypessubtypes{LE}{Small}{humanoid}{reptilian}{1}{Ambush}
\monsensesspecial{3}{\plus 1}{Low-light vision}
%\monsensesfull{Init}{\plus Perception}{\plus Spellcraft}{Special senses}
\monspace{5}{5}; \monspeed{30 ft.}%;\monaura{Aura}
\monlanguages{Draconic}

\monlinerule 

\monacspecial{16}{14}{14}{10}{\plus4 vs attacks of opportunity from Dodge target}{\plus1 size, \plus3 Dex, \plus2 armor, \plus1 natural}
\monhp{4}{1}
%\monimmune{immunities}
%\monresist{resistances}
\monsaves{0}{3}{\minus3}

\monlinerule

\monmelee{Shortspear \plus4 (d4-1)}
\monrange{Shortspear (20 ft.) \plus4 (d4-1)}
\moncmb{1}{\minus5}

\monlinerule 

%Original: 0,2,0,-1,1,-2
%Racial: -2 str, -1 con, +1 dex
\monattributes{\minus2}{3}{\minus1}{\minus1}{1}{\minus2}
\monfeats{Dodge}
\monskills{Jump \plus0, Escape Artist \plus3, Craft (trapmaking) \minus1}
\monitems{Leather, shortspear}

\monlinerule

\mondescription{Kobold Warrior}
Kobolds are short, reptilian humanoids with cowardly and sadistic tendencies. A kobold's scaly skin ranges from dark rusty brown to a rusty black color. It has glowing red eyes. Its tail is nonprehensile. Kobolds wear ragged clothing, favoring red and orange. A kobold is 2 to 2-1/2 feet tall and weighs 35 to 45 pounds. Kobolds speak with a voice that sounds like that of a yapping dog.

\monbehavior{Kobold Warrior}
Kobolds like to attack with overwhelming odds -- at least two to one -- or trickery; should the odds fall below this threshold, they usually flee. However, they attack gnomes on sight if their numbers are equal.
They begin a fight by slinging bullets, closing only when they can see that their foes have been weakened. Whenever they can, kobolds set up ambushes near trapped areas.

\subsection{Bear, Black}
\montypes{TN}{Medium}{animal}{4}{Brute}
%\montypessubtypes{Align}{Size}{type}{subtypes}{CR}{Archetypes}
\monsenses{\plus1}{\plus Perception; scent}
%\monsensesfull{Init}{\plus Perception}{\plus Spellcraft}{Special senses}
\monspace{5 ft.}{5 ft.}; \monspeed{30 ft.}%;\monaura{Aura}

\monlinerule 

\monac{Normal}{Touch}{Flat}{CMD}{modifiers}
\monhp{32}{4} %HV is 6
%\monimmune{immunities}
%\monresist{resistances}
\monsaves{Fort}{Ref}{Will}

\monlinerule

\monmelee{Claws \plus7 (d10+2/d10)}
\moncmb{\plus3}{\plus7}
%\monsa{}

\monlinerule 

\monattributes{4}{1}{4}{\minus5}{0}{0}
\monfeats{Feats}
\monskills{Climb \plus16, Stealth \plus5, Perception \plus8}

\monlinerule

\monability{Name}{Effect}

\mondescription{Example}

\monbehavior{Example}
\section{Monster Feats}

These feats apply to abilities most commonly found amongst monsters or are related to monsters.

\subsection{Ability Focus [Monstrous]}
Choose one of the creature's special attacks.
\parhead{Prerequisite:} Special attack.
\parhead{Benefit:} The creature gains a \plus2 competence bonus to the DC for all saving throws against the special attack chosen.
\parhead{Special:} A creature can gain this feat multiple times. Its effects do not stack. Each time the creature takes the feat, it applies to a different special attack.

\subsection{Awesome Blow [Combat, Monstrous]}
\parhead{Prerequisites:} Str 25, Power Attack, Improved Bull Rush, size Large or larger.
\parhead{Benefit:} As a standard action, the creature may choose \my{make a single attack to} deliver an awesome blow. If the creature hits a corporeal opponent smaller than itself with an awesome blow, \my{it may make a bull rush attack as a swift action, adding the damage dealt on the attack as a circumstance bonus. An affected creature is sent} flying \my{a number of feet based on the check result} in a direction of the attacking creature's choice and falls prone. The attacking creature can only push the opponent in a straight line, and the opponent can't move closer to the attacking creature than the square it started in. If an obstacle prevents the completion of the opponent's move, the opponent and the obstacle each take 1d6 points of damage \my{per 5 feet of movement remaining}, and the opponent stops in the space adjacent to the obstacle.

\subsection{Craft Construct [Item Creation]}
\parhead{Prerequisites:} Craft Magic Arms and Armor, Craft Wondrous Item.
\parhead{Benefit:} A creature with this feat can create any construct whose prerequisites it meets. Enchanting a construct takes one day for each 1,000 gp in its market price. To enchant a construct, a spellcaster must spend 1/25 the item's price in XP and use up raw materials costing half of this price (see individual construct monster entries for details).

A creature with this feat can repair constructs that have taken damage. In one day of work, the creature can repair up to 20 points of damage by expending 50 gp per point of damage repaired.

A newly created construct has average hit points for its Hit Dice.

\subsection{Empower Spell-Like Ability [Monstrous]}
\parhead{Prerequisite:} Spell-like ability at caster level 6th or higher.
\parhead{Benefit:} Choose one of the creature's spell-like abilities, subject to the restrictions below. The creature can use that ability as an empowered spell-like ability three times per day (or less, if the ability is normally usable only once or twice per day).

When a creature uses an empowered spell-like ability, all variable, numeric effects of the spell-like ability are increased by one half. Saving throws and opposed rolls are not affected. Spell-like abilities without random variables are not affected.

The creature can only select a spell-like ability duplicating a spell with a level less than or equal to half its caster level (round down) \minus2. For a summary, see the table in the description of the Quicken Spell-Like Ability feat. 

\parhead{Special:} This feat can be taken multiple times. Each time it is taken, the creature can apply it to a different one of its spell-like abilities.

\subsection{Flyby Attack [General]}
\parhead{Prerequisite:} Fly speed.
\parhead{Benefit:} When flying, the creature can take a move action (including a dive) and another standard action at any point during the move. The creature cannot take a second move action during a round when it makes a flyby attack.
\parhead{Normal:} Without this feat, the creature takes a standard action either before or after its move.

\subsection{Hover [Monstrous]}
\parhead{Prerequisite:} Fly speed.
\parhead{Benefit:} When flying, the creature can halt its forward motion and hover in place as a move action. It can then fly in any direction, including straight down or straight up, at half speed, regardless of its maneuverability.

If a creature begins its turn hovering, it can hover in place for the turn and take a full-round action. A hovering creature cannot make wing attacks, but it can attack with all other limbs and appendages it could use in a full attack. The creature can instead use a breath weapon or cast a spell instead of making physical attacks, if it could normally do so.

If a creature of Large size or larger hovers within 20 feet of the ground in an area with lots of loose debris, the draft from its wings creates a hemispherical cloud with a radius of 60 feet. The winds so generated can snuff torches, small campfires, exposed lanterns, and other small, open flames of non-magical origin. Clear vision within the cloud is limited to 10 feet. Creatures have concealment at 15 to 20 feet (20\% miss chance). At 25 feet or more, creatures have total concealment (50\% miss chance, and opponents cannot use sight to locate the creature).

Those caught in the cloud must succeed on a Concentration check (DC 10 \add \my{creature's special size modifier} \my{\add double spell level}) to cast a spell.

\parhead{Normal:} Without this feat, a creature must keep moving while flying unless it has perfect maneuverability.

\subsection{Improved Natural Armor [Monstrous]}
\parhead{Prerequisites:} Natural armor, Con 13.
\parhead{Benefit:} The creature gains a \plus1 competence bonus to natural armor class.
\parhead{Special:} A creature can gain this feat multiple times. Each time the creature takes the feat its \my{bonus to natural armor} increases by another point \my{and the Con requirement increases by 2}.

\subsection{Improved Natural Attack [Monstrous]}
\parhead{Prerequisite:} Natural weapon, base attack bonus \plus4.
\parhead{Benefit:} Choose one of the creature's natural attack forms. The damage for \my{all of its natural weapons of that type} increases by one step: 1d2, 1d3, 1d4, 1d6, 1d8, \my{d10, 2d6, 2d8, 2d10, 4d6, 4d8, 4d10, 8d6, 8d8, 8d10}.
\parhead{Special:} A creature can gain this feat multiple times. Each time the creature takes the feat, it applies to a different natural weapon.

\subsection{Multiattack [Combat, Monstrous]}
\parhead{Prerequisite:} Two or more natural weapons of the same type.
\parhead{Benefit:} The creature gains a \plus2 circumstance bonus to attack when making flurry attacks with natural weapons.
\parhead{Normal:} Without this feat, the creature's \my{flurry} attacks with natural weapons take a \my{\minus2} penalty, \my{or no penalty if the natural weapons are light}.

\subsection{Multiweapon Fighting [Combat, Monstrous]}
\parhead{Prerequisites:} \my{Dex 15}, three or more hands.
\parhead{Benefit:} The creature can make flurry attacks when wielding two or more manufactured weapons of the same type as if they were natural weapons.
\my{\parhead{Normal:}} A creature without this feat \my{can only make flurry attacks with natural weapons of the same type}.
\parhead{Special:} This feat replaces the Two-Weapon Fighting feat for creatures with more than two arms.

\subsection{Quicken Spell-Like Ability [Monstrous]}
\parhead{Prerequisite:} Spell-like ability at caster level 10th or higher.
\parhead{Benefit:} Choose one of the creature's spell-like abilities, subject to the restrictions described below. The creature can use that ability as a quickened spell-like ability three times per day (or less, if the ability is normally usable only once or twice per day).

Using a quickened spell-like ability is a \my{swift} action that does not provoke an attack of opportunity. The creature can perform another action -- including the use of another spell-like ability -- in the same round that it uses a quickened spell-like ability. The creature may use only one quickened spell-like ability per round.

The creature can only select a spell-like ability duplicating a spell with a level less than or equal to half its caster level (round down) \minus4. For a summary, see the table below.

In addition, a spell-like ability that duplicates a spell with a casting time greater than 1 full round cannot be quickened.

\parhead{Normal:} Normally the use of a spell-like ability requires a standard action and provokes an attack of opportunity unless noted otherwise.

\parhead{Special:} This feat can be taken multiple times. Each time it is taken, the creature can apply it to a different one of its spell-like abilities.

\begin{dtable}
\lcaption{Empower and Quicken Spell-Like Ability}
\begin{tabularx}{\columnwidth}{l X X}
\thead{Spell Level} & \thead{Caster Level to Empower} & \thead{Caster Level to Quicken} \\
0 & 4th & 8th \\
1st & 6th & 10th \\
2nd & 8th & 12th \\
3rd & 10th & 14th \\
4th & 12th & 16th \\
5th & 14th & 18th \\
6th & 16th & 20th \\
7th & 18th & \x \\
8th & 20th & \x \\
9th & \x & \x
\end{tabularx}
\end{dtable}

\subsection{Snatch [General]}
\parhead{Prerequisite:} Size Huge or larger.
\parhead{Benefits:} The creature can choose to start a grapple \my{as a swift action} when it hits with a claw or bite attack, as though it had the improved grab special attack. If the creature gets a hold on a creature three or more sizes smaller, it squeezes each round for automatic bite or claw damage. A snatched opponent held in the creature's mouth is not allowed a Reflex save against the creature's breath weapon, if it has one.

The creature can drop a creature it has snatched as a free action or use a standard action to fling it aside. A flung creature travels 1d6 \mtimes 10 feet, and takes 1d6 points of damage per 10 feet traveled. If the creature flings a snatched opponent while flying, the opponent takes this amount or falling damage, whichever is greater.


\subsection{Versatile Multiweapon Fighting [Combat, Monstrous]}
\parhead{Prerequisite:} Dex 17, Multiweapon Fighting
\parhead{Benefits:} The creature can make flurry attacks with any combination of manufactured weapons as if they were natural weapons of the same type. All of the weapons used must be light to gain the benefits of using light weapons when flurrying.


\subsection{Wingover [Monstrous]}
\parhead{Prerequisite:} Fly speed.
\parhead{Benefits:} A flying creature with this feat can change direction quickly once each round as a free action. This feat allows it to turn up to 180 degrees regardless of its maneuverability, in addition to any other turns it is normally allowed. A creature cannot gain altitude during a round when it executes a wingover, but it can dive.

The change of direction consumes 10 feet of flying movement.

\section{Types, Subtypes, and Abilities}


\subsection{Monster Types}
\subsubsection{Aberration Type} An aberration has a bizarre anatomy, strange abilities, an alien mindset, or any combination of the three.
\subparhead{Features} An aberration has the following features.
\begin{itemize*}
\item Hit Value 5.
\item Base attack bonus equal to 3/4 total \my{Hit Values} (average progression).
\item Good Will saves.
\item 4 skill points. The following are class skills for aberrations: Climb, Jump, Swim, Hide, Move Silently, Knowledge (any one), Spellcraft, Listen, Spot, Survival, Intimidate 
\end{itemize*}
\subparhead{Traits} An aberration possesses the following traits (unless otherwise noted in a creature's entry).
\begin{itemize*}
\item Darkvision out to 60 feet.
\item Proficient with its natural weapons. If generally humanoid in form, proficient with all simple weapons and any weapon \my{group} it is described as using.
\item Proficient with whatever type of armor (light, medium, or heavy) it is described as wearing, as well as all lighter types. Aberrations not indicated as wearing armor are not proficient with armor. Aberrations are proficient with shields if they are proficient with any form of armor.
\item Aberrations eat, sleep, and breathe.
\end{itemize*}

\subsubsection{Animal Type} An animal is a living, nonhuman creature, usually a vertebrate with no magical abilities and no innate capacity for language or culture.
\subparhead{Features} An animal has the following features (unless otherwise noted in a creature's entry).
\begin{itemize*}
\item Hit Value 5
\item Base attack bonus equal to \my{Hit Values (good progression)}.
\item Good Fortitude and Reflex saves (certain animals have different good saves).
\item 2 skill points. The following are class skills for animals: Climb, Jump, Swim, Balance, Hide, Move Silently, Listen, Spot, Survival
\end{itemize*}
\subparhead{Traits} An animal possesses the following traits (unless otherwise noted in a creature's entry).
\begin{itemize*}
\item Intelligence score of 1 or 2 (no creature with an Intelligence score of 3 or higher can be an animal).
\item Low-light vision.
\item Alignment: Always neutral.
\item Treasure: None.
\item Proficient with its natural weapons only. A noncombative herbivore \my{is not proficient with its natural weapons}.
\item Proficient with no armor unless trained for war.
\item Animals eat, sleep, and breathe.
\end{itemize*}

\subsubsection{Construct Type} A construct is an animated object or artificially constructed creature.
\subparhead{Features} A construct has the following features.
\begin{itemize*}
\item Hit Value 5.
\item Base attack bonus equal to 3/4 total \my{Hit Values} (average progression).
\item No good saving throws.
\item 2 skill points. However, most constructs are mindless and gain no skill points or feats. Constructs do not have any class skills, regardless of their Intelligence scores.
\end{itemize*}
\subparhead{Traits} A construct possesses the following traits (unless otherwise noted in a creature's entry).
\begin{itemize*}
\item No Constitution score.
\item Low-light vision.
\item Darkvision out to 60 feet.
\item Immunity to all mind-affecting effects (charms, compulsions, phantasms, patterns, and morale effects).
\item Immunity to poison, sleep effects, paralysis, stunning, disease, death effects, and necromancy effects.
\item Cannot heal damage on their own, but often can be repaired by exposing them to a certain kind of effect (see the creature's description for details) or through the use of the Craft Construct feat. A construct with the fast healing special quality still benefits from that quality.
\item Not subject to critical hits, nonlethal damage, ability damage, ability drain, fatigue, exhaustion, or energy drain.
\item Immunity to any effect that requires a Fortitude save (unless the effect also works on objects, or is harmless).
\item Not at risk of death from massive damage. Immediately destroyed when reduced to 0 hit points or less.
\item Since it was never alive, a construct cannot be raised or resurrected.
\item Because its body is a mass of unliving matter, a construct is hard to destroy. It gains bonus hit points based on size, as shown on the following table.
\item Proficient with its natural weapons only, unless generally humanoid in form, in which case proficient with any weapon mentioned in its entry.
\item Proficient with no armor.
\item Constructs do not eat, sleep, or breathe.
\end{itemize*}
\begin{dtable}
\begin{tabularx}{\columnwidth}{l >{\lcol}X l >{\lcol}X}
\thead{Construct Size} & \thead{Bonus Hit Points} & \thead{Construct Size} & \thead{Bonus Hit Points Per HV} \\
Fine       & \x & Large      & 4 \\
Diminutive & \x & Huge       & 6 \\
Tiny       & \x & Gargantuan & 8 \\
Small      & 1  & Colossal   & 10 \\
Medium     & 2  &            &
\end{tabularx}
\end{dtable}

\subsubsection{Dragon Type} A dragon is a reptilelike creature, usually winged, with magical or unusual abilities.
\subparhead{Features} A dragon has the following features.
\begin{itemize*}
\item Hit Value 7
\item Base attack bonus equal to total \my{Hit Values} (good progression).
\item Good Fortitude and Will saves.
\item 8 skill points. The following are class skills for dragons: Climb, Jump, Swim, Hide, Move Silently, Concentration, Appraise, Craft, Knowledge (all), Speak Language, Heal, Listen, Sense Motive, Spot, Survival, Bluff, Diplomacy, Intimidate, Use Magic Device
\end{itemize*}
\subparhead{Traits} A dragon possesses the following traits (unless otherwise noted in the description of a particular kind).
\begin{itemize*}
\item Darkvision out to 60 feet and low-light vision.
\item Immunity to magic sleep effects and paralysis effects.
\item Proficient with its natural weapons only unless humanoid in form (or capable of assuming humanoid form), in which case proficient with all simple weapons and any weapons mentioned in its entry.
\item Proficient with no armor.
\item Dragons eat, sleep, and breathe.
\end{itemize*}

\subsubsection{Fey Type} A fey is a creature with supernatural abilities and connections to nature or to some other force or place. Fey are usually human-shaped.
\subparhead{Features} A fey has the following features.
\begin{itemize*}
\item Hit Value 5
\item Base attack bonus equal to 3/4 total \my{Hit Values} (average progression).
\item Good Reflex and Will saves.
\item 8 skill points. The following are class skills for fey: Climb, Jump, Swim, Escape Artist, Hide, Move Silently, Sleight of Hand, Concentration, Craft, Knowledge (geography, local, nature), Listen, Sense Motive, Spot, Bluff, Diplomacy, Perform, Use Magic Device
\end{itemize*}
\subparhead{Traits} A fey possesses the following traits (unless otherwise noted in a creature's entry).
\begin{itemize*}
\item Low-light vision.
\item Proficient with all simple weapons and any weapons mentioned in its entry.
\item Proficient with whatever type of armor (light, medium, or heavy) that it is described as wearing, as well as all lighter types. Fey not indicated as wearing armor are not proficient with armor. Fey are proficient with shields if they are proficient with any form of armor.
\item Fey eat, sleep, and breathe.
\end{itemize*}

\subsubsection{Giant Type} A giant is a humanoid-shaped creature of great strength, usually of at least Large size.
\subparhead{Features} A giant has the following features.
\begin{itemize*}
\item Hit Value 6
\item Base attack bonus equal to total \my{Hit Values} (good progression).
\item Good Fortitude saves.
\item 2 skill points. The following are class skills for giants: Climb, Jump, Swim, Listen, Spot, Intimidate
\end{itemize*}
\subparhead{Traits} A giant possesses the following traits (unless otherwise noted in a creature's entry).
\begin{itemize*}
\item Low-light vision.
\item Proficient with all simple and martial weapons, as well as any natural weapons.
\item Proficient with whatever type of armor (light, medium or heavy) it is described as wearing, as well as all lighter types. Giants not described as wearing armor are not proficient with armor. Giants are proficient with shields if they are proficient with any form of armor.
\item Giants eat, sleep, and breathe.
\end{itemize*}

\subsubsection{Humanoid Type} A humanoid usually has two arms, two legs, and one head, or a humanlike torso, arms, and a head. Humanoids have few or no supernatural or extraordinary abilities, but most can speak and usually have well-developed societies. They usually are Small or Medium. Every humanoid creature also has a subtype.
\par Humanoids with 1 Hit Die exchange the features of their humanoid Hit Die for the class features of a PC or NPC class. Humanoids of this sort are presented as 1st-level warriors, which means that they have average combat ability and poor saving throws.
\par Humanoids with more than 1 Hit Die are the only humanoids who make use of the features of the humanoid type.
\subparhead{Features} A humanoid has the following features (unless otherwise noted in a creature's entry).
\begin{itemize*}
\item Hit Value 5, or by character class.
\item Base attack bonus equal to 3/4 total \my{Hit Values} (average progression).
\item Good Will saves (usually; a humanoid's good save varies).
\item 4 skill points. The following are class skills for humanoids without a character class: Climb, Swim, Ride, Craft, Heal, Survival, Handle Animal
\end{itemize*}
\subparhead{Traits} A humanoid possesses the following traits (unless otherwise noted in a creature's entry).
\begin{itemize*}
\item Proficient with all simple weapons, or by character class.
\item Proficient with whatever type of armor (light, medium, or heavy) it is described as wearing, or by character class. If a humanoid does not have a class and wears armor, it is proficient with that type of armor and all lighter types. Humanoids not indicated as wearing armor are not proficient with armor. Humanoids are proficient with shields if they are proficient with any form of armor.
\item Humanoids breathe, eat, and sleep.
\end{itemize*}

\subsubsection{Magical Beast Type} Magical beasts are similar to animals but can have Intelligence scores higher than 2. Magical beasts usually have supernatural or extraordinary abilities, but sometimes are merely bizarre in appearance or habits.
\subparhead{Features} A magical beast has the following features.
\begin{itemize*}
\item Hit Value 6
\item Base attack bonus equal to total \my{Hit Values} (good progression).
\item Good Fortitude and Reflex saves.
\item 2 skill points. The following are class skills for magical beasts: Climb, Jump, Swim, Balance, Hide, Move Silently, Listen, Spot
\end{itemize*}
\subparhead{Traits} A magical beast possesses the following traits (unless otherwise noted in a creature's entry).
\begin{itemize*}
\item Darkvision out to 60 feet and low-light vision.
\item Proficient with its natural weapons only.
\item Proficient with no armor.
\item Magical beasts eat, sleep, and breathe.
\end{itemize*}

\subsubsection{Monstrous Humanoid Type} Monstrous humanoids are similar to humanoids, but with monstrous or animalistic features. They often have magical abilities as well.
\subparhead{Features} A monstrous humanoid has the following features.
\begin{itemize*}
\my{\item Hit Value 5.}
\item Base attack bonus equal to total \my{Hit Values} (as fighter).
\item Usually good Fortitude and sometimes Will saves.
\item 4 skill points.
\end{itemize*}
\subparhead{Traits} A monstrous humanoid possesses the following traits (unless otherwise noted in a creature's entry).
\begin{itemize*}
\item Proficient with simple weapons and any weapons mentioned in its entry.
\item Proficient with whatever type of armor (light, medium, or heavy) it is described as wearing, as well as all lighter types. Monstrous humanoids not indicated as wearing armor are not proficient with armor. Monstrous humanoids are proficient with shields (except tower shields) if they are proficient with any form of armor.
\item Monstrous humanoids eat, sleep, and breathe.
\end{itemize*}

\subsubsection{Ooze Type} An ooze is an amorphous or mutable creature, usually mindless.
\subparhead{Features} An ooze has the following features.
\begin{itemize*}
\item 10-sided \my{Hit Values}.
\item Base attack bonus equal to 3/4 total \my{Hit Values} (as cleric).
\item No good saving throws.
\item Skill points equal to (2 \add Int, minimum 1) per Hit Die, with quadruple skill points for the first Hit Die, if the ooze has an Intelligence score. However, most oozes are mindless and gain no skill points or feats.
\end{itemize*}
\subparhead{Traits} An ooze possesses the following traits (unless otherwise noted in a creature's entry).
\begin{itemize*}
\item Mindless: No Intelligence score, and immunity to all mind-affecting effects (charms, compulsions, phantasms, patterns, and morale effects).
\item Blind (but have the blindsight special quality), with immunity to gaze attacks, visual effects, illusions, and other attack forms that rely on sight.
\item Immunity to poison, sleep effects, paralysis, polymorph, and stunning.
\item Some oozes have the ability to deal acid damage to objects. In such a case, the amount of damage is equal to 10 \add 1/2 ooze's HD \add ooze's Con per full round of contact.
\item Not subject to critical hits or flanking.
\item Proficient with its natural weapons only.
\item Proficient with no armor.
\item Oozes eat and breathe, but do not sleep.
\end{itemize*}

\subsubsection{Outsider Type} An outsider is at least partially composed of the essence (but not necessarily the material) of some plane other than the Material Plane. Some creatures start out as some other type and become outsiders when they attain a higher (or lower) state of spiritual existence.
\subparhead{Features} An outsider has the following features.
\begin{itemize*}
\item Hit Value 6.
\item Base attack bonus equal to total \my{Hit Values} (as fighter).
\item Two good saving throws, usually Fortitude and Will.
\item 8 skill points.
\end{itemize*}
\subparhead{Traits} An outsider possesses the following traits (unless otherwise noted in a creature's entry).
\begin{itemize*}
\item Darkvision out to 60 feet.
\item Unlike most other living creatures, an outsider does not have a dual nature\item its soul and body form one unit. When an outsider is slain, no soul is set loose. Spells that restore souls to their bodies, such as raise dead, reincarnate, and resurrection, don't work on an outsider. It takes a different magical effect, such as limited wish, wish, miracle, or true resurrection to restore it to life. An outsider with the native subtype can be raised, reincarnated, or resurrected just as other living creatures can be.
\item Proficient with simple weapons and any weapon groups mentioned in its entry.
\item Proficient with whatever type of armor (light, medium, or heavy) it is described as wearing, as well as all lighter types. Outsiders not indicated as wearing armor are not proficient with armor. Outsiders are proficient with shields (but not tower shields, unless mentioned in its entry) if they are proficient with any form of armor.
\item Outsiders breathe, but do not need to eat or sleep (although they can do so if they wish). Native outsiders breathe, eat, and sleep.
\end{itemize*}

\subsubsection{Plant Type} This type comprises vegetable creatures. Note that regular plants, such as one finds growing in gardens and fields, lack Wisdom and Charisma scores (see Nonabilities, above) and are not creatures, but objects, even though they are alive. 
\subparhead{Features} A plant creature has the following features. 
\begin{itemize*}
\item Hit Value 5.
\item Base attack bonus equal to 3/4 total \my{Hit Values} (average progression).
\item Good Fortitude saves.
\item 2 skill points.
\end{itemize*}
\subparhead{Traits} A plant creature possesses the following traits (unless otherwise noted in a creature's entry).
\begin{itemize*}
\item Low-light vision.
\item Immunity to all mind-affecting effects (charms, compulsions, phantasms, patterns, and morale effects).
\item Immunity to poison, sleep effects, paralysis, polymorph, and stunning.
\item Not subject to critical hits.
\item Proficient with its natural weapons only.
\item Proficient with no armor.
\item Plants breathe and eat, but do not sleep.
\end{itemize*}

\subsubsection{Undead Type} Undead are once-living creatures animated by spiritual or supernatural forces.
\subparhead{Features} An undead creature has the following features.
\begin{itemize*}
\item \my{Hit Value 7}.
\item Base attack bonus equal to \my{3/4 total Hit Values (average progression)}.
\item Good Will saves.
\item 2 skill points. Undead have Intimidate as a class skill.
\end{itemize*}
\subparhead{Traits} An undead creature possesses the following traits (unless otherwise noted in a creature's entry).
\begin{itemize*}
\item No Constitution score.
\item Darkvision out to 60 feet.
\item Immunity to all mind-affecting effects (charms, compulsions, phantasms, patterns, and morale effects).
\item Immunity to poison, sleep effects, paralysis, stunning, disease, and death effects.
\item Not subject to nonlethal damage, ability drain, or energy drain. Immune to damage to its physical attribute scores (Strength, Dexterity, and Constitution), as well as to fatigue and exhaustion effects.
\item Cannot heal damage on its own if it has no Intelligence score, although it can be healed. Negative energy (such as an inflict spell) can heal undead creatures. The fast healing special quality works regardless of the creature's Intelligence score.
\item Immunity to any effect that requires a Fortitude save (unless the effect also works on objects or is harmless).
\item Uses its Charisma for Concentration checks.
\item Not at risk of death from massive damage, but when reduced to 0 hit points or less, it is immediately destroyed.
\item Not affected by raise dead and reincarnate spells or abilities. Resurrection and true resurrection can affect undead creatures. These spells turn undead creatures back into the living creatures they were before becoming undead.
\item Proficient with its natural weapons, all simple weapons, and any weapons mentioned in its entry.
\item Proficient with whatever type of armor (light, medium, or heavy) it is described as wearing, as well as all lighter types. Undead not indicated as wearing armor are not proficient with armor. Undead are proficient with shields if they are proficient with any form of armor.
\item Undead do not breathe, eat, or sleep.
\end{itemize*}

\subsubsection{Vermin Type} This type includes insects, arachnids, other arthropods, worms, and similar invertebrates.
\subparhead{Features} Vermin have the following features.
\begin{itemize*}
\item Hit Value 5.
\item Base attack bonus equal to 3/4 total \my{Hit Values} (average progression).
\item Good Fortitude saves.
\item Skill points equal to (2 \add Int, minimum 1) per Hit Die, with quadruple skill points for the first Hit Die, if the vermin has an Intelligence score. However, most vermin are mindless and gain no skill points or feats.
\end{itemize*}
\subparhead{Traits} Vermin possess the following traits (unless otherwise noted in a creature's entry).
\begin{itemize*}
\item Mindless: No Intelligence score, and immunity to all mind-affecting effects (charms, compulsions, phantasms, patterns, and morale effects).
\item Darkvision out to 60 feet.
\item Proficient with their natural weapons only.
\item Proficient with no armor.
\item Vermin breathe, eat, and sleep.
\end{itemize*}

\subsection{Monster Subtypes}

\subsubsection{Air Subtype} This subtype usually is used for elementals and outsiders with a connection to the Elemental Plane Air. Air creatures always have fly speeds and usually have perfect maneuverability.

\subsubsection{Angel Subtype} Angels are a race of celestials, or good outsiders, native to the good-aligned Outer Planes.
\subparhead{Traits} An angel possesses the following traits (unless otherwise noted in a creature's entry).
\begin{itemize*}
\item Darkvision out to 60 feet and low-light vision.
\item Immunity to acid, cold, and petrification.
\item Resistance to electricity 10 and fire 10.
\item \plus4 racial bonus on saves against poison.
\item Protective Aura (Su): Against attacks made or effects created by evil creatures, this ability provides a \plus4 deflection bonus to AC and a \plus4 resistance bonus on saving throws to anyone within 20 feet of the angel. Otherwise, it functions as a magic circle against evil effect and a lesser globe of invulnerability, both with a radius of 20 feet (caster level equals angel's HD). (The defensive benefits from the circle are not included in an angel's statistics block.) 
\item Tongues (Su): All angels can speak with any creature that has a language, as though using a tongues spell (caster level equal to angel's \my{Hit Values}). This ability is always active.
\end{itemize*}

\subsubsection{Aquatic Subtype} These creatures always have swim speeds and thus can move in water without making Swim checks. An aquatic creature can breathe underwater. It cannot also breathe air unless it has the amphibious special quality. 

\subsubsection{Archon Subtype} Archons are a race of celestials, or good outsiders, native to lawful good-aligned Outer Planes.
\subparhead{Traits} An archon possesses the following traits (unless otherwise noted in a creature's entry).
\begin{itemize*}
\item Darkvision out to 60 feet and low-light vision.
\item Aura of Menace (Su): A righteous aura surrounds archons that fight or get angry. Any hostile creature within a 20-foot radius of an archon must succeed on a Will save to resist its effects. The save DC varies with the type of archon, is Charisma-based, and includes a \plus2 racial bonus. Those who fail take a \minus2 penalty on attacks, AC, and saves for 24 hours or until they successfully hit the archon that generated the aura. A creature that has resisted or broken the effect cannot be affected again by the same archon's aura for 24 hours.
\item Immunity to electricity and petrification.
\item  \plus4 racial bonus on saves against poison.
\item Magic Circle against Evil (Su): A magic circle against evil effect always surrounds an archon (caster level equals the archon's \my{Hit Values}). (The defensive benefits from the circle are not included in an archon's statistics block.)
\item Teleport (Su): Archons can use greater teleport at will, as the spell (caster level 14th), except that the creature can transport only itself and up to 50 pounds of objects.
\item Tongues (Su): All archons can speak with any creature that has a language, as though using a tongues spell (caster level 14th). This ability is always active.
\end{itemize*}

\subsubsection{Augmented Subtype} A creature receives this subtype whenever something happens to change its original type. Some creatures (those with an inherited template) are born with this subtype; others acquire it when they take on an acquired template. The augmented subtype is always paired with the creature's original type. A creature with the augmented subtype usually has the traits of its current type, but the features of its original type.

\subsubsection{Chaotic Subtype} A subtype usually applied only to outsiders native to the chaotic-aligned Outer Planes. Most creatures that have this subtype also have chaotic alignments; however, if their alignments change they still retain the subtype. Any effect that depends on alignment affects a creature with this subtype as if the creature has a chaotic alignment, no matter what its alignment actually is. The creature also suffers effects according to its actual alignment. A creature with the chaotic subtype overcomes damage reduction as if its natural weapons and any weapons it wields were chaotic-aligned (see Damage Reduction, below).

\subsubsection{Cold Subtype} A creature with the cold subtype has immunity to cold. It has vulnerability to fire, which means it takes half again as much (\plus50\%) damage as normal from fire, regardless of whether a saving throw is allowed, or if the save is a success or failure. 

\subsubsection{Goblinoid Subtype} Goblinoids are stealthy humanoids who live by hunting and raiding and who all speak Goblin.

\subsubsection{Good Subtype} A subtype usually applied only to outsiders native to the good-aligned Outer Planes. Most creatures that have this subtype also have good alignments; however, if their alignments change, they still retain the subtype. Any effect that depends on alignment affects a creature with this subtype as if the creature has a good alignment, no matter what its alignment actually is. The creature also suffers effects according to its actual alignment. A creature with the good subtype overcomes damage reduction as if its natural weapons and any weapons it wields were good-aligned (see Damage Reduction, above).

\subsubsection{Earth Subtype} This subtype usually is used for elementals and outsiders with a connection to the Elemental Plane of Earth. Earth creatures usually have burrow speeds, and most earth creatures can burrow through solid rock.

\subsubsection{Elemental Subtype} An elemental is a being composed of one of the four classical elements: air, earth, fire, or water.
\subparhead{Features} An elemental has the following features.
\begin{itemize*}
\item Immunity to paralysis, poison, sleep effects, and stunning.
\item Not subject to critical hits. Does not take additional damage from precision-based attacks, such as sneak attack.
\item Proficient with natural weapons only, unless generally humanoid in form, in which case proficient with all simple weapons and any weapons mentioned in its entry.
\item Proficient with whatever type of armor (light, medium, or heavy) it is described as wearing, as well as all lighter types. Elementals not indicated as wearing armor are not proficient with armor. Elementals are proficient with shields if they are proficient with any form of armor.
\item Elementals do not breathe, eat, or sleep.
\end{itemize*}

\subsubsection{Evil Subtype} A subtype usually applied only to outsiders native to the evil-aligned Outer Planes. Evil outsiders are also called fiends. Most creatures that have this subtype also have evil alignments; however, if their alignments change, they still retain the subtype. Any effect that depends on alignment affects a creature with this subtype as if the creature has an evil alignment, no matter what its alignment actually is. The creature also suffers effects according to its actual alignment. A creature with the evil subtype overcomes damage reduction as if its natural weapons and any weapons it wields were evil-aligned (see Damage Reduction, above).

\subsubsection{Fire Subtype} A creature with the fire subtype has immunity to fire. It has vulnerability to cold, which means it takes half again as much (\plus50\%) damage as normal from cold, regardless of whether a saving throw is allowed, or if the save is a success or failure.

\subsubsection{Incorporeal Subtype} An incorporeal creature has no physical body. It can be harmed only by other incorporeal creatures, magic weapons or creatures that strike as magic weapons, and spells, spell-like abilities, or supernatural abilities. It is immune to all nonmagical attack forms. Even when hit by spells or magic weapons, it has a 50\% chance to ignore any damage from a corporeal source (except for positive energy, negative energy, force effects such as magic missile, or attacks made with ghost touch weapons). Although it is not a magical attack, holy water can affect incorporeal undead, but a hit with holy water has a 50\% chance of not affecting an incorporeal creature.

\par An incorporeal creature has no natural armor bonus but has a deflection bonus equal to its Charisma bonus (always at least \plus1, even if the creature's Charisma score does not normally provide a bonus). 

\par An incorporeal creature can enter or pass through solid objects, but must remain adjacent to the object's exterior, and so cannot pass entirely through an object whose space is larger than its own. It can sense the presence of creatures or objects within a square adjacent to its current location, but enemies have total concealment (50\% miss chance) from an incorporeal creature that is inside an object. In order to see farther from the object it is in and attack normally, the incorporeal creature must emerge. An incorporeal creature inside an object has total cover, but when it attacks a creature outside the object it only has cover, so a creature outside with a readied action could strike at it as it attacks. An incorporeal creature cannot pass through a force effect.

\par An incorporeal creature's attacks pass through (ignore) natural armor, armor, and shields, although deflection bonuses and force effects (such as \spell{mage armor}) work normally against it. Incorporeal creatures pass through and operate in water as easily as they do in air. Incorporeal creatures cannot fall or take falling damage. Incorporeal creatures cannot make trip or grapple attacks, nor can they be tripped or grappled. In fact, they cannot take any physical action that would move or manipulate an opponent or its equipment, nor are they subject to such actions. Incorporeal creatures have no weight and do not set off traps that are triggered by weight.

An incorporeal creature moves silently and cannot be heard with Listen checks if it doesn't wish to be. It has no Strength score, so its Dexterity applies to both its melee attacks and its ranged attacks. Nonvisual senses, such as scent and blindsight, are either ineffective or only partly effective with regard to incorporeal creatures. Incorporeal creatures have an innate sense of direction and can move at full speed even when they cannot see.

Incorporeal creatures have Hide as a class skill.

\subsubsection{Lawful Subtype} A subtype usually applied only to outsiders native to the lawful-aligned Outer Planes. Most creatures that have this subtype also have lawful alignments; however, if their alignments change, they still retain the subtype. Any effect that depends on alignment affects a creature with this subtype as if the creature has a lawful alignment, no matter what its alignment actually is. The creature also suffers effects according to its actual alignment. A creature with the lawful subtype overcomes damage reduction as if its natural weapons and any weapons it wields were lawful-aligned (see Damage Reduction, above).

\subsubsection{Native Subtype} A subtype applied only to outsiders. These creatures have mortal ancestors or a strong connection to the Material Plane and can be raised, reincarnated, or resurrected just as other living creatures can be. Creatures with this subtype are native to the Material Plane (hence the subtype's name). Unlike true outsiders, native outsiders need to eat and sleep. 

\subsubsection{Reptilian Subtype} These creatures are scaly and usually coldblooded. The reptilian subtype is only used to describe a set of humanoid races, not all animals and monsters that are truly reptiles.

\subsubsection{Swarm Subtype} A swarm is a collection of Fine, Diminutive, or Tiny creatures that acts as a single creature. A swarm has the characteristics of its type, except as noted here. A swarm has a single pool of \my{Hit Values} and hit points, a single initiative modifier, a single speed, and a single Armor Class. A swarm makes saving throws as a single creature. A single swarm occupies a square (if it is made up of nonflying creatures) or a cube (of flying creatures) 10 feet on a side, but its reach is 0 feet, like its component creatures. In order to attack, it moves into an opponent's space, which provokes an attack of opportunity. It can occupy the same space as a creature of any size, since it crawls all over its prey. A swarm can move through squares occupied by enemies and vice versa without impediment, although the swarm provokes an attack of opportunity if it does so. A swarm can move through cracks or holes large enough for its component creatures.

A swarm of Tiny creatures consists of 300 nonflying creatures or 1,000 flying creatures. A swarm of Diminutive creatures consists of 1,500 nonflying creatures or 5,000 flying creatures. A swarm of Fine creatures consists of 10,000 creatures, whether they are flying or not. Swarms of nonflying creatures include many more creatures than could normally fit in a 10-foot square based on their normal space, because creatures in a swarm are packed tightly together and generally crawl over each other and their prey when moving or attacking. Larger swarms are represented by multiples of single swarms. The area occupied by a large swarm is completely shapeable, though the swarm usually remains in contiguous squares.

\subparhead{Traits} A swarm has no clear front or back and no discernable anatomy, so it is not subject to critical hits or flanking. A swarm made up of Tiny creatures takes half damage from slashing and piercing weapons. A swarm composed of Fine or Diminutive creatures is immune to all weapon damage. Reducing a swarm to 0 hit points or lower causes it to break up, though damage taken until that point does not degrade its ability to attack or resist attack. Swarms are never staggered or reduced to a dying state by damage. Also, they cannot be tripped, grappled, or bull rushed, and they cannot grapple an opponent.

A swarm is immune to any spell or effect that targets a specific number of creatures (including single-target spells such as disintegrate), with the exception of mind-affecting effects (charms, compulsions, phantasms, patterns, and morale effects) if the swarm has an Intelligence score and a hive mind. A swarm takes half again as much damage (\plus50\%) from spells or effects that affect an area, such as splash weapons and many evocation spells.

Swarms made up of Diminutive or Fine creatures are susceptible to high winds such as that created by a gust of wind spell. For purposes of determining the effects of wind on a swarm, treat the swarm as a creature of the same size as its constituent creatures. A swarm rendered unconscious by means of nonlethal damage becomes disorganized and dispersed, and does not reform until its hit points exceed its nonlethal damage.

\subparhead{Swarm Attack} Creatures with the swarm subtype don't make standard melee attacks. Instead, they deal automatic damage to any creature whose space they occupy at the end of their move, with no attack roll needed. Swarm attacks are not subject to a miss chance for concealment or cover. A swarm's statistics block has ``swarm'' in the Attack and Full Attack entries, with no attack bonus given. The amount of damage a swarm deals is based on its \my{Hit Values}, as shown below.

\begin{dtable}
\begin{tabularx}{\columnwidth}{l >{\lcol}X}
\thead{Swarm HD} & \thead{Swarm Base Damage} \\
1-5 & 1d6 \\
6-10 & 2d6 \\
11-15 & 3d6 \\
16-20 & 4d6 \\
21 or more & 5d6
\end{tabularx}
\end{dtable}

A swarm's attacks are nonmagical, unless the swarm's description states otherwise. Damage reduction sufficient to reduce a swarm attack's damage to 0, being incorporeal, and other special abilities usually give a creature immunity (or at least resistance) to damage from a swarm. Some swarms also have acid, poison, blood drain, or other special attacks in addition to normal damage.

Swarms do not threaten creatures in their square, and do not make attacks of opportunity with their swarm attack. However, they distract foes whose squares they occupy, as described below.

\subparhead{Distraction (Ex)} Any living creature vulnerable to a swarm's damage that begins its turn with a swarm in its square is nauseated for 1 round; a Fortitude save (DC 10 \add 1/2 swarm's HD \add swarm's Con; the exact DC is given in a swarm's description) negates the effect. Spellcasting or concentrating on spells within the area of a swarm requires a Concentration check (DC 20 \add \my{double} spell level). Using skills that involve patience and concentration requires a DC 20 Concentration check.

\subsubsection{Water Subtype} This subtype usually is used for elementals and outsiders with a connection to the Elemental Plane of Water. Creatures with the water subtype always have swim speeds and can move in water without making Swim checks. A water creature can breathe underwater and usually can breathe air as well.

\subsection{Monster Abilities}

\subsubsection{Attribute Score Loss (Su)} Some attacks reduce the opponent's score in one or more abilities. This loss can be temporary (ability damage) or permanent (ability drain).
\subparhead{Ability Damage} This attack damages an opponent's attribute score. The creature's descriptive text gives the ability and the amount of damage. If an attack that causes ability damage scores a critical hit, it deals twice the indicated amount of damage (if the damage is expressed as a die range, roll two dice). Ability damage returns at the rate of 1 point per day for each affected ability.
\subparhead{Ability Drain} This effect permanently reduces a living opponent's attribute score when the creature hits with a melee attack. The creature's descriptive text gives the ability and the amount drained. If an attack that causes ability drain scores a critical hit, it drains twice the indicated amount (if the damage is expressed as a die range, roll two dice). Unless otherwise specified in the creature's description, a draining creature gains 5 temporary hit points \my{per ability point drained}. Temporary hit points gained in this fashion last for a maximum of 1 hour.
\par Some ability drain attacks allow a Fortitude save (DC 10 \add 1/2 draining creature's racial HD \add draining creature's Cha; the exact DC is given in the creature's descriptive text). If no saving throw is mentioned, none is allowed.

\subsubsection{Alternate Form (Su)} A creature with this special quality has the ability to assume one or more specific alternate forms. A \spell{true seeing} spell or ability reveals the creature's natural form. A creature using alternate form reverts to its natural form when killed, but separated body parts retain their shape. A creature cannot use alternate form to take the form of a creature with a template. 

Assuming an alternate form results in the following changes to the creature:
\begin{itemize*}
\item The creature retains the type and subtype of its original form. It gains the size of its new form. If the new form has the aquatic subtype, the creature gains that subtype as well.
\item The creature loses the natural weapons, natural armor, and movement modes of its original form, as well as any extraordinary special attacks of its original form not derived from class levels (such as the barbarian's rage class feature). 
\item The creature gains the natural weapons, natural armor, movement modes, and extraordinary special attacks of its new form.
\item The creature retains the special qualities of its original form. It does not gain any special qualities of its new form.
\item The creature retains the spell-like abilities and supernatural attacks of its old form (except for breath weapons and gaze attacks). It does not gain the spell-like abilities or supernatural attacks of its new form.
\item The creature gains the physical attribute scores (Str, Dex, Con) of its new form. It retains the mental attribute scores (Int, Wis, Cha) of its original form. Apply any changed physical attribute score modifiers in all appropriate areas with one exception: the creature retains the hit points of its original form despite any change to its Constitution.
\item Except as described elsewhere, the creature retains all other game statistics of its original form, including (but not necessarily limited to) HD, hit points, skill ranks, feats, base attack bonus, and base save bonuses. 
\item The creature retains any spellcasting ability it had in its original form, although it must be able to speak intelligibly to cast spells with verbal components and it must have humanlike hands to cast spells with somatic components.
\item The creature is effectively camouflaged as a creature of its new form, and it gains a \plus10 \my{enhancement} bonus on Disguise checks if it uses this ability to create a disguise.
\item Any gear worn or carried by the creature that can't be worn or carried in its new form instead falls to the ground in its space. If the creature changes size, any gear it wears or carries that can be worn or carried in its new form changes size to match the new size. (Nonhumanoid-shaped creatures can't wear armor designed for humanoid-shaped creatures, and vice-versa.) Gear returns to normal size if dropped.
\end{itemize*}

\subsubsection{Blindsense (Ex)} Using nonvisual senses, such as acute smell or hearing, a creature with blindsense notices things it cannot see. The creature usually does not need to make Spot or Listen checks to pinpoint the location of a creature within range of its blindsense ability, provided that it has line of effect to that creature. Any opponent the creature cannot see still has total concealment against the creature with blindsense, and the creature still has the normal miss chance when attacking foes that have concealment. Visibility still affects the movement of a creature with blindsense. A creature with blindsense is still denied its Dexterity bonus \my{and dodge modifier} to Armor Class against attacks from creatures it cannot see.

\subsubsection{Blindsight (Ex)} This ability is similar to blindsense, but is far more discerning. Using nonvisual senses, such as sensitivity to vibrations, keen smell, acute hearing, or echolocation, a creature with blindsight maneuvers and fights as well as a sighted creature. Invisibility, darkness, and most kinds of concealment are irrelevant, though the creature must have line of effect to a creature or object to discern that creature or object. The ability's range is specified in the creature's descriptive text. The creature usually does not need to make Spot or Listen checks to notice creatures within range of its blindsight ability. Unless noted otherwise, blindsight is continuous, and the creature need do nothing to use it. Some forms of blindsight, however, must be triggered as a free action. If so, this is noted in the creature's description. If a creature must trigger its blindsight ability, the creature gains the benefits of blindsight only during its turn.

\subsubsection{Breath Weapon (Su)} A breath weapon attack usually deals damage and is often based on some type of energy.
\par Such breath weapons allow a Reflex save for half damage (DC 10 \add 1/2 breathing creature's racial HD \add breathing creature's Con; the exact DC is given in the creature's descriptive text). A creature is immune to its own breath weapon unless otherwise noted. Some breath weapons allow a Fortitude save or a Will save instead of a Reflex save.

\subsubsection{Change Shape (Su)} A creature with this special quality has the ability to assume the appearance of a specific creature or type of creature (usually a humanoid), but retains most of its own physical qualities. A \spell{true seeing} spell or ability reveals the creature's natural form. A creature using change shape reverts to its natural form when killed, but separated body parts retain their shape. A creature cannot use change shape to take the form of a creature with a template. Changing shape results in the following changes to the creature:
\begin{itemize*}
\item The creature retains the type and subtype of its original form. It gains the size of its new form.
\item The creature loses the natural weapons, natural armor, and movement modes of its original form, as well as any extraordinary special attacks of its original form not derived from class levels (such as the barbarian's rage class feature).
\item The creature gains the natural weapons, movement modes, and extraordinary special attacks of its new form.
\item The creature retains all other special attacks and qualities of its original form, except for breath weapons and gaze attacks.
\item The creature retains the attribute scores of its original form.
\item Except as described elsewhere, the creature retains all other game statistics of its original form, including (but not necessarily limited to) HD, hit points, skill ranks, feats, base attack bonus, and base save bonuses. 
\item The creature retains any spellcasting ability it had in its original form, although it must be able to speak intelligibly to cast spells with verbal components and it must have humanlike hands to cast spells with somatic components.
\item The creature is effectively camouflaged as a creature of its new form, and gains a \plus10 \my{enhancement} bonus on Disguise checks if it uses this ability to create a disguise.
\end{itemize*}

\subsubsection{Constrict (Ex)} A creature with this special attack can crush an opponent, dealing bludgeoning damage, after making a successful grapple check. The amount of damage is given in the creature's entry. If the creature also has the improved grab ability, it deals constriction damage in addition to damage dealt by the weapon used to grab.

\subsubsection{Damage Reduction (Ex or Su)} A creature with this special quality \my{mitigates} damage from weapons and natural attacks. \my{Some of the damage taken is nonlethal instead of lethal}. The creature takes normal damage from energy attacks (even nonmagical ones), spells, spell-like abilities, and supernatural abilities.

The entry indicates the amount of damage \my{mitigated} (usually 5 to 15 points).


\subsubsection{Damage Resistance (Ex or Su)} A creature with this special quality \my{takes only nonlethal damage} from most weapons and natural attacks. The creature takes normal damage from energy attacks (even nonmagical ones), spells, spell-like abilities, and supernatural abilities. A certain kind of weapon can \my{always} damage the creature normally, as noted below.

Some monsters are vulnerable to piercing, bludgeoning, or slashing damage.

Some monsters are vulnerable to certain materials, such as alchemical silver, adamantine, or cold-forged iron.

Some monsters are vulnerable to magic weapons. Any weapon with at least a \plus1 magical enhancement bonus on attack and damage rolls overcomes the \my{damage resistance} of these monsters. Such creatures' natural weapons (but not their attacks with weapons) are treated as magic weapons for the purpose of overcoming \my{damage resistance}.

A few very powerful monsters are vulnerable only to epic weapons; that is, magic weapons with at least a \plus6 enhancement bonus. Such creatures' natural weapons are also treated as epic weapons for the purpose of overcoming \my{damage resistance}.

Some monsters are vulnerable to chaotic-, evil-, good-, or lawful-aligned weapons. When a cleric casts \spell{align weapon}, affected weapons might gain one or more of these properties, and certain magic weapons have these properties as well. A creature with an alignment subtype (chaotic, evil, good, or lawful) can overcome this type of damage reduction with its natural weapons and weapons it wields as if the weapons or natural weapons had an alignment (or alignments) that match the subtype(s) of the creature.

A few creatures are harmed by more than one kind of weapon. A weapon of either type overcomes this damage reduction.

A few other creatures require combinations of different types of attacks to overcome their damage reduction. A weapon must be both types to overcome this damage reduction. A weapon that is only one type is still subject to damage reduction.

\subsubsection{Energy Drain (Su)} This attack saps a living opponent's vital energy and happens automatically when a melee or ranged attack hits. Each successful energy drain bestows one or more negative levels (the creature's description specifies how many). If an attack that includes an energy drain scores a critical hit, it drains twice the given amount. Unless otherwise specified in the creature's description, a draining creature gains 5 temporary hit points for each negative level it bestows on an opponent. These temporary hit points last for a maximum of 1 hour. An affected opponent takes a \minus1 penalty on all \my{checks}, attack rolls, and saving throws, \my{loses 5 hit points}, and loses one effective level or \my{Hit Value} (whenever level is used in a die roll or calculation) for each negative level. \my{The hit points lost decrease the creature's maximum hit points for as long as the negative level persists.} In addition, a spellcaster loses one spell slot of the highest level of spells she can cast and (if applicable) one prepared spell of that level; this loss persists until the negative level is removed. \my{A creature recovers from negative levels at a rate of one per day.}

\subsubsection{Extraplanar Subtype} A subtype applied to any creature when it is on a plane other than its native plane. A creature that travels the planes can gain or lose this subtype as it goes from plane to plane. Monster entries assume that encounters with creatures take place on the Material Plane, and every creature whose native plane is not the Material Plane has the extraplanar subtype (but would not have when on its home plane). Every extraplanar creature in this book has a home plane mentioned in its description. Creatures not labeled as extraplanar are natives of the Material Plane, and they gain the extraplanar subtype if they leave the Material Plane. No creature has the extraplanar subtype when it is on a transitive plane, such as the Astral Plane, the Ethereal Plane, and the Plane of Shadow.

\subsubsection{Fast Healing (Ex)} A creature with the fast healing special quality regains hit points at an exceptionally fast rate, usually 1 or more hit points per round, as given in the creature's entry. Except where noted here, fast healing is just like natural healing. Fast healing does not restore hit points lost from starvation, thirst, or suffocation, and it does not allow a creature to regrow lost body parts. Unless otherwise stated, it does not allow lost body parts to be reattached.

\subsubsection{Fear (Su or Sp)} Fear attacks can have various effects.
\subparhead{Fear Aura (Su)} The use of this ability is a free action. The aura can freeze an opponent (such as a mummy's despair) or function like the fear spell. Other effects are possible. A fear aura is an area effect. The descriptive text gives the size and kind of area.
\subparhead{Fear Cones (Sp) and Rays (Su)} These effects usually work like the \spell{fear} spell. 
\par If a fear effect allows a saving throw, it is a Will save (DC 10 \add 1/2 fearsome creature's racial HD \add creature's Cha; the exact DC is given in the creature's descriptive text). All fear attacks are mind-affecting fear effects.

\subsubsection{Flight (Ex or Su)} A creature with this ability can cease or resume flight as a free action. If the ability is supernatural, it becomes ineffective in an antimagic field, and the creature loses its ability to fly for as long as the antimagic effect persists.

\subsubsection{Frightful Presence (Ex)} This special quality makes a creature's very presence unsettling to foes. It takes effect automatically when the creature performs some sort of dramatic action (such as charging, attacking, or snarling). Opponents within range who witness the action may become frightened or shaken. Actions required to trigger the ability are given in the creature's descriptive text. The range is usually 30 feet, and the duration is usually 5d6 rounds. This ability affects only opponents with fewer \my{Hit Values} or levels than the creature has. An affected opponent can resist the effects with a successful Will save (DC 10 \add 1/2 frightful creature's racial HD \add frightful creature's Cha; the exact DC is given in the creature's descriptive text). An opponent that succeeds on the saving throw is immune to that same creature's frightful presence for 24 hours. Frightful presence is a mind-affecting fear effect. 


\subsubsection{Flurry Attack} If a creature has more than one natural weapon of the same type, it can make a special attack called a flurry attack to attack with all of its natural weapons of the same type at once.

To make a flurry attack, a creature makes a single attack roll with a \minus2 penalty. If the natural weapon is light, the attack is made with no penalty. For every weapon used with the flurry attack beyond the second, the attack also gets a \plus2 circumstance bonus.

If the flurry attack hits, the creature deals damage with the main weapon used for the flurry. This damage includes the creature's full Strength. For every 5 points that the attack succeeds by, the creature can deal damage with an additional weapon used for the flurry, to a limit of the number of weapons that the creature has of that type. Each hit after the main hit includes half the creature's Strength.

If a creature normally makes a flurry attack, it will specify the number of weapons used to make the attack, and two damage values will be listed; one for the damage with the first hit, and a second value for each hit after the first.


\subsubsection{Gaze (Su)} A gaze special attack takes effect when opponents look at the creature's eyes. The attack can have almost any sort of effect: petrification, death, charm, and so on. The typical range is 30 feet, but check the creature's entry for details. The type of saving throw for a gaze attack varies, but it is usually a Will or Fortitude save (DC 10 \add 1/2 gazing creature's racial HD \add gazing creature's Cha; the exact DC is given in the creature's descriptive text). A successful saving throw negates the effect. A monster's gaze attack is described in abbreviated form in its description. Each opponent within range of a gaze attack must attempt a saving throw each round at the beginning of his or her turn in the initiative order. Only looking directly at a creature with a gaze attack leaves an opponent vulnerable. Opponents can avoid the need to make the saving throw by not looking at the creature, in one of two ways. 
\subparhead{Averting Eyes} The opponent avoids looking at the creature's face, instead looking at its body, watching its shadow, tracking it in a reflective surface, and so on. Each round, the opponent has a 50\% chance to not need to make a saving throw against the gaze attack. The creature with the gaze attack, however, gains concealment against that opponent.
\subparhead{Wearing a Blindfold} The opponent cannot see the creature at all (also possible to achieve by turning one's back on the creature or shutting one's eyes). The creature with the gaze attack gains total concealment against the opponent. 
\par A creature with a gaze attack can actively gaze as an attack action by choosing a target within range. That opponent must attempt a saving throw but can try to avoid this as described above. Thus, it is possible for an opponent to save against a creature's gaze twice during the same round, once before the opponent's action and once during the creature's turn. 
\par Gaze attacks can affect ethereal opponents. A creature is immune to gaze attacks of others of its kind unless otherwise noted.
\par Allies of a creature with a gaze attack might be affected. All the creature's allies are considered to be averting their eyes from the creature with the gaze attack, and have a 50\% chance to not need to make a saving throw against the gaze attack each round. The creature also can veil its eyes, thus negating its gaze ability.

\subsubsection{Improved Grab (Ex)} If a creature with this special attack hits with a melee weapon (usually a claw or bite attack), it deals normal damage and attempts to start a grapple as a \my{swift} action \my{that never provokes} an attack of opportunity. Unless otherwise noted, improved grab works only against opponents at least one size category smaller than the creature. \my{If the creature succeeds by 10 or more, it can simply use the part of its body it used in the improved grab to hold the opponent}. If it \my{does so}, it takes a \my{\minus10} penalty on grapple checks, but is not considered grappled itself; the creature does not lose its Dexterity bonus \my{or dodge modifier} to AC, still threatens an area, and can use its remaining attacks against other opponents. A successful hold does not deal any extra damage unless the creature also has the constrict special attack. If the creature does not constrict, each successful grapple check it makes during successive rounds automatically deals the damage indicated for the attack that established the hold. Otherwise, it deals constriction damage as well (the amount is given in the creature's descriptive text).

\subsubsection{Low-Light Vision (Ex)} A creature with low-light vision can see twice as far as a human in starlight, moonlight, torchlight, and similar conditions of shadowy illumination. It retains the ability to distinguish color and detail under these conditions.

\subsubsection{Manufactured Weapons} Some monsters employ manufactured weapons when they attack. Creatures that use swords, bows, spears, and the like follow the same rules as characters, including those for additional attacks from a high base attack bonus and two-weapon fighting penalties. This category also includes ``found items,'' such as rocks and logs, that a creature wields in combat -- in essence, any weapon that is not intrinsic to the creature.

\subsubsection{Movement Modes} Creatures may have modes of movement other than walking and running. These are natural, not magical, unless specifically noted in a monster description.
\subparhead{Burrow} A creature with a burrow speed can tunnel through dirt, but not through rock unless the descriptive text says otherwise. Creatures cannot charge or run while burrowing. Most burrowing creatures do not leave behind tunnels other creatures can use (either because the material they tunnel through fills in behind them or because they do not actually dislocate any material when burrowing); see the individual creature descriptions for details.
\subparhead{Climb} A creature with a climb speed has a \plus8 racial bonus on all Climb checks. The creature must make a Climb check to climb any wall or slope with a DC of more than 0, but it always can choose to take 10 even if rushed or threatened while climbing. The creature climbs at the given speed while climbing. If it chooses an accelerated climb it moves at double the given climb speed (or its base land speed, whichever is lower) and makes a single Climb check at a \minus5 penalty. Creatures cannot run while climbing. A creature retains its Dexterity bonus to Armor Class (if any) while climbing, and opponents get no special bonus on their attacks against a climbing creature.
\subparhead{Fly} A creature with a fly speed can move through the air at the indicated speed if carrying no more than a light load. (Note that medium armor does not necessarily constitute a medium load.) All fly speeds include a parenthetical note indicating maneuverability, as follows:
\begin{itemize*}
\item Perfect: The creature can perform almost any aerial maneuver it wishes. It moves through the air as well as a human moves over smooth ground.
\item Good: The creature is very agile in the air (like a housefly or a hummingbird), but cannot change direction as readily as those with perfect maneuverability.
\item Average: The creature can fly as adroitly as a small bird. 
\item Poor: The creature flies as well as a very large bird.
\item Clumsy: The creature can barely maneuver at all.
\end{itemize*}
A creature that flies can make dive attacks. A dive attack works just like a charge, but the diving creature must move a minimum of 30 feet and descend at least 10 feet. It can make only claw or talon attacks, but these deal double damage. A creature can use the run action while flying, provided it flies in a straight line.

\subparhead{Swim} A creature with a swim speed can move through water at its swim speed without making Swim checks. It has a \plus8 racial bonus on any Swim check to perform some special action or avoid a hazard. The creature can always can choose to take 10 on a Swim check, even if distracted or endangered. The creature can use the run action while swimming, provided it swims in a straight line. 

\subsubsection{Natural Weapons} Natural weapons are weapons that are physically a part of a creature. A creature making a melee attack with a natural weapon is considered armed and does not provoke attacks of opportunity. Natural weapons are used just like manufactured weapons. Creatures get a number of attacks determined by their base attack bonus, and they can use natural weapons to take attacks interchangeably with manufactured weapons, other natural weapons, or with the same natural weapon.

If a creature has more than one weapon of the same type, it can make a flurry attack. Some natural weapons are considered light weapons, as noted in their descriptions. This makes them easier to flurry attack with. See \pref{Flurry Attack}.

Unless otherwise noted, a natural weapon threatens a critical hit on a natural attack roll of 20.

Natural weapons have types just as other weapons do. The most common are summarized below.
\subparhead{Bite} The creature attacks with its mouth, dealing piercing, slashing, and bludgeoning damage.
\subparhead{Claw or Talon} The creature rips with a sharp appendage, dealing piercing and slashing damage.
\subparhead{Gore} The creature spears the opponent with an antler, horn, or similar appendage, dealing piercing damage.
\subparhead{Slap or Slam} The creature batters opponents with an appendage, dealing bludgeoning damage.
\subparhead{Sting} The creature stabs with a stinger, dealing piercing damage. Sting attacks usually deal damage from poison in addition to hit point damage.
\subparhead{Tentacle} The creature flails at opponents with a powerful tentacle, dealing bludgeoning (and sometimes slashing) damage. 

\subsubsection{Nonabilities} Some creatures lack certain attribute scores. These creatures do not have an attribute score of 0 -- they lack the ability altogether. The modifier for a nonability is \plus0. Other effects of nonabilities are detailed below.
\subparhead{Strength} Any creature that can physically manipulate other objects has at least 1 point of Strength. A creature with no Strength score can't exert force, usually because it has no physical body or because it doesn't move. The creature automatically fails Strength checks. If the creature can attack, it applies its Dexterity to its base attack bonus instead of a Strength.
\subparhead{Dexterity} Any creature that can move has at least 1 point of Dexterity. A creature with no Dexterity score can't move. If it can perform actions (such as casting spells), it applies its Intelligence to initiative checks instead of a Dexterity. The creature automatically fails Reflex saves and Dexterity checks.
\subparhead{Constitution} Any living creature has at least 1 point of Constitution. A creature with no Constitution has no body or no metabolism. It is immune to any effect that requires a Fortitude save unless the effect works on objects or is harmless. The creature is also immune to ability damage, ability drain, and energy drain, and automatically fails Constitution checks. A creature with no Constitution cannot tire and thus can run indefinitely without tiring (unless the creature's description says it cannot run).
\subparhead{Intelligence} Any creature that can think, learn, or remember has at least 1 point of Intelligence. A creature with no Intelligence score is mindless, an automaton operating on simple instincts or programmed instructions. It has immunity to mind-affecting effects (charms, compulsions, phantasms, patterns, and morale effects) and automatically fails Intelligence checks.
\par Mindless creatures do not gain feats or free skill points, although they may have skill points from attributes, bonus feats, or racial skill bonuses.
\subparhead{Wisdom} Any creature that can perceive its environment in any fashion has at least 1 point of Wisdom. Anything with no Wisdom score is an object, not a creature. Anything without a Wisdom score also has no Charisma score.
\subparhead{Charisma} Any creature capable of telling the difference between itself and things that are not itself has at least 1 point of Charisma. Anything with no Charisma score is an object, not a creature. Anything without a Charisma score also has no Wisdom score.

\subsubsection{Paralysis (Ex or Su)} This special attack renders the victim immobile. Paralyzed creatures cannot move, speak, or take any physical actions. The creature is rooted to the spot, frozen and helpless. Paralysis works on the body, and a character can usually resist it with a Fortitude saving throw (the DC is given in the creature's description). Unlike hold person and similar effects, a paralysis effect does not allow a new save each round. A winged creature flying in the air at the time that it is paralyzed cannot flap its wings and falls. A swimmer can't swim and may drown. 

\subsubsection{Poison (Ex)} Poison attacks deal initial damage, such as ability damage (see page 305) or some other effect, to the opponent on a failed Fortitude save. Unless otherwise noted, another saving throw is required 1 minute later (regardless of the first save's result) to avoid secondary damage. A creature's descriptive text provides the details.
A creature with a poison attack is immune to its own poison and the poison of others of its kind.
The Fortitude save DC against a poison attack is equal to 10 \add 1/2 poisoning creature's racial HD \add poisoning creature's Con (the exact DC is given in the creature's descriptive text).
A successful save avoids (negates) the damage.

\subsubsection{Pounce (Ex)} When a creature with this special attack makes a charge, it can follow with a full attack -- including rake attacks if the creature also has the rake ability.

\subsubsection{Powerful Charge (Ex)} When a creature with this special attack makes a charge, its attack deals extra damage in addition to the normal benefits and hazards of a charge. The amount of damage from the attack is given in the creature's description.

\subsubsection{Psionics (Sp)} These are spell-like abilities that a creature generates with the power of its mind. Psionic abilities are usually usable at will.

\subsubsection{Rake (Ex)} A creature with this special attack gains extra natural attacks when it grapples its foe. Normally, a monster can attack with only one of its natural weapons while grappling, but a monster with the rake ability usually gains two additional claw attacks that it can use only against a grappled foe. Rake attacks are not subject to the usual –4 penalty for attacking with a natural weapon in a grapple.

A monster with the rake ability must begin its turn grappling to use its rake -- it can't begin a grapple and rake in the same turn.

\subsubsection{Ray (Su or Sp)} This form of special attack works like a ranged attack. Hitting with a ray attack requires a successful ranged touch attack roll, ignoring armor, natural armor, and shield and using the creature's ranged attack bonus. Ray attacks have no range increment. The creature's descriptive text specifies the maximum range, effects, and any applicable saving throw.

\subsubsection{Regeneration (Ex)} A creature with this ability is difficult to kill. Damage dealt to the creature is treated as nonlethal damage. The creature automatically heals nonlethal damage at a fixed rate per round, as given in the entry. Certain attack forms, typically fire and acid, deal lethal damage to the creature, which doesn't go away. The creature's descriptive text describes the details. A regenerating creature that has been rendered unconscious through nonlethal damage can be killed with a coup de grace. The attack cannot be of a type that automatically converts to nonlethal damage.
Attack forms that don't deal hit point damage ignore regeneration. Regeneration also does not restore hit points lost from starvation, thirst, or suffocation. Regenerating creatures can regrow lost portions of their bodies and can reattach severed limbs or body parts; details are in the creature's descriptive text. Severed parts that are not reattached wither and die normally.
A creature must have a Constitution score to have the regeneration ability.

\subsubsection{Resistance to Energy (Ex)} A creature with this special quality ignores some damage of the indicated type each time it takes damage of that kind (commonly acid, cold, fire, or electricity). The entry indicates the amount and type of damage ignored.

\subsubsection{Scent (Ex)} This special quality allows a creature to detect approaching enemies, sniff out hidden foes, and track by sense of smell. Creatures with the scent ability can identify familiar odors just as humans do familiar sights.

The creature can detect opponents within 30 feet by sense of smell. If the opponent is upwind, the range increases to 60 feet; if downwind, it drops to 15 feet. Strong scents, such as smoke or rotting garbage, can be detected at twice the ranges noted above. Overpowering scents, such as skunk musk or troglodyte stench, can be detected at triple normal range.

When a creature detects a scent, the exact location of the source is not revealed -- only its presence somewhere within range. The creature can take a move action to note the direction of the scent. Whenever the creature comes within 5 feet of the source, the creature pinpoints the source's location.

A creature with the Track feat and the scent ability can follow tracks by smell, making a Wisdom (or Survival) check to find or follow a track. The typical DC for a fresh trail is 10 (no matter what kind of surface holds the scent). This DC increases or decreases depending on how strong the quarry's odor is, the number of creatures, and the age of the trail. For each hour that the trail is cold, the DC increases by 2. The ability otherwise follows the rules for the Track feat. Creatures tracking by scent ignore the effects of surface conditions and poor visibility. 

\subsubsection{Shapechanger Subtype} A shapechanger has the supernatural ability to assume one or more alternate forms. Many magical effects allow some kind of shape shifting, and not every creature that can change shapes has the shapechanger subtype. 
\subparhead{Traits} A shapechanger possesses the following traits (unless otherwise noted in a creature's entry).
\begin{itemize*}
\item Proficient with its natural weapons, with simple weapons, and with any weapons mentioned in the creature's description.
\item Proficient with any armor mentioned in the creature's description, as well as all lighter forms. If no form of armor is mentioned, the shapechanger is not proficient with armor. A shapechanger is proficient with shields if it is proficient with any type of armor.
\end{itemize*}

\subsubsection{Sonic Attacks (Su)} Unless otherwise noted, a sonic attack follows the rules for spreads. The range of the spread is measured from the creature using the sonic attack. Once a sonic attack has taken effect, deafening the subject or stopping its ears does not end the effect. Stopping one's ears ahead of time allows opponents to avoid having to make saving throws against mind-affecting sonic attacks, but not other kinds of sonic attacks (such as those that deal damage). Stopping one's ears is a full-round action and requires wax or other soundproof material to stuff into the ears.

\subsubsection{Special Abilities} A special ability is either extraordinary (Ex), spell-like (Sp), or supernatural (Su).
\subparhead{Extraordinary} Extraordinary abilities are nonmagical, don't become ineffective in an antimagic field, and are not subject to any effect that disrupts magic. Using an extraordinary ability is a free action unless otherwise noted.

\subparhead{Spell-Like} Spell-like abilities are magical and work just like spells (though they are not spells and so have no verbal, somatic, material, or focus). They go away in an antimagic field and are subject to spell resistance if the spell the ability resembles or duplicates would be subject to spell resistance.

Using a spell-like ability is a standard action unless noted otherwise, and doing so while threatened provokes attacks of opportunity. It is possible to make a Concentration check to use a spell-like ability defensively and avoid provoking an attack of opportunity, just as when casting a spell. A spell-like ability can be disrupted just as a spell can be. Spell-like abilities cannot be used to counterspell, nor can they be counterspelled.

Creatures with spell-like abilities have a specific mental attribute score which they use to determine the saving throw DC and the number of times per day they can use those abilities. The attribute score used is specified in the creature's description. Creatures are limited in the number of times per day they can use their spell-like abilities. Most creatures can use any combination of their spell-like abilities a number of times per day equal to half the creature's HV \add the creature's relevant attribute. Some spell-like abilities can be used without limit, or require additional effort to use. This is noted in the creature's description.

For creatures with spell-like abilities, a designated caster level defines how difficult it is to dispel their spell-like effects and to define any level-dependent variables (such as range and duration) the abilities might have. The creature's caster level never affects which spell-like abilities the creature has; sometimes the given caster level is lower than the level a spellcasting character would need to cast the spell of the same name. If no caster level is specified, the caster level is equal to the creature's \my{Hit Values}. The saving throw (if any) against a spell-like ability is 10 \add \my{half the creature's HV} \add \my{the creature's attribute}.

Some spell-like abilities duplicate spells that work differently when cast by characters of different classes. A monster's spell-like abilities are presumed to be the sorcerer/wizard versions. If the spell in question is not a sorcerer/wizard spell, then default to cleric, druid, bard, paladin, and ranger, in that order.

\subparhead{Supernatural} Supernatural abilities are magical and go away in an antimagic field but are not subject to spell resistance. Supernatural abilities cannot be dispelled. Using a supernatural ability is a standard action unless noted otherwise. Supernatural abilities may have a use limit or be usable at will, just like spell-like abilities. However, supernatural abilities do not provoke attacks of opportunity and never require Concentration checks. Unless otherwise noted, a supernatural ability has an effective caster level equal to the creature's \my{Hit Values}. The saving throw (if any) against a supernatural ability is 10 \add 1/2 the creature's HD \add the creature's attribute (usually Charisma).

\subsubsection{Spell Immunity (Ex)} A creature with spell immunity avoids the effects of spells and spell-like abilities that directly affect it. This works exactly like spell resistance, except that it cannot be overcome. Sometimes spell immunity is conditional or applies to only spells of a certain kind or level. Spells that do not allow spell resistance are not affected by spell immunity.

\subsubsection{Spell Resistance (Ex)} A creature with spell resistance can avoid the effects of spells and spell-like abilities that directly affect it.To determine if a spell or spell-like ability works against a creature with spell resistance, the caster must make a caster level check (1d20 \add caster level). If the result equals or exceeds the creature's spell resistance, the spell works normally, although the creature is still allowed a saving throw.

\subsubsection{Spells} Sometimes a creature can cast arcane or divine spells just as a member of a spellcasting class can (and can activate magic items accordingly). Such creatures are subject to the same spellcasting rules that characters are, except as follows. 

A spellcasting creature that lacks hands or arms can provide any somatic component a spell might require by moving its body. Such a creature also does need material components for its spells. The creature can cast the spell by either touching the required component (but not if the component is in another creature's possession) or having the required component on its person. Sometimes spellcasting creatures utilize the Eschew Materials feat to avoid fussing with noncostly components.

A spellcasting creature is not actually a member of a class unless its entry says so, and it does not gain any class abilities. A creature with access to cleric spells must prepare them in the normal manner and receives domain spells if noted, but it does not receive domain granted powers unless it has at least one level in the cleric class.

\subsubsection{Summon (Sp)} A creature with the summon ability can summon specific other creatures of its kind much as though casting a summon monster spell, but it usually has only a limited chance of success (as specified in the creature's entry). Roll d\%: On a failure, no creature answers the summons. Summoned creatures automatically return whence they came after 1 hour. A creature that has just been summoned cannot use its own summon ability for 1 hour. Most creatures with the ability to summon do not use it lightly, since it leaves them beholden to the summoned creature. In general, they use it only when necessary to save their own lives. An appropriate spell level is given for each summoning ability for purposes of Concentration checks and attempts to dispel the summoned creature. No experience points are awarded for summoned monsters.

\subsubsection{Swallow Whole (Ex)} If a creature with this special attack begins its turn with an opponent held in its mouth (see Improved Grab), it can attempt a new grapple check (as though attempting to pin the opponent). If it succeeds, it swallows its prey, and the opponent takes bite damage. Unless otherwise noted, the opponent can be up to one size category smaller than the swallowing creature. Being swallowed has various consequences, depending on the creature doing the swallowing. A swallowed creature is considered to be grappled, while the creature that did the swallowing is not. A swallowed creature can try to cut its way free with any light slashing or piercing weapon (the amount of cutting damage required to get free is noted in the creature description), or it can just try to escape the grapple. The Armor Class of the interior of a creature that swallows whole is normally 10 \add 1/2 its natural armor bonus, with no modifiers for size or Dexterity. If the swallowed creature escapes the grapple, success puts it back in the attacker's mouth, where it may be bitten or swallowed again.

\subsubsection{Telepathy (Su)} A creature with this ability can communicate telepathically with any other creature within a certain range (specified in the creature's entry, usually 100 feet) that has a language. It is possible to address multiple creatures at once telepathically, although maintaining a telepathic conversation with more than one creature at a time is just as difficult as simultaneously speaking and listening to multiple people at the same time.
Some creatures have a limited form of telepathy, while others have a more powerful form of the ability.

\subsubsection{Trample (Ex)} As a full-round action, a creature with this special attack can move up to twice its speed and literally run over any opponents at least one size category smaller than itself. The creature merely has to move over the opponents in its path; any creature whose space is completely covered by the trampling creature's space is subject to the trample attack. If a target's space is larger than 5 feet, it is only considered trampled if the trampling creature moves over all the squares it occupies. If the trampling creature moves over only some of a target's space, the target can make an attack of opportunity against the trampling creature at a –4 penalty. A trampling creature that accidentally ends its movement in an illegal space returns to the last legal position it occupied, or the closest legal position, if there's a legal position that's closer.

A trample attack deals bludgeoning damage (the creature's slam damage \add 1-1/2 times its Str). The creature's descriptive text gives the exact amount.

Trampled opponents can attempt attacks of opportunity, but these take a \minus4 penalty. If they do not make attacks of opportunity, trampled opponents can attempt Reflex saves to take half damage.

The save DC against a creature's trample attack is 10 \add 1/2 creature's HD \add creature's Str (the exact DC is given in the creature's descriptive text). A trampling creature can only deal trampling damage to each target once per round, no matter how many times its movement takes it over a target creature.

\subsubsection{Tremorsense (Ex)} A creature with tremorsense is sensitive to vibrations in the ground and can automatically pinpoint the location of anything that is in contact with the ground. Aquatic creatures with tremorsense can also sense the location of creatures moving through water. The ability's range is specified in the creature's descriptive text.

\subsubsection{Treasure} This entry in a monster description describes how much wealth a creature owns. In most cases, a creature keeps valuables in its home or lair and has no treasure with it when it travels. Intelligent creatures that own useful, portable treasure (such as magic items) tend to carry and use these, leaving bulky items at home. Treasure can include coins, goods, and items. 
\begin{comment}
Creatures can have varying amounts of each, as follows.
\subparhead{Standard} Refer to the treasure tables and roll d\% once for each type of treasure (Coins, Goods, Items) on the Level section of the table that corresponds to the creature's Challenge Rating (for groups of creatures, use the Encounter Level for the encounter instead). Some creatures have double, triple, or even quadruple standard treasure; in these cases, roll for each type of treasure two, three, or four times.
\subparhead{None} The creature collects no treasure of its own.
\subparhead{Nonstandard} Some creatures have quirks or habits that affect the types of treasure they collect. These creatures use the same
treasure tables, but with special adjustments.
\subparhead{Fractional Coins} Roll on the Coins column in the section corresponding to the creature's Challenge Rating, but divide the result as indicated.
\subparhead{\% Goods or Items} The creature has goods or items only some of the time. Before checking for goods or items, roll d\% against the given percentage. On a success, make a normal roll on the appropriate Goods or Items column (which may still result in no goods or items).
\subparhead{Double Goods or Items} Roll twice on the appropriate Goods or Items column.
\subparhead{Parenthetical Notes} Some entries for goods or items include notes that limit the types of treasure a creature collects.

When a note includes the word ``no,'' it means the creature does not collect or cannot keep that thing. If a random roll generates such a result, treat the result as ``none'' instead. 

When a note includes the word ``only,'' the creature goes out of its way to collect treasure of the indicated type. Treat all results from that column as the indicated type of treasure.
It's sometimes necessary to reroll until the right sort of item appears. 
\end{comment}

\subsubsection{Turn Resistance (Ex)} A creature with this special quality (usually an undead) is less easily affected by clerics or paladins. When resolving a turn, rebuke, command, or bolster attempt, add the indicated number to the creature's \my{Hit Values} total.

\subsubsection{Vulnerability to Energy} Some creatures have vulnerability to a certain kind of energy effect (typically either cold or fire). Such a creature takes half again as much (\plus50\%) damage as normal from the effect, regardless of whether a saving throw is allowed, or if the save is a success or failure.
