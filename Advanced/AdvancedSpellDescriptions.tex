\section{Advanced Spell Descriptions}

\pdfbookmark[2]{A}{SpellDescriptionsA}
\begin{comment}
\subsubsection{A}
\end{comment}

\spellsection{Ablate Impact}
\spellschool{Abjuration (Shielding)}
\spelllvl{Abjur 2}
\spelltime{1 immediate action}
\spellrng{Personal}
\spelltgt{You}
\spelldur{1 round}
\begin{spelleffect}
  You gain physical damage reduction 10/force. This damage reduction increases by 1 per caster level above 4th.
\end{spelleffect}
\begin{spellnotes}
  This spell's damage reduction allows the subject to ignore the first 10 physical damage it takes each round. If it is hit by a attack that deals force damage, such as \spell{magic missile}, it cannot use its damage reduction for 1 round.

  You can cast this spell instantaneously, quickly enough react to an opponent attacking you (but before the attack is rolled).
\end{spellnotes}

\spellsection{Ablative Fortress}
\spelldesc{You instantly create a simmering field of magical energy, protecting you and your allies from hostile magic.}
\spellschool{Abjuration (Negation) [Magic]}
\spelllvl{Abjur 4, Magic 4}
\spellarea{\areamed radius limit centered on you}
\spelltgts{All allies within the area}
\spellsave{Will negates (harmless)}
\spellsr{Yes (Will)}
\begin{spelleffect}
  This spell functions like \spell{ablative shield}, except that it affects multiple creatures.
\end{spelleffect}

\spellsection{Ablative Shield}
\spelldesc{You instantly encase yourself a shimmering field of magical energy, protecting you from hostile magic.}
\spellschool{Abjuration (Negation) [Magic]}
\spelllvl{Abjur 1, Magic 1}
\spellcmp{V}
\spelltime{1 immediate action}
\spellrng{\rngpers}
\spelltgt{You}
\spelldur{1 round}
\begin{spelleffect}
  You gain spell damage reduction 5/force. This damage reduction increases by 1 per caster level above 2nd.
\end{spelleffect}
\begin{spellnotes}
  This spell's damage reduction allows the subject to ignore the first 5 spell damage it takes each round, such as from spells and spell-like abilities. If it is hit by a attack that deals force damage, such as \spell{magic missile}, it cannot use its damage reduction for 1 round.
  
  Spells that are not subject to spell resistance are not affected by \spell{ablative shield}. You can cast this spell instantly - quickly enough to gain its benefits in an emergency. Casting the spell is an immediate action, so you can use this spell even when it's not your turn.
\end{spellnotes}

\spellsection{Acid Arrow}
\spelldesc{You fire a magical arrow of acid from your hand that speeds to its target.}
\spellschool{Conjuration (Creation) [Acid]}
\spelllvl{Sor/Wiz 2}
\spellrng{\rngmed}
\spelleff{One arrow of acid}
\spelldur{1 round per two levels}
\spellsave{None}
\spellsr{No}
\spelldmg{2d6 acid damage \add d6 per round}
\begin{spelleffect}
  You must succeed on a ranged touch attack to hit your target. The acid remains on the target after the initial impact, dealing damage each round on your turn.
\end{spelleffect}
\begin{spellnotes}
  If the target becomes submerged in water or takes at least ten points of cold or fire damage, this spell's effect ends.
\end{spellnotes}


\spellsectioncomma{Acid Arrow}{Greater}
\spelldesc{You fire a magical arrow of acid from your hand that speeds to its target.}
\spellschool{Conjuration (Creation) [Acid]}
\spelllvl{Sor/Wiz 5}
\spellrng{\rngfar}
\spellsave{None/Fortitude negates}
\spelldmg{5d6 acid damage \add 2d6 per round}
\begin{spelleffect}
  This spell functions as \spell{acid arrow}, except that the target is also vulnerable for the duration of the spell if it fails a Fortitude save.
\end{spelleffect}
\begin{spellnotes}
  A vulnerable creature takes a \minus2 penalty to attack rolls, saving throws, checks, DCs, and AC. If the target becomes submerged in liquid or takes at least twenty points of fire or cold damage, this spell's effect ends.
\end{spellnotes}

\spellsection{Acid Fog}
\spelldesc{A billowing mass of acidic vapors fills the area, slowing creatures down and obscuring sight.}
\spellschool{Conjuration (Creation) [Acid]}
\spelllvl{Sor/Wiz 8, Destruction 8}
\spellrng{\rngmed}
\spellarea{\areamed radius spread}
\spelleff{Fog in the area}
\spelldur{\durshort}
\spellsave{Fortitude half}
\spellsr{No}
\spelldmg{4d6 acid damage per round}
\begin{spelleffect}
  This spell functions like \spell{solid fog}, except that the spell's vapors are highly acidic, dealing damage to all creatures and objects within the area on each round at the start of your turn. The fog does not do damage in the round it is cast. A successful Fortitude save halves the damage.
\end{spelleffect}

\spellsection{Agony}
\spelldesc{You inflict debilitating pain on your foe, crippling its ability to act.}
\spellschool{Necromancy (Flesh)}
\spelllvl{Sor/Wiz 4}
\spellrng{\rngclose}
\spellsave{None}
\spellsr{Yes (Fortitude)}
\begin{spelleffect}
  The subject suffers a \minus4 penalty to attack rolls, saving throws, checks, DCs, and AC.
\end{spelleffect}

\spellsection{Aid}
\spelldesc{You fill the target with confidence, improving its resilience and stamina in combat.}
\spellschool{Enchantment (Emotion) [Mind-Affecting, Morale]}
\spelllvl{Clr 2, Pal 2}
\spellrng{\rngclose}
\spelltgt{One creature}
\spelldur{\durshort}
\spellsave{Will negates (harmless)}
\spellsr{Yes (Will)}
\begin{spelleffect}
  The subject gains a \plus2 bonus to attack rolls and temporary hit points equal 10 \add 1 per caster level above 4th. \bonusscalingdescription  If you take life damage, you lose all temporary hit points provided by this spell before applying the damage.
\end{spelleffect}

\spellsection{Air Walk}
\spelldesc{You imbue the subject with the ability to walk on nothing but air.}
\spellschool{Transmutation (Imbuement) [Air]}
\spelllvl{Air 4, Clr 4, Drd 4, Travel 4}
\spellrng{\rngtouch}
\spelltgt{Creature (Gargantuan or smaller) touched}
\spelldur{\durshort}
\spellsave{Fortitude negates (harmless)}
\spellsr{Yes (Fortitude)}
\begin{spelleffect}
  The subject can walk on air as if it were solid ground. The magic only affects the subject's legs, and does not grant the ability to climb vertically through the air.
  \par Should the spell end while the subject is still aloft, the magic fails slowly. The subject floats downward 60 feet per round for 1d6 rounds. If it reaches the ground in that amount of time, it lands safely. If not, it falls the rest of the distance, taking 1d6 damage per 10 feet of fall.
\end{spelleffect}
\begin{spellnotes}
  A strong wind (21\add mph) can push the subject along or hold it back. At the end of its turn each round, the wind blows the air walker 5 feet for each 5 miles per hour of wind speed. The creature may be subject to additional penalties in exceptionally strong or turbulent winds, such as loss of control over movement or physical damage from being buffeted about.
  \par You can cast \spell{air walk} on a specially trained mount so it can be ridden through the air. You can train a mount to move with the aid of \spell{air walk} (counts as a trick; see Handle Animal skill) with one week of work and a DC 25 Handle Animal check.
\end{spellnotes}

\spellsection{Align Weapon}
\spelldesc{You enhance a weapon while bringing it closer to your ideals.}
\spellschool{Evocation/Transmutation (Augment, Channeling) [see text]}
\spelllvl{Chaos 2, Evil 2, Good 2, Law 2}
\spellrng{Touch}
\begin{spelleffect}
  This spell functions like \spell{magic weapon}, except that it also makes a weapon good, evil, lawful, or chaotic, as you choose, allowing it to overcome damage reduction of the appropriate type. When cast on a weapon that already has an alignment, this spell overrides the alignment of the weapon unless the weapon makes a Will save.
\end{spelleffect}
\begin{spellnotes}
  When you make a weapon good, evil, lawful, or chaotic, \spell{align weapon} is a good, evil, lawful, or chaotic spell, respectively.
\end{spellnotes}

\spellsection{Analyze Dweomer}
\spelldesc{You discern all spells and magical properties present in a number of creatures or objects.}
\spellschool{Divination (Awareness, Knowledge) [Magic]}
\spelllvl{Div 6, Knowledge 6, Magic 6}
\spellcmp{V, S, F}
\spellrng{\rngclose}
\spelltgt{One object or creature per round}
\spelldur{\durmed (D)}
\spellsave{None or Will negates; see text}
\spellsr{No}
\begin{spelleffect}
  Each round, you may examine a single creature or object that you can see as a swift action. In the case of a magic item, you learn its functions, how to activate its functions (if appropriate), and how many charges are left (if it uses charges). In the case of an object or creature with active spells cast upon it, you learn each spell, its effect, and its caster level.
  \par An attended object may attempt a Will save to resist this effect if its holder so desires. If the save succeeds, you learn nothing about the object except what you can discern by looking at it. An object that makes its save cannot be affected by any other \spell{analyze dweomer} spells for 24 hours.
\end{spelleffect}
\begin{spellnotes}
  \spell{Analyze dweomer} gives only partial information when used on an artifact.
\end{spellnotes}
\spellfocus{A tiny lens of ruby or sapphire set in a small golden loop. The gemstone must be worth at least 1,500 gp.}

\spellsection{Animal Growth}
\spelldesc{You cause a number of animals grow to twice their normal size and eight times their normal weight.}
\spellschool{Transmutation (Polymorph) [Size-Affecting]}
\spelllvl{Drd 7, Nature 7}
\spellrng{\rngmed}
\spellarea{\areamed radius limit}
\spelltgts{Five animals (Gargantuan or smaller) within the area}
\begin{spelleffect}
  This spell functions like \spell{enlarge person}, except that it affects multiple animals, as noted above.
\end{spelleffect}

\spellsection{Animate Objects}
\spelldesc{You imbue inanimate objects with mobility and a semblance of life.}
\spellschool{Transmutation (Animation)}
\spelllvl{Chaos 5, Trans 5}
\spellrng{\rngmed}
\spellarea{\areamed radius limit}
\spelltgts{One Small object/level within the area; see text}
\spelldur{\durshort}
\spellsave{None}
\spellsr{No}
\begin{spelleffect}
  Each animated object immediately attacks whomever or whatever you initially designate. Your control of the objects is limited to simple commands (``Attack,'' ``Defend,'' ``Stop,'' and so forth).
  \par An animated object can be of any nonmagical material. You may animate one Small or smaller object or an equivalent number of larger objects per caster level. A Medium object counts as two Small or smaller objects, a Large object as four, a Huge object as eight, a Gargantuan object as sixteen, and a Colossal object as thirty-two. You can change the designated target or targets as a move action, as if directing an active spell.
\end{spelleffect}
\begin{spellnotes}
  This spell cannot animate objects carried or worn by a creature.
  \par \spell{Animate objects} can be made permanent with a \spell{permanency} spell.
\end{spellnotes}

\spellsection{Animate Plants}
\spelldesc{You imbue inanimate plants with mobility and a semblance of life.}
\spellschool{Transmutation (Animation)}
\spelllvl{Drd 5, Plant 5}
\spelltgts{One Small plant/level within the area; see text}
\begin{spelleffect}
  This spell functions as \spell{animate objects}, except that you animate plants instead of inanimate objects.
\end{spelleffect}
\begin{spellnotes}
  \spell{Animate plants} cannot affect plant creatures, nor does it affect nonliving vegetable material.
\end{spellnotes}

\spellsection{Antilife Shell}
\spelldesc{You create an immobile, spherical energy field that hedges out living creatures.}
\spellschool{Abjuration (Interdiction) [Barrier]}
\spelllvl{Clr 7, Drd 6, Nature 7}
\spellarea{\areasmall radius emanation, centered on your location}
\spelldur{\durlong (D)}
\spellsave{None}
\spellsr{Yes (Will)}
\begin{spelleffect}
  Living creatures cannot enter the spell's area. Nonliving creatures, such as constructs and undead, suffer no ill effect.
\end{spelleffect}
\begin{spellnotes}
  Barrier spells may be used only defensively, not aggressively. Creatures in the area at the time that the spell is cast are unaffected by the spell.
\end{spellnotes}

\spellsection{Antimagic Field}
\spelldesc{You create a mobile, spherical energy field that suppresses magic.}
\spellschool{Abjuration (Negation) [Magic]}
\spelllvl{Clr 8, Magic 7, Sor/Wiz 7}
\spellarea{\areasmall radius emanation, centered on you}
\spelldur{\durlong (D)}
\spellsave{None}
\spellsr{No}
\begin{spelleffect}
  All spells, spell-like abilities, supernatural abilities, and magic items fail to function within the area of this spell. They cannot be activated from within the field, and any existing effects brought into or cast into the area are suppressed. Time spent within an \spell{antimagic field} counts against a suppressed spell's duration.
  \par Summoned creatures of any type and incorporeal undead disappear if they enter an \spell{antimagic field}. They reappear in the same spot once the field goes away. (The effects of instantaneous conjurations, such as \spell{create water}, are not affected by an \spell{antimagic field} because the conjuration itself is no longer in effect, only its result.)
  \par Creatures within an \spell{antimagic field} cannot dismiss spells. However, you can dismiss your own antimagic field.
\end{spelleffect}
\begin{spellnotes}
  A normal creature can enter the area, as can normal missiles. Furthermore, while a magic sword does not function magically within the area, it is still a sword. The spell has no effect on golems and other constructs that are imbued with magic during their creation process and are thereafter self-supporting (unless they have been summoned, in which case they are treated like any other summoned creatures). Elementals, corporeal undead, and outsiders are likewise unaffected unless summoned.
  \par \spell{Dispel magic} does not remove the field. Two or more \spell{antimagic fields} sharing any of the same space have no effect on each other. Certain spells, such as \spell{wall of force}, \spell{prismatic sphere}, and \spell{prismatic wall}, remain unaffected by \spell{antimagic field} (see the individual spell descriptions).
  \par Any part of a creature that lies outside the field is unaffected by the field.
  \par Artifacts and deities are unaffected by mortal magic such as this. 
\end{spellnotes}

\spellsection{Aqueous Blade}
\spelldesc{You transform the active part of your ally's weapon into water, weakening its blows but allowing it penetrate your foe's defenses more easily.}
\spellschool{Transmutation (Alteration) [Water]}
\spelllvl{Drd 2, Water 2}
\spellrng{\rngclose}
\spelltgt{One weapon}
\spelldur{\durshort (D)}
\spellsave{Will negates}
\spellsr{Yes (Will)}
\begin{spelleffect}
  Attacks with the affected weapon are made as touch attacks. However, damage with the weapon is halved, including any bonuses to weapon damage.
\end{spelleffect}

\spellsection{Arcane Sight}
\spelldesc{Your eyes glow blue with power. All nearby magical auras become apparent to you.}
\spellschool{Divination (Awareness) [Magic]}
\spelllvl{Sor/Wiz 2}
\spellrng{\rngpers}
\spelltgt{You}
\spelldur{\durlong (D)}
\begin{spelleffect}
  You know the location and power of all magical auras that you can see within \rngmed range of you. An aura's power depends on a spell's functioning level or an item's caster level, as noted in the description of the Spellcraft skill. If the items or creatures bearing the auras are in line of sight, you can make Spellcraft skill checks to determine the school of magic involved in each. (Make one check per aura; DC 15 \add spell level, or 15 \add one-half caster level for a nonspell effect.)
  \par If you concentrate on a specific creature within \rngmed range of you as a standard action, you can determine whether it has any spellcasting or spell-like abilities, whether these are arcane or divine (spell-like abilities register as arcane), and the strength of the most powerful spell or spell-like ability the creature currently has available for use.
\end{spelleffect}
\begin{spellnotes}
  \spell{Arcane sight} can be made permanent with a \spell{permanency} spell.
\end{spellnotes}

\spellsectioncomma{Arcane Sight}{Greater}
\spelldesc{Your eyes glow an intense blue as you gain the ability to discern all nearby magical auras at a glance.}
\spellschool{Divination (Awareness) [Magic]}
\spelllvl{Sor/Wiz 7}
\spelldur{\durext (D)}
\begin{spelleffect}
  This spell functions like \spell{arcane sight}, except that you automatically know which spells or magical effects are active upon any individual or object you see, and you can concentrate on specific creatures to learn about their spellcasting abilities as a swift action. In addition, you automatically recognize any spells being cast within the area.
\end{spelleffect}
\begin{spellnotes}
  \par \spell{Greater arcane sight} doesn't let you identify magic items. Unlike \spell{arcane sight}, this spell cannot be made permanent with a permanency spell.
\end{spellnotes}

\spellsection{Assimilate}
\spelldesc{Your pointing finger turns black as obsidian. You touch a creature and it dissolves into dust as you assimilate its form into your own body.}
\spellschool{Necromancy/Transmutation (Augment, Life)}
\spelllvl{Evil 9, Sor/Wiz 9}
\spellrng{\rngtouch}
\spelltgt{Living creature touched}
\spelldur{Instantaneous and one hour; see text}
\spellsave{Fortitude half}
\spellsr{Yes (Fortitude)}
\spelldmg{18d8 life damage \add d6 per four caster levels above 18th}
\par Any creature reduced to 0 or fewer hit points by this spell is entirely assimilated into your form, leaving behind only a trace of fine dust. An assimilated creature's equipment is unaffected.
\par If the creature has at least 1 hit point following your use of this power, you gain temporary hit points equal to half the damage you dealt for 1 hour.
\par If the creature is completely assimilated, you gain a number of temporary hit points equal to the damage you dealt and a \plus4 bonus to each of your attributes for 1 hour. In addition, you gain the appearance of the creature for 1 hour, granting you a \plus10 bonus on Disguise checks made to appear as that creature during that time.

If you take life damage, you lose all temporary hit points provided by this spell before applying the damage.

\spellsection{Attraction}
\spelldesc{You cause the subject to feel attracted to something.}
\spellschool{Enchantment (Emotion) [Mind-Affecting]}
\spelllvl{Ench 1}
\spellrng{\rngmed}
\spelltgt{One creature}
\spelldur{\durext}
\spellsave{Will negates}
\spellsr{Yes (Will)}
\begin{spelleffect}
  An affected creature feels attracted to a particular person or object. The subject will take reasonable steps to meet, get close to, attend, or find the object of its implanted attraction. For the purpose of this spell, ``reasonable'' means that, while attracted, the subject doesn't suffer from blind obsession. He will act on this attraction only when not engaged in combat. The subject won't perform obviously suicidal actions. He can still recognize danger but will not flee unless the threat is immediate. If you make the subject feel an attraction to yourself, you can't command him indiscriminately, although he will be willing to listen to you (even if he disagrees).
  \par This spell grants you a \plus4 circumstance bonus on any social interaction checks you make involving the subject (such as Bluff, Diplomacy, Intimidate, and Sense Motive).
\end{spelleffect}

\spellsection{Aversion}
\spelldesc{You make the subject want to avoid something.}
\spellschool{Enchantment (Emotion) [Mind-Affecting]}
\spelllvl{Ench 2}
\spellrng{\rngmed}
\spelltgt{One creature}
\spelldur{\durext}
\spellsave{Will negates}
\spellsr{Yes (Will)}
\begin{spelleffect}
  An affected creature feels an aversion to a particular person or object. If the object of the implanted aversion is an individual or a physical object, she will prefer not to approach within 30 feet of it. If it is a word, she will try not to utter it; if it is an action, she will not willingly attempt to perform it; and if it is an event, she will not willingly attend it. The subject will take reasonable steps to avoid the object of its aversion, but will not put herself in jeopardy by doing so.
  \par If the subject is forced into taking an action she has an aversion to, she is bewildered as long as she performs the action, making her vulnerable.
\end{spelleffect}
\begin{spellnotes}
  A vulnerable creature takes a \minus2 penalty to attack rolls, saving throws, checks, DCs, and AC.
\end{spellnotes}

\pdfbookmark[2]{B}{SpellDescriptionsB}
\begin{comment}
\subsubsection{B}
\end{comment}

\spellsection{Backbiter}
\spelldesc{You subtly animate a weapon so that it strikes its wielder instead of its intended target.}
\spellschool{Transmutation (Animation)}
\spelllvl{Trans 1}
\spellrng{\rngmed}
\spelltgt{One weapon}
\spelldur{\durshort or until discharged}
\spellsave{Will negates (object)}
\spellsr{Yes (Will)}
\begin{spelleffect}
  The next time the affected weapon is used to make a melee attack, it twists around so that the weapon automatically strikes the wielder instead. The wielder gets no warning or knowledge of the spell's effect on his weapon, and though he makes the attack, the self-dealt damage can't be consciously reduced (though damage reduction applies) or changed to nonlethal damage.
  \par Once the weapon attacks its wielder (whether successfully or not), the spell is discharged.
\end{spelleffect}

\spellsection{Baleful Polymorph}
\spelldesc{You transmute your foe into a small, insignificant animal.}
\spellschool{Transmutation (Polymorph)}
\spelllvl{Drd 6, Trans 5}
\spellrng{Touch}
\spelltgt{Creature touched}
\spelldur{\durshort or Permanent (D); see text}
\spellsave{Fortitude negates, Will partial; see text}
\spellsr{Yes (Fortitude)}
\begin{spellhealthy}
  The subject is sickened, making it vulnerable.
\end{spellhealthy}
\begin{spellblood}
  You change the subject into a Small or smaller animal of no more than 1 HD (such as a dog, lizard, monkey, or toad). The subject takes on all the statistics and special abilities of an average member of the new form in place of its own except as follows:
  \begin{itemize*} 
    \item The target retains its own alignment, personality, and mental attributes.
    \item If the target has the shapechanger subtype, it retains that subtype. 
    \item The target retains its own hit points. 
    \item The target is treated has having its normal Hit Values for purpose of adjudicating effects based on HV, though it uses the new form's base attack bonus, base save bonuses, and all other statistics derived from Hit Values. 
    \item The target also retains the ability to understand (but not to speak) the languages it understood in its original form. It can write in the languages it understands, but only the form is capable of writing in some manner (such as drawing in the dirt with a paw). 
  \end{itemize*}
  With those exceptions, the target's normal game statistics are replaced by those of the new form. The target loses all the special abilities it has in its normal form, including its class features.

  All items worn or carried by the subject fall to the ground at its feet, even if they could be worn or carried by the new form. 

  If the new form would prove fatal to the creature (for example, if you polymorphed a landbound target into a fish, or an airborne target into a toad), the subject gets a \plus4 bonus on the save. 

  If the spell succeeds, the subject must also make a Will save. If the second save fails, the transformation is permanent. Otherwise, the creature reverts to its true form after the \durshort duration.

  If the subject remains in the new form for 24 consecutive hours, it must attempt another Will save. If this save fails, it loses its ability to understand language, as well as all other memories of its previous form, and its Hit Values, hit points, and mental attributes change to match an average creature of its new form. These abilities and statistics return to normal if the effect is later ended. The subject must repeat this save every 24 hours that it remains in its new form. 
\end{spellblood}
\begin{spellnotes}
  A vulnerable creature takes a \minus2 penalty to attack rolls, saving throws, checks, DCs, and AC.
  Incorporeal or gaseous creatures are immune to \spell{baleful polymorph}, and a creature with the shapechanger subtype (such as a lycanthrope or a doppelganger) can revert to its natural form as a standard action (which ends the spell's effect). 
\end{spellnotes}

\spellsection{Bane}
\spelldesc{You fill your enemies with dismay, impairing their ability to fight.}
\spellschool{Enchantment (Emotion) [Mind-Affecting, Morale]}
\spelllvl{Clr 1, Evil 1, War 1}
\spellarea{\areamed radius burst}
\spelldur{Instantaneous}
\spellsave{None}
\spellsr{Yes (Will)}
\begin{spelleffect}
  All enemies within the area take a \minus2 penalty to attack rolls for 5 rounds.
\end{spelleffect}
\begin{spellnotes}
  \spell{Bane} counters and dispels \spell{bless}.
\end{spellnotes}

\spellsection{Banishment}
\spelldesc{You force extraplanar creatures back to their home plane.}
\spellschool{Abjuration/Conjuration (Interdiction, Translocation) [Planar]}
\spelllvl{Clr 6, Sor/Wiz 6}
\spellcmp{V, S, F}
\spellrng{\rngmed}
\spelltgts{One extraplanar creature/round}
\spelldur{Concentration}
\begin{spelleffect}
  This spell functions like \spell{dismissal}, except that you can banish one additional extraplanar creature each round that you concentrate on the spell. An individual creature can only be targeted once per casting of this spell.
\end{spelleffect}
\begin{spellnotes}
  You can improve the spell's chance of success by presenting at least one object or substance that the target hates, fears, or otherwise opposes. For each such object or substance, you gain a \plus2 circumstance bonus on your caster level with the spell. For example, if this spell were cast on a demon that hated light and was vulnerable to holy water and cold iron weapons, you might use iron, holy water, and a torch in the spell. The three items would give you a \plus6 bonus on your caster level. 
  \par Certain rare items might work twice as well as a normal item for the purpose of the bonuses (each providing a \plus4 circumstance bonus to your caster level).
\end{spellnotes}
\spellfocus{Any item that is distasteful to the subject (optional, see above)}

\spellsection{Barkskin}
\spelldesc{You toughen a creature's skin, giving it the appearance of tree bark.}
\spellschool{Transmutation (Augment)}
\spelllvl{Drd 2, Nature 2}
\spellrng{\rngtouch}
\spelltgt{Living creature touched}
\spelldur{\durshort}
\spellsave{Fortitude negates (harmless)}
\spellsr{Yes (Fortitude)}
\begin{spelleffect}
  The subject gains a \plus2 bonus to its natural armor modifier. \bonusscalingdescription In addition, the subject gains physical damage reduction 2/adamantine or fire. This damage reduction increases by 1 for every four levels above 4th.
\end{spelleffect}
\begin{spellnotes}
  This spell's damage reduction allows the subject to ignore the first 2 physical damage it takes each round. If it is hit by a adamantine weapon or an attack that deals fire damage, it cannot use its damage reduction for 1 round.
\end{spellnotes}

\spellsection{Bestow Curse}
\spelldesc{You place a curse on your foe, crippling its ability to act.}
\spellschool{Necromancy (Life) [Curse]}
\spelllvl{Clr 5, Death 5, Evil 5, Necro 5}
\spellrng{\rngclose}
\spelltgt{One creature}
\spelldur{Permanent}
\spellsave{Will negates}
\spellsr{Yes (Will)}
\begin{spelleffect}
  The subject suffers one of the following three effects, chosen by you:
  \begin{itemize*}
    \item \minus6 penalty to an attribute.
    \item \minus4 penalty on attack rolls, saving throws, checks, DCs, and AC.
    \item Each turn, the target has a 25\% chance to take no action; otherwise, it acts normally.
  \end{itemize*}
  \par You may also invent your own curse, but it should be no more powerful than those described above.
\end{spelleffect}
\begin{spellnotes}
  Curses cannot be dispelled. \spell{Bestow curse} counters \spell{remove curse}.
\end{spellnotes}

\spellsection{Black Tentacles}
\spelldesc{You conjure a field of rubbery black tentacles, each 5 feet long. These waving members seem to spring forth from the earth, floor, or whatever surface is underfoot -- including water. They grasp and entwine around creatures that enter the area, holding them fast and crushing them with great strength.}
\spellschool{Conjuration/Transmutation (Animation, Creation)}
\spelllvl{Sor/Wiz 7}
\spellrng{\rngmed}
\spellarea{\areasmall radius spread}
\spelleff{Black tentacles in the area}
\spelldur{\durshort (D)}
\spellsave{None}
\spellsr{No}
\begin{spelleffect}
  At the start of your turn, every creature within the area of the spell is the target of a grapple attack. This attack is also made in the round that \spell{black tentacles} is cast. Treat the tentacles attacking a particular target as a Medium creature with a base attack bonus equal to your caster level and a Strength score of 8. Thus, its grapple attack modifier is equal to your caster level \plus8. Roll only once each round for the entire spell effect, and apply the result to all creatures within the area of effect.
  \par If the tentacles succeed in grappling a foe, that foe takes 1d8\plus4 damage. Each round that black tentacles succeeds on a grapple attack, it deals an additional 1d8\plus4 damage. The tentacles continue to crush the opponent until the spell ends or the opponent escapes.
  \par The tentacles are immune to all types of damage. The entire area of effect is considered difficult terrain while the tentacles last, and any creature that enters the area of the spell is immediately attacked by the tentacles.
\end{spelleffect}

\spellsection{Blade Barrier}
\spelldesc{You create an immobile, vertical curtain of whirling blades shaped of pure force.}
\spellschool{Evocation (Energy) [Force, Wall]}
\spelllvl{Clr 6, War 6}
\spellrng{\rngmed}
\spelleff{Wall of whirling blades up to 100 ft. long, or a ringed wall of whirling blades with a radius of up to 20 ft.; either form 20 ft. high}
\spelldur{\durshort (D)}
\spellsave{Reflex half or Reflex negates; see text}
\spellsr{Yes (Reflex)}
\spelldmg{6d6 force damage \add d6 per four caster levels above 12th}
\begin{spelleffect}
  Any creature passing through the wall takes damage, with a Reflex save for half. If you create the wall so that it appears where creatures are, each creature takes damage as if passing through the wall. Each such creature can avoid the wall (ending up on the side of its choice) and thus take no damage by making a successful Reflex save.
  \par A \spell{blade barrier} provides cover (\plus4 circumstance bonus to AC, \plus2 circumstance bonus on Reflex saves) against attacks made through it.
\end{spelleffect}

\spellsection{Blasphemy}
\spelldesc{You speak an unholy utterance of great power, afflicting all those nearby who do not share your allegiance to evil.}
\spellschool{Evocation (Channeling) [Evil]}
\spelllvl{Cleric 7, Evil 7}
\spellcmp{V}
\spellarea{\arealarge radius spread centered on you}
\spelldur{Instantaneous}
\spellsave{None}
\spellsr{Yes (Will)}
\begin{spellhealthy}
  \par Each nonevil creature in the area is sickened, making it vulnerable for 5 rounds.
\end{spellhealthy}
\begin{spellblood}
  \par Each nonevil creature in the area suffers one or more of the following ill effects, depending on its Hit Values.
  \begin{dtable}
    \begin{tabularx}{\columnwidth}{l >{\lcol}X}
      \par \thead{HV} & \thead{Effect} \\
      \par Equal to caster level & Sickened \\
      \par Up to caster level \minus5 & Nauseated, sickened \\
      \par Up to caster level \minus10 & Paralyzed, nauseated, sickened \\
      \par Up to caster level \minus15 & Killed\fn{1}
    \end{tabularx}
    1 Living creatures die. Nonliving creatures are destroyed.
  \end{dtable}
  \par \subspell{Sickened} The creature is sickened, making it vulnerable for 5 rounds.
  \par \subspell{Nausated} The creature is nauseated for 1 round.
  \par \subspell{Paralyzed} The creature is paralyzed and helpless for 5 rounds.
  \par \subspell{Killed} Living creatures die. Nonliving creatures are destroyed.
\end{spellblood}
\begin{spellnotes}
  A vulnerable creature takes a \minus2 penalty to attack rolls, saving throws, checks, DCs, and AC.
  Creatures whose Hit Values exceed your caster level are unaffected by \spell{blasphemy}.
\end{spellnotes}

\spellsection{Bless}
\spelldesc{You fill your allies with confidence, improving their prowess in combat.}
\spellschool{Enchantment (Emotion) [Mind-Affecting, Morale]}
\spelllvl{Clr 2, Good 2, Pal 2, War 2}
\spellarea{\areamed radius burst}
\spelldur{Instantaneous}
\spellsave{None}
\spellsr{Yes (Will)}
\begin{spelleffect}
  All allies within the area gain a \plus2 bonus to attack rolls for 5 rounds. \bonusscalingdescription
\end{spelleffect}
\begin{spellnotes}
  \spell{Bless} counters and dispels \spell{bane}.
\end{spellnotes}

\spellsection{Bless Weapon}
\spelldesc{You imbue a weapon with divine power, causing it to strike true against evil foes.}
\spellschool{Evocation/Transmutation (Channeling, Imbuement) [Good]}
\spelllvl{Pal 2}
\spellcmp{V}
\begin{spelleffect}
  This spell functions like \spell{magic weapon}, except that the weapon also becomes good, which means it can bypass the damage reduction of certain creatures. (This effect overrides and suppresses any other alignment the weapon might have.)
\end{spelleffect}

\spellsection{Blindness/Deafness}
\spelldesc{You afflict one of the subject's senses.}
\spellschool{Necromancy (Flesh)}
\spelllvl{Clr 3, Death 3, Pal 3, Sor/Wiz 2}
\spellrng{Touch}
\spelltgt{Creature touched}
\spelldur{\durlong (D)}
\spellsave{Fortitude negates}
\spellsr{Yes (Fortitude)}
\begin{spellhealthy}
  The subject is sickened, making it vulnerable.
\end{spellhealthy}
\begin{spellblood}
  The subject is blinded, deafened, or sickened, as you choose.
\end{spellblood}
\begin{spellnotes}
  A vulnerable creature takes a \minus2 penalty to attack rolls, saving throws, checks, DCs, and AC. A blinded character cannot see. She takes a \minus2 penalty to attack rolls, Armor Class, and any checks which involve sight. In addition, she is flat-footed and moves at half speed. All checks and activities that rely on vision (such as reading and Spot checks) automatically fail. A blinded character has a 50\% miss chance on all attacks. A deafened character cannot hear. She automatically fails Listen checks, takes a \minus2 penalty to any checks which involve hearing, and has a 20\% chance of spell failure when casting spells with verbal components.
  
  The choice of bloodied conditions is made at the time the spell is cast.
\end{spellnotes}

\spellsection{Blink}
\spelldesc{You rapidly blink in and out of reality, confounding your foes and protecting you from their attacks.}
\spellschool{Conjuration (Translocation) [Planar]}
\spelllvl{Sor/Wiz 4}
\spellrng{\rngpers}
\spelltgt{You}
\spelldur{\durshort (D)}
\begin{spelleffect}
  You ``blink" back and forth between the Material Plane and the Ethereal Plane. This has several effects, as follows.
  \begin{itemize*}
    \item All attacks made against you and spells targeted on you have a 50\% chance to fail. This failure chance is reduced to 20\% if the attack can strike ethereal targets or if the attacker can see ethereal targets. If both are true, the attack suffers no chance of failure. Force effects can strike ethereal targets.
    \item You take half damage from area attacks (but full damage from those that extend onto the Ethereal Plane).
    \item You take half damage from falling, since you fall only while you are material.
    \item All of your attacks and spells have a 20\% chance to happen while you are in the Ethereal Plane, which usually means they have no effect.
    \item You can move at only three-quarters speed (because movement on the Ethereal Plane is at half speed, and you spend about half your time there and half your time material.)
    \item You can step through (but not see through) solid objects. For each 5 feet of solid material you walk through, there is a 50\% chance that you become material. If this occurs, you are shunted off to the nearest open space and take 1d6 damage per 5 feet so traveled. 
    \item You can see and interact with ethereal creatures in roughly the same way you interact with material ones.
  \end{itemize*}
\end{spelleffect}

\spellsection{Blur}
\spelldesc{You distort the subject's outline so it appears blurred, shifting, and wavering.}
\spellschool{Illusion (Glamer)}
\spelllvl{Sor/Wiz 2}
\spellrng{\rngclose}
\spelltgt{One creature}
\spelldur{\durshort (D)}
\spellsave{Will negates (harmless)}
\spellsr{Yes (Will)}
\begin{spelleffect}
  The subject gains concealment, granting it a \plus4 circumstance bonus to AC. This concealment allows the subject to use Stealth without other cover or concealment, though other restrictions apply as normal.
\end{spelleffect}
\begin{spellnotes}
  A \spell{see invisibility} spell does not counteract the blurring effect, but a \spell{true seeing} spell does.
  \par Opponents that cannot see the subject ignore the spell's effect (though fighting an unseen opponent carries penalties of its own).
\end{spellnotes}

\spellsection{Burning Hands}
\spelldesc{You expel a cone of searing flame shoots from your fingertips, searing creatures in front of you.}
\spellschool{Evocation (Energy) [Fire]}
\spelllvl{Destruction 1, Fire 1, Sor/Wiz 1}
\spellarea{\areamed cone-shaped burst}
\spelldur{Instantaneous}
\spellsave{Reflex half}
\spellsr{Yes (Reflex)}
\spelldmg{1d6 fire damage \add 1d6 per four caster levels above 2nd}
\begin{spelleffect}
  Everything in the area takes damage. Unattended flammable objects burn if the flames touch them. A character can extinguish burning items as a full-round action.
\end{spelleffect}

\pdfbookmark[2]{C}{SpellDescriptionsC}
\begin{comment}
\subsubsection{C}
\end{comment}

\spellsection{Call Lightning}
\spelldesc{You repeatedly call bolts of lightning that flash down from thin air to smite your foes.}
\spellschool{Evocation (Energy) [Electricity]}
\spelllvl{Air 3, Drd 3}
\spelltime{Full-round action}
\spellrng{\rngmed}
\spellarea{\arealarge vertical line of lightning, 5 ft. wide}
\spelldur{Instantaneous and \durmed (D); see text}
\spellsave{Reflex half}
\spellsr{Yes (Reflex)}
\spelldmg{3d8 electricity damage \add d8 per four caster levels above 6th}
\begin{spelleffect}
  Immediately upon completion of the spell, and once per round thereafter, you may call down a vertical bolt of lightning which deals damage to anyone in its path. Calling a bolt is a standard action that requires concentration. You may call a total number of bolts equal to your caster level.
  \par If you are outdoors and in a stormy area -- a rain shower, clouds and wind, hot and cloudy conditions, or even a tornado (including a whirlwind formed by a djinni or an air elemental of at least Large size) -- each bolt deals 3d8 electricity damage \add d8 per four caster levels above 6th instead.
\end{spelleffect}
\begin{spellnotes}
  This spell functions indoors or underground, but not underwater.
\end{spellnotes}

\spellsectioncomma{Call Lightning}{Greater}
\spelldesc{You repeatedly call intense bolts of lightning that flash down from thin air to smite your foes.}
\spellschool{Evocation (Energy) [Electricity]}
\spelllvl{Air 5, Drd 5}
\spellsave{Reflex half/Reflex negates}
\spelldmg{5d8 electricity damage \add d8 per four caster levels above 10th}
\begin{spellblood}
  A creature struck by a bolt is also staggered for 1 round. It can take a move action or a standard action each round, but not both. A successful Reflex save to halves the damage and negates the staggering.
\end{spellblood}
\begin{spelleffect}
  This spell functions like \spell{call lightning}, except as noted above. If you are outdoors in a stormy area, each bolt deals 5d8 electricity damage \add d8 per four caster levels above 10th instead.
\end{spelleffect}
\begin{spellnotes}
 A staggered character may take a single move action or standard action each round, but not both. She cannot take full-round actions, but she may take swift actions. In addition, she is vulnerable, causing her to take a \minus2 penalty on attack rolls, saving throws, checks, DCs, and AC.
\end{spellnotes}

\spellsection{Calm Emotions}
\spelldesc{You calm a group of creatures, preventing the situation from getting out of hand.}
\spellschool{Enchantment (Emotion) [Mind-Affecting]}
\spelllvl{Sor/Wiz 2}
\spellrng{\rngmed}
\spellarea{\areamed radius spread}
\spelldur{Concentration}
\spellsave{Will negates}
\spellsr{Yes (Will)}
\begin{spelleffect}
  Creatures in the area have their emotions calmed. Creatures so affected cannot take violent actions (although they can defend themselves) or do anything destructive.
\end{spelleffect}
\begin{spellnotes}
  Any aggressive action against or damage dealt to a calmed creature immediately breaks the spell on all calmed creatures.

  This spell automatically suppresses (but does not dispel) any effects of spells or abilities that affect or require emotions, including all other enchantment (emotion) spells.
\end{spellnotes}

\spellsection{Cause Fear}
\spelldesc{You fill your enemy with fear.}
\spellschool{Enchantment (Emotion) [Fear, Mind-Affecting]}
\spelllvl{Clr 1, Ench 1}
\spellrng{\rngclose}
\spelltgt{One creature}
\spelldur{\durshort/1 round (D)}
\spellsave{Will negates}
\spellsr{Yes (Will)}
\begin{spellhealthy}
  The subject is shaken, causing it to be vulnerable.
\end{spellhealthy}
\begin{spellblood}
  As the healthy effect, plus the subject is frightened for 1 round.
\end{spellblood}
\begin{spellnotes}
  A vulnerable creature takes a \minus2 penalty to attack rolls, saving throws, checks, DCs, and AC.
\end{spellnotes}

\spellsection{Chain Lightning}
\spelldesc{You create a stroke of lightning which strikes a single foe before arcing to hit a number of other foes of your choice.}
\spellschool{Evocation (Energy) [Electricity]}
\spelllvl{Destruction 5, Drd 5, Sor/Wiz 5}
\spellrng{\rngmed}
\spellarea{\areamed radius centered on the primary target}
\spelltgts{One primary target, plus five secondary targets within the area}
\spelldur{Instantaneous}
\spellsave{Reflex half}
\spellsr{Yes (Reflex)}
\spelldmg{5d10 electricity damage \add d10 per four caster levels above 10th}
\begin{spelleffect}
  This spell deals full damage to the primary target and half damage to each of the secondary targets. No secondary target can be struck more than once. You can choose to affect fewer secondary targets than the maximum.
\end{spelleffect}

\spellsection{Changestaff}
\spelldesc{You plant your staff in the ground and transform it into a massive tree-like creature which obeys your every command.}
\spellschool{Transmutation (Alteration, Animation)}
\spelllvl{Drd 7, Nature 8}
\spellcmp{V, S, F}
\spelltime{Full-round action}
\spellrng{\rngtouch}
\spelltgt{Your touched staff}
\spelldur{\durmed (D)}
\spellsave{None}
\spellsr{No}
\begin{spelleffect}
  Your staff turns into a creature that looks and fights just like a treant. The staff-treant defends you and obeys any spoken commands. However, it is by no means a true treant; it cannot converse with actual treants or control trees. If the staff-treant is reduced to 0 or fewer hit points, it crumbles to powder and the staff is destroyed. Otherwise, the staff returns to its normal form when the spell duration expires (or when the spell is dismissed), and it can be used as the focus for another casting of the spell. The staff-treant is always at full strength when created, despite any wounds it may have incurred the last time it appeared.
\end{spelleffect}
\spellfocus{The quarterstaff, which must be specially prepared. The staff must be a sound limb cut from an ash, oak, or yew, then cured, shaped, carved, and polished (a process requiring twenty-eight days). You cannot adventure or engage in other strenuous activity during the shaping and carving of the staff.}

\spellsection{Chaos Hammer}
\spelldesc{You unleash a multicolored explosion of leaping, ricocheting energy to smite your foes.}
\spellschool{Evocation (Channeling) [Chaotic]}
\spelllvl{Chaos 4}
\spellrng{\rngmed}
\spelltgt{One creature}
\spelldur{Instantaneous/5 rounds}
\spellsave{None/Will half}
\spellsr{Yes (Will)}
\spelldmg{8d6 divine damage \add d6 per two caster levels above 8th}
\begin{spelleffect}
  If the target is not chaotic, it takes damage and is bewildered for 5 rounds. A successful Will save halves the damage.
\end{spelleffect}

\spellsection{Charm Monster}
\spelldesc{You manipulate a creature's mind so it thinks of you as a trusted friend and ally.}
\spellschool{Enchantment (Emotion) [Charm, Mind-Affecting]}
\spelllvl{Ench 5}
\spelltgt{One creature}
\begin{spelleffect}
  This spell functions like \spell{charm person}, except that the effect is not restricted by creature type and has a shorter duration.
\end{spelleffect}

\spellsection{Charm Person}
\spelldesc{You manipulate a person's mind so he thinks of you as a trusted friend and ally.}
\spellschool{Enchantment (Emotion) [Charm, Mind-Affecting]}
\spelllvl{Ench 2}
\spellrng{\rngmed}
\spelltgt{One humanoid creature}
\spelldur{\durlong}
\spellsave{Will negates}
\spellsr{Yes (Will)}
\begin{spelleffect}
  This charm makes a humanoid creature regard you as its trusted friend and ally. If it is currently faced with any obvious threat from you or your allies, such as someone drawing a weapon, casting a spell, or aiming a ranged weapon at the creature, it receives a \plus5 circumstance bonus on its saving throw.
  \par The spell does not enable you to control the subject as if it were an automaton, but it perceives your words and actions in the most favorable way. You can try to give the subject orders, but you must succeed at a Diplomacy check to convince it to do anything it wouldn't ordinarily do. (Retries are not allowed.) Treat the target as a friend (a \plus10 relationship modifier) for the purpose of the Diplomacy check. An affected creature never obeys suicidal or obviously harmful orders, but it might be convinced that something very dangerous is worth doing.
\end{spelleffect}
\begin{spellnotes}
  Any act by you or your apparent allies that threatens the \spell{charmed} person breaks the spell. A creature that makes its saving throw against \spell{charm person} is immune to all further attempts by the same spellcaster for 24 hours.
\end{spellnotes}

\spellsectioncomma{Charm Person}{Mass}
\spelldesc{You manipulate the minds of many people so they think of you as a trusted friend and ally.}
\spellschool{Enchantment (Emotion) [Charm, Mind-Affecting]}
\spelllvl{Ench 6}
\spellarea{\areamed radius}
\spelltgts{Five humanoid creatures within the area}
\begin{spelleffect}
  This spell functions like \spell{charm person}, except that it affects multiple creatures at a longer range.
\end{spelleffect}

\spellsection{Circle of Death}
\spelldesc{You snuff out the life force of your weakened foes by flooding them with negative energy.}
\spellschool{Necromancy (Vitalism) [Death, Negative]}
\spelllvl{Clr 6, Death 6}
\spellcmp{V, S, M}
\spellrng{\rngmed}
\spellarea{\areamed radius limit}
\spelltgts{Several living creatures within the area}
\spelldur{Instantaneous}
\spellsave{Fortitude negates}
\spellsr{Yes (Fortitude)}
\begin{spellblood}
  The subjects immediately die.
\end{spellblood}
\begin{spellnotes}
  This spell can affect 2 HV worth of living creatures per caster level. Creatures with the fewest HV are affected first; among creatures with equal HV, those who are closest to the burst's point of origin are affected first. No creature of more HV than half your caster level can be affected, and Hit Values that are not sufficient to affect a creature are wasted. Healthy creatures are not affected by the spell, and do not count against the spell's HV limit.
\end{spellnotes}
\spellmat{The powder of a crushed black pearl with a minimum value of 750 gp.}

\spellsection{Clenched Fist}
\spelldesc{You create a floating, disembodied hand made of magical force that strikes your foe.}
\spellschool{Evocation (Control) [Force]}
\spelllvl{Evoc 9, Strength 9}
\spellsave{None/Fortitude negates}
\spellsr{Yes (Fortitude)}
\spelldmg{2d10 force damage \add half casting attribute}
\begin{spelleffect}
  This spell functions like \spell{interposing hand}, except that the hand can also strike one opponent that you select. The floating hand can move as far as 60 feet and can attack in the same round. Since this hand is directed by you, its ability to notice or attack invisible or concealed creatures is no better than yours.
  \par The hand attacks once per round, and its attack bonus equals your caster level \add your casting attribute, which is the hand's Strength score, \minus1 for being Large.
\end{spelleffect}
\begin{spellblood}
  The struck creature is dazed for 1 round. A save negates the dazing, but not the damage.
\end{spellblood}
\begin{spellnotes}
  Directing the spell to a new target is a swift action.
\end{spellnotes}

\spellsection{Cloak of Chaos}
\spelldesc{You shield your allies with an an powerful aura that resembles a random pattern of color -- an affront to your lawful foes.}
\spellschool{Abjuration (Shielding) [Chaotic]}
\spelllvl{Chaos 8, Clr 8}
\spellcmp{V, S, F}
\spellarea{\areamed radius limit centered on you}
\spelltgts{Five creatures within the area}
\spelldur{\durshort (D)}
\spellsave{Will negates (harmless)}
\spellsr{Yes (Will)}
\begin{spelleffect}
  This spell has four effects.
  \par First, each shielded creature gains a \plus5 bonus to its saving throws.
  \par Second, each shielded creature gains spell resistance against lawful spells and spells cast by lawful creatures.
  \par Third, the abjuration blocks possession and mental influence, just as \spell{protection from law} does.
  \par Finally, if a lawful creature within \rngmed range of the shielded creature successfully attacks it in any way, the offending attacker takes 4d6 damage. Any single creature can take this damage only once per round.
\end{spelleffect}
\spellfocus{A tiny reliquary containing some sacred relic, such as a scrap of parchment from a chaotic text. The reliquary costs at least 500 gp.}

\spellsection{Cloudkill}
\spelldesc{You conjure a yellowish green fog bank that obscures vision and slowly poisons creatures inside.}
\spellschool{Conjuration (Creation) [Fog, Poison]}
\spelllvl{Sor/Wiz 7}
\spellsave{None/Fortitude negates}
\begin{spelleffect}
  This spell functions like \spell{fog cloud}, except that living creatures inside the fog take 1d4 Constitution damage on your turn each round. A successful Fortitude save negates the damage for that round.
  \par Unlike a \spell{fog cloud}, the \spell{cloudkill} moves away from you at 10 feet per round, rolling along the surface of the ground. Figure out the cloud's new spread each round based on its new point of origin, which is 10 feet farther away from the point of origin where you cast the spell.
\end{spelleffect}
\begin{spellnotes}
  Holding one's breath doesn't help against the poison, but creatures immune to poison are unaffected.
  \par Because the vapors are heavier than air, they sink to the lowest level of the land, even pouring down den or sinkhole openings. It cannot penetrate liquids, nor can it be cast underwater.
\end{spellnotes}

\spellsection{Color Spray}
\spelldesc{You project a vivid cone of clashing colors from your outstretched hand, striking creatures in front of you.}
\spellschool{Illusion (Figment) [Light]}
\spelllvl{Sor/Wiz 1}
\spellarea{\areamed cone-shaped burst}
\spelldur{1d4 rounds}
\spellsave{Will negates}
\spellsr{Yes (Will)}
\begin{spelleffect}
  Creatures in the area are dazzled and bewildered.
\end{spelleffect}
\begin{spellnotes}
  A dazzled creature has a 20\% miss chance on all attack rolls and takes a \minus4 penalty to Spot checks. He is also unable to see with darkvision. A bewildered creature is mentally affected in a way that detracts from its ability to act, causing it to be vulnerable. It takes a \minus2 penalty to attack rolls, saving throws, checks, DCs, and AC.

  Creatures who cannot see the light are not affected by this spell. Merely closing one's eyes is insufficient protection.
\end{spellnotes}

\spellsection{Combat Transformation}
\spelldesc{You become a virtual fighting machine -- stronger, tougher, faster, and more skilled in combat. Your mind-set changes so that you relish combat instead of casting spells, even from magic items.}
\spellschool{Transmutation (Augment)}
\spelllvl{Sor/Wiz 5}
\spellcmp{V, S, M}
\spellrng{\rngpers}
\spelltgt{You}
\spelldur{\durshort (D)}
\begin{spelleffect}
  You gain a \plus3 bonus to Strength, Dexterity, Constitution, natural armor, and Fortitude saves. This bonus increases to \plus4 at 14th level and to \plus5 at 20th level. In addition, you gain proficiency with any weapons you hold (except exotic weapons).
\end{spelleffect}
\begin{spellnotes}
  If you cast a spell or use a spell activation or spell completion magic item, the spell immediately ends.
\end{spellnotes}
\spellmat{A potion of \spell{totemic power} (which costs 400 gp), which you drink (and whose effects are subsumed by the spell effects).}

\spellsection{Command}
\spelldesc{You compel a foe to obey a single command of your choice.}
\spellschool{Enchantment (Compulsion) [Language-Dependent, Mind-Affecting, Sound-Dependent]}
\spelllvl{Clr 1, Law 1, Pal 1, Sor/Wiz 1}
\spellcmp{V}
\spellrng{\rngmed}
\spelltgt{One creature}
\spelldur{1 round}
\spellsave{Will negates}
\spellsr{Yes (Will)}
\begin{spellhealthy}
  The subject is bewildered, making it vulnerable.
\end{spellhealthy}
\begin{spellblood}
  The subject must perform one of the following actions of your choice.
  \par \subspell{Approach} On its turn, the subject moves toward you as quickly and directly as possible. The creature may do nothing but move during its turn, and it provokes attacks of opportunity for this movement as normal.
  \par \subspell{Drop} As soon as possible, the subject drops whatever it is holding. It may act normally on its turn, except that it can't pick up any dropped items.
  \par \subspell{Fall} As soon as possible, the subject falls to the ground. It may act normally on its turn, except that it can't get up from its prone position.
  \par \subspell{Flee} On its turn, the subject moves away from you as quickly as possible. It may do nothing but move during its turn, and it provokes attacks of opportunity for this movement as normal.
  \par \subspell{Halt} On its turn, the subject can take no actions, but it can defend itself normally.
\end{spellblood}
\begin{spellnotes}
  A vulnerable creature takes a \minus2 penalty to attack rolls, saving throws, checks, DCs, and AC.
  If the subject can't understand or carry out your command, the spell automatically fails.
\end{spellnotes}

\spellsectioncomma{Command}{Mass}
\spelldesc{You compel many foes to obey your command.}
\spellschool{Enchantment (Compulsion) [Language-Dependent, Mind-Affecting, Sound-Dependent]}
\spelllvl{Clr 5, Law 5, Pal 4}
\spellarea{\areamed radius limit}
\spelltgts{Five creatures within the area}
\begin{spelleffect}
  This spell functions like \spell{command}, except that it affects multiple creatures.
\end{spelleffect}

\spellsectioncomma{Cone of Cold}{Lesser}
\spelldesc{You create an area of extreme cold that drains heat from creatures in the area.}
\spellschool{Evocation (Energy) [Cold]}
\spelllvl{Drd 2, Sor/Wiz 2}
\spellarea{\areamed cone-shaped burst}
\spelldur{Instantaneous and 1 round}
\spellsave{None/Reflex half}
\spellsr{Yes (Reflex)}
\spelldmg{2d6 cold damage \add d6 per four caster levels above 4th.}
\begin{spelleffect}
  Everything in the area takes damage. Creatures damaged by the spell are fatigued for 1 round.
\end{spelleffect}

\spellsection{Cone of Cold}
\spelldesc{You create an area of extreme cold that drains heat from creatures in the area, diminishing their ability to move and fight.}
\spellschool{Evocation (Energy) [Cold]}
\spelllvl{Drd 5, Sor/Wiz 5}
\spellsave{Reflex half/None}
\spelldmg{5d6 cold damage \add d6 per four caster levels above 10th.}
\begin{spelleffect}
  This spell functions as \spell{lesser cone of cold}, except that affected creatures are fatigued for 5 rounds.
\end{spelleffect}

\spellsectioncomma{Cone of Cold}{Greater}
\spelldesc{You create a massive area of extreme cold that drains heat from creatures in the area, diminishing their ability to move and fight.}
\spellschool{Evocation (Energy) [Cold]}
\spelllvl{Drd 8, Sor/Wiz 8}
\spellarea{\arealarge cone-shaped burst}
\spelldmg{8d6 cold damage \add d6 per four caster levels above 16th.}
\begin{spelleffect}
  This spell functions as \spell{cone of cold}, except that it affects a larger area.
\end{spelleffect}

\spellsection{Confusion}
\spelldesc{You compel a creature to act randomly, sowing confusion in your foes' ranks.}
\spellschool{Enchantment (Compulsion) [Mind-Affecting]}
\spelllvl{Chaos 3, Sor/Wiz 3, Trickery 3}
\spellrng{\rngmed}
\spelltgt{One creature}
\spelldur{\durshort}
\spellsave{Will negates}
\spellsr{Yes (Will)}
\begin{spellhealthy}
  The subject is bewildered, making it vulnerable.
\end{spellhealthy}
\begin{spellblood}
  The subject is confused. \confusionexplanation
\end{spellblood}
\begin{spellnotes}
  A vulnerable creature takes a \minus2 penalty to attack rolls, saving throws, checks, DCs, and AC.
  \par Attackers are not at any special advantage when attacking a \spell{confused} character. A \spell{confused} character will not make attacks of opportunity against any creature that it is not already devoted to attacking (either because of its most recent action or because it has just been attacked).
\end{spellnotes}

\spellsectioncomma{Confusion}{Mass}
\spelldesc{You compel a group of creatures to act randomly, sowing confusion in your foes' ranks.}
\spellschool{Enchantment (Compulsion) [Mind-Affecting]}
\spelllvl{Sor/Wiz 7, Trickery 7}
\spellarea{\areamed radius limit}
\spelltgts{Five creatures within the area}
\begin{spelleffect}
  This spell functions like \spell{confusion}, except that it affects multiple creatures. If there are more creatures in the area than you can affect, randomly determine which creatures are affected.
\end{spelleffect}

\spellsection{Contagion}
\spelldesc{You infect your foe with a contagious disease.}
\spellschool{Necromancy (Flesh) [Disease]}
\spelllvl{Clr 3, Destruction 3, Drd 3, Sor/Wiz 3}
\spellrng{\rngmed}
\spelltgt{One living creature}
\spelldur{Instantaneous}
\spellsave{Fortitude negates}
\spellsr{Yes (Fortitude)}
\begin{spelleffect}
  The subject contracts a disease selected from the table below, which strikes immediately (no incubation period). The DC for both the initial and subsequent saving throws is equal to this spell's save DC.  
  \begin{dtable}
    \begin{tabularx}{\columnwidth}{l X}
      \thead{Disease} & \thead{Damage} \\
      Blinding sickness & 1d4 Str\footnotetemp{1} \\
      Cackle fever & 1d6 Wis \\
      Filth fever & 1d3 Dex and 1d3 Con \\
      Mindfire & 1d6 Int \\
      Red ache & 1d6 Str \\
      Shakes & 1d6 Dex \\
      Slimy doom & 1d6 Con
    \end{tabularx}
    1 Each time a victim takes 3 or more Strength damage from blinding sickness, he or she must make another Fortitude save or be permanently blinded.	 
  \end{dtable}
\end{spelleffect}

\spellsection{Control Water}
\spelldesc{You manipulate elemental forces to control water around you.}
\spellschool{Evocation (Control) [Water]}
\spelllvl{Drd 2, Water 2}
\spellrng{\rngfar}
\spellarea{Water in one volume/level of 10 ft. by 10 ft. by 2 ft. (S)}
\spelldur{\durmed (D)}
\spellsave{None; see text}
\spellsr{No}
\begin{spelleffect}
  Depending on the version you choose, the \spell{control water} spell raises or lowers water.
  \par \subspell{Lower Water} This causes water or similar liquid to reduce its depth by as much as 2 feet per caster level (to a minimum depth of 1 inch). The water is lowered within a squarish depression whose sides are up to caster level \mtimes 10 feet long. In extremely large and deep bodies of water, such as a deep ocean, the spell creates a whirlpool that sweeps ships and similar craft downward, putting them at risk and rendering them unable to leave by normal movement for the duration of the spell.
  \par \subspell{Raise Water} This causes water or similar liquid to rise in height, just as the lower water version causes it to lower. Boats raised in this way slide down the sides of the hump that the spell creates. If the area affected by the spell includes riverbanks, a beach, or other land nearby, the water can spill over onto dry land.
\end{spelleffect}
\begin{spellnotes}
  With either version, you may reduce one horizontal dimension by half and double the other horizontal dimension.
\end{spellnotes}

\spellsection{Create Sound}
\spellschool{Illusion (Figment) [Unreal]}
\spelllvl{Illus 1}
\spellrng{\rngclose}
\spelleff{Illusory sounds}
\spelldur{\durshort (D)}
\spellsave{Will disbelief (if interacted with)}
\spellsr{No}
\begin{spelleffect}
  This spell allows you to create a volume of sound that rises, recedes, approaches, or remains at a fixed place. You choose what type of sound this spell creates when casting it and cannot thereafter change the sound's basic character.
  \par The volume of sound created depends on your level. You can produce as much noise as two normal humans per caster level. Thus, talking, singing, shouting, walking, marching, or running sounds can be created. The noise a \spell{ghost sound} spell produces can be virtually any type of sound within the volume limit, including speech. A horde of rats running and squeaking is about the same volume as eight humans running and shouting. A roaring lion is equal to the noise from sixteen humans, while a roaring dire tiger is equal to the noise from twenty humans.
\end{spelleffect}
\begin{spellnotes}
  \spell{Create sound} can be made permanent with a \spell{permanency} spell.
\end{spellnotes}

\spellsection{Creeping Doom}
\spelldesc{You summon uncountable hordes of centipedes to overwhelm your foes.}
\spellschool{Conjuration (Summoning)}
\spelllvl{Drd 7}
\spelltime{Full-round action}
\spellrng{\rngclose; see text}
\spelleff{One swarm of centipedes per two levels}
\spelldur{\durmed}
\spellsave{None}
\spellsr{No}
\begin{spelleffect}
  This spell creates one centipede swarm per two caster levels. They must all be adjacent at least one other swarm. You may summon the centipede swarms so that they share the area of other creatures. The swarms remain stationary, attacking any creatures in their area, unless you command the creeping doom to move (a standard action). As a standard action, you can command any number of the swarms to move toward any prey within \rngmed range of you. You cannot command any swarm to move more than \rngmed range away from you, and if you exceed that distance, the swarm remains stationary, attacking any creatures in its area (but it can be commanded again if you move within range).
\end{spelleffect}

\spellsection{Cripple}
\spelldesc{You render your foe's limbs useless.}
\spellschool{Necromancy (Flesh)}
\spelllvl{Sor/Wiz 6}
\spellrng{\rngmed}
\spelltgt{One creature}
\spelldur{\durshort}
\spellsave{Fortitude negates}
\spellsr{Yes (Fortitude)}
\begin{spellhealthy}
  The subject is staggered. It can take a move action or a standard action each round, but not both. 
\end{spellhealthy}
\begin{spellblood}
  The subject cannot move its limbs. Generally, that means it is paralyzed, except that it can move its head and mouth.
\end{spellblood}
\begin{spellnotes}
 A staggered character may take a single move action or standard action each round, but not both. She cannot take full-round actions, but she may take swift actions. In addition, she is vulnerable, causing her to take a \minus2 penalty on attack rolls, saving throws, checks, DCs, and AC.
\end{spellnotes}

\spellsection{Crush Life}
\spelldesc{You attack the life force of a single foe directly, allowing no possibility for escape.}
\spellschool{Necromancy (Life)}
\spelllvl{Death 1, Necro 1}
\spellrng{\rngmed}
\spelltgt{One living creature}
\spelldur{Instantaneous}
\spellsave{None}
\spellsr{Yes (Fort)}
\spelldmg{1d10 life damage \add d10 per four caster levels above 2nd}
\begin{spelleffect}
  The target takes damage.
\end{spelleffect}

\spellsectioncomma{Crush Life}{Greater}
\spelldesc{You obliterate the life force of a single foe directly, allowing no possibility for escape.}
\spellschool{Necromancy (Life)}
\spelllvl{Necro 4}
\spelldmg{4d10 life damage \add d10 per four caster levels above 8th}
\begin{spelleffect}
  This spell functions like \spell{crush life}, except that the target is also sickened for 5 rounds.
\end{spelleffect}

\spellsection{Crushing Despair}
\spelldesc{You fill a number of creatures with sadness and gloom.}
\spellschool{Enchantment (Emotion) [Mind-Affecting]}
\spelllvl{Sor/Wiz 3}
\spellarea{\areamed cone-shaped burst}
\spelldur{\durmed}
\spellsave{None}
\spellsr{Yes (Will)}
\begin{spelleffect}
  Each creature in the area is demoralized.
\end{spelleffect}
\begin{spellnotes}
  A demoralized creature is vulnerable, causing it to take a \minus2 penalty on attack rolls, saving throws, checks, DCs, and AC. \spellindirect{crushing despair}{Crushing despair} counters and dispels \spell{good hope}.
\end{spellnotes}

\spellsection{Crushing Hand}
\spelldesc{You create a floating, disembodied hand made of magical force that crushes your foe in its grasp.}
\spellschool{Evocation (Control) [Force]}
\spelllvl{Evoc 8}
\spellsave{Fortitude partial}
\spellsr{Yes (Fortitude)}
\spelldmg{2d6 \add half casting attribute}
\begin{spelleffect}
  This spell functions like \spell{grasping hand}, except that the hand deals lethal damage on each successful grapple attack against an opponent.
\end{spelleffect}
\begin{spellnotes}
  Directing the spell to a new target is a swift action.
\end{spellnotes}

\spellsection{Cure Critical Wounds}
\spelldesc{You lay your hand on a creature and channel positive energy into it, healing even the most grievous injuries.}
\spellschool{Necromancy (Vitalism) [Healing, Positive]}
\spelllvl{Clr 4, Drd 4, Life 4, Pal 4}
\spellheal{8d6 damage \add d6 per two caster levels above 8th}
\begin{spelleffect}
  This spell functions like \spell{cure light wounds}, except that for every 10 points of healing granted by the spell, it can instead cure 1 point of critical damage.
\end{spelleffect}

\spellsectioncomma{Cure Critical Wounds}{Mass}
\spelldesc{You stretch out your hand and channel positive energy into all of your allies, healing even their most grievous injuries.}
\spellschool{Necromancy (Vitalism) [Healing, Positive]}
\spelllvl{Clr 8, Drd 8, Life 8}
\spellheal{8d6 damage \add d6 per four caster levels above 16th}
\begin{spelleffect}
  This spell functions like \spell{mass cure light wounds}, except that for every 10 points of healing granted by the spell, it can instead cure 1 point of critical damage.
\end{spelleffect}

\spellsection{Cure Light Wounds}
\spelldesc{You lay your hand on a creature and channel positive energy into it, healing some of its wounds.}
\spellschool{Necromancy (Vitalism) [Healing, Positive]}
\spelllvl{Clr 1, Drd 1, Pal 1}
\spellrng{\rngclose}
\spelltgt{One creature}
\spelldur{Instantaneous}
\spellsave{Fortitude half (harmless) or Fortitude half; see text}
\spellsr{Yes (Fortitude)}
\spellheal{2d6 damage \add d6 per two caster levels above 2nd}
\begin{spelleffect}
  You heal the target. Since undead are powered by negative energy, this spell deals positive damage to them instead of curing their wounds.
\end{spelleffect}

\spellsectioncomma{Cure Light Wounds}{Mass}
\spelldesc{You stretch out your hand and channel positive energy into all of your allies, healing some of their wounds.}
\spellschool{Necromancy (Vitalism) [Healing, Positive]}
\spelllvl{Clr 5, Drd 5, Life 5}
\spellrng{\rngmed}
\spellarea{\areamed radius limit}
\spelltgts{Five creatures within the area}
\spelldur{Instantaneous}
\spellsave{Fortitude half (harmless) or Fortitude half; see text}
\spellsr{Yes (Fortitude)}
\spellheal{5d6 damage \add d6 per four caster levels above 10th}
\begin{spelleffect}
  You heal the targets. Like other \spellindirect{cure light wounds}{cure} spells, this spell deals positive damage to affected undead rather than curing them.
\end{spelleffect}

\spellsection{Cure Moderate Wounds}
\spelldesc{You lay your hand on a creature and channel positive energy into it, healing its wounds.}
\spellschool{Necromancy (Life) [Healing, Positive]}
\spelllvl{Clr 2, Drd 2, Life 2, Pal 2}
\spellheal{4d6 damage \add d6 per two caster levels above 4th}
\begin{spelleffect}
  This spell functions like \spell{cure light wounds}, except that for every 20 points of healing granted by the spell, it can instead cure 1 point of critical damage.
\end{spelleffect}

\spellsectioncomma{Cure Moderate Wounds}{Mass}
\spelldesc{You stretch out your hand and channel positive energy into all of your allies, healing their wounds.}
\spellschool{Necromancy (Vitalism) [Healing, Positive]}
\spelllvl{Clr 6, Drd 6, Life 6}
\spellheal{6d6 damage \add d6 per four caster levels above 12th}
\begin{spelleffect}
  This spell functions like \spell{mass cure light wounds}, except that for every 20 points of healing granted by the spell, it can instead cure 1 point of critical damage.
\end{spelleffect}

\spellsection{Cure Serious Wounds}
\spelldesc{You lay your hand on a creature and channel positive energy into it, healing even serious injuries.}
\spellschool{Necromancy (Vitalism) [Healing, Positive]}
\spelllvl{Clr 3, Drd 3, Life 3, Pal 3}
\spellheal{6d6 damage \add d6 per two caster levels above 6th}
\begin{spelleffect}
  This spell functions like \spell{cure light wounds}, except that for every 15 points of healing granted by the spell, it can instead cure 1 point of critical damage.
\end{spelleffect}

\spellsectioncomma{Cure Serious Wounds}{Mass}
\spelldesc{You stretch out your hand and channel positive energy into all of your allies, healing even serious injuries.}
\spellschool{Necromancy (Vitalism) [Healing, Positive]}
\spelllvl{Clr 7, Drd 7, Life 7}
\spellheal{7d6 damage \add d6 per four caster levels above 14th}
\begin{spelleffect}
  This spell functions like \spell{mass cure light wounds}, except that for every 15 points of healing granted by the spell, it can instead cure 1 point of critical damage.
\end{spelleffect}

\pdfbookmark[2]{D}{SpellDescriptionsD}
\begin{comment}
\subsubsection{D}
\end{comment}

\spellsection{Dancing Lights}
\spellschool{Illusion (Figment) [Light]}
\spelllvl{Sor/Wiz 1}
\spellrng{\rngmed}
\spellarea{\areasmall radius limit}
\spelleff{Up to four lights within the area}
\spelldur{\durshort (D)}
\spellsave{None}
\spellsr{No}
\begin{spelleffect}
  Depending on the version selected, you create up to four lights that resemble lanterns or torches (and cast that amount of light), or up to four glowing spheres of light (which look like will-o'-wisps), or one faintly glowing, vaguely humanoid shape. The \spell{dancing lights} must stay within a \areasmall radius in relation to each other. You can spend a swift action on your turn to move the lights as you desire: forward or back, up or down, straight or turning corners, or the like. The lights can move up to 100 feet per round. A light winks out if the distance between you and it exceeds the spell's range.
\end{spelleffect}
\begin{spellnotes}
  \spell{Dancing lights} can be made permanent with a \spell{permanency} spell.
\end{spellnotes}

\spellsection{Darkness}
\spellschool{Illusion (Glamer) [Darkness]}
\spelllvl{Sor/Wiz 2, Trickery 2}
\spellcmp{V}
\spellrng{\rngtouch}
\spelltgt{Object touched}
\spelldur{\durmed (D)}
\spellsave{None}
\spellsr{No}
\begin{spelleffect}
  This spell causes an object to radiate shadowy illumination out to a \areamed radius. This causes the level of illumination to drop to shadowy illumination or the current prevailing condition, whichever is lower. Darkvision is ineffective in magical darkness, and confers no advantage over normal vision.
\end{spelleffect}
\begin{spellnotes}
  If \spell{darkness} is cast on a small object that is then placed inside or under a lightproof covering, the spell's effect is blocked until the covering is removed.

  Normal lights (torches, candles, lanterns, and so forth) are incapable of brightening the area or shining through it, as are light spells of lower level. Such effects are also suppressed if they originate from within the area of the darkness, preventing them from shining light elsewhere. Higher level light spells are not affected by darkness.

  \par \spell{Darkness} counters or dispels any light spell of equal or lower spell level.
\end{spellnotes}

\spellsection{Darkvision}
\spellschool{Divination (Awareness)}
\spelllvl{Sor/Wiz 2}
\spellrng{\rngtouch}
\spelltgt{Creature touched}
\spelldur{\durlong}
\spellsave{Fortitude negates (harmless)}
\spellsr{Yes (Fortitude)}
\begin{spelleffect}
  The subject gains the ability to see 60 feet even in total darkness. Beyond 60 feet, the subject can see dimly, treating areas of darkness as shadowy illumination. Darkvision does not function if a creature is in an area of bright light or is dazzled. Darkvision is black and white only, but otherwise like normal sight.
\end{spelleffect}
\begin{spellnotes}
  \spell{Darkvision} does not grant one the ability to see in magical darkness.

  \par \spell{Darkvision} can be made permanent with a \spell{permanency} spell.
\end{spellnotes}

\spellsection{Daylight}
\spellschool{Illusion (Figment) [Light]}
\spelllvl{Clr 2, Pal 2}
\spellrng{\rngtouch}
\spelltgt{Object touched}
\spelldur{\durlong (D)}
\spellsave{None}
\spellsr{No}
\begin{spelleffect}
  The object touched sheds light as bright as full daylight in a \arealarge radius, and dim light for an additional 50 feet beyond that. Creatures that take penalties in bright light also take them while within the radius of this magical light. Despite its name, this spell is not the equivalent of sunlight for the purposes of creatures that are damaged or destroyed by bright light.
  \par If \spell{daylight} is cast on a small object that is then placed inside or under a light-proof covering, the spell's effects are blocked until the covering is removed.
\end{spelleffect}
\begin{spellnotes}
  \spell{Daylight} brought into an area of magical darkness (or vice versa) is temporarily negated, so that the otherwise prevailing light conditions exist in the overlapping areas of effect.
  \par \spell{Daylight} counters or dispels any darkness spell of equal or lower level, such as darkness.
\end{spellnotes}

\spellsection{Daze}
\spelldesc{You cloud the mind of your foe, preventing it from taking any actions.}
\spellschool{Enchantment (Inhibition) [Mind-Affecting]}
\spelllvl{Sor/Wiz 3}
\spellrng{\rngmed}
\spelltgt{One creature}
\spelldur{\durshort}
\spellsave{None/Will negates}
\spellsr{Yes (Will)}
\begin{spellhealthy}
  The subject is bewildered, making it vulnerable.
\end{spellhealthy}
\begin{spellblood}
  As the healthy effect, and the subject is also dazed for 1 round if it fails a Will save. A dazed creature can take no actions, though it can defend itself normally.
\end{spellblood}
\begin{spellnotes}
  A vulnerable creature takes a \minus2 penalty to attack rolls, saving throws, checks, DCs, and AC.
\end{spellnotes}

\spellsectioncomma{Daze}{Mass}
\spelldesc{You cloud the mind of your foes, preventing them from taking any actions.}
\spellschool{Enchantment (Inhibition) [Mind-Affecting]}
\spelllvl{Sor/Wiz 8}
\spellrng{\rngmed}
\spellarea{\areamed limit}
\spelltgts{Five creatures within the area}
\begin{spelleffect}
  This spell functions like \spell{daze}, except that it affects multiple creatures.
\end{spelleffect}

\spellsection{Death Knell}
\spelldesc{You draw forth the ebbing life force of a creature and use it to fuel your own power.}
\spellschool{Necromancy (Life) [Death]}
\spelllvl{Death 2, Evil 2, Necro 2}
\spellrng{\rngmed}
\spelltgt{Living creature}
\spelldur{\durshort; see text}
\spellsave{Fortitude negates}
\spellsr{Yes (Fortitude)}
\begin{spellblood}
  The subject becomes vulnerable. If it drops to 0 hit points, it dies immediately, and you gain 20 temporary hit points \add 2 per caster level above 4th. These temporary hit points last for 1 round per HV the subject had.

 If you take life damage, you lose all temporary hit points provided by this spell before applying the damage.
\end{spellblood}

\spellsection{Death Ward}
\spellschool{Abjuration/Necromancy (Shielding, Vitalism) [Positive]}
\spelllvl{Clr 3, Death 3, Good 3, Pal 3, Protection 3}
\spellrng{\rngclose}
\spelltgt{One living creature}
\spelldur{\durshort}
\spellsave{Fortitude negates (harmless)}
\spellsr{Yes (Fortitude)}
\begin{spelleffect}
  The subject is immune to all death spells, magical death effects, energy drain, and any negative energy effects.
\end{spelleffect}
\begin{spellnotes}
  This spell doesn't remove negative levels that the subject has already gained, nor does it affect the saving throw necessary 24 hours after gaining a negative level.
  \par \spell{Death ward} does not protect against other sorts of attacks, even if those attacks might be lethal.
\end{spellnotes}

\spellsectioncomma{Death Ward}{Mass}
\spellschool{Abjuration/Necromancy (Shielding, Vitalism) [Positive]}
\spelllvl{Clr 7, Death 7}
\spellrng{\rngmed}
\spellarea{\areamed radius limit}
\spelltgts{Five living creatures within the area}
\begin{spelleffect}
  This spell functions like \spell{death ward}, except that it affects multiple creatures.
\end{spelleffect}

\spellsection{Deep Slumber}
\spelldesc{You fill your foe with an overpowering urge to sleep, inevitably rendering him comatose.}
\spellschool{Enchantment (Compulsion) [Mind-Affecting]}
\spelllvl{Sor/Wiz 7}
\spellrng{\rngmed}
\spelltgt{One creature}
\spelldur{\durlong}
\spellsave{Will negates}
\spellsr{Yes (Will)}
\begin{spellhealthy}
  The subject is bewildered, making it vulnerable.
\end{spellhealthy}
\begin{spellblood}
  The subject immediately falls asleep. If left undisturbed, it will sleep until it dies. As long as it remains bloodied, it cannot be awakened until the spell's duration expires, though it can be awakened normally after that point.
\end{spellblood}
\begin{spellnotes}
  A vulnerable creature takes a \minus2 penalty to attack rolls, saving throws, checks, DCs, and AC.
\end{spellnotes}

\spellsection{Delay Poison}
\spellschool{Necromancy (Flesh)}
\spelllvl{Clr 1, Drd 1, Pal 1}
\spelltime{1 swift action}
\spellrng{\rngclose}
\spelltgt{Creature touched}
\spelldur{\durshort}
\spellsave{Fortitude negates (harmless)}
\spellsr{Yes (Fortitude)}
\begin{spelleffect}
  The subject becomes temporarily immune to the effects of poison. It does not make any saving throws against poison during this spell's duration. This effect does not prevent the subject from becoming poisoned, and any poisons in the subject's system when the spell ends will continue their effects normally. 
\end{spelleffect}
\begin{spellnotes}
  This spell does not cure any damage that poison may have already done.
\end{spellnotes}

\spellsection{Delayed Blast Fireball}
\spellschool{Evocation (Energy) [Fire]}
\spelllvl{Fire 6, Sor/Wiz 6}
\spellarea{\areamed radius spread}
\spelldur{5 rounds or less; see text}
\spelldmg{6d6 fire damage \add d6 per four caster levels above 12th}
\begin{spelleffect}
  This spell functions like \spell{fireball}, except that it is larger and can detonate up to 5 rounds after the spell is cast. You select the amount of delay upon completing the spell, and that time cannot change once it has been set unless someone touches the bead (see below). For every round that this spell is delayed, your caster level with it increases by 2.

  If you choose a delay, a glowing bead sits at the point of origin until it detonates. A creature can pick up and hurl the bead as a thrown weapon (range increment 10 feet). If a creature handles and moves the bead within 1 round of its detonation, there is a 25\% chance that the bead detonates while being handled. A creature holding the bead (not merely standing next to or even touching the bead) receives no saving throw against the spell's effect.
\end{spelleffect}

\spellsection{Destruction}
\spellschool{Necromancy (Flesh) [Death]}
\spelllvl{Clr 7, Destruction 7}
\spellcmp{V, S, F}
\spellrng{\rngclose}
\spelltgt{One creature}
\spelldur{Instantaneous}
\spellsave{Fortitude negates}
\spellsr{Yes (Fortitude)}
\begin{spellhealthy}
  The target is staggered for 5 rounds. It can take a move action or a standard action each round, but not both.
\end{spellhealthy}
\begin{spellblood}
  The target is instantly slain.
\end{spellblood}
\begin{spellnotes}
  The reamins of a creature killed by this spell are consumed utterly (but not its equipment or possessions). The only way to restore life such a creature is to use \spell{true resurrection}, a carefully worded \spell{wish} spell followed by \spell{resurrection}, or \spell{miracle}.
\end{spellnotes}
\spellfocus{A special holy (or unholy) symbol of silver marked with verses of anathema (cost 500 gp).}

\spellsection{Detect Animals or Plants}
\spellschool{Divination (Awareness) [Detection]}
\spelllvl{Drd 1, Nature 1}
\spellarea{\arealarge cone-shaped emanation from you}
\spelldur{Concentration}
\spellsave{None}
\spellsr{No}
\begin{spelleffect}
  You know the direction to any animals in the area by seeing their auras. If you concentrate on a particular aura, you learn its location. You must choose to detect either animals or plants. Alternately, you can choose to detect a particular kind of animal or plant. Each round, you can change what you are trying to detect.
\end{spelleffect}
\begin{spellnotes}
  Each round, you can turn to detect animals or plants in a new area. A detection spell can penetrate barriers, but 1 foot of stone, 1 inch of common metal, a thin sheet of lead, or 3 feet of wood or dirt blocks it.
\end{spellnotes}

\spellsection{Detect Chaos}
\spellschool{Divination (Awareness) [Detection]}
\spelllvl{Clr 1, Pal 1}
\begin{spelleffect}
  This spell functions like \spell{detect evil}, except that it detects chaotic auras, and you are vulnerable to an overwhelming chaotic aura if you are lawful.
\end{spelleffect}

\spellsection{Detect Evil}
\spelldesc{You sense the presence of evil.}
\spellschool{Divination (Awareness) [Detection]}
\spelllvl{Clr 1, Pal 1}
\spellarea{\arealarge cone-shaped emanation from you}
\spelldur{Concentration}
\spellsave{None}
\spellsr{No}
\begin{spelleffect}
  You know the direction to any evil creatures or objects in the area by seeing their auras. If you concentrate on a particular aura, you learn how powerful it is, as determined by the table below.
  \par If the HV or level of the aura's source is at least twice your caster level, the power of the aura increases by one step, with strong auras becoming overwhelming. If you are good, and you concentrate on a creature with an overwhelming aura, you must make a Will save or be stunned for 1 round (which typically breaks your concentration, ending the spell).
  \begin{dtable}
    \begin{tabularx}{\columnwidth}{l >{\lcol}X}
      \thead{Creature/Object} & \thead{Aura Power} \\
      Evil creature & Faint \\
      Undead & Moderate \\
      Evil magic item or spell & Moderate\footnotetemp{1} \\
      Evil outsider & Strong \\
      Cleric of an evil deity\footnotetemp{2} & Strong
    \end{tabularx}
    \par 1 Use the item or spell's caster level to determine whether the power of the aura us unusually strong.
    \par 2 Some characters who are not clerics (such as blackguards) may radiate an aura of equivalent power. The class description will indicate whether this applies.
  \end{dtable}
  \par \subspell{Lingering Aura} An evil aura can linger after its original source dissipates (in the case of a spell) or is destroyed (in the case of a creature or magic item). If \spell{detect evil} is cast and directed at such a location, the spell indicates an aura strength of dim (even weaker than a faint aura). Most auras only linger for a few rounds, but strong or overwhelming auras can linger for days.
\end{spelleffect}
\begin{spellnotes}
  Animals, traps, poisons, and other potential perils are not evil, and as such this spell does not detect them.
  \par Each round, you can turn to detect evil in a new area. A detection spell can penetrate barriers, but 1 foot of stone, 1 inch of common metal, a thin sheet of lead, or 3 feet of wood or dirt blocks it.
\end{spellnotes}

\spellsection{Detect Good}
\spellschool{Divination (Awareness) [Detection]}
\spelllvl{Clr 1}
\begin{spelleffect}
  This spell functions like \spell{detect evil}, except that it detects good auras, and you are vulnerable to an overwhelming good aura if you are evil.
\end{spelleffect}
\begin{spellnotes}
  Healing potions, antidotes, and similar beneficial items are not good, and as such this spell does not detect them.
\end{spellnotes}

\spellsection{Detect Law}
\spellschool{Divination (Awareness) [Detection]}
\spelllvl{Clr 1}
\begin{spelleffect}
  This spell functions like \spell{detect evil}, except that it detects lawful auras, and you are vulnerable to an overwhelming lawful aura if you are chaotic.
\end{spelleffect}

\spellsection{Detect Secret Doors}
\spelldesc{You can detect secret doors, compartments, caches, and so forth.}
\spellschool{Divination (Awareness) [Detection]}
\spelllvl{Sor/Wiz 1}
\spellarea{\arealarge cone-shaped emanation from you}
\spelldur{Concentration}
\spellsave{None}
\spellsr{No}
\begin{spelleffect}
  You know the direction to any hidden passages, doors, or openings in the area. If you concentrate on a particular aura, you learn its location. This does not automatically grant you the ability to see or open the door -- merely the knowledge that such a door exists in that location.
\end{spelleffect}
\begin{spellnotes}
  Each round, you can turn to detect secret doors in a new area. A detection spell can penetrate barriers, but 1 foot of stone, 1 inch of common metal, a thin sheet of lead, or 3 feet of wood or dirt blocks it.
\end{spellnotes}

\spellsection{Detect Thoughts}
\spellschool{Divination (Awareness) [Detection] [Mind-Affecting]}
\spelllvl{Knowledge 3, Sor/Wiz 3}
\spellarea{\arealarge cone-shaped emanation from you}
\spelldur{Concentration}
\spellsave{Will negates; see text}
\spellsr{Yes (Will)}
\begin{spelleffect}
  You detect surface thoughts. The amount of information revealed depends on how long you study a particular area or subject.
  \par \subspell{1st Round} Presence or absence of thoughts (from conscious creatures with Intelligence scores of 1 or higher).
  \par \subspell{2nd Round} Number of thinking minds and the Intelligence score of each. If the highest Intelligence is 20 or higher and at least 10 points higher than your own Intelligence score, you are stunned for 1 round and the spell ends. This spell does not let you determine the location of the thinking minds if you can't see the creatures whose thoughts you are detecting.
  \par \subspell{3rd Round} Surface thoughts of any mind in the area. A target's Will save prevents you from reading its thoughts, and you must cast detect thoughts again to have another chance. Creatures of animal intelligence (Int 1 or 2) have simple, instinctual thoughts that you can pick up. You need not be able to see a creature to detect thoughts from it. You gain a \plus4 circumstance bonus to Bluff, Diplomacy, and Intimidate checks against creatures whose mind you are reading.
\end{spelleffect}
\begin{spellnotes}
  Each round, you can turn to detect thoughts in a new area. A detection spell can penetrate barriers, but 1 foot of stone, 1 inch of common metal, a thin sheet of lead, or 3 feet of wood or dirt blocks it.
\end{spellnotes}

\spellsectioncomma{Detect Thoughts}{Greater}
\spellschool{Divination (Awareness) [Detection] [Mind-Affecting]}
\spelllvl{Knowledge 8, Sor/Wiz 8}
\spelldur{\durlong (D)}
\begin{spelleffect}
  This spell functions as \spell{detect thoughts}, except that it does not require concentration to maintain. You automatically detect the presence or absence of thoughts, the number of thinking minds, and the Intelligence score of each. You must concentrate to detect surface thoughts, but it only takes you a single round. 
\end{spelleffect}

\spellsection{Detect Undead}
\spellschool{Divination (Awareness) [Detection]}
\spelllvl{Clr 1, Pal 1, Sor/Wiz 1}
\spellarea{\arealarge cone-shaped emanation from you}
\spelldur{Concentration}
\spellsave{None}
\spellsr{No}
\begin{spelleffect}
  You know the direction of all undead creatures in the spell's area. If you concentrate on a particular undead creature, you learn the strength of its aura, determined by the table below.
  You can detect the aura that surrounds undead creatures. The amount of information revealed depends on how long you study a particular area.
  \par \subspell{1st Round} Presence or absence of undead auras.
  \par \subspell{2nd Round} Number of undead auras in the area and the strength of the strongest undead aura present. If you are of good alignment, and the strongest undead aura's strength is overwhelming (see below), and the creature has HV of at least twice your character level, you are stunned for 1 round and the spell ends.
  \par \subspell{3rd Round} The strength and location of each undead aura. If an aura is outside your line of sight, then you discern its direction but not its exact location.
  \par \subspell{Aura Strength} The strength of an undead aura is determined by the HV of the undead creature, as given on the following table:
  \begin{dtable}
    \begin{tabularx}{\columnwidth}{*{2}{>{\lcol}X}}
      \par HV & Strength \\ 
      \par 1 or lower & Faint \\ 
      \par 2--4 & Moderate \\ 
      \par 5--10 & Strong \\ 
      \par 11 or higher & Overwhelming
    \end{tabularx}
  \end{dtable}
  \par \subspell{Lingering Aura} An undead aura can linger after its original source is destroyed. If detect undead is cast and directed at such a location, the spell indicates an aura strength of dim (even weaker than a faint aura). How long the aura lingers at this dim level depends on its original power. Most auras only linger for a few rounds, but strong or overwhelming auras can linger for days.
\end{spelleffect}
\begin{spellnotes}
  Each round, you can turn to detect undead in a new area. A detection spell can penetrate barriers, but 1 foot of stone, 1 inch of common metal, a thin sheet of lead, or 3 feet of wood or dirt blocks it.
\end{spellnotes}

\spellsection{Dictum}
\spellschool{Evocation (Channeling) [Lawful]}
\spelllvl{Clr 7, Law 7}
\spellcmp{V}
\spellarea{40 foot cube-shaped spread centered on you}
\spelldur{Instantaneous/5 rounds}
\spellsave{None}
\spellsr{Yes (Will)}
\begin{spellhealthy}
  \par Each nonlawful creature in the area is deafened for 5 rounds.
\end{spellhealthy}
\begin{spellblood}
  \par Each nonlawful creature in the area suffers one or more of the following ill effects, depending on its Hit Values.
  \begin{dtable}
    \begin{tabularx}{\columnwidth}{l >{\lcol}X}
      \par \thead{HV} & \thead{Effect} \\
      \par Equal to caster level & Staggered \\
      \par Up to caster level \minus5 & Stunned, staggered \\
      \par Up to caster level \minus10 & Paralyzed, stunned, staggered \\
      \par Up to caster level \minus15 & Killed\fn{1}
    \end{tabularx}
    1 Living creatures die. Nonliving creatures are destroyed.
  \end{dtable}
  \par \subspell{Staggered} The creature is staggered for 5 rounds. It can take a move action or a standard action each round, but not both.
  \par \subspell{Stunned} The creature is stunned for 1 round.
  \par \subspell{Paralyzed} The creature is paralyzed and helpless for 5 rounds.
  \par \subspell{Killed} Living creatures die. Nonliving creatures are destroyed.
\end{spellblood}
\begin{spellnotes}
  Creatures whose Hit Values exceed your caster level are unaffected by this spell.
\end{spellnotes}

\spellsection{Dimension Door}
\spellschool{Conjuration (Translocation) [Teleportation]}
\spelllvl{Travel 5, Sor/Wiz 4}
\spellrng{\rngext}
\spelltgt{You}
\spelldur{Instantaneous}
\begin{spelleffect}
  You instantly transfer yourself from your current location to any other spot within range. You always arrive at exactly the spot desired -- whether by simply visualizing the area or by stating direction. After using this spell, you are dazed until the start of your next turn. You can bring along objects as long as their weight doesn't exceed your maximum load.
\end{spelleffect}
\begin{spellnotes}
  \par If you arrive in a place that is already occupied by a solid body, you take 2d6 damage and are shunted to a random open space on a suitable surface within 100 feet of the intended location that is within the range of the spell.
  \par  If there is no free space within 100 feet, you take an additional 4d6 damage and the spell simply fails.
\end{spellnotes}

\spellsectioncomma{Dimension Door}{Mass}
\spellschool{Conjuration (Translocation) [Teleportation]}
\spelllvl{Conj 7, Travel 8}
\spellarea{\areamed radius limit centered on you}
\spelltgts{You and up to five other willing creatures within the area}
\spellsave{None}
\spellsr{No}
\begin{spelleffect}
  This spell functions like \spell{dimension door}, except that it affects multiple creatures. Creatures must be willing to be teleported. You choose the destinations for each affected creature freely, within the range of the spell. Each affected creature is dazed until the start of your next turn.
\end{spelleffect}

\spellsection{Dimension Slide}
\spellschool{Conjuration (Translocation) [Teleportation]}
\spelllvl{Conj 3, Travel 3}
\spellrng{\rngclose}
\spelltgt{You; see text}
\spelldur{Instantaneous}
\begin{spelleffect}
  You instantly transfer yourself from your current location to any other spot within range to which you have line of sight. You can bring along objects as long as their weight doesn't exceed your maximum load. Movement caused by the use of dimension slide does not provoke attacks of opportunity.
  \par If you somehow attempt to transfer yourself to a location occupied by a solid body or a location you can't see, the spell simply fails to function.
\end{spelleffect}

\spellsection{Dimensional Anchor}
\spelldesc{You surround your foe in a shimmering emerald field that completely blocks extradimensional travel, preventing it from escaping you.}
\spellschool{Abjuration (Negation)}
\spelllvl{Clr 3, Magic 3, Sor/Wiz 3}
\spellrng{\rngmed}
\spelltgt{One creature}
\spelldur{\durlong/5 rounds}
\spellsave{Will partial}
\spellsr{Yes (Will)}
\begin{spelleffect}
  The subject cannot travel extradimensionally for an hour. A successful Will save reduces the duration to 5 rounds. Effects barred by a \spell{dimensional anchor} include \spell{astral projection}, \spell{blink}, \spell{dimension door}, \spell{dissipating touch}, \spell{ethereal jaunt}, \spell{gate}, \spell{maze}, \spell{plane shift}, \spell{shadow walk}, \spell{teleport}, and similar spell-like, psionic, or supernatural abilities.
\end{spelleffect}
\begin{spellnotes}
  A dimensional anchor does not interfere with the movement of creatures already in ethereal or astral form when the spell is cast, nor does it block extradimensional perception or attack forms, such as summoning monsters. Also, dimensional anchor does not prevent summoned creatures from disappearing at the end of a summoning spell. 
\end{spellnotes}

\spellsection{Discern Lies}
\spelldesc{You can discern subtle magical disturbances caused by lying.}
\spellschool{Divination (Awareness) [Detection]}
\spelllvl{Clr 3, Law 3, Pal 3}
\spellarea{\arealarge cone-shaped emanation from you}
\spelldur{Concentration}
\spellsave{None}
\spellsr{No}
\begin{spelleffect}
  You know when any creature in the area deliberately and knowingly speaks a lie. The spell does not reveal the truth, uncover unintentional inaccuracies, or necessarily reveal evasions.
\end{spelleffect}
\begin{spellnotes}
  Each round, you can turn to discern lies in a new area. A detection spell can penetrate barriers, but 1 foot of stone, 1 inch of common metal, a thin sheet of lead, or 3 feet of wood or dirt blocks it.
\end{spellnotes}

\spellsection{Discern Vulnerability}
\spellschool{Divination (Knowledge)}
\spelllvl{Div 4, Sor/Wiz 5}
\spelltime{1 swift action}
\spellrng{\rngmed}
\spelltgt{One creature}
\spelldur{Instantaneous}
\spellsave{None}
\spellsr{No}
\begin{spelleffect}
  You instantly recognize all of the target's vulnerabilities. This grants you a \plus2 circumstance bonus to attack rolls, weapon damage rolls, save DCs, and spell resistance checks against that creature. In addition, you learn any significant weaknesses the creature has. This includes, but is not limited to, the following information:
  \begin{itemize*}
    \item Which of the target's saving throws is lowest
    \item If the target has any vulnerabilities to specific damage types
    \item How to overcome the target's damage reduction, regeneration, or other similar abilities
  \end{itemize*}
\end{spelleffect}
\begin{spellnotes}
  This spell gives no information about a creature's strengths or abilities -- only its weaknesses.
\end{spellnotes}

\spellsection{Disguise Self}
\spellschool{Illusion (Glamer) [Unreal]}
\spelllvl{Illus 1, Trickery 1}
\spellrng{\rngpers}
\spelltgt{You}
\spelldur{\durlong (D)}
\begin{spelleffect}
  You make yourself -- including clothing, armor, weapons, and equipment -- look different. You can seem 20\% (about 1 foot for an average human) shorter or taller, thin, fat, or in between. You cannot change your body type. Otherwise, the extent of the apparent change is up to you. You could add or obscure a minor feature or look like an entirely different person.
  \par The spell does not provide the abilities or mannerisms of the chosen form, nor does it alter the perceived tactile (touch) or audible (sound) properties or  you or your equipment. 
  \par If you use this spell to create a disguise, you get a \plus10 bonus on the Disguise check.
\end{spelleffect}
\begin{spellnotes}
  A creature that interacts with the effect gets a Will save to recognize it as an illusion. In order to interact with the illusion with a Perception check, the creature must make a Perception check that beats your saving throw DC with this spell or your Disguise check (if used as part of a disguise), whichever is higher. You cannot change your disguise once the spell is cast. 
\end{spellnotes}

\spellsectioncomma{Disguise Self}{Greater}
\spellschool{Illusion (Glamer) [Unreal]}
\spelllvl{Illus 3}
\spelldur{\durext (D)}
\begin{spelleffect}
  This spell functions like \spell{disguise self}, except that it lasts longer and you can change the disguise at will. By concentrating on the spell as a standard action, you can take on an entirely new appearance, just as if you has cast \spell{disguise self}. 
\end{spelleffect}

\spellsection{Disintegrate}
\spelldesc{You shoot a thin, green ray from your pointing finger that completely destroys whatever it hits.}
\spellschool{Transmutation (Alteration)}
\spelllvl{Destruction 6, Sor/Wiz 6}
\spellrng{\rngclose}
\spelleff{Ray}
\spelldur{Instantaneous}
\spellsave{Fortitude half (object)}
\spellsr{Yes (Fortitude)}
\spelldmg{12d8 physical damage \add d8 per two caster levels above 12th}
\begin{spelleffect}
  Any creature reduced to 0 hit points by this spell is entirely disintegrated, leaving behind only a trace of fine dust. A disintegrated creature's equipment is unaffected.
  \par When used against an object, the ray simply disintegrates as much as one 10-foot cube of nonliving matter. Thus, the spell disintegrates only part of any very large object or structure targeted. The ray affects even objects constructed entirely of force, such as \spell{forceful hand} or a \spell{wall of force}, but not magical effects such as a \spell{globe of invulnerability} or an \spell{antimagic field}.
\end{spelleffect}
\begin{spellnotes}
  Only the first creature or object struck can be affected; that is, the ray affects only one target per casting.
\end{spellnotes}

\spellsection{Dismissal}
\spellschool{Abjuration/Conjuration (Interdiction, Translocation) [Planar]}
\spelllvl{Clr 4, Sor/Wiz 4}
\spellrng{\rngclose}
\spelltgt{One extraplanar creature}
\spelldur{Instantaneous}
\spellsave{Will negates; see text}
\spellsr{Yes (Will)}
\begin{spelleffect}
  This spell forces an extraplanar creature, including any summoned creature, back to its proper plane. If the spell is successful, the creature is instantly whisked away, but there is a 20\% chance of actually sending the subject to a plane other than its own.
\end{spelleffect}

\spellsection{Dispel Magic}
\spellschool{Abjuration (Negation) [Magic]}
\spelllvl{Clr 3, Drd 4, Magic 3, Pal 3, Sor/Wiz 3}
\spellrng{\rngmed}
\spellarea{\areamed radius burst; see text}
\spelltgt{One creature or object; or everything in the area}
\spelldur{Instantaneous; see text}
\spellsave{None}
\spellsr{No}
\begin{spelleffect}
  You can use \spell{dispel magic} to end ongoing spells that have been cast on a creature or object, to temporarily suppress the magical abilities of a magic item, to end ongoing spells (or at least their effects) within an area, or to counter another spellcaster's spell. A dispelled spell ends as if its duration had expired. Some spells, as detailed in their descriptions, can't be defeated by dispel magic. Dispel magic can dispel (but not counter) spell-like effects just as it does spells.
  \par Note: The effect of a spell with an instantaneous duration can't be dispelled, because the magical effect is already over before the dispel magic can take effect. 
  \par You choose to use dispel magic in one of three ways: a targeted dispel, an area dispel, or a counterspell.

  \par \subspell{Targeted Dispel} One object, creature, or spell is the target of the dispel magic spell. You make a single dispel check (1d20 \add your caster level) which applies against all spells or effects currently active on the target. The DC for this dispel check is 11 \add the caster level of the effect. Your check is compared against each effect's DC. If you succeed on the check, each effect with that DC is dispelled.

  \par If you target an object or creature that is the effect of an ongoing spell (such as a monster summoned by monster summoning) and you succeed on your dispel check, you end the spell that conjured the object or creature.
  \par If the object that you target is a magic item, you compare your dispel check against the item's caster level. If you succeed, all the item's magical properties are suppressed for 5 rounds, after which the item recovers on its own. A suppressed item becomes nonmagical for the duration of the effect. An interdimensional interface (such as a bag of holding) is temporarily closed. A magic item's physical properties are unchanged: A suppressed magic sword is still a sword (a masterwork sword, in fact). Artifacts and deities are unaffected by mortal magic such as this.
  \par You may choose to automatically succeed on your dispel check against any spell that you cast yourself.
  \par \subspell{Area Dispel} When dispel magic is used in this way, the spell affects everything within a \areamed radius.

  \par This functions as a targeted dispel against every creature, object, and ongoing spell in the area, except that you can only dispel one effect from each target in the area. The effect dispelled is the one with the highest spell level that your dispel check would succeed against. If multiple spells qualify, choose randomly. Attended magic items are unaffected by an area dispel.

  \par \subspell{Counterspell} When dispel magic is used in this way, the spell targets a spellcaster and is cast as a counterspell. Unlike a true counterspell, however, dispel magic may not work; you must make a dispel check to counter the other spellcaster's spell.
\end{spelleffect}

\spellsectioncomma{Dispel Magic}{Greater}
\spellschool{Abjuration (Negation) [Magic]}
\spelllvl{Clr 6, Drd 6, Magic 6, Sor/Wiz 6}
\begin{spelleffect}
  This spell functions like \spell{dispel magic}, except that it affects every spell and effect in the area when used as an area dispel, as if a targeted dispel had been cast on every creature, object, and ongoing spell in the area. Attended magic items are unaffected.
  \par Additionally, this spell has a chance to dispel any effect that \spell{remove curse} can remove, even if \spell{dispel magic} can't dispel that effect.
\end{spelleffect}

\spellsectioncomma{Dispel Magic}{Lesser}
\spellschool{Abjuration (Negation) [Magic]}
\spelllvl{Clr 1, Drd 2, Magic 1, Sor/Wiz 1}
\begin{spelleffect}
  This spell functions like a targeted \spell{dispel magic}, except that you add half your caster level to your dispel check.
\end{spelleffect}

\spellsection{Displacement}
\spellschool{Illusion (Glamer)}
\spelllvl{Sor/Wiz 4}
\spellrng{\rngclose}
\spelltgt{One creature}
\spelldur{\durshort (D)}
\spellsave{Will negates (harmless)}
\spellsr{Yes (Will)}
\begin{spelleffect}
  The subject of this spell appears to be about 2 feet away from its true location. Attacks against the subject have a 50\% miss chance as if it were invisible. However, unlike invisibility, this spell does not prevent enemies from targeting the creature normally, and it does not allow the creature to hide.
\end{spelleffect}
\begin{spellnotes}
  True seeing reveals the subject's true location.
\end{spellnotes}

\spellsection{Disrupting Weapon}
\spellschool{Necromancy/Transmutation (Imbuement, Positive)}
\spelllvl{Clr 4, Pal 3}
\spellrng{\rngclose}
\spelltgt{One melee weapon}
\spelldur{\durshort}
\spellsave{Will negates (object)/Fortitude negates}
\spellsr{Yes (Will)/Yes (Fortitude)}
\begin{spelleffect}
  This spell infuses a melee weapon with positive energy, making it deadly to undead. Each round, the first bloodied undead creature struck by this weapon must succeed on a Fortitude save or be destroyed utterly. Healthy undead creatures suffer no ill effect.
\end{spelleffect}

\spellsection{Dissipating Touch}
\spelldesc{Your mere touch can disperse the surface material of your foe, sending a tiny portion of it far away.}
\spellschool{Conjuration (Translocation) [Teleportation]}
\spelllvl{Conj 2}
\spellrng{\rngtouch}
\spelltgt{Creature or object touched}
\spelldur{Instantaneous/1 round}
\spellsave{Will half (object)}
\spellsr{Yes (Will)}
\spelldmg{4d8 physical damage \add d8 per two caster levels above 4th}
\begin{spelleffect}
  The touched target takes damage and is sickened for 1 round. This damage ignores the hardness and damage reduction.
\end{spelleffect}

\spellsection{Divine Favor}
\spelldesc{You imbue yourself with skill in combat by calling upon the divine power of your patron.}
\spellschool{Transmutation (Augment)}
\spelllvl{Clr 1, Pal 1, Strength 1, War 1}
\spellrng{\rngpers}
\spelltgt{You}
\spelldur{\durshort}
\begin{spelleffect}
  You gain a \plus2 bonus on attack and weapon damage rolls. \bonusscalingdescription
\end{spelleffect}

\spellsection{Divine Power}
\spelldesc{You imbue yourself with great strength and skill in combat by calling upon the divine power of your patron.}
\spellschool{Transmutation (Augment)}
\spelllvl{Clr 4, Pal 4, Strength 4, War 4}
\begin{spelleffect}
  This spell functions like \spell{divine favor}, except that you also gain temporary hit points equal to 20 \add 1 per caster level above 8th, and a \plus3 bonus to Strength. This bonus increases to \plus4 at 14th level and to \plus5 at 20th level.

  If you take life damage, you lose all temporary hit points provided by this spell before applying the damage.
\end{spelleffect}

\spellsection{Dominate Monster}
\spellschool{Enchantment (Compulsion) [Domination, Mind-Affecting]}
\spelllvl{Ench 8}
\spelltgt{One creature}
\begin{spelleffect}
  This spell functions like \spell{dominate person}, except that the spell is not restricted by creature type.
\end{spelleffect}

\spellsection{Dominate Person}
\spellschool{Enchantment (Compulsion) [Domination, Mind-Affecting]}
\spelllvl{Ench 6}
\spelltime{Full-round action}
\spellrng{\rngclose}
\spelltgt{One humanoid}
\spelldur{One day}
\spellsave{Will negates}
\spellsr{Yes (Will)}
\begin{spelleffect}
  You can control the actions of any humanoid creature through a telepathic link that you establish with the subject's mind.
  \par If you and the subject have a common language, you can generally force the subject to perform as you desire, within the limits of its abilities. If no common language exists, you can communicate only basic commands, such as ``Come here," ``Go there," ``Fight," and ``Stand still." If you concentrate on the spell, you know what the subject is experiencing, but you do not receive direct sensory input from it, nor can it communicate with you telepathically.
  \par Once you have given a dominated creature a command, it continues to attempt to carry out that command to the exclusion of all other activities except those necessary for day-to-day survival (such as sleeping, eating, and so forth). Because of this limited range of activity, a Sense Motive check against DC 15 (rather than DC 25) can determine that the subject's behavior is being influenced by an enchantment effect (see the Sense Motive skill description).
  It takes time for the link to be established. For the first hour after the spell is cast, you must concentrate on the spell (a standard action) to control the subject's actions. While you are not concentrating on the spell, the creature acts as if confused, as the \spell{confusion} spell, except that it never attacks you. If the subject would randomly attack you, it instead is forced to follow your commands. At the end of the hour, the creature makes a second saving throw against the spell effect. If you concentrate on the spell during this time, it takes a \minus4 penalty on the saving throw. If it succeeds, it ignores the spell effect; otherwise, you dominate it fully for the remainder of the spell duration.
  \par After the first hour, changing your instructions or giving a dominated creature a new command is the equivalent of redirecting a spell, so it is a move action.
  \par By concentrating fully on the spell (a standard action), you can receive full sensory input as interpreted by the mind of the subject, though it still can't communicate with you. You can't actually see through the subject's eyes, so it's not as good as being there yourself, but you still get a good idea of what's going on.
  \par Subjects resist this control, and any subject forced to take actions against its nature receives a new saving throw. This does not apply when a subject is merely ordered to perform an action it disagrees with -- the action must be directly opposed to the subject's beliefs. Ordering a paladin to murder an innocent would grant the paladin a saving throw, but ordering him to build a bridge that would allow an evil army to cross a river would not grant him a saving throw. Obviously self-destructive orders are never carried out. Once control is established, the range at which it can be exercised is unlimited, as long as you and the subject are on the same plane. You need not see the subject to control it.
  \par If you recast this spell on a subject you have dominated before it escapes your control, you can extend the duration of the spell indefinitely. The subject does not get a new saving throw when you renew your control in this fashion.
\end{spelleffect}
\begin{spellnotes}
  \spell{Protection from evil} or a similar spell can prevent you from exercising control or using the telepathic link while the subject is so shielded, but such an effect neither prevents the establishment of domination nor dispels it.
\end{spellnotes}

\pdfbookmark[2]{E}{SpellDescriptionsE}
\begin{comment}
\subsubsection{E}
\end{comment}

\spellsection{Earth's Pull}
\spelldesc{You intensify the pull of gravity on your foe, causing it to feel much heavier and making its movements sluggish.}
\spellschool{Evocation (Control) [Earth]}
\spelllvl{Drd 1, Earth 1}
\spellrng{\rngmed}
\spelltgt{One Large or smaller creature}
\spelldur{\durshort}
\spellsave{No}
\spellsr{Yes (Will)}
\begin{spelleffect}
  The subject moves at half speed and takes a \minus2 penalty to armor class. If it is flying within 10 feet of the ground, the subject falls to the ground.
\end{spelleffect}
\begin{spellnotes}
  If the subject gets farther than 10 feet from the ground, the spell's effect is broken. As a result, the spell cannot affect creatures flying high above the ground.
\end{spellnotes}

\spellsection{Earthen Blade}
\spellschool{Transmutation (Alteration, Augment) [Earth]}
\spelllvl{Drd 2, Earth 2}
\spellrng{0 ft.}
\spelleff{Earthen weapon}
\spelldur{\durlong (D)}
\spellsave{None}
\spellsr{Yes (Fortitude)}
\begin{spelleffect}
    This spell creates a weapon from the ground. The weapon can be of any type you are proficient with. In addition, the weapon is magical, as the \spell{magic weapon} spell.
\end{spelleffect}

\spellsection{Earth Glide}
\spellschool{Transmutkation (Imbuement) [Earth]}
\spelllvl{Earth 5, Drd 5, Sor/Wiz 5}
\spellrng{Touch}
\spelltgt{Touched creature}
\spelldur{\durshort}
\spellsave{Fortitude negates (harmless)}
\spellsr{Yes (Fortitude)}
\begin{spelleffect}
  The subject gains the earth glide ability, as an earth elemental. This allows it to glide through stone, dirt, or almost any other sort of earth except metal  as if it were air. The subject can walk, run, or climb at any angle in the earth. However, the subject generally cannot breathe, speak, or hear while gliding. While gliding, a creature can remain partially within the earth, granting it cover.
\end{spelleffect}
\begin{spellnotes}
  The subject's burrowing leaves behind no tunnel or hole, nor does it create any ripple or other signs of its presence.
\end{spellnotes}

\spellsection{Earthquake}
\spellschool{Evocation (Control) [Earth]}
\spelllvl{Destruction 8, Drd 8, Clr 8, Earth 7}
\spellrng{\rngfar}
\spellarea{\arealarge radius spread (S)}
\spelldur{1 round}
\spellsave{See text}
\spellsr{No}
\begin{spelleffect}
  An intense but highly localized tremor rips the ground. The shock knocks creatures down, collapses structures, opens cracks in the ground, and more. The effect lasts for 1 round, during which time creatures on the ground can't move or attack. A spellcaster on the ground who attempts to cast a spell must make a Concentration check against a DC equal to (this spell's save DC \add double the level of the spell being cast) or lose the spell. The earthquake affects all terrain, vegetation, structures, and creatures in the area. The specific effect of an earthquake spell depends on the nature of the terrain where it is cast.
  \par \subspell{Cave, Cavern, or Tunnel} The spell collapses the roof, dealing 7d8 bludgeoning damage \add d8 per four caster levels above 14th to any creature caught under the cave-in (Reflex half) and pinning that creature beneath the rubble (see below). An earthquake cast on the roof of a very large cavern could also endanger those outside the actual area but below the falling debris.
  \par \subspell{Cliffs} Earthquake causes a cliff to crumble, creating a landslide that travels horizontally as far as it fell vertically. Any creature in the path takes 7d6 bludgeoning damage \add d6 per four caster levels above 14th (Reflex half) and is pinned beneath the rubble (see below).
  \par \subspell{Open Ground} Each creature standing in the area must make a Reflex save or fall down. Fissures open in the earth, and every creature on the ground has a 25\% chance to fall into one (Reflex save to avoid, whether the creature fell down or not). At the end of the spell, all fissures grind shut, dealing 7d10 bludgeoning damage \add d10 per four caster levels above 14th to any creatures trapped in them and ejecting their bodies (dead or alive).
  \par \subspell{Structure} Any structure standing on open ground takes 10 damage per caster level, enough to collapse a typical wooden or masonry building, but not a structure built of stone or reinforced masonry. Hardness does not reduce this damage, nor is it halved as damage dealt to objects normally is. Any creature caught inside a collapsing structure takes 7d6 bludgeoning damage \add d6 per four caster levels above 14th (Reflex half) and is pinned beneath the rubble (see below).
  \par \subspell{River, Lake, or Marsh} Fissures open underneath the water, draining away the water from that area and forming muddy ground. Soggy marsh or swampland becomes quicksand for the duration of the spell, sucking down creatures and structures. Each creature in the area must make a Reflex save or sink down in the mud and quicksand. At the end of the spell, the rest of the body of water rushes in to replace the drained water, possibly drowning those caught in the mud.
\end{spelleffect}
\begin{spellnotes}
  Any creature pinned beneath rubble takes 1d6 nonlethal damage per minute while pinned. If a pinned character falls unconscious, he or she must make a DC 15 Constitution check or take 1d6 lethal damage each minute thereafter until freed or dead.
\end{spellnotes}

\spellsection{Elemental Swarm}
\spellschool{Conjuration (Summoning) [see text]}
\spelllvl{Air 9, Drd 9, Earth 9, Fire 9, Water 9}
\spellrng{\rngmed}
\spelleff{Two or more summoned creatures in a \areamed radius}
\spelldur{\durlong (D)}
\spellsave{None}
\spellsr{No}
\begin{spelleffect}
  This spell opens a portal to an Elemental Plane and summons elementals from it. A druid can choose the plane (Air, Earth, Fire, or Water); a cleric opens a portal to the plane matching his domain.
  \par When the spell is complete, 2d4 Large elementals appear. Five minutes later, 1d4 Huge elementals appear. Five minutes after that, one greater elemental appears. Each elemental has maximum hit points per HV. Once these creatures appear, they serve you for the duration of the spell.
  \par The elementals obey you explicitly and never attack you, even if someone else manages to gain control over them. You do not need to concentrate to maintain control over the elementals. You can dismiss them singly or in groups at any time.
  \par When you use a summoning spell to summon an air, earth, fire, or water creature, it is a spell of that type.
\end{spelleffect}

\spellsection{Energy Conversion}
\spellschool{Abjuration/Evocation (Energy, Shielding) [see text]}
\spelllvl{Protection 7, Sor/Wiz 7}
\spellrng{Personal and \rngclose; see text}
\spelleff{Ray; see text}
\spelldur{\durlong or until discharged}
\spellsave{None}
\spellsr{Yes (Fortitude)}
\begin{spelleffect}
  This spell functions like \spell{greater resist energy}, except that you store up the energy you absorb and can later discharge it as a ray. To discharge a ray requires a standard action. You can choose to fire any number of rays during the spell's duration. The ray you fire must be of one of the energy types you have stored (if you have stored more than one type, you can choose what kind of energy to use for each ray). If a ray successfully strikes its target (requiring a ranged touch attack), the target takes damage equal to the amount of energy damage of that type you have stored, up to a maximum of three times your caster level. As long as this power remains in effect, you can continue to absorb energy damage and fire additional rays using the stored damage.
\end{spelleffect}
\begin{spellnotes}
  This spell's descriptor is the same as the type of energy you discharge in a ray; thus, its subtype can change during the course of the spell's duration.
\end{spellnotes}

\spellsection{Energy Drain}
\spellschool{Necromancy (Vitalism) [Negative]}
\spelllvl{Clr 8, Death 8, Evil 8, Sor/Wiz 8}
\begin{spelleffect}
  This spell functions like \spell{enervation}, except that the target gains six negative levels.
  \par An undead creature struck by the ray instead gains temporary hit points equal to 40 \add twice your caster level and physical damage reduction 16/positive. 
\end{spelleffect}
\begin{spellnotes}
  The damage reduction allows an undead subject to ignore the first 16 physical damage it takes each round. If it is hit by an attack that deals positive damage, such as \spell{cure light wounds}, it cannot use its damage reduction for 1 round.
\end{spellnotes}

\spellsection{Enervation}
\spelldesc{Your foe's body loses its color momentarily as you drain its life force away.}
\spellschool{Necromancy (Vitalism) [Negative]}
\spelllvl{Death 4, Sor/Wiz 4}
\spellrng{\rngclose}
\spelltgt{One creature} 
\spelldur{\durlong}
\spellsave{Fortitude negates}
\spellsr{Yes (Fortitude)}
\begin{spelleffect}
  The target gains three negative levels.
  \par Each negative level gives a creature a \minus1 penalty on attack rolls, saving throws, checks, and effective level (for determining the power, DC, and other details of spells or special abilities). Additionally, a spellcaster loses one spell or spell slot from his or her highest available level. If the subject has at least as many negative levels as HV, it dies.
  \par An undead creature struck by the ray gains physical damage reduction 8/positive instead. This damage reduction increases by 1 per two caster levels above 8th.
\end{spelleffect}
\begin{spellnotes}
  This spell stacks with any effect that bestows negative levels, including itself.

  The damage reduction allows an undead subject to ignore the first 8 physical damage it takes each round. If it is hit by an attack that deals positive damage, such as \spell{cure light wounds}, it cannot use its damage reduction for 1 round.
\end{spellnotes}

\spellsection{Enfeeblement}
%\spelldesc{You fire a coruscating ray from your hand. When it strikes your foe, he becomes weaker.}
\spellschool{Necromancy (Flesh)}
\spelllvl{Death 1, Sor/Wiz 1}
\spellrng{\rngmed}
\spelltgt{One creature}
\spelldur{\durshort}
\spellsave{Fortitude negates}
\spellsr{Yes (Fortitude)}
\begin{spelleffect}
  The subject takes a \minus4 penalty to your choice of Strength, Dexterity, or Constitution.
\end{spelleffect}
\begin{spellnotes}
  This spell cannot reduce the subject's attributes below \minus9.
\end{spellnotes}

\spellsection{Enlarge Person}
\spellschool{Transmutation (Polymorph) [Size-Affecting]}
\spelllvl{Strength 3, Trans 3}
\spelltime{Full-round action}
\spellrng{\rngclose}
\spelltgt{One humanoid creature}
\spelldur{\durshort (D)}
\spellsave{Fortitude negates}
\spellsr{Yes (Fortitude)}
\begin{spelleffect}
  This spell causes instant growth of a humanoid creature, doubling its height and multiplying its weight by 8. This increase changes the creature's size category to the next larger one. This has several effects.
  \begin{itemize*} 
    \item \plus10 ft. inherent bonus to movement speed.
    \item \minus1 penalty to attack rolls and AC due to its increased size.
  \item \minus2 penalty to Dexterity.
  \item \plus2 bonus to Strength. \bonusscalingdescription
  \end{itemize*}
  \par A humanoid creature whose size increases to Large has a space of 10 feet and a natural reach of 10 feet.
  \par If insufficient room is available for the desired growth, the creature attains the kmaximum possible size and may make a Strength check (using its increased Strength) to burst any enclosures in the process. If it fails, it is constrained without harm by the materials enclosing it -- the spell cannot be used to crush a creature by increasing its size.
  \par All equipment worn or carried by a creature is similarly enlarged by the spell. Melee and projectile weapons affected by this spell deal more damage. Other magical properties are not affected by this spell. Any enlarged item that leaves an enlarged creature's possession (including a projectile or thrown weapon) instantly returns to its normal size. This means that thrown weapons deal their normal damage, and projectiles deal damage based on the size of the weapon that fired them. Magical properties of enlarged items are not increased by this spell.
\end{spelleffect}
\begin{spellnotes}
  Multiple magical effects that increase size do not stack.
  \par \spellindirect{enlarge person}{Enlarge person} counters and dispels \spell{reduce person}.
  \par \spellindirect{enlarge person}{Enlarge person} can be made permanent with a \spell{permanency} spell.
\end{spellnotes}

\spellsectioncomma{Enlarge Person}{Mass}
\spellschool{Transmutation (Polymorph) [Size-Affecting]}
\spelllvl{Strength 7, Trans 7}
\spellrng{\rngmed}
\spellarea{\areamed radius limit}
\spelltgts{Five humanoid creatures within the area}
\begin{spelleffect}
  This spell functions like \spell{enlarge person}, except that it affects multiple creatures.
\end{spelleffect}

\spellsection{Entangle}
\spellschool{Transmutation (Animation)}
\spelllvl{Drd 1, Nature 1}
\spellrng{\rngmed}
\spellarea{\areasmall radius spread}
\spelldur{\durshort (D)}
\spellsave{Reflex partial}
\spellsr{No}
\begin{spelleffect}
  Grasses, weeds, bushes, and even trees wrap, twist, and entwine about creatures in the area or those that enter the area, holding them fast and causing them to become entangled. The creature can break free and move half its normal speed by using a standard action to make a combat maneuver check or an Escape Artist check against this spell's save DC. A creature that succeeds on a Reflex save is not entangled but can still move at only half speed through the area. Each round on your turn, the plants once again attempt to entangle all creatures that have avoided or escaped entanglement.
\end{spelleffect}
\begin{spellnotes}
  The effects of the spell may be altered somewhat based on the nature of the entangling plants. If no plants exist in the area, the spell has no effect.
\end{spellnotes}

\spellsection{Entangling Growth}
\spellschool{Transmutation (Alteration, Animation)}
\spelllvl{Drd 4, Nature 4}
\spellarea{\areamed radius spread}
\begin{spelleffect}
  This spell functions like \spell{entangle}, except that it affects a wider area and also grows new plants in the area. These plants grow from any terrain, even if it would not normally support plant life, and entangle creatures in the area for the duration of the spell. When the magic fades, the plants with and recede into the ground, leaving no trace that they were ever there.
\end{spelleffect}

\spellsection{Entropic Shield}
\spelldesc{You surround your ally with a magical field that glows with a chaotic blast of multicolored hues. This field deflects incoming ranged attacks, causing them to randomly swerve away from their intended target.}
\spellschool{Abjuration (Shielding)}
\spelllvl{Chaos 2, Clr 2}
\spellrng{\rngclose}
\spelltgt{Touched creature}
\spelldur{\durshort (D)}
\begin{spelleffect}
  Each ranged attack directed at the subject for which the attacker must make an attack roll has a 50\% miss chance (similar to the effects of active cover). Other attacks that simply work at a distance are not affected.
\end{spelleffect}

\spellsection{Ethereal Jaunt}
\spellschool{Conjuration (Translocation) [Planar]}
\spelllvl{Sor/Wiz 5, Travel 5}
\spellrng{\rngpers}
\spelltgt{You}
\spelldur{\durshort (D)}
\begin{spelleffect}
  You become ethereal, along with your equipment. For the duration of the spell, you are in a place called the Ethereal Plane, which overlaps the normal, physical, Material Plane. When the spell expires, you return to material existence.
  \par An ethereal creature is invisible, insubstantial, and capable of moving in any direction, even up or down, albeit at half normal speed. As an insubstantial creature, you can move through solid objects, including living creatures. An ethereal creature can see and hear on the Material Plane, but everything looks gray and ephemeral. Sight and hearing onto the Material Plane are limited to 60 feet.
  \par Force effects and abjurations affect an ethereal creature normally. Their effects extend onto the Ethereal Plane from the Material Plane, but not vice versa. An ethereal creature can't attack material creatures, and spells you cast while ethereal affect only other ethereal things. Certain material creatures or objects have attacks or effects that work on the Ethereal Plane (such as a basilisk's gaze attack). Treat other ethereal creatures and ethereal objects as if they were material. 
  \par If you end the spell and become material while inside a material object (such as a solid wall), you are shunted off to the nearest open space and take 1d6 damage per 5 feet that you so travel.
\end{spelleffect}

\spellsection{Etherealness}
\spellschool{Conjuration (Translocation) [Planar]}
\spelllvl{Sor/Wiz 9, Travel 9}
\spellrng{Touch; see text}
\spelltgts{You and one other touched creature per three levels}
\spellsave{None}
\spellsr{Yes (Will)}
\begin{spelleffect}
  This spell functions like \spell{ethereal jaunt}, except that you and other willing creatures joined by linked hands (along with their equipment) become ethereal. Besides yourself, you can bring one creature per three caster levels to the Ethereal Plane. Once ethereal, the subjects need not stay together.
\end{spelleffect}
\begin{spellnotes}
  When the spell expires, all affected creatures on the Ethereal Plane return to material existence.
\end{spellnotes}

\spellsection{Expeditious Retreat}
\spellschool{Transmutation (Temporal)}
\spelllvl{Trans 1}
\spellrng{\rngclose}
\spelltgt{One creature}
\spelldur{\durshort (D)}
\spellsave{Will negates (harmless)}
\spellsr{Yes (Will)}
\begin{spelleffect}
  Your base land speed doubles, to a maximum of a \plus30 foot increase. (This adjustment is treated as an enhancement bonus.) There is no effect on other modes of movement.
\end{spelleffect}
\begin{spellnotes}
 As with any effect that increases your speed, this spell affects your jumping distance.
 \end{spellnotes}

\pdfbookmark[2]{F}{SpellDescriptionsF}
\begin{comment}
\subsubsection{F}
\end{comment}

\spellsection{Faerie Fire}
\spellschool{Illusion (Figment) [Light, Unreal]}
\spelllvl{Drd 1}
\spellrng{\rngmed}
\spellarea{\areasmall radius limit}
\spelleff{Dim lights in the area}
\spelldur{\durshort (D)}
\spellsave{None}
\spellsr{Yes (Will)}
\begin{spelleffect}
  A pale glow surrounds and outlines all creatures and objects in the area. Outlined subjects shed light as candles. Outlined creatures do not benefit from the concealment normally provided by darkness (though a 3rd-level or higher magical darkness effect functions normally), \spell{blur}, \spell{displacement}, \spell{invisibility}, or similar effects. Illusionary figments such as \spell{silent image} are not outlined, which may reveal them for what they are.
  
  The light is too dim to have any special effect on undead or dark-dwelling creatures vulnerable to light. The \spell{faerie fire} can be blue, green, or violet, according to your choice at the time of casting. This spell does not cause any harm to the objects or creatures thus outlined.
\end{spelleffect}

\spellsection{False Life}
\spelldesc{You harness the power of life to grant yourself a limited ability to avoid death.}
\spellschool{Necromancy (Life)}
\spelllvl{Necro 1}
\spellrng{\rngpers}
\spelltgt{You}
\spelldur{\durshort}
\begin{spelleffect}
  You gain 10 temporary hit points \add 2 per caster level above 2nd. If you take life damage, you lose all temporary hit points provided by this spell before applying the damage.
\end{spelleffect}

\spellsection{Farsight}
\spelldesc{You grant the subject the ability to see farther and more accurately.}
\spellschool{Divination (Awareness)}
\spelllvl{Div 1}
\spellrng{\rngtouch}
\spelltgt{Creature touched}
\spelldur{\durlong (D)}
\spellsave{Will negates (harmless)}
\spellsr{Yes (Will)}
\begin{spelleffect}
  The subject gains a \plus2 bonus to Spot checks and takes half the normal penalty for range increments and for Spot checks made at a distance. \bonusscalingdescription
\end{spelleffect}

\spellsection{Fear}
\spelldesc{You project an invisible cone that drives creatures away from you in abject fear.}
\spellschool{Enchantment (Emotion) [Fear, Mind-Affecting]}
\spelllvl{Ench 5}
\spellarea{\areamed cone-shaped burst}
\spelldur{\durshort (D)}
\spellsave{Will negates}
\spellsr{Yes (Will)}
\begin{spellhealthy}
  Creatures in the area are shaken, causing them to be vulnerable.
\end{spellhealthy}
\begin{spellblood}
  Creatures in the area are frightened.
\end{spellblood}
\begin{spellnotes}
  A vulnerable creature takes a \minus2 penalty to attack rolls, saving throws, checks, DCs, and AC.
\end{spellnotes}

\spellsection{Feather Fall}
\spellschool{Evocation (Control) [Air]}
\spelllvl{Air 1, Evoc 1, Travel 1}
\spellcmp{V}
\spelltime{1 immediate action}
\spellrng{\rngmed}
\spellarea{\areamed radius limit}
\spelltgts{Five Medium or smaller freefalling object or creatures within the area}
\spelldur{\durshort or until landing}
\spellsave{Will negates (harmless) or Will negates (object)}
\spellsr{Yes (Will)}
\begin{spelleffect}
  The affected creatures or objects fall slowly. Feather fall instantly changes the rate at which the targets fall to a mere 60 feet per round (equivalent to the end of a fall from a few feet), and the subjects take no damage upon landing while the spell is in effect. However, when the spell duration expires, a normal rate of falling resumes.
  \par The spell affects one or more Medium or smaller creatures (including gear and carried objects up to each creature's maximum load) or objects, or the equivalent in larger creatures: A Large creature or object counts as two Medium creatures or objects, a Huge creature or object counts as two Large creatures or objects, and so forth.
  \par If the spell is cast on a falling item, the object does half normal damage based on its weight, with no bonus for the height of the drop.
\end{spelleffect}
\begin{spellnotes}
  You can cast this spell instantaneously, quickly enough to save yourself if you unexpectedly fall. 
  \par Feather fall works only upon free-falling objects. It no special effect on ranged weapons unless they are falling quite a distance. It does not affect a sword blow or a charging or flying creature.
\end{spellnotes}

\spellsection{Feeblemind}
\spellschool{Enchantment (Inhibition) [Mind-Affecting]}
\spelllvl{Sor/Wiz 5}
\spellrng{\rngtouch}
\spelltgt{Touched creature}
\spelldur{Instantaneous}
\spellsave{Will negates}
\spellsr{Yes (Will)}
\begin{spellhealthy}
  The target is bewildered, making it vulnerable for 5 rounds.
\end{spellhealthy}
\begin{spellblood}
  The target's Intelligence drops to \minus9, giving it roughly the intellect of a lizard. It is unable to use Intelligence-based skills, cast spells, understand language, or communicate coherently. Still, it knows who its friends are and can follow them and even protect them. The subject remains in this state until a \spell{heal}, \spell{limited wish}, \spell{miracle}, or \spell{wish} spell is used to cancel the effect of the \spell{feeblemind}. A creature that can cast arcane spells, such as a sorcerer or a wizard, takes a \minus4 penalty on its saving throw.
\end{spellblood}
\begin{spellnotes}
  A vulnerable creature takes a \minus2 penalty to attack rolls, saving throws, checks, DCs, and AC.

  The target must be bloodied when the spell is cast to suffer the bloodied effect.
\end{spellnotes}

\spellsection{Finger of Death}
\spellschool{Necromancy (Life) [Death]}
\spelllvl{Death 7, Necro 7}
\spellrng{\rngclose}
\spelltgt{One living creature}
\spelldur{Instantaneous}
\spellsave{Fortitude negates}
\spellsr{Yes (Fortitude)}
\begin{spellhealthy}
  The target is staggered for 5 rounds. It can take a move action or a standard action each round, but not both.
\end{spellhealthy}
\begin{spellblood}
  The target is instantly slain.
\end{spellblood}

\spellsection{Fire Seeds}
\spellschool{Evocation/Transmutation (Energy, Imbuement) [Fire]}
\spelllvl{Drd 6, Fire 6, Nature 6}
\spellrng{\rngtouch}
\spellarea{\areasmall or \areamed radius burst from the touched objects; see text}
\spelltgts{Up to four touched acorns or up to eight touched holly berries}
\spelldur{\durlong or until used}
\spellsave{None or Reflex half; see text}
\spellsr{Yes (Reflex)}
\spelldmg{6d6 fire damage \add d6 per four caster levels above 12th (acorn grenades);\par
6d8 fire damage \add d8 per four caster levels above 12th (holly berry bombs)}
\begin{spelleffect}
  Depending on the version of fire seeds you choose, you turn acorns into splash weapons that you or another character can throw, or you turn holly berries into bombs that you can detonate on command.
  \par \subspell{Acorn Grenades} As many as four acorns turn into special splash weapons that can be hurled as far as 100 feet. A ranged touch attack roll is required to strike the intended target. If you miss, the acorn detonates in a random corner of the intended target square. Together, the acorns are capable of dealing 6d6 fire damage \add d6 per four caster levels above 12th, divided up among the acorns as you wish.
  \par Each acorn explodes upon striking any hard surface, damaging all creatures in a \areasmall radius burst. A creature within this area that makes a successful Reflex saving throw takes only half damage; a creature struck directly is not allowed a saving throw.
  \par \subspell{Holly Berry Bombs} You turn as many as eight holly berries into special bombs. The holly berries are usually placed by hand, since they are too light to make effective thrown weapons (they can be tossed only 5 feet). Together, the holly berries are capable of dealing 6d6 fire damage \add d6 per four caster levels above 12th, divided up among the berries as you wish.
  \par If you are within \rngmed range and speak a word of command (as a standard action), each berry instantly bursts into flame, striking every creature in a \areamed radius burst. A creature in the area that makes a successful Reflex saving throw takes only half damage.
\end{spelleffect}
\begin{spellnotes}
  You can only have one \spell{fire seeds} active at any time.
\end{spellnotes}
\par \mfx{Material Component} The acorns or holly berries.

\spellsection{Fire Shield}
\spelldesc{You appear to immolate yourself in a wreath of flame that lashes out at anyone who tries to harm you.}
\spellschool{Abjuration/Evocation (Energy, Shielding) [Fire or Cold]}
\spelllvl{Fire 4, Sor/Wiz 4}
\spellrng{\rngpers}
\spelltgt{You}
\spelldur{\durshort (D)}
\spellsave{None/Reflex half}
\spellsr{No/Yes (Reflex)}
\spelldmg{4d6 fire or cold damage \add d6 per four levels above 8th}
\begin{spelleffect}
  Any creature that hits you with its body or a melee weapon takes damage. Each individual creature can take this damage only once per round. The damage type and other effects depend on which kind of \spell{fire shield} is used. This decision must be made at the time the spell is cast.

  \par \subspell{Warm Shield} The flames are warm to the touch and deal fire damage. You gain cold damage reduction 20 \add 1 per caster level above 8th.
  \par \subspell{Chill Shield} The flames are cool to the touch and deal cold damage. You gain fire damage reduction 20 \add 1 per caster level above 8th.

  Regardless of the version, the flames are thin and wispy, giving off light equal to only half the illumination of a normal torch (10 feet).
\end{spelleffect}
\begin{spellnotes}
  The damage reduction allows the subject to ignore the first 20 energy damage it takes each round of the appropriate type. Creatures wielding weapons with exceptional reach are not subject to this spell's damage if they attack you.
\end{spellnotes}

\spellsection{Fire Storm}
\spelldesc{You fill a massive area with sheets of roaring flame, burning everyone who opposes you.}
\spellschool{Evocation (Energy) [Fire]}
\spelllvl{Destruction 8, Drd 8, Fire 8, War 8}
\spellrng{\rngmed}
\spellarea{\arealarge spread}
\spelldur{Instantaneous}
\spellsave{Reflex half}
\spellsr{Yes (Reflex)}
\spelldmg{8d6 fire damage \add d6 per four caster levels above 16th}
\begin{spelleffect}
  The spell deals damage to all enemies in the area, leaving your allies unscathed.
\end{spelleffect}

\spellsection{Fireball}
\spelldesc{You create an explosion of flame that detonates with a low roar, damaging nearby creatures and objects.}
\spellschool{Evocation (Energy) [Destructive, Fire]}
\spelllvl{Fire 3, Sor/Wiz 3}
\spellrng{\rngmed}
\spellarea{\areasmall radius spread}
\spelldur{Instantaneous}
\spellsave{Reflex half}
\spellsr{Yes (Reflex)}
\spelldmg{3d6 fire damage \add d6 per four caster levels above 6th.}
\begin{spelleffect}
  Everything in the area takes damage.
\end{spelleffect}
\begin{spellnotes}
  If a destructive spell deals enough damage to an interposing barrier to shatter or breaks through it, its effects may continue beyond the barrier if the area permits; otherwise, it stops at the barrier just as any other spell effect does.
\end{spellnotes}

\spellsection{Flame Blade}
\spellschool{Evocation (Energy) [Fire]}
\spelllvl{Drd 2, Fire 2}
\spellrng{0 ft.}
\spelleff{Sword-like beam}
\spelldur{\durlong (D)}
\spellsave{None}
\spellsr{Yes (Fortitude)}
\begin{spelleffect}
  A 3 foot long beam of red-hot fire springs forth from your hand. In addition to providing illumination like a torch, you can wield this bladelike beam as a weapon. It is treated like a scimitar, except that all damage dealt with it is fire damage, you add half your casting attribute to damage in place of half your Strength, and it is treated as a light weapon, so you can use Dexterity to attack with it. Alternately, you can hurl flames from the weapon up to \rngmed range as if it were a thrown weapon.
\end{spelleffect}
\begin{spellnotes}
  Fire spells do not function underwater. A \spell{flame weapon} can ignite combustible materials such as parchment, straw, dry sticks, and cloth. Spell resistance applies when a foe is struck by the weapon, but not when the blade is created.
\end{spellnotes}

\spellsection{Flame Strike}
\spelldesc{You call a vertical column of divine fire that roars downward, consuming your unworthy foes.}
\spellschool{Evocation (Channeling, Energy) [Fire]}
\spelllvl{Clr 5, Destruction 5, Fire 5, War 5}
\spellrng{\rngclose}
\spellarea{\areamed radius cylinder, 40 ft. high}
\spelldur{Instantaneous}
\spellsave{Reflex half}
\spellsr{Yes (Reflex)}
\spelldmg{5d6 fire and divine damage \add d6 per four caster levels above 8th; see text}
\begin{spelleffect}
  Half the damage is fire damage, but the other half results directly from divine power. Your allies in the area take half damage.
\end{spelleffect}

\spellsection{Fly}
\spellschool{Transmutation (Imbuement)}
\spelllvl{Sor/Wiz 4}
\spellrng{\rngtouch}
\spelltgt{Creature touched}
\spelldur{\durshort}
\spellsave{Fortitude negates (harmless)}
\spellsr{Yes (Fortitude)}
\begin{spelleffect}
  The subject can fly at a speed of 60 feet (or 40 feet if it wears medium or heavy armor, or if it carries a medium or heavy load). It can ascend at half speed and descend at double speed, and its maneuverability is good. Using a \spell{fly} spell requires only as much concentration as walking, so the subject can attack or cast spells normally. The subject of a \spell{fly} spell can charge but not run, and it cannot carry aloft more weight than its maximum load.
\end{spelleffect}

\spellsection{Fog Cloud}
\spelldesc{You conjure a bank of fog from a location you choose, concealing those inside.}
\spellschool{Conjuration (Creation) [Fog]}
\spelllvl{Drd 3, Sor/Wiz 3, Water 3}
\spellrng{\rngmed}
\spellarea{\areamed radius cylinder-shaped spread}
\spelleff{Fog in the area}
\spelldur{\durshort}
\spellsave{None}
\spellsr{No}
\begin{spelleffect}
  Everything within the spell's area has concealment (\plus4 AC). The cloud is stationary once created.
\end{spelleffect}
\begin{spellnotes}
  Fog spells do not function underwater and can be dispersed by wind. A moderate wind (11\add mph) disperses the fog in 5 rounds; a strong wind (21\add mph) disperses the fog in 1 round. A fire spell burns away the fog in the area into which it deals damage.
\end{spellnotes}

\spellsection{Forcecage}
\spellschool{Evocation (Control) [Force]}
\spelllvl{Evoc 7}
\spellrng{\rngmed}
\spelleff{Barred cage (20 ft. cube) or windowless cell (10 ft. cube)}
\spelldur{\durlong (D)}
\spellsave{Reflex negates; see text}
\spellsr{No}
\begin{spelleffect}
  This powerful spell brings into being an immobile, invisible cubical prison composed of either bars of force or solid walls of force (your choice).
  \par Creatures within the area are caught and contained unless they are too big to fit inside, in which case the spell automatically fails. Teleportation and other forms of astral travel provide a means of escape, but the force walls or bars extend into the Ethereal Plane, blocking ethereal travel.
  \par A creature who makes a Reflex save chooses whether it wants to be inside or outside of the forcecage when it forms. The forcecage is formed regardless.
  \par Like a \spell{wall of force} spell, a forcecage resists \spell{dispel magic}, but it is vulnerable to a \spell{disintegrate} spell, and it can be destroyed by a \magicitem{sphere of annihilation} or a \magicitem{rod of cancellation}.
  \par \subspell{Barred Cage} This version of the spell produces a 20-foot cube made of bands of force (similar to a wall of force spell) for bars. The bands are a half-inch wide, with half-inch gaps between them. Any creature capable of passing through such a small space can escape; others are confined. You can't attack a creature in a barred cage with a weapon unless the weapon can fit between the gaps. Even against such weapons (including arrows and similar ranged attacks), a creature in the barred cage has cover. All spells and breath weapons can pass through the gaps in the bars.
  \par \subspell{Windowless Cell} This version of the spell produces a 10-foot cube with no way in and no way out. Solid walls of force form its six sides.
\end{spelleffect}

\spellsection{Forceful Hand}
\spellschool{Evocation (Control) [Force]}
\spelllvl{Evoc 4}
\begin{spelleffect}
  This spell functions like \spell{interposing hand}, except that it can also pursue and bull rush one opponent you select. You must direct the hand to bull rush an opponent as a swift action. If you do, the \spell{forceful hand} may make a bull rush attack that does not provoke an attack of opportunity. Its CMA to bull rush equals your caster level \add your casting attribute, \plus4 for being Large. Its CMD is equal to 10 \add its CMA. 
  \par The hand always moves with the opponent to push them back as far as possible. It has no movement limit for this purpose, but it cannot exceed the spell's range.
  \par If you do not direct the hand to bull rush, it simply provides cover as \spell{interposing hand}.
\end{spelleffect}
\begin{spellnotes}
  A very strong creature could not push the hand out of its way because the latter would instantly reposition itself between the creature and you, but an opponent could push the hand up against you by successfully bull rushing it.
\end{spellnotes}

\spellsection{Foresight}
\spelldesc{You bestow a powerful sixth sense to your ally, giving them clear visions of any imminent danger.}
\spellschool{Divination (Knowledge)}
\spelllvl{Knowledge 9, Protection 9, Div 9}
\spellrng{Touch}
\spelltgt{Touched creature}
\spelldur{\durlong (D)}
\spellsave{None or Will negates (harmless)}
\spellsr{No or Yes (Will)}
\begin{spelleffect}
  The subject receives instantaneous warnings of impending danger or harm that would befall them. She is never surprised or flat-footed, and gains a \plus20 bonus on initiative checks. In addition, the spell gives the subject a general idea of what action she might take to best protect herself and bestows a \plus5 bonus to Reflex saves and dodge modifier.
\end{spelleffect}
\begin{spellnotes}
  \spell{Foresight} is a difficult spell to cast, since it requires maintaining a constant channel into the future. You may only have one \spell{foresight} spell active at once. If you cast the spell again before the duration wears off, the old spell is dismissed and only the new casting is active. 
\end{spellnotes}

\spellsection{Freedom}
\spellschool{Transmutation (Imbuement)}
\spelllvl{Clr 4, Drd 4, Travel 4}
\spellrng{\rngtouch}
\spelltgt{Creature touched}
\spelldur{\durshort}
\spellsave{Will negates (harmless)}
\spellsr{Yes (Will)}
\begin{spelleffect}
  The subject can move and attack normally for the duration of the spell, even under the influence of magic that usually impedes movement, such as paralysis, \spell{solid fog}, \spell{slow}, and \spell{web}. The subject gains a \plus20 bonus to CMD against grapple attacks, as well as on grapple attacks or Escape Artist checks made to escape a grapple or a pin.
  \par The spell also allows the subject to move and attack normally while underwater, provided that the weapon is wielded in the hand rather than hurled.
\end{spelleffect}

\spellsectioncomma{Freedom}{Mass}
\spellschool{Transmutation (Imbuement)}
\spelllvl{Clr 8, Drd 8, Travel 8}
\spellrng{\rngmed}
\spellarea{\areamed radius limit}
\spelltgts{Five creatures within the area}
\begin{spelleffect}
  This spell functions like \spell{freedom}, except that it affects multiple creatures.
\end{spelleffect}

\spellsection{Freezing Sphere}
\spelldesc{You create a frigid globe of cold energy that streaks from your fingertips to a location you select and explodes.}
\spellschool{Evocation (Energy) [Cold]}
\spelllvl{Sor/Wiz 6, Water 6}
\spellrng{\rngmed}
\spellarea{\areamed radius burst}
\spelldur{Instantaneous/5 rounds; see text}
\spellsave{Reflex half; see text}
\spellsr{Yes (Reflex)}
\spelldmg{6d6 cold damage \add d6 per four levels above 12th}
\begin{spelleffect}
  Creatures in the area take damage. 
  
  If the \spell{freezing sphere} strikes a body of water or a liquid that is principally water (not including water-based creatures), it freezes the liquid to a depth of 6 inches over an area equal to 100 square feet (a 10-foot square) per caster level (maximum 1,500 square feet). This ice lasts for 5 rounds. Creatures that were swimming on the surface of frozen water become trapped in the ice. Attempting to break free is a full-round action. A trapped creature must make a Strength check or an Escape Artist check against this spell's save DC to do so.
\end{spelleffect}

\pdfbookmark[2]{G}{SpellDescriptionsG}
\begin{comment}
\subsubsection{G}
\end{comment}

\spellsection{Gaseous Form}
\spellschool{Transmutation (Polymorph)}
\spelllvl{Air 3, Trans 3, Travel 3}
\spellcmp{S}
\spellrng{\rngtouch}
\spelltgt{Willing corporeal creature touched}
\spelldur{\durshort (D)}
\spellsave{None}
\spellsr{No}
\begin{spelleffect}
  The subject and all its gear become insubstantial, misty, and translucent. Its material armor (including natural armor) becomes worthless, though other modifiers continue to apply normally. The subject gains physical damage reduction 10/magic and becomes immune to critical hits. It can't attack or cast spells with verbal, somatic, material, or focus components while in gaseous form. (This does not rule out the use of certain spells that the subject may have prepared using the feats Silent Spell or Still Spell.) If it has a touch spell ready to use, that spell is discharged harmlessly when the gaseous form spell takes effect.
  \par A gaseous creature can't run, but it can fly at a speed of 10 feet (maneuverability perfect). It can pass through small holes or narrow openings, even mere cracks, with all it was wearing or holding in its hands, as long as the spell persists. The creature is subject to the effects of wind, and it can't enter water or other liquid. It also can't manipulate objects or activate items, even those carried along with its gaseous form. Continuously active items remain active, though in some cases their effects may be moot.
\end{spelleffect}
\begin{spellnotes}
  This spell's damage reduction allows the subject to ignore the first 10 physical damage it takes each round. If it is hit by a magical attack, such as a damaging spell or magic weapon, it cannot use its damage reduction for 1 round.
\end{spellnotes}

\spellsection{Gentle Descent}
\spelldesc{You grant your ally ephemeral wings which allow him to glide.}
\spellschool{Transmutation (Imbuement) [Air]}
\spelllvl{Air 2, Drd 2}
\spellrng{\rngmed}
\spelltgt{One creature}
\spelldur{\durlong}
\spellsave{Will negates (harmless)}
\spellsr{Yes (Will)}
\begin{spelleffect}
  The subject gains a 30 foot glide speed. It must spend a move action each round to glide.
\end{spelleffect}
\begin{spellnotes}
  A creature with a glide speed can glide while in the air. Each round, a gliding creature moves forward at a rate equal to its glide speed and moves five feet down. It may choose to move slower, to a minimum of half its normal glide speed. It may alternately choose to dive, allowing it to move forward at a rate equal to twice its glide speed but also moving twenty feet down. A gliding creature cannot run.
\end{spellnotes}

\begin{comment}
\spellsection{Ghoul Touch}
\spellschool{Necromancy (Flesh)}
\spelllvl{Sor/Wiz 3}
\spellrng{\rngtouch}
\spelltgt{Living creature touched}
\spelldur{\durshort}
\spellsave{None/Fortitude negates}
\spellsr{Yes (Fortitude)}
\begin{spelleffect}
  The subject is sickened, making it vulnerable.
\end{spelleffect}
\begin{spellblood}
  In addition, the subject is paralyzed if it fails a Fortitude save. Each round that it is paralyzed, the subject can make a new saving throw. If it succeeds, it is no longer paralyzed by the spell, though it is still sickened. In addition, as long as it is paralyzed, the subject exudes a carrion stench that causes all living creatures (except you) in a \areasmall radius spread to become sickened (Fortitude negates) for 5 rounds.
\end{spellblood}
\begin{spellnotes}
  A vulnerable creature takes a \minus2 penalty to attack rolls, saving throws, checks, DCs, and AC.
\end{spellnotes}
\end{comment}

\spellsection{Giant Vermin}
\spellschool{Transmutation (Polymorph)}
\spelllvl{Drd 4, Nature 4}
\spellrng{\rngclose}
\spellarea{\areamed radius limit}
\spelltgts{Up to three vermin within the area}
\spelldur{\durmed}
\spellsave{None}
\spellsr{Yes (Fortitude)}
\begin{spelleffect}
  You turn three normal-sized centipedes, two normal-sized spiders, or a single normal-sized scorpion into Large-sized forms. Only one type of vermin can be transmuted (so a single casting cannot affect both a centipede and a spider), and all must be grown to the same size.
  \par Any giant vermin created by this spell do not attempt to harm you, but your control of such creatures is limited to simple commands (``Attack," ``Defend," ``Stop," and so forth). Orders to attack a certain creature when it appears or guard against a particular occurrence are too complex for the vermin to understand. Unless commanded to do otherwise, the giant vermin attack whoever or whatever is near them.
\end{spelleffect}

\spellsection{Glibness}
\spelldesc{Your speech becomes more fluent and believable.}
\spellschool{Enchantment/Transmutation (Imbuement)}
\spelllvl{Brd 3}
\spellcmp{S}
\spellrng{\rngpers}
\spelltgt{You}
\spelldur{\durlong (D)}
\begin{spelleffect}
  You gain a \plus20 bonus on Bluff checks made to convince another of the truth of your words. (This bonus doesn't apply to other uses of the Bluff skill, such as creating a diversion to hide or communicating a hidden message via innuendo.)
  \par If a magical effect is used against you that would detect your lies or force you to speak the truth the user of the effect must succeed on a caster level check (1d20 \add caster level) against a DC of 15 \add your caster level to succeed. Failure means the effect does not detect your lies or force you to speak only the truth.
  \par At the end of the duration of the spell, anyone who only believed your words because of the bonus from \spell{glibness} realizes that they have been lied to.
\end{spelleffect}

\spellsection{Glitterdust}
\spellschool{Conjuration (Creation)}
\spelllvl{Sor/Wiz 2}
\spellrng{\rngmed}
\spellarea{\areasmall radius spread}
\spelleff{Glittering particles in the area}
\spelldur{\durshort}
\spellsave{None}
\spellsr{No}
\begin{spelleffect}
  A cloud of golden particles covers everyone and everything in the area, visibly outlining invisible things for the duration of the spell. It likewise negates the effects of \spell{blur} and \spell{displacement}, and reveals illusionary figments such as \spell{silent image} for what they are. All within the area at the time that the spell is cast are covered by the dust, which continues to sparkle until it fades.
  \par Any creature covered by the dust takes a \minus40 penalty on Hide checks.
\end{spelleffect}
\begin{spelleffect}
  Water and similar substances can remove the dust.
\end{spelleffect}

\spellsectioncomma{Glitterdust}{Greater}
\spellschool{Conjuration (Creation)}
\spelllvl{Sor/Wiz 5}
\spellsave{None}
\begin{spelleffect}
  This spell functions like \spell{glitterdust}, except that creatures in the area are also dazzled for the duration of the spell.
\end{spelleffect}
\begin{spellnotes}
  A dazzled creature has a 20\% miss chance on all attack rolls, and takes a \minus4 penalty to Spot checks. He is also unable to see with darkvision.
\end{spellnotes}

\spellsection{Globe of Invulnerability}
\spellschool{Abjuration (Negation) [Magic]}
\spelllvl{Sor/Wiz 5}
\begin{spelleffect}
  This spell functions like \spell{lesser globe of invulnerability}, except that it also excludes 4th level spells and spell-like effects.
\end{spelleffect}

\spellsectioncomma{Globe of Invulnerability}{Lesser}
\spellschool{Abjuration (Negation) [Magic]}
\spelllvl{Sor/Wiz 4}
\spellarea{\areasmall radius emanation, centered on you}
\spelldur{\durshort (D)}
\spellsave{None}
\spellsr{No}
\begin{spelleffect}
  An immobile, faintly shimmering magical sphere surrounds you and excludes all spell effects of 3rd level or lower. The area or effect of any such spells does not include the area of the lesser globe of invulnerability. Such spells fail to affect any target located within the globe. Excluded effects include spell-like abilities and spells or spell-like effects from items. However, any type of spell can be cast through or out of the magical globe. Spells of 4th level and higher are not affected by the globe, nor are spells already in effect when the globe is cast. The globe can be brought down by a \spell{dispel magic} spell or similar effects. You can leave and return to the globe without penalty.
\end{spelleffect}
\begin{spellnotes}
  Spell effects are not disrupted unless their effects enter the globe, and even then they are merely suppressed, not dispelled. 
  \par If a given spell has more than one level depending on which character class is casting it, use the level appropriate to the caster to determine whether lesser globe of invulnerability stops it.
\end{spellnotes}

\spellsection{Good Hope}
\spelldesc{You instill powerful hope and confidence in nearby allies.}
\spellschool{Enchantment (Compulsion) [Mind-Affecting, Morale]}
\spelllvl{Brd 3}
\spellarea{\areasmall radius limit centered on you}
\spelltgt{Five creatures within the area}
\spelldur{Instantaneous}
\spellsave{Will negates (harmless)}
\spellsr{Yes (Will)}
\begin{spelleffect}
  The subjects gain a \plus2 bonus on attack rolls and temporary hit points equal to 15 \add 1 per caster level above 6th for 5 rounds. \bonusscalingdescription

   If you take life damage, you lose all temporary hit points provided by this spell before applying the damage.
\end{spelleffect}
\begin{spellnotes}
  \spell{Good hope} counters and dispels \spell{crushing despair}.
\end{spellnotes}

\spellsection{Grasping Hand}
\spellschool{Evocation (Control) [Force]}
\spelllvl{Evoc 6}
\begin{spelleffect}
  This spell functions like \spell{interposing hand}, except the hand can also grapple one opponent that you select. You must direct the hand to grapple an opponent as a swift action. If you do, the \spell{grasping hand} may make a grapple attack as a swift action. Its CMA to grapple equals your caster level \add your casting attribute, \plus4 for being Large. Its CMD is equal to 10 \add its CMA.
  \par The hand holds but does not harm creatures it grapples. While the hand is grappling a foe, you must spend a swift action each round to sustain the hand's grapple; otherwise, the grappled creature escapes automatically.
  \par If you do not direct the hand to bull rush, it simply provides cover as \spell{interposing hand}.
\end{spelleffect}
\begin{spellnotes}
  Directing the spell to a new target is a swift action.
\end{spellnotes}

\spellsection{Grease}
\spelldesc{You conjure a layer of slippery grease on the ground, tripping up your foes.}
\spellschool{Conjuration (Creation)}
\spelllvl{Sor/Wiz 1}
\spellrng{\rngclose}
\spelltgtorarea{One object or a 10 ft. square}
\spelldur{\durshort (D)}
\spellsave{See text}
\spellsr{No}
\begin{spelleffect}
  Any creature in the area when the spell is cast must make a successful Reflex save or fall. A creature can walk within or through the area of grease at half normal speed with a DC 10 Balance check. Failure means it can't move that round, while failure by 5 or more means it falls (see the Balance skill for details). A creature standing in a greased area loses its Dexterity and dodge modifiers to AC due to the slippery surface.
  \par The spell can also be used to create a greasy coating on an item. Material objects not in use are always affected by this spell, while an object wielded or employed by a creature receives a Reflex saving throw to avoid the effect entirely. If the initial saving throw fails, the creature immediately drops the item. If the item is successfully greased, a saving throw must be made in each round that the creature attempts to pick up or use the greased item. A creature wearing greased armor or clothing gains a \plus10 bonus on Escape Artist checks and on grapple attacks made to resist or escape a grapple or to escape a pin.
\end{spelleffect}

\spellsection{Greater (Spell Name)}
\par Any spell whose name begins with greater is alphabetized in this chapter according to the second word of the spell name. Thus, the description of a greater spell appears near the description of the spell on which it is based. Spell chains that have greater spells in them include those based on the spells arcane sight, command, dispel magic, glyph of warding, invisibility, magic fang, magic weapon, planar ally, planar binding, prying eyes, restoration, scrying, shadow conjuration, shadow evocation, shout, and teleport.

\spellsection{Gust of Wind}
\spellschool{Evocation (Control) [Air]}
\spelllvl{Air 1, Drd 1, Evoc 1}
\spellarea{\arealarge line-shaped emanation from you}
\spelleff{Wind within the area}
\spelldur{1 round}
\spellsave{Fortitude partial; see text.}
\spellsr{No}
\begin{spelleffect}
  This spell creates a severe blast of air (approximately 50 mph) that originates from you, affecting all creatures in its path. Creatures are affected according to their size category. A successful Fortitude save causes a creature to be affected as if it were one size category larger. Flying creatures are affected as if one size category smaller.
  \begin{itemize*}
    \item Tiny or smaller creatures are knocked prone and blown to the edge of the spell's range.
    \item Small creatures are knocked prone by the force of the wind.
    \item Medium creatures are unable to move forward against the force of the wind.
    \item Large or larger creatures may move normally.
  \end{itemize*}
  \par Any creature, regardless of size, takes a \minus4 penalty on ranged attacks and Listen checks in the spell's area.
  \par In addition to the effects noted, a gust of wind can do anything that a sudden blast of wind would be expected to do. It can extinguish open flames, create a stinging spray of sand or dust, fan a large fire, overturn delicate awnings or hangings, heel over a small boat, and blow gases or vapors to the edge of its range.
\end{spelleffect}
\begin{spellnotes}
  \spell{Gust of wind} can be made permanent with a \spell{permanency} ritual.
\end{spellnotes}

\pdfbookmark[2]{H}{SpellDescriptionsH}
\begin{comment}
\subsubsection{H}
\end{comment}

\spellsection{Harm}
\spelldesc{You fill your foe with a massive influx of negative energy, crippling its body.}
\spellschool{Necromancy (Vitalism) [Negative]}
\spelllvl{Clr 6, Death 6, Evil 6, Vitality 6, Sor/Wiz 6}
\spellrng{\rngtouch}
\spelltgt{Creature touched}
\spelldur{Instantaneous}
\spellsave{Fortitude half/None}
\spellsr{Yes (Fortitude)}
\spelldmg{12d8 negative energy damage \add d8 per two caster levels above 12th}
\begin{spelleffect}
  The touched creature takes damage. In addition, it takes four points of Constitution damage. A successful Fortitude save halves the negative energy damage but does not mitigate the Constitution damage.
\end{spelleffect}
\begin{spellnotes}
  This effect can cause a creature to begin dying without being disabled first.
\end{spellnotes}
\begin{spellnotes}
  If used on an undead creature, \spell{harm} acts like \spell{heal}.
\end{spellnotes}

\spellsection{Haste}
\spelldesc{You accelerate your ally's motions, causing her to move and act more quickly than normal.}
\spellschool{Transmutation (Temporal)}
\spelllvl{Trans 4}
\spellrng{Touch}
\spelltgts{One creature}
\spelldur{\durshort}
\spellsave{Will negates (harmless)}
\spellsr{Yes (Will)}
\begin{spelleffect}
  The subject is hasted. This has two effects. First, when making a full attack action, a hasted creature may make one extra attack at a \minus5 penalty.
  \par Second, all of the hasted creature's modes of movement (including land movement, burrow, climb, fly, and swim) double in speed, to a maximum of an additional 30 ft. of movement. This increase counts as an enhancement bonus, and it affects the creature's jumping distance as normal for increased speed.
\end{spelleffect}
\begin{spellnotes}
  \spell{Haste} dispels and counters \spell{slow}. The extra attack granted is not cumulative with similar effects, such as that provided by a weapon of speed, nor does it actually grant an extra action, so you can't use it to cast a second spell or otherwise take an extra action in the round.
\end{spellnotes}

\spellsectioncomma{Haste}{Mass}
\spelldesc{You accelerate your allies' motions, causing them to move and act more quickly than normal.}
\spellschool{Transmutation (Temporal)}
\spelllvl{Trans 8}
\spellarea{\areamed radius limit}
\spelltgts{Five creatures within the area}
\begin{spelleffect}
  This spell functions like \spell{haste}, except that it affects multiple creatures.
\end{spelleffect}

\spellsection{Heal}
\spelldesc{You fill the subject with a massive influx of positive energy, restoring its body to its fullest.}
\spellschool{Necromancy (Vitalism) [Healing, Positive]}
\spelllvl{Clr 6, Drd 7, Good 6, Vitality 6}
\spellrng{\rngtouch}
\spelltgt{Creature touched}
\spelldur{Instantaneous}
\spellsave{Fortitude negates (harmless) or Fortitude negates; see text}
\spellsr{Yes (Fortitude)}
\spellheal{12d8 \add d8 per two caster levels above 12th}
\begin{spelleffect}
  This spell heals the subject and immediately ends any and all of the following adverse conditions affecting the target: ability damage, blinded, confused, dazed, dazzled, deafened, diseased, exhausted, fatigued, feebleminded, insanity, nauseated, sickened, stunned, and poisoned.

  \par In addition, for every 10 points of healing granted by the spell, it can instead cure 1 point of critical damage.
\end{spelleffect}
\begin{spellnotes}
  \spell{Heal} does not remove negative levels, restore permanently drained levels, or restore permanently drained attribute points.
  \par If used against an undead creature, \spell{heal} instead acts like \spell{harm}.
\end{spellnotes}

\spellsection{Heal Mount}
\spelldesc{You fill your mount with a massive influx of positive energy, restoring its body to its fullest.}
\spellschool{Necromancy (Vitalism) [Healing, Positive]}
\spelllvl{Pal 3}
\spellcmp{V}
\spellrng{\rngtouch}
\spelltgt{Your mount touched}
\spelldur{Instantaneous}
\spellsave{Fortitude negates (harmless)}
\spellsr{Yes (Fortitude)}
\spellheal{6d8 \add d8 per two caster levels above 6th}
\begin{spelleffect}
  This spell functions like \spell{heal}, but it affects only the paladin's special mount.
\end{spelleffect}

\spellsection{Heat Metal}
\spellschool{Evocation (Energy) [Fire]}
\spelllvl{Drd 2}
\spellrng{\rngmed}
\spelltgt{Metal equipment of one creature within the area}
\spelldur{\durshort (D)}
\spellsave{See text}
\spellsr{Yes (Will)}
\spelldmg{2d6 fire damage per round \add 1d6 per four levels above 4th; see text}
\begin{spelleffect}
  This spell makes metal burning hot, causing it to deal damage each round. A creature not touching metal takes no damage from this spell. A creature wielding metal equipment can attempt a Fortitude save for half damage each round. A creature wearing metal armor receives no saving throw, and is also vulnerable for the duration of the spell.
\end{spelleffect}
\begin{spellnotes}
A vulnerable creature takes a \minus2 penalty to attack rolls, saving throws, checks, DCs, and AC.

  If the subject is underwater, this spell deals half damage, boiling the surrounding water, and the subject is not vulnerable. Any cold intense enough to damage the creature negates fire damage from the spell (and vice versa) on a point-for-point basis.
\end{spellnotes}

\spellsection{Heroism}
\spelldesc{You imbue your ally with great bravery and morale in battle.}
\spellschool{Enchantment (Emotion) [Mind-Affecting, Morale]}
\spelllvl{Ench 3}
\spellrng{\rngclose}
\spelltgt{One creature}
\spelldur{\durshort (D)}
\spellsave{Will negates (harmless)}
\spellsr{Yes (Will)}
\begin{spelleffect}
  The subject gains a \plus2 bonus on attack rolls, checks, and saving throws. \bonusscalingdescription
\end{spelleffect}

\spellsectioncomma{Heroism}{Greater}
\spellschool{Enchantment (Emotion) [Mind-Affecting, Morale]}
\spelllvl{Ench 7}
\begin{spelleffect}
  This spell functions like \spell{heroism}, except the subject also gains temporary hit points equal to 60 \add 2 per caster level above 12th. In addition, the subject is immune to fear and hostile morale effects.
\end{spelleffect}

\spellsection{Hideous Laughter}
\spelldesc{You force the subject to collapse into gales of manic laughter with an unnaturally amusing joke.}
\spellschool{Enchantment (Compulsion) [Mind-Affecting]}
\spelllvl{Brd 2}
\spellrng{\rngclose}
\spelltgt{One creature}
\spelldur{\durshort}
\spellsave{Will negates}
\spellsr{Yes (Will)}
\begin{spelleffect}
  The subject is bewildered, making it vulnerable.
\end{spelleffect}
\begin{spellblood}
  In addition, the subject is flat-footed and must spend a standard action each round to do nothing but laugh uncontrollably. After each time it laughs, the affected creature can attempt a new saving throw. If it succeeds, it can stop laughing, though it is still bewildered.
\end{spellblood}
\begin{spellnotes}
  A creature with an Intelligence score of \minus8 or lower is not affected. A creature whose type is different from the caster's receives a \plus4 circumstance bonus on its saving throw, because humor doesn't ``translate'' well.
\end{spellnotes}
\begin{spellnotes}
  A vulnerable creature takes a \minus2 penalty to attack rolls, saving throws, checks, DCs, and AC.
\end{spellnotes}

\spellsection{Hold Monster}
\spellschool{Enchantment (Inhibition) [Mind-Affecting]}
\spelllvl{Law 4, Sor/Wiz 4}
\spellrng{\rngmed}
\spelltgt{One living creature}
\begin{spelleffect}
  This spell functions like \spell{hold person}, except that it is not limited by creature type.
\end{spelleffect}

\spellsectioncomma{Hold Monster}{Mass}
\spellschool{Enchantment (Inhibition) [Mind-Affecting]}
\spelllvl{Sor/Wiz 9}
\spellarea{\areamed radius limit}
\spelltgts{Five creatures within the area}
\begin{spelleffect}
  This spell functions like \spell{hold monster}, except that it affects multiple creatures.
\end{spelleffect}

\spellsection{Hold Person}
\spellschool{Enchantment (Inhibition) [Mind-Affecting]}
\spelllvl{Clr 2, Pal 2, Sor/Wiz 2, War 2}
\spellrng{\rngclose}
\spelltgt{One humanoid creature}
\spelldur{\durshort (D); see text}
\spellsave{Will negates; see text}
\spellsr{Yes (Will)}
\begin{spellhealthy}
  The subject is bewildered, making it vulnerable.
\end{spellhealthy}
\begin{spellblood}
  As the healthy effect, and the subject is paralyzed and unable to act. Each round on its turn, the subject may attempt a new saving throw to end the paralysis. If it succeeds, it is no longer paralyzed, though it is still bewildered and can take no other actions that round.
\end{spellblood}
\begin{spellnotes}
  A vulnerable creature takes a \minus2 penalty to attack rolls, saving throws, checks, DCs, and AC.
\end{spellnotes}

\spellsectioncomma{Hold Person}{Mass}
\spellschool{Enchantment (Inhibition) [Mind-Affecting]}
\spelllvl{Clr 7, Sor/Wiz 7}
\spellrng{\rngmed}
\spellarea{\areamed radius limit}
\spelltgts{Five creatures within the area}
\begin{spelleffect}
  This spell functions like \spell{hold person}, except that it affects multiple creatures.
\end{spelleffect}

\spellsection{Holy Aura}
\spellschool{Abjuration (Interdiction) [Good]}
\spelllvl{Clr 8, Good 8}
\spellcmp{V, S, F}
\spellarea{\areamed radius limit centered on you}
\spelltgts{Five creatures within the area}
\spelldur{\durshort (D)}
\spellsave{See text}
\spellsr{Yes (Will)}
\begin{spelleffect}
  A brilliant divine radiance surrounds the subjects, protecting them from attacks, granting them resistance to spells cast by evil creatures, and damaging evil creatures when they strike the subjects. This abjuration has four effects.
  \par First, each shielded creature gains a \plus5 bonus to its saving throws.
  \par Second, each shielded creature gains spell resistance 10 against evil spells and spells cast by evil creatures.
  \par Third, the abjuration blocks possession and mental influence, just as protection from evil does.
  \par Finally, if a evil creature within \rngmed range of the shielded creature successfully attacks it in any way, the offending attacker takes 4d6 damage. Any single creature can take this damage only once per round.
\end{spelleffect}
\spellfocus{A tiny reliquary containing some sacred relic. The reliquary costs at least 500 gp.}

\spellsection{Holy Smite}
\spellschool{Evocation (Channeling) [Good]}
\spelllvl{Good 4}
\spellrng{\rngmed}
\spelltgt{One creature}
\spelldur{Instantaneous/5 rounds}
\spellsave{None/Will half}
\spellsr{Yes (Will)}
\spelldmg{8d6 divine damage \add d6 per two caster levels above 8th}
\begin{spelleffect}
  If the target is not good, it takes damage and is bewildered for 5 rounds. A successful Will save halves the damage.
\end{spelleffect}

\spellsection{Holy Sword}
\spelldesc{You channel holy power into your sword, or any other melee weapon you choose, allowing it to smite your foes with ease.}
\spellschool{Evocation/Transmutation (Imbuement, Channeling) [Good]}
\spelllvl{Pal 4}
\spellcmp{V}
\spellrng{\rngtouch}
\spelltgt{Melee weapon touched}
\spelldur{\durmed}
\spellsave{Will negates (object)}
\spellsr{Yes (Will)}
\begin{spelleffect}
  The affected weapon acts as a \plus5 holy weapon. The spell is automatically cancelled 1 round after the weapon leaves your hand. You cannot have more than one \spell{holy sword} at a time.
  \par If this spell is cast on a magic weapon, the powers of the spell supersede any that the weapon normally has, rendering the normal bonus and powers of the weapon inoperative for the duration of the spell.
\end{spelleffect}
\begin{spellnotes}
  This spell is not cumulative with any other spell that might modify the weapon in any way. It does not work on artifacts.
  %What is a holy weapon?
\end{spellnotes}

\spellsection{Holy Word}
\spellschool{Evocation (Channeling) [Good]}
\spelllvl{Clr 7, Good 7}
\spellcmp{V}
\spellarea{\arealarge radius spread centered on you}
\spelldur{Instantaneous}
\spellsave{None}
\spellsr{Yes (Will)}
\begin{spellhealthy}
  \par Each nongood creature in the area is deafened for 5 rounds.
\end{spellhealthy}
\begin{spellblood}
  \par Each nongood creature in the area suffers one or more of the following ill effects, depending on its Hit Values.
  \begin{dtable}
    \begin{tabularx}{\columnwidth}{l >{\lcol}X}
      \par \thead{HV} & \thead{Effect} \\
      \par Equal to caster level & Deafened \\
      \par Up to caster level \minus5 & Blinded, deafened \\
      \par Up to caster level \minus10 & Paralyzed, blinded, deafened \\
      \par Up to caster level \minus15 & Killed\fn{1}
    \end{tabularx}
    1 Living creatures die. Nonliving creatures are destroyed.
  \end{dtable}
  \par \subspell{Deafened} The creature is deafened for 5 rounds.
  \par \subspell{Blinded} The creature is blinded for 2 rounds.
  \par \subspell{Paralyzed} The creature is paralyzed and helpless for 5 rounds.
  \par \subspell{Killed} Living creatures die. Nonliving creatures are destroyed.
\end{spellblood}
\begin{spellnotes}
  Creatures whose Hit Values exceed your caster level are unaffected by \spell{holy word}.
\end{spellnotes}

\spellsection{Horrid Wilting}
\spelldesc{You dessicate your foes from a great distance, shriveling their bodies.}
\spellschool{Necromancy (Flesh)}
\spelllvl{Necro 8, Water 8}
\spellrng{\rngfar}
\spellarea{\arealarge radius limit}
\spelltgts{Ten living creatures within the area}
\spellsave{Fortitude half}
\spellsr{Yes (Fortitude)}
\spelldmg{8d6 physical damage \add d6 per four caster levels above 16th}
\begin{spelleffect}
  Each target takes damage. Plants and creature with the water subtype take a \minus5 penalty on their saving throw. 
\end{spelleffect}

\spellsection{Hypnotic Pattern}
\spelldesc{You create a twisting pattern of subtle, shifting colors that weaves through the air, fascinating creatures within it.}
\spellschool{Enchantment/Illusion (Compulsion, Figment) [Light, Mind-Affecting]}
\spelllvl{Sor/Wiz 3}
\spellrng{\rngmed}
\spellarea{\areasmall radius spread}
\spelleff{Colorful lights in the area}
\spelldur{\durshort}
\spellsave{Will negates}
\spellsr{Yes (Will)}
\begin{spelleffect}
  Creatures within the spell's area are fascinated. Each fascinated creature stands or sits quietly, taking no actions other than to pay attention to the fascinating effect for the duration of the spell. It takes a \minus4 penalty on skill checks made as reactions, such as Listen and Spot checks. Any potential threat, such as a hostile creature approaching, allows the fascinated creature a new saving throw. Any obvious threat, such as noticing someone draw a weapon, cast a spell, or aim a ranged weapon at the fascinated creature automatically breaks the effect. A fascinated creature's ally may shake it free of the spell as a standard action.
\end{spelleffect}
\begin{spellnotes}
  Creatures who cannot see the lights are not affected by this spell.
\end{spellnotes}

\pdfbookmark[2]{I}{SpellDescriptionsI}
\begin{comment}
\subsubsection{I}
\end{comment}

\spellsection{Ice Storm}
\spelldesc{You conjure magical hailstones that pound down, smashing and chilling creatures in their path.}
\spellschool{Conjuration/Evocation (Creation, Energy) [Cold]}
\spelllvl{Destruction 4, Drd 4, Sor/Wiz 4, Water 4}
\spellrng{\rngmed}
\spellarea{\areasmall radius cylinder, 20 ft. high}
\spelldur{Instantaneous/1 round}
\spellsave{None}
\spellsr{Yes (Reflex)}
\spelldmg{4d4 cold and bludgeoning damage \add d4 per four caster levels above 8th}
\begin{spelleffect}
  All creatures within the area take damage. The area is difficult terrain for 1 round.
\end{spelleffect}

\spellsection{Implosion}
\spelldesc{You create a destructive responance in your foe's body that destroys it from the inside out.}
\spellschool{Evocation (Control)}
\spelllvl{Clr 9, Destruction 9}
\spellrng{\rngclose}
\spelltgts{One corporeal creature/round}
\spelldur{Instantaneous and concentration (up to 5 rounds); see text}
\spellsave{None/Fortitude negates}
\spellsr{Yes (Fortitude)}
\begin{spellhealthy}
  The target creature you concentrate on is staggered for 5 rounds. It can take a move action or a standard action each round, but not both.
\end{spellhealthy}
\begin{spellblood}
  The target is instantly slain.
\end{spellblood}
\begin{spellnotes}
  You can concentrate on one creature per round. You can target a particular creature only once with each casting of the spell.
  \par \spell{Implosion} has no effect on creatures in \spell{gaseous form} or on incorporeal creatures.
\end{spellnotes}

\spellsection{Imprisonment}
\spellschool{Conjuration/Transmutation (Time, Translocation) [Teleportation]}
\spelllvl{Earth 9, Law 9, Sor/Wiz 9}
\spellrng{\rngclose}
\spelltgt{One creature}
\spelldur{See text}
\spellsave{Will negates}
\spellsr{Yes (Will)}
\spelldmg{18d8 physical damage \add d8 per two caster levels above 18th}
\begin{spelleffect}
  The target takes damage as its body is partially teleported away, and it is slowed for 5 rounds. This damage ignores hardness and damage reduction.
\end{spelleffect}
\begin{spellblood}
  If the creature is touching the ground, it becomes permanently entombed in a state of suspended animation (as the \spell{temporal stasis} spell) in a small sphere far beneath the surface of the earth. It remains there unless an \spell{emancipation} spell is cast at the locale where the imprisonment took place.
\end{spellblood}
\begin{spellnotes}
  A slowed creature can take only a single move action or standard action each turn, but not both. Additionally, it takes a \minus2 penalty to attack rolls, Strength and Dexterity-based skill checks, and armor class.

  The subject must be bloodied at the time that the spell is cast to be imprisoned. Magical search by a crystal ball, a \spell{locate creature} spell, or some other similar divination does not reveal the fact that a creature is imprisoned, but \spell{discern location} does. A \spell{wish} or \spell{miracle} spell will not free the recipient, but will reveal where it is entombed.
\end{spellnotes}

\spellsection{Inertial Shield}
\spelldesc{You create a barrier around your ally that resists physical intrusion.}
\spellschool{Abjuration (Shielding)}
\spelllvl{Sor/Wiz 2}
\spellrng{\rngclose}
\spelltgt{One creature}
\spelldur{\durshort}
\spellsave{Will negates (harmless)}
\spellsr{Yes (Will)}
\begin{spelleffect}
  The subject gains physical damage reduction 4/force. This damage reduction increases by 1 per two caster levels above 4th.
\end{spelleffect}
\begin{spellnotes}
  This spell's damage reduction allows the subject to ignore the first 4 physical damage it takes each round. If it is hit by a attack that deals force damage, such as \spell{magic missile}, it cannot use its damage reduction for 1 round.
\end{spellnotes}

\spellsection{Inflict Critical Wounds}
\spellschool{Necromancy (Vitalism) [Negative]}
\spelllvl{Clr 4, Sor/Wiz 4}
\spelldmg{8d6 negative energy damage \add d6 per two caster levels above 8th}
\begin{spelleffect}
  This spell functions like \spell{inflict light wounds}, except that for every 10 points of damage dealt in excess of the subject's hit points, it can instead inflict 1 point of critical damage.
\end{spelleffect}
\begin{spellnotes}
  This effect can cause a creature to begin dying without being disabled first.
\end{spellnotes}

\spellsectioncomma{Inflict Critical Wounds}{Mass}
\spellschool{Necromancy (Vitalism) [Negative]}
\spelllvl{Clr 8, Sor/Wiz 8}
\spelldmg{8d6 negative energy damage \add d6 per four caster levels above 10th}
\begin{spelleffect}
  This spell functions like \spell{mass inflict light wounds}, except that for every 10 points of damage dealt in excess of the subject's hit points, it can instead inflict 1 point of critical damage.
\end{spelleffect}
\begin{spellnotes}
  This effect can cause a creature to begin dying without being disabled first.
\end{spellnotes}

\spellsection{Inflict Light Wounds}
\spellschool{Necromancy (Vitalism) [Negative]}
\spelllvl{Clr 1, Sor/Wiz 1}
\spellrng{\rngtouch}
\spelltgt{Creature touched}
\spelldur{Instantaneous}
\spellsave{Fortitude half}
\spellsr{Yes (Fortitude)}
\spelldmg{2d6 negative energy damage \add d6 per two caster levels above 2nd}
\begin{spelleffect}
  The touched creature takes damage. Since undead are powered by negative energy, this spell heals them instead of dealing damage. You must succeed on a melee touch attack to hit a target that does not allow you to touch it. 
\end{spelleffect}

\spellsectioncomma{Inflict Light Wounds}{Mass}
\spellschool{Necromancy (Vitalism) [Negative]}
\spelllvl{Clr 5, Sor/Wiz 5}
\spellrng{\rngclose}
\spellarea{\areamed radius limit}
\spelltgts{Five creatures within the area}
\spelldur{Instantaneous}
\spellsave{Fortitude half}
\spellsr{Yes (Fortitude)}
\spelldmg{5d6 negative energy damage \add d6 per four caster levels above 10th}
\begin{spellnotes}
  The targets take damage. Like other \spellindirect{inflict light wounds}{inflict} spells, \spell{mass inflict light wounds} heals undead instead of dealing damage. A cleric capable of spontaneously casting \spellindirect{inflict light wounds}{inflict} spells can also spontaneously cast \spellindirect{mass inflict light wounds}{mass inflict} spells.
\end{spellnotes}

\spellsection{Inflict Moderate Wounds}
\spellschool{Necromancy (Vitalism) [Negative]}
\spelllvl{Clr 2, Sor/Wiz 2}
\spelldmg{4d6 negative energy damage \add d6 per two caster levels above 4th}
\begin{spelleffect}
  This spell functions like \spell{inflict light wounds}, except that for every 20 points of damage dealt in excess of the subject's hit points, it can instead inflict 1 point of critical damage.
\end{spelleffect}
\begin{spellnotes}
  This effect can cause a creature to begin dying without being disabled first.
\end{spellnotes}

\spellsectioncomma{Inflict Moderate Wounds}{Mass}
\spellschool{Necromancy (Vitalism) [Negative]}
\spelllvl{Clr 6, Sor/Wiz 6}
\spelldmg{6d6 negative energy damage \add d6 per four caster levels above 10th}
\begin{spelleffect}
  This spell functions like \spell{mass inflict light wounds}, except that for every 20 points of damage dealt in excess of the subject's hit points, it can instead inflict 1 point of critical damage.
\end{spelleffect}
\begin{spellnotes}
  This effect can cause a creature to begin dying without being disabled first.
\end{spellnotes}

\spellsection{Inflict Serious Wounds}
\spellschool{Necromancy (Vitalism) [Negative]}
\spelllvl{Clr 3, Sor/Wiz 3}
\spelldmg{6d6 negative energy damage \add d6 per four caster levels above 6th}
\begin{spelleffect}
  This spell functions like \spell{inflict light wounds}, except that for every 15 points of damage dealt in excess of the subject's hit points, it can instead inflict 1 point of critical damage.
\end{spelleffect}
\begin{spellnotes}
  This effect can cause a creature to begin dying without being disabled first.
\end{spellnotes}

\spellsectioncomma{Inflict Serious Wounds}{Mass}
\spellschool{Necromancy (Vitalism) [Negative]}
\spelllvl{Clr 7, Sor/Wiz 7}
\spelldmg{7d6 negative energy damage \add d6 per four caster levels above 10th}
\begin{spelleffect}
  This spell functions like \spell{mass inflict light wounds}, except that for every 15 points of damage dealt in excess of the subject's hit points, it can instead inflict 1 point of critical damage.
\end{spelleffect}
\begin{spellnotes}
  This effect can cause a creature to begin dying without being disabled first.
\end{spellnotes}

\spellsection{Insanity}
\spellschool{Enchantment (Compulsion) [Mind-Affecting]}
\spelllvl{Chaos 6, Ench 6}
\spellrng{Touch}
\spelltgt{Living creature touched}
\spelldur{Permanent}
\spellsave{Will negates}
\spellsr{Yes (Will)}
\begin{spellhealthy}
  The creature is bewildered, making it vulnerable.
\end{spellhealthy}
\begin{spellblood}
  \par The affected creature is confused (see the \spell{confusion} spell).
\end{spellblood}
\begin{spellnotes}
  A vulnerable creature takes a \minus2 penalty to attack rolls, saving throws, checks, DCs, and AC. \spell{Remove curse} and \spell{dispel magic} do not remove \spell{insanity}. \spell{Greater restoration}, \spell{heal}, \spell{limited wish}, \spell{miracle}, or \spell{wish} can restore the creature.
\end{spellnotes}

\spellsection{Interposing Hand}
\spelldesc{You create a floating, disembodied hand made of magical force that shields you from your foe's blows.}
\spellschool{Evocation (Control) [Force]}
\spelllvl{Evoc 2}
\spellrng{\rngmed}
\spelleff{Large hand made of force}
\spelldur{\durshort (D)}
\spellsave{None}
\spellsr{Yes (Fortitude)}
\begin{spelleffect}
  The hand created by this spell stays between you and one opponent, providing you with cover (\plus4 AC) from that creature. In addition, if the creature is Large size or smaller, it moves at half speed while moving towards you. 
  \par If you cannot see the hand's target, it will stop moving until it is directed to a visible target. The hand does not pursue opponents.
  \par An interposing hand is 10 feet long and about that wide with its fingers outstretched. It has half as many hit points as you do when you're undamaged, and its AC is 15 (\minus1 size, \plus6 natural). It takes damage as a normal creature, but most magical effects that don't cause damage do not affect it.
\end{spelleffect}
\begin{spellnotes}
  \par The hand never provokes attacks of opportunity from opponents. It cannot push through a \spell{wall of force} or enter an \spell{antimagic field}, but it suffers the full effect of a \spell{prismatic wall} or \spell{prismatic sphere}. The hand makes saving throws as its caster.
  \spell{Disintegrate} or a successful \spell{dispel magic} destroys the hand without a saving throw. Directing the hand to a new target is a swift action.
\end{spellnotes}

\spellsection{Invest Magic}
\spellschool{Transmutation (Augment)}
\spelllvl{Clr 4, Pal 4, Sor/Wiz 4, War 4}
\spellrng{\rngclose}
\spelltgt{One creature}
\spelldur{\durshort}
\spellsave{Will negates (harmless)}
\spellsr{Yes (Will)}
\begin{spelleffect}
  All weapons and armor that the subject wields gain a \plus3 bonus for as long as she wields them. This bonus increases to \plus4 at 14th level, and to \plus5 at 20th level.
\end{spelleffect}

\spellsection{Invisibility}
\spellschool{Illusion (Glamer)}
\spelllvl{Sor/Wiz 3, Trickery 3}
\spellrng{\rngclose}
\spelltgt{A creature or object weighing no more than 100 lb./level}
\spelldur{\durshort (D)}
\spellsave{Will negates (harmless) or Will negates (harmless, object)}
\spellsr{Yes (Will)}
\begin{spelleffect}
  The creature or object touched becomes invisible, vanishing from sight, even from darkvision. If the recipient is a creature carrying gear, that vanishes, too.
  \par Items dropped or put down by an invisible creature become visible; items picked up disappear if tucked into the clothing or pouches worn by the creature. Light, however, never becomes invisible, although a source of light can become so (thus, the effect is that of a light with no visible source). Any part of an item that the subject carries but that extends more than 5 feet from it becomes visible.
  \par Of course, the subject is not magically silenced, and certain other conditions can render the recipient detectable (such as stepping in a puddle). The spell ends if the subject attacks any creature. For purposes of this spell, an attack includes any spell targeting a foe or whose area or effect includes a foe. (Exactly who is a foe depends on the invisible character's perceptions.) Actions directed at unattended objects do not break the spell. Causing harm indirectly is not an attack. Thus, an invisible being can open doors, talk, eat, climb stairs, summon monsters and have them attack, cut the ropes holding a rope bridge while enemies are on the bridge, remotely trigger traps, open a portcullis to release attack dogs, and so forth. If the subject attacks directly, however, it immediately becomes visible along with all its gear. Spells such as \spell{bless} that specifically affect allies but not foes are not attacks for this purpose, even when they include foes in their area.
\end{spelleffect}
\begin{spellnotes}
  \spell{Invisibility} can be made permanent (on objects only) with a \spell{permanency} ritual.
\end{spellnotes}

\spellsectioncomma{Invisibility}{Greater}
\spellschool{Illusion (Glamer)}
\spelllvl{Illus 6}
\begin{spelleffect}
  This spell functions like \spell{invisibility}, except that the subject becomes invisible again at the start of each of its turns, even if it attacked a creature during its previous turn.
\end{spelleffect}

\spellsectioncomma{Invisibility}{Mass}
\spellschool{Illusion (Glamer)}
\spelllvl{Sor/Wiz 7, Trickery 7}
\spellrng{\rngmed}
\spellarea{\areamed radius limit}
\spelltgts{Five creatures or objects weighing no more than 100 lb./level in the area}
\begin{spelleffect}
  This spell functions like \spell{invisibility}, except that it affects multiple creatures. If the effct is broken for one creature, the other subjects remain invisible.
\end{spelleffect}

\spellsection{Invisibility Purge}
\spellschool{Abjuration (Negation)}
\spelllvl{Clr 2, Sor/Wiz 2}
\spellarea{\arealarge radius emanation, centered on you}
\spelldur{\durlong (D)}
\begin{spelleffect}
  You surround yourself with a mobile sphere of power that suppresses all forms of invisibility. Anything invisible becomes visible while in the area.
\end{spelleffect}

\spellsection{Invisibility Sphere}
\spellschool{Illusion (Glamer)}
\spelllvl{Sor/Wiz 5}
\spellarea{\areasmall radius emanation around the creature or object touched}
\begin{spelleffect}
  This spell functions like \spell{invisibility}, except that this spell confers invisibility upon all creatures within a \areasmall radius emanation of the recipient. The center of the effect is mobile with the recipient.
  \par Those affected by this spell can see each other and themselves as if unaffected by the spell. Any affected creature moving out of the area becomes visible, but creatures moving into the area after the spell is cast do not become invisible. Affected creatures (other than the recipient) who attack negate the invisibility only for themselves. If the spell recipient attacks, the \spell{invisibility sphere} ends.
\end{spelleffect}

\spellsection{Iron Body}
\spellschool{Transmutation (Polymorph)}
\spelllvl{Earth 8, Sor/Wiz 8, Strength 8}
\spellrng{\rngpers}
\spelltgt{You}
\spelldur{\durshort (D)}
\begin{spelleffect}
  This spell transforms your body into living iron, which grants you several powerful resistances and abilities.
  \par You gain physical damage reduction 15/adamantine. You are immune to blindness, critical hits, attribute damage, deafness, disease, drowning, electricity, poison, stunning, and all spells or attacks that affect your physiology or respiration, because you have no physiology or respiration while this spell is in effect. You take only half damage from acid and fire of all kinds.
  \par You gain a \plus5 bonus to your Strength score, but you take a \minus5 penalty to Dexterity as well, and your speed is reduced to half normal. You have a \minus8 armor check penalty. You cannot drink (and thus can't use potions) or play wind instruments.
  \par Your unarmed attacks deal damage equal to a warhammer sized for you (1d6 for Small characters or 1d8 for Medium characters), and you are considered armed when making unarmed attacks.
  \par Your weight increases by a factor of ten, causing you to sink in water like a stone. However, you could survive the crushing pressure and lack of air at the bottom of the ocean -- at least until the spell duration expires.
\end{spelleffect}
\begin{spellnotes}
  This spell's damage reduction allows the subject to ignore the first 15 physical damage it takes each round. If it is hit by an adamantine weapon, it cannot use its damage reduction for 1 round.
\end{spellnotes}

\spellsection{Irresistible Dance}
\spelldesc{You fill your enemy with an overpowering urge to dance and caper in place. Against its will, it begins doing so, complete with foot shuffling and tapping.}
\spellschool{Enchantment (Compulsion) [Mind-Affecting]}
\spelllvl{Ench 9}
\spellrng{\rngclose}
\spelltgt{One creature}
\spelldur{1d4 rounds}
\spellsave{None}
\spellsr{Yes (Will)}
\begin{spelleffect}
  The subject is flat-footed and must spend a standard action each round to do nothing but dance, which provokes attacks of opportunity.
\end{spelleffect}

\pdfbookmark[2]{J-L}{SpellDescriptionsJ-L}
\begin{comment}
\subsubsection{J-L}
\end{comment}

\spellsection{Knock}
\spellschool{Evocation (Control)}
\spelllvl{Evoc 2}
\spellcmp{V}
\spellrng{\rngclose}
\spelltgt{One Medium or smaller object}
\spelldur{Instantaneous; see text}
\spellsave{None}
\spellsr{No}
\begin{spelleffect}
  The knock spell telekinetically opens stuck, barred, locked, held, or arcane locked objects. If the object is stuck or held, you can immediately make an Strength check to break it open, using your caster level instead of your Strength. Others can aid you on this check as normal. In addition, if the object is locked, you can immediately make a Disable Device check to open the lock as if you had rolled a 20 on the check. You get a bonus on the Disable Device check equal to half your caster level.
\end{spelleffect}
\begin{spellnotes}
  If knock is cast on an \spellindirect{arcane lock}{arcane locked} door, make a caster level check against a DC of 11 \add the caster level of the \spell{arcane lock}. If you succeed, the \spell{arcane lock} is suppressed for 10 minutes. If you fail, you may still bypass the door with the checks above, if possible.
\end{spellnotes}

\spellsection{Lesser (Spell Name)}
\par Any spell whose name begins with lesser is alphabetized in this chapter according to the second word of the spell name. Thus, the description of a lesser spell appears near the description of the spell on which it is based. Spell chains that have lesser spells in them include those based on the spells confusion, geas, globe of invulnerability, planar ally, planar binding, and restoration.

\spellsection{Levitate}
\spellschool{Evocation (Control)}
\spelllvl{Evoc 3}
\spellrng{\rngclose}
\spelltgt{You or one willing creature or one object (total weight up to 100 lb./level)}
\spelldur{\durshort (D)}
\spellsave{None}
\spellsr{Yes (Will)}
\begin{spelleffect}
  This spell allows you to telekinetically move the subject up and down as you wish. A creature must be willing to be levitated, and an object must be unattended or possessed by a willing creature. You can mentally direct the recipient to move up or down as much as 20 feet each round; doing so is a swift action. You cannot move the recipient horizontally, but the recipient could clamber along the face of a cliff, for example, or push against a ceiling to move laterally (generally at half its base land speed).
\end{spelleffect}

\spellsection{Lifelink}
\spelldesc{You bind your foe's life force to yours, leaving them vulnerable to your magic.}
\spellschool{Necromancy (Life)}
\spelllvl{Necro 1}
\spellrng{\rngfar}
\spelltgt{One living creature}
\spelldur{\durshort (D)}
\spellsave{Will negates}
\spellsr{Yes (Will)}
\begin{spelleffect}
  The subject is considered to be within \rngclose range of you for determining the range of your spells and spell-like abilities.
\end{spelleffect}

\spellsection{Lifeseeking Missile}
\spellschool{Evocation/Necromancy (Control, Life) [Force]}
\spelllvl{Sor/Wiz 3}
\spellrng{\rngmed}
\spelldmg{3d10 force damage \add d10 per four caster levels above 6th}
\begin{spelleffect}
  This spell functions like \spell{magic missile}, except that the spell creates three missiles that automatically seek out living creatures in the area. Each missile deals 1d10 force damage. If you specify a target for a missile, it will strike the target. Otherwise, it will strike a living creature within the area.
  
  \spell{Invisibility}, \spell{displacement}, and any other forms of cover or concealment do not fool the missiles. You can form one additional missile per four caster levels above 6th.
\end{spelleffect}

\spellsection{Light}
\spellschool{Illusion (Figment) [Light]}
\spelllvl{Clr 1, Drd 1, Pal 1, Sor/Wiz 1}
\spellrng{\rngtouch}
\spelltgt{Object touched}
\spelldur{\durlong (D)}
\spellsave{None}
\spellsr{No}
\begin{spelleffect}
  This spell causes an object to glow like a torch, shedding bright light in a \areamed radius (and dim light for an additional 20 feet) from the point you touch. The effect is immobile, but it can be cast on a movable object.
  
  As a swift action, you can suppress or intensify the light, preventing the object from shedding light or causing it to shed light in up to a \arealarge radius (and dim light for an additional 50 feet). Either effect lasts for 1 round.
\end{spelleffect}
\begin{spellnotes}
  A light spell (one with the light descriptor) counters and dispels a darkness spell (one with the darkness descriptor) of an equal or lower level.\spell{Light} taken into an area of magical darkness does not function.
\end{spellnotes}

\spellsection{Lightning Bolt}
\spellschool{Evocation (Energy) [Electricity]}
\spelllvl{Destruction 3, Drd 3, Sor/Wiz 3}
\spellarea{100 ft. line}
\spelldur{Instantaneous}
\spellsave{Reflex half}
\spellsr{Yes (Reflex)}
\spelldmg{3d6 electricity damage \add d6 per four caster levels above 6th}
\begin{spelleffect}
  You release a powerful stroke of electrical energy that deals damage to each creature within its area. The bolt begins at your fingertips.
  \par The lightning bolt sets fire to combustibles and damages objects in its path. It can melt metals with a low melting point, such as lead, gold, copper, silver, or bronze. If the damage caused to an interposing barrier shatters or breaks through it, the bolt may continue beyond the barrier if the spell's range permits; otherwise, it stops at the barrier just as any other spell effect does.
\end{spelleffect}

\spellsection{Limited Wish}
\spellschool{Universal}
\spelllvl{Sor/Wiz 7}
\spellcmp{V, S, M}
\spellrng{See text}
\spelltgteffarea{See text}
\spelldur{See text}
\spellsave{None; see text}
\spellsr{Yes (Will)}
\begin{spelleffect}
  A limited wish lets you create nearly any type of effect. For example, a limited wish can do any of the following things.
  \begin{itemize*}
    \item Duplicate any general sorcerer/wizard spell of 6th level or lower, provided the spell is not of a school prohibited to you.
    \item Duplicate any general sorcerer/wizard spell of 5th level or lower, even if it's of a prohibited school.
    \item Duplicate any other spell of 4th level or lower, provided the spell is not of a school prohibited to you.
    \item Duplicate any other spell of 3rd level or lower, even if it's of a prohibited school.
    \item Undo the harmful effects of many spells, such as geas/quest or insanity.
    \item Produce any other effect whose power level is in line with the above effects, such as a single creature automatically hitting on its next attack or taking a \minus5 penalty on its next saving throw.
  \end{itemize*}
  \par When casting a limited wish, you do not specify the exact spell or effect you wish to duplicate. Instead, you make a wish, describing what you want to have happen, and make a DC 15 Wisdom check. If the check fails, your intent is redirected or perverted in some way. For example, a \spell{limited wish} to turn a foe to stone would normally mimic the \spell{flesh to stone} effect of the \spell{transmute flesh and stone} spell. However, if the Wisdom check failed, your foe might gain the benefit of a \spell{stoneskin} spell instead.
  \par A duplicated spell allows saving throws and spell resistance as normal for the spell. When a limited wish spell duplicates a spell with a material component that costs more than 1,000 gp, you must provide that component (in addition to the 1,500 gp cost for this spell).
\end{spelleffect}
\spellmat{A diamond worth no less than 1,500 gp (see above).}

\spellsection{Link Vitality}
\spellschool{Necromancy (Life)}
\spelllvl{Necro 3}
\spellarea{\areamed radius limit centered on you}
\spelltgts{Any two living creatures within the area}
\spelldur{\durshort}
\spellsave{Will negates}
\spellsr{Yes (Will)}
\begin{spelleffect}
  You temporarily link the fates of any two creatures, if both fail their saving throws. If either linked creature experiences pain, both feel it. When one loses hit points, the other loses the same amount. If one takes nonlethal damage, so does the other. Likewise, when one regains hit points, the other heals the same amount. Excess healing is simply lost. If one creature is subjected to an effect to which it is immune (such as a type of energy damage), the linked creature is not subjected to it.
\end{spelleffect}
\begin{spellnotes}
  No other effects are transferred by \spell{link vitality}.
\end{spellnotes}

\spellsectioncomma{Link Vitality}{Mass}
\spellschool{Necromancy (Life)}
\spelllvl{Sor/Wiz 7}
\spelltgts{Five living creatures within the area}
\begin{spelleffect}
  This spell functions as \spell{link vitality}, except that it affects many creatures. The spell links all creatures who fail their saving throws. If any of the linked creatures lose or gain hit points, all linked creatures lose or gain the same amount, and so on.
\end{spelleffect}

\spellsection{Locate Entity}
\spellschool{Divination (Awareness) [Detection]}
\spelllvl{Knowledge 6, Sor/Wiz 6}
\spellrng{\rngext}
\spelldur{\durlong (D)}
\begin{spelleffect}
  This spell functions as \spell{locate object}, except that it can also detect creatures, as \spell{locate creature}. When you cast this spell, you choose to locate an object or creature, following the restrictions stated in the respective location spells.
\end{spelleffect}

\spellsection{Locate Creature}
\spellschool{Divination (Awareness) [Detection]}
\spelllvl{Knowledge 4, Sor/Wiz 4}
\spelldur{\durlong (D)}
\begin{spelleffect}
  This spell functions like \spell{locate object}, except this spell locates a known or familiar creature.
  \par You slowly turn and sense when you are facing in the direction of the creature to be located, provided it is within range. You also know in which direction the creature is moving, if any.
  \par The spell can locate a creature of a specific kind or a specific creature known to you. It cannot find a creature of a certain type. To find a kind of creature, you must have seen such a creature up close (within 30 feet) at least once.
\end{spelleffect}
\begin{spellnotes}
  Detection spells spell are blocked by 1 foot of stone, 1 inch of common metal, a thin sheet of lead, or 3 feet of wood or dirt. In addition, running water blocks \spell{locate creature}. It cannot detect objects. It can be fooled by \spell{mislead} and \spell{nondetection} spells.
\end{spellnotes}

\spellsection{Locate Object}
\spellschool{Divination (Awareness) [Detection]}
\spelllvl{Clr 2, Knowledge 2, Sor/Wiz 2}
\spellrng{\rngfar}
\spelldur{\durmed (D)}
\spellsave{None}
\spellsr{No}
\begin{spelleffect}
  You sense the direction of a well-known or clearly visualized object. You can search for general items, in which case you locate the nearest one of its kind if more than one is within range. Attempting to find a certain item requires a specific and accurate mental image; if the image is not close enough to the actual object, the spell fails. You cannot specify a unique item unless you have observed that particular item firsthand (not through divination).
\end{spelleffect}
\begin{spellnotes}
  The spell is blocked by even a thin sheet of lead, but not by other materials. Creatures cannot be found by this spell.
\end{spellnotes}

\spellsection{Longstrider}
\spellschool{Transmutation (Augment)}
\spelllvl{Drd 1, Travel 1}
\spellrng{\rngpers}
\spelltgt{You}
\spelldur{\durlong (D)}
\begin{spelleffect}
  This spell increases your base land speed by 10 feet. (This adjustment counts as an enhancement bonus.) It has no effect on other modes of movement, such as burrow, climb, fly, or swim.
\end{spelleffect}

\pdfbookmark[2]{M}{SpellDescriptionsM}
\begin{comment}
\subsubsection{M}
\end{comment}

\spellsection{Mage Armor}
\spelldesc{You create an invisible but tangible field of force that surrounds you, protecting you from attacks.}
\spellschool{Abjuration (Shielding) [Force]}
\spelllvl{Sor/Wiz 1}
\spellrng{Personal}
\spelltgt{You}
\spelldur{\durlong (D)}
\begin{spelleffect}
  You gain a \plus2 armor modifier to AC. \bonusscalingdescription
  \par Unlike mundane armor, \spell{mage armor} entails no armor check penalty, arcane spell failure chance, or speed reduction.
\end{spelleffect}
\begin{spellnotes}
  This armor is treated as a separate piece or armor from any other armor the creature is wearing, so it does not stack with any existing armor modifier. Since \spell{mage armor} is made of force, incorporeal creatures can't bypass it the way they do normal armor.
  
  If you become subject to the \spell{shield} spell during the duration of this spell, the \spell{shield} spell lasts until this spell's duration ends.
\end{spellnotes}

\spellsection{Mage Hand}
\spellschool{Evocation (Control)}
\spelllvl{Sor/Wiz 1}
\spellrng{\rngclose}
\spelltgt{One nonmagical, unattended object weighing up to 5 lb.}
\spelldur{\durshort}
\spellsave{None}
\spellsr{No}
\begin{spelleffect}
  You point your finger at an object and can lift it and move it in any direction from a distance. By directing the spell as a swift action, you can propel the object as far as 15 feet in any direction each round, though the spell ends if the distance between you and the object ever exceeds the spell's range.
\end{spelleffect}
\begin{spellnotes}
  Fine manipulation, including any motion other than simply moving the object in a particular direction, is not possible with this spell.
\end{spellnotes}

\spellsection{Mage's Disjunction}
\spellschool{Abjuration (Negation) [Magic]}
\spelllvl{Magic 9, Abjur 9}
\spellrng{\rngmed}
\spelltgtorarea{One magic item or \areamed radius burst}
\spelldur{Instantaneous}
\spellsave{Will negates (object)}
\spellsr{No}
\begin{spelleffect}
  All magical effects within the radius of the spell, except for those on you, are disjoined. That is, spells and spell-like effects are separated into their individual components (ending the effect as a \spell{dispel magic} spell does).
  \par You also have a 2\% chance per caster level of destroying an \spell{antimagic field}.
  \par You can also use this spell to target a single item. The item gets a Will save at a \minus5 penalty to avoid being permanently rendered nonmagical. Even artifacts are subject to this use of disjunction, though there is only a 1\% chance per caster level of actually affecting such powerful items. Additionally, if an artifact is destroyed, you permanently lose the ability to cast \spell{mage's disjunction}. (This ability cannot be recovered by mortal magic, not even \spell{miracle} or \spell{wish}.)
  \par \subspell{Note} Destroying artifacts is a dangerous business, and it is 95\% likely to attract the attention of some powerful being who has an interest in or connection with the device.
\end{spelleffect}

%need to run expected damage analysis
%basic math: attack is like save half, so ``should'' deal 49 damage
%assume 32 AC, 8 attr
%assume 4 rounds of attacking at 50% chance to hit = 36 damage.
%benefits strongly from +CL.
%At 20 CL: 85 damage vs 64 damage
\spellsection{Mage's Sword}
\spellschool{Evocation (Control) [Force]}
\spelllvl{Evoc 7}
\spellrng{\rngmed}
\spelleff{One sword}
\spelldur{\durshort (D)}
\spellsave{None}
\spellsr{Yes (Will)}
\begin{spelleffect}
  This spell brings into being a shimmering, swordlike plane of force. The sword strikes at any opponent within its range, as you desire, starting in the round that you cast the spell. The sword attacks its designated target once each round on your turn. Its attack bonus is equal to your caster level \add your casting attribute. It deals 4d6 points of force damage \add half your casting attribute \add 1d6 per four caster levels above 14th, with a threat range of 19--20 and a critical multiplier of \mtimes2.
  \par The sword always strikes from your direction. It does not contribute to overwhelm penalties, but it benefits from any that exist. If the sword goes beyond the spell range from you, if it goes out of your sight, or if you are not directing it, the sword returns to you and hovers.
\end{spelleffect}
\begin{spellnotes}
  Each round after the first, you can redirect the sword to a new target as a swift action. If you do not, the sword continues to attack the previous round's target.
  \par As a force effect, the sword can strike ethereal and incorporeal creatures. It cannot be attacked or harmed by physical attacks, but \spell{dispel magic}, \spell{disintegrate}, a sphere of annihilation, or a rod of cancellation affects it. The sword's AC is 10 (10, \plus0 size bonus for being a Medium object)
  \par If an attacked creature has spell resistance, the resistance is checked the first time \spell{mage's sword} strikes it. If the sword is successfully resisted, the spell is dispelled. If not, the sword has its full effect on that creature for the duration of the spell.
\end{spellnotes}

\spellsection{Magic Circle against Chaos}
\spellschool{Abjuration (Interdiction) [Barrier, Lawful]}
\spelllvl{Clr 5, Chaos 5, Pal 4, Sor/Wiz 5}
\begin{spelleffect}
  This spell functions like \spell{magic circle against evil}, except that it is similar to \spell{protection from chaos} instead of \spell{protection from evil}, and it hedges out nonlawful summoned creatures.
\end{spelleffect}

\spellsection{Magic Circle against Evil}
\par Abjuration (Interdiction) [Barrier, Good]
\spelllvl{Clr 5, Good 5, Pal 4, Sor/Wiz 5}
\spellrng{Touch}
\spellarea{\areasmall emanation from touched creature}
\spelldur{\durshort (D)}
\spellsave{Will negates (harmless)}
\spellsr{Yes (Will)}
\begin{spelleffect}
  All creatures within the area gain the effects of a \spell{protection from evil} spell. In addition, no nongood summoned creatures can enter the area unless they make a successful Will save.
\end{spelleffect}

\spellsection{Magic Circle against Good}
\spellschool{Abjuration (Interdiction) [Barrier, Evil]}
\spelllvl{Clr 5, Evil 5 Sor/Wiz 5}
\begin{spelleffect}
  This spell functions like \spell{magic circle against evil}, except that it is similar to \spell{protection from good} instead of \spell{protection from evil}, and it hedges out nonevil summoned creatures.
\end{spelleffect}

\spellsection{Magic Circle against Law}
\spellschool{Abjuration (Interdiction) [Barrier, Chaotic]}
\spelllvl{Clr 5, Chaos 5, Sor/Wiz 5}
\begin{spelleffect}
  This spell functions like \spell{magic circle against evil}, except that it is similar to \spell{protection from law} instead of \spell{protection from evil}, and it hedges out nonchaotic summoned creatures.
\end{spelleffect}

\spellsection{Magic Fang}
\spellschool{Transmutation (Augment)}
\spelllvl{Drd 2}
\spellrng{\rngclose}
\spelltgt{One creature}
\spelldur{\durshort}
\spellsave{Will negates (harmless)}
\spellsr{Yes (Will)}
\begin{spelleffect}
  This spell makes one of the subject's natural weapons a \plus2 magic weapon, granting a \plus2 bonus to attack and damage rolls. \bonusscalingdescription
\end{spelleffect}
\begin{spellnotes}
  The spell does not change an unarmed strike's damage from nonlethal damage to lethal damage. \spell{Magic fang} can be made permanent with a \spell{permanency} spell.
\end{spellnotes}

\spellsectioncomma{Magic Fang}{Greater}
\spellschool{Transmutation (Augment)}
\spelllvl{Drd 4}
\begin{spelleffect}
  This spell functions like \spell{magic fang}, except that it affects one of the creature's natural weapons per four caster levels.
\end{spelleffect}
\begin{spellnotes}
  \spell{Greater magic fang} can be made permanent with a permanency spell.
\end{spellnotes}

\spellsection{Magic Missile}
\spellschool{Evocation (Control) [Force]}
\spelllvl{Sor/Wiz 1}
\spellrng{\rngclose}
\spellarea{\areamed radius limit}
\spelltgts{Creatures in the area}
\spelldur{Instantaneous}
\spellsave{None}
\spellsr{Yes (Fortitude)}
\spelldmg{2d4 force damage \add d4 per two levels above 2nd; see text}
\begin{spelleffect}
  Two missiles of magical energy dart forth from your fingertip and strike creatures you designate in the area, dealing 1d4 damage each. A single missile can strike only one creature. For every two caster levels above 2nd, you gain an additional missile.
  The missiles strike unerringly, even if the target has cover or concealment. Specific parts of a creature can't be singled out. Inanimate objects are not damaged by the spell. You must designate targets before you check for spell resistance or roll damage.
\end{spelleffect}

\spellsection{Magic Vestment}
\spellschool{Transmutation (Augment)}
\spelllvl{Clr 1, Sor/Wiz 1}
\spellrng{\rngclose}
\spelltgt{One suit of armor or shield}
\spelldur{\durmed}
\spellsave{Will negates (harmless, object)}
\spellsr{Yes (Will)}
\begin{spelleffect}
  You imbue body armor or a shield with a \plus2 enhancement bonus, giving its bearer a \plus2 bonus to AC. \bonusscalingdescription
\end{spelleffect}
\begin{spellnotes}
  An outfit of regular clothing counts as armor that grants no AC bonus for the purpose of this spell.
\end{spellnotes}

\spellsection{Magic Weapon}
\spellschool{Transmutation (Augment)}
\spelllvl{Clr 2, Sor/Wiz 2}
\spellrng{\rngclose}
\spelltgt{One weapon or fifty projectiles (all of which must be in contact with each other at the time of casting)}
\spelldur{\durshort}
\spellsave{Will negates (harmless, object)}
\spellsr{Yes (Will)}
\begin{spelleffect}
  You imbue a weapon or stack of projectiles with a \plus2 enhancement bonus, giving its wielder a \plus2 bonus to attack and damage. \bonusscalingdescription
\end{spelleffect}
\begin{spellnotes}
  You can't cast this spell on a natural weapon, such as an unarmed strike (instead, see \spell{magic fang}). A monk's unarmed strike is considered a weapon, and thus it can be enhanced by this spell.
  \par If you use this spell to enhance projectiles, the projectiles must be of the same kind, and they have to be together (in the same quiver or other container). Projectiles, but not thrown weapons, lose their transmutation when used. (Treat darts and shuriken as projectiles, rather than as thrown weapons, for the purpose of this spell.)
\end{spellnotes}

\spellsection{Major Image}
\spellschool{Illusion (Figment) [Unreal]}
\spelllvl{Illus 4}
\spellrng{\rngfar}
\spelldur{\durshort}
\begin{spelleffect}
  This spell functions like \spell{silent image}, except that sound, smell, and thermal illusions are included in the spell effect. By concentrating on the spell, you can move the image within the range.
\end{spelleffect}
\begin{spellnotes}
  The image disappears when struck by an opponent unless you cause the illusion to react appropriately. Even then, the opponent who struck the image gets a Will save to disbelieve the illusion for interacting with the image.
\end{spellnotes}

\spellsection{Mass (Spell Name)}
\par Any spell whose name begins with mass is alphabetized in this chapter according to the second word of the spell name. Thus, the description of a mass spell appears near the description of the spell on which it is based. Spell chains that have mass spells in them include those based on the spells charm monster, cure critical wounds, cure light wounds, cure moderate wounds, cure serious wounds, enlarge person, heal, hold monster, hold person, inflict critical wounds, inflict light wounds, inflict moderate wounds, inflict serious wounds, invisibility, reduce person, suggestion, totemic mind, and totemic power.

\spellsection{Maze}
\spellschool{Conjuration (Translocation) [Planar]}
\spelllvl{Conj 8, Trickery 9}
\spellrng{\rngclose}
\spelltgt{One creature}
\spelldur{Instantaneous; see text}
\spellsave{Will partial}
\spellsr{Yes (Will)}
\begin{spelleffect}
  You banish the subject into an extradimensional labyrinth of force planes. Each round on its turn, it may attempt a DC 20 Intelligence check to escape the labyrinth as a full-round action. If the subject doesn't escape, the maze disappears after 5 minutes, forcing the subject back to the location where it was originally banished. A successful Will save prevents you from placing it in the middle of the labyrinth, lowering the DC of the Intelligence check to 15.
  \par On escaping or leaving the maze, the subject reappears where it had been when the maze spell was cast. If this location is filled with a solid object, the subject appears in the nearest open space.
\end{spelleffect}
\begin{spellnotes}
  Spells and abilities that move a creature within a plane, such as \spell{teleport} and \spell{dimension door}, do not help a creature escape a \spell{maze} spell, although a \spell{plane shift} spell allows it to exit to whatever plane is designated in that spell. Minotaurs can escape the spell automatically.
\end{spellnotes}

\spellsection{Mental Retribution}
\spellschool{Abjuration/Enchantment (Inhibition, Shielding) [Mind-Affecting]}
\spelllvl{Sor/Wiz 1}
\spellrng{\rngclose}
\spelltgt{One creature; see text}
\spellarea{\rngmed radius limit centered on the subject; see text}
\spelldur{\durshort or until discharged/5 rounds}
\spellsave{Will negates (harmless)}
\spellsr{Yes (Will)}
\begin{spelleffect}
  The subject gains a faintly shimmering aura. The first time it is attacked by a creature within the area, the spell is discharged, and the attacking creature is bewildered for 5 rounds. A successful Will save can prevent the subject from gaining the aura, but there is no saving throw against the bewildering effect.
\end{spelleffect}

\spellsection{Meld into Stone}
\spellschool{Transmutation (Polymorph) [Earth]}
\spelllvl{Drd 3, Earth 3}
\spellrng{Personal}
\spelltgt{You}
\spelldur{\durlong}
\begin{spelleffect}
  This spell enables you to meld your body and possessions into a single block of stone. The stone must be large enough to accommodate your body in all three dimensions. When the casting is complete, you and not more than 100 pounds of nonliving gear merge with the stone. If either condition is violated, the spell fails and is wasted.
  \par While in the stone, you remain in contact, however tenuous, with the face of the stone through which you melded. You remain aware of the passage of time and can cast spells on yourself while hiding in the stone. Nothing that goes on outside the stone can be seen, but you can still hear what happens around you. Minor physical damage to the stone does not harm you, but its partial destruction (to the extent that you no longer fit within it) expels you and deals you 5d6 points of damage. The stone's complete destruction expels you and slays you instantly unless you make a DC 20 Fortitude save.
  \par Any time before the duration expires, you can step out of the stone through the surface that you entered. If the spell's duration expires or the effect is dispelled before you voluntarily exit the stone, you are violently expelled and take 5d6 points of damage.
\end{spelleffect}
\begin{spellnotes}
  The following spells harm you if cast upon the stone that you are occupying: \spell{transmute flesh and stone} expels you and deals 6d6 points of damage. \spellindirect{shape stone}{Shape stone} deals 3d6 points of damage but does not expel you. \spell{Passwall} expels you without damage.
\end{spellnotes}

\spellsection{Message}
\spellschool{Divination (Communication)}
\spelllvl{Sor/Wiz 1}
\spellcmp{S}
\spellrng{\rngmed}
\spellarea{\areamed radius limit}
\spelltgts{Five creatures within the area}
\spelldur{\durlong}
\spellsave{None}
\spellsr{No}
\begin{spelleffect}
  You can whisper messages and receive whispered replies with little chance of being overheard. You point your finger at each creature you want to receive the message. When you whisper, the whispered message is audible to all targeted creatures within range. Magical silence, 1 foot of stone, 1 inch of common metal (or a thin sheet of lead), or 3 feet of wood or dirt blocks the spell. The message does not have to travel in a straight line. It can circumvent a barrier if there is an open path between you and the subject, and the path's entire length lies within the spell's range. The creatures that receive the message can whisper a reply that you hear. The spell transmits sound, not meaning. It doesn't transcend language barriers.
\end{spelleffect}
\begin{spellnotes}
  To speak a message, you must mouth the words and whisper, possibly allowing observers the opportunity to read your lips.
\end{spellnotes}

\spellsection{Meteor Swarm}
\spelldesc{You call a swarm of meteors that streak down from the heavens, leaving a fiery trail behind them. The meteors crash into your foes, driving flying creatures to the ground and knocking your foes off their feet.}
\spellschool{Evocation (Energy) [Fire]}
\spelllvl{Destruction 9, Fire 9, Evoc 9}
\spellrng{\rngfar}
\spellarea{A \arealarge radius cylinder, 100 ft. high}
\spelldur{Instantaneous}
\spellsave{Reflex half/Reflex negates}
\spellsr{Yes (Reflex)}
\spelldmg{9d6 fire damage \add d6 per four caster levels after 18th}
\begin{spelleffect}
  Every creature and object in the area takes damage. Flying creatures within the area of size Huge or smaller that fail their Reflex saves are driven to the ground, taking falling damage appropriate to the distance they descended. Creatures on the ground that fail their Reflex saves are knocked prone.
\end{spelleffect}
\begin{spellnotes}
  This spell functions indoors or underground, but not underwater.
\end{spellnotes}

\spellsection{Mind Fog}
\spelldesc{You conjure a fog bank that hampers the mental acuity of those caught in it.}
\spellschool{Conjuration/Enchantment (Creation, Inhibition) [Fog, Mind-Affecting]}
\spelllvl{Sor/Wiz 5, Trickery 5}
\spellrng{\rngclose}
\spelldur{\durlong and 5 rounds; see text}
\spellsave{None/Will negates}
\spellsr{None/Yes (Will)}
\begin{spelleffect}
  This spell functions like \spell{fog cloud}, except each creature in the fog take a \minus5 penalty to Wisdom unless it makes a Will save. A creature that successfully saves against the fog is not affected, but if it remains in the fog, it must make a new save each minute to avoid being affected. Affected creatures take the penalty as long as they remain in the fog and for 5 rounds thereafter. The fog is stationary and lasts for 1 hour (or until dispersed by wind).
\end{spelleffect}
\begin{spellnotes}
  A moderate wind (11\add mph) disperses the fog in 5 rounds; a strong wind (21\add mph) disperses the fog in 1 round.
\end{spellnotes}

\spellsection{Minor Image}
\spellschool{Illusion (Figment) [Unreal]}
\spelllvl{Illus 3}
\spellrng{\rngmed}
\spelldur{\durshort}
\begin{spelleffect}
  This spell functions like \spell{silent image}, except that it includes some minor sounds but not understandable speech.
\end{spelleffect}
\begin{spellnotes}
  The image disappears when struck by an opponent unless you cause the illusion to react appropriately. Even then, the opponent who struck the image gets a Will save to disbelieve the illusion for interacting with the image.
\end{spellnotes}

\spellsection{Miracle}
\spellschool{Evocation (Channeling)}
\spelllvl{Clr 9}
\spellrng{See text}
\spelltgteffarea{See text}
\spelldur{See text}
\spellsave{See text}
\spellsr{Yes (varies; see text)}
\begin{spelleffect}
  You don't so much cast a miracle as request one. You state what you would like to have happen and request that your deity (or the power you pray to for spells) intercede.
  \par A miracle can do any of the following things.
  \begin{itemize*}
    \item Duplicate any cleric spell of 8th level or lower (including spells to which you have access because of your domains). 
    \item Duplicate any other spell of 7th level or lower.
    \item Undo the harmful effects of certain spells, such as feeblemind or insanity.
    \item Have any effect whose power level is in line with the above effects.
  \end{itemize*}
  \par Alternatively, a cleric can make a very powerful request. Examples of especially powerful miracles of this sort could include the following.
  \begin{itemize*}
    \item Swiinging the tide of a battle in your favor by raising fallen allies to continue fighting.
    \item Moving you and your allies, with all your and their gear, from one plane to another through planar barriers to a specific locale with no chance of error.
    \item Protecting a city from an earthquake, volcanic eruption, flood, or other major natural disaster.
  \end{itemize*}
  \par In any event, a request that is out of line with the deity's (or alignment's) nature is refused.
\end{spelleffect}
\begin{spellnotes}
  If you request a miracle, your deity (or the power you pray to) will expect something of you in return. You must cast commune to learn what this is within 24 hours, or you will lose the ability to cast any cleric spells other than commune. For more moderate miracles, you may be required to offer 25,000gp worth of incense and gems. For especially powerful miracles, or multiple moderate miracles, you may geased with a task to complete.
  \par A duplicated spell allows saving throws and spell resistance as normal, but the save DCs are as for a 9th-level spell. When a miracle spell duplicates a spell with a material component that costs more than 10,000 gp, you must provide that component.
\end{spellnotes}

\spellsection{Mirror Image}
\spelldesc{You create illusory duplicates of yourself that make it difficult for enemies to know which image to attack.}
\spellschool{Illusion (Figment)}
\spelllvl{Illus 2}
\spellrng{Personal; see text}
\spelltgt{You}
\spelldur{\durshort (D)}
\begin{spelleffect}
  This spell creates an illusory duplicate of yourself that mimics your movements perfectly. Enemies attempting to attack you or cast spells at you must select which to attack. Generally, roll randomly to see whether the selected target is real or a figment. An image's AC is 10 \add your size modifier. You gain an additional image at 8th, 14th, and 20th level. 
  \par If an image is hit, it is destroyed. If you are hit, your attacker knows the attack was successful, and can ignore the image. You can create new images to replace destroyed images as a swift action, preventing your foes from knowing which image to attack.
  \par You can move into and through your duplicates on your turn. When you and the image separate, observers can't use vision or hearing to tell which one is you and which the image. The duplicates may also move through each other. The figments mimic your actions, pretending to cast spells when you cast a spell, drink potions when you drink a potion, levitate when you levitate, and so on.
  \par Mirror images can be attacked like any other creature. They count as separate creatures, and can be targeted separately by spells like \spell{magic missile} or feats like Whirlwind Attack, though they are not destroyed by area spells. Destroying an image counts as dropping a creature for the purpose of the Cleave feat and similar abilities.
\end{spelleffect}
\begin{spellnotes}
  An attacker must be able to see the images to be fooled. If you are invisible or an attacker shuts his or her eyes, the spell has no effect. (Being unable to see carries the same penalties as being blinded.)
\end{spellnotes}

\spellsection{Mislead}
\spellschool{Illusion (Figment, Glamer) [Unreal]}
\spelllvl{Sor/Wiz 6, Trickery 6}
\spellrng{Personal/\rngmed}
\spelltgt{You}
\spelleff{One illusory double}
\spelldur{\durshort (D); see text}
\spellsave{None/Will disbelief (if interacted with); see text}
\spellsr{No}
\begin{spelleffect}
  You become invisible (as \spell{invisibility}, a glamer), and at the same time, an illusory double of you (as \spell{major image}, an unreal figment) appears. You are then free to go elsewhere while your double moves away. The double appears within range but thereafter moves as you direct it (which requires concentration beginning on the first round after the casting). You can make the figment appear superimposed perfectly over your own body so that observers don't notice an image appearing and you turning invisible. You and the figment can then move in different directions. The double moves at your speed and can talk and gesture as if it were real, but it cannot attack or cast spells, though it can pretend to do so.
  \par The illusory double lasts as long as you concentrate upon it, plus 5 additional rounds. After you cease concentration, the illusory double continues to carry out the same activity until the duration expires. The invisibility lasts for 5 minutes, regardless of concentration.
\end{spelleffect}

\spellsection{Missile Storm}
\spelldesc{You unleash an immense swarm of missiles which seek out and destroy all of your foes.}
\spellschool{Evocation (Control) [Force]}
\spelllvl{Sor/Wiz 7}
\spellarea{\arealarge radius limit centered on you}
\spelltgts{Any number of creatures in the area}
\spelldur{Instantaneous}
\spellsave{None}
\spellsr{Yes (Fortitude)}
\spelldmg{7d4 force damage \add d4 per four levels above 14th}
\begin{spelleffect}
  Each target is struck by seven missiles like those created by the \spell{magic missile} spell. Each missile deals 1d4 damage. You can create one additional missile to strike each target per four levels above 14th.
\end{spelleffect}

\spellsection{Modify Memory}
\spellschool{Enchantment [Mind-Affecting]}
\spelllvl{Brd 4}
\spelltime{Full-round action; see text}
\spellrng{\rngclose}
\spelltgt{One living creature}
\spelldur{Permanent}
\spellsave{Will negates}
\spellsr{Yes (Will)}
\begin{spelleffect}
  You reach into the subject's mind and modify as many as 5 minutes of its memories in one of the following ways.
  \begin{itemize*}
    \item Eliminate all memory of an event the subject actually experienced. This spell cannot negate charm, geas/quest, suggestion, or similar spells.
    \item Allow the subject to recall with perfect clarity an event it actually experienced.
    \item Change the details of an event the subject actually experienced.
    \item Implant a memory of an event the subject never experienced.
  \end{itemize*}
  \par Casting the spell takes 1 round. If the subject fails to save, you proceed with the spell by spending as much as 5 minutes (a period of time equal to the amount of memory time you want to modify) visualizing the memory you wish to modify in the subject. If your concentration is disturbed before the visualization is complete, or if the subject is ever beyond the spell's range during this time, the spell is lost.
  \par A modified memory does not necessarily affect the subject's actions, particularly if it contradicts the creature's natural inclinations. An illogical modified memory is dismissed by the creature as a bad dream or a memory muddied by too much wine.
\end{spelleffect}

\spellsection{Moment of Prescience}
\spellschool{Divination (Knowledge)}
\spelllvl{Div 6, Knowledge 7, Sor/Wiz 7}
\begin{spelleffect}
  This spell functions like \spell{lesser moment of prescience}, except that you also gain a circumstance bonus equal to half your caster level on the roll. Alternately, you can expend the spell to protect yourself. If you do, you gain a circumstance bonus to your dodge modifier equal to half your caster level, and you stop being flat-footed if you were. This effect can be used even if you are flat-footed, which would normally prevent you from using immediate actions.
\end{spelleffect}
\begin{spellnotes}
  You can't have more than one \spellindirect{lesser moment of prescience}{moment of prescience} effect active on you at the same time.
\end{spellnotes}

\spellsectioncomma{Moment of Prescience}{Greater}
\spellschool{Divination (Knowledge)}
\spelllvl{Div 9}
\begin{spelleffect}
  This spell functions like \spell{moment of prescience}, except that the bonus and extra rolls apply to all attack rolls, opposed checks, and saving throws you make until the beginning of your next turn.
\end{spelleffect}
\begin{spellnotes}
  You can't have more than one \spellindirect{lesser moment of prescience}{moment of prescience} effect active on you at the same time.
\end{spellnotes}

\spellsectioncomma{Moment of Prescience}{Lesser}
\spellschool{Divination (Knowledge)}
\spelllvl{Div 3, Knowledge 4, Sor/Wiz 4}
\spellrng{Personal}
\spelltgt{You}
\spelldur{\durext or until discharged}
\begin{spelleffect}
  This spell grants you a powerful sixth sense in relation to yourself. Once during the spell's duration, you may choose to use its effect. You may roll twice on any single attack roll, opposed check, or saving throw. Activating the effect takes an immediate action, so you can even activate it on another character's turn if needed. Once activated, the spell ends.
\end{spelleffect}
\begin{spellnotes}
  You can't have more than one \spellindirect{lesser moment of prescience}{moment of prescience} effect active on you at the same time.
\end{spellnotes}

\pdfbookmark[2]{O-P}{SpellDescriptionsO-P}
\begin{comment}
\subsubsection{O-P}
\end{comment}

\spellsection{Oak Body}
\spellschool{Transmutation (Polymorph)}
\spelllvl{Druid 7, Sor/Wiz 7}
\spellrng{Personal}
\spelltgt{You}
\spelldur{\durshort (D)}
\begin{spelleffect}
  This power transforms your body into living oak, which grants you several advantages.
  \par You gain physical damage reduction 15/fire or adamantine and a \plus5 bonus to natural armor. You are immune to ability damage, blindness, deafness, disease, drowning, poison, stunning, and all spells or attacks that affect your physiology or respiration, because you have no physiology or respiration while this power is in effect.
  \par You take only half damage from cold effects of all kinds. However, you become susceptible to all special attacks that affect wood, and you gain vulnerability to fire.
  \par You gain a \plus4 bonus to Strength and Constitution, but you take a \minus2 penalty to Dexterity (to a minimum Dexterity score of \minus9), and your speed is reduced to half normal. You can speak but cannot drink (and thus can't use potions). You have an armor check penalty of \minus4 and an arcane spell failure chance of 25%.
  \par Your unarmed attacks deal damage equal to a club sized for you (1d4 for Small characters, 1d6 for Medium characters), and you are considered armed when making unarmed attacks. When you make a full attack against an object or structure using your unarmed strike, you deal double damage.
\end{spelleffect}
\begin{spellnotes}
  This spell's damage reduction allows the subject to ignore the first 15 physical damage it takes each round. If it is hit by a adamantine weapon or fire attack, it cannot use its damage reduction for 1 round.
\end{spellnotes}

\spellsection{Obscuring Mist}
\spelldesc{You conjure a bank of fog that arises around you, concealing you and your allies.}
\spellschool{Conjuration (Creation) [Fog]}
\spelllvl{Clr 1, Drd 1, Sor/Wiz 1, Water 1}
\spellarea{\areamed radius cylinder-shaped spread centered on you}
\begin{spelleffect}
  This spell functions like \spell{fog cloud}, except that the fog created is centered on you.
\end{spelleffect}

\spellsection{Order's Wrath}
\spellschool{Evocation (Channeling) [Lawful]}
\spelllvl{Law 4}
\spellrng{\rngmed}
\spelltgt{One creature}
\spelldur{Instantaneous/5 rounds}
\spellsave{None/Will half}
\spellsr{Yes (Will)}
\spelldmg{8d6 divine damage \add d6 per two caster levels above 8th}
\begin{spelleffect}
  If the target is not lawful, it takes damage and is bewildered for 5 rounds. A successful Will save halves the damage.
\end{spelleffect}
\begin{spellnotes}
  A vulnerable creature takes a \minus2 penalty to attack rolls, saving throws, checks, DCs, and AC.
\end{spellnotes}

\spellsection{Overland Flight}
\spellschool{Transmutation (Imbuement)}
\spelllvl{Sor/Wiz 6}
\spellrng{Personal}
\spelltgt{You}
\spelldur{\durext}
\begin{spelleffect}
  At any point during the duration of the spell, you can concentrate as a standard action to fly for 1 round, as the \spell{fly} spell. When using this spell for long-distance movement, you can concentrate to fly each round without taking nonlethal damage, but you cannot take a forced march. This means you can cover 60 miles in an ten-hour period of flight (or 40 miles at a speed of 40 feet).
\end{spelleffect}

\spellsection{Passwall}
\spellschool{Transmutation (Alteration)}
\spelllvl{Sor/Wiz 5, Travel 5}
\spellrng{Touch}
\spelleff{5 ft. by 8 ft. opening, 10 ft. deep plus 5 ft. deep per three additional levels}
\spelldur{\durext (D)}
\spellsave{None}
\spellsr{No}
\begin{spelleffect}
  You create a passage through wooden, plaster, or stone walls, but not through metal or other harder materials. The passage is 10 feet deep plus an additional 5 feet deep per three caster levels above 9th (15 feet at 12th, 20 feet at 15th, and a maximum of 25 feet deep at 18th level). If the wall's thickness is more than the depth of the passage created, then a single \spell{passwall} simply makes a niche or short tunnel. Several \spell{passwall} spells can then form a continuing passage to breach very thick walls. When \spell{passwall} ends, creatures within the passage are ejected out the nearest exit.
\end{spelleffect}
\begin{spellnotes}
  If someone dispels the \spell{passwall} or you dismiss it, creatures in the passage are ejected out the far exit, if there is one, or out the sole exit if there is only one.
\end{spellnotes}

\spellsection{Persistent Image}
\spellschool{Illusion (Figment)}
\spelllvl{Illus 6}
\spellrng{\rngfar}
\spelldur{\durmed (D)}
\begin{spelleffect}
  This spell functions like \spell{silent image}, except that the figment includes visual, auditory, olfactory, and thermal components, and the figment follows a script determined by you. The figment follows that script without your having to concentrate on it. The illusion can include intelligible speech if you wish.
\end{spelleffect}

\spellsection{Phantasmal Killer}
\spelldesc{You create a phantasmal image of the most fearsome creature imaginable to the subject simply by forming the fears of the subject's subconscious mind into something that its conscious mind can visualize: this most horrible beast.}
\spellschool{Enchantment/Illusion (Emotion, Phantasm) [Death, Fear, Mind-Affecting, Unreal]}
\spelllvl{Sor/Wiz 4, Trickery 4}
\spellrng{\rngmed}
\spelltgt{One creature}
\spelldur{Instantaneous}
\spellsave{Will disbelief and Fortitude negates; see text}
\spellsr{Yes (Will)}
\begin{spelleffect}
  The subject is shaken, causing it to be vulnerable for 5 rounds.
\end{spelleffect}
\begin{spellblood}
  The subject must also make a Fortitude save. If it fails, it immediately dies.
\end{spellblood}
\begin{spellnotes}
  A vulnerable creature takes a \minus2 penalty to attack rolls, saving throws, checks, DCs, and AC.
\end{spellnotes}

\spellsection{Phantom Maze}
\spelldesc{You manipulate the subject's perceptions, causing it to believe that it is trapped in a labyrinth.}
\spellschool{Illusion (Phantasm) [Unreal]}
\spelllvl{Sor/Wiz 5, Trickery 5}
\spellrng{\rngclose}
\spelltgt{One creature}
\spelldur{\durmed}
\spellsave{Will disbelief}
\spellsr{Yes (Will)}
\begin{spelleffect}
  The subject perceives itself to be banished to an extradimensional labyrinth of force planes, as the \spell{maze} spell. It cannot see or hear anything to the contrary, causing it to be treated as if blinded and deafened for most purposes. Typically, this means the subject moves in a random direction each round to escape the maze. If it encounters any physical resistance in its movements or takes any damage, it may immediately make a Will save to disbelieve the effect.
\end{spelleffect}

\spellsection{Phantom Steed}
\spelldesc{You create a quasi-real horselike creature to serve you or one of your allies. It has a black head and body, gray mane and tail, and smoke-colored, insubstantial hooves that make no sound. On its body, it bears what seems to be a saddle, bit, and bridle sized perfectly for its intended rider.}
\spellschool{Illusion/Transmutation (Imbuement, Shadow)}
\spelllvl{Sor/Wiz 3}
\spelltime{1 standard action}
\spellrng{\rngclose}
\spelleff{One quasi-real, horselike creature}
\spelldur{\durext (D)}
\spellsave{None}
\spellsr{No}
\begin{spelleffect}
  This spell creates a Large, horselike creature that can only be ridden by you or one person you designate. The mount cannot fight, and has an AC of 18 (\minus1 size, \plus4 natural armor, \plus5 Dex) and 10 hit points \add 1 per caster level. If it loses all its hit points, the phantom steed disappears. A phantom steed has a speed of 10 feet per two caster levels. It can bear its rider's weight plus up to 10 pounds per caster level.
  \par These mounts gain certain powers according to caster level. A mount's abilities include those of mounts of lower caster levels. 
  \par 8th level: The mount can ride over sandy, muddy, or even swampy ground without difficulty or decrease in speed.

  \par 12th level: The mount can use \spell{water walk} at will (as the spell, no action required to activate this ability).

  \par 16th level: The mount can use \spell{air walk} at will (as the spell, no action required to activate this ability) for up to 1 round at a time, after which it falls to the ground.

  \par 20th level: The mount can fly at its speed (good maneuverability) by concentrating, as the \spell{overland flight} spell.
\end{spelleffect}

\spellsection{Phase Door}
\spellschool{Conjuration (Creation/Translocation) [Planar]}
\spelllvl{Conj 6, Travel 7}
\spellcmp{V}
\spellrng{0 ft.}
\spelleff{Ethereal 5 ft. by 8 ft. opening, 10 ft. deep \add 5 ft. deep per three levels}
\spelldur{\durext or until discharged}
\spellsave{None}
\spellsr{No}
\begin{spelleffect}
  This spell creates an ethereal passage through wooden, plaster, or stone walls, but not other materials. The phase door is invisible and inaccessible to all creatures except you, and only you can use the passage. It can be used a number of times equal to half your caster level before the spell ends. You disappear when you enter the phase door and appear when you exit. If you desire, you can take one other creature (Medium or smaller) through the door. This counts as two uses of the door. The door does not allow light, sound, or spell effects through it, nor can you see through it without using it. Thus, the spell can provide an escape route, though certain creatures, such as phase spiders, can follow with ease.
  \par You can allow other creatures to use the phase door by setting some triggering condition for the door. Such conditions can be as simple or elaborate as you desire. They can be based on a creature's name, identity, or alignment, but otherwise must be based on observable actions or qualities. Intangibles such as level, class, Hit Values, and hit points don't qualify.
\end{spelleffect}
\begin{spellnotes}
  The \spell{true seeing} spell or similar magic reveals the presence of a phase door but does not allow its use. A \spell{phase door} is subject to \spell{dispel magic}. If anyone is within the passage when it is dispelled, he is harmlessly ejected just as if he were inside a \spell{passwall} effect.
  \par \spell{Phase door} can be made permanent with a \spell{permanency} spell.
\end{spellnotes}

\spellsection{Poison}
\spellschool{Necromancy (Flesh) [Poison]}
\spelllvl{Clr 4, Drd 3}
\spellrng{Touch}
\spelltgt{Living creature touched}
\spelldur{Instantaneous; see text}
\spellsave{Fortitude negates; see text}
\spellsr{Yes (Fortitude)}
\begin{spelleffect}
  Calling upon the venomous powers of natural predators, you infect the subject with a horrible poison that drains its life force by making a successful melee touch attack. The poison deals 1d6 points of Constitution damage immediately. A Fortitude save negates this damage. The spell continues dealing another 1d6 points of Constitution damage every two rounds until the subject makes two successful Fortitude saves to resist the poison.
\end{spelleffect}

\spellsection{Polar Ray}
\spelldesc{You fire a blue-white ray of frigid air and ice, freezing your foe in place.}
\spellschool{Evocation (Energy) [Cold]}
\spelllvl{Sor/Wiz 8, Water 8}
\spellrng{\rngclose}
\spelleff{Ray}
\spelldur{Instantaneous/5 rounds}
\spellsave{None}
\spellsr{Yes (Fortitude)}
\spelldmg{16d6 cold damage \add d6 per three caster levels above 16th}
\begin{spellhealthy}
  The struck target takes damage and is slowed for 5 rounds.
\end{spellhealthy}
\begin{spellblood}
  The struck target takes damage and is frozen solid, causing it to be paralyzed for 5 rounds.
\end{spellblood}
\begin{spellnotes}
   A slowed creature can take only a single move action or standard action each turn, but not both. Additionally, it takes a \minus2 penalty to attack rolls, Strength and Dexterity-based skill checks, and armor class.

   A paralyzed creature cannot take any action that requires motion. It has effective Dexterity and Strength scores of \minus10 and is helpless, but can take purely mental actions. A winged creature flying in the air at the time that it becomes paralyzed cannot flap its wings and falls. A paralyzed swimmer can't swim and may drown. A creature can move through a space occupied by a paralyzed creature -- ally or otherwise. Each square occupied by a paralyzed creature, however, counts as 2 squares.
 \end{spellnotes}

\spellsection{Power Word Blind}
\spellschool{Necromancy (Flesh)}
\spelllvl{Sor/Wiz 6}
\spellcmp{V}
\spellrng{\rngclose}
\spelltgt{One creature}
\spelldur{Instantaneous/5 rounds}
\spellsave{None}
\spellsr{Yes (Fortitude)}
\begin{spellhealthy}
  The target is sickened, making it vulnerable for 5 rounds.
\end{spellhealthy}
\begin{spellblood}
  The target is blinded for 5 rounds.
\end{spellblood}
\begin{spellnotes}
  A vulnerable creature takes a \minus2 penalty to attack rolls, saving throws, checks, DCs, and AC.
  The target must be bloodied when the spell is cast to suffer the bloodied effect.
\end{spellnotes}

\spellsection{Power Word Kill}
\spelldesc{You utter a single word of power that instantly kills your foe, whether it can hear the word or not.}
\spellschool{Necromancy (Life) [Death]}
\spelllvl{Death 9, Sor/Wiz 9}
\spellcmp{V}
\spellrng{\rngclose}
\spelltgt{One living creature}
\spelldur{Instantaneous}
\spellsave{None}
\spellsr{Yes (Fortitude)}
\begin{spellhealthy}
  The target is sickened for 5 rounds, making it vulnerable.
\end{spellhealthy}
\begin{spellblood}
  If the target's HV does not exceed your caster level, it is instantly slain. Otherwise, it is sickened for 5 rounds.
\end{spellblood}
\begin{spellnotes}
  A vulnerable creature takes a \minus2 penalty to attack rolls, saving throws, checks, DCs, and AC.

  The target must be bloodied when the spell is cast to suffer the bloodied effect.
\end{spellnotes}

\spellsection{Power Word Confuse}
\spellschool{Enchantment (Compulsion) [Mind-Affecting]}
\spelllvl{Sor/Wiz 5}
\spellcmp{V}
\spellrng{\rngclose}
\spelltgt{One creature}
\spelldur{Instantaneous}
\spellsave{None}
\spellsr{Yes (Will)}
\begin{spellhealthy}
  The target is bewildered, making it vulnerable for 5 rounds.
\end{spellhealthy}
\begin{spellblood}
  The target is confused for 5 rounds. \confusionexplanation
\end{spellblood}
\begin{spellnotes}
  A vulnerable creature takes a \minus2 penalty to attack rolls, saving throws, checks, DCs, and AC.
  The target must be bloodied when the spell is cast to suffer the bloodied effect.
\end{spellnotes}

\spellsection{Power Word Stun}
\spelldesc{You utter a single word of power that instantly causes your foe to become stunned, whether the creature can hear the word or not.}
\spellschool{Enchantment (Inhibition) [Mind-Affecting]}
\spelllvl{Sor/Wiz 7}
\spellcmp{V}
\spellrng{\rngclose}
\spelltgt{One creature}
\spelldur{Instantaneous}
\spellsave{None}
\spellsr{Yes (Will)}
\begin{spellhealthy}
  The target is bewildered, making it vulnerable for 5 rounds.
\end{spellhealthy}
\begin{spellblood}
  The target is stunned for 5 rounds.
\end{spellblood}
\begin{spellnotes}
  A vulnerable creature takes a \minus2 penalty to attack rolls, saving throws, checks, DCs, and AC.
  The target must be bloodied when the spell is cast to suffer the bloodied effect.
\end{spellnotes}

\spellsectioncomma{Precognition}{Lesser}
\spelldesc{You extend your mind a fraction of a second into the future, allowing you to strike at your foes more effectively.}
\spellschool{Divination (Knowledge)}
\spelllvl{Div 1}
\spellrng{Personal}
\spelltgt{You}
\spelldur{\durshort (D)}
\begin{spelleffect}
  You gain a \plus2 bonus to your attack and weapon damage rolls. \bonusscalingdescription
\end{spelleffect}

\spellsection{Precognition}
\spelldesc{You extend your mind a fraction of a second into the future, allowing you to strike at your foes more effectively and avoid hostile attacks more easily.}
\spellschool{Divination (Knowledge)}
\spelllvl{Div 4}
\begin{spelleffect}
  This spell functions like \spell{lesser precognition}, except that it also affects your saving throws and dodge modifier to AC.
\end{spelleffect}

\spellsectioncomma{Precognition}{Greater}
\spelldesc{You extend your mind a short time into the future, allowing you to strike at your foes more effectively and avoid hostile attacks more easily.}
\spellschool{Divination (Knowledge)}
\spelllvl{Div 7}
\begin{spelleffect}
  This spell functions like \spell{lesser precognition}, except that it also affects your saving throws and dodge modifier to AC. In addition, when making a full attack, you may make an additional attack at a \minus5 penalty.
\end{spelleffect}

\spellsection{Prismatic Sphere}
\spellschool{Evocation (Control, Energy) [Light]}
\spelllvl{Sor/Wiz 9}
\spelleff{\areasmall radius hollow sphere centered on you}
\begin{spelleffect}
  This spell functions like \spell{prismatic wall}, except you conjure up an immobile, opaque globe of shimmering, multicolored light that surrounds you and protects you from all forms of attack. The sphere flashes in all colors of the visible spectrum. 
  \par You can pass into and out of the prismatic sphere and remain near it without harm. However, the sphere blocks any attempt to project something through it (including spells). Other creatures that attempt to attack you or pass through suffer the effects of each color, one at a time. You can fight from partially within the sphere. If you do, you gain cover from anyone outside the sphere.
  \par Typically, only the upper hemisphere of the globe will exist, since you are at the center of the sphere, so the lower half is usually excluded by the floor surface you are standing on.
  \par The colors of the sphere have the same effects as the colors of a prismatic wall.
\end{spelleffect}
\begin{spellnotes}
  \spellindirect{prismatic sphere}{Prismatic sphere} can be made permanent with a \spell{permanency} spell.
\end{spellnotes}

\spellsection{Prismatic Spray}
\spellschool{Evocation (Control, Energy) [Light]}
\spelllvl{Chaos 7, Sor/Wiz 7}
\spellarea{\arealarge cone-shaped burst}
\spelldur{Instantaneous}
\spellsave{See text}
\spellsr{Yes (varies)}
\begin{spelleffect}
  This spell causes seven shimmering, intertwined, multicolored beams of light to spray from your hand. Each beam has a different power. Every creature in the area is randomly struck by one or more beams, which have unique effects.
  \begin{dtable}
    \begin{tabularx}{\columnwidth}{l >{\lcol}p{4em} >{\lcol}X}
      \thead{1d8} & \thead{Color of Beam} & \thead{Effect} \\
      1 & Red & 15 points fire damage (Reflex half) \\
      2 & Orange & 30 points acid damage (Reflex half) \\
      3 & Yellow & 45 points electricity damage (Reflex half) \\
      4 & Green & 40 damage and nauseated for 1 round (Fortitude negates) \\
      5 & Blue & Petrified if bloodied, slowed for 5 rounds if healthy (Fortitude negates) \\
      6 & Indigo & Insane, as \spell{insanity} spell (Will negates) \\
      7 & Violet & Sent to another plane, as \spell{plane shift} ritual (Will negates) \\
      8 & & Struck by two rays; roll twice more, ignoring any ``8" results.
    \end{tabularx}
  \end{dtable}
\end{spelleffect}

\spellsection{Prismatic Wall}
\spellschool{Evocation (Control, Energy) [Light]}
\spelllvl{Chaos 8, Sor/Wiz 8}
\spellrng{\rngclose}
\spelleff{Wall up to 50 ft. wide, 30 ft. high}
\spelldur{\durshort (D)}
\spellsave{See text}
\spellsr{See text}
\begin{spelleffect}
  This spell creates a vertical, opaque wall -- a shimmering, multicolored plane of light that protects you from all forms of attack. The wall flashes with seven colors, each of which has a distinct power and purpose. The wall is immobile, and you can pass through and remain near the wall without harm. However, any other creature with less than 8 HV that is within 20 feet of the wall is blinded for 1 minute by the colors if it looks at the wall.
  \par The wall's maximum proportions are 50 feet wide and 30 feet high. A \spell{prismatic wall} spell cast to materialize in a space occupied by a creature is disrupted, and the spell is wasted.
  \par Each color in the wall has a special effect. The accompanying table shows the seven colors of the wall, the order in which they appear, their effects on creatures trying to attack you or pass through the wall, and the magic needed to negate each color.
  \par The wall can be destroyed, color by color, in consecutive order, by various magical effects; however, the first color must be brought down before the second can be affected, and so on. A rod of cancellation or a mage's disjunction spell destroys a prismatic wall, but an antimagic field fails to penetrate it. Dispel magic and greater dispel magic cannot dispel the wall or anything beyond it. Spell resistance is effective against a prismatic wall, but the caster level check must be repeated for each color present.
  \begin{dtable*}
    \begin{tabularx}{\textwidth}{l l >{\lcol}X l}
      \thead{Color} & \thead{Order} & \thead{Effect of Color} & \thead{Negated By} \\
      Red & 1st & Stops nonmagical ranged weapons.
      Deals 15 points of fire damage (Reflex half). & \spellindirect{cone of cold}{cone of cold} \\
      Orange & 2nd & Stops magical ranged weapons.
      Deals 30 points of acid damage (Reflex half). & \spellindirect{gust of wind}{Gust of wind} \\
      Yellow & 3rd & Stops poisons, gases, and petrification.
      Deals 45 points of electricity damage (Reflex half). & \spell{Disintegrate} \\
      Green & 4th & Stops breath weapons.
      Poison (40 damage and nauseated for 1 round; Fortitude negates). & \spell{Passwall} \\
      Blue & 5th & Stops divination and mental attacks.
      Petrified if bloodied, slowed and entangled for 1 minute if healthy (Fortitude negates). & \spellindirect{magic missile}{Magic missile} \\
      Indigo & 6th & Stops all spells.
      Will save or become insane (as \spell{insanity} spell). & \spell{Daylight} \\
      Violet & 7th & Energy field destroys all objects and effects.\footnotetemp{1}
      Creatures sent to another plane (as \spell{plane shift} ritual) (Will negates). & \spellindirect{dispel magic}{Dispel magic} \\
    \end{tabularx}
    1 The violet effect makes the special effects of the other six colors redundant, but these six effects are included here because certain magical effects can create prismatic effects one color at a time, and spell resistance might render some colors ineffective (see above).
  \end{dtable*}
\end{spelleffect}
\begin{spellnotes}
  \spellindirect{prismatic wall}{Prismatic wall} can be made permanent with a permanency spell.
\end{spellnotes}

\spellsection{Project Image}
\spellschool{Illusion (Shadow)}
\spelllvl{Illus 6}
\spellrng{\rngmed}
\spelleff{One shadow duplicate}
\spelldur{\durshort (D)}
\spellsave{Will disbelief (if interacted with)}
\spellsr{No}
\begin{spelleffect}
  You tap energy from the Plane of Shadow to create a quasi-real, illusory version of yourself. The projected image looks, sounds, and smells like you but is intangible. The projected image mimics your actions (including speech) unless you direct it to act differently (which is a move action).
  \par You can see through its eyes and hear through its ears as if you were standing where it is, and during your turn you can switch from using its senses to using your own, or back again, as a swift action. While you are using its senses, your body is considered blinded and deafened.
  \par If you desire, any spell you cast whose range is touch or greater can originate from the projected image instead of from you. The projected image can't cast any spells on itself except for illusion spells. The spells affect other targets normally, despite originating from the projected image.
\end{spelleffect}
\begin{spellnotes}
  Objects are affected by the projected image as if they had succeeded on their Will save.
  \par You must maintain line of effect to the projected image at all times. If your line of effect is obstructed, the spell ends. If you use \spell{dimension door}, \spell{teleport}, \spell{plane shift}, or a similar spell that breaks your line of effect, even momentarily, the spell ends.
\end{spellnotes}

\spellsection{Protection from Chaos}
\spellschool{Abjuration (Interdiction) [Lawful]}
\spelllvl{Clr 1, Law 1, Pal 1, Sor/Wiz 1}
\begin{spelleffect}
  This spell functions like \spell{protection from evil}, except that it protects against lawful effects.
\end{spelleffect}

\spellsection{Protection from Energy}
\spellschool{Abjuration (Shielding)}
\spelllvl{Clr 3, Drd 3, Pal 3, Protection 3, Sor/Wiz 3}
\spellrng{Touch}
\spelltgt{Creature touched}
\spelldur{\durlong or until discharged}
\spellsave{Fortitude negates (harmless)}
\spellsr{Yes (Fortitude)}
\begin{spelleffect}
  This spell grants temporary immunity to the type of energy you specify when you cast it (acid, cold, electricity, fire, or sonic). When the spell absorbs 10 points per caster level of energy damage, it is discharged.
\end{spelleffect}
\begin{spellnotes}
  Protection from energy overlaps (and does not stack with) \spell{resist energy}. If a character is shielded by \spell{protection from energy} and \spell{resist energy}, the protection spell absorbs damage until its power is exhausted.
\end{spellnotes}

\spellsectioncomma{Protection from Energy}{Greater}
\spellschool{Abjuration (Shielding)}
\spelllvl{Clr 6, Drd 6, Protection 6, Sor/Wiz 6}
\begin{spelleffect}
  This spell functions like \spell{protection from energy}, except that it protects from all five types of energy. When the spell absorbs 10 points per caster level of damage in total, regardless of its type, it is discharged.
\end{spelleffect}

\spellsection{Protection from Evil}
\spelldesc{You guard your ally with a faint pure white aura, shielding him from evil influence.}
\spellschool{Abjuration (Interdiction) [Good]}
\spelllvl{Clr 1, Good 1, Pal 1, Sor/Wiz 1}
\spellrng{\rngclose}
\spelltgt{One creature}
\spelldur{\durshort (D)}
\spellsave{Will negates (harmless)}
\spellsr{No; See text}
\begin{spelleffect}
  The subject gains a \plus2 bonus on saving throws. \bonusscalingdescription

  In addition, the spell blocks any evil attempt to possess or exercise mental control over the creature (such as any domination effect). The protection does not prevent such effects from targeting the protected creature, but it suppresses the effect for the duration of the \spell{protection from evil} spell. If the \spell{protection from evil} spell ends before the effect granting mental control does, the would-be controller would then be able to mentally command the controlled creature. Likewise, the barrier keeps out a possessing life force but does not expel one if it is in place before the spell is cast. This effect works only against attacks by evil creatures or from evil effects.
\end{spelleffect}

\spellsection{Protection from Good}
\spellschool{Abjuration (Interdiction) [Evil]}
\spelllvl{Clr 1, Evil 1, Sor/Wiz 1}
\begin{spelleffect}
  This spell functions like\spell{protection from evil}, except that it protects against good effects.
\end{spelleffect}

\spellsection{Protection from Law}
\spellschool{Abjuration (Interdiction) [Chaotic]}
\spelllvl{Chaos 1, Clr 1, Sor/Wiz 1}
\begin{spelleffect}
  This spell functions like \spell{protection from evil}, except that it protects against chaotic effects.
\end{spelleffect}

\spellsection{Protection from Spells}
\spellschool{Abjuration (Shielding) [Magic]}
\spelllvl{Magic 8, Sor/Wiz 8}
\begin{spelleffect}
  This spell functions like \spell{spell resistance}, except that the subject also gains a \plus5 bonus on saving throws against spells and spell-like abilities (but not against supernatural and extraordinary abilities).
\end{spelleffect}

\spellsection{Quiet Mind}
\spellschool{Transmutation (Augment)}
\spelllvl{Sor/Wiz 1}
\spelltime{1 swift action}
\spellrng{Personal}
\spelltgt{You}
\spelldur{1 round or until discharged}
\begin{spelleffect}
  You gain a \plus10 bonus to Concentration checks. After you cast a spell, this spell ends.
\end{spelleffect}

\pdfbookmark[2]{Q-R}{SpellDescriptionsQR}
\begin{comment}
\subsubsection{Q-R}
\end{comment}

\spellsection{Rainbow Pattern}
\spelldesc{You create a glowing, rainbow-hued pattern of interweaving colors that fascinates those within it.}
\spellschool{Enchantment/Illusion (Compulsion, Figment) [Light, Mind-Affecting, Sight-Dependent]}
\spelllvl{Sor/Wiz 4}
\spellrng{\rngmed}
\spellarea{\areasmall radius spread}
\spelleff{Colorful lights in the area}
\spelldur{Concentration}
\spellsave{Will negates}
\spellsr{Yes (Will)}
\begin{spelleffect}
  Creatures in the spell's area are fascinated. While concentrating on the spell, you can make the pattern move up to 30 feet per round (moving its effective point of origin). All fascinated creatures follow the moving rainbow of light, trying to get or remain within the effect. Fascinated creatures who are restrained and removed from the pattern still try to follow it. If the pattern leads its subjects into a dangerous area each fascinated creature gets a second save. If the view of the lights is completely blocked, creatures who can't see them are no longer affected.
\end{spelleffect}
\begin{spellnotes}
  The spell does not affect sightless creatures.
\end{spellnotes}

\spellsection{Ray of Clumsiness}
\spelldesc{You fire a coruscating ray from your hand. When it strikes your foe, he becomes clumsier and less agile.}
\spellschool{Necromancy (Flesh)}
\spelllvl{Sor/Wiz 1}
\spellrng{\rngclose}
\spelleff{Ray}
\spelldur{\durshort}
\spellsave{Fortitude half}
\spellsr{Yes (Fortitude)}
\begin{spelleffect}
  You must succeed on a ranged touch attack. The subject takes a \minus4 penalty to Dexterity.
\end{spelleffect}
\begin{spellnotes}
  The subject's Dexterity score cannot drop below 1.
\end{spellnotes}

\begin{comment}
\spellsection{Ray of Exhaustion}
\spelldesc{You fire a black ray at your foe, depleting his stamina.}
\spellschool{Necromancy (Flesh)}
\spelllvl{Sor/Wiz 3}
\spellrng{\rngclose}
\spelleff{Ray}
\spelldur{\durmed}
\spellsave{Fortitude partial; see text}
\spellsr{Yes (Fortitude)}
\begin{spelleffect}
  If you succeed on a ranged touch attack with the ray, the subject is immediately exhausted. A successful Fortitude save means the creature is only fatigued.
\end{spelleffect}
\begin{spellnotes}
  A creature that is already fatigued instead becomes exhausted. A creature that is already exhausted suffers no further penalties. Unlike normal exhaustion or fatigue, the effect ends as soon as the spell's duration expires.
\end{spellnotes}
\end{comment}

\spellsection{Redirection}
\spellschool{Abjuration (Shielding)}
\spelllvl{Abjur 3}
\spellrng{Personal}
\spelltgt{You}
\spelldur{\durshort (D)}
\begin{spelleffect}
  Attacks made against you have a 20\% miss chance. This miss chance stacks with and is rolled before any other miss chance, such as from active cover. Any attack that misses you because of this miss chance is instead made against a new creature of your choice, other than the attacker.
  \par The new target must be a creature adjacent to you that the attacker threatens (if using a melee weapon) or can target (if using a ranged weapon). The new target must also be a creature that you can see and target. If there is no such creature, the attack simply misses.
\end{spelleffect}

\spellsection{Reduce Person}
\spellschool{Transmutation (Polymorph) [Size-Affecting]}
\spelllvl{Trans 2}
\spelltime{Full-round action}
\spellrng{\rngclose}
\spelltgt{One humanoid creature}
\spelldur{\durshort (D)}
\spellsave{Fortitude negates}
\spellsr{Yes (Fortitude)}
\begin{spelleffect}
  This spell causes instant diminution of a humanoid creature, halving its height, length, and width and dividing its weight by 8. This decrease changes the creature's size category to the next smaller one. This has several effects.
  \begin{itemize*} 
    \item \minus10 ft. inherent bonus to movement speed.
    \item \plus1 inherent bonus to attack rolls and AC due to its decreased size.
  \item \minus2 penalty to Strength.
  \item \plus2 bonus to Dexterity. \bonusscalingdescription
  \end{itemize*}
  \par A Small humanoid creature whose size decreases to Tiny has a space of 2-1/2 feet and a natural reach of 0 feet (meaning that it must enter an opponent's square to attack). A Large humanoid creature whose size decreases to Medium has a space of 5 feet and a natural reach of 5 feet.
  \par All equipment worn or carried by a creature is similarly reduced by the spell. Melee and projectile weapons deal less damage. Other magical properties are not affected by this spell. Any reduced item that leaves the reduced creature's possession (including a projectile or thrown weapon) instantly returns to its normal size. This means that thrown weapons deal their normal damage (projectiles deal damage based on the size of the weapon that fired them).
\end{spelleffect}
\begin{spellnotes}
  Multiple magical effects that reduce size do not stack.
  \par \spellindirect{reduce person}{Reduce person} counters and dispels \spell{enlarge person}.
  \par \spellindirect{reduce person}{Reduce person} can be made permanent with a \spell{permanency} spell.
\end{spellnotes}

\spellsectioncomma{Reduce Person}{Mass}
\spellschool{Transmutation (Polymorph) [Size-Affecting]}
\spelllvl{Trans 6}
\spellrng{\rngmed}
\spellarea{\areamed radius limit}
\spelltgts{Five humanoid creatures within the area}
\begin{spelleffect}
  This spell functions like \spell{reduce person}, except that it affects multiple creatures.
\end{spelleffect}

\spellsection{Regenerate}
\spellschool{Necromancy (Flesh) [Healing]}
\spelllvl{Clr 8, Drd 8}
\spellrng{Touch}
\spelltgt{Living creature touched}
\spelldur{\durshort}
\spellsave{Fortitude negates (harmless)}
\spellsr{Yes (Fortitude)}
\begin{spelleffect}
  You grant immense healing power to a creature with a touch. The target of this spell automatically heals a number of hit points each round equal to your caster level.
  \par You can also use this spell to regrow lost portions of the subject's body and to reattach severed limbs or body parts, if you do nothing but concentrate on regrowing the lost body part or reattaching the severed limb for 5 minutes.
\end{spelleffect}

\spellsection{Repulsion}
\spellschool{Abjuration (Shielding) [Barrier]}
\spelllvl{Abjur 6, Protection 6, Travel 6}
\spellarea{Up to a \arealarge radius emanation centered on you}
\spelldur{\durshort (D)}
\spellsave{Will negates}
\spellsr{Yes (Will)}
\begin{spelleffect}
  An invisible, mobile field surrounds you and prevents creatures from approaching you. You decide how big the field is at the time of casting. Any creature within or entering the field must attempt a save. If it fails, it becomes unable to move toward you for the duration of the spell. Repelled creatures' actions are not otherwise restricted.
  \par They can fight other creatures and can cast spells and attack you with ranged weapons. The creature is free to make melee attacks against you if you come within reach. If a repelled creature moves away from you and then tries to turn back toward you, it cannot move any closer if it is still within the spell's area.
\end{spelleffect}
\begin{spellnotes}
  Unlike most barrier spells, this spell does not collapse if you move towards a creature held at bay by the barrier. The spell continues to prevent that creature from approaching you, but the creature suffers no other ill effect.
\end{spellnotes}

\spellsection{Resilient Sphere}
\spellschool{Evocation (Control) [Force]}
\spelllvl{Evoc 5}
\spellrng{\rngmed}
\spelleff{5 ft. radius sphere, centered around creatures or objects}
\spelldur{\durshort (D)}
\spellsave{Reflex negates}
\spellsr{Yes (Reflex)}
\begin{spelleffect}
  This spell creates a globe of shimmering force centered around a creature or object. The sphere persists for the spell's duration, containing any creatures or objects held inside, provided they are small enough to fit within the diameter of the sphere. It is not subject to damage of any sort.
  \par The subject may struggle, but the sphere cannot be physically moved either by people outside it or by the struggles of those within.
\end{spelleffect}
\begin{spellnotes}
  The sphere can only be affected a \spell{disintegrate} spell, a targeted \spell{dispel magic} spell, or similar effects. These effects destroy the sphere without harm to the subject. Nothing can pass through the sphere, inside or out, though the subject can breathe normally.
\end{spellnotes}

\spellsection{Resist Energy}
\spellschool{Abjuration (Shielding)}
\spelllvl{Clr 2, Drd 2, Pal 2, Protection 2, Sor/Wiz 2}
\spellrng{Touch}
\spelltgt{Creature touched}
\spelldur{\durlong or until discharged}
\spellsave{Fortitude negates (harmless)}
\spellsr{Yes (Fortitude)}
\begin{spelleffect}
  The subject gains energy damage reduction 10 against whichever of the five energy types that you select: acid, cold, electricity, fire, or sonic. This damage reduction increases by 1 per caster level above 4th.
  \par The spell can absorb a maximum amount of damage equal to 10 points per caster level. After it absorbs its maximum amount of damage, the spell ends.
\end{spelleffect}
\begin{spellnotes}
  This spell's damage reduction allows the subject to ignore the first 10 energy damage it takes each round of the appropriate type.

  Resist energy absorbs only damage. The subject could still suffer unfortunate side effects. The spell protects the recipient's equipment as well.
  \par \spellindirect{resist energy}{Resist energy} overlaps (and does not stack with) \spell{protection from energy}. If a character is shielded by both spells, the \spellindirect{protection from energy}{protection} spell absorbs damage until its power is exhausted. A character can only be affected by one \spell{resist energy} spell at once.
\end{spellnotes}

\spellsectioncomma{Resist Energy}{Greater}
\spellschool{Abjuration (Shielding)}
\spelllvl{Clr 4, Drd 4, Pal 4, Sor/Wiz 4}
\begin{spelleffect}
  This spell functions like \spell{resist energy}, except that the creature gains protection from all five energy types at once. The spell can absorb a total amount of damage equal to 10 points per caster level.
\end{spelleffect}
\begin{spellnotes}
  A character can only be affected by one \spell{resist energy} spell at once.
\end{spellnotes}

\spellsection{Retributive Shield}
\spellschool{Abjuration/Necromancy (Life, Shielding)}
\spelllvl{Sor/Wiz 4}
\spellrng{\rngclose/\rngmed}
\spelltgt{One creature}
\spelldur{\durshort}
\spellsave{Will negates (harmless)}
\spellsr{Yes (Will)}
\begin{spelleffect}
  A subject within \rngclose range gains physical damage reduction 8/life. This damage reduction increases by 1 per two caster levels above 8th. In addition, the spell reflects the damage back at the creature's attackers. Any creature within \rngmed range of the subject that attacks it takes life damage equal to the amount of damage resisted by this spell.
\end{spelleffect}
\begin{spellnotes}
  This spell's damage reduction allows the subject to ignore the first 8 physical damage it takes each round. If it is hit by an attack that deals life damage, such as \spell{crush life}, it cannot use its damage reduction for 1 round.
\end{spellnotes}

\spellsection{Retrieve}
\spellschool{Conjuration (Translocation) [Teleportation]}
\spelllvl{Conj 1}
\spellrng{\rngclose}
\spelltgt{One object you can hold or carry in one hand, weighing up to 10 lb./level}
\spelldur{Instantaneous}
\spellsave{None (object)}
\spellsr{Yes (Will)}
\begin{spelleffect}
  You teleport an item you can see within range directly to your hand. If the object is attended, this spell automatically fails.
\end{spelleffect}

\spellsectioncomma{Retrieve}{Greater}
\spellschool{Conjuration (Translocation) [Teleportation]}
\spelllvl{Conj 5}
\spellrng{\rngmed}
\spellsave{Will negates (object)}
\begin{spelleffect}
  This spell functions like \spell{retrieve}, except that if the object is attended, it comes to your hand if the creature holding the item fails a Will save.
\end{spelleffect}

\spellsection{Reveal Death}
\spelldesc{You grant a creature a vision of its death - whether immediate or far in the future.}
\spellschool{Divination (Knowledge)}
\spelllvl{Death 2, Div 2}
\spellrng{\rngmed}
\spelltgt{One creature}
\spelldur{\durshort}
\spellsave{None}
\spellsr{Yes (Will)}
\begin{spelleffect}
    This spell has different effects depending on the version chosen.
    \par \subspell{Distant Demise} The subject gains a \plus2 bonus to saving throws. In addition, it is not staggered while at 0 hit points. Further damage is still critical damage and can cause the creature to begin dying as normal. 
    \par \subspell{Imminent Demise} The subject becomes vulnerable.
\end{spelleffect}
\begin{spellnotes}
  \vulnerableexplanation
\end{spellnotes}

\spellsection{Revelation}
\spellschool{Divination (Awareness, Knowledge)}
\spelllvl{Div 8, Knowledge 9, Sor/Wiz 9}
\spellrng{\rngmed}
\spelltgt{One creature}
\spelldur{\durshort}
\spellsave{None}
\spellsr{Yes (Will)}
\begin{spelleffect}
  You grant the target a powerful revelatory vision of a possible future. This spell has different effects depending on the version chosen. Creatures without an Intelligence score are not affected by this spell.
  \par \subspell{Revelation of Destruction} You inflict a vision of a terrible future upon the target. It takes a \minus4 penalty to attack rolls, checks, saving throws, DCs, and AC as it struggles to avoid the certainty of its own doom.
  \par \subspell{Revelation of Prowess} You show the target a vision of its success in the combat to come. It gains the benefits of a \spell{greater precognition} spell.
  \par \subspell{Revelation of Truth} You show the target the truth of the world around it. It gains the benefits of a \spell{true seeing} spell.
\end{spelleffect}

\spellsection{Reverse Gravity}
\spellschool{Transmutation}
\spelllvl{Air 8, Trickery 8, Sor/Wiz 8}
\spellrng{\rngclose}
\spellarea{Up to five 10 ft. cubes (S)}
\spelldur{Concentration (up to 5 rounds)}
\spellsave{None; see text}
\spellsr{No}
\begin{spelleffect}
  This spell reverses gravity in an area, causing all unattached objects and creatures within that area to fall upward and reach the top of the area in 1 round. If some solid object (such as a ceiling) is encountered in this fall, falling objects and creatures strike it in the same manner as they would during a normal downward fall. If an object or creature reaches the top of the area without striking anything, it remains there, oscillating slightly, until the spell ends. At the end of the spell duration, affected objects and creatures fall downward.
  \par A creature caught in the area can attempt a Reflex save to react to the shift in gravity. Common reactions include securing oneself if possible, or jumping to reach more stable ground.
\end{spelleffect}
\begin{spellnotes}
  Creatures who can fly or levitate can keep themselves from falling, though the shift in gravity can be disorienting. A creature that reacts by jumping does not actually move until its turn, but it moves in the direction of its jump, rather than simply falling upwards.
\end{spellnotes}

\spellsection{Revivify}
\spelldesc{You reconnect a corpse's soul with its body before the soul has completely passed on.}
\spellschool{Necromancy (Life, Soul)}
\spelllvl{Cleric 5}
\spellcmp{V, S, M}
\spellrng{Touch}
\spelltgt{Dead creature touched}
\spelldur{Instantaneous}
\spellsave{None}
\spellsr{No}
\begin{spelleffect}
  This spell restores a creature to life like the \spell{raise dead} ritual, except that the affected creature suffers no negative effects for having died. However, the spell must be cast within one round of the creature's death per four caster levels. After that time, it has no effect (and the material components are not consumed).

  The creature has 0 hit points and 1 point of critical damage (but is stable) after being restored to life.
\end{spelleffect}
\spellmat{Diamonds worth at least 1,000 gp.}

\spellsection{Righteous Might}
\spellschool{Transmutation (Augment, Polymorph) [Size-Affecting]}
\spelllvl{Clr 5, Good 5, Pal 4, Strength 5}
\spellrng{Personal}
\spelltgt{You}
\begin{spelleffect}
  This spell functions like \spell{enlarge person}, except that it affects only you, regardless of your creature type. In addition, you gain a \plus4 bonus to Strength (which replaces the bonus to Strength from \spell{enlarge person}) and physical damage reduction equal to 10 \add 1 per two caster levels above 10th. This damage reduction is overcome evil attacks if you are good or neutral, and by good attacks if you are evil.
\end{spelleffect}
\begin{spellnotes}
  This spell's damage reduction allows the subject to ignore the first 10 physical damage it takes each round. If it is hit by an attack that deals appropriately aligned damage, such as a weapon affected by \spell{align weapon} or a spell with the appropriate descriptor, it cannot use its damage reduction for 1 round.
  Multiple magical effects that increase size do not stack.
\end{spellnotes}

\spellsection{Sanctuary}
\spellschool{Abjuration/Enchantment (Compulsion, Shielding)}
\spelllvl{Abjur 1, Clr 1, Pal 1, Protection 1}
\spellrng{Touch}
\spelltgt{Creature touched}
\spelldur{\durshort}
\spellsave{Will negates (harmless) and Will negates; see text}
\spellsr{Yes (Will)}
\begin{spelleffect}
  Any opponent attempting to strike or otherwise directly attack the shielded creature, even with a targeted spell, must attempt a Will save. If the save succeeds, the opponent can attack normally and is unaffected by that casting of the spell. If the save fails, the opponent can't follow through with the attack, that part of its action is lost, and it can't directly attack the shielded creature for the duration of the spell. Those not attempting to attack the subject remain unaffected. This spell does not prevent the shielded creature from being attacked or affected by area or effect spells. The subject cannot attack without breaking the spell but may use nonattack spells or otherwise act.
\end{spelleffect}

\pdfbookmark[2]{S}{SpellDescriptionsS}
\begin{comment}
\subsubsection{S}
\end{comment}

\spellsection{Scintillating Pattern}
\spelldesc{You create a massive spread of colorful lights that spin and whirl in a complex pattern that bewilders your foes.}
\spellschool{Enchantment/Illusion (Compulsion, Figment) [Mind-Affecting, Sight-Dependent]}
\spelllvl{Sor/Wiz 8}
\spellarea{\arealarge radius spread centered on you}
\spelleff{Colorful lights in the area}
\spelldur{\durshort}
\spellsave{None}
\spellsr{Yes (Will)}
\begin{spelleffect}
  All enemies within the spell's area are bewildered for as long as they can see the lights, and for 5 rounds thereafter. In addition, the area is illuminated in bright light out to a 100 ft. radius, and dim light extends an additional 100 ft. beyond that.
\end{spelleffect}
\begin{spellnotes}
  A vulnerable creature takes a \minus2 penalty to attack rolls, saving throws, checks, DCs, and AC. Your allies, and creatures unable to see the lights, are unaffected.
\end{spellnotes}

\spellsection{Scorching Ray}
\spelldesc{You blast your enemies with fiery rays.}
\spellschool{Evocation (Energy) [Fire]}
\spelllvl{Fire 2, Sor/Wiz 2}
\spellrng{\rngclose}
\spelleff{One or more rays}
\spelldur{Instantaneous}
\spellsave{None}
\spellsr{Yes (Reflex)}
\spelldmg{4d6 fire damage \add d6 per two caster levels above 4th}
\begin{spelleffect}
  You may fire up to three rays at the same or separate targets. Each ray requires a ranged touch attack to hit. You may split the damage among the rays as you choose. The rays may be fired at the same or different targets, but all must be aimed at targets within 30 feet of each other and fired simultaneously. Precision damage can only be applied with one of the rays.
\end{spelleffect}

\spellsection{Sculpt Sound}
\spellschool{Illusion (Glamer)}
\spelllvl{Brd 3}
\spellrng{\rngclose}
\spellarea{\areamed radius limit}
\spelltgts{Five creatures or objects within the area}
\spelldur{\durext (D)}
\spellsave{Will negates (object)}
\spellsr{Yes (Will)}
\begin{spelleffect}
  You change the sounds that creatures or objects make. You can deaden sounds that exist or transform sounds into other sounds, but you cannot create new sounds where none existed. All affected creatures or objects must have their sounds altered in the same way. Once the effect is chosen, you cannot change it.
\end{spelleffect}
\begin{spellnotes}
  You can change the qualities of sounds but cannot create words with which you are unfamiliar yourself. A spellcaster whose voice is changed dramatically is treated as deafened when casting spells (20\% chance of failure)
\end{spellnotes}

\spellsection{Sea of Fog}
\spellschool{Conjuration (Creation)}
\spelllvl{Drd 8, Sor/Wiz 8}
\spellarea{200 ft. radius spread centered on you, 50 ft. high}
\spelleff{Fog in the area}
\begin{spelleffect}
  This spell functions like \spell{obscuring mist}, except that the effect is much larger.
\end{spelleffect}
\begin{spellnotes}
  A severe wind disperses the fog within 1 minute, a windstorm disperses it within 5 rounds, and a hurricane disperses it within a round.
\end{spellnotes}

\spellsection{Searing Light}
\spelldesc{You channel divine power into a searing blast of light that erupts your palm, striking your unworthy foe.}
\spellschool{Evocation (Channeling) [Light]}
\spelllvl{Clr 3, Pal 3}
\spellrng{\rngclose}
\spelleff{Ray}
\spelldur{Instantaneous and see text}
\spellsave{Reflex partial}
\spellsr{Yes (Reflex)}
\spelldmg{6d6 divine damage \add d6 per two caster levels above 6th; see text}
\begin{spelleffect}
  If you succeed on a ranged touch attack to hit with the ray, the target takes damage and is dazzled for 1 round. An undead creature takes 6d8 points of damage \add d8 per two caster levels above 6th, and an undead creature particularly vulnerable to bright light takes 6d10 points of damage \add d10 per two caster levels above 6th and is blinded for 1 round instead.
\end{spelleffect}
\begin{spellnotes}
  A dazzled creature has a 20\% miss chance on all attack rolls and takes a \minus4 penalty to Spot checks. He is also unable to see with darkvision.
\end{spellnotes}

\spellsection{See Invisibility}
\spellschool{Divination (Revelation)}
\spelllvl{Sor/Wiz 2}
\spellrng{Touch}
\spelltgt{Touched creature}
\spelldur{\durlong (D)}
\begin{spelleffect}
  You grant the touched creature the ability to see any objects or beings that are invisible within its range of vision, as well as any that are ethereal, as if they were normally visible. Such creatures are visible as translucent shapes, allowing the target to easily discern the difference between visible, invisible, and ethereal creatures.
\end{spelleffect}
\begin{spellnotes}
  The spell does not reveal the method used to obtain invisibility. It does not reveal illusions or enable you to see through opaque objects. It does not reveal creatures who are simply hiding, concealed, or otherwise hard to see.
  \par See invisibility can be made permanent with a permanency spell.
\end{spellnotes}

\spellsection{Seeming}
\spellschool{Illusion (Glamer) [Unreal]}
\spelllvl{Illus 5, Trickery 6}
\spellrng{\rngclose}
\spellarea{\areamed radius limit}
\spelltgts{One creature per level within the area}
\spelldur{\durext (D)}
\spellsave{Will negates (harmless)}
\spellsr{Yes}
\begin{spelleffect}
  This spell functions like \spell{disguise self}, except that it affects multiple creatures. Affected creatures resume their normal appearances if slain.
\end{spelleffect}

\spellsection{Shadow Body}
\spellschool{Illusion/Transmutation (Polymorph, Shadow)}
\spelllvl{Sor/Wiz 8}
\spellrng{Personal}
\spelltgt{You}
\spelldur{\durmed (D)}
\begin{spelleffect}
  Your body and all your equipment are subsumed by your shadow. As a living shadow, you blend perfectly into any other shadow and vanish in darkness. You appear as an unattached shadow in areas of full light.
  \par You can move at your normal speed, on any surface, including walls and ceilings, as well as across the surfaces of liquids -- even up the face of a waterfall.
  \par Your space does not change, so you cannot move into locations you would not normally be able to move into.
  \par While in your shadow body, you gain damage reduction 15/solar and darkvision out to 60 feet. You are immune to ability damage, disease, drowning, and poison. You take only half damage from acid, electricity, and fire of all kinds.
  \par While affected by this spell, you can be detected by spells that read thoughts, life, or presences (including true seeing), or if you make suspicious movements in lighted areas.
  \par You cannot harm anyone physically or manipulate any objects, but you can use your spells normally. Doing so may attract notice, but if you remain in a shadowed area, you get a \plus15 bonus on your Hide check to remain unnoticed.
\end{spelleffect}
\begin{spellnotes}
  This spell's damage reduction allows the subject to ignore the first 15 physical damage it takes each round. If it is hit by an attack that deals solar damage, such as \spell{sunbeam}, it cannot use its damage reduction for 1 round.
\end{spellnotes}

\spellsection{Shadow Conjuration}
\spelldesc{You use material from the Plane of Shadow to shape quasi-real illusions of one or more creatures, objects, or forces.}
\spellschool{Illusion (Shadow)}
\spelllvl{Illus 4}
\spellrng{See text}
\spelleff{See text}
\spelldur{See text}
\spellsave{Will disbelief (if interacted with); varies, see text}
\spellsr{Yes (Will); See text}
\begin{spelleffect}
  Shadow conjuration can mimic any non-restricted sorcerer or wizard conjuration (summoning) or conjuration (creation) spell of 3rd level or lower. If you summon a creature, as with the \spell{summon monster} spells, you may only summon a creature that you know how to summon with such a spell.
  \par Shadow conjurations are actually half as strong as the real things, though creatures who believe the shadow conjurations to be real are affected by them at full strength.
  \par Any creature that interacts with the conjured object, force, or creature can make a Will save to recognize its true nature.
  \par Spells that deal damage have normal effects unless an affected creature succeeds on a Will save. Each disbelieving creature takes only half damage from the attack. If the disbelieved attack has a special effect other than damage, that effect is half as strong (if applicable) or only half as likely to occur. Regardless of the result of the save to disbelieve, an affected creature is also allowed any save (or spell resistance) that the spell being simulated allows.
  \par A shadow creature has half the hit points of a normal creature of its kind (regardless of whether it's recognized as shadowy). It deals normal damage and has all normal abilities and weaknesses. Against a creature that recognizes it as a shadow creature, however, the shadow creature deals half damage, and all special abilities that do not deal lethal damage are only 50\% likely to work. (Roll for each use and each affected character separately.)
\end{spelleffect}
\begin{spellnotes}
  A creature that succeeds on its save sees the shadow conjurations as transparent images superimposed on vague, shadowy forms.
  \par When you learn this spell, you choose three creatures from the 3rd-level or lower lists on the Summon Monster table, one at each level. You can summon those creatures with this or any summon monster spell.
  \par Objects automatically succeed on their Will saves against this spell.
\end{spellnotes}

\spellsectioncomma{Shadow Conjuration}{Greater}
\spellschool{Illusion (Shadow)}
\spelllvl{Illus 7}
\begin{spelleffect}
  This spell functions like \spell{shadow conjuration}, except that it can duplicate any non-restricted sorcerer or wizard conjuration (summoning) or conjuration (creation) spell of 6th level or lower.
\end{spelleffect}
\begin{spellnotes}
  When you learn this spell, you choose six creatures from the 6th-level or lower lists on the Summon Monster table, one at each level. You can summon those creatures with this or any summon monster spell.
\end{spellnotes}

\spellsection{Shadow Evocation}
\spellschool{Illusion (Shadow)}
\spelllvl{Illus 5}
\spellrng{See text}
\spelleff{See text}
\spelldur{See text}
\spellsave{Will disbelief (if interacted with); varies, see text}
\spellsr{Yes (Will); see text}
\begin{spelleffect}
  You tap energy from the Plane of Shadow to cast a quasi-real, illusory version of a non-restricted sorcerer or wizard evocation spell of 4th level or lower. (For a spell with more than one level, use the best one applicable to you.)
  \par Spells that deal damage have normal effects unless an affected creature succeeds on a Will save. Each disbelieving creature takes only half damage from the attack. If the disbelieved attack has a special effect other than damage, that effect is half as strong (if applicable) or only half as likely to occur. Regardless of the result of the save to disbelieve, an affected creature is also allowed any save (or spell resistance) that the spell being simulated allows.
\end{spelleffect}
\begin{spellnotes}
  Objects automatically succeed on their Will saves against this spell.
\end{spellnotes}

\spellsectioncomma{Shadow Evocation}{Greater}
\spellschool{Illusion (Shadow)}
\spelllvl{Illus 8}
\begin{spelleffect}
  This spell functions like \spell{shadow evocation}, except that it enables you to create partially real, illusory versions of non-restricted sorcerer or wizard evocation spells of 7th level or lower.
\end{spelleffect}

\spellsection{Shadow Puppet}
\spellschool{Conjuration/Illusion (Shadow, Translocation) [Planar, Unreal]}
\spelllvl{Illus 9}
\spellrng{Personal/\rngfar; see text}
\spelltgt{You}
\spelleff{One shadow duplicate}
\spelldur{\durmed}
\spellsave{None/Will disbelief (if interacted with)}
\spellsr{No}
\begin{spelleffect}
  You step into the Plane of Shadow (as \spell{shadow walk}, a planar translocation effect), and at the same time, you create a quasi-real, illusory version of yourself (as \spell{project image}, an unreal shadow effect). The double appears superimposed over your body so that observers don't notice an image appearing and you disappearing. You can then control the image and cast spells through it even though you are on a different plane.
\end{spelleffect}
\begin{spellnotes}
  If the image moves farther than \rngfar range away from where it was originally created, or if you leave the Plane of Shadow, the image ceases to exist.
\end{spellnotes}

\spellsection{Shape Stone}
\spellschool{Transmutation (Alteration) [Earth]}
\spelllvl{Drd 3, Earth 3, Sor/Wiz 3}
\spellrng{Touch}
\spelltgt{Stone or stone object touched, up to 10 cu. ft. \add 1 cu. ft./level}
\spelldur{Instantaneous}
\spellsave{None}
\spellsr{No}
\begin{spelleffect}
  You can form an existing piece of stone into any shape that suits your purpose. While it's possible to make crude coffers, doors, and so forth with shape stone, fine detail isn't possible. There is a 30\% chance that any shape including moving parts simply doesn't work.
\end{spelleffect}

\spellsection{Shape Wood}
\spellschool{Transmutation (Alteration)}
\spelllvl{Drd 2}
\spellrng{Touch}
\spelltgt{One touched piece of wood no larger than 10 cu. ft. \add 1 cu. ft./level}
\spelldur{Instantaneous}
\spellsave{Fortitude negates (object)}
\spellsr{Yes (Fortitude)}
\begin{spelleffect}
  This spell enables you to form one existing piece of wood into any shape that suits your purpose. While it is possible to make crude coffers, doors, and so forth, fine detail isn't possible. There is a 30\% chance that any shape that includes moving parts simply doesn't work.
\end{spelleffect}

\spellsection{Share Pain}
\spellschool{Abjuration/Necromancy (Life, Shielding)}
\spelllvl{Clr 2, Pal 2, Protection 2, Sor/Wiz 2}
\spellrng{\rngmed}
\spelltgts{You and one willing creature}
\spelldur{\durlong (D)}
\spellsave{None}
\spellsr{Yes (Will)}
\begin{spelleffect}
  This spell creates a connection between you and a willing subject. As you cast the spell, you decide whether you will take half of the subject's damage, or whether the subject will take half of your damage. All attacks that deal hit point damage are redirected in this way, but no other forms of attack, including critical damage and ability damage, are redirected.
  
  If the subject is out of range of you, the spell is suppressed until the subject returns within the spell's range.
\end{spelleffect}
\begin{spellnotes}
  When this spell ends, subsequent damage is no longer divided between the subject and you, but damage already shared is not reassigned.
\end{spellnotes}

\spellsectioncomma{Share Pain}{Forced}
\spellschool{Abjuration/Necromancy (Life, Shielding)}
\spelllvl{Clr 3, Sor/Wiz 3}
\spellsave{Will negates}
\begin{spelleffect}
  This spell functions like \spell{share pain}, except that it can affect unwilling creatures. 
\end{spelleffect}

\spellsectioncomma{Share Pain}{Greater}
\spellschool{Abjuration/Necromancy (Life, Shielding)}
\spelllvl{Abjur 6}
\spelldur{\durshort (D)}
\begin{spelleffect}
  This spell functions like \spell{share pain}, except that it redirects all damage that one creature would take instead of redirecting half of the damage.
\end{spelleffect}

\spellsection{Shatter}
\spelldesc{You create a loud, ringing noise that sunders solid objects.}
\spellschool{Evocation (Energy) [Sonic]}
\spelllvl{Destruction 2, Sor/Wiz 2}
\spellrng{\rngclose}
\spelltgtorarea{One solid object or one crystalline creature; or \areasmall radius spread}
\spelldur{Instantaneous}
\spellsave{Will negates (object)/Will negates (object) or Fortitude half; see text}
\spellsr{Yes (Will)}
\spelldmg{4d6 sonic damage \add d6 per two levels after 4th}
\begin{spelleffect}
  Used as an area attack, shatter destroys nonmagical objects of crystal, glass, ceramic, or porcelain. All such objects within a \areasmall radius of the point of origin are smashed into dozens of pieces by the spell. Objects weighing more than 1 pound per your level are not affected, but all other objects of the appropriate composition are shattered.
  \par Alternatively, you can target a single solid object or crystalline creature. In the case of large objects, such as walls, you target a 5 ft. cube. The target takes damage, with a Fortitude save for half damage.
  \par A creature holding vulnerable objects can attempt a Will save to negate any effect on to those objects.
\end{spelleffect}

\spellsection{Shield}
\spelldesc{You create an invisible, heavy shield-sized mobile disk of force. It hovers in front of your ally, automatically moving to ward off enemy blows.}
\spellschool{Abjuration (Shielding) [Force]}
\spelllvl{Sor/Wiz 1}
\spellrng{Touch}
\spelltgt{Creature touched}
\spelldur{\durshort (D); see text}
\spellsave{Will negates (harmless)}
\spellsr{Yes (Will)}
\begin{spelleffect}
  The subject gains a \plus2 shield modifier to AC. \bonusscalingdescription The subject is not encumbered or hindered in any way by the shield.
\end{spelleffect}
\begin{spellnotes}
  This shield is considered to be separate from any other shields the creature is using, so it never stacks with existing shield modifiers. Since the \spell{shield} is made of force, incorporeal creatures can't bypass it the way they do normal shields.
  
  If you cast this spell on a creature subject to the \spell{mage armor} spell, its duration lasts until the \spell{mage armor} spell expires. 
\end{spellnotes}

\spellsection{Shield of Faith}
\spelldesc{You create a shimmmering, magical shield that protects your ally as long as you maintain faith.}
\spellschool{Abjuration (Shielding)}
\spelllvl{Clr 1, Pal 1, Protection 1}
\begin{spelleffect}
  This spell functions like \spell{shield}, except that it is not a force effect, so it does not protect against incorporeal touch attacks. It has no special effect when cast on a creature with \spell{mage armor}.
\end{spelleffect}
\begin{spelleffect}
  You can maintain concentration on this spell as a swift action.
\end{spelleffect}

\spellsection{Shield of Law}
\spellschool{Abjuration (Shielding) [Lawful]}
\spelllvl{Clr 8, Law 8}
\spellcmp{V, S, F}
\spellarea{\areamed radius limit centered on you}
\spelltgts{Five creatures within the area}
\spelldur{\durshort (D)}
\spellsave{See text}
\spellsr{Yes (Will)}
\begin{spelleffect}
  A dim, blue glow surrounds the subjects, protecting them from attacks, granting them resistance to spells cast by chaotic creatures, and slowing chaotic creatures when they strike the subjects. This abjuration has four effects.
  \par First, each shielded creature gains a \plus5 bonus to its saving throws.
  \par Second, each shielded creature gains spell resistance 10 against chaotic spells and spells cast by chaotic creatures.
  \par Third, the abjuration blocks possession and mental influence, just as \spell{protection from chaos} does.
  \par Finally, if a chaotic creature within \rngmed range of the shielded creature successfully attacks it in any way, the offending attacker takes 4d6 damage. Any single creature can take this damage only once per round.
\end{spelleffect}
\spellfocus{A tiny reliquary containing some sacred relic, such as a scrap of parchment from a lawful text. The reliquary costs at least 500 gp.}

\spellsection{Shillelagh}
\spellschool{Transmutation}
\spelllvl{Drd 1}
\spellrng{Touch}
\spelltgt{One touched nonmagical oak club or quarterstaff}
\spelldur{\durshort}
\spellsave{Will negates (object)}
\spellsr{Yes (Will)}
\begin{spelleffect}
  Your own nonmagical club or quarterstaff becomes a weapon with a \plus2 enhancement bonus on attack and damage rolls. \bonusscalingdescription (A quarterstaff gains this enhancement for both ends of the weapon.) In addition, the weapon deals damage as if it were one size category larger (a Small club or quarterstaff so transmuted deals 1d6 points of damage, a Medium 1d8, and a Large 1d10).
\end{spelleffect}
\begin{spellnotes}
  These effects only occur when the weapon is wielded by you. If you do not wield it, the weapon behaves as if unaffected by this spell.
\end{spellnotes}

\spellsection{Shocking Grasp}
\spelldesc{You deliver a powerful electrical shock to your foe.}
\spellschool{Evocation (Energy) [Electricity]}
\spelllvl{Destruction 1, Sor/Wiz 1}
\spellrng{Touch}
\spelltgt{Creature or object touched}
\spelldur{Instantaneous}
\spellsave{None/Fortitude negates}
\spellsr{Yes (Fortitude)}
\spelldmg{2d6 electricity damage \add d6 per two caster levels above 2nd}
\begin{spelleffect}
  If you hit with a touch attack, the target takes damage. If it fails a Fortitude save, it is also staggered for 1 round. When delivering the jolt, you gain a \plus2 circumstance bonus to attack if the opponent is wearing metal armor (or made out of metal, carrying a lot of metal, or the like).
\end{spelleffect}
\begin{spellnotes}
   A staggered character may take a single move action or standard action each round, but not both. She cannot take full-round actions, but she may take swift actions. In addition, she is vulnerable, causing her to take a \minus2 penalty on attack rolls, saving throws, checks, DCs, and AC.
\end{spellnotes}

\spellsection{Shout}
\spelldesc{You emit an ear-splitting yell that deafens and damages creatures in its path.}
\spellschool{Evocation (Energy) [Sonic]}
\spelllvl{Destruction 4, Sor/Wiz 4, Strength 4}
\spellcmp{V}
\spellarea{\areamed cone-shaped burst}
\spelldur{Instantaneous}
\spellsave{Fortitude half/Fortitude negates}
\spellsr{Yes (Fortitude)}
\spelldmg{4d6 sonic damage \add d6 per four caster levels above 8th; see text}
\begin{spelleffect}
  Any creature in the area takes damage and is deafened for 5 rounds. A successful save negates the deafness and reduces the damage by half. Any exposed brittle or crystalline object or crystalline creature takes 4d8 points of sonic damage \add d8 per four caster levels above 8th.
\end{spelleffect}

\spellsectioncomma{Shout}{Greater}
\spellschool{Evocation (Energy) [Sonic]}
\spelllvl{Destruction 7, Sor/Wiz 7, Strength 7}
\spellarea{\arealarge cone-shaped burst}
\spellsave{Fortitude partial or Reflex negates (object); See text}
\spelldmg{7d6 sonic damage \add d6 per four caster levels above 14th; see text}
\begin{spelleffect}
  This spell functions like \spell{shout}, except that it is larger and the deafness lasts for 5 rounds. Any exposed brittle or crystalline object or crystalline creature takes 7d8 points of sonic damage \add d8 per four caster levels above 14th.
\end{spelleffect}

\spellsection{Shrink Item}
\spellschool{Transmutation (Alteration)}
\spelllvl{Trans 3}
\spellrng{Touch}
\spelltgt{One Small (or larger) nonmagical object; see text}
\spelldur{24 hours; see text}
\spellsave{Will negates (object)}
\spellsr{Yes (Will)}
\begin{spelleffect}
  You are able to shrink one nonmagical item to 1/16 of its normal size in each dimension (to about 1/4,000 the original volume and mass). This change effectively reduces the object's size by four categories. Optionally, you can also change its now shrunken composition to a clothlike one. Objects changed by a \spell{shrink item} spell can be returned to normal composition and size merely by tossing them onto any solid surface or by a word of command from the original caster. The object must be resting on a stable surface to return to its original size; if the command word is spoken while the object is not stable (such as while it is in the air), the object returns to its original size as soon as it finds a resting point. Even a burning fire and its fuel can be shrunk by this spell. Restoring the shrunken object to its normal size and composition ends the spell.
  \par You can shrink a Medium object at 8th level, a Large object at 12th level, a Huge object at 16th level, or a Gargantuan object at 24th level.
\end{spelleffect}
\begin{spellnotes}
  \spellindirect{shrink item}{Shrink item} can be made permanent with a \spell{permanency} spell, in which case the affected object can be shrunk and expanded an indefinite number of times, but only by the original caster. If you recast the spell each day on an object, you can keep it at its small size indefinitely.
\end{spellnotes}

\spellsection{Silence}
\spellschool{Illusion (Glamer)}
\spelllvl{Clr 2, Trickery 2}
\spellrng{\rngmed}
\spellarea{\areamed radius emanation centered on a creature, object, or point in space}
\spelldur{\durshort (D)}
\spellsave{Will negates; see text or none (object)}
\spellsr{Yes (Will); see text}
\begin{spelleffect}
  Upon the casting of this spell, complete silence prevails in the affected area. No sound can be heard or made in the area, but sound passes through the area normally. Spellcasters are treated as being deafened for the purpose of casting spells with verbal components, and suffer a 20\% chance of spell failure. The spell can be cast on a point in space, but the effect is stationary unless cast on a mobile object. The spell can be centered on a creature, and the effect then radiates from the creature and moves as it moves. An unwilling creature who enters the spell's area can attempt a Will save to negate the spell's effect on them and can use spell resistance, if any. A creature who successfully resists the spell can hear and make sound normally, but still cannot be hear or be heard by other creatures in the area (unless they also resisted the spell). Items in a creature's possession or magic items that emit sound receive the benefits of saves and spell resistance, but unattended objects and points in space do not. 
\end{spelleffect}
\begin{spellnotes}
  This spell provides a defense against sound-dependent effects. Sonic effects are too powerful for magic such as this to muffle, and function normally.
\end{spellnotes}

\spellsection{Silent Image}
\spellschool{Illusion (Figment)}
\spelllvl{Illus 2}
\spellrng{\rngmed}
\spellarea{\areamed radius limit}
\spelleff{Visual figment within the area}
\spelldur{Concentration}
\spellsave{Will disbelief (if interacted with)}
\spellsr{No}
\begin{spelleffect}
  This spell creates the visual illusion of an object, creature, or force, as visualized by you. The illusion does not create sound, smell, texture, or temperature. You can move the image within the limits of the size of the effect.
\end{spelleffect}

\spellsection{Skysmite}
\spelldesc{You call down lightning from the heavens, unerringly striking your foes, even if you cannot see them.}
\spellschool{Evocation (Energy) [Electricity]}
\spelllvl{Air 6, Destruction 6, Drd 6, Sor/Wiz 6}
\spellrng{\rngext}
\spellarea{\arealarge vertical line of lightning, 5 ft. wide, and \areamed radius limit}
\spelldur{Instantaneous}
\spellsave{Reflex half}
\spellsr{Yes (Reflex)}
\spelldmg{12d6 electricity damage \add d6 per two caster levels above 12th}
\begin{spelleffect}
  Lightning strikes where you direct, dealing damage to all creatures and objects in its path. If no creatures or objects lie in its path, the lightning will instead strike the closest occupied square within a \areamed radius limit.
\end{spelleffect}
\begin{spellnotes}
 \spell{Invisibility} and other forms of concealment do not protect creatures from the lightning, but it does not differentiate between friend, foe, and inanimate object.
\end{spellnotes}

\spellsection{Slay Living}
\spelldesc{Your hand seethes with an eerie dark fire as you reach out to touch your foe, instantly snuffing out his life.}
\spellschool{Necromancy (Life) [Death]}
\spelllvl{Clr 6, Death 6}
\spellrng{Touch}
\spelltgt{Living creature touched}
\spelldur{Instantaneous}
\spellsave{Fortitude negates}
\spellsr{Yes (Fortitude)}
\begin{spellhealthy}
  The target is staggered for 5 rounds. It can take a move action or a standard action each round, but not both.
\end{spellhealthy}
\begin{spellblood}
  The target is instantly slain.
\end{spellblood}
\begin{spellnotes}
 A staggered character may take a single move action or standard action each round, but not both. She cannot take full-round actions, but she may take swift actions. In addition, she is vulnerable, causing her to take a \minus2 penalty on attack rolls, saving throws, checks, DCs, and AC.
\end{spellnotes}

\spellsection{Sleep}
\spellschool{Enchantment (Compulsion) [Mind-Affecting, Sleep]}
\spelllvl{Sor/Wiz 1}
\spellrng{\rngmed}
\spelltgt{One living creature}
\spelldur{\durshort}
\spellsave{Will negates}
\spellsr{Yes (Will)}
\begin{spelleffect}
  The subject is fatigued and attempts to go to sleep as soon as possible, though it will not stop fighting to do so. Awakening a creature put to sleep by this spell is difficult, and requires a standard action.
\end{spelleffect}

\spellsectioncomma{Sleep}{Mass}
\spellschool{Enchantment (Compulsion) [Mind-Affecting, Sleep]}
\spelllvl{Sor/Wiz 4}
\spellarea{\areamed radius burst}
\spelltgts{Five creatures within the area}
\begin{spelleffect}
  This spell functions like \spell{sleep}, except that it affects multiple creatures.
\end{spelleffect}

\spellsection{Slow}
\spelldesc{You decelerate your enemy's motions, causing her to move and act more slowly than normal.}
\spellschool{Transmutation (Temporal)}
\spelllvl{Sor/Wiz 2}
\spellrng{\rngclose}
\spelltgt{One creature}
\spelldur{\durshort}
\spellsave{Will negates}
\spellsr{Yes (Will)}
\begin{spelleffect}
  The subject is slowed. This has two effects.
  \par A slowed creature can take only a single move action or standard action each turn, but not both (nor may it take full-round actions).
  \par A slowed creature takes a \minus2 penalty to attack rolls, Strength and Dexterity-based checks, and armor class.
\end{spelleffect}
\begin{spellnotes}
  \spell{Slow} counters and dispels \spell{haste}.
\end{spellnotes}

\spellsectioncomma{Slow}{Mass}
\spelldesc{You decelerate your enemies' motions, causing them to move and act more slowly than normal.}
\spellschool{Transmutation (Temporal)}
\spelllvl{Sor/Wiz 7}
\spellrng{\rngmed}
\spelltgts{Five creatures in an \areamed radius}
\begin{spelleffect}
  This spell functions like \spell{slow}, except that it affects multiple creatures.
\end{spelleffect}

\spellsection{Soften Earth and Stone}
\spellschool{Transmutation (Alteration) [Earth]}
\spelllvl{Drd 2, Earth 2}
\spellrng{\rngclose}
\spellarea{\arealarge radius}
\spelldur{Instantaneous}
\spellsave{None}
\spellsr{No}
\begin{spelleffect}
  When this spell is cast, all natural, undressed earth or stone in the spell's area is softened. Wet earth becomes thick mud, dry earth becomes loose sand or dirt, and stone becomes soft clay that is easily molded or chopped. You affect a 10-foot square area to a depth of 1 to 4 feet, depending on the toughness or resilience of the ground at that spot. Magical, enchanted, dressed, or worked stone cannot be affected. Earth or stone creatures are not affected.
  \par A creature in mud must succeed on a Reflex save or be caught for 1 round and unable to move, attack, or cast spells. A creature that succeeds on its save can move through the mud at half speed, and it can't run or charge.
  \par Loose dirt is not as troublesome as mud, but All creatures within the area can move at only half their normal speed and can't run or charge over the surface.
  \par Stone softened into clay does not hinder movement, but it does allow characters to cut, shape, or excavate areas they may not have been able to affect before.
  \par While \spell{soften earth and stone} does not affect dressed or worked stone, cavern ceilings or vertical surfaces such as cliff faces can be affected. Usually, this causes a moderate collapse or landslide as the loosened material peels away from the face of the wall or roof and falls.
  \par A moderate amount of structural damage can be dealt to a manufactured structure by softening the ground beneath it, causing it to settle. However, most well-built structures will only be damaged by this spell, not destroyed.
\end{spelleffect}

\spellsection{Solid Fog}
\spellschool{Conjuration (Creation)}
\spelllvl{Druid 6, Sor/Wiz 6, Water 6}
\spelldur{\durmed}
\spellsr{No}
\begin{spelleffect}
  This spell functions like \spell{fog cloud}, but in addition to obscuring sight, the fog is so thick that any creature attempting to move through it progresses at a speed of 5 feet, regardless of its normal speed, and it takes a \minus2 penalty on all melee attack and melee damage rolls. The vapors prevent effective ranged weapon attacks (except for magic rays and the like). A creature or object that falls into solid fog is slowed, so that each 10 feet of vapor that it passes through reduces falling damage by 1d6.
  \par A creature in the fog can take a full-round action to make a Strength check, moving 5 feet for every 5 by which the result exceeds DC 0. This movement is affected by any other effects which impede movement, as normal.
\end{spelleffect}
\begin{spellnotes}
  A severe wind (31\add mph) disperses the fog in 5 rounds, and a hurricane force wind disperses the fog in 1 round.
  \par Solid fog can be made permanent with a permanency spell. A permanent solid fog dispersed by wind reforms in 10 minutes.
\end{spellnotes}

\spellsection{Song of Discord}
\spellschool{Enchantment (Compulsion) [Auditory, Mind-Affecting]}
\spelllvl{Brd 6}
\spellrng{\rngmed}
\spellarea{\areamed radius spread}
\spelldur{\durshort}
\spellsave{Will negates}
\spellsr{Yes (Will)}
\begin{spelleffect}
  This spell causes all creatures within the area to turn on each other rather than attack their foes. Each affected creature has a 50\% chance to attack the nearest target each round. (Roll to determine each creature's behavior every round at the beginning of its turn.) A creature that does not attack its nearest neighbor is free to act normally for that round. After each round that a subject is compelled to attack the nearest target, it may make a saving throw to throw off the effect.
  \par Creatures forced by a \spell{song of discord} to attack their fellows employ all methods at their disposal, choosing their deadliest spells and most advantageous combat tactics. They do not, however, harm targets that have fallen unconscious.
\end{spelleffect}
\begin{spellnotes}
  Creatures with HV in excess of your caster level are immune to this spell.
\end{spellnotes}

\spellsection{Soulrend}
\spelldesc{You attack your foe's soul directly.}
\spellschool{Necromancy (Soul)}
\spelllvl{Necro 6}
\spellrng{\rngfar}
\spelltgt{One living creature}
\spelldur{Instantaneous}
\spellsave{Will half}
\spellsr{Yes (Will)}
\begin{spellhealthy}
  The target takes 1 Charisma damage per three caster levels.
\end{spellhealthy}
\begin{spellblood}
  The target takes 1 Charisma damage per two caster levels.
\end{spellblood}
\begin{spellnotes}
  A creature with a Charisma of \minus10 is unable to act. Undead can take Charisma damage from this spell despite being immune to ability damage. 
\end{spellnotes}

\spellsection{Sound Burst}
\spelldesc{You blast an area with a cacophony of sound.}
\spellschool{Evocation (Energy) [Sonic]}
\spelllvl{Brd 2}
\spellrng{\rngclose}
\spellarea{\areasmall radius spread}
\spelldur{Instantaneous}
\spellsave{Fortitude half/Fortitude negates}
\spellsr{Yes (Fortitude)}
\spelldmg{2d6 sonic damage \add d6 per four levels above 4th}
\begin{spelleffect}
  Creatures in the area take damage and are deafened for 5 rounds. A successful Fortitude save halves the damage and negates the deafening.
\end{spelleffect}

\spellsection{Spell Immunity}
\spellschool{Abjuration (Shielding) [Magic]}
\spelllvl{Clr 3, Magic 3, Protection 3, Sor/Wiz 3}
\spellrng{\rngclose}
\spelltgt{One creature}
\spelldur{\durshort}
\spellsave{Will negates (harmless)}
\spellsr{Yes (Will)}
\begin{spelleffect}
  The subject is immune to the effects of one school of magic. Any spell from the chosen school which allows spell resistance simply fails to affect the subject. This applies to spells of 4th level or lower. At 8th level, and every four levels thereafter, the maximum spell level affected increases by one.
\end{spelleffect}
\begin{spellnotes}
   This spell protects against spells, spell-like effects of magic items, and innate spell-like abilities of creatures. It does not protect against supernatural or extraordinary abilities, such as breath weapons or gaze attacks, nor against spells which do not allows spell resistance.
  \par A creature can have only one \spell{spell immunity} effect on it at a time.
\end{spellnotes}

\spellsectioncomma{Spell Immunity}{Greater}
\spellschool{Abjuration (Shielding) [Magic]}
\spelllvl{Clr 8, Magic 8, Protection 8, Sor/Wiz 8}
\begin{spelleffect}
  This spell functions like \spell{spell immunity}, except that it protects the subject from two schools, and the immunity applies to spells of any level.
\end{spelleffect}
\begin{spellnotes}
  A creature can have only one \spell{spell immunity} effect on it at a time.
\end{spellnotes}

\spellsection{Spell Resistance}
\spellschool{Abjuration (Shielding) [Magic]}
\spelllvl{Clr 4, Magic 4, Protection 4, Sor/Wiz 4}
\spellrng{Touch}
\spelltgt{Creature touched}
\spelldur{\durshort}
\spellsave{Will negates (harmless)}
\spellsr{Yes (Will)}
\begin{spelleffect}
  The subject gains spell resistance against all spells.
\end{spelleffect}
\begin{spellnotes}
  A creature with spell resistance may always make a saving throw when a spell is cast on it. If it succeeds, the spell has no effect on it. The type of saving throw made is indicated by the spell. If the spell also allows a saving throw of the same type, only one roll is made.
\end{spellnotes}

\spellsection{Spelltheft}
\spellschool{Abjuration (Negation) [Magic]}
\spelllvl{Abjur 5, Magic 5}
\spelltgt{One spellcaster, creature, or object}
\begin{spelleffect}
  This spell functions like a targeted \spell{dispel magic}, except that you can choose to gain the effects of any spells you dispel as if they had been originally cast on you. The effects last for the remainder of their original durations or for 5 rounds, whichever is shorter. Spells that cannot be cast on you, such as spells which have a range of personal, are simply dispelled.
\end{spelleffect}

\spellsectioncomma{Spelltheft}{Greater}
\spellschool{Abjuration (Negation) [Magic]}
\spelllvl{Abjur 8}
\spelltgt{One spellcaster, creature, or object}
\begin{spelleffect}
  This spell functions like \spell{greater dispel magic}, except that you can choose to gain the effects of any spells you dispel or counterspell as if they had been originally cast on you. The effects last for the remainder of their original durations or for 5 rounds, whichever is shorter. Spells that cannot be cast on you, such as spells which have a range of personal, are simply dispelled.
\end{spelleffect}

\spellsectioncomma{Spelltheft}{Lesser}
\spellschool{Abjuration (Negation) [Magic]}
\spelllvl{Abjur 2, Magic 2}
\spelltgt{One spellcaster, creature, or object}
\begin{spelleffect}
  This spell functions like \spell{lesser dispel magic}, except that you can choose to gain the effects of any spells you dispel as if they had been originally cast on you. The effects last for the remainder of their original durations or for 5 rounds, whichever is shorter. Spells that cannot be cast on you, such as spells which have a range of personal, are simply dispelled.
\end{spelleffect}

\spellsection{Spell Turning}
\spellschool{Abjuration (Shielding) [Magic]}
\spelllvl{Magic 7, Protection 7, Abjur 7}
\spellrng{Personal}
\spelltgt{You}
\spelldur{\durlong or until expended}
\begin{spelleffect}
  Spells and spell-like effects targeted on you are turned back upon the original caster. The abjuration turns only spells that have you as a target. Effect and area spells are not affected. Spell turning also fails to stop touch range spells. 
  \par From seven to ten (1d4\plus6) spell levels are affected by the turning. The exact number is rolled secretly.
  \par When you are targeted by a spell of higher level than the amount of spell turning you have left, that spell is partially turned; both you and the caster each take half damage. For all effects other than damage, there is a 50\% chance that you suffer the effects; otherwise, the caster suffers the effects.
\end{spelleffect}
\begin{spellnotes}
  If you and a spellcasting attacker are both shielded by spell turning effects in operation, a resonating field is created.
  \par Roll randomly to determine the result.
  \begin{dtable}
    \begin{tabularx}{\columnwidth}{l >{\lcol}X}
      \thead{d\%} & \thead{Effect} \\
      01--70 & Spell drains away without effect. \\
      71--80 & Spell affects both of you equally at full effect. \\
      81--97 & Both turning effects are rendered nonfunctional for 1d4 minutes. \\
      98--100 & Both of you go through a rift into another plane.
    \end{tabularx}
  \end{dtable}
\end{spellnotes}

\spellsection{Spider Climb}
\spellschool{Transmutation (Imbuement)}
\spelllvl{Drd 2, Sor/Wiz 2, Travel 2}
\spellrng{Touch}
\spelltgt{Creature touched}
\spelldur{\durmed}
\spellsave{Fortitude negates (harmless)}
\spellsr{Yes (Fortitude)}
\begin{spelleffect}
  The subject can climb and travel on vertical surfaces or even traverse ceilings as well as a spider does. The affected creature must have its hands free to climb in this manner. The subject gains a climb speed of 20 feet; furthermore, it need not make Climb checks to traverse a vertical or horizontal surface (even upside down). A spider climbing creature retains its Dexterity and dodge modifiers to Armor Class (if any) while climbing, and opponents get no special bonus to their attacks against it. It cannot, however, use the run action while climbing.
\end{spelleffect}

\spellsection{Spike Growth}
\spellschool{Transmutation (Alteration)}
\spelllvl{Drd 2}
\spellrng{\rngmed}
\spellarea{\areasmall radius}
\spelldur{\durshort (D)}
\spellsave{None/Reflex negates}
\spellsr{Yes (Reflex)}
\begin{spelleffect}
  Any ground-covering vegetation in the spell's area becomes very hard and sharply pointed. In areas of bare earth, roots and rootlets act in the same way. Typically, spike growth can be cast in any outdoor setting except open water, ice, heavy snow, sandy desert, or bare stone. Any foe moving on foot into or through the spell's area takes 1d4 points of physical piercing damage for each 5 feet of movement through the spiked area. Allies suffer no ill effects.
  \par Any creature that takes damage from this spell must also succeed on a Reflex save or suffer injuries to its feet and legs that slow its land speed by one-half. The Reflex save must be repeated each round that the creature moves through the area. This speed penalty lasts for 12 hours or until the injured creature receives magical healing. Another character can remove the penalty by taking 10 minutes to dress the injuries and succeeding on a Heal check against the spell's save DC.
\end{spelleffect}

\spellsection{Spike Stones}
\spellschool{Transmutation (Alteration)}
\spelllvl{Drd 4}
\spellarea{\areamed radius}
\begin{spelleffect}
  This spell functions like \spell{spike growth}, except that it deals 1d8 physical piercing damage to creatures moving through it and it can also be cast on rocky ground, stone floors, and similar surfaces.
\end{spelleffect}

%need to run expected damage calculation
\spellsection{Spiritual Weapon}
\spelldesc{You bring into being a weapon made of pure force which attacks your foes of its own volition.}
\spellschool{Evocation (Energy) [Force]}
\spelllvl{Clr 2, Pal 2, War 2}
\spellrng{\rngmed}
\spelleff{Magic weapon of force}
\spelldur{\durshort (D)}
\spellsave{None}
\spellsr{Yes (Will)}
\begin{spelleffect}
  The weapon created by this spell attacks once each round on your turn. This functions just as if you were attacking with the weapon, except that you use your casting ability in place of your Strength and you never get multiple attacks with the weapon.
  \par The weapon attacks the same target until you redirect it (a swift action). The weapon is treated as a separate creature for the purpose of overwhelm penalties.
  \par If an attacked creature has spell resistance, you make a spell penetration check the first time the spiritual weapon strikes it. If the weapon is successfully resisted, it cannot harm that creature. If not, the weapon has its normal full effect on that creature for the duration of the spell.
  \par The weapon takes the shape of a weapon favored by your deity or a weapon with some spiritual significance or symbolism to you (see below), and has the same threat range and critical multipliers as a real weapon of its form.
\end{spelleffect}
\begin{spellnotes}
  The \spell{spiritual weapon} strikes as a spell, not as a weapon, so, for example, ignores physical damage reduction. As a force effect, it can strike incorporeal creatures without the normal miss chance associated with incorporeality. If the weapon goes beyond the spell range, if it goes out of your sight, or if you are not directing it, the weapon returns to you and hovers. Even if the spiritual weapon is a ranged weapon, use the spell's range, not the weapon's normal range increment, and switching targets still is a move action.
  \par A \spell{spiritual weapon} cannot be attacked or harmed by physical attacks, but \spell{dispel magic}, \spell{disintegrate}, and similar effects can affect it. A spiritual weapon's AC against touch attacks is 12 (10 \add size bonus for Tiny object).
  \par The weapon that you get is usually a force replica of any weapon from your deity's weapon group. A cleric without a deity gets a weapon based on his alignment. A neutral cleric without a deity can create a spiritual weapon of any alignment, provided he is acting at least generally in accord with that alignment at the time. The weapon groups associated with each alignment are as follows.
  \par Chaos: Axes
  \par Evil: Flexible weapons
  \par Good: Headed weapons
  \par Law: Heavy blades
\end{spellnotes}

\spellsection{Stampede}
\spellschool{Conjuration (Summoning)}
\spelllvl{Drd 9, Nature 9}
\spelltime{Full-round action}
\spellrng{\rngfar}
\spellarea{\arealarge radius limit}
\spelleff{Nine or more Large summoned creatures within the area}
\spelldur{\durshort (D)}
\spellsave{Reflex half; see text}
\spellsr{No}
\spelldmg{9d6 bludgeoning damage \add d6 per four levels above 18th}
\begin{spelleffect}
  This spell summons a stampede of nine bison to trample your foes. Creatures trampled by the herd of bison take 1d6 damage per bison in the herd. You can summon one additional bison per four levels above 18th.
  \par The bison are summoned in a place that you designate within the spell's area, with each creature being summoned in the closest free space to the point of origin. If there is insufficient room for all of the bison to appear while standing on stable ground, the spell will summon fewer bison than the maximum. The herd of bison always moves directly away from you, trampling anything of Large size or smaller that gets in their way. If the herd is thinned to fewer than 5 bison, they stop stampeding and scatter in random directions.
  \par The bison do not attack, even if cornered; they will only stampede. At the end of the spell's duration, the bison disappear.
\end{spelleffect}
\begin{spellnotes}
  Under normal circumstances, the bison can travel 800 feet over the duration of the spell.
\end{spellnotes}

\spellsection{Stinking Cloud}
\spellschool{Conjuration/Necromancy (Creation, Flesh)}
\spelllvl{Sor/Wiz 5}
\spellsave{None/Fortitude negates}
\spellsr{None/Yes (Fortitude)}
\begin{spelleffect}
  This spell functions like \spell{fog cloud}, except that creatures within the cloud are sickened, making them vulnerable. A successful Fortitude save negates the sickening. The condition lasts as long as the creature remains in the cloud and for 5 rounds after it leaves. Any creature that succeeds on its save but remains in the cloud must continue to save each round on your turn.
\end{spelleffect}
\begin{spellnotes}
  A vulnerable creature takes a \minus2 penalty to attack rolls, saving throws, checks, DCs, and AC.
  
  \spellindirect{stinking cloud}{Stinking cloud} can be made permanent with a \spell{permanency} spell. A permanent \spell{stinking cloud} dispersed by wind reforms in 10 minutes.
\end{spellnotes}

\spellsection{Stoneskin}
\spelldesc{You dramatically toughen a creature's skin, giving it the appearance of stone.}
\spellschool{Transmutation (Alteration) [Earth]}
\spelllvl{Drd 4, Earth 4, Protection 4, Trans 4}
\begin{spelleffect}
  This spell functions like \spell{barkskin}, except that it grants physical damage reduction 8/adamantine. This damage reduction increases by 1 per two caster levels above 8th.
\end{spelleffect}
\begin{spellnotes}
  This spell's damage reduction allows the subject to ignore the first 8 physical damage it takes each round. If it is hit by an adamantine weapon, it cannot use its damage reduction for 1 round.
\end{spellnotes}

\spellsection{Storm of Vengeance}
\spellschool{Conjuration/Evocation (Energy, Control, Creation)}
\spelllvl{Air 9, Drd 9, Clr 9, War 9, Water 9}
\spelltime{Full-round action}
\spellrng{\rngfar}
\spellarea{360 ft. radius cylinder, 200 ft. high}
\spelleff{Supernatural weather in the area}
\spelldur{Concentration (maximum 10 rounds)}
\spellsave{See text}
\spellsr{Yes (varies)}
\spelldmg{Varies}
\begin{spelleffect}
  This spell creates an enormous black storm cloud. Lightning and crashing claps of thunder appear within the storm. Each creature beneath the cloud must succeed on a Fortitude save or be deafened for 5 minutes. Violent rain and wind gusts obscure all sight beyond 100 feet. A creature less than 100 feet away has concealment (\plus4 AC). Ranged attacks within the area of the storm take a \minus4 penalty, and spells cast within the area are disrupted unless the caster succeeds on a Concentration check against a DC equal to 20 \add double the level of the spell.
  \par If you do not maintain concentration on the spell after casting it, the spell ends. If you continue to concentrate, the spell generates new effects in each following round, as noted below. Each effect occurs during your turn.
  \par 2nd Round: Acid rains down, dealing 1d10 acid damage to everything in the area (no save).
  \par 3rd Round: You call three bolts of lightning down from the cloud. You decide where the bolts strike. No two bolts may strike the same target. Each bolt deals 9d6 electricity damage \add d6 per four levels after 18th. A creature struck can attempt a Reflex save for half damage. If you do not direct the lightning bolts, each bolt automatically targets the largest available target in the area.
  \par 4th Round: Hailstones rain down, dealing 5d6 bludgeoning damage to all enemies in the area.
  \par 5th through 10th Rounds: Acid rains down, dealing 1d10 damage to everything in the area (no save).
\end{spelleffect}

\spellsection{Stormlord}
\spelldesc{You surround yourself in a whirlwind which deflects ranged attacks and batters your foes.} 
\spellschool{Abjuration/Evocation (Control, Shielding)}
\spelllvl{Air 7, Drd 7}
\spellrng{Personal}
\spelltgt{You}
\spelldur{\durshort (D)}
\spellsave{None/Fortitude half}
\spellsr{None/Yes (Fortitude)} 
\spelldmg{7d6 bludgeoning damage \add d6 per four levels above 14th}
\begin{spelleffect}
  You gain physical damage reduction 35 against ranged attacks such as projectile weapons and thrown weapons. This damage reduction increases by 1 per caster level above 14th. In addition, any creature that hits you with its body or a melee weapon takes damage. Each individual creature can take this damage only once per round.
\end{spelleffect}
\begin{spellnotes}
    This spell's damage reduction allows the subject to ignore the first 35 physical damage it takes each round from ranged attacks. The saving throw and spell resistance apply against the damage dealt, but not against this spell's other effects.
\end{spellnotes}

\spellsection{Strip the Flesh}
\spelldesc{You rend parts of your foe's skin off its body, inflicting grievous wounds and leaving it vulnerable.}
\spellschool{Necromancy (Flesh)}
\spelllvl{Sor/Wiz 7}
\spellrng{\rngclose}
\spelltgt{One creature}
\spelldur{Instantaneous/5 rounds}
\spellsave{None/Fortitude negates}
\spellsr{Yes (Fortitude)}
\spelldmg{7d10 physical damage \add d10 per four caster levels above 14th}
\begin{spelleffect}
  The target takes damage. In addition, if it fails a Fortitude save, for 5 rounds all damage it takes is doubled. This does not double the initial damage dealt by this spell.
\end{spelleffect}
\begin{spellnotes}
  A successful Heal check with a DC equal to this spell's save DC negates the doubling of damage. 
\end{spellnotes}

\spellsection{Suggestion}
\spellschool{Enchantment (Compulsion) [Language-Dependent, Mind-Affecting, Sound-Dependent]}
\spelllvl{Sor/Wiz 5}
\spellcmp{V, M}
\spellrng{\rngclose}
\spelltgt{One living creature}
\spelldur{\durext or until completed}
\spellsave{Will negates}
\spellsr{Yes (Will)}
\begin{spelleffect}
  You influence the actions of the target creature by suggesting a course of activity (limited to a sentence or two). The suggestion must be worded in such a manner as to make the activity sound reasonable. Asking the creature to do some obviously harmful act automatically negates the effect of the spell. Additionally, any obvious threat, such as someone drawing a weapon, casting a spell, or aiming a ranged weapon at the fascinated creature, grants the creature a new saving throw with a \plus5 bonus.
  \par The suggested course of activity can continue for the entire duration. If the suggested activity can be completed in a shorter time, the spell ends when the subject finishes what it was asked to do. You can instead specify conditions that will trigger a special activity during the duration. If the condition is not met before the spell duration expires, the activity is not performed.
\end{spelleffect}
\begin{spellnotes}
  A very reasonable suggestion can cause the save to be made with a \minus2 or greater penalty. A creature that makes its saving throw against \spell{suggestion} is immune to all further attempts by the same spellcaster for 24 hours.
\end{spellnotes}

\spellsectioncomma{Suggestion}{Mass}
\spellschool{Enchantment (Compulsion) [Language-Dependent, Mind-Affecting, Sound-Dependent]}
\spelllvl{Sor/Wiz 8}
\spelldur{\durshort}
\spellarea{\areamed radius limit}
\spelltgts{Five creatures within the area}
\begin{spelleffect}
  This spell functions like \spell{suggestion}, except that it can affect multiple creatures and has a shorter duration. The same suggestion applies to all subjects.
\end{spelleffect}

\spellsection{Summon Monster I}
\spellschool{Conjuration (Summoning) [see text]}
\spelllvl{Clr 1, Sor/Wiz 1}
\spelltime{Full-round action}
\spellrng{\rngclose}
\spelleff{One summoned creature}
\spelldur{\durshort (D)}
\spellsave{None}
\spellsr{No}
\begin{spelleffect}
  This spell summons an extraplanar creature (typically an outsider, elemental, or magical beast native to another plane). It appears where you designate and acts on your next turn. You must spend a swift action each round to control the creature summoned by this spell. If you do, it attacks your opponents to the best of its ability. You can direct the creature not to attack, to attack particular enemies, or to perform other actions if you can communicate with it. If you do not actively control the creature summoned by this spell, it acts according to its nature.
  \par When you learn this spell, you choose a creature from the 1st-level list on the Summon Monster table. In the case of creatures with multiple options, such as elementals, you must choose one specific kind of creature. You can summon that creature with this or any other summon monster spell.
  \par A summoned monster cannot summon or otherwise conjure another creature, nor can it use any teleportation or planar travel abilities. Creatures cannot be summoned into an environment that cannot support them.
  \par When you use a summoning spell to summon an air, chaotic, earth, evil, fire, lawful, or water creature, it is a spell of that type.
\end{spelleffect}

\spellsection{Summon Monster II}
\spellschool{Conjuration (Summoning) [see text for summon monster I]}
\spelllvl{Clr 2, Conj 1, Sor/Wiz 2}
\spellarea{\areamed radius limit}
\spelleff{One or more summoned creatures within the area}
\begin{spelleffect}
  This spell functions like \spell{summon monster I}, except that you can summon one creature from the 2nd-level list or 1d3 creatures of the same kind from the 1st-level list. When you learn this spell, you choose two creatures from the 2nd-level or lower lists on the Summon Monster table, one at each level. You can summon those creatures with this or any other \spellindirect{summon monster I}{summon monster} spell.
\end{spelleffect}

\spellsection{Summon Monster III}
\spellschool{Conjuration (Summoning) [see text for summon monster I]}
\spelllvl{Chaos 3, Clr 3, Conj 2, Evil 3, Good 3, Law 3, Sor/Wiz 3}
\spellarea{\areamed radius limit}
\spelleff{One or more summoned creatures within the area}
\begin{spelleffect}
  This spell functions like \spell{summon monster I}, except that you can summon one creature from the 3rd-level list or 1d3 creatures of the same kind from a lower-level list. When you learn this spell, you choose three creatures from the 3rd-level or lower lists on the Summon Monster table, one at each level. You can summon those creatures with this or any other \spellindirect{summon monster I}{summon monster} spell.
\end{spelleffect}

\spellsection{Summon Monster IV}
\spellschool{Conjuration (Summoning) [see text for summon monster I]}
\spelllvl{Clr 4, Conj 3, Sor/Wiz 4}
\spellarea{\areamed radius limit}
\spelleff{One or more summoned creatures within the area}
\begin{spelleffect}
  This spell functions like \spell{summon monster I}, except that you can summon one creature from the 4th-level list or 1d3 creatures of the same kind from a lower-level list. When you learn this spell, you choose four creatures from the 4th-level or lower lists on the Summon Monster table, one at each level. You can summon those creatures with this or any other \spellindirect{summon monster I}{summon monster} spell.
\end{spelleffect}

\spellsection{Summon Monster V}
\spellschool{Conjuration (Summoning) [see text for summon monster I]}
\spelllvl{Air 4, Clr 5, Conj 4, Earth 4, Fire 4, Sor/Wiz 5, Water 4}
\spellarea{\areamed radius limit}
\spelleff{One or more summoned creatures within the area}
\begin{spelleffect}
  This spell functions like \spell{summon monster I}, except that you can summon one creature from the 5th-level list or 1d3 creatures of the same kind from a lower-level list. When you learn this spell, you choose five creatures from the 5th-level or lower lists on the Summon Monster table, one at each level. You can summon those creatures with this or any other \spellindirect{summon monster I}{summon monster} spell.
\end{spelleffect}

\spellsection{Summon Monster VI}
\spellschool{Conjuration (Summoning) [see text for summon monster I]}
\spelllvl{Chaos 6, Clr 6, Conj 5, Evil 6, Good 6, Law 6, Sor/Wiz 6}
\spellarea{\areamed radius limit}
\spelleff{One or more summoned creatures within the area}
\begin{spelleffect}
  This spell functions like \spell{summon monster I}, except you can summon one creature from the 6th-level list or 1d3 creatures of the same kind from a lower-level list. When you learn this spell, you choose six creatures from the 6th-level or lower lists on the Summon Monster table, one at each level. You can summon those creatures with this or any other \spellindirect{summon monster I}{summon monster} spell.
\end{spelleffect}

\spellsection{Summon Monster VII}
\spellschool{Conjuration (Summoning) [see text for summon monster I]}
\spelllvl{Clr 7, Conj 6, Sor/Wiz 7}
\spellarea{\areamed radius limit}
\spelleff{One or more summoned creatures within the area}
\begin{spelleffect}
  This spell functions like \spell{summon monster I}, except that you can summon one creature from the 7th-level list or 1d3 creatures of the same kind from a lower-level list. When you learn this spell, you choose seven creatures from the 7th-level or lower lists on the Summon Monster table, one at each level. You can summon those creatures with this or any other \spellindirect{summon monster I}{summon monster} spell.
\end{spelleffect}

\spellsection{Summon Monster VIII}
\spellschool{Conjuration (Summoning) [see text for summon monster I]}
\spelllvl{Air 7, Clr 8, Conj 7, Earth 7, Fire 7, Sor/Wiz 8, Water 7}
\spellarea{\areamed radius limit}
\spelleff{One or more summoned creatures within the area}
\begin{spelleffect}
  This spell functions like \spell{summon monster I}, except that you can summon one creature from the 8th-level list or 1d3 creatures of the same kind from a lower-level list. When you learn this spell, you choose eight creatures from the 8th-level or lower lists on the Summon Monster table, one at each level. You can summon those creatures with this or any other \spellindirect{summon monster I}{summon monster} spell.
\end{spelleffect}

\spellsection{Summon Monster IX}
\spellschool{Conjuration (Summoning) [see text for summon monster I]}
\spelllvl{Chaos 9, Clr 9, Conj 8, Evil 9, Good 9, Law 9, Sor/Wiz 9}
\spellarea{\areamed radius limit}
\spelleff{One or more summoned creatures within the area}
\begin{spelleffect}
  This spell functions like \spell{summon monster I}, except that you can summon one creature from the 9th-level list or 1d3 creatures of the same kind from a lower-level list. When you learn this spell, you choose nine creatures from the 9th-level or lower lists on the Summon Monster table, one at each level. You can summon those creatures with this or any other \spellindirect{summon monster I}{summon monster} spell.

  \begin{dtable!*}
    \lcaption{Summon Monster List}
    \begin{tabularx}{\textwidth}{>{\lcol}X c >{\lcol}X c >{\lcol}X c}
      \thead{1st Level} &  & \thead{4th Level} &  & Fiendish monstrous spider, Huge & CE \\
      Celestial dog & LG & Archon, lantern & LG & Fiendish snake, giant constrictor & CE \\
      Celestial owl & LG & Celestial giant owl & LG &  &  \\
      Celestial giant fire beetle & NG & Celestial giant eagle & CG & \thead{7th Level} &  \\
      Celestial porpoise\fn{1} & NG & Celestial lion & CG & Celestial elephant & LG \\
      Celestial badger & CG & Mephit (any) & N & Avoral (guardinal) & NG \\
      Celestial monkey & CG & Fiendish dire wolf & LE & Celestial baleen whale\fn{1} & NG \\
      Fiendish dire rat & LE & Fiendish giant wasp & LE & Djinni (genie) & CG \\
      Fiendish raven & LE & Fiendish giant praying mantis & NE & Elemental, Huge (any) & N \\
      Fiendish monstrous centipede, Medium & NE & Fiendish shark, Large\fn{1} & NE & Invisible stalker & N \\
      Fiendish monstrous scorpion, Small & NE & Yeth hound & NE & Devil, bone & LE \\
      Fiendish hawk & CE & Fiendish monstrous spider, Large & CE & Fiendish megaraptor & LE \\
      Fiendish monstrous spider, Small & CE & Fiendish snake, Huge viper & CE & Fiendish monstrous scorpion, Huge & \\ NE
      Fiendish octopus\fn{1} & CE & Howler & CE & Babau (demon) & CE \\
      Fiendish snake, Small viper & CE &  &  & Fiendish giant octopus\fn{1} & CE \\
      &  & \thead{5th Level} &  & Fiendish girallon & CE \\
      \thead{2nd Level} &  & Archon, hound & K &  &  \\
      Celestial giant bee & LG & Celestial brown bear & LG &  &  \\
      Celestial giant bombardier beetle & NG & Celestial giant stag beetle & LG & \thead{8th Level} &  \\
      Celestial riding dog & NG & Celestial sea cat\fn{1} & NG & Celestial dire bear & LG \\
      Celestial eagle & CG & Celestial griffon & NG & Celestial cachalot whale\fn{1} & NG \\
      Lemure (devil) & LE & Elemental, Medium (any) & CG & Celestial triceratops & NG \\
      Fiendish squid\fn{1} & LE & Achaierai & N & Lillend & CG \\
      Fiendish wolf & LE & Devil, bearded & LE & Elemental, greater (any) & N \\
      Fiendish monstrous centipede, Large & NE & Fiendish deinonychus & LE & Fiendish giant squid\fn{1} & LE \\
      Fiendish monstrous scorpion, Medium & NE & Fiendish dire ape & LE & Hellcat & LE \\
      Fiendish shark, Medium\fn{1} & NE & Fiendish dire boar & LE & Fiendish monstrous centipede, Colossal & NE \\
      Fiendish monstrous spider, Medium & CE & Fiendish shark, Huge & NE & Fiendish dire tiger & CE \\
      Fiendish snake, Medium viper & CE & Fiendish monstrous scorpion, Large & NE & Fiendish monstrous spider, Gargantuan & CE \\
      &  & Shadow mastiff & NE & Fiendish tyrannosaurus & CE \\
      \thead{3rd Level} &  & Fiendish dire wolverine & NE & Vrock (demon) & CE \\
      Celestial black bear & LG & Fiendish giant crocodile & CE &  &  \\
      Celestial bison & NG & Fiendish tiger & CE &  &  \\
      Celestial dire badger & CG &  &  & \thead{9th Level} &  \\
      Celestial hippogriff & CG & \thead{6th Level} &  & Couatl & LG \\
      Elemental, Small (any) & N & Celestial polar bear & LG & Leonal (guardinal) & NG \\
      Fiendish ape & LE & Celestial orca whale\fn{1} & NG & Celestial roc & CG \\
      Fiendish dire weasel & LE & Bralani (eladrin) & CG & Elemental, elder (any) & N \\
      Hell hound & LE & Celestial dire lion & CG & Devil, barbed & LE \\
      Fiendish snake, constrictor  & LE & Elemental, Large (any) & N & Fiendish dire shark\fn{1} & NE \\
      Fiendish boar & NE & Janni (genie) & N & Fiendish monstrous scorpion, Gargantuan & NE \\
      Fiendish dire bat & NE & Chaos beast & CN & Night hag & NE \\
      Fiendish monstrous centipede, Huge & NE & Devil, chain & LE & Bebilith (demon) & CE \\
      Fiendish crocodile & CE & Xill & LE & Fiendish monstrous spider, Colossal & CE \\
      Dretch (demon) & CE & Fiendish monstrous centipede, Gargantuan & NE & Hezrou (demon) & CE \\
      Fiendish snake, Large viper & CE & Fiendish rhinoceros & NE & & \\
      Fiendish wolverine & CE & Fiendish elasmosaurus\fn{1} & CE & &
    \end{tabularx}
    1 May be summoned only into an aquatic or watery environment.
  \end{dtable!*}
\end{spelleffect}

\spellsection{Summon Nature's Ally I}
\spellschool{Conjuration (Summoning)}
\spelllvl{Drd 1}
\spelltime{Full-round action}
\spellrng{\rngclose}
\spelleff{One summoned creature}
\spelldur{\durshort (D)}
\spellsave{None}
\spellsr{No}
\begin{spelleffect}
  This spell summons a natural creature. It appears where you designate and acts on your next turn. You must spend a swift action each round to control the creature summoned by this spell. If you do, it attacks your opponents to the best of its ability. You can direct the creature not to attack, to attack particular enemies, or to perform other actions if you can communicate with it. If you do not actively control the creature summoned by this spell, it acts according to its nature.
  \par When you learn this spell, you choose a creature from the 1st-level list on the Summon Nature's Ally table. In the case of creatures with multiple options, such as elementals, you must choose one specific kind. You can summon that creature with this or any other \spellindirect{summon nature's ally i}{summon nature's ally} spell. 
  \par A summoned monster cannot summon or otherwise conjure another creature, nor can it use any teleportation or planar travel abilities. Creatures cannot be summoned into an environment that cannot support them.
  \par All the creatures on the table are neutral unless otherwise noted.
\end{spelleffect}

\spellsection{Summon Nature's Ally II}
\spellschool{Conjuration (Summoning)}
\spelllvl{Drd 2}
\spellarea{\areamed radius limit}
\spelleff{One or more summoned creatures within the area}
\begin{spelleffect}
  This spell functions like \spellindirect{summon nature's ally i}{summon nature's ally I}, except that you can summon one 2nd-level creature or 1d3 1st-level creatures of the same kind. When you learn this spell, you choose two creatures from the 2nd-level or lower lists on the Summon Nature's Ally table, one at each level. You can summon those creatures with this or any other \spellindirect{summon nature's ally i}{summon nature's ally} spell.
\end{spelleffect}

\spellsection{Summon Nature's Ally III}
\spellschool{Conjuration (Summoning) [see text]}
\spelllvl{Drd 3, Nature 3}
\spellarea{\areamed radius limit}
\spelleff{One or more summoned creatures within the area}
\begin{spelleffect}
  This spell functions like \spellindirect{summon nature's ally i}{summon nature's ally I}, except that you can summon one 3rd-level creature, 1d3 2nd-level creatures of the same kind, or 1d4\plus1 1st-level creatures of the same kind. When you learn this spell, you choose three creatures from the 3rd-level or lower lists on the Summon Nature's Ally table, one at each level. You can summon those creatures with this or any other \spellindirect{summon nature's ally i}{summon nature's ally} spell.
\end{spelleffect}

\spellsection{Summon Nature's Ally IV}
\spellschool{Conjuration (Summoning) [see text]}
\spelllvl{Drd 4}
\spellarea{\areamed radius limit}
\spelleff{One or more summoned creatures within the area}
\begin{spelleffect}
  This spell functions like \spellindirect{summon nature's ally i}{summon nature's ally I}, except that you can summon one 4th-level creature or 1d3 creatures of the same kind from a lower-level list. When you learn this spell, you choose four creatures from the 4th-level or lower lists on the Summon Nature's Ally table, one at each level. You can summon those creatures with this or any other \spellindirect{summon nature's ally i}{summon nature's ally} spell.
\end{spelleffect}

\spellsection{Summon Nature's Ally V}
\spellschool{Conjuration (Summoning) [see text]}
\spelllvl{Drd 5}
\spellarea{\areamed radius limit}
\spelleff{One or more summoned creatures within the area}
\begin{spelleffect}
  This spell functions like \spellindirect{summon nature's ally i}{summon nature's ally I}, except that you can summon one 5th-level creature or 1d3 creatures of the same kind from a lower-level list. When you learn this spell, you choose five creatures from the 5th-level or lower lists on the Summon Nature's Ally table, one at each level. You can summon those creatures with this or any other \spellindirect{summon nature's ally i}{summon nature's ally} spell.
\end{spelleffect}

\spellsection{Summon Nature's Ally VI}
\spellschool{Conjuration (Summoning) [see text]}
\spelllvl{Drd 6, Nature 6}
\spellarea{\areamed radius limit}
\spelleff{One or more summoned creatures within the area}
\begin{spelleffect}
  This spell functions like \spellindirect{summon nature's ally i}{summon nature's ally I}, except that you can summon one 6th-level creature or 1d3 creatures of the same kind from a lower-level list. When you learn this spell, you choose six creatures from the 6th-level or lower lists on the Summon Nature's Ally table, one at each level. You can summon those creatures with this or any other summon nature's ally spell.
\end{spelleffect}

\spellsection{Summon Nature's Ally VII}
\spellschool{Conjuration (Summoning) [see text]}
\spelllvl{Drd 7}
\spellarea{\areamed radius limit}
\spelleff{One or more summoned creatures within the area}
\begin{spelleffect}
  This spell functions like \spellindirect{summon nature's ally i}{summon nature's ally I}, except that you can summon one 7th-level creature or 1d3 creatures of the same kind from a lower-level list. When you learn this spell, you choose seven creatures from the 7th-level or lower lists on the Summon Nature's Ally table, one at each level. You can summon those creatures with this or any other \spellindirect{summon nature's ally i}{summon nature's ally} spell.
\end{spelleffect}

\spellsection{Summon Nature's Ally VIII}
\spellschool{Conjuration (Summoning) [see text]}
\spelllvl{Drd 8}
\spellarea{\areamed radius limit}
\spelleff{One or more summoned creatures within the area}
\begin{spelleffect}
  This spell functions like \spellindirect{summon nature's ally i}{summon nature's ally I}, except that you can summon one 8th-level creature or 1d3 creatures of the same kind from a lower-level list. When you learn this spell, you choose eight creatures from the 8th-level or lower lists on the Summon Nature's Ally table, one at each level. You can summon those creatures with this or any other \spellindirect{summon nature's ally i}{summon nature's ally} spell.
\end{spelleffect}

\spellsection{Summon Nature's Ally IX}
\spellschool{Conjuration (Summoning) [see text]}
\spelllvl{Drd 9, Nature 9}
\spellarea{\areamed radius limit}
\spelleff{One or more summoned creatures within the area}
\begin{spelleffect}
  This spell functions like \spellindirect{summon nature's ally i}{summon nature's ally I}, except that you can summon one 9th-level creature or 1d3 creatures of the same kind from a lower-level list. When you learn this spell, you choose nine creatures from the 9th-level or lower lists on the Summon Nature's Ally table, one at each level. You can summon those creatures with this or any other \spellindirect{summon nature's ally i}{summon~nature's~ally} spell.

  \begin{dtable*}
    \lcaption{Summon Nature's Ally List}
    \begin{tabularx}{\textwidth}{>{\lcol}X >{\lcol}X >{\lcol}X >{\lcol}X}
      \thead{1st Level} & Eagle, giant [NG] & \thead{5th Level} & \thead{7th Level} \\
      Dire rat & Lion & Arrowhawk, adult & Arrowhawk, elder \\
      Eagle (animal) & Owl, giant [NG] & Bear, polar (animal) & Dire tiger \\
      Monkey (animal) & Satyr [CN; without pipes] & Dire lion & Elemental, greater (any) \\
      Octopus\fn{1} (animal) & Shark, Large\fn{1} (animal) & Elasmosaurus\fn{1} (dinosaur) & Djinni (genie) [NG] \\
      Owl (animal) & Snake, constrictor (animal) & Elemental, Large (any) & Invisible stalker \\
      Porpoise\fn{1} (animal) & Snake, Large viper (animal) & Griffon & Pixie\fn{2} (sprite) [NG; with sleep arrows] \\
      Snake, Small viper (animal) & Thoqqua & Janni (genie) & Squid, giant\fn{1} (animal) \\
      Wolf (animal) &  & Rhinoceros (animal) & Triceratops (dinosaur) \\
      & \thead{4th Level} & Satyr [CN; with pipes] & Tyrannosaurus (dinosaur) \\
      \thead{2nd Level} & Arrowhawk, juvenile & Snake, giant constrictor (animal) & Whale, cachalot\fn{1} (animal) \\
      Bear, black (animal) & Bear, brown (animal) & Nixie (sprite) & Xorn, elder \\
      Crocodile (animal) & Crocodile, giant (animal) & Tojanida, adult\fn{1} &  \\
      Dire badger & Deinonychus (dinosaur) & Whale, orca\fn{1} (animal) & \thead{8th Level} \\
      Dire bat & Dire ape &  & Dire shark\fn{1} \\
      Elemental, Small (any) & Dire boar & \thead{6th Level} & Roc \\
      Hippogriff & Dire wolverine & Dire bear & Salamander, noble [NE] \\
      Shark, Medium\fn{1} (animal) & Elemental, Medium (any) & Elemental, Huge (any) & Tojanida, elder \\
      Snake, Medium viper (animal) & Salamander, flamebrother [NE] & Elephant (animal) &  \\
      Squid\fn{1} (animal) & Sea cat\fn{1} & Girallon & \thead{9th Level} \\
      Wolverine (animal) & Shark, Huge\fn{1} (animal) & Megaraptor (dinosaur) & Elemental, elder \\
      & Snake, Huge viper (animalo) & Octopus, giant\fn{1} (animal) & Grig [NG; with fiddle] (sprite) \\
      \thead{3rd Level} & Tiger (animal) & Pixie\fn{2} (sprite) [NG; no special arrows] & Pixie\fn{3} (sprite) [NG; with sleep and memory loss arrows] \\
      Ape (animal) & Tojanida, juvenile\fn{1} & Salamander, average [NE] & Unicorn, celestial charger \\
      Dire weasel & Unicorn [CG] & Whale, baleen\fn{1} &  \\
      Dire wolf & Xorn, minor & Xorn, average & 
    \end{tabularx}
    1 May be summoned only into an aquatic or watery environment. \\
    2 Can't cast irresistible dance \\
    3 Can cast irresistible dance
  \end{dtable*}
\end{spelleffect}

\spellsection{Summon Nature's Army}
\spellschool{Conjuration (Summoning)}
\spelllvl{Drd 8, Nature 8}
\spellarea{\areamed radius limit}
\spelleff{One or more summoned creatures within the area}
\begin{spelleffect}
  This spell functions like \spellindirect{summon nature's ally i}{summon nature's ally I}, except that you can summon up to one creature per caster level from the 4th-level list or lower.
  \par When you learn this spell, you choose a creature from the 4th-level list or lower on the Summon Nature's Ally table. You can only summon that creature with this spell.
\end{spelleffect}

\spellsection{Sunbeam}
\spelldesc{You evoke a dazzling beam of intense light, blinding your foes with the power of the sun itself.}
\spellschool{Evocation (Control) [Light]}
\spelllvl{Drd 5}
\spellarea{\arealarge line}
\spelldur{Instantaneous/5 rounds}
\spellsave{Reflex half/Reflex negates}
\spellsr{Yes (Reflex)}
\spelldmg{5d6 solar damage \add d6 per four caster levels above 10th; see text}
\begin{spelleffect}
  Each creature in the beam takes damage and is dazzled for 5 rounds. Any creatures to which sunlight is harmful or unnatural instead take 5d10 points of damage \add d10 per four caster levels above 10th and are blinded for 5 rounds. A successful Reflex save negates the dazzling (or blindness) and reduces the damage by half.
\end{spelleffect}
\begin{spellnotes}
  A dazzled creature has a 20\% miss chance on all attack rolls and takes a \minus4 penalty to Spot checks. He is also unable to see with darkvision.
  
  \spell{Sunbeam} dispels any darkness spells of 5th level or lower within its area.
\end{spellnotes}

\spellsection{Sunburst}
\spelldesc{You cause a globe of searing radiance to explode silently from a point you select.}
\spellschool{Evocation (Control) [Light]}
\spelllvl{Drd 8}
\spellrng{\rngmed}
\spellarea{\areamed radius burst}
\spelldmg{8d6 solar damage \add d6 per four caster levels above 16th; see text}
\begin{spelleffect}
  This spell functions as \spell{sunbeam}, except that it affects a \areamed radius and deals more damage. Any creatures to which sunlight is harmful or unnatural take 8d10 points of damage \add d10 per three caster levels above 16th.
\end{spelleffect}
\begin{spellnotes}
  \spell{Sunburst} dispels any darkness spells of 8th level or lower within its area.
\end{spellnotes}

\pdfbookmark[2]{T}{SpellDescriptionsT}
\begin{comment}
\subsubsection{T}
\end{comment}

\spellsection{Telekinesis}
\spelldesc{You move objects or creatures by concentrating on them.}
\spellschool{Evocation (Control)}
\spelllvl{Evoc 6}
\spellrng{\rngmed}
\spelltgtortgts{See text}
\spelldur{Concentration, up to \durmed/Instantaneous; see text}
\spellsave{Will negates (object)/None; see text}
\spellsr{Yes (Will)/None; see text}
\begin{spelleffect}
  Depending on the version selected, the spell can provide a gentle, sustained force, perform a variety of combat maneuvers, or exert a single short, violent thrust.
  \par \subspell{Sustained Force} As the \spell{telekinetic force} spell.

  \par \subspell{Combat Maneuver} As the \spell{telekinetic maneuver} spell.

  \par \subspell{Violent Thrust} Alternatively, the spell energy can be spent in a single round, as the \spell{telekinetic thrust} spell.
\end{spelleffect}

\spellsection{Telekinetic Force}
\spellschool{Evocation (Control)}
\spelllvl{Evoc 4}
\spellrng{\rngmed}
\spelltgt{One object or creature at a time}
\spelldur{Concentration, up to 5 minutes}
\spellsave{Will negates (object); see text}
\spellsr{Yes (Will)}
\begin{spelleffect}
  You may manipulate objects or creatures at a distance as if you were holding the object in your hands. When doing so, your effective Strength is equal to half your casting attribute, and your effective Dexterity is equal to half your Intelligence. You can move objects at a speed of up to 20 feet per round in any direction.

   A creature can negate the effect on itself or an object it possesses with a successful Will save. Each round, the subject can attempt a new saving throw to negate the effect. If you are prevented from affecting a target in this way, it and any of its possessions are immune to your attempts for the duration of the spell, though you can still attempt to affect other creatures or objects. 
\end{spelleffect}
\begin{spellnotes}
  This spell generally moves objects too slowly for them to be used as weapons. However, some indirect weapons, such as crossbows, may be used to attack with this spell. 
\end{spellnotes}

\spellsection{Telekinetic Maneuver}
\spellschool{Evocation (Control)}
\spelllvl{Evoc 3}
\spellrng{\rngmed}
\spelltgt{One creature}
\spelldur{Concentration, up to \durmed}
\spellsave{None}
\spellsr{Yes (Will)}
\begin{spelleffect}
  Once per round, you can telekinetically attack a foe of your choice. You can perform a bull rush, a disarm, a dirty trick, a grapple (including a pin, if you have already grappled a foe), or a trip. Resolve these attempts as normal, except that they don't provoke attacks of opportunity, you use your caster level in place of your base attack bonus, and you use your casting attribute in place of your Strength. In addition, you get a \plus2 bonus to combat maneuvers with this spell. \bonusscalingdescription
\end{spelleffect}

\spellsection{Telekinetic Thrust}
\spellschool{Evocation (Control)}
\spelllvl{Evoc 5}
\spellrng{\rngmed}
\spelltgtortgts{Five objects or creatures in a \areamed radius \add one per four caster levels after 8th}
\spelldur{Instantaneous}
\spellsave{Will negates (object); see text}
\spellsr{Yes (Will)}
\begin{spelleffect}
  You can throw the affected objects or creatures anywhere within the spell's range. All subjects of this spell must be thrown to the same place. You can hurl up to a total weight of 25 pounds per caster level.
  \par You must succeed on ranged attack rolls (one per creature or object thrown) to hit the target of the hurled items with the items, applying your caster level and casting attribute to the attack roll instead of your base attack bonus and Dexterity. Hurled weapons deal their normal damage. Other objects deal damage ranging from 1 point per 25 pounds of weight (for less dangerous objects such as an empty barrel) to 1d6 points per 25 pounds of weight (for hard, dense objects such as a boulder).
  \par Creatures are allowed Will saves (and spell resistance) to avoid being hurled or having their held possessions be targeted by this power.
  \par If you use this power to hurl a creature against a solid surface, it takes damage as if it had fallen 50 feet (5d6 damage).
\end{spelleffect}

\spellsection{Telepathic Bond}
\spelldesc{You forge a mental link binding two allies together.}
\spellschool{Divination/Transmutation (Communication, Imbuement)}
\spelllvl{Sor/Wiz 3}
\spellrng{\rngclose}
\spelltgts{You and one willing creature, or two willing creatures}
\spelldur{\durext (D)}
\spellsave{None}
\spellsr{Yes (Will)}
\begin{spelleffect}
  The subjects can communicate mentally through telepathy. The communication is instantaneous across any distance within the same plane.
\end{spelleffect}
\begin{spellnotes}
  No special influence is established as a result of the bond. \spellindirect{telepathic bond}{Telepathic bond} can be made permanent with a \spell{permanency} ritual.
\end{spellnotes}

\spellsection{Telepathic Bond, Mass}
\spellschool{Divination/Transmutation (Communication, Imbuement)}
\spelllvl{Sor/Wiz 6}
\spelltgts{You plus up to five willing creatures in a \areamed radius }
\begin{spelleffect}
  This spell functions like \spell{telepathic bond}, except that it links multiple creatures together into the same bond. Each affected creature can communicate with all other creatures, either privately or to the group as a whole. If desired, you may leave yourself out of the bond forged. This decision must be made at the time of casting.
\end{spelleffect}
\begin{spellnotes}
  \spellindirect{telepathic bond}{Telepathic bond} can be made permanent with a \spell{permanency} ritual. If you cast this spell multiple times, you may link each casting of the spell together such that all subjects may telepathically communicate with each other. 
\end{spellnotes}

\spellsection{Temporal Stasis}
\spellschool{Transmutation (Temporal)}
\spelllvl{Sor/Wiz 8}
\spellrng{Touch}
\spelltgt{Creature touched}
\spelldur{\durshort/Permanent}
\spellsave{None/Will negates}
\spellsr{Yes (Will)}
\begin{spelleffect}
  If you succeed on a melee touch attack, the subject is slowed for a \durshort duration.
\end{spelleffect}
\begin{spellblood}
  In addition, the subject is placed into a state of suspended animation unless it makes a successful Will save. For the creature, time ceases to flow and its condition becomes fixed. The creature does not grow older. Its body functions virtually cease, and no force or effect can harm it. This state persists until the magic is removed (such as by a successful \spell{dispel magic} spell or an \spell{emancipation} spell).
\end{spellblood}

\spellsection{Power Word Fear}
\spell{You fill your foe with an inescapable fear, forcing it to flee from your presence.}
\spellschool{Enchantment (Emotion) [Fear, Mind-Affecting]}
\spelllvl{Sor/Wiz 6}
\spellrng{\rngclose}
\spelltgt{One creature}
\spelldur{\durshort}
\spellsave{None}
\spellsr{Yes (Will)}
\begin{spellhealthy}
  The subject is shaken, causing it to be vulnerable.
\end{spellhealthy}
\begin{spellblood}
  The subject is frightened.
\end{spellblood}
\begin{spellnotes}
  A vulnerable character takes a \minus2 penalty on attack rolls, saving throws, checks, DCs, and AC. A frightened creature is the same, except that it also flees from the source of its fear as best it can. If unable to flee, it may fight.
  \par A character shaken by multiple sources becomes frightened. A character frightened by multiple sources becomes panicked.
\end{spellnotes}

\spellsection{Time Stop}
\spellschool{Transmutation (Temporal)}
\spelllvl{Sor/Wiz 9}
\spellrng{Personal}
\spelltgt{You}
\spelldur{1d3\plus1 rounds (apparent time); see text}
\begin{spelleffect}
  This spell seems to make time cease to flow for everyone but you. In fact, you step into an alternate timestream, causing you to speed up so greatly that all other creatures seem frozen, though they are actually still moving at their normal speeds. You are free to act for 1d3\plus1 rounds of apparent time. You are still vulnerable to danger, such as from heat or dangerous gases, but your actions have no effect on anything in the world other than yourself. Objects and creatures appear frozen in place. You cannot cast spells that affect any targets except yourself; the temporal magic is too strong to permit interference from lesser magic, and attempts to cast magic beyond the accelerated time surrounding you simply fail. The only exception is for temporal spells, which can be cast normally inside a \spell{time stop}. The subjects are not affected and do not attempt to resist the effects until the end of the \spell{time stop}, so you do not know whether they are affected by any spells you cast until the effect has expired.
\end{spelleffect}
\begin{spellnotes}
  Most spellcasters use the additional time to improve their defenses or flee from combat. You are undetectable while \spell{time stop} lasts. You cannot enter an area protected by an \spell{antimagic field} while under the effect of \spell{time stop}.
\end{spellnotes}

\spellsection{Totemic Mind}
\spellschool{Transmutation (Augment)}
\spelllvl{Clr 2, Drd 2, Sor/Wiz 2}
\spellrng{\rngclose}
\spelltgt{One creature}
\spelldur{\durshort}
\spellsave{Will negates (harmless)}
\spellsr{Yes (Will)}
\begin{spelleffect}
  This spell grants creatures the mental power of a totem animal. It has three forms, each of which grants a \plus2 bonus to a mental attribute. \bonusscalingdescription
  \par \subspell{Eagle's Splendor} The transmuted creature becomes more persuasive and personally forceful, gaining a bonus to Charisma.
  \par \subspell{Fox's Cunning} The transmuted creature becomes smarter, gaining a bonus to Intelligence.
  \par \subspell{Owl's Wisdom} The transmuted creature becomes more perceptive, gaining a bonus to Wisdom.
\end{spelleffect}

\spellsectioncomma{Totemic Mind}{Mass}
\spellschool{Transmutation (Augment)}
\spelllvl{Clr 6, Drd 6, Sor/Wiz 6}
\spellrng{\rngmed}
\spellarea{\areamed radius limit}
\spelltgts{Five creatures within the area}
\begin{spelleffect}
  This spell functions like \spell{totemic mind}, except that it affects multiple creatures. All affected creatures must gain a bonus to the same attribute.
\end{spelleffect}

\spellsection{Totemic Power}
\spellschool{Transmutation (Augment)}
\spelllvl{Clr 2, Drd 2, Sor/Wiz 2, Strength 2}
\spellrng{\rngclose}
\spelltgt{One creature}
\spelldur{\durshort}
\spellsave{Fortitude negates (harmless)}
\spellsr{Yes (Fortitude)}
\begin{spelleffect}
  This spell grants creatures the physical power of an animal. It has three forms, each of which grants a \plus2 bonus to a mental attribute. \bonusscalingdescription
  \par \subspell{Bear's Endurance} The transmuted creature gains greater vitality and stamina, gaining a bonus to Constitution. Hit points gained by a temporary increase in Constitution score are not temporary hit points. They go away when the subject's Constitution drops back to normal. They are not lost first as temporary hit points are.
  \par \subspell{Bull's Strength} The transmuted creature becomes stronger, gaining a bonus to Strength.
  \par \subspell{Cat's Grace} The transmuted creature becomes more graceful, agile, and coordinated, gaining a bonus to Dexterity.
\end{spelleffect}

\spellsectioncomma{Totemic Power}{Mass}
\spellschool{Transmutation (Augment)}
\spelllvl{Clr 6, Drd 6, Sor/Wiz 6}
\spellrng{\rngmed}
\spellarea{\areamed radius limit}
\spelltgts{Five creatures within the area}
\begin{spelleffect}
  This spell functions like \spell{totemic power}, except that it affects multiple creatures.
\end{spelleffect}

\spellsection{Touch of Idiocy}
\spellschool{Enchantment (Inhibition) [Mind-Affecting]}
\spelllvl{Sor/Wiz 2}
\spellrng{\rngclose}
\spelltgt{One creature}
\spelldur{\durshort}
\spellsave{Will half}
\spellsr{Yes (Will)}
\begin{spelleffect}
  With a touch, you reduce the target's mental faculties. Your successful melee touch attack applies a \minus4 penalty to the target's Intelligence, Wisdom, and Charisma scores. This penalty can't reduce any of these scores below \minus9.
\end{spelleffect}
\begin{spellnotes}
  This spell's effect may make it impossible for the target to cast some or all of its spells, if the requisite attribute drops below the minimum required to cast spells of that level.
\end{spellnotes}

\spellsection{Transfer Suffering}
\spellschool{Necromancy (Life) [Healing]}
\spelllvl{Necro 4}
\spellrng{Touch}
\spelltgt{One creature}
\spelldmg{8d6 life damage \add d8 per two caster levels above 8th}
\begin{spelleffect}
  The touched creature takes damage, and you immediately regain hit points equal the amount of damage you transfer. You cannot transfer more damage than you have taken, and you cannot use this spell to gain hit points in excess of your full normal total.
\end{spelleffect}

\spellsection{Transmute Any Object}
\spellschool{Transmutation (Alteration, Polymorph)}
\spelllvl{Sor/Wiz 9}
\spellrng{\rngmed}
\spelltgt{One creature, or one nonmagical object of up to 1000 cu. ft.}
\spelldur{See text}
\spellsave{Fortitude negates (object); see text}
\spellsr{Yes (Fortitude)}
\begin{spelleffect}
  This spell can be used to duplicate the effects of \spell{fabricate}, \spell{major creation}, \spell{passwall}, \spell{shape stone}, \spell{transmute flesh and stone}, or \spell{wall of stone}. The object or creature to be transformed must meet any requirements of the spell to be duplicated, except that it must be within \rngmed range.
\end{spelleffect}

\spellsection{Transmute Flesh and Stone}
\spellschool{Transmutation (Polymorph)}
\spelllvl{Earth 6, Trans 6}
\spellrng{\rngmed}
\spelltgt{One creature or a cylinder of stone from 1 ft. to 3 ft. in diameter and up to 10 ft. long}
\spelldur{\durshort/Instantaneous}
\spellsave{Fortitude negates (object); see text}
\spellsr{Yes (Fortitude)}
\spelldmg{3d8 damage per round; see text}
\begin{spelleffect}
  This spell has different effects depending on the version chosen.
  \par \subspell{Flesh to Stone} The subject is slowed for the duration of the spell, and takes 3d8 physical damage each round as its body gradually turns to stone. A Fortitude save negates this effect. If the subject reaches 0 hit points before the spell ends, it becomes a mindless, inert statue, along with all its carried gear. If the statue resulting from this effect is broken or damaged, the subject (if ever returned to its original state) has similar damage or deformities. The creature is not dead, but it is not considered alive either.
  \par Only creatures made of flesh are affected by this effect.
  \par \subspell{Stone to Flesh} This effect restores a petrified creature to its normal state, restoring life and goods. The creature must make a DC 15 Fortitude save to survive the process. Any petrified creature, regardless of size, can be restored. A restored creature has as many hit points as it had when it was petrified. Stone which was not originally a petrified creature is unaffected.
\end{spelleffect}

\spellsection{Tree Shape}
\spellschool{Transmutation (Polymorph)}
\spelllvl{Drd 2}
\spellrng{Personal}
\spelltgt{You}
\spelldur{\durext (D)}
\begin{spelleffect}
  You become able to assume the form of a Large living tree or shrub or a Large dead tree trunk with a small number of limbs. The closest inspection cannot reveal that the tree in question is actually a magically concealed creature. To all normal tests you are, in fact, a tree or shrub, although a Spellcraft check can reveal a faint transmutation on the tree. While in tree form, you can observe all that transpires around you just as if you were in your normal form, and your hit points and save bonuses remain unaffected. You gain a \plus10 bonus to natural armor, but you have an effective Dexterity score of 0 and a speed of 0 feet. You are immune to critical hits while in tree form. All clothing and gear carried or worn changes with you.
\end{spelleffect}
\begin{spellnotes}
  You can dismiss tree shape as a free action (instead of as a standard action).
\end{spellnotes}

\spellsection{Tremorsense}
\spellschool{Transmutation (Imbuement)}
\spelllvl{Drd 1, Earth 1}
\spellrng{Personal/\arealarge limit}
\spelltgt{You}
\spelldur{Concentration}
\begin{spelleffect}
  You gain the tremorsense ability. If you are touching a surface, you can automatically pinpoint the location of anything within the area of the spell that is in contact with the surface, including inanimate objects.
\end{spelleffect}
\begin{spellnotes}
  Tremorsense functions on surfaces of any kind, regardless of lighting conditions.
\end{spellnotes}

\spellsection{True Seeing}
\spellschool{Divination (Awareness)}
\spelllvl{Clr 6, Div 5, Knowledge 5, Law 6, Sor/Wiz 6}
\spellcmp{V, S, M}
\spellrng{Touch}
\spelltgt{Creature touched}
\spelldur{\durshort}
\spellsave{Will negates (harmless)}
\spellsr{Yes (Will)}
\begin{spelleffect}
  You confer on the subject the ability to see all things as they actually are. The subject sees through normal and magical darkness, notices secret doors hidden by magic, sees the truth behind visual figments and glamers, and sees the true form of polymorphed, changed, or transmuted things. Further, the subject can focus its vision to see into the Ethereal Plane (but not into extradimensional spaces). The effect extends out to \rngmed range.
\end{spelleffect}
\begin{spellnotes}
  \spellindirect{true seeing}{True seeing} does not penetrate solid objects. It in no way confers X-ray vision or its equivalent. It does not negate concealment, including that caused by fog and the like. True seeing does not help the viewer see through mundane disguises, spot creatures who are simply hiding, or notice secret doors hidden by mundane means. In addition, the spell effects cannot be further enhanced with known magic, so one cannot use \spell{true seeing} through a scrying effect.
\end{spellnotes}
\spellmat{An ointment for the eyes that costs 250 gp and is made from mushroom powder, saffron, and fat.}

\spellsection{True Strike}
\spellschool{Divination (Knowledge)}
\spelllvl{Div 5}
\spelltime{1 swift action}
\spellcmp{V}
\spellrng{Personal}
\spelltgt{You}
\spelldur{See text}
\begin{spelleffect}
  You gain temporary, intuitive insight into the immediate future during your next attack. Your next single attack roll (if it is made before the end of the next round) gains a \plus20 bonus. Additionally, you are not affected by the miss chance that applies to attackers trying to strike a concealed target.
\end{spelleffect}

\pdfbookmark[2]{U-Z}{SpellDescriptionsU-Z}
\begin{comment}
\subsubsection{U-Z}
\end{comment}

\spellsection{Undeath to Death}
\spellschool{Necromancy (Vitalism) [Positive]}
\spelllvl{Clr 6}
\spellarea{\areamed radius limit}
\spelltgts{Several undead creatures within the area}
\begin{spelleffect}
  This spell functions like \spell{circle of death}, except that it destroys undead creatures.
\end{spelleffect}
\spellmat{The powder of a crushed diamond worth at least 750 gp.}

\spellsection{Unholy Aura}
\spellschool{Abjuration (Interdiction) [Evil]}
\spelllvl{Clr 8, Evil 8}
\spellcmp{V, S, F}
\spellarea{\areamed radius limit centered on you}
\spelltgts{Five creatures within the area}
\spelldur{\durshort (D)}
\spellsave{See text}
\spellsr{Yes (Will)}
\begin{spelleffect}
  A malevolent darkness surrounds the subjects, protecting them from attacks, granting them resistance to spells cast by good creatures, and weakening good creatures when they strike the subjects. This abjuration has four effects.
  \par First, each shielded creature gains a \plus5 bonus to its saving throws.
  \par Second, each shielded creature gains spell resistance 10 against chaotic spells and spells cast by good creatures.
  \par Third, the abjuration blocks possession and mental influence, just as protection from good does.
  \par Finally, if a good creature within \rngmed range of the shielded creature successfully attacks it in any way, the offending attacker takes 4d6 damage. Any single creature can take this damage only once per round.
\end{spelleffect}
\spellfocus{A tiny reliquary containing some sacred relic, such as a piece of parchment from an unholy text. The reliquary costs at least 500 gp.}

\spellsection{Unholy Blight}
\spellschool{Evocation (Channeling) [Evil]}
\spelllvl{Evil 4}
\spellrng{\rngmed}
\spelltgt{One creature}
\spelldur{Instantaneous/5 rounds}
\spellsave{Will half/Will negates}
\spellsr{Yes (Will)}
\spelldmg{8d6 divine damage \add d6 per two caster levels above 8th}
\begin{spelleffect}
  If the target is not evil, it takes damage and is sickened for 5 rounds, making it vulnerable. A successful Will save halves the damage.
\end{spelleffect}
\begin{spellnotes}
  A vulnerable creature takes a \minus2 penalty to attack rolls, saving throws, checks, DCs, and AC.
\end{spellnotes}

\spellsection{Unliving Eyes}
\spellschool{Divination/Necromancy (Awareness, Life)}
\spelllvl{Necro 2}
\spellrng{Touch}
\spelltgt{One creature}
\spelldur{\durshort (D)}
\spellsave{Will negates (harmless)}
\spellsr{Yes (Will)}
\begin{spelleffect}
  The subject gains the ability to ``see'' any living creatures and their equipment within 60 feet perfectly, regardless of lighting conditions, physical barriers, invisibility, or any other means of concealment.
\end{spelleffect}

%gaining full temp HP would be level 5
\spellsection{Vampiric Touch}
\spellschool{Necromancy (Life)}
\spelllvl{Necro 3}
\spellrng{Touch}
\spelltgt{Living creature touched}
\spelldur{Instantaneous/\durmed}
\spellsave{Fortitude half}
\spellsr{Yes (Fortitude)}
\spelldmg{6d8 life damage \add d8 per two caster levels above 6th}
\begin{spelleffect}
  The touched creature takes damage. You gain temporary hit points equal to half the damage you deal. However, you can't gain more hit points than the damage required to kill the subject. The temporary hit points disappear 1 hour later.

  If you take life damage, you lose all temporary hit points provided by this spell before applying the damage.
\end{spelleffect}

\spellsection{Veil}
\spellschool{Illusion (Glamer) [Unreal]}
\spelllvl{Sor/Wiz 2}
\spellrng{\rngclose}
\spelltgt{One humanoid creature}
\spelldur{\durshort}
\spellsave{Will negates}
\spellsr{Yes (Will)}
\begin{spelleffect}
  The subject's arms and torso are masked in illusion, causing onlookers to perceive whatever movements you project instead of the creature's true actions. For example, the subject might draw a dagger and attack another creature, but anyone watching would only see the subject folding its arms, even as the dagger strikes true.
\end{spelleffect}
\begin{spellnotes}
  A creature that interacts with the effect gets a Will save to recognize it as an illusion. In order to interact with the illusion with a Perception check, the creature must make a Perception check that beats your saving throw DC with this spell. Anyone witnessing the subject perform an impossible action, such as attacking or climbing without the use of its hands, receives a Will save with a \plus10 bonus.
\end{spellnotes}

\spellsection{Ventriloquism}
\spellschool{Illusion (Figment)}
\spelllvl{Sor/Wiz 1, Trickery 1}
\spellcmp{V, F}
\spellrng{\rngclose}
\spelleff{Intelligible sound, usually speech}
\spelldur{\durshort (D)}
\spellsave{Will disbelief (if interacted with)}
\spellsr{No}
\begin{spelleffect}
  You can make your voice (or any sound that you can normally make vocally) seem to issue from someplace else. You can speak in any language you know. With respect to such voices and sounds, anyone who hears the sound and rolls a successful save recognizes it as illusory (but still hears it).
\end{spelleffect}

\begin{comment}
\spellsection{Vestments of the Mage}
\spelldesc{You imbue a set of armor with magical power, preventing it from interfering with your spellcasting.}
\spellschool{Transmutation (Imbuement)}
\spelllvl{Sor/Wiz 2}
\spellrng{Touch}
\spelltgt{Touched nonmagical armor or shield}
\spelldur{\durext (D)}
\spellsave{Will negates (harmless, object)}
\spellsr{Yes (Will)}
\begin{spelleffect}
  The armor or shield's chance of arcane spell failure decreases by 10\% as long as you are wearing or using it. If any other creature wears the armor, it receives no benefit from this spell.
\end{spelleffect}
\begin{spellnotes}
  This is considered an enhancement bonus.
\end{spellnotes}
\end{comment}

%complicated
\spellsection{Wail of the Banshee}
\spelldesc{You emit a terrible scream that kills anyone that hears it.}
\spellschool{Necromancy (Life) [Death, Sound-Dependent]}
\spelllvl{Death 9, Necro 9}
\spellcmp{V}
\spelltgts{Living creatures in a \arealarge spread centered on you, up to Five creatures}
\spelldur{Concentration, up to 2 rounds; see text}
\spellsave{Fortitude negates}
\spellsr{Yes (Fortitude)}
\begin{spellhealthy}
  The subjects are sickened, making them vulnerable for 5 rounds. If you concentrate for a second round, subjects still in the area are nauseated for 1 round.
\end{spellhealthy}
\begin{spellblood}
  The subjects are nauseated for 1 round. If you concentrate for a second round, subjects still in the area immediately die.
\end{spellblood}
\begin{spellnotes}
  A vulnerable creature takes a \minus2 penalty to attack rolls, saving throws, checks, DCs, and AC.
  This spell affects a maximum number of creatures equal to your caster level. Creatures closest to you are affected first, so creatures farther away may be unaffected if there are enough intervening creatures. Each creature makes only one saving throw against the effect.
\end{spellnotes}

\spellsection{Wall of Fire}
\spellschool{Evocation (Energy) [Fire, Wall]}
\spelllvl{Drd 5, Fire 5, Sor/Wiz 5}
\spellrng{\rngmed}
\spelleff{Opaque sheet of flame up to 100 ft. long or a ring of fire with a radius of up to 20 ft.; either form 20 ft. high}
\spelldur{\durshort}
\spellsave{Reflex half or Reflex negates; see text}
\spellsr{Yes (Reflex)}
\spelldmg{5d6 \add d6 per four caster levels above 10th; see text}
\begin{spelleffect}
  An immobile, blazing curtain of shimmering violet fire springs into existence. The wall deals damage to any creature passing through it. A successful Reflex save halves this damage. In addition, the wall radiates heat, dealing 2d6 points of fire damage to creatures within 10 feet and 1d6 points of fire damage to those past 10 feet but within 20 feet. No save is allowed against this damage. The wall deals this damage at the start of each of your turns to All creatures within the area.
  \par If you evoke the wall so that it appears where creatures are, each creature takes damage as if passing through the wall. If any 5-foot length of wall takes 20 points of cold damage or more in 1 round, that length goes out.
\end{spelleffect}
\begin{spellnotes}
  \spellindirect{wall of fire}{Wall of fire} can be made permanent with a \spell{permanency} ritual. A permanent \spell{wall of fire} that is extinguished by cold damage becomes inactive for 10 minutes, then reforms at normal strength.
\end{spellnotes}

\spellsection{Wall of Force}
\spellschool{Evocation (Control) [Force, Wall]}
\spelllvl{Sor/Wiz 5}
\spellrng{\rngmed}
\spelleff{Wall whose area is up to ten 10 ft. squares}
\spelldur{\durshort (D)}
\spellsave{None}
\spellsr{No}
\begin{spelleffect}
  This spell creates an invisible wall made of force. Nothing can pass through or alter the wall. It forms a flat, vertical plane, and it must be continuous and unbroken when formed. If the surface is broken by any object or creature, the spell fails.
\end{spelleffect}
\begin{spellnotes}
  The wall is unaffected by most spells, including \spell{dispel magic}. However, \spell{disintegrate} immediately destroys it, as does a \magicitem{rod of cancellation}, a \magicitem{sphere of annihilation}, or a \spell{mage's disjunction} spell. As a force effect, it blocks ethereal creatures as well as material ones.
  \par \spellindirect{wall of force}{Wall of force} can be made permanent with a \spell{permanency} ritual.
\end{spellnotes}

\spellsection{Wall of Ice}
\spellschool{Conjuration/Evocation (Creation, Energy) [Cold, Wall]}
\spelllvl{Sor/Wiz 4, Water 5}
\spellrng{\rngmed}
\spelleff{Anchored plane of ice, up to ten 10 ft. squares, or hemisphere of ice with a radius of up to 10 ft.}
\spelldur{\durmed}
\spellsave{Reflex negates; see text}
\spellsr{Yes (Reflex)}
\spelldmg{4d6 \add d6 per four caster levels above 8th}
\begin{spelleffect}
  This spell creates an anchored plane of ice or a hemisphere of ice, depending on the version selected. A wall of ice cannot form in an area occupied by physical objects or creatures. Its surface must be smooth and unbroken when created. Any creature adjacent to the wall when it is created may attempt a Reflex save to disrupt the wall as it is being formed. A successful save indicates that the spell automatically fails. Fire can melt a wall of ice, and it deals full damage to the wall (instead of the normal half damage taken by objects). Suddenly melting a wall of ice creates a great cloud of steamy fog that lasts for 10 minutes.
  \par \subspell{Ice Plane} A sheet of strong, hard ice appears. The wall is 1 foot thick. It covers up to ten 10-foot square areas (so it can create a wall of ice 100 feet long and 10 feet high, a wall 50 feet long and 20 feet high, or some other combination of length and height that does not exceed 1,000 square feet). The plane can be oriented in any fashion as long as it is anchored. A vertical wall need only be anchored on the floor, while a horizontal or slanting wall must be anchored on two opposite sides.
  \par Each 10-foot square of wall has 3 hit points per inch of thickness, or 36 hit points total. Creatures can hit the wall automatically. A section of wall whose hit points drop to 0 is breached. If a creature tries to break through the wall with a single attack, the DC for the Strength check is 15 \add 1 per inch of thickness remaining.
  \par Even when the ice has been broken through, a sheet of frigid air remains. Any creature stepping through it (including the one who broke through the wall) takes damage (no save).
  \par \subspell{Hemisphere} The wall takes the form of a hemisphere whose maximum radius is 10 feet. The hemisphere is as hard to break through as the ice plane form, but it does not deal damage to those who go through a breach.
\end{spelleffect}

\spellsection{Wall of Stone}
\spellschool{Transmutation (Alteration) [Earth, Wall]}
\spelllvl{Drd 5, Earth 5, Sor/Wiz 5}
\spellrng{\rngmed}
\spelleff{Stone wall whose area is up to ten 5 ft. squares (S)}
\spelldur{Instantaneous}
\spellsave{See text}
\spellsr{No}
\begin{spelleffect}
  This spell forms a wall of stone atop existing rock surfaces. A wall of stone is 4 inches thick and composed of up to ten 5-foot squares. You can double the wall's area by halving its thickness. The wall cannot be conjured so that it occupies the same space as a creature or another object.
  \par Unlike a wall of iron, you can create a wall of stone in almost any shape you desire. The wall created need not be vertical, nor rest upon any firm foundation; however, it must merge with and be solidly supported by existing stone. It can be used to bridge a chasm, for instance, or as a ramp. For this use, if the span is more than 20 feet, the wall must be arched and buttressed. This requirement reduces the spell's area by half. The wall can be crudely shaped to allow crenellations, battlements, and so forth by likewise reducing the area.
  \par Like any other stone wall, this one can be destroyed by a disintegrate spell or by normal means such as breaking and chipping. Each 5-foot square of the wall has 15 hit points per inch of thickness and hardness 8. A section of wall whose hit points drop to 0 is breached. If a creature tries to break through the wall with a single attack, the DC for the Strength check is 20 \add 2 per inch of thickness.
  \par It is possible to trap mobile opponents within or under a wall of stone, provided the wall is shaped so it can hold the creatures. Creatures can avoid entrapment with successful Reflex saves.
\end{spelleffect}

\spellsection{Wall of Thorns}
\spellschool{Conjuration (Creation) [Wall]}
\spelllvl{Drd 5, Nature 5}
\spellrng{\rngmed}
\spelleff{Wall of thorny brush, up to ten 10 ft. cubes (S)}
\spelldur{\durlong (D)}
\spellsave{None}
\spellsr{No}
\begin{spelleffect}
  This spell creates a barrier of very tough, pliable, tangled brush bearing needle-sharp thorns as long as a human's finger. Any creature forced into or attempting to move through a wall of thorns takes slashing damage per square of movement equal to 25 minus the creature's flat-footed AC. (Creatures with a flat-footed Armor Class of 25 or higher, take no damage from contact with the wall.)
  \par You can make the wall as thin as 5 feet thick, which allows you to shape the wall as twenty 10\mult10\mult5 foot blocks. This has no effect on the damage dealt by the thorns, but any creature attempting to break through takes that much less time to force its way through the barrier.
  \par Creatures can force their way slowly through the wall by making a combat maneuver attack or Escape Artist check as a full-round action. The creature moves 5 feet for each full 5 points by which the check result exceeds 20, up to a maximum distance equal to its normal land speed. Of course, moving or attempting to move through the thorns incurs damage as described above. A creature trapped in the thorns can choose to remain motionless in order to avoid taking any more damage.
  \par If you have at least 5 feet of thorns between you and an opponent, it provides cover. If you have at least 20 feet of thorns between you, it provides total cover.
  \par Any creature within the area of the spell when it is cast takes damage as if it had moved into the wall and is caught inside. In order to escape, it must attempt to push its way free, or it can wait until the spell ends. Creatures with the ability to pass throughs overgrown areas unhindered can pass through a wall of thorns at normal speed without taking damage.
\end{spelleffect}
\begin{spellnotes}
  A \spell{wall of thorn} can be breached by slow work with edged weapons or fire. It has hardness 8 and 30 hit points per square foot of thickness.
  \par Despite its appearance, a \spell{wall of thorns} is not actually a living plant, and thus is unaffected by spells that affect plants.
\end{spellnotes}

\spellsection{Warp Wood}
\spellschool{Transmutation (Alteration)}
\spelllvl{Destruction 2, Drd 2}
\spellrng{\rngclose}
\spellarea{\areamed radius limit}
\spelltgt{1 Small nonmagical wooden object/level within the area}
\spelldur{Instantaneous}
\spellsave{Will negates (object)}
\spellsr{Yes (Will)}
\begin{spelleffect}
  You cause wood to bend and warp, permanently destroying its straightness, form, and strength. A warped door springs open (or becomes stuck, requiring a Strength check to open, at your option). A boat or ship springs a leak. Warped ranged weapons are useless. A warped melee weapon imposes a \minus4 penalty on attack rolls.
  \par You may warp one Small or smaller object or its equivalent per caster level. A Medium object counts as two Small objects, a Large object as four, a Huge object as eight, a Gargantuan object as sixteen, and a Colossal object as thirty-two.
  \par Alternatively, you can unwarp wood (effectively warping it back to normal) with this spell, straightening wood that has been warped by this spell or by other means. \spellindirect{make whole}{Make whole}, on the other hand, does no good in repairing a warped item.
\end{spelleffect}
\begin{spellnotes}
  You can combine multiple consecutive \spell{warp wood} spells to warp (or unwarp) an object that is too large for you to warp with a single spell. Until the object is completely warped, it suffers no ill effects.
\end{spellnotes}

\spellsection{Water Walk}
\spellschool{Transmutation (Imbuement) [Water]}
\spelllvl{Druid 3, Water 3}
\spellrng{Touch}
\spelltgts{Five touched creatures}
\spelldur{\durlong (D)}
\spellsave{Fortitude negates (harmless)}
\spellsr{Yes (Fortitude)}
\begin{spelleffect}
  The transmuted creatures can tread on any liquid as if it were firm ground. Mud, oil, snow, quicksand, running water, ice, and even lava can be traversed easily, since the subjects' feet hover an inch or two above the surface. (Creatures crossing molten lava still take damage from the heat because they are near it.) The subjects can walk, run, charge, or otherwise move across the surface as if it were normal ground.
  \par If the spell is cast underwater (or while the subjects are partially or wholly submerged in whatever liquid they are in), the subjects are borne toward the surface at 60 feet per round until they can stand on it.
\end{spelleffect}

\spellsection{Waves of Exhaustion}
\spellschool{Necromancy (Flesh)}
\spelllvl{Death 8, Sor/Wiz 8, War 8}
\spellarea{\arealarge cone-shaped burst}
\spelldur{\durshort}
\spellsave{Fortitude partial}
\spellsr{Yes (Fortitude)}
\begin{spelleffect}
  Living creatures in the area are exhausted. A successful Fortitude save causes a creature to be fatigued instead. This spell has no effect on a creature that is already exhausted.
\end{spelleffect}

\spellsection{Waves of Fatigue}
\spellschool{Necromancy (Flesh)}
\spelllvl{Death 5, Sor/Wiz 5, War 5}
\spellarea{\arealarge cone-shaped burst}
\spelldur{\durshort}
\spellsave{No}
\spellsr{Yes (Fortitude)}
\begin{spelleffect}
  Living creatures in the area are fatigued. This spell has no effect on a creature that is already fatigued.
\end{spelleffect}

\spellsection{Web}
\spelldesc{You create a many-layered mass of strong, stricky strands that entangle creatures caught within them. The strands are similar to spider webs, but larger and tougher.}
\spellschool{Conjuration (Creation)}
\spelllvl{Sor/Wiz 3}
\spellrng{\rngclose}
\spellarea{\areamed radius spread}
\spelleff{Webs in the area}
\spelldur{\durshort (D)}
\spellsave{Reflex negates; see text}
\spellsr{No}
\begin{spelleffect}
  Each creature in the spell's area are entangled unless it makes a successful Reflex save. This save must be repeated each round that the creature moves or fights within the area. An entangled creature can spend a standard action to make a grapple attack or Escape Artist attempt against the spell's save DC to break the webs holding it, preventing it from being entangled. A creature entangled by the spell remains entangled until it breaks the webs holding it or escapes the spell's area.
  \par If the strands can be anchored to two or more solid and diametrically opposed structures, such as walls, the strands are much more sturdy. A creature entangled within a sturdy web is unable to move from its square until it stops being entangled.
\end{spelleffect}
\begin{spellnotes}
  An entangled creature moves at half speed, cannot run or charge, and takes a \minus2 penalty to attack rolls, Strength and Dexterity-based checks, and armor class. If it attempts to cast a spell must make a Concentration check (DC 10 \add double the spell's level) or lose the spell.
  The strands are too widely spaced to significantly obscure sight, but are flammable. A magic flaming sword can slash them away as easily as a hand brushes away cobwebs. Any fire can set the webs alight and burn away 5 square feet in 1 round. All creatures within flaming webs take 2d4 points of fire damage from the flames.
  \spell{Web} can be made permanent with a \spell{permanency} ritual. A permanent \spell{web} that is destroyed regrows in 10 minutes.
\end{spellnotes}

\spellsection{Weird}
\spellschool{Enchantment/Illusion (Emotion, Phantasm) [Death, Fear, Mind-Affecting, Unreal]}
\spelllvl{Sor/Wiz 9, Trickery 9}
\spellarea{\areamed radius limit}
\spelltgts{Five creatures within the area}
\begin{spelleffect}
  This spell functions like \spell{phantasmal killer}, except that it affects multiple creatures.
\end{spelleffect}

\spellsection{Windstrike}
\spelldesc{You command the air to bludgeon the target, sending it flying.}
\spellschool{Evocation (Control) [Air]}
\spelllvl{Air 2, Drd 2}
\spellrng{\rngmed}
\spelltgt{One creature or object}
\spelldur{Instantaneous}
\spellsave{Fortitude half}
\spellsr{Yes (Fortitude)}
\spelldmg{4d6 bludgeoning damage \add d6 per two levels after 4th}
\begin{spelleffect}
  The target takes damage from the powerful winds. A successful Fortitude save halves the damage. In addition, you may make a bull rush attack with a bonus equal to your caster level \add your casting attribute. If you succeed, you may have the wind bull rush the target in any direction -- even vertically. Moving the target up takes twice as much movement as moving the target horizontally.
\end{spelleffect}

\spellsection{Windstrike, Greater}
\spelldesc{You command the air to bludgeon the target with tremendous force, sending it flying.}
\spellschool{Evocation (Control) [Air]}
\spelllvl{Air 5, Drd 5}
\spelldmg{10d6 bludgeoning damage \add d6 per two levels after 10th}
\begin{spelleffect}
  This spell functions like \spell{windstrike}, except that the bull rush is much more powerful. You make a bull rush attack with a bonus equal to your caster level \add your casting attribute \add 12, treating the wind as a Gargantuan creature.
  
  If you succeed, you knock the target prone and may have the wind bull rush the target in any direction -- even vertically. Moving the target up does not require more movement than moving the target horizontally.
\end{spelleffect}

\spellsection{Wish}
\spellschool{Universal}
\spelllvl{Magic 9, Sor/Wiz 9}
\spellcmp{V, S, M}
\spellrng{See text}
\spelltgteffarea{See text}
\spelldur{See text}
\spellsave{See text}
\spellsr{Yes (varies)}
\begin{spelleffect}
  This spell is the mightiest spell a wizard or sorcerer can cast. By simply speaking your desires aloud, you can alter reality to better suit you.
  \par Even wish, however, has its limits.
  \par A wish can produce any one of the following effects.
  \begin{itemize*}
    \item Duplicate any general wizard or sorcerer spell of 8th level or lower, provided the spell is not of a school prohibited to you.
    \item Duplicate any general wizard or sorcerer spell of 7th level or lower even if it's of a prohibited school.
    \item Duplicate any other spell of 6th level or lower, provided the spell is not of a school prohibited to you.
    \item Duplicate any other spell of 5th level or lower even if it's of a prohibited school. 
    \item Undo the harmful effects of many other spells, such as geas/quest or insanity.
    \item Create a nonmagical item of up to 25,000 gp in value.
    \item Create a magic item, or add to the powers of an existing magic item.
    \item Grant a creature a \plus1 inherent bonus to an attribute. Two to five wish spells cast in immediate succession can grant a creature a \plus2 to \plus5 inherent bonus to an attribute (two wishes for a \plus2 inherent bonus, three for a \plus3 inherent bonus, and so on). Inherent bonuses are instantaneous, so they cannot be dispelled. Note: An inherent bonus may not exceed \plus5 for a single attribute, and inherent bonuses to a particular attribute do not stack, so only the best one applies.
    \item Remove injuries and afflictions. A single wish can aid one creature per caster level, and all subjects are cured of the same kind of affliction. For example, you could heal all the damage you and your companions have taken, or remove all poison effects from everyone in the party, but not do both with the same wish. A wish can never restore the experience point loss from casting a spell or the level or Constitution loss from being raised from the dead.
    \item Revive the dead. A wish can bring a dead creature back to life by duplicating a resurrection spell. A wish can revive a dead creature whose body has been destroyed, but the task takes two wishes, one to recreate the body and another to infuse the body with life again. A wish cannot prevent a character who was brought back to life from losing an experience level.
    \item Transport travelers. A wish can lift one creature per caster level from anywhere on any plane and place those creatures anywhere else on any plane regardless of local conditions. An unwilling target gets a Will save to negate the effect, and spell resistance (if any) applies.
    \item Undo misfortune. A wish can undo a single recent event. The wish forces a reroll of any roll made within the last round (including your last turn). Reality reshapes itself to accommodate the new result. For example, a wish could undo an opponent's successful save, a foe's successful critical hit (either the attack roll or the critical roll), a friend's failed save, and so on. The reroll, however, may be as bad as or worse than the original roll. An unwilling target gets a Will save to negate the effect, and spell resistance (if any) applies.
  \end{itemize*}
  \par When casting a wish, you do not specify the exact spell or effect you wish to duplicate. Instead, you make a wish, describing what you want to have happen, and make a DC 20 Wisdom check. If the check fails, your intent is redirected or perverted in some way. For example, a wish to turn a foe to stone would normally mimic the flesh to stone effect of the transmute flesh and stone spell. However, if the Wisdom check failed, your foe might gain the benefit of a stoneskin spell instead.
  \par You may try to use a wish to produce greater effects than these, but doing so is dangerous. The DC of the Wisdom check increases to 25, and the negative consequences for failing the check increase in proportion to the potency of the effect you try to create.
\end{spelleffect}
\begin{spellnotes}
  Duplicated spells allow saves and spell resistance as normal.
\end{spellnotes}
\spellmat{25,000gp of diamonds. In addition, when a \spell{wish} duplicates a spell with a material component that costs more than 10,000 gp, you must provide that component.}

\spellsection{Word of Chaos}
\spellschool{Evocation (Channeling) [Chaotic]}
\spelllvl{Chaos 7}
\spellcmp{V}
\spellarea{\arealarge radius spread centered on you}
\spelldur{Instantaneous}
\spellsave{None or Will negates; see text}
\spellsr{Yes (Will)}
\begin{spellhealthy}
  Each nonchaotic creature in the area is bewildered, making it vulnerable for 5 rounds.
\end{spellhealthy}
\begin{spellblood}
  Each nonchaotic creature in the area suffers one or more of the following ill effects, depending on its Hit Values.
  \begin{dtable}
    \begin{tabularx}{\columnwidth}{l >{\lcol}X}
      \par \thead{HV} & \thead{Effect} \\
      \par Equal to caster level & Bewildered \\
      \par Up to caster level \minus5 & Confused, bewildered \\
      \par Up to caster level \minus10 & Paralyzed, nauseated, sickened \\
      \par Up to caster level \minus15 & Killed\fn{1}
    \end{tabularx}
    1 Living creatures die. Nonliving creatures are destroyed.
  \end{dtable}
  \par \subspell{Bewildered} The creature is bewildered, making it vulnerable for 5 rounds.
  \par \subspell{Confused} The creature is confused for 2 rounds.
  \par \subspell{Paralyzed} The creature is paralyzed and helpless for 5 rounds.
  \par \subspell{Killed} Living creatures die. Nonliving creatures are destroyed.
\end{spellblood}
\begin{spellnotes}
  A vulnerable creature takes a \minus2 penalty to attack rolls, saving throws, checks, DCs, and AC. Creatures whose Hit Values exceed your caster level are unaffected by \spell{word of chaos}.
\end{spellnotes}

\spellsection{Word of Recall}
\spellschool{Conjuration (Translocation) [Teleportation]}
\spelllvl{Clr 6}
\spellcmp{V}
\spellrng{Unlimited}
\spelltgt{You}
\begin{spelleffect}
  This spell teleports you instantly back to your sanctuary. You must designate the sanctuary when you ready the spell for the day, and it must be a very familiar place. The actual point of arrival is a designated area no larger than 10 feet by 10 feet. You can be transported any distance within a plane but cannot travel between planes. You can transport, in addition to yourself, any objects you carry, as long as their weight doesn't exceed your maximum load. Exceeding this limit causes the spell to fail.
\end{spelleffect}

\spellsection{Zephyr Blade}
\spelldesc{You imbue a weapon with the power of the wind, allowing it to manipulate air currents as it strikes.}
\spellschool{Evocation/Transmutation (Augment, Control) [Air]}
\spelllvl{Air 3, Drd 3}
\spellrng{Touch}
\spelltgt{Touched melee weapon}
\spelldur{\durshort}
\spellsave{Will negates (harmless)}
\spellsr{Yes (Will)}
\begin{spelleffect}
  This spell functions as \spell{magic weapon}, except that the affected weapon also gains an additional five feet of reach, extending the wielder's threatened area. Attacks outside the weapon's normal range deal half damage, but are otherwise treated exactly as if the wielder was attacking with the weapon normally.
\end{spelleffect}
\begin{spellnotes}
  Despite the name of the spell, it can affect melee weapons of any type, even reach weapons. The weapon's extended reach is visible, and opponents can defend themselves normally against the attacks.
\end{spellnotes}

\spellsectioncomma{Zephyr Blade}{Greater}
\spelldesc{You imbue a weapon with the full might of the wind, allowing it to shred opponents with nothing but the air itself.}
\spellschool{Evocation/Transmutation (Augment, Control) [Air]}
\spelllvl{Air 6, Drd 6}
\begin{spelleffect}
  This spell functions like \spell{zephyr blade}, except that it extends the weapon's reach by ten feet, and attacks outside the weapon's normal range deal full damage.
\end{spelleffect}

\spellsection{Zone of Silence}
\spellschool{Illusion (Glamer)}
\spelllvl{Brd 3}
\spellrng{Personal}
\spellarea{\areasmall radius emanation centered on you}
\spelldur{\durlong (D)}
\begin{spelleffect}
  By casting this spell, you manipulate sound waves in your immediate vicinity so that you and those within the spell's area can converse normally, yet no one outside can hear your voices or any other noises from within, including language-dependent or sound-dependent spell effects. This effect is centered on you and moves with you. Anyone who enters the zone immediately becomes subject to its effects, but those who leave are no longer affected.
\end{spelleffect}
\begin{spellnotes}
  This spell provides a defense against sound-dependent effects. Sonic effects are too powerful for magic such as this to muffle, and function normally.
\end{spellnotes}

\spellsection{Zone of Truth}
\spellschool{Enchantment (Inhibition) [Mind-Affecting]}
\spelllvl{Clr 2, Law 2, Pal 2}
\spellrng{\rngmed}
\spellarea{\areamed radius emanation}
\spelldur{\durmed}
\spellsave{Will negates}
\spellsr{Yes (Will)}
\begin{spelleffect}
  Creatures within the emanation area (or those who enter it) can't speak any deliberate and intentional lies. Each potentially affected creature is allowed a save to avoid the effects when the spell is cast or when the creature first enters the emanation area. Affected creatures are aware of this enchantment. Therefore, they may avoid answering questions to which they would normally respond with a lie, or they may be evasive as long as they remain within the boundaries of the truth. Creatures who leave the area are free to speak as they choose.
\end{spelleffect}
