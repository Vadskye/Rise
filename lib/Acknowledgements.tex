A number of marvelous people have helped me make this revision possible, and many clever minds produced ideas I that have incorporated into the new system. I cannot hope to properly credit everyone who contributed, knowingly or otherwise, but I can at least make an attempt here. The following individuals have my gratitude:

Connor Haines, for being a great brainstorming partner and excellent roommate.

Zontan Ferrah, for frequently telling me when my ideas were stupid. Sometimes he was even right.

Desmond Henderson, for his remarkably thorough and insightful feedback, and for being the first person brave enough to run a game with Rise.

Kyle McCauley, Linchaun Zhang, and Scott Kottkamp for helping to correct foolish errors that once littered this document and for providing ideas about how to improve the system.

Tarkisflux, for his excellent \href{http://dnd-wiki.org/wiki/Tome_of_Prowess_(3.5e_Sourcebook)}{Tome of Prowess}, feedback on spell design, and more.

Kholai, for keeping me on my toes in our monstrously long discussions of rules minutia.

Douglas Milewski, for his insightful essay on the nature of fighters. It helped inspire me to think that I could make D\&D right.

Rich Burlew, for providing a forum for house rule enthusiasts like myself to gather and for his insightful articles that guided my thought.

Deepbluediver, for our extended discussions on spell design and other topics. Many good ideas came from those conversations.

ElementsOfOrder, Jiriku, RobbyPants, Rogue\_Shadows, PlzBreakMyCampaign, Nail, Yitzi, nonsi256, Justin Alexander, James Flanagan, and others for the house rules and ideas that I borrowed from their work.

Ideasmith, for assistance in arranging the subschools of magic.

Shannon Carty, for the donation of her vocabulary.

Dave Rosenberg, for his insight into the nature of druids.

Paizo Publishing, LLC and the people who made the Pathfinder Roleplaying Game, for several excellent ideas, particularly Combat Maneuver Bonus.

Wizards of the Coast, for making a great game and releasing it under the OGL license, which makes all of this tinkering possible.
