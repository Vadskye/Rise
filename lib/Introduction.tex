\chapter{Introduction}

\section{What Is Rise?}
Rise is a role-playing game. In Rise, you play a character of your own design. Your character can try to do anything you can imagine in a world that the game master, or GM, creates. Of course, you won't always succeed. Your character's abilities are defined in the pages ahead; when you're done creating a character, it will have a personality of its own, along with strengths, weaknesses, and special abilities. Usually, your character will go on adventures with other characters, and together, you will create and experience a story with the game master.

\section{How To Take Actions}

If you want your character to do something in Rise, just tell the game master what you want your character to do. Sometimes, you will have to wait until it's your turn. In many cases, your character will simply take the action -- you don't need to roll or check your character's abilities to walk downstairs and order a dwarven ale.

However, sometimes you'll want to take a dramatic action with a chance of failure, such as picking a lock or swinging from a chandelier. In that case, you roll a a twenty-sided die, or d20, and add a number to the number on the die. This result is compared to a number representing how difficult the action is, or \term{Difficulty Class} (DC). If your result is at least as high as the DC, your character's action succeeds. If it is lower, your character's action fails.

The number you add to the d20 roll represents how likely your character is to succeed at the action. If your character is very strong, she will probably succeed at breaking down a door - but if she is not very perceptive, she will probably not notice the trap! Your character's abilities can be modified in many ways, but they are most affected by three things: attributes, skills, and classes.

\subsection{Opposed Actions}
Sometimes, you're competing with another creature, such as when you are trying to hold a door closed and a ferocious ogre is trying to shove it open. In that case, you both roll a d20 and add your modifiers, and the creature with the higher result wins. This is called an opposed check. If you get the same result, the creature with the higher modifier wins. If you have the same modifier, roll again to break the tie.

\subsection{Turns}
Normally, you can just say that your character is going to do something, and the GM will take care of what happens and when. Sometimes, particularly during combat, it's important to keep track of exactly what order things are happening in. When that happens, your character will get a ``turn'', where she can take actions. The specific actions you can take during a combat turn are covered in more detail in \pcref{Combat Overview}.

\section{Principles of Rise}

\begin{itemize}
    \item \textbf{The Game Master controls the world.} Everything about the game world is up to the GM: the people your character meets, the challenges she will overcome, and the very ground under her feet. Even the ``rules'' of the game are completely subject to the GM's whim.
    \item \textbf{You control your character.} A GM should never tell you how your character feels or what he tries to do -- unless, of course, your character is being controlled by hostile magic or some other power.
    \item \textbf{Respect and trust is critical.} The GM has a great deal of power in Rise, and the players have to trust that the GM knows what she is doing. Likewise, the GM needs to let the players do what they want -- even if it doesn't suit her idea of what ``should'' happen. Some of the most memorable events happen when players do things that are totally unexpected.
    \item \textbf{Everything is flexible.} Rise contains hundreds of pages of rules. We think they're pretty good rules. But sometimes, you don't need them all -- or you think you've come up with something better. Do whatever works for you and your group.
    \item \textbf{Do what makes sense.} This book doesn't contain rules for how to drink water, sleep, or blink. If something isn't described explicitly here, assume that it works the same way it does in reality.
    \item \textbf{It's just a game, so have fun.}
\end{itemize}

\subsection{Rule of Drama}

The Rule of Drama is simple: \textbf{Only dramatic actions matter}. This has several effects.

\parhead{Taking 20} It's not impossible for you to trip over a tree root while walking normally, but you shouldn't roll every time you try to take a step. You should only roll when your character is trying to do something where the chance of failure matters.

This applies even when there are specific game mechanics for the action, such as when searching for a hidden key. If the time required to perform the action doesn't matter, and there are no consequences for failure, it's not meaningful to roll. Instead, your character should just spend some time performing the action and automatically get the best result she can, as if you had rolled a 20. This is called \term{taking 20}. When possible, the GM should automatically have your character take 20 to minimize the amount of time you spend doing things that aren't dramatic.

\parhead{Narrative Time} In most cases, the exact time of day, and exactly how long an action takes, is not very important. No one really cares whether the orc horde broke through the town gates at the crack of dawn, or at 6:55 AM. Likewise, travelling from a tavern in a city to the blacksmith probably takes a few minutes, depending on the size of the city, but it's not usually important.

Rise contains rules which you can use to decide exactly how much time things take, but when it's not important, it's generally better to only worry about time in broad strokes. It makes everyone's life a bit easier -- especially for the GM.

\section{Character Creation}

The first thing you will probably want to do in Rise is create a character. There are four important decisions required to create a character.

\begin{enumerate*}
    \item \textbf{Personality}: Who is your character? What do they want? What is their background? It's good to have some idea of what kind of character you want to play, or else the number of options available can feel overwhelming. However, you don't need to decide everything about your character before you start creating it. In fact, you can figure out their personality as you start playing the game. You should keep your character's personality and style in mind as you make other decisions, however.
    \item \textbf{Attributes}: Is your character strong and stupid, frail but agile, or just generally talented? There are six core attributes in Rise: Strength, Dexterity, Constitution, Intelligence, Perception, and Willpower. These attributes determine a great deal about your character's strengths and weaknesses.
    \item \textbf{Race}: Is your character a surly dwarf, an agile elf, or a dim-witted half-orc? There are six core races in Rise: Humans, dwarves, elves, gnomes, halflings, and half-orcs. Your character's race doesn't have a strong effect on your character's abilities, but it can be important when thinking about your personality and background.
    \item \textbf{Class}: Your character's class is what they have chosen to focus on, and their source of power -- the fundamental element that makes them rise above a mere commoner. 
\end{enumerate*}

\section{Other Rules}

\subsection{Rounding}
In general, if you encounter a fractional number, you round it down.

\subsection{Multipliers}
Sometimes a rule makes you multiply a number or a die roll. As long as you�re applying a single multiplier, multiply the number normally. When two or more multipliers apply to any abstract value (such as a modifier or a die roll), however, combine them into a single multiple, with each extra multiple adding 1 less than its value to the first multiple. Thus, a double (\mult2) and a double (\mult2) applied to the same number results in a triple (\mult3, because 2 \add 1 = 3).

When applying multipliers to real-world values (such as weight or distance), normal rules of math apply instead. A creature whose size doubles (thus multiplying its weight by 8) and then is turned to stone (which would multiply its weight by a factor of roughly 3) now weighs about 24 times normal, not 10 times normal. Similarly, a blinded creature attempting to negotiate difficult terrain would need to spend 20 feet of movement to move 5 feet (doubling the cost twice, for a total multiplier of \mult4), rather than as 15 feet (adding 100\% twice).
