\chapter{Races and Backgrounds}

Each character has a race.

\section{Racial Traits}

\subsection{Racial Bonus Feats}
Each race grants a bonus feat at 1st level. Most races can only choose from a small group of feats, listed in the description of the race. A character must meet any prerequisites for these bonus feats, as normal.

\subsection{Favored Weapons}
The names of some exotic weapons, such as the orcish double axe, include the name of a race. Members of the named race are treated as being proficient with exotic weapons for the purpose of wielding those weapons.

\subsection{Race and Languages}
All characters know how to speak Common. A dwarf, elf, gnome, half-elf, half-orc, or halfling also speaks a racial language, as appropriate. For each point of Intelligence a character has at 1st level, they also know one additional language of their choice. See \pcref{Linguistics}, for details about languages.

\subsection{Small Characters}\label{Small Characters}
A Small character has the following effects based on their size.
  \begin{itemize} 
    \item \minus4 penalty to maneuver accuracy and defense.
    \item \plus1 bonus to other physical attacks and defenses.
    \item \plus4 bonus to Stealth checks.
    \item Carrying capacity is three-quarters that of a Medium character (see \pcref{Encumbrance}).
  \end{itemize}

In addition, a Small character generally moves about two-thirds as fast as a Medium character. A Small character must also use smaller weapons than a Medium character.

\section{Race Descriptions}

\subsection{Humans}
\parhead{Size} Medium.
\parhead{Attributes} No change.
\parhead{Speed} 30 feet.
\parhead{Special Abilities}
\begin{itemize}
    \itemhead*{Skilled}: Humans get 2 bonus skill points at 1st level. They can spend those skill points on any skills.
\end{itemize}
\parhead{Racial Bonus Feat} A human may choose any feat as a bonus feat.
\parhead{Automatic Language} Common.

\subsection{Dwarves}
\parhead{Size} Medium.
\parhead{Attributes} \plus1 Constitution, \minus1 Dexterity.
\parhead{Speed} 20 feet.
\parhead{Special Abilities}
\begin{itemize}
    \itemhead*{Darkvision}: Dwarves can see in the dark clearly up to 50 feet.   Beyond that, they can see dimly, treating areas of darkness as shadowy illumination. Darkvision does not function if a dwarf is in a brightly lit area, and does not resume functioning until 1 round after the dwarf leaves the brightly lit area.
    \itemhead*{Dwarven Endurance}: Dwarves do not move slower in heavy armor, and can run normally in both medium and heavy armor.
    \itemhead*{Stability}: While standing on solid ground, dwarves gain a \plus2 bonus to Maneuver defense against attacks that would move them.
\end{itemize}
\parhead{Racial Bonus Feat} Any from the following list: \featref*{Armor Proficiency}, \featref*{Endurance}, \featref*{Diehard}, \featref*{Dwarven Resilience}, \featref*{Giantfighter}, \featref*{Perfect Health}, \featref*{Stonecunning}, \featref*{Toughness}, \featref*{Weapon Proficiency (axes)}.
\parhead{Automatic Languages} Common, Dwarven.

\subsection{Elves}
\parhead{Size} Medium.
\parhead{Attributes} \plus1 Dexterity, \minus1 Constitution.
\parhead{Speed} 30 feet.
\parhead{Special Abilities}
\begin{itemize}
    \itemhead*{Keen Senses}: \plus2 bonus on Awareness checks.
    \itemhead*{Low-light Vision}: Elves treat sources of light as if they had double their normal illumination range.
    \itemhead*{Trance}: Elves do not sleep, and are immune to sleep effects. Instead of sleeping, elves can trance for 4 hours. An elf in trance may make Perception-based checks at a \minus5 penalty. Elves must still avoid strenuous activity for 8 hours to heal, avoid fatigue, and gain other benefits of resting.
\end{itemize}
\parhead{Racial Bonus Feat} Any from the following list: \featref*{Armor Proficiency} (light), \featref*{Battlecaster}, \featref*{Combat Casting}, \featref*{Dilettante}, \featref*{Focused Mind}, \featref*{Lightning Reflexes}, \featref*{Ritual Caster}, \featref*{Swift}, \featref*{Weapon Proficiency} (bows, heavy blades, or light blades)
\parhead{Automatic Languages} Common, Elven.

\subsection{Gnomes}
\parhead{Size} Small. This gives several benefits and penalties, as described at \pcref{Small Characters}.
\parhead{Attributes} \plus1 Constitution, \minus1 Strength.
\parhead{Speed} 20 feet.
\parhead{Special Abilities}
\begin{itemize}
    \itemhead*{Low-light Vision}: Gnomes treat sources of light as if they had double their normal illumination range.
\end{itemize}
\parhead{Racial Bonus Feat} Any magic feat, spellgift feat, or gnomish racial feat.
\parhead{Automatic Languages} Common, Gnome.

\subsection{Half-Elves}
\parhead{Size} Medium.
\parhead{Attributes} No change.
\parhead{Speed} 30 feet.
\parhead{Special Abilities}
\begin{itemize}
    \itemhead*{Dual Heritage}: For all effects related to race, a half-elf is considered both a human and an elf.
    \itemhead*{Low-light Vision}: Half-elves treat sources of light as if they had double their normal illumination range.
    \itemhead*{Skill Affinity}: Half-elves treat all skills as class skills.
\end{itemize}
\parhead{Racial Bonus Feat} Any skill feat or elven or human racial feat.
\parhead{Automatic Languages} Common, Elven.

\subsection{Half-Orcs}
\parhead{Size} Medium.
\parhead{Attributes} \plus1 Strength, \minus1 Intelligence, \minus1 Perception.
\parhead{Speed} 30 feet.
\parhead{Special Abilities}
\begin{itemize}
    \itemhead*{Darkvision}: Half-orcs can see in the dark clearly up to 50 feet.   Beyond that, they can see dimly, treating areas of darkness as shadowy illumination. Darkvision does not function if a half-orc is in a brightly lit area, and does not resume functioning until 1 round after the half-orc leaves the brightly lit area.
    \itemhead*{Dual Heritage}: For all effects related to race, a half-orc is considered both a human and an orc.
\end{itemize}
\parhead{Racial Bonus Feat} Any combat feat or orc or human racial feat.
\parhead{Automatic Languages} Common, Orc.

\subsection{Halflings}
\parhead{Size} Small.
\parhead{Attributes} \plus1 Dexterity, \minus1 Strength.
\parhead{Speed} 20 feet. This gives several benefits and penalties, as described at \pcref{Small Characters}.
\parhead{Special Abilities}
\begin{itemize}
    \itemhead*{Halfling Luck}: \plus1 to Fortitude and Mental defenses.
\end{itemize}
\parhead{Racial Bonus Feat} Any from the following list: \featref*{Giantfighter}, \featref*{Lightning Reflexes}, \featref*{Iron Will}, \featref*{Swift}, \featref*{Weapon Proficiency} (thrown).
\parhead{Automatic Languages} Common, Halfling.

\section{Backgrounds}
In addition to a race, each character also has at least one background. A background describes what a character has done before the start of the story. Suggested backgrounds are given below, but you can also create new backgrounds. You can choose anything your character might reasonably have done as a background. You can also choose to have multiple backgrounds if your character has done a variety of things.

Regardless of how you choose your background or backgrounds, choose any two skills related to what your character has done. You may treat those skills as class skills.

\subsection{Civilized Backgrounds}

\subsubsection{Bodyguard}
\parhead{Skills} Perception, Sense Motive.

\subsubsection{Commoner}
\parhead{Skill} Profession (any).

\subsubsection{Linguist}
\parhead{Skills} Linguistics, Knowledge (local).

\subsubsection{Jester}
\parhead{Skills} Acrobatics, Perform (comedy).

\subsubsection{Mage's Apprentice}
\parhead{Skills} Knowledge (arcana), Spellcraft.

\subsubsection{Merchant}
\parhead{Skills} Persuasion, Knowledge (local).

\subsubsection{Nobility}
\parhead{Skills} Bluff, Knowledge (local).

\subsubsection{Priest}
\parhead{Skill} Heal, Knowledge (religion).

\subsubsection{Scholar}
\parhead{Skill} Knowledge (any).

\subsubsection{Scribe}
\parhead{Skill} Craft (manuscript), Linguistics.

\subsubsection{Smith}
\parhead{Skill} Craft (any).

\subsubsection{Spy}
\parhead{Skills} Bluff, Disguise.

\subsubsection{Watchman}
\parhead{Skills} Knowledge (local), Perception.

\subsection{Military Backgrounds}

\subsubsection{Border Guard}
\parhead{Skill} Knowledge (geography), Survival.

\subsubsection{Cavalry}
\parhead{Skill} Creature Handling, Ride.

\subsubsection{Combat Engineer}
\parhead{Skill} Craft (any), Knowledge (engineering).

\subsubsection{Diplomat}
\parhead{Skills} Persuasion, Sense Motive.

\subsubsection{Infiltrator}
\parhead{Skills} Disguise, Stealth.

\subsubsection{Officer}
\parhead{Skills} Intimidate, Persuasion.

\subsubsection{Saboteur}
\parhead{Skills} Devices, Stealth.

\subsubsection{Scout}
\parhead{Skills} Perception, Stealth.

\subsection{Uncivilized Backgrounds}

\subsubsection{Bandit}
\parhead{Skills} Intimidate, Stealth.

\subsubsection{Explorer}
\parhead{Skills} Knowledge (geography), Survival.

\subsubsection{Hermit}
\parhead{Skill} Knowledge (nature), Survival.

\subsubsection{Minstrel}
\parhead{Skill} Perform (any).

\subsubsection{Primitive}
\parhead{Skill} Survival.

\subsubsection{Thief}
\parhead{Skills} Sleight of Hand, Stealth.
