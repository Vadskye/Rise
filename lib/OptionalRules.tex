\chapter{Optional Rules}

\section{Attributes}

\subsection{Other Methods of Attribute Generation}
Point buy offers the fairest and most customizable system for determining attribute scores, ensuring that players can be almost any character they want to be. However, some groups may wish to determine attribute scores differently. Other options are provided below.

\subsubsection{Semi-Randomized Point Buy}
With this method, you have only a small degree of control over your character's attribute scores, but all characters generated in this way are equally powerful. As with the point buy method, all your character's attribute scores start at 0, and you get 10 points to distribute among your character's attribute scores. However, you do not have full control over how to distribute those points.

Roll 4d6 for each attribute score, dropping a die of your choice with each roll. First roll for the attribute scores that you care about most, and save the least important attribute scores for last. After rolling for an attribute score, sum results on the three highest dice and consult \trefnp{Semi-Randomized Point Buy Results} and spend the appropriate number of points to yield an attribute score, as indicated by \trefnp{Attribute Score Point Costs}. If you do not have enough points remaining to spend the amount indicated by the die roll, spend as many as you can and move on to the next ability.

If you have points remaining after rolling all of your attribute scores, you may distribute the points freely among your abilities, using the normal point buy rules. You cannot increase any attribute above 3 during this stage.

After all of your points have been spent, you may swap any two of your attribute scores.

\begin{dtable}
\lcaption{Semi-Randomized Point Buy Results}
\begin{dtabularx}{\columnwidth}{X X X}
\thead{Roll} & \thead{Attribute Score} & \thead{Point Cost} \\
\hline
3-7   & \minus2 & \minus2\fn{1} \\
8-9   & \minus1 & \minus1\fn{1} \\
10-11 & 0  & 0 \\
12-13 & 1  & 1 \\
13-14 & 2  & 2 \\
15-16 & 3  & 3 \\
17    & 4  & 5 \\
18    & 5  & 8 \\
\end{dtabularx}
1 You gain extra points for having low stats. You can gain these points any number of times per character. \\
\end{dtable}

For characters with more extreme attribute scores, use the following approach for each attribute score, starting with 10 points as normal:
\begin{itemize}
  \item Roll 2d8
  \item Take the average, rounding down
  \item Subtract 3
  \item Spend the points as indicated on \trefnp{Attribute Score Point Costs} until you have no points left.
\end{itemize}

\subsubsection{Random Point Buy}
This method gives you no control over the character whatsoever, while still ensuring that all characters generated are equally powerful. It functions as the semi-randomized point buy method, except that you also randomize the order in which the attribute scores are rolled.

\begin{comment}
\subsubsection{Weighted Semi-Randomized Point Buy}
This method gives you more control over your attribute scores, while still requiring you to deal with attribute score advantages or disadvantages that you might not have chosen. It functions like the semi-randomized point buy method, except that you do not roll 3d6 for every attribute score. You have a pool of 18 dice. You can distribute those dice as you choose between each attribute score, except the last one, which is automatically assigned your remaining points. If you roll more than three dice for an attribute score, keep only the highest 3 dice. If you assign fewer than three dice to roll for an attribute score, add 1 to the result for every die less than 3 you are rolling. You can roll as many as six dice for a single attribute score, and you cannot roll less than one die for an attribute score.

For example, player hoping to play a barbarian might assign six dice to Strength, five dice to Constitution, three dice to Dexterity, and two dice to each of Intelligence and Wisdom, leaving the rest of the points to Charisma.
\end{comment}

\subsubsection{Classic Hardcore}

This method is completely random and can generate very overpowered or underpowered chracters. It represents the unfairness of the world, where some people are just better or worse than others. Roll 1d8 for each attribute score and subtract 3 from each result. The result is the attribute score.

\newpage

\section{Races}

\subsection{Awakened Animal}

Awakened animals are animals that have been granted sentience by the \spell{awaken} ritual.
The abilities of an awakened animal depend on the type of animal it is.

\parhead{Size} Tiny, Small, or Medium, as original animal.
\parhead{Attributes} The attributes of an awakened animal depend on its size.
\subparhead{Medium} No change.
\subparhead{Small} \plus1 Dexterity, \minus1 Strength.
\subparhead{Tiny} \plus2 Dexterity, \minus2 Strength.
\parhead{Speed} As the original animal.
\parhead{Special Abilities} As the original animal.
\parhead{Racial Bonus Feat} No racial bonus feat.

\subsubsection{Sample Awakened Animals}

\parhead{Cat}

\subparhead{Size} Tiny. As a Tiny character, a cat gains several benefits and penalties, as described at \pcref{Small Characters}.
\subparhead{Attributes} \plus2 Dexterity, \minus2 Strength.
\subparhead{Speed} 20 feet.
\subparhead{Special Abilities}
\begin{itemize}
    \itemhead{Scent} A cat has the scent ability (see \pcref{Scent}).
    \itemhead{Claws} A cat's paws end in claws, which it can use to attack (see \pcref{Natural Weapons}). A cat's claws do 1d3 damage.
    \itemhead{Low-light Vision} A cat treats sources of light as if they had double their normal illumination range.
\end{itemize}

\begin{comment}
\subsection{Changeling}

\parhead{Size} Medium.
\parhead{Attributes} No change.
\parhead{Speed} 30 feet.
\parhead{Special Abilities}
\begin{itemize}
    \itemhead{Alter Shape} A changeling can alter its physical form in minor ways. It gains a \plus10 bonus on Disguise checks, and it can disguise its body as a standard action. This ability does not alter the changeling's equipment, which may give away its identity unless disguised normally.
\end{itemize}
\parhead{Racial Bonus Feat} Any one from the following list: Open Minded, Skill Focus (Bluff, Disguise, Intimidate, Persuasion, or Sense Motive).
\parhead{Automatic Languages} Common and any one language (except Druidic).
\parhead{Bonus Languages} Any.
\end{comment}

\subsection{Dryaidi}

Dryaidi are humanoid creatures that resemble plants. They are descended from dryads.

\parhead{Size} Medium.
\parhead{Attributes} \plus1 Constitution, \minus1 Dexterity.
\parhead{Speed} 20 feet.
\parhead{Special Abilities}
\begin{itemize}
    \itemhead{Ingrain} As a standard action, a dryaidi can plant roots into natural earth. While ingrained, a dryaid's land speed becomes 5 feet, but she gains a \plus2 bonus to her Maneuver defense against attacks that would move her. Resting for 4 hours while ingrained gains the same benefits that a human would gain from 8 hours of rest, including healing. In addition, the dryaidi acquires nutrients sufficient to replace a day's worth of food and water. Withdrawing ingrained roots is a full-round action.
    \itemhead{Photosynthesis} While in sunlight, a dryaidi gains a \plus10 foot bonus to land speed.
\end{itemize}
\parhead{Racial Bonus Feat} Dryad Heritage.

\subsection{Tieflings}

Tieflings are humanoid creatures descended from fiends.
\parhead{Size} Medium.
\parhead{Attributes} No change.
\parhead{Speed} 30 feet.
\parhead{Special Abilities}
\begin{itemize}
    \itemhead{Darkvision} Tieflings can see in the dark clearly up to 50 feet. Beyond that, they can see dimly, treating areas of darkness as shadowy illumination. Darkvision does not function if a tiefling is in a brightly lit area, and does not resume functioning until 1 round after the tiefling leaves the brightly lit area.
    \itemhead{Energy Resistance} A tiefling has damage reduction against cold, electricity, and fire equal to twice its level.
\end{itemize}
\parhead{Racial Bonus Feat} Fiendish Heritage.

\section{Feats}

\feat{Body of the Bending Willow}{Bloodline, Fae}
\featpre Dryad Heritage.
\featben You gain a \plus2 bonus to Escape Artist and Stealth checks.

If you have three or more fae bloodline feats, you can also walk between trees. As a move action, you can step into an adjacent of at least Medium tree and out of any 

\feat{Body of the Mighty Oak}{Bloodline, Fae}
\featpre Dryad Heritage.
\featben You gain a \plus1 bonus to Armor defense.

If you have three or more fae bloodline feats, you can also ingrain in natural earth or stone.

\feat{Deep Ingrain}{Bloodline, Fae}
\featpres Dryad Heritage, Con 3.
\featben When you ingrain, you may spend a fae point to deeply ingrain your roots. While deeply ingrained, your bonus to Maneuver defense increases to \plus5. In addition, you can draw nutrients from the earth to heal hit points equal to your fae power as a swift action. You can only regain hit points in this way 5 times before you deplete the available nutrients in the area. 

\feat{Dryad Heritage}{Bloodline, Fae}
\featpre Dryaidi.
\featben As a standard action, you can gain the ability to speak with trees. This ability functions like the druid's wild speech ability, except that it only allows you to communicate with trees. You can use this ability a number of times per day equal to the number of fae bloodline feats you possess.

\feat{Fiendish Heritage}{Bloodline, Fiendish}
\featpre Tiefling or nongood alignment.
\featben You have the blood of a fiendish creature in your veins, granting you fiendish power.
Your fiendish power is equal to your Willpower, or your level \add the number of fiendish bloodline feats you possess, whichever is higher.
You have a pool with a number of fiend points equal to the number of fiendish bloodline feats you possess.

As a standard action, you can spend a fiend point to surround yourself in \areamed radius emanation of darkness for \durshort duration.
All light within the area is reduced to be no brighter than shadowy illumination.
This typically grants you concealment, allowing you to hide.
