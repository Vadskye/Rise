\chapter{Glossary}\label{Glossary}

\glossdef{attack}[attacks] Anything that affects another creature in a potentially harmful way. There are two kinds of attacks: physical attacks and special attacks.

\glossdef{attack roll}[attack rolls] A roll required to succeed with an attack. If the result of the attack roll equals or exceeds the target's \glossterm{defense}, the attack succeeds. Some attacks, especially magical attacks, have effects even if the attack roll fails.

\glossdef{attribute} A core representation of a character's capacity in a wide range of areas. There are six attributes: \glossterm{Strength}, \glossterm{Dexterity}, \glossterm{Constitution}, \glossterm{Intelligence}, \glossterm{Perception}, and \glossterm{Willpower}.

\glossdef{base bonus} A base bonus is the value of a bonus before any temporary modifiers or effects. For example, your base Fortitude defense bonus is equal to the Fortitude defense bonus granted by your class.

\glossdef{class skill}[class skills] A class skill is a skill which you can train with using skill points from your class. See \pcref{Skill Training}.

\glossdef{compulsion}[compulsions] Compulsion effects force you to take or avoid specific actions. All compulsions are \glossterm{mind} effects.

\glossdef{critical damage penalties} If you have \glossterm{critical damage}, you take a penalty to attacks, checks, and defenses equal to the amount of critical damage you have.

\glossdef{damage reduction} Damage reduction allows you to ignore a certain amount of incoming damage.
Each round, you ignore the first points of damage you would take.
Damage reduction always specifies an amount of damage it reduces.
Once it reduces that much damage, it stops functioning until the end of the round.

Most sources of damage reduction only apply against a specific type of attack.
For example, a barbarian's damage reduction only applies against physical damage.
If an attack deals multiple types of damage, you must have damage reduction against every type of damage dealt.
For example, damage reduction against piercing damage would not help if you are struck by a morningstar, since it deals both bludgeoning and piercing damage.

Many sources of damage reduction can be ignored and negated by a specific type of attack.
For example, the \spell{barkskin} spell grants damage reduction that can be ignored and negated by fire and slashing damage.
If you are hit an attack that negates your damage reduction, you cannot apply your damage reduction against any other attacks that round.
This includes other attacks that resolve simultaneously, but not attacks that resolved earlier in the round.
For example, if you had the \spell{barkskin} spell active, and you were hit by a club (bludgeoning damage) and a longsword (slashing damage), you would take full damage both attacks.
However, if you were instead hit by a club and a \spell{fireball} spell (fire damage), you would reduce the damage from the club, because the spell resolves later in the round.

\glossdef{darkvision} A creature with darkvision can see in the dark clearly up to a given range.
Beyond that, it can see dimly, treating areas of darkness as shadowy illumination.
Darkvision does not function if a creature is in a brightly lit area, and does not resume functioning until 1 round after the creature leaves the brightly lit area.

\glossdef{defense}[defenses] A defense is a static number which represents how difficult you are to affect with attacks. See \glossterm{attack rolls}.

\glossdef{delusion}[delusions] Delusion effects change your opinions or how your mind works.
All delusions are \glossterm{mind} effects.

\glossdef{difficult terrain} Difficult terrain costs double the normal movement cost to move through.

\glossdef{energy damage} Damage from one of the four types of energy: acid, cold, electricity, and fire.

\glossdef{enhancement bonus} Magic armor and weapons can have enhancement bonuses.
Each \plus1 of enhancement bonus on magic armor grants temporary hit points equal the item's power, and grants you an additional defensive legend point each day.

You gain a bonus to damage on physical attacks using a magic weapon equal to the weapon's enhancement bonus.
In addition, each \plus1 of enhancement bonus on a weapon grants you an additional offensive legend point each day.

See \pcref{Armor Enhancement Bonuses} and \pcref{Weapon Enhancement Bonuses} for details.

\glossdef{falling damage} For every 10 feet you fall, you take 1d6 bludgeoning damage, to a maximum of 20d6 damage.
If you control your fall with a successful Jump or Tumble check, you can reduce the falling damage you take (see \pcref{Jump}, and \pcref{Tumble}).

\glossdef{hit value} Your hit value is the number of hit points you gain per level. Your hit value is equal to half your Constitution, half your Willpower, or half your base Fortitude bonus, whichever is higher.

\glossdef{mind} Mind effects alter something about your mind. Creatures without minds, such as oozes and golems, are immune to mind effects.

\glossdef{potency} The potency of a poison, disease, or similar effect determines its attack bonus

\glossdef{random effect}[random effects] Random effects change what they do based on a specific die roll.
This does not include effects which require a successful attack or similar roll.
The \spell{prismatic beam} spell is an example of a random effect.
In addition, the random retargeting of certain miscast spells, such as \spell{scorching ray}, is a random effect.

\glossdef{restricted spell}[restricted spells] Arcane spells are divided into two categories: restricted spells, and unrestricted spells. Most spells are unrestricted spells, which can be be learned by both sorcerers and wizards. Restricted spells are listed in italics. Wizards have a limited ability to learn and cast restricted spells with the Arcane Insight class feature (see \pcref{Arcane Insight} for details).

\glossdef{wild magic roll} Whenever a sorcerer casts a spell, he must make a wild magic roll.
Success means he gains a bonus to his spellpower with the spell.
Failure means the spell's miscast effect occurs in addition to its normal effect, and the sorcerer loses the ability to cast spells of the same level.
See \pcref{Wild Magic}, for details.

\glossdef{unrestricted spell}[unrestricted spells] The most common kind of arcane spell. See \glossterm{restricted spells}.
