\chapter{Glossary}\label{Glossary}

\glossdef{acid}[Acid] A kind of \glossterm{energy}. Acid damage is very effective against most objects. For the Acid spell tag, see \tref{Spell Tags}.

\glossdef{action phase} The action phase is the second of two \glossterm{phases} in a combat \glossterm{round}.
During the action phase, creatures can \glossterm{attack}, cast \glossterm{spells}, and take other major combat actions.

\glossdef{accuracy} The bonus added to a \glossterm{attack roll}.

\glossdef{Air} See \tref{Spell Tags}.

\glossdef{Antimagic} See \tref{Spell Tags}.

\glossdef{attack}[attacks] Anything that affects another creature in a potentially harmful way. There are two kinds of attacks: physical attacks and special attacks.

\glossdef{attack roll}[attack rolls] A roll required to succeed with an attack.
To make an attack roll, roll 1d20 \add your \glossterm{accuracy} with the attack.
If the result of the attack roll equals or exceeds the target's \glossterm{defense}, the attack succeeds.
Some attacks, especially magical attacks, have effects even if the attack roll fails.

\glossdef{attribute} A core representation of a character's capacity in a wide range of areas. There are six attributes: \glossterm{Strength}, \glossterm{Dexterity}, \glossterm{Constitution}, \glossterm{Intelligence}, \glossterm{Perception}, and \glossterm{Willpower}.

\glossdef{Auditory} See \tref{Spell Tags}.

\glossdef{Augment} See \tref{Spell Tags}.

\glossdef{Barrier} See \tref{Spell Tags}.

\glossdef{base bonus} A base bonus is the value of a bonus before any temporary modifiers or effects. For example, your base Fortitude defense bonus is equal to the Fortitude defense bonus granted by your class.

\glossdef{Charm} See \tref{Spell Tags}.

\glossdef{chaotic}[Chaotic] Relating to chaos, one of the four \glossterm{alignment} components. For the Chaotic spell tag, see \tref{Spell Tags}.

\glossdef{class skill}[class skills] A class skill is a skill which you can train with using skill points from your class. See \pcref{Skill Training}.

\glossdef{cold}[Cold] A kind of \glossterm{energy}. For the Cold spell tag, see \tref{Spell Tags}.

\glossdef{combat maneuver}[combat maneuvers] A combat maneuver is an attempt to physically hinder your foe with your body, such as by shoving or tripping it.
Most combat maneuvers are made in place of a \glossterm{strike}.

\glossdef{Compulsion}[Compulsions] See \tref{Spell Tags}.

\glossdef{Creation} See \tref{Spell Tags}.

\glossdef{Curse} See \tref{Spell Tags}.

\glossdef{critical damage penalties} If you have \glossterm{critical damage}, you take a penalty to accuracy, checks, and defenses equal to the amount of critical damage you have.

\glossdef{damage reduction} Damage reduction allows you to ignore a certain amount of incoming damage.
Each \glossterm{round}, you ignore the first points of damage you would take.
Damage reduction always specifies an amount of damage it reduces.
Once it reduces that much damage, it stops functioning until the end of the round.

Most sources of damage reduction only apply against a specific type of attack.
For example, a barbarian's damage reduction only applies against physical damage.
If an attack deals multiple types of damage, you must have damage reduction against every type of damage dealt.
For example, damage reduction against piercing damage would not help if you are struck by a morningstar, since it deals both bludgeoning and piercing damage.

Many sources of damage reduction can be ignored and negated by a specific type of attack.
For example, the \spell{barkskin} spell grants damage reduction that can be ignored and negated by fire and slashing damage.
If you are hit an attack that negates your damage reduction, you cannot apply your damage reduction against any other attacks that round.
This includes other attacks that resolve simultaneously, but not attacks that resolved earlier in the round.
For example, if you had the \spell{barkskin} spell active, and you were hit by a club (bludgeoning damage) and a longsword (slashing damage), you would take full damage both attacks.
However, if you were instead hit by a club and a \spell{fireball} spell (fire damage), you would reduce the damage from the club, because the spell resolves later in the round.

\glossdef{darkvision} A creature with darkvision can see in the dark clearly up to a given range.
Beyond that, it can see dimly, treating areas of darkness as shadowy illumination.
Darkvision does not function if a creature is in a brightly lit area, and does not resume functioning until 1 round after the creature leaves the brightly lit area.

\glossdef{Death} See \tref{Spell Tags}.

\glossdef{defense}[defenses] A defense is a static number which represents how difficult you are to affect with attacks. See \glossterm{attack rolls}.

\glossdef{Delusion} See \tref{Spell Tags}.

\glossdef{Detection} See \tref{Spell Tags}.

\glossdef{difficult terrain} Difficult terrain costs double the normal movement cost to move out of.

\glossdef{disease} An affliction of the body, causing a steady deterioration over time. For the Disease spell tag, see \tref{Spell Tags}.

\glossdef{Earth} See \tref{Spell Tags}.

\glossdef{electricity}[Electricity] A kind of \glossterm{energy}. For the Electricity spell tag, see \tref{Spell Tags}.

\glossdef{energy}[energy damage] There are four types of energy: acid, cold, electricity, and fire. Energy effects often deal damage.

\glossdef{Enchantment} Enchantment spells affect the minds of others, influencing or controlling their behavior or mental capabilities. Almost all enchantment spells are \glossterm{Mind} spells, and many of them are \glossterm{Subtle} as well.

\glossdef{enhancement bonus} Magic armor and weapons can have enhancement bonuses.
Each \plus1 of enhancement bonus on magic armor grants temporary hit points equal the item's power, and grants you an additional defensive legend point each day.

You gain a bonus to damage on physical attacks using a magic weapon equal to the weapon's enhancement bonus.
In addition, each \plus1 of enhancement bonus on a weapon grants you an additional offensive legend point each day.

See \pcref{Armor Enhancement Bonuses} and \pcref{Weapon Enhancement Bonuses} for details.

\glossdef{evil}[Evil] One of the four \glossterm{alignment} components. For the Evil spell tag, see \tref{Spell Tags}.

\glossdef{falling damage} For every 10 feet you fall, you take 1d6 bludgeoning damage, to a maximum of 20d6 damage.
If you control your fall with a successful Jump or Tumble check, you can reduce the falling damage you take (see \pcref{Jump}, and \pcref{Tumble}).

\glossdef{fast healing} A creature with fast healing automatically heals hit points at the end of every round.
Like other healing, this healing offsets damage taken during the round for the purposes of taking \glossdef{critical damage} and becoming \disabled.

\glossdef{Fear} See \tref{Spell Tags}.

\glossdef{fire}[Fire] A kind of \glossterm{energy}. For the Fire spell tag, see \tref{Spell Tags}.

\glossdef{Figment} See \tref{Spell Tags}.

\glossdef{Flesh} See \tref{Spell Tags}.

\glossdef{Fog} See \tref{Spell Tags}.

\glossdef{Force} See \tref{Spell Tags}.

\glossdef{Glamer} See \tref{Spell Tags}.

\glossdef{good}[Good] One of the four \glossterm{alignment} components. For the Evil spell tag, see \tref{Spell Tags}.

\glossdef{hit value} Your hit value is the number of hit points you gain per level. Your hit value is equal to half your Constitution, half your Willpower, or half your base Fortitude bonus, whichever is higher.

\glossdef{Instantaneous} See \tref{Spell Tags}.

\glossdef{lawful}[Lawful] Relating to law, one of the four \glossterm{alignment} components. For the Lawful spell tag, see \tref{Spell Tags}.

\glossdef{legend point}[legend points] Legend points can be used to reroll failed rolls, or force your foes to reroll successful rolls against you. See \pcref{Legend Points}, for details.

\glossdef{Life} See \tref{Spell Tags}.

\glossdef{Light} See \tref{Spell Tags}.

\glossdef{Mind} See \tref{Spell Tags}.

\glossdef{melee attack} A melee attack is a physical \glossterm{attack} against a creature within your \glossterm{reach}.

\glossdef{Morale} See \tref{Spell Tags}.

\glossdef{move} When you move, you usually travel a distance equal to your speed.
See \pcref{Movement and Positioning}, for details.
For specific actions that involve movement, see \glossterm{move action}.

\glossdef{move action}[move actions] A move action is a minor action that requires motion, such as drawing a sword.
You can take move actions during the \glossterm{movement phase}.
For the act of moving from one place to another, see \glossterm{move}.

\glossdef{movement phase} The movement phase is the first of two \glossterm{phases} in a combat \glossterm{round}.
During the movement phase, creatures can \glossterm{move} and take \glossterm{move actions}.
The movement phase is followed by the \glossterm{action phase}.

\glossdef{Negative} See \tref{Spell Tags}.

\glossdef{overwhelmed} An overwhelmed creature is suffering \glossterm{overwhelm penalties}.

\glossdef{overwhelm penalties} A creature \glossterm{threatened} by at least two creatures suffers a penalty to physical defenses (Armor, Manuever, Reflex).
The size of the penalty is equal to the number of creatures threatening it, to a maximum of \minus8.
These penalties are called overwhelm penalties.
A creature suffering overwhelm penalties is \glossterm{overwhelmed}.

\glossdef{phase}[phases] A phase is part of the combat \glossterm{round}.
There are two phases: the \glossterm{movement phase} and the \glossterm{action phase}.
A phase does not represent a fixed span of time.
It is an abstract concept designed to represent a variety of actions that all take place nearly simultaneously.

\glossdef{Physical} See \tref{Spell Tags}.

\glossdef{Planar} See \tref{Spell Tags}.

\glossdef{poison} For a description of poisons and how they work, see \pcref{Poisons}. For the Poison spell tag, see \tref{Spell Tags}.

\glossdef{Positive} See \tref{Spell Tags}.

\glossdef{potency} The potency of a poison, disease, or similar effect determines its attack bonus.

\glossdef{random effect}[random effects] Random effects change what they do based on a specific die roll.
This does not include effects which require a successful attack or similar roll.
The \spell{prismatic beam} spell is an example of a random effect.
In addition, the random retargeting of certain miscast spells, such as \spell{scorching ray}, is a random effect.

\glossdef{range} The range of an ability determines how far away it can be used.
You can't use abilities on a target outside of the ability's range.

\glossdef{Retritutive} See \tref{Spell Tags}.

\glossdef{reach} Your reach is how far away from your body you can make melee attacks.
A typical Medium creature has a five-foot reach.

\glossdef{restricted spell}[restricted spells] Arcane spells are divided into two categories: restricted spells, and unrestricted spells. Most spells are unrestricted spells, which can be be learned by both sorcerers and wizards. Restricted spells are listed in italics. Wizards have a limited ability to learn and cast restricted spells with the Arcane Insight class feature (see \pcref{Arcane Insight} for details).

\glossdef{round} Combat takes place in a series of rounds, which represent about six seconds of action.
Rounds are divided into two \glossterm{phases}: the \glossterm{movement phase}, and the \glossterm{action phase}.

\glossdef{Scrying} See \tref{Spell Tags}.

\glossdef{Shielding} See \tref{Spell Tags}.

\glossdef{Sizing} See \tref{Spell Tags}.

\glossdef{Speech} See \tref{Spell Tags}.

\glossdef{square} A square represents a single 5-ft. by 5-ft. space.
A typical Medium creature occupies a single square in combat.

\glossdef{standard action} You can use a standard action to attack with a weapon, cast a spell, drink a potion, and do most other things that take concentration and effort.

\glossterm{standard attack} A standard attack is the most common way to attack with weapons you wield.
During the action phase, you can make a standard attack to \glossterm{strike} foes with your weapons.
If you have a high \glossterm{combat prowess}, you may be able to make multiple strikes during a standard attack (see \pcref{Multiple Attacks}).

\glossdef{strike}[strikes] A strike is a single physical attack with a weapon.
You usually make strikes during a \glossterm{standard attack} in the \glossterm{action phase}.
You can make some special attacks in place of strikes, such as \glossterm{combat maneuvers}.

\glossdef{Subtle} See \tref{Spell Tags}.

\glossdef{Summoning} See \tref{Spell Tags}.

\glossdef{Teleportation} See \tref{Spell Tags}.

\glossdef{threaten}[threatened] When using a melee weapon, you threaten any creatures within the weapon's \glossterm{reach}.
A typical Medium creature threatens creatures in all adjacent squares.
If you threaten a creature, you can make \glossterm{melee} attacks against it, and you can make it suffer \glossterm{overwhelm penalties}.

\glossdef{threatened area} The area that you can make melee attacks into, as determined by your \glossterm{reach}.
The threatened area of a typical Medium creature consists of all squares adjacent to the creature.

\glossdef{Trap} See \tref{Spell Tags}.

\glossdef{Unreal} See \tref{Spell Tags}.

\glossdef{Water} See \tref{Spell Tags}.

\glossdef{wild magic roll} Whenever a sorcerer casts a spell, he must make a wild magic roll.
Success means he gains a bonus to his spellpower with the spell.
Failure means the spell's miscast effect occurs in addition to its normal effect, and the sorcerer loses the ability to cast spells of the same level.
See \pcref{Wild Magic}, for details.

\glossdef{unrestricted spell}[unrestricted spells] The most common kind of arcane spell. See \glossterm{restricted spells}.
